    
\chapter*{~ために II}

\begin{center}
\begin{Large}
第???課: ~ために II: Due to\dothyp{}\dothyp{}\dothyp{} 
\end{Large}
\end{center}
 
\par{ ~ために, as hinted at in the introduction of the previous lesson, is not limited to expressing an objective. Another primary interpretation that may or may not coincide with an objective is expressing a cause. This lesson will help you understand exactly how to use and understand this "cause-marking" ~ために and how it contrasts and coincides with the "objective-marking" ~ために. }
      
\section{Marking a Cause}
 
\par{ The cause-marking ~ために is used to express the cause of something in a reserved manner. It is equivalent to the English phrase “due to.” It can be seen after nouns, adjectival nouns, adjectives, and verbs. For conjugatable parts of speech, it can be seen after both the non-past and past tenses. }

\begin{ltabulary}{|P|P|P|}
\hline 

Part of Speech & Non-Past Tense & Past Tense \\ \cline{1-3}

Nouns & N + の + ため(に) & N + だった + ため(に) \\ \cline{1-3}

Adjectival  Nouns & Adj. N + な + ため(に) & Adj. N + だった + ため(に) \\ \cline{1-3}

Adjectives & Adj. + ため(に) & Adj. + かった + ため(に) \\ \cline{1-3}

Verbs & V + ため(に) & V + た + ため(に) \\ \cline{1-3}

\end{ltabulary}

\begin{center}
 \textbf{Semantic Restrictions }\hfill\break

\end{center}

\par{ The cause-marking ~ために is only used in declarative sentences and the questions made from said sentences. The basic sentence pattern with the cause-marking ために is “A + ために + B.” The A expresses cause if B is the result of A having occurred. Generally speaking, when the agent of A and B are the same, then ~ために expresses an objective, but when the agent of A and B differ, it expresses cause. This rule of thumb only works, though, with the non-past tense. When you see ~ため(に) after the past tense or the progressive form (~ている), or whenever both clauses lack a true active agent, then it is always the cause-marking interpretation. }

\par{\textbf{Particle Note }: The に in ~ために is frequently omitted, and when the statement is a positive one, it\textquotesingle s most natural to omit it. When に isn\textquotesingle t present, the statement is as neutral as can be. When に is present, the cause stands out more, but there is no emphasis as to whether the cause is good or bad. }

\par{\textbf{Usage Note }: This pattern is almost entirely used in the written language and\slash or in news(-like speech). }

\par{1. ${\overset{\textnormal{げんざい}}{\text{現在}}}$ は ${\overset{\textnormal{こうじ}}{\text{工事}}}$ \textbf{のために }${\overset{\textnormal{へいさちゅう}}{\text{閉鎖中}}}$ です。 \hfill\break
(Road) currently closed \textbf{due to }construction. }

\par{2. ${\overset{\textnormal{もっと}}{\text{最}}}$ も ${\overset{\textnormal{とくひょう}}{\text{得票}}}$ の ${\overset{\textnormal{おお}}{\text{多}}}$ い ${\overset{\textnormal{こうほしゃ}}{\text{候補者}}}$ でも ${\overset{\textnormal{とうせん}}{\text{当選}}}$ に ${\overset{\textnormal{ひつよう}}{\text{必要}}}$ な ${\overset{\textnormal{ひょうすう}}{\text{票数}}}$ に ${\overset{\textnormal{とど}}{\text{届}}}$ かなかった \textbf{ため }、 ${\overset{\textnormal{さいせんきょ}}{\text{再選挙}}}$ が ${\overset{\textnormal{おこな}}{\text{行}}}$ われることになった。 \hfill\break
 \textbf{Due to }the fact that not even the candidate who won the most votes reached the necessary count to be elected, a repeat election is to be held. }

\par{3. ${\overset{\textnormal{かのじょ}}{\text{彼女}}}$ は ${\overset{\textnormal{ようしたんれい}}{\text{容姿端麗}}}$ \{ \textbf{な・だった }\} \textbf{ため }\{に X・ø 〇\}、 ${\overset{\textnormal{かずかず}}{\text{数々}}}$ の ${\overset{\textnormal{だんしがくせい}}{\text{男子学生}}}$ の ${\overset{\textnormal{あこが}}{\text{憧}}}$ れの ${\overset{\textnormal{まと}}{\text{的}}}$ だった。 \hfill\break
She was the object of adoration of numerous male students \textbf{due to }her attractive face and figure. }

\par{4. ${\overset{\textnormal{かのじょ}}{\text{彼女}}}$ は ${\overset{\textnormal{ようしたんれい}}{\text{容姿端麗}}}$ \textbf{なために }、 ${\overset{\textnormal{かずかず}}{\text{数々}}}$ の ${\overset{\textnormal{だんしがくせいこくはく}}{\text{男子学生告白}}}$ されたりするのにうんざりしている。 \hfill\break
She is fed up with having numerous male students confess their love for her \textbf{due to }her attractive face and figure. }

\par{5. \textbf{このため }、フジテレビは ${\overset{\textnormal{こんご}}{\text{今後}}}$ の ${\overset{\textnormal{ほうそう}}{\text{放送}}}$ を ${\overset{\textnormal{と}}{\text{取}}}$ りやめることにした。 \hfill\break
Because of this, Fuji Television has decided to call off further broadcast (of the program). }

\par{6. ${\overset{\textnormal{ぎょうせき}}{\text{業績}}}$ が ${\overset{\textnormal{みぎかたさ}}{\text{右肩下}}}$ がりの ${\overset{\textnormal{じき}}{\text{時期}}}$ に ${\overset{\textnormal{にゅうしゃ}}{\text{入社}}}$ した \textbf{ためか }、 ${\overset{\textnormal{そしきへんこう}}{\text{組織変更}}}$ の繰り ${\overset{\textnormal{かえ}}{\text{返}}}$ しで ${\overset{\textnormal{ひと}}{\text{人}}}$ の ${\overset{\textnormal{でい}}{\text{出入}}}$ りが ${\overset{\textnormal{はげ}}{\text{激}}}$ しい。 \hfill\break
 \textbf{Perhaps due to }having entered the company at a time when performance was declining, turnover was relentless with repeated structure changes. }

\par{\textbf{Grammar Note }: ~ためか is equivalent to “perhaps due to.” }

\par{7. ${\overset{\textnormal{あね}}{\text{姉}}}$ が ${\overset{\textnormal{けっこん}}{\text{結婚}}}$ する \textbf{ため }、 ${\overset{\textnormal{ちち}}{\text{父}}}$ はアルバイトして、お ${\overset{\textnormal{かね}}{\text{金}}}$ を貯めている。 \hfill\break
My father is working a part-time job and saving money \textbf{because }my older sister is marrying. }

\par{\textbf{Sentence Note }: It may also be the case that the father is saving the money so that his daughter may marry, but this would be something one must confirm whether this is true with the father. }

\par{8. ${\overset{\textnormal{あね}}{\text{姉}}}$ は ${\overset{\textnormal{けっこん}}{\text{結婚}}}$ する \textbf{ために }、アルバイトして、お ${\overset{\textnormal{かね}}{\text{金}}}$ を貯めている。 \hfill\break
My older sister is working a part-time job saving money \textbf{in order to }marry. }

\par{\textbf{Sentence Note }: Because the agent is the same in all clauses of the sentence, ために is interpreted here as “in order to.” The same can be said for Ex. 9. }

\par{9. ${\overset{\textnormal{けんきゅうしゃ}}{\text{研究者}}}$ も ${\overset{\textnormal{なぞ}}{\text{謎}}}$ を ${\overset{\textnormal{さぐ}}{\text{探}}}$ る \textbf{ため(に) }${\overset{\textnormal{じっけん}}{\text{実験}}}$ に ${\overset{\textnormal{の}}{\text{乗}}}$ り ${\overset{\textnormal{だ}}{\text{出}}}$ した。 \hfill\break
Scientist have also set out to experimenting \textbf{(in order) to }probe the mystery. }

\par{10. ${\overset{\textnormal{あきさめぜんせん}}{\text{秋雨前線}}}$ が ${\overset{\textnormal{ちか}}{\text{近}}}$ づいている \textbf{ため }、 ${\overset{\textnormal{くも}}{\text{曇}}}$ り ${\overset{\textnormal{ぞら}}{\text{空}}}$ となりました。 \hfill\break
The weather has become cloudy \textbf{due to }an autumnal rain front approaching. }

\par{11. ${\overset{\textnormal{けっきょく}}{\text{結局}}}$ ${\overset{\textnormal{さいしゅう}}{\text{最終}}}$ バスに ${\overset{\textnormal{の}}{\text{乗}}}$ り ${\overset{\textnormal{おく}}{\text{遅}}}$ れた \textbf{ため }、 ${\overset{\textnormal{ほか}}{\text{他}}}$ の ${\overset{\textnormal{ともだち}}{\text{友達}}}$ と ${\overset{\textnormal{とほ}}{\text{徒歩}}}$ でキャンパスへ ${\overset{\textnormal{きかん}}{\text{帰還}}}$ 。 \hfill\break
Returned to campus by foot with other friends \textbf{due to }ultimately missing the final bus. }

\par{\textbf{Sentence Note }: This sentence is representative of blog-style writing. }

\par{12. ${\overset{\textnormal{まわ}}{\text{周}}}$ りの ${\overset{\textnormal{ひと}}{\text{人}}}$ に ${\overset{\textnormal{み}}{\text{見}}}$ せびらかしたい \textbf{ために }、ゲームがしたい \textbf{ために }${\overset{\textnormal{きしゅ}}{\text{機種}}}$ を ${\overset{\textnormal{か}}{\text{変}}}$ える、 ${\overset{\textnormal{きみ}}{\text{君}}}$ の ${\overset{\textnormal{りょうしん}}{\text{両親}}}$ はそんな ${\overset{\textnormal{りゆう}}{\text{理由}}}$ で ${\overset{\textnormal{なんまん}}{\text{何万}}}$ ( ${\overset{\textnormal{えん}}{\text{円}}}$ )も ${\overset{\textnormal{だ}}{\text{出}}}$ せるわけが ${\overset{\textnormal{な}}{\text{無}}}$ いでしょう? \hfill\break
Changing devices \textbf{because }you want to show off to people around or \textbf{because }you want to play a game… there\textquotesingle s no way that your parents could dish out tens of thousands of yen for those sort of reasons, no? }

\par{13. ${\overset{\textnormal{ゆう}}{\text{夕}}}$ べはぐっすり ${\overset{\textnormal{ね}}{\text{寝}}}$ ていた \textbf{ため }、 ${\overset{\textnormal{じしん}}{\text{地震}}}$ に ${\overset{\textnormal{き}}{\text{気}}}$ が ${\overset{\textnormal{つ}}{\text{付}}}$ きませんでした。 \hfill\break
 \textbf{As }I was fast asleep last night, I did not feel the earthquake. }

\par{14. ${\overset{\textnormal{じゅうぶん}}{\text{十分}}}$ な ${\overset{\textnormal{じょうほう}}{\text{情報}}}$ がない \textbf{ため }、サーバーを ${\overset{\textnormal{けんしょう}}{\text{検証}}}$ できません。 \hfill\break
Cannot access server \textbf{due to }insufficient information. }

\par{15. ${\overset{\textnormal{のうむ}}{\text{濃霧}}}$ \textbf{のため }、 ${\overset{\textnormal{くるま}}{\text{車}}}$ という ${\overset{\textnormal{くるま}}{\text{車}}}$ はみなライトをつけて ${\overset{\textnormal{はし}}{\text{走}}}$ っている。 \hfill\break
 \textbf{Due to }dense fog, every single car is running with its headlights on. }

\par{16. ${\overset{\textnormal{こ}}{\text{濃}}}$ い ${\overset{\textnormal{きり}}{\text{霧}}}$ \textbf{のため }、 ${\overset{\textnormal{かんせいとう}}{\text{管制塔}}}$ からの ${\overset{\textnormal{しじ}}{\text{指示}}}$ があるまで、 ${\overset{\textnormal{ちゃくりく}}{\text{着陸}}}$ を ${\overset{\textnormal{みあ}}{\text{見合}}}$ わせています。 \hfill\break
 \textbf{Due to }dense fog, landing will be postponed until there are instructions from the control tower. }

\par{17. サービスが ${\overset{\textnormal{わる}}{\text{悪}}}$ かった \textbf{ため(に) }、お ${\overset{\textnormal{きゃく}}{\text{客}}}$ さんが ${\overset{\textnormal{へ}}{\text{減}}}$ った。 \hfill\break
Customers have diminished \textbf{since }service was bad. }

\begin{center}
 \textbf{~ます + ~ため(に) }
\end{center}

\par{ There are also instances in which you will find the cause-marking ~ため(に) after polite endings such as ~ます, ~ました,  and ~ません in attempt to be more formal. This is also almost entirely used in the written language. Some speakers frown on this practice, preferring the plain speech forms; however, it is nonetheless important to know that both ~ますため and ~ませんため are used a lot. }

\par{\textbf{Particle Note }: The particle に is seldom seen after ため in this situation. }

\par{18. ${\overset{\textnormal{げんじてん}}{\text{現時点}}}$ では、まだ ${\overset{\textnormal{かんち}}{\text{完治}}}$ に ${\overset{\textnormal{いた}}{\text{至}}}$ っており \textbf{ませんため }、 ${\overset{\textnormal{こんご}}{\text{今後}}}$ は ${\overset{\textnormal{つういん}}{\text{通院}}}$ して ${\overset{\textnormal{ちりょう}}{\text{治療}}}$ を ${\overset{\textnormal{つづ}}{\text{続}}}$ ける ${\overset{\textnormal{よてい}}{\text{予定}}}$ です。 \hfill\break
 \textbf{Because }I have yet to fully recover at this time, I plan on continuing my treatment through outpatient visits going forward. }

\par{19. ${\overset{\textnormal{へんぴん}}{\text{返品}}}$ ・ ${\overset{\textnormal{こうかん}}{\text{交換}}}$ は ${\overset{\textnormal{う}}{\text{受}}}$ け ${\overset{\textnormal{つ}}{\text{付}}}$ けておりませんため、 ${\overset{\textnormal{あらかじ}}{\text{予}}}$ めご ${\overset{\textnormal{りょうしょうくだ}}{\text{了承下}}}$ さい。 \hfill\break
We do not accept returns and\slash or trades, \textbf{so }please take this into your consideration. }

\par{20. ドリンク ${\overset{\textnormal{だい}}{\text{代}}}$ は ${\overset{\textnormal{ふく}}{\text{含}}}$ まれており \textbf{ませんため }、 ${\overset{\textnormal{げんば}}{\text{現場}}}$ でお ${\overset{\textnormal{しはら}}{\text{支払}}}$ いいただきます。 \hfill\break
Drink fees are not included, and \textbf{so }you will pay on site. }

\par{21. パスワードの ${\overset{\textnormal{さいとうろく}}{\text{再登録}}}$ が ${\overset{\textnormal{ひつよう}}{\text{必要}}}$ となり \textbf{ますため }${\overset{\textnormal{たいへん}}{\text{大変}}}$ お ${\overset{\textnormal{てすう}}{\text{手数}}}$ をお ${\overset{\textnormal{か}}{\text{掛}}}$ けし ${\overset{\textnormal{もう}}{\text{申}}}$ し ${\overset{\textnormal{わけ}}{\text{訳}}}$ ございませんが、 ${\overset{\textnormal{なにとぞ}}{\text{何卒}}}$ ご ${\overset{\textnormal{りかい}}{\text{理解}}}$ とご ${\overset{\textnormal{きょうりょく}}{\text{協力}}}$ をお ${\overset{\textnormal{ねが}}{\text{願}}}$ い ${\overset{\textnormal{いた}}{\text{致}}}$ します。 \hfill\break
 \textbf{As }password re-registration is required, we ask for your understanding and cooperation. We apologize for the inconvenience this may cause you. }

\par{22. ${\overset{\textnormal{なお}}{\text{尚}}}$ 、 ${\overset{\textnormal{けいさい}}{\text{掲載}}}$ までには ${\overset{\textnormal{きやくいはん}}{\text{規約違反}}}$ がないかの ${\overset{\textnormal{しんさ}}{\text{審査}}}$ をしており \textbf{ますため }、 ${\overset{\textnormal{すこ}}{\text{少}}}$ しお ${\overset{\textnormal{じかん}}{\text{時間}}}$ がかかることをご ${\overset{\textnormal{りょうしょう}}{\text{了承}}}$ ください。 \hfill\break
Also, \textbf{because }we are inspecting as to whether they are no TOS violations before publication, please understand that it will take some time. }

\par{23. ${\overset{\textnormal{さき}}{\text{先}}}$ ほど ${\overset{\textnormal{いただ}}{\text{頂}}}$ いた ${\overset{\textnormal{ないよう}}{\text{内容}}}$ で ${\overset{\textnormal{かくにんてん}}{\text{確認点}}}$ が ${\overset{\textnormal{ござ}}{\text{御座}}}$ いました ${\overset{\textnormal{ため}}{\text{為}}}$ 、ご ${\overset{\textnormal{れんらくいた}}{\text{連絡致}}}$ しました。 \hfill\break
I'm contacting \textbf{because }of there being something to confirm in the content we received from you earlier. }

\par{\textbf{Spelling Note }: Although ため hasn\textquotesingle t been spelled as 為 elsewhere in this lesson, Ex. 23 particularly exemplifies the style of writing seen in business communications where other phrases that would otherwise not be written in 漢字 are to give off an overall formal feel. }

\begin{center}
\textbf{Ambiguity Between Objective\slash Cause-Marking ~ため(に) }
\end{center}

\par{ There are some situations in which the agent of A and B are the same. Sometimes this causes ambiguity as to whether ~ために expresses an objective or a cause. }

\par{①: When A is a noun, if the noun is one that describes an unrealized thing such as 成功 (success)・健康 (health)・幸福 (happiness), then it expresses an objective. }

\par{24. ${\overset{\textnormal{ひび}}{\text{日々}}}$ の ${\overset{\textnormal{けんこう}}{\text{健康}}}$ \textbf{のためにも }、 ${\overset{\textnormal{まいにちじゅうぶん}}{\text{毎日十分}}}$ な ${\overset{\textnormal{すいみん}}{\text{睡眠}}}$ を ${\overset{\textnormal{と}}{\text{取}}}$ りましょう。 \hfill\break
Get plenty of rest daily \textbf{for }your daily healthy \textbf{as well }. \hfill\break
 \hfill\break
②: If the noun expresses something that has happened\slash is happening such as 事故 (accident)・失敗 (failure)・あられ (hail), then it expresses cause. }

\par{25. ${\overset{\textnormal{じこ}}{\text{事故}}}$ \textbf{のため }、 ${\overset{\textnormal{でんしゃ}}{\text{電車}}}$ が ${\overset{\textnormal{おく}}{\text{遅}}}$ れております。 \hfill\break
The train has been delayed \textbf{due to }an accident. }

\par{26. ${\overset{\textnormal{あめ}}{\text{雨}}}$ \textbf{のために }${\overset{\textnormal{しあい}}{\text{試合}}}$ を ${\overset{\textnormal{えんき}}{\text{延期}}}$ します。 \hfill\break
We will postpone the game \textbf{due to }the rain. }

\par{27. ただいま ${\overset{\textnormal{しんごうま}}{\text{信号待}}}$ ち \textbf{のため(に) }、 ${\overset{\textnormal{ていしゃ}}{\text{停車}}}$ しております。お ${\overset{\textnormal{いそ}}{\text{急}}}$ ぎのところご ${\overset{\textnormal{めいわく}}{\text{迷惑}}}$ をおかけします。 \hfill\break
We\textquotesingle ve now stopped \textbf{to }wait for a traffic light. We apologize for this inconvenience as you\textquotesingle re in a hurry. }

\par{\textbf{Sentence Note }: Although the use of “to” as the English translation may suggest that Ex. 27 demonstrates an objective, it is not the case that the vehicle stopped purposely to wait for a light. Although such a scenario is theoretically possible, it is also the case in English that “to” refers to cause. Changing the translation from “to” to “in order to,” for this reason, would be wrong. }

\par{③: For nouns—or even verbal expressions—in the middle that are neither positive nor negative things, only context can tell which interpretation is meant. Both interpretations can also be meant, which is also true for the English equivalent “for” and “to.” }

\par{28. ${\overset{\textnormal{さいけんさ}}{\text{再検査}}}$ をする \textbf{ため(に) }、 ${\overset{\textnormal{せんしゅう}}{\text{先週}}}$ 、 ${\overset{\textnormal{ついたち}}{\text{一日}}}$ 、 ${\overset{\textnormal{かいしゃ}}{\text{会社}}}$ を ${\overset{\textnormal{やす}}{\text{休}}}$ みました。 \hfill\break
I took a day off last week \textbf{to }get re-examined\slash  \textbf{for }a re-examination. }

\par{29. ${\overset{\textnormal{はは}}{\text{母}}}$ を ${\overset{\textnormal{むか}}{\text{迎}}}$ える \textbf{ために }、 ${\overset{\textnormal{は}}{\text{張}}}$ り ${\overset{\textnormal{き}}{\text{切}}}$ って ${\overset{\textnormal{そうじ}}{\text{掃除}}}$ をした。 \hfill\break
I revved up and cleaned \textbf{to }welcome my mother. }

\par{30. それとも ${\overset{\textnormal{りえき}}{\text{利益}}}$ \textbf{のために }そうなったのかはわからない。 \hfill\break
And it\textquotesingle s not clear if it happened so \textbf{because of\slash for }profit. }

\par{31. ${\overset{\textnormal{ことし}}{\text{今年}}}$ の ${\overset{\textnormal{じゅう}}{\text{10}}}$ ${\overset{\textnormal{がつ}}{\text{月}}}$ の ${\overset{\textnormal{たんじょうび}}{\text{誕生日}}}$ で、 ${\overset{\textnormal{はは}}{\text{母}}}$ が ${\overset{\textnormal{かんれき}}{\text{還暦}}}$ を ${\overset{\textnormal{むか}}{\text{迎}}}$ える \textbf{ため }、 ${\overset{\textnormal{かぞく}}{\text{家族}}}$ みんなで ${\overset{\textnormal{さん}}{\text{3}}}$ ${\overset{\textnormal{ぱく}}{\text{泊}}}$ のお ${\overset{\textnormal{とま}}{\text{泊}}}$ り ${\overset{\textnormal{りょこう}}{\text{旅行}}}$ に ${\overset{\textnormal{い}}{\text{行}}}$ ってきました。 \hfill\break
Our whole family went and came back from going on a three-night trip \textbf{for }my mother turning sixty on her birthday this year in October. }
    