    
\chapter{Honorifics IV}

\begin{center}
\begin{Large}
第127課: Honorifics IV: Regular Verbs 
\end{Large}
\end{center}
 
\par{  Most Japanese think they're all equal. Nevertheless, the language still places heavy emphasis on choosing the correct register to address others. 敬語 makes oneself look more educated and better when used appropriately. Always \textbf{consider the listener's position versus that of your own }. }

\par{ Everyone has trouble with 敬語 including Japanese people. Many honorific phrases are old-fashioned, but that shouldn't downplay the importance of 敬語. Interesting phenomenons today include "Xになります" instead of "Xでございます". The first is traditionally incorrect. However, it is so common now that it is here to stay. }

\par{ So, what kind of expressions are in 敬語? As you should know by now, 敬語 is usually split up into three kinds: 尊敬語 (respectful speech), 謙譲語・謙遜語 (humble speech), and 丁寧語 (polite speech). However, there are other phrases based off words and phrases found in 敬語 that serve the opposite purpose of being rude. This can be very complicated, but we will look at some examples as to how that can be the case. }

\begin{ltabulary}{|P|P|P|P|P|}
\hline 

Type & Object of Respect & Type of respect? & Subject of Action & 例文 \\ \cline{1-5}

尊敬語 & 相手 & Yes & 相手 & 部長が \textbf{お }話し \textbf{下さい }。 \\ \cline{1-5}

謙譲語 & 自分 & No & 自分 & 私が \textbf{お }話し \textbf{します }。 \\ \cline{1-5}

丁寧語 & 聞き手 & Yes & X & こちらにペンが \textbf{ございます }。 \\ \cline{1-5}

尊大語 & 自分 & Yes* & 自分 & 俺 \textbf{様 }が言ってやろう! \\ \cline{1-5}

軽蔑語 & 相手 & No & 相手 & 早くし \textbf{やがれ }! \\ \cline{1-5}

\end{ltabulary}

\par{*: To yourself, which is why it's boastful. }
 
\par{\textbf{Terminology Note }: \hfill\break
1. 丁寧語 used intransitively is specifically called 丁重語. \hfill\break
2. 尊敬語 is \emph{used to give\slash pay respect to one's out-group or to superiors in one's in-group }. \hfill\break
3. 謙譲語 is \emph{used to lower oneself or one's in-group }. \hfill\break
4. 尊大語 takes items meant for showing respect to others to direct respect to oneself in a boastful manner. \hfill\break
5. ${\overset{\textnormal{けいべつご}}{\text{軽蔑語}}}$ does not necessarily share vocabulary with the above kinds of speech, but it is the opposite of 尊敬語. The purpose of the speech style is to look down upon or contempt. It may also be referred to as ${\overset{\textnormal{ひば}}{\text{卑罵}}}$ ${\overset{\textnormal{ご}}{\text{語}}}$ or ${\overset{\textnormal{ばりご}}{\text{罵詈語}}}$ .  }

\par{You must consider diction changes that we've studied up to this point. For example, instead of あの人, use あの方. Attitude is also very important. Your words can be elegantly chosen, but if your posture and attitude are coarse, you do nothing but show a crude side of yourself. }

\par{ Although there is a lot of information, this lesson only scratches the surface. The best way to learn honorifics is seeing and using it in real life. What you learn here, though, will give you the skills necessary to facilitate it. We will now study the intricacy of verbs in honorifics. }
      
\section{Verbs Regular in Honorifics}
 
\par{ Both respectful and humble forms }

\begin{ltabulary}{|P|P|}
\hline 

 & Respectful Levels of Politeness \\ \cline{1-2}

R1 & お(連用形)です \\ \cline{1-2}

R2 & (連用形)なさいます \\ \cline{1-2}

R3 & お(連用形)になります \\ \cline{1-2}

R4 & お(連用形)下さいます (implies favor) \\ \cline{1-2}

\end{ltabulary}
 Respectful Levels of Politeness R1 お(連用形)です R2 (連用形)なさいます R3 お(連用形)になります R4 お(連用形)下さいます (implies favor) 
\par{have four general levels of politeness. }

\begin{ltabulary}{|P|P|}
\hline 

 & Humble Levels of Politeness \\ \cline{1-2}

H1 & お(連用形)します \\ \cline{1-2}

H2 & お(連用形)いたします 
\\ \cline{1-2}

H3 & お(連用形)もうしあげます \\ \cline{1-2}

H4 & お(連用形)いただきます (implies favor) \\ \cline{1-2}

\end{ltabulary}
 Humble Levels of Politeness H1 お(連用形)します H2 
\par{お(連用形)いたします }
H3 お(連用形)もうしあげます H4 お(連用形)いただきます (implies favor) 
\par{\textbf{Patterns' Note }: These charts are not exhaustive. Others including the following. }

\begin{ltabulary}{|P|P|P|}
\hline 

Pattern & 尊敬語・謙譲語? & Usage \\ \cline{1-3}

お・ご\dothyp{}\dothyp{}\dothyp{}の & 尊敬語 & Refined feminine speech \\ \cline{1-3}

~なさる & 尊敬語 & Attaches to the 連用形; very honorific but normally seen in the command form ~なさい because people \hfill\break
tend to incorrectly mix it with R3. \\ \cline{1-3}

~れる・られる & 尊敬語 & This is light 敬語. These are the same endings \hfill\break
as for the passive. \\ \cline{1-3}

お\dothyp{}\dothyp{}\dothyp{}になられる & 誤用 & This is a misuse. \\ \cline{1-3}

お…なさる & 誤用 & Mostly deemed as a misuse. \\ \cline{1-3}

お・ご\dothyp{}\dothyp{}\dothyp{}(を)たまわる & 謙譲語 & Humble pattern for receiving. \\ \cline{1-3}

お・ご\dothyp{}\dothyp{}\dothyp{}(を)差し上げる & 謙譲語 & Humble pattern for giving. \\ \cline{1-3}

\end{ltabulary}

\par{\textbf{Prefix Note }: ご~ is used instead of お~ in Sino-Japanese expressions. }

\par{\textbf{Exception Note }: 為さる・なさる is the respectful form of する and 致す・いたす is the humble form of する. }

\par{\textbf{Usage Note }: Think of the situations where the example sentences could be practically heard used. Who is speaking? To whom is the speaker addressing? This is all important in deciding which level of politeness is expected of you. Should a waiter have to use R3\slash H3, or because the guests' statuses are unknown, should he\slash she use R1\slash H1? }

\par{\textbf{Base Note }: Again, the 連用形 of verbs is either just る taken off for 一段 verbs or the base that ends in い for 五段 verbs. }

\begin{ltabulary}{|P|P|P|}
\hline 

 & 届ける (一段) = To deliver \hfill\break
& 呼ぶ (五段) = To Call \\ \cline{1-3}

R1 & お届けです & お呼びです \\ \cline{1-3}

R2 & 届けなさいます & 呼びなさいます \\ \cline{1-3}

R3 & お届けになります & お呼びになります \\ \cline{1-3}

H1 & お届けします & お呼びします \\ \cline{1-3}

H2 & お届けいたします & お呼びいたします \\ \cline{1-3}

H3 & お届け申し上げます & お呼び申し上げます \\ \cline{1-3}

\end{ltabulary}
 
\par{\textbf{Speaker Variant Note }: Not all speakers feel お…なさる is wrong. Historically, it has been correct and survives in phrases like お帰りなさい and お休みなさい. Examples using this pattern show that it is still actually used. Oddly enough, ご…なさる is fine for whenever ご is OK. }

\begin{center}
\textbf{Examples }
\end{center}

\par{1. 何をお探しですか。(尊敬語) \hfill\break
What are you looking for? }

\par{2. お忘れですか。(尊敬語) \hfill\break
Have you forgotten? }

\par{3. お持ち帰りですか。(尊敬語) \hfill\break
Is this to-go? }

\par{4. 田中博士とお話ししましたか。(尊敬語) \hfill\break
Have you talked with Professor Tanaka? }

\par{5. このところよくお会いしますね。(謙譲語) \hfill\break
Our paths seem to cross often lately, no? }

\par{6. お出かけですか。(尊敬語) \hfill\break
Are you going out? }

\par{7. 謹んでお詫び致します。(謙譲語) \hfill\break
I humbly apologize. }

\par{8. 心からお詫び申し上げます。(謙譲語) \hfill\break
Please accept my heartfelt apology. }

\par{9. 後ほど連絡をいたします。(謙譲語) \hfill\break
I will contact you later. }

\par{10. 弊社をご案内申し上げます。(謙譲語) \hfill\break
I shall give you a tour of the company. }

\par{11. 私はあなたをお守りします。(謙譲語) \hfill\break
I will protect you. }

\par{12. 速達でお願いします。(謙譲語) \hfill\break
By special delivery, please. }

\par{13. よろしくお願いします。(謙譲語) \hfill\break
Please treat me well. }

\par{14. 岸田部長はご出張なさるそうです。(尊敬語) \hfill\break
I hear that Chief Kishida is going to go on a business trip. }

\par{15. 夕べ課長からこちらの本をお借りしたよ。(謙譲語) \hfill\break
I borrowed this book from the section manager last night. }

\par{\textbf{Usage Note }: When speaking to friends\slash colleagues, it is possible to use respectful speech in reference to a superior while the overall speech level is polite\slash plain, but not doing so isn't necessarily considered rude. You may also humble yourself when referring to a superior even when the overall conversation is in polite\slash plain speech. }

\begin{center}
 \textbf{R4 VS H4 }
\end{center}

\begin{ltabulary}{|P|P|}
\hline 

R4 お\dothyp{}\dothyp{}\dothyp{}下さる & H4 お\dothyp{}\dothyp{}\dothyp{}いただく \\ \cline{1-2}

お届け下さいます & お届けいただきます \\ \cline{1-2}

お呼び下さいます & お呼びいただきます \\ \cline{1-2}

\end{ltabulary}

\par{ R4 is expressed from the \textbf{favor-giver }side and H4 is expressed from the \textbf{favor-receiving }side. The person doing the favor in R4 is marked by が or は but is marked by に (by) or から (from) in H4. In H4, this person is from whom something is received. When に is already used in the same sentence, から is often used rather than に. Remember that に can only mark whom one is receiving, not what . After all, honorifics is improper outside the realm of people. For R4, に marks who the object\slash favor is being directed . In either case, if there is a direct object を, is used. }

\par{16. 私 \textbf{は }頭取\{ \textbf{に・から }\}ご招待いただきました。(謙譲語) \hfill\break
I was invited by the bank president. }
 
\par{17. 伊藤さんが(私達に)お話し下さいました。(尊敬語) \hfill\break
Mr. Ito told us a story. }

\begin{center}
 \textbf{Double 敬語 Alert }
\end{center}

\par{ Remember that ~ (ら)れる can be used to make sentences more polite than simple polite sentences but less polite than true honorific sentences. Using it with honorific patterns is called "Double 敬語" and is not correct. Despite this, you might hear it. }

\par{18. テニスを\{なさられますか △・なさいますか 〇・されますか 〇\}。 \hfill\break
Will you play\slash do you play tennis? }

\par{\textbf{Exception Note }: する\textquotesingle s respectful form is なさる. }

\par{\textbf{Warning Note }: There are other entire phrases to avoid in 敬語. Asking if your superior can do something is forbidden! Asking directly whether your superior wants to do something or is thinking something is also bad. Consider the following. }

\par{19a. 部長は、韓国語はお分かりになりましたか。X  (すごく失礼な言い方) \hfill\break
19b. 部長は、韓国語をお話になりますか。 〇   (礼儀正しい言い方) \hfill\break
Chief, can you understand Korean?  \textrightarrow  Chief, do you speak Korean? }

\par{20. お茶をお飲みになりたいですか。 X \textrightarrow  お茶をお飲みになりますか。 \hfill\break
Do you want to drink tea? \textrightarrow   Will you drink tea? }

\begin{center}
 \textbf{お\dothyp{}\dothyp{}\dothyp{}だ? }
\end{center}

\par{ You can also see in \textbf{less respectful situations }です replaced by だ. This is reminiscent of the use of honorifics in a mix of polite\slash plain speech that was a hallmark of wife speech. Nowadays, examples like the first are simply set expressions but those like the second are still very much signature of more refined feminine speech. Even towards colleagues, many women still use honorifics in otherwise plain speech to show their own refinement. }

\par{21. あなた、夏休みはどちらへいらしたの。(女性語) \hfill\break
Where did you go for the summer? }

\par{22. 今朝はずいぶん早く(に)お出ましだね。(女性語・女房語) \hfill\break
My, you've come quite early this morning. }

\par{23. (だれだれさんが)お帰りだよ。(決まり文句) \hfill\break
Such and such is going home! }

\par{\textbf{Sentence Note }: This last sentence sounds like it would be said by people who run establishments with close customer relations. Places like 居酒屋 or scenes from plays come to mind. }

\par{24. お帰り! (Casual) \hfill\break
Welcome home. }

\par{\textbf{Word Notes }: }

\par{1. いらす is a contracted form of the respectful いらっしゃる, which is the respectful form of 行く in this case. }

\par{2. お出まし is an honorific word for 出席. Note that the honorific word itself is not feminine. }

\begin{center}
 \textbf{One's Boss: Humble Speech? }
\end{center}

\par{  It is important that you just don't run with the title of this section. To some, including Japanese people, it is very hard to not always speak of your boss respectfully, but when you're \textbf{addressing someone in one's out-group }, you refer to people in your in-group, including your boss, with humble speech. This includes 呼び捨て, the dropping of titles. Again, this is in reference to your \textbf{in-group }. }

\par{Situation A: 岩間さん is an employee at A社. The person on the other line when he answers the phone is 高田さん of B社. 高田さん acts about whether A社長, 岩間さん's boss is available or not. The following response is what 岩間さん should say if his boss is currently out and unavailable. \hfill\break
\hfill\break
25. 申し訳ございませんが、Aさんはただいま外出しております。 \hfill\break
I'm terrible sorry, but A is currently out. }

\par{ Using respectful speech is a common mistake by newbies in the Japanese workplace. At least it's not as bad as not using 敬語 entirely. }

\par{Question: Is there any situation where you would refer to your boss to an out-group person with respectful speech? }

\par{Answer: Yes. Say you were in a huge company and rarely or never have a chance to work with your boss, it becomes even more natural and understandable to use respectful speech as your boss is no doubt very 偉い to regular workers like yourself. However, the person answering the phone for a message like above is probably in constant contact with the boss, so this would not work. }
      
\section{ある}
 
\par{ ある is a little tricky in 敬語. If you have any older textbooks that are older than the 1960s, you will notice a lot of old forms listed. If you happen to read slightly old-fashioned literature, you'll also see them. This section will show you these forms for completeness, but you must understand what is used today. In the chart below, forms not listed as 古風 are still used. As for 尊敬語, the most common and important form is ございます. おありです may be used sometimes in making questions, but it's starting to become old-fashioned as well. }

\begin{ltabulary}{|P|P|P|P|}
\hline 

丁寧語 &  謙譲語 (古風) &  尊敬語 & 尊敬語 (古風) \\ \cline{1-4}

 ございます \hfill\break
&  おありでございます & ございます \hfill\break
おありです &  おありでいらっしゃいます \hfill\break
おありになります \hfill\break
\\ \cline{1-4}

\end{ltabulary}

\par{26a. マッチがおありですか。(古風) \hfill\break
26b. マッチをお持ちですか。 (丁寧語) \hfill\break
Do you have a match? }

\par{27. 申し訳ございません。(丁寧語) \hfill\break
I'm terribly sorry. \hfill\break
Literally: I have no excuse. \hfill\break
\hfill\break
28a. お釣はおありになりますか。(古風) \hfill\break
28b. お釣りはございますか。(現代の言い方) \hfill\break
Do you have some change? }

\par{29a. 拙宅にはお炬燵がございません。(やや古風) \hfill\break
29b. うちには(お)こたがございません。(現代の女性語) \hfill\break
29c. うちにはこたつがございません。(現代の言い方) \hfill\break
I don't have a kotatsu in my house. }

\par{\textbf{Variant Note }: お炬燵 has become out of use for many speakers. Opinions differ on its correctness, and what should be used instead depends partly on gender and partly on norms for everyone, which is represented in the notes above. }

\par{30a. お姉様がおありでいらっしゃいますか。(やや古風; もはや使われていない) \hfill\break
30b. お姉様はいらっしゃいますか。 \hfill\break
Do you have an elder sister? }

\par{31. ご質問はございませんか。 \hfill\break
Do you have any questions? }
      
\section{~させていただきます}
 
\par{Scene I: You are asked by your boss to go out to a very fine dinner. You have to decline, but you don't want to be rude. You would say something like: }

\par{32. せっかくでございますが、ご遠慮させていただきます。 \hfill\break
It would be an honor, but I'm afraid I have to decline. }

\par{~させていただきます shows a feeling of \textbf{humble diffidence }to convey one's arbitrary action or intention while being \emph{considerate of the speaker and his or her status }. As such, you should use this pattern rather than other humble patterns with certain verbs when you should recognize authority . ~ていただく implies receiving a favor. Its question forms are ~させていただけませんか and ~させていただけないでしょうか. When you want to ask someone to do something like this, you don't use the causative. }

\par{33. }

\par{学生:今日はすこし頭が痛いので、早めに家へ帰らせていただけないでしょうか。 \hfill\break
先生:それはいけませんね。お大事に。 \hfill\break
Student: Since I have a headache today, may I go home early? \hfill\break
Teacher: That's no good. Take care. }

\par{34. }

\par{学生:先生、来週の金曜日は文部科学省の試験を受けに行かなければならないので、学校を休ませていただけないでしょうか。 \hfill\break
先生:ああ、いいですよ。頑張ってくださいね。 \hfill\break
Student: Teacher, I have to take a MEXT exam Friday of next week, so may I skip school (that day)? \hfill\break
Teacher: Ah, that's ok. Good luck. }

\par{\textbf{Abbreviation Note }: MEXT stands for the Ministry of Education, Culture, Sports, Science and Technology. }

\par{35. }

\par{学生:私にもコピーをさせていただけないでしょうか。 \hfill\break
先生:いいですよ。でも明日までに返してください。 \hfill\break
Student: May I make a copy of it? \hfill\break
Teacher: Yes, but you need to return it by tomorrow. }

\par{36. }

\par{Aさん: すみませんが、お写真を取らせていただけないでしょうか。 \hfill\break
Bさん:あ、いいですよ。 \hfill\break
A: Excuse me, but may I take your picture. \hfill\break
B: Ah, that's OK. }

\par{${\overset{\textnormal{}}{\text{37. 藤山先生}}}$ を ${\overset{\textnormal{}}{\text{紹介}}}$ していただけませんか。 \hfill\break
Could you introduce me to Fujiyama Sensei? }

\begin{center}
\textbf{When ~させていただけないでしょうか is not 100\% Appropriate }
\end{center}

\par{ When you use this expression, it makes the listener feel as if he or she has to say yes. Therefore, it is often the case that it is not appropriate to use this towards someone like a teacher or president. In such case, using ~ても構いませんか is the best choice. }

\par{38. }

\par{学生:申し訳ございませんが、宿題を忘れてきたので、明日提出しても構いませんか。 \hfill\break
先生:仕方がありませんねえ。じゃ、明日忘れないでくださいね。 \hfill\break
Student: I'm terribly sorry, but I forgot to bring my homework, so may I turn it in tomorrow? \hfill\break
Teacher: I guess there's no other way. Well, please don't forget it tomorrow. }

\par{39. }

\par{学生:すみません、これは日本語で言うのは難しいので、英語で言っても構わないですか。 \hfill\break
先生:いいですけど、何でしょう。 \hfill\break
Student: Sorry, I can't say this in Japanese, so may I say it in English? \hfill\break
Teacher: That's OK, but what is it? }

\par{Scene II: Your friend is speaking to you. Yet, you're irked at him. He asks if you want to go somewhere. However, you're not really interested in your friend and you say without emotion, "せっかくですが、遠慮させていただきます"。 You were extremely sarcastic and rude. This pattern is sadly effective in asking for divorce. }

\par{Scene III: As a social norm, you should always get permission before you do something, particularly in advance. However, there are those instances where you have to dig yourself out of a hole. In this case, you can use ~させていただきます. }

\par{40. いらっしゃらなかったので、代わりに転送させていただきました。 \hfill\break
Since you weren't here, I took the privilege of forwarding it. }

\par{You should have gotten permission in advance, but in this case couldn't for some reason. In such case, using the following phrase may cause the listener to reluctantly consent. }

\par{\textbf{さ入れ Note }: Some speakers add さ even when it is a 五段 verb to the expression because it sounds more polite despite the fact that the resultant expression is \textbf{technically incorrect }. }

\par{41. 急がさせていただきます。 (X) \hfill\break
Please let me hurry. }

\par{\textbf{Historical Note }: This phrase is clearly used to be more indirect about the speaker doing something. This, though, used to not be representative of Tokyo speech honorifics. In fact, this pattern comes from the 関西地方. Yes, this is 関西弁 that seeped into 標準語 which has since become standard. }
      
\section{Many More Examples of Honorifics}
   These sentences range from rather simple expressions, some of which you've already seen before, to far more complex sentences. Take note in how conjugations, particles, and other things work in honorifics.  
\par{42. 社長がお呼びです。 \hfill\break
The company president is calling you. }

\par{44. 田中さんとよくお会いになりますか。 \hfill\break
Do you often meet with Mr. Tanaka? }

\par{46. 遠慮なくお邪魔させていただきます。 \hfill\break
Allow me to come without hesitation. }

\par{48. 決めさせていただきます。 \hfill\break
I will respectfully decide. }
50. 昨日、私の英語の先生は推薦状を書いて下さいました。 \hfill\break
Yesterday, my English teacher wrote a letter of recommendation for me. 
\par{53. すぐにお送りします。 \hfill\break
I will send it right away. }

\par{54. 休業させていただきます。 \hfill\break
We are taking off for holiday. \hfill\break
\hfill\break
\textbf{誤用 Note }: This is a common misuse of honorifics. It should say 休業いたします or even just 本日休業 because it sounds as if they are asking permission from their customers. On the other hand, who can blame them for being extremely humble. }

\par{58. お車はどちらでございますか。 \hfill\break
Which is your car? }

\par{60. 少々お待ち下さい。 \hfill\break
Please wait one moment. }
 43. 心中お察しいたします。 \hfill\break
My sympathies are with you. 
\par{45. 私達は伊藤さんにお話いただきました。 \hfill\break
We were told a story by Mr. Ito. }

\par{47. 加藤さんはいらっしゃいますか。 \hfill\break
Is Mr. Kato here? }

\par{49. 彼はそちらの件をお断りなさいました。 \hfill\break
He rejected it. }

\par{51. お手数おかけしますが。 \hfill\break
I hate to be a burden. }

\par{52. 食堂は三階にございます。 \hfill\break
The cafeteria is on the third floor. }

\par{55a. いいえ、着なくてもよろしゅうございます。 \hfill\break
55b. いいえ、着る必要はございません。 \hfill\break
No, it's alright to not wear it. }

\par{56. 窓を閉めましてよろしゅうございますか? \hfill\break
Is it alright for me to shut the window? }

\par{57. 色々お世話になりました。 \hfill\break
Thank you for all the wonderful things that you have done. }

\par{59. この度は大変お世話になりありがとうございました。 \hfill\break
Thank you very much for all of your hard work. \hfill\break
\hfill\break
61. このたびは音楽会の入場券をわざわざお送り下さって誠に恐縮です。 \hfill\break
It was especially kind of you to have given me the concert ticket. }
\hfill\break

\begin{center}
\textbf{The Particle にて: The Original Form of で } 
\end{center}

\par{ で is actually the contraction of にて. にて has the same functions as で and is seen in formal situations. }

\par{62. ${\overset{\textnormal{}}{\text{神戸}}}$ にて ${\overset{\textnormal{かいさい}}{\text{開催}}}$ されます。 \hfill\break
It will be held in Kobe. }

\par{63. ${\overset{\textnormal{}}{\text{今日}}}$ の ${\overset{\textnormal{えんそく}}{\text{遠足}}}$ はこちらにて ${\overset{\textnormal{かいさん}}{\text{解散}}}$ します。 \hfill\break
Today's field trip will be dismissed here. }

\par{64. ${\overset{\textnormal{}}{\text{昨日}}}$ は、 ${\overset{\textnormal{}}{\text{風邪}}}$ にて ${\overset{\textnormal{けっせき}}{\text{欠席}}}$ しました。 \hfill\break
I was absent yesterday due to a cold. }
    