    
\chapter{~ために I}

\begin{center}
\begin{Large}
第137課: ~ために I: In order to\dothyp{}\dothyp{}\dothyp{} 
\end{Large}
\end{center}
 
\par{ The purpose-marker に attaches to a handful of nouns, creating several important grammar patterns. This is very much the case when it follows the noun 為, which means “objective\slash benefit” or “result.” Although most typically seen paired with the particle に, 為 can still be used like any other noun with these meanings as demonstrated below. }

\par{1. あなた \textbf{の }${\overset{\textnormal{ため}}{\text{為}}}$ を ${\overset{\textnormal{おも}}{\text{思}}}$ って ${\overset{\textnormal{い}}{\text{言}}}$ うのよ! \hfill\break
I\textquotesingle m saying this for your own \textbf{good }! }

\par{2. ${\overset{\textnormal{せかい}}{\text{世界}}}$ \textbf{の }${\overset{\textnormal{ため}}{\text{為}}}$ なら ${\overset{\textnormal{ぎせい}}{\text{犠牲}}}$ は ${\overset{\textnormal{しかた}}{\text{仕方}}}$ ない。 \hfill\break
If they\textquotesingle re for the sake of \textbf{ }the world, sacrifices can\textquotesingle t be helped. }

\par{3. ${\overset{\textnormal{こども}}{\text{子供}}}$ \textbf{の }${\overset{\textnormal{ため}}{\text{為}}}$ を ${\overset{\textnormal{おも}}{\text{思}}}$ って ${\overset{\textnormal{い}}{\text{言}}}$ っているつもりが、 ${\overset{\textnormal{ぎゃく}}{\text{逆}}}$ にストレスを ${\overset{\textnormal{あた}}{\text{与}}}$ えているかもしれません。 \hfill\break
Your intentions of saying that for \textbf{ }your child \textbf{\textquotesingle s benefit }may in fact be conversely stressing out your child. }

\par{4. ${\overset{\textnormal{ぶか}}{\text{部下}}}$ の ${\overset{\textnormal{ちんもく}}{\text{沈黙}}}$ は ${\overset{\textnormal{じょうし}}{\text{上司}}}$ \textbf{のために }ならないし、 ${\overset{\textnormal{そしき}}{\text{組織}}}$ にとってもその上司のキャリアにとってもプラスにはならない。 \hfill\break
Silence from subordinates is not \textbf{for the benefit of }the boss, nor will it be a plus for the organization or for that boss\textquotesingle  career. }

\par{5. ${\overset{\textnormal{なさ}}{\text{情}}}$ けは ${\overset{\textnormal{ひと}}{\text{人}}}$ の ${\overset{\textnormal{ため}}{\text{為}}}$ ならず。 \hfill\break
Compassion is not \textbf{for }other people \textbf{\textquotesingle s benefit }. }

\par{\textbf{Grammar Note }: ならず is the Classical Japanese equivalent of ではない and is commonly seen in proverbs such as Ex. 5.  }

\par{  When the purpose-marker に follows ため, it is used to express an \textbf{objective\slash goal to realize something by one\textquotesingle s utmost effort }. In doing so, it is often translated as “for” or “(in order) to” as we will soon see. Aside from showing purpose, in concurrence with its second definition, ため \textbf{may also be used to show cause }. This lesson will focus solely on the first meaning, and in the next lesson we'll focus on the second meaning. }

\par{\textbf{Orthography Note }: It\textquotesingle s important to note that 為 is typically written as ため , which is how it will be spelled for the remainder of this lesson. Nonetheless, 為 is still a common spelling. Often, it is the writer\textquotesingle s style or the medium that dictates which spelling is used. }
      
\section{Marking an Objective}
 
\par{ The primary purpose of ~ために is to express purpose. In doing so, ~ために \textbf{must be used with verbs of volition }. This does not mean that it is limited to transitive verbs. Rather, it is limited to verbs which have an agent who has control over achieving the stated goal. }

\begin{center}
\textbf{Conjugation Recap }
\end{center}

\par{ Although ~ために follows the same grammar as any other instance of a noun following a verb, because the Japanese expression itself is very different in regard to part of speech from its English counterpart “(in order) to,” we will look at how to conjugate ~ために with each kind of verb. }

\begin{ltabulary}{|P|P|P|P|}
\hline 

\slash eru\slash - \emph{Ichidan }Verb & 食べる + ために \textrightarrow  & 食べるために & (In order) to eat \\ \cline{1-4}

\slash iru\slash - \emph{Ichidan }Verb & 見る + ために \textrightarrow  & 見るために & (In order) to see \\ \cline{1-4}

\slash u\slash - \emph{Godan }Verb & 使う + ために \textrightarrow  & 使うために & (In order) to use \\ \cline{1-4}

\slash ku\slash - \emph{Godan }Verb & 行く + ために \textrightarrow  & 行くために & (In order) to go \\ \cline{1-4}

\slash gu\slash - \emph{Godan }Verb & 泳ぐ + ために \textrightarrow  & 泳ぐために & (In order) to swim \\ \cline{1-4}

\slash su\slash - \emph{Godan }Verb & 話す + ために \textrightarrow  & 話すために & (In order) to talk \\ \cline{1-4}

\slash tsu\slash - \emph{Godan }Verb & 勝つ + ために \textrightarrow  & 勝つために & (In order) to win \\ \cline{1-4}

\slash nu\slash - \emph{Godan }Verb & 死ぬ + ために \textrightarrow  & 死ぬために & (In order) to die \\ \cline{1-4}

\slash mu\slash - \emph{Godan }Verb & 読む + ために \textrightarrow  & 読むために & (In order) to read \\ \cline{1-4}

\slash ru\slash - \emph{Godan }Verb & 測る + ために \textrightarrow  & 測るために & (In order) to measure \\ \cline{1-4}

 \emph{Suru }(Verb) & する + ために \textrightarrow  & するために & (In order) to do \\ \cline{1-4}

 \emph{Kuru }& 来る + ために \textrightarrow  & 来るために & (In order) to come \\ \cline{1-4}

\end{ltabulary}

\par{\textbf{Conjugation Note }: This usage of ~ために is usually used with affirmative expressions, and it must be used with the non-past form of a verb. }
\textbf{}\textbf{The Four Scenarios of the Objective-Marking ~(の)ために } 
\par{ Regardless of how ~(の)ために expresses purpose, the agent of both clauses in the sentence are always the same. To visualize this, envision all instances of the purpose-ため as being “AためにB.” The someone doing A and the someone doing B must be the same person. In case you\textquotesingle ve forgotten what “agent” means, the “agent” of a sentence is the “someone doing X.” }

\par{\textbf{Particle Note }: Interestingly, the particle に is often dropped in literary settings from these two expressions when in declarative statements, but in interrogative sentences and\slash or comments directed at others, に is not dropped. Motivation for this is that in literary settings, grammatical connections such as に\textquotesingle s role of marking purpose that are easily deduced from context are dropped out of a necessity to be concise. Questions or statements directed toward others, though, present a need to be explicit, and so omission of latent parts of a sentence becomes out of place. }

\par{①: Noun A + の + ため: Thinking about A\textquotesingle s benefit }

\par{ This interpretation is used with nouns that either concern people or are\slash made up of people. This usage is usually translated as “for.” Even when the “A” element in the base sentence pattern “AためにB” is a noun, the particle の is conceptualized as being an abbreviation of some verbal expression. For instance, in Ex. 6, a verb that comes to mind that の before ため stands for is ${\overset{\textnormal{えんじょ}}{\text{援助}}}$ する (to aid). }

\par{6. ${\overset{\textnormal{まず}}{\text{貧}}}$ しい ${\overset{\textnormal{ひとびと}}{\text{人々}}}$ \textbf{のために }${\overset{\textnormal{なんひゃくまんえん}}{\text{何百万円}}}$ もの ${\overset{\textnormal{じぜんきふきん}}{\text{慈善寄付金}}}$ を ${\overset{\textnormal{あつ}}{\text{集}}}$ めました。 \hfill\break
We have collected millions of yen in charity contributions \textbf{for }the poor. }

\par{7. ${\overset{\textnormal{いま}}{\text{今}}}$ までで ${\overset{\textnormal{いちばんこくみん}}{\text{一番国民}}}$ \textbf{のための }${\overset{\textnormal{せいじ}}{\text{政治}}}$ を ${\overset{\textnormal{おこな}}{\text{行}}}$ ってくれた ${\overset{\textnormal{そうりだいじん}}{\text{総理大臣}}}$ は ${\overset{\textnormal{だれ}}{\text{誰}}}$ だと ${\overset{\textnormal{おも}}{\text{思}}}$ いますか? \hfill\break
Of the prime ministers up till now, who do you think has governed \textbf{for }the people the most? }

\par{8. ${\overset{\textnormal{しゃかいこうけんかつどう}}{\text{社会貢献活動}}}$ を ${\overset{\textnormal{とお}}{\text{通}}}$ して、 ${\overset{\textnormal{あんぜん}}{\text{安全}}}$ ・ ${\overset{\textnormal{かいてき}}{\text{快適}}}$ な ${\overset{\textnormal{しゃかい}}{\text{社会}}}$ \textbf{のために }、 ${\overset{\textnormal{さまざま}}{\text{様々}}}$ な ${\overset{\textnormal{かつどう}}{\text{活動}}}$ を ${\overset{\textnormal{おこな}}{\text{行}}}$ っています。 \hfill\break
Through social action programs, we are conducting various operations \textbf{for }a safe and pleasant society. }

\par{9. ${\overset{\textnormal{ことば}}{\text{言葉}}}$ の ${\overset{\textnormal{かべ}}{\text{壁}}}$ により、 ${\overset{\textnormal{たか}}{\text{高}}}$ い ${\overset{\textnormal{のうりょく}}{\text{能力}}}$ を ${\overset{\textnormal{い}}{\text{生}}}$ かせない ${\overset{\textnormal{がいこくじんせいと}}{\text{外国人生徒}}}$ \textbf{のため }、 ${\overset{\textnormal{と}}{\text{取}}}$ り ${\overset{\textnormal{だ}}{\text{出}}}$ し ${\overset{\textnormal{じゅぎょう}}{\text{授業}}}$ や ${\overset{\textnormal{がいこくじんせいと}}{\text{外国人生徒}}}$ サポーターなどによる ${\overset{\textnormal{きょういくしえんたいせい}}{\text{教育支援体制}}}$ を ${\overset{\textnormal{ととの}}{\text{整}}}$ えている。 \hfill\break
 \textbf{For }foreign students who cannot make the best of their high-level abilities due to language barrier, (they) have arranged an education support structure by means of separate classes and foreign-student supporters. }

\par{10. ${\overset{\textnormal{かいしゃ}}{\text{会社}}}$ \textbf{のために }${\overset{\textnormal{はたら}}{\text{働}}}$ いていることが、お ${\overset{\textnormal{きゃくさま}}{\text{客様}}}$ に ${\overset{\textnormal{よろこ}}{\text{喜}}}$ ばれ、それが ${\overset{\textnormal{じぶん}}{\text{自分}}}$ や ${\overset{\textnormal{かぞく}}{\text{家族}}}$ \textbf{のために }なるのであればそれでいいし、 ${\overset{\textnormal{じぶん}}{\text{自分}}}$ や ${\overset{\textnormal{かぞく}}{\text{家族}}}$ \textbf{のために }${\overset{\textnormal{はたら}}{\text{働}}}$ くことでお ${\overset{\textnormal{きゃくさま}}{\text{客様}}}$ から ${\overset{\textnormal{しじ}}{\text{支持}}}$ され、それが ${\overset{\textnormal{かいしゃ}}{\text{会社}}}$ にとっても ${\overset{\textnormal{ゆうえき}}{\text{有益}}}$ ならそれでいい。また、お ${\overset{\textnormal{きゃくさま}}{\text{客様}}}$ \textbf{のために }${\overset{\textnormal{はたら}}{\text{働}}}$ いていることが、 ${\overset{\textnormal{じぶん}}{\text{自分}}}$ の ${\overset{\textnormal{よろこ}}{\text{喜}}}$ びとなり、それが ${\overset{\textnormal{けっかてき}}{\text{結果的}}}$ に ${\overset{\textnormal{かいしゃ}}{\text{会社}}}$ の ${\overset{\textnormal{りえき}}{\text{利益}}}$ に ${\overset{\textnormal{つな}}{\text{繋}}}$ がり、 ${\overset{\textnormal{きゅうりょう}}{\text{給料}}}$ が ${\overset{\textnormal{あ}}{\text{上}}}$ がることで ${\overset{\textnormal{かぞく}}{\text{家族}}}$ の ${\overset{\textnormal{よろこ}}{\text{喜}}}$ びとなればそれでいい。 \hfill\break
Working \textbf{for }the company will delight one\textquotesingle s clientele, which is fine if it becomes beneficial \textbf{for }oneself and one\textquotesingle s family, and if one receives support from one\textquotesingle s clients by working \textbf{for }oneself and one\textquotesingle s family and that is also profitable to the company, then that\textquotesingle s fine. Also, if working \textbf{for }one\textquotesingle s clientele becomes one\textquotesingle s joy, which is then consequently tied to the company\textquotesingle s benefit, and leads to a raise that becomes one\textquotesingle s family\textquotesingle s delight, that too is fine. }

\par{②: Noun A + の + ため: Objective to realizing A }

\par{ In this usage, “Noun A” refers to some entity that can be viewed as a purpose\slash goal, and what follows is the objective for realizing that said purpose\slash goal. This usage can also have its parts reversed to have the objective stated first with the sentence ending in ~ためだ (Ex. 11). This usage is also typically translated as “for.” }

\par{11. なぜ ${\overset{\textnormal{かいしゃ}}{\text{会社}}}$ \textbf{のために }${\overset{\textnormal{しごと}}{\text{仕事}}}$ (を)するかというと、 ${\overset{\textnormal{じしん}}{\text{自身}}}$ の ${\overset{\textnormal{しゅうにゅう}}{\text{収入}}}$ \textbf{のため }です。 \hfill\break
As to why one works \textbf{for }one\textquotesingle s company, it\textquotesingle s \textbf{for }one\textquotesingle s own income. }

\par{\textbf{Grammar Note }: Though the first instance of ため is of Usage 1, the sentence-ending instance is of Usage 2. “Working for one\textquotesingle s computer” is the objective for realizing one\textquotesingle s source of income. }

\par{12. ${\overset{\textnormal{あんぜん}}{\text{安全}}}$ な ${\overset{\textnormal{しゃかい}}{\text{社会}}}$ の ${\overset{\textnormal{じつげん}}{\text{実現}}}$ \textbf{のためには }、すべての ${\overset{\textnormal{しみん}}{\text{市民}}}$ の ${\overset{\textnormal{りかい}}{\text{理解}}}$ と ${\overset{\textnormal{きょうりょく}}{\text{協力}}}$ が ${\overset{\textnormal{ひつよう}}{\text{必要}}}$ です。 \hfill\break
 \textbf{For }the realization of a safe society, understanding and cooperation of all townspeople is necessary. }

\par{13. ${\overset{\textnormal{じぶんじしん}}{\text{自分自身}}}$ の ${\overset{\textnormal{けんこう}}{\text{健康}}}$ と ${\overset{\textnormal{こうふく}}{\text{幸福}}}$ \textbf{のために }${\overset{\textnormal{てき}}{\text{敵}}}$ を ${\overset{\textnormal{ゆる}}{\text{赦}}}$ し、 ${\overset{\textnormal{わす}}{\text{忘}}}$ れましょう。 \hfill\break
 \textbf{For }one\textquotesingle s own health and happiness, forgive one\textquotesingle s enemies and forget. }

\par{14. ${\overset{\textnormal{なん}}{\text{何}}}$ \textbf{のために }${\overset{\textnormal{たたか}}{\text{闘}}}$ うのか。 ${\overset{\textnormal{けんりょく}}{\text{権力}}}$ \textbf{のため }か。 ${\overset{\textnormal{じゆう}}{\text{自由}}}$ \textbf{のため }か。 ${\overset{\textnormal{へいわ}}{\text{平和}}}$ \textbf{のため }か。 \hfill\break
What do you fight \textbf{for }? \textbf{For }influence? \textbf{For }freedom? \textbf{For }peace? }

\par{15. ${\overset{\textnormal{せきにん}}{\text{責任}}}$ ある ${\overset{\textnormal{かがくしゃ}}{\text{科学者}}}$ は、 ${\overset{\textnormal{かがく}}{\text{科学}}}$ の ${\overset{\textnormal{けんぜん}}{\text{健全}}}$ な ${\overset{\textnormal{はってん}}{\text{発展}}}$ \textbf{のために }、こうした ${\overset{\textnormal{じたい}}{\text{事態}}}$ に ${\overset{\textnormal{みずか}}{\text{自}}}$ ら ${\overset{\textnormal{てきせつ}}{\text{適切}}}$ に ${\overset{\textnormal{たいおう}}{\text{対応}}}$ していく ${\overset{\textnormal{ひつよう}}{\text{必要}}}$ がある。 \hfill\break
As for the scientists responsible, it is necessary that they themselves adequately handle these situations \textbf{for }the healthy development of science. }

\par{念のため、確認したいのですが、上記の認識で大丈夫でしょうか。 \hfill\break
I want to make sure just in case. Is my understanding above okay? }

\par{③: Verb + ための Noun B: Noun B which realizes stated goal }

\par{ Although this can also be seen as Noun A + ための Noun B, the point of Usage 3 is that when ~(の)ため is followed directly by another noun, that Noun B is what realizes the stated goal seen before ため. This usage may be translated as “for” or “to.” }

\par{16. ${\overset{\textnormal{じゅうみん}}{\text{住民}}}$ の ${\overset{\textnormal{ふくし}}{\text{福祉}}}$ を ${\overset{\textnormal{こうじょう}}{\text{向上}}}$ させる \textbf{ための }${\overset{\textnormal{せいさく}}{\text{政策}}}$ を ${\overset{\textnormal{じっし}}{\text{実施}}}$ する。 \hfill\break
To implement political measures \textbf{for }advancing the welfare of residents. }

\par{17. ${\overset{\textnormal{しあわ}}{\text{幸}}}$ せになる \textbf{ための }たったひとつの ${\overset{\textnormal{みち}}{\text{道}}}$ とは? \hfill\break
What is the one single path \textbf{to }becoming happy? }

\par{18. ${\overset{\textnormal{ばしょ}}{\text{場所}}}$ によっては、 ${\overset{\textnormal{きが}}{\text{着替}}}$ える \textbf{ための }${\overset{\textnormal{へや}}{\text{部屋}}}$ を ${\overset{\textnormal{ていきょう}}{\text{提供}}}$ してくれるところもある。 \hfill\break
Depending on the location, there are also places that will offer you a room \textbf{to }change. }

\par{19. ${\overset{\textnormal{ひとりぐ}}{\text{一人暮}}}$ らしをする \textbf{ための }${\overset{\textnormal{へや}}{\text{部屋}}}$ を ${\overset{\textnormal{か}}{\text{借}}}$ りるときには、 ${\overset{\textnormal{へや}}{\text{部屋}}}$ を ${\overset{\textnormal{か}}{\text{貸}}}$ してくれる ${\overset{\textnormal{おおや}}{\text{大家}}}$ さんと ${\overset{\textnormal{ちんたいけいやく}}{\text{賃貸契約}}}$ を ${\overset{\textnormal{むす}}{\text{結}}}$ ぶことになります。 \hfill\break
When renting a room \textbf{to }live alone, you will sign a lease agreement with the landlord who is renting you the room. }

\par{20. ${\overset{\textnormal{すべ}}{\text{全}}}$ ては、 ${\overset{\textnormal{かぞく}}{\text{家族}}}$ により ${\overset{\textnormal{よ}}{\text{良}}}$ い ${\overset{\textnormal{せいかつ}}{\text{生活}}}$ をもたらす \textbf{ため }だ。 \hfill\break
It is all [ \textbf{for }bringing\slash  \textbf{to }bring] about a better life for (their) families. }

\par{\textbf{漢字 Note }: The verb もたらす (to bring about) is only seldom spelled as 齎す. }

\par{④: Verb of Volition + ためには + Expression of Obligation }

\par{ ~ためには involves willful action\slash control by the agent\slash speaker in question. For verbs that have intransitive-transitive pairs, only the transitive form should be used with this pattern. This is further enforced by the sentences ending in an expression of obligation. }

\par{21. ${\overset{\textnormal{せいこう}}{\text{成功}}}$ する \textbf{ためには }、 ${\overset{\textnormal{めんどうぐさ}}{\text{面倒臭}}}$ いことをたくさんしなければなりませんよ。 \hfill\break
 \textbf{(In order) to }success, you must do lots of tiresome things. }

\par{22. ${\overset{\textnormal{ゆうりょう}}{\text{優良}}}$ な ${\overset{\textnormal{みこ}}{\text{見込}}}$ み ${\overset{\textnormal{きゃく}}{\text{客}}}$ を ${\overset{\textnormal{かくとく}}{\text{獲得}}}$ する \textbf{ためには }、これを ${\overset{\textnormal{みきわ}}{\text{見極}}}$ めなくてはなりません。 \hfill\break
 \textbf{(In order) to }acquire excellent prospective [customers\slash clients], you must get to the bottom of this. }

\par{23. ${\overset{\textnormal{おおがたせんぱく}}{\text{大型船舶}}}$ を ${\overset{\textnormal{うんこう}}{\text{運航}}}$ する \textbf{ためには }${\overset{\textnormal{かいぎし}}{\text{海技士}}}$ の ${\overset{\textnormal{めんきょ}}{\text{免許}}}$ が ${\overset{\textnormal{ひつよう}}{\text{必要}}}$ です。 \hfill\break
 \textbf{(In order to) }operate a large seacraft, [a mariner license is necessary\slash you must be a licensed mariner]. }

\par{24. ${\overset{\textnormal{しんぞうびょう}}{\text{心臓病}}}$ を ${\overset{\textnormal{よぼう}}{\text{予防}}}$ する \textbf{ためには }、どうすればいいですか。 \hfill\break
 \textbf{(In order) to }prevent heart disease, what should one do? }

\par{25. ${\overset{\textnormal{えいご}}{\text{英語}}}$ を ${\overset{\textnormal{しゅうとく}}{\text{習得}}}$ する \textbf{ためには }、あなたの ${\overset{\textnormal{せいかつ}}{\text{生活}}}$ の ${\overset{\textnormal{なか}}{\text{中}}}$ で ${\overset{\textnormal{えいご}}{\text{英語}}}$ の ${\overset{\textnormal{ゆうせんじゅんい}}{\text{優先順位}}}$ を ${\overset{\textnormal{たか}}{\text{高}}}$ くしましょう。 \hfill\break
 \textbf{(In order) to }learn English, increase the priority of English in your daily life. }

\begin{center}
\textbf{~ないために }
\end{center}

\par{ As mentioned earlier, ~ないために is seldom used. This is because for it to be grammatical, 100\% or near 100\% confidence that the agent has control over the non-realization of “A” must be implied. Remember, the base sentence pattern for ~ために is “AためにB.” Just as how ため carries a very affirmative nuance of “A” being realized by a stated objective “B,” when paired with a negative sentence, this becomes a very affirmative statement that “A” won\textquotesingle t realize by means of the stated objective “B.” }

\par{26. スズメバチに ${\overset{\textnormal{おそ}}{\text{襲}}}$ われないためには、どのようなことに ${\overset{\textnormal{き}}{\text{気}}}$ をつけたらよいのでしょうか。 \hfill\break
What sort of things should one be careful of to not be attacked by wasps? }

\par{\textbf{Sentence Note }: This question asks about what actions—“B”—can be taken to give the agent control over the non-realization of “A”—being attacked by wasps. }

\par{27. ${\overset{\textnormal{ひがい}}{\text{被害}}}$ に ${\overset{\textnormal{あ}}{\text{遭}}}$ わないために、 ${\overset{\textnormal{さまざま}}{\text{様々}}}$ な ${\overset{\textnormal{さぎ}}{\text{詐欺}}}$ の ${\overset{\textnormal{てぐち}}{\text{手口}}}$ について ${\overset{\textnormal{し}}{\text{知}}}$ っておきましょう! \hfill\break
Know about the various scam tricks in order not to become victimized! }

\par{\textbf{Sentence Note }: The use of ~ために is appropriate so long the speaker implies that “B”—actively finding out and learning about the various scam tricks—brings about person control over not becoming a victim. }

\par{28. そうならない \textbf{ため }にも専門家以外の人に相談するのはやめましょう。 \hfill\break
Refrain from consulting with non-experts to make it \textbf{so }it also doesn\textquotesingle t happen that way. }

\par{\textbf{Grammar Note }: Although なる typically doesn\textquotesingle t imply volition, this sentence implies that “B” is the means by which the agent has control over “A” not realizing, thus making the sentence grammatical. }

\par{29. ${\overset{\textnormal{し}}{\text{死}}}$ なないために ${\overset{\textnormal{い}}{\text{生}}}$ きるのは ${\overset{\textnormal{ごめん}}{\text{御免}}}$ だ。 \hfill\break
I'll have nothing to do with living in order not to die. }

\par{30. ~パリ ${\overset{\textnormal{きょうていご}}{\text{協定後}}}$ の ${\overset{\textnormal{とうゆうし}}{\text{投融資}}}$ を ${\overset{\textnormal{あやま}}{\text{誤}}}$ らないために~ \hfill\break
—In Order not to Make Investment and Lending Mistakes Post-Paris Accords— }

\par{ \textbf{Sentence Note }: The use of ~ないために is most common in headlines such as this. }
    