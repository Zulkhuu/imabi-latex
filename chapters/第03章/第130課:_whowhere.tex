    
\chapter{Who \& Where}

\begin{center}
\begin{Large}
第130課: Who \& Where: 誰, どなた, どちら, どいつ, \& どこ 
\end{Large}
\end{center}
 
\par{ In this lesson, the words we will look at are 誰, どなた, どちら, and どいつ. After we learn about the words for "who," we will go over the two words for "where," どこ and どちら. }
      
\section{Who's Who?}
 
\begin{center}
\textbf{誰 }
\end{center}

\par{ The basic word for “who,” 誰, may be used in casual, plain, and polite speech. It is for all intended purposes, the generic word for asking “who” someone is. }

\par{1. それは ${\overset{\textnormal{だれ}}{\text{誰}}}$ がやったの? \hfill\break
Who did that? }

\par{2. これを ${\overset{\textnormal{だれ}}{\text{誰}}}$ に ${\overset{\textnormal{わた}}{\text{渡}}}$ せばいいの? \hfill\break
Who should I pass this to? }

\par{3. ${\overset{\textnormal{だれ}}{\text{誰}}}$ に ${\overset{\textnormal{き}}{\text{聞}}}$ けばいいでしょうか。 \hfill\break
Who should I ask? }

\par{4. これって ${\overset{\textnormal{だれ}}{\text{誰}}}$ のものですか。 \hfill\break
Whose is this? }

\par{5. ${\overset{\textnormal{きみ}}{\text{君}}}$ 、 ${\overset{\textnormal{だれ}}{\text{誰}}}$ ? \hfill\break
Who are you? }

\par{\textbf{Sentence Note }: This sentence would likely be spoken by a guy in an extremely informal, abrupt fashion, most likely to someone who he views to be inferior to himself. }

\par{6. ${\overset{\textnormal{せかい}}{\text{世界}}}$ で ${\overset{\textnormal{いちばんうつく}}{\text{一番美}}}$ しいのは ${\overset{\textnormal{だれ}}{\text{誰}}}$ だい。 \hfill\break
Who is the fairest of them all? }

\par{\textbf{Sentence Note }: The use of だい at the end of the sentence indicates a rather coarse, abrupt, masculine question. It is rather bombastic and isn\textquotesingle t used so much anymore, but you will hear it frequently used in anime, music, etc. }

\begin{center}
\textbf{どなた } 
\end{center}

\par{ どなた is the respectful version of 誰. Its purpose is to ask who someone is. However, it is normally not used at someone directly. For instance, you might ask your boss about who someone is by using どなた, but you wouldn\textquotesingle t walk up to someone and say どなたですか. However, if you were to ask someone who happened to just appear to you and are perplexed as to who that person is, you could respond with Ex.7 }

\par{7.(あなた、)どなた(ですか)? \hfill\break
Who are you? }

\par{8. お ${\overset{\textnormal{はな}}{\text{花}}}$ を ${\overset{\textnormal{くだ}}{\text{下}}}$ さったのはどなたですか。 \hfill\break
Who is it that gave the flowers? \hfill\break
 \hfill\break
9. ${\overset{\textnormal{つぎ}}{\text{次}}}$ の ${\overset{\textnormal{かた}}{\text{方}}}$ は、どなたですか。 \hfill\break
Who is it that\textquotesingle s next? }

\par{\textbf{Sentence Note }: In Ex. 9, the speaker is merely asking for whoever is next to point himself\slash herself out. It is almost undoubtedly the case that the speaker knows who the person next is, and it\textquotesingle s also likely that the case that the person next is physically near the speaker. }

\par{10. すみませんが、 ${\overset{\textnormal{せきにんしゃ}}{\text{責任者}}}$ の ${\overset{\textnormal{かた}}{\text{方}}}$ はどなたですか。 \hfill\break
Excuse me, but who is the person in charge? }

\par{11. どなたか、このお ${\overset{\textnormal{こ}}{\text{子}}}$ さんをご ${\overset{\textnormal{ぞんじ}}{\text{存知}}}$ の ${\overset{\textnormal{かた}}{\text{方}}}$ はいらっしゃいませんか。 \hfill\break
Anyone, is there someone who knows who this child is? }

\begin{center}
\textbf{どちら }
\end{center}

\par{ When used to mean “who,” どちら is almost always used asどちら様. This is used to ask not just who someone is, but also who they are affiliated with. This is because it comes from a \emph{kosoado }which literally means “where?” }

\par{ Say if you were to receive a phone call from someone you don\textquotesingle t know. The natural response in Japanese to this is どちら様ですか. Perhaps you\textquotesingle re at work and there is a visitor waiting to see your boss. As a secretary who is in a hurry to tell his\slash her boss who has come by, you may ask Ex. 12. }

\par{12. どちら ${\overset{\textnormal{さま}}{\text{様}}}$ でいらっしゃいますか。 \hfill\break
May I ask who you are? }

\par{ As you can see, this variation is more respectful. At times, though, it\textquotesingle s not always appropriate to ask “who are you?” to a client or someone you should be giving the utmost respect\slash courtesy. In such a situation, you may wish to use something like in Ex. 13a or 13b. \hfill\break
 \hfill\break
13a. ${\overset{\textnormal{しつれい}}{\text{失礼}}}$ ですが、お ${\overset{\textnormal{なまえ}}{\text{名前}}}$ を ${\overset{\textnormal{ちょうだい}}{\text{頂戴}}}$ できますか。 \hfill\break
Excuse me, but could I have your name? \hfill\break
13b. ${\overset{\textnormal{しつれい}}{\text{失礼}}}$ ですが、お ${\overset{\textnormal{なまえ}}{\text{名前}}}$ を ${\overset{\textnormal{うかが}}{\text{伺}}}$ えますでしょうか。 \hfill\break
Excuse me, but could I ask your name? }

\par{ どちら様, nevertheless, plays an important role in “who” questions in respectful speech. This is because of how it is nuanced towards asking for affiliation and not just one\textquotesingle s personal name. }

\par{14. あちらのお ${\overset{\textnormal{がた}}{\text{方}}}$ はどちら ${\overset{\textnormal{さま}}{\text{様}}}$ でしょうか。 \hfill\break
Who is that person over there (from)? }

\par{15. ${\overset{\textnormal{おそ}}{\text{恐}}}$ れ ${\overset{\textnormal{い}}{\text{入}}}$ りますが、 ${\overset{\textnormal{ほんじつ}}{\text{本日}}}$ のご ${\overset{\textnormal{らいてん}}{\text{来店}}}$ はどちら ${\overset{\textnormal{さま}}{\text{様}}}$ のご ${\overset{\textnormal{しょうかい}}{\text{紹介}}}$ ですか。 \hfill\break
Pardon me, but who was it that introduced you to come to our store today? }

\par{16. どちら様までご ${\overset{\textnormal{れんらくざ}}{\text{連絡差}}}$ し ${\overset{\textnormal{あ}}{\text{上}}}$ げればよろしいでしょうか。 \hfill\break
Who should I get in contact with? }

\par{17.「すみません。 ${\overset{\textnormal{でんわちゅう}}{\text{電話中}}}$ ですので ${\overset{\textnormal{お}}{\text{折}}}$ り ${\overset{\textnormal{かえ}}{\text{返}}}$ します。」「あ、わかった」「どちら ${\overset{\textnormal{さま}}{\text{様}}}$ まで? ${\overset{\textnormal{ねん}}{\text{念}}}$ のために ${\overset{\textnormal{でんわばんごう}}{\text{電話番号}}}$ をお ${\overset{\textnormal{ねが}}{\text{願}}}$ いします。」 \hfill\break
“I apologize, but I will need to call you back because I\textquotesingle m on the phone.” “Oh, ok.” “To whom am I speaking? Just in case, may I have your phone number?” }

\par{18. 「 ${\overset{\textnormal{おそ}}{\text{恐}}}$ れ ${\overset{\textnormal{い}}{\text{入}}}$ りますが、お ${\overset{\textnormal{なまえ}}{\text{名前}}}$ とご ${\overset{\textnormal{れんらくさき}}{\text{連絡先}}}$ をお ${\overset{\textnormal{き}}{\text{聞}}}$ きしてもよろしいでしょうか。」 \hfill\break
“Pardon me, but may I ask your name and contact information?” }

\begin{center}
\textbf{どなた様 }
\end{center}

\par{ どなた様ですか is also a thing. As you can see from the examples below, it\textquotesingle s used as the respectful form of 誰 in very honorific yet indirect circumstances. }

\par{19. どなた ${\overset{\textnormal{さま}}{\text{様}}}$ か ${\overset{\textnormal{ぞん}}{\text{存}}}$ じませんが、 ${\overset{\textnormal{さきほど}}{\text{先程}}}$ ご ${\overset{\textnormal{しんせつ}}{\text{親切}}}$ に ${\overset{\textnormal{みちあんない}}{\text{道案内}}}$ をしていただいた ${\overset{\textnormal{かた}}{\text{方}}}$ にお ${\overset{\textnormal{れい}}{\text{礼}}}$ を ${\overset{\textnormal{もう}}{\text{申}}}$ し ${\overset{\textnormal{あ}}{\text{上}}}$ げたいのですが。 \hfill\break
I don\textquotesingle t know who it was, but I would like to show my appreciation to the person I had kindly show me the way a moment ago. }

\par{20. どなた様にお送りいたしますか。 \hfill\break
Who should I send it to? }

\par{21. どなた様をお訪ねしたらよろしいですか。 \hfill\break
Who should I visit? }

\par{22. どなた様でもお申し込みいただけます。 \hfill\break
Anyone can apply. }

\par{23. どなた様もご利用OK! \hfill\break
It\textquotesingle s OK for anyone to use! }

\par{\textbf{Sentence Note }: Ex. 23 is indicative of an advertisement. }

\begin{center}
\textbf{In Humble Speech } 
\end{center}

\par{ When it\textquotesingle s necessary that you humbly ask who in one\textquotesingle s company someone wishes to speak to in a business situation, you may useどちらの者. }

\par{24. どちらの ${\overset{\textnormal{もの}}{\text{者}}}$ にご ${\overset{\textnormal{よう}}{\text{用}}}$ でしょうか。 \hfill\break
Who do you have business with? }

\par{ In reality, you might wish to ask the following question, but there are certainly situations in which asking whom one has business with is appropriate. In Ex. 24, you also implicitly ask about the department of the company the person you are speaking to has business with. }

\par{25. どのようなご ${\overset{\textnormal{ようけん}}{\text{用件}}}$ でしょうか。 \hfill\break
What sort of business do you have (with us)? }

\par{ When asking someone who you need to call, you could just use 誰. どの者 may also be used, but some speakers find this to be stiff. Because this is a humble setting, there is no grammatical need to use anything more than 誰. }

\par{26.\{ ${\overset{\textnormal{だれ}}{\text{誰}}}$ ・どの ${\overset{\textnormal{もの}}{\text{者}}}$ \}をお ${\overset{\textnormal{よ}}{\text{呼}}}$ び ${\overset{\textnormal{いた}}{\text{致}}}$ しましょうか。 \hfill\break
Who shall I call for you? }

\begin{center}
\textbf{Rude Language: どいつ } 
\end{center}

\par{ Although outdated outside of certain set expressions, another word for "who" is どいつ. If you think this sounds like ドイツ (Germany), you'd be right. Japanese people also love making puns with it because of this! }
 
\par{27. ${\overset{\textnormal{うわさ}}{\text{噂}}}$ の ${\overset{\textnormal{おんな}}{\text{女}}}$ はどいつだ。 \hfill\break
Who's the rumored woman? }
 
\par{28. どいつでもいいんじゃない! \hfill\break
Just anyone's not alright! }

\par{29. どいつもこいつもドイツ ${\overset{\textnormal{じん}}{\text{人}}}$ だ! \hfill\break
Every last one of them is a German! }

\par{\textbf{Grammar Note }: This word creates a \emph{kosoado }series, as is indicated by こいつ. }
      
\section{Where}
 
\par{ The basic word for "where" is どこ. Aside from just meaning "where," it can also mean "what part?" or ask about to "what extent" something has gone. You can see how these nuances relate to each other in the examples below. }

\par{30. トイレはどこですか。 \hfill\break
Where is the bathroom? }
 
\par{31. ${\overset{\textnormal{みかみ}}{\text{三上}}}$ さんはどこですか。 \hfill\break
Where is Mr. Mikami? }
 
\par{32. どこに ${\overset{\textnormal{す}}{\text{住}}}$ んでるの? \hfill\break
Where do you live? }
 
\par{33. どこが ${\overset{\textnormal{わる}}{\text{悪}}}$ いの? \hfill\break
What (part) is wrong about (that\slash it)? }
 
\par{34. どこも ${\overset{\textnormal{いた}}{\text{痛}}}$ くありません。 \hfill\break
It doesn't hurt anywhere. }
 
\par{35. どこの ${\overset{\textnormal{しょくば}}{\text{職場}}}$ でも ${\overset{\textnormal{にんげんかんけい}}{\text{人間関係}}}$ がうまくいかない。 \hfill\break
No matter the work place, my relations with people don't go well. }
 
\par{36. どこまで ${\overset{\textnormal{すす}}{\text{進}}}$ んでいるか ${\overset{\textnormal{おし}}{\text{教}}}$ えてくれない? \hfill\break
Could you tell me how far you've progressed? }
 
\par{37. iPhoneのどこがいいですか。 \hfill\break
What part of the iPhone is good? }

\begin{center}
\textbf{どちら }
\end{center}

\par{ As mentioned, どちら literally refers to "where." This is seen in respectful language. }

\par{38. どちらへお ${\overset{\textnormal{とま}}{\text{泊}}}$ りでしょうか。 \hfill\break
Where will you be staying? }
 
\par{39. どちらへ ${\overset{\textnormal{い}}{\text{行}}}$ かれますか。 \hfill\break
Where are you going? }
 
\par{40. どちらへおかけですか。 \hfill\break
Where will you sit? }
 
\par{41. ご ${\overset{\textnormal{しゅっしん}}{\text{出身}}}$ はどちらですか。 \hfill\break
May I ask where you were born? }
 
\par{42. ご ${\overset{\textnormal{じゅうしょ}}{\text{住所}}}$ はどちらですか。 \hfill\break
May I ask for your address? }
 
\par{43. エスカレーターはどちらですか。 \hfill\break
Where is the escalator? }

\par{44. 制服ってどちらで買えるんですか。 \hfill\break
Where can you buy uniforms? }

\par{45. 「王さんのお ${\overset{\textnormal{くに}}{\text{国}}}$ はどちらですか。」「 ${\overset{\textnormal{たいわん}}{\text{台湾}}}$ です。」 \hfill\break
“Mr. Wang, what is your country of origin?” “Taiwan.” }
 
\par{\textbf{Reading Note }: 王 is a Chinese surname. When read the Japanese way, people will read and refer to someone with this surname as オウ. If using a Mandarin rendition, people will refer to that person as ワン. If using the Cantonese rendition, people will refer to that person as ウォン. }
    