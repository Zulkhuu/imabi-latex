    
\chapter{The Particle たり}

\begin{center}
\begin{Large}
第101課: The Particle たり 
\end{Large}
\end{center}
 
\par{ The particle たり shares the same changes in conjugation with ~た and ~て. }

\begin{ltabulary}{|P|P|P|}
\hline 

 & Positive & Negative \\ \cline{1-3}

Verbs & 見る \textrightarrow  見たり & 見ない \textrightarrow  見なかったり \\ \cline{1-3}

 & 変える \textrightarrow  変えたり & 変えない \textrightarrow  変えなかったり \\ \cline{1-3}

 & 行く \textrightarrow  行ったり & 行かない \textrightarrow  行かなかったり \\ \cline{1-3}

 & 書く \textrightarrow  書いたり & 書かない \textrightarrow  書かなかったり \\ \cline{1-3}

 & 泳ぐ \textrightarrow  泳いだり & 泳がない \textrightarrow  泳がなかったり \\ \cline{1-3}

 & 話す \textrightarrow  話したり & 話さない \textrightarrow  話さなかったり \\ \cline{1-3}

 & 待つ \textrightarrow  待ったり & 待たない \textrightarrow  待たなかったり \\ \cline{1-3}

 & 死ぬ \textrightarrow  死んだり & 死なない \textrightarrow  死ななかったり \\ \cline{1-3}

 & 呼ぶ \textrightarrow  呼んだり & 呼ばない \textrightarrow  呼ばなかったり \\ \cline{1-3}

 & 読む \textrightarrow  読んだり & 読まない \textrightarrow  読まなかったり \\ \cline{1-3}

 & 帰る \textrightarrow  帰ったり & 帰らない \textrightarrow  帰らなかったり \\ \cline{1-3}

Adjectives & 新しい \textrightarrow  新しかったり & 新しくない \textrightarrow  新しくなかったり \\ \cline{1-3}

 & きれいだ \textrightarrow  きれいだったり & きれいじゃない \textrightarrow  きれいじゃなかったり \\ \cline{1-3}

だ & だ \textrightarrow  だったり △ & \{では・じゃ\}ない \textrightarrow  \{では・じゃ\}なかったり △ \\ \cline{1-3}

\end{ltabulary}
 \textbf{Contraction Note }: Sometimes the the negative form may be simplified to just なかったり. So, you may see ~たり~なかったり.       
\section{The Conjunctive Particle たり}
 
\par{ ~たり is most known for being in the common pattern ~たり~たりする. The pattern usually always ends in する. Sometimes in speaking, the する may get cut off, but its existence is nevertheless implied. }

\par{ When used with verbs, the pattern shows two actions occurring \textbf{back and forth }. Or, with one or more verb or adjective, it shows examples of actions and or states. This pattern can also be used with the negative. }
 
\par{1. 彼女は立ったり座ったりした。 \hfill\break
She stood up and down. }

\par{2. 「今週は何をしますか」「友達にメールをしたり、本を読んだりします」  \hfill\break
"What will you do this week?" "I'll do things like mail my friends and read a book" }
 
\par{3. 月が雲の ${\overset{\textnormal{あいま}}{\text{合間}}}$ に見えたり、 ${\overset{\textnormal{かく}}{\text{隠}}}$ れたりした。 \hfill\break
The moon came in and out of sight in the clouds. }
 
\par{4. 休日は ${\overset{\textnormal{どくしょ}}{\text{読書}}}$ をしたり ${\overset{\textnormal{さんぽ}}{\text{散歩}}}$ をしたりして ${\overset{\textnormal{す}}{\text{過}}}$ ごす。 \hfill\break
As for holidays, I do thing such as read books or take walks to pass time. }
 
\par{5. 男が泣いたりして ${\overset{\textnormal{きも}}{\text{気持}}}$ ちが悪いなあ。 \hfill\break
It's a little creepy for a man to be crying. }

\par{6. 昼食の時間は早かったり遅かったりだ。  \hfill\break
Lunch times goes from being early to being late. }

\par{7. 近所のお店のコーヒーは日によって濃かったり薄かったりします。 \hfill\break
The neighborhood store's coffee goes from being strong to being thin. }
 
\par{8. デパートに行ったり、ゲームをしたりした。 \hfill\break
I did things like go to the department store and play games. }
 
\par{${\overset{\textnormal{そうじ}}{\text{9. 掃除}}}$ (を)したり ${\overset{\textnormal{せんたく}}{\text{洗濯}}}$ (を)したりしました。 \hfill\break
I was cleaning, doing laundry, etc. }
 
\par{10. 本を読んだりレポートを書いたりしていた。 \hfill\break
I was reading books, writing reports, etc. }

\par{11. 株を買おうか迷ってるけど、日によって高かったり安かったりでいつ買うべきかよく分からないなぁ。 \hfill\break
I'm wavering on whether to buy the stock, and because the price is expensive and cheap from day to day, I don't know when I should buy. }

\par{\textbf{Grammar Note }: 買おう is the volitional form of 買う. }
 
\par{12. 500円といっても、人によって高く感じたり安く感じたりする。 \hfill\break
Even though it may be 500 yen, it may feel expensive or cheap depending on the person. }
 
\par{13. ガソリンの値段が場所によって高かったり、安かったりするのは何故ですか?  \hfill\break
Why is that gas prices are high and low depending on the place? }

\par{14 . 映画を ${\overset{\textnormal{み}}{\text{観}}}$ たり、本を読んだり、 ${\overset{\textnormal{ひるね}}{\text{昼寝}}}$ したりする。 \hfill\break
I do things like watch movies, read books, take naps, among other things. }
 
\par{15. 一週間雨が ${\overset{\textnormal{ふ}}{\text{降}}}$ ったり ${\overset{\textnormal{や}}{\text{止}}}$ んだりしていた。 \hfill\break
The rain was intermittent throughout the week. }

\par{16.  日によって上手だったり下手だったりするのはプロとは言えない。 \hfill\break
Going from awesome to bad makes you not a pro. }
 
\par{17. 日によって絵があったりなかったりします。 \hfill\break
There are and aren't paintings depending on the day. }
 
\par{\textbf{Grammar Note }: As you can see, the negative can be used with たり. }
 
\par{\textbf{漢字 Note }: 絵\textquotesingle s reading え is actually an 音読み. }
 
\par{18. 高かったり安かったりで ${\overset{\textnormal{あんてい}}{\text{安定}}}$ しない ${\overset{\textnormal{かかく}}{\text{価格}}}$ \hfill\break
An unsteady price that goes back and forth from high to low }
 
\par{\textbf{Grammar Note }: The sentence above shows how the conjunctive particle たり can be used without ending in する. The copula is often used, and aside from this, when you use する, it can be used in other conjugations like the て形. }
 
\par{19. 暑かったり寒かったりしていた。 \hfill\break
It was going from hot to cold. }

\par{20. マンションの部屋は広かったり狭かったり\{だ・する\}。 \hfill\break
The condo rooms were wide and narrow. }

\par{\textbf{Grammar Note }: The use of だ injects more emotion whereas the use of する states things in a more objective fashion. }
 
\par{21. クマはおりの中で行ったり来たりしていた。 \hfill\break
The bear was walking back and forth in its cage. }

\par{22. 女だと思ってよく見たら、男だったりする。 \hfill\break
You think the person is a woman, but when you look, the person is a man. }

\par{23. 新品と思って買ったら、中古品だったりすることがよくある。 \hfill\break
It's often the case that one buys something thinking it's new and finding out it's used. }
 
\par{24. どれも似たり寄ったりだな。(Idiom) \hfill\break
Nothing really stands out. }

\par{25. ${\overset{\textnormal{ねが}}{\text{願}}}$ ったり ${\overset{\textnormal{かな}}{\text{叶}}}$ ったり \textbf{の }${\overset{\textnormal{}}{\text{好条件  (Set Phrase)}}}$   \hfill\break
Favorable conditions that work out as desired }

\par{\textbf{Grammar Note }: You sometimes see ~たり~たりの in set phrases like above. }

\par{26. 飲み食い \hfill\break
Eating and drinking }

\par{\textbf{Phrasing Note }: The above phrase is an example of nominal phrases from verbs having similar meaning to ~たり. If 飲み食いする were paraphrased, you would get either 食べたり飲んだりする or 飲んだり食べたりする. Note that "drinking and eating" is far less frequent than "eating and drinking" in English, but in Japanese no such ordering other than in 飲み食い can be attested. But, for other pairings of verbs, certain orders may be expected, and they may not all be the same as in English. So, pay attention. }
    