    
\chapter{Conditional Phrases}

\begin{center}
\begin{Large}
第110課: Conditional Phrases  
\end{Large}
\end{center}
 
\par{ Now that you have learned about the conditional particles, it's time for you to learn about all the different phrases you can make with them. Now, this lesson won't be totally exhaustive in its coverage on these phrases. So, you will still learn more in the future. }
      
\section{なら Phrases}
 
\begin{center}
 \textbf{AがAならBもBだ }
\end{center}

\par{  "AがAならBもBだ" compares two things that aren't good. Similarly, "AもAならBもBだ" gives a meaning of "both being not normal". }

\par{1. ${\overset{\textnormal{おっと}}{\text{夫}}}$ も ${\overset{\textnormal{きょうし}}{\text{教師}}}$ なら ${\overset{\textnormal{つま}}{\text{妻}}}$ も ${\overset{\textnormal{きょうし}}{\text{教師}}}$ だ = 夫も妻も ${\overset{\textnormal{とも}}{\text{共}}}$ に教師だ \hfill\break
The husband and the wife are both teachers. }
 
\par{2. 犬も犬なら猫も猫だ。 \hfill\break
If dogs are dogs, then cats are also cats. }
 
\par{3. 親も親なら、子も子だ。 \hfill\break
If parents are parents, then kids are also kids. }
      
\section{たら Phrases}
 
\begin{center}
\textbf{~としたら }
\end{center}

\par{ ~としたら is like "if it happens that". Notice that it can follow verbs in the non-past or past tense, but this decision has to be solely made on the circumstances. }

\par{4. もしたくさんの ${\overset{\textnormal{いさん}}{\text{遺産}}}$ が \textbf{あった }としたらどのように使うのですか? \hfill\break
Say if it so happened that you \textbf{had }a lot of inheritance, how would you use it? }

\par{5. 仮に、お金が一億円 \textbf{ある }としたら、何に使いますか。 \hfill\break
If you were to \textbf{have }100 million yen for instance, what would you use it for? }

\begin{center}
 \textbf{~たらいい? }\textbf{~たらどう? }
\end{center}

\par{ There are many ways to make suggestions, some of which are made with たら. These phrases may show noninterference, light commands, and suggestions--"how about?". In plain speech it's more common to drop か in a question in order to not sound curt\slash rude. }

\par{6. ${\overset{\textnormal{た}}{\text{他}}}$ の人と話したらどうですか。 \hfill\break
How about talking to other people? }

\par{7. もし来なかったらどうしよう? \hfill\break
What shall we do if he doesn't come? }

\par{8. どうやって ${\overset{\textnormal{つぐな}}{\text{償}}}$ いをしたらいいのでしょうか。 \hfill\break
How can I make it up to you? }

\par{9. 先生もお使いになってみたら \textbf{いかが }ですか。(Very polite\slash formal) \hfill\break
Sensei, how about you try using this? }

\par{10. どうしたらいいですか。 \hfill\break
What shall\slash should I do? }

\par{11. どうしたらいいと思いますか。 \hfill\break
What do you think I should do? }

\par{12. 何で行ったらどうですか。 \hfill\break
How do I get there? }

\par{\textbf{Grammar Note }: This 何で is read as なにで, and it refers to the means by which you are going. As this is also expressed by the word "how" do not be confused. }

\par{13. どこでチケットを買ったらいいですか。 \hfill\break
Where should I buy a ticket? }

\par{ In casual speech, the final part after the conditional is often dropped. When this is done, the sentence ends with a rising intonation. }

\par{14. 先生に聞いてみたら? \hfill\break
How about you ask the teacher? }

\par{~てはどう and its variants can also be used to mean "how about (you)?". }

\par{15. この本を読んでみてはどうですか。 \hfill\break
How about you read this book. }

\par{16. この薬を飲んでみてはいかがでしょう。 \hfill\break
Why don't you take this medicine? }

\begin{center}
 \textbf{Conditional\dothyp{}\dothyp{}\dothyp{} }\textbf{だけ }
\end{center}

\par{With a conditional, だけ can express "the more, the\dothyp{}\dothyp{}\dothyp{}". Some patterns include したら…しただけ, …しただけ, …したらそれだけ, and ~ば…だけ". At this point, just recognize what role だけ plays. }

\par{17. ${\overset{\textnormal{きぼう}}{\text{希望}}}$ した ${\overset{\textnormal{かいしゃ}}{\text{会社}}}$ に ${\overset{\textnormal{しゅうしょく}}{\text{就職}}}$ したら就職したぶん(だけ) ${\overset{\textnormal{きぐろう}}{\text{気苦労}}}$ も ${\overset{\textnormal{ふ}}{\text{増}}}$ えるでしょう。 \hfill\break
Just by getting a job at your dream company, anxiety will also surely increase. }

\par{\textbf{Word Note }: 希望する and ${\overset{\textnormal{のぞ}}{\text{望}}}$ む both mean "to hope", but they're not used like the English word. They are mainly literary, but when spoken they show sincere desire, beseeching, etc. 希望する is more like "to desire". If wanted to say, "I hope it's cold tomorrow", you'd say something like ${\overset{\textnormal{あすさむ}}{\text{明日寒}}}$ いといいね. In short, these words are too serious to be used in more practical settings. }

\par{18. ( ${\overset{\textnormal{じかん}}{\text{時間}}}$ が) ${\overset{\textnormal{じゅっぷん}}{\text{十分}}}$ (だけ)あれば、 ${\overset{\textnormal{しあ}}{\text{仕上}}}$ げられるけど、もう ${\overset{\textnormal{れいじ}}{\text{零時}}}$ になったから ${\overset{\textnormal{とこ}}{\text{床}}}$ に ${\overset{\textnormal{つ}}{\text{就}}}$ いた方がいいです。 \hfill\break
If I had ten minutes, I would be able to finish it. But, because it's already midnight, it's best to go to bed. }

\begin{center}
 \textbf{でもしたら・でもしようものなら }
\end{center}

\par{でもしたら or でもしようものなら--"(even) if you were". This pattern attaches to the 連用形 of a verb. }

\par{19. ${\overset{\textnormal{おく}}{\text{遅}}}$ れでもしたら ${\overset{\textnormal{たいへん}}{\text{大変}}}$ ですよ。 \hfill\break
It would be awful even if you were just late. }

\par{20. でも ${\overset{\textnormal{つ}}{\text{突}}}$ っ ${\overset{\textnormal{き}}{\text{切}}}$ る時に ${\overset{\textnormal{まん}}{\text{万}}}$ が ${\overset{\textnormal{いち}}{\text{一}}}$ 子供を ${\overset{\textnormal{ひ}}{\text{轢}}}$ きでもしたら大変である。 \hfill\break
But, if by chance when I break across, it would be awful if I were to run over a kid. \hfill\break
From 雨天炎天―ギリシャ・トルコ辺境紀行 by 村上春樹. }

\begin{center}
 \textbf{Conditional\dothyp{}\dothyp{}\dothyp{} }\textbf{(た)で }
\end{center}

\par{There are three conditional patterns in which で acts as a conjunctive particle. Unlike the case particle, this classification means that it is to follow something that acts as a clause. This is completely different than just following a noun phrase. }

\par{ These phrases below that utilize this odd で show an effect on something due to a conditional. So, there is some causality implied. So, if something happens, there is a certain effect. This use of で can surface as any of the following patterns. The table is followed by example sentences. }

\begin{ltabulary}{|P|P|P|}
\hline 

With なら & 名詞・形容動詞+なら+形容動詞+で & 広大なら広大で \\ \cline{1-3}

With たら & 連用形+たら+た+で & 引っ越したら引っ越したで \hfill\break
安かったら安かったで \\ \cline{1-3}

With ば & 動詞: 已然形+ば+連用形+た+で \hfill\break
形容詞: 已然形+ば+終止形+で \hfill\break
形容動詞: 已然形+ば+語根+で \hfill\break
名詞: であれば+ 名詞 + で & あればあったで \hfill\break
広ければ広いで \hfill\break
難解であれば難解で \hfill\break
暇であれば暇で \\ \cline{1-3}

\end{ltabulary}

\par{ Say we note this pattern as A + Conditional +A + B. While you adequately assess the establishment of A, you lead into a situation B that is not foreseen. }

\par{21. 休んだら休んだで、たくさんやることがある。 \hfill\break
From having taken a break, I have a lot of stuff to do. }

\par{22. 一段落したらしたで、次の難問が待っている。 \hfill\break
Just from following down one step, the next difficult problem awaits. }

\par{23. お金というものはあったらあったで無駄に使ってしまうものだ。 \hfill\break
Money is something that when you have it, you end up wasting it. }

\par{24. 部屋が広ければ広いで掃除機なんか結構時間がかかります。 \hfill\break
The bigger the room gets, the more time it takes to do stuff like vacuuming. }

\par{ In all four of these examples, because of the event transpiring before で, there is an effect from it that leads to the result stated after で. This pattern tends to get used a lot more when the past tense precedes で. }

\par{ Because this pattern doesn't get used extremely frequently, judgments may vary on exact instances of it. This is because it should be used within the explanation provided above. If you are told that you are using it incorrectly, you should use ほど instead of で. This does not mean で \textrightarrow  ほど in most cases. Nor does this mean that these different patterns are the same by any means.  It's just that if you use this pattern incorrectly, it more than likely should be ほど. }
${\overset{\textnormal{いちだんらく}}{\text{一段落}}}$ したらしたで、 ${\overset{\textnormal{}}{\text{次}}}$ の ${\overset{\textnormal{なんもん}}{\text{難問}}}$ が ${\overset{\textnormal{}}{\text{待}}}$ っている。 \hfill\break
Just from following down one step, the next difficult problem awaits. \hfill\break
      
\section{ば Phrases}
 
\begin{center}
\textbf{~ばいい } 
\end{center}

\par{ ~ばいい is like "it would be good". In the past tense, ~ばよかった, the perspective changes to first person to show a feeling of regret or criticism at something not happening. In question form it is essentially the same thing as ~たらいいですか。 }
 
\par{25. あの家を買えばいいんです。 \hfill\break
It would be good for you to buy that house. }
 
\par{26. 私がすればよかったんですが。 \hfill\break
It would have been good if I had done it. }
 
\par{27. 時間があればよかった。 \hfill\break
It would have been nice if there were time. }
 
\par{28. もっと早く来ればよかった。 \hfill\break
It would have been nice if you had come earlier. }
 
\begin{center}
\textbf{~\{から・と・に\}すれば }
\end{center}
 
\par{ ~からすれば, ~とすれば, and ~にすれば show a particular position. There isn't really any significant differences between them, although judging from the translations below, you can get a sense of slight nuance differences based on the particle used. }
 
\par{29. 彼にすれば ${\overset{\textnormal{あくい}}{\text{悪意}}}$ から出た ${\overset{\textnormal{こうい}}{\text{行為}}}$ だったのでしょう。 \hfill\break
Considering it's him, the action was probably out of malice thought. }
 
\par{30. 彼女の顔つきからすれば、本当に美しいですよね。 \hfill\break
Starting with her looks, she's really pretty, isn't she! }
 
\begin{center}
\textbf{A }\textbf{もすればB }\textbf{もする }
\end{center}
 
\par{ "AもすればBもする" aligns similar conditional situations. }
 
\par{31. 雨も降れば風も吹く。 \hfill\break
The rain falls, and the wind blows. }
 
\par{32. 男もいれば、女もいる。 \hfill\break
There are men, and there are also women. }
    