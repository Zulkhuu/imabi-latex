    
\chapter{Hearsay}

\begin{center}
\begin{Large}
第147課: Hearsay: ~そうだ \& ~らしい 
\end{Large}
\end{center}
 
\par{ Hearsay, 伝聞, is the topic of this lesson. This should be a lot easier than phrases for expressing similarity. }
      
\section{Verb\slash Adjective + ~そうだ}
 
\par{ Directly following a verbal\slash adjectival expression with そうだ expresses hearsay. So, with a noun, follow it with だそうだ. Tense is shown before ~そうだ because this usage has no tense: it's only used with て or in the final position. Before we get to any example sentences, we will first look at the conjugation chart below. }

\begin{ltabulary}{|P|P|P|P|P|P|P|P|}
\hline 

動詞 & す \textbf{る }そうだ & 形容詞 & 古 \textbf{い }そうだ & 形容動詞 & 簡単 \textbf{だ }そうだ & 名詞 & 猫 \textbf{だ }そうだ \\ \cline{1-8}

\end{ltabulary}

\par{1. ${\overset{\textnormal{はかせ}}{\text{博士}}}$ は ${\overset{\textnormal{}}{\text{入院}}}$ したそうだ。 \hfill\break
I hear that the professor was hospitalized. }

\par{\textbf{Reading Note }: The general reading of 博士 is はかせ, but はくし is respectful and is in words such as ${\overset{\textnormal{はくしごう}}{\text{博士号}}}$ . }

\par{${\overset{\textnormal{}}{\text{2. 先生}}}$ は ${\overset{\textnormal{}}{\text{病気}}}$ だそうです。 \hfill\break
I hear that the teacher is sick. }

\par{\textbf{Part of Speech Note }: Remember that 病気 is not an adjective in Japanese! }

\par{3. ${\overset{\textnormal{うわさ}}{\text{噂}}}$ によれば、 ${\overset{\textnormal{}}{\text{二人}}}$ は ${\overset{\textnormal{りこん}}{\text{離婚}}}$ するそうです。 \hfill\break
Rumor has it that the two are going to get a divorce. }

\par{4. ${\overset{\textnormal{らしょうもん}}{\text{羅生門}}}$ はいい ${\overset{\textnormal{}}{\text{映画}}}$ だそうです。 \hfill\break
They say that Rashomon is a good movie. }

\par{\textbf{Culture Note }: 羅生門 is a very popular crime scene movie in Japan from the 50s created by the renowned producer ${\overset{\textnormal{くろさわあきら}}{\text{黒澤明}}}$ . }

\par{${\overset{\textnormal{}}{\text{5. 地震}}}$ があったそうだ。 \hfill\break
They say there was an earthquake. }

\par{6. ${\overset{\textnormal{てんきよほう}}{\text{天気予報}}}$ では ${\overset{\textnormal{}}{\text{今日}}}$ は ${\overset{\textnormal{}}{\text{雨が}}}$ ${\overset{\textnormal{}}{\text{降}}}$ らないそうです。 \hfill\break
I hear it's not going to rain today with the weather report. }

\par{${\overset{\textnormal{}}{\text{7. 日本語}}}$ は ${\overset{\textnormal{}}{\text{上手}}}$ じゃないそうです。 \hfill\break
I hear that he's not good at Japanese. }

\par{${\overset{\textnormal{}}{\text{8. 彼}}}$ はよく ${\overset{\textnormal{}}{\text{勉強}}}$ するそうです。 \hfill\break
It sounds like he often studies. }

\par{9. 「あの人は子供の欲しいものは何でも買ってやるそうですよ」「よっぽど甘やかしているんでしょう」 \hfill\break
"I hear that that person always buys whatever her kids want" "The kids must be pretty spoiled" }

\par{10.  テレビ局が来月からタガログ語の番組を放送するそうです。 \hfill\break
I hear that the TV network will air a Tagalog program starting next month. }
      
\section{~らしい}
 
\par{ The \textbf{auxiliary }~らしい shows speculation based on some sort of \textbf{foundation }\emph{ such as hearsay }. However, it is somewhat of a euphemism for a more direct, declarative statement concerning hearsay, which would be expressed by ~そうだ. }

\par{ Also, when there is a clearly defined reason or proof for something, you the speaker aren't just making conjecture. Rather, you're taking what you're hearing, seeing, or reading and making a non-intuitive statement. }

\begin{ltabulary}{|P|P|P|P|P|P|}
\hline 

形容詞 & 高い + らしい \textrightarrow   高いらしい & 形容動詞 & 危険な + らしい \textrightarrow  危険らしい & 名詞 & ガンらしい \\ \cline{1-6}

\end{ltabulary}

\begin{center}
 \textbf{Example Sentences }
\end{center}

\par{${\overset{\textnormal{}}{\text{11. 日本}}}$ へ ${\overset{\textnormal{}}{\text{来}}}$ なかったらしいです。 \hfill\break
He apparently didn't go to Japan. }
${\overset{\textnormal{}}{\text{}}}$ 
\par{12. それが ${\overset{\textnormal{げんいん}}{\text{原因}}}$ らしい。 \hfill\break
That is apparently the cause. }
13. この ${\overset{\textnormal{}}{\text{都市}}}$ の ${\overset{\textnormal{ふどうさん}}{\text{不動産}}}$ は ${\overset{\textnormal{}}{\text{高}}}$ いらしいです。 \hfill\break
The real estate properties in this city seem to be expensive. 
\par{14. その ${\overset{\textnormal{}}{\text{家}}}$ に ${\overset{\textnormal{ゆうれい}}{\text{幽霊}}}$ が ${\overset{\textnormal{}}{\text{出}}}$ るらしい。 \hfill\break
The house appears to be haunted. }

\par{${\overset{\textnormal{}}{\text{15. 彼女}}}$ はよく ${\overset{\textnormal{}}{\text{勉強}}}$ するらしいです。 \hfill\break
She appears to study often.  }

\par{ When the \textbf{suffix }~らしい attaches to nouns, it makes adjectives to show typicality. It may also follow the stem of adjectives and some adverbs to show that something brings on a certain emotion. As you can see, you must understand that the auxiliary verb and suffix ~らしい's are different. }

\par{${\overset{\textnormal{}}{\text{16. 彼は子供}}}$ らしくない。 \hfill\break
He doesn't act like a child. }

\par{17. もう ${\overset{\textnormal{}}{\text{少}}}$ し ${\overset{\textnormal{おとな}}{\text{大人}}}$ らしくしてください。 \hfill\break
Please a little more like an adult. }

\par{18. この傷は致命傷らしい。 \hfill\break
The wound seem to be mortal. }

\par{19. あいつもう ${\overset{\textnormal{}}{\text{元気}}}$ らしいじゃん?  (Casual) \hfill\break
Doesn't he seem already fine? }

\par{20. あんた、ホントに ${\overset{\textnormal{せんぱく}}{\text{浅薄}}}$ でいやらしい ${\overset{\textnormal{}}{\text{男}}}$ だわ。(Rough female speech) \hfill\break
You, you're a really shallow and disgusting man! }

\par{21. わざとらしい ${\overset{\textnormal{びしょう}}{\text{微笑}}}$ \hfill\break
An intentional\slash unnatural smile }

\par{22. そんな ${\overset{\textnormal{はつげん}}{\text{発言}}}$ はいかにも彼らしい。 \hfill\break
Speech like that is typical of him. }

\par{23. このごろあんたらしくねーよ。(乱暴な言い方) \hfill\break
You haven't been yourself lately. }
    