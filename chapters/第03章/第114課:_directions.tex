    
\chapter{Directions}

\begin{center}
\begin{Large}
第114課: Directions 
\end{Large}
\end{center}
 
\par{ Having a large vocabulary  is important. So, let's learn about places for directions! }
      
\section{Places}
  
\begin{ltabulary}{|P|P|P|P|P|P|}
\hline 

Place & 所 & ところ & Amusement park & 遊園地 & ゆうえんち \\ \cline{1-6}

Post office & 郵便局 & ゆうびんきょく & Bank & 銀行 & ぎんこう \\ \cline{1-6}

Company & 会社 & かいしゃ & Work & 仕事 & しごと \\ \cline{1-6}

Hospital & 病院 & びょういん & Graduate school & 大学院 & だいがくいん \\ \cline{1-6}

College & 大学 & だいがく & Department store & デパート & デパート \\ \cline{1-6}

School & 学校 & がっこう & Department store & 百貨店 & ひゃっかてん \\ \cline{1-6}

Theater & 映画館 & えいがかん & Convenience store & コンビニ & コンビに \\ \cline{1-6}

Coffee shop & 喫茶店 & きっさてん & Supermarket & スーパー & スーパー \\ \cline{1-6}

Park & 公園 & こうえん & Museum & 博物館 & はくぶつかん \\ \cline{1-6}

Home & 家 & いえ & Restaurant & レストラン & レストラン \\ \cline{1-6}

Police station & 交番 & こうばん & Train station & 駅 & えき \\ \cline{1-6}

Airport & 空港 & くうこう & Church & 教会 & きょうかい \\ \cline{1-6}

Mosque & モスク & モスク & Synagogue & シナゴーグ & シナゴーグ \\ \cline{1-6}

High rise & ビル & ビル & Skyscraper & 摩天楼 & まてんろう \\ \cline{1-6}

Zoo & 動物園 & どうぶつえん & Art museum & 美術館 & びじゅつかん \\ \cline{1-6}

Shrine & 神社 & じんじゃ & Chinese restaurant & 中国料理店 & ちゅうごくりょうりてん \\ \cline{1-6}

Temple & 寺 & てら & Japanese restaurant & 日本料理屋 & にほんりょうりや \hfill\break
\\ \cline{1-6}

Hotel & ホテル & ホテル & Inn & 旅館 & りょかん \\ \cline{1-6}

Family inn & 民宿 & みんしゅく & Pharmacy & 薬局 & やっきょく \\ \cline{1-6}

Public bath & 銭湯 & せんとう & Cemetery \hfill\break
& 墓地 & ぼち・はかち \hfill\break
\\ \cline{1-6}

\end{ltabulary}
 
\par{\textbf{Culture Note }: A 旅館 is a traditional Japanese inn. They are wooden structures where guests may sleep, bathe, and eat traditional food. The flooring is laid out with ${\overset{\textnormal{たたみ}}{\text{畳}}}$ . Bedding is ${\overset{\textnormal{ふとん}}{\text{布団}}}$ , and you wear ${\overset{\textnormal{ゆかた}}{\text{浴衣}}}$ to sleep.  }
      
\section{Shops}
 
\par{ Store in Japanese is ${\overset{\textnormal{みせ}}{\text{店}}}$ and shop is ${\overset{\textnormal{てんぽ}}{\text{店舗}}}$ . Store names often end in ~ ${\overset{\textnormal{や}}{\text{屋}}}$ or ~ ${\overset{\textnormal{てん}}{\text{店}}}$ . To address an employee of a store, address him or her as: }

\begin{ltabulary}{|P|P|P|}
\hline 

Kind of Store \hfill\break
& + & ~さん \hfill\break
\\ \cline{1-3}

\end{ltabulary}

\par{At times ~ ${\overset{\textnormal{いん}}{\text{員}}}$ is used to mean "employee". As it is more polite to address a person with his or her actual name, always look for a name tag if available. To say where one works, you use the phrase ~に ${\overset{\textnormal{つと}}{\text{勤}}}$ めています . }

\par{\textbf{Kinds of Stores }}

\begin{ltabulary}{|P|P|P|P|P|P|}
\hline 

Stationer & 文房具屋 & ぶんぼうぐや & Butcher shop & 肉屋 & にくや \\ \cline{1-6}

Cleaners & クリーニング屋 & クリーニングや & Photo shop & 写真屋 & しゃしんや \\ \cline{1-6}

Fish store & 魚屋 & さかなや & Grocery store & 八百屋 & やおや \\ \cline{1-6}

Fruit store & 果物屋 & くだものや & Bakery & パン屋 & パンや \\ \cline{1-6}

Cake shop & ケーキ屋 & ケーキや & Candy shop & お菓子屋 & おかしや \\ \cline{1-6}

Appliance store & 電気屋 & でんきや & Shoe store & 靴屋 & くつや \\ \cline{1-6}

Sushi shop \hfill\break
& 寿司屋 & すしや & Flower store \hfill\break
& 花屋 & はなや \\ \cline{1-6}

Drug store & 薬屋 & くすりや & Food stand & 出店 & でみせ \\ \cline{1-6}

Book store & 本屋 & ほんや & Book store & 書店 & しょてん \\ \cline{1-6}

Retail store & 小売店 & こうりてん & Liquor store & 酒屋 & さかや \\ \cline{1-6}

Supermarket & スーパー & スーパー & Toy store & 玩具屋 & おもちゃや \\ \cline{1-6}

Camera store & カメラ屋 & カメラや & Old book store & 古本屋 & ふるほんや \\ \cline{1-6}

Furniture store & 家具屋 & かぐや & Jewelry store & 宝石店 & ほうせきてん \\ \cline{1-6}

Clothing store & 衣料品屋 & いりょうひんや & Restaurant & 飲食店 & いんしょくてん \hfill\break
\\ \cline{1-6}

Convenience store & コンビニ &  &  &  &  \\ \cline{1-6}

\end{ltabulary}
      
\section{Location}
 
\par{ To describe the location of an item or place, you must use the following expressions in Japanese. Unlike in English, the location phrase is after the noun. }

\begin{ltabulary}{|P|P|P|P|P|P|}
\hline 

Xの前 & Xのまえ & In front of X & Xの上 & Xのうえ & On top of X \\ \cline{1-6}

Xの横 & Xのよこ & (To) the side of X & Xの中 & Xのなか & Inside X \\ \cline{1-6}

XとYの間 & XとYのあいだ & Between X and Y & Xの後ろ & Xのうしろ & Behind X \\ \cline{1-6}

Xの奥 & Xのおく & In the back of X & Xの側面 & Xのそくめん & At the side of X \\ \cline{1-6}

Xの下 & Xのした & Under X & Xの隣 & Xのとなり & Next to X \\ \cline{1-6}

Xの向かい & Xのむかい & Across from X & Xの周り & Xのまわり & Around X \\ \cline{1-6}

Xの正面 & Xのしょうめん & In front of X & Xの近く & Xのちかく & Near X \\ \cline{1-6}

Xの先 & Xのさき & Past X & Xの遠く & Xのとおく & Far from X \\ \cline{1-6}

Xの向こう & Xのむこう & Beyond X & Xの脇 & Xのわき & On the side of X \\ \cline{1-6}

Xの方 & Xのほう & In the direction of X & Xの近隣 & Xのきんりん & Adjacent to X \\ \cline{1-6}

Xの(直ぐ)側 & Xの(すぐ)そば & Right by X & Xの傍ら & Xのかたわら & Aside from  X \\ \cline{1-6}

Xの手前 & Xのてまえ & On this side of &  &  &  \\ \cline{1-6}

\end{ltabulary}

\par{\textbf{Word Note }: You can add ${\overset{\textnormal{なな}}{\text{斜}}}$ め at the front of a direction word to mean "diagonally". So, 斜め後ろ = diagonally behind. 向き means "facing". So, 南向き = facing south. }

\par{${\overset{\textnormal{}}{\text{1. 銀行}}}$ はどこですか。 \hfill\break
Where is the bank? }

\par{2. ${\overset{\textnormal{こうじょう}}{\text{工場}}}$ の ${\overset{\textnormal{}}{\text{隣}}}$ にあります。 \hfill\break
It's right next to the factory. }
 
\par{${\overset{\textnormal{}}{\text{3. 近}}}$ くに ${\overset{\textnormal{}}{\text{電気屋}}}$ はありますか。 \hfill\break
Is there a electronics store nearby? }

\par{4. ${\overset{\textnormal{じしょ}}{\text{辞書}}}$ は ${\overset{\textnormal{}}{\text{机}}}$ の ${\overset{\textnormal{}}{\text{上}}}$ にあります。 \hfill\break
The dictionary is on the desk. }
 
\par{${\overset{\textnormal{}}{\text{5. 交番}}}$ の ${\overset{\textnormal{}}{\text{前}}}$ に ${\overset{\textnormal{}}{\text{学校}}}$ があります。 \hfill\break
There is a school in front of the police station. }
 
\par{${\overset{\textnormal{}}{\text{6. 美術館}}}$ の ${\overset{\textnormal{}}{\text{横}}}$ に ${\overset{\textnormal{}}{\text{病院}}}$ があります。 \hfill\break
There is a hospital to the side of the art museum. }

\par{7. 川(の)向うの家 \hfill\break
House on the other side of a river }

\par{8. テーブルの ${\overset{\textnormal{}}{\text{下}}}$ に ${\overset{\textnormal{}}{\text{置}}}$ いてある。 \hfill\break
It's been placed under the desk. }
 
\par{${\overset{\textnormal{}}{\text{9. 変}}}$ な ${\overset{\textnormal{せん}}{\text{線}}}$ が ${\overset{\textnormal{}}{\text{学校}}}$ の ${\overset{\textnormal{}}{\text{周}}}$ りにあります。 \hfill\break
There is an odd line around the school. }

\par{10. あそこの ${\overset{\textnormal{ほんだな}}{\text{本棚}}}$ の下の ${\overset{\textnormal{だん}}{\text{段}}}$ の左の ${\overset{\textnormal{}}{\text{方}}}$ の小さい ${\overset{\textnormal{じびき}}{\text{字引}}}$ の ${\overset{\textnormal{}}{\text{隣}}}$ の本 \hfill\break
The book over there next to the small dictionary on the left side of the bottom shelf on the bookshelf }
 
\par{${\overset{\textnormal{}}{\text{11. 消防署}}}$ の ${\overset{\textnormal{}}{\text{向}}}$ かいに ${\overset{\textnormal{}}{\text{喫茶店}}}$ がある。 \hfill\break
There is a coffee shop across from the fire station. }

\par{12. ${\overset{\textnormal{みずうみ}}{\text{湖}}}$ のわきで ${\overset{\textnormal{さんぽ}}{\text{散歩}}}$ する。 \hfill\break
To take a walk on the side of the lake. }
 
\par{${\overset{\textnormal{}}{\text{13. 彼は扉}}}$ の ${\overset{\textnormal{}}{\text{向}}}$ こうへ ${\overset{\textnormal{}}{\text{走}}}$ り ${\overset{\textnormal{}}{\text{出}}}$ した。 \hfill\break
He began to race beyond the door. }
 
\par{14. この ${\overset{\textnormal{}}{\text{部屋}}}$ の ${\overset{\textnormal{}}{\text{前}}}$ には ${\overset{\textnormal{えんがわ}}{\text{縁側}}}$ があって、その ${\overset{\textnormal{}}{\text{前}}}$ は ${\overset{\textnormal{}}{\text{庭}}}$ です。 \hfill\break
There is a veranda in front of the room, and in front of it is a garden. }
 
\par{15. この ${\overset{\textnormal{さきつうこうど}}{\text{先通行止}}}$ め \hfill\break
Road Blocked! }
 
\par{${\overset{\textnormal{}}{\text{16. 病院}}}$ ならすぐこの ${\overset{\textnormal{}}{\text{先}}}$ です。 \hfill\break
If you're talking about the hospital, it's only a little way from here. }

\par{17. 駅の隣に大きなスーパーができました。 \hfill\break
A big supermarket has been made next to the train station. }

\par{\textbf{Word Note }: Remember that できる does not mean just "can" but can also mean "completed", "be made", "come into being", etc. }
      
\section{Asking for Directions}
 
\par{ Below are some more additional vocabulary words for the examples that follow. }

\begin{ltabulary}{|P|P|P|P|P|P|}
\hline 

Intersection & 交差点 & こうさてん & Traffic Light & 信号 & しんごう \\ \cline{1-6}

Bridge & 橋 & はし & End of the Street & 突き当り & つきあたり \\ \cline{1-6}

Meter & メートル & メートル & Train & 電車 & でんしゃ \\ \cline{1-6}

Subway & 地下鉄 & ちかてつ & Bus & バス & バス \\ \cline{1-6}

Bus Stop & バス停 & バスてい & Taxi & タクシー & タクシー \\ \cline{1-6}

Taxi Station & 乗り場 & のりば & Bullet train \hfill\break
& 新幹線 & しんかんせん \hfill\break
\\ \cline{1-6}

Street & 道 & みち & Road & 道路 & どうろ \\ \cline{1-6}

Highway & ハイウェイ & ハイウェイ & Expressway & 高速道路 & こうそくどうろ \\ \cline{1-6}

Highway & ハイウエー & ハイウエー & Main road & 幹線道路 & かんせんどうろ \\ \cline{1-6}

Corner & 角 & かど & Left & 左 & ひだり \\ \cline{1-6}

Left Side & 左側 & ひだりがわ & Right & 右 & みぎ \\ \cline{1-6}

Right Side & 右側 & みぎがわ & Left and Right & 左右 & さゆう \\ \cline{1-6}

North & 北 & きた & South & 南 & みなみ \\ \cline{1-6}

East & 東 & ひがし & West & 西 & にし \\ \cline{1-6}

Northeast & 北東 & ほくとう & Northwest & 北西 & ほくせい \\ \cline{1-6}

Southeast & 南東 & なんとう & Southwest & 南西 & なんせい \\ \cline{1-6}

North-north-east & 北北東 & ほくほくとう & North-north-west & 北北西 & ほくほくせい \\ \cline{1-6}

South-south-east & 南南東 & なんなんとう & South-south-west & 南南西 & なんなんせい \\ \cline{1-6}

South and North & 南北 & なんぼく & East and West & 東西 & とうざい \\ \cline{1-6}

All 4 directions & 東西南北 & とうざいなんぼく & Vicinity & 辺り & あたり \\ \cline{1-6}

Neighborhood & 辺 & へん & Direction & 方 & ほう \\ \cline{1-6}

To Turn & 曲がる & まがる & To cross & 渡る & わたる \\ \cline{1-6}

To Go & 行く & いく & To pass & 過ぎる & すぎる \\ \cline{1-6}

\end{ltabulary}

\par{\textbf{Word Notes }: }

\par{1. "By" in the sense of a mode of transportation is expressed by で. }

\par{2. 側 = side. This can be used to refer to direction, side of something, a position, and a third person. For the first three, it can be read as かわ and even be seen as っかわ, but these are no longer standard readings. }
 
\par{\textbf{Directions Note }: Directions like ${\overset{\textnormal{なんとう}}{\text{南東}}}$ are reversed if they represent location. For example, southeast Asia is ${\overset{\textnormal{とうなん}}{\text{東南}}}$ アジア. }
 
\par{When asking for directions, there are many different ways you can formulate your question. You may opt to say you're lost, use a conditional, or use a certain level of politeness. Below are some of the many viable ways you would do this in Japanese. }

\begin{ltabulary}{|P|P|}
\hline 

\dothyp{}\dothyp{}\dothyp{}に・へ行きたいですけど・が・けれど(も) & I want to go to\dothyp{}\dothyp{}\dothyp{}but \\ \cline{1-2}

どうやって…に・へ行ったらいいですか。 & How should I go if I am going to\dothyp{}\dothyp{}\dothyp{}? \\ \cline{1-2}

どうやって…に・へ行けばいいですか。 & How should I go if I am going to\dothyp{}\dothyp{}\dothyp{}? \\ \cline{1-2}

道に迷ってしまいました。 & I've gotten lost. \\ \cline{1-2}

…への行き方を教えてください(ませんか)。 & Could you please tell me how to get to\dothyp{}\dothyp{}\dothyp{}? \\ \cline{1-2}

…ここは、どの辺ですか。 & What neighborhood is this? \\ \cline{1-2}

…どうやって行くんですか & How do I get to\dothyp{}\dothyp{}\dothyp{}? \\ \cline{1-2}

\dothyp{}\dothyp{}\dothyp{}はどこですか。 & Where is\dothyp{}\dothyp{}\dothyp{}? \\ \cline{1-2}

\dothyp{}\dothyp{}\dothyp{}はどちらですか。 & Where is\dothyp{}\dothyp{}\dothyp{}? (Formal) \\ \cline{1-2}

\dothyp{}\dothyp{}\dothyp{}がどこか教えてください(ませんか)。 & (If you would) please tell me where\dothyp{}\dothyp{}\dothyp{}is. \\ \cline{1-2}

\end{ltabulary}

\par{\textbf{Word Note }: You can use 何で instead of どうやって, but it isn't as common. }

\par{As politeness is often heightened when asking for directions, です is often changed to でしょう, its volitional form. People will typically respond to you by using the と conditional or ~てください. }
 
\par{18. どうやって ${\overset{\textnormal{}}{\text{行}}}$ ったらいいでしょうか。 \hfill\break
How should I go there? }
 
\par{19. 2つ ${\overset{\textnormal{}}{\text{目}}}$ の ${\overset{\textnormal{}}{\text{交差点}}}$ を ${\overset{\textnormal{わた}}{\text{渡}}}$ って、40メートルぐらい ${\overset{\textnormal{}}{\text{行}}}$ ってください。 \hfill\break
Cross the second intersection, and go about forty meters. }
 
\par{20. 40メートルぐらい ${\overset{\textnormal{}}{\text{行}}}$ くと、 ${\overset{\textnormal{}}{\text{左側}}}$ に ${\overset{\textnormal{}}{\text{美術館}}}$ があります。 \hfill\break
If you go about forty meters, an art museum will be on your left. }
 
\par{21. その ${\overset{\textnormal{}}{\text{美術館}}}$ の ${\overset{\textnormal{}}{\text{角}}}$ を ${\overset{\textnormal{}}{\text{左}}}$ に ${\overset{\textnormal{}}{\text{曲}}}$ がって、しばらくまっすぐ ${\overset{\textnormal{}}{\text{行}}}$ ってください。 \hfill\break
Turn left at the corner of the art museum, and go straight for a while. }
 
\par{22. しばらくまっすぐ ${\overset{\textnormal{}}{\text{行}}}$ くと、 ${\overset{\textnormal{}}{\text{郵便局}}}$ が ${\overset{\textnormal{}}{\text{見}}}$ えますよ。 \hfill\break
If you go straight for a while, you will see a post office. }
 
\par{23. すみません、 ${\overset{\textnormal{}}{\text{私}}}$ もこの ${\overset{\textnormal{}}{\text{辺}}}$ は ${\overset{\textnormal{}}{\text{初}}}$ めてなんです。 \hfill\break
Sorry, I am new to this area. }
 
\par{${\overset{\textnormal{}}{\text{24. 旅館}}}$ はどちらですか。 \hfill\break
Where is the ryokan? \hfill\break
 \hfill\break
 \textbf{Culture Note }: In a 旅館 guests live on traditional tatami mats, take Japanese baths, and sleep on futon. You will most likely encounter Japanese style toilets, which look quite different. You have to squat on those. }
 
\par{25. 「 ${\overset{\textnormal{}}{\text{新幹線}}}$ に ${\overset{\textnormal{}}{\text{乗}}}$ ったことがありますか。」「いいえ、 ${\overset{\textnormal{}}{\text{乗}}}$ ったことがありません。ところで、 ${\overset{\textnormal{}}{\text{東京駅}}}$ に ${\overset{\textnormal{}}{\text{行}}}$ きたいですけど、どうやって ${\overset{\textnormal{}}{\text{行}}}$ けばいいですか。」「3つ ${\overset{\textnormal{}}{\text{目}}}$ の ${\overset{\textnormal{}}{\text{交差点}}}$ を ${\overset{\textnormal{}}{\text{右}}}$ に ${\overset{\textnormal{}}{\text{曲}}}$ がってください。そうすると、 ${\overset{\textnormal{}}{\text{左}}}$ にありますよ。」「どうも。」 \hfill\break
"Have you ridden the Shinkansen?" "No, I haven't". "By the way, I want to go to Tokyo Station, but how should I get there?" "Turn right on the third intersection. Then, it will be on your left". "Thanks". }
 
\par{\textbf{Culture Note }: The ${\overset{\textnormal{}}{\text{新幹線}}}$ , also known as the "bullet train", is one of the fastest passenger train services in the world. The 新幹線 opened in ${\overset{\textnormal{}}{\text{九州}}}$ in 2011. }
 
\par{26. Xへはどうやって ${\overset{\textnormal{}}{\text{行}}}$ きますか。 \hfill\break
How do I get to X? }
 
\par{${\overset{\textnormal{}}{\text{27. 道}}}$ に ${\overset{\textnormal{}}{\text{迷}}}$ いました。 ${\overset{\textnormal{}}{\text{銀行}}}$ を ${\overset{\textnormal{さが}}{\text{探}}}$ しています。どうすれば ${\overset{\textnormal{}}{\text{行}}}$ けますか。 \hfill\break
We're lost. We are trying to find a bank. How can we get there? }
 
\par{28. 「 ${\overset{\textnormal{}}{\text{遠}}}$ いですか」「いいえ5分くらいですよ」「ああ、そうですか。どうもありがとう」「どういたしまして」 \hfill\break
“Is it far?” “No, it's about five minutes (from here)” “Ah, really. Thank you very much" "You're welcome”. }
      
\section{Then}
 
\par{ "Then" is それから when you're explaining a progression of events, which is very important in giving directions and showing what happens right after in general, and そうすると is used when you do something, something else will happen. }

\par{29. 2つ ${\overset{\textnormal{}}{\text{目}}}$ の ${\overset{\textnormal{}}{\text{交差点}}}$ を ${\overset{\textnormal{}}{\text{左}}}$ に ${\overset{\textnormal{}}{\text{曲}}}$ がってください。それから、まっすぐ ${\overset{\textnormal{}}{\text{行}}}$ ってください。 \hfill\break
Please turn right on the second intersection. Then, please go straight. }
 
\par{30. この ${\overset{\textnormal{}}{\text{道}}}$ をまっすぐ ${\overset{\textnormal{}}{\text{行}}}$ ってください。そうすると、 ${\overset{\textnormal{}}{\text{右}}}$ に ${\overset{\textnormal{}}{\text{病院}}}$ があります。 \hfill\break
Please go straight on this street. Then, there will be a hospital to your right. }
 
\par{${\overset{\textnormal{}}{\text{31. 薬}}}$ を ${\overset{\textnormal{}}{\text{口}}}$ に ${\overset{\textnormal{ふく}}{\text{含}}}$ んだ。それから、 ${\overset{\textnormal{}}{\text{水}}}$ で ${\overset{\textnormal{なが}}{\text{流}}}$ し ${\overset{\textnormal{こ}}{\text{込}}}$ んだ。 \hfill\break
I swallowed the medicine. Then, I washed it down with water. }
 
\par{${\overset{\textnormal{}}{\text{32. 私}}}$ は ${\overset{\textnormal{}}{\text{仕事}}}$ で ${\overset{\textnormal{さら}}{\text{皿}}}$ を ${\overset{\textnormal{}}{\text{洗}}}$ い、それから ${\overset{\textnormal{かわ}}{\text{乾}}}$ かしました。 \hfill\break
I washed then dried the dishes at work. }
      
\section{Exercises}
 
\par{Translate into Japanese (1-5) }

\par{1. I'm going to school tomorrow. }

\par{2. Where is the bank? }

\par{3. How should I go to the movie theatres? }

\par{4. The police station is behind the department store. }

\par{5. The neighborhood is to the left side of the highway. }

\par{6.How do you say then as in continuing your directions? }

\par{7. Explain そうすると. }

\par{8. Make a paragraph using 3 locations, and 2 specific indicators where they're are. }
    