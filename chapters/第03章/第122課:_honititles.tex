    
\chapter{呼称}

\begin{center}
\begin{Large}
第122課: 呼称 
\end{Large}
\end{center}
 
\par{ In our first step into the world of 敬語 (honorific speech), the first thing to learn about are affixes used to embellish names and titles called ${\overset{\textnormal{こしょう}}{\text{呼称}}}$ . You've already been acquainted with the majority of them already, so they should be recognizable to you. A lot can be said about each of them, but for the ones that particularly matter to honorific speech, leaving them out (呼び ${\overset{\textnormal{す}}{\text{捨}}}$ て) can be drastically bad. }
      
\section{呼称}
 
\par{ 呼称 are suffixes similar to Mr. and Mrs., but aren't, unless stated, gender specific. They are suffixes added to titles or names. Like we've seen, some titles don't need them. Pay close attention to when these endings are used and for whom. }

\par{\textbf{Analysis }}

\par{We need context to know the extent 呼称 are used. There are so many niches that one could be a part in that no single person, including natives, could tell you authoritatively every single situation these endings can be used in. Luckily, however, you can get the gist of how they're used and see for yourself how they're used. }

\par{There are a few ground rules, however. First, you must never directly use them to refer to oneself. Two, you mustn't use more than one to refer to the same person in the same sentence. }

\par{1. わたしは金子さんです。X \hfill\break
わたしは金子です。 \hfill\break
I am Kaneko (surname). }

\par{2. 店員さんは裕子さんですよ。X \hfill\break
店員の人は裕子さんですよ。 \hfill\break
The clerk is Yuuko-san. }

\begin{center}
 \textbf{~さん }
\end{center}

\par{ ~さん is generally attached to the name of a person or group to show a light respect or familiarity. ~さん is a contraction of ~さま, which is why it isn't as respectful both after people's names and in set phrases. }

\par{2. もしもし、加藤さんのお宅ですか。 \hfill\break
Hello, is this the Kato residence? }

\par{3. お待ちどおさん \hfill\break
Thank you for waiting; I'm sorry to have kept you waiting. }

\par{ ~さん may be attached to someone's workplace to refer to the workers. So, if you walk into a electric appliance store, you may refer to the clerk as 電気屋さん. Other examples of this include 本屋さん, 八百屋さん, etc. Businesses may also refer to each other in the same fashion. }

\par{\textbf{Pronunciation Note }: ~さん is rendered as ~はん in certain dialects. }

\begin{center}
\textbf{~ちゃん }
\end{center}

\par{~ちゃん is a diminutive of ~さん, and should be treated as one. It shows affection and endearment. In general, \emph{~ }ちゃん is used toward infants, pets, children, grandparents, lovers, and\slash or close friends. When used toward the wrong individual, it could sound condescending, but tone of voice and one's relationship with the individual are both important factors. }

\begin{center}
 \textbf{~坊 }
\end{center}

\par{~坊 may be after a boy's name to show affection, but it is also placed after the alias of a monk. }

\begin{center}
 \textbf{~さま }
\end{center}

\par{~さま, again, is much more respectful than ~さん. It can be applied to basically anything ~さん came, but the effect is different. ~さま can be seen after おれ to refer to oneself in a very haughty manner. }

\par{4. スミス様 \hfill\break
Mr. Smith }

\par{5. 先生様 X \textrightarrow  先生 }

\par{6. 神様 \hfill\break
God }

\par{7. 天皇様  △   \textrightarrow  天皇陛下 \hfill\break
Emperor }

\par{8. 王様 \hfill\break
King }

\par{\textbf{Word Note }: It's important to note that 天皇様 is a questionable phrase in the grey zone with most Japanese speakers thinking it is inappropriate for similar reasons why 王様 is actually found to be somewhat kiddish. This is probably because ~様 through hyperbole weakening in honorific meaning for these instances. }

\par{9. ${\overset{\textnormal{じゅんさ}}{\text{巡査}}}$ は ${\overset{\textnormal{じゅんこ}}{\text{順子}}}$ のことを ${\overset{\textnormal{き}}{\text{訊}}}$ いたが、これは ${\overset{\textnormal{やどちょう}}{\text{宿帳}}}$ に彼女の名前がないからだった。「 ${\overset{\textnormal{ほかいちめいさま}}{\text{外一名様}}}$ 」とあるだけである。 \hfill\break
The police officer asked about Junko, but this was because her name wasn't in the guest book. It only said, "one other person." }

\par{\textbf{Word Note }: 外一名様 is occasionally used to refer to another person anonymously for which mention is necessary out of respect. However, many people nowadays take offense to the term. Sometimes, organizations now send out explanations to their addresses whenever the terminology is implemented to seek understanding. Part of this is due to it falling out of use, but another issue is people now generally expect all persons of interest, especially when there aren't that many to begin with, to be listed out individually. Of course, there are real world instances in which pinpointing even a small number of people specifically is difficult. }

\begin{center}
 \textbf{~君 }
\end{center}

\par{~君 used to be similar to ~の君(きみ). ~の君 refers to Lords, etc. Now, ~君 is used to address males juniors. Females may also use it towards males affectionately. With close personal friends and family, gender restrictions are removed. In business, men may refer to young employees of either gender with ~君. This works the same way in a classroom. In the Japanese legislature chairpersons address members with ~君. }

\par{10. 黒木君はどこにいるの? \hfill\break
Where's Kuroki? }

\par{11. ${\overset{\textnormal{しんゆう}}{\text{親友}}}$ の ${\overset{\textnormal{じろうくん}}{\text{次郎君}}}$ \hfill\break
 My best friend Jiro }

\begin{center}
 \textbf{~殿 }
\end{center}

\par{ Although literally meaning "Lord", it is used a lot in official documents seen after names or after an organization, department, etc. However, when you are referring to a superior by name, you will use his\slash her surname + ~様, not ~殿. When you are referring to a managerial position rather than a personal name, ~殿 will likely be your colleague. All of this, however, is about the written language. ~殿 does not get used in normal conversation. If you hear it in anime, it's because the speaker is purposely speaking in a neo-classical fashion. }

\par{12. 〇〇部長殿 \hfill\break
Department Chairman xx }

\par{ Luckily, for when you are addressing a group of people, 御中, which is read as おんちゅう, and 各位 are perfect. The former is used in documents addressed to schools, companies, company offices, groups, and organizations. 各位 is used as a header to a letter, but as an addressee, it can be used when sending something to each person without having to say every single name. There is no need to say 各位様 or 各位殿. }

\begin{center}
 \textbf{~氏 }
\end{center}

\par{ ~氏 is used to reference someone in the third person in the written language or usages of speech akin to the written language such as news reports and the like. }

\begin{center}
\textbf{Others }
\end{center}

\par{閣下: This word is realistically only used in two instances. One, you will see it in the official means of addressing heads of state or dignitaries by the Ministry of Foreign Affairs of Japan. For Japanese heads of state, the word is never used. However, for foreign figures, it follows the full name of the country followed by the full name of the person in question. For instance, the soon to be former president of Italy would be referred to as "イタリア共和国大統領ジョルジョ・ナポリターノ閣下." }

\par{Aside from being used to refer to foreign emissaries and dignitaries, it has been used in the realm of translating foreign literature for instances of "excellency." For instance, you may find "your honor" in a courtroom-sense translated as "判事閣下." However, because it has been relegated to very official circumstances, even this would be replaced with what's actually used nowadays, "裁判官." There are also other military\slash lordship titles that it may refer to, but these are all so specialized that the only use they would serve is if you were to read a novel in which they appear. At which, all you would need to know is that its purpose is being a very honorific title. }

\par{陛下: This word is reserved for the Imperial Family of Japan. Of course, you will also hear it employed for the same purpose for imperial members of other nations (even fictional places). }

\par{13.天皇皇后両陛下 \hfill\break
Their Majesties the Emperor and Empress }

\par{婦人: This word used to go after surnames to refer to the Mrs. of the household. However, due to concerns of sexism, words with 婦 are becoming less common. As such, although 婦人 would have been commonly used fifty years ago, that is no longer the case. }

\par{14. 中村婦人 \hfill\break
Mrs. Nakamura }

\par{夫人: This word is used to mean "Mrs." and is still frequently used. Despite being homophonous with 婦人, due to the difference in characters used, it has not died out just yet. }

\par{15. 加藤夫人はどちらに行かれたのか教えてください。 \hfill\break
Please tell me where Mrs. Kato went. }

\par{嬢: This is seldom used after the name of unmarried women, but it can also be more frequently found in certain occupation terms and set phrases. As お嬢さん, it is still an appropriate and respectful way of referring to a young lady. In other instances, however, the same can't be quite said. }

\par{16. 裕子はもともとキャバ嬢として働いてたよ。 \hfill\break
Yuuko used to work as a hostess. }

\par{卿: This is a very rare honorific used to refer to aristocracy. It is now relegated to historical\slash classical references. }

\par{17. アーサー卿は収入のほとんどを荘園の維持費にしていたといわれています。 \hfill\break
It is said that Lord Arthur used most of his income in the maintenance of his gardens. }

\begin{center}
 \textbf{Honorable Mentions }
\end{center}

\begin{ltabulary}{|P|P|P|}
\hline 

~たん & ~たん & Extreme slang form of ~ちゃん. \\ \cline{1-3}

~ちん & ~ちん & Yet another slang derivation that came from ~たん. \\ \cline{1-3}

~ぽん & ~ぽん & Used in jokes as a silly variant of ~さん. \\ \cline{1-3}

\end{ltabulary}
    