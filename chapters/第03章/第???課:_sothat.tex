    
\chapter*{~ように}

\begin{center}
\begin{Large}
第???課: ~ように: So That\dothyp{}\dothyp{}\dothyp{} 
\end{Large}
\end{center}
 
\par{ In this lesson, we will look at yet another grammatical expression which utilizes the purpose-marking に. This phrase is ~ように, which is composed of the noun ${\overset{\textnormal{よう}}{\text{様}}}$ and the particle に. 様 is a Sino-Japanese noun, but unlike the noun ${\overset{\textnormal{ため}}{\text{為}}}$ we learned about in the previous two lessons, 様 is not ever used in true isolation as a standalone noun—at least not in Modern Standard Japanese. }

\par{ Overall, ~ように is used to express an expectation or goal that one wishes to realize or hopes will realize if possible. This is similar to ~ために; however, ~ように oppositely implies no volition on the part of the agent of the clause that precedes it. }

\par{\textbf{Orthography Note }: The よう in ~ように is only seldom spelled as 様. When it is, it is usually in very formal writing. }
      
\section{Conjugation Recap}
 
\par{ First, let\textquotesingle s have a bit of conjugation review. Despite the ambition ~ように following the same grammar as any other nominal expression which follows a verb, because the expression itself is totally different in terms of part of speech from its English counterpart “so that,” we will look at how to conjugate ~ように with each kind of verb. Note that unlike the simile ~ように,  the ambition ~ように is only used with verbs. }

\begin{ltabulary}{|P|P|P|P|}
\hline 

\slash eru\slash - \emph{Ichidan }Verb & 見える + ように \hfill\break
見えない + ように & 見えるように \hfill\break
見えないように & So that… can be seen \hfill\break
So that… cannot be seen \\ \cline{1-4}

\slash iru\slash - \emph{Ichidan }Verb & 見る + ように \hfill\break
見ない + ように & 見るように \hfill\break
見ないように & So that… sees \hfill\break
So that… doesn\textquotesingle t see \\ \cline{1-4}

\slash u\slash - \emph{Godan }Verb & 買う + ように \hfill\break
買わない + ように & 買うように \hfill\break
買わないように & So that… buys \hfill\break
So that… doesn\textquotesingle t buy \\ \cline{1-4}

\slash ku\slash - \emph{Godan }Verb & 行く + ように \hfill\break
行かない + ように & 行くように \hfill\break
行かないように & So that… goes \hfill\break
So that… doesn\textquotesingle t go \\ \cline{1-4}

\slash gu\slash - \emph{Godan }Verb & 騒ぐ + ように \hfill\break
騒がない + ように & 騒ぐように \hfill\break
騒がないように & So that… makes a fuss \hfill\break
So that… doesn\textquotesingle t make a fuss \\ \cline{1-4}

\slash su\slash - \emph{Godan }Verb & 話す + ように \hfill\break
話さない + ように & 話すように \hfill\break
話さないように & So that… talks \hfill\break
So that… doesn\textquotesingle t talk \\ \cline{1-4}

\slash tsu\slash - \emph{Godan }Verb & 立つ + ように \hfill\break
立たない + ように & 立つように \hfill\break
立たないように & So that… stands \hfill\break
So that… doesn\textquotesingle t stand \\ \cline{1-4}

\slash nu\slash - \emph{Godan }Verb & 死ぬ + ように \hfill\break
死なない + ように & 死ぬように \hfill\break
死なないように & So that… dies \hfill\break
So that… doesn\textquotesingle t die \\ \cline{1-4}

\slash bu\slash - \emph{Godan }Verb & 運ぶ + ように \hfill\break
運ばない + ように & 運ぶように \hfill\break
運ばないように & So that… carries \hfill\break
So that… doesn\textquotesingle t carry \\ \cline{1-4}

\slash mu\slash - \emph{Godan }Verb & 飲む + ように \hfill\break
飲まない + ように & 飲むように \hfill\break
飲まないように & So that… drinks \hfill\break
So that… doesn\textquotesingle t drink \\ \cline{1-4}

\slash ru\slash - \emph{Godan }Verb & 切る + ように \hfill\break
切らない + ように & 切るように \hfill\break
切らないように & So that\dothyp{}\dothyp{}\dothyp{} cuts \hfill\break
So that\dothyp{}\dothyp{}\dothyp{} doesn't cut \\ \cline{1-4}

 \emph{Aru }& ある + ように \hfill\break
ない + ように & あるように \hfill\break
ないように & So that… there is \hfill\break
So that… there isn\textquotesingle t \\ \cline{1-4}

 \emph{Suru }(Verb) & する + ように \hfill\break
しない + ように & するように \hfill\break
しないように & So that… does \hfill\break
So that… doesn\textquotesingle t do \\ \cline{1-4}

 \emph{Kuru }& くる + ように \hfill\break
こない + ように & くるように \hfill\break
こないように & So that… comes \hfill\break
So that… doesn\textquotesingle t come \\ \cline{1-4}

\end{ltabulary}

\par{\textbf{Part of Speech Note }: Conjugation-wise, ~ように behaves like a nominal phrase, but the purpose-marking に is adverbial in nature. Thus, as a whole, ~ように can be viewed as an adverbial phrase. This contrasts with “so that,” which is treated as a conjunction in English. Even in Japanese, though, the resultant phrase can also be viewed as an adverbial conjunctive phrase. }
      
\section{The Usages of ~ように}
 
\par{ Whereas ~ために must be used with verbs of volition to express realizing a stated objective\slash goal, ~ように must be used with verbs that don\textquotesingle t express volition. In fact, although ~ように is largely used with intransitive verbs, it can be used with transitive verbs so long as those verbs don\textquotesingle t imply willful control over the action by the agent. }

\par{ Just as was the case with ~ために, the concept of “agent” is very important to understanding how to use ~ように properly. First, know that the base sentence pattern is “A + ~ように + B.” Both “A” and “B” are verbal expressions. Each verbal expression will have an agent. “A” will always be the expectation\slash goal, and “B” will either be an effort or an exertion of influence. The nature of “B” is determined by who the agent of “A” is and whether it is the same as the one in “B.” }

\par{ ~ように is not particularly complicated to understand, but these factors do add subtleties to its interpretation. For each usage discussed, we will learn about what the “agent” can be for both the “A” clause and the “B” clause, what kind of verbs “A” and “B” can be, and compare usages with each other if necessary . }

\par{\textbf{Grammar Note }: Do not confuse this with the volitional form. The volitional form is created with the auxiliary verb ~よう for only a subset of verbs in Japanese, and conjugating with it does not follow the same rules as ~ように. }

\begin{center}
\textbf{Usage 1:(目標+努力) }
\end{center}

\par{ The words 目標 (goal) and 努力 (effort) are very important words to know in understanding ~ように. For Usage 1 of “A + ~ように + B,” A is a goal that the speaker wishes to make possible by some effort expressed by B. This usage normally involves A being a verb in the potential form, negative form, or the potential-negative form. When not specifically used with a potential verb, the verb in question should be an intransitive verb of non-volition. }

\par{\textbf{Translation Note }: This usage is most often translated as “so that.” }

\par{1. ${\overset{\textnormal{らいねんふっかつ}}{\text{来年復活}}}$ できる \textbf{ように }${\overset{\textnormal{がんば}}{\text{頑張}}}$ りたいと ${\overset{\textnormal{おも}}{\text{思}}}$ います。 \hfill\break
I want to do my best \textbf{so that }I can make a come-back next year. }

\par{2. ${\overset{\textnormal{かんこく}}{\text{韓国}}}$ へ ${\overset{\textnormal{い}}{\text{行}}}$ ける \textbf{ように }${\overset{\textnormal{ちょきん}}{\text{貯金}}}$ をしています。 \hfill\break
I am saving money \textbf{so that }I can go to Korea. }

\par{3. ( ${\overset{\textnormal{じぶん}}{\text{自分}}}$ が) ${\overset{\textnormal{じぶん}}{\text{自分}}}$ であることをいつか ${\overset{\textnormal{ほこ}}{\text{誇}}}$ れる \textbf{ように }、 ${\overset{\textnormal{きょう}}{\text{今日}}}$ も ${\overset{\textnormal{あす}}{\text{明日}}}$ も ${\overset{\textnormal{むね}}{\text{胸}}}$ を ${\overset{\textnormal{は}}{\text{張}}}$ るんだ。 \hfill\break
I shall be confident today and tomorrow \textbf{so that }I one day can be proud of being myself. }

\par{4. ${\overset{\textnormal{よ}}{\text{良}}}$ い ${\overset{\textnormal{にほんごきょうし}}{\text{日本語教師}}}$ になれる \textbf{ように }、 ${\overset{\textnormal{けいけん}}{\text{経験}}}$ を ${\overset{\textnormal{つ}}{\text{積}}}$ んでいきたいと ${\overset{\textnormal{おも}}{\text{思}}}$ います。 \hfill\break
I\textquotesingle d like to gain experience \textbf{so that }I may become a good Japanese teacher. }

\par{5. ${\overset{\textnormal{いち}}{\text{1}}}$ ${\overset{\textnormal{にち}}{\text{日}}}$ \{も・でも\} ${\overset{\textnormal{はや}}{\text{早}}}$ く ${\overset{\textnormal{かいけつ}}{\text{解決}}}$ できる \textbf{ように }やっていきたい。 \hfill\break
I'd like (for us) to move forward \textbf{so that }I\slash we can resolve this as soon as possible. }

\par{6. クビにならないよう(に)しっかり ${\overset{\textnormal{しごと}}{\text{仕事}}}$ を ${\overset{\textnormal{おぼ}}{\text{覚}}}$ えるのが ${\overset{\textnormal{さき}}{\text{先}}}$ だ。 \hfill\break
Properly learning the job comes first \textbf{(so as) } \textbf{to }not getting fired. }

\par{\textbf{Grammar Note }: Although the person that would potentially fire the agent of B would not be that said individual, the agent of B is the experiencer of A, and A is still a goal that the agent of B would be trying to make sure of. }

\par{\textbf{Particle Note }: Ex. 6 is an example of the agent of A and B being “one,” which is a third-person pronoun and not a first-person pronoun like all the other examples in this section. When this is not used in first person, it is possible to omit the particle に. This omission is commonplace in very formal writing and oration. It is worth noting that when this usage is used in second-person, it brings about Usage 3. }

\par{7. よく ${\overset{\textnormal{み}}{\text{見}}}$ えるように( ${\overset{\textnormal{からだ}}{\text{体}}}$ を) ${\overset{\textnormal{たか}}{\text{高}}}$ く ${\overset{\textnormal{も}}{\text{持}}}$ ち ${\overset{\textnormal{あ}}{\text{上}}}$ げてもらった。 \hfill\break
I had myself lifted up \textbf{so that }I could see well. }

\par{\textbf{Grammar Note }: In this example, B involves an effort which has someone do something for the speaker, but the experiencer of A and the causer of B is still the same individual, and B is still an action that person does to achieve goal A. }

\par{8. ${\overset{\textnormal{かぜ}}{\text{風邪}}}$ やインフルエンザなどにならない \textbf{ように }、 ${\overset{\textnormal{めんえきりょく}}{\text{免疫力}}}$ をアップさせることを ${\overset{\textnormal{いしき}}{\text{意識}}}$ して ${\overset{\textnormal{せいかつ}}{\text{生活}}}$ していきたいと ${\overset{\textnormal{おも}}{\text{思}}}$ います。 \hfill\break
I want to live while consciously thinking of increasing my immunity \textbf{so that }I don\textquotesingle t get colds or the flu. }

\par{\textbf{Grammar Note }: In this sentence, the experiencer of A is the agent of B. Even though the viruses that would cause the speaker to become ill would be the technical agent of A, Japanese does not typically treat non-human things such as this as active agents. Thus, this creates a basis for concluding that if the agent of B is not present in A that A must be the same person as B but as the role of experiencer. }

\par{9. ${\overset{\textnormal{す}}{\text{擦}}}$ れ ${\overset{\textnormal{ちが}}{\text{違}}}$ う \textbf{ように }${\overset{\textnormal{きみ}}{\text{君}}}$ に ${\overset{\textnormal{あ}}{\text{会}}}$ いたい。 \hfill\break
I want to meet you \textbf{so that }we may pass by each other. }

\par{\textbf{Grammar Note }: In this sentence, the experiencer of A includes another individual, but because one of those experiencers is the agent of B, it is still representative of this grammar. }

\begin{center}
\textbf{~ようにする }
\end{center}

\par{ In the examples above, the agent\slash experiencer of A and B was limited to “I” and “one.” These examples also demonstrated how this grammar point doesn\textquotesingle t limit A to potential verbs. The one stipulation stated was that A would otherwise need to be a non-volitional intransitive verb. However, there are in fact instances when A is a transitive verb or volitional intransitive verb. When this is the case, though, B is limited to  する by default, creating the grammatical pattern ~ようにする. This phrase is translated as “to try to…” as in putting effort into making some goal a habitual action. }

\par{ The thought process behind this is that the volition that would otherwise be wholly expressed by the transitive verb of A is shifted to the verb in B, する. This is the case even when the verb in A also happens to be する. }

\par{\textbf{Particle Note }: Ex. 6 presented a case in which the particle に can be omitted after よう, but it is important that you not run with this. This is only true for second-person and third-person in formal writing and speaking and is not applicable to ~ようにする or anything involving ~ように with first person. Although ~ようにする can be used in second-person (in question form) and third-person, ~ようにする is a set phrase in which に is obligatory and should be treated as such. }

\par{10. ${\overset{\textnormal{ぎゅうにゅう}}{\text{牛乳}}}$ を ${\overset{\textnormal{の}}{\text{飲}}}$ まない \textbf{ようにしています }。 \hfill\break
I am trying not to drink milk. \hfill\break
 \hfill\break
11. ${\overset{\textnormal{すこ}}{\text{少}}}$ しずつ ${\overset{\textnormal{やさい}}{\text{野菜}}}$ を ${\overset{\textnormal{と}}{\text{摂}}}$ る \textbf{ようにしています }。 \hfill\break
I am trying to have vegetables little by little. }

\par{12. ${\overset{\textnormal{ぎんこうこうざ}}{\text{銀行口座}}}$ はまめにチェックする \textbf{ようにしています }。 \hfill\break
As for my bank account, I am trying to check it diligently. }

\begin{center}
\textbf{Usage 2: (目標+働きかけ) }
\end{center}

\par{ The word ${\overset{\textnormal{はたら}}{\text{働}}}$ きかけ means “pressure\slash encouragement.” The concept of putting pressure on someone becomes very important when the agents of A and B don\textquotesingle t match. When this is the case, the subject will manifest oneself in B, and B will be an action that puts pressure, encourages, and\slash or expresses hope\slash anticipation that goal A is realized. The goal of A could be several things. It could be the agent of A becoming able to do something (with potential verbs), doing some action (with transitive verbs or volitional intransitive verbs), or the experiencer of A becoming a certain state (non-volitional intransitive verbs). Regardless of what the goal of A is, B will always involve the subject of B either hoping or doing some action to get that goal realized. }

\par{\textbf{Translation Note }: Potential translations for this usage include “(so) that” and “so as to.” }

\par{\textbf{Grammar Note }: In many ways, Usage 2 is no different from Usage 1 except that the agents of A and B don\textquotesingle t match. Putting pressure on someone for A to be fulfilled is in and of itself an effort. }

\par{13. ${\overset{\textnormal{えいご}}{\text{英語}}}$ の ${\overset{\textnormal{ぶんぽう}}{\text{文法}}}$ が ${\overset{\textnormal{わ}}{\text{分}}}$ かる \textbf{ように }、 ${\overset{\textnormal{せんせい}}{\text{先生}}}$ は ${\overset{\textnormal{え}}{\text{絵}}}$ を ${\overset{\textnormal{つか}}{\text{使}}}$ って ${\overset{\textnormal{せつめい}}{\text{説明}}}$ してくれました。 \hfill\break
The teacher explained using pictures \textbf{so that }(they) could understand English grammar. \hfill\break
 \hfill\break
\textbf{Sentence Note }: The agent of B is “the teacher” and the agent\slash experiencer of A (understanding) is the implied “they.” }

\par{14. ${\overset{\textnormal{つま}}{\text{妻}}}$ が ${\overset{\textnormal{ぶじしゅっさん}}{\text{無事出産}}}$ できる \textbf{ように }、 ${\overset{\textnormal{かみさま}}{\text{神様}}}$ にお ${\overset{\textnormal{ねが}}{\text{願}}}$ いしています。 \hfill\break
I\textquotesingle ve asked God \textbf{that }my wife may deliver safely. }

\par{15. スマートに ${\overset{\textnormal{かいわ}}{\text{会話}}}$ を ${\overset{\textnormal{はじ}}{\text{始}}}$ めることができる \textbf{ように }、 ${\overset{\textnormal{じこしょうかい}}{\text{自己紹介}}}$ で ${\overset{\textnormal{つか}}{\text{使}}}$ われるフレーズをいくつか ${\overset{\textnormal{しょうかい}}{\text{紹介}}}$ してみました。 \hfill\break
I tried introducing (to them) several phrases used in self-introductions so that (they) could begin to be able to converse skillfully. }

\par{16. ${\overset{\textnormal{さよくしそう}}{\text{左翼思想}}}$ の ${\overset{\textnormal{ひとびと}}{\text{人々}}}$ は、 ${\overset{\textnormal{さいぶんぱい}}{\text{再分配}}}$ を ${\overset{\textnormal{きょうか}}{\text{強化}}}$ する \textbf{ように }${\overset{\textnormal{ぜいせい}}{\text{税制}}}$ を ${\overset{\textnormal{か}}{\text{変}}}$ えようと ${\overset{\textnormal{しゅちょう}}{\text{主張}}}$ している。 \hfill\break
Leftists are insistent about changing the tax system so as to reinforce redistribution. }

\par{17. ${\overset{\textnormal{かんちょう}}{\text{官庁}}}$ は、 ${\overset{\textnormal{どうろ}}{\text{道路}}}$ の ${\overset{\textnormal{けんせつ}}{\text{建設}}}$ に ${\overset{\textnormal{ちから}}{\text{力}}}$ を ${\overset{\textnormal{い}}{\text{入}}}$ れ、 ${\overset{\textnormal{しほんか}}{\text{資本家}}}$ や ${\overset{\textnormal{きぎょう}}{\text{企業}}}$ を ${\overset{\textnormal{ほご}}{\text{保護}}}$ する \textbf{ように }${\overset{\textnormal{せいさく}}{\text{政策}}}$ を ${\overset{\textnormal{てんかん}}{\text{転換}}}$ した。 \hfill\break
The authorities changed its policy so as to put effort into road construction and protect capitalists and corporations. }

\par{18. ${\overset{\textnormal{じしんなど}}{\text{地震等}}}$ の ${\overset{\textnormal{しんどう}}{\text{振動}}}$ で ${\overset{\textnormal{てんとう}}{\text{転倒}}}$ しない \textbf{よう }、 ${\overset{\textnormal{こてい}}{\text{固定}}}$ する ${\overset{\textnormal{ひつよう}}{\text{必要}}}$ のあるものは ${\overset{\textnormal{かなら}}{\text{必}}}$ ず ${\overset{\textnormal{こてい}}{\text{固定}}}$ する。 \hfill\break
Always secure things that ought to be secured so that they do not fall down due to vibrations caused by earthquakes and such. }

\par{\textbf{Grammar Note }: This example, in contrast with Exs. 16 and 17, shows how に is only omissible when the sentence is interpreted as a directive to someone, which was also the case with Ex. 6. These examples all hint at Usage 3, which we are about to come to shortly. }

\par{19. ${\overset{\textnormal{こうせいろうどうしょう}}{\text{厚生労働省}}}$ は、インフルエンザと ${\overset{\textnormal{しんだん}}{\text{診断}}}$ されてから ${\overset{\textnormal{すく}}{\text{少}}}$ なくとも ${\overset{\textnormal{ふつ}}{\text{2}}}$ ${\overset{\textnormal{かかん}}{\text{日間}}}$ は ${\overset{\textnormal{きょくりょく}}{\text{極力}}}$ ${\overset{\textnormal{ひと}}{\text{1}}}$ ${\overset{\textnormal{り}}{\text{人}}}$ にしない \textbf{よう }${\overset{\textnormal{ちゅうい}}{\text{注意}}}$ するほか、マンションやアパートの ${\overset{\textnormal{ばあい}}{\text{場合}}}$ は ${\overset{\textnormal{まど}}{\text{窓}}}$ や ${\overset{\textnormal{げんかん}}{\text{玄関}}}$ を ${\overset{\textnormal{せじょう}}{\text{施錠}}}$ し、ベランダに ${\overset{\textnormal{めん}}{\text{面}}}$ していない ${\overset{\textnormal{へや}}{\text{部屋}}}$ で ${\overset{\textnormal{ね}}{\text{寝}}}$ かせる \textbf{よう }${\overset{\textnormal{よ}}{\text{呼}}}$ びかけている。また、 ${\overset{\textnormal{おお}}{\text{多}}}$ くの ${\overset{\textnormal{ひと}}{\text{人}}}$ が ${\overset{\textnormal{はや}}{\text{早}}}$ めにワクチンを ${\overset{\textnormal{せっしゅ}}{\text{接種}}}$ できる \textbf{よう }、 ${\overset{\textnormal{じゅうさん}}{\text{13}}}$ ${\overset{\textnormal{さいいじょう}}{\text{歳以上}}}$ の ${\overset{\textnormal{ひと}}{\text{人}}}$ は ${\overset{\textnormal{げんそく}}{\text{原則}}}$ 、 ${\overset{\textnormal{いっ}}{\text{1}}}$ ${\overset{\textnormal{かい}}{\text{回}}}$ だけの ${\overset{\textnormal{せっしゅ}}{\text{接種}}}$ にしてほしいと ${\overset{\textnormal{よ}}{\text{呼}}}$ びかけている。 \hfill\break
The Ministry of Health, Labour and Welfare is asking that, in addition to being careful not to leave someone diagnosed with influenza alone as much as possible for at least two days, they are also calling that for those in apartment buildings or condominium high-rises have (said individuals) rest in rooms which are not facing verandas. They are also calling for people 13 years or older to limit their vaccination to just once as a rule so that more people can be vaccinated ahead of time. }

\par{20. イエレン ${\overset{\textnormal{ぎちょう}}{\text{議長}}}$ は、 ${\overset{\textnormal{じゆうぼうえき}}{\text{自由貿易}}}$ を ${\overset{\textnormal{そこ}}{\text{損}}}$ なうことがない \textbf{よう }、トランプ ${\overset{\textnormal{せいけん}}{\text{政権}}}$ を ${\overset{\textnormal{けんせい}}{\text{牽制}}}$ した。 \hfill\break
Chairman Yellen checked the Trump Administration so that they don\textquotesingle t mar free trade. }

\par{21. ${\overset{\textnormal{さいがいはっせいじ}}{\text{災害発生時}}}$ 、 ${\overset{\textnormal{つと}}{\text{勤}}}$ め ${\overset{\textnormal{さき}}{\text{先}}}$ や ${\overset{\textnormal{がいしゅつちゅう}}{\text{外出中}}}$ の ${\overset{\textnormal{かぞく}}{\text{家族}}}$ の ${\overset{\textnormal{あんぴかくにん}}{\text{安否確認}}}$ をとるために ${\overset{\textnormal{れんらくさき}}{\text{連絡先}}}$ をみんながわかる \textbf{ように }していますか? \hfill\break
Do you have your contact information set up so that you know everyone in order to confirm the safety of family at work or out and about for when a natural disaster occurs? \hfill\break
 \hfill\break
22. ${\overset{\textnormal{かがくこうじょう}}{\text{科学工場}}}$ の ${\overset{\textnormal{かさい}}{\text{火災}}}$ なので、 ${\overset{\textnormal{しょくば}}{\text{職場}}}$ の ${\overset{\textnormal{なかま}}{\text{仲間}}}$ にはマスクをしたうえで ${\overset{\textnormal{そと}}{\text{外}}}$ に ${\overset{\textnormal{で}}{\text{出}}}$ ない \textbf{よう }${\overset{\textnormal{ちゅうい}}{\text{注意}}}$ をした。 \hfill\break
With it being a chemical factory fire, I warned my friends at the workplace not \textbf{to }go outside upon having put on masks. }

\par{23. ${\overset{\textnormal{さら}}{\text{更}}}$ に ${\overset{\textnormal{どりょく}}{\text{努力}}}$ してプレミアムフライデーを ${\overset{\textnormal{ていちゃく}}{\text{定着}}}$ させるよう ${\overset{\textnormal{がんば}}{\text{頑張}}}$ っていきたい。 \hfill\break
I want to do my best strive more to have Premium Friday take hold (in society). }

\par{\textbf{Grammar Note }: Although A is a causative verb, the agent of A and B is still the main speaker, making it an application of a “goal” being sought for by the effort expressed in B. The causative verb, however, does imply that the goal involves directing people to make it so, thus a grammatical motivation for why に can be omitted here. }

\par{24. ${\overset{\textnormal{こども}}{\text{子供}}}$ が ${\overset{\textnormal{さわ}}{\text{触}}}$ らない \textbf{ように }、 ${\overset{\textnormal{あぶ}}{\text{危}}}$ ない ${\overset{\textnormal{もの}}{\text{物}}}$ は ${\overset{\textnormal{たか}}{\text{高}}}$ いところに ${\overset{\textnormal{お}}{\text{置}}}$ く \textbf{ようにしています }。 \hfill\break
I\textquotesingle m \textbf{trying }to place dangerous objects high up \textbf{so that }(the) children won\textquotesingle t touch them. }

\par{\textbf{Grammar Note }: This example as well as Ex. 25 show that it is possible to have more than one ~ように in a sentence. As for the first instance in Ex. 24, its purpose is to indicate the goal of having children not touch dangerous objects which is achieved by the effort of placing them up high, which is an indirect pressure on said children not to get into such trouble. The purpose of ~ようにしている indicates that this is a pressure the speaker is actively exerting and trying to maintain. }

\par{25. ${\overset{\textnormal{はは}}{\text{母}}}$ に ${\overset{\textnormal{しんぱい}}{\text{心配}}}$ をかけない\{ \textbf{よう }・ため\} \textbf{に }、 ${\overset{\textnormal{まいにち}}{\text{毎日}}}$ 、 ${\overset{\textnormal{はや}}{\text{早}}}$ く ${\overset{\textnormal{いえ}}{\text{家}}}$ に ${\overset{\textnormal{かえ}}{\text{帰}}}$ る \textbf{ようにしている }。 \hfill\break
I am trying to return home quickly every day [ \textbf{so as }\slash in order] not \textbf{to }worry my mother. }

\par{\textbf{Grammar Note }: The difference between ~ように and ~ために is that although both indicate fulfilling a certain goal, only ~ために implies complete confidence of that goal being made so. ~ように puts more emphasis on the hope that the goal will be matched, but hope is not a guarantee. }

\par{26. ${\overset{\textnormal{がくせい}}{\text{学生}}}$ がフェイクニュースに ${\overset{\textnormal{まど}}{\text{惑}}}$ わされない \textbf{ように }しようと ${\overset{\textnormal{おも}}{\text{思}}}$ って、 ${\overset{\textnormal{ちょうさ}}{\text{調査}}}$ を ${\overset{\textnormal{はじ}}{\text{始}}}$ めました。 \hfill\break
I began an investigation hoping \textbf{to }have students not be misled by fake news. }

\par{\textbf{Grammar Note }: This example exemplifies how “effort” and “pressure” are one of the same thing. Do note that the よう in しよう is the auxiliary verb for affirmative volition. Its purpose is to emphasize the will of the speaker to have it so that students aren\textquotesingle t misled. }

\par{27. ${\overset{\textnormal{しゃっきん}}{\text{借金}}}$ は ${\overset{\textnormal{な}}{\text{無}}}$ くならないように ${\overset{\textnormal{でき}}{\text{出来}}}$ ているものだ。 \hfill\break
Debt is made so that it doesn\textquotesingle t go away. }

\par{\textbf{Grammar Note }: 出来る, here, means “to be made,” which implicitly suggests an agent who had the goal of making said liability one that wouldn\textquotesingle t go away. }

\begin{center}
\textbf{~ようになる }
\end{center}

\par{ Ex. 27 implies that it is possible to use ~ように in contexts in which goal A has already been achieved. This is no truer than when it is used in ~ようになる, which is intrinsically the non-volitional, intransitive form of ~ようにする. ~ようになる can be after transitive verbs to express a meaning that is close to “to come to.” }

\par{28. ${\overset{\textnormal{まいにち}}{\text{毎日}}}$ お ${\overset{\textnormal{ちゃ}}{\text{茶}}}$ を ${\overset{\textnormal{の}}{\text{飲}}}$ むようになった。 \hfill\break
I\textquotesingle ve come to drink tea daily. \hfill\break
 \hfill\break
 ~ようになる may also be after potential verbs to mean “to become able to.” \hfill\break
 \hfill\break
29. やっと ${\overset{\textnormal{りかい}}{\text{理解}}}$ できるようになった。 \hfill\break
I\textquotesingle ve finally become able to comprehend it. }

\par{ ~ようになる may also be after intransitive verbs to mean “to be (made) so that.” In this situation, it isn\textquotesingle t so much so that A is a goal that is already realized, but it may very well be an anticipation that has been borne out. }

\par{30. その ${\overset{\textnormal{けいたいでんわ}}{\text{携帯電話}}}$ は、しばらく ${\overset{\textnormal{ほうち}}{\text{放置}}}$ すると ${\overset{\textnormal{でんげん}}{\text{電源}}}$ が ${\overset{\textnormal{き}}{\text{切}}}$ れるようになっている。 \hfill\break
That cellphone is made so that it turns off when you leave it idle for a while. \hfill\break
 \hfill\break
\textbf{Grammar Note }: Like Ex. 27, the agent that would have programmed the phone had しばらく放置すると電源が切れる as a goal, and that success is embodied in ~ようになっている. The use of ~ている suggests that this success is not instantaneous with the statement but has been so. }

\begin{center}
\textbf{Usage 3: Euphemized Command }
\end{center}

\par{ When ~ように is directed toward someone, it expresses a command that indicates the manner you want the listener (agent) to do in order to achieve goal A. When the verb before ~ように is a transitive verb or volitional intransitive verb, the following effort B doesn\textquotesingle t have to be stated. However, when the verb is a non-volitional intransitive, the effort described by B cannot be omitted as the statement becomes a directive for the state expressed by the verb of A to come true (Ex. 32). }

\par{\textbf{Grammar Note }: This usage may also be interpreted as an application of the simile ~ように\textquotesingle s use of illustrating an example. In the examples below, ~ように is translated as “so that,” which plays emphasis on the interpretation of goal A being carried by effort B. However, “so that” can easily be paraphrased to “in a way that.” Thus, Usage 3 demonstrates that there is in fact overlap between these two different ~ように. }

\par{31. せっかく ${\overset{\textnormal{ねつ}}{\text{寝付}}}$ いた ${\overset{\textnormal{あか}}{\text{赤}}}$ ちゃんを ${\overset{\textnormal{お}}{\text{起}}}$ こさない \textbf{ように }、( ${\overset{\textnormal{しず}}{\text{静}}}$ かにしてよ)。 \hfill\break
(Be quiet \textbf{so that }you) don\textquotesingle t wake up the baby who just went to sleep. }

\par{\textbf{Translation Note }: When “B” is omitted from this sentence, it can be translated as “Try not to wake up the baby who just went to sleep.” }

\par{32. ${\overset{\textnormal{こわ}}{\text{壊}}}$ さない \textbf{ように }${\overset{\textnormal{はこ}}{\text{運}}}$ んでください。 \hfill\break
Please carry it [ \textbf{so that }you don\textquotesingle t break\slash  \textbf{with }out breaking] it. }

\par{33. ${\overset{\textnormal{ふだん}}{\text{普段}}}$ から、 ${\overset{\textnormal{あや}}{\text{怪}}}$ しい ${\overset{\textnormal{もう}}{\text{儲}}}$ け ${\overset{\textnormal{ばなし}}{\text{話}}}$ などの ${\overset{\textnormal{ゆうわく}}{\text{誘惑}}}$ に ${\overset{\textnormal{の}}{\text{乗}}}$ らない \textbf{ように }${\overset{\textnormal{こうどう}}{\text{行動}}}$ する \textbf{よう }${\overset{\textnormal{こころ}}{\text{心}}}$ がけましょう。 \hfill\break
Let\textquotesingle s keep in mind on a routine basis \textbf{to }conduct ourselves \textbf{so that }we are not yielded to temptation by shady get-rich-quick schemes. }

\par{\textbf{Particle Note }: Aside from the second よう having に omitted because of it marking a directive to others, another motivation for why に is omitted here is that it gets rid of grammatical awkwardness that would be felt with two too similar instances of ~ように. }

\par{ The next three examples show how this usage can also be used when the verb preceding ~ように is in polite speech. Thus, both ~ますよう(に)and  ~ませんよう(に) are possible. However, the use of the auxiliary ~ます is only acceptable when ~ように is used to create a command or to simply express a hope (Usage 5) }

\par{34. ${\overset{\textnormal{こんご}}{\text{今後}}}$ ともよろしくご ${\overset{\textnormal{しどう}}{\text{指導}}}$ くださいます \textbf{ように }(お ${\overset{\textnormal{ねが}}{\text{願}}}$ い ${\overset{\textnormal{もう}}{\text{申}}}$ し ${\overset{\textnormal{あ}}{\text{上}}}$ げます)。 \hfill\break
I ask \textbf{for }your continued guidance going forward. }

\par{35. ご ${\overset{\textnormal{かくにんいただ}}{\text{確認頂}}}$ きますよう、お ${\overset{\textnormal{ねが}}{\text{願}}}$ い ${\overset{\textnormal{いた}}{\text{致}}}$ します。 \hfill\break
I ask that you please confirm this. }

\par{36. ${\overset{\textnormal{さむ}}{\text{寒}}}$ い ${\overset{\textnormal{ひ}}{\text{日}}}$ が ${\overset{\textnormal{つづ}}{\text{続}}}$ きますので、お ${\overset{\textnormal{かぜ}}{\text{風邪}}}$ など ${\overset{\textnormal{め}}{\text{召}}}$ しませんよう、ご ${\overset{\textnormal{じあい}}{\text{自愛}}}$ のほどお ${\overset{\textnormal{ねが}}{\text{願}}}$ い ${\overset{\textnormal{もう}}{\text{申}}}$ し上げます。 \hfill\break
Because cold days continue, please take care of yourself so that you do not catch a cold. }

\begin{center}
\textbf{Usage 4: ~ようにと }
\end{center}

\par{ As an extension of the previous usage of creating euphemized commands regarding manner, ~ように can be optionally followed by the quotation-marking と and then followed by a request verb of some kind. This is largely restricted to transitive verbs or volitional intransitive verbs. }

\par{37. ${\overset{\textnormal{ともだち}}{\text{友達}}}$ に ${\overset{\textnormal{にもつ}}{\text{荷物}}}$ を ${\overset{\textnormal{はこ}}{\text{運}}}$ ぶように(と) ${\overset{\textnormal{たの}}{\text{頼}}}$ みました。 \hfill\break
I asked my friend to carry [the\slash my] luggage. }

\par{38. ${\overset{\textnormal{がくせい}}{\text{学生}}}$ たちに ${\overset{\textnormal{じゅぎょう}}{\text{授業}}}$ を ${\overset{\textnormal{けっせき}}{\text{欠席}}}$ しない \textbf{ように(と) }${\overset{\textnormal{ちゅうい}}{\text{注意}}}$ しました。 \hfill\break
I warned the students not \textbf{to }be absent from class. }

\par{39. ${\overset{\textnormal{わたし}}{\text{私}}}$ は ${\overset{\textnormal{かれ}}{\text{彼}}}$ に ${\overset{\textnormal{げんこう}}{\text{原稿}}}$ を ${\overset{\textnormal{か}}{\text{書}}}$ く \textbf{ように(と) }${\overset{\textnormal{いらい}}{\text{依頼}}}$ しました。 \hfill\break
I requested to him \textbf{that }he write draft\slash manuscript. }

\par{40. ${\overset{\textnormal{かれ}}{\text{彼}}}$ はみんなに ${\overset{\textnormal{ちゅうい}}{\text{注意}}}$ して ${\overset{\textnormal{うんてん}}{\text{運転}}}$ する \textbf{ように(と) }${\overset{\textnormal{うなが}}{\text{促}}}$ した。 \hfill\break
I urged everyone \textbf{to }be careful driving. }

\par{41. 花が咲く \textbf{ように }(と思って)毎日水をやっている。 \hfill\break
I water the flowers daily \textbf{so that }they may bloom. }

\par{\textbf{Grammar Note }: Ex. 41 is an example of this being used with a non-volitional verb. When that is the case, と is obligatorily followed by a verb of thought. This puts one\textquotesingle s hope into a mental quotation. \textbf{\hfill\break
}}

\begin{center}
\textbf{Usage 5: Hope\slash Prayer }
\end{center}

\par{ In the same way ~ように can indicate a command to someone so that a goal can be met, when the verb involved is a non-volitional verb and nothing else follows, the resultant phrase solely expresses a hope that that goal is realized. This was hinted at in Ex. 41, but the examples below explicitly show how ~ように without “B” expresses hope. This usage is used extensively when praying and wishing for something to happen. }

\par{ Typically, this usage is used with the verb preceding ~ように being in polite speech, thus creating ~ますように. If a plain form speech were to be used, then grammar resembling that of Ex. 41 would need to follow. Unsurprisingly, ~ませんように is also possible with this usage. }

\par{42. ${\overset{\textnormal{あ}}{\text{当}}}$ たりますように。 \hfill\break
May [I\slash you] hit (the lottery). }

\par{43. ${\overset{\textnormal{すてき}}{\text{素敵}}}$ な ${\overset{\textnormal{いちねん}}{\text{一年}}}$ になりますように。 \hfill\break
May this be a wonderful year. }

\par{44. ずっと ${\overset{\textnormal{いっしょ}}{\text{一緒}}}$ にいられますように。 \hfill\break
May we always be together. }

\par{45. どうか ${\overset{\textnormal{あや}}{\text{怪}}}$ しまれませんように。 \hfill\break
May I not be suspected. }

\par{46. お ${\overset{\textnormal{つか}}{\text{疲}}}$ れの ${\overset{\textnormal{で}}{\text{出}}}$ ませんように。 \hfill\break
May you not become wary (in your endeavors). }

\par{\textbf{Phrase Note }: Ex. 47 is the honorific means of expressing ${\overset{\textnormal{がんば}}{\text{頑張}}}$ ってください to superiors. It is not so ubiquitous among younger speakers, but knowledge of it and proper use of it in the workplace is highly appreciated by older generations among adult speakers. }

\begin{center}
\textbf{Variation with Negative Auxiliaries }
\end{center}

\par{ In conclusion, it\textquotesingle s worth noting that ~ない is not the only form that negation takes and that those other forms can all be used in conjunction with ~ように. Aside from ~ませんよう, there is also ~ぬよう(に), seen most frequently in literature and song as a more poetic variant. There is also ~んよう(に) which is merely dialectical, but to many can give off the impression of a man in his 40s or older. Lastly, ~へんよう(に) is possible in Kansai Dialects. }

\par{47. ${\overset{\textnormal{じせつがら}}{\text{時節柄}}}$ 、 ${\overset{\textnormal{たいちょう}}{\text{体調}}}$ を ${\overset{\textnormal{くず}}{\text{崩}}}$ されませぬようご ${\overset{\textnormal{じあい}}{\text{自愛}}}$ くださいませ。 \hfill\break
The season being what is, please take care of yourself so that you do not upset your health. }

\par{48. この ${\overset{\textnormal{おも}}{\text{想}}}$ いが ${\overset{\textnormal{き}}{\text{消}}}$ えぬように、 ${\overset{\textnormal{なが}}{\text{流}}}$ れ ${\overset{\textnormal{ぼし}}{\text{星}}}$ に ${\overset{\textnormal{ねが}}{\text{願}}}$ うよ。 \hfill\break
I wish to a shooting star so that this hope may not die. }

\par{49. ${\overset{\textnormal{おな}}{\text{同}}}$ じような ${\overset{\textnormal{じこ}}{\text{事故}}}$ が ${\overset{\textnormal{お}}{\text{起}}}$ きんようにしっかりと ${\overset{\textnormal{がんば}}{\text{頑張}}}$ ってほしいな。 \hfill\break
I want for them to work firmly so that similar accidents don\textquotesingle t occur. }

\par{50. ${\overset{\textnormal{ライン}}{\text{LINE}}}$ の ${\overset{\textnormal{つうち}}{\text{通知}}}$ を ${\overset{\textnormal{こ}}{\text{来}}}$ ーへんようにしたよ。 \hfill\break
I made it so that LINE notifications don\textquotesingle t show up. }

\par{\textbf{Dialect Note }: こーへん = こない in Kansai Dialects. }

\par{\textbf{Curriculum Note }: One cliffhanger that remains from this lesson pertains the usages of ~ようになる and ~ようにする which fall outside their definitions presented here due to the simile definition of  ~ように. In Lesson ??? (which will be made within the coming month), we will direct specific attention to ~ようになる and ~ようにする, and in doing so we will learn more about how the simile ~ように and the ambition ~ように are interrelated. }
    