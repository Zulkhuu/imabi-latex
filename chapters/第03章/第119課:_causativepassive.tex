    
\chapter{Causative-Passive}

\begin{center}
\begin{Large}
第119課: Causative-Passive 
\end{Large}
\end{center}
 
\par{ Adding the causative and passive forms together creates the 使役受け身形. As the name suggests, the ordering of the endings is causative then passive. Because there is a short causative form, there is a short causative-passive form. So, we will need to look into detail on that. Also, because the causative can either be interpreted as make and let depending on context (remember that natives often do not sense a difference between the two concepts), the causative-passive reflects this as well. }
      
\section{Causative-Passive}
 
\par{ The 受身形 (passive form) and the 使役形 may be put together by combining ~させる・せる and ~られる together. }

\begin{ltabulary}{|P|P|P|P|}
\hline 

Class & Verb & 動詞 & 使役受身形 \\ \cline{1-4}

一段 & To wear & 着る & 着させられる \\ \cline{1-4}

一段 & To close & 閉める & 閉めさせられる \\ \cline{1-4}

五段 & To eat (俗語) & 食う & 食わせられる \\ \cline{1-4}

五段 & To scratch & 掻く & 掻かせられる \\ \cline{1-4}

五段 & To sniff & 嗅ぐ & 嗅がせられる \\ \cline{1-4}

五段 & To pierce\slash sting & 刺す & 刺させられる \\ \cline{1-4}

五段 & To shoot & 撃つ & 撃たせられる \\ \cline{1-4}

五段 & To die & 死ぬ & 死なせられる \\ \cline{1-4}

五段 & To jump & 跳ぶ & 跳ばせられる \\ \cline{1-4}

五段 & To carve & 刻む & 刻ませられる \\ \cline{1-4}

五段 & To print & 刷る & 刷らせられる \\ \cline{1-4}

\end{ltabulary}

\par{ To begin our grammar analysis, let's consider the particle differences to expect between the passive, causative, and causative-passive conjugations. Take note of the first three very similar example sentences demonstrating this point. }

\par{ The particle に marks the forcer and the person made to do the action is simply the subject of the sentence. Notice the differences in translation and particle usages in the following. }

\par{1. 先生 \textbf{に }発音を直された。(Passive) \hfill\break
My pronunciation got corrected by the teacher. }

\par{\textbf{Sentence Note }: 先生 is the doer of the correction; the pronunciation is the direct object of what was corrected. }

\par{2. \textbf{人に }ほかの人の発音を直させる。(Causative) \hfill\break
To make a person correct someone else's pronunciation. }

\par{\textbf{Sentence Note }: 人 is the person being made; you are the person making the person do so. }

\par{3. \textbf{先生に }ほかの学生の発音を直させられた。(Causative-passive) \hfill\break
I was made to correct the pronunciation of the other students by the teacher. }

\par{\textbf{Sentence Note }: The teacher is who forced you to correct the pronunciation of the other students. }

\par{ \textbf{Contracting the Causative-Passive Form }}

\par{Remember the short causative endings ~さす and ~す? For the latter, you can add the ending ~れる to create the short causative-passive. Because these endings are using the 未然形, some students mistakenly think that ~される is somehow the ending, which is not the case. However, using the legitimate short causative-passive form when you should be using the passive form because of confusing this as a longer passive form is most certainly incorrect. }

\par{ As we know, ~す only attaches to 五段 verbs. The short causative-passive form has another restriction. When a verb ending in す takes ~す, you use the 未然形 さ. Because ~す  will also be in its未然形 さ to take ~れる, two さ would be next to each other. So, you cannot say things like 話さされる. Therefore, only 五段 verbs not ending in す can have a short causative-passive form. The chart below illustrates the causative, short causative, causative-passive, and short causative-passive forms for all kinds of verbs. }

\begin{ltabulary}{|P|P|P|P|P|P|}
\hline 

活用 & 動詞 & 使役形 & 使役縮約形 & 使役受身形 & 使役受身縮約形 \\ \cline{1-6}

五段カ行 & 描く & 描かせる & 描かす & 描かせられる & 描かされる \\ \cline{1-6}

五段ガ行 & 泳ぐ & 泳がせる & 泳がす & 泳がせられる & 泳がされる \\ \cline{1-6}

五段タ行 & 待つ & 待たせる & 待たす & 待たせられる & 待たされる \\ \cline{1-6}

五段ナ行 & 死ぬ & 死なせる & 死なす & 死なせられる & 死なされる △ \\ \cline{1-6}

五段バ行 & 遊ぶ & 遊ばせる & 遊ばす & 遊ばせられる & 遊ばされる \\ \cline{1-6}

五段マ行 & 読む & 読ませる & 読ます & 読ませられる & 読まされる \\ \cline{1-6}

五段ラ行 & 帰る & 帰らせる & 帰らす & 帰らせられる & 帰らされる \\ \cline{1-6}

上一段 & 見る & 見させる & 見さす (△) & 見させられる & 見さされる X \\ \cline{1-6}

下一段 & 食べる & 食べさせる & 食べさす (△) & 食べさせられる & 食べさされる X \\ \cline{1-6}

サ変 & する & させる & さす (△) & させられる & さされる X \\ \cline{1-6}

カ変 & 来る & 来させる & 来さす (△) & 来させられる & 来さされる X \\ \cline{1-6}

五段サ行 & 話 \textbf{す }& 話させる & 話さす & 話せられる & 話さされる X \\ \cline{1-6}

\end{ltabulary}

\par{ The final three short causative-passives being wrong is not surprising. After all, two さ is unnatural and is resultant of using this shortcut with the wrong kind of verb. Surprisingly, though, one combination was given a △. This means that 死なされる is not common Japanese and is usually avoided. For instance, say you have 本人の意思で医者の手により死なされた. This is odd considering the antonym 生かされる exists. This is because rephrasing with 死なせる is more common\slash natural. Thus, 死なされる isn't grammatically faulty but hardly ever used and thus usually unnatural. }

\par{ ~す should only be used with 五段 verbs. However, you often hear things like 食べさして, although this sort of speech sounds dialectical. You could view this as a corruption of 食べさせて or a use of the short ~す in the て形. Thus, △ should actually be X in terms of 標準語. }

\begin{center}
\textbf{Examples }
\end{center}

\par{4. 兄が母に部屋を掃除させられました。 \hfill\break
My brother was made by his mother to clean the room. }

\par{5. 学生は先生に漢字を練習させられる。 \hfill\break
The students are made to practice Kanji by the teacher. }

\par{6. 兄にクッキーをすべて食べさせられた。 \hfill\break
I was made to eat all of the cookies by my big brother. }

\par{7. ガールフレンドに待たされた。 \hfill\break
I was made to wait by my girlfriend. }

\par{8. ${\overset{\textnormal{どれい}}{\text{奴隷}}}$ は毎日ただで働かせられていたということです。 \hfill\break
The slaves were made to work for no pay every day. }

\par{9. 僕は子供の時、母親にニンジンを食べさせられなかったから、今でも大嫌いだよ。 \hfill\break
Since I wasn't forced to eat carrots by my mom when I was a kid, I still hate them. }

\par{10. 熱に浮かされる。(Idiomatic) \hfill\break
To be in a craze. }

\par{11. 彼は学校をやめさせられた。 \hfill\break
He was expelled from school. }

\begin{center}
 \textbf{X makes Y get\dothyp{}\dothyp{}\dothyp{}by Z }
\end{center}

\par{ In English, we can say the following sentences. }

\begin{itemize}

\item The apple was eaten by the student. (Passive) 
\item The teacher made the student eat the apple. (Causative) 
\item The student was made to eat the apple by the teacher. (Passive-causative) 
\item The teacher made the apple get eaten by the student. (Causative-Passive) 
\end{itemize}
 In Japanese, there is a fixed ordering of causative and passive. They go in that order, which is why we have させられる and not られさせる. How, then, can we have Japanese mimic the final possibility? The labels are for English, not Japanese. So, although we can easily translate the third sentence by keeping the word order as is minus "was made" being flipped, the fourth is exceptionally difficult.   The first worry we have is if we try to translate 4 directly, we'll get a literally translated expression that sounds foreign. This means unnatural speech, which is not what we want. The following two sentences would be the options a native speaker would suggest for 4. Contextual details are meant to make up for any lack of meaning in 4. 
\par{12. 学生は先生にリンゴを食べさせられた。◎ }

\par{13. 先生はリンゴを学生に食べさせた。 ◎ }

\par{ It appears already that a direct reflection of 4 is impossible. Consider, though, the following which relies on introducing a different word to the equation: 仕向ける. }

\par{14. 先生はリンゴを生徒に食べさせるように仕向けた。〇 But not ◎ }

\par{15. 先生はリンゴを生徒に食べられるように仕向けた。X }

\par{16. 先生はリンゴが生徒に食べられるようにした。 X }

\par{17. 先生は、リンゴが生徒に食べられるように仕向けた。 X }

\par{ ~られるように仕向ける and ~させるように仕向ける both exist, but the former is incorrect in this context as the speaker is not being affected. This is because there is will involved in making something be acted upon. Now, we'll see how ~させるように仕向ける can go from 〇 to △ because of what 仕向ける means and not because of the existence of the phrase. }

\par{ 仕向ける is working up someone to doing something (tempting\slash inducing) and it's a lot more work than just a simple scene of a witch giving an apple, eat, and then done. Even if the witch was pushy at that moment, the work behind for 仕向ける situations would be a mismatch. With this in mind, consider the following. }

\par{18. ${\overset{\textnormal{まじょ}}{\text{魔女}}}$ は ${\overset{\textnormal{しらゆき}}{\text{白雪}}}$ ${\overset{\textnormal{ひめ}}{\text{姫}}}$ に ${\overset{\textnormal{どく}}{\text{毒}}}$ リンゴを食べさせるように仕向けた。 (ちょっと変) }

\par{19. 魔女は白雪姫が毒リンゴを食べるよう(に)仕向けた。 }

\par{20. 魔女は白雪姫に毒リンゴを ${\overset{\textnormal{わた}}{\text{渡}}}$ し、食べるように言った。(一番自然) }

\par{ Japanese may not allow you to explicitly say what you can in English with "the teacher made the apple get eaten by a student", but you can supplement this semantic information with context. This is what we have to do when languages don't meet eye to eye. }

\begin{center}
 \textbf{Causative + Potential }
\end{center}

\par{  It just so happens that ~られる also has the "potential" usage, meaning that what looks like the "causative-passive" may in fact be the "causative potential." Because the particles would differ, there usually isn't ever confusion as tow hat is meant. Note that the abbreviated causative cannot be used with this because す \textrightarrow  せる would simply revert them back to the non-abbreviated causative form. However, both the short and long causative forms can be followed by ~ことができる. }

\par{21. 私は犬にえさを食べさせられますよ。 \hfill\break
I can feed the dog. }
    