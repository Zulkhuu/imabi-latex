    
\chapter{ニ格 VS カラ格}

\begin{center}
\begin{Large}
第117課: ニ格 VS カラ格 
\end{Large}
\end{center}
 
\par{ 格 refers to grammatical case. Case marking is used to state the role of individual phrases (arguments) in the sentence. Many different particles overlap each other, and に and から are no different. In this lesson, we'll learn when they are interchangeable with each other and how they differ in those situations. }
      
\section{に VS から}
 
\par{ There are three kinds of expressions in which we find に and から interchangeable. }

\par{1. ~てもらう \hfill\break
2. Passive phrases \hfill\break
3. 借りる, もらう, 教える, 聞く, etc. }

\par{ Differentiating between に and から can be done several ways, but it's best to keep all factors in mind. }

\begin{center}
\textbf{Starting Point }
\end{center}

\par{ The first thing to consider is whether the agent (doer) of the action is the same thing as the starting point of a transfer of some kind. For physical borrowing or exchange of something via favor, that makes sense because the giver is where the transaction begins. }

\par{1. 友里はセス\{に・から\}英語を教えてもらった。 \hfill\break
Yuri had Seth teach her English. }

\par{2. 智哉がセス\{に・から\}科学の教科書を借りた。 \hfill\break
Tomoya borrowed a\slash the science textbook from Seth. }

\par{ In these sentences, we can see how the agent of the action is the same as the starting point of a transfer. In Ex. 1, the person giving knowledge of English is Seth, and the receiver is Yuri. In Example 2, Seth is the one actually giving the book and Tomoya is the end point of the physical transfer. Not all "transfers" are physical, however. }

\par{3. 百華は上司から褒められたよ。 \hfill\break
Momoka was praised by her boss. }

\begin{center}
 \textbf{Only に }
\end{center}

\par{ Let's assume, then, that the actual motivation for when you can use both has to deal with the kind of recipient. The agent would still be the person doing the action and then we would have direct and indirect recipients possible. We see that if there is only an indirect recipient in the sentence, から cannot be used. }

\par{4. 秀晃は光太郎\{に 〇・から X\}パーティーに来てもらった。 \hfill\break
Hidemitsu had Kotaro come to the party. }

\par{5. 有紀子は時計屋\{に 〇・から X\}時計を直してもらった。 \hfill\break
Yukiko had a watch fixed by a watch shop. }

\par{ Furthermore, if there is only a direct recipient, に can only be used if that recipient has a change in its condition. The recipient in this case also happens to be the subject. }

\par{6. 私はきのう、医者\{に 〇・から X\}診てもらった。 \hfill\break
I had the doctor see me yesterday. }

\par{7. 私は去年、愛犬\{に 〇・から X\}死なれました。 \hfill\break
Literally: I was died on by my beloved dog last year. }

\begin{center}
\textbf{When Interchangeable }
\end{center}

\par{ When the subject is receiving a direct effect from an agent but no condition change occurs, both particles may be used. }

\par{8. 私は親切な人\{に・から\}席を譲ってもらいました。 \hfill\break
I got a nice person to lend me his\slash her seat\slash I had a seat given to me from a nice person. }

\par{9. 喜久子は潤子\{に・から\}お金を出してもらった。 \hfill\break
Kikuko had Junko give her money\slash Kikuko had money given to her by Junko. }

\par{ So long as the direct recipient doesn't undergo a state change, there could be an indirect recipient in the sentence. Another dynamic that makes から more likely is if the agent and the direct recipient are both marked by に, then から would appear more frequently to prevent the doubling of the same particle. This does not mean not doing so is grammatically incorrect. }

\par{10. 藤川さんは担任教師\{に・から\}息子を叱ってもらった。 \hfill\break
Fujikawa had his son scolded by the homeroom teacher. }

\par{11. 悠子は級友\{から・に\}代表に選ばれました。 \hfill\break
Yuuko was elected as representative by her classmates. }

\par{12. リーさんは友達\{に・から\}笑われた。 \hfill\break
Lee-san was laughed at by his friend(s). }

\par{13. 謙太は俊太朗\{から・に\}小田原先生にそのメッセージを伝えてもらった。 \hfill\break
Kenta had the message sent to Odawara-Sensei by Shuntaro. }

\begin{center}
\textbf{Only から? }
\end{center}

\par{  If there are situations when you can only use に, there must also be situations when you can only use から. An agent of a transfer can always be marked with に. If no agent is even used at all, then に would not be possible. }

\par{14. セスは図書館から本を借りてもらった。 \hfill\break
Seth had a book borrowed for him from the library. }

\par{ The only known giving-receiving verb pair that does not take に is 預かる. In this case, you must use から. However, this appears to be the only case in which から must be used. }

\par{15. 有希から荷物を預かった。 \hfill\break
My luggage was entrusted with Yuki. }

\par{ If we look at the meaning of this verb pair more closely, we see that the action is only a one-way deal. In other words, when you are entrusted with something, you don't have any way of definitively making that be the case (for the sake of the use of these words). In a few of the sentences earlier, we saw how this sort of nuance splitting could be done with に vs. から (though for many utterances no difference is typically felt when interchangeable). }

\begin{center}
\textbf{Putting Verb and Particle Meaning Together }
\end{center}

\par{ Nevertheless, it is certain that から is the only particle that can imply receiving something or an action without personal want. With this in mind, the meanings of these verbs involving giving and receiving combined with what you know about these particles now determine which one you use.  }

\par{16. 誰から聞いたの? \hfill\break
Who did you hear that from? }

\par{17. 誰に聞いたの? \hfill\break
Who did you ask? }

\par{18. 智哉はセス\{に 〇・ から X\}古典の辞書を貸した。 \hfill\break
Tomoya lent Seth a classics' dictionary. }

\par{19. 花束が先生に贈られました。 \hfill\break
A bouquet was presented to the teacher. }

\par{20. 先生からプレゼントを贈っていただきました。 \hfill\break
I received a present from the teacher. }

\par{21. そのメダルは藤原から杉村に贈られました。 \hfill\break
The medal was presented to Sugimura by Fujiwara. }

\par{22. セスは智哉\{に 〇・から X\}古典の辞書を借りた。 \hfill\break
Seth borrowed a classics' dictionary from Tomoya.  }

\par{\textbf{Particle Note }: For physical borrowing, we don't see much nuance splitting at all, and it is not possible to say that speaker want is not present in borrowing when you use から. So the pairing of it and 借りる essentially overrides から\textquotesingle s possible role of inferring non-volitional receiving. It is important to note, over all, though, that when both particles are possible, に is typically the most common. }

\par{\textbf{参照 }: http:\slash \slash ir.nul.nagoya-u.ac.jp\slash jspui\slash bitstream\slash 2237\slash 5645\slash 1\slash BZ001907107.pdf }
    