    
\chapter{Which}

\begin{center}
\begin{Large}
第132課: Which: どれ, どちら, どっち, \& いずれ 
\end{Large}
\end{center}
 
\par{ In this lesson, we will discover how to say the question word “which” in Japanese. This lesson will be about the “which” used in questions. Although “which” may at times correspond to other things in Japanese, the words you will learn in this lesson cover the fundamental meaning of “which.” }
      
\section{Which 'Which'?}
 
\par{ There are three primary words that mean “which” in Japanese: どれ, どちら, and どっち. Generally speaking, they are distinguished in the following manner. }

\par{どれ: Used for when pointing out from three or more things. \hfill\break
どちら: Used when pointing out from two things. \hfill\break
どっち: Used when casually point out from two things. }

\par{1. ${\overset{\textnormal{かがわ}}{\text{香川}}}$ さんの ${\overset{\textnormal{かばん}}{\text{鞄}}}$ はどれですか。 \hfill\break
Which is Mr. Kagawa\textquotesingle s bag\slash briefcase? }

\par{\textbf{Sentence Note }: In Ex. 1, there are most likely more than two bags\slash briefcases present. }

\par{2. どちらが ${\overset{\textnormal{ほ}}{\text{欲}}}$ しいですか。 \hfill\break
Which do you want? }

\par{\textbf{Sentence Note }: In Ex. 2, only two options are likely being offered. }

\par{3. 日本語を勉強し始める前は、どちらが中国語の教科書、日本語の教科書かが分からなかったでしょう。 \hfill\break
Before studying Japanese, I probably wouldn\textquotesingle t have figured out which is the Chinese textbook and which is the Japanese textbook. }

\par{4. この ${\overset{\textnormal{かいわ}}{\text{会話}}}$ では、どっちがいいか ${\overset{\textnormal{わ}}{\text{分}}}$ かりません。 \hfill\break
I don\textquotesingle t know which to use in this conversation. }

\par{\textbf{Sentence Note }: Although どっち is used colloquially, it still works well in polite speech. However, in formal situations, it gets completely replaced with どちら. }

\par{5. アルコールの ${\overset{\textnormal{よわ}}{\text{弱}}}$ いお ${\overset{\textnormal{さけ}}{\text{酒}}}$ はどれですか。 \hfill\break
Which drinks are low in alcohol? }

\par{ What we have seen thus far are the forms for “which” when used as a noun. When used adjectivally, you will need to change どれ to どの and add the particle の to どちら and どっち. }

\par{6. どの ${\overset{\textnormal{みち}}{\text{道}}}$ を ${\overset{\textnormal{えら}}{\text{選}}}$ ぶか ${\overset{\textnormal{まよ}}{\text{迷}}}$ っています。 \hfill\break
I\textquotesingle m wavering on which path to choose. }

\par{7. \{どの・どちらの\} ${\overset{\textnormal{としょかん}}{\text{図書館}}}$ に ${\overset{\textnormal{へんきゃく}}{\text{返却}}}$ してもいいんでしょうか。 \hfill\break
Which library would it be alright to return (it) to? }

\par{8. どの ${\overset{\textnormal{えき}}{\text{駅}}}$ で ${\overset{\textnormal{お}}{\text{降}}}$ りればいいですか。 \hfill\break
Which station should I get off at? }

\par{9. どのチームを ${\overset{\textnormal{おうえん}}{\text{応援}}}$ してるの? \hfill\break
Which team do you support? }

\par{10. どの ${\overset{\textnormal{かもく}}{\text{科目}}}$ を ${\overset{\textnormal{せんたく}}{\text{選択}}}$ するかとても ${\overset{\textnormal{なや}}{\text{悩}}}$ んでます。 \hfill\break
I\textquotesingle m really worried about which subjects to choose. }

\par{11. ( ${\overset{\textnormal{わたし}}{\text{私}}}$ には)どの ${\overset{\textnormal{いろ}}{\text{色}}}$ が ${\overset{\textnormal{いちばんにあ}}{\text{一番似合}}}$ うと ${\overset{\textnormal{おも}}{\text{思}}}$ います? \hfill\break
Which color(s) do you think fits me the best? }

\par{12. ${\overset{\textnormal{けっきょく}}{\text{結局}}}$ 、どれにすればいいの? \hfill\break
In the end, which should I go with? }

\par{13. どれにしようか ${\overset{\textnormal{まよ}}{\text{迷}}}$ っています。 \hfill\break
I\textquotesingle m wavering on which to go with. }

\par{\textbf{Grammar Note }: しよう is the volitional form of する (to do). If you were to remove what\textquotesingle s after か from the sentence, you would get a sentence that equates to “which shall I go with?” }

\par{\emph{ }どれの is seldom possible, but it does exist. For instance, it can be seen in どれのこと. In this case, のこと is added to emphasize ‘what\textquotesingle  is being discussed. }

\par{14. ログインIDってどれのことを ${\overset{\textnormal{さ}}{\text{指}}}$ すんですか。 \hfill\break
Which thing does login ID indicate? }

\par{15. どれのことを ${\overset{\textnormal{い}}{\text{言}}}$ っているのかわかりません。 \hfill\break
I don\textquotesingle t know which thing (he\slash she) is talking about. }

\par{ Aside from the truly basic sentences we\textquotesingle ve seen thus far, somewhat more grammar needs to be introduced when listing the actual things in a sentence with “which.” Because the underlying grammar is the same across the three ‘which\textquotesingle  words, we\textquotesingle ll use the English word ‘which\textquotesingle  in discussing the grammar patterns used with them. }

\par{ The basic grammatical pattern used when listing the actual things ‘which\textquotesingle  may refer to is “ \emph{X }と \emph{Y ( }と \emph{) }では.” When listing things with the particle と , it was once the case that Y was always marked by と. In making comparisons, this archaic grammar remains relatively used. Even when there are three or more elements, in which case (と)では would be after the final element, it is still often used. }

\par{16. ドメインは、.comと.net(と)ではどっちがいいですか。 \hfill\break
For a domain, between “.com” and “.net,” which is good? }

\par{17. ジョギングは、 ${\overset{\textnormal{あさ}}{\text{朝}}}$ と ${\overset{\textnormal{よる}}{\text{夜}}}$ ではどっちがいいの \hfill\break
As for jogging, which is best, morning or night? }

\par{\textbf{Sentence Note }: The lack of と before では is largely due to the sentence being more casual. After all, the use of と after the second element is optional and indicative of older grammar. }

\par{18. ${\overset{\textnormal{たいじゅう}}{\text{体重}}}$ も ${\overset{\textnormal{しんちょう}}{\text{身長}}}$ も ${\overset{\textnormal{あさ}}{\text{朝}}}$ と ${\overset{\textnormal{よる}}{\text{夜}}}$ (と)では ${\overset{\textnormal{かなら}}{\text{必}}}$ ず ${\overset{\textnormal{ちが}}{\text{違}}}$ います。 \hfill\break
Both one\textquotesingle s weight and one\textquotesingle s height always differ between day and night. }

\par{\textbf{Sentence Note }: Although “which” is not in the sentence, (と)では is still used to create the phrase “between…and…” }

\par{19. ( ${\overset{\textnormal{どうしつ}}{\text{同室}}}$ と) ${\overset{\textnormal{べっしつ}}{\text{別室}}}$ とどっちがいい? \hfill\break
(Between same room or) separate rooms, which is better? }

\par{\textbf{Sentence Note }: Whenever the first element is deemed to be obvious, it may be omitted from the sentence. In this example, the only thing that could reasonably be compared with別室 (separate room) is the phrase for “same room,” which is 同室. Though rather unrelated, also note that the “which” in “which is” is neither a noun nor an adjective, meaning it doesn\textquotesingle t correspond to the “which” phrases in this lesson. }

\par{ When listing three or more phrases, you can use \emph{~ }の中で(は)instead of (と)では. The difference in nuance by using the word 中 is that the translation will usually include “among” rather than “between.” }

\par{20. ${\overset{\textnormal{いぬ}}{\text{犬}}}$ と ${\overset{\textnormal{ねこ}}{\text{猫}}}$ と、ウサギ\{(と)・の ${\overset{\textnormal{なか}}{\text{中}}}$ \}では、どれが ${\overset{\textnormal{いちばんす}}{\text{一番好}}}$ きですか。 \hfill\break
Between\slash among dogs, cats, and rabbits, which do you like the most? }

\par{\textbf{Grammar Note }: The use of the particle は is to emphasize the choice being made. In effect, it highlights the pool of choices. In the next sentence, は is not used. The reason for this is because the four seasons are quite finite and there is no need to add special emphasis to the options at hand. However, adding は would not make the sentence odd or ungrammatical. }

\par{21. ${\overset{\textnormal{はる}}{\text{春}}}$ 、 ${\overset{\textnormal{なつ}}{\text{夏}}}$ 、 ${\overset{\textnormal{あき}}{\text{秋}}}$ 、 ${\overset{\textnormal{ふゆ}}{\text{冬}}}$ の ${\overset{\textnormal{なか}}{\text{中}}}$ で、\{どれ・どちら\}の ${\overset{\textnormal{きせつ}}{\text{季節}}}$ が ${\overset{\textnormal{いちばん}}{\text{一番}}}$ お ${\overset{\textnormal{す}}{\text{好}}}$ きですか。 \hfill\break
Among spring, summer, autumn, and winter, which season do you like the most? }

\par{\textbf{Grammar Note }: Although どちら is almost always used in situations in which there are only two things ‘which\textquotesingle  may refer to, it is sometimes treated as the formal form of どれ. This becomes apparent when clearly formal elements are introduced in the sentence. Here, we see that the prefix お~ is added to 好きだ (to like) to create an honorific expression. }

\par{22. モンストとパズドラってどっちが ${\overset{\textnormal{す}}{\text{好}}}$ き? \hfill\break
Between Monster Strike and Puzzle \& Dragon, which do you like? }

\par{\textbf{Grammar Note }: The second\slash last と may actually be seen as って. As we know, this is not the only instance in which と can be replaced by って. Its use here is to lessen the old-fashioned nature of marking the second element. }

\par{\textbf{Phrase Note }: Monster Strike and Puzzle \& Dragon are two of the most popular mobile app games in Japan. Colloquially, they are referred to by their contractions as is demonstrated in Ex. 22. }

\par{ The use of と to mark the second\slash last element, as mentioned, is grammatically unnecessary. This is also the case when one or both elements are nominalized phrases. When it\textquotesingle s missing, you\textquotesingle ll see that the nominalized phrase essentially connects with the ‘which\textquotesingle  word with the particle の. }

\par{23. ${\overset{\textnormal{たま}}{\text{玉}}}$ ねぎは、 ${\overset{\textnormal{なま}}{\text{生}}}$ と ${\overset{\textnormal{かねつ}}{\text{加熱}}}$ するの(と)、どっちがいいの? \hfill\break
As for onions, which is better, they being raw or heating them up? }

\par{24. ${\overset{\textnormal{いえ}}{\text{家}}}$ は、 ${\overset{\textnormal{う}}{\text{売}}}$ るのと ${\overset{\textnormal{か}}{\text{貸}}}$ すの(と)、どっちがいいの? \hfill\break
As for homes, between selling and leasing, which is better? }

\par{25. ${\overset{\textnormal{そ}}{\text{剃}}}$ るのと ${\overset{\textnormal{ぬ}}{\text{抜}}}$ くの(と)、どっちがいいの? \hfill\break
Between shaving and plucking, which is better? }

\par{ Another facet of grammar surrounding ‘which\textquotesingle  is the optional addition of \emph{~ }の方 to the sentence after a ‘which\textquotesingle  word to emphasize “which \emph{one }.” }

\par{\textbf{Orthography Note }: To avoid confusion with its other usages, 方 may be read as ほう. }

\par{26. ワインとビール\{(と)では・の\}どちら(のほう)が ${\overset{\textnormal{す}}{\text{好}}}$ きですか。 \hfill\break
Between wine and beer, which (one) do you like? }

\par{27. ${\overset{\textnormal{いま}}{\text{今}}}$ のところ、モンストとパズドラ(と)では、\{どっち・どちら\}のほうが ${\overset{\textnormal{にんき}}{\text{人気}}}$ があるんですか。 \hfill\break
Currently, between Monster Strike or Puzzle \& Dragons, which is more popular? }

\par{28. ${\overset{\textnormal{きつえんせき}}{\text{喫煙席}}}$ と ${\overset{\textnormal{きんえんせき}}{\text{禁煙席}}}$ 、どちらになさいますか。 \hfill\break
Which would you like, smoking or non-smoking? }

\par{ The options before the ‘which\textquotesingle  word may alternatively be listed with the particle か. In this case, か is usually stated after the second\slash last element, although it may be omitted if the following ‘which\textquotesingle  phrase sounds like a separate statement. Regarding difference in nuance, the sense of comparison is lessened to a listing of options which may not be exclusive which may also be emphatically based. }

\par{29. お ${\overset{\textnormal{はし}}{\text{箸}}}$ かフォークかどちらになさいますか。 \hfill\break
Will you go with chopsticks or fork? }

\par{\textbf{Sentence Note }: Although the general options may indeed just be chopsticks or forks, the option of neither is also implied. It is also open-ended enough for the customer to choose something else not explicitly mentioned. }

\par{30. ガチャって、 ${\overset{\textnormal{たんぱつ}}{\text{単発}}}$ か ${\overset{\textnormal{じゅう}}{\text{10}}}$ ${\overset{\textnormal{れん}}{\text{連}}}$ か、どっちがいい? \hfill\break
As for gacha, which is better, single shot or 10 shot? }

\par{\textbf{Sentence Note }: As mentioned, the sense of comparison is open-ended but is also more subjective and emphatically based. By posing the question this way, the speaker may expect follow-up questions regarding the nature of the options or personal experiences from others with said options. }

\par{\textbf{Word Note }: In mobile games, \emph{gacha }is an internal prize that you get, often by utilizing in-game currency. }

\par{31. ${\overset{\textnormal{あい}}{\text{愛}}}$ かお ${\overset{\textnormal{かね}}{\text{金}}}$ かどちらにしますか。 \hfill\break
Which will you go with, love or money? }

\par{\textbf{Sentence Note }: Ex. 31 is a perfect example for the subjective, emphatic nature of か. Clearly, there is more behind “love” and “money” alone, but it is this fact that also drives the emphasis behind the question itself. }

\par{32. モンストかパズドラ、どっちがおススメですか? \hfill\break
Monster Strike or Puzzle \& Dragon, which do you recommend? }

\par{33. ${\overset{\textnormal{ちんたい}}{\text{賃貸}}}$ か ${\overset{\textnormal{も}}{\text{持}}}$ ち ${\overset{\textnormal{いえ}}{\text{家}}}$ かどっちが ${\overset{\textnormal{とく}}{\text{得}}}$ かなど ${\overset{\textnormal{ひかく}}{\text{比較}}}$ してはいけません。 \hfill\break
You mustn\textquotesingle t compare on the lines of like whether renting or owning one\textquotesingle s home is the better bargain. }

\par{ In addition to what we\textquotesingle ve seen, it is also appropriate and very common for the second\slash last element of a sentence with ‘which\textquotesingle  to be followed by a conditional phrase. }

\par{34a. ${\overset{\textnormal{たいへいよう}}{\text{太平洋}}}$ と ${\overset{\textnormal{たいせいよう}}{\text{大西洋}}}$ だと、どちら(のほう)が ${\overset{\textnormal{おお}}{\text{大}}}$ きいのでしょうか。 \hfill\break
If it\textquotesingle s the Atlantic Ocean and Pacific Ocean, which (one) is larger? \hfill\break
34b. ${\overset{\textnormal{たいへいよう}}{\text{太平洋}}}$ と ${\overset{\textnormal{たいせいよう}}{\text{大西洋}}}$ (と)では、どちら(のほう)が ${\overset{\textnormal{おお}}{\text{大}}}$ きいのでしょうか。 \hfill\break
Between the Atlantic Ocean and the Pacific Ocean, which (one) is larger? }

\par{\textbf{Grammar Note }: The use of だと adds to the sense that a definitive comparison is being made. \hfill\break
 \hfill\break
35. ${\overset{\textnormal{けいえいしゃ}}{\text{経営者}}}$ と ${\overset{\textnormal{ろうどうしゃ}}{\text{労働者}}}$ なら、どちら(のほう)が ${\overset{\textnormal{よ}}{\text{良}}}$ いでしょうか。 \hfill\break
If it\textquotesingle s between being a manager and being a worker, which (one) is better? }

\par{\textbf{Grammar Note }: The use of the particle なら is used in order to ask for a suggestion. }

\par{37. ${\overset{\textnormal{くるま}}{\text{車}}}$ を ${\overset{\textnormal{か}}{\text{買}}}$ うなら、 ${\overset{\textnormal{しんしゃ}}{\text{新車}}}$ か ${\overset{\textnormal{ちゅうこしゃ}}{\text{中古車}}}$ かどっちがお ${\overset{\textnormal{とく}}{\text{得}}}$ なんですか。 \hfill\break
If you\textquotesingle re buying a new car, which is a better bargain, a new car or a used car? \hfill\break
 \hfill\break
38. ${\overset{\textnormal{くるま}}{\text{車}}}$ を ${\overset{\textnormal{こうにゅう}}{\text{購入}}}$ する ${\overset{\textnormal{ばあい}}{\text{場合}}}$ は、 ${\overset{\textnormal{しんしゃ}}{\text{新車}}}$ と ${\overset{\textnormal{ちゅうこしゃ}}{\text{中古車}}}$ (と)では、どちらがお ${\overset{\textnormal{とく}}{\text{得}}}$ なんでしょうか。 \hfill\break
In the case of purchasing a vehicle, which is the better bargain, a new car or a used car? }

\par{\textbf{Sentence Note }: Ex. 38 is a more formal version of Ex. 37. As you can see, the phrase \emph{~ }場合は is used to mean “in the case…” }

\begin{center}
\textbf{Deceiving Translations from English }
\end{center}

\par{ At times, the use of the word "which" doesn't correspond to either of the 'which' phrases discussed in this lesson. However, it is almost always the case that the English can be paraphrased into something else, and it will be that something else that translates smoothly into Japanese. }

\par{39. ${\overset{\textnormal{しんしふくう}}{\text{紳士服売}}}$ り ${\overset{\textnormal{ば}}{\text{場}}}$ は何階ですか。 \hfill\break
Which\slash what floor is men\textquotesingle s clothing? \hfill\break
 \hfill\break
\textbf{Word Note }: The opposite of “men\textquotesingle s clothing” is “women\textquotesingle s clothing,” which is 婦人服. }

\par{40. \{どこの・どの\} ${\overset{\textnormal{とし}}{\text{都市}}}$ に ${\overset{\textnormal{す}}{\text{住}}}$ んでみたいですか。 \hfill\break
What\slash which city would you like to live? }

\begin{center}
\textbf{\emph{Izure }いずれ } 
\end{center}

\par{ Another word meaning “which” is いずれ. It is the original form of ‘which\textquotesingle  that became どれ over time. It, though, remains as “which,” often in set phrases like いずれにしても, meaning “at any rate\slash in any case.” You will also see it frequently used as an adverb meaning “sooner or later.” }

\par{41. ${\overset{\textnormal{かのじょ}}{\text{彼女}}}$ はいずれ ${\overset{\textnormal{わす}}{\text{忘}}}$ れるでしょう。 \hfill\break
She will surely forget sooner or later. }

\par{42. ${\overset{\textnormal{しけいはいし}}{\text{死刑廃止}}}$ 、 ${\overset{\textnormal{そんぞく}}{\text{存続}}}$ 、いずれにしても ${\overset{\textnormal{むずか}}{\text{難}}}$ しい ${\overset{\textnormal{もんだい}}{\text{問題}}}$ なのだ。 \hfill\break
Abolishing the death penalty or continuing with it, at any rate, it is a difficult question. }
    