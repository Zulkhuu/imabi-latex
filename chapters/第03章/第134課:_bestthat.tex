    
\chapter{Best That}

\begin{center}
\begin{Large}
第134課: Best That: ~方 Patterns \& ~に越したことはない 
\end{Large}
\end{center}
 
\par{ For some reason, students are compelled to use the word 方 excessively. }
\textbf{}      
\section{Best\slash Better That}
 
\par{ 方がいい means “best to…” and with verbs, you should use the past tense, and with adjectives or the negative, you should use the non-past tense. However, this can still be used with the non-past of verbs, and いい can be replaced with phrases like 賢明だ (wise). \hfill\break
}
 
\par{ ~方がいい is used in the written and spoken language. It is often used in daily conversation, but it is often used by Japanese learners when giving suggestions\slash advice, which may lead some Japanese to smirk. }

\par{ Non-past +方がいい is primarily used with one\textquotesingle s in-group as it is used a lot in situations where you are giving direct advice on what is best to do. Past+方がいい is felt to be more euphemistic and is often used towards one\textquotesingle s out-group. However, some people have personal preferences for one or the other. }
 
\par{1. 今すぐ行った方がいいです。 \hfill\break
It'd be best to go right now. }
 
\par{2. 寝た方がいい。 \hfill\break
It's best to sleep. }
 
\par{3. よく考えた方がいい。 \hfill\break
You'd better think that over. }
 
\par{4. 泳ぐ前に、 ${\overset{\textnormal{ひや}}{\text{日焼}}}$ け ${\overset{\textnormal{ど}}{\text{止}}}$ めを ${\overset{\textnormal{ぬ}}{\text{塗}}}$ った方がいいよ。 \hfill\break
It's best to put on sunblock before swimming. }

\par{5. ${\overset{\textnormal{たまご}}{\text{卵}}}$ は食べない方がいい。 (Contrasting and or highlighting) \hfill\break
It's best to not eat eggs. }
 
\par{6. \{ほとんどの\}わずかな ${\overset{\textnormal{みせ}}{\text{店}}}$ しか開いていないから、今食品を買った方がいい。 \hfill\break
Since there are only but a few stores that are open, it's best to buy food now. }
 
\par{7. 彼はそこへ行った方がよい。 \hfill\break
It is best for him to go there. }
 
\par{8. 毎日運動した方がいい。 \hfill\break
It's best to exercise every day. }
 
\par{\textbf{Nuance Note }: Even when there is not a second reference mentioned, this pattern implies a comparison between things. It may sound pushy depending on the context as it can be used to impose your will\slash thought. When you want to make a simple suggestion, you can use ~たらいい. }
 
\par{ Again, other things than just いい may also appear in this fashion. }
 
\par{9. 当人に会って直接に話し合った方が ${\overset{\textnormal{こうか}}{\text{効果}}}$ はあると思います。 \hfill\break
I think that it is effective to meet with the said person and directly talk together (about it). }
 
\par{\textbf{Non-Past + }\textbf{方がいい }}
 
\par{ It is also worth noting that when you use the pattern "Non-past+方がいい", there could be several things that are good to do in the said situation, but you pick one as an example of being good. }
 
\par{10. 折りにふれて ${\overset{\textnormal{そうき}}{\text{想起}}}$ するほうがいい。 \hfill\break
Occasionally, it is good to call it to mind. }
 
\par{\textbf{~がよい }}
 
\par{ ~がよい can create a suggestion bordering on a command, and it uses older grammar by just directly following verbs without the need of a nominalizer like の. }
 
\par{11. そうするがよい。 \hfill\break
You had better do it. }
 
\begin{center}
 \textbf{~方: How to\dothyp{}\dothyp{}\dothyp{} }
\end{center}

\par{ Following the stem of a verb, ~ ${\overset{\textnormal{かた}}{\text{方}}}$ means "how to\dothyp{}\dothyp{}\dothyp{}". Sometimes this may not show up in translation. For instance, 読み方 may very well be able to be translated as "how to read," but it can also be sufficiently translated as "reading" depending on context. }

\par{12. 弟の笑い方が気に食わねー。(Slang) \hfill\break
I can't stand the way my little brother smiles. }

\par{13. 日本語の書き方をよく知っていますか。 \hfill\break
Do you know how to write Japanese well? }
14. 行き方が全く分からない。 \hfill\break
I don't know the way to go at all. 
\par{15. どの読み方を使うべきかは分かりません。 \hfill\break
I don't know which reading I'm supposed to use. }

\par{16. 本当に ${\overset{\textnormal{ひど}}{\text{酷}}}$ い言い方だよ。 \hfill\break
That's a hideous way to talk. }

\par{17. しかたがありません。(Set Phrase) \hfill\break
It can't be helped. }

\par{18. 簡単な覚え方を教えてくださいませんか。 \hfill\break
Could you teach me an easy way to remember? }
      
\section{~に越したことはないが}
 
\par{ ~に ${\overset{\textnormal{こ}}{\text{越}}}$ したことはない(が) means "can never\dothyp{}\dothyp{}\dothyp{}too much" and is used in giving common sense advice. The verb 越す has meanings all related to the idea of "surpassing". This phrase is used to show that nothing beats doing X.  So, when young people use it in certain situations, it can make them sound arrogant. }

\par{ If turned into ~に越したことはなかった, you end up showing regret. This is because you recognize what was the best thing to do, but you didn't do it. This directly follows verbs and adjectives, but for adjectival nouns, the copula is either as である or not there at all (which is most common). }

\par{19. ${\overset{\textnormal{かさ}}{\text{傘}}}$ を持つに越したことはない。 \hfill\break
It's always best to have an umbrella. }

\par{20. 貯金をするに越したことはなかったが、ちょっと生活を楽しむことができた。 \hfill\break
Nothing would have beat saving, but I got to enjoy life a little. }

\par{21. ${\overset{\textnormal{ようじん}}{\text{用心}}}$ に越したことはないが、そんな服装を着るなんて ${\overset{\textnormal{はで}}{\text{派手}}}$ すぎるじゃない? \hfill\break
You can never be too careful, but isn't wearing an outfit like that too showy? }
    