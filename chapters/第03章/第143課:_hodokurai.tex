    
\chapter{The Particles ほど \& くらい}

\begin{center}
\begin{Large}
第143課: The Particles ほど \& くらい 
\end{Large}
\end{center}
 
\par{ Although very similar, ほど and くらい―also ぐらい―are not always the same. }
      
\section{The Adverbial Particle ほど}
 
\par{ The first thing that you should know about the particle ほど is how it is attached to various parts of speech and what sort of particle combinations are allowed. }

\begin{ltabulary}{|P|P|P|}
\hline 

Class & Pattern & Example \\ \cline{1-3}

名詞 & N+ほど & 日本ほど \\ \cline{1-3}

形容詞 & 連体形+ほど & できな \textbf{い }ほど \\ \cline{1-3}

形容動詞 & 連体形+ほど & 楽 \textbf{な }ほど \\ \cline{1-3}

動詞 & 連体形+ほど & 考え \textbf{る }ほど \\ \cline{1-3}

\end{ltabulary}

\par{ As far as particles are concerned, the typical order is noun + case particle + ほど. However, が・を are deleted. If they are to be used, they are used right after ほど. However, the particles such as に, で, and から are always after it. }

\begin{center}
\textbf{Approximation } 
\end{center}

\par{ After a \emph{quantity }, ほど shows an \textbf{approximation }. This usage is interchangeable with くらい・ぐらい. }

\par{${\overset{\textnormal{}}{\text{1. 9年間}}}$ ほど \hfill\break
About nine years }

\par{${\overset{\textnormal{}}{\text{2. 旅}}}$ は10 ${\overset{\textnormal{}}{\text{時間}}}$ ほどかかるでしょう。 \hfill\break
The trip will probably take about ten hours. }

\par{3. そのロープは ${\overset{\textnormal{}}{\text{長}}}$ さが ${\overset{\textnormal{}}{\text{六}}}$ メートルほどあります。 \hfill\break
The rope is about six meters long. }

\par{4. することが ${\overset{\textnormal{}}{\text{山}}}$ ほどあるよ。 \hfill\break
I have a pile of work to do. \hfill\break
Literally: I have a mountain of stuff to do. }

\par{5. あと1 ${\overset{\textnormal{}}{\text{ヶ}}}$ ${\overset{\textnormal{}}{\text{月}}}$ ほどで ${\overset{\textnormal{}}{\text{夏休}}}$ みになる。 \hfill\break
I'll be on summer vacation in about a month. }

\par{6. この ${\overset{\textnormal{}}{\text{仕事}}}$ はあと2 ${\overset{\textnormal{}}{\text{週間}}}$ ほどあれば ${\overset{\textnormal{}}{\text{出来上}}}$ がりますか。 \hfill\break
Will this job get done in around two weeks? }

\par{${\overset{\textnormal{}}{\text{7. \{細石・小石\}}}}$ ほどの ${\overset{\textnormal{}}{\text{大}}}$ きさの ${\overset{\textnormal{}}{\text{石}}}$ \hfill\break
A stone the size of a pebble }

\par{\textbf{Reading \& Meaning Note }: 細石 = さざれいし. This is smaller than a 小石. However, in English you use the same word. }

\par{8a. あの ${\overset{\textnormal{}}{\text{橋}}}$ はどれ\{ほど・くらい\} ${\overset{\textnormal{}}{\text{長い}}}$ ${\overset{\textnormal{}}{\text{ん}}}$ ですか。(More natural) \hfill\break
8b. あの橋の長さはどれくらいですか。 \hfill\break
How long is that bridge over there? }

\par{9. お(お)よそ50人(ほど)の人 \hfill\break
Around fifty people }

\par{\textbf{Word Note }: お(お)よそ = Approximately }

\begin{center}
 \textbf{Extreme Example }
\end{center}
 
\par{It may also show an \textbf{extreme example }. It takes a specific situation and evaluates its extent. So, it is often translated as "to the extent that". It may be seen after nominal or verbal expressions. Of course, idiomatic translation may very well blur this literal definition. }
 
\par{${\overset{\textnormal{}}{\text{10. 言葉}}}$ にできないほど ${\overset{\textnormal{}}{\text{素晴}}}$ らしい。 \hfill\break
It's too wonderful for words. }
 
\par{${\overset{\textnormal{}}{\text{11. 彼}}}$ は ${\overset{\textnormal{}}{\text{歩}}}$ けないほど ${\overset{\textnormal{}}{\text{弱}}}$ ってはいなかった。 \hfill\break
He wasn't weak enough to not walk. }
 
\par{${\overset{\textnormal{}}{\text{12. 彼}}}$ を ${\overset{\textnormal{}}{\text{見}}}$ てるのが ${\overset{\textnormal{}}{\text{気}}}$ の ${\overset{\textnormal{}}{\text{毒}}}$ なほど ${\overset{\textnormal{しょげ}}{\text{悄気}}}$ てる。(Casual) \hfill\break
Just seeing him makes me feel bad. }

\par{${\overset{\textnormal{}}{\text{13. 彼}}}$ はそれを ${\overset{\textnormal{}}{\text{試}}}$ みるほど ${\overset{\textnormal{ゆうかん}}{\text{勇敢}}}$ な ${\overset{\textnormal{}}{\text{兵士}}}$ ではありません。 \hfill\break
He isn't a brave enough soldier to attempt that. }
 
\par{${\overset{\textnormal{}}{\text{14. 全国民}}}$ は ${\overset{\textnormal{}}{\text{戦争}}}$ に ${\overset{\textnormal{}}{\text{優勝}}}$ して、 ${\overset{\textnormal{なみだ}}{\text{涙}}}$ が ${\overset{\textnormal{}}{\text{出}}}$ るほど ${\overset{\textnormal{}}{\text{嬉}}}$ しかったです。 \hfill\break
The entire nation was happy to the point of tears from winning the war. }

\par{${\overset{\textnormal{}}{\text{15. 振}}}$ り ${\overset{\textnormal{ほど}}{\text{解}}}$ けないほど ${\overset{\textnormal{}}{\text{彼女}}}$ を ${\overset{\textnormal{しっか}}{\text{確}}}$ りと抱き ${\overset{\textnormal{し}}{\text{締}}}$ めた。 (叙述的な表現) \hfill\break
I hugged her so tightly that she couldn't break away. }

\par{\textbf{漢字 Note }: しっかり(と) is usually written in ひらがな. Using the 漢字 is more old-fashioned. }

\par{ When ほど is used to make a negative comparison such as in the following sentences, it cannot be replaced with くらい・ぐらい. }
 
\par{16. クリスマスほど ${\overset{\textnormal{}}{\text{待}}}$ ち ${\overset{\textnormal{どお}}{\text{遠}}}$ しいものはない。 \hfill\break
There's nothing I look forward to more than Christmas. }
 
\par{${\overset{\textnormal{}}{\text{17a. 鈴木}}}$ さんは ${\overset{\textnormal{}}{\text{外見}}}$ ほど ${\overset{\textnormal{}}{\text{年}}}$ をとっていませんね。 (Potentially rude) \hfill\break
17b. 鈴木さんは見た目より若く見えますね。 (More natural) \hfill\break
Mr. Suzuki's not as old as he looks. }
 ${\overset{\textnormal{}}{\text{18. 高橋氏}}}$ は ${\overset{\textnormal{}}{\text{外見}}}$ ほどの ${\overset{\textnormal{ねんれい}}{\text{年齢}}}$ じゃないよ。 (失礼な言い方) \hfill\break
Mr. Takahashi is not as old as he looks. 
\par{${\overset{\textnormal{}}{\text{19. 日本}}}$ に ${\overset{\textnormal{}}{\text{住}}}$ むほど ${\overset{\textnormal{}}{\text{幸}}}$ せなことはありません。 \hfill\break
There is nothing more happy than living in Japan. }

\par{20. 足はこの程度なら医者に行くほどのこともない。 \hfill\break
If your foot is only this bad, there's no need to go to the doctor. }

\par{21. ${\overset{\textnormal{おこ}}{\text{怒}}}$ るほどのことではありません。 \hfill\break
It's not something we need to get angry about. }
 
\par{${\overset{\textnormal{}}{\text{22. 心配}}}$ するほどの ${\overset{\textnormal{けが}}{\text{怪我}}}$ ではありません。 \hfill\break
It's not an injury to worry about. }
 
\par{${\overset{\textnormal{}}{\text{23. 病状}}}$ は ${\overset{\textnormal{}}{\text{手術}}}$ が ${\overset{\textnormal{}}{\text{必要}}}$ なほどではない。 \hfill\break
The condition of her illness is not to the point of surgery. }
 
\par{24. あいつは ${\overset{\textnormal{}}{\text{俺}}}$ ほどうまく ${\overset{\textnormal{}}{\text{運転}}}$ しねー。(Vulgar) \hfill\break
He doesn't drive as good as me. }
 
\par{${\overset{\textnormal{}}{\text{25. 私}}}$ の ${\overset{\textnormal{ちしき}}{\text{知識}}}$ など ${\overset{\textnormal{}}{\text{先生}}}$ のそれとは ${\overset{\textnormal{ひかく}}{\text{比較}}}$ にならないほどお ${\overset{\textnormal{そまつ}}{\text{粗末}}}$ なものでございます。 (Humble) \hfill\break
My knowledge is not even comparable to that of the teacher's. }
 
\par{${\overset{\textnormal{}}{\text{26. 散歩}}}$ ほどいいものはないね。 \hfill\break
There's nothing like a walk, right? }

\par{${\overset{\textnormal{}}{\text{27. 今日}}}$ の ${\overset{\textnormal{びぶんせきぶんがく}}{\text{微分積分学}}}$ の ${\overset{\textnormal{}}{\text{試験}}}$ は ${\overset{\textnormal{}}{\text{思}}}$ ったほど ${\overset{\textnormal{}}{\text{難}}}$ しくなかった。 \hfill\break
Today's calculus exam was less difficult than I thought. }

\par{28. 結果は後程ご一報ください。(Formal) \hfill\break
Please inform us the results later. }

\par{\textbf{敬語 Note }: 後程(のちほど) = あとで. }

\begin{center}
 \textbf{AすればAするほど }
\end{center}

\par{ This pattern means "the more\dothyp{}\dothyp{}\dothyp{}the more\dothyp{}\dothyp{}\dothyp{}". Without the conditional, you'd just show that there is a link between the extent of the pattern and the phrase that follows. The overall sentence can either be positive or negative depending on whether you are showing an increase or decrease in the degree of something respectively. You \textbf{cannot }replace ほど with くらい・ぐらい for this usage! }

\begin{ltabulary}{|P|P|}
\hline 

Verbs & 泳ぐ \textrightarrow  泳げ \textrightarrow  泳げ +ば = 泳げば + 泳ぐほど = 泳げば泳ぐほど \\ \cline{1-2}

形容詞 & 新しい \textrightarrow  新しけれ + ば = 新しければ + 新しいほど = 新しければ新しいほど \\ \cline{1-2}

形容動詞 & 簡単だ \textrightarrow  簡単な (+ば) = 簡単なら(ば) + 簡単なほど = 簡単なら(ば)簡単なほど \\ \cline{1-2}

\end{ltabulary}
 
\par{\textbf{Politeness Note }: Using であれば instead of なら is more formal\slash polite. }

\par{${\overset{\textnormal{}}{\text{29. 多}}}$ ければ ${\overset{\textnormal{}}{\text{多}}}$ いほどよい。 \hfill\break
The more, the better it is. }

\par{30. ${\overset{\textnormal{ふくざつ}}{\text{複雑}}}$ なら複雑なほど ${\overset{\textnormal{}}{\text{壊}}}$ れやすいです。 \hfill\break
The more complicated something is, the easier it is to break. }

\par{${\overset{\textnormal{}}{\text{31. 何}}}$ でも、 ${\overset{\textnormal{}}{\text{練習}}}$ すればするほど ${\overset{\textnormal{}}{\text{上手}}}$ になります。 \hfill\break
With anything, the more you practice, the better you become. }

\par{32. ${\overset{\textnormal{かかく}}{\text{価格}}}$ が高ければ高いほど、 ${\overset{\textnormal{じゅよう}}{\text{需要}}}$ は ${\overset{\textnormal{げんしょう}}{\text{減少}}}$ する。 \hfill\break
The higher the price, the smaller the demand is. }

\par{33. ${\overset{\textnormal{こっとうひん}}{\text{骨董品}}}$ は古ければ古いほど価値があがる。 \hfill\break
The older antiques are the more value it has. }

\par{34. 早ければ早いほどいいです。 \hfill\break
The sooner, the better it is. }

\par{35. 高く ${\overset{\textnormal{}}{\text{登}}}$ れば登るほど寒くなってくる。 \hfill\break
The higher you climb, the colder it gets. }

\par{36. パソコンは操作が簡単であれば簡単であるほどいいです。 \hfill\break
The easier the operation of a PC is, the better it is. }

\par{37. ${\overset{\textnormal{いなか}}{\text{田舎}}}$ に ${\overset{\textnormal{}}{\text{住}}}$ むほど ${\overset{\textnormal{}}{\text{幸}}}$ せなことはないと ${\overset{\textnormal{}}{\text{思}}}$ う。 \hfill\break
I think that nothing is happier than living in the country. }

\par{38. 人が多ければ多いほど楽しい。 \hfill\break
The more the merrier. }

\par{39. 深ければ深いほど暗くなる。 \hfill\break
The deeper you get, the darker it gets. }
      
\section{The Adverbial Particle くらい・ぐらい}
 
\par{ The way くらい・ぐらい are attached to things is the same as with ほど, but its combination with other particles is slightly different. }

\begin{ltabulary}{|P|P|P|}
\hline 

Class & Pattern & Example \\ \cline{1-3}

名詞 & N+くらい・ぐらい & 一円\{くらい・ぐらい\} \\ \cline{1-3}

形容詞 & 連体形+くらい・ぐらい & 見えない\{くらい・ぐらい\} \\ \cline{1-3}

形容動詞 & 連体形+くらい・ぐらい & 不思議な\{くらい・ぐらい\} \\ \cline{1-3}

動詞 & 連体形+くらい・ぐらい & 捨てる\{くらい・ぐらい\} \\ \cline{1-3}

\end{ltabulary}

\par{  The particles が and を if used with it are after it. The particles に, へ, で, と, から can all be seen before or after it. }

\begin{center}
 \textbf{Approximation }
\end{center}

\par{ Introduces something and then shows the degree or quantity almost equivalent to that. When you are saying that something is the same degree, you must use くらい・ぐらい. }

\par{40. ニューメキシコ州も同じ\{〇 くらい・〇 ぐらい・X ほど\}暑いですよ。 \hfill\break
New Mexico is just as hot. }
 
\par{${\overset{\textnormal{}}{\text{41. 8時}}}$ くらいです。 \hfill\break
It's about eight o'clock. }

\par{42. リーチは立原の方が極端なぐらい長い。 \hfill\break
As for reach, Tachihara's is longer to an extreme. \hfill\break
From 擬態 by 北方謙三. }
 
\par{${\overset{\textnormal{}}{\text{43. 彼}}}$ は36 ${\overset{\textnormal{}}{\text{歳}}}$ ぐらいだ。 \hfill\break
He is around thirty six years old. }
 
\par{${\overset{\textnormal{}}{\text{44. 30分}}}$ ぐらいしてから、もう ${\overset{\textnormal{}}{\text{一度電話}}}$ をかけてみました。 \hfill\break
I tried calling again after about thirty minutes. }
 
\par{${\overset{\textnormal{}}{\text{45. 私}}}$ の捕った ${\overset{\textnormal{}}{\text{魚}}}$ はこれくらいの ${\overset{\textnormal{}}{\text{大}}}$ きさでした。 \hfill\break
The fish I caught was about this big. }
 
\par{${\overset{\textnormal{}}{\text{46. 昼}}}$ くらいに ${\overset{\textnormal{}}{\text{雨}}}$ が ${\overset{\textnormal{}}{\text{止}}}$ んだ。 \hfill\break
It stopped raining around noon. \hfill\break
 \hfill\break
47. それくらいのことでめげるな。 \hfill\break
Don't be discouraged at such a thing. }

\par{48. ${\overset{\textnormal{まゆ}}{\text{眉}}}$ が ${\overset{\textnormal{}}{\text{隠}}}$ れるくらいに ${\overset{\textnormal{}}{\text{切}}}$ ってください。 \hfill\break
Please cut it so that my eyebrows are covered. }

\par{49. そして ${\overset{\textnormal{}}{\text{仲田}}}$ が、その女を自分の妹あつかいし、馬鹿にしているのを ${\overset{\textnormal{もったい}}{\text{勿体}}}$ ないことをする ${\overset{\textnormal{やつ}}{\text{奴}}}$ だ位に感じた。 \hfill\break
Furthermore, he felt that Nakata was this guy treating that (perfect woman for him) as his own sister, making her out to be stupid, having her do things unfit for her class. \hfill\break
From 友情 by ${\overset{\textnormal{むしゃのこうじさねあつ}}{\text{武者小路実篤}}}$ . }

\par{\textbf{Grammar Note }: An entire clause is modifying 位, the 漢字 spelling of くらい, and if this were written in a more modern fashion, quotations would have probably been used to reinforce a short spoken pause that aids in the grammaticality of this sentence. }

\par{\textbf{Orthography Note }: This sentence comes from a book originally published in 1920, which is why くらい is written in 漢字. It would be somewhat old-fashioned to do likewise today. }

\begin{center}
 \textbf{Belittling }
\end{center}
 
\par{ Introduces something of light degree and limits it to a certain level. This can be seen in the pattern XのはYくらいだ (Y is about the only X). Thus, again, this usage creates a minimal phrase. This belittling usage is not interchangeable with ほど. }
 
\par{50. せめてタバコくらいやめてくださいよ。 \hfill\break
The least you can do is stop smoking. }
 
\par{${\overset{\textnormal{}}{\text{51. 簡単}}}$ な ${\overset{\textnormal{}}{\text{料理}}}$ ぐらい ${\overset{\textnormal{}}{\text{私}}}$ だってできるよ。 \hfill\break
Even I can cook a simple meal (to that extent)! }
 
\par{${\overset{\textnormal{}}{\text{52. 子供}}}$ でもそんなことくらい ${\overset{\textnormal{}}{\text{分}}}$ かる。 \hfill\break
Even a child would understand (to the extent of) something like that. }

\par{53. 「韓国料理は、何でも好きですか」 「いいえ、私が好きなのはキムチくらいです」 \hfill\break
"Do you like any Korean food?". "No, kimchi is about the only thing I like". }

\par{54. 「このごろの子供は、本をあんまり読まんと聞いたが」「ええ、ほんとに。よく読むのマンガぐらいなんだよね」「そりゃ困ったな~」 (すごく砕けた言い方) \hfill\break
"I heard that kids these days don't read books much". "Yeah, that's true. Manga is just about the only thing they often read". "That's no good". \hfill\break
\hfill\break
55. 「あの、君はスポーツは何でもできるんでしょ」「いや、実は、僕ができるのは\{サッカー・卓球・野球・バドミントン・テニス\}くらいだよ」 \hfill\break
"Uh, you can play any sport, right?". "No, actually, [soccer; table tennis; baseball; badminton; tennis] is about the only thing I can". }

\par{${\overset{\textnormal{}}{\text{56. 空一面}}}$ ${\overset{\textnormal{}}{\text{真}}}$ っ ${\overset{\textnormal{か}}{\text{赤}}}$ になるくらいの ${\overset{\textnormal{みごと}}{\text{見事}}}$ な ${\overset{\textnormal{}}{\text{夕焼}}}$ けでした。 \hfill\break
The splendid sunset was to the extent that the whole sky turned completely red. }

\par{57. ${\overset{\textnormal{は}}{\text{恥}}}$ ずかしくて ${\overset{\textnormal{}}{\text{穴}}}$ があったら ${\overset{\textnormal{}}{\text{入}}}$ りたいくらいだった。 \hfill\break
I was so embarrassed to the extent that if a hole opened up I would want to enter it. }
 
\par{2. Shows an extreme matter. ~ くらいなら~ ほうがましだ = "rather\dothyp{}\dothyp{}\dothyp{}than\dothyp{}\dothyp{}\dothyp{}" . However, it should be understood that both actions are unfavorable. It's just you say that you'd rather do the latter. }
 
\par{${\overset{\textnormal{}}{\text{58. 降参}}}$ するぐらいなら ${\overset{\textnormal{}}{\text{死}}}$ んだ ${\overset{\textnormal{}}{\text{方}}}$ がましだ。 \hfill\break
I'd rather die than surrender. }
 
\par{${\overset{\textnormal{}}{\text{59. 君}}}$ を ${\overset{\textnormal{}}{\text{捨}}}$ てるぐらいなら ${\overset{\textnormal{}}{\text{死}}}$ んだ ${\overset{\textnormal{}}{\text{方}}}$ がましだ。 \hfill\break
I would rather die than abandon you. }

\par{60. ${\overset{\textnormal{きじつ}}{\text{期日}}}$ に ${\overset{\textnormal{}}{\text{遅}}}$ れるくらいなら、 ${\overset{\textnormal{てつや}}{\text{徹夜}}}$ をして ${\overset{\textnormal{かんせい}}{\text{完成}}}$ させた ${\overset{\textnormal{}}{\text{方}}}$ がましだ。 \hfill\break
I would rather have it completed overnight than be late to the deadline. }

\par{\textbf{Phrase Note }: ましになる is used to show that something is improving, but it's not fully good. As such, it's not really appropriate to a lot of people to use such a phrase to refer to someone feeling better. However, there are areas like the Kansai Region where it is rather common. }

\par{61. 痛みがましになる。 \hfill\break
The pain improved. }

\par{62. 通りから広場に出たところで、少し混雑がましになった。 \hfill\break
As I entered into the plaza from the street, the traffic got a little better. \hfill\break
From 野生の風 by 村山由佳. }
 
\par{3. ~くらい~はない shows something being the most\dothyp{}\dothyp{}\dothyp{}. }
 
\par{${\overset{\textnormal{}}{\text{63. 彼}}}$ くらい ${\overset{\textnormal{}}{\text{努力}}}$ する ${\overset{\textnormal{}}{\text{人}}}$ はいません。 \hfill\break
There is no person putting effort into it than him. }

\par{${\overset{\textnormal{}}{\text{64. 地震\{ほど・}}}$ くらい\} ${\overset{\textnormal{}}{\text{怖}}}$ いものはない。 \hfill\break
There's nothing as scary as an earthquake. }

\par{4. ~くらいで  gives that the feeling that something is at a small amount or significance. It is often used in suggesting or belittling something. }

\par{65. テストが悪かったぐらいで、泣かないで。(Casual) \hfill\break
Don't cry over a doing bad on a test. }

\par{66. スペイン語習ったことはあるけど、ちょっとしゃべれるようになったぐらいでやめてちゃったんで、本を読めるようにはならんかった。 (すごく砕けた言い方) \hfill\break
I've studied Spanish, but I quit when I became able to speak it, so I didn't end up becoming able to read a book. }

\par{67. このレッスンはやさしいから、ちょっと練習するくらいで、終わりましょう。 \hfill\break
Since this lesson is easy, let's end with a little practice. }
    