    
\chapter{Honorifics III Light Honorifics}

\begin{center}
\begin{Large}
第126課: Honorifics III: Light Honorifics: レル・ラレル敬語 
\end{Large}
\end{center}
 
\par{ All of the usages of these endings are etymologically related to each other in Japanese. However, to keep things easier in our minds, we will need to treat them separately. This is because the grammar\slash particle usage for each usage is quite different. These endings, ~られる・れる, along with appropriate phrasing to go along, help create 軽い敬語. }

\par{ Knowing when it is appropriate to only be lightly honorific is a question that natives struggle at times. It is safe to say, though, that when you are with someone you should be respectful to yet have no established connection with each other to warrant the same speech you would use to your boss, 軽い敬語 becomes very useful. }

\par{\textbf{Ambiguity Note }: There are times when the passive usage and the honorific usage of these endings becomes blurred. Because of this, the usage of 軽い敬語 is less common in comparison to other speech styles. }
      
\section{軽い敬語}
 
\par{ Before we look at how this grammar is used, let's recap on how to construct it. }

\begin{ltabulary}{|P|P|P|P|}
\hline 

Verb Class & Example Verb & 未然形 & +られる・れる \\ \cline{1-4}

上一段 & 見る & 見- & 見られる \\ \cline{1-4}

下一段 & 食べる & 食べ- & 食べられる \\ \cline{1-4}

五段 & 読む & 読ま- & 読まれる \\ \cline{1-4}

サ変 & する & さ- & される \\ \cline{1-4}

カ変 & 来る & 来(こ)- & 来られる \\ \cline{1-4}

\end{ltabulary}

\par{  The use of レル・ラレル敬語 is becoming more frowned upon as being too vague and unfitting for more and more situations. Many speakers believe that if you are to be respectful, then you should just use 敬語. However, many younger people overuse レル・ラレル敬語 and is a potential cause for 二重敬語 when these young people are trained to use more honorific means of expression as they combine what they have known with what they're being told. Of course, doubling 敬語 can come about by just being overly cautious of how respectful you're being, but this is another motivation. }

\par{ This is not to take away from the fact that レル・ラレル敬語 is frequently used because it is. Just note that this is just the lowest tier of 敬語 and that you will need to use more honorific patterns as is needed. For instance, if were are to assume that talking to your 課長 and 社長 is different, レル・ラレル敬語 would be proper for the former but not for the latter. In interviews there is a tendency to avoid レル・ラレル敬語, but there are plenty of reporters that use it. In the end, there is a lot of speaker variation. Just keep these things in mind as you use this speech pattern. }

\begin{center}
 \textbf{Examples }
\end{center}

\par{1. その ${\overset{\textnormal{}}{\text{本}}}$ はもう ${\overset{\textnormal{}}{\text{読}}}$ まれましたか。 \hfill\break
Have you already read the book? }

\par{2. どちらで ${\overset{\textnormal{}}{\text{日本語}}}$ を ${\overset{\textnormal{}}{\text{習}}}$ われましたっけ。 \hfill\break
Where again did you learn Japanese? }

\par{${\overset{\textnormal{}}{\text{3. 奈良}}}$ に ${\overset{\textnormal{}}{\text{行}}}$ かれたことがありますか。 \hfill\break
Have you ever been to Nara? }

\par{4. 柳田さんは今朝、仙台を発たれました。 \hfill\break
Yanagida-san left Sendai this morning. }

\par{5. 今朝の産経新聞を読まれましたか。 \hfill\break
Have you read this morning's Sankei Newspaper? }

\par{6. お酒を飲まれますか。 \hfill\break
Do you drink? }

\par{7. どちらに住んでおられますか。 \hfill\break
Where do you live? }

\par{8. この記事をどう思われますか。 \hfill\break
What do you think about the article? }

\par{9. 先生が来られました。 \hfill\break
The teacher has come. }

\par{10. 何時に出発されますか。 \hfill\break
At what time will you depart? }

\par{11. 鈴木副課長は朝何時に起きられましたか。 \hfill\break
What time in the morning did Chief Suzuki wake up? }

\par{12. 先生も行かれますか。(〇・?・X) \hfill\break
Will you go too, Sensei? }

\par{\textbf{Sentence Note }: For those who believe レル・ラレル敬語 should be used towards a figure like 先生, this sentence is wrong. As we will soon discuss in more detail, the sentence could also mean whether Sensei can go as well. Practically speaking, however, many people in school would say this to their Sensei. So, it's not inherently grammatically incorrect. }

\par{13. 先程社長が\{言われました X・仰いました 〇\}ように、韓国へ進出することとなりました。 \hfill\break
Just as the president has said just a moment ago, we will be expanding into Korea. }

\par{\textbf{Sentence Note }: In this sentence, it truly is deemed by the majority of speakers that レル・ラレル敬語 would be wrong because 社長 is not a title that goes with it. }

\par{14. よくゴルフを\{されます・なさいます\}か。 \hfill\break
Do you often golf? }

\par{\textbf{Word Note }: Historically なさる and 下さる didn't even exist. They are in fact from the combination of 為す+れる and 下す+れる respectively. Interestingly enough, なされる is starting to make a comeback in Double Honorifics in Modern Japanese. }

\begin{center}
 \textbf{Ambiguity }
\end{center}

\par{ As mentioned in the introduction, there are times when it is hard to tell what is meant. In the first sentence, one could argue that if traditional particle alignment (が instead of を for the object of a potential verb) were used that the second interpretation would not be likely. In reality, little distinction is made between using が or を to mark the object of a potential verb. So, the second interpretation is totally plausible. }

\par{15. 資料を見られましたか。 \hfill\break
Did you see the material? \hfill\break
Could you see the material? \hfill\break
Was the material seen? (迷惑受身) }

\par{ In this example, there is not as much ambiguity as the previous. One could say that the use of 行かれる for the potential form of 行く instead of 行ける is unlikely and old-fashioned. However, because the use of the traditional potential form would be more formal, this causes both interpretations below to be plausible. If the purpose of 敬語 is to be both respectful and clear, then there will be instances in which you should not leave room for misinterpretation. Thus, in these cases you should rephrase with a statement that is clearly only 敬語. }

\par{16. 課長は明日行かれますか。 ?    \textrightarrow  課長は明日いらっしゃいますか。 \hfill\break
Can the chief go tomorrow? \hfill\break
Will the chief go tomorrow? }

\par{17. こちらのカボチャ、食べられますか。 \textrightarrow  こちらのカボチャ、召し上がりますか。 \hfill\break
Will you eat this pumpkin? \hfill\break
Can you eat this pumpkin? }

\par{ If you were to want to add potential to an honorific phrase, it is best to do add a polite element once you have used a regular 敬語 expression. }

\par{18. お酒はどの程度召し上がれますか。 \hfill\break
How much alcohol can you drink? }
    