    
\chapter{The Volitional I}

\begin{center}
\begin{Large}
第120課: The Volitional I 
\end{Large}
\end{center}
 
\par{ Volition is the speaker's will to do or not do something. In Japanese, there are both affirmative and negative volitional forms. For starters, we will learn about the endings used to create affirmative volitional statements. }
      
\section{Plain Speech Affirmative Volitional Endings: ~よう \& ~う}
 
\par{ As far as meaning is concerned, the affirmative volitional form either translates as "let's" or "I will." If the statement implies participation from others, then the former interpretation is intended. If the statement doesn't imply the participation of others, then the latter interpretation is intended. }

\par{ Although verbs aren't the only things that have volitional forms, we will limit our discussion to verbal volitional phrases for now. In plain speech, there are two auxiliary verbs used to create the affirmative volitional form: ~よう \& ~う. The only difference between ~よう and ~う is what class of verbs they're used with. }

\begin{center}
\textbf{Conjugation Chart } 
\end{center}

\begin{ltabulary}{|P|P|P|P|}
\hline 

\slash iru\slash -Ending \emph{Ichidan }Verbs & 見る + よう \textrightarrow  & 見よう & Let's\slash I'll see \\ \cline{1-4}

\slash eru\slash -Ending \emph{Ichidan }Verbs & 食べる + よう \textrightarrow  & 食べよう & Let's\slash I'll eat \\ \cline{1-4}

\slash u\slash -Ending \emph{Godan }Verbs & 買う + う \textrightarrow  & 買おう & Let's\slash I'll buy \\ \cline{1-4}

\slash ku\slash -Ending \emph{Godan }Verbs & 書く + う \textrightarrow  & 書こう & Let's\slash I'll write \\ \cline{1-4}

\slash gu\slash -Ending \emph{Godan }Verbs & 泳ぐ + う \textrightarrow  & 泳ごう & Let's\slash I'll swim \\ \cline{1-4}

\slash su\slash -Ending \emph{Godan }Verbs & 話す + う \textrightarrow  & 話そう & Let's\slash I'll talk \\ \cline{1-4}

\slash tsu\slash -Ending \emph{Godan }Verbs & 勝つ + う \textrightarrow  & 勝とう & Let's\slash I'll win \\ \cline{1-4}

\slash nu\slash -Ending \emph{Godan }Verbs & 死ぬ + う \textrightarrow  & 死のう & Let's\slash I shall die \\ \cline{1-4}

\slash mu\slash -Ending \emph{Godan }Verbs & 読む + う \textrightarrow  & 読もう & Let's\slash I'll read \\ \cline{1-4}

\slash ru\slash -Ending \emph{Godan }Verbs & 図る + う \textrightarrow  & 図ろう & Let's\slash I'll devise \\ \cline{1-4}

 \emph{Suru }Verbs & する + よう \textrightarrow  & しよう & Let's\slash I'll do \\ \cline{1-4}

 \emph{Kuru } & 来る + よう \textrightarrow  & 来(こ)よう & Let's\slash I'll come \\ \cline{1-4}

\end{ltabulary}

\par{\textbf{Translation Note }: In English, it is not always the case that "I will" is the best phrasing to indicate personal volition to do something, especially since it also functions as the future tense auxiliary. Whenever this is the case, "shall" can be a better working translation of the affirmative volitional auxiliaries of Japanese. }

\par{\textbf{Pronunciation Note }: In casual speech, the final う at the end of these forms may be heard omitted. }

\begin{center}
 \textbf{One's Volition\slash Will } 
\end{center}

\par{  The main usage of these endings is to express one's volition\slash will to do something. This is referred to as the 意志形 in Japanese. This is either used in a sense of including those around you or simply used to solely indicate one's own intention. }

\par{${\overset{\textnormal{}}{\text{1. 寿司}}}$ を ${\overset{\textnormal{}}{\text{食}}}$ べよう。 \hfill\break
I'll eat sushi\slash Let's eat sushi. }

\par{2. これだけははっきりとさせておこう。 \hfill\break
Let's get this (much) straight. \hfill\break
}

\par{${\overset{\textnormal{}}{\text{3. 少}}}$ し ${\overset{\textnormal{}}{\text{休}}}$ もうか。 \hfill\break
How about taking a rest for a while? }

\par{4. もう ${\overset{\textnormal{}}{\text{間}}}$ に ${\overset{\textnormal{}}{\text{合}}}$ わないから、ゆっくりしよう。 \hfill\break
It's not that we have to make it on time. So, let's go slowly. }

\par{\textbf{Word Note }: 間に合う should only be used with time. }

\par{5. さらに ${\overset{\textnormal{}}{\text{食}}}$ べようっていうの? \hfill\break
You're going to eat on top of this? }

\par{${\overset{\textnormal{}}{\text{6. 出}}}$ かけよう\{では・じゃ\}ないか? \hfill\break
Why don't we go out? }
\textbf{Grammar Note }: ~ようでは・じゃないか can also make a volitional question. \hfill\break

\par{${\overset{\textnormal{}}{\text{7. 中国語}}}$ を ${\overset{\textnormal{}}{\text{勉強}}}$ しようか。 \hfill\break
How about I\slash we study Chinese? }

\par{${\overset{\textnormal{}}{\text{8. 食事}}}$ を ${\overset{\textnormal{す}}{\text{済}}}$ ませてから、 ${\overset{\textnormal{}}{\text{外}}}$ に ${\overset{\textnormal{}}{\text{出}}}$ よう。 \hfill\break
Let's go outside after we finish dinner. \hfill\break
\hfill\break
9. この ${\overset{\textnormal{}}{\text{電車}}}$ は ${\overset{\textnormal{}}{\text{込}}}$ んでるから、 ${\overset{\textnormal{}}{\text{次}}}$ のに ${\overset{\textnormal{}}{\text{乗}}}$ ろう。(Colloquial) \hfill\break
Since this train is crowded, let's get on the next one. }

\begin{center}
\textbf{Determination\slash Volition of Others }
\end{center}

\par{ ~(よ)うと思います shows that oneself is \emph{now }determined to do something. In contrast, ~(よ)うと思っています either shows that oneself has made the decision to do something some time ago or the volition of others. }
 
\par{${\overset{\textnormal{}}{\text{10. 姉}}}$ は ${\overset{\textnormal{}}{\text{中国}}}$ で ${\overset{\textnormal{}}{\text{日本語}}}$ を教えようと ${\overset{\textnormal{}}{\text{思}}}$ っています。 \hfill\break
My older sister is thinking about teaching Japanese in China. }
 
\par{11. ハワイに ${\overset{\textnormal{}}{\text{行}}}$ こうと ${\overset{\textnormal{}}{\text{思}}}$ います。 \hfill\break
I think I'm going to Hawaii. }

\par{12. ${\overset{\textnormal{しょうがくきん}}{\text{奨学金}}}$ をもらおうと ${\overset{\textnormal{}}{\text{考}}}$ えています。 \hfill\break
I'm considering receiving scholarship money. }
 
\par{13. ${\overset{\textnormal{しょうらい}}{\text{将来}}}$ は ${\overset{\textnormal{えいがかんとく}}{\text{映画監督}}}$ になろうと ${\overset{\textnormal{}}{\text{思}}}$ っている。 \hfill\break
I'm thinking of becoming a movie director in the future. }
   
\begin{center}
\textbf{Likelihood (Old-Fashioned Speech) }
\end{center}

\par{ A meaning that has fallen out of use but is still seen in old-fashioned speech and literature is the ability to show likelihood. This has largely been replaced with のだろう, which combines だ and ~う, as we will see again later in this lesson. Though this meaning is largely defunct, it can be distinguished from the others by not just differences in tone but also by context. This meaning of the volitional endings involves statements about state, not an action in which the speaker has control over. }

\par{${\overset{\textnormal{}}{\text{14a. 長時間歩}}}$ いたのでお腹もすいていよう。(Old-fashioned) \hfill\break
14b. 長時間歩いたのでお腹もすいているのだろう。 \hfill\break
You're also probably starving because you've been walking for so long. }

\par{15a. 彼らは ${\overset{\textnormal{してき}}{\text{指摘}}}$ できよう。(Old-fashioned) \hfill\break
15b. 彼らは指摘できるのだろう。 \hfill\break
They'll probably be able to point it out. }

\par{ There are some grammatical instances where this meaning of the volitional endings lives on in modern language use. These instances include the patterns ~(よ)うはずがない  and (よ)うものなら. }

\par{16. そんなことがあろうはずがない。 \hfill\break
Such a thing should not happen. }

\par{17. 黙っていようものなら、自滅するぞ。 \hfill\break
If you are to remain quiet, you will end yourself. }
 
\begin{center}
\textbf{Rhetorical Questions }
\end{center}

\par{ Another usage of these endings is making rhetorical questions when followed by the rhetorical question-marker か. In this situation, か has a sharp drop in pitch. This usage has largely been replaced by ~のだろうか, which incidentally also uses the auxiliary ~う. }

\par{${\overset{\textnormal{}}{\text{18a. 許}}}$ されようか。(Old-fashioned) \hfill\break
18b. 許されるのだろうか。 \hfill\break
Will [I\slash you\slash he\slash she\slash it] really be forgiven? \hfill\break
 \hfill\break
 ${\overset{\textnormal{}}{\text{19a. 彼}}}$ はそれができようか。(Old-fashioned) \hfill\break
19b. 彼はそれができるのだろうか。 \hfill\break
Can he really do that? }
 \textbf{No Matter What }   "\dothyp{}\dothyp{}\dothyp{}(よ)うが", "\dothyp{}\dothyp{}\dothyp{}(よ)うと", "\dothyp{}\dothyp{}\dothyp{}(よ)うとも", "(よ)うものなら", and "\dothyp{}\dothyp{}\dothyp{}(よ)うにも", show supposition and is equivalent to "しても". They are like "no matter what" or "even if".   
\par{${\overset{\textnormal{}}{\text{20. 家出}}}$ をしようにもお ${\overset{\textnormal{}}{\text{金}}}$ がないよ。 \hfill\break
Even if run away, you don't have any money. }
 
\par{${\overset{\textnormal{}}{\text{21. 真実}}}$ であろうが ${\overset{\textnormal{}}{\text{嘘}}}$ であろうが、まだ ${\overset{\textnormal{}}{\text{関係}}}$ はない。 \hfill\break
Whether it's true or a lie, I still have no part in it. }
 
\par{${\overset{\textnormal{}}{\text{22. 行}}}$ こうと ${\overset{\textnormal{}}{\text{行}}}$ くまいと ${\overset{\textnormal{}}{\text{僕}}}$ の ${\overset{\textnormal{}}{\text{勝}}}$ ちだ。 \hfill\break
Even if you go or don't go, it's my victory. }

\par{23. どんなに反対されようが、消費税の引き上げが施行される。 \hfill\break
No matter how much it's protested against, the consumer tax hike will be enforced. }

\par{24. どんな苦しみを味わおうが、自分が決めたことは変えたくない。 \hfill\break
No matter the suffering I suffer, I won't want to change what I've decided. }

\par{25. どれだけお金を損しようが、賭博し続けるよ。 \hfill\break
No matter how much money I lose, I'll continue to gamble. }
 
\par{${\overset{\textnormal{}}{\text{26. 何}}}$ をしようと ${\overset{\textnormal{}}{\text{私}}}$ の ${\overset{\textnormal{}}{\text{知}}}$ ったことではありません。 \hfill\break
Whatever you do, it's nothing that I know. }
 
\par{27. たとえ雨が ${\overset{\textnormal{}}{\text{降}}}$ ろうともフットボールをする。 \hfill\break
I play football even if it rains. }
 
\par{28. どれだけ ${\overset{\textnormal{}}{\text{時間}}}$ がかか\{ろうとも・っても\}、 ${\overset{\textnormal{}}{\text{彼}}}$ らを ${\overset{\textnormal{}}{\text{支持}}}$ します。 \hfill\break
I will support them no matter how long it takes. }
 
\par{${\overset{\textnormal{}}{\text{29. 先生}}}$ に ${\overset{\textnormal{あつ}}{\text{厚}}}$ かましくも ${\overset{\textnormal{くちごた}}{\text{口答}}}$ えをしようものなら、 ${\overset{\textnormal{おおめだま}}{\text{大目玉}}}$ を ${\overset{\textnormal{く}}{\text{食}}}$ らうでしょう。 \hfill\break
Should you ever have the nerve to talk back to the teacher, you'll surely get scolded severely. }
 
\par{${\overset{\textnormal{}}{\text{30. 火事}}}$ になろうものなら、 ${\overset{\textnormal{}}{\text{大変}}}$ だぞ。 \hfill\break
It would be grave should there be a fire.  }

\begin{center}
 \textbf{Just About To\dothyp{}\dothyp{}\dothyp{} }
\end{center}

\par{ ~(よ)うとしたら is when "just as one is about to do X, Y happens". Y is out of your control, and often includes speech modals like ~てしまった and ~きた. }

\par{31. 電車に乗ろうとしたら、ドアが閉まってしまいました。 \hfill\break
I was about to get on the train when the door (regrettably\slash accidentally) closed. }
 
\par{32. アイスクリームを買って、歩きながら食べようとしたら、「みっともないですよ」っておこられちゃったし。 \hfill\break
When I tried to eat the ice cream I bought while walked, I was scolded and told that it was "indecent". }
 
\par{33. 喫茶店へ行こうとしたら、雨が降って来た。 \hfill\break
Just as I was about to go to the coffee shop, it started to rain. }
 
\par{34. 出かけようとしたら、電話がかかって来た。 \hfill\break
Just as I was going to leave, a phone call came. }
      
\section{Polite Speech Affirmative Volitional Ending: ~ましょう}
 
\par{ The polite speech equivalent of both ~よう and ~う is ~ましょう. Conjugating with this is the same as with ~ます. In fact, this is just a combination of ~ます + ~う! Meaning-wise, it is not used in all sorts of grammar patterns like its plain speech counterparts. Instead, it is limited to showing "let's" or "I will\slash shall." Very rarely is it used to show likelihood, in which case it behaves just like its plain speech counterparts. }

\begin{center}
\textbf{Conjugation Chart }
\end{center}

\begin{ltabulary}{|P|P|P|P|}
\hline 

\slash iru\slash -Ending \emph{Ichidan }Verbs & 見る + ましょう \textrightarrow  & 見ましょう & Let's\slash I'll see \\ \cline{1-4}

\slash eru\slash -Ending \emph{Ichidan }Verbs & 食べる + ましょう \textrightarrow  & 食べましょう & Let's\slash I'll eat \\ \cline{1-4}

\slash u\slash -Ending \emph{Godan }Verbs & 買う + ましょう \textrightarrow  & 買いましょう & Let's\slash I'll buy \\ \cline{1-4}

\slash ku\slash -Ending \emph{Godan }Verbs & 書く + ましょう \textrightarrow  & 書きましょう & Let's\slash I'll write \\ \cline{1-4}

\slash gu\slash -Ending \emph{Godan }Verbs & 泳ぐ + ましょう \textrightarrow  & 泳ぎましょう & Let's\slash I'll swim \\ \cline{1-4}

\slash su\slash -Ending \emph{Godan }Verbs & 話す + ましょう \textrightarrow  & 話しましょう & Let's\slash I'll talk \\ \cline{1-4}

\slash tsu\slash -Ending \emph{Godan }Verbs & 勝つ + ましょう \textrightarrow  & 勝ちましょう & Let's\slash I'll win \\ \cline{1-4}

\slash nu\slash -Ending \emph{Godan }Verbs & 死ぬ + ましょう \textrightarrow  & 死にましょう & Let's\slash I shall die \\ \cline{1-4}

\slash mu\slash -Ending \emph{Godan }Verbs & 読む + ましょう \textrightarrow  & 読みましょう & Let's\slash I'll read \\ \cline{1-4}

\slash ru\slash -Ending \emph{Godan }Verbs & 図る + ましょう \textrightarrow  & 図りましょう & Let's\slash I'll devise \\ \cline{1-4}

 \emph{Suru }Verbs & する + ましょう \textrightarrow  & しましょう & Let's\slash I'll do \\ \cline{1-4}

 \emph{Kuru } & 来る + ましょう \textrightarrow  & 来(き)ましょう & Let's\slash I'll come \\ \cline{1-4}

\end{ltabulary}

\par{\textbf{Pronunciation Note }: Though less polite, the final う in these forms can be heard omitted. It is also possible to hear the しょ pronounced as ひょ in certain dialects, especially traditional Kyoto Dialect speech. }

\begin{center}
\textbf{Examples }
\end{center}

\par{35. 価格を下げることで売り上げが ${\overset{\textnormal{の}}{\text{伸}}}$ びるように定価から2千円を割り ${\overset{\textnormal{び}}{\text{引}}}$ きましょう。 \hfill\break
In lowering prices in order to boost sales, let's knock off 2000 yen from the price. }

\par{36. ${\overset{\textnormal{かんぱい}}{\text{乾杯}}}$ しましょう。 \hfill\break
Cheers! }
 
\par{${\overset{\textnormal{}}{\text{37. 私}}}$ から ${\overset{\textnormal{}}{\text{電話}}}$ しましょうか。 \hfill\break
Shall I call? }
 
\par{\textbf{Particle Note }: This usage of から may be replaced with が. }
 
\par{${\overset{\textnormal{}}{\text{38. 一緒}}}$ に ${\overset{\textnormal{}}{\text{外食}}}$ しましょう。 \hfill\break
Let's go out to eat together. }
 
\par{${\overset{\textnormal{}}{\text{39. 早速出}}}$ かけましょう! \hfill\break
Let's head out at once! }

\par{40. ${\overset{\textnormal{よ}}{\text{世}}}$ にある ${\overset{\textnormal{かぎ}}{\text{限}}}$ り ${\overset{\textnormal{さいぜん}}{\text{最善}}}$ を ${\overset{\textnormal{つく}}{\text{尽}}}$ しましょう。 \hfill\break
Let's do our best to live in this world as much as possible. }
 
\par{\textbf{Phrase Note }: ${\overset{\textnormal{}}{\text{世}}}$ にある means "to live in this world". }

\begin{ltabulary}{|P|P|P|P|}
\hline 

\slash iru\slash -Ending \emph{Ichidan }Verbs & 見る + よう \textrightarrow  & 見よう & Let's\slash I'll see \\ \cline{1-4}

\slash eru\slash -Ending \emph{Ichidan }Verbs & 食べる + よう \textrightarrow  & 食べよう & Let's\slash I'll eat \\ \cline{1-4}

\slash u\slash -Ending \emph{Godan }Verbs & 買う + う \textrightarrow  & 買おう & Let's\slash I'll buy \\ \cline{1-4}

\slash ku\slash -Ending \emph{Godan }Verbs & 書く + う \textrightarrow  & 書こう & Let's\slash I'll write \\ \cline{1-4}

\slash gu\slash -Ending \emph{Godan }Verbs & 泳ぐ + う \textrightarrow  & 泳ごう & Let's\slash I'll swim \\ \cline{1-4}

\slash su\slash -Ending \emph{Godan }Verbs & 話す + う \textrightarrow  & 話そう & Let's\slash I'll talk \\ \cline{1-4}

\slash tsu\slash -Ending \emph{Godan }Verbs & 勝つ + う \textrightarrow  & 勝とう & Let's\slash I'll win \\ \cline{1-4}

\slash nu\slash -Ending \emph{Godan }Verbs & 死ぬ + う \textrightarrow  & 死のう & Let's\slash I shall die \\ \cline{1-4}

\slash mu\slash -Ending \emph{Godan }Verbs & 読む + う \textrightarrow  & 読もう & Let's\slash I'll read \\ \cline{1-4}

\slash ru\slash -Ending \emph{Godan }Verbs & 図る + う \textrightarrow  & 図ろう & Let's\slash I'll devise \\ \cline{1-4}

 \emph{Suru }Verbs & する + よう \textrightarrow  & しよう & Let's\slash I'll do \\ \cline{1-4}

 \emph{Kuru } & 来る + よう \textrightarrow  & 来(こ)よう & Let's\slash I'll come \\ \cline{1-4}

\end{ltabulary}

\begin{ltabulary}{|P|P|P|P|}
\hline 

\slash iru\slash -Ending \emph{Ichidan }Verbs & 見る + よう \textrightarrow  & 見よう & Let's\slash I'll see \\ \cline{1-4}

\slash eru\slash -Ending \emph{Ichidan }Verbs & 食べる + よう \textrightarrow  & 食べよう & Let's\slash I'll eat \\ \cline{1-4}

\slash u\slash -Ending \emph{Godan }Verbs & 買う + う \textrightarrow  & 買おう & Let's\slash I'll buy \\ \cline{1-4}

\slash ku\slash -Ending \emph{Godan }Verbs & 書く + う \textrightarrow  & 書こう & Let's\slash I'll write \\ \cline{1-4}

\slash gu\slash -Ending \emph{Godan }Verbs & 泳ぐ + う \textrightarrow  & 泳ごう & Let's\slash I'll swim \\ \cline{1-4}

\slash su\slash -Ending \emph{Godan }Verbs & 話す + う \textrightarrow  & 話そう & Let's\slash I'll talk \\ \cline{1-4}

\slash tsu\slash -Ending \emph{Godan }Verbs & 勝つ + う \textrightarrow  & 勝とう & Let's\slash I'll win \\ \cline{1-4}

\slash nu\slash -Ending \emph{Godan }Verbs & 死ぬ + う \textrightarrow  & 死のう & Let's\slash I shall die \\ \cline{1-4}

\slash mu\slash -Ending \emph{Godan }Verbs & 読む + う \textrightarrow  & 読もう & Let's\slash I'll read \\ \cline{1-4}

\slash ru\slash -Ending \emph{Godan }Verbs & 図る + う \textrightarrow  & 図ろう & Let's\slash I'll devise \\ \cline{1-4}

 \emph{Suru }Verbs & する + よう \textrightarrow  & しよう & Let's\slash I'll do \\ \cline{1-4}

 \emph{Kuru } & 来る + よう \textrightarrow  & 来(こ)よう & Let's\slash I'll come \\ \cline{1-4}

\end{ltabulary}
      
\section{~だろう}
 
\par{ ~だろう comes from the volitional form of だ. It is often shortened to ~だろ. ~だろう is often not used by females due to the brisk tone it often gives. }
 
\par{1. Used to show guess. It may follow nouns, adjectives, and verbs. }
 
\par{${\overset{\textnormal{}}{\text{41. 明日}}}$ は ${\overset{\textnormal{}}{\text{雨}}}$ が ${\overset{\textnormal{}}{\text{降}}}$ るだろう。 \hfill\break
It will probably rain tomorrow. }

\par{42. ${\overset{\textnormal{けっきょく}}{\text{結局}}}$ は ${\overset{\textnormal{あっか}}{\text{悪化}}}$ するだろう。 \hfill\break
It will surely get worse. }
 
\par{${\overset{\textnormal{}}{\text{43. 出席者}}}$ は ${\overset{\textnormal{たかだか}}{\text{高々}}}$ 10 ${\overset{\textnormal{}}{\text{人}}}$ だろう。 \hfill\break
There will be no more than ten attendees. }
 
\par{44. あの ${\overset{\textnormal{ようす}}{\text{様子}}}$ からして ${\overset{\textnormal{}}{\text{離婚}}}$ は ${\overset{\textnormal{まぢか}}{\text{間近}}}$ だろう。 \hfill\break
Based on that condition, divorce is surely close. }

\par{45. ${\overset{\textnormal{めいわく}}{\text{迷惑}}}$ だろう。 \hfill\break
It's probably a bother. }
 
\par{2. ~だろうか may be used to express personal doubt, especially in one's inner monologue. This is the case for both men and women, and in this sense, it can be translated as "I wonder\dothyp{}\dothyp{}\dothyp{}" In the spoken language when paired with a rising intonation, it can be used to direct serious doubt at someone about something or that individual. }
 
\par{${\overset{\textnormal{}}{\text{46. 何時}}}$ だろうか。 \hfill\break
I wonder what time it is. }
 
\par{47. いつ ${\overset{\textnormal{}}{\text{行}}}$ うだろうか。 \hfill\break
I wonder when he'll carry it out. }
 
\par{${\overset{\textnormal{}}{\text{48. 誰}}}$ が ${\overset{\textnormal{ぎじどう}}{\text{議事堂}}}$ に ${\overset{\textnormal{}}{\text{行}}}$ くのだろうか。 \hfill\break
Who would go to the Diet? }
 
\par{49. あんな ${\overset{\textnormal{ばか}}{\text{馬鹿}}}$ な ${\overset{\textnormal{}}{\text{行為}}}$ が ${\overset{\textnormal{}}{\text{許}}}$ されるだろうか。 \hfill\break
How would such a stupid action be allowed? }
 
\par{${\overset{\textnormal{}}{\text{50. 犯人}}}$ なのではないだろうか。 \hfill\break
Isn't he supposed the criminal? }

\par{51. お ${\overset{\textnormal{}}{\text{前}}}$ も ${\overset{\textnormal{}}{\text{来}}}$ るだろう?(Casual\slash Masculine) \hfill\break
Aren't you coming too? }

\par{\textbf{Grammar Note }: When the particle か is dropped like in Ex. 51, the speaker is strongly seeking affirmation from the listener. }
  
\par{3. "\dothyp{}\dothyp{}\dothyp{}だろうが" and "\dothyp{}\dothyp{}\dothyp{}だろうと" show supposition and mean "even (in\slash as)" or "no matter". The former is used in the sense of "even as" whereas the latter is used in the sense of "even in\slash no matter." }

\par{52. ${\overset{\textnormal{うてん}}{\text{雨天}}}$ だろうと ${\overset{\textnormal{}}{\text{決行}}}$ するつもりです。 \hfill\break
I plan to carry it out even in rainy weather }
 
\par{${\overset{\textnormal{}}{\text{53. 子供}}}$ だろうが ${\overset{\textnormal{ようしゃ}}{\text{容赦}}}$ はしない。 \hfill\break
Even as children, they don't show mercy. }
 
\par{54. どんな ${\overset{\textnormal{}}{\text{人}}}$ だろうと、この ${\overset{\textnormal{}}{\text{映画}}}$ は ${\overset{\textnormal{}}{\text{楽}}}$ しめます。 \hfill\break
No matter what kind of person you are, you can enjoy this movie. }
 
\par{4. ~だろうに means "even though it's supposed to be". When seen at the end of a sentence, it is often translated as "how I wish!" }
 
\par{55. 苦しかっただろうに、よく頑張った。 \hfill\break
Even though it was supposed to be painful, he persevered well. }
 
\par{${\overset{\textnormal{}}{\text{56. 人生}}}$ をもう ${\overset{\textnormal{}}{\text{一度}}}$ やりなおせたらどんなにいいだろうに。 \hfill\break
How I wish I could live my life again! }

\par{57. もう少し早く出れば ${\overset{\textnormal{ま}}{\text{間}}}$ に合っただろうに。 \hfill\break
If we only left a little bit more early, we would have made it on time. }
      
\section{~でしょう}
 
\par{  For the most part, ~でしょう is the polite form of ~だろう. ~でしょう comes from the combination of です and ~う. It is often shortened to ~でしょ in casual speech. It has largely replaced だろう whenever it isn't followed by some particle. However, it is possible to see ~でしょうに. }
 
\par{${\overset{\textnormal{}}{\text{58. 今晩}}}$ は ${\overset{\textnormal{}}{\text{雪}}}$ でしょう。 \hfill\break
It's probably snow this evening. }
 
\par{${\overset{\textnormal{}}{\text{59. 戻}}}$ ってくるでしょう。 \hfill\break
It'll probably return. }

\par{60. ${\overset{\textnormal{くすり}}{\text{薬}}}$ で ${\overset{\textnormal{ずつう}}{\text{頭痛}}}$ は ${\overset{\textnormal{おさ}}{\text{収}}}$ まるでしょう。 \hfill\break
Headache should subside with medicine. }
 
\par{61. よろしいでしょうか。 \hfill\break
Will this be alright? }
 
\par{${\overset{\textnormal{}}{\text{62. 彼女}}}$ は ${\overset{\textnormal{}}{\text{本当}}}$ に ${\overset{\textnormal{よろこ}}{\text{喜}}}$ ぶでしょう。 \hfill\break
I think it'll probably make her really happy. }
 
\par{${\overset{\textnormal{}}{\text{63. 間}}}$ に ${\overset{\textnormal{}}{\text{合}}}$ わないのではないでしょうか。 \hfill\break
Are we not going to be able to make it? }

\begin{center}
\textbf{~ますでしょうか } 
\end{center}

\par{ ~でしょうか normally goes after the plain form, but it's occasionally after ~ます in attempts to be more honorific. }

\par{64. 日本に来られて何年になりますでしょうか。 \hfill\break
How many years has it been since you've come to Japan? }

\par{65. お分かりになりますでしょうか。 \hfill\break
Do you understand? }
    