    
\chapter{Daily Expressions II}

\begin{center}
\begin{Large}
第105課: Daily Expressions II 
\end{Large}
\end{center}
 
\par{ This lesson is a continuation of Lesson 25. Currently, coverage that was previously in the old versions of these lessons are being expanded. As this coverage is completed, this lesson will be moved next to Lesson 25. At which point, Lesson 105 will become one of the new offshoots. }
      
\section{How are You?}
 
\par{ Many Westerners are familiar with the phrase \emph{o-genki desu ka? }お元気ですか, which translates as “how are you?” The word \emph{genki }元気 is an adjectival noun meaning “lively\slash healthy.” Because one can usually tell whether someone is doing alright in this regard, \emph{o-genki desu ka? }お元気ですか is most often used over the phone or via letter where you aren\textquotesingle t able to directly ascertain how the person is (Ex. 1 and 2). Or, it is used in asking others about how people not present are doing. }
 
\par{1. お ${\overset{\textnormal{げんき}}{\text{元気}}}$ ですか。 ${\overset{\textnormal{わたし}}{\text{私}}}$ は ${\overset{\textnormal{げんき}}{\text{元気}}}$ です。 \hfill\break
 \emph{O-genki desu ka? Watashi wa genki desu. \hfill\break
 }How are you? I\textquotesingle m fine. }
 
\par{2. お ${\overset{\textnormal{ひさ}}{\text{久}}}$ しぶりです。お ${\overset{\textnormal{げんき}}{\text{元気}}}$ ですか。 \hfill\break
 \emph{O-hisashiburi desu. O-genki desu ka? \hfill\break
 }It\textquotesingle s been a while. How are you? }
 
\par{3. ご ${\overset{\textnormal{かぞく}}{\text{家族}}}$ はお ${\overset{\textnormal{げんき}}{\text{元気}}}$ ですか。 \hfill\break
 \emph{Go-kazoku wa o-genki desu ka? \hfill\break
 }How is your family? }
 
\par{4. ご ${\overset{\textnormal{りょうしん}}{\text{両親}}}$ はお ${\overset{\textnormal{げんき}}{\text{元気}}}$ ですか。 \hfill\break
 \emph{Go-ryōshin wa o-genki desu ka? \hfill\break
 }How are your parents doing? }
 
\par{5. ${\overset{\textnormal{きむら}}{\text{木村}}}$ さん、お ${\overset{\textnormal{げんき}}{\text{元気}}}$ ですか。 \hfill\break
 ${\overset{\textnormal{おく}}{\text{奥}}}$ さんもお ${\overset{\textnormal{か}}{\text{変}}}$ わりありませんか。 \hfill\break
 \emph{Kimura-san, o-genki desu ka? \hfill\break
Oku-san mo o-kawari arimasen ka? }\hfill\break
Kimura-san, how are you? \hfill\break
Is your wife also doing well? }
 
\par{\textbf{Phrase Note }: お変わりありませんか literally means “have there been any changes?” }
 
\par{6. おかげで、お ${\overset{\textnormal{かあ}}{\text{母}}}$ さんも ${\overset{\textnormal{ぼく}}{\text{僕}}}$ も ${\overset{\textnormal{げんき}}{\text{元気}}}$ です。 \hfill\break
 \emph{O-kage de, o-k }\emph{āsan mo boku mo genki desu. \hfill\break
 }Thankfully, mother and I are also well. }
 
\par{ If there is an existing relationship with who you are speaking and that individual need not be addressed with particularly formal language, then this may also be simply said as \emph{genki desu ka? }元気ですか. This is best used in situation where you\textquotesingle re addressing people informally. }
 
\par{7. みんな ${\overset{\textnormal{げんき}}{\text{元気}}}$ ですか! \hfill\break
 \emph{Min\textquotesingle na genki desu ka? \hfill\break
 }How is everyone! }
 
\par{ Like the English equivalent, it is mostly used as a greeting. It is not used to mean “are you ok?” when you\textquotesingle re worried about someone. It is also not a greeting that is used every day. Even if It is used directly with someone, it wouldn\textquotesingle t sound like you really know the individual at a personal level. Although students use it as an everyday greeting, this usage is unnatural and not reflective of how it is actually used by native speakers. }
 
\par{\textbf{Politeness Note }: To make \emph{o-genki desu ka? }お元気ですか  politer, replace \emph{desu }です with \emph{desh }\emph{ō }でしょう. }
 
\par{ If you haven\textquotesingle t seen someone in a while and want to know how they\textquotesingle ve been, it\textquotesingle s best to ask \emph{(o-)genki deshita ka? }(お)元気でしたか。Casually, this would become \emph{genki (datta)? }元気(だった)? }
 
\par{8. ${\overset{\textnormal{さっ}}{\text{幸}}}$ ちゃん、 ${\overset{\textnormal{げんき}}{\text{元気}}}$ ? \hfill\break
 \emph{Satchan, genki? \hfill\break
 }Satchan! How are you? }
 
\par{ In far more formal situations, other phrases may be appropriate. }
 
\par{9. ご ${\overset{\textnormal{ぶじ}}{\text{無事}}}$ でいらっしゃいますか。 \hfill\break
 \emph{Go-buji de irasshaimasu ka? \hfill\break
 }Have you been safe and well? }
 
\par{\textbf{Phrase Note }: This phrase would be used in writing. \emph{De irasshaimasu ka? }でいらっしゃいますか is a very formal, respectful form of the copula. You can alternatively make this not so formal by replacing it with \emph{desu }です. In the spoken language, you can ask people \emph{buji desu ka? }無事ですか to ask about the safety of everyone, which is used frequently when disasters happen. }
 
\par{10. ご ${\overset{\textnormal{きげん}}{\text{機嫌}}}$ いかがですか。 \hfill\break
 \emph{Go-kigen ikaga desu ka? \hfill\break
 }How are you? }
 
\par{\textbf{Sentence Note }: \emph{Kigen }機嫌 means “mood.” This phrase more so literally means “how do you do?” but it isn\textquotesingle t old-fashioned like this English counterpart. For the most part, it\textquotesingle s treated as a more formal, elegant replacement for \emph{o-genki desu ka? }お元気ですか. However, it isn\textquotesingle t appropriate in business because it isn\textquotesingle t the case that clients\slash customers are always in high spirits, and it isn\textquotesingle t the right place to assume this. }
 
\par{11. ご ${\overset{\textnormal{ぶさた}}{\text{無沙汰}}}$ しております。お ${\overset{\textnormal{げんき}}{\text{元気}}}$ でいらっしゃいますか。 \hfill\break
 \emph{Go-busata shite orimasu. O-genki de irasshaimasu ka? \hfill\break
 }I\textquotesingle m sorry for not hearing from you all this time. Are you doing well? }
 
\par{\textbf{Sentence Note }: Imagine if you haven\textquotesingle t spoken or heard from a superior or someone of high social status for a while. Although blame for this lack of communication could be on both sides, a very honorific opening such as this would be very appropriate. }
 
\par{12. ${\overset{\textnormal{なが}}{\text{長}}}$ らくご ${\overset{\textnormal{ぶさた}}{\text{無沙汰}}}$ してすみません。 \hfill\break
 \emph{Nagaraku go-busata shite sumimasen. \hfill\break
 }I apologize for not hearing from you in so long. }
 
\par{13. ご ${\overset{\textnormal{ぶさた}}{\text{無沙汰}}}$ \{しました・いたしました\}。 \hfill\break
 \emph{Go-busata [shimashita\slash itashimashita]. \hfill\break
 }It\textquotesingle s been a long time. }
 
\par{\textbf{Sentence Note }: This phrase would be appropriate especially when you\textquotesingle re recognizing how long it\textquotesingle s been since you\textquotesingle ve updated people. \emph{Itashimashita }いたしました is a more humble version of \emph{shimashita }しました. }
 
\par{14. いかがお ${\overset{\textnormal{す}}{\text{過}}}$ ごしでしょうか。 \hfill\break
 \emph{Ikaga o-sugoshi deshō ka? \hfill\break
 }How are things with you? }
 
\par{\textbf{Sentence Note }: This phrase is also quite honorific and is appropriate in very formal situations, both written and spoken. }
 
\par{15. しばらくでしたね。 \hfill\break
 \emph{Shibaraku deshita ne. \hfill\break
 }It\textquotesingle s been a while, hasn\textquotesingle t it? }
 
\begin{center}
\textbf{Responding }
\end{center}
 
\par{ In response to being asked how you are doing, use phrases like \emph{ē, o-kage-sama de }ええ、おかげさまで. This is equivalent to “Yes, I\textquotesingle m fine. Thank you.” This can also be used in the sense of “It went fine\slash I did, thank you.” }
 
\par{16. 「ご ${\overset{\textnormal{かぞく}}{\text{家族}}}$ の ${\overset{\textnormal{みな}}{\text{皆}}}$ さんはお ${\overset{\textnormal{げんき}}{\text{元気}}}$ ですか」「ええ、おかげさまで、みんな ${\overset{\textnormal{げんき}}{\text{元気}}}$ にしていますよ」 \hfill\break
 \emph{“Go-kazoku no mina-san wa o-genki desu ka?” “Ē, o-kage-sama de, min\textquotesingle na genki ni shite imasu yo” \hfill\break
 }“How is everyone in your family doing?” “They\textquotesingle re all doing fine, thank you.” }
 
\par{17. 「 ${\overset{\textnormal{しあい}}{\text{試合}}}$ はうまくいきましたか」「ええ、おかげさまで」 \hfill\break
 \emph{“Shiai wa umaku ikimashita ka?” “Ē, o-kage-sama de.” }\hfill\break
“Did your match go okay?” “Yes, it did, thankfully.” }
      
\section{Long Time No See}
 
\par{ The Japanese equivalent of “long time no see” is \emph{(o-)hisashiburi desu (ne) }(お)久しぶりです(ね). The use of \emph{o- }お at the beginning is determined by whether the person in question is someone you ought to give respect to. The use of \emph{ne }ね at the end is determined by whether you wish to imply a mutual understanding that it\textquotesingle s been a while since you\textquotesingle ve last seen that person. }
 
\par{ Casually, “long time no see” can be expressed simply as \emph{hisashiburi }久しぶり or alternatively as \emph{hisabisa da ne }久々だね. }
 
\par{18. 皆さん、お ${\overset{\textnormal{ひさ}}{\text{久}}}$ しぶりです。 \hfill\break
 \emph{Mina-san, o-hisashiburi desu. \hfill\break
 }Long time no see, everyone. }
 
\par{19. おお、 ${\overset{\textnormal{ひさびさ}}{\text{久々}}}$ だね! \hfill\break
 \emph{Ō, hisabisa da ne! \hfill\break
 }Whoa, long time no see! }
 
\par{20. 調子はどうですか? \hfill\break
 \emph{Ch }\emph{ōshi wa d }\emph{ō desu ka? \hfill\break
 }How are you doing? }
 
\par{\textbf{Sentence Note }: This expression is frequently used when meeting someone after a while. \emph{Ch }\emph{ōshi }調子 in this context means “state of health.” To say this casually, just drop \emph{desu ka? }ですか. }
 
\par{21. ${\overset{\textnormal{さいきん}}{\text{最近}}}$ どう? \hfill\break
 \emph{Saikin d }\emph{ō? \hfill\break
 }How\textquotesingle ve you been recently? }
 
\par{\textbf{Sentence Note }: To make this polite, just add \emph{desu ka? }ですか. A simple reply would be \emph{m }\emph{ām }\emph{ā (desu) }まあまあ(です). }
 
\par{22. ずいぶんお ${\overset{\textnormal{みかぎ}}{\text{見限}}}$ りでしたね。 \hfill\break
 \emph{Zuibun o-mikagiri deshita ne. \hfill\break
 }I haven\textquotesingle t seen you in ages. }
 
\par{\textbf{Sentence Note }: This phrase isn\textquotesingle t all that common, but whenever it is used, there is typically a slight amount of sarcasm implied, hinting at how the other person hasn\textquotesingle t been available to see. }
 
\par{23. おひさ! \hfill\break
 \emph{O-hisa! \hfill\break
 }Hey, long time no see! }
\textbf{Sentence Note }: This expression is no longer really used—a \emph{shigo }死語—but it is still a notable contraction of \emph{o-hisashiburi }お久しぶり.       
\section{Congratulations: O-iwai no Kotoba お祝いの言葉}
 
\par{ The Japanese expression for “congratulations” is \emph{omedetō (gozaimasu) }おめでとう(ございます). The full version is, of course, the polite form. This phrase is seen at the end of congratulatory phrases. }
 
\par{24. (お) ${\overset{\textnormal{たんじょうび}}{\text{誕生日}}}$ おめでとうございます。 \hfill\break
 \emph{(O-)tanj }\emph{ōbi omedet }\emph{ō gozaimasu. \hfill\break
 }Happy birthday. }
 
\par{25. ${\overset{\textnormal{あ}}{\text{明}}}$ けましておめでとうございます。 \hfill\break
 \emph{Akemashite omedet }\emph{ō gozaimasu. \hfill\break
 }Happy New Year. }
 
\par{26. ご ${\overset{\textnormal{にゅうがく}}{\text{入学}}}$ おめでとうございます。 \hfill\break
 \emph{Go-ny }\emph{ūgaku omedet }\emph{ō gozaimasu. \hfill\break
 }Congratulations on enrollment. }
 
\par{27. ご ${\overset{\textnormal{けっこん}}{\text{結婚}}}$ おめでとうございます。 \hfill\break
 \emph{Go-kekkon omedet }\emph{ō gozaimasu. \hfill\break
 }Congratulations on your marriage. }
 
\par{28. ご ${\overset{\textnormal{しゅっさん}}{\text{出産}}}$ おめでとうございます。 \hfill\break
 \emph{Go-shussan omedet }\emph{ō gozaimasu. \hfill\break
 }Congratulations on giving birth. }
      
\section{Please}
 
\par{ The word for “please” as in “by all means” is \emph{d }\emph{ōzo }どうぞ. It is often paired with the verbal ending \emph{-te kudasai }~てください, which creates a polite command\slash request. }

\par{29. どうぞ。 \hfill\break
 \emph{D }\emph{ōzo. \hfill\break
 }Please, by all means. }

\par{30. どうぞ ${\overset{\textnormal{あ}}{\text{上}}}$ がってください。 \hfill\break
 \emph{D }\emph{ōzo agatte kudasai. \hfill\break
 }Please, come in. }

\par{\textbf{Sentence Note }: This phrase is used when letting people into one\textquotesingle s home. }

\par{31. お ${\overset{\textnormal{さき}}{\text{先}}}$ にどうぞ。 \hfill\break
 \emph{O-saki ni d }\emph{ōzo. \hfill\break
 }Please go ahead. }

\par{32. どうぞお ${\overset{\textnormal{かま}}{\text{構}}}$ いなく。 \hfill\break
 \emph{D }\emph{ōzo o-kamai naku. \hfill\break
 }Please don\textquotesingle t fuss over me. }

\par{33. どうぞこちらへ。 \hfill\break
 \emph{D }\emph{ōzo kochira e. \hfill\break
 }This way, please. }

\par{34. どうぞ ${\overset{\textnormal{め}}{\text{召}}}$ し ${\overset{\textnormal{あ}}{\text{上}}}$ がってください。 \hfill\break
 \emph{D }\emph{ōzo meshiagatte kudasai. \hfill\break
 }Please enjoy your meal. }

\par{ If there is no sense of “by all means,” you should simply using \emph{-te kudasai }~てください.  Of course, in casual speech, you are free to drop \emph{kudasai }ください. }
 
\par{35. ${\overset{\textnormal{しょるい}}{\text{書類}}}$ を ${\overset{\textnormal{まと}}{\text{纏}}}$ めてください。 \hfill\break
 \emph{Shorui wo matomete kudasai. \hfill\break
 }Please compile the documents. }
      
\section{Meeting People}
 
\par{ When meeting people for the first time regardless of the appropriate speech style for the occasion, Japanese people greet each other by saying \emph{hajimemashite }初めまして. Following this, most people state their name. At the end, speakers will tell each other to keep the other in good thought. The phrase that\textquotesingle s said for this is \emph{yoroshiku onegai [shimasu\slash itashimasu] }よろしくお願い\{します・いたします\}. Although there is a lot that can be said about it, for now treat it as a set phrase set after introducing oneself. Typically, \emph{kochira koso }こちらこそ (likewise) is added when you\textquotesingle re not the first to introduce oneself. \hfill\break
}

\par{\textbf{Intonation Note }: は \textbf{じめま }して. }

\par{36. \hfill\break
 \hfill\break
はじめまして、 ${\overset{\textnormal{ささき}}{\text{佐々木}}}$ \{と ${\overset{\textnormal{もう}}{\text{申}}}$ します・です\}。 \hfill\break
よろしくお ${\overset{\textnormal{ねが}}{\text{願}}}$ い\{します・いたします\}。 \hfill\break
 \emph{Hajimemashite, Sasaki [to m }\emph{ōshimasu\slash desu]. \hfill\break
Yoroshiku o-negai [shimasu\slash itashimasu]. \hfill\break
 }Nice to meet you. I\textquotesingle m Sasaki. Please keep me in good thought. }
 
\par{\textbf{Sentence Note }: The phrase \emph{to m }\emph{ōshimasu }と申します literally means “am called.” }
      
\section{Asking OK}
 
\par{ There are two broad usages of the word “okay” that you will need to separate when speaking in Japanese. The first application of “okay” is asking if someone is alright. The second application is asking if something is alright. If you are concerned about the well-being of someone, use Ex. 37. If you are concerned about something being okay, use expressions as seen with Exs. 38-40. }
 
\par{37. ${\overset{\textnormal{だいじょうぶ}}{\text{大丈夫}}}$ ですか。 \hfill\break
 \emph{Daij }\emph{ōbu desu ka? \hfill\break
 }Are you okay? \hfill\break
Is that alright\slash okay? }
 
\par{\textbf{Sentence Note }: To make this casual, just drop \emph{desu ka? }ですか. The response, regardless of whether one is really okay or not, is usually \emph{daij }\emph{ōbu (desu) }大丈夫(です). }
 
\par{38. よろしいですか。 \hfill\break
 \emph{Yoroshii desu ka? \hfill\break
 }Is that alright\slash okay? }
 
\par{39. それでも\{いい・大丈夫\}ですか。 \hfill\break
 \emph{Sore demo [ii\slash daij }\emph{ōbu] desu ka? \hfill\break
 }Is it okay even if it\textquotesingle s that? }
 
\par{40. ${\overset{\textnormal{ほんとう}}{\text{本当}}}$ にいいの? \hfill\break
 \emph{Hont }\emph{ō ni ii no? \hfill\break
 }Are you sure it\textquotesingle s okay? }
\textbf{Sentence Note }: To make this polite, follow \emph{ii }いい with \emph{n desu ka? }んですか.     