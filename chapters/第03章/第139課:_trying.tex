    
\chapter{Try I}

\begin{center}
\begin{Large}
第139課: Try I: ~てみる・みたい・みせる 
\end{Large}
\end{center}
 
\par{  The word "try" has a lot of different meanings. When it is translated into Japanese, there are many phrases to consider. The phrases themselves are not difficult to use or understand, but because they all have the same English translation, it may become troublesome to differentiate between them. So, in the next two lessons, we will go through each applicable phrase in great detail so that you properly learn how to express the many "try" phrases of Japanese. }
      
\section{~てみる\slash ~てみたい}
 
\par{ As you know already, 見る means "to see." When it is after the particle て, it means "to try to\dothyp{}\dothyp{}\dothyp{}" in the sense of doing something for the sake of seeing what happens. There isn't any negative consequences necessarily implied by using it. }

\par{${\overset{\textnormal{}}{\text{1. 新}}}$ しいコートを ${\overset{\textnormal{}}{\text{着}}}$ てみる。 \hfill\break
I'll try a new coat on. }

\par{2. よく ${\overset{\textnormal{}}{\text{考}}}$ えてみる(よ)。 \hfill\break
I'll think it over. }

\par{3. カレーライスを ${\overset{\textnormal{}}{\text{食}}}$ べてみる。 \hfill\break
To try to eat curry rice. }
4. みんなに聞いてみたけど。 \hfill\break
I've tried asking everyone, but\dothyp{}\dothyp{}\dothyp{} 
\par{5. 一か八かやってみる。 \hfill\break
To take a chance\slash sink or swim. }

\par{\textbf{Reading Note }: 八 is read as ばち in the phrase above. Other readings of the entire phrase other than いちかばちか is wrong. }

\par{6. たまった宿題を ${\overset{\textnormal{いっぺん}}{\text{一遍}}}$ に片付けてみよう。 \hfill\break
I'm going to try to finish up this piled homework at once. }

\par{\textbf{Grammar Note }: In this sentence, ~よう is used to emphasize one's volition to finish the homework. }

\begin{center}
 \textbf{~てみたい }
\end{center}

\par{ ~てみたい means "to want to try to\dothyp{}\dothyp{}\dothyp{}" Again, this "try" is the same "try" as in "trial and error" (試行錯誤). These patterns are not intended to describe "trials" that last longer than an occurrence. }

\par{${\overset{\textnormal{}}{\text{7. 本当}}}$ に(お)すしを ${\overset{\textnormal{}}{\text{食}}}$ べてみたいです。 \hfill\break
I really want to try sushi. }

\par{\textbf{漢字 Note }: Sushi may also be spelled as 寿司, 鮨, or rarely as 鮓. }

\par{8. やってみたい。 \hfill\break
I want to try. }

\par{${\overset{\textnormal{}}{\text{9. 月見}}}$ バーガーを ${\overset{\textnormal{}}{\text{食}}}$ べてみたい。 \hfill\break
I want to try a tsukimi burger. }

\par{\textbf{Word Note }: It might as well be called egg burger. }

\par{10. 誰か ${\overset{\textnormal{}}{\text{一度会}}}$ ってみたい ${\overset{\textnormal{}}{\text{人}}}$ がいますか。 \hfill\break
Is there someone you would like to meet once? }
 
\par{11. みんなを笑いの ${\overset{\textnormal{うず}}{\text{渦}}}$ に巻き込んでみたい。 (Idiomatic) \hfill\break
I want to put a smile on everyone's faces. }

\par{\textbf{Orthography Note }: In Modern Japanese spelling, helper verbs like みる in ~てみる are written as such in ひらがな. Writing the verbs in 漢字 would not be wrong, but the verb may be interpreted literally. }

\par{\textbf{Colloquialism Note }: In casual speech, you can tell someone to try to do something by using ~てみい. }

\begin{center}
\textbf{~てごらん }
\end{center}

\par{  This is used to entice someone to try to do something. ごらん is from the honorific form of みる, ご覧になる. This pattern comes from the abbreviation of the command form of this expression, ご覧なさい. }

\par{12. おいしいから食べてごらん。 \hfill\break
Try it. It's delicious. }

\par{13. そのサボテンを ${\overset{\textnormal{と}}{\text{跳}}}$ び ${\overset{\textnormal{こ}}{\text{越}}}$ えてごらん! \hfill\break
Try jumping over the cactus! }

\par{\textbf{漢字 Note }: The 漢字 for サボテン is 仙人掌. You don't need to remember this. }

\par{\textbf{Honorifics Note }: Some speakers do not like ~てご覧なさい and feel that it lacks etiquette. However, as is found in dictionaries and is the case for most speakers, it is completely fine and the correct way of making ~てみる honorific. }

\par{14. 来てごらんなさい。 \hfill\break
Please come [forth]. }

\par{15. 耳を ${\overset{\textnormal{す}}{\text{澄}}}$ ませてご覧なさい。 \hfill\break
Please try to listen carefully. }

\par{16. 目を閉じてご覧なさい。 \hfill\break
Try closing your eyes. }

\par{ However, those who feel ~てご覧なさい lacks respect replace it with ~てみてください with an already honorific verb. So, the difference lies in where the honorific element of the verb phrase is. }

\par{17a. 訂正されてみてください。△ \hfill\break
17b. 訂正なさってみてください。〇 \hfill\break
Please try to revise\slash edit it. }

\par{ This shows that lowering honorific standards causes problems. The first sentence may make someone sound uneducated. The second sentence comes from fears of being too pushy, which ~なさい in other contexts gives. }

\par{ To complicate things more, ~てご覧なさい may not always be appropriate. However, ~てみる is typically not used in other conjugations in honorific fashion as mentioning superiors trying things is not formal. Rather, things like ~ていただく "receiving the act of\dothyp{}\dothyp{}\dothyp{}(by\slash from\dothyp{}\dothyp{}\dothyp{}.)" would be used. So, "I received the boss eating X" rather than "X tried eating X". }

\begin{center}
 \textbf{試す }
\end{center}

\par{ This verb means "to try out" and is used a lot with ~てみる. }

\par{18. 自分の力を試してみる。 \hfill\break
To try one's strength. }

\par{19. もう一度試してみて。 \hfill\break
Have another try. }

\begin{center}
\textbf{試す VS 試みる } 
\end{center}

\begin{itemize}

\item 試す is used to mean "to test" in the sense of investigating means. 
\item 試みる is used to mean "to test" in the sense of ascertaining an end result. In other words, you're taking a go at it. 
\end{itemize}

\par{ The best way to see how these two verbs differ, it's important to see what sorts of phrases they are used in. Trying to escape from jail? 試みる will be your best choice. Simply trying to test your Japanese skills? 試す will be what you want. }

\par{20. ${\overset{\textnormal{だっそう}}{\text{脱走}}}$ を試みる。 \hfill\break
To test\slash have a got at an escape. }

\par{21. 能力を試す。 \hfill\break
To test ability. }

\par{22. 日本語能力試験をして実力を試してみたらどうですか。 \hfill\break
How about testing your true skills by taking the Japanese Proficiency Test? }

\par{23. 新車の乗り ${\overset{\textnormal{ごこち}}{\text{心地}}}$ を試す。 \hfill\break
To test the feel of a new car. }

\par{24. 別の方法を試みる。 \hfill\break
To try another method. }

\begin{center}
 \textbf{Other Miscellaneous "Try" Verbs } 
\end{center}

\par{ There are plenty of other phrases in Japanese that are translated into English as "try". However, they all correspond to specific usages. Meanings like in "trying one's utmost" or "test new products" are all expressed differently in Japanese. }

\par{25. 彼は全力を尽くした。 \hfill\break
He tried his hardest. }

\par{26. ラーメンを ${\overset{\textnormal{ししょく}}{\text{試食}}}$ する。 \hfill\break
To taste-test ramen. }

\par{27. 新製品を試用する。 \hfill\break
To test new products. }

\par{28. あらゆる ${\overset{\textnormal{どりょく}}{\text{努力}}}$ をする。 \hfill\break
To make every effort. }

\par{29. 誰がこの事件を\{ ${\overset{\textnormal{さば}}{\text{裁}}}$ く・ ${\overset{\textnormal{しんり}}{\text{審理}}}$ する\}のでしょうか。 \hfill\break
Who is going to try this case? }

\par{30. 彼は ${\overset{\textnormal{だつぜい}}{\text{脱税}}}$ ${\overset{\textnormal{ようぎ}}{\text{容疑}}}$ で裁判を受けた。 \hfill\break
He was tried for tax evasion. }
      
\section{~てみせる}
 
\par{ 見せる means "to show". After the particle て, it usually isn't written in 漢字. In this sense, it shows the showing of an action to someone. }

\par{31. ${\overset{\textnormal{てじな}}{\text{手品}}}$ をしてみせましょう。 \hfill\break
I will show you a trick! }

\par{32. 微笑んでみせる。 \hfill\break
To show a smile. }

\par{33. 何としても勝ってみせるぞ。 \hfill\break
No matter what it takes, I'll show you that I'll win. }

\par{34. 彼はグラフをコンピュータ画面に\{表示して・出して\}みせた。 \hfill\break
He displayed the graph on the computer screen. }

\par{35. 彼は素晴らしい勇敢さを見せた。 \hfill\break
He showed great courage. }

\par{36. 私の名前を漢字で書いてみせた。 \hfill\break
I showed them how to write my name in Kanji. }

\par{37. 将来必ず日本語の ${\overset{\textnormal{つうやく}}{\text{通訳}}}$ になってみせるよ。 \hfill\break
As for the future, I will certainly show that I will become a Japanese interpreter. }
    