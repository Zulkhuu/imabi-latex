    
\chapter{Conditionals I}

\begin{center}
\begin{Large}
第109課: Conditionals I: The Particles と, なら(ば), たら, \& ば 
\end{Large}
\end{center}
 
\par{ Creating an "if" statement in Japanese is not easy. This lesson will showcase to you three of the most common grammar points in Japanese to make conditional phrases. By no means, though, does this mean you will be able to master them completely by reading this lesson. The differences between these patterns are highly contextual, and it will still take a lot of practice to get the hang of them. So, by all means, please take your time as you read through this lesson. }
      
\section{The Conjunctive Particle と}
 
\par{  The conjunctive と can be described as being in the part of the construction of the following constructs. For all of these constructs, the particle follows the non-past tense of a verb. }
 
\begin{enumerate}
 
\item When X happens, Y will happen. For example, "when you      turn, there will be a bank".  
\item When X happens, Y happens. For example, "when I opened the      door, snow came in".  
\item When\slash if\dothyp{}\dothyp{}\dothyp{}, then\dothyp{}\dothyp{}\dothyp{}   (Certain)  
\end{enumerate}
 
\par{So, \textbf{"when X happens, Y happens" }. Y is predictable or an unavoidable fact . This is not an "if" statement. Even when it is translated with the word "if", Y is still something that is certain. So, this means that this pattern is \textbf{not correct for }\emph{requests, judgments, etc }. These sorts of things don't have a 100\% degree of certainty, and they can be easily refuted. }

\begin{center}
 \textbf{Examples }
\end{center}
 
\par{${\overset{\textnormal{}}{\text{1. 四月}}}$ になると、 ${\overset{\textnormal{さくら}}{\text{桜}}}$ が ${\overset{\textnormal{さ}}{\text{咲}}}$ く。 \hfill\break
When it becomes April, cherry blossoms will blossom. }
 
\par{${\overset{\textnormal{}}{\text{2. 先生}}}$ だと、きっと ${\overset{\textnormal{としうえ}}{\text{年上}}}$ なんじゃないですか。 \hfill\break
If he's a teacher, surely he has to be older, right? }

\par{3. ${\overset{\textnormal{びょういん}}{\text{病院}}}$ に ${\overset{\textnormal{}}{\text{行}}}$ くと ${\overset{\textnormal{}}{\text{藤原}}}$ さんがいました。 \hfill\break
When I went to the hospital, Mr. Fujiwara was there. }

\par{${\overset{\textnormal{}}{\text{4. 今}}}$ ${\overset{\textnormal{}}{\text{出}}}$ かけないと ${\overset{\textnormal{おく}}{\text{遅}}}$ れるよ。 \hfill\break
If you don't depart now, you'll be late. }

\par{5. ${\overset{\textnormal{まど}}{\text{窓}}}$ を ${\overset{\textnormal{あ}}{\text{開}}}$ けると、 ${\overset{\textnormal{}}{\text{牛}}}$ が ${\overset{\textnormal{}}{\text{見}}}$ えます。 \hfill\break
When you open the window, you can see cows. }
 
\par{${\overset{\textnormal{}}{\text{6. 日}}}$ が ${\overset{\textnormal{のぼ}}{\text{昇}}}$ ると、 ${\overset{\textnormal{}}{\text{明}}}$ るくなります。 \hfill\break
When the sun rises, it becomes bright. }
 
\par{${\overset{\textnormal{}}{\text{7. 日}}}$ が ${\overset{\textnormal{しず}}{\text{沈}}}$ むと、 ${\overset{\textnormal{}}{\text{暗}}}$ くなります。 \hfill\break
When the sun sets, it becomes dark. }

\par{8. ${\overset{\textnormal{こんぶ}}{\text{昆布}}}$ や何かで ${\overset{\textnormal{だし}}{\text{出汁}}}$ を ${\overset{\textnormal{}}{\text{取}}}$ るといい。 \hfill\break
It would be good to gather soup stock with \textbf{ }kombu or something. }
 
\par{9. 3に2を ${\overset{\textnormal{た}}{\text{足}}}$ すと、5になります。 \hfill\break
When we add 2 to 3, we get 5. }
 
\par{${\overset{\textnormal{}}{\text{10. 雪}}}$ が ${\overset{\textnormal{ふ}}{\text{降}}}$ り ${\overset{\textnormal{}}{\text{始}}}$ めると、 ${\overset{\textnormal{どうぶつ}}{\text{動物}}}$ たちは ${\overset{\textnormal{とうみん}}{\text{冬眠}}}$ を ${\overset{\textnormal{}}{\text{始}}}$ める。 \hfill\break
When it begins to snow, the animals begin hibernation. }

\par{11. ワカメやヒジキを食べると髪が生える。 \hfill\break
Your hair grows when you eat wakame and hijiki. }

\par{\textbf{Word Note }: ワカメ and ヒジキ are different kinds of edible seaweed. }

\par{12. ♪さかなさかなさかな~魚を食べると~♪あたまあたまあたま~頭がよくなる~♪ \hfill\break
Fish, fish, fish, when you eat fish, your brain gets smart, your brain gets smart, your brain gets smart. }

\par{\textbf{Practice (1) }: Translate the following. You may use a dictionary. }
 
\par{1. If you warm ice, it melts. \hfill\break
2. ${\overset{\textnormal{}}{\text{部屋}}}$ に入るとカーテンを ${\overset{\textnormal{}}{\text{開}}}$ けた。 \hfill\break
3. ボーイフレンドと京都に行きました。 \hfill\break
4. ドアを開けると、犬がいた。 ${\overset{\textnormal{まど}}{\text{窓}}}$ を開けると、鳥がいた。そして、ゴミ ${\overset{\textnormal{ばこ}}{\text{箱}}}$ を開けると、ねずみがいた。 }

\par{ \textbf{Other Usages }}

\par{These usages are really just extensions of above, but we'll look at them later. }

\par{~ようと \textrightarrow  Lesson 80   といい \textrightarrow  Lesson 95   ~ずと \textrightarrow  Lesson 133  }

\par{\textbf{Practice (8) }: Translate the following. You may use a dictionary. }

\par{1. If you warm ice, it melts. \hfill\break
2. ${\overset{\textnormal{へや}}{\text{部屋}}}$ に ${\overset{\textnormal{}}{\text{入}}}$ るとカーテンを ${\overset{\textnormal{あ}}{\text{開}}}$ けた。 \hfill\break
3. ボーイフレンドと ${\overset{\textnormal{}}{\text{京都}}}$ に ${\overset{\textnormal{}}{\text{行}}}$ きました。 \hfill\break
4. ドアを ${\overset{\textnormal{}}{\text{開}}}$ けると、 ${\overset{\textnormal{}}{\text{犬}}}$ がいた。 ${\overset{\textnormal{まど}}{\text{窓}}}$ を ${\overset{\textnormal{}}{\text{開}}}$ けると、 ${\overset{\textnormal{}}{\text{鳥}}}$ がいた。そして、ゴミ ${\overset{\textnormal{ばこ}}{\text{箱}}}$ を ${\overset{\textnormal{}}{\text{開}}}$ けると、ねずみがいた。 }
      
\section{The Conjunctive Particle なら(ば)}
 
\par{ なら may show \textbf{advice }. It can show things that would be realistic if something were to ever be the case. なら is called the \emph{contextual conditional }because it is equivalent to " \textbf{if you are talking about\dothyp{}\dothyp{}\dothyp{}, then\dothyp{}\dothyp{}\dothyp{} }". Overall, なら(ば) is used to give \emph{suggestions, speculation, or requests }. }

\par{${\overset{\textnormal{}}{\text{13. 中国}}}$ に ${\overset{\textnormal{}}{\text{行}}}$ くなら、 ${\overset{\textnormal{ばんり}}{\text{万里}}}$ の ${\overset{\textnormal{ちょうじょう}}{\text{長城}}}$ を ${\overset{\textnormal{み}}{\text{見}}}$ てください。 \hfill\break
If you are going to China, please see the Great Wall. }
 
\par{14. カメラを ${\overset{\textnormal{}}{\text{買}}}$ いたいなら、秋葉原に ${\overset{\textnormal{}}{\text{行}}}$ った ${\overset{\textnormal{}}{\text{方}}}$ がいいです。 \hfill\break
If you want to buy a camera, it's best to go to Akihabara. }

\par{15. ${\overset{\textnormal{ゆうびんきょく}}{\text{郵便局}}}$ に ${\overset{\textnormal{}}{\text{行}}}$ くなら、60 ${\overset{\textnormal{}}{\text{円}}}$ の ${\overset{\textnormal{きって}}{\text{切手}}}$ を5 ${\overset{\textnormal{}}{\text{枚}}}$ ${\overset{\textnormal{}}{\text{買}}}$ ってきてください。 \hfill\break
If you are going to the post office, please buy five 60 yen stamps. }
 
\par{16. いいえ、 ${\overset{\textnormal{}}{\text{売}}}$ り ${\overset{\textnormal{}}{\text{切}}}$ れましたが、 ${\overset{\textnormal{じゆうせき}}{\text{自由席}}}$ ならあります。 \hfill\break
No, they are sold out, but we have unreserved seats. }

\par{17. どうしてもというなら ${\overset{\textnormal{しかた}}{\text{仕方}}}$ がない。 \hfill\break
If you must, you have no choice. }

\par{18. それなら話は早い。 \hfill\break
That makes it easy. }

\par{${\overset{\textnormal{}}{\text{19. 仕事}}}$ を ${\overset{\textnormal{}}{\text{終}}}$ えたのなら、 ${\overset{\textnormal{かえ}}{\text{帰}}}$ っていいよ。 \hfill\break
You may leave provided\slash providing that you finished your work. }

\par{${\overset{\textnormal{}}{\text{20. 知}}}$ っているんなら、 ${\overset{\textnormal{}}{\text{教}}}$ えてください。 \hfill\break
Providing that you know, please tell me. }

\par{\textbf{Grammar Note }: なら \emph{expresses the speaker's supposition or defines the basis of the statement }. There is no required temporal ordering of the two situations presented. So, it can't be used to show a consequence in the past or condition for the future. }

\par{\textbf{Nuance Note }: のなら is stronger and enforces a confirmation of the content of the condition. }

\par{\textbf{Usage Note }: なら can also be used as a conjunction at the front of a sentence in broken down speech for それなら. }

\par{\textbf{Practice (2) }: Translate the following. \hfill\break
\hfill\break
1. ${\overset{\textnormal{かんこく}}{\text{韓国}}}$ に ${\overset{\textnormal{}}{\text{行}}}$ くなら(ば)、 ${\overset{\textnormal{ひこうき}}{\text{飛行機}}}$ の ${\overset{\textnormal{}}{\text{方}}}$ がよいです。 \hfill\break
2. みなが ${\overset{\textnormal{}}{\text{行}}}$ くなら、 ${\overset{\textnormal{}}{\text{私}}}$ も ${\overset{\textnormal{}}{\text{行}}}$ きます。 \hfill\break
3. ${\overset{\textnormal{}}{\text{電車}}}$ がないなら、 ${\overset{\textnormal{}}{\text{歩}}}$ くまでだ。 \hfill\break
4. ${\overset{\textnormal{おおさか}}{\text{大阪}}}$ に ${\overset{\textnormal{}}{\text{行}}}$ くなら、 ${\overset{\textnormal{しんかんせん}}{\text{新幹線}}}$ がいいですよ。 \hfill\break
5. その話なら ${\overset{\textnormal{なんど}}{\text{何度}}}$ も聞いているよ。 }
      
\section{The Conjunctive Particle たら}
 
\par{1. Shows an unrealized event being hypothesized, not completely certain. It's \textbf{often }preceded by もし. In a temporal condition, the next action is to happen after the action with たら. So, something happens \textbf{once certain conditions are met }. }

\par{21. ${\overset{\textnormal{かいぎ}}{\text{会議}}}$ が終わったら、 ${\overset{\textnormal{でんわ}}{\text{電話}}}$ しますね。 \hfill\break
I'll call you once the meeting ends, OK? }

\par{22. ${\overset{\textnormal{かねも}}{\text{金持}}}$ ちになったら、 ${\overset{\textnormal{しろ}}{\text{城}}}$ を ${\overset{\textnormal{}}{\text{買}}}$ う! \hfill\break
If I were rich, I'd so buy a castle! }
 
\par{23. もし ${\overset{\textnormal{}}{\text{雨}}}$ が ${\overset{\textnormal{}}{\text{降}}}$ らなかったら、 ${\overset{\textnormal{}}{\text{行}}}$ きましょう。 \hfill\break
If it doesn't rain, let's go. }

\par{ ${\overset{\textnormal{}}{\text{24. 雨}}}$ が( ${\overset{\textnormal{}}{\text{降}}}$ り)やんだら、どこかに ${\overset{\textnormal{}}{\text{行}}}$ くと ${\overset{\textnormal{}}{\text{決}}}$ めた。 \hfill\break
After it finished raining, we decided to go somewhere. }
 
\par{${\overset{\textnormal{}}{\text{25. 日本}}}$ に ${\overset{\textnormal{}}{\text{行}}}$ けたらいいなあ。 \hfill\break
It would be nice if I could go to Japan. }
 
\par{26. もし ${\overset{\textnormal{ちきゅう}}{\text{地球}}}$ に ${\overset{\textnormal{いんりょく}}{\text{引力}}}$ がなかったら、 ${\overset{\textnormal{くうちゅう}}{\text{空中}}}$ に ${\overset{\textnormal{う}}{\text{浮}}}$ いているでしょう。 \hfill\break
If there wasn't gravity on earth, you would probably be floating in the air. }

\par{27. ${\overset{\textnormal{じしん}}{\text{地震}}}$ が ${\overset{\textnormal{おさ}}{\text{収}}}$ まったら、 ${\overset{\textnormal{}}{\text{正}}}$ しい ${\overset{\textnormal{じょうほう}}{\text{情報}}}$ を ${\overset{\textnormal{}}{\text{聞}}}$ いてください。 \hfill\break
Once an earthquake subsides, get accurate information. }
 
\par{2. When you do X, Y happens. The latter part is typically unexpected. }
 
\par{28. ドアを開けたら、 ${\overset{\textnormal{かとう}}{\text{加藤}}}$ さんが ${\overset{\textnormal{}}{\text{立}}}$ っていました。 \hfill\break
When I opened the door, Mr. Kato was standing there. }
 
\par{${\overset{\textnormal{}}{\text{29. 本}}}$ を ${\overset{\textnormal{}}{\text{読}}}$ んであげたら、 ${\overset{\textnormal{}}{\text{弟}}}$ が ${\overset{\textnormal{よろこ}}{\text{喜}}}$ びました。 \hfill\break
When I read the book for my younger brother, he was very happy. }
 
\par{${\overset{\textnormal{}}{\text{30. 休}}}$ んだら、 ${\overset{\textnormal{}}{\text{元気}}}$ になった。 \hfill\break
When I rested, I got better. }
 
\par{${\overset{\textnormal{}}{\text{31. 銀行}}}$ に ${\overset{\textnormal{}}{\text{行}}}$ ったら、 ${\overset{\textnormal{ゆうじん}}{\text{友人}}}$ に ${\overset{\textnormal{}}{\text{会}}}$ いました。 \hfill\break
When I went to the bank, I saw a friend. }
 
\par{3. Shows an \textbf{obvious }conditional. This is a \textbf{generic condition }. This means that something is almost always followed by some situation. This "almost always" diction suggests that it is not as certain as と. }
 
\par{32. この ${\overset{\textnormal{かいしゃ}}{\text{会社}}}$ の ${\overset{\textnormal{しゃいん}}{\text{社員}}}$ だったら、あの ${\overset{\textnormal{だいがく}}{\text{大学}}}$ で ${\overset{\textnormal{わりびき}}{\text{割引}}}$ がもらえる。 \hfill\break
If you are an employee of this company, you can get a discount at that college. }
 
\par{${\overset{\textnormal{}}{\text{33. 人}}}$ を ${\overset{\textnormal{ころ}}{\text{殺}}}$ したら、 ${\overset{\textnormal{しけい}}{\text{死刑}}}$ になります。 \hfill\break
If you kill a person, you'll get the death penalty. }

\par{34. ${\overset{\textnormal{かぜ}}{\text{風邪}}}$ を ${\overset{\textnormal{}}{\text{引}}}$ いたら、くしゃみが ${\overset{\textnormal{で}}{\text{出}}}$ ます。 \hfill\break
If we catch a cold, we sneeze. }
 
\par{4. (の)だったら emphasizes a topic because of the condition of someone else. \hfill\break
 \hfill\break
35. やりたくないんだったらやめてくれ。(Vulgar) \hfill\break
If you don't want to do it, stop! }
 
\par{36. お ${\overset{\textnormal{}}{\text{父}}}$ さんだったら ${\overset{\textnormal{}}{\text{部屋}}}$ にいるよ。 \hfill\break
If you're talking about dad, he's in the room. }

\par{ たら may be たらば in formal situations. In hypothetical situations, もし(も) is \emph{almost always }added. Unlike と or なら, たら may be in a hypothetical yet unrealistic gesture. It's also the most common conditional because of its wide usage \textbf{. }However, it's often avoided in formal writing. }
      
\section{The Adverb もし}
 
\par{ We saw that in the first usage of たら, もし was always present. もし is related to hypothetical events. If you looked in a Japanese dictionary, you would find a definition similar to the following. }
 
\par{37. ある ${\overset{\textnormal{じたい}}{\text{事態}}}$ を ${\overset{\textnormal{かてい}}{\text{仮定}}}$ して ${\overset{\textnormal{の}}{\text{述}}}$ べることを ${\overset{\textnormal{あらわ}}{\text{表}}}$ す。 \hfill\break
It shows an event of hypothesizing and stating a certain condition. }
 
\par{When a clause with たら is used to show a hypothetical condition--where it may or may not happen--it is \textbf{usually }used with もし. もし is also used with counter-factual hypothetical statements where the event is extremely unrealistic. Even here, もし simply adds more \textbf{uncertainty }. }
 
\par{${\overset{\textnormal{}}{\text{38. 毎日すし}}}$ が ${\overset{\textnormal{}}{\text{食}}}$ べられたら、 ${\overset{\textnormal{}}{\text{私}}}$ は ${\overset{\textnormal{}}{\text{幸}}}$ せでしょう。 \hfill\break
If I could eat sushi every day, I would be very happy. }
 
\par{39. もし ${\overset{\textnormal{}}{\text{雨}}}$ が ${\overset{\textnormal{}}{\text{降}}}$ れば ${\overset{\textnormal{}}{\text{私}}}$ は ${\overset{\textnormal{}}{\text{家}}}$ にいます。 \hfill\break
If it rains tomorrow, I will stay at home. }
 
\par{40. もし ${\overset{\textnormal{}}{\text{私}}}$ があなただったらすこし ${\overset{\textnormal{}}{\text{待}}}$ つでしょう。 \hfill\break
If I were you, I'd wait a while. }
 
\par{41. もし ${\overset{\textnormal{}}{\text{君}}}$ が ${\overset{\textnormal{}}{\text{僕}}}$ の ${\overset{\textnormal{たちば}}{\text{立場}}}$ だったら、どうしますか? \hfill\break
Suppose you were in my place, what would you do? }
      
\section{The Conjunctive Particle ば}
 
\par{ ば creates the \emph{provisional conditional form }with the \textbf{${\overset{\textnormal{いぜんけい}}{\text{已然形}}}$ }. Never confuse this with the potential forms of verbs. There is a \textbf{difference }between ${\overset{\textnormal{}}{\text{買}}}$ えば and ${\overset{\textnormal{}}{\text{買}}}$ えれば. The first is ば with the basic form of the verb, and the second is ば with the potential form of the verb. The chart below shows how to conjugate with the ${\overset{\textnormal{}}{\text{已然形}}}$ . You are now responsible to know what this base is. }

\begin{ltabulary}{|P|P|P|P|P|P|P|P|}
\hline 

形容詞 & 形容動詞 & 一段 & 五段 & する & 来る & である & だ \\ \cline{1-8}

新し \textbf{けれ }ば & 簡単 \textbf{であれ }ば & 食べ \textbf{れ }ば & 歌 \textbf{え }ば & す \textbf{れ }ば & 来(く) \textbf{れ }ば &  \textbf{であれ }ば & ならば \\ \cline{1-8}

\end{ltabulary}

\par{ば shows that \textbf{\emph{the previous stated condition's establishment is the condition for the latter stated condition to occur }}. The subjects of both clauses should not show volition. So, although the subject may be the same in both clauses, the resultant outcome should be natural in such instance. }

\par{ This particle is perfect for showing desired result, so it would sound unnatural if the latter clause had some negative\slash undesired result specifically stated. This, then, does not mean "negative words" used in making suggestions\slash commands are then ungrammatical because you are soliciting a desired outcome. }

\par{ ば may also show the cue for a latter stated recognition or judgment. }

\par{42. ほかに ${\overset{\textnormal{}}{\text{意見}}}$ がなければ、これで ${\overset{\textnormal{}}{\text{終}}}$ わりましょう。 \hfill\break
If you don't have any other opinions, let's end here. }
 
\par{43. まあ、 ${\overset{\textnormal{むり}}{\text{無理}}}$ \{なら・であれば\}、 ${\overset{\textnormal{}}{\text{月曜日}}}$ までに ${\overset{\textnormal{}}{\text{出}}}$ してください。 \hfill\break
Well, if it's impossibility, turn it in by Monday. }

\par{ Like たら, it can show generic, temporal, and hypothetical conditions, but it's more forceful and places emphasis on the \textbf{future aspect }. The main clause shouldn't suggest with \textbf{an action }in the following manner. }

\par{44. 京都へ来たら、ぜひ連絡してください。〇  京都へ来れば、ぜひ連絡してください。X \hfill\break
Once you come to Kyoto, please contact me. }

\par{The suggestion is to do something once the condition of reaching Kyoto is met. Both conditionals can be followed by expressions reflecting the will of the speaker, but phrases with ば tend to be stronger and directed more specifically on something. }

\par{45. ${\overset{\textnormal{お}}{\text{押}}}$ せば、 ${\overset{\textnormal{ひら}}{\text{開}}}$ きます。 \hfill\break
It will open once you push it. }
 
\par{${\overset{\textnormal{}}{\text{46. 冬}}}$ になれば、 ${\overset{\textnormal{}}{\text{雪}}}$ が ${\overset{\textnormal{}}{\text{降}}}$ る。 \hfill\break
When it becomes winter, snow will fall. }
 
\par{\textbf{Grammar Note }: たら has a sequence requirement, but this "once" nuance of ば focuses on the conditional as the instigator for the latter to occur. So, it is stronger. }
 
\par{${\overset{\textnormal{}}{\text{47. 都合}}}$ がよければ、 ${\overset{\textnormal{いっしょ}}{\text{一緒}}}$ に ${\overset{\textnormal{}}{\text{大阪}}}$ に ${\overset{\textnormal{}}{\text{行}}}$ きませんか。 \hfill\break
How about going to Osaka together if it's convenient with you? }

\par{48. ${\overset{\textnormal{じしん}}{\text{自信}}}$ があれば ${\overset{\textnormal{なか}}{\text{半}}}$ ば ${\overset{\textnormal{せいこう}}{\text{成功}}}$ したも ${\overset{\textnormal{どうぜん}}{\text{同然}}}$ だ。(Proverb) \hfill\break
Confidence is half the battle. }

\par{${\overset{\textnormal{}}{\text{49. 見上}}}$ げれば、 ${\overset{\textnormal{ほし}}{\text{星}}}$ が ${\overset{\textnormal{ちゅう}}{\text{宙}}}$ をうねり、 ${\overset{\textnormal{やみ}}{\text{闇}}}$ が ${\overset{\textnormal{おお}}{\text{覆}}}$ っている。 \hfill\break
If you look up, the stars are projected in the air, and the dark is spreading. }
 
\par{50. あれだけ ${\overset{\textnormal{}}{\text{勉強}}}$ すれば、 ${\overset{\textnormal{ごうかく}}{\text{合格}}}$ するのも ${\overset{\textnormal{とうぜん}}{\text{当然}}}$ です。 \hfill\break
If you'll study to that extent, passing is only natural. }
 
\par{${\overset{\textnormal{}}{\text{51. 酒}}}$ も ${\overset{\textnormal{}}{\text{飲}}}$ めば、タバコも ${\overset{\textnormal{}}{\text{吸}}}$ う。 \hfill\break
If you drink sake, you also smoke. }

\par{52. ${\overset{\textnormal{かしゅ}}{\text{歌手}}}$ になりたいのであれば、 ${\overset{\textnormal{がっき}}{\text{楽器}}}$ の ${\overset{\textnormal{ひ}}{\text{弾}}}$ き ${\overset{\textnormal{かた}}{\text{方}}}$ も ${\overset{\textnormal{なら}}{\text{習}}}$ った ${\overset{\textnormal{ほう}}{\text{方}}}$ がよいのです。 \hfill\break
If it is the case that you wish to become a singer, it is best that you also learn how to play an instrument. }

\par{\textbf{Pronunciation Note }: Do not pronounce ${\overset{\textnormal{}}{\text{楽器}}}$ as がき because ${\overset{\textnormal{}}{\text{餓鬼}}}$ = brat. So, be careful. }

\par{\textbf{Practice (3) }: Translate the following. }

\par{1. ${\overset{\textnormal{}}{\text{高}}}$ ければ、 ${\overset{\textnormal{}}{\text{買}}}$ わない ${\overset{\textnormal{}}{\text{方}}}$ がいいですよ。 \hfill\break
2. 僕の家に来ればどう? \hfill\break
3. 犬と話せれば、楽しいでしょう。 \hfill\break
4. 勉強すれば、Aがもらえますよ。 \hfill\break
5. 安ければ、買った方がいいと思います。 }

\par{\textbf{Proverb Note }: ${\overset{\textnormal{いっけんきょ}}{\text{一犬虚}}}$ に ${\overset{\textnormal{ほ}}{\text{吠}}}$ ゆれば ${\overset{\textnormal{ばんけんじつ}}{\text{万犬実}}}$ を ${\overset{\textnormal{い}}{\text{云}}}$ う。This proverb means that "if a single dog barks a lie, ten thousand dogs will then speak it as the truth". This is meant to show that once a rumor spreads, it will eventually be taken as the truth. 吠ゆれば is the Classical Japanese form 吠えれば, and in proverbs such as this, old forms of grammar are common. }

\par{\textbf{Slang Note }: ~ば may be turned into ~や in slang. Generally, this is seen when れ precedes ば, and then the combination becomes りゃ. For instance, どうすりゃいいの? However, you can still say things like どう言いやいいんだよ. Of course, this would be very casual and slang and is rather 乱暴. }
      
\section{Keys}
 
\par{Practice (1): }

\par{1. 氷を ${\overset{\textnormal{あたた}}{\text{温}}}$ めると、 ${\overset{\textnormal{と}}{\text{溶}}}$ ける。 \hfill\break
2. When I entered the room, I opened the curtains. \hfill\break
3. I went to Kyoto with my boyfriend. \hfill\break
4. When I opened the door, there was a dog. When I opened the window, there was a bird. And, when I opened the trash can, there was a rat. }

\par{Practice (2): }

\par{1. If you are going to Korea, it is best to go by plane. \hfill\break
2. If everyone's going, I'll go too. \hfill\break
3. If there's not a train, you just got to walk. \hfill\break
4. If you are going to Osaka, the bullet train is good. \hfill\break
5. If it's that you're talking about, I've heard it numerous times. }

\par{Practice (3): }

\par{1. It's best not to buy it if it's expensive. \hfill\break
2. How about coming to my house? \hfill\break
3. It would probably be fun if you could take to dogs. \hfill\break
4. You'll be able to receive an A if you study. \hfill\break
5. If it\textquotesingle s cheap, I think that it\textquotesingle s best to buy it. }
    