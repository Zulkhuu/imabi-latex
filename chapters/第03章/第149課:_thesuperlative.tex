    
\chapter{The Superlative}

\begin{center}
\begin{Large}
第149課: The Superlative 
\end{Large}
\end{center}
 
\par{ The superlative is relatively easy, but there is still a lot to pay attention to. }
      
\section{-est}
 
\par{ The superlative (最上級) is made with "-est" in English. In Japanese you use 一番 or 最も (in formal speech) before the adjective, not after. Superlative sentences often have some sort of restriction such as the following. }

\begin{ltabulary}{|P|P|P|}
\hline 

In the world & 世界(中)で & せかい(じゅう)で \\ \cline{1-3}

Of the three & 三の[中・内]で & さんの[なか・うち]で \\ \cline{1-3}

Of all & [全て・皆]の[中・内]で & [すべて・みんな]の[なか・うち]で \\ \cline{1-3}

\end{ltabulary}

\par{\textbf{Word Note }: 皆 is typically used in reference to people, but it can refer to everything as well. }

\par{1. その店は一番便利だと思う。 \hfill\break
I think that store is the most convenient. }

\par{2. 会社では渡辺さんが一番意地悪だね。 \hfill\break
Isn't Mr. Watanabe the meanest in the company? }

\par{3. あなたが今までの人生の中で一番悩んだ決断は何ですか? \hfill\break
What has been the most difficult decision you've made in your life? }

\par{4. こちらの教会は町で最も古いです。 \hfill\break
This church is the oldest in the town. }

\par{5. 富士山は日本中で一番高い山です。 \hfill\break
Mt. Fuji is the tallest mountain in all of Japan. }

\par{6. 最も深刻な都市問題は何ですか。 \hfill\break
What is the most serious urban problem? }

\par{7. エベレストは世界で一番高い山です。 \hfill\break
Mt. Everest is the tallest mountain in the world. }

\par{\textbf{Suffix Note }: ~山 isn't used with certain mountains outside of Japan. It's something you have to learn on a mountain by mountain basis. }

\begin{center}
 \textbf{最~ }
\end{center}

\par{There are other superlative words with the prefix 最~. These words are, thus, Sino-Japanese for the most part. However, there are some exceptions to this. For instance, 最安値 (all-time low), 最高値 (all-time high), 最大手 (largest company\slash industry leader), and 最果て(the farthest\slash furthest ends) are possible. }

\begin{ltabulary}{|P|P|P|P|P|P|}
\hline 

Strongest & 最強 & さいきょう & Worst & 最低 & さいてい \\ \cline{1-6}

Biggest & 最大 & さいだい & Worst & 最悪 & さいあく \\ \cline{1-6}

Beloved & 最愛 & さいあい & Last place & 最下位 & さいかい \\ \cline{1-6}

Best; tallest & 最高 & さいこう & Oldest & 最古 & さいこ \\ \cline{1-6}

Least & 最少 & さいしょう & First-class & 最上 & さいじょう \\ \cline{1-6}

Newest; latest & 最新 & さいしん & Best & 最善 & さいぜん \\ \cline{1-6}

Very middle & 最中 & さなか・さいちゅう & Most & 最多 & さいた \\ \cline{1-6}

\end{ltabulary}

\par{\textbf{Word Note }: 最低 is for things and 最悪 is for people. 最善 refers to greater good whereas 最高 is more broad and similar to usages like "that's the best". 最良 also means "best" as in "ideal". There is also the word 最適 which refers to something being "the most fitting". }

\par{8.最悪の場合を覚悟する。 \hfill\break
To prepare for the worst. }

\par{9. 地球の最果て \hfill\break
The utmost point of the earth }

\par{10. 現存する最古の寺 \hfill\break
The oldest temple in existence }

\par{11. 最小の努力で最大の効果を上げるのが我々の目的です。 \hfill\break
Achieving the largest results with the smallest efforts is our goal. }

\par{12. 最上階 \hfill\break
The top floor }

\par{13. 最新のニュースを聞いたか。 \hfill\break
Did you hear the latest news? }

\par{14. 彼は一試合最多奪三振の記録を作った。 \hfill\break
He set a record for strikeouts in one game. }

\par{15. 横浜に行く最短の道順を教えてあげました。 \hfill\break
I told them the shortest way to Yokohama. }

\par{16. 演説の最中に\{立ち去った・席を立った\}。 \hfill\break
I left in the middle of the speech. }

\par{17. 本州の最南端は潮岬だ。 \hfill\break
The southernmost part of Honshu is Shionomisaki. }

\par{18. ${\overset{\textnormal{おき}}{\text{沖}}}$ ノ ${\overset{\textnormal{とりしま}}{\text{鳥島}}}$ は日本の ${\overset{\textnormal{さいなんたん}}{\text{最南端}}}$ にある。 \hfill\break
Okinotorishima is at the southernmost point of Japan. }

\par{19. 正直は最良の策。 \hfill\break
Honesty is the best policy. }
あなたが今までの人生の中で一番悩んだ決断は何ですか? \hfill\break
What has been the most difficult decision you've made in your life? \hfill\break
      
\section{The Superlative Question}
 
\par{ When making a superlative question, the first thing you must decide on is which interrogative you will use. The chart below illustrates the possible options. }

\begin{ltabulary}{|P|P|P|P|}
\hline 

何 & Comparison of a group & どれ & Used in a list format \\ \cline{1-4}

誰 & About people & どこ & About a place \\ \cline{1-4}

\end{ltabulary}

\par{When referring to a group such as in sports, で always follows and の中 is often used in between for more emphasis. If ~の中 is used and you are listing things with と, the last と that would normally be in the list--on the final noun--is deleted. }

\par{20. スポーツでは何が一番好きですか。 \hfill\break
Among sports, which do you like the best? }

\par{21. 犬と、猫と、ウサギの中では、どれが一番好きですか。 \hfill\break
Among dogs, cats, and rabbits, which do you like the best? }

\par{22. 日本の歌手では誰が一番好きですか。 \hfill\break
Among Japanese singers, who do you like the most? }

\par{23. どこが世界で一番暑いですか。 \hfill\break
Which place is the hottest in the world? }

\par{\textbf{Grammar Note }: In answering these questions, you simply use the normal superlative pattern. }
    