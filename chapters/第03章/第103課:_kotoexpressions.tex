    
\chapter{Expressions with こと}

\begin{center}
\begin{Large}
第103課: Expressions with こと 
\end{Large}
\end{center}
 
\par{ There are many expressions with こと. Even though you may end up hating こと, you're going to have to use it a lot. }
      
\section{こと Expressions}
 
\par{ The nominalizer こと is used in a lot of set expressions that involve nominalization. In these expressions, you should not replace こと with の. }

\par{1 . ~ことがある: With the past tense it means "have done" as you have done something before. With the non-past tense it means that you "often" do something. Note that you can't use ~ことがある for the “have” in expressions like "I have done my homework". This would be the "past perfect," in which case you would just use the auxiliary ~た. }
 
\par{1. 富士山に登ったことがありますか。 \hfill\break
Have you climbed Mt. Fuji before? }
 
\par{2. 私は酒を飲んだことがあります。 \hfill\break
I've had sake before. }
 
\par{3. 北海道では四月でも雪が降ることがある。 \hfill\break
In Hokkaido, it snows often even in April. }
 
\par{4. 彼はジョギングすることがある。 \hfill\break
He often jogs. }
 
\par{5. 彼女を見かけることがある。 \hfill\break
I often catch a glimpse of her. }
 
\par{6. 東京で道に迷うことがあります。 \hfill\break
I often get lost in Tokyo. }
 
\par{7. 私は買い物に行くことがある。 \hfill\break
I often go shopping. }
 
\par{8. そのバスは遅れることがある。 \hfill\break
The bus is often late. }
 
\par{9. 香港に行くことがある。 \hfill\break
I often go to Hong Kong. }
 
\par{10. 私はフィンランドに行ったことがある。 \hfill\break
I've been to Finland. }
 
\par{11. 何度も寿司を食べたことがあります。 \hfill\break
I have eaten sushi many times. }
 
\par{12. あなたは歌舞伎を観たことがありますか。 \hfill\break
Have you seen Kabuki before? }
 
\par{13. この映画は前に見たことがある。 \hfill\break
I have seen this movie before. }
 
\par{2. ~ことがない: To have not done before. }
 
\par{14. 東京に行ったことがありません。 \hfill\break
I haven\textquotesingle t gone to Tokyo. }
15. あんなにきれいな ${\overset{\textnormal{にちぼつ}}{\text{日没}}}$ は見たことがありません。 \hfill\break
I've never seen such a beautiful sunset. 
\par{16. 飛行機で行ったことがない。 \hfill\break
I haven\textquotesingle t gone by airplane. }

\par{17. 秋葉原に行ったことがありませんが、一度行ってみたいです。 \hfill\break
I haven't been to Akihabara, but I would like to go once. }

\par{\textbf{Culture Note }: 秋葉原, often shortened to 秋葉, is a popular place in Tokyo known for cheap electronics as well as cafes that cater to オタク・ヲタク. In メイド喫茶 maids dress up in uniform. }
 
\par{3. ~ことはない: It shows that something is not important\slash necessary or is an emphatic version of \#2. }
 
\par{18. 気にすることはない。 \hfill\break
There is no need to worry. }
 
\par{19. 多分これからも外国に行くことはないでしょう。 \hfill\break
I probably won't go abroad from now on. }

\par{20. 彼の ${\overset{\textnormal{ほう}}{\text{方}}}$ では何ら ${\overset{\textnormal{こま}}{\text{困}}}$ ったことはない。 \hfill\break
There's no problem at his end. }

\par{4. ~ことにする・こととする: To decide. ~ことにしている shows that one makes a routine of doing something or that one is determined\slash has a strong commitment to doing something. }
 
\par{21. あいつを忘れることにしただけさ。(Very colloquial) \hfill\break
I've just decided to just forget that guy. }
 
\par{22. 私はお酒を飲むのをやめることとしました。 (Slightly formal) \hfill\break
I have decided to quit drinking liquor. }

\par{23. 毎日カルピスを飲むことにしています。 \hfill\break
I make it a rule to drink Calpis every day. }

\par{24. 英語では話さないことにしています。 \hfill\break
I am determined to not speak in \emph{English }. }
 
\par{5. ~ ことになる・こととなる: To be decided }
 
\par{25. 今年日本に行くことになりました。 \hfill\break
It has been decided that I will go to Japan this year. }
 
\par{26. 新製品は5月10日に発表することになりました。 \hfill\break
It has been decided that we will announce the new product on May 10th. }
 
\par{\textbf{Grammar Notes: }}
 
\par{1. こと is not needed with ~にする or ~になる when these phrases are used with a concrete noun. }
 
\par{2. ~ことになっている shows that the decision has been made for you. It is equivalent to "am to". It is also important in describes rules, traditions, etc. }
 
\par{3. と gives a more formal and punctual feeling. }
 
\par{6. ~ということだ:  It is said that }
 
\par{27 . ロケットが打ち上げられたということだ。 \hfill\break
It is said that a rocket was launched. }

\par{7. ~ことだろう: Probably }
 
\par{28. . さぞかし無念\{な・だった\}ことだろう。  (やや書き言葉的) \hfill\break
It was certainly regretful. }
 
\par{8. ~ことだし: Is, so }
 
\par{29. 子供の(した)ことだし、彼女を許してくれないか。 \hfill\break
It was a childish thing to do, so won't you forgive her? }
 
\par{9. ~ことこのうえない: There's nothing better than }
 
\par{30. ウケいいことこの上ない。 \hfill\break
Nothing is more popular than this. }
 
\par{10. ~ことに(は): In being }
 
\par{With the negative it shows that when you don\textquotesingle t do X, Y is an undesirable result. }
 
\par{31. ありがたいことに、全員が無事だった。 \hfill\break
In being grateful, everyone was alright. }

\par{32. 早く出発しないことには、大変なことになる。 \hfill\break
In not departing early, it'll become an ordeal. }
 
\par{11. ~ことと思う:  Hope that }
 
\par{33. ご承知のことと思いますが。 \hfill\break
I hope you are aware, but\dothyp{}\dothyp{}\dothyp{} }
 
\par{12. ~のこと: Strengthens an adverb and is typically only used in certain expressions. }
 
\par{34. もちろん(のこと)、あいつは失敗した。 \hfill\break
Of course that guy failed. }
 
\par{13. AことはAが:  It's (the case that)\dothyp{}\dothyp{}\dothyp{}but }
 
\par{35. おいしいことはおいしいが、ちょっと高すぎる。 \hfill\break
It\textquotesingle s delicious, but it\textquotesingle s a little too expensive. }
 
\par{14. Adj. + こと: Makes an adverb. It isn't really constructive and is only limited to certain phrases. }
 
\par{36. 長いこと待つ。 \hfill\break
To wait for a long time. }
 
\par{15. ~ってことよ: It creates an admonishing tone. Remember that って = という }
 
\par{37. いいってことよ。 \hfill\break
It's alright! }
 
\par{16. ~ことなく: Without }
 
\par{ This pattern should not be used with habitual actions. }
 
\par{38. ためらうことなく、消防士は燃えている家に入った。 \hfill\break
The firefighters went into the burning house without hesitation. }
 
\par{17. ~ことなしに:  Not possible without }
 
\par{39. 福田社長はしょうことなしに謝罪をした。 \hfill\break
Company President Fukuda apologied with little choice. }
 
\par{40. 試験に合格することなしに、この会社には入れません。 \hfill\break
It is not possible to get into this company without passing the exam. }
 
\par{18. ~に事寄せて: On the plea\slash context of }
 
\par{41a. 彼女は出張に事寄せて東京市内の観光をしました。 \hfill\break
41b. 彼女は出張を口実に東京市内の観光をしました。 \hfill\break
41c. 彼女は出張に託けて東京市内の観光をしました。 \hfill\break
She went sightseeing in Tokyo on the pretext of a business trip. }
 
\par{19.  ~こともなげに:  Easily; as if nothing happened }
 
\par{42. 彼は仕事を事もなげにやってのけた。 \hfill\break
He pulled the job off as if nothing happened. }
 
\par{20. \{どれだけ・どんなに\}…ことか: Oh how….! }
 
\par{43. どれだけ雨が降ったことか。 \hfill\break
Oh how it rained! }
 
\par{44. どんなに料理がおいしかったことか。 \hfill\break
Oh how the dish was delicious! }
 
\par{45. どれだけ私が急いだことか。 \hfill\break
Oh how I was in a rush! }
 
\par{46. どれだけあなたが食べたことか。 \hfill\break
Oh how you ate! }
 
\par{\textbf{Contraction Note }: ことは as well as とは can be contracted to こ(っ)たぁ・こっちゃ and たあ respectively in slang. }

\begin{center}
\textbf{~だけ(のこと) }\textbf{}
\end{center}
 ~だけ(のことは)ある shows that something is as expected, and moreover, adequate and fulfills something. A worth, skill, capability, or action is demonstrated to be suitable. Potential ways to translate this include "it is worth", "it's not surprising that (as expected)". It is often used with adverbs like \{さすが・流石\}(に), which means "as expected".           
\par{47. ${\overset{\textnormal{こうひ}}{\text{公費}}}$ を ${\overset{\textnormal{さくげん}}{\text{削減}}}$ するだけのことはある。 \hfill\break
Cutting public spending is worth it. }
 
\par{48. 高くても、買っただけのことはある。 \hfill\break
Although it was expensive, it has its worth for being bought. }
 
\par{49. 高かっただけのことはあるよね。 \hfill\break
It was expected, but it was worth it. }
 
\par{50. さすが(に)大学に行っただけのことはある。 \hfill\break
As expected, he has worth from going to college. }
 
\par{~だけ(のことは)あって describes a circumstance that fulfills expectations. It is interchangeable with ~だけに, but there is a slight nuance difference. With ~ だけに, the situation expected might not happen, making it an unexpected surprise . It may also be found in the pattern "AがAだけに". With A in mind, there is plenty of reason for a certain outcome. }
 
\par{51. ここは ${\overset{\textnormal{ゆうめい}}{\text{有名}}}$ なだけに、 ${\overset{\textnormal{たくさん}}{\text{沢山}}}$ の人が ${\overset{\textnormal{おとず}}{\text{訪}}}$ れます。 \hfill\break
Since this place is famous, a lot of people visit. }
 
\par{52. 母も書道もあまりに身近な存在だっただけに、客観視することができなかったのかもしれない。 \hfill\break
Since mom also had too close of an existence with calligraphy too, she might not have been able to see (her work) from an objective point of view. \hfill\break
By 武田双雲 in the 文藝春秋 2008 (2). }
 
\par{53. よく勉強しただけに、いい成績で合格しました。 \hfill\break
Since I studied hard, I passed with good grades as expected. }
 
\par{\textbf{Sentence Note }: This student probably normally doesn't study, but this time he\slash she did and as expected from good effort, he\slash she got a just reward. }

\par{54. 彼は ${\overset{\textnormal{せいじか}}{\text{政治家}}}$ だけに口が ${\overset{\textnormal{うま}}{\text{巧}}}$ いでしょう。 \hfill\break
He's probably a good talker since he's a politician. }
 
\par{\textbf{Word Note }: ${\overset{\textnormal{せいじや}}{\text{政治屋}}}$ would be a much more fitting word for this sentence, but it is a sensitive word to some due to implying the politician is in it for the money. }
 
\par{55. 津波で町全体が荒らされただけに、一戸の建物さえ残っているのにびっくりしました。 \hfill\break
Since the entire town was devastated by the tsunami, I was surprised that just one building remained. }
 
\par{56. 民主党が勝手に頑固たる政策方針を結束して言い張り続けただけに、選挙戦に敗戦して自民党・公明党が当選を果たしたのであろう。(Literary) \hfill\break
Since the DJP continued to insist in solidarity on their stubborn policy stance as they pleased, they lost in the election battle and the LDP and Komeito won in the language (as expected). }
 
\par{\textbf{History Note }: On December 16, 2012 the Democratic Party of Japan lost to the Liberal Democratic Party, which won the majority of seats, and Komeito in a sweeping election, which would make 安部晋三(あべしんぞう) Prime Minister for the second time. }
 
\par{57. 韓国は儒教文化が残るとされる国なだけに、パク・クネ氏が初めての女性として大統領選挙で選び出されたことは韓国社会の変化を象徴しているだけでなく、アジア諸国の社会変動も象徴しているのかもしれません。 \hfill\break
Since South Korea is deemed to be a country where Confucian culture still remains, the election of Park Geun-hye as the first female president not only symbolizes change in Korean society, but it may also symbolize societal change in all of Asia. }
 
\par{\textbf{History Note }: Park Geun-hye of the Saenuri Party of South Korea was elected with over 50\% of the vote, a first for a presidential candidate in that country, as the first woman president on December 19, 2012. }
 
\par{58. 敵の防壁を潔く突破しただけに、宮殿に入ってからすぐに殺されたという戦死の告知は母国に大嘘だと見なされた。 \hfill\break
Since (he) had gracefully burst through the enemy bulwark, the notice of his death in war after being killed once he had entered the palace was taken as a big lie in his homeland. }
 
\par{59. インドネシアは赤道に近いだけあって、うだるように蒸し暑いですね。僕の親友がジャワ島に住んでいますから、訪ねるために身軽な服装を持っていかないといけないでしょう。 \hfill\break
Since Indonesia is close to the Equator, it's seething hot, isn't it? As my close friend lives on Java Island, I'll probably have to take light clothing to visit. }
 
\par{60. ミス東大だけに美人だ。 \hfill\break
Since she's Miss Todai, she's a beauty. }
 
\par{\textbf{Word Note }: 東大 = 東京大学. }
 
\par{61. 明日はテストだけあってみんな眠そうだ。 \hfill\break
Since there is a test tomorrow, everyone is sleepy. }

\par{62. 台風の被害が大きかっただけに、都市の ${\overset{\textnormal{ふっこうぶ}}{\text{復興振}}}$ りには目を見張るものがあった。 \hfill\break
Since the damage from the typhoon was large, there was quite something to behold in the state of reconstruction. }
 
\par{\textbf{Usage Note }: The usage of だけに to show something unexpected is becoming rarer nowadays. }
 
\par{63. 被害が大きかっただけに、復旧も困難を極めた。 \hfill\break
Since the damage was large, restoration was extremely difficult. }

\begin{center}
\textbf{${\overset{\textnormal{けっこん}}{\text{結婚}}}$ \textbf{することになりました VS }\textbf{結婚することにしました }}\textbf{\textbf{}}
\end{center}
 This is one time where definitions of a phrase don't help much. When students hear the first one, they think of an arranged marriage. That may be so, but it is just a euphemism that describes the current state. It's uncertain whether the person had say in it or not. It is more appropriate to use 結婚することになりました in situations where you are informing someone or talking to someone you haven't seen in a while. 結婚することにしました clearly shows you've chosen to and you are likely telling friends in a conversation.  
\par{ This is a rule of thumb, but Japanese themselves don't always follow through all the time. Some may purposely use the wrong one to make people laugh or what not.  }
    