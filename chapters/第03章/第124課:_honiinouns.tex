    
\chapter{Honorifics I}

\begin{center}
\begin{Large}
第124課: Honorifics I: Nouns 
\end{Large}
\end{center}
 
\par{ Honorific speech 敬語 is very intricate and its usage is mandatory in many situations. There are three broad categories of honorific speech. Of these, the top two categories will be relatively new to you. }

\par{ As has been the case with polite speech seen listed as the third category, using formal speech toward those whom you ought to be casual with holds the opposite effect than otherwise intended. Therefore, as you learn how to form honorific expressions, it will be just as important to learn when to use said expressions. }

\begin{ltabulary}{|P|P|P|P|}
\hline 

尊敬語 & そんけいご & Respectful &  \emph{S }\emph{peaking to\slash of a person whom one wishes to respect }. It is not used in reference to oneself. It is associated with second and third person. \\ \cline{1-4}

謙譲語・謙譲語 & けんじょうご・けんそんご & Humble &  speaking to someone on what you or your in-group are to do . It is associated with first person. \\ \cline{1-4}

丁寧語 & ていねいご & Polite &  \\ \cline{1-4}

\end{ltabulary}

\par{\textbf{\hfill\break
Culture Note }: You should also be aware that using 敬語 to someone that one is usually cordial and casual with will cause division and a sense of separation. }
      
\section{Honorific Nouns}
 
\par{ Nouns are made honorific by the prefix 御~ which has three possible readings. }

\par{お~ }

\par{御~ is read as お~ when attached to native words as well as a number of 漢語 (Sino-Japanese words). }

\par{ Motivations for using お~ with 漢語 is being a 和製語 (word created in Japan). It's nearly impossible to know whether an 音読み phrase was made in Japan without being told. However, a great hint is that the majority of these words relate to things of the modern era. }

\par{ Yet another means of knowing whether お~ is used with a 漢語 is if its sense as a 漢語 has been lost in the general public and has essentially become Japanese (due to the lack or loss of a native equivalent). }

\begin{ltabulary}{|P|P|P|P|}
\hline 

Noun & Honorific & 和語・漢語 & Meaning \\ \cline{1-4}

名前 & お名前 & 和語 & Name \\ \cline{1-4}

世話 & お世話 & 和語 & Assistance \\ \cline{1-4}

世辞 & お世辞 & 和語 & Flattery \\ \cline{1-4}

釣り & お釣り & 和語 & Change \\ \cline{1-4}

手紙 & お手紙 & 和語 & Letter \\ \cline{1-4}

塩 & お塩 & 和語 & Salt \\ \cline{1-4}

魚 & お魚 & 和語 & Fish \\ \cline{1-4}

目 & お目 & 和語 & Eyes \\ \cline{1-4}

腹 & お腹 & 和語 & Stomach \\ \cline{1-4}

上 & お上 & 和語 & Buttress; government \\ \cline{1-4}

自宅 & お住い & 和語 & Residence \\ \cline{1-4}

\end{ltabulary}

\par{ Some words are somewhat difficult to read. For instance, お腹 is read as おなか and お上 is read as おかみ. As you can see, all words are listed. So, what about 世話? This is 当て字,  meaning せわ is native and the spelling blurs this fact.  }

\par{1. お‐ is always incompatible with 'long words' several morae long, and it is also not used with words that start with o }

\par{2. A rarer but more honorific form of お~ is おん~. Examples include 御自ら and 御母上. }

\begin{center}
\textbf{Example Sentences } 
\end{center}

\par{1. 彼女にはいろいろなことをしていただいて \textbf{お }${\overset{\textnormal{せわ}}{\text{世話}}}$ になっております。 \hfill\break
I owe her a lot for everything. }

\par{2. お ${\overset{\textnormal{しか}}{\text{叱}}}$ りのお言葉を有り難く頂戴いたします。 \hfill\break
I would appreciate it if I receive your scolding. }

\par{3. お ${\overset{\textnormal{さき}}{\text{先}}}$ にどうぞ。 \hfill\break
Please go in front. }

\par{4. 亡くなられた \textbf{お方 }の小さい ${\overset{\textnormal{おこたち}}{\text{御子達}}}$ の相手に女の ${\overset{\textnormal{めい}}{\text{姪}}}$ たちを連れて来て ${\overset{\textnormal{もら}}{\text{貰}}}$ いたいと ${\overset{\textnormal{い}}{\text{云}}}$ うのだった。 \hfill\break
She was told that [they] would like her to bring along her nieces as companions to the small children of the deceased lady. \hfill\break
From ${\overset{\textnormal{おばすて}}{\text{姨捨}}}$ by ${\overset{\textnormal{ほりおたつ}}{\text{堀尾辰}}}$ . }

\par{\textbf{Word Note }: Though お方 now refers politely to a gentleman\slash gentlewoman, in the past it was used to refer to a noble woman or daughter, as is the case in this setting. }

\par{5. 「罪が深いんですから、いくら難有いお経だって浮ばれる事は御座いませんよ」 \hfill\break
Since his sin is so grave, you cannot rest his soul no matter how great of a sutra you evoke. \hfill\break
From 吾輩は猫である by 夏目漱石. }

\par{\textbf{Word Note }: お経 is an interesting exception. 経 is a Sino-Japanese word of Chinese origin and does not take ご, which we are about to get to. This must mean that the word is treated as if it is a native word, just like other old loans such as 馬. }

\par{\textbf{Spelling Note }: Because the novel from which this example comes from is so old, spelling conventions are not quite the same. ありがたい, when it is written in 漢字 today, is typically spelled as 有り難い. As you can see, the 漢字 are flipped. }

\begin{center}
\textbf{Non-Polite Examples }
\end{center}

\par{ Although a phrase may have an honorific suffix in it, it doesn't mean that it is respectful. One example is お里が知れる, which means "to reveal one's upbringing". This refers to Japanese dialects and is felt to be a rude expression. A much more polite way is なまり・方言・アクセントで出身が分かる. }

\par{ As another example of words with お that are not honorific, consider the word お化け. This is an alternative way of saying 化け物. Both refer to Japanese-style ghosts\slash monsters. お化け can also refer to something just really abnormal. There are a lot of important お化け that you should know about. Important ones include ${\overset{\textnormal{ろくろくび}}{\text{轆轤首}}}$ , ${\overset{\textnormal{てんぐ}}{\text{天狗}}}$ , and から ${\overset{\textnormal{かさ}}{\text{傘}}}$ お ${\overset{\textnormal{ば}}{\text{化}}}$ け. }

\begin{center}
 \textbf{お + 外来語 }
\end{center}

\par{ What if you wanted to make a loanword like ワイン or コンピューター? You add nothing. However, there are several exceptions. In relation to drinks, おビール is questionable, but some people, particularly women, do use it. It is also appropriate in the business arena. There is no problem with using お酒, お茶, or お冷 because they are 'Japanese' words. Likewise,  although you can say お洋服, you cannot say おスーツ or おコート. }

\par{ おタバコ is not that common unless you're in a service industry. Similar words that are odd unless you're in a secretarial position include お車 and お薬 despite being native words. おズボン would also be seen in the trades or by women who frequently add お to more words than men typically do such as おジュース. おソース is on the same lines as おしょう油. }

\par{ A decent amount of people also say things like おトイレ. It is strange to a lot of people, though. Interestingly, お手洗い is proper Japanese. }

\par{ \textbf{お + 漢語 }}

\par{ As mentioned above, there quite a few words that are from Chinese or based on Chinese morphemes that take お. Consider the following examples. }

\begin{ltabulary}{|P|P|P|P|P|P|P|P|P|P|}
\hline 

お電話 & Phone & お時間 & Time & お砂糖 & Sugar & お宅 & Home & お礼状 & Thank you note \\ \cline{1-10}

お肉 & Meat & お邪魔 & Disturbance & お写真 & Photo & お天気 & Weather & お菓子 & Sweets \\ \cline{1-10}

お愛想* & Affability & お風呂 & Bath & お会計* & Bill & お勘定* & Bill & お返事** & Response \\ \cline{1-10}

\end{ltabulary}

\par{ Ironically, お宅 does also mean "nerd". In honorifics, though, it seriously refers to someone else's home. }

\par{*: All of these words happen to be used to mean "bill", but there is disagreement on how they should be used. おあいそ is largely felt to be rude in Tokyo, but it is frequently used in West Japan. The reason why some say it's rude is because it should be what a business should say to a customer. Meaning, it is they who may have not shown affability to you as they should have. Thus, a customer referring to this is rude. }

\par{ お会計 is the word of choice for younger generation. It is the word most likely to be on your receipt at any given place. However, there are those who think only the business side should use it because it is referring to actual accounting. お勘定 is rather neutral in being used by the customer, but the percentage of young people who use it is dropping. }

\par{**: This word can also be ご返事, which is deemed to be the original form and always proper. However, over 60\% of people no longer use it. So, it is fair to say that there is no practical difference between the two. Be aware of sticklers. }

\par{ご~  }

\par{ご~ is used only with 音読み compounds that are often considered formal and typically have a native Japanese word equivalent. }

\begin{ltabulary}{|P|P|P|}
\hline 

Noun & Honorific & Meaning \\ \cline{1-3}

主人 & ご主人 & Master* \\ \cline{1-3}

病気 & ご病気 & Sickness \\ \cline{1-3}

心配 & ご心配 & Worry \\ \cline{1-3}

旅行 & ご旅行 & Trip \\ \cline{1-3}

連絡 & ご連絡 & Contact \\ \cline{1-3}

両親 & ご両親 & Parents \\ \cline{1-3}

遠慮 & ご遠慮 & Discretion \\ \cline{1-3}

注意 & ご注意 & Caution \\ \cline{1-3}

親類 & ご親類 & Relatives \\ \cline{1-3}

相談 & ご相談 & Consulting \\ \cline{1-3}

近所 & ご近所 & Vicinity \\ \cline{1-3}

親切 & ご親切 & Kindness \\ \cline{1-3}

苗字 & ご苗字 & Last name \\ \cline{1-3}

\end{ltabulary}

\par{\textbf{Usage Notes }: }

\par{1. The adverb ゆっくり may also be used with ご~. }

\par{2. 主人 means "master" but "(someone's) husband when ご~ is attached. }

\par{3. There are times when adding ご results in Double Keigo (二重敬語). For instance, in ご芳名 (your good name), ご令息 (son), ご逝去 (death), the nouns themselves are already honorific. But, because they have for whatever reason been reanalyzed as being not honorific enough, ご is always attached to them. }

\begin{center}
 \textbf{Examples }
\end{center}

\par{6. ${\overset{\textnormal{たいへん}}{\text{大変}}}$ ご ${\overset{\textnormal{めんどう}}{\text{面倒}}}$ をおかけしてすみませんでした。 \hfill\break
I'm very sorry to have put you into any trouble. }

\par{${\overset{\textnormal{}}{\text{7. 先生}}}$ に ${\overset{\textnormal{さどう}}{\text{茶道}}}$ のご ${\overset{\textnormal{きょうじゅ}}{\text{教授}}}$ を ${\overset{\textnormal{たまわ}}{\text{賜}}}$ りたいのですが。 \hfill\break
Would you be so kind as to give me instruction on tea ceremonies? }

\par{8. ご ${\overset{\textnormal{ていせい}}{\text{訂正}}}$ を頂戴いたしまして、どうもありがとうございます。 \hfill\break
Thank you very much for giving me corrections. }

\par{9. どうぞご ${\overset{\textnormal{えんりょ}}{\text{遠慮}}}$ なく。 \hfill\break
Feel free. }

\par{み~ }

\par{ み~ is used with nouns particular to religion or grand importance. Although rarely seen, it is even possible to use おみ~ and おんみ~. }

\begin{ltabulary}{|P|P|P|}
\hline 

Comes From &  & New Meaning \\ \cline{1-3}

Treasure & 大御宝 & Imperial subjects \\ \cline{1-3}

Child & 御子 & God's son \\ \cline{1-3}

Name & 御名 & Holy Name \\ \cline{1-3}

Palanquin & 御輿・神輿 (みこし) & Portable Shrine \\ \cline{1-3}

Liquor & 御神酒 (おみき) & Sacred Wine \\ \cline{1-3}

World & 御世 & Imperial Reign \\ \cline{1-3}

Heart & 御心 & Lord's will \\ \cline{1-3}

Rock of spirit & 御影石 & Granite \\ \cline{1-3}

\end{ltabulary}
\hfill\break
\textbf{Word Note }: The last example is a little complicated. 御影 happens to mean "spirit(s) of the dead". This word also happens to be a place name where granite is manufactured. Thus, the word 御影石 came from this geographical connection. This is an important example to note, though, because this word is not honorific in any way despite the fact it clearly has the honorific prefix み- in it. 
\par{HONORIFIC NOUNS WITHOUT \textbf{御 }}

\par{There are other means to show respectful and humble speech in nouns. For respectful nouns, 貴~ is apart of many respectful Sino-Japanese words. However, most, with exception to 貴社 "your honorable company" are rarely used. Death especially has several respectful variants. }

\par{ Several humble nouns begin with 粗(そ), 拙(せつ), and 弊 (へい). All three characters have negative\slash humble connotations. 愚(ぐ)~ is a humble prefix added to the 音読み of characters used for family. For representing respectfulness in terms of family, the suffix ~上(うえ) is used. Or, you may use 賢(けん)~ . }

\begin{ltabulary}{|P|P|P|P|}
\hline 

 & Noun & HON. & H or R? \hfill\break
\\ \cline{1-4}

Person & 人 & 者 & H \\ \cline{1-4}

Person & 人 & 方 & R \\ \cline{1-4}

Refreshments & おやつ \hfill\break
&  粗 菓 & H \\ \cline{1-4}

My thoughts & 私意 & 拙意 & H \\ \cline{1-4}

Word & 言葉 & 詔 & R \hfill\break
\\ \cline{1-4}

Older brother & 兄 & 愚兄 & H \\ \cline{1-4}

Older brother & 兄 & 賢兄 & R \\ \cline{1-4}

Company & 会社 & 貴社 & R \\ \cline{1-4}

Company & 会社 & 弊社 & H \\ \cline{1-4}

Tea & 茶 & 粗茶 & H \\ \cline{1-4}

Place & 所 & 御所 & R \\ \cline{1-4}

Husband & 夫 & 宿六 & H \\ \cline{1-4}

Liquor & 酒 & 粗酒 & H \\ \cline{1-4}

Daughter & 娘 & 愚女 & H \\ \cline{1-4}

Father & 父 & 父上 & R\slash H \\ \cline{1-4}

Goods & 品 & 粗品 & H \\ \cline{1-4}

Manuscript & 原稿 & 玉稿 & R \\ \cline{1-4}

Production; work \hfill\break
& 著作 & 拙作 & H \\ \cline{1-4}

\end{ltabulary}

\par{INTERPRETING THE CHART: R = Respectful \& H = Humble }

\par{\textbf{Usage Notes }: }

\par{1. 御所 refers to the "old imperial palace". }

\par{2. 詔 refers to an "imperial decree". }

\par{3. Whenever a word refers to royalty, that said variant is not used in reference to regular people. }

\par{4. 粗品 may also mean "little gift". This would still be 書き言葉的. }

\par{5. 粗 means "crude\slash course\slash inferior" and these meanings are implied in the humble words whether the thing(s) in question are as such. }

\par{6. Sometimes using the right word can be tricky. For instance, whenever people take entrance exams for college (入学試験) and companies (入試試験), people these days are starting to want to use 御校 and 御社 respectively instead of 貴校 or 貴社. The problem with this is with 御 being read as おん, it may sound quite unnatural in the spoken language if it were not for abnormal situation in which one is in front of many people you don't know. Although 貴校 and 貴社 may get used in letters between groups in an almost equal mentality, the most important thing in 敬語 is still to be conscious overall of ways to speak politely. Thus, it is quite OK to use those words. }

\par{\textbf{HONORIFIC VARIANTS FOR DEATH }}

\begin{ltabulary}{|P|P|}
\hline 

崩御 (ほうぎょ) & The honorable death of royalty. \\ \cline{1-2}

卒去 (そっきょ) & Death of king or queen or a high ranking court official. \\ \cline{1-2}

逝去 (せいきょ) \hfill\break
& A more honorific word for the passing of an individual. \\ \cline{1-2}

薨去 (こうきょ) \hfill\break
& Death of an imperial member or high court. \\ \cline{1-2}

薨御 (こうぎょ) & The death of a crowned prince or minister. \\ \cline{1-2}

\end{ltabulary}

\par{\textbf{Word Note }: It is unlikely that you will see some of these words, but there is always that chance. "To die" is generally a euphemism for several other verbs such as 死亡する (to decease), 死去する (to part), and 亡くなる (to pass away)". }

\par{\textbf{HONORIFIC PHRASES WITH HONORIFIC PREFIXES \& SUFFIXES }}

\par{There are also many set phrases that include the prefixes お~ and ご~ and the honorific endings ~さん and ~さま. }

\begin{ltabulary}{|P|P|}
\hline 

お月\{さま・さん\} & Moon \\ \cline{1-2}

お日\{さま・さん\} & Sun \\ \cline{1-2}

お世話さま & Trouble \\ \cline{1-2}

お邪魔さま & Trouble \\ \cline{1-2}

お疲れさま & Tired One \\ \cline{1-2}

お気の毒さま & I'm very sorry \\ \cline{1-2}

ご面倒さま & Could I trouble you \\ \cline{1-2}

ご苦労さま & Much obliged for hardship \\ \cline{1-2}

\end{ltabulary}

\par{ \textbf{FORMAL SINO-JAPANESE WORDS }}

\par{Some nouns are naturally formal because they are Sino-Japanese. These examples are by no means it, and you have seen many already. You will also see more in coming lessons through example sentences. }

\begin{ltabulary}{|P|P|P|P|}
\hline 

Tomorrow & 明日 (みょうにち) & This year \hfill\break
& 今年、本年 (こんねん、ほんねん) \\ \cline{1-4}

Breakfast & 朝食 (ちょうしょく) & Day after tomorrow & 明後日 (みょうごにち) \\ \cline{1-4}

Next year & 明年 (みょうねん) & Day before yesterday & 一昨日 (いっさくじつ) \\ \cline{1-4}

Last night \hfill\break
& 昨夜 (さくや) & Today & 本日 (ほんじつ) \\ \cline{1-4}

The other day & 先日 (せんじつ) & Around\dothyp{}\dothyp{}\dothyp{} & 約 (やく) \\ \cline{1-4}

\end{ltabulary}

\par{\textbf{Word Note }: Some of the words are typically heard as well. But, when it comes down to deciding what variant of a word you should use when speaking respectfully, you would choose these words. }

\par{ \textbf{Meals }}

\par{Breakfast, lunch, and dinner, in fact, have 4 sets of expressions based on politeness with each set using a different verb for "to eat". }

\begin{ltabulary}{|P|P|P|P|P|}
\hline 

 & Breakfast & Lunch & Dinner & To Eat \\ \cline{1-5}

Casual & 朝飯(あさめし) & 昼飯(ひるめし) & [晩・夕]飯([ばん・ゆう]めし) & 食(く)う \\ \cline{1-5}

Plain & 朝飯(あさはん) & 昼飯(ひるはん) & 夕飯(ゆうはん) & 食(た)べる \\ \cline{1-5}

Plain\slash Polite & 朝御飯(あさごはん) \hfill\break
& 昼御飯(ひるごはん) \hfill\break
& [晩・夕]御飯([ばん・ゆう]ごはん) & 食べる \\ \cline{1-5}

Formal & 朝食(ちょうしょく) & 昼食(ちゅうしょく) & 夕食(ゆうしょく) & 取る \\ \cline{1-5}

\end{ltabulary}

\par{\textbf{Word Note }: 夜ご飯 is becoming more acceptable as more people are now eating dinner later in the evening.  }
      
\section{Exercises}
 
\par{1. When do you use お-? }

\par{2. When do you use ご-? }

\par{3. How do you use adjectives in honorific speech? }

\par{4. What are the two types of honorific speech? }

\par{5. Make a sentence in honorific speech with the copula. }

\par{6. Make a sentence in honorific speech with an adjective. }

\par{7. Make a sentence in honorific speech with a noun. }

\par{8. Show how a noun may change meaning when used with an honorific prefix. }

\par{9. Explain the usage of honorifics in your own words. }
    