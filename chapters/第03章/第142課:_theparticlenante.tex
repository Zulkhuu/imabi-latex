    
\chapter{The Particle なんて}

\begin{center}
\begin{Large}
第142課: The Particle なんて 
\end{Large}
\end{center}
 
\par{ The particle なんて is similar to なんか and  など because all three particles can be used to euphemize what they follow. However, the practical situations in which they are used can be very different. When it comes to how なんて is used, the tone that it gives is far more emphatic  than the other two can ever be. It also brings about unique grammar that the other two don\textquotesingle t. }

\par{ In this lesson, we will focus solely on なんて to see how it\textquotesingle s used. Ultimately, it is easier to learn this particle separately so that you don\textquotesingle t confuse it with things that can occasionally be used similarly but not ultimately bring about the same nuance. }
      
\section{How to Use なんて}
 
\par{ Grammatically speaking, なんて can pretty much following anything. One reason for this is that it is partially composed of って, which you should understand as one of the most common means of citation in speech. The なん, as you can imagine, comes from 何. This provides major context to how the phrase generally works. It\textquotesingle s emphatic; it\textquotesingle s subjective; it\textquotesingle s there to draw some form of attention. }

\par{ なんて can pretty much follow anything. It can be seen after nouns, adjectives, adjectival nouns, verbs, as well as full statements. Interestingly enough, as you will soon see, the copula だ can actually intervene at times to add even more dramatic effect to what\textquotesingle s being referred. The "referring" aspect makes だ's role akin to that of a citation particle like って, drawing upon the literal breakdown of なんて itself. }

\begin{ltabulary}{|P|P|}
\hline 

Part of Speech & Example \\ \cline{1-2}

With Nouns & 恋(だ)なんて \\ \cline{1-2}

With Adjectives & おかしい(だ)なんて \\ \cline{1-2}

With Adjectival Nouns & 新鮮(だ)なんて \\ \cline{1-2}

With Verbs & 行く(だ)なんて \\ \cline{1-2}

With Sentences & 本当に好きだよなんて \\ \cline{1-2}

\end{ltabulary}

\par{ Although overall it is fair to say that なんて is used in mostly negative situations, it is not limited to this. Regardless of whether the statement is positive or negative, it will euphemize whatever it follows. This euphemizing can then have a belittling, critical, or supportive tone depending on the sentence. }

\par{1. ${\overset{\textnormal{かのじょ}}{\text{彼女}}}$ 、 ${\overset{\textnormal{ほうちょう}}{\text{包丁}}}$ なんて ${\overset{\textnormal{つか}}{\text{使}}}$ ったことあるのかな。 \hfill\break
I wonder if she\textquotesingle s ever used a kitchen knife. }
 
\par{2. そもそも ${\overset{\textnormal{けっとう}}{\text{決闘}}}$ なんてしませんよ。 \hfill\break
We\textquotesingle re not going to be having a duel in the first place. }
 
\par{3. もう、 ${\overset{\textnormal{けんたくん}}{\text{健太君}}}$ なんて ${\overset{\textnormal{し}}{\text{知}}}$ らないわ! \hfill\break
I don\textquotesingle t know you anymore, Kentaro! }
 
\par{4. ${\overset{\textnormal{わたし}}{\text{私}}}$ なんてまだまだです。 \hfill\break
I still have a long way to go. }
 
\par{5. アライグマなんて ${\overset{\textnormal{こわ}}{\text{怖}}}$ くないよ。 \hfill\break
There\textquotesingle s nothing scary about raccoons! }
 
\par{6. バスク ${\overset{\textnormal{ご}}{\text{語}}}$ なんて ${\overset{\textnormal{かんたん}}{\text{簡単}}}$ さ。 \hfill\break
Basque is just easy! }
 
\par{7. ${\overset{\textnormal{やきゅう}}{\text{野球}}}$ なんてちょっと ${\overset{\textnormal{つ}}{\text{詰}}}$ まらなくない? \hfill\break
Isn\textquotesingle t baseball a little boring? }
 
\par{8. ${\overset{\textnormal{ゆきがっせん}}{\text{雪合戦}}}$ なんて ${\overset{\textnormal{こども}}{\text{子供}}}$ じゃあるまいし。 \hfill\break
A snowball fight…it isn\textquotesingle t as if I\textquotesingle m a child. }
 
\par{9. ${\overset{\textnormal{じんせい}}{\text{人生}}}$ なんてそんな ${\overset{\textnormal{すば}}{\text{素晴}}}$ らしいものじゃない。 \hfill\break
Human life is not such a wonderful thing. }
 
\par{10. ${\overset{\textnormal{にんげん}}{\text{人間}}}$ なんて ${\overset{\textnormal{もろ}}{\text{脆}}}$ いものなんでしょうね。 \hfill\break
People really are weak things, aren't they? }
 
\par{11. セス ${\overset{\textnormal{せんせい}}{\text{先生}}}$ なんて ${\overset{\textnormal{き}}{\text{聞}}}$ いたことねー。 \hfill\break
I\textquotesingle ve never heard of this Seth-sensei. }
 
\par{12. ${\overset{\textnormal{かんじ}}{\text{漢字}}}$ なんてどうしても ${\overset{\textnormal{か}}{\text{書}}}$ けないの! \hfill\break
I can\textquotesingle t write Kanji no matter what I do! }
 
\par{13. なんで ${\overset{\textnormal{しけん}}{\text{試験}}}$ なんてやるんだろ。 \hfill\break
Why is it that we do exams anyway? }
 
\par{14. お ${\overset{\textnormal{かね}}{\text{金}}}$ なんて ${\overset{\textnormal{い}}{\text{要}}}$ らない! \hfill\break
I don\textquotesingle t need any money! }
 
\par{15. ${\overset{\textnormal{むいみ}}{\text{無意味}}}$ な ${\overset{\textnormal{ぐち}}{\text{愚痴}}}$ なんて ${\overset{\textnormal{き}}{\text{聞}}}$ きたくもない。 \hfill\break
I don\textquotesingle t want to hear your meaningless complaining. }
 
\par{16. ${\overset{\textnormal{ほんらい}}{\text{本来}}}$ ${\overset{\textnormal{しろ}}{\text{白}}}$ いはずのものが ${\overset{\textnormal{あか}}{\text{赤}}}$ いなんてちょっと ${\overset{\textnormal{しんぴ}}{\text{神秘}}}$ ( ${\overset{\textnormal{てき}}{\text{的}}}$ )だよね。 \hfill\break
Something that\textquotesingle s originally supposed to be white being red is a little mysterious, isn\textquotesingle t it? }
 
\par{17. ${\overset{\textnormal{だんせい}}{\text{男性}}}$ じゃあるまいし、 ${\overset{\textnormal{びか}}{\text{鼻下}}}$ が ${\overset{\textnormal{あお}}{\text{青}}}$ いなんて ${\overset{\textnormal{しゃれ}}{\text{洒落}}}$ になりません。 \hfill\break
It\textquotesingle s not as if I\textquotesingle m a man; my upper-lip being blue is no joking matter. }
 
\par{18. ${\overset{\textnormal{はじ}}{\text{始}}}$ まりが ${\overset{\textnormal{こわ}}{\text{怖}}}$ いなんて、そんなの ${\overset{\textnormal{とうぜん}}{\text{当然}}}$ だからね。 \hfill\break
That\textquotesingle s ‘cause it\textquotesingle s natural for the beginning to be scary. }
 
\par{19. あれほど ${\overset{\textnormal{みにく}}{\text{醜}}}$ いなんて、 ${\overset{\textnormal{ほんとう}}{\text{本当}}}$ に ${\overset{\textnormal{ざんねん}}{\text{残念}}}$ なことだ。 \hfill\break
It truly is a shame that (he) is so ugly like that. }
 
\par{20. ${\overset{\textnormal{ずうずう}}{\text{図々}}}$ しいだなんて、とんでもございません。 \hfill\break
It\textquotesingle s absolutely not audacious (of him). }
 
\par{21. ${\overset{\textnormal{たいけい}}{\text{体型}}}$ が ${\overset{\textnormal{うつく}}{\text{美}}}$ しいだなんて、 ${\overset{\textnormal{て}}{\text{照}}}$ れるわ。 \hfill\break
I get embarrassed when I\textquotesingle m told my figure is beautiful. }
 
\par{22. ${\overset{\textnormal{かなら}}{\text{必}}}$ ずずっと ${\overset{\textnormal{いっしょ}}{\text{一緒}}}$ (だ)なんて、あり ${\overset{\textnormal{え}}{\text{得}}}$ ないんじゃない? \hfill\break
Always being together forever is impossible, no? }
 
\par{23. 日常会話が簡単(だ)なんて嘘だよ! \hfill\break
It\textquotesingle s a complete lie that daily conversation is easy! }
 
\par{24. え、まさかあいつが ${\overset{\textnormal{おれ}}{\text{俺}}}$ のことを ${\overset{\textnormal{す}}{\text{好}}}$ きだなんて・・・ \hfill\break
What? You\textquotesingle re saying that guy likes me!? }
 
\par{25. ${\overset{\textnormal{ほとけさま}}{\text{仏様}}}$ を ${\overset{\textnormal{や}}{\text{焼}}}$ き ${\overset{\textnormal{はら}}{\text{払}}}$ うなんて、 ${\overset{\textnormal{にんげん}}{\text{人間}}}$ の ${\overset{\textnormal{でき}}{\text{出来}}}$ ることじゃないと ${\overset{\textnormal{おも}}{\text{思}}}$ うがね。 \hfill\break
Though I\textquotesingle m pretty sure reducing the Buddha into ashes is not something man can do. }
 
\par{26. ${\overset{\textnormal{わ}}{\text{我}}}$ が ${\overset{\textnormal{くに}}{\text{国}}}$ の ${\overset{\textnormal{かみがみ}}{\text{神々}}}$ を ${\overset{\textnormal{うたが}}{\text{疑}}}$ うなんて、とんだ ${\overset{\textnormal{みぶん}}{\text{身分}}}$ だな。 \hfill\break
What unthinkable position you\textquotesingle re in doubting the gods of our country. }
 
\par{27. ${\overset{\textnormal{むすこ}}{\text{息子}}}$ さん ${\overset{\textnormal{わせだ}}{\text{早稲田}}}$ に受かったなんて ${\overset{\textnormal{すご}}{\text{凄}}}$ いですね。 \hfill\break
That\textquotesingle s amazing how your son got accepted to Waseda (University). }
 
\par{28. ${\overset{\textnormal{しょうがつ}}{\text{正月}}}$ に ${\overset{\textnormal{いちど}}{\text{一度}}}$ も ${\overset{\textnormal{かえ}}{\text{帰}}}$ らないなんて ${\overset{\textnormal{なに}}{\text{何}}}$ か ${\overset{\textnormal{じじょう}}{\text{事情}}}$ があるからだろうしなあ。 \hfill\break
There's probably got to be some reason behind (him) not going home even one on New Year\textquotesingle s. }
 
\par{29. ${\overset{\textnormal{じゅうよう}}{\text{重要}}}$ な ${\overset{\textnormal{かいぎ}}{\text{会議}}}$ に ${\overset{\textnormal{ちこく}}{\text{遅刻}}}$ するなんて、 ${\overset{\textnormal{ゆる}}{\text{許}}}$ せない。 \hfill\break
It\textquotesingle s unforgiveable to be late to an important meeting. }
 
\par{30. ${\overset{\textnormal{い}}{\text{行}}}$ くなんて ${\overset{\textnormal{い}}{\text{言}}}$ ってないよ。 \hfill\break
I'm not saying that I'm going. }
 
\par{31. ${\overset{\textnormal{こども}}{\text{子供}}}$ の ${\overset{\textnormal{さる}}{\text{猿}}}$ も ${\overset{\textnormal{つぎ}}{\text{次}}}$ の ${\overset{\textnormal{とし}}{\text{年}}}$ に ${\overset{\textnormal{こ}}{\text{子}}}$ をたくさん ${\overset{\textnormal{う}}{\text{産}}}$ むなんてことはできない。 \hfill\break
The young monkeys can\textquotesingle t also give birth to a lot of offspring the next year. }
 
\par{32. この ${\overset{\textnormal{さる}}{\text{猿}}}$ ちゃん、 ${\overset{\textnormal{こ}}{\text{子}}}$ をたくさん ${\overset{\textnormal{う}}{\text{産}}}$ むことなんてできないもん。 \hfill\break
This little monkey, there\textquotesingle s no way it can give birth to a lot of offspring. }
 
\par{33. ${\overset{\textnormal{ぜんいん}}{\text{全員}}}$ に ${\overset{\textnormal{す}}{\text{好}}}$ かれるのなんて ${\overset{\textnormal{むり}}{\text{無理}}}$ だよね。 \hfill\break
It\textquotesingle s impossible to be liked by everyone, huh. }
 
\par{34. ${\overset{\textnormal{は}}{\text{禿}}}$ げるのなんて ${\overset{\textnormal{ぜったいいや}}{\text{絶対嫌}}}$ だ! \hfill\break
Going bald would just be absolutely awful! }
 
\par{35. タイムマシンが ${\overset{\textnormal{でき}}{\text{出来}}}$ るのなんて ${\overset{\textnormal{ふかのう}}{\text{不可能}}}$ なんですよね? \hfill\break
Building a time machine is impossible, right? }
 
\par{36. ${\overset{\textnormal{えいご}}{\text{英語}}}$ できるのなんて ${\overset{\textnormal{せいじん}}{\text{成人}}}$ としては ${\overset{\textnormal{あ}}{\text{当}}}$ たり ${\overset{\textnormal{まえ}}{\text{前}}}$ じゃないか。 \hfill\break
Being able to speak English is only natural as an adult, is it not? }
 
\par{37. ${\overset{\textnormal{ともみ}}{\text{智美}}}$ がああいう ${\overset{\textnormal{ひと}}{\text{人}}}$ だったなんて、 ${\overset{\textnormal{し}}{\text{知}}}$ らなかったです。 \hfill\break
I had no idea Tomomi was that kind of a person. }
 
\par{38. つい ${\overset{\textnormal{さいきん}}{\text{最近}}}$ まで ${\overset{\textnormal{こうこうせい}}{\text{高校生}}}$ だったなんて ${\overset{\textnormal{しん}}{\text{信}}}$ じられませんね。 \hfill\break
I can\textquotesingle t believe that I\textquotesingle ve been a high school student up until just recently. }
 
\par{39. この ${\overset{\textnormal{よ}}{\text{世}}}$ も ${\overset{\textnormal{お}}{\text{終}}}$ わりだなんて ${\overset{\textnormal{かんが}}{\text{考}}}$ えるのは ${\overset{\textnormal{おろ}}{\text{愚}}}$ かだ。 \hfill\break
Thinking something like the world's going to end is foolish! }
 
\par{40. どうしてここに ${\overset{\textnormal{き}}{\text{来}}}$ たのかなんて、 ${\overset{\textnormal{き}}{\text{決}}}$ まっている。 \hfill\break
It\textquotesingle s a given as to why (he) came here. }
 
\par{41. ${\overset{\textnormal{よ}}{\text{呼}}}$ び ${\overset{\textnormal{す}}{\text{捨}}}$ てか、さん ${\overset{\textnormal{づ}}{\text{付}}}$ けかなんて ${\overset{\textnormal{かんが}}{\text{考}}}$ える ${\overset{\textnormal{よゆう}}{\text{余裕}}}$ はなかった。 \hfill\break
There was no leeway to ponder whether to address him with or without an honorific. }
 
\begin{center}
\textbf{As a Final Particle } \hfill\break

\end{center}

\par{ To the same effect as above, you can also find なんて at the end of a sentence\slash statement to show utter surprise\slash disbelief. Depending on the tone, this may be used in ridicule as well. }
 
\par{42. どうして ${\overset{\textnormal{きゅう}}{\text{急}}}$ にお ${\overset{\textnormal{さけ}}{\text{酒}}}$ なんて・・・ \hfill\break
Why alcohol all so suddenly? }
 
\par{43. まさか、 ${\overset{\textnormal{わぼく}}{\text{和睦}}}$ の ${\overset{\textnormal{みち}}{\text{道}}}$ を ${\overset{\textnormal{えら}}{\text{選}}}$ ぶなんて・・・ \hfill\break
Wow, to think (they\textquotesingle re) choosing the path of reconciliation… }
 
\par{44. お ${\overset{\textnormal{しょうがつ}}{\text{正月}}}$ がこんなに ${\overset{\textnormal{たいへん}}{\text{大変}}}$ (だ)なんて・・・ \hfill\break
To think New Year\textquotesingle s would be this intense… }
 
\par{45. どちらも ${\overset{\textnormal{ほんもの}}{\text{本物}}}$ だなんて。 \hfill\break
Both are the real thing?! }
 
\par{46. これからずっと ${\overset{\textnormal{おび}}{\text{怯}}}$ えて ${\overset{\textnormal{く}}{\text{暮}}}$ らさなきゃいけないなんて。 \hfill\break
The thought of us having to live from here on out in fear…. }
 
\par{47. まだまだ ${\overset{\textnormal{し}}{\text{4}}}$ ${\overset{\textnormal{がつ}}{\text{月}}}$ なのに ${\overset{\textnormal{たつまき}}{\text{竜巻}}}$ が ${\overset{\textnormal{はっせい}}{\text{発生}}}$ するなんて・・・ \hfill\break
Tornadoes forming even though it\textquotesingle s still only April… }
 
\par{48. ${\overset{\textnormal{あい}}{\text{愛}}}$ してるよなんて。 \hfill\break
Huh, I love you\dothyp{}\dothyp{}\dothyp{} }
 
\par{49. あいつが「 ${\overset{\textnormal{たけしま}}{\text{竹島}}}$ は ${\overset{\textnormal{じつ}}{\text{実}}}$ は ${\overset{\textnormal{トクト}}{\text{独島}}}$ だと ${\overset{\textnormal{しゅちょう}}{\text{主張}}}$ してる」なんてもっぱらの ${\overset{\textnormal{うわさ}}{\text{噂}}}$ だよ。 \hfill\break
The persistent rumor is that he "claims Takeshima is Dokdo". }
 
\par{\textbf{Culture Note }: 竹島 is a group of islands in the Sea of Japan called the "Liancourt Rocks" in English. However, the islands are under territorial dispute with Korea. Japan claims they\textquotesingle re part of ${\overset{\textnormal{しまねけん}}{\text{島根県}}}$ (Shimane Prefecture). However,  Korea claims them as theirs under the name 独島. }
 
\begin{center}
\textbf{With the Copula や } 
\end{center}

\par{ The particle なんて does not follow the copula や, however, which is used in dialects like 関西弁 instead of だ. }
 
\par{50a. ${\overset{\textnormal{とうきょうべん}}{\text{東京弁}}}$ : ${\overset{\textnormal{か}}{\text{買}}}$ ってあげる(だ)なんて ${\overset{\textnormal{い}}{\text{言}}}$ っていない。〇 \hfill\break
50b. ${\overset{\textnormal{きょうとべん}}{\text{京都弁}}}$ : ${\overset{\textnormal{か}}{\text{買}}}$ うたげるなんて ${\overset{\textnormal{い}}{\text{言}}}$ うてへん。〇 \hfill\break
50c. ${\overset{\textnormal{きょうとべん}}{\text{京都弁}}}$ : ${\overset{\textnormal{か}}{\text{買}}}$ うたげるやなんて ${\overset{\textnormal{い}}{\text{言}}}$ うてへん。X \hfill\break
I'm not saying that I'll buy it for you. }
 
\begin{center}
\textbf{何て }
\end{center}

\par{ As you may have already noticed and as was alluded to earlier, you can also view なんてas being equivalent to などと言って・何と言って. As you can see in the following sentences, spelling it as 何て is possible when it used more literally. }
 
\par{51. ${\overset{\textnormal{いま}}{\text{今}}}$ 、何て ${\overset{\textnormal{い}}{\text{言}}}$ ったんだろ・・・ \hfill\break
What was it that (he) said now? }
 
\par{52. ${\overset{\textnormal{なまえ}}{\text{名前}}}$ は何ていうの。 \hfill\break
What\textquotesingle s your name? }
 
\par{53. この ${\overset{\textnormal{かんじ}}{\text{漢字}}}$ 、何て ${\overset{\textnormal{よ}}{\text{読}}}$ むの? \hfill\break
How do you read this Kanji? }
 
\par{54. こんな ${\overset{\textnormal{うわさ}}{\text{噂}}}$ が ${\overset{\textnormal{ひろ}}{\text{広}}}$ まってるなんて、 ${\overset{\textnormal{じょうし}}{\text{上司}}}$ が ${\overset{\textnormal{し}}{\text{知}}}$ ったら ${\overset{\textnormal{なん}}{\text{何}}}$ て ${\overset{\textnormal{い}}{\text{言}}}$ うかな。 \hfill\break
I wonder what the boss would say if he found out this sort of rumor was surprising. }
 
\par{55. ${\overset{\textnormal{いけだ}}{\text{池田}}}$ さんなんて ${\overset{\textnormal{ひと}}{\text{人}}}$ は ${\overset{\textnormal{し}}{\text{知}}}$ りません。 \hfill\break
I don\textquotesingle t know a person by the name of Ikeda-san. }
 
\par{56. スイーツでまずいなんてことってあんまりないですよね。 \hfill\break
There really aren't that many times in which something is a sweet but actually tastes bad. }
 
\par{\textbf{Sentence Note }: In Ex. 56, you could interpret なんて as being the same as などと言う. }
 
\par{57. ${\overset{\textnormal{だんな}}{\text{旦那}}}$ なんて ${\overset{\textnormal{よ}}{\text{呼}}}$ ばれる ${\overset{\textnormal{みぶん}}{\text{身分}}}$ じゃない! \hfill\break
I'm not in the position to be called “husband”! }
 
\par{58. 「 ${\overset{\textnormal{しごと}}{\text{仕事}}}$ と ${\overset{\textnormal{わたし}}{\text{私}}}$ 、どっちが ${\overset{\textnormal{だいじ}}{\text{大事}}}$ なの?」なんて ${\overset{\textnormal{ぜったい}}{\text{絶対}}}$ に ${\overset{\textnormal{い}}{\text{言}}}$ ってはいけない ${\overset{\textnormal{きんく}}{\text{禁句}}}$ なんですよ。 \hfill\break
“Which is more important, work or me?” is a taboo that you should absolutely never utter. }
 
\par{59. すごいなんて ${\overset{\textnormal{ことば}}{\text{言葉}}}$ では ${\overset{\textnormal{ぜんぜんた}}{\text{全然足}}}$ りない。 \hfill\break
Using a word like “awesome” is totally not good enough. }
 
\par{60. ${\overset{\textnormal{わたし}}{\text{私}}}$ 、なんて ${\overset{\textnormal{こと}}{\text{事}}}$ を! \hfill\break
What have I done! }
 
\begin{center}
\textbf{Oh How\dothyp{}\dothyp{}\dothyp{} }
\end{center}

\par{ Lastly, it is also important to note that なんて can be used in its own set phrases (Ex. 64) as well as an adverb meaning “oh how…,” in which case it is synonymous with 何と (Exs. 61-63). }
 
\par{61. カナダの ${\overset{\textnormal{ふゆ}}{\text{冬}}}$ って、 ${\overset{\textnormal{ほんとう}}{\text{本当}}}$ になんて ${\overset{\textnormal{うつく}}{\text{美}}}$ しいんでしょう。 \hfill\break
Oh how beautiful winter in Canada truly is! }
 
\par{62. うちのアパートの ${\overset{\textnormal{すいどうすい}}{\text{水道水}}}$ 、なんてまずいんだ。 \hfill\break
The tap water at my apartment, it\textquotesingle s just awful. }
 
\par{63. なんてつまらないんだ! \hfill\break
How boring! }
 
\par{64. なんて、( ${\overset{\textnormal{じょうだん}}{\text{冗談}}}$ で)ね。 \hfill\break
Just joking. }
    