    
\chapter{~てしまう}

\begin{center}
\begin{Large}
第135課: ~てしまう 
\end{Large}
\end{center}
 
\par{ ~てしまう is an extremely important ending. It is often described as showing perfected actions or things done on accident, but what does this actually mean and is there more to it? This lesson will discuss the nuances that it has as well as important variants used in every speech of it. }
      
\section{~てしまう}
 
\par{ \textbf{Spelling and Other Meanings } }

\par{ The verb しまう either means "to end" or "to keep\slash put back". In 漢字, you might see it spelled as 仕舞う, 終う, or 了う. As you can imagine, the latter two spellings are perfect fits for its meaning of "to end". The only time when you should take this verb literally when after て is when there is an intentional pause. Then, you have to be careful about what other possible homophonous phrases fit--especially when listening. }

\par{ Take for instance this classic demonstration in Japanese linguistics concerning this very thing. }

\par{1. 変態な本を読んで、しまった。 \hfill\break
I read a perverted book, and (then) I put it up. }

\par{2. 変態な本を読んでしまった。 \hfill\break
I accidentally read a perverted book. }

\par{ \textbf{The 3 Nuances of ~てしまう }}

\par{ So, what is ~てしまう? It sadly doesn't have one meaning. There are basically three ways it is used. Intonation plays a role in discerning whether しまう is supplementary or not. When supplementary, you get the pattern ~てしまう, which has one of the three meanings below. }

\par{1. To show a sense of regret\slash surprise when you did have volition in doing something, but it turned out to be bad to do. This can also be sarcastically and cleverly used in a positive attitude while still being natural. Sexual relations come to mind. \hfill\break
2.  To show perfective\slash punctual achievement. This shows that an action has been completed. It should not be used with something like ~そうだ, which is a nonfactual speech modal. \hfill\break
3. To show unintentional action--"accidentally". Often used with adverbs like うっかり (absentmindedly) and ぐうぜんに (unexpectedly). }

\par{\textbf{Clarification Notes }: }

\par{1. Statives (verbs that show state not action) can be used with ~てしまう too. However, they are not perfective in any sense. When put together, the perfective is never expressed. Rather, it goes back to point 1. \hfill\break
2. It's important to view these usages as instances of the same thing interpreted differently in context because there are cases where more than one reading is possible. At times, all three can be meant together. In  "うっかり花瓶を割ってしまった", the act of breaking the vase is perfective. It could have been accidentally broken. Plus, you could feel regret over having broken it. }

\par{ \textbf{Examples }}

\par{1. スープは ${\overset{\textnormal{さ}}{\text{冷}}}$ めてしまっている。 \hfill\break
My soup has gotten cold. }

\par{2. あの ${\overset{\textnormal{}}{\text{子供}}}$ はのろのろと ${\overset{\textnormal{いす}}{\text{椅子}}}$ から ${\overset{\textnormal{}}{\text{立}}}$ ち ${\overset{\textnormal{}}{\text{上}}}$ がった ${\overset{\textnormal{}}{\text{後}}}$ 、 ${\overset{\textnormal{とつぜんたお}}{\text{突然倒}}}$ れてしまったよ。 \hfill\break
That kid over there suddenly collapsed after he rose up sluggishly from the chair. }

\par{3. ${\overset{\textnormal{こいびと}}{\text{恋人}}}$ たちは ${\overset{\textnormal{}}{\text{二人}}}$ で ${\overset{\textnormal{}}{\text{サンバ}}}$ を ${\overset{\textnormal{おど}}{\text{踊}}}$ らなくなってしまった。 \hfill\break
The couple ended up not dancing the samba together. }

\par{4. ${\overset{\textnormal{かびん}}{\text{花瓶}}}$ を ${\overset{\textnormal{こわ}}{\text{壊}}}$ してしまって、 ${\overset{\textnormal{ばつ}}{\text{罰}}}$ を ${\overset{\textnormal{のが}}{\text{逃}}}$ れたい。 \hfill\break
I accidentally broke the vase, and I hope I don't get punished. \hfill\break
More literally: I accidentally broke the vase, and I want to escape punishment. }

\par{${\overset{\textnormal{}}{\text{5. 窓}}}$ を ${\overset{\textnormal{}}{\text{閉}}}$ めなかったので、 ${\overset{\textnormal{}}{\text{風邪}}}$ を ${\overset{\textnormal{}}{\text{引}}}$ いてしまいました。 \hfill\break
I didn't close the window, and so I caught a cold. }

\par{6. ${\overset{\textnormal{なつふく}}{\text{夏服}}}$ を ${\overset{\textnormal{おしい}}{\text{押入}}}$ れにしまった。 \hfill\break
I put my summer clothes in the closet. }

\par{7. 明日までにレポートを書いてしまいます。 \hfill\break
I'll finish writing the report by tomorrow. }

\par{8. お金を使ってしまった。 \hfill\break
I regrettably used the money. }

\par{Practice: Translate the following. }

\par{1. 宿題をしてしまいました。 \hfill\break
2. I ate them all. \hfill\break
3. 僕の車が故障してしまった。 \hfill\break
4. 眠ってしまった。 \hfill\break
5. To keep information back. }

\par{\textbf{~ちゃう \& ~じゃう }}

\par{ ~てしまう may be contracted to -ちゃう (playful) or -ちまう (vulgar\slash mature men). These forms are voiced as ~じゃう and ~じまう respectively when て is used with certain 五段 verbs. }

\begin{ltabulary}{|P|P|P|}
\hline 

食べてしまう \textrightarrow  食べちゃう & 見てしまう \textrightarrow  見ちゃう & 呼んでしまう \textrightarrow  呼んじゃう \\ \cline{1-3}

読んでしまう \textrightarrow  読んじゃう & 死んでしまう \textrightarrow  死んじゃう & ばれてしまう \textrightarrow  ばれちゃう \\ \cline{1-3}

着てしまう \textrightarrow  着ちゃう & 切ってしまう \textrightarrow  切っちゃう & してしまう \textrightarrow  しちゃう \\ \cline{1-3}

\end{ltabulary}
\hfill\break
\textbf{Grammar Note }: The use of slang and polite speech is usually improper, but there are times when these casual forms are used in familiar polite speech, but this would be most common by children and women than men. \hfill\break

\par{ \textbf{Examples }}

\par{9, 見る見るうちに ${\overset{\textnormal{ろうじん}}{\text{老人}}}$ になっちゃった! \hfill\break
In the blink of an eye, I turned into an old person! }

\par{10. ガソリンが ${\overset{\textnormal{}}{\text{切}}}$ れちゃった。 ${\overset{\textnormal{もより}}{\text{最寄}}}$ の(ガソリン)スタンドまで ${\overset{\textnormal{ひ}}{\text{引}}}$ っ ${\overset{\textnormal{ぱ}}{\text{張}}}$ っててもらえる? \hfill\break
I ran out of gas. Could you tow me to the nearest gas station? }

\par{11. その ${\overset{\textnormal{}}{\text{本}}}$ を ${\overset{\textnormal{}}{\text{読}}}$ んじゃいました。 \hfill\break
I finished reading that book. }

\par{${\overset{\textnormal{}}{\text{12. 犬}}}$ が ${\overset{\textnormal{くる}}{\text{狂}}}$ っちゃった。 \hfill\break
The dog went crazy. }

\par{${\overset{\textnormal{}}{\text{13. 中国語}}}$ かと ${\overset{\textnormal{}}{\text{思}}}$ っちゃった。 \hfill\break
I mistook it for Chinese. }

\par{14. ${\overset{\textnormal{さいふ}}{\text{財布}}}$ を ${\overset{\textnormal{}}{\text{忘}}}$ れちゃったよ。 \hfill\break
I forgot my wallet. }

\par{15. どうして空き地にし\{てしま・ちゃ\}わないのか。 \hfill\break
Why won't they make it into an empty lot? }

\par{16. ${\overset{\textnormal{ふちゅうい}}{\text{不注意}}}$ からカップを ${\overset{\textnormal{おと}}{\text{落}}}$ しちゃった。 \hfill\break
I accidentally spilled the cup due to carelessness. }

\par{17. とても ${\overset{\textnormal{}}{\text{熱}}}$ かったから、 ${\overset{\textnormal{}}{\text{口}}}$ を ${\overset{\textnormal{やけど}}{\text{火傷}}}$ しちゃった。 \hfill\break
Because it was very hot, I accidentally scalded my mouth. }

\par{ \textbf{関西弁: ~て(し)もた }}

\par{ The 関西弁 version is so commonly seen in manga and anime that not mentioning it would do you a disservice. しまった and しもた share the same origin. ~た instead attached to the 終止形. This produced しまうた. The vowel sequence au simplified to a long o in Western dialects, and this was then shortened in this phrase and many others. The し is then dropped in even more colloquial speech. }

\par{${\overset{\textnormal{}}{\text{18. 忘}}}$ れて(し)もた! \hfill\break
I completely forgot! }

\par{19. 買うて(し)もた。 \hfill\break
I accidentally bought it. \hfill\break
\hfill\break
\textbf{関西弁の音便 (Kansai Dialect Sound Change) }: かって \textrightarrow  こうて. }

\par{\textbf{はめになる }}

\par{This phrase is equivalent to "end up; wind up; (come down) to". はめ may be written in 漢字 as 羽目 or 破目. はめ  in this case actually means "bind" as in an awkward situation. }

\par{${\overset{\textnormal{}}{\text{20. 彼は言}}}$ いなりになるはめになった。 \hfill\break
He ended up giving in to him. }

\par{${\overset{\textnormal{}}{\text{21. 僕}}}$ は ${\overset{\textnormal{ばっきん}}{\text{罰金}}}$ を ${\overset{\textnormal{しはら}}{\text{支払}}}$ うはめになった。 \hfill\break
I ended up paying a fine. }

\par{22. ねずみは ${\overset{\textnormal{おぼ}}{\text{溺}}}$ れるはめになった。 \hfill\break
The mouse ended up drowning. }

\par{23. 疑うはめになる。 \hfill\break
To come to suspect. }
      
\section{Key}
 
\par{Practice }

\par{1.    I (have) completed by homework. \hfill\break
2.    全部食べてしまった。 \hfill\break
3.    My car broke down. \hfill\break
4.    I accidentally slept. \hfill\break
5.    情報をしまっておく。 }
    