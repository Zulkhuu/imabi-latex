    
\chapter{Too\slash Continue}

\begin{center}
\begin{Large}
第107課: Too\slash Continue: ~すぎる, ~続ける, \& ~急ぐ 
\end{Large}
\end{center}
 
\par{  Yes, the title is a pun. }
      
\section{~すぎる}
 
\par{ 過ぎる means "to pass" and is used both transitively and intransitively. It may be used in basically any situation that relates to someone or something passing by. ~過ぎる shows something "is too\dothyp{}\dothyp{}\dothyp{}”or someone is doing something "too much". When used with adjectives, you drop the く or に in the 連用形 altogether. Likewise, with 形容動詞, you don't use the copula. }

\begin{ltabulary}{|P|P|}
\hline 

Verb & 食べる + すぎる \textrightarrow  食べすぎる \\ \cline{1-2}

形容詞 & 小さい + すぎる \textrightarrow  小さすぎる \\ \cline{1-2}

形容動詞 & 簡単 + すぎる \textrightarrow  簡単すぎる \\ \cline{1-2}

\end{ltabulary}

\par{\textbf{漢字 Note }: This is usually written in ひらがな when used as an ending. }

\par{1. 酒を飲みすぎて、 ${\overset{\textnormal{ふつかよ}}{\text{二日酔}}}$ いがある。 \hfill\break
I drank too much sake, and I have a hangover. }

\par{2. 小さすぎる。大きいのがある? \hfill\break
This is too small. Do you have a bigger one? }

\par{3. 食べ過ぎないでくださいね。 \hfill\break
Please don't overeat. }

\par{${\overset{\textnormal{まよなか}}{\text{4. 真夜中}}}$ を過ぎる。 \hfill\break
To pass through midnight. }

\par{\textbf{漢字 Note }: Be careful to not confuse this 中 with the suffix ちゅう・じゅう. It turns out that 夜中, which is read as やちゅう, means "at night", but it's a 書き言葉. }

\par{5. あんた、頭がよすぎるよ。(Casual; potentially rude) \hfill\break
You're too smart! }

\par{6. この問題は難しすぎる。 \hfill\break
This question is too difficult. }

\par{7. そのカメラは高すぎるね。 \hfill\break
Isn't that camera too high? }

\par{8. 明日では ${\overset{\textnormal{おそ}}{\text{遅}}}$ すぎるでしょう。 \hfill\break
Tomorrow will probably be too late. }

\par{${\overset{\textnormal{どうろ}}{\text{9. 道路}}}$ の ${\overset{\textnormal{おうだん}}{\text{横断}}}$ にはいくら ${\overset{\textnormal{ちゅうい}}{\text{注意}}}$ してもしすぎることはない。 \hfill\break
You can never be too careful when crossing the street. }

\par{10. 君は彼女に ${\overset{\textnormal{きたい}}{\text{期待}}}$ をかけすぎる。 \hfill\break
You expect too much of her. }

\par{11. いくら好きでも、食べすぎると、体に悪いです。 \hfill\break
No matter how much you like it, eating too much is bad for your health. }

\par{12. 彼はやりすぎたよ。 \hfill\break
He went too far. }

\par{13. 時間が過ぎた。 \hfill\break
Time passed. }

\par{14. この物理学の問題は難しすぎて、理解するのは無理です。 \hfill\break
This physics problem is too difficult, and understanding it is useless. }

\par{15. コンピューターの画面に近すぎないことが大切だ。 \hfill\break
It's important to not get too close to the computer screen. }

\par{16. 考え事をしながら歩いていたら、自分の家の前を通り過ぎてしまった。 \hfill\break
Lost in thought, I walked past my house. }

\par{\textbf{Word Note: }通り過ぎる usually means to "pass by", but it can also have the sense "going too far". }

\par{\textbf{Grammar Note }: What about the negative? Take the following two similar phrases into consideration. 読まなすぎる vs. 読みすぎない. The first one states that one "reads too little". The second states that one "doesn't read too much". There may also be cases when ~なすぎる is inappropriate for pragmatic reasons in particular contexts. }

\par{17. 彼は何もできなすぎる。 △ \hfill\break
He can't do anything. }

\begin{center}
 \textbf{The Intensifier ~ない }
\end{center}

\par{ There is also a suffix ~ない that increases the intensity of a given adjective. Inserting さ when using them with ~すぎる is wrong, but speakers occasionally do so anyway. }

\par{ The confusing part about this is that this does come from the negative ない. It so happened that late in Classical Japanese it acquired the meaning of just being an intensifier to particular phrases. }

\par{18. だらしない生活をする。 \hfill\break
To lead a sloven lifestyle. }

\par{19. あの映画はまったくえげつないよ。 \hfill\break
That movie is just completely dirty. }

\par{20. はしたなく言い争う。 \hfill\break
To immodestly quarrel. }

\par{21. しがないサラリーマンの人生 \hfill\break
The humble life of a salary-man }

\par{22. あどけない子供の笑顔を見る。 \hfill\break
To look at the angelic smile of a child, }

\par{23. ぎこちなく運転しちゃだめだ。 \hfill\break
Don't drive all clumsy. }

\par{24. \{いとけない・無邪気な\}子 \hfill\break
An innocent child }

\par{25. 滅相もない・滅相なことをいうものじゃないよ! \hfill\break
Don't say something so absurd! }

\par{ There are a few cases where the original adjective and the adjective with the intensifier ~ない exist, just like above. Another example is ${\overset{\textnormal{せわ}}{\text{忙}}}$ しない and 忙しい. The first means "seems busy" and the other means "really busy", but it is still the case that the former is more intense. }

\par{26. 彼はせわしい人だ。 \hfill\break
He's a real busybody. }

\par{27. 忙しない季節 \hfill\break
A season so busy with no time to rest }

\par{  Another odd pair is 切な versus 切ない. 切な is now typically 切なる, odd giving that this is more Classical in form. The word means "earnest", and you would think 切ない would mean that too. It did, but over time it gained more negative undertones, and now it refers to heartrending sadness. This, though, sprouted out from the meaning of "earnest". }

\par{28. 切な(る)顔 \hfill\break
An earnest face }

\par{29. 切なさを堪える。 \hfill\break
To withhold heart-wrenching. }

\par{ Another weird word is 怪しからん. This comes from the old verb 怪しかる, but rather than being opposites, they accidentally became the same thing, both meaning "inexcusable". }

\par{30. 親切に扱ってくれた人の不満をいうとは怪しからん。(Dialectical\slash older person) \hfill\break
Complaining about those who have treated you well is inexcusable. }
      
\section{~続ける}
 
\par{ The 一段 verb ${\overset{\textnormal{つづ}}{\text{続}}}$ ける means "to continue". The verb is normally used for "one's own" actions. If the verb happens naturally, the verb 続く is used instead. Although you would think that this distinction would be carried in compounds, ~続く is essentially only seen with the verb ${\overset{\textnormal{ふ}}{\text{降}}}$ る. }

\begin{ltabulary}{|P|P|P|}
\hline 

一段 Verbs & 見る + 続ける \textrightarrow  &  見 続ける \\ \cline{1-3}

五段 Verbs & 泳ぐ + 続ける \textrightarrow  &  泳ぎ 続ける \\ \cline{1-3}

Noun + Copula & Nounだ + 続ける \textrightarrow  &  男性であり 続ける \\ \cline{1-3}

形容詞 & 美しい + 続ける \textrightarrow  &  美しくあり 続ける \\ \cline{1-3}

形容動詞 & 自由な + 続ける \textrightarrow  &  自由であり 続ける \\ \cline{1-3}

\end{ltabulary}

\par{31. 日本語を勉強し続けてください。 \hfill\break
Please continue studying Japanese. }

\par{${\overset{\textnormal{わ}}{\text{32. 我}}}$ がままを ${\overset{\textnormal{とお}}{\text{通}}}$ し続ける。 \hfill\break
Continue one's own way. }

\par{33. 彼氏んちまで歩き続けた。(Casual 東京弁) \hfill\break
I continued walking up to my boyfriend's house. }

\par{34. 電話のベルは鳴り続けた。 \hfill\break
The phone kept ringing. }

\par{35. ハチ ${\overset{\textnormal{こう}}{\text{公}}}$ は来る日も来る日も ${\overset{\textnormal{しゅじん}}{\text{主人}}}$ の帰りを待ち続けた。 \hfill\break
Hachiko waited day after day for his master to return. }

\par{\textbf{Culture Note }: ハチ公 is a dog that was so loyal to his master, it waited for him to return at the station even after the owner's death. }

\par{36. 彼らは ${\overset{\textnormal{はたら}}{\text{働}}}$ き続けた。 \hfill\break
They continued working. }

\par{37. 赤ちゃんはぐっすりと ${\overset{\textnormal{ねむ}}{\text{眠}}}$ り続けていた。 \hfill\break
The baby was continuing to sleep soundly. }

\par{38. 火が ${\overset{\textnormal{も}}{\text{燃}}}$ え続けている。 \hfill\break
The fire is continuing to burn. }

\par{39. 雨が降り続いている。 \hfill\break
It is continuing to rain. }

\par{${\overset{\textnormal{すで}}{\text{40. 既}}}$ に ${\overset{\textnormal{むくろ}}{\text{骸}}}$ になってしまったとはいえ、やはり母は、こうして祥子の頭をいためる存在であり続けるのだ。 \hfill\break
Although she has already become a corpse, mother will after all continue an existence of having Sachiko ache this way. \hfill\break
From 冷たい ${\overset{\textnormal{ゆうわく}}{\text{誘惑}}}$ by ${\overset{\textnormal{のなみ}}{\text{乃南}}}$ アサ. }

\par{\textbf{漢字 Notes }: }

\par{1. すでに is often not spelled in 漢字. However, 既に is more formal and literary. \hfill\break
2. 骸 means "corpse" and is stronger than other words like 遺体, which is typically used in news reports, or 死体. \hfill\break
3. 祥 is a name character meaning "auspicious". \hfill\break
4. 乃 is also a name character. Its most important reading is の. This is used to write the particle の in many names and in old writing. }

\begin{center}
 \textbf{~続ける VS ~のを続ける }
\end{center}

\par{ These two patterns translate the same as "to continue\dothyp{}\dothyp{}\dothyp{}", but they're not exactly the same. ~続ける shows an action\slash state that is ongoing. The latter simply states that there is a continuation of some sort. For example, a way to distinguish this in English would be similar to "she is continuing to watch the show"  versus "she continues to watch the show". The first shows the ongoing state of her watching the show whereas the latter just states that she continues to regularly watch the show. }

\par{41. 彼女はその番組を観続けている。 \hfill\break
She's continuing to watch that show. }

\par{42. 彼女は、その番組を観るのを続けている。 \hfill\break
She continues to watch that show. }

\par{\textbf{漢字 Note }: You can use 見 instead of 観, but the latter is often used for watching things like movies, shows, etc. }
      
\section{~急ぐ}
 
\par{${\overset{\textnormal{}}{\text{急}}}$ ぐ means "to hurry" and can be seen in both transitive and intransitive contexts. As a transitive verb, it can be used in compounds to show that one is trying to hurry and finish something. Its –て form 急いで can also be used to show this as an adverbial phrase. It is best for this ending to see it used in a compound verb before using it as such. }

\par{43. 売り急ぐ。 \hfill\break
To be in a hurry to sell. }

\par{44a. リンゴを買い急ぐ。(ちょっと不自然) \hfill\break
44b. 急いでリンゴを買う。(もっと自然) \hfill\break
To buy apples in a hurry. }

\par{45. 彼は死に急いだ。 \hfill\break
He hastened to his death. }

\par{46. 急いで ${\overset{\textnormal{ふく}}{\text{服}}}$ を ${\overset{\textnormal{き}}{\text{着}}}$ た。 \hfill\break
I hurried and got dressed. }

\par{47. 学校へ急ぐ。 \hfill\break
To hurry to school. }

\par{\textbf{Word Note }: ${\overset{\textnormal{あせ}}{\text{焦}}}$ る is not the same thing. This implies frustration that things might not go as planned. }
 
\par{48. 彼は焦って ${\overset{\textnormal{しっぱい}}{\text{失敗}}}$ した。 \hfill\break
He hurried it and failed. }
    