    
\chapter{The Particle Nagara ながら I}

\begin{center}
\begin{Large}
第104課: The Particle Nagara ながら I: Simultaneous Action 
\end{Large}
\end{center}
 
\par{ In this lesson, you will be introduced to the particle \emph{nagara }ながら. This particle is a conjunctive particle that is typically used to mean “while” in the sense of doing two things at the same time or in the same time span. Because using it will require conjugation, we\textquotesingle ll look at that first before delving into how it\textquotesingle s used. }

\begin{ltabulary}{|P|P|}
\hline 

 \emph{Ichidan }Verbs & 見る +ながら = 見ながら \emph{\emph{Miru } +  \emph{nagara } =  \emph{Minagara }}\\ \cline{1-2}

 \emph{Godan }Verbs & 飲む + ながら = 飲みながら \emph{\emph{Nomu } +  \emph{nagara } =  \emph{Nominagara }}\\ \cline{1-2}

 \emph{Suru }& する + ながら = しながら \emph{\emph{Suru } +  \emph{nagara } =  \emph{shinagara }}\\ \cline{1-2}

 \emph{Kuru }& 来る + ながら = きながら \emph{\emph{Kuru } +  \emph{nagara } =  \emph{kinagara }}\\ \cline{1-2}

\end{ltabulary}

\par{ As you can see, using \emph{nagara }ながら from the point of conjugation is no different than using – \emph{masu }ます. }
      
\section{Simultaneous Action}
 
\par{ The primary use of \emph{nagara }ながら is to show simultaneous action or in the same time span. The clauses of the sentence you make must have the same subject, and the main verb of the sentence is found in the latter clause. }
 
\par{1. テレビを ${\overset{\textnormal{み}}{\text{見}}}$ ながらご ${\overset{\textnormal{はん}}{\text{飯}}}$ を ${\overset{\textnormal{た}}{\text{食}}}$ べる。 \hfill\break
 \emph{Terebi wo minagara gohan wo taberu. }\hfill\break
To eat one\textquotesingle s meal while watching TV. }
 
\par{2. ご ${\overset{\textnormal{はん}}{\text{飯}}}$ を ${\overset{\textnormal{た}}{\text{食}}}$ べながらテレビを ${\overset{\textnormal{み}}{\text{見}}}$ る。 \hfill\break
 \emph{Gohan wo tabenagara terebi wo miru. }\hfill\break
To watch TV while eating one\textquotesingle s meal. }
 
\par{ As you can see, \emph{nagara }ながら stands for the “while” in these two sentences. The primary action is also clearly different. This is an incredibly important point because it is not always the case that switching the verb that has \emph{nagara }ながら results in a sensible statement. }
 
\par{3a. ${\overset{\textnormal{な}}{\text{泣}}}$ きながら ${\overset{\textnormal{たま}}{\text{玉}}}$ ねぎを ${\overset{\textnormal{き}}{\text{切}}}$ っていました。○ \hfill\break
 \emph{Nakinagara tamanegi wo kitte imashita. } \hfill\break
I was cutting up onions with tears streaming down. \hfill\break
3b. ${\overset{\textnormal{たま}}{\text{玉}}}$ ねぎを ${\overset{\textnormal{き}}{\text{切}}}$ りながら ${\overset{\textnormal{な}}{\text{泣}}}$ いていました。?? \hfill\break
 \emph{Tamanegi wo kirinagara naite imashita. \hfill\break
 }I was crying as I was cutting up onions. }
 
\par{ You are cutting onions while crying because the onions are undoubtedly making you cry. In Japanese, this is expressed with 3a. Let's say, for some odd reason, your other half decided to break up with you over Facebook right before you began to prepare yourself a meal containing onion. Way before you cut into the first onion, you're already shedding a large volume of tears. In this situation, 3b. would make sense. Although the English could still imply cause-and-effect between "crying" and "cutting onions," the onions are not the cause of crying in 3b as is the case in 3a. }
 
\par{4. ${\overset{\textnormal{じしょ}}{\text{辞書}}}$ で ${\overset{\textnormal{し}}{\text{知}}}$ らない ${\overset{\textnormal{たんご}}{\text{単語}}}$ を ${\overset{\textnormal{しら}}{\text{調}}}$ べながら、フランス ${\overset{\textnormal{ご}}{\text{語}}}$ の ${\overset{\textnormal{ほん}}{\text{本}}}$ を ${\overset{\textnormal{よ}}{\text{読}}}$ みたいと ${\overset{\textnormal{おも}}{\text{思}}}$ います。○ \hfill\break
 \emph{Jisho wo shiranai tango wo shirabenagara, Furansugo no hon wo yomitai to omoimasu. \hfill\break
 }I want to read a French book while looking up words I don\textquotesingle t know with a dictionary. \hfill\break
フランス ${\overset{\textnormal{ご}}{\text{語}}}$ の ${\overset{\textnormal{ほん}}{\text{本}}}$ を ${\overset{\textnormal{よ}}{\text{読}}}$ みながら ${\overset{\textnormal{じしょ}}{\text{辞書}}}$ で ${\overset{\textnormal{し}}{\text{知}}}$ らない ${\overset{\textnormal{たんご}}{\text{単語}}}$ を ${\overset{\textnormal{しら}}{\text{調}}}$ べたいと ${\overset{\textnormal{おも}}{\text{思}}}$ います。?? \hfill\break
 \emph{Furansugo no hon wo yominagara jisho de shiranai tango wo shirabetai to omoimasu. \hfill\break
 }I want to look up words I don\textquotesingle t know about as I\textquotesingle m reading a French book. }
 
\par{ You've just received a French novel for the first time. Excited, you also realize that a dictionary will be your best friend to look up words you don\textquotesingle t know in said book. Because reading the book is the primary action, the first sentence is what makes sense. Now, say you are randomly looking up words you don't happen to know irrespective of the French book you\textquotesingle re reading. Maybe you decide to look up Spanish words randomly instead. In such an odd situation, the latter may make sense, but you\textquotesingle d best explain yourself. }

\begin{center}
\textbf{More Examples } 
\end{center}
 
\par{5. ${\overset{\textnormal{てんしん}}{\text{天津}}}$ で ${\overset{\textnormal{はたら}}{\text{働}}}$ きながら ${\overset{\textnormal{ちゅうごくご}}{\text{中国語}}}$ を ${\overset{\textnormal{まな}}{\text{学}}}$ んでいます。 \hfill\break
 \emph{Tenshin de hatarakinagara chūgokugo wo manande imasu. }\hfill\break
I\textquotesingle m studying Chinese while working in Tianjin. }
 
\par{6. ${\overset{\textnormal{ある}}{\text{歩}}}$ きながら ${\overset{\textnormal{どくしょ}}{\text{読書}}}$ をするのはとても ${\overset{\textnormal{きけん}}{\text{危険}}}$ です。 \hfill\break
 \emph{Arukinagara dokusho wo suru no wa totemo kiken desu. }\hfill\break
Reading a book while walking is very dangerous. }
 
\par{7. コーヒーを ${\overset{\textnormal{の}}{\text{飲}}}$ みながら、ゆっくりと ${\overset{\textnormal{かいわ}}{\text{会話}}}$ しましょう。 \hfill\break
 \emph{Kōhii wo nominagara, yukkuri to kaiwa shimashō. }\hfill\break
Let\textquotesingle s converse slowly while we drink coffee. }
 
\par{8. ${\overset{\textnormal{くるま}}{\text{車}}}$ を ${\overset{\textnormal{うんてん}}{\text{運転}}}$ しながら ${\overset{\textnormal{でんわ}}{\text{電話}}}$ をしてはいけません。 \hfill\break
 \emph{Kuruma wo unten shinagara, denwa wo shite wa ikemasen. }\hfill\break
You mustn\textquotesingle t talk on the phone while driving a car. }
 
\par{9. ${\overset{\textnormal{せんせい}}{\text{先生}}}$ の ${\overset{\textnormal{はなし}}{\text{話}}}$ を ${\overset{\textnormal{き}}{\text{聞}}}$ きながら、ノートにメモを ${\overset{\textnormal{と}}{\text{取}}}$ ります。 \hfill\break
 \emph{Sensei no hanashi wo kikinagara, nōto ni memo wo torimasu. }\hfill\break
I take notes in my notebook while listening to my teacher. }
 
\par{10. ご ${\overset{\textnormal{はんた}}{\text{飯食}}}$ べながら ${\overset{\textnormal{はな}}{\text{話}}}$ さないで、 ${\overset{\textnormal{ぎょうぎ}}{\text{行儀}}}$ が ${\overset{\textnormal{わる}}{\text{悪}}}$ いのよ! (Feminine) \hfill\break
 \emph{Gohan tabenagara hanasaide, gyōgi ga warui no yo! }\hfill\break
Don\textquotesingle t talk while eating your food; what bad manners! }
 
\par{11. ${\overset{\textnormal{でんわ}}{\text{電話}}}$ で ${\overset{\textnormal{はな}}{\text{話}}}$ しながらタバコを ${\overset{\textnormal{す}}{\text{吸}}}$ うのはマナー ${\overset{\textnormal{いはん}}{\text{違反}}}$ だと ${\overset{\textnormal{おも}}{\text{思}}}$ いませんか。 \hfill\break
 \emph{Denwa de hanashinagara tabako wo sū no wa manā ihan da to omoimasen ka? }\hfill\break
Don\textquotesingle t you think smoking while talking on the phone is a breach of etiquette? }
 
\par{12. ${\overset{\textnormal{しゅくだい}}{\text{宿題}}}$ しながらピザを ${\overset{\textnormal{く}}{\text{食}}}$ ってて ${\overset{\textnormal{き}}{\text{気}}}$ づいたらなくなっていた。 \hfill\break
 \emph{Shukudai shinagara piza wo kuttete kizuitara nakunatte ita. }\hfill\break
I was eating pizza while doing my homework, but once I noticed it was gone. }
 
\par{13. ${\overset{\textnormal{は}}{\text{晴}}}$ れた ${\overset{\textnormal{ひ}}{\text{日}}}$ に ${\overset{\textnormal{す}}{\text{好}}}$ きな ${\overset{\textnormal{おんがく}}{\text{音楽}}}$ を ${\overset{\textnormal{き}}{\text{聴}}}$ きながらドライブ(を)するのって ${\overset{\textnormal{しふく}}{\text{至福}}}$ のひと ${\overset{\textnormal{とき}}{\text{時}}}$ だと ${\overset{\textnormal{おも}}{\text{思}}}$ いません? \hfill\break
 \emph{Hareta hi ni suki na ongaku wo kikinagara doraibu (wo) suru no tte shifuku no hitotoki da to omoimasen ka? }\hfill\break
Don\textquotesingle t you think driving while listening to your favorite music on a clear day is a blissful moment? }
 
\par{14. ${\overset{\textnormal{わたし}}{\text{私}}}$ は ${\overset{\textnormal{しんぶん}}{\text{新聞}}}$ を ${\overset{\textnormal{よ}}{\text{読}}}$ みながら、バスを ${\overset{\textnormal{ま}}{\text{待}}}$ っていました。 \hfill\break
 \emph{Watashi wa shimbun wo yominagara, basu wo matte imashita. }\hfill\break
I was waiting for the bus while reading the newspaper. }
 
\par{15. ${\overset{\textnormal{まち}}{\text{町}}}$ の ${\overset{\textnormal{ぎんこう}}{\text{銀行}}}$ で ${\overset{\textnormal{はたら}}{\text{働}}}$ きながら、 ${\overset{\textnormal{きしょうよほうし}}{\text{気象予報士}}}$ の ${\overset{\textnormal{べんきょう}}{\text{勉強}}}$ をしています。 \hfill\break
 \emph{Machi no ginkō de hatarakinagara,  kishō yohōshi no benkyō wo shite imasu. \hfill\break
 }I\textquotesingle m studying to be a weather forecaster while working at the town bank. }
 
\par{16. ともかく、 ${\overset{\textnormal{ひと}}{\text{人}}}$ は ${\overset{\textnormal{なや}}{\text{悩}}}$ みながら ${\overset{\textnormal{い}}{\text{生}}}$ きるしかないんですか。 \hfill\break
 \emph{Tomokaku, hito wa nayaminagara ikiru shika nai n desu ka? }\hfill\break
Anyhow, do people have no choice but to live in worry? }
 
\par{17. ${\overset{\textnormal{ちょうしゅう}}{\text{聴衆}}}$ の ${\overset{\textnormal{ひと}}{\text{人}}}$ たちも ${\overset{\textnormal{えがお}}{\text{笑顔}}}$ を ${\overset{\textnormal{み}}{\text{見}}}$ せながら ${\overset{\textnormal{ねっしん}}{\text{熱心}}}$ に ${\overset{\textnormal{き}}{\text{聞}}}$ いていた。 \hfill\break
 \emph{Chōshū no hitotachi mo egao wo misenagara nesshin ni kiite ita. }\hfill\break
The people in the audience were also listening with enthusiasm while showing smiles in their faces. }
 
\par{18. ${\overset{\textnormal{ぶかつ}}{\text{部活}}}$ もやりながら、 ${\overset{\textnormal{じゅく}}{\text{塾}}}$ に ${\overset{\textnormal{かよ}}{\text{通}}}$ っていました。 \hfill\break
 \emph{Bukatsu mo yarinagara, juku ni kayotte imashita. \hfill\break
 }I was attended cram school while also doing club activities. }
 
\par{19. ${\overset{\textnormal{だいがく}}{\text{大学}}}$ に ${\overset{\textnormal{かよ}}{\text{通}}}$ いながら、 ${\overset{\textnormal{つき}}{\text{月}}}$ に ${\overset{\textnormal{ろくじゅう}}{\text{60}}}$ ${\overset{\textnormal{まんえん}}{\text{万円}}}$ くらい ${\overset{\textnormal{かせ}}{\text{稼}}}$ いでいました。 \hfill\break
 \emph{Daigaku ni kayoinagara, tsuki ni rokujūman\textquotesingle en kurai kaseide imashita. }\hfill\break
I was earning about 600,000 yen a month while attending college. }
 
\par{20. ${\overset{\textnormal{せんたく}}{\text{洗濯}}}$ が ${\overset{\textnormal{お}}{\text{終}}}$ わるのを ${\overset{\textnormal{ま}}{\text{待}}}$ ちながら、 ${\overset{\textnormal{あす}}{\text{明日}}}$ の ${\overset{\textnormal{じゅんび}}{\text{準備}}}$ をしています。 \hfill\break
 \emph{Sentaku ga owaru no wo machinagara, ashita no jumbi wo shite imasu. }\hfill\break
I\textquotesingle m doing preparations for tomorrow while waiting for the wash to finish. }
 
\par{21. ${\overset{\textnormal{じょうきゃく}}{\text{乗客}}}$ の ${\overset{\textnormal{まえ}}{\text{前}}}$ でカップ ${\overset{\textnormal{めん}}{\text{麺}}}$ を ${\overset{\textnormal{た}}{\text{食}}}$ べながら ${\overset{\textnormal{うんてん}}{\text{運転}}}$ する。 \hfill\break
 \emph{Jōkyaku no mae de kappumen wo tabenagara unten suru. \hfill\break
 }To drive while eating cup noodles in front of passengers. }
 
\par{22. ${\overset{\textnormal{りょうり}}{\text{料理}}}$ しながら、 ${\overset{\textnormal{どうじ}}{\text{同時}}}$ に ${\overset{\textnormal{かたづ}}{\text{片付}}}$ けも ${\overset{\textnormal{でき}}{\text{出来}}}$ るんですよ。 \hfill\break
 \emph{Ryōri shinagara, dōji ni katazuke mo dekiru n desu yo. }\hfill\break
It\textquotesingle s possible to clean up simultaneously while cooking, you know. }
 
\par{23. ${\overset{\textnormal{わたし}}{\text{私}}}$ は ${\overset{\textnormal{まいにち}}{\text{毎日}}}$ 、 ${\overset{\textnormal{としょかん}}{\text{図書館}}}$ で ${\overset{\textnormal{いろいろ}}{\text{色々}}}$ な ${\overset{\textnormal{しりょう}}{\text{資料}}}$ を ${\overset{\textnormal{しら}}{\text{調}}}$ べながら、 ${\overset{\textnormal{かんじ}}{\text{漢字}}}$ や ${\overset{\textnormal{ぶんぽう}}{\text{文法}}}$ など ${\overset{\textnormal{べんきょう}}{\text{勉強}}}$ しています。 \hfill\break
 \emph{Watashi wa mainichi, toshokan de iroiro na shiryō wo shirabenagara, kanji ya bumpō nado benkyō shite imasu. }\hfill\break
I\textquotesingle m studying Kanji and grammar in the library every day while examining various materials. }
 
\par{24. ${\overset{\textnormal{でんわ}}{\text{電話}}}$ をかけながらや、LINEなどのテキストメッセージを ${\overset{\textnormal{う}}{\text{打}}}$ ちながら ${\overset{\textnormal{ちゅうもん}}{\text{注文}}}$ をすることは ${\overset{\textnormal{てんいん}}{\text{店員}}}$ さんに ${\overset{\textnormal{たい}}{\text{対}}}$ して、あまりにも ${\overset{\textnormal{しつれい}}{\text{失礼}}}$ な ${\overset{\textnormal{こうい}}{\text{行為}}}$ なんですよ。 \hfill\break
 \emph{Denwa wo kakenagara ya, Rain nado no tekisuto messēji wo uchinagara chūmon wo suru koto wa ten\textquotesingle intachi ni tai shite, amari ni mo shitsurei na kōi na n desu yo. } \hfill\break
Ordering while being on the phone or typing text messages on LINE and such is beyond rude behavior toward employees. }
 
\par{\textbf{Grammar Note }: As this sentence demonstrates, \emph{nagara }ながら can interestingly accept certain particles like \emph{ya }や at times. This shows that it is somewhat noun-like. In this example, there is a need to refer to more than one ancillary verb phrase that is done with the same verb, and rather than repeating the same thing twice with slightly different context, just using \emph{nagara ya }ながらや makes everything easier. }
 
\begin{center}
\textbf{Instantaneous Verbs X } 
\end{center}

\par{ Using \emph{nagara }ながら with verbs that describe instant occurrences is incorrect. Therefore, it cannot be used with verbs like \emph{shinu }死ぬ, \emph{tsuku }つく, \emph{kieru }消える, etc. For verbs that take hardly any time at all to take place, if the main action takes more time than the secondary verb in the first clause with \emph{nagara }ながら, then the usage is incorrect. }
 
\par{25. ${\overset{\textnormal{すわ}}{\text{座}}}$ りながら ${\overset{\textnormal{はな}}{\text{話}}}$ しませんか。X \hfill\break
 \emph{Suwarinagara hanashimasen ka? }\hfill\break
 ${\overset{\textnormal{すわ}}{\text{座}}}$ って ${\overset{\textnormal{はな}}{\text{話}}}$ しませんか。○ \hfill\break
 \emph{Suwatte hanashimasen ka? }\hfill\break
How about we sit and talk? }

\begin{center}
\emph{\textbf{T }\textbf{e }}\textbf{て or \emph{Nagara }ながら? }
\end{center}

\par{ Sometimes, using \emph{nagara }ながら may resemble using the particle \emph{te }て since the verb that goes with \emph{nagara }ながら is a secondary action. However, ancillary actions described with \emph{te }て tend to be means by which to do something or condition by which things happen. This difference means that \emph{te }て and \emph{nagara }ながら can indeed coexist in the same sentence. }
 
\par{26. ${\overset{\textnormal{くるま}}{\text{車}}}$ での ${\overset{\textnormal{つうきんじ}}{\text{通勤時}}}$ に ${\overset{\textnormal{かなら}}{\text{必}}}$ ずテープを ${\overset{\textnormal{き}}{\text{聞}}}$ いて ${\overset{\textnormal{えいご}}{\text{英語}}}$ を ${\overset{\textnormal{べんきょう}}{\text{勉強}}}$ しています。 \hfill\break
 \emph{Kuruma de no tsūkinji ni kanarazu tēpu wo kiite eigo wo benkyō shite imasu. }\hfill\break
I\textquotesingle m studying English by always listening to tapes when I commute by car. }
 
\par{27. ${\overset{\textnormal{くるま}}{\text{車}}}$ で ${\overset{\textnormal{つうきん}}{\text{通勤}}}$ しているときは ${\overset{\textnormal{かなら}}{\text{必}}}$ ず ${\overset{\textnormal{えいかいわ}}{\text{英会話}}}$ のテープを ${\overset{\textnormal{き}}{\text{聴}}}$ きながら ${\overset{\textnormal{うんてん}}{\text{運転}}}$ します。 \hfill\break
 \emph{Kuruma de tsūkin shite iru toki wa kanarazu eikaiwa no tēpu wo kikinagara unten shite imasu. }\hfill\break
I\textquotesingle m driving while always listening to English conversation times when commuting by car. }
 
\par{28. ${\overset{\textnormal{ぶんか}}{\text{文化}}}$ も、 ${\overset{\textnormal{じだい}}{\text{時代}}}$ の ${\overset{\textnormal{なが}}{\text{流}}}$ れに ${\overset{\textnormal{あ}}{\text{合}}}$ わせて ${\overset{\textnormal{しんか}}{\text{進化}}}$ しながら ${\overset{\textnormal{か}}{\text{変}}}$ わっていかなければなりません。 \hfill\break
 \emph{Bunka mo, jidai no nagare ni awasete shinka shinagara kawatte ikanakereba narimasen. \hfill\break
 }Culture must also change as it evolves along with the trends of the times. }
 
\par{29. ${\overset{\textnormal{いぬ}}{\text{犬}}}$ の ${\overset{\textnormal{さんぽ}}{\text{散歩}}}$ をしながら ${\overset{\textnormal{ある}}{\text{歩}}}$ きタバコをする ${\overset{\textnormal{ひと}}{\text{人}}}$ は ${\overset{\textnormal{めいわく}}{\text{迷惑}}}$ だ。 \hfill\break
 \emph{Inu no sampo wo shinagara aruki-tabako wo suru hito wa meiwaku da. \hfill\break
 }People who smoke and walk while walking their dogs are bothers. }
 
\par{\textbf{Grammar Note }: Some \emph{nagara }ながら expressions refer to two actions done in tandem so much so that they result in set phrases without \emph{nagara }ながら. This explains \emph{aruki-tabako }歩きタバコ.  Other examples include aruki-sumaho 歩きスマホ (using one\textquotesingle s smart phone while walking), \emph{tabearuki }食べ歩き (eating while walking), etc. }
 
\begin{center}
\textbf{\emph{Nagara-zoku }ながら族 }
\end{center}
 
\par{\emph{ Nagara-zoku }ながら族 is a set phrase that refers to people who do all sorts of things while they\textquotesingle re supposed to be focusing on work or the like. As is the case in this sentence, it can be used as a nice context to string several \emph{nagara }ながら phrases together. }
 
\par{30. ながら ${\overset{\textnormal{ぞく}}{\text{族}}}$ って、 ${\overset{\textnormal{おおぜい}}{\text{大勢}}}$ いるんじゃないですか。 ${\overset{\textnormal{おれ}}{\text{俺}}}$ だって、 ${\overset{\textnormal{けいたいでんわ}}{\text{携帯電話}}}$ を ${\overset{\textnormal{み}}{\text{見}}}$ ながら、テレビを見ながら、お ${\overset{\textnormal{さけ}}{\text{酒}}}$ を ${\overset{\textnormal{の}}{\text{飲}}}$ みながら、お ${\overset{\textnormal{かし}}{\text{菓子}}}$ を ${\overset{\textnormal{た}}{\text{食}}}$ べながら、 ${\overset{\textnormal{あたま}}{\text{頭}}}$ を ${\overset{\textnormal{か}}{\text{掻}}}$ きながら、パソコンでRedditの ${\overset{\textnormal{しつもん}}{\text{質問}}}$ に ${\overset{\textnormal{かいとう}}{\text{回答}}}$ しています。 \hfill\break
 \emph{Nagara-zoku tte, ōzei iru n ja nai desu ka? Ore datte, keitai denwa wo minagara, terebi wo minagara,osake wo nominagara, okashi wo tabenagara, atama wo kakinagara, pasokon de reditto no shitsumon ni kaitō shite imasu. \hfill\break
}Aren\textquotesingle t there lots of people who do all sorts of stuff while they\textquotesingle re working? Like even me, I\textquotesingle m answering questions on Reddit on my computer while looking at my cellphone, while watching TV, while drinking liquor, while eating sweets, while scratching my head, etc. }
    