    
\chapter{Titles}

\begin{center}
\begin{Large}
第123課: Titles 
\end{Large}
\end{center}
 
\par{ One's title is important in society, especially Japanese society. Sometimes, it\textquotesingle s so important that it\textquotesingle s how you\textquotesingle re addressed. Last lesson, you were introduced to embellishments called ${\overset{\textnormal{こしょう}}{\text{呼称}}}$ that attach to people\textquotesingle s names and provide insight to their relation with you or whoever\textquotesingle s speaking. You were also introduced to the concept of 呼び ${\overset{\textnormal{す}}{\text{捨}}}$ て, which is not using a 呼称 at all, and when it\textquotesingle s okay to address people solely by their names (casual circumstances). }
      
\section{The World of Titles}
 
\par{ The most recognizable title in Japanese is 先生, an honorific word which literally means "one born earlier." 先生 refers to teachers, professionals, and leaders. Like a 呼称, it must not be used to refer to oneself. The reason for this is that in honorifics, you are supposed to be humble in speaking about yourself. }

\par{Titles may differ in wording depending on whether you are referring to someone else or oneself, but there is always a way to phrase someone\textquotesingle s title, putting aside who exactly one is talking about. Embellishing with  呼称 is a good way at bridging the gap between referring to one\textquotesingle s own occupation to referring to someone else; however, many occupations suffice alone as titles. This is because the large majority of occupation terms are neutral in honorifics, and using them in tangent with a surname is usually all you need to do. However, it\textquotesingle s important to study how individual words are used to get all this correct. }

\par{Now it\textquotesingle s time to expand your vocabulary by absorbing many of the common occupation words you may encounter. Some may function like 呼称 while others will simply function as standalone words. For instance, you may talk about your boss as 上司, but you\textquotesingle ll refer to him in person by his actual position. }

\begin{center}
\textbf{Title?\slash Occupation? }
\end{center}

\par{The best way to learn when to use what and how is not by staring at a list of words but by seeing them used in action. Word notes will be provided per example sentence to give you information about how titles are used and how they may be used outside the example itself. }

\par{1. 課長、 ${\overset{\textnormal{しんき}}{\text{新規}}}$ の見積(書)をご覧になりましたか。 \hfill\break
Chief, have you seen the new quote? }

\par{\textbf{Word Note }: The word 課長, meaning "(department) chief," is both a title and an occupation. }
 
\par{2. アメリカのトランプ ${\overset{\textnormal{じきだいとうりょう}}{\text{次期大統領}}}$ にメキシコ工場の ${\overset{\textnormal{けんせつけいかく}}{\text{建設計画}}}$ を ${\overset{\textnormal{ひはん}}{\text{批判}}}$ されているトヨタ自動車の ${\overset{\textnormal{とよだあきおしゃちょう}}{\text{豊田章男社長}}}$ は9日、アメリカのデトロイトで開かれているモーターショーに出席し、アメリカで今後5年の間に100 ${\overset{\textnormal{おく}}{\text{億}}}$ ドル( ${\overset{\textnormal{いっ}}{\text{1}}}$ ${\overset{\textnormal{ちょうえんいじょう}}{\text{兆円以上}}}$ )を ${\overset{\textnormal{とうし}}{\text{投資}}}$ する計画になっていることを明らかにしました。 \hfill\break
Akio Toyoda, the president of Toyota Motor Corporation, who has been criticized by President-elect (Donald) Trump for plans to construct a factory in Mexico, while attending an auto show being held in Detroit in America on the ninth, revealed that the company has made plans to invest over 10 billion dollars (over 1 trillion yen) into America in coming five years. }

\par{From NHK on 1\slash 9\slash 17. }
 
\par{\textbf{Word Note }: トランプ次期大統領 incorporates a “surname + occupation.” (次期)大統領 may be used as a standalone word. }
 
\par{\textbf{Word Note }: The company president of Toyota Motor Corporation is Akio Toyoda. He is referred to initially by the “full name of the company + full name + title.” Workers of his company would undoubtedly refer to him as 社長. Other mentions of him in the article would have abbreviated his full address to just “surname + title” (豊田社長). }

\par{3. ${\overset{\textnormal{やくしじ}}{\text{薬師寺}}}$ の ${\overset{\textnormal{むらかみたいいんかんしゅ}}{\text{村上太胤管主}}}$ は「 ${\overset{\textnormal{とうとう}}{\text{東塔}}}$ の ${\overset{\textnormal{さいけん}}{\text{再建}}}$ が ${\overset{\textnormal{じゅんちょう}}{\text{順調}}}$ に進んでいくものと心強く思っています」と話していました。 \hfill\break
Tai'in Murakami, the chief abbot of Yakushi-ji, spoke of the matter saying, “I\textquotesingle m strongly reassured that the reconstruction of the East Tower is to proceed steadily.” \hfill\break
From NHK on 1\slash 9\slash 17. }
 
\par{\textbf{Word Note }: 管主 refers to the chief abbot of a temple, and depending on the sect, it may alternatively be ${\overset{\textnormal{かんじゅ}}{\text{貫首}}}$ , ${\overset{\textnormal{ざす}}{\text{座主}}}$ , etc. }

\par{4. ${\overset{\textnormal{たいわん}}{\text{台湾}}}$ の ${\overset{\textnormal{さiえいぶんそうとう}}{\text{蔡英文総統}}}$ は、 ${\overset{\textnormal{がいこうかんけい}}{\text{外交関係}}}$ のある ${\overset{\textnormal{ちゅうべい}}{\text{中米}}}$ 4か国を ${\overset{\textnormal{ほうもん}}{\text{訪問}}}$ するのを前に8日、 ${\overset{\textnormal{けいゆち}}{\text{経由地}}}$ のアメリカ南部テキサス ${\overset{\textnormal{しゅう}}{\text{州}}}$ のヒューストンで、 ${\overset{\textnormal{じもとせんしゅつ}}{\text{地元選出}}}$ で ${\overset{\textnormal{わかて}}{\text{若手}}}$ の ${\overset{\textnormal{ゆうりょくせいじか}}{\text{有力政治家}}}$ の1人である ${\overset{\textnormal{きょうわとう}}{\text{共和党}}}$ のテッド・クルーズ上院議員と ${\overset{\textnormal{かいだん}}{\text{会談}}}$ しました。 \hfill\break
Taiwanese President Tsai Ing-wen on the eighth, before traveling to four Central American nations with which Taiwan has diplomatic relations, met with Republican Senator Ted Cruz, a young, local elected political star, at her transit point in Houston, which is located in the south of the U.S. \hfill\break
From NHK on 1\slash 9\slash 17. }
 
\par{\textbf{Word Note }: The president of some countries, Taiwan being an example, is called 総統. Senator in Japanese is “上院議員.” 議員 simply means “assemblyman” and can be used in a variety of government\slash bureaucracy related terminology. }
 
\par{5. おととし、 ${\overset{\textnormal{ちょうちょう}}{\text{町長}}}$ と数人の議員たちが ${\overset{\textnormal{たいほ}}{\text{逮捕}}}$ されるという ${\overset{\textnormal{ふしょうじ}}{\text{不祥事}}}$ が起きました。 \hfill\break
Two years ago, there was a scandal in which the town mayor and several assembly members were arrested. }
 
\par{\textbf{Word Note }: 町長 is both a title and an occupation. }
 
\par{6. これに ${\overset{\textnormal{かんれん}}{\text{関連}}}$ して ${\overset{\textnormal{ちゅうごくがいむしょう}}{\text{中国外務省}}}$ の ${\overset{\textnormal{りくこう}}{\text{陸慷}}}$ ${\overset{\textnormal{ほうどうかん}}{\text{報道官}}}$ は9日の ${\overset{\textnormal{きしゃかいけん}}{\text{記者会見}}}$ で \hfill\break
In the context of this, Press Secretary of Chinese Foreign Affairs, Lu Kang, in a press interview on the ninth… \hfill\break
From NHK on 1\slash 9\slash 17. }
 
\par{\textbf{Word Note }:  中国外務省の陸慷報道官 utilizes the occupation\slash title 報道官. ~官 is a suffix meaning “officer” that you\textquotesingle ll see in several other titles\slash occupations. }
 
\par{7. FIFAは9日、去年の ${\overset{\textnormal{ねんかんさいゆうしゅうせんしゅ}}{\text{年間最優秀選手}}}$ を発表し、男子は、ポルトガル ${\overset{\textnormal{だいひょう}}{\text{代表}}}$ のキャプテンで、スペイン ${\overset{\textnormal{いち}}{\text{1}}}$ ${\overset{\textnormal{ぶ}}{\text{部}}}$ リーグ、レアルマドリードのエース、クリスチアーノロナウド ${\overset{\textnormal{せんしゅ}}{\text{選手}}}$ が選ばれました。 \hfill\break
On the ninth, FIFA announced the Player of the Year of 2016, choosing Cristiano Ronaldo for man of the year, who is a leading player in Real Madrid, a soccer league of Spain, as a captain representing Portugal. \hfill\break
From NHK on 1\slash 10\slash 17. }
 
\par{\textbf{Word Note }: Many titles\slash occupations are mentioned here. Firstly, 代表 is used after some “group\slash organization\slash country name” to demarcate a representative (代表者). キャプテン is, clearly, a title for “captain (of a team),” and 選手 is used after the name of athletes. }

\par{8. ${\overset{\textnormal{あおもりやまだこうこう}}{\text{青森山田高校}}}$ の ${\overset{\textnormal{くろだごうかんとく}}{\text{黒田剛監督}}}$ は「去年3位に終わってから、リベンジしたいと頑張ってきた」と ${\overset{\textnormal{うれ}}{\text{嬉}}}$ しそうに話していました。 \hfill\break
Couch Gō Kuroda of Aomoriyamada High School spoke happily saying, “Since ending in third place last year, we\textquotesingle ve worked hard to get our revenge.” \hfill\break
From NHK on 1\slash 9\slash 17. }
 
\par{\textbf{Word Note }: 監督 is a title. Aside from meaning “coach,” it may also be used to mean director as in 映画監督. The occupation word for this is 監督者. For instance, 管理監督者 would mean “Management Superintendent.” }
 
\par{\textbf{Word Note }: ~者 is a suffix meaning “person,” but using it to refer to oneself is typically inappropriate. You should always rephrase it out in that case. Often times, ~人 is appropriate. Or, you could just not use a suffix after the occupation word in question. }
 
\par{9. 私はマンションの管理(人)をやっています。 \hfill\break
I manage apartments. }

\par{10. ${\overset{\textnormal{あべそうりだいじん}}{\text{安倍総理大臣}}}$ はサウジアラムコの ${\overset{\textnormal{とうしょう}}{\text{東証}}}$ への ${\overset{\textnormal{じょうじょう}}{\text{上場}}}$ を ${\overset{\textnormal{ちょくせついらい}}{\text{直接依頼}}}$ した。 \hfill\break
Prime Minister Abe directly requested for Saudi Aramco\textquotesingle s to be listed in the Tokyo Stock Exchange. \hfill\break
From NHK on 1\slash 6\slash 17. }
 
\par{\textbf{Word Note }: 総理大臣 is both a title and occupation. }

\par{11. ${\overset{\textnormal{きよた}}{\text{清田}}}$ CEOは ${\overset{\textnormal{しゅと}}{\text{首都}}}$ リヤドに ${\overset{\textnormal{とうちゃくご}}{\text{到着後}}}$ 、ようやく待ち望んだ ${\overset{\textnormal{ろうほう}}{\text{朗報}}}$ を受け取りました。 \hfill\break
CEO Kiyota at last received the good news he had been waiting for after arriving at Riyadh the capital. \hfill\break
From NHK on 1\slash 6\slash 17. }
 
\par{\textbf{Word Note }: CEO is translated into Japanese as ${\overset{\textnormal{さいこうけいえいせきにんしゃ}}{\text{最高経営責任者}}}$ . Although this word is also used, it is far more practical and common to see CEO used instead. }

\par{12. 清田CEOは、 ${\overset{\textnormal{さいだい}}{\text{最大}}}$ のキーマンと ${\overset{\textnormal{もく}}{\text{目}}}$ するムハンマド ${\overset{\textnormal{ふくこうたいし}}{\text{副皇太子}}}$ と30分間、サウジアラムコの ${\overset{\textnormal{かいちょう}}{\text{会長}}}$ を ${\overset{\textnormal{けんむ}}{\text{兼務}}}$ するファリハ・エネルギー ${\overset{\textnormal{さんぎょうこうぶつしげんしょう}}{\text{産業鉱物資源相}}}$ と ${\overset{\textnormal{いち}}{\text{1}}}$ ${\overset{\textnormal{じかんじゃく}}{\text{時間弱}}}$ 、それぞれ ${\overset{\textnormal{かいだん}}{\text{会談}}}$ し、東証の ${\overset{\textnormal{みりょく}}{\text{魅力}}}$ をアピールしました。CEO Kiyota respectively met with the Deputy Crown Prince Mohammed, who is also deemed the most key influential person, for thirty minutes as well as Al-Falih, the minister of the Ministry of Energy, Industry and Mineral Resources as well as the chairman Saudi Aramco, for a little less than an hour to appeal to the glamour of the Tokyo Stock Exchange. \hfill\break
From NHK on 1\slash 6\slash 17. }
 
\par{\textbf{Word Note }: 副~  is a prefix attached to titles\slash occupations meaning “vice-\slash deputy…” }
 
\par{\textbf{Word Note }: 皇太子 means “Crown Prince” and is both a title and an occupation. }
 
\par{\textbf{Word Note }: 会長 is a title and may mean “the president (of a society)” or “chairman (of the board of directors).” }
 
\par{\textbf{Word Note }: ~ ${\overset{\textnormal{しょう}}{\text{相}}}$ is a suffix meaning “minister” that appears at the end of the official name of a ministry. }
 
\par{13. サウジアラビアの ${\overset{\textnormal{じゅうようかくりょう}}{\text{重要閣僚}}}$ は、日本の ${\overset{\textnormal{おおてせきゆもとう}}{\text{大手石油元売}}}$ り ${\overset{\textnormal{がいしゃ}}{\text{会社}}}$ の ${\overset{\textnormal{しゅのう}}{\text{首脳}}}$ にも電話をかけ、 ${\overset{\textnormal{ほうあんてっかい}}{\text{法案撤回}}}$ への日本の協力を ${\overset{\textnormal{ようせい}}{\text{要請}}}$ しました。 \hfill\break
Key cabinet members of the Saudi Arabian cabinet even gave calls to the head leaders of major Japanese oil refiner-distributors seeking Japan\textquotesingle s support to overturn the law. \hfill\break
From NHK on 1\slash 6\slash 17. }
 
\par{\textbf{Word Note }: 閣僚 means “cabinet members” and is inherently plural. Together they form the cabinet ( ${\overset{\textnormal{ないかくかくりょう}}{\text{内閣閣僚}}}$ ). A member (構成員) of the cabinet is referred to as a ${\overset{\textnormal{こくむだいじん}}{\text{国務大臣}}}$ . In direct reference to being a cabinet member, you would use ${\overset{\textnormal{かくりょういいん}}{\text{閣僚委員}}}$ . 閣僚 is only an occupation, and in order to make it a title, you would need to use 閣僚委員. }
 
\par{\textbf{Word Note }: 首脳 is an occupation and may refer to a head of state or head leader(s) of an organization\slash company. To use it in reference to someone, you might see it used in a sentence like below. }
 
\par{14. 世界には ${\overset{\textnormal{いっこく}}{\text{一国}}}$ の政府首脳を ${\overset{\textnormal{つと}}{\text{務}}}$ める女性が ${\overset{\textnormal{いがい}}{\text{意外}}}$ とたくさんいます。 \hfill\break
There are surprisingly a lot of women who are heads of state in the world. }
 
\par{15. ムハンマド副皇太子は、中国で ${\overset{\textnormal{しゅうきんぺい}}{\text{習近平}}}$ ${\overset{\textnormal{こっかしゅせき}}{\text{国家主席}}}$ と ${\overset{\textnormal{かいだん}}{\text{会談}}}$ 、ロシアではプーチン大統領と会談するなど、 ${\overset{\textnormal{せいりょくてき}}{\text{精力的}}}$ に世界を飛び回っています。 \hfill\break
Deputy Crown Prince Mohammed energetically flew around the world meeting with President Xi Jinping of China, President Putin of Russia, etc. \hfill\break
From NHK on 1\slash 6\slash 17. }
 
\par{\textbf{Word Note }: The word for “president” in reference to China is 国家主席. Other words for president yet to be mentioned in this lesson include 頭取 for president of a bank, 理事長 meaning “director,” and 総裁, which is the president of a major organization such as the Bank of Japan, hospitals, etc. }

\par{16. ${\overset{\textnormal{らくせいしき}}{\text{落成式}}}$ では ${\overset{\textnormal{あべひでおしちょう}}{\text{阿部秀保市長}}}$ が「6年生が卒業する前に、新しい ${\overset{\textnormal{こうしゃ}}{\text{校舎}}}$ で学んでもらおうと ${\overset{\textnormal{かんせい}}{\text{完成}}}$ を目指し、努力してきました。この校舎は ${\overset{\textnormal{ふっこう}}{\text{復興}}}$ のシンボルです」とあいさつしました。 \hfill\break
City Mayor Hideo Abe greeted (attendees) at the completion ceremony saying, "We have worked hard aiming to complete this new schoolhouse for our sixth graders to learn in before graduating (primary school).  This school building is a symbol of restoration.” \hfill\break
From NHK on 1\slash 9\slash 17. }
 
\par{\textbf{Word Note }: 市長, meaning “city mayor,” is both a title and an occupation. }
 
\par{17. ${\overset{\textnormal{みやのもりしょうがっこう}}{\text{宮野森小学校}}}$ の ${\overset{\textnormal{あいざわひでおこうちょう}}{\text{相澤日出夫校長}}}$ は「すばらしい校舎が完成し、感謝しています。 ${\overset{\textnormal{ちいき}}{\text{地域}}}$ の復興に向けて、いい学びの ${\overset{\textnormal{ば}}{\text{場}}}$ にしていきたい」と話していました。 \hfill\break
Principal Hideo Aizawa of Miyanomori Elementary School spoke saying, “We are greatly thankful that this wonderful schoolhouse has been completed. We look forward to having this be a great learning place as we head towards restoring the region.” \hfill\break
From NHK on 1\slash 9\slash 17. }
 
\par{\textbf{Word Note }: 校長, meaning “principal (of a school),” is both a title and an occupation. }
 
\par{18. 私は20年 ${\overset{\textnormal{きょうし}}{\text{教師}}}$ をしています。 \hfill\break
I\textquotesingle ve been a teacher for twenty years. }
 
\par{\textbf{Word Note }: 教師 is the occupation of “teaching.” Teachers are referred to as 先生. However, one must never refer to oneself as such. }
 
\par{19. オーキド ${\overset{\textnormal{はかせ}}{\text{博士}}}$ からポケモンを ${\overset{\textnormal{もら}}{\text{貰}}}$ えるって聞いたけど、本当? \hfill\break
I heard that you can get a Pokemon from Professor Oak, but is that true? }
 
\par{\textbf{Word Note }: 博士 in professional words like Doctorate Degree (博士号) it is pronounced as はくし, but in general use, it is usually pronounced as はかせ. }
 
\par{\textbf{Final Notes }}
 
\par{When an occupation is used as a title in conjunction with someone\textquotesingle s name, a 呼称 is not necessary. Whenever you are calling someone by his or her occupation, a 呼称 is often necessary. For instance, a store clerk is a 店員, but you\textquotesingle ll need to refer to him\slash her as 店員さん. In the same token, your doctor is an 医者, but you refer to him\slash her as お医者さん. }
 
\par{Not all occupations can or ought to be used as titles. This means “occupation +さん” is not a fix all solution, but ~さん・様 will still likely be needed. Your lawyer is a ${\overset{\textnormal{べんごし}}{\text{弁護士}}}$ . However, you won\textquotesingle t call him\slash her in person as 弁護士さん. You may refer to him\slash her as such in writing, but in person, you\textquotesingle d refer to him\slash her by surname +さん. }
 
\par{Some occupations sound quite wordy and technical if used in the spoken language. In such cases, different phrases may be used altogether. For instance, ${\overset{\textnormal{けいさつかん}}{\text{警察官}}}$ means “police officer.” You may see people refer to police officers as 警察官さん(たち), but in conversation, you\textquotesingle ll hear the word お ${\overset{\textnormal{まわ}}{\text{巡}}}$ りさん to prevent the awkwardness of using such a wordy title. }
  
\par{\textbf{先輩 }\textbf{vs. }\textbf{後輩 }}
 
\par{Lastly, to conclude this lesson, we will study the well-known words 先輩 and 後輩, which have gained worldwide attention due to their overuse in anime and manga. Most people outside Japan, however, don\textquotesingle t really understand how they\textquotesingle re used. }
 
\par{If I were your boss, I wouldn't be your 先輩. I would be your 上司. Even then, you would need to call me by the appropriate title. You don't call your teacher 先輩 either. Even if your 先生 happened to be younger than you, you would not call him\slash him your 後輩. }
 
\par{先輩 means "senior" as in being ahead in rank. In school everyone in grades above you is your 先輩. In reverse, you are their 後輩. Everyone in the school, though, is a 学生. 先輩 and 後輩 can be used like 呼称 and as stand-alone words. This is also the case for words like 先生. Teacher and student are complete opposites, and the teacher has a position that should be respected. Thus, you should use neither 先輩 or 後輩 in speaking to him\slash her. }
 
\par{Whether or not you can refer to athletes or what not as your 先輩 or 後輩 is dictated by the social circumstances at hand. Are you part of the team? Are you an extremely obsessed female fan that adores a member and affectionately refers to him as 先輩? There is a lot to keep in mind, but it is safe to say that if you find a usage in the wild that fits your situation, it is probably safe to use it likewise, assuming that you don't solely read odd manga series. }
    