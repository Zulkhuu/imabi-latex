    
\chapter{Nominalization}

\begin{center}
\begin{Large}
第102課: Nominalization 
\end{Large}
\end{center}
 
\par{  \emph{Nominalization is making \textbf{another part of speech-- }verbs, adjectives, phrases, abstract nouns, etc.-- \textbf{noun-like }}. Nominalization is mainly done with the help of ${\overset{\textnormal{けいしきめいし}}{\text{形式名詞}}}$ (nominal\slash dummy nouns). }
      
\section{The Nominalizers の and こと}
 
\par{ The nominalizers の and こと are both used to nominalize phrases to make them noun-like \emph{to an extent }. At the most basic understanding, the main difference between the two is that the former is more subjective and the latter is more objective. This small yet important difference places a major role in how they are chosen and how they fit with other expressions. }

\par{ An objective perspective is one that is not influenced by personal emotions whereas a subject perspective is one that does involve emotion. こと literally means "matter\slash circumstance," and so it is used in grammatical situations where one needs to nominalize verbal\slash adjectival expressions and concretely talk about them as entities. For instance, if you want to talk about "new developments" or "dancing," こと would appear. }

\par{ However, a more subjective circumstance will most likely call for の. It is a dummy noun, meaning it doesn't have all the freedoms of a true noun. That's why we see it in grammar patterns like のだ. When there is no need to portray an objective stance, の will also be the appropriate choice. }

\par{\textbf{Orthography Note }: こと is frequently spelled as 事. }

\begin{center}
\textbf{Examples of こと \& の } 
\end{center}

\par{1. 見ることは信ずることである。 \hfill\break
Seeing is believing. }

\par{\textbf{Grammar Note }: Whereas の is a dummy noun, こと is a true noun. Although Ex. 1 is a rather subjective thing to say, there is a sense of certainty that can be felt with the use of こと. However, at a more basic understanding of grammar,the use of こと is also necessary to set up the grammatical parallelism and comparison being made between "seeing" and "believing." These entities need to be expressed with true nouns, which is something that only こと can make possible. }

\par{2. お二人が会うのは初めてですか。 \hfill\break
Is this the first time you two have met? }

\par{3. カナダが寒いことに気づいた。(Objective) \hfill\break
I just noticed that Canada is cold. }

\par{4. カナダがペンギンも生きれるほど寒いのに気づいたばっかりだわ。(Feminine) \hfill\break
I just realized that Canada is cold enough for penguins to even live. }

\par{  Objectivity\slash subjectivity, although often a part of the equation, may not be the most important thing at work. In the following sentences, の just acts as a dummy noun for something else, but こと is more concrete, referring specifically to a situation. However, one can say that this is related to subjectivity being related to abstractness and objectivity being related to concreteness. }

\par{5. どんなのを聞きましたか。 \hfill\break
What sort of thing did you listen to? \hfill\break
}

\par{6. どんなことを聞きましたか。 \hfill\break
What sort of thing did you hear\slash ask? }

\par{ In the first sentence の replaces some noun such as 曲. In the second, こと refers to a situation (状態). The response could be something like the following. }

\par{7. 彼が京都に行った\{と・か\}きました。 \hfill\break
I heard\slash asked that\slash whether he went to Kyoto. }

\begin{center}
\textbf{Noun + のこと }  \hfill\break

\end{center}

\par{ At a basic understanding, there are two different yet intertwined functions of こと. The first is to focus on a certain action or state as the object of attention. The other is to expand the scope of something to encompass the situation surrounding it. In this latter sense, it abstracts. At face value, this seems like the opposite of focusing. However, this isn't really the case. }

\par{ When こと attaches itself to verbs and adjectives, which in both cases could involve an entire sentence, こと packs up these expressions into a noun phrase, making it refer to the action\slash state at hand. This is the focusing aspect. When it follows a noun, however, it's following something that is already concrete to some extent. By using こと, which literally means "matter\slash circumstance," you're no longer literally talking about the physical entity at hand. Rather, you're talking about its essence, which is truly the matter at hand. }

\par{8. あの事件のことをよく覚えています。 \hfill\break
I remember that incident well. }

\par{\textbf{Sentence Note }: In this example, the speaker remembers a lot about the incident. If のこと were omitted, the speaker would still remember that the incident happened, but it wouldn't necessarily sound that he\slash she knows much about the incident. }

\par{9. ピザのことをピッツアーと言う人がいます。 \hfill\break
There are some people who pronounce "piza" as "pittsā." }

\par{10. この文は現在のことを言っています。 \hfill\break
This sentence is talking about the present. }

\par{11. 誰の事を指しているんですか。 \hfill\break
Who are you referring to? }

\par{12. 私たちはまだまだこの大きな地球のことを理解していないのです。 \hfill\break
We still do not understand this great planet of ours. }

\par{13. 自分のことをどのくらい知っていますか。 \hfill\break
How much do you know about yourself? }

\par{14. 僕のことがすき? \hfill\break
Do you like me? }

\par{\textbf{Sentence Note }: In this example, the speaker is asking the listener not just that the listener likes "him" in the mere sense of physical attraction. The speaker is asking if that individual likes him at a personal level as well. In the case of 好きだ and 嫌いだ, the use of のこと also helps solve semantic ambiguity between two possible interpretations. If the sentence were just 僕が好き?, it would normally be interpreted as "do you like me?" but it could also mean "Do I like\dothyp{}\dothyp{}\dothyp{}?" depending on the context. The use of のこと, thus, helps bring clarity to what is the object at hand. }

\begin{center}
\textbf{More Examples }
\end{center}
 
\par{${\overset{\textnormal{}}{\text{14. 彼}}}$ が ${\overset{\textnormal{ゆうめい}}{\text{有名}}}$ な ${\overset{\textnormal{おんがくか}}{\text{音楽家}}}$ だということを ${\overset{\textnormal{}}{\text{知}}}$ りませんでした。 \hfill\break
I didn't know that he is a famous musician. \hfill\break
 \hfill\break
15. あたし、そんなことはいわんかったわ。(Feminine; dialectical) \hfill\break
I didn't say anything like that! }
 
\par{16. もう ${\overset{\textnormal{}}{\text{一}}}$ つ ${\overset{\textnormal{}}{\text{君}}}$ に ${\overset{\textnormal{たず}}{\text{尋}}}$ ねたいことがある。  (Male) \hfill\break
There's another thing I'd like to ask you about. }
 
\par{\textbf{Speech Note }: To a teacher say something more respectful like ${\overset{\textnormal{}}{\text{先生、}}}$ もう ${\overset{\textnormal{}}{\text{一}}}$ つ ${\overset{\textnormal{うかが}}{\text{伺}}}$ いたいことがあるんですけれども(よろしいですか)。 }
 
\par{17. (あなたが) ${\overset{\textnormal{}}{\text{見}}}$ たことを ${\overset{\textnormal{}}{\text{話}}}$ してくださいませんか。 \hfill\break
Could you tell me what you saw? }

\par{18. ${\overset{\textnormal{こいびと}}{\text{恋人}}}$ たちは ${\overset{\textnormal{いっしょ}}{\text{一緒}}}$ にサンバを ${\overset{\textnormal{おど}}{\text{踊}}}$ るのをやめました。 \hfill\break
The couple stopped dancing the samba together. }

\par{19. ${\overset{\textnormal{まいにちさんぽ}}{\text{毎日散歩}}}$ するのは ${\overset{\textnormal{けんこう}}{\text{健康}}}$ にとてもいいようです。 \hfill\break
It seems that going for a walk every day is very good for your health. }
 
\par{${\overset{\textnormal{}}{\text{20. 彼女}}}$ が ${\overset{\textnormal{さくやおそ}}{\text{昨夜遅}}}$ く ${\overset{\textnormal{}}{\text{帰}}}$ ってきたことを ${\overset{\textnormal{}}{\text{知}}}$ ってた。 \hfill\break
I've known that she came home late last night. }
 
\par{${\overset{\textnormal{}}{\text{21. 彼女}}}$ のいうことを ${\overset{\textnormal{ま}}{\text{真}}}$ に ${\overset{\textnormal{う}}{\text{受}}}$ けたな。 \hfill\break
You took her at her word, didn't you? }

\par{22. 何か ${\overset{\textnormal{しんぱい}}{\text{心配}}}$ なことがあるんですか。 \hfill\break
Is anything bothering you? }

\par{23. ${\overset{\textnormal{しんせん}}{\text{新鮮}}}$ な ${\overset{\textnormal{いき}}{\text{息}}}$ を ${\overset{\textnormal{す}}{\text{吸}}}$ うことは ${\overset{\textnormal{すば}}{\text{素晴}}}$ らしい。 \hfill\break
Breathing fresh air is wonderful. }
 
\par{24. これからのことを ${\overset{\textnormal{}}{\text{考}}}$ えるのは ${\overset{\textnormal{かしこ}}{\text{賢}}}$ いですね。 \hfill\break
It sure is wise to think about the future? }

\par{25. ${\overset{\textnormal{ようぎしゃ}}{\text{容疑者}}}$ がその ${\overset{\textnormal{}}{\text{部屋}}}$ から ${\overset{\textnormal{あわ}}{\text{慌}}}$ てて ${\overset{\textnormal{}}{\text{出}}}$ てくるのを ${\overset{\textnormal{}}{\text{見}}}$ ましたよ。 \hfill\break
I saw the suspect come out flustered from the room. }
 
\par{${\overset{\textnormal{}}{\text{26. 宿題}}}$ をしないで ${\overset{\textnormal{じゅぎょう}}{\text{授業}}}$ に ${\overset{\textnormal{}}{\text{来}}}$ るのはよくないです。 \hfill\break
It's not good to come to class without having done our homework. }

\par{27. ${\overset{\textnormal{う}}{\text{生}}}$ まれたのも育ったのもニューヨークです。 \hfill\break
I was born and raised in New York. }
 
\par{\textbf{Usage Note }: It may be the case that の is used in place of a thing, person, or place. In this case の refers to the city that the speaker was raised in. }
 
\par{28. 彼女の電話番号を調べるのに時間がかかった。 \hfill\break
It took a lot of time to find her phone number. }
 
\par{29. 歌うことは ${\overset{\textnormal{あきら}}{\text{諦}}}$ めていただけに ${\overset{\textnormal{ぐうぜん}}{\text{偶然}}}$ (に) ${\overset{\textnormal{はつぶたい}}{\text{初舞台}}}$ を ${\overset{\textnormal{ふ}}{\text{踏}}}$ めたことはとても ${\overset{\textnormal{うれ}}{\text{嬉}}}$ しかったです。 \hfill\break
Since I had given up singing, being able to debut by chance was very delightful. }

\par{30. ${\overset{\textnormal{きょういく}}{\text{教育}}}$ を受けるのは ${\overset{\textnormal{じゅうよう}}{\text{重要}}}$ です。 \hfill\break
Receiving an education is important. }

\par{31. ${\overset{\textnormal{けいさつ}}{\text{警察}}}$ が今ここにあることは ${\overset{\textnormal{うたが}}{\text{疑}}}$ えない。 \hfill\break
You can't doubt that there are police here now. }
32. 日本語を話せる人の間では ${\overset{\textnormal{じょうしき}}{\text{常識}}}$ のことだ。 \hfill\break
It's common sense among Japanese speakers. 
\par{33. 日本が経済大国になれたのは、何と言ってもアメリカのおかげではないと思います。 \hfill\break
No matter what they say, I don't think that Japan was able to become an economic power thanks to America. }

\par{34. 単純労働でお金を稼ぐのは大変だ。 \hfill\break
It is difficult to earn money by simple labor. }

\par{35. 生まれたのは中国だが、国籍はインドネシアだ。 \hfill\break
I was born in China, but my nationality is Indonesian. }

\par{36. ${\overset{\textnormal{あき}}{\text{明}}}$ らかに ${\overset{\textnormal{じょうほう}}{\text{情報}}}$ をもう一度目を ${\overset{\textnormal{とお}}{\text{通}}}$ すのは ${\overset{\textnormal{とうぜん}}{\text{当然}}}$ ですよ。 \hfill\break
It's evident that we clearly look over the information once more. }
 
\par{\textbf{Phrase Note }: 目を ${\overset{\textnormal{とお}}{\text{通}}}$ す = To look\slash scan over. }

\par{お ${\overset{\textnormal{}}{\text{二人}}}$ が ${\overset{\textnormal{}}{\text{会}}}$ うのは、 ${\overset{\textnormal{}}{\text{初}}}$ めてですか。 \hfill\break
Is this the first time that you two have met? }

\par{\textbf{Honorifics Note }: お- makes the sentence more polite to the addresses. }

\par{お ${\overset{\textnormal{}}{\text{二人}}}$ が ${\overset{\textnormal{}}{\text{会}}}$ うのは、 ${\overset{\textnormal{}}{\text{初}}}$ めてですか。 \hfill\break
Is this the first time that you two have met? }

\par{\textbf{Honorifics Note }: お- makes the sentence more polite to the addresses. }
      
\section{The Nominalizer もの}
 
\par{ もの nominalizes something to show that something is undoubtedly true. }

\par{37. ${\overset{\textnormal{あぶら}}{\text{油}}}$ は ${\overset{\textnormal{}}{\text{水}}}$ に ${\overset{\textnormal{う}}{\text{浮}}}$ くものだ。 \hfill\break
Oil floats on water. }
 
\par{${\overset{\textnormal{}}{\text{38. 子犬}}}$ になりたいものですわ!(Feminine) \hfill\break
I want to become a puppy! }
 
\par{39. よく ${\overset{\textnormal{}}{\text{金曜日}}}$ に ${\overset{\textnormal{}}{\text{海}}}$ で ${\overset{\textnormal{}}{\text{泳}}}$ いだものだね。 \hfill\break
I used to swim in the sea on Fridays, you know. }
 
\par{\textbf{Phrase Note }: よく…たものだ = Used to. \textbf{\hfill\break
Contraction Note }: もの may be seen colloquially as もん. \textbf{\hfill\break
Origin Note }: もの comes from the noun 物. }
 
\par{\textbf{もの・物・者 }}
 
\par{もの can be any \textbf{thing }. This もの can be in particle constructions like above. 物 is a tangible and perhaps living thing. It can even be a spiritual force. It's in many set expressions. It can even be used as a prefix. 者 refers to a person humbly or in a condescending manner. }

\par{40. ${\overset{\textnormal{もの}}{\text{物}}}$ の ${\overset{\textnormal{け}}{\text{怪}}}$ にとりつかれる。 \hfill\break
To be possessed by an evil spirit. }
 
\par{${\overset{\textnormal{}}{\text{41. 物}}}$ も ${\overset{\textnormal{}}{\text{言}}}$ いようで ${\overset{\textnormal{かど}}{\text{角}}}$ が ${\overset{\textnormal{}}{\text{立}}}$ つ。 \hfill\break
People may be offended by the way you speak. }
 
\par{${\overset{\textnormal{}}{\text{42. 物}}}$ の数分もしないうちに \hfill\break
In no more than a few minutes. }

\par{43. ${\overset{\textnormal{ものい}}{\text{物言}}}$ えば ${\overset{\textnormal{くちびる}}{\text{唇}}}$ ${\overset{\textnormal{さむ}}{\text{寒}}}$ し。(Proverb) \hfill\break
Least said, soonest mended. }
 
\par{${\overset{\textnormal{}}{\text{44. 早}}}$ い ${\overset{\textnormal{もの}}{\text{者}}}$ ${\overset{\textnormal{が}}{\text{勝}}}$ ち。 \hfill\break
First come, first served. }
    