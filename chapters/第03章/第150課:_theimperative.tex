    
\chapter{The Imperative}

\begin{center}
\begin{Large}
第150課: The Imperative  
\end{Large}
\end{center}
 
\par{ The imperative form of a verb or adjective is the 命令形. However, there are other patterns that make commands. }

\par{\textbf{Part of Speech Note }: Adjectives are usually used with "連用形 + しろ" instead. }
      
\section{命令形}
 
\begin{ltabulary}{|P|P|P|P|P|P|P|P|P|P|P|}
\hline 

 一段 & 五段 & する & 来る & くれる & いらっしゃる & 下さる & 仰る & なさる & 形容詞 & 形容動詞 \\ \cline{1-11}

-ろ・よ & -え & しろ・せい・せよ & 来い & くれ(ろ・よ) & いらっしゃい & 下さい & 仰い & なさい & -かれ & -であれ \\ \cline{1-11}

\end{ltabulary}

\par{\textbf{Usage Notes }: }

\par{1. ~よ is the original 命令形, but ~ろ has been used in East Japan, at least, for centuries and has since supplanted the original form as the standard variant. Now, the original form gives a nostalgic feel when used today. }

\par{2. As for する, \textbf{ }しろ is typical, and せい (mostly used by higher up and older people) and せよ (which is quite formal) also give a more nostalgic feel. }

\par{3. The 命令形 of くれる is typically seen as くれ. ~てくれ can make a \textbf{vulgar command }. }

\par{4. いらっしゃる, 下さる, 仰る are \textbf{honorific verbs }. Their 命令形 are simplified, but in old-style speech and in some regions of Japan, you may still hear the original, non-contracted 命令形, which would be いらっしゃれ, くだされ, and おっしゃれ respectively. }

\par{5. Again, adjectives are \textbf{rarely }used in the 命令形, but there are some common set expressions where adjectives are in it. In this sense, it is bested to view them as set phrases. }

\par{1. 善かれ悪しかれ \hfill\break
Right or wrong }

\par{\textbf{Form Note }: あしかれ is the 命令形 of the Classical Japanese 形容詞 悪し which has fallen out of use. }

\par{2. よかれと思って \hfill\break
All for the best }

\par{6. であれ is often used as in the following. If it were used to make a command, it would be somewhat old-fashioned to say the least. }

\par{3. ${\overset{\textnormal{りゆう}}{\text{理由}}}$ が ${\overset{\textnormal{なん}}{\text{何}}}$ であれ \hfill\break
Whatever the reason may be }
      
\section{The Imperative}
 
\par{Respectful }

\par{The grammatically correct form of the respectful command style is formed by adding an honorific prefix to the stem of a verb, then using the honorific verb なさる, to do, adding ~ます in its 命令形, ませ. ~下さい(ませ) may also be used as such, thus implying favor. }

\begin{ltabulary}{|P|}
\hline 

お+連用形 +なさいませ \\

お+連用形 +下さい(ませ) \\

\end{ltabulary}

\par{Even in formal situations, the likelihood of you having to ask a superior to do something is unlikely. Note that for する verbs,these items are directly attached. So, ご受験くださいませ = "please take your honorable test". }

\par{Polite  }

\par{The most common way to make the polite imperative is ~てください. To be less frank, you can use 下さいませんか. You may also attach ~なさい to the 連用形 of a verb. However, this usage is only used by a superior to an inferior. }

\begin{ltabulary}{|P|}
\hline 

連用形+-て+ください(ませんか) \\

連用形+-なさい \\

\end{ltabulary}

\par{\textbf{Etymology\slash Speech Style Note }: ~なさい comes from ~なされ, and ~なされ can still be used by older people. This can then be contracted to ~な in casual speech. }

\par{\textbf{Orthography Note }: ください is generally only in ひらがな when used as a supplementary verb in this fashion. }

\par{\textbf{Base Constraint Note }: One would think that you should be able to just use -ませ to make a polite imperative. However, this is highly restricted to set phrases like お帰りなさいませ (an extremely honorific way of saying welcome home), いらっしゃいませ (welcome!). To supplement this, ~なさい is used. }

\par{Plain \& Vulgar }

\par{Although dependent on tone, the following patterns listed below from least to most vulgar may be used to tell someone to do something in plain speech. }

\begin{ltabulary}{|P|}
\hline 

て形(+語尾) \\

命令形(+語尾) \\

連用形+-たまえ \\

連用形+-て+くれ \\

\end{ltabulary}

\par{\textbf{Usage Note }: ~たまえ is used by mainly older men and it is often demanding and taunting. }

\par{\textbf{Intonation Note }: You can also use the non-past form with a high intonation and force to make a command. Ex. さっさと食べる! (East fast!) }

\par{ The 命令形 can be used when commanding training\slash exercise sessions. It is also used heavily among men, especially by superiors to underlings. It is also used when cheering for sporting events. It is also used when you have to be brief like during in an emergency and in traffic signs. All these situations apply to the negative imperative as well. }

\par{ \textbf{Examples }}

\par{${\overset{\textnormal{}}{\text{1. 早}}}$ く ${\overset{\textnormal{}}{\text{泳}}}$ いでくれ。 \hfill\break
Swim faster! }
 
\par{3. このボタンを ${\overset{\textnormal{お}}{\text{押}}}$ せと ${\overset{\textnormal{}}{\text{説明書}}}$ に ${\overset{\textnormal{}}{\text{書}}}$ いてある。 \hfill\break
It says to press this button in the instruction manual. }
 
\par{${\overset{\textnormal{}}{\text{4. 父}}}$ が ${\overset{\textnormal{}}{\text{私}}}$ に ${\overset{\textnormal{}}{\text{勉強}}}$ しろと ${\overset{\textnormal{}}{\text{言}}}$ いました。 \hfill\break
My father told me to study. }

\par{${\overset{\textnormal{}}{\text{5. 暑}}}$ いのなら ${\overset{\textnormal{うわぎ}}{\text{上着}}}$ を ${\overset{\textnormal{ぬ}}{\text{脱}}}$ ぎなさい。 \hfill\break
If you're hot, then take off your coat. }
 
\par{${\overset{\textnormal{}}{\text{6. 早}}}$ く ${\overset{\textnormal{}}{\text{食}}}$ べなさい。 \hfill\break
Eat quickly. }

\par{${\overset{\textnormal{}}{\text{7. 息}}}$ を ${\overset{\textnormal{}}{\text{吸}}}$ って、 ${\overset{\textnormal{は}}{\text{吐}}}$ いて! \hfill\break
Inhale! Exhale! }
 
\par{${\overset{\textnormal{}}{\text{8. 名前}}}$ を ${\overset{\textnormal{}}{\text{書}}}$ きなさい。 \hfill\break
Write your name. }
 
\par{${\overset{\textnormal{}}{\text{9. 安心}}}$ しなよ。 \hfill\break
Calm down. }
 
\par{10. この ${\overset{\textnormal{こづつみ}}{\text{小包}}}$ を ${\overset{\textnormal{ふなびん}}{\text{船便}}}$ で ${\overset{\textnormal{}}{\text{送}}}$ ってくださいね。 \hfill\break
Please send this package by surface mail, k? }
 
\par{${\overset{\textnormal{}}{\text{11. 11時}}}$ までに ${\overset{\textnormal{}}{\text{帰}}}$ りなさい。 \hfill\break
Come back by 11 o'clock. }

\par{12. 急ぐなら新幹線で行きなさい。 \hfill\break
If you're in a hurry, take the bullet train. }

\par{13. もしも空を ${\overset{\textnormal{と}}{\text{飛}}}$ べるなら、 ${\overset{\textnormal{がけ}}{\text{崖}}}$ から飛んでいけ! \hfill\break
If you can fly, go fly off the cliff! }
 
\par{14. いらっしゃい!(Especially at home) \hfill\break
Welcome! }

\par{15. いらっしゃいませ。(Especially in a store) \hfill\break
Welcome! }
 
\par{${\overset{\textnormal{}}{\text{16. 入学}}}$ の ${\overset{\textnormal{}}{\text{前}}}$ 、コンピューターを ${\overset{\textnormal{}}{\text{買}}}$ ってください。 \hfill\break
Before you enter college, please buy a computer. }
 
\par{17. テレビを ${\overset{\textnormal{}}{\text{左側}}}$ に ${\overset{\textnormal{}}{\text{置}}}$ いてください。 \hfill\break
Please place the TV to the left side. }
 
\par{18. ご ${\overset{\textnormal{}}{\text{家族}}}$ の ${\overset{\textnormal{みなさま}}{\text{皆様}}}$ によろしくお ${\overset{\textnormal{}}{\text{伝}}}$ え ${\overset{\textnormal{}}{\text{下}}}$ さい。 \hfill\break
Please give my best wishes to your family. }

\par{19. ${\overset{\textnormal{まん}}{\text{満}}}$ タンにしてください。オイルもチェックしてください。クレジットカードで ${\overset{\textnormal{}}{\text{払}}}$ えますか。 \hfill\break
Please fill it up. Also, please check the oil. Can I pay by credit card? }

\par{20. ${\overset{\textnormal{と}}{\text{止}}}$ まれ! \hfill\break
Stop! }

\par{21. どうぞお ${\overset{\textnormal{}}{\text{入}}}$ りください。 \hfill\break
Please come in. }
 
\par{${\overset{\textnormal{}}{\text{22. 早}}}$ く ${\overset{\textnormal{}}{\text{寝}}}$ ろ! \hfill\break
Go to bed now! }

\par{${\overset{\textnormal{}}{\text{23a. 名前}}}$ だけ(を) ${\overset{\textnormal{おおもじ}}{\text{大文字}}}$ で書いてください。 \hfill\break
23b. 名前だけ(を)大文字で書きなさい。 \hfill\break
Please write only your name in capital letters. }
 
\par{\textbf{Sentence Note }: 23b is likely to be used by a superior to an inferior or on instructions on something. }

\par{24. しかし、 \textbf{飲んだくれて }歌舞伎町で寝ている織江はあなたではない。 \hfill\break
However, Orie, who was dead drunk sleeping at Kabukichou, is not you. \hfill\break
From the commentary by 池田清彦 for 冷たい誘惑 by 乃南アサ. }

\par{\textbf{Word Note }: 飲んだくれる is a verb inspired by the slurring of 飲んでくれ, a slang word used ever since the Edo Period to describe drunkards that drink all day. It's no doubt the word was inspired by the slurred speech of people drunk, and as it is based off of an imperative phrase, it is mentioned here. }

\par{25. あとを黒めてたもれ。  (古語的) \hfill\break
Smooth over the rest. \hfill\break
From 髭櫓 by 虎明狂. }

\par{\textbf{Phrase Note }: This sentence has the interesting phrase たもれ. This is from a sound change of \{賜・給\}れ. So, it is related to たまえ. This was used by elite people to lower individuals. Contexts are those in which the speaker makes it known that there won't be any dissent in doing so. }
      
\section{The Negative Imperative}
 
\par{ The negative imperative, which is telling someone not to do something, is very similar to the imperative. }

\par{Respectful }

\par{The way to create the respectful negative imperative is by adding an honorific prefix to the stem of a verb and adding なさいますな・下さいますな. Like the respectful imperative, the respectful negative imperative is reserved to most becoming situations. }

\par{26. 私のことは御心配下さいますな。 \hfill\break
Please do not worry about me. }

\par{\textbf{Usage Note }: Though more honorific and polite in nature, ~ないで下さい is far more common even in really formal situations. However, there are still situations where respectful negative commands may be expected such as in letters and situations where highly sophisticated and proper 敬語 is expected. }

\begin{center}
 Polite 
\end{center}

\par{Just like the polite imperative, there are two ways to show the polite negative imperative. The first way is by adding ~ないで to the 連用形 and then ending with 下さい(ませんか). The second way is by adding ~なさる to the 連用形 and ending with the particle な. However, this method is \textbf{rare }. }

\begin{ltabulary}{|P|}
\hline 

連用形+-ないで+下さい(ませんか) \\

連用形+なさる+な \textrightarrow  -なさいますな \\

\end{ltabulary}

\par{Plain \& Vulgar }

\par{The particle な makes the negative imperative. ~ないでくれ makes a vulgar negative command. ~ないで itself can make a casual negative command. 語尾 can greatly change the mood of a command. \hfill\break
}

\par{\textbf{Examples }}

\par{${\overset{\textnormal{}}{\text{27. 屋内走}}}$ らないでくださいね。 \hfill\break
Please don't run in the house. }
 
\par{${\overset{\textnormal{}}{\text{28. 切}}}$ らないでよ。 \hfill\break
Don't hang up! }

\par{29. ${\overset{\textnormal{な}}{\text{舐}}}$ めるな。 \hfill\break
Don't underestimate me. }

\par{30. ${\overset{\textnormal{えんりょ}}{\text{遠慮}}}$ しないで。 \hfill\break
Don't hold back. }
 
\par{${\overset{\textnormal{}}{\text{31. (お)酒}}}$ は ${\overset{\textnormal{}}{\text{飲}}}$ まないでください。 \hfill\break
Please do not drink sake\slash alcohol. }
${\overset{\textnormal{}}{\text{32. 盗}}}$ みだけはするな。 \hfill\break
Just don't steal! 
\par{33. ああ、 ${\overset{\textnormal{}}{\text{触}}}$ らないでくれよ、すけべー! \hfill\break
Ahhh, don't touch me you freak! }
 
\par{${\overset{\textnormal{}}{\text{34. 教会中}}}$ にお ${\overset{\textnormal{}}{\text{話}}}$ しなさいますな。 \hfill\break
Please, no talking during church. }
 
\par{${\overset{\textnormal{}}{\text{35. 心配}}}$ するな。 \hfill\break
Don't worry. }

\par{ ${\overset{\textnormal{}}{\text{36. 馬鹿}}}$ なことをすんな。 \hfill\break
Don't be silly. }
 
\par{${\overset{\textnormal{}}{\text{37. 入}}}$ るな! \hfill\break
Don't enter! }
 
\par{${\overset{\textnormal{}}{\text{38. 誰}}}$ も ${\overset{\textnormal{}}{\text{動}}}$ くな。 \hfill\break
No one move. }
 
\par{${\overset{\textnormal{}}{\text{39. 決}}}$ して ${\overset{\textnormal{やくそく}}{\text{約束}}}$ を ${\overset{\textnormal{やぶ}}{\text{破}}}$ るなよ。 \hfill\break
Never break a promise! }
 
\par{${\overset{\textnormal{}}{\text{40. 金}}}$ に ${\overset{\textnormal{いじきたな}}{\text{意地汚}}}$ く ${\overset{\textnormal{}}{\text{手}}}$ を ${\overset{\textnormal{}}{\text{出}}}$ すな。 \hfill\break
Don't put your greedy hands on money. }
 
\par{${\overset{\textnormal{}}{\text{41. 死}}}$ なないでください。 \hfill\break
Please don't die. }
 
\par{${\overset{\textnormal{}}{\text{42. 無理}}}$ しないでね。 \hfill\break
Take it easy. }

\par{43. ${\overset{\textnormal{ごかい}}{\text{誤解}}}$ しないでください。 \hfill\break
Don't get me wrong. }

\par{44. ${\overset{\textnormal{よけい}}{\text{余計}}}$ なことをべらべらしゃべるな。 \hfill\break
Don't chatter unnecessary things. }

\par{45. ${\overset{\textnormal{おどろ}}{\text{驚}}}$ かないでください。 \hfill\break
Please don't be surprised. }
 
\par{46. パーティーに ${\overset{\textnormal{}}{\text{行}}}$ っても ${\overset{\textnormal{て}}{\text{照}}}$ れるな。 \hfill\break
If you go to the party, don't be shy. }
    