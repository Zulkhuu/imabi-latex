    
\chapter{Seem}

\begin{center}
\begin{Large}
第146課: Seem: ~そうだ 
\end{Large}
\end{center}
 
\par{ ~そうだ acts like a 形容動詞 and is only used with adjectives or verbs. It just can't be used with plain nouns.Its usages have significant differences in grammar. There are two grammatical ways it can be used. In this lesson, we will focus on how it is used when it attaches to the 連用形. }
      
\section{Attaching to the 連用形: Conjecture}
 
\par{ From the outward appearance, ~そうだ expresses the speaker's intuitive subjective guess on the nature, quality, or properties of something or someone. The statement isn't proven. So, if obvious, it's not applicable. This is like "seem to be (doing)". ~そうだ attaches to the 連用形 for verbs but the stems of adjectives for conjecture. }

\begin{ltabulary}{|P|P|P|P|P|P|}
\hline 

動詞 & 降る+そうだ\textrightarrow 降りそうだ & 形容詞 & 面白い+そうだ\textrightarrow 面白そうだ & 形容動詞 & 簡単+そうだ\textrightarrow 簡単そうだ \\ \cline{1-6}

\end{ltabulary}

\par{ The negative can be ~なさそうだ or ~そうじゃない. One may be more preferable depending on the situation and or the opinion of the speaker. With \textbf{adjectives in the negative and よい }, さ should be inserted. So, ~な \textbf{さ }そうだ and よ \textbf{さ }そうだ. }

\par{${\overset{\textnormal{}}{\text{1. 新}}}$ しいマネージャーは ${\overset{\textnormal{こわ}}{\text{怖}}}$ そうな ${\overset{\textnormal{かた}}{\text{方}}}$ ですね。 \hfill\break
The new manager is a scary-looking guy, isn't he? }

\par{2. このビデオゲームは ${\overset{\textnormal{}}{\text{面白}}}$ くなさそうだ。 \hfill\break
This video game doesn't seem interesting. }

\par{${\overset{\textnormal{}}{\text{3. 今日}}}$ は ${\overset{\textnormal{ちょうし}}{\text{調子}}}$ がよさそうだね。 \hfill\break
You look good today. }

\par{4. マレーシアは、 ${\overset{\textnormal{}}{\text{今日}}}$ は ${\overset{\textnormal{}}{\text{天気}}}$ がよさそうです。 \hfill\break
The weather seems nice today in Malaysia. }

\par{${\overset{\textnormal{}}{\text{5. 雨}}}$ は ${\overset{\textnormal{}}{\text{降}}}$ らなさそうです。 \hfill\break
It doesn't look like it's going to rain. }

\par{6. もうしばらく続きそうです。 \hfill\break
It seems like it will continue for a little longer. }

\par{7. このステーキは ${\overset{\textnormal{おい}}{\text{美味}}}$ しそうです! \hfill\break
This steak looks delicious. }

\par{8. ${\overset{\textnormal{きんぱつ}}{\text{金髪}}}$ の ${\overset{\textnormal{}}{\text{女}}}$ が ${\overset{\textnormal{}}{\text{好}}}$ きそうだな。(Masculine; casual) \hfill\break
I bet you like blond women! }

\par{9. まだ ${\overset{\textnormal{}}{\text{使}}}$ えそうだ。 \hfill\break
It looks like you can still use it. }

\par{${\overset{\textnormal{}}{\text{10. 難}}}$ しそうな ${\overset{\textnormal{}}{\text{本}}}$ だ。 \hfill\break
It's a difficult looking book. }

\par{11. なんだかずいぶんよさそうじゃないか? (Casual) \hfill\break
Somehow it doesn't really seem amazing does it? }

\par{${\overset{\textnormal{}}{\text{12. 彼}}}$ は ${\overset{\textnormal{}}{\text{苦}}}$ しくなさそうだ。 \hfill\break
He doesn't seem to be feeling painful. }

\par{${\overset{\textnormal{}}{\text{13. 建物}}}$ は ${\overset{\textnormal{}}{\text{静}}}$ かそうです。 \hfill\break
The building seems quiet. }

\par{${\overset{\textnormal{}}{\text{14. 彼}}}$ は ${\overset{\textnormal{}}{\text{来}}}$ そうだ。 \hfill\break
It seems that he's going to come. }

\par{${\overset{\textnormal{}}{\text{15. 彼}}}$ は ${\overset{\textnormal{}}{\text{楽}}}$ しそうではなかった。 \hfill\break
He didn't look happy. }

\par{16. この ${\overset{\textnormal{}}{\text{犬}}}$ は ${\overset{\textnormal{かわいそう}}{\text{可哀相}}}$ 。 \hfill\break
This poor dog. }

\par{\textbf{Meaning Not }e: Notice how the meaning of かわいい changes to "poor" with ~そうだ. }

\par{\textbf{漢字 Note }: 可哀相 is 当て字. 可哀想 is also possible. }

\par{${\overset{\textnormal{}}{\text{17. 一見}}}$ したところではその ${\overset{\textnormal{}}{\text{本}}}$ はやさしそうだ。 \hfill\break
At first sight, the book seems easy. }

\par{18. 忙しそうです。      VS  忙しいようです。 \hfill\break
Seems to be busy.  Looks busy. }

\par{\textbf{Nuance Note }: As this shows, ~そうだ shows something from an intuitive judgment based on observation.You get the sense that the person is busy. Maybe he's constantly looking at his clock. ~ようだ indicates that the judgment is based on actual knowledge of the person's situation. You may have heard that he was busy, or you know information concerning his work schedule. }

\par{19. ${\overset{\textnormal{しかた}}{\text{仕方}}}$ なく一成は ${\overset{\textnormal{じゅわき}}{\text{受話器}}}$ を置いた。 ${\overset{\textnormal{しょろう}}{\text{初老}}}$ の ${\overset{\textnormal{しゅえい}}{\text{守衛}}}$ が ${\overset{\textnormal{けげん}}{\text{怪訝}}}$ \textbf{そうに }しているので、すぐにその場を立ち去ることにした。 \hfill\break
Kazunari reluctantly put the phone receiver down. The middle-aged security guard was giving a questioning look, so Kazunari immediately deciding to leave the area. \hfill\break
From 白夜行 by 東野圭吾. }

\par{\textbf{Grammar Note }: Remember that this structure is adjectival. So, you can get forms like ~そうな and ~そうに. }

\par{\textbf{Word Note }: 初老 originally meant someone in one's forties. As people now live longer, some consider it to mean mid-fifties to even early sixties. }

\begin{center}
\textbf{Inserting - }\textbf{さ }
\end{center}

\par{Again, with よい and ない, さ is inserted. This gives よさそう and なさそう. As for the auxiliaries ~たい and ~ない, ~そうだ attaches to the stem. However, さ is being attached more and more. Even for adjectives that end in ない like ぎこちない (awkward), ~さ is being attached. The negative is ~そうではない, but you can also use ~な(さ)そうだ. However, something like "知らなそうだ" is correct. You still get examples like 20, though. }

\par{20. 成瀬はつまらなさそうに言った。 \hfill\break
Naruse said in a disappointed tone. \hfill\break
From 顔に降りかかる雨 by 桐野夏生. }

\par{2. Shows a judgment based on circumstances or experience. The negative is そう[に・も]ない. }

\par{${\overset{\textnormal{}}{\text{21. このままだと、帰れ}}}$ そうにもない。 \hfill\break
It would seem that at this rate, there is no way that [I\slash we] will be able to go home. }

\par{${\overset{\textnormal{}}{\text{22. 今}}}$ ならまだ ${\overset{\textnormal{ま}}{\text{間}}}$ に ${\overset{\textnormal{}}{\text{合}}}$ いそうです。 \hfill\break
If now, we can still be there on time. }

\par{23. 景気は依然として好転しそうにない。(ちょっと硬い文章語) \hfill\break
The economy still appears that it won't improve.  }

\par{3. Shows that something looks like \emph{ it's going to or has\dothyp{}\dothyp{}\dothyp{}just as before }. There is a basis with some experience. After all, how can you know if it's going to rain if you've never seen rain? This is an affirmative application of 1 seen with verbs. This can be seen with the past tense, and time phrases are often used to specify when something is thought to occur. }

\par{${\overset{\textnormal{}}{\text{24. 赤}}}$ ちゃんは ${\overset{\textnormal{}}{\text{泣}}}$ き ${\overset{\textnormal{}}{\text{出}}}$ しそうな ${\overset{\textnormal{}}{\text{顔}}}$ をしていた。 \hfill\break
The baby had a face as if he was going to burst out crying. }

\par{25. ${\overset{\textnormal{とうぶん}}{\text{当分}}}$ の間、 ${\overset{\textnormal{こんかい}}{\text{今回}}}$ の ${\overset{\textnormal{ねっぱ}}{\text{熱波}}}$ は ${\overset{\textnormal{いす}}{\text{居据}}}$ わりそうだ。 \hfill\break
For the time being, this heat wave will probably settle. }

\par{26. バランスが ${\overset{\textnormal{くず}}{\text{崩}}}$ れて、 ${\overset{\textnormal{いっしゅんたお}}{\text{一瞬倒}}}$ れそうだったさ。 \hfill\break
I lost my balance, and for a moment it seemed I would fall. }

\par{${\overset{\textnormal{}}{\text{27. 彼}}}$ は100 ${\overset{\textnormal{}}{\text{歳}}}$ まで ${\overset{\textnormal{}}{\text{生}}}$ きられそうだ。 \hfill\break
He is likely to live to 100. }

\par{28. このカメラは ${\overset{\textnormal{}}{\text{壊}}}$ れていそうだ。 \hfill\break
This camera looks broken. }

\par{${\overset{\textnormal{}}{\text{29. 電気}}}$ が ${\overset{\textnormal{}}{\text{消}}}$ えそうだ。 \hfill\break
The lights appear to be dying. }

\par{30. 元気になれそうです。 \hfill\break
It looks like it will be able to become better. }

\par{31. 今にも雨が ${\overset{\textnormal{ふ}}{\text{降}}}$ りそうだ。 \hfill\break
It looks like it's going to rain any moment. }

\par{32. もうすぐ ${\overset{\textnormal{さくら}}{\text{桜}}}$ の ${\overset{\textnormal{}}{\text{花}}}$ が ${\overset{\textnormal{さ}}{\text{咲}}}$ きそうだよ。 \hfill\break
The cherry blossom trees are about to bloom. }

\par{${\overset{\textnormal{}}{\text{33. 試験}}}$ に ${\overset{\textnormal{}}{\text{合格}}}$ できそうだ。 \hfill\break
I feel that I can pass the exam. }

\par{34. ${\overset{\textnormal{ちょうせんはんとう}}{\text{朝鮮半島}}}$ で、 ${\overset{\textnormal{せんそう}}{\text{戦争}}}$ や何か ${\overset{\textnormal{ぶっそう}}{\text{物騒}}}$ なことが ${\overset{\textnormal{お}}{\text{起}}}$ こりそうだ。 \hfill\break
It appears that war \textbf{or something }dangerous is going to occur in the Korean peninsula. }
    