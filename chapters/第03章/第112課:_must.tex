    
\chapter{Must}

\begin{center}
\begin{Large}
第112課: Must 
\end{Large}
\end{center}
 
\par{  This lesson will now look at the "must" phrases. There will be more variation in forms to be aware of in this section, and the grammar complications will be harder than with the "must not" forms, but don't worry. }
      
\section{Must}
 
\par{ The basic patterns are ~なければならない, ~なければいけない, ~なくてはならない, and ~なくてはいけない. Just as with must not phrases, ~ては contracts to ちゃ in colloquial speech. Other contractions include ~なければ \textrightarrow  なけりゃ and ~なくてはいけない shortened to just ~なくちゃ.  }

\begin{ltabulary}{|P|P|P|}
\hline 

Part of Speech & Example Phrase & Conjugations \\ \cline{1-3}

動詞 & 食べる & 食べなくては + ならない・いけない \hfill\break
食べなければ + ならない・いけない \\ \cline{1-3}

形容詞 & 新しい & 新しくなくては + ならない・いけない \\ \cline{1-3}

形容動詞 \hfill\break
名詞 & 新鮮だ & 新鮮でなければ + ならない・いけない \\ \cline{1-3}

\end{ltabulary}

\par{\textbf{Chart Note }: Example verbs may not be perfect for demonstrating the actual sensitive nuances of the options to be discussed in this lesson. }

\begin{center}
 \textbf{Colloquial Variants }
\end{center}

\begin{ltabulary}{|P|P|P|}
\hline 

~なければ \textrightarrow  ~なきゃ & ~なければ \textrightarrow  ~なけりゃ & ~なくては \textrightarrow  ~なくちゃ \\ \cline{1-3}

\end{ltabulary}

\par{\textbf{Variant Note }: ~なきゃ and ~なくちゃ are often used without anything after them. }
 
\begin{center}
\textbf{~なくてはならない VS ~なければならない } 
\end{center}
 
\par{ In the written language, these two "must" phrases are both most likely to be used. This is due to the use of ならない. They're both used not to describe personal matters, but to describe societal duties in regards to law and societal norms. ~なければならない is used for \emph{affirmative }statements. When t he sense of duty you're wishing to command somewhat is directed towards individual responsibility rather than society or some institution as a whole, ~なくてはならない is acceptable. }

\par{1. 国家は ${\overset{\textnormal{ちょくせつ}}{\text{直接}}}$ 、国民すべてに、 ${\overset{\textnormal{さいてい}}{\text{最低}}}$ ${\overset{\textnormal{げんど}}{\text{限度}}}$ の生活を ${\overset{\textnormal{ほしょう}}{\text{保障}}}$ しなければならない。 \hfill\break
The nation must directly ensure a bare minimal lifestyle to all citizens. }
 
\par{2. ${\overset{\textnormal{じゅうぎょういん}}{\text{従業員}}}$ は毎日、 ${\overset{\textnormal{しゅったいきん}}{\text{出退勤}}}$ ${\overset{\textnormal{じ}}{\text{時}}}$ を、タイムレコーダーに ${\overset{\textnormal{きろく}}{\text{記録}}}$ し\{なければならない・なくてはならない\}。 \hfill\break
Employees must record in the time recorder one's leave time. }
 
\begin{center}
\textbf{Interchangeability in the Spoken Language } 
\end{center}

\par{ However, unlike straight-up prohibition, we find all four options ~なければならない・~なければいけない・~なくてはならない・~なくてはいけない acceptable in the spoken language. This is a perfect opportunity to get familiar with casual speech. }
 
\par{3. 国は、国民すべてに、最低限度の生活を保障し\{なければならない・なくてはならない・なければいけない・なくてはいけない\}よ。 \hfill\break
The country must provide some minimal lifestyle to all its citizens. }
 
\par{4. 早く帰って、妹の世話を し\{なきゃなんない・なくちゃなんない・なきゃいけない・なくちゃいけない\}よ。 \hfill\break
I have to go home and take care of my little sister. }

\par{5. 本を ${\overset{\textnormal{かえ}}{\text{返}}}$ さなくちゃ。(Casual) \hfill\break
I have to return the book. }

\par{6. てめーは死ななきゃ。(Masculine; vulgar) \hfill\break
You have to die. }
 
\begin{center}
 \textbf{使い分け }
\end{center}

\par{ In everyday conversation, these four patterns have great interchangeability. If we are to talk about the extremities of the usages of ~なくてはならない and ~なければならない, the difference would be this. }
 
\par{\textbf{~なくてはいけない }:Duty or command to the addressee (which may be oneself or another party) regarding common sense, morality, societal common wisdom, or current conditions. The speaker is directing the addressee. When used towards oneself, you treat yourself as the 'addressee'. If referring to oneself, a first pronoun is usually needed to tell that you mean this. }
 
\par{\textbf{~なければならない }:A duty of the speaker in regards to common sense, morality, and or societal common wisdom. It will show responsibility burdened upon the speaker. The person bearing the responsibility is not excluding to the speaker, but it is best described as being the subject. This may be be used in the spoken language as well, and this is also when you may see the polite form ~なければなりません. }
 
\par{ When you use ~なければ, you imply an obvious duty with no penalty involved for not complying. ~なくては is used when there is a penalty for not complying. いけない is in regards to the listener or one\textquotesingle s conscious and ならない is used in regards to the speaker. So, when we say things interchangeable, we\textquotesingle re saying that if we change one for the other, the new nuance still makes sense. It is not the case that the meaning stays completely the same. }

\par{7. 急がなければいけない。 \hfill\break
You must hurry. }

\par{8. ${\overset{\textnormal{こどもたち}}{\text{子供達}}}$ は早く ${\overset{\textnormal{ね}}{\text{寝}}}$ なくてはいけない。 \hfill\break
Children must go to bed early. }

\par{9. あまりの ${\overset{\textnormal{むだ}}{\text{無駄}}}$ は ${\overset{\textnormal{さ}}{\text{避}}}$ け なければならない。 \hfill\break
We must avoid an excess of waste. }

\par{10. 弟と遊ばなくてはいけないから。 \hfill\break
It's because I have to play with my little brother. }

\par{11. 毎日水を飲まなければなりません。 \hfill\break
You have to drink water every day.  (General you, which includes oneself) }

\par{12. 私はボートに乗らなくてはいけないんです。 \hfill\break
I have to catch the boat. }

\par{13. 新刊を買わなくてはいけません。 \hfill\break
You have buy the new book. }

\par{\textbf{Word Note }: \{その・あの\}新しい本 may be more natural in many contexts than 新刊, which can also be interpreted as "new publication". Nevertheless, 新刊 is still a commonly used word. }

\begin{center}
\textbf{~ないといけない \& ~ないとだめだ }
\end{center}
 
\par{ As you would imagine, the rules governing these phrases is just like when used for must not. ~ないとならない doesn\textquotesingle t exist. And, these two phrases are quite removed semantically from the others. You can\textquotesingle t tell whether the speaker\textquotesingle s duty or addressee\textquotesingle s duty is being referred to without relying on context, so it is typically only used in the spoken language or writing which can reflect back on the conditions in a dialogue. As this phrase particularly blurs the existence of penalty, it has a lighter feeling and is used a lot in the spoken language because of this. }
 
\par{ Because the particle と is used, though, your statement should be one based on experience or knowledge of some sort. Though the situation is somewhat more ambiguous than with must not phrases, you can still see how ~ないといけない refers to rule-like behavior and ~ないとだめだ may hint at the existence of a consequence for not following through. }
 
\par{15. 早く帰って、妹の ${\overset{\textnormal{せわ}}{\text{世話}}}$ を しないとだめなの。(女性語) \hfill\break
I gotta go home quick and take care of my little sister. }
 
\par{16. 明日、試験だから、早く起きないといけないんだよ。 \hfill\break
I have to wake up early tomorrow 'cause I have an exam. }

\par{17. ${\overset{\textnormal{しかつもんだい}}{\text{死活問題}}}$ だから、 ${\overset{\textnormal{たいさく}}{\text{対策}}}$ を ${\overset{\textnormal{こう}}{\text{講}}}$ じないといけないのだ。 \hfill\break
It's a matter of life or death, so I\slash you need to take measures. }

\par{18. ${\overset{\textnormal{あす}}{\text{明日}}}$ までに ${\overset{\textnormal{ほうこくしょ}}{\text{報告書}}}$ を書かないとだめだよ。(Casual yet serious) \hfill\break
I\slash You have to write a report by tomorrow! }

\par{\textbf{Casual Speech Note }: In colloquial speech, you can simply use ~ないと. }

\par{19. もう行かないと。(Casual) \hfill\break
I've got to go. }

\begin{center}
\textbf{Older Variants }
\end{center}
 
\par{ There are many other older and dialectical variations of what we have just learned. As for old phrases, we have ~ねば\{ならない・ならぬ\},and ~なければならぬ which are old-fashioned but not so archaic that they don\textquotesingle t frequently show up in the written language. You may also see the very stiff ~なるまい. ~ねばならない in particular is extremely common in literature. }

\par{20. これからは、自分たちのことは自分たちで決めるんだと思った。なぜなら三人だけで生きていかねばならないからだ。 \hfill\break
I felt that from now on that we would have decide on our own accord concerning ourselves. That is because we must live on just the three of us. }

\par{21. 何としても ${\overset{\textnormal{ごうかく}}{\text{合格}}}$ せんばならんねな。(Old-fashioned; dialectical) \hfill\break
You have to pass no matter what. \hfill\break
From ${\overset{\textnormal{そうぼう}}{\text{蒼氓}}}$ by ${\overset{\textnormal{いしかわたつぞう}}{\text{石川達三}}}$ . }
 
\par{\textbf{Base Note }: The old みぜんけい of する, せ-, must be used with ~ぬ. }
 
\par{\textbf{Variant Note }: There are several important notes about the above sentence. The ~ね in せねば is contracted to ~ん and ~ない in ならない becomes \textrightarrow  ~んね, which would be realized today in Modern Japanese as the casual ~んねー such as in 分かんねー・分かんない. }

\begin{center}
 \textbf{Important Dialectical Variants }
\end{center}
 
\par{Sometimes "must" is expressed completely differently in other dialects. The most common dialect pattern is " ${\overset{\textnormal{みぜんけい}}{\text{未然形}}}$ +なあかんで". You can also see ~ねば contracted to ~にゃ in Western Japanese Dialects. }

\par{ いけない is very sensitive in respect to dialect. Below is a chart of the most common variations you may encounter. }

\begin{ltabulary}{|P|}
\hline 

いかん、いけん、あかん \\

\end{ltabulary}
22. ${\overset{\textnormal{かたみち}}{\text{片道}}}$ ${\overset{\textnormal{きっぷ}}{\text{切符}}}$ を ${\overset{\textnormal{}}{\text{買}}}$ わなあかんで。 \hfill\break
I have to buy a one-way ticket.  
\par{${\overset{\textnormal{}}{\text{23. 歌}}}$ わなくてはいけん。 \hfill\break
I have to sing. }

\par{\textbf{More Notes }: }

\par{1. せにゃならん = しなければならない }

\par{2. せないかん = しなければならない }

\par{3. Note 2 is also a Western Japanese dialect example. Here, we see a rare example of ~ない following the old 未然形 of the verb する, せ-, and あかん contracted as かん. }
    