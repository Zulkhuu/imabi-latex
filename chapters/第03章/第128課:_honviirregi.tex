    
\chapter{Honorifics V}

\begin{center}
\begin{Large}
第128課: Honorifics V: Irregular Verbs I  
\end{Large}
\end{center}
 
\par{ Whereas for most conjugations する and 来る are the only verbs that are irregular, there are many exceptional verbs in honorific speech. Drastic change in conjugation for honorifics makes it more honorific in nature. So, if anything can be drastically changed, it is more preferable. }
      
\section{The Exceptions}
 
\par{ Too little information is typically given for irregular honorific verbs. There are definitely a lot of them, and not even all "irregular" honorific verbs are to be mentioned in this section. That's because many are restricted to the written language, and the biggest priority that you should have is know how to use honorifics effectively and correctly in the spoken language. }

\par{\textbf{Chart Note }: The chart below illustrates the most important irregularities in respect to 尊敬語 and 謙譲語. When one form is normal, it will be marked with (N) in the chart below. There is no ordering to the chart.Variants often differ in nuance, so the chart is followed by many notes. The most important verbs to remember are in bold. }
 
\begin{ltabulary}{|P|P|P|}
\hline 

 \textbf{動詞 }& \textbf{尊敬語 }& \textbf{謙譲語・謙遜語 }\\ \cline{1-3}

行く & \textbf{いらっしゃる・お出でになる }・お越しになる &  \textbf{参る・伺う }・ 参上する・ 上がる \\ \cline{1-3}

来る &  \textbf{いらっしゃる・お出でになる }・お越しになる・ \textbf{見える }・お見えになる &  \textbf{参る }\\ \cline{1-3}

いる & \textbf{いらっしゃる・お出でになる }& \textbf{おる }\\ \cline{1-3}

見る &  \textbf{ご覧になる }& \textbf{拝見する }\\ \cline{1-3}

読む & \textbf{お読みになる (N) }& \textbf{拝読する } \\ \cline{1-3}

言う & \textbf{仰る }& \textbf{申す }\\ \cline{1-3}

食べる & \textbf{召し上がる }・お上がりになる・上がる & \textbf{頂く } \\ \cline{1-3}

飲む & \textbf{召し上がる }・お上がりになる・上がる & \textbf{頂く }\\ \cline{1-3}

受ける & \textbf{お受けになる (N) }& 拝受する \\ \cline{1-3}

助ける & \textbf{ご支援なさる・お助けになる (N) }& \textbf{お手伝いさせていただく } \\ \cline{1-3}

決める & \textbf{ご決定なさる・お決めになる (N) }& \textbf{決めさせていただく (N) }\\ \cline{1-3}

する & \textbf{なさる }& \textbf{致す }\\ \cline{1-3}

知る & \textbf{ご存知だ }& \textbf{存じる・存じ上げる・承知する } \\ \cline{1-3}

知らない & \textbf{ご存知ない }& \textbf{存じない } \\ \cline{1-3}

知らせる & \textbf{お知らせになる (N) }& \textbf{お耳に入れる }\\ \cline{1-3}

見せる & お示しになる・お見せになる (N) & \textbf{お目にかける・ご覧に入れる } \\ \cline{1-3}

訪問する & \textbf{お出でになる・訪問なさる }& \textbf{伺う・参上する }\\ \cline{1-3}

聞く &  \textbf{お聞きになる (N) }& \textbf{伺う・承る・拝聴する }\\ \cline{1-3}

会う & \textbf{お会いになる (N) }& \textbf{お目にかかる・お会いする (N) }\\ \cline{1-3}

買う & \textbf{お求めになる }・ご利用になる & \textbf{お買いする (N) } \\ \cline{1-3}

あげる & おあげになる (N) & \textbf{差し上げる }\\ \cline{1-3}

くれる & \textbf{下さる }& NOT APPLICABLE \\ \cline{1-3}

思う・考える & \textbf{お考えになる (N) \hfill\break
お思いになる (N) }& \textbf{存じる }\\ \cline{1-3}

もらう & \textbf{おもらいになる (N) }& \textbf{頂く・頂戴する }\\ \cline{1-3}

尋ねる & お尋ねになる (N) & \textbf{伺う・参上する } \\ \cline{1-3}

着る & \textbf{お召しになる・ご着用なさる }& \textbf{着させていただく (N) }\\ \cline{1-3}

教える & お教えになる (N) & \textbf{ご案内する }\\ \cline{1-3}

承知する & \textbf{ご承知なさる・ご理解なさる }& \textbf{かしこまる }\\ \cline{1-3}

寝る & \textbf{お休みになる }& \textbf{寝させていただく (N) }\\ \cline{1-3}

許す & \textbf{ご容赦なさる・お許しになる (N) }& \textbf{許させていただく (N) } \\ \cline{1-3}

死ぬ & \textbf{お亡くなりになる }& \textbf{死なせていただく (N) }\\ \cline{1-3}

\end{ltabulary}
 
\par{\textbf{Usage Notes }: }
 
\par{1. As for 行く, if the speaker is going to the place of the addressee, 伺う is used instead of 参る. Also, 参上する and 上がる can be used in both contexts but are not as common. }
 
\par{2. 拝見する is only used if the item belongs to the addressee. }
 
\par{3. ご容赦なさる means "to pardon". }
 
\par{4.  ご尽力なさる = "to be instrumental in". }
 
\par{5. One could say that phrases like お亡くなりになる are just regular honorific forms of euphemisms. }
 
\par{6. 訪ねる can also be お邪魔する in humble speech. }
 
\par{7. お上がりになる is seen a lot in お上がりです and お上がりなさい. }
 
\par{8. 申し上げる \& 申す are not 100\% the same. Although the first is even more humble, it is showing respect to someone in which an action is being extended to. }
 
\par{1a. わたくしはスミスと申し上げます。X \hfill\break
1b. わたくしはスミスと申します。〇 \hfill\break
I am Smith. }

\par{9. 亡くなる relates to 無くなる, 失くす, and 亡くす. Here we have a classic battle between script and nuance. }

\par{\textbf{ 無くなる }is a more literary spelling for the intransitive verb なくなる, which comes from the 連用形 of the adjective 無い・ない and the verb なる.  It can be used to show that something has become no longer in existence, or something is used up, something is lost. }
 
\par{2. 時間がなくなった。 \hfill\break
Time has run up. }
 
\par{\textbf{Spelling Note }: When なくなる is used for ないようになる (to become not\dothyp{}\dothyp{}\dothyp{}), it is viewed as being separate and is not written in Kanji. }

\par{ 亡くなる comes from the same source as above but refers to the passing away of an individual in a \textbf{respectful }\slash euphemistic fashion. }
 
\par{3. いつお亡くなりになりましたか。( \textbf{Honorific }; superfluous to some because of the doubling of なる) \hfill\break
When did [he] die? }
 
\par{ 無くす = 失(く)す = (喪す). It is the transitive form of above and has the following definitions. }
 
\par{1. To lose something that you've had up till now. \hfill\break
2. To get rid of a bad situation. \hfill\break
3. When used in the sense of 亡くす (see below), it can be spelled as 喪す. However, due to the government's attempts to lower spelling options, it is no longer prevalent. }
 
\par{4. 財布をなくしてしまう。 \hfill\break
To accidentally lose one's wallet. \hfill\break
 \hfill\break
5. 不正をなくす。 \hfill\break
To get rid of injustice. }
 
\par{ 亡くす is being died on by someone in your loved ones, and losing that individual.This comes from 亡く + す(る) as expected of the same source as 無くす, which can be broken down likewise. }
 
\par{6. 幼時に父を亡くす。 \hfill\break
To lose one's father in early childhood. }
 
\par{ Now, we have 失う・喪う. It has a lot of commonalities with the transitive なくす. Thus, there is heavy interchangeability between the two. }
 
\par{1. To lose something that had been in your possession or on you(r person). \hfill\break
2. To lose the chance at getting something. \hfill\break
3. To be gone from having been taken\slash stolen from (you). \hfill\break
4. To end up not knowing the path to take. \hfill\break
5. To lose (a loved one). This is where the spelling 喪う comes into play. }
 
\par{7. \{バランス・視力・昇進の機会\}を失う。 \hfill\break
To lose \{balance\slash eyesight\slash the chance of promotion\}. }
 
\par{8. 交通事故で7人の命が\{失・喪\}われました。 \hfill\break
Seven lives were lost in the traffic accident. }
 
\par{9. 鍵が\{無くなる・なくなる\}。 \hfill\break
For the key to be lost\slash disappear\slash be gone. }
 
\par{10. 鍵を\{なくす・無くす・失(く)す・失う\}。 \hfill\break
To lose a key. }

\par{\textbf{Nuance Note }: 鍵を失う sounds more like the key has been in your possession or on your person for a significant period of time. }

\begin{center}
 \textbf{Contractions with ~ます }
\end{center}

\par{  Though it may seem counter-intuitive for there to be contractions in honorifics, it turns out that there are with ~ます when attached to the 連用形 of a handful of 五段 verbs that end in る. }

\begin{ltabulary}{|P|P|P|P|}
\hline 

いらっしゃる\textrightarrow いらっしゃいます & 仰る\textrightarrow 仰います  下さる\textrightarrow 下さいます & ござる\textrightarrow ございます & なさる\textrightarrow なさいます \\ \cline{1-4}

\end{ltabulary}

\begin{center}
 \textbf{Examples }
\end{center}
 
\par{11. クッキーを召し上がりませんか。 \hfill\break
Will you have a cookie? }
 
\par{12. ありがとうございます。じゃあ、ひとつだけいただきます。 \hfill\break
Thank you very much. Well then, I'll just have one. }
 
\par{13. 「野山先生でいらっしゃいますか」 「はい、そうですが」「益田と申しますが、ちょっと伺いたいことがあるんですが」 \hfill\break
”Are you Noyama Sensei?" "Yes, I am" "I'm Masuda. May I ask you something?" }
 
\par{14. 「この写真をご覧になりますか」「ええ、ぜひ拝見したいです」 \hfill\break
"Will you see the photo?" "Yes, I definitely want to see it". }

\par{\textbf{Sound Change Note }: A similar sound change occurs with the imperatives of these verbs. れ \textrightarrow  い. So, for example, 仰る \textrightarrow  仰れ \textrightarrow  仰い. The original imperative is quite old-fashioned for these verbs. }
      
\section{おいでになる VS いらっしゃる}
 
\par{ おいでになる is the honorific version of 行く, いる, and 来る. It is common to see おいで as a command for 行く and 来る. This is a contraction of おいでなさい. おいで in some contexts behaves like いる. }
 
\par{15. お子様はおいででしょうか。 \hfill\break
Do you have children? }
 
\par{16. ちょっと寄っておいで。 \hfill\break
Drop by for a bit. }
 
\par{17. 何名おいでですか。 \hfill\break
How many people are there? }
 
\par{18. ご主人は酔っておいでです。 \hfill\break
Your husband is drunk. }
 
\par{19. 我慢しておいでなさい。 \hfill\break
Have patience. }
 
\par{ Unlike おいでになる, いらっしゃる, the contraction of いらせられる, may be used to make adjectives honorific! The pattern for this is "お~ Adjective 連用形 + ていらっしゃる". Lastly, いらっしゃる is in でいらっしゃる, the honorific copula. }
 
\par{20. お若くていらっしゃいます。 \hfill\break
You are young. }

\par{21. そちら(の方)は彼の友人でいらっしゃいます。 \hfill\break
He is his friend. }
 
\par{22. どちら様でいらっしゃいますか。 \hfill\break
Who is this? }
 
\begin{center}
 \textbf{Differences }
\end{center}
 
\par{ おいでになる may be seen as おいで and おいでなさい to call someone over or to tell someone to go. いらっしゃい, on the other hand, means "welcome!". いらっしゃる is used to make the progressive. When おいでになる is used similarly, it shows some sort of state and is normally seen as おいで in this case. }
      
\section{Honorific Progressive}
 
\par{ So, ~ている is how you make the progressive. In humble speech, you use ~ておる, and in honorific speech you use ~ていっらしゃる. If you want to be extremely honorific, you combine this with the honorific form of a verb. So, you have two big changes to the verb. Only one is necessary to make something honorific. Therefore, these extremely long variants are often not used as much. }
 
\par{23. この雑誌を読んでいらっしゃいますか。 \hfill\break
Are you reading this magazine? }
 
\par{24. 何をお作りになっていらっしゃいますか。 \hfill\break
What are you making? }
 
\par{25. 何もしておりません。 \hfill\break
I am not doing anything. }
      
\section{The Problems with Making a Command with 敬語}
 
\par{ In Classical Japanese, it was very common and completely proper to make a command like 書かせたまえ because Double 敬語 was used. Sadly, however, this reasoning and direct command to a superior in the first place have become bad in Modern Japanese. }

\par{ With all the ways to tell someone in Japanese aside from the 命令形 further complicated by the need to be as polite and indirect as possible to superiors, Japanese has developed many ways to say single commands with various degrees of politeness and uses. }

\par{ It is impossible to list all possible euphemisms of some expressions, but just with the verb 来る, 23 phrases related to its command form 来い can be cited. The following chart is a chart modified from 敬語マニュアル (1996) by 浅田秀子. This is not all powerful, and it doesn't take long to think of counterexamples, but that I will leave you the reader to ponder on. }

\begin{ltabulary}{|P|P|P|P|P|P|}
\hline 

# & 「来い」の表現 & 対象 \hfill\break
(Object) & 文体 \hfill\break
(Writing Style) & 尊敬度 \hfill\break
(Level of Respect) & 新密度 \hfill\break
(Level of intimacy) \\ \cline{1-6}

1 & 来い & Various &  & 0\% & 100\% \\ \cline{1-6}

2 & 来るの・来るんだ & 子供・目下 & 会話 & 5\% & 95\% \\ \cline{1-6}

3 & 来ること & 目下 & 文章 & 10\% & 90\% \\ \cline{1-6}

4 & 来なさい & 目下・子供 & 会話 & 15\% & 85\% \\ \cline{1-6}

5 & 来るように & 目下 &  & 20\% & 80\% \\ \cline{1-6}

6 & 来てくれ & 同輩・目下 & 会話 & 25\% & 75\% \\ \cline{1-6}

7 & 来てちょうだい & 目下・子供 & 会話 & 30\% & 70\% \\ \cline{1-6}

8 & 来てほしい & 目下 & 会話 & 35\% & 65\% \\ \cline{1-6}

9 & 来て & 同輩・目下 & 会話 & 40\% & 60\% \\ \cline{1-6}

10 & 来てください & 他人 & 会話 & 45\% & 55\% \\ \cline{1-6}

11 & おいでなさい & 目下 & 会話 & 50\% & 50\% \\ \cline{1-6}

12 & おいで & 目下・子供 & 会話 & 55\% & 45\% \\ \cline{1-6}

13 & いらっしゃい & 目下・子供 & 会話 & 60\% & 40\% \\ \cline{1-6}

14 & いらっしゃいませ & 同輩・他人 & 会話 & 65\% & 35\% \\ \cline{1-6}

15 & おいでください & 同輩・他人 &  & 70\% & 30\% \\ \cline{1-6}

16 & いらしてください & 同輩・他人 &  & 75\% & 25\% \\ \cline{1-6}

17 & おいでを乞う & 目上・他人 & 文章 & 80\% & 20\% \\ \cline{1-6}

18 & おいでいただきたい & 目上・他人 &  & 85\% & 15\% \\ \cline{1-6}

19 & おいで願いたい & 目上・他人 & 文章 & 90\% & 10\% \\ \cline{1-6}

20 & お運びいただきたい & 目上 & 文章 & 93\% & 7\% \\ \cline{1-6}

21 & 御足労いただきたい & 目上 & 文章 & 96% & 4\% \\ \cline{1-6}

22 & 御来臨を賜りたい & VIP & 文章 & 98\% & 2\% \\ \cline{1-6}

23 & 御来駕賜りたい & VIP & 文章 & 100\% & ~0\% \\ \cline{1-6}

\end{ltabulary}

\par{ Respectfulness is not necessarily the same thing as politeness, and it's definitely hard to define all of these expressions in one chart. So, please keep all of what you know about Japanese in mind. Just to think, this chart doesn't even have 来させ給え! }
      
\section{Other Conjugations}
 
\par{ The volitional, conditional, provisional, potential, and the "must" patterns must be used with the exceptional verbs if they exist. }

\par{26. ご意見をいただけると助かります。 (Potential) \hfill\break
We would appreciate your comments. }

\par{26. よろしかったら明日伺います。(~たら Conditional) \hfill\break
I will be over tomorrow if you like. }

\par{27. 彼のご意見を伺ったところ、彼はその件に対してあまり快く思っていないようです。 \hfill\break
Having asked his opinion, he seemed to not be pleased about the case. }

\par{28. 伝言を伺いましょうか。 (Volitional) \hfill\break
Shall I take a message? }

\par{29. 是非伺いたく存じます。 (With ~たい) \hfill\break
I would like to come. }

\par{30. 長らくお待たせいたしました。 (Causative) \hfill\break
I am sorry to have kept you waiting so long. }

\par{31. ご質問がございましたら、今承ります。  (~たら Conditional) \hfill\break
I will take questions now if you have them. }

\par{32. 早くこちらへいらっしゃい。 (Imperative) \hfill\break
Come here quickly. }
      
\section{Many More Examples}
 
\par{ These sentences range from rather simple expressions, some of which you've already seen before, to far more complex sentences. Take note in how conjugations, particles, and other things work in honorifics. }

\par{33. 玉稿を賜る。(古風) \hfill\break
To be granted your honorable manuscript. }

\par{34. お手紙を拝受いたしました。 \hfill\break
I received your letter. }

\par{35. よくおいでくださいました。 \hfill\break
Thank you for taking the time to come. }

\par{36. お刺し身をいただきます。 \hfill\break
I will eat sashimi. }

\par{38. お部屋をご覧になりますか。 \hfill\break
Would you look in the room? }

\par{39. ご迷惑をおかけして、本当に申し訳なく存じます。 \hfill\break
I appreciate all your trouble. }

\par{40. 長らくご無沙汰して申し訳ございません。 \hfill\break
I am terribly sorry for my long absence. }

\par{41. あの方を存じ上げません。 \hfill\break
I don't know that person. }

\par{42. 明日お目にかかります。 \hfill\break
I'll see you tomorrow. }

\par{42. 初めまして、十和田と申します。 \hfill\break
Nice to meet you. My name is Towada. }

\par{43. 自分は専門家だなどと生意気なことは申しません。 \hfill\break
I don't profess to be an expert. }

\par{44. 手品をお目にかけます。 \hfill\break
I'll show you a trick. }

\par{45. 彼が心配していらっしゃることをもう存じております。 \hfill\break
I already know that he's worrying. }

\par{46. ご入用でしたら差し上げます。 \hfill\break
You may have it for the asking. }

\par{47. こちらの本を差し上げましょう。 \hfill\break
I shall give you this book. }

\par{48. 日本にいついらっしゃいましたか。 \hfill\break
When did you come to Japan? }

\par{49. 弟が参っております。 \hfill\break
My little brother has come. }

\par{50. 時間がございましたら、伺います。 \hfill\break
I would come if I had time. }

\par{51. お礼を申し上げます。 \hfill\break
I wish to express my appreciation. }

\par{52. この度は大変お世話になりありがとうございました。 \hfill\break
Thank you very much for all of your hard work. }

\par{53. このたびは音楽会の入場券をわざわざお送り下さって誠に恐縮です。 \hfill\break
It was especially kind of you to have given me the concert ticket. }

\par{54. お求めになりやすい値段でございます。 \hfill\break
It's at a price apt for you to buy. }

\par{55. 何を召し上がりますか。 \hfill\break
What are you going to eat? }

\par{56. ご注文を承ります。 \hfill\break
I will take your order. }

\par{57. かしこまりました。 \hfill\break
Certainly! }

\par{58. またお越し下さい(ませ)。 \hfill\break
Please come again. }

\par{59. 正直に仰っていただいたことに心よりお礼申し上げます。 \hfill\break
I would like to express my heartfelt gratitude that you spoke honestly about it. }

\par{60. 申し訳ございません。お待たせいたしました。 \hfill\break
I am terribly sorry that I had you wait. }

\par{61. 万年筆を拝借します。 \hfill\break
I will borrow your fountain pen. }

\par{\textbf{Orthography Note }: 漢字 like 御 and 宜 are normally seen in really formal writing. }
 
\par{62. 心中お察しいたします。 \hfill\break
My sympathies are with you. }

\par{63. 喜んで参ります。 \hfill\break
I'll be happy to come. }
 
\par{64. 明日そちらへ持って参ります。 \hfill\break
I'll bring it to you tomorrow. }
 
\par{65. 栄光に存じます。 \hfill\break
It is an honor. }
 
\par{66a. 以降お見知り置きを \hfill\break
66b. 以後お見知り置きください。 \hfill\break
I look forward to getting to know you. }
 
\par{66. 冥利に尽きます。 \hfill\break
I am quite lucky.  }
 
\par{67. 工場をご案内申し上げます。 \hfill\break
I will give you a tour of the factory. }

\par{68. 私たちは伊藤さんにお話しいただきました。 \hfill\break
We were told a story by Mr. Ito. }
 
\par{69. そんなに仰らないでください。私はただ当たり前のことをしたまででございます。 \hfill\break
Please don't say such. I merely did what was natural. }

\par{70. 月曜日にどちらへいらっしゃいますか。 \hfill\break
Where are you going Monday? }
 
\par{71. 結構なお品を賜り、ありがとうございます。 \hfill\break
We are thankful for receiving from you these excellent items. }
 
\par{72. 陛下がお言葉を賜る。 \hfill\break
To receive His Majesty's words. }
 
\par{73. 思し召してご容赦下さい。 \hfill\break
Please forgive me for thinking such. }
 
\par{74. そちらをお断りなさいました。 \hfill\break
(He) rejected it. }
 
\par{75. お手数おかけしますが。 \hfill\break
I hate to be a burden. }
 
\par{76. 京都から参りました。 \hfill\break
I came from Kyoto. }
 
\par{77. 明日伺います。 \hfill\break
I will come tomorrow. }
 
\par{78. ご承知のように私は泳ぐことができません。 \hfill\break
As you know, I cannot swim. }
 
\par{79. 米国のどちらからお出でになりましたか。 \hfill\break
Where in America do you come from? }
 
\par{80. どのようなお仕事をなさいますか。 \hfill\break
What kind of work do you do? }
 
\par{81. はい、よく存じております。 \hfill\break
Yes, I know (him) very well.  }
 
\par{82. 鈴木さん、韓国語をご存じですか。 \hfill\break
Mr. Suzuki, do you know Korean? }
 
\par{83. お酒をお上がりになりますか。 \hfill\break
Will you have some sake?  }
 
\par{84. ご飯を頂こうといたしましたが、お箸がございませんでした。 \hfill\break
I tried to eat (rice), but I didn't have any chopsticks. }
 
\par{85 お名前は何と仰いますか。 \hfill\break
What is your name? }
 
\par{86. おたばこはいただきません。 \hfill\break
I don't smoke. }

\par{87. お越しいただいて恐縮です。 \hfill\break
Thank you for coming. }

\par{88. 何をお貰いになりましたか。 \hfill\break
What did you receive? }
 
\par{89. 呼んで来て差し上げましょう。 \hfill\break
I'll call him here for you.  }
 
\par{\textbf{Dialect Note }: Honorifics is different depending on dialect. Just be aware of this.  }

\begin{center}
\textbf{瀧野と女中との対話の抜粋 }
\end{center}

\par{ This excerpt is from a conversation between Takino-san and a 女中. 女中 is now a politically incorrect word for maid\slash female servant. However, the use of 敬語 by 女中 is quite intriguing in literature. This piece is a typical example of male characters speaking in casual speech while the female combines feminine spoken language aspects to her 敬語. }

\par{90. }

\par{「瀧野さん今晩、金彌にお約束なすってるの?」 \hfill\break
"Takino-san, do you have a engagement with Kanaya tonight?" \hfill\break
「うん」 \hfill\break
"Yes" \hfill\break
と、彼はとっさにうなづいた。 \hfill\break
He promptly acknowledged. \hfill\break
「そうならそうと、早く仰って下さればいいのに」 \hfill\break
"It would be nice if you would say that you did early". \hfill\break
\hfill\break
\textbf{Variant Note }: なすっている is still a commonly used variant of なさっている, but it is typically viewed as dialectical and or not exactly proper. However, many people, especially in the older population, use this variant as it is common throughout East Japan. }

\begin{center}
 \textbf{読み物: Letter in 敬語 }
\end{center}
 
\par{91. }

\par{わたしは若葉学園の卒業生です。都合によって本名が名乗れないのは残念ですが、愛校心は人一倍 強うございます 。今は大阪で家庭の者 となっています が、母校のことは一日も忘れたことはありません。 }

\par{I am a graduate of Wakaba Academy. By convenience, I regret not being able to have my real name known, but my school spirit is twice that of most people. I am now a housewife in Osaka, but I haven't forgotten a day of my old school. }

\par{92. }

\par{このような 手紙を出すのをずいぶん迷いましたが、愛校心のため思い切って書くことに いたしました 。 \hfill\break
I wavered at sending such a letter for a while, but due to school spirit, I daringly decided to write it. }

\par{93. }

\par{先日のこと でございます 。わたくしは三重県の鳥羽に主人といっしょに子供を つれて参り 、あるホテルに泊まりました。そのホテルは鳥羽湾を見渡す景色のいい所 でございます 。 わたくし が朝、飽かずに海のほうを眺め ておりますと 、ふと、下の道を通る男女づれが眼に止まりました。そして思わず、あっ、と叫んだ ことでございます 。 \hfill\break
It was the other day. I went together with my husband and children to Toba in Mie Prefecture, and we stayed at a certain hotel. The hotel is a scenic place to survey Toba Bay. When I was untiringly looking towards the ocean, suddenly, my eyes stopped on a man and woman company passing through the street below. And, without even thinking I let out an "ah". }

\par{94. }

\par{ここから書くのは 忍びない ことですが、今も申した通り、愛校心のために勇を鼓して書くことにします。その 男の方 は若葉学園の大島理事長さんです。それから、同伴の女性はなんと学生課の秋山千鶴子さんではありませんか。秋山さんは、 わたくし の在学中、よくその顔を見 ておりますし 、それに秋山さんに呼びつけられて叱言を喰ったことがありますので、よく 存じあげています 。 \hfill\break
From here I can't bring myself to write, but as I've said now, I will write due to school spirit and take it to heart. That man is Wakaba Academy's Board Chairman Ohshima. And, the female companion was none other than Chizuko Akiyama of the student affairs office. I saw her face a lot when I was at school. Also, I have been called on and scolded on by Ms. Akiyama. So, I know well. }

\par{95. }

\par{わたくし は、そのとき自分の眼を疑いました。まさか大島理事長と秋山さんとが、こんな所を仲よくアベックで散歩しているとは思いませんでした。でも、眼を凝らして見ると、まさにお二人に間違いありません。朝のことでしたが、 お二人 は手をつなぎ合って、まるで若い恋人同士のようでした。 \hfill\break
I doubted my own eyes at that time. I didn't think that Board Chairman Ohshima and Akiyama were walking nicely together as a couple in a place like this. When I tried to fixate my eyes on them, there was no doubt that it was the exact two. Although it was the morning, the two were holding hands and looking just like two young lovers. }

\par{96. }

\par{わたくし は大島理事長が秋山さんと結婚したことも聞かないし、また大島理事長が奥さんを 亡くされたことも承っておりません 。 わたくし はぼんやりと お二人 のうしろ姿を見送っ ておりました 。 \hfill\break
I haven't heard that Board Chairman Ohshima and Akiyama have married, and I haven't heard that his wife has passed. I absent-mindedly saw the two off from behind. }

\par{97. }

\par{それでも、 わたくし は お二人 がそういう お仲 になっているとは信じられず、あるいは学生の引率で こられて 、その間の軽い気持の散歩だと信じたかったのです。けれど、ホテルのフロントにきいてみて、 お二人 の名前もまったく違うし、もちろん、学生の引率でもないことを知って、眼の前が暗くなりました。 \hfill\break
Even so, I couldn't believe that the two were together, and I wanted to believe that perhaps it was a walk with light emotion with them coming as student escorts. But, when I checked at the front desk, the two's names were completely different, and I knew that they were obviously not student escorts, and the front of my eyes darkened. }

\par{ This passage is an excerpt from 混声の森 (下) by 松本清張, one of the best novelists in Modern Japanese literature. This section was in quotations, so other things you would expect with a letter were omitted. Although some of the wording may be somewhat old-fashioned, this passage is a great demonstration of how 敬語 is typically used. }

\begin{center}
\textbf{読み物 2 } 
\end{center}

\par{  Though older literature may have older 敬語, such 敬語 is typically viewed as the "purest" form of it, if such thing even exists in a realistic sense. Regardless, there are many things that you can learn about 敬語 usage in reading texts and listening to dialogues. The following is the first page of the famous story 雪国 by 川端康成. }

\par{ Not all of this passage is in 敬語, but it's important to have preceding context if possible. The spelling of the original text has been altered to be more of use to you. Blocks of text will be given Japanese and then English by the original paragraph spacing of the text. Translating this passage in particular into English has caused many controversies, but the translations provided try to adhere to the original meaning of the text as close as possible while still sounding natural in English. }

\par{98. }

\par{国境の長いトンネルを抜けると雪国であった。夜の底が白くなった。信号所に止まった。 \hfill\break
Once (they) came out of the long border tunnel, there was the Land of Snow. The depths of the night had turned white. The train stopped at the signal station. }

\par{99. }

\par{向う側の座席から娘が立って来て、島村のガラス窓を落とした。雪の冷気が流れこんだ。娘は窓いっぱいに乗り出して、遠くへ叫ぶように、 \hfill\break
「駅長さあん、駅長さあん」 \hfill\break
A young lady stood up and came over from the seat on the other side and opened the glass window in front of Shimamura. The cold snowy air rushed in. The girl leaned all out of the window, and as if she were yelling afar, she shouted “Conductor, conductor”. }

\par{100. }

\par{明かりをさげてゆっくり雪を踏んで来た男は、襟巻で鼻の上まで包み、耳に帽子の毛皮を垂れていた。 \hfill\break
A man who had slowly walked there in the snow covered himself to the top of his nose with a scarf and had a cap pelt drooping on his ears. }

\par{101. }
 
\par{もうそんな寒さかと島村は外を眺めると、鉄道の官舎らしいバラックが山裾に寒々と散らばっているだけで、雪の色はそこまで行かぬうちに闇に呑まれていた。 \hfill\break
When Shimamura gazed outside, thinking it had already gotten that cold, railroad residence-looking barracks were just desolately dispersed at the foot of the mountains, and before the snow hues could reach that far, the barracks were swallowed by darkness. }

\par{102. 駅長さん、私です、 \textbf{御機嫌よろしゅうございます。 }」 \hfill\break
“Conductor, it\textquotesingle s me. How do you do?”. }
 
\par{103. ああ、葉子さんじゃないか。お帰りかい。また寒くなったよ。」 \hfill\break
“Ah, why isn\textquotesingle t it Youko? Welcome home, it\textquotesingle s gotten cold again”. }
 
\par{104. 弟が今度こちらに \textbf{勤めさせていただいております }のですってね。 \textbf{お世話さまです }わ。」 \hfill\break
”I heard that my younger brother has been allowed to work here next time. Thank you for your care”. }
 
\par{105. 「こんなところ、今に寂しくて参るだろうよ。若いのに可哀想だな。」 \hfill\break
“He\textquotesingle ll probably be lonely and troubled before long now. He\textquotesingle s pitiful yet so young. }
 
\par{106. 「ほんの子供ですから、駅長さんから \textbf{教えてやっていただいて、よろしくお願いいたします }わ。」 \hfill\break
“Since he\textquotesingle s merely a kid, I ask that you teach him well, Conductor”. }

\par{ 葉子\textquotesingle s language is indicative of 敬語 speech in the early half of the 20th century. The rule that polite endings should not have 連体形 didn't really exist then, and even today, this rule is still ignored. As you can also see from this passage, traditional feminine speech is maintained even in 敬語. This is now not really the case anymore for obvious post-modern attitude changes towards the role of women. }

\begin{center}
 \textbf{紹介 }
\end{center}

\par{107.  葉山先生をご紹介いたします。葉山先生は、ワシントン大学の大学院で勉強なさった後、 \hfill\break
 ずっとカナダで日本語を教えていらっしゃいましたが、五年前に名古屋大学にいらっしゃいました。 \hfill\break
 最近はとてもお忙しくて、テレビをご覧になる時間もないと仰っています。 \hfill\break
 この間、ピアノをお買いになったそうです。今日は「ほうき星」を歌ってくださいます。 \hfill\break
 昨日もお宅で練習なさったそうです。では、葉山先生、よろしくお願いいたします。 }

\par{1. Who is being introduced? \hfill\break
2. When did Hayama visit Nagoya University? \hfill\break
3. Where did Hayama graduate from for graduate school? \hfill\break
4. Where did Hayama teach Japanese for a long time? \hfill\break
5. What song is she going to be singing in English? \hfill\break
6. Underline the honorific speech. Change the person being introduced to your friend, and then change the honorific speech to polite speech. }

\begin{center}
 \textbf{リー先生 }
\end{center}

\par{108.  リー先生は、韓国語を教えています。 韓国から来たばかりのころは、 \hfill\break
 韓国の方がいいと思ったそうですが、今はアメリカの方が住みやすいと言っています。 \hfill\break
 スポーツはあんまりしないそうですが、音楽が好きだそうです。 \hfill\break
 毎日遅くまで研究室にいるので、いつ行っても会えますよ。 }

\par{1. Underline the parts that need to be changed to 敬語. \hfill\break
2. Change the underlined parts to an appropriate form. \hfill\break
3. What does Lee think about America? \hfill\break
4. What does Lee not like to do often? \hfill\break
5. Why is it that "we" can visit Lee whenever? }
      
\section{Keys}
 
\par{紹介: Key }

\par{1. Hayama Sensei is being introduced. \hfill\break
2. Hayama visited Nagoya University five years ago. \hfill\break
3. Hayama graduated from Washington University for graduate school. \hfill\break
4. Hayama taught Japanese in Canada for a long time. \hfill\break
5. She is to sing "Shouting Star". \hfill\break
6.  葉山先生を ご紹介いたします 。葉山先生は、ワシントン大学の大学院で勉強 なさった 後、 \hfill\break
 ずっとカナダで日本語を教えて いらっしゃいました が、五年前に名古屋大学に いらっしゃいました 。 \hfill\break
 最近はとても お忙しくて 、テレビを ご覧になる 時間もないと 仰っています 。 \hfill\break
 この間、ピアノを お買いになった そうです。今日は「ほうき星」を歌って くださいます 。 \hfill\break
 昨日も お宅 で練習 なさった そうです。では、葉山先生、 よろしくお願いいたします 。 }

\par{友達の___を紹介します。___は、ワシントン大学の大学院で勉強した後、ずっとカナダで日本語を教えていましたが、五年前に名古屋大学に来ました。最近はとても忙しくて、テレビを見る時間もないと言っています。この間、ピアノをお買いになったそうです。今日は「ほうき星」を歌ってくれます。昨日も家で練習したそうです。じゃあ、____、よろしくお願いします。 }

\par{リー先生: Key }

\par{1. Underline the parts that need to be changed to 敬語. }

\par{ リー先生は、韓国語を 教えています 。 韓国から 来た ばかりのころは、 \hfill\break
 韓国の方がいいと 思った そうですが、今はアメリカの方が住みやすいと 言っています 。 \hfill\break
 スポーツは あんまりしない そうですが、音楽が 好きだ そうです。 \hfill\break
 毎日遅くまで研究室に いる ので、いつ行っても 会えますよ 。 }

\par{2.  リー先生は、韓国語を教えていらっしゃいます。韓国からいらっしゃったばかりのころは、 \hfill\break
 韓国の方がいいとお思いになったそうですが、今はアメリカの方が住みやすいと仰っています。 \hfill\break
 スポーツはあまりなさらないそうですが、音楽がお好きだそうです。 \hfill\break
 毎日遅くまで研究室にいらっしゃるので、いつ行ってもお会いできます。 }

\par{3. Lee thinks that it is easier to live in America than Korea. }

\par{4. Lee doesn't play sports often. }

\par{5. Lee is always in his office. }
    