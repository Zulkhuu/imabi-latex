    
\chapter{いかない vs いけない}

\begin{center}
\begin{Large}
第113課: いかない vs いけない 
\end{Large}
\end{center}
 
\par{ To be or not to be, or in this case, to choose いかない or いけない is a not so simple choice for the Japanese learner to make at times. They both come from the verb 行く, which has the basic meaning of “to go.” Unsurprisingly, the basic meanings of 行かない and 行けない are “to not go” and “cannot go” respectively. }

\par{1. ${\overset{\textnormal{なに}}{\text{何}}}$ もかもうまくいかないときは、 ${\overset{\textnormal{じぶん}}{\text{自分}}}$ を ${\overset{\textnormal{か}}{\text{変}}}$ える、いいチャンスではないでしょうか。 \hfill\break
Isn\textquotesingle t when everything goes wrong a great chance to change oneself? }

\par{2. でも、そうかといって ${\overset{\textnormal{おれ}}{\text{俺}}}$ が ${\overset{\textnormal{とうひょう}}{\text{投票}}}$ (をし)に ${\overset{\textnormal{い}}{\text{行}}}$ っても ${\overset{\textnormal{い}}{\text{行}}}$ かなくても ${\overset{\textnormal{べつ}}{\text{別}}}$ に ${\overset{\textnormal{いっぴょう}}{\text{一票}}}$ の ${\overset{\textnormal{おも}}{\text{重}}}$ みを ${\overset{\textnormal{かん}}{\text{感}}}$ じない。 \hfill\break
But even so, whether I go vote or not, I don\textquotesingle t particularly feel the weight of one vote. }

\par{3. ${\overset{\textnormal{よ}}{\text{読}}}$ んでもらえないだけならまだしも、 ${\overset{\textnormal{ほんとう}}{\text{本当}}}$ に ${\overset{\textnormal{ちゅうもく}}{\text{注目}}}$ してほしいところに ${\overset{\textnormal{ちゅうい}}{\text{注意}}}$ が ${\overset{\textnormal{い}}{\text{行}}}$ かないってことが ${\overset{\textnormal{もんだい}}{\text{問題}}}$ なんですよ。 \hfill\break
If it were just not being able to have him read that would be one thing, but the problem is that his attention doesn\textquotesingle t go to the areas that I really want paid attention to. }

\par{4. トイレに ${\overset{\textnormal{い}}{\text{行}}}$ けないときに ${\overset{\textnormal{かぎ}}{\text{限}}}$ って ${\overset{\textnormal{ふくつう}}{\text{腹痛}}}$ が ${\overset{\textnormal{お}}{\text{起}}}$ こるのはなぜでしょうか。 \hfill\break
Why is it that occurrences of abdominal pain is limited to when I can\textquotesingle t go to the bathroom? }

\par{5. ${\overset{\textnormal{くうこう}}{\text{空港}}}$ に ${\overset{\textnormal{みおく}}{\text{見送}}}$ りに ${\overset{\textnormal{い}}{\text{行}}}$ けなくてごめんね。 \hfill\break
I\textquotesingle m sorry for not being able to see you off at the airport. }

\par{\textbf{Orthography Note }: The uses of these two phrases that are to be discussed are typically written in Hiragana. }
      
\section{わけにはいかない}
 
\par{ Translated as “there\textquotesingle s no way…can…” or “…cannot (afford to) …” this expression is used to state that an action is not reasonable\slash proper for obvious reasons and cannot be done as a result. }

\par{6. ${\overset{\textnormal{にほんご}}{\text{日本語}}}$ の ${\overset{\textnormal{じゅぎょう}}{\text{授業}}}$ では ${\overset{\textnormal{えいご}}{\text{英語}}}$ を ${\overset{\textnormal{はな}}{\text{話}}}$ すわけにはいきません。 \hfill\break
You cannot speak English in Japanese class. }

\par{7. アメリカ ${\overset{\textnormal{し}}{\text{史}}}$ の授業ではガムを ${\overset{\textnormal{か}}{\text{噛}}}$ むわけにはいきません。 \hfill\break
You cannot chew gum in American history. }

\par{8. ${\overset{\textnormal{じゅぎょう}}{\text{授業}}}$ では ${\overset{\textnormal{ね}}{\text{寝}}}$ るわけにはいきません。 \hfill\break
You cannot afford to sleep in class. }

\par{9. ${\overset{\textnormal{いそが}}{\text{忙}}}$ しくても ${\overset{\textnormal{ね}}{\text{寝}}}$ ないわけにはいかないよ。 \hfill\break
Even if you're busy, you can\textquotesingle t afford to not sleep. \hfill\break
}

\par{10. }

\par{A: ${\overset{\textnormal{かんじ}}{\text{漢字}}}$ は ${\overset{\textnormal{べんきょう}}{\text{勉強}}}$ するのに ${\overset{\textnormal{じかん}}{\text{時間}}}$ がかかりますねえ。 \hfill\break
B:  ええ、でも ${\overset{\textnormal{おぼ}}{\text{覚}}}$ えないわけにはいかないしねえ。 \hfill\break
A: It must take a lot of time to study Kanji. \hfill\break
B: Yeah, but I can\textquotesingle t afford not to remember them. }

\par{11. 「この ${\overset{\textnormal{ごろ}}{\text{頃}}}$ サークル ${\overset{\textnormal{かつどう}}{\text{活動}}}$ で ${\overset{\textnormal{いそが}}{\text{忙}}}$ しいようだね」「うん、でも、 ${\overset{\textnormal{いそが}}{\text{忙}}}$ しいといって ${\overset{\textnormal{しゅくだい}}{\text{宿題}}}$ をしないわけにはいかないしね」 \hfill\break
“You seem pretty busy these days with club activities, don't you?” “Yeah but, just because I\textquotesingle m busy with them, I can\textquotesingle t afford not to do my homework.” }

\par{12. 「 ${\overset{\textnormal{かんじ}}{\text{漢字}}}$ は ${\overset{\textnormal{むずか}}{\text{難}}}$ しいねえ」「うん、でもそうかと ${\overset{\textnormal{い}}{\text{言}}}$ って ${\overset{\textnormal{かんじ}}{\text{漢字}}}$ で ${\overset{\textnormal{か}}{\text{書}}}$ かないわけにはいかないしねえ」 \hfill\break
"Kanji are difficult, aren't they?” “Yeah, but even so, I can\textquotesingle t afford not to write with them, you know…” }

\par{13. 「 ${\overset{\textnormal{まいにちいそが}}{\text{毎日忙}}}$ しくて ${\overset{\textnormal{ね}}{\text{寝}}}$ る ${\overset{\textnormal{じかん}}{\text{時間}}}$ もない」「でもそうかといって ${\overset{\textnormal{ぜんぜんね}}{\text{全然寝}}}$ ないわけにはいかないでしょ」 \hfill\break
"I'm so busy every day that there's no time to sleep." “Yeah, but even so, there\textquotesingle s no way you can just not sleep at all.” }
      
\section{いけない}
 
\par{ The not so literal usages of いけない revolve around being a euphemism of 悪い. }

\par{14. ${\overset{\textnormal{かれ}}{\text{彼}}}$ は ${\overset{\textnormal{じこ}}{\text{事故}}}$ に ${\overset{\textnormal{あ}}{\text{遭}}}$ ったって?それはいけない、あいつは ${\overset{\textnormal{だいじょうぶ}}{\text{大丈夫}}}$ なの? \hfill\break
He got in an accident? That\textquotesingle s not good; is he alright? }

\par{15. ${\overset{\textnormal{さいきん}}{\text{最近}}}$ 、 ${\overset{\textnormal{にほんしゅ}}{\text{日本酒}}}$ を ${\overset{\textnormal{の}}{\text{飲}}}$ めるようになったんですが、 ${\overset{\textnormal{あつかん}}{\text{熱燗}}}$ はあまりいけない ${\overset{\textnormal{くち}}{\text{口}}}$ なんですね。 \hfill\break
Recently, I\textquotesingle ve become able to drink sake, but I can\textquotesingle t really handle hot sake. }

\par{16. ${\overset{\textnormal{いぶつこんにゅう}}{\text{異物混入}}}$ なんか ${\overset{\textnormal{ぜんぶほうどう}}{\text{全部報道}}}$ していたら、どこの ${\overset{\textnormal{みせ}}{\text{店}}}$ もいけなくなるわ。 \hfill\break
If every single instance of contamination were reported (in the media), all stores would go out. }

\par{17. ${\overset{\textnormal{ばか}}{\text{馬鹿}}}$ な ${\overset{\textnormal{びようし}}{\text{美容師}}}$ だなあ、この ${\overset{\textnormal{みせ}}{\text{店}}}$ もいけなくなるだろう。 \hfill\break
What a stupid stylist…well that shop will probably go under. }

\par{18. ${\overset{\textnormal{の}}{\text{飲}}}$ みすぎは ${\overset{\textnormal{からだ}}{\text{体}}}$ にいけない。 \hfill\break
Drinking too much is not good for the body. }

\begin{center}
 \textbf{ていけない }
\end{center}

\par{ The phrase ていけない is used to show that something is “undesirable\slash unpleasant.” }

\par{19. ${\overset{\textnormal{くち}}{\text{口}}}$ の ${\overset{\textnormal{なか}}{\text{中}}}$ が ${\overset{\textnormal{しおから}}{\text{塩辛}}}$ くていけないので、 ${\overset{\textnormal{みず}}{\text{水}}}$ を ${\overset{\textnormal{の}}{\text{飲}}}$ んだ。 \hfill\break
I drank water since having the inside of my mouth being salty would be unpleasant. }

\par{20. あのくらいの ${\overset{\textnormal{ねんれい}}{\text{年齢}}}$ の ${\overset{\textnormal{じょせい}}{\text{女性}}}$ は ${\overset{\textnormal{ほんとう}}{\text{本当}}}$ にお ${\overset{\textnormal{しゃべ}}{\text{喋}}}$ りでいけないっすね。 \hfill\break
Women at that age are chatterboxes, which is really unpleasant, you know. }

\par{21. ${\overset{\textnormal{ふゆ}}{\text{冬}}}$ は ${\overset{\textnormal{さむ}}{\text{寒}}}$ くていけないね。 \hfill\break
Winter is cold and unpleasant, you know. }

\par{22. ${\overset{\textnormal{きみ}}{\text{君}}}$ がいなきゃ ${\overset{\textnormal{い}}{\text{生}}}$ きていけない。 \hfill\break
I can\textquotesingle t live without you. }

\par{23. ${\overset{\textnormal{さいきんなに}}{\text{最近何}}}$ かとやりたいことが ${\overset{\textnormal{おお}}{\text{多}}}$ いせいか、どうにも ${\overset{\textnormal{き}}{\text{気}}}$ が ${\overset{\textnormal{ち}}{\text{散}}}$ っていけない ${\overset{\textnormal{き}}{\text{気}}}$ がする。 \hfill\break
Recently one way or another, perhaps from having a lot of stuff I want to do, I feel like getting distracted by no means would be good. }

\begin{center}
\textbf{てはいけない } 
\end{center}

\par{ The phrase てはいけない is used to show that some state\slash action is not permissible\slash acceptable. It is typically translated as “must\slash may not.” }

\par{24. ${\overset{\textnormal{しばふ}}{\text{芝生}}}$ に ${\overset{\textnormal{はい}}{\text{入}}}$ ってはいけません。 \hfill\break
You must not get on the lawn. }

\par{25. ${\overset{\textnormal{きょうしつ}}{\text{教室}}}$ の ${\overset{\textnormal{なか}}{\text{中}}}$ でタバコを ${\overset{\textnormal{す}}{\text{吸}}}$ ってはいけません。 \hfill\break
You must not smoke inside the classroom. }

\par{26. ${\overset{\textnormal{みず}}{\text{水}}}$ と ${\overset{\textnormal{いっしょ}}{\text{一緒}}}$ に ${\overset{\textnormal{の}}{\text{飲}}}$ んではいけません。 \hfill\break
You must not take (the medicine) with water. }

\par{27. ${\overset{\textnormal{ねこ}}{\text{猫}}}$ はネギを ${\overset{\textnormal{た}}{\text{食}}}$ べてはいけません。 \hfill\break
Cats must not eat scallions. }

\par{28. ${\overset{\textnormal{そと}}{\text{外}}}$ で ${\overset{\textnormal{あそ}}{\text{遊}}}$ んじゃいけないぞ。(Masculine) \hfill\break
You mustn't play outside. }

\begin{center}
\textbf{\emph{To ikenai\slash Tara ikenai }と・たらいけない } 
\end{center}

\par{ The phrase といけない is used to show that if said state comes to be, the speaker and or persons involved will be troubled. たらいけない is also possible, and as expected, it simply shows that a certain state\slash action would not be pleasant if it happens\slash is done. }

\par{29. ${\overset{\textnormal{ちこく}}{\text{遅刻}}}$ するといけない ので、 ${\overset{\textnormal{なに}}{\text{何}}}$ より ${\overset{\textnormal{じかんげんしゅ}}{\text{時間厳守}}}$ です。 \hfill\break
Being late is bad, and so strict observance of time is above all. }

\par{30. ${\overset{\textnormal{やさい}}{\text{野菜}}}$ が ${\overset{\textnormal{くさ}}{\text{腐}}}$ るといけないから、 ${\overset{\textnormal{いったんいえ}}{\text{一旦家}}}$ に ${\overset{\textnormal{かえ}}{\text{帰}}}$ って ${\overset{\textnormal{れいぞうこ}}{\text{冷蔵庫}}}$ に ${\overset{\textnormal{い}}{\text{入}}}$ れてから ${\overset{\textnormal{で}}{\text{出}}}$ かけましょう。 \hfill\break
It won\textquotesingle t be good if the vegetables go bad, so let\textquotesingle s return to the house for a moment, put them in the refrigerator, and then go out. }

\par{31. ${\overset{\textnormal{まいにち}}{\text{毎日}}}$ ${\overset{\textnormal{は}}{\text{歯}}}$ を ${\overset{\textnormal{みが}}{\text{磨}}}$ かないといけない。 \hfill\break
You must brush your teeth every day. \hfill\break
Literally: It\textquotesingle ll be bad if you don\textquotesingle t brush your teeth every day. }

\par{32. ${\overset{\textnormal{しょうみきげん}}{\text{賞味期限}}}$ が ${\overset{\textnormal{き}}{\text{切}}}$ れたら、 ${\overset{\textnormal{ぜったい}}{\text{絶対}}}$ に ${\overset{\textnormal{た}}{\text{食}}}$ べたらいけないものといえば、 ${\overset{\textnormal{たし}}{\text{確}}}$ かに ${\overset{\textnormal{たまご}}{\text{卵}}}$ や ${\overset{\textnormal{とりにく}}{\text{鶏肉}}}$ とかですね。 \hfill\break
In speaking of foods you should never eat once it\textquotesingle s passed its expiration date, there\textquotesingle s definitely eggs and poultry, right? }

\par{33. ${\overset{\textnormal{こうけつあつ}}{\text{高血圧}}}$ の ${\overset{\textnormal{くすり}}{\text{薬}}}$ を ${\overset{\textnormal{ふくよう}}{\text{服用}}}$ しているときに、グレープフルーツなど ${\overset{\textnormal{た}}{\text{食}}}$ べたらいけないとよく ${\overset{\textnormal{い}}{\text{言}}}$ われます。 \hfill\break
When taking high blood pressure medicine, you\textquotesingle re often told not to eat things like grape fruit. }

\begin{center}
\textbf{なくては・なければいけない }
\end{center}

\par{ The phrase なくては・なければいけない is used to express some obligation\slash necessity to do something and it is typically translated as “must do…” As we have discussed once before when we discussed the “must\slash must not” phrases, double negatives like in this situation create positive expressions just as in English. }

\par{34. ${\overset{\textnormal{あしもと}}{\text{足下}}}$ に ${\overset{\textnormal{ちゅうい}}{\text{注意}}}$ しなくてはいけません。 \hfill\break
You must watch your step. }

\par{35. ${\overset{\textnormal{ひつよう}}{\text{必要}}}$ な ${\overset{\textnormal{しょるい}}{\text{書類}}}$ を ${\overset{\textnormal{ていしゅつ}}{\text{提出}}}$ しなくてはいけません。 \hfill\break
You must submit necessary documents. }

\par{36. ${\overset{\textnormal{ゆうりょうどうろ}}{\text{有料道路}}}$ を ${\overset{\textnormal{とお}}{\text{通}}}$ らなければいけません。 \hfill\break
We must go through a toll road. }

\par{37. そうしたチャンスは ${\overset{\textnormal{まれ}}{\text{稀}}}$ なので、 ${\overset{\textnormal{み}}{\text{見}}}$ つけたときには ${\overset{\textnormal{おお}}{\text{大}}}$ きく ${\overset{\textnormal{とうし}}{\text{投資}}}$ しなくてはいけません。 \hfill\break
Such chances are rare, and so when you find them, you need to invest a lot. }

\par{38. タオルも ${\overset{\textnormal{たいりょう}}{\text{大量}}}$ に ${\overset{\textnormal{こうにゅう}}{\text{購入}}}$ しなくてはいけません。 \hfill\break
We have to purchase a large quantity of towels as well. }
      
\section{いかん・いかぬ}
 
\par{ In addition to what we\textquotesingle ve seen, it\textquotesingle s important to note that いかない sometimes becomes いかん・いかぬ. The latter is more dialectical and or potentially old-fashioned, but you will still frequently encounter it in anime\slash manga among other places. Interestingly enough, despite having seen the different usages of いかない and いけない, いかん・いかぬ can be used in the same instances as both いかない and いけない. }

\par{39. 背中を見せるわけにかんぞ。 \hfill\break
You cannot show your backside. }

\par{40. 通すわけにはいかぬぞ。 \hfill\break
I cannot have you pass by. }

\par{41. やられたわ、もういかん。 \hfill\break
Ugh, I\textquotesingle m done for. }

\par{42. 威張ってはいかん。 \hfill\break
Don\textquotesingle t be stuck up. }

\par{43. 無理をしてはいかん。 \hfill\break
You mustn\textquotesingle t push yourself. }
      
\section{あかん}
 
\par{ Another noteworthy variant that can be used for any of the usages we\textquotesingle ve discussed thus far is あかん. This word is famously used in Kansai dialects and likely derives from the set phrase 埒が明かない, which means “to not get anywhere.” This word, not being in Standard Japanese, has a peculiar pitch accent in that all three syllables are high pitch. This word is so well-known and frequently used that there is no way that you can watch TV for more than an hour before it's used. }

\par{ As far as grammar is concerned, many of the example sentences below will incorporate dialectical grammar that we have yet to ever cover, but contrary to the “must” patterns you\textquotesingle ve grown accustomed to, あかん follows verb by adding ない and dropping the \slash i\slash . }

\begin{ltabulary}{|P|P|}
\hline 

 \emph{Ichidan }Verbs & 食べる + あかん = 食べなあかん \\ \cline{1-2}

 \emph{Godan }Verbs & 飲む + あかん = 飲まなあかん \\ \cline{1-2}

 \emph{Suru }& する + あかん = せなあかん \\ \cline{1-2}

 \emph{Kuru }& 来る + あかん = 来なあかん \\ \cline{1-2}

\end{ltabulary}

\par{45. そないことしたら、あかん。 \hfill\break
You shouldn\textquotesingle t do stuff like that. }

\par{\textbf{Dialect Note }: そんな becomes そない for most Kansai dialect speakers. }

\par{46. 子共は見たらあかん。 \hfill\break
Kids don\textquotesingle t need to watch. }

\par{47. 一回ぐらいは見に来なあかんで。 \hfill\break
You must come see at least once! }

\par{\textbf{Dialect Note }: The particle で the end of a sentence in Kansai dialect is similar in meaning to the particle よ. }

\par{48. あかんゆうたら、あかん。 \hfill\break
If you say it\textquotesingle s no good, then it\textquotesingle s no good. }

\par{\textbf{Dialect Note }: 言ったら tends to be pronounced as \emph{y }\emph{ūtara }in Kansai dialects. The verb 言う also tends to be used without the quotation particle と. }

\par{49. あかん子やなあ。 \hfill\break
What a messed up kid. }

\par{\textbf{Dialect Note }: The copula verb is usually や in Kansai dialects. }

\par{50. 体重を減らすどころか増やさなあかんぐらいやで。 \hfill\break
Rather than losing weight, you\textquotesingle ve gotten to the point you need to be gaining some. }

\par{51. あかん、雨降ってきた。 \hfill\break
Crap, it\textquotesingle s started to rain. }

\par{52. こんだけ頼んでもあかん。 \hfill\break
You can\textquotesingle t entrust anything to (him) even if it's just something like this. }

\par{\textbf{Dialect Note }: これだけ becomes こんだけ in most Kansai dialects. }

\par{53. しっかり勉強せなあかんで。 \hfill\break
I have to properly study. }

\par{54. こら、あかんわ。 \hfill\break
This is no good. }

\par{\textbf{Dialect Note }: これは is frequently contracted to こら in most Kansai dialects. }

\par{ 55. 何で朝起きなあかんのか。 \hfill\break
Why do I have to wake up in the morning? }
    