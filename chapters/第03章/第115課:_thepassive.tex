    
\chapter{The Passive I}

\begin{center}
\begin{Large}
第115課: The Passive I 
\end{Large}
\end{center}
 
\par{ This lesson is actually our second lesson concerning ~ られる and ~れる but sadly not the last. They not only make the potential, but they also make the passive voice. The Japanese passive is used in ways the English passive is not. This is aside from the obvious fact that the times when English speakers and Japanese speakers use it won't be the same across the board. }
      
\section{~られる and ~れる}
 
\par{ ~られる and ~れる attach to the 未然形 and may be used for the following things. ~られる is used with 一段 verbs and 来る and ~れる is used with 五段 verbs and する. }

\begin{ltabulary}{|P|P|P|P|P|P|P|P|}
\hline 

一段 & 食べられる & 五段 & 飲まれる & する & される & 来る & こられる \\ \cline{1-8}

\end{ltabulary}

\par{ As you can see, 来る is intransitive but is shown here. You will learn in the next lesson on the passive how the Japanese passive endings can be used with intransitive verbs. You'll also learn in a future lessons how to create spontaneity and light honorific phrases. }
 
\par{\textbf{Conjugation Notes }: }
 
\par{1. For する, the older passive form is せられる (せ-未然形 + ~られる). Some verbs like 課す (to tax; levy) are still sometimes used with the old form せられる. \hfill\break
2. The voiced forms of する, ずる is conjugated as follows: ずる + られる = ぜられる (Ex. 感ぜられる). Using ずる is old-fashioned and restricted to 書き言葉. }

\begin{center}
 \textbf{More Conjugation Examples }
\end{center}

\begin{ltabulary}{|P|P|P|P|P|P|}
\hline 

五段 & To take & 取る \textrightarrow  取られる & 一段 & To eat & 食べる \textrightarrow  食べられる \\ \cline{1-6}

一段 & To see & 見る \textrightarrow  見られる & 五段 & To swim & 泳ぐ \textrightarrow  泳がれる \\ \cline{1-6}

カ変 & To come & 来る \textrightarrow  来られる & 五段 & To buy & 買う \textrightarrow  買われる \\ \cline{1-6}

サ変 & To do & する \textrightarrow  される & 五段 & To change (int.) & 変わる \textrightarrow  変わられる \\ \cline{1-6}

五段 & To carry & 運ぶ \textrightarrow  運ばれる & 一段 & To slip off & 脱げる \textrightarrow  脱げられる \\ \cline{1-6}

五段 & To steal & 盗む \textrightarrow  盗まれる & 五段 & To indulge & 貪る \textrightarrow  貪られる \\ \cline{1-6}

五段 & To fasten & 繋ぐ \textrightarrow  繋がれる & 五段 & To wait & 待つ \textrightarrow  待たれる \\ \cline{1-6}

\end{ltabulary}

\par{\textbf{Caution Note }: 五段 verbs ending in る―remember that all verbs in this category etymologically just end in -u―appear to end in ~られる. However, as their 未然形 is ら-, this is not the case. }

\par{\textbf{Usage Note }: 変わられる would only be correct Japanese if used in honorifics or compound verbs like 移り変わる. }

\begin{ltabulary}{|P|P|P|P|P|P|}
\hline 

五段 & To take & 取る \textrightarrow  取られる & 一段 & To eat & 食べる \textrightarrow  食べられる \\ \cline{1-6}

一段 & To see & 見る \textrightarrow  見られる & 五段 & To swim & 泳ぐ \textrightarrow  泳がれる \\ \cline{1-6}

カ変 & To come & 来(く)る \textrightarrow  来(こ)られる & 五段 & To buy & 買う \textrightarrow  買われる \\ \cline{1-6}

サ変 & To do & する \textrightarrow  される & 一段 & To change & 変わる \textrightarrow  変われる \\ \cline{1-6}

五段 & To carry & 運ぶ \textrightarrow  運ばれる & 一段 & To slip off \hfill\break
& 脱げる \textrightarrow  脱げられる \\ \cline{1-6}

五段 & To steal & 盗む \textrightarrow  盗まれる & 五段 & To indulge & 貪(むさぼ)る \textrightarrow  貪られる \\ \cline{1-6}

五段 & To fasten & 繋(つな)ぐ \textrightarrow  繋がれる & 五段 & To wait & 待つ \textrightarrow  待たれる \\ \cline{1-6}

\end{ltabulary}

\par{Note: 五段 verbs that end in -る seem to use -られる, but it's just that their 未然形 is ら-. }

\begin{ltabulary}{|P|P|P|P|P|P|}
\hline 

五段 & To take & 取る \textrightarrow  取られる & 一段 & To eat & 食べる \textrightarrow  食べられる \\ \cline{1-6}

一段 & To see & 見る \textrightarrow  見られる & 五段 & To swim & 泳ぐ \textrightarrow  泳がれる \\ \cline{1-6}

カ変 & To come & 来(く)る \textrightarrow  来(こ)られる & 五段 & To buy & 買う \textrightarrow  買われる \\ \cline{1-6}

サ変 & To do & する \textrightarrow  される & 一段 & To change & 変わる \textrightarrow  変われる \\ \cline{1-6}

五段 & To carry & 運ぶ \textrightarrow  運ばれる & 一段 & To slip off \hfill\break
& 脱げる \textrightarrow  脱げられる \\ \cline{1-6}

五段 & To steal & 盗む \textrightarrow  盗まれる & 五段 & To indulge & 貪(むさぼ)る \textrightarrow  貪られる \\ \cline{1-6}

五段 & To fasten & 繋(つな)ぐ \textrightarrow  繋がれる & 五段 & To wait & 待つ \textrightarrow  待たれる \\ \cline{1-6}

\end{ltabulary}

\par{Note: 五段 verbs that end in -る seem to use -られる, but it's just that their 未然形 is ら-. }

\begin{ltabulary}{|P|P|P|P|P|P|}
\hline 

五段 & To take & 取る \textrightarrow  取られる & 一段 & To eat & 食べる \textrightarrow  食べられる \\ \cline{1-6}

一段 & To see & 見る \textrightarrow  見られる & 五段 & To swim & 泳ぐ \textrightarrow  泳がれる \\ \cline{1-6}

カ変 & To come & 来(く)る \textrightarrow  来(こ)られる & 五段 & To buy & 買う \textrightarrow  買われる \\ \cline{1-6}

サ変 & To do & する \textrightarrow  される & 一段 & To change & 変わる \textrightarrow  変われる \\ \cline{1-6}

五段 & To carry & 運ぶ \textrightarrow  運ばれる & 一段 & To slip off \hfill\break
& 脱げる \textrightarrow  脱げられる \\ \cline{1-6}

五段 & To steal & 盗む \textrightarrow  盗まれる & 五段 & To indulge & 貪(むさぼ)る \textrightarrow  貪られる \\ \cline{1-6}

五段 & To fasten & 繋(つな)ぐ \textrightarrow  繋がれる & 五段 & To wait & 待つ \textrightarrow  待たれる \\ \cline{1-6}

\end{ltabulary}

\par{Note: 五段 verbs that end in -る seem to use -られる, but it's just that their 未然形 is ら-. }

\begin{ltabulary}{|P|P|P|P|P|P|}
\hline 

五段 & To take & 取る \textrightarrow  取られる & 一段 & To eat & 食べる \textrightarrow  食べられる \\ \cline{1-6}

一段 & To see & 見る \textrightarrow  見られる & 五段 & To swim & 泳ぐ \textrightarrow  泳がれる \\ \cline{1-6}

カ変 & To come & 来(く)る \textrightarrow  来(こ)られる & 五段 & To buy & 買う \textrightarrow  買われる \\ \cline{1-6}

サ変 & To do & する \textrightarrow  される & 一段 & To change & 変わる \textrightarrow  変われる \\ \cline{1-6}

五段 & To carry & 運ぶ \textrightarrow  運ばれる & 一段 & To slip off \hfill\break
& 脱げる \textrightarrow  脱げられる \\ \cline{1-6}

五段 & To steal & 盗む \textrightarrow  盗まれる & 五段 & To indulge & 貪(むさぼ)る \textrightarrow  貪られる \\ \cline{1-6}

五段 & To fasten & 繋(つな)ぐ \textrightarrow  繋がれる & 五段 & To wait & 待つ \textrightarrow  待たれる \\ \cline{1-6}

\end{ltabulary}

\par{Note: 五段 verbs that end in -る seem to use -られる, but it's just that their 未然形 is ら-. }

\begin{ltabulary}{|P|P|P|P|P|P|}
\hline 

五段 & To take & 取る \textrightarrow  取られる & 一段 & To eat & 食べる \textrightarrow  食べられる \\ \cline{1-6}

一段 & To see & 見る \textrightarrow  見られる & 五段 & To swim & 泳ぐ \textrightarrow  泳がれる \\ \cline{1-6}

カ変 & To come & 来(く)る \textrightarrow  来(こ)られる & 五段 & To buy & 買う \textrightarrow  買われる \\ \cline{1-6}

サ変 & To do & する \textrightarrow  される & 一段 & To change & 変わる \textrightarrow  変われる \\ \cline{1-6}

五段 & To carry & 運ぶ \textrightarrow  運ばれる & 一段 & To slip off \hfill\break
& 脱げる \textrightarrow  脱げられる \\ \cline{1-6}

五段 & To steal & 盗む \textrightarrow  盗まれる & 五段 & To indulge & 貪(むさぼ)る \textrightarrow  貪られる \\ \cline{1-6}

五段 & To fasten & 繋(つな)ぐ \textrightarrow  繋がれる & 五段 & To wait & 待つ \textrightarrow  待たれる \\ \cline{1-6}

\end{ltabulary}

\par{Note: 五段 verbs that end in -る seem to use -られる, but it's just that their 未然形 is ら-. }
      
\section{受身形}
 
\begin{center}
 \textbf{'Direct' Passive }
\end{center}

\par{ Basic passive sentences are like Ex. 2, deriving from a non-passive sentence like Ex. 1. However, just as how English speakers can say "the tournament will be held in Paris", there doesn't even have to necessarily be a subject with overt emotion. First, though, consider this basic example yet comical example. }

\par{${\overset{\textnormal{}}{\text{1. 犬}}}$ が ${\overset{\textnormal{くじら}}{\text{鯨}}}$ を ${\overset{\textnormal{く}}{\text{食}}}$ った。 \hfill\break
The dog \textbf{ate }the whale. }
 
\par{${\overset{\textnormal{}}{\text{2. 犬}}}$ が ${\overset{\textnormal{}}{\text{鯨}}}$ に ${\overset{\textnormal{}}{\text{食}}}$ われた。 \hfill\break
The dog \textbf{was eaten }by the whale. }
 
\par{Although odd, the sentences show the differences well. In a passive sentence there is an \textbf{action receiver and an action performer }. The action received is always there: it's the passive verb. What \emph{may or may not }be there is the receiver and or the performer. The performer (agent) is marked by に --"by". The \emph{subject is the action receiver }. In the \textbf{past tense }the subject is the "doer". The \textbf{(direct) passive voice }has the whale be the "doer" and the dog the one being eaten. }

\par{\textbf{Curriculum Note }: For instances of the agent being marked by から in the passive, click link  . }

\par{ Lastly, \textbf{によって }shows what an action was done under. It's especially used for showing when something is created, discovered, or named by someone. If there is a direct object in the sentence, you will see を too. }

\begin{center}
 \textbf{Examples }
\end{center}
 
\par{3. アリは ${\overset{\textnormal{}}{\text{弟}}}$ に ${\overset{\textnormal{く}}{\text{食}}}$ われた。 \hfill\break
The ant was eaten by my brother. }
 
\par{4. ネックレスが ${\overset{\textnormal{どろぼう}}{\text{泥棒}}}$ に ${\overset{\textnormal{}}{\text{盗}}}$ まれました。 \hfill\break
The necklace was stolen by a thief. }
 
\par{5. この ${\overset{\textnormal{}}{\text{本}}}$ は ${\overset{\textnormal{}}{\text{僕}}}$ に ${\overset{\textnormal{}}{\text{読}}}$ まれました。 \hfill\break
This book was read by me. }

\par{6. ボールがパスされ、彼女はそれを受けた。 \hfill\break
The ball was passed and she received it. }

\par{${\overset{\textnormal{}}{\text{7. 子供}}}$ は ${\overset{\textnormal{}}{\text{引}}}$ っ ${\overset{\textnormal{か}}{\text{掻}}}$ かれて、 ${\overset{\textnormal{}}{\text{泣}}}$ いた。 \hfill\break
The kid was scratched, and he cried. }
 
\par{${\overset{\textnormal{}}{\text{8. 親}}}$ にまで ${\overset{\textnormal{}}{\text{見限}}}$ られた。 \hfill\break
Even his parents have turned their backs on him. }
 
\par{9. このため ${\overset{\textnormal{}}{\text{開発}}}$ が ${\overset{\textnormal{}}{\text{急}}}$ がれている。 \hfill\break
Because of this, the development is being hurried. }
 
\par{${\overset{\textnormal{}}{\text{10. 彼女}}}$ に ${\overset{\textnormal{だま}}{\text{騙}}}$ されたよ。 \hfill\break
I got cheated by my girlfriend. }
 
\par{${\overset{\textnormal{}}{\text{11. 姿}}}$ を ${\overset{\textnormal{たにん}}{\text{他人}}}$ に ${\overset{\textnormal{}}{\text{見}}}$ られる。 \hfill\break
To have one's figure seen by others. }

\par{12. ${\overset{\textnormal{ふたり}}{\text{二人}}}$ だけ取り ${\overset{\textnormal{のこ}}{\text{残}}}$ された。 \hfill\break
The two were left all alone. }

\par{13. アメリカに住んでいるのは ${\overset{\textnormal{めぐ}}{\text{恵}}}$ まれた ${\overset{\textnormal{かんきょう}}{\text{環境}}}$ にあるということです。 \hfill\break
Living in America is in a blessed environment. }

\par{14. ${\overset{\textnormal{いっく}}{\text{一句}}}$ の ${\overset{\textnormal{うち}}{\text{内}}}$ に ${\overset{\textnormal{ばんかん}}{\text{万感}}}$ の思いが ${\overset{\textnormal{ぎょうしゅく}}{\text{凝縮}}}$ されていた。 \hfill\break
Floods of emotions were condensed into a single phrase. }
15. 彼女は ${\overset{\textnormal{やと}}{\text{雇}}}$ われて ${\overset{\textnormal{}}{\text{五}}}$ ${\overset{\textnormal{}}{\text{ヶ}}}$ ${\overset{\textnormal{}}{\text{月}}}$ もしないうちに ${\overset{\textnormal{くび}}{\text{首}}}$ になったさ。(Casual; Tokyo Dialect) \hfill\break
She hadn't been employed for five months before she got fired. 
\par{16. この ${\overset{\textnormal{とう}}{\text{塔}}}$ は ${\overset{\textnormal{}}{\text{200}}}$ ${\overset{\textnormal{}}{\text{年前}}}$ に ${\overset{\textnormal{た}}{\text{建}}}$ てられました。(Passive) \hfill\break
This tower was built two hundred years ago. }

\par{17. 一人だけ取り ${\overset{\textnormal{}}{\text{残}}}$ された孤独な学生が ${\overset{\textnormal{まいご}}{\text{迷子}}}$ になった。 \hfill\break
The left behind lonely student got lost. }

\par{18. すりに ${\overset{\textnormal{さいふ}}{\text{財布}}}$ をすられた。 \hfill\break
I got my wallet pocket picked by a pickpocket. }

\par{19. ${\overset{\textnormal{じょうし}}{\text{上司}}}$ に ${\overset{\textnormal{しか}}{\text{叱}}}$ られました。 \hfill\break
I got scolded by my boss. }

\par{20. ${\overset{\textnormal{のろ}}{\text{呪}}}$ われた ${\overset{\textnormal{}}{\text{島}}}$ だよ。 \hfill\break
It is a cursed island. }

\par{${\overset{\textnormal{}}{\text{21. 日本}}}$ の ${\overset{\textnormal{}}{\text{車}}}$ は ${\overset{\textnormal{せかいじゅう}}{\text{世界中}}}$ へ ${\overset{\textnormal{}}{\text{輸出}}}$ されています。 \hfill\break
Japanese cars are exported throughout the world. }
 
\par{22. その ${\overset{\textnormal{}}{\text{計画}}}$ はあの ${\overset{\textnormal{}}{\text{人}}}$ たちに ${\overset{\textnormal{}}{\text{思}}}$ い ${\overset{\textnormal{}}{\text{付}}}$ かれたものです。 \hfill\break
That plan is something that was thought up of by those people. }
 
\par{23. この「 ${\overset{\textnormal{}}{\text{火車}}}$ 」という ${\overset{\textnormal{}}{\text{小説}}}$ は ${\overset{\textnormal{}}{\text{宮部}}}$ によって ${\overset{\textnormal{}}{\text{書}}}$ かれました。 \hfill\break
This novel called "Kasha" was written by Miyabe. }
 
\par{24. その ${\overset{\textnormal{}}{\text{仕事}}}$ は ${\overset{\textnormal{かとう}}{\text{加藤}}}$ さんによってなされました。 \hfill\break
The work was done by Mr. Kato. }
 
\par{25. あの ${\overset{\textnormal{}}{\text{島}}}$ はコロンブスによって ${\overset{\textnormal{}}{\text{名}}}$ づけられた。 \hfill\break
That island was named by Columbus. }
 
\par{${\overset{\textnormal{}}{\text{26. 彼女}}}$ は ${\overset{\textnormal{}}{\text{皆}}}$ から ${\overset{\textnormal{}}{\text{愛}}}$ されています。 \hfill\break
She is loved by everybody. }

\par{27. ${\overset{\textnormal{じゅうたい}}{\text{渋滞}}}$ に ${\overset{\textnormal{}}{\text{巻}}}$ き ${\overset{\textnormal{}}{\text{込}}}$ まれたからだよ。 \hfill\break
(It's) because I got stuck in traffic! }
 
\par{28. この ${\overset{\textnormal{}}{\text{歌}}}$ は ${\overset{\textnormal{}}{\text{英語}}}$ で ${\overset{\textnormal{}}{\text{歌}}}$ われました。 \hfill\break
This song was sung in English. }
 
\par{${\overset{\textnormal{}}{\text{29. 道}}}$ を ${\overset{\textnormal{}}{\text{聞}}}$ かれた。 \hfill\break
I was asked the way. }

\par{30. ${\overset{\textnormal{な}}{\text{済}}}$ し ${\overset{\textnormal{}}{\text{崩}}}$ しに ${\overset{\textnormal{}}{\text{変更}}}$ される。 \hfill\break
It will be changed little by little. }
 
\par{${\overset{\textnormal{}}{\text{31. 彼}}}$ の ${\overset{\textnormal{}}{\text{名前}}}$ は ${\overset{\textnormal{}}{\text{決}}}$ して ${\overset{\textnormal{}}{\text{忘}}}$ れられないだろう。 \hfill\break
His name will never be forgotten. }

\par{${\overset{\textnormal{}}{\text{32. 昔日本}}}$ の ${\overset{\textnormal{}}{\text{家}}}$ は ${\overset{\textnormal{}}{\text{木}}}$ で ${\overset{\textnormal{つく}}{\text{造}}}$ られました。 \hfill\break
In the past Japanese homes were made with wood. }
 
\par{33. ブロードバンドルータが ${\overset{\textnormal{}}{\text{取}}}$ り ${\overset{\textnormal{}}{\text{付}}}$ けられました。 \hfill\break
The broadband router was installed. }
 
\par{${\overset{\textnormal{}}{\text{34. 保管庫}}}$ に ${\overset{\textnormal{かぎ}}{\text{鍵}}}$ がかけられた。 \hfill\break
The vault was secured with a key. }

\par{35. ${\overset{\textnormal{きん}}{\text{金}}}$ から ${\overset{\textnormal{しろ}}{\text{城}}}$ を ${\overset{\textnormal{}}{\text{作}}}$ ったといわれている。 \hfill\break
It is said that he built the castle from gold. }
 
\par{${\overset{\textnormal{}}{\text{36. 町}}}$ を ${\overset{\textnormal{}}{\text{歩}}}$ いていると ${\overset{\textnormal{}}{\text{呼}}}$ び ${\overset{\textnormal{}}{\text{止}}}$ められた。 \hfill\break
I was halted when I was walking through the town. \hfill\break
 \hfill\break
 ${\overset{\textnormal{}}{\text{37. 先生}}}$ に ${\overset{\textnormal{}}{\text{叱}}}$ られて ${\overset{\textnormal{}}{\text{泣}}}$ いている。 \hfill\break
He is crying because he was scolded by his teacher. }

\par{38. 私は ${\overset{\textnormal{}}{\text{先生}}}$ に ${\overset{\textnormal{ほ}}{\text{褒}}}$ められました。 \hfill\break
I was praised by my teacher. }
 
\par{${\overset{\textnormal{}}{\text{39. 何}}}$ の ${\overset{\textnormal{}}{\text{料金}}}$ は ${\overset{\textnormal{}}{\text{含}}}$ まれていますか。 \hfill\break
What fines are included? }
 
\par{${\overset{\textnormal{}}{\text{40. 悪口}}}$ を ${\overset{\textnormal{}}{\text{言}}}$ われてもへこたれんな!(Slang) \hfill\break
Don't be discouraged even if you're bombarded by slurs! }

\par{41. 今でも、この習慣はまだ行われています。 \hfill\break
Even now this custom is still carried out. }

\par{42. これは ${\overset{\textnormal{}}{\text{何}}}$ に ${\overset{\textnormal{}}{\text{使}}}$ われるのですか。 \hfill\break
What is this used for? }
 
\par{${\overset{\textnormal{}}{\text{43. 犬}}}$ が ${\overset{\textnormal{くさり}}{\text{鎖}}}$ で ${\overset{\textnormal{つな}}{\text{繋}}}$ がれている。 \hfill\break
The dog is chained up. }

\par{44. ${\overset{\textnormal{とうやこ}}{\text{洞爺湖}}}$ サミットは ${\overset{\textnormal{}}{\text{抗議}}}$ とともに ${\overset{\textnormal{}}{\text{行}}}$ われた。 \hfill\break
The G8 Summit in Toyako (Hokkaido) started out with protests. }
 
\par{${\overset{\textnormal{}}{\text{45. 彼}}}$ の ${\overset{\textnormal{}}{\text{理論}}}$ はでたらめな ${\overset{\textnormal{}}{\text{調査}}}$ に ${\overset{\textnormal{もと}}{\text{基}}}$ づいて ${\overset{\textnormal{}}{\text{立}}}$ てられている。 \hfill\break
His theory is based on haphazard inquiry. }

\par{46. ${\overset{\textnormal{えがお}}{\text{笑顔}}}$ に ${\overset{\textnormal{つ}}{\text{釣}}}$ られて ${\overset{\textnormal{ほほえ}}{\text{微笑}}}$ む。 \hfill\break
To grin from looking at a smiley face. }

\par{47. 大名に ${\overset{\textnormal{め}}{\text{召}}}$ し ${\overset{\textnormal{かか}}{\text{抱}}}$ えられる。 \hfill\break
To be employed as a retainer by a daimyo. }

\par{48. 彼女は男に ${\overset{\textnormal{こ}}{\text{恋}}}$ い ${\overset{\textnormal{こ}}{\text{焦}}}$ がれられる女だね。 \hfill\break
She's certainly a woman deeply loved by men, isn't she? }

\par{\textbf{WARNING Note }: 焦がれる (To yearn for) is not in the passive form. Its passive would be 焦がれられる. The verb 焦がれる can also be used as a supplementary verb in expressions such as 思い焦がれる (to pine for) and  恋(い)焦がれる (to be deeply in love with). So, be careful with these phrases as well. }

\begin{center}
 \textbf{とされる }
\end{center}

\par{  とされる shows that an idea is held by people in general. }

\par{49. ${\overset{\textnormal{いっぱん}}{\text{一般}}}$ に ${\overset{\textnormal{}}{\text{成功}}}$ は ${\overset{\textnormal{}}{\text{難}}}$ しいとされる。 \hfill\break
It is believed that success is generally difficult. }

\par{\textbf{Word Note }: There are some words that are not verbal expressions but are passive in nature. One such phrase is と ${\overset{\textnormal{おぼ}}{\text{思}}}$ しい, which is equivalent to と ${\overset{\textnormal{}}{\text{思}}}$ われる. }

\par{\textbf{Grammar Note }: The copula is not used with the passive voice. }

\par{\textbf{Contraction Note }: For passive meanings in the negative, you could see ~らんない. }
      
\section{The Potential}
 
\par{ The potential and passive verbs go hand in hand. The difference lies in context and particle usage. Even earlier forms of these endings had the same functions. However, the use of ~れる for the potential has rapidly fallen in recent times, though it is still seen in many set phrases and old-fashioned speech. }

\par{50. 越すに越されぬ大井川 \hfill\break
The Ooi River that one just can't cross }

\par{51a. 見捨てておかれようか。 \hfill\break
51b. 見捨てておけるだろうか。(More Modern) \hfill\break
Can you leave and forsake it? }

\par{52. 鬼が怖くて行かれない。 \hfill\break
The oni is scared and can't move. }
    