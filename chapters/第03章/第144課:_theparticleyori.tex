    
\chapter{The Particle より}

\begin{center}
\begin{Large}
第144課: The Particle より 
\end{Large}
\end{center}
 
\par{ In this lesson we will learn how to compare things with the particle より. }
      
\section{The Case Particle より}
 
\par{ より creates a comparison and it translates to "than". There is no change to the adjective in Japanese like there is in English with the suffix -er. As illustrated in the chart, the Japanese word order is still flexible. Notice the differences in word order between English and Japanese. }

\begin{ltabulary}{|P|P|}
\hline 

English & X is adj -er than Y. \\ \cline{1-2}

Japanese & X(は・の方が)Yより(か・も) adj. \\ \cline{1-2}

Japanese & Yより(か・も)X(の方)が adj. \\ \cline{1-2}

\end{ltabulary}

\par{\textbf{Contraction Note }: よりか can be contracted to よか in slang. }

\par{\textbf{Usage Note }: よりか is a くだけた言い方. }

\par{1. このリンゴはあのリンゴより大きい。 \hfill\break
This apple is bigger than that apple. }

\par{2. 俺はお前より背が高いぞ。(Rough male speech) \hfill\break
I'm taller than you are!  }

\par{3. 彼女は臆病者が何よりもいやだった。 \hfill\break
She hated a coward more than anything. }

\par{4. 君は誰よりも早く走ったな。(Male speech) \hfill\break
You ran faster than anyone else, didn't you. }

\par{5. 祖父は父よりゆっくり話す。 \hfill\break
My grandfather speaks more slowly than my father. }

\par{6. 俺にはあんたよりずっと多くの歌があるんだ。(Rough male speech) \hfill\break
I have much more songs than you do. }

\par{7. 彼女は僕より\{4つ・4年・4歳\}年下だ。 \hfill\break
She is 4 years younger than me. }

\par{8. 昨日よりも今日の方がずっと暑い。 \hfill\break
Today is much hotter than yesterday. \hfill\break
}

\par{9. あいつは、不良よりマヌケだぞ。(Rough male speech) \hfill\break
He's more so a fool than a rogue. }

\par{10. 薬を飲むより仕事してくれ!   (乱暴な言い方) \hfill\break
Work instead of drinking medicine! }

\par{11. 彼は僕よりお前を選んだのか。(Male speech) \hfill\break
He chose you instead of me? }

\par{12. ここに残ってるより出て行きたいの? \hfill\break
You'd rather go than stay? }

\begin{center}
 \textbf{より仕方がない }
\end{center}

\par{ より仕方がない = There is no other choice but to. }

\par{13. 「車が故障したんですが」「じゃ、歩いて行くより仕方がありませんね」 \hfill\break
"My car broke down". "Then, there's no other choice but to walk". }

\par{14. 「僕、料理は、ぜんぜんできないよ」「じゃ、自分で作るよりしょうがないね」 \hfill\break
"I can't cook at all". "Then, you have no choice but to do it yourself". }

\par{15. 「今日の宿題を忘れてきちゃった」「じゃ、早くやりなおすよりしょうがないよ」 \hfill\break
"I accidentally forgot today's homework". "Then, you have no other choice but to redo it quickly". }

\par{2. "From" in the sense of starting, distance, space, or quantity. This is a more polite version of から. }

\par{16. 四時よりも後なら結構です。 \hfill\break
It'll be good if its after 4 o' clock. }

\par{17. ホームの白線より内側でお待ちください。 \hfill\break
Please wait inside the white line on the platform. }

\par{18. 真ん中より後ろに置いてください。 \hfill\break
Put it behind the middle of it. }

\par{20. これより先は立入禁止。 \hfill\break
It's forbidden to pass by from here! }

\par{21. 学校は駅より手前にあるのです。 \hfill\break
The school is before the station. }

\par{22. 会議は五時より行います。 \hfill\break
The meeting will be held at 5 o' clock. }

\par{23. 雪は未明より降り続いています。 \hfill\break
The snow will continue to fall from dawn. }

\par{24. 私共は朝6時半より営業しております。 \hfill\break
We are open from 6:30 A.M. }

\par{25. 子どものころよりお互いに知っています。 \hfill\break
We've known each since we were children. }

\par{26. 羽田空港より出発しました。 \hfill\break
We departed at Haneda Airport. }

\par{27. 横浜港より出港しました。 \hfill\break
We departed from Port Yokohama. }

\par{28. 吉田さまよりお電話がございました。 \hfill\break
You got a telephone call from Mr. Yoshida. }

\par{3. "Verb + ~よりほかない" means "nothing else to do but\dothyp{}\dothyp{}\dothyp{}" This is a more formal rendition of simply using ~ほかない (Ex. 30). }

\par{29. どうしても電話が通じない。こうなっては、仲谷君の家へ行くよりほかない。(やや書き言葉的) \hfill\break
I can't get through on the phone no matter what. This being the case, there's nothing I can do besides going to Nakaya's house. }

\par{30. ケーブルカーが故障で動かへん。直るまで待つほかあらへんわ。(Kansai Dialect) \hfill\break
The tram is out of order due to a collision. There's nothing else to do besides wait until it's repaired. }

\par{\textbf{Dialect Note }: ~へん = ~ない. }
    