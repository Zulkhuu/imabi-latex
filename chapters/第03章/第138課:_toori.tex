    
\chapter{~とおりに}

\begin{center}
\begin{Large}
第138課: ~とおりに 
\end{Large}
\end{center}
 
\par{ In this lesson, we will learn about a phrase composed of a generic noun and the standard-marking particle に. The noun in question this time is 通り, which literally refers to a road in a town in which people and cars pass along. First, we will study more literal definitions of the word and then learn about the expression ~とおりに, the true focus of this lesson. }

\par{\textbf{Intonation Note }: と \textbf{おり }↓ }
      
\section{The Literal Definitions of 通り}
  \hfill\break
 The most basic meaning of 通り is “road” as in a road that runs through a town in which people and things (cars, carts, animals, etc.) pass along. In meaning street, it can be used as a suffix following place names. In this situation, it is pronounced as どおり. \hfill\break
 
\par{1. ${\overset{\textnormal{ぜんめん}}{\text{前面}}}$ は ${\overset{\textnormal{ほどう}}{\text{歩道}}}$ のある ${\overset{\textnormal{ひかくてきおお}}{\text{比較的大}}}$ きな ${\overset{\textnormal{とお}}{\text{通}}}$ りに ${\overset{\textnormal{めん}}{\text{面}}}$ している。 \hfill\break
The front part faces a relatively large street with a sidewalk. }
 
\par{2. ${\overset{\textnormal{ぎんざどお}}{\text{銀座通}}}$ りに ${\overset{\textnormal{ちか}}{\text{近}}}$ いので、 ${\overset{\textnormal{りようしゃ}}{\text{利用者}}}$ も ${\overset{\textnormal{おお}}{\text{多}}}$ い。 \hfill\break
Because it is close to Ginza Street, there are lots of consumers. }
 
\par{ 通り not only can mean “road,” but it can also refer to the flow of people and things trafficking said road. Thus, it is akin to “street traffic,” which isn\textquotesingle t the same “traffic” as in one is stuck in traffic. That is technically referred to as “traffic conjugation,” which translated as 渋滞. }
 
\par{3. ${\overset{\textnormal{みち}}{\text{道}}}$ が ${\overset{\textnormal{せま}}{\text{狭}}}$ く、 ${\overset{\textnormal{くるま}}{\text{車}}}$ の ${\overset{\textnormal{とお}}{\text{通}}}$ りが ${\overset{\textnormal{おお}}{\text{多}}}$ い。 \hfill\break
The street is narrow, and there is a lot of vehicle traffic. }
 
\par{ From this concept of the flowing of things, 通り by extension can also refer to the flow of liquid and gases. It can also refer to the transmission of sound and by extension the reach of someone\textquotesingle s voice. And from extension of this, the meaning of reputation also comes about. After all, one\textquotesingle s reputation is formed by the dissemination and coalescence of what others are saying. }
 
\par{4. ${\overset{\textnormal{くうき}}{\text{空気}}}$ や ${\overset{\textnormal{みず}}{\text{水}}}$ の ${\overset{\textnormal{とお}}{\text{通}}}$ りを ${\overset{\textnormal{よ}}{\text{良}}}$ くして ${\overset{\textnormal{しょくぶつ}}{\text{植物}}}$ の ${\overset{\textnormal{ね}}{\text{根}}}$ に ${\overset{\textnormal{さんそ}}{\text{酸素}}}$ を ${\overset{\textnormal{あた}}{\text{与}}}$ えるという ${\overset{\textnormal{しく}}{\text{仕組}}}$ みです。 \hfill\break
It\textquotesingle s a mechanism that gives oxygen to the plant roots by bettering the flow of air and water. }
 
\par{5. ${\overset{\textnormal{とお}}{\text{通}}}$ りの ${\overset{\textnormal{わる}}{\text{悪}}}$ い ${\overset{\textnormal{こえ}}{\text{声}}}$ で ${\overset{\textnormal{こま}}{\text{困}}}$ っています。 \hfill\break
I\textquotesingle m bothered with my voice not carrying well. }
 
\par{6. ${\overset{\textnormal{じぶん}}{\text{自分}}}$ の ${\overset{\textnormal{ぎょうかい}}{\text{業界}}}$ で ${\overset{\textnormal{とお}}{\text{通}}}$ りが ${\overset{\textnormal{よ}}{\text{良}}}$ さそうな ${\overset{\textnormal{なまえ}}{\text{名前}}}$ を ${\overset{\textnormal{つか}}{\text{使}}}$ えばいいのではないかと ${\overset{\textnormal{おも}}{\text{思}}}$ います。 \hfill\break
I think it\textquotesingle d be a good idea to choose a name that seems reputable in one\textquotesingle s industry. }
 
\par{ By further extension, the very flow of ideas that compose one\textquotesingle s understanding is yet another meaning of 通り. With understanding comes methods, which is how 通り can be used as a counter to count “method.” }
 
\par{7. ${\overset{\textnormal{いま}}{\text{今}}}$ のところ ${\overset{\textnormal{とお}}{\text{通}}}$ りの ${\overset{\textnormal{よ}}{\text{良}}}$ い ${\overset{\textnormal{せつめい}}{\text{説明}}}$ だろうか。 \hfill\break
Would it really be a sensible explanation currently? }
 
\par{8. ${\overset{\textnormal{しらが}}{\text{白髪}}}$ になるパターンは ${\overset{\textnormal{さん}}{\text{3}}}$ ${\overset{\textnormal{とお}}{\text{通}}}$ りあります。 \hfill\break
There are three patterns to turning gray. }
      
\section{~とおりに \& ~どおりに}
 
\par{ It is this last meaning of “method” that forms the semantic basis for the grammar pattern ~とおりに meaning “just as.” ~とおりに  follows after nouns or verbal expressions. }

\begin{ltabulary}{|P|P|P|}
\hline 

Part of Speech & Pattern Forms & Examples \\ \cline{1-3}

Nouns & ~のとおりに \hfill\break
~どおりに & 地図\{のとおりに・どおりに\}(according to the map) \\ \cline{1-3}

Verbs & ~るとおりに \hfill\break
~たとおりに & 言うとおりに (just as…says) \hfill\break
言ったとおりに (just as…said) \\ \cline{1-3}

\end{ltabulary}

\par{ As demonstrated, when ~とおりに attaches to verbal expressions, the tense of the verb can either be in the non-past or the past tense. However, as is also demonstrated by the chart, one mustn\textquotesingle t interchange the tenses for the other because the meaning will change accordingly. }

\par{ When ~とおりに attaches to nouns, there are two options. One can use the particle の to attach it to the noun, or one can just append とおりに to the noun with it being consequently voiced as ~どおりに, with the latter option being the most prevalent choice. Grammatically speaking, the particle の here can be viewed as substituting for a fully worded verbal expression. Using ~どおりに, by comparison, causes this semantic simplification to be understood analytically. To understand what this means, consider the following. }

\par{9a. ${\overset{\textnormal{ちず}}{\text{地図}}}$ のとおりに ${\overset{\textnormal{すす}}{\text{進}}}$ んでください。 \hfill\break
9b. 地図どおりに ${\overset{\textnormal{すす}}{\text{進}}}$ んでください。 \hfill\break
9c. 地図に ${\overset{\textnormal{か}}{\text{書}}}$ いてあるとおりに ${\overset{\textnormal{すす}}{\text{進}}}$ んでください。 \hfill\break
Please proceed according to the map. }

\par{ Grammatically speaking, all three versions of this sentence are perfectly sound, and you can hear all three versions. 9c, although the wordier version, is still just as viable and usable as the others, and this may very well be because it forms the grammatical basis for why the shorter versions are possible. How one wishes to phrase things is typically based on the flow of the given context. }

\par{ Another point that must be noted is that Ex. 9 exemplifies the basic sentence pattern X + とおりに + Y, but it\textquotesingle s important to understand that the pattern X + とおりだ is also possible and used profusely. }

\par{\textbf{Orthography Note }: Unlike many other grammatical patterns, this one can be just as easily written in \emph{Kanji }as it can in \emph{Hiragana }. }

\par{\textbf{Translation Note }: ~\{とおり・どおり\}に is most often translated as “just as\slash according to.” It implies that one is doing X a certain way 100\%. }

\par{10. ${\overset{\textnormal{ぼく}}{\text{僕}}}$ \{が・の\} ${\overset{\textnormal{そうぞう}}{\text{想像}}}$ したとおりだ。 \hfill\break
It\textquotesingle s just as I had imagined. }

\par{11. ${\overset{\textnormal{じじつ}}{\text{事実}}}$ 、 ${\overset{\textnormal{ぼく}}{\text{僕}}}$ の ${\overset{\textnormal{そうぞうどお}}{\text{想像通}}}$ りだ。 \hfill\break
As a matter of fact, it is just as I imagined. }

\par{12. ${\overset{\textnormal{みち}}{\text{道}}}$ はまさしく ${\overset{\textnormal{ちず}}{\text{地図}}}$ どおりに ${\overset{\textnormal{つな}}{\text{繋}}}$ がっていた。 \hfill\break
The route was evidently connected just like the map. \hfill\break
 \hfill\break
13. ${\overset{\textnormal{とりあつかいせつめいしょ}}{\text{取扱説明書}}}$ どおりに ${\overset{\textnormal{せっち}}{\text{設置}}}$ したのに、 ${\overset{\textnormal{せつぞく}}{\text{接続}}}$ されません。 \hfill\break
It won\textquotesingle t connect even though I set it up just like how it was written in the user\textquotesingle s manual. \hfill\break
 \hfill\break
14. ${\overset{\textnormal{い}}{\text{言}}}$ われたとおりにお ${\overset{\textnormal{はし}}{\text{箸}}}$ で ${\overset{\textnormal{た}}{\text{食}}}$ べてみました。 \hfill\break
I tried eating it with chopsticks just as I was told. }

\par{15. ${\overset{\textnormal{わたし}}{\text{私}}}$ \{が・の\} ${\overset{\textnormal{い}}{\text{言}}}$ うとおりに ${\overset{\textnormal{うご}}{\text{動}}}$ いてください。 \hfill\break
Move as I say. }

\par{16. ${\overset{\textnormal{わたし}}{\text{私}}}$ \{が・の\} ${\overset{\textnormal{い}}{\text{言}}}$ ったとおりに ${\overset{\textnormal{うご}}{\text{動}}}$ いてください。 \hfill\break
Move as I said. }

\par{17. ${\overset{\textnormal{わたし}}{\text{私}}}$ の ${\overset{\textnormal{い}}{\text{言}}}$ ったとおりになったでしょう。 \hfill\break
It happened just as I said, right? }

\par{18. ${\overset{\textnormal{ぜんじつ}}{\text{前日}}}$ の ${\overset{\textnormal{てんきよほう}}{\text{天気予報}}}$ のとおりに ${\overset{\textnormal{あめもよう}}{\text{雨模様}}}$ 。 \hfill\break
Signs of rain just like the weather report the other day said there would be. }

\par{19. ${\overset{\textnormal{ごご}}{\text{午後}}}$ からは、 ${\overset{\textnormal{てんきよほう}}{\text{天気予報}}}$ どおりに ${\overset{\textnormal{あめ}}{\text{雨}}}$ が ${\overset{\textnormal{ふ}}{\text{降}}}$ ってきました。 \hfill\break
Just like the weather report said, the rain came starting in the afternoon. }

\par{20. なぜ、 ${\overset{\textnormal{しごと}}{\text{仕事}}}$ が ${\overset{\textnormal{よてい}}{\text{予定}}}$ どおりに ${\overset{\textnormal{お}}{\text{終}}}$ わらないのか。 \hfill\break
Why doesn\textquotesingle t work end like planned? }

\par{21. ${\overset{\textnormal{せんせい}}{\text{先生}}}$ が ${\overset{\textnormal{おし}}{\text{教}}}$ えてくださった ${\overset{\textnormal{とお}}{\text{通}}}$ りにやったら ${\overset{\textnormal{と}}{\text{解}}}$ けました。 \hfill\break
When I did it just how my teacher taught me, I was able to solve it. }

\par{22. ${\overset{\textnormal{}}{\text{本案件}}}$ の ${\overset{\textnormal{せいかぶつ}}{\text{成果物}}}$ は、 ${\overset{\textnormal{きげん}}{\text{期限}}}$ どおりに ${\overset{\textnormal{のうひん}}{\text{納品}}}$ された。 \hfill\break
The results of this project were delivered in due course. }

\par{23. ${\overset{\textnormal{じぶん}}{\text{自分}}}$ の ${\overset{\textnormal{つごう}}{\text{都合}}}$ で ${\overset{\textnormal{あいて}}{\text{相手}}}$ に ${\overset{\textnormal{きたい}}{\text{期待}}}$ し、 ${\overset{\textnormal{じぶん}}{\text{自分}}}$ の ${\overset{\textnormal{おも}}{\text{思}}}$ い ${\overset{\textnormal{どお}}{\text{通}}}$ りにならないと ${\overset{\textnormal{はら}}{\text{腹}}}$ を ${\overset{\textnormal{た}}{\text{立}}}$ てる。 \hfill\break
To get angry when one expects of someone on one\textquotesingle s own circumstances but things don\textquotesingle t turn the way one thought. }

\par{24. ${\overset{\textnormal{しんりがく}}{\text{心理学}}}$ の ${\overset{\textnormal{りろん}}{\text{理論}}}$ \textbf{どおりに }${\overset{\textnormal{ひと}}{\text{人}}}$ が ${\overset{\textnormal{うご}}{\text{動}}}$ くとは ${\overset{\textnormal{かぎ}}{\text{限}}}$ らないし、 ${\overset{\textnormal{けいざいがく}}{\text{経済学}}}$ の ${\overset{\textnormal{りろん}}{\text{理論}}}$ \textbf{どおりに }${\overset{\textnormal{けいざい}}{\text{経済}}}$ が ${\overset{\textnormal{うご}}{\text{動}}}$ くとは ${\overset{\textnormal{かぎ}}{\text{限}}}$ らない。 \hfill\break
It isn\textquotesingle t always true that people will operate according to the theories of psychology, and it isn\textquotesingle t always true that the economy will operate according to the theories of economics. }

\par{25. ${\overset{\textnormal{きおく}}{\text{記憶}}}$ がある ${\overset{\textnormal{じじつ}}{\text{事実}}}$ は、 ${\overset{\textnormal{きおく}}{\text{記憶}}}$ \textbf{どおりに }${\overset{\textnormal{こた}}{\text{答}}}$ えます。 \hfill\break
For facts you have memory of, answer \textbf{exactly according to }your memory. }

\par{26. ${\overset{\textnormal{おし}}{\text{教}}}$ えたとおりにやりなさい。 \hfill\break
Do it exactly how I\textquotesingle ve taught you. }

\par{27. ${\overset{\textnormal{かみさま}}{\text{神様}}}$ は ${\overset{\textnormal{かなら}}{\text{必}}}$ ずしも ${\overset{\textnormal{せいしょ}}{\text{聖書}}}$ どおりに ${\overset{\textnormal{こうどう}}{\text{行動}}}$ するわけではない。 \hfill\break
It is not the case that God always acts exactly according to the Bible. }

\par{28. ${\overset{\textnormal{じんせい}}{\text{人生}}}$ は ${\overset{\textnormal{おも}}{\text{思}}}$ い ${\overset{\textnormal{どお}}{\text{通}}}$ りに、 ${\overset{\textnormal{よそうつう}}{\text{予想通}}}$ りにいくことばかりではないのだ。 \hfill\break
Life isn\textquotesingle t full of things going as one wants or expects. }

\par{29. ${\overset{\textnormal{まった}}{\text{全}}}$ く ${\overset{\textnormal{おっしゃ}}{\text{仰}}}$ る ${\overset{\textnormal{とお}}{\text{通}}}$ りです。 \hfill\break
You\textquotesingle re absolutely right. }

\par{30. その ${\overset{\textnormal{とお}}{\text{通}}}$ りだ。 \hfill\break
That\textquotesingle s exactly right. }

\par{31. ${\overset{\textnormal{あす}}{\text{明日}}}$ から ${\overset{\textnormal{もとどお}}{\text{元通}}}$ りだね。 \hfill\break
Things will be back to normal starting tomorrow, huh. }

\par{32. \{ ${\overset{\textnormal{いか}}{\text{以下}}}$ ・ ${\overset{\textnormal{かき}}{\text{下記}}}$ \}の ${\overset{\textnormal{とお}}{\text{通}}}$ りです。 \hfill\break
(It) is as follows: }

\par{33. ${\overset{\textnormal{わたし}}{\text{私}}}$ はいつも ${\overset{\textnormal{どお}}{\text{通}}}$ りです。 \hfill\break
Same as usual. }

\par{34. ${\overset{\textnormal{わたし}}{\text{私}}}$ にできる ${\overset{\textnormal{かぎ}}{\text{限}}}$ り、 ${\overset{\textnormal{ぶちょう}}{\text{部長}}}$ のご ${\overset{\textnormal{しじどお}}{\text{指示通}}}$ りに ${\overset{\textnormal{おこな}}{\text{行}}}$ いました。 \hfill\break
I carried it out according to the chief\textquotesingle s instructions to the best of my ability. }

\par{35. スケジュール ${\overset{\textnormal{どお}}{\text{通}}}$ りに ${\overset{\textnormal{しごと}}{\text{仕事}}}$ を ${\overset{\textnormal{すす}}{\text{進}}}$ めていきたいと ${\overset{\textnormal{おも}}{\text{思}}}$ います。 \hfill\break
I would like to proceed with the work according to the schedule. }

\begin{center}
\textbf{Omission of に }
\end{center}

\par{ The に after とおり・どおり is the standard-marking に. Unsurprisingly, because the noun itself expresses adherence to a standard by 100\%, it is possible to delete it with no effect to the grammaticality of the sentence. However, it is more often used than not. }

\par{ In fact, it\textquotesingle s quite unnatural to drop に when using ~のとおりに. One motivation for this is that the に used in this expression is also adverbial in nature. It is this adverbial nature that makes it grammatical redundant and thus omissible in expressions like いつもどおり (Ex. 36). }

\par{ There are also situations where omitting に brings about a more formal tone, similarly to how when it is omitted in other expressions such as ~ために (Ex. 37). }

\par{36. ${\overset{\textnormal{きゅうじつ}}{\text{休日}}}$ でもいつもどおり(に) ${\overset{\textnormal{お}}{\text{起}}}$ きてしまう。 \hfill\break
I end up waking up as usual even on weekends. }

\par{37. ${\overset{\textnormal{じぶん}}{\text{自分}}}$ が ${\overset{\textnormal{き}}{\text{決}}}$ めたことを ${\overset{\textnormal{けいかくどお}}{\text{計画通}}}$ り(に) ${\overset{\textnormal{じっこう}}{\text{実行}}}$ できれば、なんでもやれる。 \hfill\break
You can do anything if you\textquotesingle re able to carry out according to plan what you\textquotesingle ve decided to do. }

\par{38. イギリスの ${\overset{\textnormal{ことわざ}}{\text{諺}}}$ にもある ${\overset{\textnormal{とお}}{\text{通}}}$ り(に)、 ${\overset{\textnormal{じぶん}}{\text{自分}}}$ の ${\overset{\textnormal{くにいがい}}{\text{国以外}}}$ の ${\overset{\textnormal{ほか}}{\text{他}}}$ のことを ${\overset{\textnormal{し}}{\text{知}}}$ ることで、 ${\overset{\textnormal{じぶん}}{\text{自分}}}$ の ${\overset{\textnormal{とくちょう}}{\text{特徴}}}$ がより ${\overset{\textnormal{ふか}}{\text{深}}}$ く ${\overset{\textnormal{りかい}}{\text{理解}}}$ できるのだ。 \hfill\break
 \textbf{Just as is }also in the English proverb, one can more deeply understand one\textquotesingle s own characteristics by knowing other things outside one\textquotesingle s own country. \textbf{}}

\begin{center}
\textbf{~\{とおり・どおり\}の }
\end{center}

\par{ With X + とおりだ being a valid sentence pattern, it is only logical that it can be used to directly modify a noun phrase, and in doing so, one gets ~\{とおり・どおり\}の. }

\par{39. ${\overset{\textnormal{きぼうどお}}{\text{希望通}}}$ りの ${\overset{\textnormal{しょくしゅ}}{\text{職種}}}$ に ${\overset{\textnormal{つ}}{\text{就}}}$ きながらも、 ${\overset{\textnormal{ふまん}}{\text{不満}}}$ を ${\overset{\textnormal{かん}}{\text{感}}}$ じている ${\overset{\textnormal{わかてしゃいん}}{\text{若手社員}}}$ は ${\overset{\textnormal{いがい}}{\text{意外}}}$ に ${\overset{\textnormal{すく}}{\text{少}}}$ なくない。 \hfill\break
The number of young employees who feel frustrated despite being employed in the occupation they hoped for is surprisingly not few. }

\par{40. ${\overset{\textnormal{ひょうばん}}{\text{評判}}}$ \textbf{どおりの }${\overset{\textnormal{よ}}{\text{良}}}$ いホテルだった。 \hfill\break
It was a good hotel that \textbf{lived up to }its reputation. }

\begin{center}
\textbf{~てのとおり }
\end{center}

\par{ Interestingly, there are two expressions in which ~のとおり follows verbs in the gerund form made with the particle て. These phrases are 見ての通り (as you can see) and 知っての通り (as you know). Grammatically, the particle の can be viewed as taking the place of some other phrase. \hfill\break
 \hfill\break
 As for 見ての通り, one such “other phrase” that would fit perfectly here is わかる. In fact, 見てわかる通り is another way of saying “as you can see.” As for 知っての通り, の can be thought of as a substitute for いる. }

\par{ It\textquotesingle s important to understand that 見る and 知る can still be used just like other verbs in other forms provided the semantic combination of said form with 通り makes sense. Such examples will be sprinkled in with examples of ~てのとおり below. }

\par{\textbf{Particle Note }: Partly due to being set phrases, neither 見ての通り nor  知っての通り are followed by the particle に. }

\par{43. ${\overset{\textnormal{み}}{\text{見}}}$ ての ${\overset{\textnormal{とお}}{\text{通}}}$ り、 ${\overset{\textnormal{はな}}{\text{話}}}$ すのは ${\overset{\textnormal{とくい}}{\text{得意}}}$ じゃなくてね。 \hfill\break
As you can see, I\textquotesingle m not good at speaking. }

\par{44. ${\overset{\textnormal{み}}{\text{見}}}$ てわかる ${\overset{\textnormal{とお}}{\text{通}}}$ り、 ${\overset{\textnormal{まいにちひま}}{\text{毎日暇}}}$ です。 \hfill\break
As you can see, I\textquotesingle m free with my time every day. }

\par{45. グラフ\{に ${\overset{\textnormal{み}}{\text{見}}}$ る・で ${\overset{\textnormal{わ}}{\text{分}}}$ かる・の\} ${\overset{\textnormal{どお}}{\text{通}}}$ りに、 ${\overset{\textnormal{じゅうご}}{\text{15}}}$ ${\overset{\textnormal{ねんかん}}{\text{年間}}}$ の ${\overset{\textnormal{ひりつ}}{\text{比率}}}$ は ${\overset{\textnormal{よん}}{\text{4}}}$ . ${\overset{\textnormal{ご}}{\text{5}}}$ だった。 \hfill\break
[As can be seen in the graph\slash as can be understood with the graph\slash exactly as the graph says], the fifteen-year ratio was 4.5. }

\par{46. ${\overset{\textnormal{わたし}}{\text{私}}}$ が ${\overset{\textnormal{てほん}}{\text{手本}}}$ を ${\overset{\textnormal{しめ}}{\text{示}}}$ すから、 ${\overset{\textnormal{み}}{\text{見}}}$ たとおりにやってみてください。 \hfill\break
I\textquotesingle m going to show an example, so try doing it just as you see it. }

\par{47. ${\overset{\textnormal{し}}{\text{知}}}$ っての ${\overset{\textnormal{とお}}{\text{通}}}$ り、 ${\overset{\textnormal{エッチピー}}{\text{HP}}}$ ・ ${\overset{\textnormal{こうげき}}{\text{攻撃}}}$ ・ ${\overset{\textnormal{ぼうぎょ}}{\text{防御}}}$ は ${\overset{\textnormal{いじょう}}{\text{異常}}}$ に ${\overset{\textnormal{たか}}{\text{高}}}$ いのです。 \hfill\break
As you know, its HP, attack, and defense are abnormally high. }

\par{48. ご ${\overset{\textnormal{ぞんち}}{\text{存知}}}$ の ${\overset{\textnormal{とお}}{\text{通}}}$ り、 ${\overset{\textnormal{ひ}}{\text{冷}}}$ えは ${\overset{\textnormal{しんたい}}{\text{身体}}}$ に ${\overset{\textnormal{よ}}{\text{良}}}$ くありません。 \hfill\break
As you know, chilliness is not good for the body. }

\par{\textbf{Variant Note }: ご存知の通り, alternatively also seen as ご存じの通り, is the honorific form of 知っての通り, and it is actually more frequently used due to it being politer as 知っての通り can often sound too direct. }

\par{49. ${\overset{\textnormal{そと}}{\text{外}}}$ は、ご ${\overset{\textnormal{らん}}{\text{覧}}}$ の ${\overset{\textnormal{とお}}{\text{通}}}$ りの ${\overset{\textnormal{ちょうだ}}{\text{長蛇}}}$ の ${\overset{\textnormal{れつ}}{\text{列}}}$ です。 \hfill\break
Outside, there\textquotesingle s a long line, as you can see. }

\par{\textbf{Variant Note }: ご覧の通り is the honorific form of 見ての通り, and is arguably used more than its non-honorific form. As Ex. 49 demonstrates, ご覧の通り can even be used to modify noun phrases with the help of the particle の. The same thing can be said for ご\{存知・存じ\}の通り, but not for either phrase\textquotesingle s non-honorific form. }

\par{50. ${\overset{\textnormal{みなし}}{\text{皆知}}}$ っている ${\overset{\textnormal{とお}}{\text{通}}}$ り、 ${\overset{\textnormal{こきゃく}}{\text{顧客}}}$ なしではビジネスは ${\overset{\textnormal{な}}{\text{成}}}$ り ${\overset{\textnormal{た}}{\text{立}}}$ たない。 \hfill\break
As everyone knows, business is not viable without customers. }

\par{51. ${\overset{\textnormal{だれ}}{\text{誰}}}$ も ${\overset{\textnormal{し}}{\text{知}}}$ る ${\overset{\textnormal{とお}}{\text{通}}}$ り、 ${\overset{\textnormal{かみ}}{\text{紙}}}$ は ${\overset{\textnormal{おも}}{\text{主}}}$ に ${\overset{\textnormal{じゅし}}{\text{樹脂}}}$ からとられる。 \hfill\break
As known by everyone, paper is largely taken from resin. }

\begin{center}
\textbf{~ように } \hfill\break

\end{center}

\par{ As an aside, it is worth noting that ~とおりに is similar to ~ように, with the latter being translatable as “as.” Just as ~とおりに establishes some standard X by which the agent is to act 100\% accordingly to, ~ように can also establish a standard X by which the agent is to act accordingly to, but the expectation is not set at 100\%. }

\par{52. ${\overset{\textnormal{せつめい}}{\text{説明}}}$ したように ${\overset{\textnormal{てきとう}}{\text{適当}}}$ にしてください。 \hfill\break
Do as you see fit as explained. }

\par{53. ${\overset{\textnormal{まえ}}{\text{前}}}$ にも ${\overset{\textnormal{い}}{\text{言}}}$ ったように、この ${\overset{\textnormal{あんけん}}{\text{案件}}}$ は ${\overset{\textnormal{らいしゅう}}{\text{来週}}}$ の ${\overset{\textnormal{きんようび}}{\text{金曜日}}}$ に ${\overset{\textnormal{お}}{\text{終}}}$ わる ${\overset{\textnormal{よてい}}{\text{予定}}}$ です。 \hfill\break
As was said earlier, this project is scheduled to end on Friday of next week. }

\par{54. よくも ${\overset{\textnormal{わる}}{\text{悪}}}$ くも、 ${\overset{\textnormal{そうぞう}}{\text{想像}}}$ したようにはまったくいかないな。 \hfill\break
Good or bad, it just doesn\textquotesingle t go at all like I imagine. \hfill\break
 \hfill\break
55. ご ${\overset{\textnormal{らん}}{\text{覧}}}$ のように、 ${\overset{\textnormal{ぐ}}{\text{具}}}$ と ${\overset{\textnormal{めん}}{\text{麺}}}$ が ${\overset{\textnormal{べつべつ}}{\text{別々}}}$ に ${\overset{\textnormal{で}}{\text{出}}}$ てきます。 \hfill\break
As you can see, the noodles and ingredients come out separately. }

\par{\textbf{Phrase Note }: Due to it being more indirect, ご覧のように is somewhat politer than ご覧の通り. }
    