    
\chapter{What \& When}

\begin{center}
\begin{Large}
第129課: What \& When 
\end{Large}
\end{center}
 
\par{ In this lesson, we will take a second yet closer look at the words for “what” and “when.” This time, we will look at how to express these words outside of polite speech, in which case some variation will have to be taken into consideration. }
      
\section{What}
 
\par{ When looking up “what” in plain speech, most people will be find that the expression is なんだ. As you can see, なに becomes なん. This is because \slash ni\slash  becomes \slash n\slash  to make pronunciation easier. This causes some problems, but for now, let\textquotesingle s see how なんだ is used. }

\par{1. ${\overset{\textnormal{なん}}{\text{何}}}$ だ? \hfill\break
What (do you want\slash is it)? }

\par{\textbf{Sentence Note }: Ex. 1 would most likely be said by a male speaker. All by itself, なんだ shows irritation at someone. }

\par{2. ${\overset{\textnormal{なん}}{\text{何}}}$ だよ! \hfill\break
What the heck! }

\par{\textbf{Sentence Note }: Ex. 2 also shows irritation, which is amplified with the use of the particle よ. With that being the case, this isn\textquotesingle t a literal question. }

\par{ Of course, we already know that なん is the form you use in polite speech. This is simply because です also starts with \slash d\slash . }

\par{3. ${\overset{\textnormal{しゅみ}}{\text{趣味}}}$ は ${\overset{\textnormal{なん}}{\text{何}}}$ ですか。 \hfill\break
What are your hobbies? }

\par{4. お ${\overset{\textnormal{しごと}}{\text{仕事}}}$ は ${\overset{\textnormal{なん}}{\text{何}}}$ ですか。 \hfill\break
What is your job? }

\par{5. ${\overset{\textnormal{かみ}}{\text{神}}}$ の ${\overset{\textnormal{おうこく}}{\text{王国}}}$ とは ${\overset{\textnormal{なん}}{\text{何}}}$ ですか。 \hfill\break
What is “God\textquotesingle s kingdom”? }

\par{\textbf{Grammar Note }: The use of とは is mean to seek a definition of what precedes it. }

\par{ When not used in isolation, なんだ isn\textquotesingle t limited to irritated responses. Rather, the question tends to be philosophical. They also tend to be more commonly stated this way in the written language, but you can imagine sentences like Ex. 4 being spoken in slightly dramatic soliloquies. }

\par{6. ${\overset{\textnormal{にんげん}}{\text{人間}}}$ とは ${\overset{\textnormal{なん}}{\text{何}}}$ だ。 \hfill\break
What is \emph{mankind }? }

\par{ Another neat phrase that utilizes なん instead of なに is なんぞや. This utilizes pretty old grammar, which means most people typically only use it when they\textquotesingle re purposely trying to sound old-fashioned, but it can also be seen in things like textbooks to draw attention to a topic. For instance, if you see a heading that says “What is biochemical engineering?” you might see this used. }

\par{7. ${\overset{\textnormal{われ}}{\text{我}}}$ とは ${\overset{\textnormal{なん}}{\text{何}}}$ ぞや。 \hfill\break
What am I? }

\par{\textbf{Word Note }: Keep in mind that 我 is the original word meaning “I” in Japanese and is still occasionally used in purposely old-fashioned expressions such as in Ex. 7. }

\par{ In isolation, 何 is how “what” is usually expressed in casual expressions. Kids and female speakers tend to drag out the \slash a\slash , resulting in な~に, but this isn\textquotesingle t common in male speech. なんなの, however, tends to show up as well. This adds more emphasis to getting an explanation for “what” something is. }

\par{ Of course, as a regular noun that can take on any case particles, you use なに. At the end of a sentence, it is rarely followed by the particle か. When it is, the question sounds as if it is a part of narration or the title of some discussion in some form of presentation\slash writing. }

\par{8. お ${\overset{\textnormal{みやげ}}{\text{土産}}}$ は ${\overset{\textnormal{なに}}{\text{何}}}$ がいいですか。 \hfill\break
What would be good for souvenirs? }

\par{9. あれって ${\overset{\textnormal{なに}}{\text{何}}}$ ? \hfill\break
What is that? }

\par{10. ${\overset{\textnormal{つなみ}}{\text{津波}}}$ とは ${\overset{\textnormal{なに}}{\text{何}}}$ か。 \hfill\break
What is a tsunami? }

\par{ With some particles, なに may emphatically alternatively become なんに with certain particles, particularly も. However, it\textquotesingle s also important to note that the combination にでも causes なに to become なん most of the time (as seen in Ex. 12). }

\par{11. いや、な(ん)にもない。 \hfill\break
Oh, no, it\textquotesingle s nothing. }

\par{12. ${\overset{\textnormal{なん}}{\text{何}}}$ にでも ${\overset{\textnormal{な}}{\text{成}}}$ れるよ。 \hfill\break
You can become anything. }

\par{ In compounds, なん and なに are used in fundamentally different situations. なん is used with counter phrases to mean “how many…” なに is used to mean “what kind…” There is one exception in particular that must be noted, which is何曜日 (what day of the week?). Although its traditional reading is なにようび, it is most frequently pronounced asなんようび. This is because most speakers find this reading easier to pronounce. Now, let\textquotesingle s return to the main difference between these two readings with the following examples. }

\par{13. ${\overset{\textnormal{ぜんぶ}}{\text{全部}}}$ で ${\overset{\textnormal{なんしょく}}{\text{何色}}}$ ありますか。 \hfill\break
In total, how many colors are there? }

\par{14. デンマークの ${\overset{\textnormal{こっき}}{\text{国旗}}}$ は、 ${\overset{\textnormal{2}}{\text{2}}}$ ${\overset{\textnormal{しょくつか}}{\text{色使}}}$ われている。 \hfill\break
As for the national flag of Denmark, two colors are used. }

\par{15. ${\overset{\textnormal{なにいろ}}{\text{何色}}}$ のペンキを ${\overset{\textnormal{か}}{\text{買}}}$ ったらいいですか。 \hfill\break
What color paint should I buy? }

\par{16. ${\overset{\textnormal{なにいろ}}{\text{何色}}}$ に ${\overset{\textnormal{み}}{\text{見}}}$ えますか。 \hfill\break
What color does it look like? }

\par{17. ${\overset{\textnormal{にほん}}{\text{日本}}}$ には ${\overset{\textnormal{ぜんぶ}}{\text{全部}}}$ で ${\overset{\textnormal{なんけん}}{\text{何県}}}$ ありますか。 \hfill\break
In total, how many prefectures are there in Japan? }

\par{\textbf{Phrase Note }: Most people will answer this question by giving the number of prefectures that are actual 県, not those that are 都道府. }

\par{18. ${\overset{\textnormal{たけしま}}{\text{竹島}}}$ は ${\overset{\textnormal{なにけん}}{\text{何県}}}$ にありますか。 \hfill\break
What prefecture is Takeshima in? }

\par{19. ${\overset{\textnormal{なにぶ}}{\text{何部}}}$ に ${\overset{\textnormal{しょぞく}}{\text{所属}}}$ してるの? \hfill\break
What club\slash department do you belong to? }

\par{20. ${\overset{\textnormal{なんぶ}}{\text{何部}}}$ くらい ${\overset{\textnormal{つく}}{\text{作}}}$ ればいいですか。 \hfill\break
About how many copies should I make? }

\par{21. ${\overset{\textnormal{かれ}}{\text{彼}}}$ は ${\overset{\textnormal{なにじん}}{\text{何人}}}$ ですか。 \hfill\break
What nationality is he? }

\par{22. ${\overset{\textnormal{きょうだい}}{\text{兄弟}}}$ は ${\overset{\textnormal{なんにん}}{\text{何人}}}$ いますか。 \hfill\break
How many siblings do you have? }

\begin{center}
\textbf{なにで vs なんで }
\end{center}

\par{ One rather difficult challenge presented by “what” that confuses students is the difference between なにで and なんで. The use of the particle で here is used to show means\slash method\slash composition. In this sense, なにで is almost always the reading used. }

\par{23. ${\overset{\textnormal{くちべに}}{\text{口紅}}}$ の ${\overset{\textnormal{いろ}}{\text{色}}}$ はなにで ${\overset{\textnormal{き}}{\text{決}}}$ まる? \hfill\break
What determines the color of lipstick? }

\par{24. このジュースってなにで ${\overset{\textnormal{つく}}{\text{作}}}$ ったの? \hfill\break
What did you make this juice with? }

\par{25. ${\overset{\textnormal{つめ}}{\text{爪}}}$ は ${\overset{\textnormal{なに}}{\text{何}}}$ で ${\overset{\textnormal{でき}}{\text{出来}}}$ ているの? \hfill\break
What are nails made of? }

\par{26. ${\overset{\textnormal{にゅうし}}{\text{入試}}}$ に ${\overset{\textnormal{で}}{\text{出}}}$ る ${\overset{\textnormal{かんじ}}{\text{漢字}}}$ は、なにで ${\overset{\textnormal{べんきょう}}{\text{勉強}}}$ すればいいですか。 \hfill\break
What should I use to study with for the Kanji that appear in the entrance exam? }

\par{27. ${\overset{\textnormal{こうぶん}}{\text{構文}}}$ は ${\overset{\textnormal{なに}}{\text{何}}}$ で ${\overset{\textnormal{べんきょう}}{\text{勉強}}}$ すればいいんでしょうか。 \hfill\break
What should I use to study with for sentence structure? }

\par{\textbf{Pronunciation Note }: なにで may alternatively be replaced with なんで. However, most speakers avoid this as なんで typically means "why?" Although なんで seldom replaces なにで, it does occasionally happen in contexts regarding transportation. }

\par{28. }

\par{${\overset{\textnormal{さかい}}{\text{坂井}}}$ :「 ${\overset{\textnormal{さだむら}}{\text{定村}}}$ さん、いつ ${\overset{\textnormal{よこはまししゃ}}{\text{横浜支社}}}$ に ${\overset{\textnormal{い}}{\text{行}}}$ きますか。」 \hfill\break
 ${\overset{\textnormal{さだむら}}{\text{定村}}}$ :「 ${\overset{\textnormal{すいようび}}{\text{水曜日}}}$ に ${\overset{\textnormal{い}}{\text{行}}}$ きます。」 \hfill\break
 ${\overset{\textnormal{さかい}}{\text{坂井}}}$ :「\{なにで・なんで\} ${\overset{\textnormal{い}}{\text{行}}}$ きますか。」 \hfill\break
 ${\overset{\textnormal{さだむら}}{\text{定村}}}$ :「 ${\overset{\textnormal{ひこうき}}{\text{飛行機}}}$ で行きます。」 \hfill\break
Sakai: Mr. Sadamura, when will you be going to the Yokohama office? \hfill\break
Sadamura: I\textquotesingle m going on Wednesday. \hfill\break
Sakai: [By what means\slash how] will you be going? \hfill\break
Sadamura: I\textquotesingle m going by plane. \hfill\break
 \hfill\break
 As for other means to say “how,” there is a caveat to using なにで over the usual どうやって or some other expression. As stated above, it simply asks by what means someone travels. The answer shouldn\textquotesingle t describe manner. }

\par{29. なにで ${\overset{\textnormal{き}}{\text{来}}}$ たの? \hfill\break
How\textquotesingle d you get here? }

\par{30. どうやって ${\overset{\textnormal{き}}{\text{来}}}$ たの? \hfill\break
How did you come? }

\par{\textbf{Sentence Note }. In Ex. 30, the question is open-ended enough for the listener to respond with something like “by camouflaging myself,” which would be an inappropriate response to Ex. 31. }

\par{31. なにを ${\overset{\textnormal{つか}}{\text{使}}}$ って ${\overset{\textnormal{き}}{\text{来}}}$ たの? \hfill\break
What did you use to come here? }

\par{\textbf{Sentence Note }: In Ex. 31, the question is out-of-place as a typical question one would ask in Japanese, but if you were to ask this to someone, you would inevitably get a smart-alecky reply on the lines of “by using my legs.” }

\par{ In the same vein of thought, even when a verb primarily used for movement is used in a different sense, なにで・なんで can be seen, again, with なにで being most preferred. }

\par{32. さて、何で ${\overset{\textnormal{い}}{\text{行}}}$ きますか。 \hfill\break
Alright, what will we go with? }
      
\section{When}
 
\par{ The three expressions that you will need to pay most attention to not confuse are いつ (when?), 何時 (what time?), and 何時間 (how many hours?). }

\par{33. いつ ${\overset{\textnormal{ね}}{\text{寝}}}$ ますか。 \hfill\break
When do you go to sleep? }

\par{34. ${\overset{\textnormal{なんじかんね}}{\text{何時間寝}}}$ ますか。 \hfill\break
How many hours do you sleep? }

\par{35. ${\overset{\textnormal{なんじ}}{\text{何時}}}$ に ${\overset{\textnormal{ね}}{\text{寝}}}$ ますか。 \hfill\break
What time do you go to sleep? }

\par{36. ${\overset{\textnormal{かいぎ}}{\text{会議}}}$ はいつ ${\overset{\textnormal{お}}{\text{終}}}$ わりますか。 \hfill\break
When does the meeting end? }

\par{37. ${\overset{\textnormal{きょう}}{\text{今日}}}$ は ${\overset{\textnormal{なんじ}}{\text{何時}}}$ に ${\overset{\textnormal{かえ}}{\text{帰}}}$ るの? \hfill\break
What time will you return home today? }

\par{38. ${\overset{\textnormal{じんるい}}{\text{人類}}}$ が ${\overset{\textnormal{かせい}}{\text{火星}}}$ に ${\overset{\textnormal{ぎょう}}{\text{行}}}$ けるようになるのはいつだろうか。 \hfill\break
When will mankind become able to go to Mars, I wonder? \hfill\break
 \hfill\break
39. いつの ${\overset{\textnormal{ま}}{\text{間}}}$ にか ${\overset{\textnormal{ねむ}}{\text{眠}}}$ り ${\overset{\textnormal{こ}}{\text{込}}}$ んでいた。 \hfill\break
I had fallen asleep before I knew it. }

\par{\textbf{Phrase Note }: いつの間にか is a set phrase meaning “before one knows it.” }

\par{ Aside from these three basic expressions, there are also the phrases いつ頃 (about when?) and いつなんどき (at any moment). As you can see, the latter comes from an emphatic version of いつ, which isn\textquotesingle t really used in literal questions. However, you may notice that it peculiarly has なんどき in it, which does happen to be an old-fashioned variation of 何時. This is rarely used outside the phrase いつなんどき, but if you do see it elsewhere, the context will be very specialized. }

\par{40. いつ ${\overset{\textnormal{ごろかんせい}}{\text{頃完成}}}$ しますか。 \hfill\break
About when will it be completed\slash you will complete it? }

\par{41. ${\overset{\textnormal{いま}}{\text{今}}}$ 、 ${\overset{\textnormal{なん}}{\text{何}}}$ どきですか。はい、ラーメンどきよ! \hfill\break
What time is it? It\textquotesingle s ramen time! \hfill\break
 \hfill\break
\textbf{Sentence Note }: This was a line to an old ramen commercial on TV. As you can see, when なんどき is used to ask “what time is it” as in what\textquotesingle s supposed to be going on. This, in normal conversation, would be conveyed by どの時. }

\par{42. いつなんどき ${\overset{\textnormal{ひつよう}}{\text{必要}}}$ になるかも分からない。 \hfill\break
I also have no clue when it\textquotesingle ll become needed. }

\par{43. (何時)何時 ${\overset{\textnormal{じこ}}{\text{事故}}}$ に ${\overset{\textnormal{あ}}{\text{遭}}}$ わないとも ${\overset{\textnormal{かぎ}}{\text{限}}}$ らない。 \hfill\break
It is not necessarily the case that you will never get into an accident. }

\par{\textbf{Spelling Note }: When written in 漢字, いつなんどき usually becomes いつ何時, but it may also be written as 何時何時. }

\par{\textbf{Grammar Note }: ~とも限らない is a verbal expression that follows adjectives\slash verbs to indicate “it is not necessarily that…” }

\par{44. ${\overset{\textnormal{すいぶんほきゅう}}{\text{水分補給}}}$ はいつなんどきでも ${\overset{\textnormal{わす}}{\text{忘}}}$ れないでください。 \hfill\break
Don\textquotesingle t ever forget to be hydrated . }
    