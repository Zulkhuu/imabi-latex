    
\chapter{The Particles しか \& ほか}

\begin{center}
\begin{Large}
第108課: The Particles しか \& ほか 
\end{Large}
\end{center}
 
\par{  These particles are very similar to だけ, but they're syntactically different. }
      
\section{The Particles しか \& ほか}
 
\par{ しか must be used with the negative form of a verb. It may follow nouns and even other particles such as those with に, で, and まで.  In translation, it is close to "just\slash only" but with a nuance of "there is no choice but to," especially after verbal phrases. }

\par{\textbf{Particle Note }: The particle に may be deleted before しか! }

\par{1. ${\overset{\textnormal{あと}}{\text{後}}}$ たった(の) ${\overset{\textnormal{いっしゅうかん}}{\text{一週間}}}$ しかありません。 \hfill\break
There's only one week left. }

\par{2. セミナーのレポートの ${\overset{\textnormal{し}}{\text{締}}}$ め ${\overset{\textnormal{き}}{\text{切}}}$ りは ${\overset{\textnormal{あした}}{\text{明日}}}$ だ。 ${\overset{\textnormal{てつや}}{\text{徹夜}}}$ するしかない。 \hfill\break
The deadline for the seminar report is tomorrow. I have no choice but to work on it all night long. }

\par{3. 今日のおやつこれしかないの。(Casual) \hfill\break
Is this all there is for snacks today? }

\par{4. ( ${\overset{\textnormal{じぜん}}{\text{事前}}}$ に・あらかじめ) ${\overset{\textnormal{つた}}{\text{伝}}}$ えるしかない。 \hfill\break
I have no choice but to tell. }

\par{5. 歩くしかありません。 \hfill\break
You have no choice but to just walk. }

\par{6a. ローマ字だけ書けます。                  I can only write Roman letters. \hfill\break
6b. ローマ字しか書けません。              I can't write anything but Roman letters. }

\par{\textbf{Particle Note }: Although だけを is possible and relatively common, the equivalent with しか, をしか, is very literary and doesn't show up often. }

\par{7. 私が目に見える美をしか信じなかった以上、この態度は当然である。 \hfill\break
Since I could only believe the beauty that I could see in my eyes, this attitude is natural. \hfill\break
From 金閣寺 by 三島由紀夫. }

\par{8. いつしか時が流れた。 \hfill\break
Time slid by. }

\par{\textbf{Set Phrase Note }: いつしか = いつの間にか知らないうちに. It doesn't need to be used to be with the negative. }

\par{ You may also use だけ and しか together creating だけしか. Also, しか may be seen as しきゃ and っきゃ in slang. }

\par{9. やるっきゃないな。 \hfill\break
I have no choice but to do it. }

\begin{center}
 \textbf{ほか Continued }
\end{center}

\par{ほか means "aside from." Technically, it can also be seen in ~ほかない, which is synonymous to ~しかない. }

\par{10. ${\overset{\textnormal{しゃちょう}}{\text{社長}}}$ ほか ${\overset{\textnormal{ごめい}}{\text{五名}}}$ が ${\overset{\textnormal{しゅっせき}}{\text{出席}}}$ \hfill\break
 Five in attendance aside from the company president. }

\par{11. どこかほかを ${\overset{\textnormal{さが}}{\text{探}}}$ す。 \hfill\break
To search the rest. }

\par{12. 彼はこのほかに ${\overset{\textnormal{なん}}{\text{何}}}$ と言った? \hfill\break
What else did he say? }

\par{13a. ${\overset{\textnormal{べんごし}}{\text{弁護士}}}$ は ${\overset{\textnormal{ぶんしょ}}{\text{文書}}}$ だけでなく、 ${\overset{\textnormal{こうとう}}{\text{口頭}}}$ でも ${\overset{\textnormal{せつめい}}{\text{説明}}}$ してくれた。 \hfill\break
13b. 弁護士は文書をもってするほか、口頭でも説明してくれた。 \hfill\break
The lawyer explained orally aside from the use of documents.  }
    