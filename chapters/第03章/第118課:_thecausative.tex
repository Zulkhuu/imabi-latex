    
\chapter{The Causative I}

\begin{center}
\begin{Large}
第118課: The Causative I: ~させる \& ~せる 
\end{Large}
\end{center}
 
\par{ The auxiliary verbs ~させる and ~せる are primarily used to create the \textbf{使役形 (causative form) }, which equates to " \textbf{to make\slash let X do Y }". B efore we look at what "causation" means and the grammatical restrictions, know that 作る is \textbf{not }used to make the causative form because it means "to make". 作る is used in the physical sense. Furthermore, 作る is a verb, not an auxiliary verb. Knowing that Japanese uses auxiliaries to make forms such as this, you would not be tempted to make a mistake like 1a. }

\par{1a. 怒る作る X \hfill\break
1b. 怒らせる 〇 \hfill\break
To make angry }

\par{ Again, the \textbf{causative }make is "to make X do Y".  Do not confuse the causative with the similar pattern XをYにする, which is used to show that one "makes\slash has a thing (X) be a certain way (Y). If you were to couple this with XがYになる, you should be able to avoid confusion. Consider Exs. 2~4 which differentiates these three patterns. }

\par{2. 音楽が小さくなった。 \hfill\break
The music('s volume) got smaller\slash quieter. }

\par{3. 音楽を小さくした。 \hfill\break
(I) turned down the music. }

\par{4. 彼に音楽を小さくさせた。 \hfill\break
I made him turn down the music. }
      
\section{The Auxiliaries ~させる \& ~せる: 使役形}
 
\par{ The 使役形 is used with both transitive and intransitive verbs, but the grammar is a little different. These endings attach to the 未然形 and conjugate as 一段 verbs. In \textbf{contracted speech }~させる and ~せる become ~さす and ~す respectively, which are conjugated as 五段 verbs. }

\par{\textbf{Chart Note }: In the chart below, class refer to the class of the verb being used as the example. }

\begin{ltabulary}{|P|P|P|P|P|}
\hline 

Class & Auxiliary & Example & Auxiliary & Example \\ \cline{1-5}

一段 & ~させる & 食べさせる & ~さす & 食べさす \\ \cline{1-5}

五段 & ~せる & 行かせる & ~す & 行かす \\ \cline{1-5}

する & ~せる & させる & ~す & さす (X\slash △) \\ \cline{1-5}

来る & ~させる & 来させる & ~さす & 来さす (X\slash △) \\ \cline{1-5}

\end{ltabulary}

\par{\textbf{Grammar Notes }: }

\par{1. The copula is not used in the causative form and should be replaced with ~にさせる. }

\par{Ex. バカ\{だ・である\} \textrightarrow  バカにさせる (to make\dothyp{}\dothyp{}\dothyp{}out to be stupid). }

\par{2. さす is rather uncommon by itself, but when used in conjugations it may be seen. For instance, 勉強さした (made\dothyp{}\dothyp{}\dothyp{}study). This, however, just like 来さす are \textbf{dialectical }at best and are not correct in Standard Japanese. }

\par{3. To make an \textbf{adjectival causative expression }, add させる after the く-連用形 of 形容詞 and after the に-連用形 of the copula for 形容動詞. }

\par{ \textbf{\emph{Conjugation Chart }}}

\begin{ltabulary}{|P|P|P|P|P|P|P|P|P|}
\hline 

Class & Verb & 動詞・ \hfill\break
形容詞 &  \textbf{使役形 }& Neg & Past &  \textbf{Short \hfill\break
使役形 }& Neg & Past \\ \cline{1-9}

一段 & To see & 見る & 見させる & 見させない & 見させた & 見さす (X\slash △) & X & 見さした \\ \cline{1-9}

一段 & To endure & 耐える & 耐えさせる & 耐えさせない & 耐えさせた & 耐えさす (X\slash △) & X & 耐えさした \\ \cline{1-9}

五段 & To say & 言う & 言わせる & 言わせない & 言わせた & 言わす & X & 言わした \\ \cline{1-9}

五段 & To strike\slash beat & 叩く & 叩かせる & 叩かせない & 叩かせた & 叩かす & X & 叩かした \\ \cline{1-9}

五段 & To clamor & 騒ぐ & 騒がせる & 騒がせない & 騒がせた & 騒がす & X & 騒がした \\ \cline{1-9}

五段 & To knock down & 倒す & 倒させる & 倒させない & 倒させた & 倒さす & X & 倒さした \\ \cline{1-9}

五段 & To share\slash divide \hfill\break
& 分かつ & 分かたせる & 分かたせない & 分かたせた & 分かたす & X & 分かたした \\ \cline{1-9}

五段 & To die & 死ぬ & 死なせる & 死なせない & 死なせた & 死なす & X & 死なした \\ \cline{1-9}

五段 & To be delighted & 喜ぶ & 喜ばせる & 喜ばせない & 喜ばせた & 喜ばす & X & 喜ばした \\ \cline{1-9}

五段 & To smile & 微笑む & 微笑ませる & 微笑ませない & 微笑ませた & 微笑ます & X & 微笑ました \\ \cline{1-9}

五段 & To rub & 擦る & 擦らせる & 擦らせない & 擦らせた & 擦らす & X & 擦らした \\ \cline{1-9}

形容詞 & New & 新しい & 新しくさせる & 新しくさせない & 新しくさせた & X & X & X\slash △ \\ \cline{1-9}

形容動詞 & Easy & 簡単だ & 簡単にさせる & 簡単にさせない & 簡単にさせた & X & X & X\slash △ \\ \cline{1-9}

\end{ltabulary}
      
\section{使役形}
 
\begin{center}
\textbf{With Transitive Verbs }
\end{center}

\par{ If the original verb is \textbf{transitive }, the direct object is marked with を if there is one mentioned. Remember, though, that if there is a transitive verb in a sentence or clause, a direct object must be implied. Japanese just doesn't require one to be used explicitly if it is information recoverable in context. The subject of the action, in other words, the person\slash performer being made to do something, is \textbf{marked by }\textbf{に }. }
 
\par{5. 弟にゴミを出させる。 \hfill\break
I make my younger brother take out the trash. }

\par{6. 太郎くんにカメを殺させる。 \hfill\break
To make Taro-kun kill a turtle. }

\par{7. 母は弟に犬にエサをやらせた。 \hfill\break
My mom made my little brother feed the dog. }

\par{\textbf{Sentence Note }: The first に marks the brother as the person being made to do the action and the second に marks the dog as the recipient of the food. }

\begin{center}
 \textbf{With Intransitive Verbs }
\end{center}
 
\par{ If the original verb is \textbf{intransitive, }\textbf{を and }\textbf{に }may seem to be interchangeable to mark the person being made to do something (被使役者). Most resources have failed in addressing the issue that adequately matches the actual decision between these two by regular native speakers. What is certain, though, is that the deciding factor for using them is determined by semantics. }

\par{ As for the particle を, it may imply a \emph{compelling force }. It may also be accidental\slash overbearing in nature. However, as you will see though, there is speaker variation in this interpretation. It is chosen, however, with 非情物 (things lacking emotion=inanimate objects) subjects and with emotion phrases. }

\par{8. 友達を怒らせる。 \hfill\break
To anger a friend. }

\par{9. 風が木を揺らせていた。 \hfill\break
The wind shook the tree(s). }

\par{10. ネズミを死なせる。 \hfill\break
To make the rat die. }

\par{\textbf{Sentence Note }: This last sentence fits in this description as bringing about an effect such as making something die is a means of a compelling force. If we think of animal abuse, using 死なせる in the case of accusing the owner of causing the pet's death would be completely logical. }

\par{ を would be expected in situations in which the direct object is something that could never have any will to act on its own (Ex. 11). }
11. 心を乱れさせる。 \hfill\break
To upset someone's feelings. 
\par{ に shows self-discipline and recognition of the will of the person being made to do something, which is why it goes with the "let" definition. Movement verbs are great examples of this. Nevertheless, natives don't actually consciously differentiate between let and make. So, in reality, what this means is that に can give the interpretation of "let", but in context, the speaker could still just mean "make". In fact, as the next example shows, even when a modal like ~てあげる is added, many speakers would still choose を to mark the 被使役者. }

\par{12. 父親は子ども\{を・に\} プールで泳がせてあげました。 \hfill\break
The father let his child swim in the pool. }

\par{13. 今回は山本に行かせましょうか。 \hfill\break
Next time, let's have Yamamoto go. }

\par{\textbf{Sentence Note }: It's unlikely that Ex. 13 implies the speaker(s) are thinking of Yamamoto's intents, but as the English translation suggests, it's possible to be vague on this matter. }

\par{14. 子どもをベンチに座らせる。 \hfill\break
To make the\slash a child(ren) sit on the bench. }

\par{15a. 子どもを座らせる。 \hfill\break
15b. 子どもに座らせる。(△) \hfill\break
To make the\slash a child(ren) sit. }

\par{\textbf{Sentence Note }: Most speakers would say 15a is more assertive, and some speakers would think 15b is wrong. There is also natural ambiguity in the exact interpretation of 15a and 15b. 15a may sound like a command or as if a parent is physically forcing the child to sit. 15b may sound like someone is having something sit in the kid's lap. }

\par{16. 先生は吉田さん\{を・に\}そこへ座らせました。 \hfill\break
The teacher made\slash let Mr. Yoshida sit there. }
17. 犬に公園を散歩させます。 \hfill\break
I let my dog walk in the park. 
\par{ As far as realistic usage of Japanese, a large motivation of which particle to use has to deal with not using the same one twice in a clause for the same purpose. Japanese has no problems using the same particle twice if it's for something different like we saw in Ex. 7, but it does not tolerate the doubling of the same thing for the same purpose. }

\par{18. 子供\{を・(△・X)に\}おつかいに行かせる。 \hfill\break
To make\slash let a kid go do an errand. }

\par{\textbf{Sentence Note }: Despite the grammatical correctness of this sentence, some speakers would say that Ex. 18 is wrong due to the doubling of に. }

\par{19. 親は子供\{を・(△・X)に\}買い物に行かせた。 \hfill\break
The parent made\slash had his\slash her child(ren) go shopping. }

\par{20. ワンちゃんに道の内側を歩かせる。 \hfill\break
To make the dog walk inside the road. }

\par{\textbf{Sentence Note }: Though the particle を is not used in this sentence to mark a direct object, the particle に is used to mark the 被使役者 is used first and foremost to avoid from using を twice. Unlike the particle に, native speakers are not aware of there being different usages of を. }

\par{ The particle に is also used with verbs that are actions regarding people such as 答える, しゃべる, etc. }

\par{21. 先生は、学生に質問に答えさせる。 \hfill\break
The teacher made\slash had his\slash her students answer (to) the questions. }

\par{\textbf{Sentence Note }: The status of に as a must with 答える is the same as with 行く. Yet, some speakers would still try to replace at least one of the に with を. Grammatically speaking, the first would be the best to change, but in reality, some speakers would say ~を答えさせる. }

\par{\textbf{Particle Note }: Overall, native speakers use を considerably more often even when either particle is possible. When a place is added into the sentence marked with に, the number of people who would chose を to mark the 被使役者 drastically goes up. In the end, as we've seen, there is certainly individual differences in exact usage of these particles in this situation. So, if you get corrected harshly, do not be discouraged. It's easiest to say that in the case for intransitive verbs, you may mark a 被使役者 with either を or に, but it is most important to most speakers to avoid the doubling of either one of them. }

\begin{center}
 \textbf{More Examples }
\end{center}

\par{22. 先生は学生を帰らせました。 \hfill\break
The teacher sent his\slash her students home. }

\par{23. 毎朝犬を散歩させます。 \hfill\break
I walk the dog every morning. }

\par{24. 妹を大阪に来させる。 \hfill\break
I'll make\slash have my younger sister come to Osaka. }

\par{25. 公園に犬を散歩させます。 \hfill\break
I walk the dog in the park. }
 
\par{26. 気分が悪そうなので生徒\{を・に\}先に帰らせた。 \hfill\break
Since the student seemed to feel bad, I had\slash let him\slash her go home earlier. }
 
\par{27a. 人に立たせる。? \hfill\break
27b. 人 \textbf{を }立たせる \hfill\break
To make a person stand. \hfill\break
\hfill\break
\textbf{Sentence Note }: 27a sounds like you're making someone stand something up. So, you should use 27b. }

\begin{center}
 \textbf{Had Someone Do }
\end{center}
 
\par{ When you say you \textbf{had someone do something for you }, instead of using the causative, you use ~てもらう・~ていただく. With expressions concerning emotions, those things are not in your control to cause. So, using them with the causative is fine. So, you can say something like 母を心配させた. This is because the causative does not have to be intentional. ~てもらう・~ていただく always are. }
 
\par{28. 先生に ${\overset{\textnormal{すいせんじょう}}{\text{推薦状}}}$ を書いていただきました。(You didn't make the teacher do it.) \hfill\break
I had my teacher write me a letter of recommendation. }
 
\par{29. 先生に推薦状を書かせました。(Not for your benefit at all) \hfill\break
I made the teacher write a letter of recommendation. }
 
\par{ Also note that the use of に is different. In the first sentence it shows from whom you received the action. In the second sentence it marks who was forced to do the action. }

\begin{center}
\textbf{More Examples }
\end{center}

\par{30. 笑わせないでください。 \hfill\break
Please don't make me laugh. }

\par{31. 頭をすっきりさせる。 \hfill\break
To clear the head. }

\par{32. 人を ${\overset{\textnormal{よ}}{\text{酔}}}$ わせる。 \hfill\break
To intoxicate a person. }

\par{33. 方言は日本語をより美しくて、素晴らしい言語として成り立たせていると思います。 \hfill\break
I think that dialects makes Japanese a more beautiful and wonderful language. }

\par{34. 母が妹に ${\overset{\textnormal{そうじ}}{\text{掃除}}}$ させたのにがらがらと笑ったんだ。 \hfill\break
I laughed so hard about my mom making my younger sister wash the dishes. }

\par{35. でも、父が僕に空手を習わせたのはいやだったよ。 \hfill\break
But, I hated that my dad made me take karate. }

\par{36. 僕にもやらせてもらえない(か)。(Casual) \hfill\break
Could you let me play it? }

\par{37. 彼をあっと言わせたい。 \hfill\break
I want to surprise him. }

\par{38. あたしを悲しくさせないで。(Feminine) \hfill\break
Don't make me sad. }

\par{39. 森林を焼けさせる。 \hfill\break
To have the forests burn. }

\par{40. 祖母を休ませました。 \hfill\break
We made our grandmother take a rest. }

\par{41. ${\overset{\textnormal{かぜ}}{\text{風邪}}}$ を ${\overset{\textnormal{こじ}}{\text{拗}}}$ らせて、寝込む。 \hfill\break
To complicate a cold and stay in bed (because of it). }

\par{\textbf{Word Note }: 拗らせる shouldn't be used in its basic form. Use the intransitive 拗れる in that case. }

\par{42. 弟は野菜が ${\overset{\textnormal{きら}}{\text{嫌}}}$ いだが、母は弟に毎日野菜を食べさせるつもりだ。 \hfill\break
My younger brother hates vegetables, but my mother plans to make him eat them every day. }

\par{43. 今働き始めたら、それだけ早く済ませるだろう。 \hfill\break
If we were to start working now, the quicker we would probably finish. }

\par{44. 何と言っても金が ${\overset{\textnormal{よ}}{\text{世}}}$ の中を動かすのさ。 \hfill\break
No matter what, money makes the world go round. }

\par{45. ${\overset{\textnormal{えんしゅつか}}{\text{演出家}}}$ は ${\overset{\textnormal{じょげん}}{\text{助言}}}$ を ${\overset{\textnormal{あた}}{\text{与}}}$ えるため ${\overset{\textnormal{たびたび}}{\text{度々}}}$ リハーサルを ${\overset{\textnormal{ちゅうだん}}{\text{中断}}}$ させた。 \hfill\break
The producer interrupted rehearsals frequently in order to give advice. }

\par{46. }

\par{A. あなたが、日本語の先生だったら、学生に何をさせますか。 \hfill\break
B. 日本語の先生だったら、学生に\{日本語の本・和書\}を読むようにさせるでしょう。 \hfill\break
A. If you were a Japanese teacher, what would you make your students do? \hfill\break
B. If I were a Japanese teacher, I'd make my students try to read Japanese words. }

\par{47. }

\par{A. あなたが、社長になったら、秘書に何をさせますか。 \hfill\break
B. 週ごとに報告書を書かせるでしょう。 \hfill\break
A. If you were a company president, what would you make your secretary do? \hfill\break
B. I would make (the secretary) write a report a week. }

\par{48. }

\par{A. 頭のいいロボットがいたら、何をさせますか。 \hfill\break
B. 掃除をさせるでしょう。 \hfill\break
A. If you had a smart robot, what would you make it do? \hfill\break
B. I'd make it do my laundry. }

\par{49. }

\par{A. 親になったら、子供に何をさせてあげますか。 \hfill\break
B. 日本語と ${\overset{\textnormal{かんこくご}}{\text{韓国語}}}$ を習わせてあげるでしょう。 \hfill\break
A. When you become a parent, what would you have your children do? \hfill\break
B. I would have (them) study Japanese and Korean. }

\par{50. }

\par{A. 結婚したら、相手に何をさせてあげますか。 \hfill\break
B. ハワイに旅行させてあげるでしょう。 \hfill\break
A. When you're married, what would you have your partner do? \hfill\break
B. I'd take him\slash her on a trip to Hawaii. }
      
\section{Unconcern or Let}
 
\par{ "Making" someone do something and "letting" someone do something are very similar. In Japanese, context largely plays a role in determining whether ~させる and ~せる actually stand for either or because in both cases someone is causing something\slash someone else to do something. Contexts for when "let" is the natural interpretation include "neglect" or "unintentional" circumstances. }
${\overset{\textnormal{}}{\text{51. 昼}}}$ に ${\overset{\textnormal{}}{\text{全生徒}}}$ を ${\overset{\textnormal{}}{\text{帰}}}$ らせました。 I let all of my students go home at noon. 52. お ${\overset{\textnormal{ふろ}}{\text{風呂}}}$ の ${\overset{\textnormal{}}{\text{水}}}$ を ${\overset{\textnormal{あふ}}{\text{溢}}}$ れさせてしまったの!(Old person) I accidentally let the water in the tub overflow!  ${\overset{\textnormal{}}{\text{53. 育}}}$ て ${\overset{\textnormal{}}{\text{方}}}$ を ${\overset{\textnormal{}}{\text{間違え}}}$ て ${\overset{\textnormal{}}{\text{子供}}}$ をいじけさせるのは ${\overset{\textnormal{}}{\text{親}}}$ の ${\overset{\textnormal{}}{\text{最悪}}}$ の ${\overset{\textnormal{}}{\text{悪夢}}}$ だ。 Incorrectly raising up one's child and having that child grow perverse is a parent's worst nightmare. 
\par{54. ちょっと ${\overset{\textnormal{}}{\text{考}}}$ えさせてください。 \hfill\break
Please let me think about it. }
${\overset{\textnormal{}}{\text{55. 車}}}$ を ${\overset{\textnormal{}}{\text{使}}}$ わせてください。 \hfill\break
Please let me use your car. 
\par{ When paired with ~てもらう・ていただく, the speaker humbly takes permission to do something. Rather than simply saying "I will do something", you are saying that you are taking it upon yourself, but you are aware of the person\slash people responsible for allowing you to even do it in the first place. This is why ~させていただきます is becoming ever more common in honorifics. }
${\overset{\textnormal{}}{\text{56. 会議}}}$ を ${\overset{\textnormal{}}{\text{始}}}$ めさせていただきます。(謙譲語) Allow me to start this meeting. 
\par{ When the causative form is paired with ~てあげる・やる, ~てくれる, ~てもらう, the sense of permission becomes clear and distinguishable from the "make" definition. }

\par{57. 先生は学生\{を・に\}トイレに行かせてあげた。 \hfill\break
The teacher let the student go to the restroom. }
${\overset{\textnormal{}}{\text{58. 父}}}$ は ${\overset{\textnormal{}}{\text{僕}}}$ に ${\overset{\textnormal{}}{\text{車}}}$ を ${\overset{\textnormal{}}{\text{運転}}}$ させてくれた。 My dad let me drive the car.  59.  A. ${\overset{\textnormal{}}{\text{子供}}}$ の ${\overset{\textnormal{}}{\text{時}}}$ 、 ${\overset{\textnormal{}}{\text{皆}}}$ の ${\overset{\textnormal{}}{\text{両親}}}$ は、 ${\overset{\textnormal{}}{\text{何}}}$ をさせてくれませんでしたか。 B. ${\overset{\textnormal{}}{\text{子供}}}$ の ${\overset{\textnormal{}}{\text{時}}}$ 、 ${\overset{\textnormal{}}{\text{一人}}}$ で ${\overset{\textnormal{}}{\text{遊}}}$ ばせてくれませんでした。 A. As a child, what did everyone's parents not let them do? B. When I was a child, I wasn't allowed to play alone.  60.  A. ${\overset{\textnormal{}}{\text{高校}}}$ の ${\overset{\textnormal{}}{\text{時}}}$ 、 ${\overset{\textnormal{}}{\text{皆}}}$ の ${\overset{\textnormal{}}{\text{両親}}}$ は、 ${\overset{\textnormal{}}{\text{何}}}$ をさせてくれませんでしたか。 B. \{ ${\overset{\textnormal{}}{\text{酒}}}$ ・アルコール\}を ${\overset{\textnormal{}}{\text{飲}}}$ ませてくれませんでした。 A In high school, what did everyone's parents not let them do? B. They wouldn't let me drink.     