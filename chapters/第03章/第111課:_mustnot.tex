    
\chapter{Must Not}

\begin{center}
\begin{Large}
第111課: Must Not 
\end{Large}
\end{center}
 
\par{ In this lesson, we will learn about the phrases for “must not”. Now, contrary to what many textbooks may say, there are actually differences among the several options that you as the listener must understand. Before we study “must” phrases, it\textquotesingle s important to learn about the “must not” phrases because there are fewer of them and less information overall to go through. So, let\textquotesingle s begin.  }
      
\section{禁止:~てはならない・~てはいけない・~てはだめだ}
 
\begin{center}
\textbf{Conjugations }
\end{center}

\begin{itemize}

\item 動詞: 連用形 + ては+ならない・いけない・だめだ 
\item 形容詞: 連用形(く) + ては+ならない・いけない・だめだ 
\item 名詞・形容動詞: 連用形(で) + は+いけない・だめだ\slash ではあっては+ならない・いけない 
\end{itemize}
\textbf{Contraction Note }: In casual speech ては \textrightarrow  ちゃ では \textrightarrow じゃ 
\begin{ltabulary}{|P|P|}
\hline 

使っては \textrightarrow  使っちゃ & 死んでは \textrightarrow  死んじゃ \\ \cline{1-2}

\end{ltabulary}

\begin{center}
 \textbf{使い分け }
\end{center}

\par{\textbf{~てはならない }: A prohibitory command accompanied with a sense of societal duty. This is the only pattern appropriate for law documents or judicial verdicts. It is obligatory in the written language in showing commands that regard society. }

\par{\textbf{~てはいけない }: A prohibitory command accompanied with a sense of personal duty. ~てはならない would be more stern than ~てはいけない in this situation in the written language. }

\par{\textbf{~てはだめだ }: A prohibitory command projecting warning, chastise, or criticism to the listener. It may not always be a literal command. It may simply suggest prohibition or avoidance of an action. It may also be used to encourage someone like in “don\textquotesingle t give up”: 諦めてはだめだ!This pattern is used primarily in the spoken language. }

\par{\textbf{Spelling Note }: These phrases are typically spelled in ひらがな. However, だめ is frequently spelled in 漢字 as 駄目. }

\par{ In the spoken language, there is a lot of interchangeability between ~てはいけない and ~てはだめだ, and in the written language, there is a lot of interchangeability between ~てはならない and ~てはいけない. }

\par{1. 日本は、 ${\overset{\textnormal{けんぽう}}{\text{憲法}}}$ に ${\overset{\textnormal{て}}{\text{照}}}$ らして、いかなる ${\overset{\textnormal{ぶき}}{\text{武器}}}$ も保持してはならない。(Formal document; 書き言葉) \hfill\break
Japan must not maintain any arms in light of the constitution. }

\par{2a. ${\overset{\textnormal{ちょうみん}}{\text{町民}}}$ ${\overset{\textnormal{とう}}{\text{等}}}$ は、いかなる ${\overset{\textnormal{りゆう}}{\text{理由}}}$ にせよ、 ${\overset{\textnormal{はいきぶつ}}{\text{廃棄物}}}$ を ${\overset{\textnormal{ふほう}}{\text{不法}}}$ ${\overset{\textnormal{とうき}}{\text{投棄}}}$ してはならない。 (法律; 書き言葉) \hfill\break
2b. 市民は、いかなる理由であっても、廃棄物を不法投棄してはならない。(書き言葉) \hfill\break
Townspeople must by no reason unlawfully dump waste matter. }

\par{3. この薬は ${\overset{\textnormal{ふくさよう}}{\text{副作用}}}$ があり、 ${\overset{\textnormal{けっとうち}}{\text{血糖値}}}$ の上昇を引き起こす ${\overset{\textnormal{かのうせい}}{\text{可能性}}}$ があるので、2日 ${\overset{\textnormal{いじょう}}{\text{以上}}}$ 、 ${\overset{\textnormal{れんぞく}}{\text{連続}}}$ して ${\overset{\textnormal{ふくよう}}{\text{服用}}}$ しては\{ならない・いけない\}。(書き言葉) \hfill\break
There are side effects to this drug with one being a possible rise in blood sugar. So, you mustn't use it more than two days consecutively. }

\begin{center}
 \textbf{The Person Commanding }
\end{center}

\par{ If we think of who makes the command, the person is of a higher status than the listener. It does not necessarily have to be the speaker who is the commander. The commander is likely as such based on what exactly you use. }

\begin{ltabulary}{|P|P|P|}
\hline 

Pattern & Style? & Description of Use \\ \cline{1-3}

 ~てはならない &  書き言葉 &  Commanded person is an overseeing\slash administering agency. \\ \cline{1-3}

 ~てはいけない &  Both &  C ommanded person is a responsible party. \hfill\break
In the spoken language, it is mainly used by men. \\ \cline{1-3}

 ~てはいけません &  話し言葉 & U sed mainly by women: mothers, teachers, and bosses \\ \cline{1-3}

 ~てはだめだ &  話し言葉 & U sed mainly by men: fathers, teachers, bosses, and sempai. \\ \cline{1-3}

 ~てはだめです &  話し言葉 & U sed mainly by women: mothers, teachers, bosses, and sempai. \\ \cline{1-3}

\end{ltabulary}

\par{\textbf{Speech Style Note }: If a man is to use the non-敬語 phrases ~てはいけない and ~てはだめだ and they are not a person of ultimate authority, one can imagine the speaker becoming in an emotional state in which he just ignores the character of the addressee. }
 ~てはだめ(だよ・よ)when used among friend and family becomes advice rather than a straight out command and can be used to one\textquotesingle s technical superior and inferior in this situation. 
\par{4. お母さん、そんなに働いちゃだめよ。体を壊しちゃうわ。(女性語) \hfill\break
Mom, don't work so hard like that. You'll get sick. }

\par{5. そのぬいぐる み 、まだ持ってたの?ルパートは危ないから、持ってちゃだめだよ。捨てなさい。(Family Guy reference) \hfill\break
You still have that stuffed animal? I told you that Rupert was dangerous, so you can't keep him. Throw him away. }

\begin{center}
 \textbf{Gender Trend? }
\end{center}

\par{ It is not that ~てはいけません and ~てはだめです are exclusively used by women, but women are often expected to be more polite. So, in this sense, regardless of whether one is male or female, the use of these polite forms emphasizes ones class more than actually showing respect to the addressee. This is expected of any well-educated 成人. }

\par{6a. 今日は、自分の車で帰るつもりでしょう?お酒を飲んでは\{いけません・だめです\}よ。( ${\overset{\textnormal{けいい}}{\text{敬意}}}$ ) \hfill\break
6 b. 今日は、車で帰る?じゃ、飲んじゃ\{いけません・だめです\}よ。( ${\overset{\textnormal{ひんかく}}{\text{品格}}}$ ${\overset{\textnormal{ほじ}}{\text{保持}}}$ ) }

\par{6a. Aren't you going home in your own car today? You mustn't drink. \hfill\break
6b. You're going home by car today? Well, you mustn't drink. }

\par{\textbf{Grammar Note }: Not using polite forms of conjunctive phrases makes it clear whether someone is showing 敬意 (respect) or 品格保持 (societal standing). This would show the latter. }

\begin{center}
 \textbf{More on ~てはならない }
\end{center}

\par{ Using ~てはならない in things based on personal experience and knowledge in spoken, colloquial contexts is unnatural except in dramas and movies. }

\par{7. 敵に後ろを見せるのかい。そりゃならぬ。武士として、あるまじきことじゃ。  (Samurai talk) \hfill\break
Show our rear to the enemy? That mustn't happen. That is unworthy of a warrior. }

\begin{center}
\textbf{The Difference between ~てはいけない \& ~てはだめだ }
\end{center}

\par{ When ~てはいけない is used, there is a rule that needs to be protected\slash followed, and if one breaks the rule, punishment follows. When ~てはだめだ is used, it is most often the case that there is no penalty or responsibility implied for the addressee. Even if there isn\textquotesingle t a rule to follow, if you are scolded or are to run into an unpleasant circumstance, it becomes understood to the speaker and listener that there is an unwritten rule to abide by. }

\par{8. ご飯を食べたら、すぐにお風呂! それまでゲームをやっては\{いけません・だめです\}よ。 \hfill\break
Bath right after supper! Until then, you must not play video games. }

\par{9. こんなに雨が降ってきては\{だめだ 〇・いけない X\}、計画は中止だ。 \hfill\break
This can't do with all rain falling like this; plans are cancelled. }

\par{ It\textquotesingle s impossible for man to actually command the weather. So, in this sentence, using ~てはいけない is just wrong. }

\begin{center}
\textbf{Summary Chart }
\end{center}

\begin{ltabulary}{|P|P|P|P|P|P|}
\hline 

 & 書き言葉 & 話し言葉 &  ${\overset{\textnormal{ばつ}}{\text{罰}}}$ ・ペナルティ &  ${\overset{\textnormal{めいれい}}{\text{命令}}}$ ${\overset{\textnormal{ぶん}}{\text{文}}}$ &  ${\overset{\textnormal{ひめいれい}}{\text{非命令}}}$ ${\overset{\textnormal{ぶん}}{\text{文}}}$ \\ \cline{1-6}

~てはならない & 〇 & X & 〇 & 〇 & X \\ \cline{1-6}

~てはいけない & 〇 & 〇 & 〇 & 〇 & △ \\ \cline{1-6}

~てはだめだ & X & 〇 & X & 〇 & 〇 \\ \cline{1-6}

\end{ltabulary}
\hfill\break
\textbf{Definition Notes }: 罰 = Punishment; 命令文 = Imperative sentence; 非命令文 = Non-imperative sentence \hfill\break

\begin{center}
 \textbf{~といけない・~とだめだ }
\end{center}

\par{ First, before we look at how these phrases differ with the ones above, consider the conjugation rules below. }

\begin{itemize}

\item 動詞:終止形+と\{いけない・だめだ\} 
\item 形容詞・形容動詞:終止形+\{といけない・だめだ\} 
\item 名詞:である・だ+(といけない・だめだ) 
\end{itemize}

\par{ As you can see, there is no ~とならない. You can\textquotesingle t use ~といけないand ~とだめだ to make sentences that urge someone. They may be used in declarative and affirmation sentences, but they can\textquotesingle t be interpreted as prohibitory commands. }

\par{ と marks a general condition, and it can only be used with confidence based on personal experience and or knowledge. So, even if you were to tell a listener a message involving having been prohibited, the use of these options would still be unnatural. As ~てはならない exists as an absolute prohibitory imperative phrase, there\textquotesingle s no way a declarative ~とならない could exist. }

\par{10. ${\overset{\textnormal{しんぞう}}{\text{心臓}}}$ が 悪いから、歩くのはいいが、走ると\{いけない・だめな\}のだ。   (書き言葉) \hfill\break
My heart is not good, so it's OK for me to walk, but I can't run. }

\begin{center}
\textbf{More Examples }
\end{center}

\par{11. 子供はお酒を飲んではいけない。 \hfill\break
Children must not drink sake. }

\par{12. ${\overset{\textnormal{}}{\text{家}}}$ の中では走ってはいけませんよ。 \hfill\break
You must not run in the middle of the house! }

\par{13. ${\overset{\textnormal{おく}}{\text{遅}}}$ れちゃだめ(だよ)。 \hfill\break
You can't be late! }
    