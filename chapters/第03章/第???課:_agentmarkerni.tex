    
\chapter*{The Particle に III}

\begin{center}
\begin{Large}
第???課: The Particle に III: Agent Marker  
\end{Large}
\end{center}
 
\par{ In Lesson 21, we learned how the particle が functions as an object marker for stative-transitive predicates. In this lesson, we will be learning how the particle に marks the agent of such predicates. }

\par{ \textbf{Stative-transitive predicates }are \emph{predicates that describe states\slash conditions but still have objects }. The term itself presents a dynamic often missed when discussing transitivity and the concepts of “object” and “agent.” }

\par{ The “ \textbf{agent }” is \emph{either the cause or the initiator\slash doer of the sentence }. In the context of this lesson, only the latter purpose will be relevant. A doer doesn\textquotesingle t always have to be actively doing anything. }

\par{ For example, when an animal “needs” food, it has no control over its need for sustenance, but grammatically speaking, “food” is the object of the verb “to need.” In English, “animals need food” is easily explained as involving the transitive verb “to need” and its object “food,” but in Japanese, the primary equivalent of “to need” is 要る. }

\par{ In Japanese, 要る is categorized as a 自動詞. This is because the object is marked with が. The concept of “object” becomes befuddled here, though, because が for all other kinds of predicates marks the subject. }

\par{1. お ${\overset{\textnormal{かね}}{\text{金}}}$ が ${\overset{\textnormal{い}}{\text{要}}}$ る。 \hfill\break
i. Money is needed. \hfill\break
ii. I need money. }

\par{ ~が要る can be interpreted as being no different as is evident in translation i. Regardless of whether が is a subject or object marker here, お金 is not the agent. The understood “I” is. An agent\slash doer, though, need not be the subject of a sentence. The same can be said in English such as in passive expressions. }

\par{i. [The man] was chased by \slash the dog\slash  for hours. \hfill\break
ii. [The cat] was sold to \slash an elderly couple\slash . \hfill\break
iii. [The turtle] was released back into the ocean by \slash its rescuers\slash . }

\par{ In these sentences, the nouns in [] are subjects and the nouns in \slash \slash  are agents. If we were to view the が with stative-transitive verbs as a subject marker, then the verbs would remarkably resemble passive verbs. After all, “needed” is the passive form of “to need,” and translation i. could easily be reworded as “Money is needed by me.” }

\par{ The next question to ask is how the “agent” of these predicates is marked by に. }

\par{2. ${\overset{\textnormal{きみ}}{\text{君}}}$ のいる ${\overset{\textnormal{ばしょ}}{\text{場所}}}$ が ${\overset{\textnormal{ぼく}}{\text{僕}}}$ \textbf{には }わかるんだ。 \hfill\break
i. I know where you are. \hfill\break
ii. Where you are is known by me. }

\par{ In English, translation ii. is strange, but it\textquotesingle s exactly how Japanese expresses the emotion that can be felt in the English of translation i. The use of に exemplifies 僕 as the agent of knowledge concerning [君のいる場所]. The use of に, however, is not obligatory. As such, Ex. 2 can be reworded as follows: }

\par{3. ${\overset{\textnormal{ぼく}}{\text{僕}}}$ \textbf{は }${\overset{\textnormal{きみ}}{\text{君}}}$ のいる ${\overset{\textnormal{ばしょ}}{\text{場所}}}$ がわかる。 \hfill\break
I know where you are. }

\par{ Nuance-wise, Exs. 2 and 3 are almost the same. However, Ex. 2 is more emphatic. Grammatically speaking, the “agent” of both Ex. 2 and Ex. 3 is 僕. Just as we learned in Lesson 12, the particle は, when it is in fact the topic marker, is best understood as only marking the topic. As such, the deep structure (the form of a sentence that represents the interworking grammar in the mind) of Ex. 3 would still, in fact, look like: “boku-wa (ø-ni) kimi-no iru basho-ga wakaru.” }

\par{ Motivation for why the particle に marks the agent of a sentence can be found in its basic use of denoting the place where something exists. Although long-winded and not representative of typical speech, one could say “it is within me where the knowledge of your location resides.” By extension of this, one can view the particle に as purposely marking the existence of an agent\slash doer as agent\slash doer = someone. }

\par{ We will now look at each verbal stative-transitive predicate that uses に to optionally mark the agent. This is a precursor to the next lesson in which we\textquotesingle ll finally learn about the passive form of verbs where に obligatorily marks the agent of a sentence. The verbs to be discussed in this lesson are as follows: }

\par{・分かる \hfill\break
・聞こえる \hfill\break
・見える \hfill\break
・ある \hfill\break
・いる \hfill\break
・出来る \hfill\break
・要る }

\par{ Because you already have a basic understanding of what these verbs mean and how they\textquotesingle re used, each section will mainly focus on particle issues the agent-marker に presents in relation to other kinds of に and other particles. }

\par{\textbf{Particle Note }: T he agent-marking function of に is most often accompanied with the contrast-marker は. When one feels it is necessary to explicitly state the agent, one is doing so because the natural happenstance of the verb is tied to the person being stated in contrast with others. }
      
\section{分かる}
 
\par{ The verb わかる, as we have learned, refers to non-intentional, natural understanding. Its lack of volition is why Japanese categorizes it as a 自動詞.  The basic patterns a sentence can take with わかる are as follows: }

\par{①XはYがわかる \hfill\break
②Xに(は)Yがわかる \hfill\break
③XはYをわかる \hfill\break
 \hfill\break
 In these patterns, “Y” is understood to be an “object.” Even if it can be conceptualized as a “subject” due to the use of が to mark it, “X” is what is conceptualized as the subject, and as such, the exhaustive-listing function of が can be used to mark X. This can be seen in Exs. 13 and 14 below. It is also impossible to forget how there are indeed cases when the object of わかる is indeed marked by を when the verb itself is no longer treated as being non-volitional. }

\par{ Ultimately, however, it is Pattern 2 that is the basic pattern of わかる. Pattern 1 is simply a derivative with the marking of the agent treated as an unspoken, understood element. }

\begin{center}
\textbf{Examples } 
\end{center}

\par{4. ${\overset{\textnormal{わ}}{\text{分}}}$ かる ${\overset{\textnormal{ひと}}{\text{人}}}$ \textbf{には }${\overset{\textnormal{わ}}{\text{分}}}$ かるけど、 ${\overset{\textnormal{わ}}{\text{分}}}$ からない ${\overset{\textnormal{ひと}}{\text{人}}}$ \textbf{には }${\overset{\textnormal{まった}}{\text{全}}}$ く ${\overset{\textnormal{わ}}{\text{分}}}$ からない ${\overset{\textnormal{はなし}}{\text{話}}}$ をします。 \hfill\break
I'm going to talk about something that will be understood by those who understand it but will be totally not understood by those who don\textquotesingle t understand it. }

\par{\textbf{Particle Note }: には is obligatory. Using は would disrupt the single-dependent clause modifying the noun 話. には is also obligatory due to the contrast grammar in the sentence. Lastly, using が instead of には would be too syntactically ambiguous and would not be able to express the contrast the sentence expresses. }

\par{5. ${\overset{\textnormal{にんげん}}{\text{人間}}}$ の ${\overset{\textnormal{きも}}{\text{気持}}}$ ち \textbf{が }お ${\overset{\textnormal{まえ}}{\text{前}}}$ \textbf{には }${\overset{\textnormal{わ}}{\text{分}}}$ かるものか。 \hfill\break
You would understand what people are feeling? }

\par{\textbf{Particle Note }: The use of ものか is very emphatic, which matches well with the emphasis provided by には. The flipping of constituents so that お前には is not at the front of the sentence is so that が can be interpreted as being the exhaustive-listing function. }

\par{6. ${\overset{\textnormal{おれ}}{\text{俺}}}$ \textbf{には }わかるさ。 \hfill\break
I can tell. }

\par{7. ちなみに、 ${\overset{\textnormal{わたし}}{\text{私}}}$ \textbf{は }ドイツ ${\overset{\textnormal{ご}}{\text{語}}}$ \textbf{が }${\overset{\textnormal{わ}}{\text{分}}}$ かるから、 ${\overset{\textnormal{かし}}{\text{歌詞}}}$ の ${\overset{\textnormal{いみ}}{\text{意味}}}$ も ${\overset{\textnormal{りかい}}{\text{理解}}}$ できます。 \hfill\break
Incidentally, because I understand German, I can also understand the meaning of the lyrics. }

\par{\textbf{Particle Note }: There is no heightened sense of emotion packed into this sentence, so there is no reason to use に(は) after 私 even if it would still be grammatically correct. The same can be said for Ex. 8. }

\par{8. イルカ \textbf{は }${\overset{\textnormal{にんげん}}{\text{人間}}}$ の ${\overset{\textnormal{ことば}}{\text{言葉}}}$ が ${\overset{\textnormal{わ}}{\text{分}}}$ かるのだろうか。 \hfill\break
I wonder if dolphins can understand human speech? }

\par{9. ${\overset{\textnormal{にんき}}{\text{人気}}}$ があるの \textbf{は }${\overset{\textnormal{わ}}{\text{分}}}$ かるが、 ${\overset{\textnormal{びみょう}}{\text{微妙}}}$ だね。 \hfill\break
I understand \emph{that it\textquotesingle s possible }, but it\textquotesingle s iffy, huh. }

\par{\textbf{Particle Note }: In this example, the は after the nominal phrase 人気があるの is the contrastive は, which means that the sentence implies that the speaker may not understand other aspects of the discussion at hand. }

\par{10. ${\overset{\textnormal{あじ}}{\text{味}}}$ の ${\overset{\textnormal{ちが}}{\text{違}}}$ い \textbf{は }${\overset{\textnormal{わ}}{\text{分}}}$ かるんですか。 \hfill\break
You understand the difference in taste? }

\par{\textbf{Particle Note }: The は of this sentence should not be interpreted as the contrastive は. }

\par{11. ${\overset{\textnormal{きみ}}{\text{君}}}$ \textbf{に }${\overset{\textnormal{おれ}}{\text{俺}}}$ の ${\overset{\textnormal{なに}}{\text{何}}}$ が ${\overset{\textnormal{わ}}{\text{分}}}$ かる? \hfill\break
What about me do you understand? }

\par{\textbf{Particle Note }: The particle は doesn\textquotesingle t follow the particle に in this sentence because there is no reason for the speaker to contrast the target agent\textquotesingle s (you) understanding of himself with another person. }

\par{12. ${\overset{\textnormal{わたし}}{\text{私}}}$ \{ \textbf{に・が }\}わかるように ${\overset{\textnormal{おし}}{\text{教}}}$ えてください。 \hfill\break
Please teach so that [it can be understood by me\slash I can understand it]. }

\par{\textbf{Particle Note }: The use of the particle が should invoke the exhaustive-listing interpretation in this sentence, but that doesn\textquotesingle t have to be the case, especially if the object of discussion—the it—isn\textquotesingle t at the forefront of the speaker\textquotesingle s mind. In which case, が would just be marking the subject of a dependent clause like it usually does. However, if the exhaustive-listing meaning isn\textquotesingle t intended, the choice of が creates unnecessary grammatical ambiguity that is solved by using に. }

\par{13. ${\overset{\textnormal{ほんとう}}{\text{本当}}}$ の ${\overset{\textnormal{わたし}}{\text{私}}}$ \textbf{は }${\overset{\textnormal{だれ}}{\text{誰}}}$ \textbf{にも }わからない。そして、 ${\overset{\textnormal{ほんとう}}{\text{本当}}}$ のあなた\{ \textbf{を・が }\} ${\overset{\textnormal{わたし}}{\text{私}}}$ \{ \textbf{が・には }\}わかることもできない。 \hfill\break
The true me is understood by no one. And, I cannot understand the true you. }

\par{\textbf{Particle Notes }: \hfill\break
1. The は after 私 is the contrastive は, meaning that 私 is still the object of わからない. \hfill\break
2. Provided を marks 本当のあなた as the object, the following 私 would typically be marked with が as it is established via を that わかる is being treated as a typical transitive verb. には can still be used to mark 私 in this situation, but the sentence would go from sounding like “It is indeed I who doesn\textquotesingle t understand the true you” to “Others may, but it is not I who understands the true you.” \hfill\break
3. If 本当のあなた is properly marked with が, then には ought to be used. This is because the も preceding できない technically deletes another instance of が, and although the exhaustive-listing function of が does allow for two が in a clause, Japanese never allows three instances of が in a single clause. }

\par{14. ${\overset{\textnormal{かれ}}{\text{彼}}}$ (に)もわかっているはずだ。 \hfill\break
He should also understand (the situation). }

\par{\textbf{Particle Note }: Without context, simply using も would cause the sentence to be ambiguous, making 彼 the object of understanding. This is why when も is used, に is frequently employed. }
      
\section{聞こえる}
 
\par{ The verb 聞こえる is often mistaken as a potential verb, and although it is translatable as “can hear,” it merely describes the natural phenomenon of something being audible, making it fundamentally different from both 聞ける and 聞くことが出来る, which are true potential expressions in the sense that they imply volition over the exercise of said ability. }

\par{ The basic patterns of a sentence with 聞こえる is as follows: }

\par{①XにはYが聞こえる \hfill\break
②XはYが聞こえる \hfill\break
③XにはYがZ(のよう)に聞こえる }

\par{ ①② can be translated as “Y is audible to X\slash Y can be heard by X.” Typically, the “X” element is not ever stated with 聞こえる in patterns for ①②. When it is, though, the first pattern is overwhelmingly the most used of the two. The second pattern would be limited to non-emphatic instances in which one wishes to contrast what\textquotesingle s audible to oneself with others (Ex. 24). }

\par{ ③ means “Y sounds like Z to X.” The reason why it has a difference in meaning is because there\textquotesingle s an additional argument (element) added to the sentence, and with it a different function of に comes into play, which can also be potentially modified by は. Whereas the に that follows X is the agent-marking function, the に after Z indicates an (indirect) object of comparison. Z can also be an adjective or verb instead of a noun, in which case it must be followed by よう so it can be treated as a nominal phrase. }

\par{ Although Japanese often drops elements out of a sentence if they're deemed unimportant or understood, if you see a sentence that only has 〇〇が〇〇に聞こえる, it\textquotesingle s going to be ③. However, just 〇〇に(は)聞こえる could have either interpretation depending on the context. One way to differentiate Xに(は)聞こえる and Zに(は)聞こえる is that X must be a sentient agent (human or personified non-human), whereas Z is usually some physical\slash abstract entity. }

\begin{center}
\textbf{Examples } 
\end{center}

\par{15. ${\overset{\textnormal{かれ}}{\text{彼}}}$ らは ${\overset{\textnormal{わたし}}{\text{私}}}$ \textbf{に }${\overset{\textnormal{き}}{\text{聞}}}$ こえるように ${\overset{\textnormal{わるくち}}{\text{悪口}}}$ を ${\overset{\textnormal{い}}{\text{言}}}$ っていた。 \hfill\break
They were speaking bad of me to where I could hear them. }

\par{\textbf{Particle Note }: There is no logical reason to think that “they” were speaking ill of the speaker in a way that could be heard by him\slash her but not others. Although に is the proper particle after 私, it could be seen replaced with が, especially if one wishes to strongly emphasis the exhaustive-listing of oneself—italicizing of “I.” }

\par{16. ${\overset{\textnormal{わか}}{\text{若}}}$ い ${\overset{\textnormal{ひと}}{\text{人}}}$ \textbf{には }${\overset{\textnormal{き}}{\text{聞}}}$ こえる ${\overset{\textnormal{おと}}{\text{音}}}$ が、 ${\overset{\textnormal{わたし}}{\text{私}}}$ \textbf{(に)は }${\overset{\textnormal{き}}{\text{聞}}}$ こえない。 \hfill\break
I can\textquotesingle t hear sounds that can be heard by young people. }

\par{\textbf{Particle Note }: For parallelism, には would be the preferred choice after 私. However, there is nothing wrong with just using は. This would be indicative of the spoken language where people are not always going to necessarily stay grammatically consistent. }

\par{17. ${\overset{\textnormal{おとな}}{\text{大人}}}$ \textbf{には }${\overset{\textnormal{き}}{\text{聞}}}$ こえず、 ${\overset{\textnormal{わか}}{\text{若}}}$ い ${\overset{\textnormal{ひと}}{\text{人}}}$ \textbf{にしか }${\overset{\textnormal{き}}{\text{聞}}}$ こえない ${\overset{\textnormal{ふしぎ}}{\text{不思議}}}$ な ${\overset{\textnormal{おと}}{\text{音}}}$ 、それがモスキート ${\overset{\textnormal{おん}}{\text{音}}}$ です! \hfill\break
A strange noise that cannot be heard but by the ears of young people and not adults, and that is the mosquitone! }

\par{\textbf{Particle Note }: It\textquotesingle s important to understand that しか cancels out は, meaning that it should be viewed as a derivative. しか and ~ない, and this agreement is what cancels out は. は and しか are incidentally both bound particles. }

\par{18. ${\overset{\textnormal{にんげん}}{\text{人間}}}$ \textbf{には }${\overset{\textnormal{き}}{\text{聞}}}$ こえないけれど ${\overset{\textnormal{いぬ}}{\text{犬}}}$ \textbf{には }${\overset{\textnormal{き}}{\text{聞}}}$ こえる ${\overset{\textnormal{おと}}{\text{音}}}$ というのは、 ${\overset{\textnormal{じっさい}}{\text{実際}}}$ \textbf{には }${\overset{\textnormal{そんざい}}{\text{存在}}}$ する。 \hfill\break
There are, in fact, sounds that can be heard by dogs but not by people. }

\par{\textbf{Particle Note }: The には after 実際 is simply the emphatic は after the adverb 実際に. }

\par{19. \{ ${\overset{\textnormal{なつ}}{\text{夏}}}$ \} \textbf{には }${\overset{\textnormal{せみ}}{\text{蝉}}}$ の ${\overset{\textnormal{こえ}}{\text{声}}}$ が ${\overset{\textnormal{き}}{\text{聞}}}$ こえる。 \hfill\break
The buzz of cicadas can be heard in the summer. }

\par{\textbf{Particle Note }: This example is meant to remind you that には could simply emphatically denote temporal setting. This can ascertained from には not being attached to a sentient agent (people\slash personified non-humans). }

\par{20. ${\overset{\textnormal{おっと}}{\text{夫}}}$ が ${\overset{\textnormal{わたし}}{\text{私}}}$ \textbf{に }${\overset{\textnormal{き}}{\text{聞}}}$ こえるような ${\overset{\textnormal{こごえ}}{\text{小声}}}$ で ${\overset{\textnormal{わたし}}{\text{私}}}$ の ${\overset{\textnormal{わるくち}}{\text{悪口}}}$ を ${\overset{\textnormal{い}}{\text{言}}}$ うのが ${\overset{\textnormal{くつう}}{\text{苦痛}}}$ です。 \hfill\break
It\textquotesingle s painful that my husband insults me in a whisper where I can hear him. }

\par{\textbf{Particle Note }: Understand that the 私に in this sentence is in a dependent clause modifying よう which then modifies 小声, which is why が is used beforehand with 夫. }

\par{21. ${\overset{\textnormal{みぎみみ}}{\text{右耳}}}$ が ${\overset{\textnormal{き}}{\text{聞}}}$ こえるようになった。 \hfill\break
I\textquotesingle ve become able to hear out of my right ear. }

\par{\textbf{Particle Note }: The use of に(は) would change the meaning to “I then became able to hear it in my right ear.” This sentence exemplifies how ② can be altered to Xが聞こえる, leaving out “Y” entirely. Of course, Xは聞こえる is possible as well. }

\par{22. ${\overset{\textnormal{とつぜんき}}{\text{突然聞}}}$ こえなくなったあと、 ${\overset{\textnormal{みみ}}{\text{耳}}}$ は ${\overset{\textnormal{き}}{\text{聞}}}$ こえるようになったが、 ${\overset{\textnormal{みみな}}{\text{耳鳴}}}$ りが ${\overset{\textnormal{のこ}}{\text{残}}}$ った。 \hfill\break
After suddenly becoming unable to hear, my ears became able to hear to again, but severe tinnitus remained. }

\par{\textbf{Particle Note }: The topic of this sentence is the speaker\textquotesingle s ears, which is why there is no 私は or the like that accompanies the first clause. If it were stated, it would be accompanied with 耳が, creating 私は耳が〇〇. Although this is an example of ②, this is not XやYが聞こえる. Rather, it\textquotesingle s XはXが聞こえる. Although the nouns for X are different, they are parts of a single entity. }

\par{23. ${\overset{\textnormal{ぼく}}{\text{僕}}}$ の ${\overset{\textnormal{みみ}}{\text{耳}}}$ \textbf{が }${\overset{\textnormal{き}}{\text{聞}}}$ こえにくいと ${\overset{\textnormal{わ}}{\text{分}}}$ かるなら、 ${\overset{\textnormal{ぼく}}{\text{僕}}}$ \{ \textbf{が・に }\} ${\overset{\textnormal{き}}{\text{聞}}}$ こえるように ${\overset{\textnormal{はな}}{\text{話}}}$ すのが ${\overset{\textnormal{すじ}}{\text{筋}}}$ じゃないか。 \hfill\break
If you understand that I\textquotesingle m hard of hearing, then wouldn\textquotesingle t it be logical to speak su ch that [I can hear what\textquotesingle s being said\slash what\textquotesingle s being said can be heard by me]? }

\par{\textbf{Particle Note }: The が after 耳 shouldn't be replaced, but because the use of が after 僕 may imply the exhaustive-listing function,the typical particle choice there would be に. }

\par{24. ( ${\overset{\textnormal{わたし}}{\text{私}}}$ \textbf{は }) ${\overset{\textnormal{いまへん}}{\text{今変}}}$ な ${\overset{\textnormal{おと}}{\text{音}}}$ \textbf{が }${\overset{\textnormal{き}}{\text{聞}}}$ こえました。 \hfill\break
I heard a strange noise\slash sound just now. }

\par{25. めっちゃ ${\overset{\textnormal{はたら}}{\text{働}}}$ いている \textbf{ように }${\overset{\textnormal{き}}{\text{聞}}}$ こえるけれど、 ${\overset{\textnormal{じっさい}}{\text{実際}}}$ のところ、ダレて ${\overset{\textnormal{しごと}}{\text{仕事}}}$ (を)してる。 \hfill\break
It may sound like I\textquotesingle m working real hard, but in actuality, I\textquotesingle m slacking at my job. }

\par{26. ずいぶん ${\overset{\textnormal{せいちょう}}{\text{成長}}}$ が ${\overset{\textnormal{はや}}{\text{早}}}$ い \textbf{ように }${\overset{\textnormal{き}}{\text{聞}}}$ こえるかもしれないですね。 \hfill\break
It might sound like growth has been quite fast, huh. }

\par{27. ${\overset{\textnormal{あらし}}{\text{嵐}}}$ を ${\overset{\textnormal{はこ}}{\text{運}}}$ んでくる ${\overset{\textnormal{らくらい}}{\text{落雷}}}$ \textbf{も }、 ${\overset{\textnormal{なんきょく}}{\text{南極}}}$ の ${\overset{\textnormal{こおり}}{\text{氷}}}$ が ${\overset{\textnormal{くだ}}{\text{砕}}}$ ける ${\overset{\textnormal{おと}}{\text{音}}}$ \textbf{も }、このコ \textbf{には }、この ${\overset{\textnormal{せかい}}{\text{世界}}}$ の ${\overset{\textnormal{こどう}}{\text{鼓動}}}$ \textbf{に }${\overset{\textnormal{き}}{\text{聞}}}$ こえているのだと ${\overset{\textnormal{おも}}{\text{思}}}$ う。 \hfill\break
Lightning which carry storms, the sound of Antarctic ice breaking, they must sound like the heartbeat of this world to this kid. }

\par{\textbf{Particle Note }: The pattern of this sentence is YがXにはZに聞こえる. Don\textquotesingle t let the fact that the Y elements come before the X element confuse you. }

\par{\textbf{Grammar Note }: The use of ている emphasizes how the truth of the statement holds for more than just the present point in time and has been so for some period of time. }

\par{28. とても ${\overset{\textnormal{にほんご}}{\text{日本語}}}$ \textbf{には }${\overset{\textnormal{き}}{\text{聞}}}$ こえないなあ。 \hfill\break
It sounds absolutely nothing like \emph{Japanese }. }

\par{29. ${\overset{\textnormal{とお}}{\text{遠}}}$ くにいてもあなたの ${\overset{\textnormal{こえ}}{\text{声}}}$ \textbf{は }${\overset{\textnormal{わたし}}{\text{私}}}$ \textbf{にだけは }${\overset{\textnormal{き}}{\text{聞}}}$ こえるの。 \hfill\break
No matter if you are afar, your voice can only be heard by me. \hfill\break
 \textbf{\hfill\break
Particle Notes }: \hfill\break
1. The は after あなたの声 replaces the が that would typically follow the “Y” element to topicalize\slash emphasize it. \hfill\break
2. だけに(は) is also possible, but this would shift emphasis off 聞こえる to the noun phrase that precedes the combination particle, which would be 私 in this sentence. Although 私 is emphasized thanks to だけは, the presence of に helps bring a stronger emphasis between the agent and 聞こえる. }

\par{30. ${\overset{\textnormal{にっぽん}}{\text{日本}}}$ の ${\overset{\textnormal{みつびしでんきかぶしきがいしゃ}}{\text{三菱電機株式会社}}}$ が、ブラインドサッカーに ${\overset{\textnormal{てきおう}}{\text{適応}}}$ する ${\overset{\textnormal{しこうせい}}{\text{指向性}}}$ スピーカーの ${\overset{\textnormal{かいはつ}}{\text{開発}}}$ を ${\overset{\textnormal{すす}}{\text{進}}}$ めている。 ${\overset{\textnormal{いち}}{\text{位置}}}$ が ${\overset{\textnormal{すこ}}{\text{少}}}$ しずれるだけで、 ${\overset{\textnormal{おと}}{\text{音}}}$ \textbf{が }${\overset{\textnormal{き}}{\text{聞}}}$ こえたり ${\overset{\textnormal{き}}{\text{聞}}}$ こえなかったりするものだ。ピッチでは ${\overset{\textnormal{おと}}{\text{音}}}$ \textbf{は }${\overset{\textnormal{き}}{\text{聞}}}$ こえないけれど、 ${\overset{\textnormal{かんきゃくせき}}{\text{観客席}}}$ \textbf{だけには }${\overset{\textnormal{き}}{\text{聞}}}$ こえるようにすることができる。 \hfill\break
Japan\textquotesingle s Mitsubishi Corporation is developing a directional speaker adapted for blind soccer. Just by shifting direction slightly, sound can be heard or not heard. Although sound won\textquotesingle t be heard on the field, one can have it so that sound will only be heard in the spectator stands. \hfill\break
 \hfill\break
\textbf{Particle Note }: \hfill\break
1. The は after 音 in the third sentence replaces the が that would otherwise follow the “Y” element so that it can be topicalized and used as the “Y” element for the next clause without having to be restated. \hfill\break
2. だけに(は) emphasizes the distinctiveness of the person\slash people who are a part of the phenomenon described by 聞こえる, which in this sentence is 観客席. Although not a typical person, the people who sit in the stands are people. }
      
\section{見える}
 
\par{ 見える is also often mistaken as a potential verb, and although it is translatable as “can see,” it merely describes the natural phenomenon of something being visible, making it fundamentally different from both 見(ら)れる and 見ることが出来る. }

\par{ The basic sentence patterns for 見える are as follows: }

\par{①XにはYが見える \hfill\break
②XはYが見える \hfill\break
③XにはYがZ(のよう)に見える。 }

\par{ The grammatical dynamics for 見える are the same as with 聞こえる. ①② are interpreted as “Y is visible to\slash can be seen by X” and ③ is interpreted as “Y looks\slash seems like Z to X.” The Z is an (indirect) object of comparison and can be made up of just a noun or with an adjective or verb that is then purposed into a nominal phrase with the help of よう. So for instance, 働いているように見える = “to seem like (one) is working.” }

\par{ Of ① and ②, the latter is most common, with には only really explicitly marking the agent if there is a strong, emotional appeal being made (Exs. 31 and 32). To distinguish ①② from ③ when various elements may be missing, remember that X is always a topicalized argument (sentence element) that is a sentient agent (human or personified non-human). Z has no restriction as both people and non-human entities are frequently used as objects of comparison. However, Z must be directly followed by 見える for the comparison-marker function of に to be made clear. }

\par{\textbf{Translation Note }: There are two translations provided for each example. The only difference being showcased is the variation in the interpretation of が with 見える. }

\begin{center}
\textbf{Examples } 
\end{center}

\par{31. ${\overset{\textnormal{しんこうしゃ}}{\text{信仰者}}}$ \textbf{には }${\overset{\textnormal{かみ}}{\text{神}}}$ が ${\overset{\textnormal{み}}{\text{見}}}$ える ${\overset{\textnormal{りゆう}}{\text{理由}}}$ は ${\overset{\textnormal{なぜ}}{\text{何故}}}$ ですか。 \hfill\break
i. Why is it that God is visible to believers? \hfill\break
ii. Why is it that believers can see God? }

\par{32. ${\overset{\textnormal{かみ}}{\text{神}}}$ が ${\overset{\textnormal{しんこうしゃ}}{\text{信仰者}}}$ \textbf{\{にだけ(は)・だけに(は)\} }${\overset{\textnormal{み}}{\text{見}}}$ えるようにしているんでしょうか。 \hfill\break
i. Is it that (they) are making it so that God is only visible to believers? \hfill\break
ii. Is it that (they) are making it so that only believers can see God? }

\par{\textbf{Particle Note }: だけに(は) is also possible, but this would shift emphasis off 見える to the noun phrase that precedes the combination particle, which is 信仰者 in this sentence. にだけ(は) emphasizes the distinctiveness in the phenomenon which is described by 見える, whereas だけに(は) emphasizes the distinctiveness of the person\slash people who are a part of the phenomenon described by 見える. }

\par{33. お ${\overset{\textnormal{まえ}}{\text{前}}}$ \textbf{に }${\overset{\textnormal{み}}{\text{見}}}$ えるか、 ${\overset{\textnormal{おれ}}{\text{俺}}}$ の ${\overset{\textnormal{かな}}{\text{哀}}}$ しい ${\overset{\textnormal{かお}}{\text{顔}}}$ が… \hfill\break
Can you see, my sad face…? }

\par{\textbf{Particle Note }: The sentence pattern here is a derivative of ①. There is some inversion in here, which is why the “Y” element appears after the verb. Although this means that the “X” element is right next to 見える, because it is お前, there really isn\textquotesingle t ambiguity here to confuse it with ③. }

\par{34. ${\overset{\textnormal{どうぶつ}}{\text{動物}}}$ \textbf{(に)は }${\overset{\textnormal{ゆうれい}}{\text{幽霊}}}$ が ${\overset{\textnormal{み}}{\text{見}}}$ えるのですか。 \hfill\break
i. Can ghost be seen by animals? \hfill\break
ii. Can animals see ghosts? }

\par{\textbf{Particle Note }: The importance of being able to see spirits\slash ghosts can be deemed important enough to emphasize “animals” as the agent of 見える. }

\par{35. ${\overset{\textnormal{しりょく}}{\text{視力}}}$ の ${\overset{\textnormal{わる}}{\text{悪}}}$ い ${\overset{\textnormal{ひと}}{\text{人}}}$ \textbf{にしか }${\overset{\textnormal{み}}{\text{見}}}$ えない ${\overset{\textnormal{がぞう}}{\text{画像}}}$ をご ${\overset{\textnormal{しょうかい}}{\text{紹介}}}$ していきます! \hfill\break
i. I\textquotesingle m going to introduce you to images that are invisible but to people whose eyesight is poor! \hfill\break
ii. I\textquotesingle m going to introduce you to images that only people with poor eyesight can see! }

\par{\textbf{Particle Note }: The fact that にしか appears in a dependent clause modifying 画像 indicates that 画像 is the “Y” element of ①. Furthermore, the use of しか emphasizes the fact that “X” is the agent of 見える. }

\par{36. この ${\overset{\textnormal{てそう}}{\text{手相}}}$ の ${\overset{\textnormal{も}}{\text{持}}}$ ち ${\overset{\textnormal{ぬし}}{\text{主}}}$ \textbf{(に)は }${\overset{\textnormal{ゆうれい}}{\text{幽霊}}}$ が ${\overset{\textnormal{み}}{\text{見}}}$ える。 \hfill\break
i. Ghosts are visible to possessors of these palm lines. \hfill\break
ii. Possessors of these palm lines can see ghosts. }

\par{37. ${\overset{\textnormal{こんちゅう}}{\text{昆虫}}}$ \textbf{も }カラス \textbf{も }${\overset{\textnormal{しがいせん}}{\text{紫外線}}}$ が ${\overset{\textnormal{み}}{\text{見}}}$ える。 \hfill\break
i. Ultra-violet rays are visible to even insects and crows. \hfill\break
ii. Even insects and crows can see ultra-violet rays. }

\par{\textbf{Particle Note }: As this is a generic statement concerning fact, there is no semantic reason for why 昆虫 and カラス to be marked with にも instead. }

\par{38. ${\overset{\textnormal{じつ}}{\text{実}}}$ は、 ${\overset{\textnormal{いぬ}}{\text{犬}}}$ \textbf{は }${\overset{\textnormal{いろ}}{\text{色}}}$ が ${\overset{\textnormal{み}}{\text{見}}}$ えるが、 ${\overset{\textnormal{しきじゃく}}{\text{色弱}}}$ の ${\overset{\textnormal{にんげん}}{\text{人間}}}$ と ${\overset{\textnormal{おな}}{\text{同}}}$ じ ${\overset{\textnormal{ていど}}{\text{程度}}}$ \textbf{に }${\overset{\textnormal{み}}{\text{見}}}$ える。 \hfill\break
i. As for dogs, colors are in fact visible to them, but only to the same degree as a person who has slight color-blindness. \hfill\break
ii. Dogs can in fact see color, but only to the same degree as a person who has slight color-blindness. }

\par{\textbf{Particle Note }: The に after 程度 functions as a comparison-marker. It can also be seen as being adverbial, creating a degree-adverb out of 色弱の人間と同じ程度. }

\par{39. ${\overset{\textnormal{みつばち}}{\text{蜜蜂}}}$ \textbf{(に)は }${\overset{\textnormal{しがいせん}}{\text{紫外線}}}$ が ${\overset{\textnormal{み}}{\text{見}}}$ え、それを ${\overset{\textnormal{りよう}}{\text{利用}}}$ して ${\overset{\textnormal{じぶん}}{\text{自分}}}$ の ${\overset{\textnormal{この}}{\text{好}}}$ みの ${\overset{\textnormal{はな}}{\text{花}}}$ を ${\overset{\textnormal{み}}{\text{見}}}$ つけることができる。 \hfill\break
i. Ultra-violet rays are visible to honeybees, and they can find their favorite flowers by utilizing them. \hfill\break
ii. Honeybees can see ultra-violet rays, and they can find their favorite flowers by utilizing them. }

\par{\textbf{Particle Note }: The use of には accounts for the agent of both 見える and 見つけることができる. Using it over は helps emphasize the uniqueness of honeybees being explained by both predicates. }

\par{40. ${\overset{\textnormal{あたら}}{\text{新}}}$ しい ${\overset{\textnormal{きょうりゅう}}{\text{恐竜}}}$ \textbf{は }アライグマ \textbf{に }${\overset{\textnormal{み}}{\text{見}}}$ えるかもしれない。 \hfill\break
A new dinosaur might look like a raccoon. }

\par{\textbf{Particle Note }: The は of this sentence actually mark the “Y” element of Pattern 3. The “X” would be an implied あなたには. }

\par{41. トムソンガゼルの ${\overset{\textnormal{む}}{\text{群}}}$ れを ${\overset{\textnormal{おそ}}{\text{襲}}}$ うクロコダイルたち \textbf{が }${\overset{\textnormal{かんぜん}}{\text{完全}}}$ に ${\overset{\textnormal{ば}}{\text{化}}}$ け ${\overset{\textnormal{もの}}{\text{物}}}$ \textbf{にしか }${\overset{\textnormal{み}}{\text{見}}}$ えない。 \hfill\break
Crocodiles attacking a herd of Thomson\textquotesingle s gazelle just completely look like monsters. }

\par{\textbf{Particle Note }: Due to the elements in the sentence and word order, にしか is after the “Z” element and not the “X” element, like has been the case for other examples in this lesson. }

\par{42. ${\overset{\textnormal{ねこ}}{\text{猫}}}$ \textbf{には }${\overset{\textnormal{ひと}}{\text{人}}}$ の ${\overset{\textnormal{め}}{\text{目}}}$ \textbf{に }${\overset{\textnormal{み}}{\text{見}}}$ えない「あるもの」 \textbf{が }${\overset{\textnormal{み}}{\text{見}}}$ えているらしい! \hfill\break
There appears to be a “certain something” not visible to the human eye that is visible to cats! }

\par{\textbf{Particle Note }: The phrase 目に見える means “visible to the eye.” The use of には after 猫 establishes grammatical parallelism with 目, as both are agents. 猫 is the agent of 見えている. 見えている is used instead of見える to emphasize the ability cats (unintentionally) utilize that isn\textquotesingle t accessible to humans. }

\par{43. ${\overset{\textnormal{にんげん}}{\text{人間}}}$ のようなものに ${\overset{\textnormal{ば}}{\text{化}}}$ けてはいるが、とても ${\overset{\textnormal{にんげん}}{\text{人間}}}$ \textbf{には }${\overset{\textnormal{み}}{\text{見}}}$ えない。 \hfill\break
Although it has shape-changed to something like a person, it hardly looks like a person. }

\par{44. ${\overset{\textnormal{め}}{\text{目}}}$ \textbf{が }${\overset{\textnormal{み}}{\text{見}}}$ えない ${\overset{\textnormal{じょうたい}}{\text{状態}}}$ でメイクするなんてできません。 \hfill\break
I couldn\textquotesingle t possibly do my make-up blind. }

\par{\textbf{Particle Note }: This example demonstrates how ② can be modified to Xが(Yが)見える, especially when the “Y” element is irrelevant, such as is the case here. Furthermore, が is obligatory here because it is in a dependent clause modifying the noun 状態. }

\par{45. ${\overset{\textnormal{こんや}}{\text{今夜}}}$ は ${\overset{\textnormal{いざよい}}{\text{十六夜}}}$ の ${\overset{\textnormal{つき}}{\text{月}}}$ 、 ${\overset{\textnormal{いえ}}{\text{家}}}$ の ${\overset{\textnormal{ちか}}{\text{近}}}$ く \textbf{では }${\overset{\textnormal{み}}{\text{見}}}$ えるかな。 \hfill\break
Tonight is the sixteenth-day moon; I wonder if I can see it near my\slash the house. }

\par{\textbf{Particle Notes }: \hfill\break
1. This sentence reminds us not to forget other particle expressions that could appear before 見える. では shows the location of 見える and has nothing to do with the “X” or “Y” elements of the basic sentence patterns with 見える. The same can be said for 聞こえる. \hfill\break
2. The “Y” element of the sentence is 十六夜の月. The “X” element is an understood 私. However, the thing topicalized in the sentence is 今夜, the time which the phenomenon of 見える is to potentially occur. }
      
\section{ある \& いる}
 
\par{ In the sense of “to have” as in indicating possession, there are two basic sentence patterns with ある and いる, which are as follows: }

\par{①Xに(は)Yが\{ある・いる\} \hfill\break
②XはYが\{ある・いる\} }

\par{ The use of には emphasizes the agent of possession, in other words, who is doing the possessing. “X” must be a human agent or a personified non-human agent. Both ①② are equally common with ある. For when the object of possession is people, いる almost always refer to having family members. In this situation, the agent can either take には or は. Of course, if the exhaustive-listing function of が needs to be employed, it too can follow “X.” }

\par{ As we learned in Lesson 21, the possession of other animate, living things such as pets is usually not expressed with いる. However, when it is, には is used but with words such as うち and 我が家 which both refers to oneself and one\textquotesingle s household. }

\par{\textbf{Particle Notes }: \hfill\break
1. The use of に to mark the agent with ある and いる also calls upon に\textquotesingle s function of indicating where something exists, which is why には is so emphatic here. \hfill\break
2. に isn\textquotesingle t usually by itself to mark the agent with these verbs except when it\textquotesingle s in a dependent clause with just the “X” element that modifies the “Y” element. }

\begin{center}
\textbf{Examples } 
\end{center}

\par{46. うちにはお ${\overset{\textnormal{かね}}{\text{金}}}$ が ${\overset{\textnormal{な}}{\text{無}}}$ いから。 \hfill\break
‘Cause I don\textquotesingle t have money. }

\par{47. ${\overset{\textnormal{わたし}}{\text{私}}}$ たち \textbf{には }、 ${\overset{\textnormal{きんしゅちゅう}}{\text{禁酒中}}}$ 、ぼーっと ${\overset{\textnormal{よ}}{\text{酔}}}$ っ ${\overset{\textnormal{ぱら}}{\text{払}}}$ っていたころでは ${\overset{\textnormal{かんが}}{\text{考}}}$ えられないほど、 ${\overset{\textnormal{じかん}}{\text{時間}}}$ \textbf{が }あるんです。 \hfill\break
When we\textquotesingle re abstaining from alcohol, we have so much time that we couldn\textquotesingle t have ever thought when we\textquotesingle d be spaced out drunk. }

\par{\textbf{Particle Note: }The adverbial phrase ending with the particle ほど does not interfere with the basic sentence pattern XにはYがある. It is just one of two adverb phrases in between Xには and Yが, the other being 禁酒中. }

\par{48. ${\overset{\textnormal{じぶん}}{\text{自分}}}$ \textbf{には }お ${\overset{\textnormal{かね}}{\text{金}}}$ があるのに、なぜ ${\overset{\textnormal{まわ}}{\text{周}}}$ りの ${\overset{\textnormal{ひと}}{\text{人}}}$ \textbf{は }お ${\overset{\textnormal{かね}}{\text{金}}}$ を ${\overset{\textnormal{も}}{\text{持}}}$ っていないだろう。 \hfill\break
Why is it that the people around me don\textquotesingle t have money when one oneself does? }

\par{\textbf{Particle Note }: The reason for why には doesn\textquotesingle t follow 周りの人 is because 持っている is not a stative-transitive predicate. Although it expresses the continuous state of having money, it also implies there is an active volition behind the agent carrying money around, which is contrary to ある, which implies that the possession in question is just a natural circumstance. }

\par{49. ${\overset{\textnormal{だいがくせい}}{\text{大学生}}}$ \textbf{は }${\overset{\textnormal{いっぱんてき}}{\text{一般的}}}$ \textbf{には }${\overset{\textnormal{じかん}}{\text{時間}}}$ \textbf{が }あるように ${\overset{\textnormal{み}}{\text{見}}}$ えるが、サークル ${\overset{\textnormal{かつどう}}{\text{活動}}}$ やバイトなど ${\overset{\textnormal{さまざま}}{\text{様々}}}$ なこと \textbf{が }あり、 ${\overset{\textnormal{いがい}}{\text{意外}}}$ と ${\overset{\textnormal{いそが}}{\text{忙}}}$ しい。 \hfill\break
It may seem that college students, generally, have time, but they are surprisingly quite busy with various things such as club activities and part-time jobs. }

\par{\textbf{Particle Notes }: \hfill\break
1. The use of には after 大学生 would be strange as the clause it\textquotesingle s in is just a generic statement. In such situations, the “agent” of possession should just be marked by は and not には. \hfill\break
2. Note that the agent of the second が phrase in the second clause is the same as the first. \hfill\break
3. The には after 一般的 is simply the adverbial form of 一般的だ (to be typical) followed by the contrast-marker は. }

\par{50. ${\overset{\textnormal{とくそうはん}}{\text{特捜班}}}$ \textbf{には }フランス ${\overset{\textnormal{ご}}{\text{語}}}$ が ${\overset{\textnormal{わ}}{\text{分}}}$ かるメンバー \textbf{が }${\overset{\textnormal{だれ}}{\text{誰}}}$ もいないんだ! \hfill\break
There\textquotesingle s not a single member in the squad that understands French! }

\par{\textbf{Particle Note }: 特捜班 can be treated as a location of affiliation and\slash or a sentient, human agent of possession of people, which is why it is marked には. Using は here would be ungrammatical. }

\par{51. うち \textbf{には }${\overset{\textnormal{いぬ}}{\text{犬}}}$ がいます。 \hfill\break
i. I have a dog. \hfill\break
ii. [I\slash we] have a dog at home. }

\par{\textbf{Particle Note }: The translation that best reflects the particle grammar of the sentence is translation ii. }

\par{52. ${\overset{\textnormal{わ}}{\text{我}}}$ が ${\overset{\textnormal{や}}{\text{家}}}$ \textbf{は }${\overset{\textnormal{いぬ}}{\text{犬}}}$ \textbf{が }いるから、 ${\overset{\textnormal{よけいそうじ}}{\text{余計掃除}}}$ するようになった。 \hfill\break
Our house has begun to excessively clean because we have a dog. }

\par{\textbf{Particle Note }: This is an instance where いる can show possession of a non-human object (pet)  with the agent being marked by は. For one, 我が家 is a personified noun referring to both oneself and one\textquotesingle s household. Two, 我が家 is both the topic of the first clause and the second clause. Although it can be viewed as the subject and agent of both clauses, the second clause is not a stative-transitive predicate, and because には doesn\textquotesingle t mark the agent of a regular transitive predicate, using it would be ungrammatical here. }

\par{53. ${\overset{\textnormal{じんせい}}{\text{人生}}}$ \textbf{は }${\overset{\textnormal{いぬ}}{\text{犬}}}$ \textbf{が }いることによって ${\overset{\textnormal{うつく}}{\text{美}}}$ しく ${\overset{\textnormal{かがや}}{\text{輝}}}$ いて ${\overset{\textnormal{み}}{\text{見}}}$ える! \hfill\break
Life seems to shine so beautifully by having a dog. }

\par{\textbf{Particle Note }: This sentence demonstrates that you mustn\textquotesingle t ignore independent-dependent clause boundaries. This is not an example of ② because 犬がいる is in a dependent clause modifying こと. However, this is an example of いる indicating the possession of pets, which is common with こと with the “X” element being omitted as an understood—oneself and\slash or people in general. }

\par{54. ${\overset{\textnormal{じぶん}}{\text{自分}}}$ \textbf{にしか }ないものは ${\overset{\textnormal{なん}}{\text{何}}}$ だろう。 \hfill\break
What is it that only I myself have? }

\par{55. ${\overset{\textnormal{じぶん}}{\text{自分}}}$ \textbf{に }ないものを ${\overset{\textnormal{かぞ}}{\text{数}}}$ えるより、 ${\overset{\textnormal{じぶん}}{\text{自分}}}$ \textbf{に }あるものを ${\overset{\textnormal{かんが}}{\text{考}}}$ えましょう。 \hfill\break
Rather than count what you don\textquotesingle t have, count what you do have. }
      
\section{出来る}
 
\par{ As we learned in Lesson 83, 出来る is the potential verb of する and is the representative verb of potential in general in Japanese. The agent of a potential sentence can be marked with に(は) so long as the potential phrase in question can be perceived as being a natural phenomenon. Meaning, when you see に(は), the sense of “volition” that may or may not otherwise be present becomes voided out. If cancelling out the nuance of volition is not possible, then so too is the use of には in that circumstance. To demonstrate this, consider Ex. 56. }

\par{56. ${\overset{\textnormal{わたし}}{\text{私}}}$ \textbf{(には ??・は 〇) }テニスができますよ。 \hfill\break
I can play tennis! }

\par{ Although it is not entirely ungrammatical to use には to mark 私 as the agent of this sentence due to できる being a stative-transitive predicate at its basic understanding, it is incredibly unnatural here because the strong volitional nuance of the transitiveする. }

\par{ Though at this point it goes without saying, there are two basic sentence patterns involving potential phrases, which are as follows: }

\par{①Xに(は)Yができる \hfill\break
②XはYができる }

\par{ As for ①, に is typically only seen when in a dependent clause were “X” is only present to directly modify “Y,” or when “Y” and “X” flip places. Having に directly next to the predicate helps emphasize the agent to the point that は isn\textquotesingle t needed unless one wishes to also contrast the agent with other people. Otherwise, には is typically seen. ① is frequently employed when showing the inherent abilities of one agent over that of another (Exs. 59 and 61). Another reason for using には is when は is used as the contrast-marker and not as the topic marker. }

\par{ As for ②, this is the most common pattern for potential phrases because most people generally imply volitional control of the execution of their abilities. As we have already seen with Ex. 56, the presence of this nuance will cause には to sound unnatural. Because we have already seen plenty of examples of ② in previous lessons, all the examples below will be of ①. }

\begin{center}
\textbf{Examples } 
\end{center}

\par{57. ${\overset{\textnormal{わたし}}{\text{私}}}$ \textbf{に }できることはなんだろう。 \hfill\break
i. What is it that I can do? \hfill\break
ii. What is it that can be done by me? }

\par{58. ${\overset{\textnormal{わたし}}{\text{私}}}$ \textbf{には }できることとできないこと \textbf{が }あります。 \hfill\break
i. There are things that I can and cannot do. \hfill\break
ii. There are things that can and cannot be done by me. }

\par{\textbf{Particle Note }: This example is a misnomer as には is used with the predicate ある. However, the presence of できる in the sentence, emphasizes the necessity of には in the speaker\textquotesingle s mind. Simply using は would cause the sentence to sound rather unnatural. }

\par{59. ${\overset{\textnormal{ひと}}{\text{人}}}$ \textbf{には }${\overset{\textnormal{でき}}{\text{出来}}}$ ないが、 ${\overset{\textnormal{かみ}}{\text{神}}}$ \textbf{には }${\overset{\textnormal{でき}}{\text{出来}}}$ る。 \hfill\break
It cannot be done by man, but it can be done by God. }

\par{60. ${\overset{\textnormal{なん}}{\text{何}}}$ でこんな ${\overset{\textnormal{ばかえら}}{\text{馬鹿偉}}}$ そうな ${\overset{\textnormal{こと}}{\text{事}}}$ \textbf{が }お ${\overset{\textnormal{まえ}}{\text{前}}}$ \textbf{に }${\overset{\textnormal{か}}{\text{書}}}$ けるんや! \hfill\break
How is it that you can write such absurdly cocky crap like this?! }

\par{\textbf{Particle Note }: Topicalizing お前 and having it come at the front of the sentence to make it follow ② would ruin the chastising tone. Although お前 could theoretically be marked by が with the exhaustive-listing function, the use of に is further enforced by the standard phenomenon of avoiding particle duplication in the same clause. }

\par{61. そり \textbf{ゃ }やれる ${\overset{\textnormal{ひと}}{\text{人}}}$ \textbf{には }やれるし、できない ${\overset{\textnormal{ひと}}{\text{人}}}$ \textbf{には }できないとしか ${\overset{\textnormal{こた}}{\text{答}}}$ えよう \textbf{が }ない。 \hfill\break
That's something that can be done by those capable of doing it, and the only answer is that it cannot be done by those incapable. }

\par{\textbf{Particle Notes }: }

\par{1. そりゃ is a contraction of それは. The それ is the “Y” element to やれる人にはやれる. It also happens to be “X” element to ない in the second clause, but it\textquotesingle s important to note that the purely existential meaning of ある, with which there isn\textquotesingle t an agent\slash owner to the existence. \hfill\break
2. Just as に is common when “X” and “Y” are flipped in ①, the same can be said for には. Additionally, the contrast-marker は is obligatory here due to the contrast being made between やれる人 and できない人. }

\par{62. どっちにしろ、 ${\overset{\textnormal{にほんご}}{\text{日本語}}}$ にはない ${\overset{\textnormal{ことば}}{\text{言葉}}}$ だから ${\overset{\textnormal{にほんじん}}{\text{日本人}}}$ \textbf{には }${\overset{\textnormal{はつおん}}{\text{発音}}}$ できないんだよね。 \hfill\break
Either way, since it\textquotesingle s a word that isn\textquotesingle t in Japanese, Japanese people can\textquotesingle t pronounce it, no? }

\par{\textbf{Particle Note }: The “Y” element here is 日本語にはない言葉. It is not repeated after 日本人には due to the lack of “it” in Japanese grammar. }

\par{63. もはや、エラーコードを ${\overset{\textnormal{にんげん}}{\text{人間}}}$ \textbf{に }${\overset{\textnormal{よ}}{\text{読}}}$ める ${\overset{\textnormal{けいしき}}{\text{形式}}}$ に ${\overset{\textnormal{へんかん}}{\text{変換}}}$ する ${\overset{\textnormal{ひつよう}}{\text{必要}}}$ はない。 \hfill\break
No longer is there a need to convert error codes into expressions that are human-readable. }

\par{\textbf{Particle Note }: This sentence demonstrates how ① is the base pattern of potential expressions. Although using が instead is possible here, because the conditions for using に are all met, and because it is the base pattern, に is the preferred particle choice. }

\par{64. ${\overset{\textnormal{あんごう}}{\text{暗号}}}$ とは、 ${\overset{\textnormal{かいどくほうほう}}{\text{解読方法}}}$ を ${\overset{\textnormal{し}}{\text{知}}}$ らない ${\overset{\textnormal{ひと}}{\text{人}}}$ \textbf{には }${\overset{\textnormal{かいどく}}{\text{解読}}}$ できないものだ。 \hfill\break
An encryption is something that cannot be deciphered by someone who doesn\textquotesingle t know how to decode it. }

\par{65. ${\overset{\textnormal{た}}{\text{食}}}$ べなければならないのは ${\overset{\textnormal{にんげん}}{\text{人間}}}$ \textbf{には }${\overset{\textnormal{つく}}{\text{作}}}$ れない ${\overset{\textnormal{ぶっしつ}}{\text{物質}}}$ があるからです。 \hfill\break
We must eat because there are substances that cannot be created in people. }

\par{\textbf{Particle Note }: には is used to contrast with how other organisms may very well be able to create substances that the human body cannot. }

\par{66. それは ${\overset{\textnormal{われわれ}}{\text{我々}}}$ \textbf{には }${\overset{\textnormal{そうぞう}}{\text{想像}}}$ ができないような ${\overset{\textnormal{くる}}{\text{苦}}}$ しみだろう。 \hfill\break
That is surely a hardship that is unimaginable to us. }

\par{\textbf{Particle Notes }: \hfill\break
1. With する verbs, が is often inserted between the base-noun and できる, making the base-noun the object of できる. When this happens, the presence of が after “X” for the exhaustive-listing function becomes less likely. \hfill\break
2. Because special emphasis is being placed on 我々 for being the agent of 想像ができない, simply marking the agent with は would be inappropriate, especially also given that the topic is already established as being それ. }
      
\section{要る}
 
\par{ To bring this lesson to a conclusion, the last stative-transitive predicate that we will learn about will be the very one used as an example in the opening: 要る. As we know, this is the basic verb for “to need.” As has been the case for the other verbs, 要る too has two basic sentence patterns: }

\par{①Xに(は)Yが要る \hfill\break
②XはYが要る }

\par{ Although ① is inherently the base pattern, ② is most commonly used at a ratio of ten-to-one. Whenever the agent of “to need” is actually present, に will always be optional. には is used to emphasis the relation between the agent and 要る. As is to be expected, XにYが要る is exceedingly rare and would only be seen in dependent clauses in which “X” directly modifies “Y” (Ex. 70) or when “Y” and “X” flip places followed by some emphatic sentence ending (Ex. 71). }

\begin{center}
\textbf{Examples } 
\end{center}

\par{67. お ${\overset{\textnormal{としよ}}{\text{年寄}}}$ り \textbf{(に)は }お ${\overset{\textnormal{かね}}{\text{金}}}$ が ${\overset{\textnormal{い}}{\text{要}}}$ らないんじゃないか? \hfill\break
Don\textquotesingle t old people not need money? }

\par{68. ${\overset{\textnormal{むすこ}}{\text{息子}}}$ \textbf{には }、お ${\overset{\textnormal{かね}}{\text{金}}}$ \textbf{は }${\overset{\textnormal{い}}{\text{要}}}$ らないからその ${\overset{\textnormal{か}}{\text{代}}}$ わり ${\overset{\textnormal{りょこう}}{\text{旅行}}}$ に ${\overset{\textnormal{つ}}{\text{連}}}$ れて ${\overset{\textnormal{い}}{\text{行}}}$ ってって ${\overset{\textnormal{たの}}{\text{頼}}}$ んでるのさ。 \hfill\break
I ask my son to take me on trips instead since I don\textquotesingle t need money. }

\par{\textbf{Particle Note }: This example is a misnomer. The には after 息子 goes with 頼んでる. The sentence pattern for 要らない is simply ② with only “Y” present. は is used to imply that of the things the speaker needs, money is not one of those things. However, the fact that the speaker does need things is not negated, just the need for money. }

\par{69. ${\overset{\textnormal{どれい}}{\text{奴隷}}}$ \textbf{(に)は }、お ${\overset{\textnormal{かね}}{\text{金}}}$ は ${\overset{\textnormal{い}}{\text{要}}}$ らない。 \hfill\break
i. Slaves don\textquotesingle t need money. \hfill\break
ii. Money isn\textquotesingle t needed by slaves. }

\par{\textbf{Particle Note }: Translation i. is most appropriate for when は is used, whereas translation ii. is most appropriate when には is used. }

\par{70. ${\overset{\textnormal{われ}}{\text{我}}}$ ら \textbf{に }${\overset{\textnormal{い}}{\text{要}}}$ るものは、 ${\overset{\textnormal{ぎんが}}{\text{銀河}}}$ を ${\overset{\textnormal{つつ}}{\text{包}}}$ むエネルギーである。 \hfill\break
What is needed by us is the energy that envelops the galaxy. \hfill\break
 \hfill\break
71. ${\overset{\textnormal{なに}}{\text{何}}}$ かを ${\overset{\textnormal{つか}}{\text{使}}}$ う ${\overset{\textnormal{まえ}}{\text{前}}}$ に、 ${\overset{\textnormal{じぶん}}{\text{自分}}}$ \textbf{に }${\overset{\textnormal{い}}{\text{要}}}$ るものか(を) ${\overset{\textnormal{かくにん}}{\text{確認}}}$ してから ${\overset{\textnormal{つか}}{\text{使}}}$ ったほうがいいと ${\overset{\textnormal{おも}}{\text{思}}}$ う。 \hfill\break
Before using something, I think it\textquotesingle s best to use it after verifying whether it\textquotesingle s something you yourself need. }

\begin{center}
\textbf{~に必要だ: The Purpose-Marker に } 
\end{center}

\par{ The one adjectival stative-transitive predicate that can take に is 必要だ, unsurprisingly synonymous with 要る. 必要だ means “to be necessary,” and the use of に indicates “to whom” “Y” is necessary. However, this use of に isn\textquotesingle t quite the agent-marker. This is because “X” is not limited to sentient entities. In fact, “X” can be anything, including nominalized expressions (Ex. 73). This に marks a purpose\slash reason for an action\slash state. }

\par{72. ${\overset{\textnormal{うんめい}}{\text{運命}}}$ の ${\overset{\textnormal{ひと}}{\text{人}}}$ に ${\overset{\textnormal{であ}}{\text{出会}}}$ うため \textbf{に }、 ${\overset{\textnormal{いま}}{\text{今}}}$ の ${\overset{\textnormal{わたし}}{\text{私}}}$ \textbf{に }${\overset{\textnormal{ひつよう}}{\text{必要}}}$ なことは ${\overset{\textnormal{なん}}{\text{何}}}$ でしょうか。 \hfill\break
What is it that my current self needs to meet my soulmate? }

\par{\textbf{Particle Note }: Thematically, both に in bold can be viewed as marking purpose\slash reason. }

\par{73. ギターを ${\overset{\textnormal{はじ}}{\text{始}}}$ める \textbf{のに }${\overset{\textnormal{ひつよう}}{\text{必要}}}$ なものをまとめてみました。 \hfill\break
I\textquotesingle ve tried to compile the things necessary for beginning guitar. }

\par{ The greatest cliffhanger in the introduction of this lesson was the lack of a Japanese translation for “animals need food.” Without further ado, the translation of this sentence is as follows: }

\par{74. ${\overset{\textnormal{どうぶつ}}{\text{動物}}}$ が ${\overset{\textnormal{い}}{\text{生}}}$ きる \textbf{には }${\overset{\textnormal{しょくもつ}}{\text{食物}}}$ が ${\overset{\textnormal{い}}{\text{要}}}$ る。 \hfill\break
i. Animals need food (to live). \hfill\break
ii. For animals to live, they need food. \hfill\break
iii. For animals to live, food is needed. }

\par{ This sentence represents another facet of には. This には is thematically the same as the ones in Exs. 72 and 73, making it equivalent to “for\slash to.” The reason why には can directly follow a verb is that the verb is actually nominalized as an effect. This makes には equal to のに(は). }

\par{ The three translat ions above tie back to various things detailed in this lesson. The insertion of ~が生きる re-purposes に as the purpose-marker because 動物 isn't a true agent. However, because 生きる ends up being nominalized, it is not illogical to treat this as an extension of the agent-marker に. In fact, it\textquotesingle s contexts like these where the purpose-marker に likely derived. A way to conceptualize this as such is translating Ex. 74 as “The state of animals living requires that there be food.” }

\par{75. ${\overset{\textnormal{ひと}}{\text{人}}}$ が ${\overset{\textnormal{い}}{\text{生}}}$ きる \textbf{には }${\overset{\textnormal{いみ}}{\text{意味}}}$ があり、そして ${\overset{\textnormal{かち}}{\text{価値}}}$ があるのです。 \hfill\break
There is meaning and value to people living. \hfill\break
 \hfill\break
\textbf{Particle Note }: The relation between には and ある can be interpreted as both being its existential and its possessive meaning. Where "meaning" resides is within people living, and it is the act of man living which has meaning. Additionally, if we view the purpose-marker に as being one in the same as the agent-marker, then one could translate this as “The state of people living has meaning and value.” }

\begin{center}
\textbf{Verb + には } 
\end{center}

\par{ Thinking this deep into how and why には is used after verbs helps in understanding how に truly works in its various functions. Although one could create grammatically literal translations for any instance of には like the ones just presented, Verb + には is typically translated as “in + verb + -ing…” }

\par{76. ${\overset{\textnormal{いんしょくてん}}{\text{飲食店}}}$ をやる \textbf{には }、 ${\overset{\textnormal{ちょうりしめんきょ}}{\text{調理師免許}}}$ が ${\overset{\textnormal{ひつよう}}{\text{必要}}}$ だと ${\overset{\textnormal{おも}}{\text{思}}}$ っている ${\overset{\textnormal{ひと}}{\text{人}}}$ が ${\overset{\textnormal{おお}}{\text{多}}}$ いですが、 ${\overset{\textnormal{じつ}}{\text{実}}}$ はそんな ${\overset{\textnormal{ひつよう}}{\text{必要}}}$ はありません。 \hfill\break
There are many people who think a chef certificate is needed [in running\slash to run] a restaurant, in reality, there is no such need. }

\par{77. ${\overset{\textnormal{じぶん}}{\text{自分}}}$ に ${\overset{\textnormal{ひつよう}}{\text{必要}}}$ な ${\overset{\textnormal{ほけん}}{\text{保険}}}$ を ${\overset{\textnormal{かんが}}{\text{考}}}$ える \textbf{には }、まず ${\overset{\textnormal{ほけん}}{\text{保険}}}$ に ${\overset{\textnormal{かにゅう}}{\text{加入}}}$ する ${\overset{\textnormal{もくてき}}{\text{目的}}}$ を ${\overset{\textnormal{かくにん}}{\text{確認}}}$ することが ${\overset{\textnormal{たいせつ}}{\text{大切}}}$ だ。 \hfill\break
In thinking about what insurance is needed for oneself, it is important to first confirm one\textquotesingle s aim in subscribing for insurance. }

\par{\textbf{Curriculum Note }: As has been alluded to, there is overlap between “Verb + のに(は)” and “Verb + には.” However, due to the complexity of the topic, this will be foregone for now and will thoroughly discussed in a later lesson. }
    