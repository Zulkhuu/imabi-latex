    
\chapter{~ていく \& ~てくる}

\begin{center}
\begin{Large}
第133課: ~ていく \& ~てくる 
\end{Large}
\end{center}
   These patterns are a little easier than the ones from the last lesson, but these are easier to confuse with each other. So, 注意してください!       
\section{~ていく \& ~てくる}
 
\par{ 行く (to go) and 来る (to come) with て creates constructions that are not as straightforward as one would expect. What dictates the usage of ~ていく and ~てくる is highly context driven. It's rather impossible for even the best linguists to properly define all of the possible situations these phrases can be used in. "Highly context driven" refers to the fact that how you interpret them is not just determined by the transitivity of the verb phrase but by what exactly it is "semantically". }

\par{ Categorical properties of what exactly the verb means is very important to keep in mind, and this also means that you can't just assume that English and Japanese match perfectly on a word-by-word basis. Usually, even this is never the case. After all, the two languages are NOT related to each other. }

\par{ When \emph{physical action }is involved, ~ていく and ~てくる show action away and action towards the speaker\textquotesingle s current location respectively. In this case, the verbs are typically written in 漢字, but this is not always the case. At times, these phrases result in sentences that do not sound quite like anything one would say in English. In such cases, think really hard about the real world context in which they are used to better understand the semantic properties at work. }

\par{1. ちょっと ${\overset{\textnormal{}}{\text{出}}}$ かけて ${\overset{\textnormal{}}{\text{来}}}$ ます。 \hfill\break
I'm going out for a bit (and will come back). }

\par{2. 公園に歩いて行った。 \hfill\break
I went to the park by foot. }

\par{\textbf{Grammar Note }: Remember that some て phrases act like adverbs in Japanese, which then can then like in this sentence equate to a prepositional phrase, in this case 歩いて = by foot. }

\par{3. 朝ご飯を食べて行きました。 \hfill\break
I went having eaten breakfast. }

\par{${\overset{\textnormal{}}{\text{4. 行}}}$ って ${\overset{\textnormal{}}{\text{来}}}$ ます。 \hfill\break
I'm going and coming back. }

\par{\textbf{Usage Note }: The above phrase is used when leaving your residence, knowing that you will be coming back. Though the plain form is acceptable with those you are close to, given its status as a set phrase, the polite form is also used without any repercussions. }

\par{5. ${\overset{\textnormal{だいどころ}}{\text{台所}}}$ からいすを ${\overset{\textnormal{と}}{\text{取}}}$ ってきます。 \hfill\break
I'm going to get a chair from the kitchen. }

\par{6. 10分もすれば、帰ってくる。 \hfill\break
If I have but 10 minutes, I'll come home. }

\par{ When there is no physical motion involved, ~ていく indicates disappearance whereas ~てくる indicates a process of emergence. Both can show a process of change or continuation, but they are different in this manner. ~ていく would indicate something that \textbf{will }change and continue on into the future whereas ~てくる would represent that \textbf{has been }happening for something and extends to the present. }

\par{7. ミルズさんは ${\overset{\textnormal{}}{\text{日本語}}}$ がうまくなってきましたね。 \hfill\break
Mr. Mills has been getting better in Japanese, hasn't he? }

\par{${\overset{\textnormal{}}{\text{8. 君}}}$ の ${\overset{\textnormal{}}{\text{英語}}}$ はきっと ${\overset{\textnormal{うま}}{\text{上手}}}$ くなっていくでしょう。(Familiar) \hfill\break
Your English will surely get better! }

\par{9. ${\overset{\textnormal{じぶんかって}}{\text{自分勝手}}}$ に ${\overset{\textnormal{せんたく}}{\text{選択}}}$ したい ${\overset{\textnormal{よく}}{\text{欲}}}$ は、 ${\overset{\textnormal{かなら}}{\text{必}}}$ ず ${\overset{\textnormal{}}{\text{消}}}$ えていくのであろう。 \hfill\break
Your desire to want to selfishly choose will surely disappear. }

\par{10. ${\overset{\textnormal{しゅうじんたち}}{\text{囚人達}}}$ は ${\overset{\textnormal{どうくつ}}{\text{洞窟}}}$ からぞろぞろと ${\overset{\textnormal{}}{\text{出}}}$ てきた。 \hfill\break
The prisoners came filtering out of the cave. }
11. その木は、芽が出てきた。 \hfill\break
The tree has begun to sprout.  12. 最後の希望が消えていく。 \hfill\break
There goes our last hope.  \textbf{Inception }\hfill\break

\par{ ~てくる can also indicate the inception of a process. As this section will show, understanding this can get really tricky. So, pay attention to detail and be open to differences between Japanese and English grammar. }

\par{ ${\overset{\textnormal{}}{\text{13a. 雪}}}$ が ${\overset{\textnormal{ふ}}{\text{降}}}$ ってきた。 \hfill\break
${\overset{\textnormal{}}{\text{13b. 雪}}}$ が ${\overset{\textnormal{}}{\text{降}}}$ り ${\overset{\textnormal{}}{\text{始}}}$ めた。 \hfill\break
It started to snow. }

\begin{center}
 \textbf{~てくる・~ていく Interchangeability } 
\end{center}

\par{ Tense can be problematic in making ~てくる and ~ていく seem interchangeable. The semantic differences between the verbs of these phrases and differences in the main verbs used are important. For instance, the fact that 消える is used below is very important. }

\par{14. 暖炉の火が消えて\{きた・いく\}。 \hfill\break
The fire in the fireplace is about to go out. }

\par{ The first option indicates inception of a change. The status of the fire gets changed, and the process is "going out". With ~た, you indicate that the process has just started. A strong wind, cutting off of fuel source, or lots of water put on it would determine how quickly it goes out. Nevertheless, the English can be used in the same situations. The second option indicates a process of disappearance. The flame will shortly start to fade and then go out. In a real world situation, it would be very hard to tell all these details, especially because these judgments are all subjective to what the speaker feels. }

\par{ When considering the English equivalent, you should be able to see how it could be used for both. For brevity, more literal\slash less natural English equivalents are not shown. With the information given thus far, you should be able to construct more literal interpretations at your own will. }

\par{ Consider a less puzzling example. Here we have a sentence in reference to the weather. This,again, is an important detail to keep in mind. }

\par{15. 段々暑くなって\{くる・いく\}。 \hfill\break
It will become gradually hotter. }

\par{ In this case, you could say that the former indicates an inception of a process that is to happen. It's more vivid and doesn't feel that it's going to take long for this change to occur. Once it has become quite hotter, it could just reach a point and stay that way. The latter option doesn't imply this. Rather, the latter shows a slower, gradual change that will occur and continue on that way. So, even if both sentences were used in the context of the temperature reaching 100F, with the latter, it could just continue to steadily get higher. }

\begin{center}
 \textbf{More Examples }
\end{center}

\par{\textbf{Variant Note }: It is also important to realize that ~ていく is often contracted to ~てく in casual speech in the same way ~ている is contracted to ~てる. Be sure to recognize it in different conjugations. For example, ~てった = ~ていった. However, this contraction is avoided in conjugations in which it might get confused with ~てくる. For instance, ~ていきなさい would accidentally become ~てきなさい if you dropped い. }

\par{16. ${\overset{\textnormal{ひと}}{\text{独}}}$ りで ${\overset{\textnormal{こども}}{\text{子供}}}$ を ${\overset{\textnormal{そだ}}{\text{育}}}$ ててきました。 \hfill\break
I have brought up my children alone. }
 
\par{${\overset{\textnormal{}}{\text{17. 帰}}}$ ってくるまでここで ${\overset{\textnormal{}}{\text{待}}}$ っててね。(Casual) \hfill\break
Just wait here until I come back. }
 
\par{${\overset{\textnormal{}}{\text{18. 先生}}}$ が ${\overset{\textnormal{はい}}{\text{入}}}$ ってくるのを ${\overset{\textnormal{}}{\text{待}}}$ つ。 \hfill\break
To wait for the teacher to come in. }

\par{19. 氷がどんどんと ${\overset{\textnormal{せま}}{\text{迫}}}$ ってきた。 \hfill\break
The ice steadily approached and came (here). }

\par{${\overset{\textnormal{}}{\text{20. 彼は新聞}}}$ を ${\overset{\textnormal{}}{\text{持}}}$ ってくるのを ${\overset{\textnormal{}}{\text{忘}}}$ れた。 \hfill\break
He forgot about bringing the newspaper. }
 
\par{${\overset{\textnormal{}}{\text{21. 寒}}}$ くなってきたみたいだ。 \hfill\break
It looks like it\textquotesingle s becoming colder. }

\par{22. ガスが( ${\overset{\textnormal{まわ}}{\text{回}}}$ って) ${\overset{\textnormal{き}}{\text{来}}}$ た。 \hfill\break
Gas came. }

\par{\textbf{Word Note }: 回る in this context emphasizes the \emph{circulation }of gas. }
 
\par{23. カーテンの ${\overset{\textnormal{うし}}{\text{後}}}$ ろから ${\overset{\textnormal{}}{\text{近}}}$ づいてくるなんてこわい。 \hfill\break
Approaching from behind the curtain like that is scary. }
 
\par{${\overset{\textnormal{}}{\text{24. 時}}}$ が ${\overset{\textnormal{や}}{\text{矢}}}$ のように ${\overset{\textnormal{す}}{\text{過}}}$ ぎていきました。 \hfill\break
Time started to pass by like an arrow. }
 
\par{${\overset{\textnormal{}}{\text{25. 彼}}}$ は ${\overset{\textnormal{}}{\text{雨}}}$ の ${\overset{\textnormal{}}{\text{中}}}$ を ${\overset{\textnormal{}}{\text{駅}}}$ まで ${\overset{\textnormal{}}{\text{歩}}}$ いていきました。 \hfill\break
He walked to the station through the rain. }

\par{${\overset{\textnormal{}}{\text{26. 持}}}$ ってきてください。 \hfill\break
Please bring it. }

\par{${\overset{\textnormal{}}{\text{27. 持}}}$ っていってください。 \hfill\break
Please take it (away). }

\par{28. ペン一本も持ってきませんでした。 \hfill\break
I didn't bring a single pen. }

\par{29. プレゼントに銀ドルを持っていきました。 \hfill\break
I took a silver dollar as a gift. }

\par{30. ちょっとお金を忘れてきちゃった。貸してくれない。(Casual) \hfill\break
I forgot to bring some money. Could you lend me some? }

\par{${\overset{\textnormal{}}{\text{31. 雪}}}$ が ${\overset{\textnormal{}}{\text{降}}}$ ってきたかと ${\overset{\textnormal{}}{\text{思}}}$ うともうやんだ。 \hfill\break
Just as I thought it had started to snow, it had already stopped. }
 
\par{\textbf{Phrase Note }: かと思うと = "just as I thought". }

\par{${\overset{\textnormal{}}{\text{32. 彼}}}$ を ${\overset{\textnormal{つ}}{\text{連}}}$ れてきてください。 \hfill\break
Please bring him. }

\par{${\overset{\textnormal{}}{\text{33. 彼}}}$ を ${\overset{\textnormal{}}{\text{連}}}$ れていきなさい。 \hfill\break
Take him away. }

\par{34. 今夜は岡田さんを連れてきました。 \hfill\break
I brought Mr. Okada along tonight. }

\par{35. シネコンに行くんだったら、僕も連れて行ってくれない。 \hfill\break
If you're going to the cinema complex, can't you take me along too? }

\par{\textbf{Word Note }: Notice the contrast in definitions between 連れて[いく・くる] and 持って[いく・くる] }

\par{${\overset{\textnormal{}}{\text{36. 彼}}}$ が ${\overset{\textnormal{}}{\text{東京}}}$ を ${\overset{\textnormal{た}}{\text{発}}}$ ってから ${\overset{\textnormal{}}{\text{五日後}}}$ に ${\overset{\textnormal{}}{\text{私}}}$ は ${\overset{\textnormal{}}{\text{帰}}}$ ってきた。 \hfill\break
I came back home after five days after he left Tokyo. }

\par{37. ${\overset{\textnormal{にちじょう}}{\text{\{日常}}}$ ・日々\}の ${\overset{\textnormal{せいかつもんだい}}{\text{生活問題}}}$ が\{ ${\overset{\textnormal{かさ}}{\text{重}}}$ なってくる・ ${\overset{\textnormal{つ}}{\text{積}}}$ み ${\overset{\textnormal{}}{\text{重}}}$ なる\}でしょう。 \hfill\break
Normal daily life problems will pile on. }

\par{38. ヒマワリの種をまいてから、一週間とたたないうちに芽が出てきました。 \hfill\break
After I planted the sunflower seeds, they sprouted in no less than a week. }

\par{${\overset{\textnormal{}}{\text{39. 彼女}}}$ は ${\overset{\textnormal{}}{\text{先}}}$ ${\overset{\textnormal{}}{\text{月}}}$ ${\overset{\textnormal{}}{\text{神}}}$ ${\overset{\textnormal{}}{\text{戸}}}$ に( ${\overset{\textnormal{}}{\text{引}}}$ っ) ${\overset{\textnormal{こ}}{\text{越}}}$ していきました。 \hfill\break
She moved to Kobe last month. }

\par{40. これからも頑張っていきたいと思います。 \hfill\break
I'd like to continue to do my best from now on. }

\par{${\overset{\textnormal{}}{\text{41a. 彼}}}$ が ${\overset{\textnormal{}}{\text{私}}}$ の ${\overset{\textnormal{へや}}{\text{部屋}}}$ に ${\overset{\textnormal{と}}{\text{飛}}}$ び ${\overset{\textnormal{こ}}{\text{込}}}$ んできました。 \hfill\break
41b. He rushed into my room. }

\begin{center}
\textbf{~ていく \textrightarrow  ゆく }
\end{center}

\par{ ~ゆく may seldom follow the stem of verbs in a poetic fashion. This is in fact the original form the pattern took, which is why it is deemed poetic\slash nostalgic. ~てゆく also exists, which is quite nostalgic and is very common in songs and literature. }
 
\par{42. 大事な思い出も記憶から消え(て)ゆくのであろう。 \hfill\break
Even important memories will slip from your memory. }

\par{43. 花燃えゆく。 \hfill\break
Flowers, burn away. }

\par{${\overset{\textnormal{}}{\text{44. 死}}}$ にゆく。 \hfill\break
To be dying. }

\begin{center}
\textbf{なってくる VS なっていく } 
\end{center}

\par{ Although the previous information may be sufficient enough to make the differences between these two phrases easy enough to ascertain, given that they are still often misunderstood, it's best to go through this as an independent point of discussion. First, consider the following. }

\par{48. こちらも、こんなに寒くなって\{きた・いったX\}のですから、そちらは、これからますます寒くなって\{いく・くるX\}ことでしょうね。 \hfill\break
Since it has become cold like this here, it will definitely get steadily cold there after this. }

\par{ This is a wonderful example of how the two can even be seen in the same complex sentence but yet have zero interchangeability. The first part states the change that it has become cold and still is cold \textbf{at the speaker's location }. In saying so, the speaker makes a conjecture that it \textbf{will get }cold somewhere else \textbf{away }from the speaker. These distinctions are crucial to keep in mind. Here are some more sentences to consider. }

\par{49a. 10月だ。故郷は、もう寒くなってきただろうか。 〇 \hfill\break
49b. 10月だ。故郷は、もう寒くなっていっただろうか。X \hfill\break
It's October. Hasn't it already gotten cold at our hometown? }

\par{50. 5月も半ばだから、そろそろ暑くなってくるね。梅雨が明けると、それこそ、ますます暑くなってくるんだろうけど。 \hfill\break
Since it's already the middle of May, it should be getting hot soon, right? Once the rainy season lifts, that alone should make it get hotter gradually. }

\par{51. 6ヶ月もダイエット続けてるのに、太ってくばかりで、ちっともやせてこないんだよ。 \hfill\break
Although I've been on this diet for 6 months, I've only continued to gain and haven't gotten a bit skinnier! }

\par{ Though the last may be harder to follow from the free translation, it would be a great mistake to change the speech modals to one or the other. When used with intransitive verbs that express change, you are showing a natural change of events by a given process. When you use ~ていく, you say that at the time the change starts, there will continue to be change from that point onward. ~てくる gives a more punctual feel to the instant of change. So, ~ていった would be like "had been getting\dothyp{}\dothyp{}\dothyp{}" whereas ~てきた would be like "got". }
    