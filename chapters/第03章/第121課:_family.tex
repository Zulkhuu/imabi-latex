    
\chapter{Family}

\begin{center}
\begin{Large}
第121課: Family 
\end{Large}
\end{center}
 
\par{ Family terms are different for one's family and someone's family. One's family is ${\overset{\textnormal{かぞく}}{\text{家族}}}$ . Another person's is ご ${\overset{\textnormal{}}{\text{家族}}}$ . ご is an honorific prefix that attaches to most Sino-Japanese words. }
 
\par{The traditional Japanese house structure, ${\overset{\textnormal{いえ}}{\text{家}}}$ , is very different from a nuclear family.  The ${\overset{\textnormal{}}{\text{家}}}$ is responsible for making contracts with Buddhist temples and is comprised of generations of married couples and unmarried children. Ancestors are thought to watch over their descendants. }
 
\par{The wife joins her husband's 家 and has no ritual obligations for her nuclear family. In real life one's nuclear family is still important, but death and lineage is dominated by one's ${\overset{\textnormal{}}{\text{家}}}$ . Only one child is a successor, and it's usually the oldest male child. Females are rarely chosen because they would be obligated for both families' rituals. Upon death, individuals who don't receive ritual care become ${\overset{\textnormal{むえんぼとけ}}{\text{無縁仏}}}$ . These people include singles, childless couples, etc. Family altar is ${\overset{\textnormal{ぶつだん}}{\text{仏壇}}}$ and family grave is ${\overset{\textnormal{はか}}{\text{墓}}}$ . Ashes of past generations are kept there. }
 
\par{Nowadays, some are veering from this. Some have their ashes scattered and others have a grave plot, but these are hard to get. }
      
\section{家族}
 
\par{ There are many important people in your family tree ${\overset{\textnormal{かけいず}}{\text{家系図}}}$ . Not all family terms have separate terms for one's family and someone else's family. When there is a term that has both an 音読み and a 訓読み, the first is usually for referring to them or in written terms. }

\begin{ltabulary}{|P|P|P|}
\hline 

Great-grandparents & 曾祖父母 & そうそふぼ \\ \cline{1-3}

Great-grandfather & 曽祖父 & そうそふ・ひ(い)じじ・ひおおじ \\ \cline{1-3}

Great-grandmother & 曾祖母 & そうそぼ・ひ(い)ばば・ひおおば \\ \cline{1-3}

Great-grandchild(ren) & 曾孫 & そうそん・ひ(い)まご・ひこ (Rare) \\ \cline{1-3}

Grandparents & 祖父母 & そふぼ \\ \cline{1-3}

Grandfather & 祖父 & そふ・おじ (Rare)・じじ(Dialect)・じい (~Old man)・おおじ (Rare) \\ \cline{1-3}

Grandmother & 祖母 & そぼ・ばば (Dialect\slash rude)・おおば (Rare) \\ \cline{1-3}

Grandchild(ren) & 孫 & まご \\ \cline{1-3}

Granduncle & 大叔・伯父 & おおおじ \\ \cline{1-3}

Grandaunt & 大叔・伯母 & おおおば \\ \cline{1-3}

Parents & 両親 & りょうしん \\ \cline{1-3}

Father & 父(親) & ちち(おや) \\ \cline{1-3}

Mother & 母(親) & はは(おや) \\ \cline{1-3}

Dad & パパ、おやじ & パパ、おやじ \\ \cline{1-3}

Mom & ママ、お袋 & ママ、おふくろ \\ \cline{1-3}

Uncle & 叔・伯父 & おじ \\ \cline{1-3}

Aunt & 叔・伯母 & おば \\ \cline{1-3}

Husband & 夫、旦那 & おっと、だんな \\ \cline{1-3}

Wife & 妻、女房、家内 & つま、にょうぼう、かない \\ \cline{1-3}

Son & 息子 & むすこ \\ \cline{1-3}

Daughter & 娘 & むすめ \\ \cline{1-3}

Older Brother & 兄(貴) & あに(き) \\ \cline{1-3}

Younger Brother & 弟 & おとうと \\ \cline{1-3}

Older Sister & 姉(貴) & あね(き) \\ \cline{1-3}

Younger Sister & 妹 & いもうと \\ \cline{1-3}

\end{ltabulary}

\par{\textbf{Word Notes }: }
 
\par{1. ${\overset{\textnormal{}}{\text{家内}}}$ is felt to be condescending because it suggests that women should be in the house. }
 
\par{2. For writing aunt and uncle, 叔 is used when they're younger and 伯 is used when they're older than your parents. }
 
\par{3. ${\overset{\textnormal{}}{\text{母親}}}$ and ${\overset{\textnormal{}}{\text{父親}}}$ shouldn't be used to address one's parents. ${\overset{\textnormal{}}{\text{母}}}$ and ${\overset{\textnormal{}}{\text{父}}}$ are similarly used. ${\overset{\textnormal{}}{\text{母}}}$ and ${\overset{\textnormal{}}{\text{父}}}$ may refer to the mother or father of anything also. }
 
\par{4. ~ ${\overset{\textnormal{}}{\text{貴}}}$ attached to ${\overset{\textnormal{}}{\text{兄}}}$ and ${\overset{\textnormal{}}{\text{姉}}}$ adds respect. }

\par{5. As you will learn, you will often or usually call family members with the titles originally for someone else's family. Some here include one's mother, father, grandfather, and grandmother. 兄 is often like this and 姉 is even more so replaced by the term originally reserved for someone else's older sister. }

\begin{center}
\textbf{Examples } 
\end{center}

\par{${\overset{\textnormal{}}{\text{1. 私}}}$ は ${\overset{\textnormal{きょうだい}}{\text{兄弟}}}$ が3 ${\overset{\textnormal{}}{\text{人}}}$ います。 \hfill\break
I have three siblings. }

\par{${\overset{\textnormal{}}{\text{2. 母}}}$ なる ${\overset{\textnormal{}}{\text{大地 (Set Phrase)}}}$ \hfill\break
Mother Earth }

\par{${\overset{\textnormal{}}{\text{3. 姉}}}$ は ${\overset{\textnormal{}}{\text{母}}}$ に似ている。 \hfill\break
My older sister resembles my mom . }
      
\section{ご家族}
 
\par{ The main difference between the words for one's family with that of another is that someone's family is addressed with more honorific terms. The list below illustrates these terms with the same rules of age applying for aunt and uncle as before. }

\begin{ltabulary}{|P|P|P|}
\hline 

Great-grandfather & 曾お祖父さん & ひおじいさん \\ \cline{1-3}

Great-grandmother & 曾お祖母さん & ひおばあさん \\ \cline{1-3}

Great-grandchild(ren) & 曾孫さん & ひまごさん \\ \cline{1-3}

Grandfather & お祖父さん & おじいさん \\ \cline{1-3}

Grandmother & お祖母さん & おばあさん \\ \cline{1-3}

Grandchild(ren) & お孫さん & おまごさん \\ \cline{1-3}

Granduncle & 大叔・伯父さん & おおおじさん \\ \cline{1-3}

Grandaunt & 大叔・伯母さん & おおおばさん \\ \cline{1-3}

Parents & ご両親 & ごりょうしん \\ \cline{1-3}

Father & お父さん & おとうさん \\ \cline{1-3}

Mother & お母さん & おかあさん \\ \cline{1-3}

Uncle & 叔・伯父さん & おじさん \\ \cline{1-3}

Aunt & 叔・叔母さん & おばさん \\ \cline{1-3}

Husband & ご主人、旦那さん & ごしゅじん、だんなさん \\ \cline{1-3}

Wife & 奥さん & おくさん \\ \cline{1-3}

Child & お子さん & おこさん \\ \cline{1-3}

Son & 息子さん & むすこさん \\ \cline{1-3}

Daughter & 娘さん、お嬢さん & むすめさん、おじょうさん \\ \cline{1-3}

Older Brother & お兄さん & おにいさん \\ \cline{1-3}

Younger Brother & 弟さん & おとうとさん \\ \cline{1-3}

Older Sister & お姉さん & おねえさん \\ \cline{1-3}

Younger Sister & 妹さん & いもうとさん \\ \cline{1-3}

Siblings & ご兄弟 & ごきょうだい \\ \cline{1-3}

\end{ltabulary}

\par{4. 彼女は ${\overset{\textnormal{がいけん}}{\text{外見}}}$ はお ${\overset{\textnormal{ねえ}}{\text{姉}}}$ さんと似ているが、 ${\overset{\textnormal{せいかく}}{\text{性格}}}$ は ${\overset{\textnormal{こと}}{\text{異}}}$ なるよ。 \hfill\break
She takes after her sister in appearance, but their characters differ. }
 \textbf{会話: }\textbf{アメリカ人 ${\overset{\textnormal{りゅう}}{\text{留}}}$ \textbf{${\overset{\textnormal{がく}}{\text{学}}}$ \textbf{${\overset{\textnormal{せい}}{\text{生}}}$ \textbf{、セス・クンロードがホームステイ ${\overset{\textnormal{さき}}{\text{先}}}$ \textbf{のお ${\overset{\textnormal{}}{\text{母}}}$ \textbf{さんと話している。 }}}}}} 
\par{5. }

\par{お母さん: セス、 ${\overset{\textnormal{}}{\text{家族}}}$ の ${\overset{\textnormal{しゃしん}}{\text{写真}}}$ (を)持ってる? \hfill\break
セス:ええ、あります。 \hfill\break
お母さん: ちょっと見せてもらってもいい? \hfill\break
セス:ええ。 \hfill\break
お母さん:へえ。これがお ${\overset{\textnormal{}}{\text{父}}}$ さん? \hfill\break
セス:ええ、 ${\overset{\textnormal{}}{\text{父}}}$ です。 \hfill\break
お母さん:セスは、お父さんによく似て(い)るね。 \hfill\break
セス:ええ、よく言われます。それからこっちが ${\overset{\textnormal{}}{\text{弟}}}$ です。 \hfill\break
お母さん:ああ、弟さんか。今、大学生? \hfill\break
セス:いいえ、弟は私より4つ下で、今高校生です。 \hfill\break
お母さん:ずいぶん若く見えるわねえ。とってもかわいい弟です。 }
 
\begin{center}
${\overset{\textnormal{ごい}}{\text{語彙}}}$ Vocabulary 
\end{center}
 
\par{写真 = Picture    高校生 = High school student    若い = Young   よく言われます =  said a lot \hfill\break
~に似ている = To resemble\dothyp{}\dothyp{}\dothyp{} }
      
\section{Neutral Family Terms}
 
\par{ Despite there being a great divide in family terminology, there are still  many terms that are neutral and may be used for both one's own family  and someone's family. Below is a list of the most common of these  neutral terms. }

\begin{ltabulary}{|P|P|P|}
\hline 

Nephew & 甥 & おい \\ \cline{1-3}

Niece & 姪 & めい \\ \cline{1-3}

Cousin & いとこ & いとこ \\ \cline{1-3}

Second cousin & はとこ・またいとこ・いとこの子 & はとこ・またいとこ・いとこのこ \\ \cline{1-3}

Father and Mother & 父母 & ふぼ \\ \cline{1-3}

Sisters & 姉妹 & しまい \\ \cline{1-3}

Brothers\slash Siblings & 兄弟 & きょうだい \hfill\break
 \\ \cline{1-3}

Younger Siblings & 弟妹 & ていまい \\ \cline{1-3}

Brother-in-law & 義兄弟 & ぎきょうだい \\ \cline{1-3}

Brother-in-law & 義弟 & ぎてい \\ \cline{1-3}

Older sister-in-law & 義姉 & ぎし \\ \cline{1-3}

Younger sister-in-law & 義妹 & ぎまい \\ \cline{1-3}

Son-in-law & 娘の夫、(婿)婿、女婿 & むすめのおっと、(むすめ)むこ、じょせい \hfill\break
\\ \cline{1-3}

Daughter-in-law & 息子の妻、嫁、義理の娘 & むすこのつま、よめ、ぎりのむすめ \\ \cline{1-3}

Father-in-law & 義理の父、義父、舅 & ぎりのちち、ぎふ、しゅうと \\ \cline{1-3}

Mother-in-law & 義理の母、義母、姑 & ぎりのはは、ぎぼ、しゅうと \\ \cline{1-3}

Relatives & 親族、身寄り & しんぞく、みより \\ \cline{1-3}

Ancestors & 祖先、先祖 & そせん、せんぞ \\ \cline{1-3}

\end{ltabulary}

\par{Family members in this group can still be made polite in reference to someone else. For example, the nephew and niece of another person can be referred to as ${\overset{\textnormal{}}{\text{甥}}}$ ごさん and ${\overset{\textnormal{}}{\text{姪}}}$ ごさん respectively. Other family members are simply discussed, if necessary, with ~さん to be polite. ${\overset{\textnormal{おやご}}{\text{親御}}}$ is a similar looking word that may respectfully be used to address someone's parents. }

\begin{center}
 \textbf{Ancestor }
\end{center}

\par{  先祖 typically refers to one's ancestors, particularly those in your 家. You may hear people refer to them respectfully as ご先祖さま. You never hear ご祖先さま because 祖先 is impersonal. 祖 refers to the founder of a lineage, dynasty, or even a field of study. "Ancestor" as in biological origin is normally 始祖. }

\begin{center}
\textbf{After Grandchildren }
\end{center}

\par{ It's interesting to know that Japanese has words for those who have not just great-grandchildren, but offspring for 8 generations ahead of you (just in case you live that long). The terms up to great-great-grandchildren are usually known by most speakers. The others are just fun to look at. }

\begin{ltabulary}{|P|P|P|}
\hline 

Child & 子 & こ \\ \cline{1-3}

Grandchild & 孫 & まご \\ \cline{1-3}

Great-grandchild & ひ孫 & ひまご \\ \cline{1-3}

Great-great-grandchild & 玄孫 & やしゃご \\ \cline{1-3}

Great-great-great-grandchild & 来孫 & らいそん \\ \cline{1-3}

Great-great-great-great-grandchild & 昆孫 & こんそん \\ \cline{1-3}

Great-great-great-great-great-grandchild & 仍孫 & じょうそん \\ \cline{1-3}

Great-great-great-great-great-great-grandchild & 雲孫 & うんそん \\ \cline{1-3}

\end{ltabulary}
      
\section{How to Write いとこ}
 
\par{ The word for cousin, いとこ, is a very odd word to write in Japanese. It is written differently depending on the age and gender relationship with the speaker. }

\begin{ltabulary}{|P|P|P|P|}
\hline 

Older Girls\slash Younger Guys & 従姉弟 & Older Guys\slash Younger Girls & 従兄妹 \\ \cline{1-4}

Older Girls & 従姉 & Older Guys & 従兄 \\ \cline{1-4}

Younger Girls & 従妹 & Younger Guys & 従弟 \\ \cline{1-4}

Guys & 従兄弟 & Girls & 従姉妹 \\ \cline{1-4}

\end{ltabulary}

\par{\textbf{Word Note }: You may also use the 音読み of the characters to distinguish. }
      
\section{Using Terms for Someone Else's Family for Your Own}
 
\par{ Many speakers will call their mom, dad, older brother or sister, and grandparents with the terms that are supposed to be for someone's family. But, if you use them for you own, you may slightly change their appearance. For example, instead of -さん you might choose -ちゃん or drop -さん altogether. It\textquotesingle s also even possible to drop the お- at the beginning of these phrases. }
 
\par{6. お ${\overset{\textnormal{}}{\text{母}}}$ さん、おもちゃほしい! \hfill\break
Mom, I want a toy! }
 
\par{7. お ${\overset{\textnormal{}}{\text{父}}}$ さん、ほんとにしゃべりすぎだ。 \hfill\break
Dad, you really talk too much. }
 
\par{${\overset{\textnormal{}}{\text{8. 姉}}}$ さん、 ${\overset{\textnormal{}}{\text{何}}}$ してる? \hfill\break
Hey sis', what are you doin'? }
      
\section{Stepfamily}
 \hfill\break

\begin{ltabulary}{|P|P|P|}
\hline 

 & Sino-Japanese & Native \\ \cline{1-3}

Stepmother & 継母(けいぼ) & 継母(ままはは) \\ \cline{1-3}

Stepfather & 継父(けいふ) & 継父(ままちち) \\ \cline{1-3}

Stepchild & 継子(けいし) & 継子(ままこ) \\ \cline{1-3}

Stepson &  & 継息子(ままむすこ) \\ \cline{1-3}

Stepdaughter &  & 継娘(ままむすめ) \\ \cline{1-3}

\end{ltabulary}
      
\section{Half Family}
 
\par{ To create the terms for half-brother and half-sister, you first must decide whether they are from a stepfather or a stepmother. For each combination, there is a technical, conversational, and insensitive term available. }

\par{ The technical terms utilize Sino-Japanese vocabulary. The conversational terms are the native vocabulary equivalents of the Sino-Japanese terms. The insensitive terms utilize the terms 種違い (of different seed) and 腹違 い (of a different womb) to refer to "by a different father" and "by a different mother" respectively. }

\begin{ltabulary}{|P|P|P|P|}
\hline 

Meaning & Technical Term & Conversational Term & Insensitive Term \\ \cline{1-4}

 \textbf{By a different father }& 異父(いふ)・・・ & 父親違いの・・・ & 種違いの・・・ \\ \cline{1-4}

Siblings & 異父兄弟(いふきょうだい) & 父親違いの兄弟 & 種違いの兄弟 \\ \cline{1-4}

Sisters & 異父姉妹(いふしまい) & 父親違いの姉妹 & 種違いの姉妹 \\ \cline{1-4}

Older brother & 異父兄(いふけい) & 父親違いの兄 & 種違いの兄 \\ \cline{1-4}

Older sister & 異父姉(いふし) & 父親違いの姉 & 種違いの姉 \\ \cline{1-4}

Younger brother & 異父弟(いふてい) & 父親違いの弟 & 種違いの弟 \\ \cline{1-4}

Younger sister & 異父妹(いふまい) & 父親違いの妹 & 種違いの妹 \\ \cline{1-4}

 \textbf{By a different mother }& 異母・・・ & 母親違いの・・・ & 腹違いの・・・ \\ \cline{1-4}

Siblings & 異母兄弟(いぼきょうだい) & 母親違いの兄弟 & 腹違いの兄弟 \\ \cline{1-4}

Sisters & 異母姉妹(いぼしまい) & 母親違いの姉妹 & 腹違いの姉妹 \\ \cline{1-4}

Older brother & 異母兄(いぼけい) & 母親違いの兄 & 腹違いの兄 \\ \cline{1-4}

Older sister & 異母姉(いぼし) & 母親違いの姉 & 腹違いの姉 \\ \cline{1-4}

Younger brother & 異母弟(いぼてい) & 母親違いの弟 & 腹違いの弟 \\ \cline{1-4}

Younger sister & 異母妹(いぼまい) & 母親違いの妹 & 腹違いの妹 \\ \cline{1-4}

\end{ltabulary}
    