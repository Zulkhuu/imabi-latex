    
\chapter{The Particles とか, など, \& なんか}

\begin{center}
\begin{Large}
第141課: The Particles とか, など, \& なんか 
\end{Large}
\end{center}
 
\par{ In this lesson we'll learn about additional particles that are used to list things colloquially: とか, など, and なんか. }
      
\section{The Adverbial Particle とか}
 
\par{ とか means "such as." It can also be interpreted as "something like". This particle may be seen after nouns, the end of conjugations such as the て形 and the 命令形. It is somewhat colloquial, and you can often see it at the end of a sentence fragment. }

\par{1. 美由紀さんはギターとか、ドラムとか、ピアノとかたくさんお ${\overset{\textnormal{けいこ}}{\text{稽古}}}$ に行ってるのよ。(女性言葉) \hfill\break
Miyuki is taking many lessons such as guitar, drum, and piano. }

\par{2. 民主主義は食料とか飲料のように輸出できるものではない。 \hfill\break
Democracy is not exportable like food or drink. }

\par{3. まだ用意とかできてねえよ。(乱暴な言い方) \hfill\break
I still can't, uh, get prepared. }

\par{4. Youtubeを見てないで、少しは犬を散歩につれてくとか、本を読むとか、してはどう?(Casual) \hfill\break
How about doing something like walk the dog or read a book without watching Youtube? }

\par{5. テニスとかサッカーとかアメフトとかの球技が大好きだからね。(Casual) \hfill\break
Because I love ball sports like tennis, soccer, and (American) football. \hfill\break
}

\par{6. 宏君のおばあさんは80いくつとかでまだフランス語を教えているんですって。 \hfill\break
Hiroshi's grandmother is something like 80 years old and is still teaching French. }
      
\section{The Adverbial Particle など}
 
\par{ など becomes なんか in colloquial settings. It is used to give related examples and may be seen after nouns or verbs. This particle is not usually used in list form like other similar particles such as と. However, when the need arises, it is often used together with the particle や. }

\par{\textbf{Translation Note }: The particle など is often also translated as "etc." }

\par{7. 私は休日は ${\overset{\textnormal{ざっし}}{\text{雑誌}}}$ を読むなどして ${\overset{\textnormal{す}}{\text{過}}}$ ごしました。 \hfill\break
I spent the holidays doing stuff like reading magazines. }
 
\par{8. 東京やニューヨーク市などの大都市には ${\overset{\textnormal{たいせい}}{\text{大勢}}}$ の人が住んでいます。 \hfill\break
A lot of people live in metropolises such as Tokyo and New York City. }
9. 音楽やビデオゲームなどが若者に愛用されている。(堅苦しい言い方) \hfill\break
Things such as music and video game are cherished by young people. 
\par{10. お茶・ ${\overset{\textnormal{さとう}}{\text{砂糖}}}$ ・塩など \hfill\break
Tea, sugar, salt and so on }
 
\par{11. レモンやミカンやタンジェリンなどの ${\overset{\textnormal{くだもの}}{\text{果物}}}$ を買い集めた。 \hfill\break
I bought up fruits such as lemons, mandarin oranges, and tangerines. }
 
\par{\textbf{漢字 Note }: Lemon and mandarin oranges can be written in 漢字 as 檸檬 and 蜜柑 respectively, but they are usually just spelled in カタカナ. }
 
\par{12. 彼など ${\overset{\textnormal{てきにんしゃ}}{\text{適任者}}}$ だね。 \hfill\break
Someone like him is suitable, right? }
 
\par{13. このパソコンなどお ${\overset{\textnormal{か}}{\text{買}}}$ い ${\overset{\textnormal{どく}}{\text{得}}}$ です。 \hfill\break
Something like this personal computer is a bargain. }
 
\par{14. ヨーロッパの ${\overset{\textnormal{しんよう}}{\text{信用}}}$ 不安で海外経済が ${\overset{\textnormal{げんそく}}{\text{減速}}}$ していることなどが ${\overset{\textnormal{えいきょう}}{\text{影響}}}$ している。 \hfill\break
Events such as overseas economies decelerating due to uneasiness of confidence in Europe are taking an effect. }
 
\par{15. こりゃなんかええじゃん? (Casual) \hfill\break
Isn't something like this OK? }
 
\par{16. 彼女になんか会わなければよかった。 \hfill\break
If I had somehow not met her, it would have been good. }
 
\par{17. 私はテレビでアニメなどは見ません。 \hfill\break
I don't watch things like anime on TV. }

\par{18. ${\overset{\textnormal{きゅうか}}{\text{休暇}}}$ 中に ${\overset{\textnormal{せんたく}}{\text{洗濯}}}$ や ${\overset{\textnormal{そうじ}}{\text{掃除}}}$ などしなくてもいい。(堅苦しい) \hfill\break
休みなのに洗濯とか掃除とかしなくてもいいじゃない。(Casual) \hfill\break
It's all right to not do laundry, cleaning and such on vacation. }

\par{19. ${\overset{\textnormal{ひょうしょう}}{\text{表彰}}}$ なんか受けたくもねーし。( ${\overset{\textnormal{くだ}}{\text{砕}}}$ けた言い方) \hfill\break
I don't want any public acknowledgement. }
 
\par{20. 泣いてなどいられん。(Casual; dialectical) \hfill\break
I can't just cry. }

\par{21. 「ああ、佐伯あさっちゃんのところに行ったの。あんなヤツの言うことなんか、嘘っぱちだよ」 \hfill\break
"Ahh, Sahaku went to Satchan's place? What that guy says is a downright lie. \hfill\break
From 冷たい誘惑 by 乃南アサ. }
 
\par{22. 今さら行くなどと言ってももう ${\overset{\textnormal{おそ}}{\text{遅}}}$ いよ。 \hfill\break
Even if you say you're going to go now, you're already later. }

\par{23. ${\overset{\textnormal{かっちゅう}}{\text{甲冑}}}$ なども ${\overset{\textnormal{せいこう}}{\text{精巧}}}$ に ${\overset{\textnormal{でき}}{\text{出来}}}$ ている。 \hfill\break
Things like the armor are also exquisitely made. }

\par{\textbf{漢字 Note }: 冑 is not a 常 用漢字. So, don't worry about it. }

\par{\textbf{Variants' Note }: など may also be seen as なんど, なぞ and なんぞ. These are all dialectical. So, who uses them is dependent on the region in question. }
    