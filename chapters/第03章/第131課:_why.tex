    
\chapter{Why}

\begin{center}
\begin{Large}
第131課: Why: なぜ, どうして, \& なんで 
\end{Large}
\end{center}
 
\par{ The words for “why” in Japanese are relatively easy to understand, but in Japanese culture, “why” questions are frequently avoided so as not to be too direct, pushy, or rude. Sometimes, asking why can be offensive or uncalled for. Nevertheless, asking why is also sometimes unavoidable. As such, we will spend time looking at the words for “why” so that you can both ask and be asked why. }
      
\section{なぜ}
 
\par{ The basic word for “why?” in Japanese is なぜ. In 漢字,  this is spelled as 何故. For the most part, なぜ is logically\slash objectively based. This makes it appropriate for all speech styles, although tone will drastically change how people perceive it. }

\par{ To use these words grammatically, there are a few places where なぜ can be placed. When at the front of the sentence, the topic of the question is marked by は. The sentence itself ends in some form of のか. の is usually contracted to ん in the spoken language, but in the written language this is generally not the case. }

\par{1. なぜ ${\overset{\textnormal{ちきゅう}}{\text{地球}}}$ は ${\overset{\textnormal{まる}}{\text{丸}}}$ いのですか。 \hfill\break
Why is the Earth round? }

\par{2. なぜ ${\overset{\textnormal{ほうげん}}{\text{方言}}}$ や ${\overset{\textnormal{なま}}{\text{訛}}}$ りなどが ${\overset{\textnormal{う}}{\text{生}}}$ まれたのですか。また、なぜ ${\overset{\textnormal{とうきょう}}{\text{東京}}}$ の ${\overset{\textnormal{ほうげん}}{\text{方言}}}$ は ${\overset{\textnormal{ひょうじゅんご}}{\text{標準語}}}$ なのですか。 \hfill\break
Why do dialects, accents, and the like come into existence? Also, why is the dialect of Tokyo the standard dialect? }

\par{3. ${\overset{\textnormal{そら}}{\text{空}}}$ はなぜ ${\overset{\textnormal{あお}}{\text{青}}}$ いの? \hfill\break
Why is the sky blue? }

\par{\textbf{Grammar Note }: As Ex. 3 demonstrates, the topic and なぜ can be flipped with no changes made to particle usage in the sentence. }

\par{4. なぜワシントン ${\overset{\textnormal{しゅう}}{\text{州}}}$ に引っ ${\overset{\textnormal{こ}}{\text{越}}}$ すんですか。 \hfill\break
Why did you move to the State of Washington? }

\par{\textbf{Grammar Note }: Ex. 4 would be spoken, not written. The question innocently asks why the person in question moved to the State of Washington. The question does not imply interrogation or pressure for the person to spill some sort of truth. }

\par{ A college classmate lover began dating another woman, and she couldn\textquotesingle t fathom that she herself was dumped and why the person chosen was some other person and not herself. }

\par{5. ちなみになぜこの ${\overset{\textnormal{しょうひん}}{\text{商品}}}$ を ${\overset{\textnormal{き}}{\text{気}}}$ に ${\overset{\textnormal{い}}{\text{入}}}$ って ${\overset{\textnormal{いただ}}{\text{頂}}}$ いたのですか? \hfill\break
By the way, why did you take interest in this product? }

\par{6. お ${\overset{\textnormal{かね}}{\text{金}}}$ が貯まる ${\overset{\textnormal{ひと}}{\text{人}}}$ はなぜ ${\overset{\textnormal{じかん}}{\text{時間}}}$ の ${\overset{\textnormal{つか}}{\text{使}}}$ い ${\overset{\textnormal{かた}}{\text{方}}}$ がうまいのか。 \hfill\break
Why is it that people whose money is being saved up good at using time? }

\par{\textbf{Nuance Note }: Ex. 6 sounds as if it is a header to some document\slash post. It is not the plain speech のか that you hear in coarse conversations. That usage is to be seen shortly in this lesson. }

\par{7. ${\overset{\textnormal{おな}}{\text{同}}}$ じアマゾン ${\overset{\textnormal{ない}}{\text{内}}}$ で、 ${\overset{\textnormal{おな}}{\text{同}}}$ じ ${\overset{\textnormal{しょうひん}}{\text{商品}}}$ なのに ${\overset{\textnormal{かかく}}{\text{価格}}}$ が ${\overset{\textnormal{ちが}}{\text{違}}}$ うのはなぜですか。 \hfill\break
In Amazon itself, why is it that something may be the same product but have a different price? }

\par{8. なぜ、この ${\overset{\textnormal{ことば}}{\text{言葉}}}$ を ${\overset{\textnormal{き}}{\text{聞}}}$ くと ${\overset{\textnormal{ふかい}}{\text{不快}}}$ に ${\overset{\textnormal{おも}}{\text{思}}}$ ってしまうのでしょうか? \hfill\break
“Why,” do you not feel uncomfortable when you hear this word? }

\par{\textbf{Nuance Note }: As with any word in Japanese, “why” can also be the word of conversation. }

\par{9. なぜなら、 ${\overset{\textnormal{こた}}{\text{答}}}$ えは、 ${\overset{\textnormal{じぶん}}{\text{自分}}}$ のすぐ「 ${\overset{\textnormal{となり}}{\text{隣}}}$ 」にあったからです。 \hfill\break
That is because the answer is right \emph{beside }oneself. }

\par{\textbf{Grammar Note }: When used with the particle なら, なぜ is used to explain reasoning. You can see it as literally meaning “if you ask why…” }

\par{ In Ex. 10 and Ex. 11, なぜ is used in embedded questions. It is typically the word of choice in this situation due its objective nature. }

\par{10. ${\overset{\textnormal{かのじょ}}{\text{彼女}}}$ は、 ${\overset{\textnormal{だいがく}}{\text{大学}}}$ の ${\overset{\textnormal{どうきゅうせい}}{\text{同級生}}}$ である ${\overset{\textnormal{こいびと}}{\text{恋人}}}$ が、 ${\overset{\textnormal{べつ}}{\text{別}}}$ の ${\overset{\textnormal{おんな}}{\text{女}}}$ と ${\overset{\textnormal{こうさい}}{\text{交際}}}$ をはじめて、 ${\overset{\textnormal{じぶん}}{\text{自分}}}$ は ${\overset{\textnormal{す}}{\text{捨}}}$ てられた、なぜ ${\overset{\textnormal{えら}}{\text{選}}}$ ばれたのが ${\overset{\textnormal{たほう}}{\text{他方}}}$ であり ${\overset{\textnormal{じぶん}}{\text{自分}}}$ でないのかがどう ${\overset{\textnormal{かんが}}{\text{考}}}$ えても ${\overset{\textnormal{わ}}{\text{分}}}$ からなかった。 \hfill\break
A college classmate's lover began dating another woman, and she couldn\textquotesingle t fathom that she herself was dumped and why the person chosen was some other person and not herself. \hfill\break
 \hfill\break
11. ${\overset{\textnormal{みな}}{\text{皆}}}$ がなぜ、いきなり ${\overset{\textnormal{しゅうだん}}{\text{集団}}}$ ヒステリーのように ${\overset{\textnormal{な}}{\text{泣}}}$ き ${\overset{\textnormal{だ}}{\text{出}}}$ したのか、 ${\overset{\textnormal{ふしぎ}}{\text{不思議}}}$ でたまらなかった。 \hfill\break
It was a wonder to me why everyone started suddenly crying out in mass hysteria. }

\par{ We as humans intuitively infer the tone of a question based on its content. Below, each example is confrontational. This is when なぜ will go from sounding objective and calm to sounding confrontational and out of anger. }

\par{12. 「なぜ ${\overset{\textnormal{ちこく}}{\text{遅刻}}}$ をしたの?」 \hfill\break
“Why were you late?” }

\par{13. 「なぜ ${\overset{\textnormal{きづ}}{\text{気付}}}$ かなかったの?」 \hfill\break
“Why didn\textquotesingle t you notice?” }

\par{14. 「なぜ ${\overset{\textnormal{わす}}{\text{忘}}}$ れてきたんだ?」 \hfill\break
“Why did you come and forget it?” }

\begin{center}
\textbf{なにゆえ } 
\end{center}

\par{ As you may have noticed, the characters of 何故 imply that it should normally be read as なにゆえ. It just so happens that this is also a valid reading, but it is an archaism. Its place in Modern Japanese is relegated to jokingly sounding old-fashioned like in Ex. 15 or in neo-classical contexts. }

\par{15. なにゆえここに ${\overset{\textnormal{きみ}}{\text{君}}}$ がいるのか。 \hfill\break
Why is it that you are here? }
      
\section{どうして}
 
\par{ どうして is likely the word for “why”  you have incidentally heard over and over. This is because it is far more frequently emotionally based. It also happens to be somewhat more casual than なぜ as an effect. }

\par{ This word is composed of どう meaning “how\slash in what way?” and して, the gerund of する. Thus, it intrinsically means “how.” This is demonstrated in the following examples. }

\par{16. ${\overset{\textnormal{まいあさ}}{\text{毎朝}}}$ はどうして ${\overset{\textnormal{じかん}}{\text{時間}}}$ を ${\overset{\textnormal{つぶ}}{\text{潰}}}$ そうかと ${\overset{\textnormal{かんが}}{\text{考}}}$ えていた。 \hfill\break
Every morning I pondered how I would pass the time. }

\par{\textbf{Grammar Note }: When used with the volitional form + the particle か, the meaning of “how” becomes especially obvious. }

\par{17. どうして ${\overset{\textnormal{ぼく}}{\text{僕}}}$ が ${\overset{\textnormal{しけん}}{\text{試験}}}$ に ${\overset{\textnormal{う}}{\text{受}}}$ かったって ${\overset{\textnormal{し}}{\text{知}}}$ ってますか? \hfill\break
Do you know how I passed the exam? }

\par{\textbf{Sentence Note }: When the sentence ends in “do you know…?” it makes more sense to interpret どうして as “how” rather than “why” because even if you were to rephrase the English translation to use “why” instead, the question still intrinsically revolves around method.  This means that even なぜ to some extent may even ask “how” like in Ex. 18. }

\par{18. なぜ ${\overset{\textnormal{せんぷうき}}{\text{扇風機}}}$ は ${\overset{\textnormal{はたら}}{\text{働}}}$ くのか. \hfill\break
How fans works\slash how do fans work\slash why (do) fans work(?) }

\par{\textbf{Sentence Note }: Even in English, there is some fluctuation that can be seen between “why” and “how.” The Japanese phrasing could be used in introducing a question of discussion or even as the title of something. }

\par{ In related phrases, the literal breakdown of どうして meaning “how” becomes even more apparent; especially in どうしても, which either means “no matter what cost” or “no matter what,” the latter being used in a negative connotation. }

\par{19. どうしても ${\overset{\textnormal{きみ}}{\text{君}}}$ を ${\overset{\textnormal{うしな}}{\text{失}}}$ いたくない。 \hfill\break
I don\textquotesingle t want to lose you no matter what. }

\par{20. どうしても ${\overset{\textnormal{なっとく}}{\text{納得}}}$ できません。 \hfill\break
I can\textquotesingle t accept this by any means. }

\par{ Unlike all other words for “why,” どうして has evolved other usages. Occasionally, you will find it used in complex adverbial phrases with a meaning of “contrary to (expectations).” }

\par{21. なかなかどうしてうまくいかないものだ。 \hfill\break
Things aren\textquotesingle t going well as expected. }

\par{\textbf{Phrase Note }: なかなかどうして is a set phrase which is synonymous with ${\overset{\textnormal{おも}}{\text{思}}}$ ったより. }

\par{22. ちょっと ${\overset{\textnormal{しんぱい}}{\text{心配}}}$ だったのですが、どうしてなかなかしっかりしてるよ。 \hfill\break
I was a little worried, but (it\textquotesingle s) holding quite sturdy contrary to expectations. }

\par{\textbf{Grammar Note }: As you can see, for this meaning to make sense, the sentence itself must not be formed as a question but as a declarative statement. Furthermore, for どうして to be used for this meaning, it needs to be followed by some line that brings up a past expectation. }

\par{ Another grammar pattern that works on the sense of “how” is どうして…られよう(か).  This utilizes the potential form with the ending よう, which you may remember as originally also being used to indicate “surely” like in だろう・でしょう. This means we get an expression on the lines of “how could…possibly…?” It is inherently rhetorical and emphatic. }

\par{23. そんな ${\overset{\textnormal{すがた}}{\text{姿}}}$ を ${\overset{\textnormal{み}}{\text{見}}}$ てどうして ${\overset{\textnormal{だま}}{\text{黙}}}$ っていられようか。 \hfill\break
How could one stay silent after having seen that appearance? }

\par{24. どうして ${\overset{\textnormal{わす}}{\text{忘}}}$ れられよう。どうして ${\overset{\textnormal{わす}}{\text{忘}}}$ れていられようか。 \hfill\break
How could I forget? How I could I ever forget? }

\par{25. こんな ${\overset{\textnormal{よ}}{\text{世}}}$ の ${\overset{\textnormal{なか}}{\text{中}}}$ でどうして ${\overset{\textnormal{わら}}{\text{笑}}}$ っていられようか。 \hfill\break
How could one ever remain laughing in such a world as this? }

\par{26. どうして ${\overset{\textnormal{ひと}}{\text{人}}}$ の ${\overset{\textnormal{し}}{\text{死}}}$ を ${\overset{\textnormal{よろこ}}{\text{喜}}}$ んでいられようか。 \hfill\break
How could one ever be rejoiceful in the death of another person? }

\par{ It may come as no surprise that at one point, どうして could also be used as an interjection similar to “oh my!” like in Ex. 27. This, however, is not so common nowadays, but it still readily understood due to its occasional use in writing. }

\par{27. どうして、 ${\overset{\textnormal{たいそう}}{\text{大層}}}$ な ${\overset{\textnormal{あくやく}}{\text{悪役}}}$ ぶりだな。 \hfill\break
Oh my, what the villain-type you are. }

\par{ Now, you be may thinking, is どうして used to mean “how”? No, not quite… In the same way the concept of “why” and “how” are intertwined, so is the case with どうして. As a stand-alone sentence, all you need is どうして or the addition of ですか if used in semi-casual polite speech. However, when more is added to the sentence, some form of のか is grammatically necessary. }

\par{ Remember that unlike なぜ, どうして tends to be emotionally and\slash or subjectively based. Although it is far more common in conversation due to its semi-casual nature, its overuse will draw upon more emotion than you intend. }

\par{28. どうしてですか。 \hfill\break
Why (is that)? }

\par{29. どうしてまた ${\overset{\textnormal{おそ}}{\text{遅}}}$ くなったんですか。 \hfill\break
Why were you late again? }

\par{30. どうして ${\overset{\textnormal{じぶん}}{\text{自分}}}$ を ${\overset{\textnormal{おとこ}}{\text{男}}}$ だと ${\overset{\textnormal{おも}}{\text{思}}}$ うんですか。 \hfill\break
Why do you see yourself as male? }

\par{31. ${\overset{\textnormal{きのう}}{\text{昨日}}}$ 、どうして ${\overset{\textnormal{こ}}{\text{来}}}$ なかったの? \hfill\break
How come you didn\textquotesingle t come yesterday? }

\par{\textbf{Sentence Note }: As Ex. 31 demonstrates, it is sometimes more natural to translate どうして as “how come?” This makes the intertwined dichotomy of how-why even more apparent. }

\par{32. どうして ${\overset{\textnormal{な}}{\text{泣}}}$ くの。どうして ${\overset{\textnormal{な}}{\text{泣}}}$ き ${\overset{\textnormal{や}}{\text{止}}}$ まない。 \hfill\break
Why do you cry? Why do you not stop…? }

\par{\textbf{Sentence Note }: The lack of のか, interestingly enough, creates a sense of despair. It also makes it clear that the sentence is part of the speaker\textquotesingle s inner monologue. The speaker is distraught. One can imagine the next thought of the speaker being tied to whatever troubles are besieging the person. }

\par{33. どうしてあくびが ${\overset{\textnormal{で}}{\text{出}}}$ るの? \hfill\break
Why are you yawning? }
      
\section{なんで}
 
\par{ なんで is composed of 何 meaning “what” and the particle で, which can be seen as marking the instrumental case in this phrase. The instrumental case marks means\slash method. This means that なんで is akin to “how come?” }

\par{ なんで is more casual than anything else. It doesn\textquotesingle t particularly have any sense of objectivity or subjectivity. This is why it is used heavily in casual conversation. However, just as with the other words for “why,” it too can be used in tense situations or in questions based in intimidation. Nevertheless, think of how teenagers say “why” and equate it to how people would generally use なんで. By doing so, and by reviewing the examples below, you should get a good picture of how なんで is used. }

\par{ Grammatically speaking, a sentence with なんで ends with some form of のか. Without it, the question will sound brisk and abrupt. Of course, in a stand-alone sentence, it can be said by itself or followed by ですか in semi-casual speech. }

\par{34. ${\overset{\textnormal{とく}}{\text{特}}}$ に、 ${\overset{\textnormal{じんせいけいけん}}{\text{人生経験}}}$ 、 ${\overset{\textnormal{せいこうたいけん}}{\text{成功体験}}}$ の ${\overset{\textnormal{すく}}{\text{少}}}$ ない ${\overset{\textnormal{こ}}{\text{子}}}$ どもたちに ${\overset{\textnormal{む}}{\text{向}}}$ かって、なんで?なぜ?どうして?の ${\overset{\textnormal{と}}{\text{問}}}$ いかけは ${\overset{\textnormal{ひゃくがい}}{\text{百害}}}$ あって ${\overset{\textnormal{いちり}}{\text{一利}}}$ なしです。 \hfill\break
Especially towards child, who have little life experience and experience succeeding, questioning them with “why, why, why?” brings a hundred harms and not a single gain. }

\par{35. なんでテストで ${\overset{\textnormal{ひゃく}}{\text{100}}}$ ${\overset{\textnormal{てん}}{\text{点}}}$ ${\overset{\textnormal{と}}{\text{取}}}$ ったんだ? \hfill\break
How come you got a 100 on your test? }

\par{\textbf{Sentence Note }: Ex. 35 is an example of how being asked about something good can sound intimidating when seemingly interrogating someone. }

\par{36. なんでちゃんとやらないの? \hfill\break
Why won\textquotesingle t you do it properly? }

\par{37. すごいじゃないか!なんでそんなによかったんだ? \hfill\break
Wow, that\textquotesingle s awesome! How come it was so good like that? }

\par{38. すごいですね!なんでそんなに ${\overset{\textnormal{こうちょう}}{\text{好調}}}$ なんですか? \hfill\break
Wow, that\textquotesingle s great! How come things are going so well? }

\par{39. なんでこれが ${\overset{\textnormal{お}}{\text{起}}}$ きた? \hfill\break
How come this happened? }

\par{40. なんでそうなった? \hfill\break
How did it end up that way? }

\par{41. なんで?なんで? \hfill\break
How come? How come? }

\par{42. なんでだよ! \hfill\break
How come!? }

\par{43. ねーねー?なんで? \hfill\break
Hey, hey… how come? }

\par{44. なんでじゃねーんだよ? \hfill\break
Don\textquotesingle t you “how come?” me; you hear me? }

\par{ One grammatical situation that makes なんで very similar to なぜ but in a far more casual way is when it is used at the end of the sentence to mean “why is that\slash how come…?” }

\par{45. ${\overset{\textnormal{にんげん}}{\text{人間}}}$ が ${\overset{\textnormal{と}}{\text{跳}}}$ べないのは ${\overset{\textnormal{なん}}{\text{何}}}$ でですか。 \hfill\break
Why is it that\slash how come humans can\textquotesingle t jump up? }

\par{46. ${\overset{\textnormal{おな}}{\text{同}}}$ じ ${\overset{\textnormal{かた}}{\text{型}}}$ の ${\overset{\textnormal{でんしゃ}}{\text{電車}}}$ でも、 ${\overset{\textnormal{どあ}}{\text{ドア}}}$ の ${\overset{\textnormal{かいへいそくど}}{\text{開閉速度}}}$ が ${\overset{\textnormal{こと}}{\text{異}}}$ なるのは ${\overset{\textnormal{なん}}{\text{何}}}$ でですか。 \hfill\break
Why is that\slash how come the opening and closing speed of doors differ even when it\textquotesingle s the same model train? }

\par{47. ${\overset{\textnormal{おんな}}{\text{女}}}$ と ${\overset{\textnormal{おとこ}}{\text{男}}}$ の ${\overset{\textnormal{けっこんりつ}}{\text{結婚率}}}$ が ${\overset{\textnormal{ちが}}{\text{違}}}$ うのは ${\overset{\textnormal{なん}}{\text{何}}}$ でですか。 \hfill\break
Why is that\slash how come the marriage rates for men and women are different? }

\par{48. ${\overset{\textnormal{みせ}}{\text{店}}}$ によって ${\overset{\textnormal{ちが}}{\text{違}}}$ うのは ${\overset{\textnormal{なん}}{\text{何}}}$ でですか。 \hfill\break
Why is that\slash how come it\textquotesingle s different based on the store? }

\par{49. ${\overset{\textnormal{あせ}}{\text{汗}}}$ で ${\overset{\textnormal{しろ}}{\text{白}}}$ シャツの ${\overset{\textnormal{わき}}{\text{脇}}}$ が ${\overset{\textnormal{きいろ}}{\text{黄色}}}$ くなるのは ${\overset{\textnormal{なん}}{\text{何}}}$ でなの? \hfill\break
Why is that\slash how come the sides of white T-shirts become yellow from sweat? }

\par{50. ${\overset{\textnormal{かたち}}{\text{形}}}$ が ${\overset{\textnormal{ちが}}{\text{違}}}$ うのは ${\overset{\textnormal{なん}}{\text{何}}}$ でですか。 \hfill\break
Why is that\slash how come the shape is different? }
      
\section{理由}
 
\par{ Another phrase that brings on a translation of “why” is the word for reason, 理由. In English, we say “the reason why…” In Japanese, the word “why” just isn\textquotesingle t grammatically needed. The sense of this naturally comes about from 理由 being combined with other phrases. }

\par{51. ${\overset{\textnormal{たいしょく}}{\text{退職}}}$ (した) ${\overset{\textnormal{りゆう}}{\text{理由}}}$ を ${\overset{\textnormal{き}}{\text{聞}}}$ かせてください。 \hfill\break
Please let me hear why you left your (previous) job. }

\par{52. あなたが ${\overset{\textnormal{ひと}}{\text{人}}}$ を ${\overset{\textnormal{ころ}}{\text{殺}}}$ した ${\overset{\textnormal{りゆう}}{\text{理由}}}$ を ${\overset{\textnormal{おし}}{\text{教}}}$ えてください。 \hfill\break
Please tell me why you killed a\slash the person. }
    