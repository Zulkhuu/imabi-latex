    
\chapter{The Passive II}

\begin{center}
\begin{Large}
第116課: The Passive II: 不可抗力の受身 
\end{Large}
\end{center}
 
\par{ Throughout the site, extremely complicated names for things are avoided. However, in this discussion, it is important to bring up proper terminology as common names for what is to be discussed cause the confusion. }

\par{ The use of "passive" endings on intransitive verbs puzzled Western scholars from the very beginning. Common names for this have been 迷惑の受身 (trouble passive), 被害の受身 (painful passive), and 間接受身 (indirect passive). Though these definitions are right within specific examples, they not only fail at explaining all instances of this passive sentence type, nor does it address the fact that it is not exclusive to intransitives. The passive sentences we will see all have one thing in common. What might that be? Find out next! }
      
\section{不可抗力性}
 
\par{ Calling the 受身文 below "indirect" is very tempting, especially when you see examples like Ex. 1, but this does us little good. "Indirect" passive, grammatically speaking, means that the agent, which is marked by に, is not directly affecting the subject of the passive sentence. This is very vague, and the relation of "influence" between the doer and the person affected could vary greatly from sentence to sentence. }

\par{1. 畠中さんは大統領に才能を買われました。 \hfill\break
Hatanaka-san's abilities were appreciated by the president. }

\par{ This sentence, though, doesn't describe a bad thing. Yet, the overall grammar has been called the "painful passive." For one, there are very few verbs, much less intransitive verbs, referring to painful things. Some of the most painful verbs are transitive like 殺す. That's not to say this pattern never describes 被害. }

\par{2. 私は愛犬に死なれて悲嘆に暮れた。 \hfill\break
My beloved dog died, and I suffered from heartache. }

\par{ Because many examples describe 迷惑, some people are fine with calling any usage some sort of 迷惑 even if it's a stretch, but does this really create a good analysis? We also haven't even touched on why を can show up. From an English point of view, the use of intransitives is already bizarre. }

\par{3. 私たちは雨に降られた。 \hfill\break
We were rained on. }

\par{${\overset{\textnormal{}}{\text{4. 弟}}}$ は ${\overset{\textnormal{}}{\text{地下鉄}}}$ の ${\overset{\textnormal{}}{\text{中}}}$ で ${\overset{\textnormal{}}{\text{中年}}}$ の ${\overset{\textnormal{}}{\text{女性}}}$ に ${\overset{\textnormal{}}{\text{足}}}$ を ${\overset{\textnormal{ふ}}{\text{踏}}}$ まれた。 \hfill\break
My little brother got his foot stepped on by some middle aged woman on the subway. }

\par{ What if we were to say that what is really at work is the existence of a sense of inevitability or something being irresistible which does not exist in a typical passive like in Ex. 5 }

\par{5. カエルがワニに食われた \hfill\break
The\slash a frog was eaten by the alligator. }

\par{ The frog just happened to get eaten by the alligator. In Ex. 4, your brother being stepped on by the woman was inevitable and something he couldn't have avoided. The frog, again, just got straight up eaten in the jaws of a beast. With this grammar, the agent does some action which is stated in full, which is why you can see を. Depending on what this is, you are unable resist being put in a situation, and you let into it. Sometimes it's a good thing, but it is very frequently a bad thing. }

\begin{center}
 \textbf{More Examples }
\end{center}

\par{6. またあいつに家に ${\overset{\textnormal{}}{\text{来}}}$ られたようさ。(東京弁) \hfill\break
It seems that guy came to my house again. (You don't like it) }

\par{7. 両親に死なれた。 (Will be rude and inappropriate) \hfill\break
Literally: I was died on by my parents. \hfill\break
My parents died on me. }

\par{${\overset{\textnormal{}}{\text{8. 雨}}}$ に ${\overset{\textnormal{}}{\text{降}}}$ られて、びしょ ${\overset{\textnormal{ぬ}}{\text{濡}}}$ れになった。 \hfill\break
I was rained on and became soaking wet. }

\par{${\overset{\textnormal{}}{\text{9. 妹}}}$ に ${\overset{\textnormal{}}{\text{先}}}$ に ${\overset{\textnormal{}}{\text{卒業}}}$ された。 \hfill\break
My younger sister graduated before me (and I was really embarrassed). }

\par{${\overset{\textnormal{}}{\text{10. 私}}}$ は ${\overset{\textnormal{}}{\text{犬}}}$ に ${\overset{\textnormal{}}{\text{手}}}$ を ${\overset{\textnormal{か}}{\text{噛}}}$ まれました。 \hfill\break
I (had\slash got) my hand bitten by a dog. }

\par{11. 私が河田に水藤へ藤原を紹介された。 \hfill\break
Kawada introducing Fujiwara to Suito was inconvenient to me. }

\par{12. 先に子供に死なれたら、どうなるんだろう。 \hfill\break
What would happen if my child\slash the child were to die first? }
    