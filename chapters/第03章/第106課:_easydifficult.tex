    
\chapter{Easy \& Difficult}

\begin{center}
\begin{Large}
第106課: Easy \& Difficult: ~やすい, ~にくい, \& ~づらい 
\end{Large}
\end{center}
 
\par{ Who knew that expressing easy and difficult could both in fact be slightly difficult in Japanese? }

\par{\textbf{Part of Speech Note }: These endings are all adjectival. }
      
\section{Easy and Difficult}
 
\par{ ~やすい, when used with transitive verbs, shows that something is easy to do.When used with intransitive verbs, it means that something is easy to occur. }

\par{1. このサンドイッチは小さくて、 ${\overset{\textnormal{}}{\text{食}}}$ べやすいよ。 \hfill\break
This sandwich is small and easy to eat. }

\par{2. そのペンはとても ${\overset{\textnormal{}}{\text{書}}}$ きやすいです。 \hfill\break
That pen is very easy to write with. }
3. あの ${\overset{\textnormal{}}{\text{山}}}$ は ${\overset{\textnormal{のぼ}}{\text{登}}}$ りやすくない。 \hfill\break
That mountain (over there) isn't easy to climb. 
\par{${\overset{\textnormal{}}{\text{4. 白}}}$ いシャツは ${\overset{\textnormal{よご}}{\text{汚}}}$ れやすい。 \hfill\break
White shirts easily get dirty.  }

\par{5. インターネットを ${\overset{\textnormal{}}{\text{使}}}$ いやすくしたのは誰でしょうか。 \hfill\break
Who was it that made the internet easier to use. }

\par{${\overset{\textnormal{}}{\text{6. 僕}}}$ にはこの ${\overset{\textnormal{}}{\text{車}}}$ は ${\overset{\textnormal{うんてん}}{\text{運転}}}$ しやすい。 \hfill\break
This car is easy for me to drive. }
\textbf{漢字 Note }: The 漢字 spelling is no longer commonly used. It is usually found in older writing. 
\par{ Instead of ~やすい, the adjective よい may also be used. This, though, is very old-fashioned or unheard to speakers depending on where the person is from. Some natives may not even recognize it as being correct Japanese, but it does indeed exist. In fact, it is still extremely common even in the form ~いい in 関東弁. Its usage in the capital region is still prevalent in people 60 and older. }

\par{7. ${\overset{\textnormal{せきど}}{\text{咳止}}}$ め ${\overset{\textnormal{ぐすり}}{\text{薬}}}$ は ${\overset{\textnormal{}}{\text{飲}}}$ みよいはずである。(あまり使われていない) \hfill\break
${\overset{\textnormal{}}{\text{咳止}}}$ め ${\overset{\textnormal{}}{\text{薬}}}$ は ${\overset{\textnormal{}}{\text{飲}}}$ みやすいはずだ。(Natural) \hfill\break
The cough medicine should be easy to take. }

\par{ ~ ${\overset{\textnormal{にく}}{\text{難}}}$ い shows either that something is difficult to do or to occur. ~ ${\overset{\textnormal{がた}}{\text{難}}}$ いonly shows the inability to do something, and it is usually restricted to writing and set expressions . Lastly, ~辛い may be used instead of ~にくい to describes undesirable circumstances. These endings attach to the 連用形 of verbs and conjugate as 形容詞. }
 
\par{8  この ${\overset{\textnormal{}}{\text{本}}}$ は ${\overset{\textnormal{}}{\text{読}}}$ みにくいです。 \hfill\break
This book is difficult to read. }
 
\par{${\overset{\textnormal{}}{\text{9a. 歩}}}$ き ${\overset{\textnormal{}}{\text{辛}}}$ い ${\overset{\textnormal{}}{\text{道}}}$ は ${\overset{\textnormal{さかみち}}{\text{坂道}}}$ だ。(ちょっと ${\overset{\textnormal{ふしぜん}}{\text{不自然}}}$ ) \hfill\break
 ${\overset{\textnormal{}}{\text{9b. 坂道}}}$ は、 ${\overset{\textnormal{}}{\text{歩}}}$ きづらい。(もっと ${\overset{\textnormal{}}{\text{自然}}}$ ) \hfill\break
A road difficult to walk on is a hill. }
 
\par{${\overset{\textnormal{}}{\text{10a. 書}}}$ きよい ${\overset{\textnormal{}}{\text{漢字}}}$ は ${\overset{\textnormal{そんざい}}{\text{存在}}}$ するのかな? (Old-fashioned) \hfill\break
 ${\overset{\textnormal{}}{\text{10b. 書}}}$ きやすい ${\overset{\textnormal{}}{\text{漢字}}}$ は ${\overset{\textnormal{}}{\text{存在}}}$ するのかな?(Natural) \hfill\break
I wonder if a Kanji that is easy to write exists. }
 
\par{${\overset{\textnormal{}}{\text{11. 考}}}$ えにくいと ${\overset{\textnormal{}}{\text{思}}}$ います。 \hfill\break
I think that is difficult to consider. }
 
\par{12. この ${\overset{\textnormal{}}{\text{戸}}}$ は ${\overset{\textnormal{}}{\text{開}}}$ きにくい。 \hfill\break
This door is difficult to open. }
 
\par{${\overset{\textnormal{}}{\text{13. 字}}}$ が ${\overset{\textnormal{うす}}{\text{薄}}}$ くて、 ${\overset{\textnormal{}}{\text{読}}}$ み ${\overset{\textnormal{}}{\text{づら}}}$ いです。 \hfill\break
The letters are faded, and it is difficult to read. }
 
\par{${\overset{\textnormal{}}{\text{14. 生徒達}}}$ は ${\overset{\textnormal{みな}}{\text{皆}}}$ ${\overset{\textnormal{}}{\text{教}}}$ え ${\overset{\textnormal{がた}}{\text{難}}}$ くて、ぜんぜん言うことに ${\overset{\textnormal{}}{\text{聞}}}$ かなくて、『いまび』を ${\overset{\textnormal{めった}}{\text{滅多}}}$ に ${\overset{\textnormal{}}{\text{読}}}$ まなくて、 ${\overset{\textnormal{こま}}{\text{困}}}$ ったんです。 \hfill\break
My students are all very hard to teach because they never listen to what I say and rarely read IMABI; I'm very troubled by this. }
 
\par{${\overset{\textnormal{}}{\text{15. 言}}}$ い\{ ${\overset{\textnormal{がた}}{\text{難}}}$ い・にくい・づらい\}ことだ。 \hfill\break
It's hard to say. }
 
\par{${\overset{\textnormal{}}{\text{16. 彼}}}$ は ${\overset{\textnormal{}}{\text{付}}}$ き ${\overset{\textnormal{}}{\text{合}}}$ い ${\overset{\textnormal{がた}}{\text{難}}}$ いやつだな。 \hfill\break
He is a difficult person to get along with isn't he? }
 
\par{${\overset{\textnormal{}}{\text{17. 彼女}}}$ の ${\overset{\textnormal{}}{\text{言葉}}}$ は ${\overset{\textnormal{}}{\text{聞}}}$ き ${\overset{\textnormal{}}{\text{取}}}$ り ${\overset{\textnormal{がた}}{\text{難}}}$ かった。 \hfill\break
Her speech was difficult to hear. }

\par{${\overset{\textnormal{}}{\text{18. 日本}}}$ は住みにくいです。 \hfill\break
Living in Japan is difficult. }

\par{19. 彼女はその職の適任者とは言いがたい。 \hfill\break
It's difficult to say that she is the right candidate for the job. }
    