    
\chapter{Honorifics II}

\begin{center}
\begin{Large}
第125課: Honorifics II: Adjectives \& the Copula 
\end{Large}
\end{center}
 
\par{ Pay close attention to what is deemed old-fashioned and what is not. }
      
\section{Adjectives}
 
\par{  The "honorific" use of adjectives is normally seen in the pattern お+Adjective+です. This is often felt to just be really polite speech rather than honorific speech, but it suffices for the most part. }

\par{\textbf{お・ご~ Note }: The omission of the honorific prefixes is primarily determined by whether you are showing respect\slash humility or using 丁寧語. As you could imagine, the latter case is for when they should be omitted. }

\begin{center}
 \textbf{Examples }
\end{center}

\par{ In the examples below, you will see various other patterns that hint at what has traditionally been the usage for adjectives. More will be said about them later in this lesson. So, for now, treat options as synonyms. }
 
\par{1. お忙しいですか。(普通) \hfill\break
Are you busy? }
 
\par{2. インドへ行けるのは嬉し\{ゅうございます・い限りです・い所存でございます\}。(謙譲語) \hfill\break
I would be happy to go to India. \hfill\break
\hfill\break
\textbf{Sentence Note }: This example shows how you can make an adjective more humble. 所存 means intention, and it is a good formal choice. We will get to what exactly ゅうございます is later in this lesson. }
 
\par{3a. 皆様本日は \textbf{お忙しい }ところご参加 ${\overset{\textnormal{いただ}}{\text{頂}}}$ きありがとうございます。(謙譲語) \hfill\break
3b. 皆様本日はお忙しいところご ${\overset{\textnormal{りんせき}}{\text{臨席}}}$ を ${\overset{\textnormal{たまわ}}{\text{賜}}}$ りありがとうございます。(上位者などに対する尊敬語) \hfill\break
3c. 皆様本日はお忙しいところご出席下さりありがとうございます。(尊敬語) \hfill\break
Thank you for attending. }
 
\par{4a. 開けても \textbf{よろしい }ですか。 \hfill\break
4b. 開けても \textbf{よろしい }でしょうか。(△\slash More polite) \hfill\break
Is it alright if I open up (the window)? }
 
\par{\textbf{自然さ Note }: よろしいでしょう to many speakers is incorrect 敬語 because it is a doubling of two patterns at once to make it such. However, to some, the politeness of よろしい seems insufficient, thereby making this a reasonable solution. }
 
\par{\textbf{Exception Note }: The honorific form of よい is よろしい・宜しい. }
 
\par{5a. 皆さん、お静かに願います。(とても丁寧) \hfill\break
5b. 皆様、ご ${\overset{\textnormal{せいしゅく}}{\text{静粛}}}$ にお願い申し上げます。(Very respectful) \hfill\break
Everyone, please be quiet. }
 
\par{6a. ご立派でいらっしゃいます。 \hfill\break
6b. ご立派であらせられます。(Old-fashioned; very respectful) \hfill\break
You are great. }
 
\par{\textbf{Traditional Adjective Honorific Conjugations: Adj. + ~ }\textbf{ございます }}
 
\par{ The traditional means of conjugating an adjective into honorific speech, which is now deemed old-fashi oned and potentially grandiose, involves dropping い and adding うございます. Whenever the adjective ends in しい・じい, then you drop い and add ゅございます. You've already seen this grammar in the phrase ありがとうございます. }

\begin{ltabulary}{|P|P|P|P|}
\hline 

If it ends in\dothyp{}\dothyp{}\dothyp{} & Drop & Add & Then Add \\ \cline{1-4}

―いい & ―いい & Small ゅ +う & ございます \\ \cline{1-4}

―あい & ―あい & ―おう & ございます \\ \cline{1-4}

―おい & ―おい & ―おう & ございます \\ \cline{1-4}

―うい & ―うい & ―うう & ございます \\ \cline{1-4}

\end{ltabulary}

\begin{center}
 \textbf{ Examples }
\end{center}

\begin{ltabulary}{|P|P|P|P|P|P|}
\hline 

大きい & 大きゅうございます & Big & 嬉しい & 嬉しゅうございます & Happy \\ \cline{1-6}

早い & 早うございます & Early & よろしい & よろしゅうございます & Good \\ \cline{1-6}

新しい & 新しゅうございます & New & 白い & 白ございます & White \\ \cline{1-6}

難しい & 難しゅうございます & Difficult & 古い & 古うございます & Old \\ \cline{1-6}

赤い & 赤うございます & Red & 悪い & 悪うございます & Bad \\ \cline{1-6}

\end{ltabulary}
 
\par{\hfill\break
7. 部屋は夏でも涼しゅうございます。(古風) \hfill\break
The room is cool even in the summer. }

\par{8. お早うございます。 \hfill\break
Good morning. \hfill\break
Literally: You're early. }
 
\par{9. 旅行は ${\overset{\textnormal{なご}}{\text{長}}}$ うございます。(古風) \hfill\break
The trip is long. }
 
\par{10. お友達と別れるのは本当に悲しゅうございます。(古風) \hfill\break
Separating from one's friends is really sad. }
 
\par{11. このお茶は ${\overset{\textnormal{うす}}{\text{薄}}}$ うございます。(古風) \hfill\break
The tea is flat. }
 
\par{12. 地震や ${\overset{\textnormal{よしん}}{\text{余震}}}$ は ${\overset{\textnormal{おそ}}{\text{恐}}}$ ろしゅうございますね。(古風) \hfill\break
Earthquakes and aftershocks are scary, aren't they? }
 
\par{13. 空は青うございます。(古風) \hfill\break
The sky is blue. }

\par{14. 「はあ、一人三円で参ります。少しお高うございますが、翌る日寝てしまいますから。」 \hfill\break
"Yes, I'll come for three yen a person. It is a little high, but it is because I sleep the next day". \hfill\break
From 死体紹介人 by 川端康成. }

\par{\textbf{History Note }: There was a time in Japan's history when the yen was far more valuable than it is today just as the dollar and penny once were in America. }
 
\par{15. 日本は大きくございません。( ${\overset{\textnormal{ひんかく}}{\text{品格}}}$ のある丁寧さ; ちょっと古風) \hfill\break
Japan is not large. }
 
\par{\textbf{Pattern Note }: Avoiding the contractions is a way to make honorific adjectives sound less old-fashioned and yet at the same time be more polite. Remember that sentences like this are examples of 丁寧語, not 尊敬語. }

\par{16. 「それで、好ござんすとも」と御米は答えた。 \hfill\break
"Then, that's fine", Oyome replied. \hfill\break
From 門 by 夏目漱石. }

\par{\textbf{Contraction Note }: よござんす is a shortening of よろしゅうございます and was a common 江戸言葉 in light honorifics a century ago. Always be on the lookout for odd stuff in literature. }
 
\par{\textbf{~く\{思って・ ${\overset{\textnormal{ぞん}}{\text{存}}}$ \textbf{じて\}おります }}}
 
\par{ Many older speakers cannot get used to adding です or a copula of any sort after a 形容詞. The coming of this pattern, like with many generational changes, can be explained by contractions. In this case, the common practice of the new typing age and East Japanese dialect habits became standard. However, as you learn more about how to make your speech ever more honorific, one way to overcome this issue of grammaticality is by changing the adjective to an adverb and using \{思って・存じて\}おります. }
 
\par{17. 有り ${\overset{\textnormal{がた}}{\text{難}}}$ く存じております。 \hfill\break
I am very grateful. \hfill\break
Literally: I think very gratefully. }

\par{18. ${\overset{\textnormal{うれ}}{\text{嬉}}}$ しく思っております。 \hfill\break
I am very happy. \hfill\break
Literally: I'm thinking happily. }
      
\section{The Copula}
 
\par{ The copula doesn't have a true 謙譲語 form. When you wish to show more politeness with the copula, you use the form \textbf{でございます }. However, this is classified as 丁寧語. It is politer than です・ます調, and it can be used to refer to third person intentions along with sentences regarding oneself. In fact, ございます as a stand alone verb is classified as 尊敬語 or 丁寧語. As far as the copula is concerned, its 尊敬語 is \textbf{でいらっしゃいます }. Referring to their plain forms for convenience, you'll learn that いらっしゃる and ござる happen to be honorific verbs. So, it's no surprise that they would be used this way with the copula.   }

\par{19. 社長、こちらは藤原常務でいらっしゃいます。 \hfill\break
President, this is Director Fujiwara.  }

\par{20. 皇帝から頼まれたものでございます。 \hfill\break
This is something that was entrusted to me from the Emperor. }

\par{21. そうではございません。 \hfill\break
It's not so. }

\par{22. 私はこの子の父でございます。 \hfill\break
I am the father of this child. }

\par{23. はい、鈴木でございます。 \hfill\break
Yes, this is Suzuki. }
    