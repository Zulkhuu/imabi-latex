    
\chapter{Similarity}

\begin{center}
\begin{Large}
第145課: Similarity: ~ようだ, ~みたいだ,  \& ~っぽい 
\end{Large}
\end{center}
 
\par{ Though all of these phrases have a common theme, their grammar and usages do vary significantly. So, please be sure to not rush through them even if you've seen them show up many times in your studies thus far. }
      
\section{~ようだ}
 
\par{ 様 is a noun that literally means "appearance". ~ようだ is a 形容動詞 conjugating ending that comes from it, and it has many applications. }

\begin{center}
\textbf{~ようだ }
\end{center}

\par{ Overall, ~ようだ is like "like" or "seem". よう is a noun itself, so when used after nouns, precede it with の. When after 形容動詞, use な! }
 
\par{1. 僕の犬は言葉が分かるようだよ。 \hfill\break
My dog \textbf{seems }to understand words! }

\par{2. ${\overset{\textnormal{はちゅうるい}}{\text{爬虫類}}}$ のようで、実は、 ${\overset{\textnormal{ほにゅうるい}}{\text{哺乳類}}}$ だ。 \hfill\break
It \textbf{looks like }a reptile but in reality is a mammal. }

\par{\textbf{漢字 Note }: 爬 and 哺 are rare. So, you don't need to worry about them. }
 
\par{3. すこし ${\overset{\textnormal{しおから}}{\text{塩辛}}}$ いようですね。 \hfill\break
It's \textbf{kind }of salty, don't you think? }

\par{\textbf{Word Note }: 辛い is often used to mean "salty" in West Japan. The word 塩っぱい also exists. In this word, 塩 is pronounced as しょ. }
 
\par{4. どうも ${\overset{\textnormal{こわ}}{\text{壊}}}$ れているようだ。 \hfill\break
Presumably, it's broken. }
 
\par{5. みな元気なようです。 \hfill\break
Everyone \textbf{seems }fine. }
 
\par{6. 彼はとても悲しんでいるようです。 \hfill\break
He \textbf{seems }to be very sad. }

\par{7.何か ${\overset{\textnormal{かく}}{\text{隠}}}$ しているようです。 \hfill\break
He \textbf{appears }to be hiding something. }
 
\par{8. あの人を見たような気がしました。 \hfill\break
I felt \textbf{like }I saw that person before. }
 
\par{9. ちょっと ${\overset{\textnormal{おこ}}{\text{怒}}}$ ったように聞こえた。 \hfill\break
I heard her \textbf{like }she was a little angry. }

\par{10. ステーキかまたはそれと同じようなものを食べる。 \hfill\break
To eat steak or something just \textbf{like }it. }

\par{11. 彼は、何も言いたくないような、 ${\overset{\textnormal{お}}{\text{押}}}$ し ${\overset{\textnormal{だま}}{\text{黙}}}$ った表情をしてる。 \hfill\break
He's reticent \textbf{as if }he doesn't want to say anything. }
 
\par{12. 彼は、何も言いたくないようで、ぶすっとしてる。 \hfill\break
He's in a bad mood not saying anything. }

\par{13. 難しかったようです。 \hfill\break
It \textbf{looks like }it was difficult. }
 
\par{14. 対策もなく、 ${\overset{\textnormal{まないた}}{\text{俎板}}}$ の ${\overset{\textnormal{こい}}{\text{鯉}}}$ のようなものだ。(Idiomatic) \hfill\break
It's \textbf{basically }a hopeless situation without a countermeasure. }

\begin{center}
 \textbf{まるで \& ~かのようだ } 
\end{center}

\par{ ~ようだ is also often paired with まるで to make "just like", and when used with the particle か, it creates "as if". }

\par{15. マイクさんはまるで天使のような人ですね。 \hfill\break
Isn't Mike just \textbf{like }an angel? }

\par{16. まるで人でも殺したかのように、ぼうぜんとしてる。 \hfill\break
He's dumbfounded as if he's killed a person. }

\par{17. 彼の病気はまるで神様が治してくれたかのようにきれいに消えましたよ。 \hfill\break
His sickness disappeared \textbf{as if }it was by God. }

\par{18. 地に ${\overset{\textnormal{ひざ}}{\text{膝}}}$ をつくかのように木が風に ${\overset{\textnormal{ゆ}}{\text{揺}}}$ れていた。 \hfill\break
The trees swayed in the wind \textbf{as if }they knelled to the ground. }

\begin{center}
\textbf{~よう\{なら・だったら\} }
\end{center}
 
\par{ ~よう\{なら・だったら\} euphemizes a supposition. }

\par{19. この ${\overset{\textnormal{ちりょう}}{\text{治療}}}$ \{手当て・ ${\overset{\textnormal{しょち}}{\text{処置}}}$ \}で治らないようだったら病院に行ってくださいね。 \hfill\break
If you don't get better with this medical treatment, please go to the hospital, OK? }

\begin{center}
 \textbf{~ようになる }
\end{center}
 
\par{ ~ようになる shows that a state changes. It indicates a state that could not be done has become one that can be done. It can also be used in the negative. Also, ~ないようになる can be contracted to ~なくなる. The pattern can also be used to show acquisition of a habit or custom. }
 
\par{20. 六年ぶりに ${\overset{\textnormal{きゅうゆう}}{\text{旧友}}}$ と会 \textbf{えるようになった }。 \hfill\break
After six years I was \textbf{able to }meet an old friend. }

\par{21. 日本語で ${\overset{\textnormal{あいさつ}}{\text{挨拶}}}$ ができるようになりましたよ。 \hfill\break
I have become able to greet in Japanese. }

\par{22. 「日本語の勉強はどうですか」「宿題が簡単にできるようになりましたし、それに日本人と毎日会話をしてい      ます」 \hfill\break
"How are you Japanese studies?” “Homework has become easier to do, and furthermore, I converse with Japanese people every day". }
 
\par{23. 最近、彼女はよく話すようになりました。 \hfill\break
She recently \textbf{started to }talk a lot. }
 
\par{24. 彼はテレビを見なくなりました。 \hfill\break
He stopped watching TV. }

\begin{center}
\textbf{The Noun }\textbf{よう }
\end{center}
 
\par{The noun よう may be after the 連用形 of verbs to show appearance or method. It can also function as a nominalizer. }
 
\par{25. 言おう様ない (Rare) = 言い様のない \hfill\break
Indescribable }
 
\par{26. その ${\overset{\textnormal{}}{\text{苦}}}$ しみようは ${\overset{\textnormal{}}{\text{見}}}$ ていられない。 \hfill\break
I can't stand watching that miserable condition (of that man). }
      
\section{~みたいだ}
 
\par{ ~みたいだ is for the most part a more casual variant of よう for comparison, but since it does come from 見る (to see), it shows that something "resembles\dothyp{}\dothyp{}\dothyp{}". It acts as a 形容動詞. So, you can see ~みたいな (adjectival) and ~みたいに (adverbial). The negative should be ~ないみたいだ. but in slang you may still see ~みたくない. }
 
\par{${\overset{\textnormal{}}{\text{27. 兄}}}$ は ${\overset{\textnormal{さむらい}}{\text{侍}}}$ みたいな ${\overset{\textnormal{}}{\text{人}}}$ だね。 \hfill\break
My older brother is like a samurai. }

\par{28. ${\overset{\textnormal{ま}}{\text{真}}}$ っ ${\overset{\textnormal{}}{\text{暗}}}$ で ${\overset{\textnormal{}}{\text{雨}}}$ が ${\overset{\textnormal{}}{\text{降}}}$ ったみたいだ。 \hfill\break
It was like rain had fallen in pitch black. }
 
\par{29. ガムみたいに ${\overset{\textnormal{}}{\text{伸}}}$ びるものだ。 \hfill\break
It's a thing that stretches like gum. }
 
\par{30. もう ${\overset{\textnormal{}}{\text{売}}}$ り ${\overset{\textnormal{}}{\text{切}}}$ れみたい。 \hfill\break
It looks like it's already sold out. }
 
\par{31. 誰もいないみたいだが。 \hfill\break
It seems that nobody is here, but\dothyp{}\dothyp{}\dothyp{} }
 
\par{${\overset{\textnormal{}}{\text{32. 夢}}}$ みたいな ${\overset{\textnormal{}}{\text{話}}}$ だ。 \hfill\break
That's a story resembling a dream. }

\par{33. ${\overset{\textnormal{かべ}}{\text{壁}}}$ は ${\overset{\textnormal{}}{\text{紙}}}$ みたいに ${\overset{\textnormal{うす}}{\text{薄}}}$ いよ。 \hfill\break
The walls are thin like paper. }
 
\par{${\overset{\textnormal{}}{\text{34. 味}}}$ がすこし ${\overset{\textnormal{}}{\text{薄}}}$ いみたいです。 \hfill\break
The taste seems to slightly be weak. }

\par{\textbf{Origin Note }: ~みたい comes from ~みた様, which can still be seen in literature.  }
      
\section{~っぽい}
 
\par{ The 形容詞 conjugating suffix ~っぽい is often discarded as slang. However, it is becoming very popular. It is very similar to "-ish". After the stems of adjectives, it shows a distinct nature. With nouns, it shows a strong impression about what something is like. However, when attached to the 連用形 of verbs, it shows tendency. Now it can be seen after the 終止形 to mean "looks like". }
 
\par{${\overset{\textnormal{}}{\text{35. 彼女}}}$ は ${\overset{\textnormal{}}{\text{全然女}}}$ っぽくないな。 \hfill\break
She's not womanly at all! }
 
\par{${\overset{\textnormal{}}{\text{36. 君}}}$ の ${\overset{\textnormal{けってん}}{\text{欠点}}}$ は ${\overset{\textnormal{}}{\text{忘}}}$ れっぽいことだぞ。(Masculine) \hfill\break
Your fault is that you're forgetful. }
 
\par{37. なんて ${\overset{\textnormal{うそ}}{\text{嘘}}}$ っぽい ${\overset{\textnormal{}}{\text{話}}}$ や! (関西弁) \hfill\break
What a lie! }

\par{38. ${\overset{\textnormal{ほこり}}{\text{埃}}}$ っぽい \hfill\break
Dusty }
 
\par{${\overset{\textnormal{}}{\text{39. 最近忘}}}$ れっぽくなった。 \hfill\break
I tended to forget the recent times. }

\par{40. ${\overset{\textnormal{あら}}{\text{荒}}}$ っぽい ${\overset{\textnormal{}}{\text{行動}}}$ 。 \hfill\break
Rough conduct. }

\par{41. 彼は ${\overset{\textnormal{ねつ}}{\text{熱}}}$ っぽく ${\overset{\textnormal{}}{\text{身}}}$ が ${\overset{\textnormal{}}{\text{震}}}$ えていた。 \hfill\break
His body was feverishly shaking. }
 
\par{42. この ${\overset{\textnormal{}}{\text{酒}}}$ は ${\overset{\textnormal{}}{\text{水}}}$ っぽいの。(Feminine) \hfill\break
This sake is watery. }

\par{43. ${\overset{\textnormal{りくつ}}{\text{理屈}}}$ っぽい \hfill\break
Argumentative }
 
\par{44. マジで ${\overset{\textnormal{}}{\text{安}}}$ っぽいテレビだな。 (Casual) \hfill\break
It's really a cheap TV, isn't it? }
 
\par{45. やるっぽい! (Casual) \hfill\break
It looks like he's doing it! }
 
\par{${\overset{\textnormal{}}{\text{46. 病院}}}$ っぽい ${\overset{\textnormal{ふんいき}}{\text{雰囲気}}}$ \hfill\break
A hospital-like atmosphere. }
 
\par{47. 子供っぽいことすんなよ。 (Masculine) \hfill\break
Don't do something childish. }
 
\par{48. バナナっぽい ${\overset{\textnormal{かお}}{\text{香}}}$ りが ${\overset{\textnormal{}}{\text{好}}}$ きだわ。(女性言葉) \hfill\break
I like banana-like fragrances. }
 
\par{${\overset{\textnormal{}}{\text{49. 黒}}}$ っぽい ${\overset{\textnormal{}}{\text{車}}}$ でした。 \hfill\break
It was a blackish car. }
 
\par{${\overset{\textnormal{}}{\text{50a. 哀}}}$ れっぽい声 \hfill\break
50b. 哀れめいた声 (More serious) \hfill\break
A piteous voice  }
    