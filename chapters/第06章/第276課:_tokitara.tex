    
\chapter{”When it Comes to\dothyp{}\dothyp{}\dothyp{}" III}

\begin{center}
\begin{Large}
第276課: ”When it Comes to\dothyp{}\dothyp{}\dothyp{}" III: と\{きたら・きては・くると・きた日には・きている(ものだ)・くれば\} 
\end{Large}
\end{center}
 
\par{ This lesson is a continuation of the last lesson but with a twist. It just so happens that the verb 来る is not as easy as it has been made out to be. At this point, we know that it is not always used in the literal sense of “coming” to a location. For instance, we know it can be used in idiomatic expressions such as 頭にくる (to get mad). }

\par{ One usage that the verb has had for a very long time in Japanese history but which has been in decline in recent decades is using it to mean いう. This has been done so in Japanese to bring about extremely emphatic commentary about things that are deemed sensitive\slash of importance to the speaker. }

\par{ So far, this sounds like random commentary out of context, but remember how といったら・いえば were used in the last lesson. These phrases are both used to conjure up some topic to emphatically make a statement about it. The phrases in this lesson are more or less synonymous to this degree, but the nuances implied, of course, are not 100\% the same. }
      
\section{くる ≒ いう}
 
\begin{center}
\textbf{ときたら }
\end{center}

\par{ The purpose of ときたら is to bring up into conversation a very specific circumstance\slash event\slash matter\slash topic of some significance to the speaker in an emphatic manner, which is then followed by commentary true in the eyes of the speaker for the situation mentioned. This commentary is usually negative in connotation; however, regardless of whether the sentence is positive or negative in nuance, it always brings about some sense of astonishment\slash amazement. After all, you can be astounded by how awesome or how horrible something is. }

\par{Basic Translation: “When it comes to…”; “Regarding…” }

\par{ As the basic translations suggest, you can view this phrase to be a very emphatic variation of phrases like といえば and に関しては. }

\par{1. ${\overset{\textnormal{ちゅうごく}}{\text{中国}}}$ での ${\overset{\textnormal{せいかつ}}{\text{生活}}}$ ときたら、 ${\overset{\textnormal{ほんとう}}{\text{本当}}}$ に ${\overset{\textnormal{さんざん}}{\text{散々}}}$ だった。 \hfill\break
As far as my livelihood in China went, it was truly miserable. }

\par{2. あの ${\overset{\textnormal{まわ}}{\text{回}}}$ らない ${\overset{\textnormal{すしや}}{\text{寿司屋}}}$ ときたら、 ${\overset{\textnormal{りえきりつ}}{\text{利益率}}}$ が ${\overset{\textnormal{たか}}{\text{高}}}$ そうだな。 \hfill\break
When it comes to sushi restaurants that aren\textquotesingle t conveyor ones, it seems that the profit ratio is high, you know. }

\par{3. この ${\overset{\textnormal{くるま}}{\text{車}}}$ ときたら、 ${\overset{\textnormal{なつ}}{\text{夏}}}$ はエンジンの ${\overset{\textnormal{ねっき}}{\text{熱気}}}$ で ${\overset{\textnormal{がいき}}{\text{外気}}}$ より ${\overset{\textnormal{あつ}}{\text{暑}}}$ いくせに ${\overset{\textnormal{ふゆ}}{\text{冬}}}$ は ${\overset{\textnormal{すきまかぜ}}{\text{隙間風}}}$ で ${\overset{\textnormal{そと}}{\text{外}}}$ より ${\overset{\textnormal{さむ}}{\text{寒}}}$ い。 \hfill\break
Regarding this car, despite it being hotter than the outside due to the heat of the engine in the summer, it\textquotesingle s colder than the outside due to draft in the winter. }

\par{4. メキシコの ${\overset{\textnormal{てんき}}{\text{天気}}}$ ときたら、 ${\overset{\textnormal{ひど}}{\text{酷}}}$ い ${\overset{\textnormal{あつ}}{\text{暑}}}$ さだ。 \hfill\break
When it comes to the weather in Mexico, the heat is awful. }

\par{5. ${\overset{\textnormal{せかい}}{\text{世界}}}$ では ${\overset{\textnormal{ひど}}{\text{酷}}}$ い ${\overset{\textnormal{じけん}}{\text{事件}}}$ が ${\overset{\textnormal{ひんぱつ}}{\text{頻発}}}$ してるのに、 ${\overset{\textnormal{にほん}}{\text{日本}}}$ の ${\overset{\textnormal{けいさつ}}{\text{警察}}}$ ときたらこんなことを・・・ \hfill\break
Even though horrible incidents are frequently happening in the world, to think that the Japanese police of all things would… }

\par{6. あの ${\overset{\textnormal{せんせい}}{\text{先生}}}$ ときたら、 ${\overset{\textnormal{じゅぎょうちゅう}}{\text{授業中}}}$ も ${\overset{\textnormal{じょうだん}}{\text{冗談}}}$ ばかりで、 ${\overset{\textnormal{こま}}{\text{困}}}$ るなあ。 \hfill\break
When it comes to that teacher, he\textquotesingle s always joking in class, which is really bothersome. }

\par{7. このパソコンときたら、 ${\overset{\textnormal{か}}{\text{買}}}$ ったばかりなのに、もう ${\overset{\textnormal{こわ}}{\text{壊}}}$ れ ${\overset{\textnormal{はじ}}{\text{始}}}$ めてて、 ${\overset{\textnormal{こま}}{\text{困}}}$ ってるよ。 \hfill\break
When it comes to this PC, even though I\textquotesingle ve just bought it, it\textquotesingle s already beginning to break, which is causing me a great bit of trouble. }

\par{8. まったく、 ${\overset{\textnormal{さいきん}}{\text{最近}}}$ の ${\overset{\textnormal{わかもの}}{\text{若者}}}$ ときたら、なっとらん! \hfill\break
Ugh, young people these days, they just won\textquotesingle t cut it. }

\par{9. このチェーン ${\overset{\textnormal{いざかや}}{\text{居酒屋}}}$ ときたら、 ${\overset{\textnormal{じゅうぎょういん}}{\text{従業員}}}$ は ${\overset{\textnormal{さいてい}}{\text{最低}}}$ だよ。 \hfill\break
When it comes to this chain izakaya, the employees are the worst. }

\par{10. ${\overset{\textnormal{おっと}}{\text{夫}}}$ ときたら、 ${\overset{\textnormal{わたし}}{\text{私}}}$ を ${\overset{\textnormal{かたとき}}{\text{片時}}}$ とも ${\overset{\textnormal{ひとり}}{\text{一人}}}$ にしてくれないのです。 \hfill\break
When it comes to my husband, he won\textquotesingle t even let me have a moment alone by myself. }

\par{11. あいつときたら、いっつも ${\overset{\textnormal{うそ}}{\text{嘘}}}$ を ${\overset{\textnormal{つ}}{\text{突}}}$ くんだ。 \hfill\break
When it comes to that guy, he always lies. }

\par{12. あいつときたら、 ${\overset{\textnormal{まいあさ}}{\text{毎朝}}}$ ${\overset{\textnormal{さんじゅっぷん}}{\text{30}}}$ ${\overset{\textnormal{ぷんいじょうちこく}}{\text{分以上遅刻}}}$ してくるんだよ。 \hfill\break
When it comes to that guy, he always shows up 30 minutes or more late every morning. }

\par{13. ${\overset{\textnormal{となり}}{\text{隣}}}$ の ${\overset{\textnormal{いえ}}{\text{家}}}$ の ${\overset{\textnormal{いぬ}}{\text{犬}}}$ ときたら、いつも ${\overset{\textnormal{ほ}}{\text{吠}}}$ えてばかりで ${\overset{\textnormal{こま}}{\text{困}}}$ ってます。 \hfill\break
When it comes to my neighbor\textquotesingle s dog, it\textquotesingle s always barking, and it\textquotesingle s bothering me. }

\par{14. うちの ${\overset{\textnormal{つま}}{\text{妻}}}$ ときたら、またもや ${\overset{\textnormal{しゅっちょうちゅう}}{\text{出張中}}}$ だ。 \hfill\break
Concerning my wife, she\textquotesingle s on a business trip \emph{again }. }

\par{ と言ったら is used in two kinds of situations. It either calls out\slash talks to a listener, or it talks about a certain situation. In both situations, commentary follows. The entire situation implies familiarity with the person\slash situation at hand. The commentary\slash critique\slash outburst can have a variety of emotions packed into it: anxiety, worry, rebuke, jealousy, pride, resignation, etc. Regardless of the emotion, all this holds together. }

\par{ ときたら, on the other hand, brings up some topic as a prerequisite for a comment that follows. The comment that follows is deemed to be obvious\slash natural\slash absolutely certain. This ‘opinion\textquotesingle  is deeply felt by the speaker, demonstrating that the topic is of some significance to the speaker.  Although と言ったら can be used to give positive or negative feedback in a familial tone, the same cannot be said of ときたら. The difference is that ときたら does not guarantee a tone of familiarity. In fact, the more negative the statement, the more visceral, cold line of sight you feel from the speaker. It is very easy to belittle someone by using ときたら whereas と言ったら doesn\textquotesingle t go beyond jokingly chastising someone. }

\begin{center}
\textbf{ときては }\hfill\break

\end{center}

\par{ ときては is a variation of ときたら which only differs in the fact that it brings into mind a cause-effect relationship. We know that the basic understand of ときたら is “With A being a prerequisite, B is only natural.” With ときては, you more explicitly state this on the lines of “Since A is so, B is only natural (as an effect).” Although very subjective, there is a very “as a matter of fact” tone to this pattern. }

\par{15. ${\overset{\textnormal{み}}{\text{見}}}$ た ${\overset{\textnormal{め}}{\text{目}}}$ も ${\overset{\textnormal{おな}}{\text{同}}}$ じ、 ${\overset{\textnormal{つか}}{\text{使}}}$ われ ${\overset{\textnormal{かた}}{\text{方}}}$ も ${\overset{\textnormal{おな}}{\text{同}}}$ じ、 ${\overset{\textnormal{やっこう}}{\text{薬効}}}$ まで ${\overset{\textnormal{おな}}{\text{同}}}$ じときては ${\overset{\textnormal{こんどう}}{\text{混同}}}$ するのも ${\overset{\textnormal{むり}}{\text{無理}}}$ はないでしょう。 \hfill\break
When it comes down to their appearance, how they\textquotesingle re used, and even to their efficacy being the same, it would not be reasonable to also confuse them. }

\par{16. ${\overset{\textnormal{ざんぎょうだい}}{\text{残業代}}}$ は ${\overset{\textnormal{まんがく}}{\text{満額}}}$ でないときては、 ${\overset{\textnormal{なっとく}}{\text{納得}}}$ できない ${\overset{\textnormal{ひと}}{\text{人}}}$ が ${\overset{\textnormal{おお}}{\text{多}}}$ い。 \hfill\break
There are many people who cannot accept it when their overtime pay is not the full amount. }

\par{17. この ${\overset{\textnormal{ざっし}}{\text{雑誌}}}$ ときては、 ${\overset{\textnormal{はんぶんいじょう}}{\text{半分以上}}}$ が ${\overset{\textnormal{せんでんこうこく}}{\text{宣伝広告}}}$ だよ。 \hfill\break
Regarding this magazine, over half of it is advertisements. }

\par{18. おまけに ${\overset{\textnormal{さつじん}}{\text{殺人}}}$ ときては ${\overset{\textnormal{わら}}{\text{笑}}}$ うに ${\overset{\textnormal{わら}}{\text{笑}}}$ えぬ ${\overset{\textnormal{はなし}}{\text{話}}}$ だ。 \hfill\break
What\textquotesingle s more is that this story is one that not even murders are able to laugh at. }

\par{19. しかも ${\overset{\textnormal{しょうばい}}{\text{商売}}}$ にならないほどの ${\overset{\textnormal{やすね}}{\text{安値}}}$ ときてはどうにも ${\overset{\textnormal{しょうひ}}{\text{消費}}}$ しきれないだろう。 \hfill\break
Moreover, especially when it\textquotesingle s at such a low price it won\textquotesingle t even turn into business, you can\textquotesingle t possibly go through all of it. }

\par{20. おまけにハンサムときては ${\overset{\textnormal{みな}}{\text{皆}}}$ の ${\overset{\textnormal{ねた}}{\text{妬}}}$ みを ${\overset{\textnormal{か}}{\text{買}}}$ っている。 \hfill\break
What\textquotesingle s more, his handsomeness makes everyone envious. }

\par{21. 自らで ${\overset{\textnormal{あさ}}{\text{朝}}}$ から ${\overset{\textnormal{ばん}}{\text{晩}}}$ まで ${\overset{\textnormal{くろう}}{\text{苦労}}}$ して ${\overset{\textnormal{しょくじ}}{\text{食事}}}$ の ${\overset{\textnormal{ようい}}{\text{用意}}}$ をしても「 ${\overset{\textnormal{てん}}{\text{天}}}$ にましますお父さま、 ${\overset{\textnormal{きょう}}{\text{今日}}}$ の ${\overset{\textnormal{しょくじ}}{\text{食事}}}$ を ${\overset{\textnormal{かんしゃ}}{\text{感謝}}}$ します」ときては、 ${\overset{\textnormal{かみさま}}{\text{神様}}}$ も ${\overset{\textnormal{こま}}{\text{困}}}$ ってしまうだろう。 \hfill\break
When you still pray, "Father who art in heaven, we thank you for this meal,” despite having worked on your own from morning to night and prepare the meal, God will be surely troubled as well. }

\begin{center}
\textbf{とくると }
\end{center}

\par{ This variation is the most objective form, and this is because of the use of と rather than the other conditional particles. As an effect, it doesn\textquotesingle t get used at scolding statements directed at others. It is essentially the same as と言うと with the only difference being that it is not near as common. This is simply due to the fact that the use of くる to stand for いう is in decline overall. }

\par{22. ${\overset{\textnormal{さけ}}{\text{酒}}}$ とくると、からっきし ${\overset{\textnormal{だめ}}{\text{駄目}}}$ だ。 \hfill\break
When it comes to alcohol, I\textquotesingle m absolutely hopeless. }

\par{23. ${\overset{\textnormal{おんがく}}{\text{音楽}}}$ とくると、やはりモーツァルトですね。 \hfill\break
When it comes to music, Mozart is definitely where it\textquotesingle s at, you know. }

\begin{center}
\textbf{ときた日にゃ(あ) }
\end{center}

\par{ The compound particle には, in either dialectical and\slash or old-fashioned speech, can be contracted as にゃ(あ). Putting this aside, ときた日\{には・にゃ(あ)\} is simply a somewhat old-fashioned variant of ときたら. The 日 in this phrase is equivalent to 場合. Many speakers do not even know what this phrase is anymore, but it does appear in literature as well as in Early Modern Japanese. Meaning, if you like reading things from Natsume Sōseki (夏目漱石), you will find it. }

\par{24. うちの人と ${\overset{\textnormal{き}}{\text{来}}}$ た ${\overset{\textnormal{ひ}}{\text{日}}}$ にゃ、 ${\overset{\textnormal{たいへん}}{\text{大変}}}$ なヤキモチやきでね。 \hfill\break
When it comes to my partner, he\slash she is extremely jealous, you see. }

\par{25. うちの ${\overset{\textnormal{むすこ}}{\text{息子}}}$ と ${\overset{\textnormal{き}}{\text{来}}}$ た ${\overset{\textnormal{ひ}}{\text{日}}}$ には、 ${\overset{\textnormal{さき}}{\text{先}}}$ が ${\overset{\textnormal{おも}}{\text{思}}}$ いやられる。 \hfill\break
When it comes to my son, I have no idea what is going to happen. }

\par{26. ${\overset{\textnormal{まつおばしょう}}{\text{松尾芭蕉}}}$ と ${\overset{\textnormal{き}}{\text{来}}}$ た ${\overset{\textnormal{ひ}}{\text{日}}}$ にゃあ、 ${\overset{\textnormal{おおばか}}{\text{大馬鹿}}}$ じゃ。 \hfill\break
When it comes to Basho Matsuo, he is an utter fool. }

\par{27. ${\overset{\textnormal{あ}}{\text{会}}}$ ったのは ${\overset{\textnormal{だんせい}}{\text{男性}}}$ の ${\overset{\textnormal{こうこうせい}}{\text{高校生}}}$ 、ついでに ${\overset{\textnormal{びしょうねん}}{\text{美少年}}}$ ときた ${\overset{\textnormal{ひ}}{\text{日}}}$ にゃあ、 ${\overset{\textnormal{よる}}{\text{夜}}}$ も ${\overset{\textnormal{ねむ}}{\text{眠}}}$ れない。 \hfill\break
Who I met is a guy in high school, and incidentally he\textquotesingle s a handsome guy; I can\textquotesingle t even sleep at night. }

\par{28. ${\overset{\textnormal{おとな}}{\text{大人}}}$ になるときた ${\overset{\textnormal{ひ}}{\text{日}}}$ にゃ、まったくしょうがない。 \hfill\break
When it comes to becoming an adult, it absolutely can\textquotesingle t be helped. }

\begin{center}
\textbf{ときている(もんだ) }
\end{center}

\par{ When ときたら is paraphrased to come at the end of a sentence, you get ときている(ものだ). The use of \{もの・もん\}だ is there to simply imply that the statement is common sense, but the principles in understanding the phrase at large mentioned above still apply. Typically, the phrase is partnered with phrases that equate to “in addition to,” “what\textquotesingle s more,” etc. These phrases include ~うえに, ~に加えて, おまけに~, etc. The best way to translate this, although translation is not always necessary to reflect its meaning, is “to boot.” }

\par{31. ${\overset{\textnormal{かれし}}{\text{彼氏}}}$ は ${\overset{\textnormal{かねづか}}{\text{金遣}}}$ いが ${\overset{\textnormal{あら}}{\text{荒}}}$ いのに ${\overset{\textnormal{くわ}}{\text{加}}}$ えて、 ${\overset{\textnormal{わ}}{\text{我}}}$ が ${\overset{\textnormal{まま}}{\text{儘}}}$ ときている。 \hfill\break
In addition to his use of money being wasteful, my boyfriend is selfish to boot. }

\par{\textbf{Sentence Note }: The prerequisite for the boyfriend\textquotesingle s selfishness would be the person\textquotesingle s inherent nature which not only doesn't stop at wasting money but which also leads to being selfish as a natural effect. }

\par{32. ${\overset{\textnormal{かしこ}}{\text{賢}}}$ い ${\overset{\textnormal{うえ}}{\text{上}}}$ に、 ${\overset{\textnormal{せいかく}}{\text{性格}}}$ もいいときてるもんだから、 ${\overset{\textnormal{こま}}{\text{困}}}$ ってます。 \hfill\break
On top of being wise, since (his\slash her) personality is also nice to boot, which is what I\textquotesingle m grappling with. }

\par{33. サラリーマンは ${\overset{\textnormal{きらく}}{\text{気楽}}}$ な ${\overset{\textnormal{かぎょう}}{\text{家業}}}$ ときたもんだ。 \hfill\break
Salary-men have a carefree line of work to boot. }

\par{\textbf{Sentence Note }: The prerequisite of the “carefree nature” indicative of a salary-man have would be based on the nature of their work. In today\textquotesingle s Japan, this statement would not be true, but in the past, this was a very prominent critique of the leisure many saw in their lifestyles. }

\par{34. とんだタイミングときたもんだ。 \hfill\break
What unthinkable timing. }

\par{\textbf{Sentence Note }: Although the prerequisite is not mentioned, one can imagine that the situation with the “horribly unthinkable timing” was a domino effect of bad circumstances. }

\par{35. 水も滴るいい男ときたもんだわ。 \hfill\break
What a breathtakingly beautiful, nice guy he is! }

\par{36. この ${\overset{\textnormal{ゆかた}}{\text{浴衣}}}$ と ${\overset{\textnormal{い}}{\text{言}}}$ ったら ${\overset{\textnormal{きれい}}{\text{綺麗}}}$ な ${\overset{\textnormal{うえ}}{\text{上}}}$ に ${\overset{\textnormal{やす}}{\text{安}}}$ いときている。 \hfill\break
Talking about this yukata, on top of it being so pretty, it\textquotesingle s cheap to boot. }

\par{37. ${\overset{\textnormal{ぜんぜんおい}}{\text{全然美味}}}$ しくないときたもんだ。 \hfill\break
It\textquotesingle s absolutely disgusting to boot. (taste) }

\par{38. どちらもしっかり ${\overset{\textnormal{いち}}{\text{1}}}$ ${\overset{\textnormal{にんまえ}}{\text{人前}}}$ で、おまけに ${\overset{\textnormal{きゅうひゃく}}{\text{900}}}$ ${\overset{\textnormal{えん}}{\text{円}}}$ ときたもんだ。 \hfill\break
Both are full portions, and what\textquotesingle s more is that they\textquotesingle re 900 yen! }

\par{39. あの ${\overset{\textnormal{ひと}}{\text{人}}}$ は、 ${\overset{\textnormal{なに}}{\text{何}}}$ をするか ${\overset{\textnormal{わ}}{\text{分}}}$ からない ${\overset{\textnormal{うえ}}{\text{上}}}$ に、 ${\overset{\textnormal{ものおぼ}}{\text{物覚}}}$ えが ${\overset{\textnormal{わる}}{\text{悪}}}$ い。おまけにコーヒーの ${\overset{\textnormal{あじ}}{\text{味}}}$ でさえ ${\overset{\textnormal{まいにちちが}}{\text{毎日違}}}$ うときたもんです。 \hfill\break
On top of not knowing what he\textquotesingle s doing, that person has a terrible memory. To make matters worse, the taste of the coffee (he makes) is different each day. }

\par{40. あいつは、いつも ${\overset{\textnormal{じぶん}}{\text{自分}}}$ のことしか ${\overset{\textnormal{かんが}}{\text{考}}}$ えてないのに ${\overset{\textnormal{こま}}{\text{困}}}$ ったときだけ ${\overset{\textnormal{たす}}{\text{助}}}$ けてくれときたもんだ。 \hfill\break
Even though this guy is always thinking just about himself, what\textquotesingle s more is that he only cries for help only when he\textquotesingle s in trouble. }

\par{41. ${\overset{\textnormal{じょおうばち}}{\text{女王蜂}}}$ は ${\overset{\textnormal{かこく}}{\text{過酷}}}$ な ${\overset{\textnormal{かぎょう}}{\text{家業}}}$ ときたものだ。 \hfill\break
Queen bees have a cruel occupation to boot. }

\par{42. おまけに ${\overset{\textnormal{べんきょう}}{\text{勉強}}}$ も ${\overset{\textnormal{でき}}{\text{出来}}}$ て ${\overset{\textnormal{うんどう}}{\text{運動}}}$ もそこそこときてるもんだ。 \hfill\break
What\textquotesingle s more, I\textquotesingle ve been able to study as well as reasonably exercise. }

\begin{center}
\textbf{とくれば }
\end{center}

\par{ We know that with the use of the particle ば, you can create a conditional phrase that is subjective, relating to present\slash future situation, and that also leads to a desire of the speaker. As such, this is never used in a negative connotation. This, as you can imagine, is a more emphatic version of と言えば. }

\par{44. ${\overset{\textnormal{うみ}}{\text{海}}}$ とくれば、 ${\overset{\textnormal{かいすいよく}}{\text{海水浴}}}$ だ。 \hfill\break
When it comes to the ocean, you think of sea bathing. }

\par{45. ${\overset{\textnormal{おおさか}}{\text{大阪}}}$ とくれば、やっぱりたこ ${\overset{\textnormal{や}}{\text{焼}}}$ きが ${\overset{\textnormal{いちばん}}{\text{一番}}}$ に ${\overset{\textnormal{おも}}{\text{思}}}$ い ${\overset{\textnormal{う}}{\text{浮}}}$ かびますね。 \hfill\break
When it comes to Ōsaka, takoyaki definitely first comes to mind. }

\par{46. 前回は99点とくれば、次は100点が取れそうだね。 \hfill\break
With it being that I got a 99 last time, it seems that I will get a 100 next time, huh. }

\par{47. クエン酸とくれば疲労回復! \hfill\break
When it comes to citric acid, think recovery from fatigue! }
    