    
\chapter{The Particle まで II}

\begin{center}
\begin{Large}
第251課: The Particle まで II 
\end{Large}
\end{center}
 
\par{ This small lesson will be about difficult usages of the particle まで. Half of this lesson will focus on grammar points that are still quite important to know but are more difficult to use. The rest of the lesson involves grammar that is either outdated or restricted to the written language. However, because these usages have not died out entirely, they are worth mentioning. As we are now well into our advanced studies, it's more important for you to know what all is out there so that you are not confused by small things.  }
      
\section{~てまで}
 
\par{ Some people get utterly confused when they see this pattern. Nothing has really changed, but the implication of using this is slightly different than a phrase with just "Non-past+まで". This shows that one goes all the way to doing action X. This is often used with the particle のに, which means "although". }

\par{1. ${\overset{\textnormal{けんみんたいかい}}{\text{県民大会}}}$ を ${\overset{\textnormal{}}{\text{開}}}$ いてまで ${\overset{\textnormal{}}{\text{反対}}}$ を ${\overset{\textnormal{うった}}{\text{訴}}}$ えたのに、オスプレイは ${\overset{\textnormal{はいび}}{\text{配備}}}$ されてしまった。 \hfill\break
Although we voiced our protest by going as far as to open a meeting of the prefecture citizens, the Ospreys ended up being deployed. \hfill\break
From NHK on January 22, 2012. }

\par{2. 10万ドル ${\overset{\textnormal{はら}}{\text{払}}}$ ってまでテキサス大学に入りたくないよ。 \hfill\break
I don't want to go to the University of Texas to the point of paying 100,000 dollars. }

\par{3. 体を壊してまで、仕事を続ける。 \hfill\break
To continue the\slash a job until one hurts one's health (literally: until one breaks one's body). }

\par{4. 「苦肉の策」は自分の身を傷めてまでも敵を欺くことを指す言葉です。 \hfill\break
"Kuniku no saku" is a phrase that refers to deceiving an enemy even to the point of hurting oneself. }
      
\section{~まで(のこと)だ}
 
\par{ ~まで(のこと)だ shows that nothing else extends out of the action in question. This can be used with the past tense because the particle is literally only being used in the sense of "extent". So, just as you can "to the extent I have done" in English, you can say in Japanese したまでだ. This, though, does not negate the fact that there are still situations just like earlier in which using the past tense with まで would be unnatural. The のこと that is optional in the phrase is essentially just a filler phrase. }

\par{5. ${\overset{\textnormal{}}{\text{考}}}$ え ${\overset{\textnormal{}}{\text{事}}}$ をしていたまでです。 \hfill\break
I was just thinking (and nothing more).   }

\par{ Used with the negative, まで shows a meaning of "no need to do so". With this, there are two important situations. ~まで(のこと)もない means "there's no need to\dothyp{}\dothyp{}\dothyp{}" and ~ないまでも means "even though it's not necessary to\dothyp{}\dothyp{}\dothyp{}at least\dothyp{}\dothyp{}\dothyp{}". ~たまで is impossible. }

\par{6. それをするまでもない。 \hfill\break
There's no need to do that. }

\par{7. ${\overset{\textnormal{}}{\text{手紙}}}$ に書くまでもないことだから、 ${\overset{\textnormal{}}{\text{電話}}}$ で ${\overset{\textnormal{}}{\text{伝}}}$ える。 \hfill\break
Because it's not necessary to write a letter, I'll tell you on the telephone. }

\par{8. ${\overset{\textnormal{}}{\text{確}}}$ かめるまでもなく ${\overset{\textnormal{}}{\text{明}}}$ らかだった。 \hfill\break
It was clear without even having to make sure. }

\par{9. それは言うまでもないよ。 \hfill\break
That goes without saying. }
      
\section{までが}
 
\par{ Particles like が and を \emph{can }follow まで even though まで is a case particle. This, though, has to deal with the fact that this particle was once a noun. In which, this wouldn't be that exceptional. This pattern is somewhat uncommon. After all, this quote comes from a book that is somewhat old. It's simply fun to know that such phrasing is still possible every now and then. }

\par{10.足の下の ${\overset{\textnormal{たたみ}}{\text{畳}}}$ \textbf{までが }${\overset{\textnormal{ひ}}{\text{冷}}}$ えて来るので、一人で ${\overset{\textnormal{ゆ}}{\text{湯}}}$ に行こうとすると、 \hfill\break
「待って下さい。私も行きます。」と、今度は女が ${\overset{\textnormal{すなお}}{\text{素直}}}$ について来た。 \hfill\break
While I was about to go bathe alone since even the tatami mat underneath my feet was getting \hfill\break
cold, she finally came by this time and meekly said , "Wait for me. I'm going too". }

\par{From 雪国 by 川端康成. }

\par{7. 足の下の ${\overset{\textnormal{たたみ}}{\text{畳}}}$ \textbf{までが }${\overset{\textnormal{ひ}}{\text{冷}}}$ えて来るので、一人で ${\overset{\textnormal{ゆ}}{\text{湯}}}$ に行こうとすると、 \hfill\break
「待って下さい。私も行きます。」と、今度は女が ${\overset{\textnormal{すなお}}{\text{素直}}}$ について来た。 \hfill\break
While I was about to go bathe alone since even the tatami mat underneath my feet was getting \hfill\break
cold, she finally came by this time and meekly said , "Wait for me. I'm going too". }

\par{From 雪国 by 川端康成. }

\par{\textbf{Grammar Notes }: }

\par{1. When reading things from actual books, you are going to inevitably encounter grammar you've not seen before. In this sentence, ~までが shows us that particles like が and を can follow まで even though まで is a case particle. This, though, has to deal with the fact that this particle was once a noun. In which, this wouldn't be that exceptional. This pattern is somewhat uncommon. After all, this quote comes from a book that is somewhat old. It's simply fun to know that such phrasing is still possible every now and then. }

\par{2. ので shows reason in this sentence. }

\par{3. 行こうとすると is a combination of several things, but the phrase as a whole shows that when the speaker starts to try to go, the following thing happens. }

\par{7. 足の下の ${\overset{\textnormal{たたみ}}{\text{畳}}}$ \textbf{までが }${\overset{\textnormal{ひ}}{\text{冷}}}$ えて来るので、一人で ${\overset{\textnormal{ゆ}}{\text{湯}}}$ に行こうとすると、 \hfill\break
「待って下さい。私も行きます。」と、今度は女が ${\overset{\textnormal{すなお}}{\text{素直}}}$ について来た。 \hfill\break
While I was about to go bathe alone since even the tatami mat underneath my feet was getting \hfill\break
cold, she finally came by this time and meekly said , "Wait for me. I'm going too". }

\par{From 雪国 by 川端康成. }

\par{\textbf{Grammar Notes }: }

\par{1. When reading things from actual books, you are going to inevitably encounter grammar you've not seen before. In this sentence, ~までが shows us that particles like が and を can follow まで even though まで is a case particle. This, though, has to deal with the fact that this particle was once a noun. In which, this wouldn't be that exceptional. This pattern is somewhat uncommon. After all, this quote comes from a book that is somewhat old. It's simply fun to know that such phrasing is still possible every now and then. }

\par{2. ので shows reason in this sentence. }

\par{3. 行こうとすると is a combination of several things, but the phrase as a whole shows that when the speaker starts to try to go, the following thing happens. }

\par{7. 足の下の ${\overset{\textnormal{たたみ}}{\text{畳}}}$ \textbf{までが }${\overset{\textnormal{ひ}}{\text{冷}}}$ えて来るので、一人で ${\overset{\textnormal{ゆ}}{\text{湯}}}$ に行こうとすると、 \hfill\break
「待って下さい。私も行きます。」と、今度は女が ${\overset{\textnormal{すなお}}{\text{素直}}}$ について来た。 \hfill\break
While I was about to go bathe alone since even the tatami mat underneath my feet was getting \hfill\break
cold, she finally came by this time and meekly said , "Wait for me. I'm going too". }

\par{From 雪国 by 川端康成. }

\par{\textbf{Grammar Notes }: }

\par{1. When reading things from actual books, you are going to inevitably encounter grammar you've not seen before. In this sentence, ~までが shows us that particles like が and を can follow まで even though まで is a case particle. This, though, has to deal with the fact that this particle was once a noun. In which, this wouldn't be that exceptional. This pattern is somewhat uncommon. After all, this quote comes from a book that is somewhat old. It's simply fun to know that such phrasing is still possible every now and then. }

\par{2. ので shows reason in this sentence. }

\par{3. 行こうとすると is a combination of several things, but the phrase as a whole shows that when the speaker starts to try to go, the following thing happens. }
      
\section{~ほどまでに}
 
\par{ When it is after the 連体形 of a verb or in (ほど)までに, it shows to what extent something has reached. Unlike the case までに, what it is after isn't necessarily the actual end point. For instance, the speaker in the example obviously didn't actually die. Having ほど there, which is a particle from a noun meaning "extent", helps. However, the sentence could easily not have it. }

\par{11. わたしは死ぬほど(までに) ${\overset{\textnormal{くる}}{\text{苦}}}$ しんだ。 \hfill\break
I suffered to the death.  }
      
\section{~までをも \& ~までもを}
 
\par{ ~までも, of course, is quite emphatic, but you may be surprised to find out that ~までをも and ~までもを also exist, and if your teachers deny it, they need to face reality the same way people need to recognize that "ain't" is indeed a word in the English speaking world. }

\par{ Remember that we are using the adverbial particle まで. When being very formal in one's speech, the particle を surfaces in this pattern. Why? With the change in category, no matter how arbitrary it may seem, it is a logical trigger for を to appear. }

\par{ Some speakers don't like either patterns, which is fine. If speakers just don't like one of them, it'll be ~までもを that they don't like. Compounds like までに are fine, but other combinations like はを or もに are just wrong. These facts lead to speakers' opinions of liking ~までをも but not ~までもを. }

\par{ Nevertheless, both are and have been used, and even the egregious ~までもを  appears in the famous novel 海辺のカフカ by the renowned author 村上春樹. This man is no idiot, and although this doesn't mean he never messes up his grammar, it's highly likely that he thinks it's quite OK and probably used it on purpose. }

\par{12. 日本人までをも ${\overset{\textnormal{みりょう}}{\text{魅了}}}$ した ${\overset{\textnormal{いぶんか}}{\text{異文化}}}$ の ${\overset{\textnormal{しゅうかん}}{\text{習慣}}}$ \hfill\break
A tradition of another culture that has even fascinated Japanese people }

\par{13. あの ${\overset{\textnormal{}}{\text{戦争}}}$ で愛する ${\overset{\textnormal{おっと}}{\text{夫}}}$ と父をなくし、 ${\overset{\textnormal{しゅうせんご}}{\text{終戦後}}}$ の ${\overset{\textnormal{こんらん}}{\text{混乱}}}$ の中で母 \textbf{までもを }なくし、あわただしい結婚生活の中で子どもをもうけるいとまなく、 ${\overset{\textnormal{いらいてんがいこどく}}{\text{以来天涯孤独}}}$ の ${\overset{\textnormal{み}}{\text{身}}}$ をかかえて生きて ${\overset{\textnormal{まい}}{\text{参}}}$ りました。(Humble, written speech) \hfill\break
I lost my beloved husband and father in that war, and I even lost my mother in the turmoil after the end of the war, and with there being no time to bear a child in my busy married life, I have since picked myself up and lived on alone with no relatives. \hfill\break
From 海辺のカフカ 上 by 村上春樹. }

\par{\textbf{Translation Note }: If you go word for word in the original Japanese, you\textquotesingle ll see that it doesn\textquotesingle t match literally with the English translation. However, if you were to give a literal English translation, you\textquotesingle d end up with a bad translation. Sometimes you have to take liberties in translating, and that\textquotesingle s completely fine so long as you still captivate the meaning of the original text as best as can be done. }

\par{ \textbf{~までをも・~までもを }}

\par{ ~までも, of course, is quite emphatic, but you may be surprised to find out that ~までをも and ~までもを also exist, and if your teachers deny it, they need to face reality the same way people need to recognize that "ain't" is indeed a word in the English speaking world. }

\par{ Remember that we are using the adverbial particle まで. When being very formal in one's speech, the particle を surfaces in this pattern. Why? With the change in category, no matter how arbitrary it may seem, it is a logical trigger for を to appear. }

\par{ Some speakers don't like either patterns, which is fine. If speakers just don't like one of them, it'll be ~までもを that they don't like. Compounds like までに are fine, but other combinations like はを or もに are just wrong. These facts lead to speakers' opinions of liking ~までをも but not ~までもを. }

\par{ Nevertheless, both are and have been used, and even the egregious ~までもを  appears in the famous novel 海辺のカフカ by the renowned author 村上春樹. This man is no idiot, and although this doesn't mean he never messes up his grammar, it's highly likely that he thinks it's quite OK and probably used it on purpose. }

\par{21. 日本人までをも ${\overset{\textnormal{みりょう}}{\text{魅了}}}$ した ${\overset{\textnormal{いぶんか}}{\text{異文化}}}$ の ${\overset{\textnormal{しゅうかん}}{\text{習慣}}}$ \hfill\break
A tradition of another culture that has even fascinated Japanese people }

\par{22. あの ${\overset{\textnormal{せんそう}}{\text{戦争}}}$ で愛する ${\overset{\textnormal{おっと}}{\text{夫}}}$ と父をなくし、 ${\overset{\textnormal{しゅうせんご}}{\text{終戦後}}}$ の ${\overset{\textnormal{こんらん}}{\text{混乱}}}$ の中で母 \textbf{までもを }なくし、あわただしい結婚生活の中で子どもをもうけるいとまなく、 ${\overset{\textnormal{いらいてんがいこどく}}{\text{以来天涯孤独}}}$ の ${\overset{\textnormal{み}}{\text{身}}}$ をかかえて生きて ${\overset{\textnormal{まい}}{\text{参}}}$ りました。(Humble, written speech) \hfill\break
I lost my beloved husband and father in that war, and I even lost my mother in the turmoil after the       end of the war, and with there being no time to bear a child in my busy married life, I have since           picked myself up and lived on alone with no relatives. \hfill\break
From 海辺のカフカ 上 by 村上春樹. }

\par{\textbf{Translation Note }: If you go word for word in the original Japanese, you\textquotesingle ll see that it doesn\textquotesingle t match literally with the English translation. However, if you were to give a literal English translation, you\textquotesingle d end up with a bad translation. Sometimes you have to take liberties in translating, and that\textquotesingle s completely fine so long as you still captivate the meaning of the original text as best as can be done.  }

\par{\textbf{~てまで }}
 
\par{ \textbf{~までをも・~までもを }}

\par{ ~までも, of course, is quite emphatic, but you may be surprised to find out that ~までをも and ~までもを also exist, and if your teachers deny it, they need to face reality the same way people need to recognize that "ain't" is indeed a word in the English speaking world. }

\par{ Remember that we are using the adverbial particle まで. When being very formal in one's speech, the particle を surfaces in this pattern. Why? With the change in category, no matter how arbitrary it may seem, it is a logical trigger for を to appear. }

\par{ Some speakers don't like either patterns, which is fine. If speakers just don't like one of them, it'll be ~までもを that they don't like. Compounds like までに are fine, but other combinations like はを or もに are just wrong. These facts lead to speakers' opinions of liking ~までをも but not ~までもを. }

\par{ Nevertheless, both are and have been used, and even the egregious ~までもを  appears in the famous novel 海辺のカフカ by the renowned author 村上春樹. This man is no idiot, and although this doesn't mean he never messes up his grammar, it's highly likely that he thinks it's quite OK and probably used it on purpose. }

\par{21. 日本人までをも ${\overset{\textnormal{みりょう}}{\text{魅了}}}$ した ${\overset{\textnormal{いぶんか}}{\text{異文化}}}$ の ${\overset{\textnormal{しゅうかん}}{\text{習慣}}}$ \hfill\break
A tradition of another culture that has even fascinated Japanese people }

\par{22. あの ${\overset{\textnormal{せんそう}}{\text{戦争}}}$ で愛する ${\overset{\textnormal{おっと}}{\text{夫}}}$ と父をなくし、 ${\overset{\textnormal{しゅうせんご}}{\text{終戦後}}}$ の ${\overset{\textnormal{こんらん}}{\text{混乱}}}$ の中で母 \textbf{までもを }なくし、あわただしい結婚生活の中で子どもをもうけるいとまなく、 ${\overset{\textnormal{いらいてんがいこどく}}{\text{以来天涯孤独}}}$ の ${\overset{\textnormal{み}}{\text{身}}}$ をかかえて生きて ${\overset{\textnormal{まい}}{\text{参}}}$ りました。(Humble, written speech) \hfill\break
I lost my beloved husband and father in that war, and I even lost my mother in the turmoil after the       end of the war, and with there being no time to bear a child in my busy married life, I have since           picked myself up and lived on alone with no relatives. \hfill\break
From 海辺のカフカ 上 by 村上春樹. }

\par{\textbf{Translation Note }: If you go word for word in the original Japanese, you\textquotesingle ll see that it doesn\textquotesingle t match literally with the English translation. However, if you were to give a literal English translation, you\textquotesingle d end up with a bad translation. Sometimes you have to take liberties in translating, and that\textquotesingle s completely fine so long as you still captivate the meaning of the original text as best as can be done.  }

\par{\textbf{~てまで }}
     