    
\chapter{Potential IV}

\begin{center}
\begin{Large}
第278課: Potential IV: ~かねる・かねない VS ~きれる・きれない VS ~える\slash うる・~えない 
\end{Large}
\end{center}
 
\par{ Now that you have learned more Japanese, it is now time to delve into differences with phrases related to "can" or "can't” that are more advanced and somewhat difficult to get straight. Be careful with the role of ~ない in this lesson. }

\par{  Certainly, when a student realizes that 理解しきれない, 理解しえない, and 理解しかねる are possible on top of 理解できない, to the student, it seems at first glance that they must be interchangeable because they all translate as “can\textquotesingle t understand”. However, as with any similar phrases in Japanese, there will always be differences. Let\textquotesingle s get started at figuring out what these differences are. }
 
\par{ Although there are clear nuance differences and writing style differences between them, as the example with 理解する demonstrates, ~きれない, ~えない, and ~かねる have the basic meaning of “can\textquotesingle t”. It's now time to look at them individually and then compare. }
      
\section{~かねる・かねない}
 
\par{ The 一段 verb かねる means " \textbf{ }to serve\slash combine two or more functions or roles simultaneously". Other verbs that could be used instead for this include 組み合わせる and 兼ね備える. }

\par{1. 彼は教師と教育理事長を兼ねています。 \hfill\break
He is both a teacher and the education chair. }

\par{2. 彼はその二つの部の部長を兼ねています。 \hfill\break
He is in charge of the two departments. }

\par{3. 総理が外相を兼ねた。 \hfill\break
The Prime Minister concurrently held the Foreign Minister position. }

\par{4. 朝ごはんと昼ごはんをかねて遅めに食べることが多い。 \hfill\break
I eat breakfast with lunch late a lot. }

\par{5. 応接間と ${\overset{\textnormal{しょさい}}{\text{書斎}}}$ を兼ねた部屋です。 \hfill\break
This is a room that serves as a parlor and a library. }

\par{6. 趣味を兼ねた仕事 \hfill\break
Job combined with hobbies }

\par{7. 観光と取材をかねて旅行していた。 \hfill\break
We were traveling to both sight-see and cover an event. }

\par{\textbf{Meaning Note }: News' crews go places all the time not to just cover the story, but also to enjoy the place they're at. }

\par{~かねる shows that something is impossible. This is more so impossibility of doing something in a certain situation. Say you really hate someone, and you are asked if you're going to agree with him, you might say "I can't agree with such a person". So, it expresses the inability, reluctance, or refusal to do something. It can be translated as "cannot" or "(am) not able." }

\par{8. 彼は、「表明には ${\overset{\textnormal{しょうふく}}{\text{承服}}}$ しかねる」と怒りの ${\overset{\textnormal{おもも}}{\text{面持}}}$ ちで言ったそうだ。 \hfill\break
I hear that he said that he could not consent with the assertion with an angry face. }

\par{9. そんな ${\overset{\textnormal{あくとく}}{\text{悪徳}}}$ 業者には賛成しかねる。 \hfill\break
I cannot agree with such an unscrupulous businessman. }

\par{10. そんな方法には賛成しかねる。 \hfill\break
I can't approve of such methods. }

\par{11. 動機を図りかねる。 \hfill\break
To not be able to fathom the motive. }

\par{12. この場では決めかねますので、また ${\overset{\textnormal{べっと}}{\text{別途}}}$ 会議を設けましょう。 \hfill\break
Since it is impossible to decide at this place, let's set up a separate meeting again. }

\par{13. 彼女は彼氏が来るのを待ちかねていた。 \hfill\break
She couldn't wait for her boyfriend to come. }

\par{14. 申しかねますが。 \hfill\break
I hesitate to say but\dothyp{}\dothyp{}\dothyp{} }

\par{~かねない is the negative form of -かねる. Since -かねる shows the impossibility, -かねない shows that something very well "can" happen. This ending is especially used when the situation is bad. So, it means that there is a danger\slash possibility that the \emph{bad }thing might occur. }

\par{15. 彼は嘘をつきかねない男だ。 \hfill\break
He is a man capable of telling a lie. }

\par{16. 彼なら嘘をつきかねないと思う。 \hfill\break
I wouldn't put it past him to lie. }

\par{17. このまま進めば、人間の絶滅をも引き起こしかねない。 \hfill\break
If it continues at this rate, it may lead to human extinction. }

\par{18. このままでは、個人情報が ${\overset{\textnormal{ろうえい}}{\text{漏洩}}}$ しかねない。 \hfill\break
At this rate, personal information might be leaked. }

\par{19. このレフェリーは規則にうるさいので試合を台無しにしかねない。 \hfill\break
This referee is fussy about the rules, so he could ruin the match. }

\par{20. 彼はせっかくのチャンスを ${\overset{\textnormal{ぼう}}{\text{棒}}}$ に振ってしまいかねない。 \hfill\break
He might end up making a complete waste of an opportune moment. }
      
\section{~きれる・きれない}
 
\par{ This is the potential form of ~切る. It is used to show that you can completely do something, of which a lot is to be done. So, it\textquotesingle s negative form, ~きれない shows that you can\textquotesingle t do this. }

\par{ So, as for 理解しきれない, you can still understand somewhat, but you can\textquotesingle t understand entirely. It is often used with expressions that show quantity and degree. It is often written in 漢字 as ~切れない. }

\par{21. 学校でいくら注意しても子供を守りきれないんだよ。 \hfill\break
No matter how much we warn them at school, we cannot completely protect the kids. }

\par{22. 彼の ${\overset{\textnormal{こうい}}{\text{行為}}}$ は言葉で ${\overset{\textnormal{ほ}}{\text{褒}}}$ め切れないほど立派ですよ。 \hfill\break
His actions are beyond praise. }

\par{23. 言い切れます。 \hfill\break
I can completely say (that…). }

\par{24. 1日で食べきれないほどのバナナを買いました。 \hfill\break
I bought more bananas than can be eaten all in a day. }

\par{25. 待ちきれなかった。 \hfill\break
We couldn\textquotesingle t wait (completely). }
      
\section{~える・うる\slash ~えない}
 
\par{ Though this has already been taught in Potential II, remember that this pattern is old-fashioned and used in the written language. It doesn't show that something is literally impossible, but that since the conditions don\textquotesingle t match up, the possibility is slim to none. With non-volitional verbs the interpretation changes slightly to more of “it shouldn\textquotesingle t\slash can\textquotesingle t be that…”. }

\par{26. 期待しえないことだ。 \hfill\break
It\textquotesingle s something that you can\textquotesingle t anticipate. }

\par{27. 忘れ得ぬ日 \hfill\break
A day I can never forget }

\par{28. ${\overset{\textnormal{かこくじこ}}{\text{過酷事故}}}$ は ${\overset{\textnormal{げんりじょう}}{\text{原理上}}}$ 起こりえない。 \hfill\break
Severe accidents in theory can't occur. }
      
\section{Compare and Contrast}
 
\par{ All of these endings show impossibility in the forms ~かねる, ~きれない, and ~えない, but let\textquotesingle s see how well they can be used in a situation of stating one\textquotesingle s actual abilities. }

\par{29. }

\par{A: 韓国に来てどのくらいですか。 \hfill\break
B: 半年です。 \hfill\break
A: 韓国語はどの程度話せますか。 \hfill\break
B: まだ十分話\{〇 せません・△ しきれません・X しえません・X しかねます\}。 \hfill\break
A: How long has it been since you came to Korea? \hfill\break
B: It's been half a year. \hfill\break
A: How much Korean can you speak? \hfill\break
B: I still can't speak well enough. }

\par{ From this data we can see that these endings cannot simply be used to show capacity. }

\par{ The situation is not saying that you can\textquotesingle t speak a certain quantity of something completely. The situation is asking for your competency of the language itself. So, ~きれない is not correct. }

\par{ Now, consider the following. }

\par{30. 田中が犯人だと断定できましたか。 \hfill\break
Could you conclude that Tanaka is the criminal? \hfill\break
 }

\par{31. }

\par{Response 1: いえ、彼が犯人だとは断定できません。 \hfill\break
Response 2: いえ、彼が犯人だとは断定しきれません。 \hfill\break
Response 3: いえ、彼が犯人だとは断定しえません。 \hfill\break
Response 4: いえ、彼が犯人だとは断定しかねます。 \hfill\break
Translation: No, I cannot conclude that he is the criminal. \hfill\break
\hfill\break
 In this case, all four responses are correct, but they are all slightly different. In the first response, you are objective stating that it is impossible to decide. In the second situation, you state that you can\textquotesingle t 100\% call him the criminal because there is still some doubt. In the third response, you can\textquotesingle t say that he is the criminal given the surrounding circumstances. It is also very uncommon to use this in 話し言葉. The fourth response implies that you refrain from stating definitively that he is the criminal. }

\par{ With the last comment, you can see how ~かねる can be used in polite contexts in denying\slash rejecting things. }

\par{32. いろいろ考慮しましたが、お引き受けしかねます。 \hfill\break
I've thought hard on this, but I cannot undertake it. }

\par{ So, even in situations where all are possible, there will be differences as defined thus far. They are all used in showing that one can\textquotesingle t decide one way or another, as demonstrated by the criminal example. }

\par{33. 生徒が本当に分かっているのかどうか判断し\{きれない・えない・かねる\}。 \hfill\break
I can't determine whether or not the students really understand. }

\par{ They can also be used in situations where because of a certain reason or cause, you can\textquotesingle t do something or it is extremely difficult to do something. The fine details, of course, are predicated on the pattern you choose. }

\par{34. 大きすぎて、 ${\overset{\textnormal{はあく}}{\text{把握}}}$ し\{きれない・えない ・かねる\}問題だ。 \hfill\break
It's a problem too big that I cannot fully grasp. }

\par{ Remember that ~かねる implies that the speaker has a feeling of reluctance, which is part of the reason why they “can\textquotesingle t”. Again, as for ~えない, a consensus of the various circumstances can\textquotesingle t be made. So, in this example that means that the speaker can\textquotesingle t grasp certain parts due to conflicting issues. }

\begin{center}
 \textbf{Words Used }
\end{center}

\par{ It is now time to examine what kinds of words these items are frequently used with to even further distinguish them. }

\par{\textbf{X~きれない }: Used with verbs concerning with degree, quantity. It is also used with volitional verbs in regards to recognition and decision. Ex. 数える, 押さえる, 対処する, 理解する, 等. }

\par{35. 彼のアリバイ説明で納得しきれないところがあります。 \hfill\break
There are parts I\textquotesingle m not completely convinced in his alibi\textquotesingle s explanation. }

\par{\textbf{X~えない }: Used with volitional verbs in concern with recognition, progress, and remarking. It is not used with simple action verbs like 見る, 食べる, 飲む, 等. Verbs it is used with include 予期する, 期待する, 達成する, 等. It can also be used with non-volitional verbs such as ある, 存在する, 起こる, 等. }

\par{36. そんなことはありえないでしょう。 \hfill\break
Isn\textquotesingle t something like that impossible? }

\par{37. 理解し得ない感情 \hfill\break
Feelings that one cannot understand }

\par{38. 金にならないものは評価され得ない。 \hfill\break
Something that won't make money is cannot be assessed. }

\par{39. 客観的中立メディアなど存在し得ない。 \hfill\break
An objective, neutral media cannot exist. }

\par{\textbf{X~かねる }: It too is used with volitional verbs of decision but also agreement, and it is also used in a lot of other expressions given its sense of reluctance. Ex. 理解する、決める, 耐える, 等. }

\par{40. すぐには決めかねます。 \hfill\break
I can\textquotesingle t decide immediately. }

\par{41. ${\overset{\textnormal{げきつう}}{\text{激痛}}}$ に ${\overset{\textnormal{た}}{\text{耐}}}$ えかねて ${\overset{\textnormal{うめ}}{\text{呻}}}$ く。 \hfill\break
To not be able to bear the great pain and moan. }
 
\par{42. 先週、夫の ${\overset{\textnormal{ぎゃくたい}}{\text{虐待}}}$ に耐えかねて、花子は実家に帰った。 \hfill\break
Last week, unable to stand her husband\textquotesingle s abuse, Hanako returned to her parent\textquotesingle s house. }
    