    
\chapter{相槌(あいづち)}

\begin{center}
\begin{Large}
第294課: 相槌(あいづち) 
\end{Large}
\end{center}
 
\par{ 相槌 allow conversations to be prolonged and flow in Japanese. They are often misunderstood by foreigners as showing agreement. Sometimes this may be this case, but this feature of conversation is far from limited to this. }
      
\section{相槌}
 
\begin{center}
\textbf{Agreement }
\end{center}

\par{ 頷き (nodding) can show agreement as well as using words like はあ、はい、ええ、うん. Of course, you must realize that these words have different politeness implications. For instance, you shouldn't say うん to your boss. はい is the most appropriate in this situation. はあ is rather ceremonious. All of these should be used with a falling intonation for this instance. }

\par{1. はあ、分かりました。 \hfill\break
Yes, I understand\slash understood. }

\begin{center}
 \textbf{I See\dothyp{}\dothyp{}\dothyp{} }
\end{center}

\par{そうですか。  そうなんですか。  なるほど。  へえ  ふーん }

\par{  There are small differences. The first two have a slight sense of surprise with the second being slightly more emphatic. なるほど is used when something from another person is said that you yourself hadn't realized\slash known and you recognize or agree with it. へえ can be used in a somewhat reserved feeling in responding. ふーん・ふうん is used when one is a little impressed, agree or acknowledgment something. }

\par{2. ふーん、なるほど。 \hfill\break
Hm\dothyp{}\dothyp{}\dothyp{}indeed. }

\begin{center}
 \textbf{When Surprised }
\end{center}

\begin{ltabulary}{|P|P|P|P|P|P|}
\hline 

ええっ? & Eh? & 本当ですか? & Really? & 嘘 & You're lying! \\ \cline{1-6}

信じられない & I can't believe it & 冗談でしょう? & You're kidding? &  &  \\ \cline{1-6}

\end{ltabulary}

\begin{center}
 \textbf{Showing Doubt }
\end{center}

\begin{ltabulary}{|P|P|P|}
\hline 

そうですか? & そうでしょうか & そうかな(~) \\ \cline{1-3}

\end{ltabulary}

\par{\textbf{Intonation Note }: Rising intonation for these phrases. }

\par{ These phrases can be described as echo questions (おうむ返し疑問文) because they aren't true questions and are merely used to try to get at some understanding. }

\begin{center}
 \textbf{Others }
\end{center}

\par{  Anything like それで、どういうわけでしょうか、それは____ですね, etc. are also considered 相槌. Essentially, any response from another person to keep the conversation going and making the transitions in details more natural is an 相槌. The typical Japanese way of explaining starts out broad and then getting to details by the end. Although dialogues in books, anime, and manga may have quite a lot, true natural spoken Japanese is one of the best ways to hear these in use. }

\par{ Understand that 相槌 are the way to make you a good listener in Japanese. Rather than just standing there quiet, it's important to use these small phrases at appropriate stops from the speaker. Otherwise, the speaker might think that the listener isn't taking any particular attention or understanding. This is one of the easiest ways to end up with a 不安にさせる会話, and that's not a good thing. }
      
\section{Examples (相槌 in Bold)}
 
\par{3. }

\par{水谷: 土曜日に夕食に来ませんか。 \hfill\break
相田: ああ、行きたいんですが、土曜日の夜は遅くまで勉強しないといけないんです。 \hfill\break
水谷: \textbf{それは残念ですね。それでは、また今度 }。 \hfill\break
相田: \textbf{うん }、また次の機会にどこかで食べましょう。 }

\par{Mizutani: Do you want to come to dinner on Saturday? \hfill\break
Aida: Ah, I want to go, but I have to study real late Saturday night. \hfill\break
Mizutani: That's too bad. Well, next time then. \hfill\break
Aida: Sure, let's eat somewhere some other time. }

\par{4. }

\par{学生A : 最近、勉強が捗ってないんだ。 \hfill\break
学生B: \textbf{そうなんだ、 }捗ってないんだって。。。 \hfill\break
学生A: あと、1ヶ月で試験なのに \hfill\break
学生B: あと1ヶ月で試験なのか。。。Aは、今まで一生懸命勉強してきたんだから大丈夫だよ! }

\par{Student A: My students haven't been getting along recently? \hfill\break
Student B: Is that so? Not getting along\dothyp{}\dothyp{}\dothyp{} \hfill\break
Student A: Even though I have an exam in a month. \hfill\break
Student B: You have a test in a month. Well, since you've studying so hard up till now, you'll do fine! }

\begin{center}
\textbf{相槌 Gone Wrong }
\end{center}

\par{  Using 相槌 alone isn't going to magically create a nice, flowing, and friendly conversation. You can't seem distant to the speaker. Some people don't intend to be distant or mean, but it can come off that way if you act like Speaker B below. }

\par{5. }

\par{Aさん: 今日は、お会いできて光栄です。 \hfill\break
Bさん: ああ。 \hfill\break
Aさん: やはりパーティーに、ご出席される機会が多いのですか。 \hfill\break
Bさん: まあ。 \hfill\break
Aさん: たくさんの方が、お見えになっていますね。 \hfill\break
Bさん: そうですねえ。  }

\par{Aさん: It's an honor to be able to meet you today. \hfill\break
Bさん: Ah. \hfill\break
Aさん: Just as I thought, there are a lot of opportunities to attend at parties. \hfill\break
Bさん: So-so. \hfill\break
Aさん: A lot of people have come. \hfill\break
Bさん: That's true. }

\par{  Speaker A is being so formal and nice and trying to keep the conversation going, but it doesn't seem like Speaker B really even cares. You can just imagine the change of facial expressions by Speaker A. The problem is that light 相槌 can let you get out of a conversation quicker. If in a situation like this where the more courteous thing is to extend the conversation, then you should do so. It's only proper. }

\par{ The first thing to really try to do is pay close attention to what the speaker says at the very beginning of the conversation so that you can have an attempt at being able to repeat certain parts or ask questions about it. Adding opinion or emotion like surprise or disbelief is much better than being silent. Be natural, though. It's best to give a small pause to sound more natural. }

\par{ When repeating what the other person says, feel free to change up the sentence endings to show an appropriate reaction. You don't want to sound like a robot or like you can't relate. Even adding a question after giving an appropriate reaction is good too. So, Student B from earlier could have added something like 本当は、合格の自信あるんだろ?Questions like どういうわけかな, それってどういうことかな, etc. can become very useful. }

\par{\textbf{Common Mistake Note }: To use 相槌 should be 相槌を打つ, but even natives often mistakenly use 相槌を入れる instead. 相槌 comes from the concept of two blacksmiths alternately hitting the hammer together. So, it is an idiomatic expression. Since the verb you use with 槌 is 打つ, it's only natural that it would also be used with 相槌. }
    