    
\chapter{込む}

\begin{center}
\begin{Large}
第264課: 込む 
\end{Large}
\end{center}
 
\par{ In this lesson we'll learn all about 込む. It's interesting to know that the character 込 is a Japanese-made 漢字. This verb and how it's used often throws people off, especially when trying to translate stuff. }
      
\section{込む}
   込む means "to be crowded" or "to continue in the same state". Of all of the compound verb endings in Japanese, -込む is by far one of the most important. This is because it is used to create hundreds of unique verbs. ~込む has four broad usages.  
\par{\textbf{THE 4 USAGES }: 1: Going inside     2: Put inside     3: Keep as is     4: Doing enough }

\begin{ltabulary}{|P|P|P|P|P|P|}
\hline 

(1) To plunge & 飛び込む & とびこむ & (2) To corner into & 追い込む & おいこむ \\ \cline{1-6}

(3) To mope & 塞ぎ込む & ふさぎこむ & (4) To drill (teaching) & 教え込む & おしえこむ \\ \cline{1-6}

(4) To boil well & 煮込む & にこむ & (3) To age & 老け込む & ふけこむ \\ \cline{1-6}

(2) To cram & 詰め込む & つめこむ & (1) To invade & 攻め込む & せめこむ \\ \cline{1-6}

(1) To run in yelling & 怒鳴り込む & どなりこむ & (2) To entice & 誘い込む & さそいこむ \\ \cline{1-6}

(1) To march in & 繰り込む & くりこむ & (1) To roll in & 転がり込む & ころがりこむ \\ \cline{1-6}

(3) To sit down & 座り込む & すわりこむ & (2) To count on & 見込む & みこむ \\ \cline{1-6}

\end{ltabulary}
\hfill\break
 
\begin{center}
\textbf{Examples } 
\end{center}

\par{1. ${\overset{\textnormal{がけ}}{\text{崖}}}$ から海に飛び込むのは危ない。 \hfill\break
Plunging into the sea from a cliff is dangerous. }

\par{2. ${\overset{\textnormal{てき}}{\text{敵}}}$ を ${\overset{\textnormal{きゅうち}}{\text{窮地}}}$ に追い込むのは重要な ${\overset{\textnormal{せんりゃく}}{\text{戦略}}}$ です。 \hfill\break
Cornering the enemy is an important war strategy. }
 
\par{3. 冷たい空気が流れ込んでいます。 \hfill\break
Cold air is flowing in. }
 
\par{4. 我々の大学が ${\overset{\textnormal{てっていてき}}{\text{徹底的}}}$ に全教科課程を学生の頭に教え込むためには、そういった教授を雇用する ${\overset{\textnormal{}}{\text{必要があります。}}}$ \hfill\break
It is necessary that our university employ professors that exhaustively drill the curriculum into the minds of their students. }
 
\par{\textbf{Ending Note }: More compound verbs are based off of the solely transitive form ~こめる. }
 
\par{5. 部屋に閉じ込められた。 \hfill\break
I was locked in a room. }

\par{6. ${\overset{\textnormal{さくどう}}{\text{策動}}}$ を ${\overset{\textnormal{ふう}}{\text{封}}}$ じ込める。(Rare) \hfill\break
To contain a scheme. }
込・籠める 
\par{1. To load bullets. }

\par{8. 実弾を込める。 \hfill\break
To hold live fire. }

\par{2. To pour emotion into. This is where を込めて mostly applies. }

\par{9. 彼は満身の力を込めて大石を投げた。 \hfill\break
He threw a great stone with all his strength. \hfill\break
}

\par{10. 心をこめて持て成す。 \hfill\break
To make welcome with (all) one's heart. }

\par{3. To include another item. Using 含める is more common. }

\par{11. 消費税を込めた料金 \hfill\break
Fee with sales tax included. }

\par{4. As an intransitive verb, it means "enshroud (clouds, mist, etc.)". }

\par{12a. 霧が町に込めた。(Old-fashioned) \hfill\break
12b. 霧が町に立ち込めた。(Natural) \hfill\break
Mist enshrouded the village. }

\par{~込める }

\par{~込める may mean to "put into" or "to do away with". Examples include the following. \hfill\break
}

\begin{ltabulary}{|P|P|}
\hline 

To imprison & 閉じ込める \\ \cline{1-2}

To seal up & 塗り込める \\ \cline{1-2}

To argue down & 言い込める \\ \cline{1-2}

To envelop & 立ち込める \\ \cline{1-2}

To confine & 封じ込める \\ \cline{1-2}

To shut up & 押し込める \\ \cline{1-2}

\end{ltabulary}

\par{13. 雲の立ち込めた山を見よ。 \hfill\break
Look at the cloud enveloped mountains! }

\par{14. 相手を言いこめた。 \hfill\break
I argued down my opponent. }

\par{15. 強盗は人質を物置に押し込めた。 \hfill\break
The burglar pushed the hostages down into a closet. }

\par{16. 強盗は人質を物置きに押し込んだ。 \hfill\break
The burglar pushed the hostages into the closet. }

\par{17. 神社には、壁に塗り込められた経文がある。 \hfill\break
There are sutras painted over in walls at shrines. }
    