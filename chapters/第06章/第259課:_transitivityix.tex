    
\chapter{Intransitive \& Transitive}

\begin{center}
\begin{Large}
第259課: Intransitive \& Transitive: Part 8 
\end{Large}
\end{center}
 
\par{ In this eighth and final installment on verbs that are both intransitive and transitive, we will master five more verbs that deserve special attention. As has been the case for the last two lessons, their usages with the particle を can more or less be categorized in the three categories below: }

\par{1. Either the intransitive or the transitive usage is relatively new in the language. Meaning, some speakers will think it\textquotesingle s wrong to use it a certain way but many speakers still do. \hfill\break
2. The use of the verb in a transitive sense is done so to implicitly show a connection between an agent and an action. \hfill\break
3. The use of the verb in a transitive sense is done so to emphasize the agent\textquotesingle s volition in said action. }
      
\section{垂れる vs 垂らす}
 
\par{ The verb 垂れる is an intransitive verb with the basic meanings of “to hang\slash droop\slash drip” and its transitive pair is 垂らす, whose basic meanings are “to suspend\slash hang down\slash dribble.” However, the two happen to overlap each other in the sense of “to hang\slash droop.” When the act of suspension is involuntary to a reasonable degree but there is an agent for said suspension, 垂れる is used over 垂らす. This is because \emph{ }垂らす shows an active effort to suspend something. }

\par{ When you use \emph{ }垂れる, there is a general nuance that the object is not totally detached from the agent. In a sense, it\textquotesingle s as if it is an extension of said agent. For instance, when you\textquotesingle re dangling a fishing lure into a lake, it and the rod that\textquotesingle s undoubtedly in your hand can be viewed as an extension of your hand. }

\par{ Ironically, however, there are some things that even if they are somehow appended to you, you can\textquotesingle t use 垂れる because there\textquotesingle s no way you aren\textquotesingle t purposely hanging said thing down. One example of this is “bangs” (前髪). Men and women alike who have bangs consciously let their bangs down. Even if they aren\textquotesingle t, idiomatically speaking, only 前髪を垂らす is used. }

\par{ When, then, can only 垂れる be used? In addition to the basic meanings described already, it can also be used as a condescending variant of the verb 言う (to say). }

\par{1. ${\overset{\textnormal{かれ}}{\text{彼}}}$ は ${\overset{\textnormal{りょううで}}{\text{両腕}}}$ を ${\overset{\textnormal{た}}{\text{垂}}}$ れて ${\overset{\textnormal{しず}}{\text{静}}}$ かに ${\overset{\textnormal{た}}{\text{立}}}$ っていた。 \hfill\break
He stood silently with both arms drooping. }

\par{2. ${\overset{\textnormal{りょううで}}{\text{両腕}}}$ を ${\overset{\textnormal{まえ}}{\text{前}}}$ に ${\overset{\textnormal{た}}{\text{垂}}}$ らします。 \hfill\break
Dangle both arms in front of you. }

\par{\textbf{Nuance Note }: This would be something that you would hear in an exercise program. }

\par{3. ${\overset{\textnormal{かれ}}{\text{彼}}}$ は ${\overset{\textnormal{くち}}{\text{口}}}$ をだらりと ${\overset{\textnormal{ひら}}{\text{開}}}$ けたまま ${\overset{\textnormal{よだれ}}{\text{涎}}}$ を ${\overset{\textnormal{た}}{\text{垂}}}$ らしている。 \hfill\break
He is slobbering all over with his mouth languidly open. }

\par{4. ${\overset{\textnormal{もんく}}{\text{文句}}}$ を\{ ${\overset{\textnormal{た}}{\text{垂}}}$ れる・ ${\overset{\textnormal{い}}{\text{言}}}$ う\}な。 \hfill\break
Don\textquotesingle t complain. }

\par{5. ${\overset{\textnormal{むいしき}}{\text{無意識}}}$ のうちに ${\overset{\textnormal{よだれ}}{\text{涎}}}$ が ${\overset{\textnormal{た}}{\text{垂}}}$ れているのは ${\overset{\textnormal{なに}}{\text{何}}}$ かの ${\overset{\textnormal{しっかん}}{\text{疾患}}}$ でしょうか。 \hfill\break
Is unconsciously having slobber drool a sign of some ailment? }

\par{6. ${\overset{\textnormal{にほん}}{\text{日本}}}$ では、 ${\overset{\textnormal{じょせい}}{\text{女性}}}$ が ${\overset{\textnormal{まえがみ}}{\text{前髪}}}$ を ${\overset{\textnormal{た}}{\text{垂}}}$ らして ${\overset{\textnormal{ひたい}}{\text{額}}}$ を ${\overset{\textnormal{だ}}{\text{出}}}$ さずに ${\overset{\textnormal{よこ}}{\text{横}}}$ に ${\overset{\textnormal{なが}}{\text{流}}}$ すのが ${\overset{\textnormal{はや}}{\text{流行}}}$ っている。 \hfill\break
In Japan, for women who have bangs, putting them sideways without covering the forehead is a fashion trend. }

\par{7. ${\overset{\textnormal{なに}}{\text{何}}}$ かにつけて ${\overset{\textnormal{うんちく}}{\text{ウンチク}}}$ を ${\overset{\textnormal{た}}{\text{垂}}}$ れまくるやつって ${\overset{\textnormal{ほんとう}}{\text{本当}}}$ にうるさいよね。 \hfill\break
People who just have to say anything and everything they know are really annoying, huh. }

\par{\textbf{Spelling Notes }: ウンチク can seldom be spelled as 薀蓄 or 蘊蓄. Also, うるさい may seldom be spelled as 五月蝿い or 煩い. }

\par{8. ${\overset{\textnormal{しょうゆ}}{\text{醤油}}}$ を ${\overset{\textnormal{た}}{\text{垂}}}$ らしてしっかり ${\overset{\textnormal{ひ}}{\text{火}}}$ を ${\overset{\textnormal{とお}}{\text{通}}}$ してください。 \hfill\break
Drip soy sauce on it and heat it thoroughly. }

\par{9. ${\overset{\textnormal{こいぬ}}{\text{子犬}}}$ がしっぽを\{ ${\overset{\textnormal{た}}{\text{垂}}}$ れて・ ${\overset{\textnormal{た}}{\text{垂}}}$ らして\} ${\overset{\textnormal{ある}}{\text{歩}}}$ いている。 \hfill\break
The puppy has his tail down while walking. }
      
\section{気づく}
 
\par{ The verb 気付く means “to notice\slash recognize\slash become aware of” and historically only takes the particle に. However, some people use the particle を instead. There are a few grammatical situations, though, that must be put into consideration. }

\par{\textbf{Orthography Note }: 気づく may alternatively be spelled as 気付く. }

\par{1. When using the causative, there is a direct object and an indirect object. The direct object is who you\textquotesingle re making realize what. The “what” is the indirect object. However, there is no requirement that the indirect object always be in the sentence. This will make it seem as if 気づく is just used with を. }

\par{10. ${\overset{\textnormal{かれし}}{\text{彼氏}}}$ を ${\overset{\textnormal{こうしゅう}}{\text{口臭}}}$ に ${\overset{\textnormal{き}}{\text{気}}}$ づかせるためにどうしたらいいですか。 \hfill\break
What should I do in order to get my boyfriend to notice his bad breathe? }

\par{11. ${\overset{\textnormal{こうしゅう}}{\text{口臭}}}$ \{に・を??\} ${\overset{\textnormal{き}}{\text{気}}}$ づかせるためにどうしたらいいですか。 \hfill\break
What should I do in order to get him\slash her to notice his\slash her bad breathe? }

\par{\textbf{Sentence Note }: In Ex. 10, you could imagine the sentence making sense without 口臭 if it were already mentioned and it was clear in context that that is the indirect object. In Ex. 11, the lack of the direct object causes confusion to some speakers as there is a tendency for を to be used to emphasize the agent\textquotesingle s role in an action. However, in this case, Ex. 11 would literally mean that it\textquotesingle s the breathe odor you\textquotesingle re trying to convince of something, which doesn\textquotesingle t make sense. }

\par{12. ${\overset{\textnormal{しゃしん}}{\text{写真}}}$ は ${\overset{\textnormal{き}}{\text{気}}}$ づいていないこと\{に・を?\} ${\overset{\textnormal{き}}{\text{気}}}$ づかせる。 \hfill\break
Pictures make you notice things you hadn\textquotesingle t noticed before. }

\par{\textbf{Sentence Note }: Ex. 12 has the same issue as Ex. 11 when wo を is used. The one being made to notice is “you” and not the “things one hadn\textquotesingle t noticed before.” }

\par{2. The use of を instead of に is often used when using the passive. However, even in this situation, the particle に is still the ‘correct\textquotesingle  one to use. The use of を highlights the seriousness of the situation. Although who you're noticed by and what is being noticed end up both being marked by に, this poses no real problem to the naturalness of a sentence. }

\par{13. ${\overset{\textnormal{じょせい}}{\text{女性}}}$ は ${\overset{\textnormal{かみ}}{\text{髪}}}$ を ${\overset{\textnormal{き}}{\text{切}}}$ ったこと\{に・を ?\} ${\overset{\textnormal{き}}{\text{気}}}$ づかれると ${\overset{\textnormal{うれ}}{\text{嬉}}}$ しいものですか。 \hfill\break
Are women happy when their haircuts are noticed? }

\par{14. ${\overset{\textnormal{だんせい}}{\text{男性}}}$ でも ${\overset{\textnormal{かみ}}{\text{髪}}}$ を ${\overset{\textnormal{き}}{\text{切}}}$ ったこと\{に・を ?\} ${\overset{\textnormal{き}}{\text{気}}}$ づかれたら ${\overset{\textnormal{うれ}}{\text{嬉}}}$ しいもんですよ。 \hfill\break
Even men are happy when their haircuts are noticed. }

\par{15. 何かを探っていること\{に・を ?\}気づかれたら、夫婦喧嘩になることはもとより、浮気調査を行うことさえ危うくなってしまう場合があります。 \hfill\break
There are instance in which being found out about searching for something results even with performing an infidelity investigation becomes dangerous, let alone the marital disputes. }

\par{16. ${\overset{\textnormal{だんせい}}{\text{男性}}}$ は、 ${\overset{\textnormal{じぶん}}{\text{自分}}}$ が ${\overset{\textnormal{しっと}}{\text{嫉妬}}}$ していること\{に・を?\} ${\overset{\textnormal{き}}{\text{気}}}$ づかれるのが ${\overset{\textnormal{は}}{\text{恥}}}$ ずかしいと ${\overset{\textnormal{おも}}{\text{思}}}$ っている。 \hfill\break
Men think it\textquotesingle s embarrassing when they\textquotesingle re noticed being jealous. }

\par{3. Sometimes 気づく is part of an adverbial phrase, which makes it seem it\textquotesingle s with を but it isn\textquotesingle t. }

\par{17. ${\overset{\textnormal{おお}}{\text{多}}}$ くの ${\overset{\textnormal{ひと}}{\text{人}}}$ が、「 ${\overset{\textnormal{おんな}}{\text{女}}}$ らしさ」「 ${\overset{\textnormal{おとこ}}{\text{男}}}$ らしさ」を ${\overset{\textnormal{き}}{\text{気}}}$ づかないうちに信じ ${\overset{\textnormal{こ}}{\text{込}}}$ んでいるのです。 \hfill\break
Many people believe implicitly in “femininity” and “masculinity” without even realizing it. }

\par{4. Another instance of を being used with \emph{ }気づく is in 気づいてもらう. This is done to avoid two に in the same sentence, but as mentioned in the second point, this isn\textquotesingle t a valid grammatical issue. }

\par{18. どうしても ${\overset{\textnormal{もと}}{\text{元}}}$ カレに ${\overset{\textnormal{ふくえん}}{\text{復縁}}}$ したいこと\{に・を?\} ${\overset{\textnormal{き}}{\text{気}}}$ づいてもらいたいんです。 \hfill\break
No matter what it takes, I want my ex-boyfriend to realize that I want to reconcile things with him. }

\par{19. ${\overset{\textnormal{どんかん}}{\text{鈍感}}}$ な ${\overset{\textnormal{あいて}}{\text{相手}}}$ に ${\overset{\textnormal{じぶん}}{\text{自分}}}$ の ${\overset{\textnormal{こうい}}{\text{好意}}}$ \{に・を ?\} ${\overset{\textnormal{き}}{\text{気}}}$ づいてもらいたい。 \hfill\break
I want this dull person to notice my affection to him\slash her. }

\par{20. ${\overset{\textnormal{かげぐち}}{\text{陰口}}}$ を ${\overset{\textnormal{い}}{\text{言}}}$ う ${\overset{\textnormal{ひと}}{\text{人}}}$ は ${\overset{\textnormal{ばれ}}{\text{バレ}}}$ てるの\{に・ ${\overset{\textnormal{さんかく}}{\text{△}}}$ を・Ø\} ${\overset{\textnormal{き}}{\text{気}}}$ づいてないの? \hfill\break
Do people that gossip not even realize that the cat\textquotesingle s out of the bag? }
      
\section{生きる}
 
\par{ The intransitive verb “to live” is used to mean so in several ways. It can be taken literally, be used to mean “to be live” as in baseball, indicate what you dedicate your life to, etc. When used with the particle を, 生きる  indicates a more dynamic outlook on living out one\textquotesingle s life on a certain stage. }

\par{21. ウナギは ${\overset{\textnormal{かくち}}{\text{各地}}}$ の ${\overset{\textnormal{みずうみ}}{\text{湖}}}$ や ${\overset{\textnormal{かせん}}{\text{河川}}}$ に ${\overset{\textnormal{す}}{\text{住}}}$ む ${\overset{\textnormal{たんすいぎょ}}{\text{淡水魚}}}$ ですが、 ${\overset{\textnormal{たんすい}}{\text{淡水}}}$ だけでなく ${\overset{\textnormal{かいすい}}{\text{海水}}}$ でも ${\overset{\textnormal{い}}{\text{生}}}$ きることができます。 \hfill\break
The eel is a freshwater fish that lives in lakes and rivers everywhere, but it it doesn\textquotesingle t live in just fresh water. It can even life in seawater. }

\par{\textbf{Spelling Note }: ウナギ may also be spelled as 鰻. }

\par{22. ${\overset{\textnormal{しゅみ}}{\text{趣味}}}$ に ${\overset{\textnormal{い}}{\text{生}}}$ きる ${\overset{\textnormal{わかもの}}{\text{若者}}}$ はダメ! \hfill\break
Young people who subsist on their hobbies are no good! }

\par{23. ${\overset{\textnormal{かこ}}{\text{過去}}}$ を ${\overset{\textnormal{す}}{\text{捨}}}$ てて ${\overset{\textnormal{いま}}{\text{今}}}$ を ${\overset{\textnormal{い}}{\text{生}}}$ きてください。 \hfill\break
Throw away the past and live the now. }

\par{24. ${\overset{\textnormal{く}}{\text{悔}}}$ いのない ${\overset{\textnormal{せいしゅん}}{\text{青春}}}$ を ${\overset{\textnormal{い}}{\text{生}}}$ きてほしい。 \hfill\break
I want you to live out a regret-free youth. }

\par{25. ${\overset{\textnormal{たにん}}{\text{他人}}}$ の ${\overset{\textnormal{じんせい}}{\text{人生}}}$ を ${\overset{\textnormal{い}}{\text{生}}}$ きないと ${\overset{\textnormal{き}}{\text{決}}}$ めましょう。 \hfill\break
Decide not to live another\textquotesingle s life. }

\par{26. ${\overset{\textnormal{しごと}}{\text{仕事}}}$ でただ ${\overset{\textnormal{ていじ}}{\text{定時}}}$ を ${\overset{\textnormal{ま}}{\text{待}}}$ つ ${\overset{\textnormal{じんせい}}{\text{人生}}}$ を ${\overset{\textnormal{い}}{\text{生}}}$ きたくない。 \hfill\break
I don\textquotesingle t want to live a life where I\textquotesingle m just waiting for the end of the day at work. }

\par{\textbf{Word Note }: 定時 refers to the set time one gets off at in Ex. 26. }

\par{27. ${\overset{\textnormal{こども}}{\text{子供}}}$ のような ${\overset{\textnormal{たんきゅうしん}}{\text{探求心}}}$ を ${\overset{\textnormal{も}}{\text{持}}}$ って ${\overset{\textnormal{じんせい}}{\text{人生}}}$ を ${\overset{\textnormal{い}}{\text{生}}}$ きれば、 ${\overset{\textnormal{せかい}}{\text{世界}}}$ があなたを ${\overset{\textnormal{たの}}{\text{楽}}}$ しませてくれるでしょう。 \hfill\break
If you live your life with the heart of exploration like that of a child, the world will surely delight you. }
      
\section{触る \& 触れる}
 
\par{ The verbs 触る and 触れる both mean “to touch, “and they can both incidentally be used with either に or を, but that doesn\textquotesingle t mean these four combinations are freely interchangeable. }

\par{1. に触る: \emph{ }触る specifically refers to purposely touching\slash feeling something, usually using one\textquotesingle s hand(s). In a non-literal sense, it can also mean “to be involved with.” Additionally, when spelled as 障る, it refers to hurting someone\textquotesingle s feelings. In these senses, the typical particle used is に. }

\par{28. ${\overset{\textnormal{くら}}{\text{暗}}}$ い ${\overset{\textnormal{なか}}{\text{中}}}$ で ${\overset{\textnormal{て}}{\text{手}}}$ が ${\overset{\textnormal{なに}}{\text{何}}}$ かに ${\overset{\textnormal{さわ}}{\text{触}}}$ った。 \hfill\break
Something touched my hand in the dark. }

\par{29. コタツの ${\overset{\textnormal{なか}}{\text{中}}}$ で ${\overset{\textnormal{だれ}}{\text{誰}}}$ かの ${\overset{\textnormal{あし}}{\text{足}}}$ が ${\overset{\textnormal{わたし}}{\text{私}}}$ の ${\overset{\textnormal{あし}}{\text{足}}}$ に ${\overset{\textnormal{さわ}}{\text{触}}}$ った。 \hfill\break
Someone\textquotesingle s leg touched my leg inside the kotatsu. }

\par{\textbf{Spelling Note }: コタツ may also be spelled as 炬燵 or 火燵. }

\par{30. ${\overset{\textnormal{じゅぎょうちゅう}}{\text{授業中}}}$ に ${\overset{\textnormal{つくえ}}{\text{机}}}$ の ${\overset{\textnormal{した}}{\text{下}}}$ で ${\overset{\textnormal{あし}}{\text{足}}}$ を ${\overset{\textnormal{まえ}}{\text{前}}}$ に ${\overset{\textnormal{の}}{\text{伸}}}$ ばして ${\overset{\textnormal{すわ}}{\text{座}}}$ っていたら、 ${\overset{\textnormal{なに}}{\text{何}}}$ か ${\overset{\textnormal{つめ}}{\text{冷}}}$ たい ${\overset{\textnormal{もの}}{\text{物}}}$ が ${\overset{\textnormal{あし}}{\text{足}}}$ に ${\overset{\textnormal{さわ}}{\text{触}}}$ った。 \hfill\break
As I was sitting with my legs extended in front of me under the desk during class, something cold touched my legs. }

\par{31. ${\overset{\textnormal{しんけい}}{\text{神経}}}$ に ${\overset{\textnormal{さわ}}{\text{障}}}$ ることを ${\overset{\textnormal{い}}{\text{言}}}$ う ${\overset{\textnormal{あいて}}{\text{相手}}}$ にはどう ${\overset{\textnormal{たいおう}}{\text{対応}}}$ したらいいのか。 \hfill\break
How should you handle people who say things that get on your nerves? }

\par{32. その ${\overset{\textnormal{ぶん}}{\text{文}}}$ の ${\overset{\textnormal{なに}}{\text{何}}}$ があなたの ${\overset{\textnormal{しゃく}}{\text{癪}}}$ に ${\overset{\textnormal{さわ}}{\text{障}}}$ ったの? \hfill\break
What about that sentence offended you? }

\par{33. ${\overset{\textnormal{てんじぶつ}}{\text{展示物}}}$ には ${\overset{\textnormal{さわ}}{\text{触}}}$ らないでください。 \hfill\break
Please do not touch the display. }

\par{34. ${\overset{\textnormal{あや}}{\text{怪}}}$ しいものには ${\overset{\textnormal{さわ}}{\text{触}}}$ らないほうがいいよ。 \hfill\break
It\textquotesingle s best not to touch\slash get involved with suspicious things. }

\par{\textbf{Sentence Note }: If this were spoken out-loud, one could interpret this as referring to people because もの can also be 者. }

\par{35. ${\overset{\textnormal{べい}}{\text{米}}}$ ドルには ${\overset{\textnormal{さわ}}{\text{触}}}$ らないほうがいいかな。 \hfill\break
It\textquotesingle s maybe best not to touch the American dollar. }

\par{2. を触る: The use of を is meant to emphasize the purposeful touching\slash feeling of something. It is frequently used when expressing touching in which the thing\slash person touched is not a fan of being touched. }

\par{36. ${\overset{\textnormal{かれし}}{\text{彼氏}}}$ が ${\overset{\textnormal{むね}}{\text{胸}}}$ を ${\overset{\textnormal{さわ}}{\text{触}}}$ ってきてとても ${\overset{\textnormal{いや}}{\text{嫌}}}$ です。 \hfill\break
I really hate it when my boyfriend comes and touches my chest. }

\par{37. ${\overset{\textnormal{とうがらし}}{\text{唐辛子}}}$ に ${\overset{\textnormal{さわ}}{\text{触}}}$ った ${\overset{\textnormal{て}}{\text{手}}}$ で ${\overset{\textnormal{はな}}{\text{鼻}}}$ を ${\overset{\textnormal{さわ}}{\text{触}}}$ ってしまいました。 \hfill\break
I accidentally touched my noise with the hand that had touched cayenne pepper. }

\par{38. ${\overset{\textnormal{だんせい}}{\text{男性}}}$ が ${\overset{\textnormal{じょせい}}{\text{女性}}}$ の ${\overset{\textnormal{はな}}{\text{鼻}}}$ を ${\overset{\textnormal{さわ}}{\text{触}}}$ りたがるのはなぜでしょうか。 \hfill\break
Why is that men want to touch women\textquotesingle s noses? }

\par{39. この ${\overset{\textnormal{けん}}{\text{剣}}}$ の ${\overset{\textnormal{はさき}}{\text{刃先}}}$ を ${\overset{\textnormal{さわ}}{\text{触}}}$ ってみて。 \hfill\break
Try feeling the edge of this sword. }

\par{40. ウサギを ${\overset{\textnormal{さわ}}{\text{触}}}$ ろうとしたら、 ${\overset{\textnormal{さわ}}{\text{触}}}$ らせてくれなかった。 \hfill\break
When I tried to touch the bunny, it wouldn\textquotesingle t let me touch it. }

\par{\textbf{Spelling Note }: ウサギ may alternatively be spelled as 兎. }

\par{3. に触れる: In a literal sense, 触れる means “to lightly touch.” In a less literal sense, it encompasses other sense of “touch” such as “to touch on.” You may see it in plenty of set phrases like 目に触れる  (to cross one\textquotesingle s eyes). Another important application is when it means “to violate” as in a law or regulation of some sort. }

\par{41. ${\overset{\textnormal{はだ}}{\text{肌}}}$ が ${\overset{\textnormal{さんご}}{\text{珊瑚}}}$ に ${\overset{\textnormal{ふ}}{\text{触}}}$ れたら ${\overset{\textnormal{かゆ}}{\text{痒}}}$ みとかぶれで ${\overset{\textnormal{ねむ}}{\text{眠}}}$ れなくなった。 \hfill\break
When my skin touched the coral reef, I ended up not being able to sleep because of itchiness and a rash. }

\par{\textbf{Spelling Note }: かぶれ may also be spelled as 気触れ. }

\par{42. それ ${\overset{\textnormal{いご}}{\text{以後}}}$ も ${\overset{\textnormal{おり}}{\text{折}}}$ に ${\overset{\textnormal{ふ}}{\text{触}}}$ れて ${\overset{\textnormal{さん}}{\text{3}}}$ ${\overset{\textnormal{にん}}{\text{人}}}$ で ${\overset{\textnormal{かつどう}}{\text{活動}}}$ している。 \hfill\break
Even since then, the three of us campaign\slash do activities occasionally. }

\par{\textbf{Phrase Note }: 折に触れて is a rather literary set phrase meaning “occasionally.” }

\par{43. ${\overset{\textnormal{けんきゅうしつ}}{\text{研究室}}}$ で ${\overset{\textnormal{はたら}}{\text{働}}}$ いている ${\overset{\textnormal{かぎ}}{\text{限}}}$ りは ${\overset{\textnormal{りんしょう}}{\text{臨床}}}$ の ${\overset{\textnormal{もんだい}}{\text{問題}}}$ に ${\overset{\textnormal{ふ}}{\text{触}}}$ れる ${\overset{\textnormal{きかい}}{\text{機会}}}$ はほとんどありません。 \hfill\break
So long as you are working in the research lab, there are hardly any opportunities to personally touch on clinical issues. }

\par{44. 「 ${\overset{\textnormal{じぶん}}{\text{自分}}}$ は ${\overset{\textnormal{かみ}}{\text{神}}}$ より ${\overset{\textnormal{すぐ}}{\text{優}}}$ れている」と ${\overset{\textnormal{くち}}{\text{口}}}$ に ${\overset{\textnormal{だ}}{\text{出}}}$ せば、 ${\overset{\textnormal{かみ}}{\text{神}}}$ の ${\overset{\textnormal{いか}}{\text{怒}}}$ りに ${\overset{\textnormal{ふ}}{\text{触}}}$ れる ${\overset{\textnormal{こと}}{\text{事}}}$ になる。 \hfill\break
If you verbalize that oneself is superior to gods, you will offend the gods. }

\par{45. ${\overset{\textnormal{わいろぼうしじょうれい}}{\text{賄賂防止条例}}}$ に ${\overset{\textnormal{ふ}}{\text{触}}}$ れる ${\overset{\textnormal{かのうせい}}{\text{可能性}}}$ がある。 \hfill\break
There is the possibility that it violates the bribe prevention ordinance. }

\par{41. ${\overset{\textnormal{ほうりつ}}{\text{法律}}}$ に ${\overset{\textnormal{ふ}}{\text{触}}}$ れる ${\overset{\textnormal{こうい}}{\text{行為}}}$ をしても、 ${\overset{\textnormal{はんざいしゃ}}{\text{犯罪者}}}$ にならない ${\overset{\textnormal{ほうりつ}}{\text{法律}}}$ は ${\overset{\textnormal{おどろ}}{\text{驚}}}$ くことにわんさかあります。 \hfill\break
Surprisingly, there are a lot of laws that you won\textquotesingle t become a criminal for even if you do something that violates the law. }

\par{4. を触れる: The basic pattern of using 触れる can be viewed as being “object + に + body part +が・を・で.” The use of が implies zero volition in a body part breezing against something. As for Ex. 42, you might wonder why で isn\textquotesingle t used. If で were used, then the listener could just interpret the request as simply banning touching with one\textquotesingle s hands but touching with one\textquotesingle s feet is fair game. By using を, this is not the first that would come to mind. It simply recognizes a willful act of lightly touching something and that act of touching happens to be carried out by one\textquotesingle s hands. }

\par{42. ${\overset{\textnormal{て}}{\text{手}}}$ を ${\overset{\textnormal{ふ}}{\text{触}}}$ れないでください。 \hfill\break
Please don\textquotesingle t touch with your hands. }

\par{43. ${\overset{\textnormal{ろじょう}}{\text{路上}}}$ にあったアライグマの ${\overset{\textnormal{しがい}}{\text{死骸}}}$ を ${\overset{\textnormal{ぼう}}{\text{棒}}}$ で ${\overset{\textnormal{さわ}}{\text{触}}}$ ってみた。 \hfill\break
I tried touching the carcass of a raccoon that was on the side of the road with a stick. }

\par{44. はやり目は、その名の通り、非常に感染力が強く、感染者が手を目に触れて、テーブルに手を置いて、そこが乾燥し切る前に別の人が手を触れて、その人が手を目に触れたら、感染してしまう可能性もあります。 \hfill\break
Pink eye, literally 'endemic eye' (in Japanese, has an extremely great infectious capacity; by an infect person touching his\slash her hand on the eyes and then placing said hand on a table, if another person touches the table by the hand before it can fully dry out and then touches his\slash her eye, there is even the potential of that person getting infected. }

\par{5: が触れる: In the phrase 脈が触れる, the verb \emph{ }触れる is used to mean “to feel a pulse.” In this context, there is no に or を used with it. }

\par{45. ${\overset{\textnormal{てくびいがい}}{\text{手首以外}}}$ で ${\overset{\textnormal{みゃく}}{\text{脈}}}$ が ${\overset{\textnormal{ふ}}{\text{触}}}$ れるのはこめかみ、 ${\overset{\textnormal{のどほとけ}}{\text{喉仏}}}$ の ${\overset{\textnormal{りょうわき}}{\text{両脇}}}$ 、 ${\overset{\textnormal{ひじ}}{\text{肘}}}$ の ${\overset{\textnormal{うちがわ}}{\text{内側}}}$ 、 ${\overset{\textnormal{だいたい}}{\text{大腿}}}$ の ${\overset{\textnormal{つ}}{\text{付}}}$ け ${\overset{\textnormal{ね}}{\text{根}}}$ 、 ${\overset{\textnormal{ひざ}}{\text{膝}}}$ の ${\overset{\textnormal{うら}}{\text{裏}}}$ 、 ${\overset{\textnormal{あし}}{\text{足}}}$ の ${\overset{\textnormal{こう}}{\text{甲}}}$ などが ${\overset{\textnormal{だいひょうてき}}{\text{代表的}}}$ な部分です。 \hfill\break
Aside from the wrists, representative places where you can feel your pulse include the temple, both sides of one\textquotesingle s Adam\textquotesingle s apple, the root of one\textquotesingle s thigh, the back of the knee, and the top of the feet. }

\par{\textbf{Spelling Note }: こめかみ \emph{ }may seldom be alternatively spelled as 顳顬. }
    