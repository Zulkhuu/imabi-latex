    
\chapter{Advanced -べき Phrases}

\begin{center}
\begin{Large}
第295課: Advanced -べき Phrases: ~べくもない, ~べからず・ざる, ~べく, ~べくして, \& ~べし 
\end{Large}
\end{center}
 
\par{ ~べきだ is actually an evolved form of the auxiliary ~べし. With that said, we will first look at the bases which were neglected the first time to make the different forms of it found in more advanced\slash rarer usages easier to make and use. }
      
\section{The Bases of ~べし}
 
\par{ The modern form of ~べし is ~べきだ. Now it is important to see what the bases are for this auxiliary because rarer\slash archaic usages employed today use them. }

\begin{ltabulary}{|P|P|P|P|P|P|}
\hline 

未然形 & 連用形 & 終止形 & 連体形 & 已然形 & 命令形 \\ \cline{1-6}

べから- & べく-・べかり- & べし & べき・べかる & べけれ- & 存在しない \\ \cline{1-6}

\end{ltabulary}

\par{\textbf{Conjugation Note }: It's usually only used with verbs, but when not it shows a strong sense of "should". -べし should follow the かる-連体形 of 形容詞 and the copula as なる for 形容動詞 and nouns. As for ~べきだ, it should follow 形容詞 like in あたらしくあるべきだ and after the copula である for 形容動詞 and nouns. }

\par{\textbf{漢字 Note }: The auxiliary can rarely be seen written as 可し. }
      
\section{~べくもない}
 
\par{ ~べくもない is used to show that there is absolutely no possibility in something you hope could happen. You just don't have the means even if you try. This is interchangeable with できるわけがない and できるはずがない. }

\par{1. 将来、東京大学の教授になろうなんて望むべくもない。 \hfill\break
There's no way that I could ever become something like a Tokyo University professor in the future. }
      
\section{~べからず・ざる}
 
\par{ ~べからず, the negative form of -べし is primarily used today to make negative imperatives. When used in ~ざるべからず with ~ざる being the 連体形 of the classical negative auxiliary ~ず, it means "must" in the same way ~なければならない is used today but with more of a command sense to it. At other times, it might be used to show incapability of something, especially with verbs like 許す (to forgive). In this usage it is usually seen in the べからざる-連体形. ~べからず can also be used like ~べくもない, but this would have to be quite archaic for it to have any importance. }

\par{2. 予測すべからざる事態。 \hfill\break
A situation that one cannot predict. }

\par{3. ${\overset{\textnormal{しば}}{\text{芝}}}$ に入るべからず。 \hfill\break
Keep off the grass. }

\par{4. ${\overset{\textnormal{はね}}{\text{羽}}}$ なければ空をも飛ぶべからず。(ちょっと古風) \hfill\break
Since you have no wings, you can't even fly in the sky. }
 
\par{${\overset{\textnormal{}}{\text{5. 天皇}}}$ を ${\overset{\textnormal{おも}}{\text{重}}}$ んぜざるべからず。(Archaic) \hfill\break
One must honor the emperor. }
 
\par{6. 「 ${\overset{\textnormal{さわ}}{\text{触}}}$ るべからず!」 (Old-fashioned) \hfill\break
"Do not touch!" }

\par{${\overset{\textnormal{てんのう}}{\text{天皇}}}$ を ${\overset{\textnormal{おも}}{\text{重}}}$ んぜざるべからず。(Archaic) \hfill\break
One must honor the emperor. }

\par{Note: The above pattern ~ざるべからず is equivalent to ~なければならない. This is an archaic pattern, and you would only see it in things like proverbs or sayings today. }

\par{「 ${\overset{\textnormal{さわ}}{\text{触}}}$ るべからず!」 (Old-fashioned) \hfill\break
"Do not touch!" }
      
\section{~べく \& ~べくして}
 
\par{ The べく-連用形 may be used similarly to ために to mean "in order to", being quite formal. It is placed after the 連体形. It may be translated as "in trying to". This is far more formal, and it is seen all the time in informative books, but you hardly ever hear it in spoken Japanese. }

\par{7. 早く帰るべく、 ${\overset{\textnormal{じゅんび}}{\text{準備}}}$ をし始めた。 \hfill\break
In trying to go home early, I started making preparations. }
 
\par{8. 我々は、将来を担うべく努力することを誓います。 \hfill\break
We vow to endeavor to bear the future. }
\textbf{\textbf{~ }\textbf{可くして }} 
\begin{itemize}
 
\item "It's only natural      as". The same verb must be repeated for this usage!  
\item Followed by an      antithetical statement, it means "even if it's possible to".  
\end{itemize}
 
\par{9. これは勝つべくして勝った試合です。 \hfill\break
This is only natural that it\textquotesingle s a match that we were going to win. }
 
\par{10. するべくして、するべからず。(Old-fashioned) \hfill\break
Even though you can do it, you should not do it. }
 
\par{11. その教科書を読むべくして理解しがたい。(Old-fashioned) \hfill\break
Even if you can read the textbook, it is hard to understand. }
 
\par{12. この ${\overset{\textnormal{せつ}}{\text{説}}}$ は言うべくして行うべからず。(Old-fashioned) \hfill\break
Even if you can say the theory, you shouldn't carry it out. }
 
\par{13. 負けるべくして負けた。 \hfill\break
It's only natural that we lost. }

\par{\textbf{- }\textbf{可くして }}

\begin{itemize}

\item "It's only natural as". The same verb must be repeated for this usage! 
\item Followed by an antithetical statement, it means "even if it's possible to". 
\end{itemize}

\par{これは ${\overset{\textnormal{}}{\text{勝}}}$ つべくして ${\overset{\textnormal{}}{\text{勝}}}$ った試合です。 \hfill\break
This is only natural that it\textquotesingle s a match that we were going to win. }

\par{するべくして、するべからず。(Old-fashioned) \hfill\break
Even though you can do it, you should not do it. }

\par{その ${\overset{\textnormal{}}{\text{教科書}}}$ を ${\overset{\textnormal{}}{\text{読}}}$ むべくして ${\overset{\textnormal{}}{\text{理解}}}$ しがたい。(Old-fashioned) \hfill\break
Even if you can read the textbook, it is hard to understand. }

\par{この ${\overset{\textnormal{せつ}}{\text{説}}}$ は ${\overset{\textnormal{}}{\text{言}}}$ うべくして ${\overset{\textnormal{}}{\text{行}}}$ うべからず。(Old-fashioned) \hfill\break
Even if you can say the theory, you shouldn't carry it out. }

\par{${\overset{\textnormal{}}{\text{負}}}$ けるべくして ${\overset{\textnormal{}}{\text{負}}}$ けた。 \hfill\break
It's only natural that we lost. }
      
\section{~べし: Command}
 
\par{ This is rarely seen, but it is of the same mindset as the other usages. It's forceful, and it has the same effect for the affirmative as ~べからず does for the negative. }
 
\par{14. 辞職すべし。 \hfill\break
Resign! }
    