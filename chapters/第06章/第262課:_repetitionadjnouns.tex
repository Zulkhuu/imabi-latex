    
\chapter{Reduplication}

\begin{center}
\begin{Large}
第262課: Reduplication: Adjectival Nouns 
\end{Large}
\end{center}
 
\par{ Reduplication in adjectival nouns is somewhat more complicated than for nouns or adjectives. The reason for this is that as we\textquotesingle ve learned before, whether to use \emph{na }な or \emph{no }の is a bit all over the place, and in addition to that, many of the phrases shown here may also have alternate uses as adverbs. }

\par{ Just as before, some examples of reduplication in adjectival nouns derive from nouns, adjectives, adjectival nouns, verbs, and even onomatopoeic expressions. However, you will be disappointed that the number of reduplicated adjectival nouns is dwindling in Modern Japanese. }

\par{・From Nouns: \emph{Kazukazu [no] }数々(の)- Numerous \hfill\break
・From Adjectives: \emph{Atsuatsu [no] }熱々(の) - Piping hot \hfill\break
・From Adjectival Nouns: \emph{Boroboro [na\slash no] }ぼろぼろ(な・の) - Run-down \hfill\break
・ From Verbs: \emph{Tobitobi [na] }飛び飛び(な) - Scattered here and there \hfill\break
・From Onomatopoeia: \emph{Pikapika [no\slash na] }ぴかぴか(の・な)- Sparkling }

\par{ Grammatical restraints are just as all over the place as the etymologies of these instances of reduplication. Most instances can be used as adverbs by using \emph{ni }に, which is no different than the majority of adjectival nouns. However, some may only be used to the extent of \emph{ni naru }になる, and so playing with each one to see how much it can be used in is necessary. }
      
\section{Examples}
 
\begin{center}
\textbf{From Nouns }
\end{center}

\par{1. ${\overset{\textnormal{しゃかい}}{\text{社会}}}$ に ${\overset{\textnormal{で}}{\text{出}}}$ ると ${\overset{\textnormal{かずかず}}{\text{数々}}}$ \textbf{の }${\overset{\textnormal{もんだい}}{\text{問題}}}$ に ${\overset{\textnormal{ちょくめん}}{\text{直面}}}$ する。 \hfill\break
 \emph{Shakai ni deru to \textbf{kazukazu no }\textbf{ }mondai ni chokumen suru. }\hfill\break
When you enter society, you encounter numerous problems. }

\par{2. ${\overset{\textnormal{しゅじゅ}}{\text{種々}}}$ \textbf{の }${\overset{\textnormal{たいさく}}{\text{対策}}}$ を ${\overset{\textnormal{こう}}{\text{講}}}$ じる。 \hfill\break
 \textbf{\emph{Shuju no }}\emph{ }\emph{taisaku wo kōjiru. }\hfill\break
To take all sorts of measures. }

\par{3. ${\overset{\textnormal{いろいろ}}{\text{色々}}}$ \textbf{な }${\overset{\textnormal{ばしょ}}{\text{場所}}}$ にポケモンが ${\overset{\textnormal{せいそく}}{\text{生息}}}$ しています。 \hfill\break
 \textbf{ \emph{Iroiro na }}\emph{\textbf{ }basho ni pokemon ga seisoku shite imasu. }\hfill\break
Pokemon inhabit various \textbf{ }places. }

\par{4. お ${\overset{\textnormal{にく}}{\text{肉}}}$ は ${\overset{\textnormal{ほどほど}}{\text{程々}}}$ \textbf{の }${\overset{\textnormal{あじ}}{\text{味}}}$ で ${\overset{\textnormal{ねだん}}{\text{値段}}}$ も ${\overset{\textnormal{わりだか}}{\text{割高}}}$ です。 \hfill\break
 \emph{Oniku wa \textbf{hodohodo no }aji de nedam mo waridaka desu. }\hfill\break
As for the meat, the price is also fairly expensive with a moderate \textbf{ }taste. }

\par{5. ${\overset{\textnormal{かれ}}{\text{彼}}}$ は ${\overset{\textnormal{もともと}}{\text{元々}}}$ 、 ${\overset{\textnormal{けっこんがんぼう}}{\text{結婚願望}}}$ がなかった。 \hfill\break
 \emph{Kare wa \textbf{motomoto }, kekkon gambō ga nakatta. }\hfill\break
He had no desire to marry from the start \textbf{. }}

\par{\textbf{Usage Note }: \emph{Motomoto }元々 may be used as an adverb without the need of any particle. }

\par{6. ${\overset{\textnormal{もともと}}{\text{元々}}}$ \textbf{の }${\overset{\textnormal{いみ}}{\text{意味}}}$ はご ${\overset{\textnormal{ぞん}}{\text{存}}}$ じですか。 \hfill\break
 \textbf{ \emph{Motomoto no }}\emph{\textbf{ }imi wa gozonji desu ka? }\hfill\break
Do you know what the original \textbf{ }meaning is? }

\par{7. ${\overset{\textnormal{きたく}}{\text{帰宅}}}$ が ${\overset{\textnormal{おそ}}{\text{遅}}}$ くなったりしますが、 ${\overset{\textnormal{えいぎょうしょく}}{\text{営業職}}}$ としては ${\overset{\textnormal{なみなみ}}{\text{並々}}}$ \textbf{の }${\overset{\textnormal{ざんぎょう}}{\text{残業}}}$ かと ${\overset{\textnormal{おも}}{\text{思}}}$ います。 \hfill\break
 \emph{Kitaku ga osoku nattari shimasu ga, eigyōshoku to shite wa }\emph{naminami }\emph{no zangyō ka to omoimasu. }\hfill\break
Coming home is sometimes late or what not, but I think that\textquotesingle s \textbf{ordinary }overtime for a business job. }

\begin{center}
\textbf{From Adjectives }
\end{center}

\par{8. ピリ ${\overset{\textnormal{から}}{\text{辛}}}$ で ${\overset{\textnormal{あつあつ}}{\text{熱々}}}$ \textbf{の }キムチをご ${\overset{\textnormal{はん}}{\text{飯}}}$ と ${\overset{\textnormal{いっしょ}}{\text{一緒}}}$ に ${\overset{\textnormal{た}}{\text{食}}}$ べました。 \hfill\break
 \emph{Pirikara de \textbf{atsuatsu no }\textbf{ }kimuchi wo gohan to issho ni tabemashita. }\hfill\break
I ate my meal\slash cooked rice with spicy and \textbf{piping hot }kimchi. }

\par{\textbf{9. }${\overset{\textnormal{ひさびさ}}{\text{久々}}}$ \textbf{に }カレーを ${\overset{\textnormal{つく}}{\text{作}}}$ った。 \hfill\break
 \textbf{\emph{Hisabisa ni }}\emph{ }\emph{karē wo tsukutta. }\hfill\break
\textbf{ }I made curry \textbf{after a long while }\textbf{. }}

\begin{center}
\textbf{From Adjectival Nouns }
\end{center}

\par{10. ${\overset{\textnormal{おっと}}{\text{夫}}}$ は ${\overset{\textnormal{けっこん}}{\text{結婚}}}$ するまで \textbf{ぼろぼろの }${\overset{\textnormal{ふく}}{\text{服}}}$ を ${\overset{\textnormal{き}}{\text{着}}}$ ていました。 \hfill\break
 \emph{Otto wa kekkon suru made \textbf{boroboro no }fuku wo kite imashita. }\hfill\break
My husband worse \textbf{ragged }clothes up until we got married. }

\begin{center}
\textbf{From Verbs }
\end{center}

\par{11. ${\overset{\textnormal{りょうたくん}}{\text{亮太君}}}$ は ${\overset{\textnormal{かのじょ}}{\text{彼女}}}$ に ${\overset{\textnormal{いき}}{\text{息}}}$ が ${\overset{\textnormal{た}}{\text{絶}}}$ \textbf{え }${\overset{\textnormal{だ}}{\text{絶}}}$ \textbf{えになる }ほど ${\overset{\textnormal{あつ}}{\text{熱}}}$ いキスをした。 \hfill\break
 \emph{Ryōta-kun wa kanojo ni iki ga \textbf{taedae ni naru }\textbf{ }hodo atsui kisu wo shita. }\hfill\break
Ryota gave her such a warm kiss so strong it left her \textbf{gasping }\textbf{. }}

\par{12. ${\overset{\textnormal{いろ}}{\text{色}}}$ \textbf{とりどりに }${\overset{\textnormal{はな}}{\text{花}}}$ が ${\overset{\textnormal{さ}}{\text{咲}}}$ いています。 \hfill\break
 \textbf{\emph{Iro toridori ni }}\emph{ }\emph{hana ga saite imasu. }\hfill\break
 \textbf{Multi-colored }flowers are blooming. }

\par{13. ${\overset{\textnormal{くうはく}}{\text{空白}}}$ が ${\overset{\textnormal{と}}{\text{飛}}}$ び ${\overset{\textnormal{と}}{\text{飛}}}$ \textbf{びになっている }。 \hfill\break
 \emph{ K }\emph{ū }\emph{haku ga \textbf{tobitobi ni natte iru }. }\hfill\break
 Empty spaces are \textbf{scattered everywhere }\textbf{. }}

\begin{center}
\textbf{From Four Character Idioms }
\end{center}

\par{14. ${\overset{\textnormal{ききかいかい}}{\text{奇々怪々}}}$ な ${\overset{\textnormal{じけん}}{\text{事件}}}$ を ${\overset{\textnormal{あざ}}{\text{鮮}}}$ やかに ${\overset{\textnormal{と}}{\text{解}}}$ き ${\overset{\textnormal{あ}}{\text{明}}}$ かしていく。 \hfill\break
 \textbf{\emph{Kikikaikai na }}\emph{ }\emph{jiken wo azayaka ni tokiakashite iku. } \hfill\break
I will vividly dispel \textbf{bizarre }cases. }

\par{\textbf{Word Note }: 奇々怪々 is rare and would usually be replaced with the simpler \emph{kikai }奇怪 from which it derives. }

\begin{center}
\textbf{From Onomatopoeia }
\end{center}

\par{15. \textbf{ }\textbf{ぴかぴかの }${\overset{\textnormal{は}}{\text{歯}}}$ を ${\overset{\textnormal{と}}{\text{取}}}$ り ${\overset{\textnormal{もど}}{\text{戻}}}$ しましょう。 \hfill\break
 \textbf{Pikapika no }ha wo torimodoshimashō. \hfill\break
Restore one\textquotesingle s \textbf{sparkling }teeth! }

\begin{center}
\textbf{From Prefixes }
\end{center}

\par{16. ${\overset{\textnormal{しゅうへんしょこく}}{\text{周辺諸国}}}$ にもその ${\overset{\textnormal{えいきょう}}{\text{影響}}}$ が ${\overset{\textnormal{でんぱ}}{\text{伝播}}}$ し、 \textbf{${\overset{\textnormal{もろもろ}}{\text{諸々}}}$ の }${\overset{\textnormal{じじょう}}{\text{事情}}}$ によって ${\overset{\textnormal{ほうかい}}{\text{崩壊}}}$ していったとのことでした。 \hfill\break
 \emph{Shūhen shokoku ni mo sono eikyō ga dempa shi, \textbf{moromoro no }jijō ni yotte hōkai shite itta to no koto deshita. }\hfill\break
That effect propagated to all the neighboring countries as well, which led to them collapsing under \textbf{all kinds of }circumstances. }

\par{17. ${\overset{\textnormal{もろもろ}}{\text{諸々}}}$ 、 ${\overset{\textnormal{りょうかい}}{\text{了解}}}$ しました。 \hfill\break
 \textbf{\emph{Moromoro }}\emph{, ryōkai shimashita. }\hfill\break
Rodger that \textbf{to everything }\textbf{. }}

\par{\textbf{Usage Note }: \emph{Moromoro }諸々 may be used as an adverb without the need of any particle. }
    