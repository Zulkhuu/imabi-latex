    
\chapter{Honorifics}

\begin{center}
\begin{Large}
第265課: Honorifics: [ni\slash to] narimasu\slash natte orimasu\{に・と\}なります\slash なっております 
\end{Large}
\end{center}
 
\par{ Honorifics is a very intricate and complicated system. Although it is without a doubt that a native Japanese speaker naturally has a decent control of honorifics, nearly half of all speakers complain of lacking confidence in using honorifics. The other half complains about how others ‘incorrectly\textquotesingle  use honorifics. Several patterns warrant special attention because they are new and or questionable. It is important to be things that are frequently used, even if those things are unorthodox in anyway. }

\par{ Having said this, in this lesson we will study the intriguing phrase \emph{ni narimasu }になります・ \emph{[ni\slash to] natte orimasu }に・となっております. }
      
\section{What is ni narimasu になります?}
 
\par{ We should know by now that the verb \emph{naru }成る is used in a wide variety of situations. }

\par{1. ${\overset{\textnormal{こおり}}{\text{氷}}}$ が ${\overset{\textnormal{みず}}{\text{水}}}$ になります。 \hfill\break
 \emph{K }\emph{ōri ga mizu ni narimasu. }\hfill\break
Ice \textbf{turns }into water. }

\par{2. ${\overset{\textnormal{かんぽうやく}}{\text{漢方薬}}}$ は ${\overset{\textnormal{からだ}}{\text{体}}}$ のためになります。 \hfill\break
 \emph{Kamp }\emph{ōyaku wa karada no tame ni narimasu. }\hfill\break
Chinese herbal medicine \textbf{benefits }the body. }

\par{3. ${\overset{\textnormal{けっこん}}{\text{結婚}}}$ して ${\overset{\textnormal{じゅう}}{\text{10}}}$ ${\overset{\textnormal{ねん}}{\text{年}}}$ になります。 \hfill\break
 \emph{Kekkon shite j }\emph{ūnen ni narimasu. }\hfill\break
It \textbf{will be }ten years since I got married. }

\par{4. お ${\overset{\textnormal{せわ}}{\text{世話}}}$ になります。 \hfill\break
 \emph{Osewa ni narimasu. \hfill\break
 }I \textbf{will be }in your aid. \hfill\break
 \hfill\break
\textbf{Phrase Note }: This is a set phrase used in formal situations when one is corresponding and or addressing someone to whom will be in your aid in some endeavor. }

\par{5. エオリア ${\overset{\textnormal{しょとう}}{\text{諸島}}}$ は ${\overset{\textnormal{なな}}{\text{7}}}$ つの ${\overset{\textnormal{かざんとう}}{\text{火山島}}}$ からなります。 \hfill\break
 \emph{Eoria shot }\emph{ō wa nanatsu no kazant }\emph{ō kara narimasu. } \hfill\break
The Aeolian Islands are made up of seven volcanic islands. }

\par{6. お ${\overset{\textnormal{きゃくさま}}{\text{客様}}}$ がお ${\overset{\textnormal{つ}}{\text{着}}}$ きになりました。 \hfill\break
 \emph{Okyaku-sama ga otsuki ni narimashita. }\hfill\break
The customer has arrived. }

\par{\textbf{Grammar Note }: \emph{O- }お+Verb Stem + \emph{ni naru }になる, as we have learned, is one of the fundamental means of making most verbs honorific. }

\par{7. ${\overset{\textnormal{ぜんぶ}}{\text{全部}}}$ で ${\overset{\textnormal{じゅう}}{\text{10}}}$ ${\overset{\textnormal{まんえん}}{\text{万円}}}$ になります。 \hfill\break
 \emph{Zembu de j }\emph{ūman\textquotesingle en ni narimasu. }\hfill\break
It will reach 100,000 yen in total. }

\par{ The usages thus far are well-established in Japanese, but we have yet to explore its unorthodox usage that is now used extensively in customer service honorific speech. }

\begin{center}
 \textbf{'Convenience Store' Honorifics: \emph{ni narimasu }になります }
\end{center}

\par{ So what is it about \emph{ni narimasu になります }that is problematic? The issue at hand is that \emph{ni narimasu }になります is a hallmark phrase of what is called \emph{baito keigo }バイト敬語. Although the name suggests that it is only used at part time jobs, \emph{baito keigo }is actually used by people in all walks of professions, especially if the job has anything to do with customer service. }

\par{ The issue at hand here is when \emph{ni narimasu }になります is used in the sense of \emph{desu }です or \emph{de gozaimasu }でございます. }

\par{8a. こちらが ${\overset{\textnormal{りょうしゅうしょ}}{\text{領収書}}}$ になります。△ \hfill\break
 \emph{Kochira ga ryōshūsho ni narimasu. }}

\par{8b. こちらが ${\overset{\textnormal{りょうしゅうしょ}}{\text{領収書}}}$ \{です・でございます\}。○ \hfill\break
 \emph{Kochira ga ryōshūsho [desu\slash de gozaimasu]. }}

\par{ Both sentences mean, “This is the receipt.” However, because \emph{naru }なる\textquotesingle s primary meaning expresses change of state, many speakers find it ungrammatical for means of simply avoiding a declarative statement. Nothing is becoming the receipt. The receipt is the receipt. }

\par{ There are important counterarguments to consider. Pretend that you are a waiter. Your customer orders iced coffee. You go to the back and fetch what your customer ordered, and then you return to the customer\textquotesingle s table and present it. In a psychological sense, one could say that the cup of iced coffee you have in your hand is simply a cup with a cold, dark liquid in it from the customer\textquotesingle s viewpoint. }

\par{9. こちら、ご ${\overset{\textnormal{ちゅうもん}}{\text{注文}}}$ のアイスコーヒーになります。 \hfill\break
 \emph{Kochira, goch }\emph{ūmon no aisuk }\emph{ōhii ni narimasu. \hfill\break
 }This is the iced coffee you ordered. \emph{}}

\par{ Instead of outright stating that it IS iced coffee, you present the iced coffee as something that has potentially gone from being an unknown entity to being the iced coffee the customer had asked for. In this sense, \emph{ni narimasu }になります becomes a conduit for presenting things without it sounding like you\textquotesingle re stating the obvious. Additionally, this serves as a means of avoiding responsibility if you messed up somehow. }

\par{10. お ${\overset{\textnormal{かいけい}}{\text{会計}}}$ 、 ${\overset{\textnormal{ろくせん}}{\text{6000}}}$ ${\overset{\textnormal{えん}}{\text{円}}}$ \{になります △・です ○・でございます ○\}。 \hfill\break
 \emph{Okaikei, rokusen\textquotesingle en [ni narimasu\slash desu\slash de gozaimasu]. \hfill\break
 }Your total is 6000 yen. }

\par{\textbf{Example Note }: The total of the bill is not inherently the worker\textquotesingle s fault. If the consumer is spending that much at said establishment, then it is the consumer\textquotesingle s responsibility to pay for the goods and or services the stated value. Therefore, it is strange at best why an employee would wish to shift blame away from himself for the cost of the consumer\textquotesingle s bill. However, there is a principle in customer service that the customer is always right. }

\par{ The pursuit of addressing all customers with the utmost respect and care also places pressure on the worker to also cover for his and the establishment\textquotesingle s own misgivings if any exist. For all one knows, the 6000-yen purchase may actually only be 5600 because a discount wasn\textquotesingle t properly taken into account. Since the possibilities of what could go wrong at the cash register are endless, one could say that the reason why \emph{ni narimasu }になります is here to stay is that it is the best means of the workers to cover face and yet be polite to the consumer at the same time. }

\par{\textbf{Terminology Note }: Customer service honorifics may also go by the names \emph{baitogo }バイト語, \emph{kombini keigo }コンビニ敬語, \emph{famiresu keigo }ファミレス敬語, \emph{famikon kotoba }ファミコン言葉, etc. }

\begin{center}
\textbf{\emph{To narimasu }となります \& \emph{[ni\slash to] natte orimasu }\{に・と\}なっております } 
\end{center}

\par{ In addition to \emph{ni narimasu }になります, there is also \emph{to narimasu }となります, and \emph{[ni\slash to] natte orimasu }に・となっております to consider. The usage of the particle \emph{to }と is meant to enhance formality. The use of に・となっております is meant to indicate that a situation is and has been so but in a manner that shifts responsibility away from the speaker. One can also say that the use of this form simply follows the general rule of thumb that the longer the phrase, the politer it is. However, with each time it\textquotesingle s used, the “but why?” response from the listener becomes ever more justified. }

\par{11. ${\overset{\textnormal{ほんじつ}}{\text{本日}}}$ のランチは、 ${\overset{\textnormal{ひ}}{\text{冷}}}$ やし ${\overset{\textnormal{ちゅうかていしょく}}{\text{中華定食}}}$ \{となっております △・でございます ○\}。 \hfill\break
 \emph{Honjitsu no ranchi wa, hiyashi ch }\emph{ūka teishoku [to natte orimasu\slash de gozaimasu]. } \hfill\break
As for today\textquotesingle s lunch, the special of the day is chilled Chinese days. }

\par{\textbf{Example Note }: When it is a waiter that is saying this, one can easily see how he or she is trying to avoid responsibility for the management\textquotesingle s decision for having the special of the day be as such. Yet, the worker is by default a spokesperson of the establishment when telling customers about the day\textquotesingle s special. The use of \emph{to natte orimasu }となっております conversely emphasizes the value in what the special is to many speakers, which is unnecessary. }

\par{12. こちらが ${\overset{\textnormal{まーぼーどうふていしょく}}{\text{麻婆豆腐定食}}}$ \{となっております △・です ○・でございます ○\}。 \hfill\break
 \emph{Kochira ga m }\emph{āb }\emph{ō-d }\emph{ōfu teishoku [to natte orimasu\slash desu\slash de gozaimasu]. \hfill\break
 }This the mapo doufu special. }

\par{\textbf{Example Note }: Just as with the previous example, too much emphasis is conversely attached to the value of the special, making this statement verging on boasting, an ironic conclusion to avoiding responsibility for what it is. }

\par{13. ${\overset{\textnormal{あした}}{\text{明日}}}$ が ${\overset{\textnormal{きげん}}{\text{期限}}}$ \{に\slash と・なっております △・です ○\}。 \hfill\break
 \emph{Asu ga kigen [ni\slash to natte orimasu\slash desu]. }\hfill\break
The deadline is tomorrow. }

\par{\textbf{Example Note }: This example sentence is indicative of a worker telling a coworker, perhaps someone of higher status at the place of employment, that the deadline is tomorrow. As such, the relationship between the speaker and listener is not as inherently distant as is the case between unacquainted customer and server. This is why the particle \emph{ni }に may manifest here. However, because the phrase \emph{ni\slash to natte orimasu }に・となっております, overall, implies shifting responsibility, that in and of itself is unpleasant in the workplace. As such, this worker will most likely be corrected to simply use \emph{desu }です instead. }

\par{14. ${\overset{\textnormal{とうてん}}{\text{当店}}}$ は、 ${\overset{\textnormal{じょせいせんよう}}{\text{女性専用}}}$ \{となっております △・とさせて ${\overset{\textnormal{いただ}}{\text{頂}}}$ いています ○・ でございます ○\}。 \hfill\break
 \emph{T }\emph{ōten wa, josei sen\textquotesingle y }\emph{ō [to natte orimasu\slash to sasete itadaite imasu\slash de gozaimasu]. }\hfill\break
This store is exclusive to women. \hfill\break
 \textbf{\hfill\break
Example Note }: In this example, the use of to \emph{natte orimasu }となっております shifts the blame away from the establishment for being women-exclusive, which at face value is ludicrous. As such, most speakers prefer that \emph{to sasete itadaite imasu }とさせていただいています be used because this pattern expresses a willful decision that was taken out of discretion toward those affected. }

\begin{center}
 \textbf{Listener Discomfort }
\end{center}

\par{ Another thing to consider is to why \emph{ni\slash to narimasu }に・となります and \emph{ni\slash to natte orimasu }に・となっております cause the listener discomfort is because \emph{naru }なる not only shows change, but the change it describes is not one caused by someone. Purposely using it in a situation where the situation (which may change) is most certainly caused by someone is yet another reason for why these phrases are interpreted as avoiding responsibility\slash blame. Using \emph{desu }です or \emph{de gozaimasu }でございます avoids this situation, and if the statement may still be perceived as being rude to the listener(s), then cushion words \emph{like ky }\emph{ōshuku desu ga }恐縮ですが (I\textquotesingle m terribly sorry to trouble you, but…), \emph{osoreimasu ga }恐れ入りますが (Pardon me for troubling you, but…), \emph{m }\emph{ōshiwake gozaimasen ga }申し訳ございませんが (I am terribly sorry, but…) may be used to enhance politeness, thus making the phrases in question unnecessary. }

\par{15. お ${\overset{\textnormal{しはら}}{\text{支払}}}$ いは ${\overset{\textnormal{ぎんこうふりこみ}}{\text{銀行振込}}}$ のみ\{となります △・です ○・ でございます ○\}。 \hfill\break
 \emph{Oshiharai wa gink }\emph{ō furikomi [to narimasu\slash desu\slash de gozaimasu]. }\hfill\break
Payment is only by bank transfer. }

\par{\textbf{Example Note }: Consumers, hearing this with \emph{to narimasu }となります, have every reason to retort, “But why? Is there not any other form of payment, and why is this the only one you offer?” }

\par{16. ${\overset{\textnormal{えきこうない}}{\text{駅構内}}}$ は ${\overset{\textnormal{きんえん}}{\text{禁煙}}}$ \{となります △・です ○・でございます ○\}。 \hfill\break
 \emph{Eki k }\emph{ōnai wa kin\textquotesingle en [to narimasu\slash desu\slash de gozaimasu]. }\hfill\break
The inside of the train station is nonsmoking. }

\par{\textbf{Example Note }: Smokers, hearing this statement with \emph{to narimasu }となります, may retort, “Well why can\textquotesingle t we smoke?” }

\begin{center}
\textbf{When \emph{n }\emph{i\slash to narimasu }に・となります \& \emph{ni\slash to natte orimasu }に・となっております are Correct }
\end{center}

\par{ As was demonstrated at the beginning of this lesson, there are indeed instances when these phrases are correct, which is when the verb \emph{naru }なる is used in its traditional means. In the following examples, \emph{naru }なる is used in the sense of showing some natural state, perhaps a change, benefit, or reaching some status. Also, as usually, the use of \emph{te imasu\slash orimasu }ています・おります is determined by normal grammar. If it is a state that has been the case, you use them. If you are stating said situation with any degree of formality \emph{orimasu }おります is used instead of \emph{imasu }います. }

\par{17. ${\overset{\textnormal{たま}}{\text{玉}}}$ ねぎを ${\overset{\textnormal{いた}}{\text{炒}}}$ めて ${\overset{\textnormal{じゅっ}}{\text{10}}}$ ${\overset{\textnormal{ふんかんひ}}{\text{分間火}}}$ を ${\overset{\textnormal{とお}}{\text{通}}}$ したものが、こちらになります。 \hfill\break
 \emph{Tamanegi wo itamete juppunkan hi wo t }\emph{ōshita mono ga, kochira ni narimasu. \hfill\break
 }This is what becomes of sautéing and cooking onions for ten minutes. }

\par{\textbf{Spelling Note }: \emph{Tamanegi }may alternatively be spelled as 玉葱. }

\par{\textbf{Reading Note }: 10分間 may also be read as \emph{jippunkan }. }

\par{18. ヴルカーノ ${\overset{\textnormal{とう}}{\text{島}}}$ は「 ${\overset{\textnormal{かざん}}{\text{火山}}}$ 」を ${\overset{\textnormal{いみ}}{\text{意味}}}$ する「Volcano」という ${\overset{\textnormal{えいたんご}}{\text{英単語}}}$ そのものの ${\overset{\textnormal{ごげん}}{\text{語源}}}$ となっています。 \hfill\break
 \emph{Buruk }\emph{ān }\emph{o-t }\emph{ō wa “kazan” wo imi suru “volcano” to iu eitango sono mono no gogen to natte imasu. }\hfill\break
Vulcano Island is the very origin of the English word “volcano” meaning “kazan.” \emph{}}

\par{19. この ${\overset{\textnormal{かんわじてん}}{\text{漢和辞典}}}$ では、 ${\overset{\textnormal{じょうようかんじ}}{\text{常用漢字}}}$ の ${\overset{\textnormal{よ}}{\text{読}}}$ みは ${\overset{\textnormal{あかもじ}}{\text{赤文字}}}$ となっております。 \hfill\break
 \emph{Kono kanwa jiten de wa, j }\emph{ōy }\emph{ō kanji no yomi wa akamoji to natte orimasu. }\hfill\break
In this Chinese-Japanese character dictionary, Joyo Kanji readings are in red. }

\begin{center}
\textbf{The Validity of \emph{ni narimasu }になります and Others }
\end{center}

\par{ The use of these phrases in honorifics to replace the copula is here to stay. Language is always evolving along with culture. Because this usage is here to stay, it is important to become familiar with them and understand what people mean when they use them. It is no longer the case that a large majority even views this usage as incorrect anymore; thus, labeling it as incorrect and ignoring it is no longer an option. As a Japanese learner, you must understand that if you ever absorb Japanese media or go to Japanese establishments, this will be used and used extensively. It is also a part of the regular honorific speech of people outside of the workplace, too. }

\par{20. お ${\overset{\textnormal{てあら}}{\text{手洗}}}$ いはあちらになります。 \hfill\break
 \emph{Otearai wa achira ni narimasu. }\hfill\break
The bathroom is over there. \hfill\break
 \hfill\break
21. こちらが ${\overset{\textnormal{かいぎしつ}}{\text{会議室}}}$ になります。 \hfill\break
 \emph{Kochira ga kaigishitsu ni narimasu. }\hfill\break
This is the meeting room }

\par{22. ${\overset{\textnormal{さんびゃくごじゅう}}{\text{350}}}$ ${\overset{\textnormal{えん}}{\text{円}}}$ のお ${\overset{\textnormal{つ}}{\text{釣}}}$ りになります。 \hfill\break
 \emph{Sambyakugoj }\emph{ūen no otsuri ni narimasu. }\hfill\break
Your change is 350 yen. }

\par{23. ${\overset{\textnormal{まいしゅうげつようび}}{\text{毎週月曜日}}}$ は ${\overset{\textnormal{ていきゅうび}}{\text{定休日}}}$ となっております。 \hfill\break
 \emph{Maish }\emph{ū getsuyobi wa teiky }\emph{ūbi to natte orimasu. }\hfill\break
Our fixed day off is Monday of every week. }

\par{24. ${\overset{\textnormal{ほんじつ}}{\text{本日}}}$ 、 ${\overset{\textnormal{あめ}}{\text{雨}}}$ の ${\overset{\textnormal{よほう}}{\text{予報}}}$ となっておりますので、お ${\overset{\textnormal{き}}{\text{気}}}$ をつけて、お ${\overset{\textnormal{こ}}{\text{越}}}$ しください。 \hfill\break
 \emph{Honjitsu, ame no yoh }\emph{ō to natte orimasu node, oki wo tsukete, okoshi kudasai. \hfill\break
 }Today, the forecast is rain, and so please be careful when you come. }

\par{25. ポイントの ${\overset{\textnormal{いちぶ}}{\text{一部}}}$ をご ${\overset{\textnormal{りよう}}{\text{利用}}}$ になりますか。 \hfill\break
 \emph{Pointo no ichibu wo goriy }\emph{ō ni narimasu ka? } \hfill\break
Would you like to use your points? }

\par{\textbf{Example Note }: This last example is an instance of the correct use of the pattern \emph{o\slash go- }お・ご + Verb Stem + \emph{ni narimasu }になります. }
    