    
\chapter{Defining Parameter}

\begin{center}
\begin{Large}
第292課: Defining Parameter: ~を始め(として), ~を皮切りに(して), ~に至るまで, ~をもって, \& ~といったところだ 
\end{Large}
\end{center}
 
\par{ Defining the beginning and limitation of a parameter is something that people do all the time. These phrases, although not an exhaustive list of phrases in Japanese that fit this description, will allow you to become even more sophisticated in how you can articulate yourself in this regard. }
      
\section{~を始め(として)}
 
\par{ As you should know, 始める means "to start". をはじめ = "starting with". It may also be expanded to をはじめとして, which is a bit more poignant. }

\par{1. 社長を始め、社員一同です。 \hfill\break
Starting with the company president, all employees are present. }

\par{2. 日本の会社を始めとして、沢山の企業が業績悪化しています。 \hfill\break
Starting with Japanese companies, a lot of businesses are in a downturn. }

\par{3. 彼は中国語を始め、日本語と韓国語にも堪能ですよ。 \hfill\break
Starting with Chinese, he is also proficient in Japanese and Korean. }

\par{4. 春になると、桜をはじめとして、いろんな花が咲きます。 \hfill\break
When it becomes spring, starting with cherry blossoms, various flowers blossom. }
      
\section{~を皮切りに(して)}
 
\par{${\overset{\textnormal{かわき}}{\text{皮切}}}$ り means the "outset of things" and is used in the speech modal を皮切りに(して)・を皮切りとして to show that starting with X, you do something in succession. This pattern shouldn't really be used with natural phenomena or bad things. What follows should be similar in nature, and it is especially used in situations where you want to clearly state such a chain of events. }

\par{5. ${\overset{\textnormal{じょうだん}}{\text{冗談}}}$ を皮切りに演説をする。 \hfill\break
To give a speech by starting out with a joke. }

\par{6. 公演を皮切りに ${\overset{\textnormal{かくち}}{\text{各地}}}$ で ${\overset{\textnormal{えんそうかい}}{\text{演奏会}}}$ を開きました。 \hfill\break
They opened up concerts in various\slash every place(s) as the start out of their public performance. }

\par{7. 今月6日の東京公演を皮切りにして、全国ツアーを予定しております。 \hfill\break
We are planning to do a nationwide tour starting with a Tokyo performance on the sixth of this month. }

\par{8. 塩川さんの発言を皮切りにして、みんなが次々に意見を言った。 \hfill\break
Starting with Shiokawa's remarks, everyone said their opinions one after another. }

\par{9. カリフォルニアへの旅行を皮切りとして、国外をあちこち旅行しています。 \hfill\break
Starting with a trip to California, I'm traveling all over outside the country. }
      
\section{~に至るまで}
 
\par{ ~に至るまで shows that a certain matter's extent reaches a surprising state. As you will see in the examples, this speech modal attaches to nouns. }

\begin{center}
\textbf{Examples }
\end{center}

\par{10. 今度の出張のスケジュール ${\overset{\textnormal{ひょう}}{\text{表}}}$ は ${\overset{\textnormal{めんみつ}}{\text{綿密}}}$ すぎる!起床時間から飛行機内の食事開始時間に至るまで書いてあるよ。 \hfill\break
This business trip's schedule is too detailed! It has written on it the time from when we wake up to the time we start eating in the plane. }

\par{11. 畑中先生には卒業後の進路はもちろん、 ${\overset{\textnormal{れんあい}}{\text{恋愛}}}$ の ${\overset{\textnormal{なや}}{\text{悩}}}$ みに至るまで何でも相談している。 \hfill\break
My path after graduation is one thing, but I'm also consulting about everything to even love problems with Hatanaka Sensei. }

\par{12. 外国で暮らすことになったので、ベッドから ${\overset{\textnormal{だいどころようひん}}{\text{台所用品}}}$ に至るまでみんなリサイクルショップに売った。 \hfill\break
As it's been decided that I live abroad, I sold everything from my bed to kitchen utensils to the recycling shop. }

\par{13. 今に至るまで作られている。 \hfill\break
Even until now they are being made. }

\par{14. 頭から足先に至るまでのほぼ全身がびしょ ${\overset{\textnormal{ぬ}}{\text{濡}}}$ れだよ。 \hfill\break
Basically my entire body is soaked from head to toe. }

\par{15. その大都市の学校はもちろんのこと、小さな分校に至るまで、統一されたカリキュラムで教育が行われている。 \hfill\break
Of course, as far as that big city's schools, education is done with a unified curriculum to even the smallest campus. }

\par{16. 彼は頭の上から ${\overset{\textnormal{つまさき}}{\text{爪先}}}$ に至るまで全身泥まみれになった。 \hfill\break
His entire body became covered in mud from the top of his head to his tiptoes. }

\par{\textbf{Usage Note }: This construction is not used in reference to animals. It is always used with something concerning humans. }

\begin{center}
 \textbf{More About 至る }
\end{center}

\par{至る means "to reach\dothyp{}\dothyp{}\dothyp{}". It may follow the 連体形 or stem of する to show that something has finally reached a certain condition. ~に至って(は) and に至ると focus on something as the topic by raising an extreme example. Lastly, 至って as an adverb means "very" or "extremely". }

\par{17. 中国語はもちろん、英語に至ってはからきしだめだよ。 \hfill\break
Not to mention Chinese, reaching English is completely hopeless. }

\par{18. 年老いたとはいえ、至って健康です。 \hfill\break
Although he has aged, he is extremely healthy. }

\par{19. ${\overset{\textnormal{じっし}}{\text{実施}}}$ するに至った状況を説明します。 \hfill\break
I'll explain the situation that's lead to this being implemented. }
      
\section{~をもって}
 
\par{ Though only the first usage relates to the topic of this lesson, as this phrase has other important usages to know about, they will also be discussed at this time. }

\par{1. ~をもって marks the end of things, and is not used in daily conversation and it is very formal and official-like. }

\par{20. 本日の営業は10時をもって閉店致します。 \hfill\break
Today's business will close at 10 o' clock. }

\par{21. 当店は10月末日をもちまして閉店させていただきました。長い間のご利用ありがとうございました。 \hfill\break
This store will close at the end of October. Thank you for your long use (of our store). }

\par{22. 以上をもちまして本日の演説大会は終了いたします。 \hfill\break
With this being the end, today's speech convention is concluded. }

\par{23. 今期をもって私は退職するつもりでございます。 \hfill\break
At the end of this term, I plan to retire. }

\par{24. ${\overset{\textnormal{てんらんかい}}{\text{展覧会}}}$ はこの時点を以って閉会とします。 \hfill\break
We have decided to close this exhibition with\slash at this occasion. }

\par{ It can also show a basis for something or object of judgment. }

\par{25. ${\overset{\textnormal{いちじ}}{\text{一事}}}$ を以って人を判断するなかれ。(Archaic) \hfill\break
Do not judge a man by one action. }

\par{26. やつを以って ${\overset{\textnormal{こうし}}{\text{嚆矢}}}$ とす。(Set phrase; archaism) \hfill\break
To hit the enemy as the aim with an arrow to signal the start of the battle. }

\par{27. ${\overset{\textnormal{ひにん}}{\text{非人}}}$ をもって ${\overset{\textnormal{にん}}{\text{任}}}$ じるは悪し。(Archaic; classical) \hfill\break
It is bad to pose as a hinin. }

\par{\textbf{Historical Note }: A 非人 was a person of the lowest rank of society. }

\par{2. In でもって meaning "furthermore". This is still used in formal writing. }

\par{28. 美人で以って頭も良い女 \hfill\break
A woman who is a beautiful person, and furthermore smart }

\par{3. In まったくもって, a stronger version of まったく. Although more literary, it's still possible to hear it used in the spoken language. }

\par{29. 宮城県は全くもって困った地方だね。 \hfill\break
Miyagi Prefecture is a completely troubled region, isn't it? }

\par{4. \textbf{Older }usages of this include being used as a conjunction to mean "thereupon" and as an equivalent to the case particle で to show means of action. }

\par{30. 以って ${\overset{\textnormal{めい}}{\text{瞑}}}$ すべし。 \hfill\break
Thereupon you should sleep. }

\par{31. 手紙を以って ${\overset{\textnormal{こうてい}}{\text{皇帝}}}$ を ${\overset{\textnormal{つうたつ}}{\text{通達}}}$ す。 \hfill\break
To notify the Emperor with a letter. }
      
\section{~といったところだ}
 
\par{ ~といったところだ is used to say that at the most, something isn't that high. It can be after nouns or verbs. It should only be used with phrases that concern a low number\slash amount. }

\par{41. せいぜい1泊で温泉に行くといったところでしょうか。 \hfill\break
How about at the most going to an onsen for a night? }

\par{42. 私の ${\overset{\textnormal{すいみん}}{\text{睡眠}}}$ 時間は4時間といったところです。 \hfill\break
My sleeping time is just at most four hours. }
    