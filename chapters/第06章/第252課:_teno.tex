    
\chapter{~ての}

\begin{center}
\begin{Large}
第252課: ~ての  
\end{Large}
\end{center}
 
\par{ Although it would seem that such a relatively small phrase would be easy, ~ての is far from simple. }
      
\section{~ての}
 
\par{ What is so complicated about this expression? As you know, て follows the 連用形 as a conjunctive particle to show continuation or parallelism. It has become so important that it is deemed an essential conjugation in Japanese, the て形, or more correctly called the 並列形 in this instance. Essentially how it works is that it attaches to the 連用形 and stops the continuation of the current clause, and you as the listener or reader expect something related to that clause to be said next. }

\par{ What about ~ての? の makes an attribute. You may have seen ~についての (about) before. }

\par{1a. 日本について話す。To talk about Japan. \hfill\break
1b. 日本についての話。 A talk about Japan.  }

\par{ In the second it's used as an attribute of 話. There are several such phrases and they're often deemed as fixed postpositions. This, though, doesn't account for everything. In reality there are four broad usages of ~ての. }

\par{1. The attribute form for stopping the 連用形 (連用中止法の連体形). \hfill\break
2. As a postposition (the opposite of a preposition) \hfill\break
3. In fixed attribute expressions. \hfill\break
4. After て phrases used adverbially like 初めて. }

\par{ However, the hard one is 1. Expressions under the third category are just deemed suffixes. An example is ~きっての (The most\dothyp{}\dothyp{}\dothyp{}of all). }

\par{2. 国内きっての敏腕家 \hfill\break
The most capable person in the country }

\par{ We also understand expressions under 4 like 初めての. }

\par{3. 生まれて初めての海外旅行はメキシコでした。 \hfill\break
My first overseas trip ever was to Mexico. }

\par{ The problem is how の really functions and the relationship between the 連用形 and 連体形. Say you have the verb 使う. 使う\textquotesingle s 連体形 is still just 使う. So, why can you say something like さいころを使っての遊び (a game played with dice)? One would assume that this could work with anything. However, there are indeed instances where this is unnatural. Now we have to find out what these restrictions are. The previous example could have easily been stated as さいころを使った遊び. }

\par{ ~ての has a deep tie to the 連体形. After all, it too is an attributive expression. Considering the ways to make an attributive expression in Japanese, you have the following options. }

\begin{enumerate}

\item Nominal + の + Noun ( ${\overset{\textnormal{こ}}{\text{木}}}$ の ${\overset{\textnormal{は}}{\text{葉}}}$ = tree leaves) 
\item 連体形 + Noun  (きれいな町; 美しい花; 勉強している ${\overset{\textnormal{せいと}}{\text{生徒}}}$ ) 
\item Attributive (この学校) 
\end{enumerate}

\par{ Let us not forget about the underlining properties of an attribute and case particles themselves as we examine this pattern further. }

\par{4a. 韓国へ旅行する。To travel to Korea. \hfill\break
4b. 韓国への旅行。 Travel to Korea. \hfill\break
4c. 韓国にの旅行。X }

\par{ As you can see, you have to use の for the attribute, and the case particle in this case stays. However, you couldn't have said にの. Consider how the following expressions below change when you change it to an attribute. }

\par{5. 高校 \textbf{を }卒業する \textrightarrow  高校 \textbf{の }卒業  \hfill\break
To graduate high school \textrightarrow  High school graduation \hfill\break
\hfill\break
6. 彼 \textbf{に }死刑 \textbf{を }宣告する \textrightarrow  彼 \textbf{への }死刑 \textbf{の }宣告  \hfill\break
To give him the death penalty \textrightarrow  The death sentence given to him }

\par{7. 武器 \textbf{が }恐ろし \textbf{い }\textrightarrow  武器 \textbf{への }恐ろし \textbf{さ }\hfill\break
Weapons are dreadful \textrightarrow  The dreadfulness of weapons }

\par{  What about when it seems that both the 連用形 and the 連体形 are logical choices. These are exceptional circumstances, and the noun in question is one that has adjectival qualities. Consider the following. }

\par{8a. 彼女はすごく美人だ。 \hfill\break
8b. 彼女はすごい美人だ。(元は間違い) }

\par{ The first focuses on the quality of beauty while the second modifies the thing. So, it's like the difference between "she's really beautiful" and "she's a real beauty". In other cases such as the ones we've seen, there is a clear back-and-forth change between the 連用形 and 連体形. However, sometimes changing one phrase to fit the other results in an unnatural sentence. This is due to semantic restraints, not the 'grammaticality' of the phrase. }

\par{9. うれしい話 〇 \textrightarrow  うれしく話す X   }

\par{ What about adverbs? What if they're used like an attribute? はじめて is a prime example of の following. しばらく (in a while) is also a good example. Yet, there is a group of adverbs that never take の even when they are with a noun. These nouns are relative and have to do with an indeterminate quantity. So, one says something like ちょっと前. However, ちょっとの前 is unacceptable. There are still other adverbs that are never used with の like 必ず. }

\par{ You have now gotten a glimpse of why ~ての may be so difficult given the complexity of the 連体形 itself and how it relates to the 連用形. However, that's not the only thing that makes it complex. What about the many usages of て? }

\par{1. Shows sequence. }

\par{10. 家に帰って、晩ご飯を食べる。 \hfill\break
To go home and eat dinner. }

\par{2. Indicates reason or cause. }

\par{11. 猫に引っ掻かれて、泣いた。 \hfill\break
I got scratched by a cat and cried. }

\par{3. Indicates method or means. }

\par{12. 肘をついて、眺める。 \hfill\break
To use one's elbows to view. }

\par{4. Comparison or contrast. }

\par{13. 赤くて大きなリンゴ \hfill\break
A red and large apple }

\par{5. Contradictory condition. }

\par{14. あいつは知っていて教えてくれなかった。 \hfill\break
He knew but didn't tell (me). }

\par{6. Shows some sort of condition. This is so in phrases like について. }

\par{ Ignoring \#6, \#1-3 can be used with ~ての. This definitely makes things still difficult. \#4 and \#5 are illogical with it. Now, let's see what happens when it's used with \#1. }

\begin{center}
\textbf{~ての: Sequential }
\end{center}

\par{ When ~ての is used to show sequence, it compensates the particle から. So, it could be replaced by ~てからの or ~たあとの. }

\par{14. 2年間休んでの予定通りの復活でした。 \hfill\break
It was a come-back as planned after a rest for two years }

\par{15. 水族館を見学しての帰り \hfill\break
Coming back from visiting an aquarium }

\par{16. 行脚を果たしての帰途 \hfill\break
Homeward after completing a pilgrimage }

\par{17a. あの日は、遊園地に遊びに行っての緊急事態だったため、現場に行くのに時間がかかってしまった。〇 \hfill\break
17b. あの日は、遊園地に遊びに行った緊急事態だったため、現場に行くのに時間がかかってしまった。X \hfill\break
That day, because there was an emergency situation after having already gone to have fun at an amusement park, it took some time to get to the scene. }

\par{ All of these can be rephrased with the 連体形. This phrase can lead to something being incidental. }

\par{18. 番組は見てのお楽しみだ。 \hfill\break
The fun in the show comes from watching it. }

\par{\textbf{Phrase Note }: This phrase indicates that one wouldn't understand the fun had you not seen it. This phrase, incidentally, no pun intended, can lead to very strong statements. }

\par{19. それは10日たってのことだ。 \hfill\break
That is something that lapsed over 10 days. }

\par{20. 許しを得\{ての・た\}上で \hfill\break
Upon receiving permission }

\par{21. 知っ\{ての・た\}上で \hfill\break
Upon knowing\slash realizing }

\par{ However, when the transition shown by て is a simple sequence, it's very hard to use ~ての. This is because as these examples have shown, it presents some sort of premise, and then something strong is supposed to come after it. }

\begin{center}
 \textbf{~ての: Reasoning }
\end{center}

\par{ When て indicates reasoning or cause, the relation is general. Since X is so, Y happens. Phrases like 心配しての言葉 can be rephrased to 心配して言った言葉. However, this is not as simple as it looks. Rather, there are three possible consequences. }

\begin{enumerate}

\item ~ての can be interchangeable with the 連体形 
\item ~ての can be replaced by the 連体形 but there is a change in meaning 
\item The two are not interchangeable and only one is correct. 
\end{enumerate}

\par{ This is very context driven, but it isn't a totally random assignment. Sometimes context must be there for ~ての to be understood. If words seem to be missing if you translate back into English, the Japanese is probably bad. }

\par{22. 借金返済に追われ\{ての・た\}無理心中 \hfill\break
A suicide forced by being pressed into repaying one's debts }

\par{23. アメリカ生活に疲れ\{ての・た\}情緒的な日本回帰 \hfill\break
An emotional return to Japan from being tired of American life. }

\par{ You can also see ~てのことだ. In similar examples, you can't just replace it with ~た. That's because you lose any sense of reasoning with something else. The の replaces a verb that can be ascertained from the context. The same goes for ~てのものだ. }

\par{24. いつも食料や水を貯めておくのは、\{自然災害・地震・緊急事態\}に備えてのことだ。 \hfill\break
Why I always store food and water is in case of a [natural disaster\slash earthquake\slash emergency situation]. }

\par{25. 明日からのことを考えての準備だ。 \hfill\break
That's in preparation for things tomorrow. }

\par{26. アベノミクスの政策が失敗に終わったら、日本の経済はすぐに復活できないかもしれない。景気が戻るには50年はかかる可能性があるだろう。なので、私が現在のアベノミクスに対して警鐘を唱えるのは、30年先の生活を心配してのことだ。 \hfill\break
If the Abenomics policy were to end in failure, the Japanese economy would possibly not be able to quickly recover. There is also the possibility that it could take even 50 years for the economy to recover. So, as for me personally, what raises an alarm at the current Abenomics is worrying about my life thirty years down. }

\begin{center}
\textbf{~あっての }
\end{center}

\par{ ~あっての is a very important application that states that without X, there is no Y. が is usually dropped. }

\par{27. 苦労あっての喜び。 \hfill\break
Joy only from having gone through hardship. }

\par{28. 海あっての漁業なんだから、海を汚してはいけないぞ。 \hfill\break
As this is a fishing industry since there is the ocean, we mustn't pollute it. }

\par{29. 両親あってのことです。 \hfill\break
It's all thanks to my parents. }

\par{30. 命あっての物種 (Proverb) \hfill\break
Where there is life, there is hope. }

\par{31. 企業あっての組合 \hfill\break
A union in existence thanks to the company }

\begin{center}
 \textbf{~ての: Means and Condition }
\end{center}

\par{ This is actually the most prevalent usage and the easiest to understand. It is often interchangeable with ~ながらの. }

\par{32. たくさんの人を案内しての登山は大変だった。 \hfill\break
Hiking while guiding a lot of people was hard. }

\par{33. 恥をさらしての暴露 \hfill\break
A revolution while disgracing oneself }

\par{ Just as was the case before, there are instances where you can replace it with the 連体形, cases you can but there is a change in meaning, and cases where they're not interchangeable. }

\par{When Interchangeable: }

\par{34. 仕事を離れ\{ての・た\}個人的な関係 \hfill\break
A personal relation aside from work }

\par{35. 予想を踏まえ\{ての・た\}決定 \hfill\break
A decision based on forecast }

\par{36. 大統領にちなん\{での・だ\}恐竜の名前 \hfill\break
A dinosaur name after the president }

\par{37a. 試験に向けての勉強 \hfill\break
37b. 試験に向けた勉強 \hfill\break
Study directed for exam  }
    