    
\chapter{The Particle から II}

\begin{center}
\begin{Large}
第256課: The Particle から II: ~からある, ~からの, \& ~からする 
\end{Large}
\end{center}
 
\par{ These phrases are all quite similar, but it's important to understand how they don't overlap. }
      
\section{Emphasizing Degree}
 
\par{ からある basically shows that a something is a certain amount, but in other contexts it may imply that the quantity is large or even more than the number phrase being used. It is used with phrase involving length, width, wideness, depth, weight, size, and quantity. }

\par{ However, because it is generally not used as much and is very 書き言葉的, it may be best to avoid it in the spoken language in order to not sound unnatural. For instance, 1a is completely grammatically correct, but many speakers would prefer you say 1b. }

\par{1a. 2000からある漢字 〇\slash △ \hfill\break
1b. 2000もの漢字 〇 \hfill\break
As much as 2000 Kanji }

\par{ One application that has not been affected much is when からある is preceded by some time phrase. }

\par{2.\{古く・昔\}からある物語 \hfill\break
A story from a long time ago }

\par{3. この神社は鎌倉時代からあります。 \hfill\break
This temple is from the Kamakura Period. }

\par{ Before seeing more examples, it\textquotesingle s important to know about the similar ~からの and ~からする which find themselves used more than ~からある. ~からの has some overlap with ~からある at times, but its main purpose to be used with cost and expenses as well as basic number statements in which ある would be inappropriate like in Ex. 13. ~からする is used to show prices as in purchases. }

\par{4. 17兆6000億ドルからの国債 \hfill\break
National debt of over 17 trillion, 600 billion dollars }

\par{5. 年間一万ドルからする特別教室 \hfill\break
Special classroom costing ten thousand dollars annually }

\par{6. この地方では毎年100㎝からある積雪のため、スキーが盛んだ。 \hfill\break
Due to snow accumulation of 100 cm or more each year in this region, skiing has flourished. }

\par{7. 夜の間に、10トンからある大量のバイオ ${\overset{\textnormal{はいきぶつ}}{\text{廃棄物}}}$ がゴミ捨て場に捨てられていた。 \hfill\break
During the night, more than 10 tons of biological waste was dumped in the dump site. }

\par{8. 今年は大雪で、ダラスでも30㎝からある積雪が観測された \hfill\break
This year has had heavy snow, and 30 cm has been recorded even in Dallas. }

\par{9. その男の人は50キロからある荷物をひょいと肩に ${\overset{\textnormal{かつ}}{\text{担}}}$ いで、大型トラックに載せた。 \hfill\break
The man hoisted luggage of more than 50 kilos with ease on his shoulders and loaded it into the semi-trailer. }

\par{10. インドネシア諸島は1万3000からある島で構成されている。 \hfill\break
The Indonesian Isles is composed of more than 13,000 islands. }

\par{11. 安くても大体1頭3、4万円からする。 \hfill\break
Even if it\textquotesingle s cheap, a head will basically cost 3-4000 yen. }

\par{12. 数時間前に、40キロからある荷物をいくつも運んだので、体中が痛くて動けない。 \hfill\break
Several hours ago, I was lifting a lot of luggage over 40 kills, and so my whole body aches, and I can\textquotesingle t move. }

\par{13. ${\overset{\textnormal{せんにん}}{\text{千人}}}$ からの ${\overset{\textnormal{かんこうきゃく}}{\text{観光客}}}$ が、 ${\overset{\textnormal{まいにちすいぞくかん}}{\text{毎日水族館}}}$ を ${\overset{\textnormal{おとず}}{\text{訪}}}$ れます。 \hfill\break
As many as a thousand tourists visit the aquarium every day. }

\par{14. 木の ${\overset{\textnormal{おも}}{\text{重}}}$ さは2百トンからある。 \hfill\break
The tree's weight is as much as 200 tons }

\par{15. 身長6メートルからあるキリンが、突然、目の前に現れた。 \hfill\break
A giraffe of over 6 meters in height suddenly appeared in front of my\slash our eyes. }

\par{16. 10メートルからある ${\overset{\textnormal{たいぼく}}{\text{大木}}}$ が集落の家のある方向へ倒れてきた。 \hfill\break
A large tree of 10 meters came crashing down in the direction where a village house was. }

\par{17a. 50 ${\overset{\textnormal{だい}}{\text{台}}}$ からのトラックが ${\overset{\textnormal{なら}}{\text{並}}}$ んでいる。 \hfill\break
17b. 50 ${\overset{\textnormal{だいいじょう}}{\text{台以上}}}$ のトラックが並んでいる。(もっと ${\overset{\textnormal{しぜん}}{\text{自然}}}$ ) \hfill\break
There are as many as fifty trucks lined up. }

\par{\textbf{Warning Note }: Do not confuse this からある with the particle から being followed by ある meaning a certain distance. }

\par{18. 至るとは、始点からある ${\overset{\textnormal{けいろ}}{\text{経路}}}$ を ${\overset{\textnormal{たど}}{\text{辿}}}$ って終点に到達するという意味です。 \hfill\break
“Itaru” means following a certain route from the start point and reaching the end point. }
    