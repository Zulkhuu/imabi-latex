    
\chapter{Combination Particles with ばかり}

\begin{center}
\begin{Large}
第254課: Combination Particles with ばかり 
\end{Large}
\end{center}
 
\par{ ばかり is a very important particle and is in many combination particles. }
      
\section{The Adverbial Particle ばかり}
 
\par{  ばかり is used in several important usages and may be ばっかり, ばかし, or ばっか in colloquial settings. }
 
\par{1. Shows degree. }
 
\par{1. リンゴを二つばかりください。 \hfill\break
Please give me (approximately) two apples. }
 
\par{2. 旅行の ${\overset{\textnormal{ひよう}}{\text{費用}}}$ は全部で50万円ばかりかかりました。 \hfill\break
The total cost of the trip came to around 500,000 yen. }
 
\par{3. 運動選手さんは10キロばかり ${\overset{\textnormal{きょうそう}}{\text{競走}}}$ に走りました。 \hfill\break
The athlete ran about 10 kilometers in the race. }
 
\par{2. Shows limitation. }

\begin{ltabulary}{|P|P|}
\hline 

1. & Used with nominals to show everything is just a certain thing. \\ \cline{1-2}

2. & After a noun or in between て and いる meaning "just doing". \\ \cline{1-2}

3. & ばかりだ expresses what's yielded in a one-sided situation. \\ \cline{1-2}

\end{ltabulary}

\par{${\overset{\textnormal{}}{\text{4. 見渡}}}$ す限り一面雪ばかりでした。 \hfill\break
There was nothing but snow as far as I could see. }

\par{5. 弟は外で遊んでばかりいます。 \hfill\break
My younger brother does nothing but play outside. }

\par{6. 兄はコンピューターを使ってばかりいる。でも、毎度クラッシュしてしまうから、お返しされるんだ。 \hfill\break
My older brother is always just using the computer. But, since every time he uses it it crashes, I get payback. }

\par{7. 国際問題は悪くなるばかりだ。 \hfill\break
The international problem is just getting worse. }

\par{8. その子犬はいつも眠ってばかりいる。 \hfill\break
That puppy is always just sleeping. }

\par{9. 家にばかりいないで、たまには外出しよう。 \hfill\break
Don't just stay home. Let's go out once in a while. }

\par{10. ${\overset{\textnormal{とつぜん}}{\text{突然}}}$ の ${\overset{\textnormal{ていでん}}{\text{停電}}}$ で人々は逃げ場を求めながら ${\overset{\textnormal{うおうさおう}}{\text{右往左往}}}$ するばかりだった。 \hfill\break
People could only move in confusion while looking for a place to escape by the sudden blackout. }

\par{3. Either after ~ん, which must follow the original 未然形 for all verbs, or the 連体形, ばかり may also show that "something is as if it is going to happen". It does not have to literally be the case. This pattern may also be used to show an excessive\slash extreme degree. }

\par{11. 雨が降り出さんばかりの空 ${\overset{\textnormal{もよう}}{\text{模様}}}$ だ。 \hfill\break
The sky is as if it's about to rain. }

\par{12. ${\overset{\textnormal{ふくろ}}{\text{袋}}}$ ははち切れるんばかりにつまっていた。 \hfill\break
The bag was filled as if it was about to burst. }

\par{13. ボールは爆発せんばかりに、 ${\overset{\textnormal{ふく}}{\text{膨}}}$ らんでいた。 \hfill\break
The ball expanded as if it were about to explode. }

\par{14. 彼は爆発せんばかりに、 ${\overset{\textnormal{ほお}}{\text{頬}}}$ を膨らませていた。 \hfill\break
He was about to explode (in anger). }

\par{\textbf{Historical Note }: The ~ん comes from the old negative ending ~ぬ, and the pattern ~ぬばかりだ was equivalent to しないだけで・・・したかと思うほど. It eventually got confused with the old volitional form ~ん・む, which brought about the sense that it looks like it's about to happen. }

\par{\textbf{Reading Note }: 頬 can also be read as ほほ. }

\par{4. After ~た it shows what is yielded right after something. It is of a moment. In other words, the event in question can't have a long duration. So, things like 徹夜したばかり are wrong. }

\par{15. 20歳になったばかりだ。 \hfill\break
I just turned 20. }

\par{16. 僕はたった今着いたばかりだ。 \hfill\break
I only just arrived. }

\par{17. 寝たばかりのようだ。 \hfill\break
It looks like he just fell asleep. }

\par{18. 寝たばかり。X \hfill\break
I just slept. }

\par{19. はい、昨日覚えたばかりですから。 \hfill\break
Yes, it's because I just learned it yesterday. }

\par{20. 起きたばかりです。 \hfill\break
I just woke up. }

\par{21. 私は習ったばかりの日本語を使ってみました。 \hfill\break
I tried using Japanese I had just learned. }

\par{22. 学期が始まったばかりなので、まだあんまり忙しくない。 \hfill\break
Since the semester started, I'm still not that busy. }

\par{23. さっき食べたばかりなのに、またすぐお ${\overset{\textnormal{なか}}{\text{腹}}}$ が空いてしまった。 \hfill\break
Although I had just eaten a little while ago, I got hungry again right away. }

\par{24. この本は出版されたばかりだ。 \hfill\break
This book has just been printed. }

\par{25. 昼ご飯を食べたばっかりなのに、もうお腹がすきました。 \hfill\break
Although I just ate lunch, I'm already hungry. }

\par{読み物: \textbf{アイヌ語を守ろう! }}

\par{アイヌ語は、単語のひとつひとつにも物語のような ${\overset{\textnormal{がいねん}}{\text{概念}}}$ があるらしいですね。アイヌ語が消えて行くのはもったいないです。アイヌ語は ${\overset{\textnormal{ぜつめつ}}{\text{絶滅}}}$ の ${\overset{\textnormal{きき}}{\text{危機}}}$ にある言語ですが、まだ ${\overset{\textnormal{ひとすじ}}{\text{一筋}}}$ の希望の光があると思います。言語は人間のごとく生きているものです。アイヌ語の命が ${\overset{\textnormal{た}}{\text{絶}}}$ たれたならば、それは世界は人間性を形成する大事な部分も失ってしまうほどのことなのです。それぞれの言語は ${\overset{\textnormal{どくじせい}}{\text{独自性}}}$ を保っていて、代々受け ${\overset{\textnormal{つ}}{\text{継}}}$ がれた情報のかたまりであり、それが記号化されたものです。アイヌ語が絶滅する可能性は高いですが、きっと ${\overset{\textnormal{よみがえ}}{\text{蘇}}}$ らせられるでしょう。アイヌ語のネイティブは ${\overset{\textnormal{こうれいしゃ}}{\text{高齢者}}}$ ばかりになってしまっているので、テープなどで保存していると以前聞いたことがありますが、どうでしょうかね…。 }

\par{1. What is the overall gist? }

\par{2. Find and explain the usage of ばかり. }

\par{3. What is the Ainu language described as? }

\par{4. What has been done according to the reading to preserve it? }

\par{\textbf{Reading Note }: 代々 = だいだい・よよ. The last reading is less common. }
      
\section{Combination Particles of ばかり}
 
\par{ The chart below illustrates the usages and meanings of the combination particles of ばかり. Then, the chart will be followed by examples. }

\begin{ltabulary}{|P|P|P|}
\hline 

 & Meaning & Description \\ \cline{1-3}

~ばかりに & Only Because & Shows what happens solely due to\dothyp{}\dothyp{}\dothyp{} \\ \cline{1-3}

~ばかりか & Not only, but & Shows a negative extreme and then another. \\ \cline{1-3}

~ばかり[で・じゃ]なく & Not only, but also & Shows an extreme and then presents another. \\ \cline{1-3}

\end{ltabulary}

\par{\textbf{Grammar Note }: ~ばかりに is only seen after a dependent clause. All other ばかり are simply the two separate particles used right next to each other. }

\begin{center}
 \textbf{Examples }
\end{center}

\par{26. あの終電車に乗ったばかりに、事故に ${\overset{\textnormal{あ}}{\text{遭}}}$ った。 \hfill\break
I was in the wreck only because I road on the last train. }

\par{27. ${\overset{\textnormal{はんざい}}{\text{犯罪}}}$ の場にたまたま居合わせたばかりに、事件に巻き込まれてしまうということはよくあります。 \hfill\break
Getting dragged into a case solely due to the fact that you just happen to be at the crime scene happens a lot. }

\par{28. 彼はお金がないばかりに友だちも少ない。 \hfill\break
His friends are even few solely because he has no money. }

\par{29. 頭ばかりか胸も痛いです。 \hfill\break
Not only my head but my chest hurts, too. }

\par{30. 彼は ${\overset{\textnormal{きょうじゅ}}{\text{教授}}}$ にゴマをするばかりか、カンニングまでしていい点を取ろうとするそうです。友人ではあるけど、信じられないよ。 \hfill\break
I hear that not only butters up his professors, but even goes so far as to cheat to try to get a good grade. Although he's a friend, I can't believe it. }

\par{31. 日本語ばかりでなく韓国語も勉強したいです。 \hfill\break
Not only do I want to study Japanese, but I also want to study Korean. }

\par{32. ${\overset{\textnormal{のど}}{\text{喉}}}$ が ${\overset{\textnormal{かわ}}{\text{渇}}}$ いたばかりじゃなく、お腹も ${\overset{\textnormal{す}}{\text{空}}}$ いた。 \hfill\break
Not only am I thirsty, I'm also hungry. }
    