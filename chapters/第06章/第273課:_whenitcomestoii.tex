    
\chapter{"When it Comes to\dothyp{}\dothyp{}\dothyp{}" II}

\begin{center}
\begin{Large}
第273課: "When it Comes to\dothyp{}\dothyp{}\dothyp{}" II: となると, となったら, \& となれば 
\end{Large}
\end{center}
 
\par{ All three of these phrases share certain commonalities. When the speaker knows\slash hears of a certain situation, the speaker feels, as an effect of that circumstance, that something important has to be noted. What is duly noted typically involves some sort of idea, decision, or reference to common knowledge. }
      
\section{~となると: When it comes to\dothyp{}\dothyp{}\dothyp{}\slash if it were to come to\dothyp{}\dothyp{}\dothyp{}}
 
\par{ ~となると is used to make assertions involving the speaker\textquotesingle s notion, idea, decision, or reference to common knowledge regarding a certain situation. This situation involves some form of change, whether it be literal or simply one\textquotesingle s change in perception. There is always some sort of contrast implied. The use of the particle と versus the other conditional particles provides a rather objective tone and definitive quality to the statement. It also makes it more appropriate in formal writing than the other variations of this pattern we\textquotesingle ll see later in this lesson. }

\par{ As far as utility is concerned, this pattern can follow nouns, verbs (non-past and past tense form), and adjectives (non-past and past tense form). }

\par{1. ${\overset{\textnormal{だいがくいん}}{\text{大学院}}}$ に ${\overset{\textnormal{すす}}{\text{進}}}$ むとなると、 ${\overset{\textnormal{いっしょうけんめいべんきょう}}{\text{一生懸命勉強}}}$ しなければならない。 \hfill\break
When it comes to proceeding to graduate school, you must study very hard. }

\par{\textbf{Sentence Note }: Here, the contrast implied is that if you were not to proceed to graduate school or pursue something else, you may not have to study near as hard, but because the change at hand is going to graduate school, the speaker feels impelled to tell the user what the natural consequence of that decision will be. }

\par{2. ${\overset{\textnormal{かいさん}}{\text{解散}}}$ となると、 ${\overset{\textnormal{そうせんきょ}}{\text{総選挙}}}$ の ${\overset{\textnormal{そうてん}}{\text{争点}}}$ は ${\overset{\textnormal{しょうひぜい}}{\text{消費税}}}$ になってしまう。 \hfill\break
If it were to come to the dissolution (of the government assembly), the consumer tax would end up becoming the issue at hand in the general election. }

\par{3. ${\overset{\textnormal{きょう}}{\text{今日}}}$ はかなり ${\overset{\textnormal{さむ}}{\text{寒}}}$ い。。。となるとラーメンかうどんが ${\overset{\textnormal{た}}{\text{食}}}$ べたくなります。 \hfill\break
Today is quite cold…with that being the case, I\textquotesingle ll want to eat ramen or udon. }

\par{\textbf{Sentence Note }: As demonstrated by Ex. 3, this pattern may be used at the initial position of a sentence. This is the case for the other variants of this grammatical pattern we have yet to study closely, which are to be detailed later in this lesson. }

\par{4. ${\overset{\textnormal{えいご}}{\text{英語}}}$ を ${\overset{\textnormal{はな}}{\text{話}}}$ せない ${\overset{\textnormal{おや}}{\text{親}}}$ が ${\overset{\textnormal{く}}{\text{来}}}$ るとなると、 ${\overset{\textnormal{にゅうこくしんさ}}{\text{入国審査}}}$ が ${\overset{\textnormal{しんぱい}}{\text{心配}}}$ です。 \hfill\break
If my parents, who can\textquotesingle t speak English, were to come, I\textquotesingle d worry about their immigration checks. }

\par{5. ${\overset{\textnormal{くるまえら}}{\text{車選}}}$ び、ましてや ${\overset{\textnormal{ちゅうこしゃ}}{\text{中古車}}}$ となると、マーケットは ${\overset{\textnormal{はばひろ}}{\text{幅広}}}$ く ${\overset{\textnormal{せんたくし}}{\text{選択肢}}}$ がいっぱいです! \hfill\break
As for choosing cars, especially more so when it comes to used cars, the market is extensive and full of options! }

\par{\textbf{Spelling Note }: ましてや may seldom be spelled as 況してや. }

\par{6. ギャンブルにおける ${\overset{\textnormal{ぜいきん}}{\text{税金}}}$ となると、あまりピンと ${\overset{\textnormal{こ}}{\text{来}}}$ ない ${\overset{\textnormal{ひと}}{\text{人}}}$ も ${\overset{\textnormal{おお}}{\text{多}}}$ いとは ${\overset{\textnormal{おも}}{\text{思}}}$ いますが、 ${\overset{\textnormal{かくとく}}{\text{獲得}}}$ した ${\overset{\textnormal{しょうきんしだい}}{\text{賞金次第}}}$ によっては ${\overset{\textnormal{ぜいきん}}{\text{税金}}}$ を ${\overset{\textnormal{しはら}}{\text{支払}}}$ う ${\overset{\textnormal{ひつよう}}{\text{必要}}}$ が ${\overset{\textnormal{で}}{\text{出}}}$ てきます。 \hfill\break
As far as taxes in gambling are concerned, I think this doesn\textquotesingle t come intuitively to a lot of people, but depending on the prize money you earn, it becomes necessary to pay taxes. }

\par{7. ${\overset{\textnormal{かがわ}}{\text{香川}}}$ さんはあまり ${\overset{\textnormal{の}}{\text{飲}}}$ まない ${\overset{\textnormal{ひと}}{\text{人}}}$ なんですが、 ${\overset{\textnormal{の}}{\text{飲}}}$ むとなると、 ${\overset{\textnormal{てっていてき}}{\text{徹底的}}}$ に ${\overset{\textnormal{の}}{\text{飲}}}$ む ${\overset{\textnormal{ひと}}{\text{人}}}$ ですよ。 \hfill\break
Mr. Kagawa doesn\textquotesingle t drink that much, but when he does drink, he\textquotesingle s the kind of person who hits the bottle. }

\par{\textbf{Sentence Note }: This pattern is perfect to establish generalizations that involve some change in circumstance. Ex. 7 is a perfect example of this. The change in circumstance is Mr. Kagawa being in the position of imbibing. The generalization is that when he does hit the bottle, he goes all out. }

\par{8. え、 ${\overset{\textnormal{ほんとう}}{\text{本当}}}$ に? ${\overset{\textnormal{あつ}}{\text{暑}}}$ いとなると、 ${\overset{\textnormal{も}}{\text{持}}}$ ってく ${\overset{\textnormal{ふく}}{\text{服}}}$ を ${\overset{\textnormal{かんが}}{\text{考}}}$ え ${\overset{\textnormal{なお}}{\text{直}}}$ さなくちゃなあ。 \hfill\break
What, really? If it\textquotesingle s going to be hot (there), I\textquotesingle ll have to rethink what clothes to bring… }

\par{\textbf{Sentence Note }: It\textquotesingle s important to note that it is not implied that the place the speaker is going to has become hot. Rather, the speaker\textquotesingle s perception of what the temperature is like has changed. }

\par{9. ${\overset{\textnormal{こじん}}{\text{個人}}}$ への ${\overset{\textnormal{えいきょう}}{\text{影響}}}$ となると、また ${\overset{\textnormal{すこ}}{\text{少}}}$ し ${\overset{\textnormal{ちが}}{\text{違}}}$ ってきます。 \hfill\break
Where effects on the individual are concerned, it is once again a little different. }

\par{ You mustn\textquotesingle t blindly associate every instance of ~となると with this grammatical pattern. The reason for this is that sometimes, like in Ex. 10, the final と is the citationと. }

\par{10. この ${\overset{\textnormal{ざいせいもんだい}}{\text{財政問題}}}$ は ${\overset{\textnormal{にっぽんけいざい}}{\text{日本経済}}}$ に ${\overset{\textnormal{おお}}{\text{大}}}$ きなマイナス ${\overset{\textnormal{えいきょう}}{\text{影響}}}$ となると ${\overset{\textnormal{い}}{\text{言}}}$ えよう。 \hfill\break
One could say that this economic problem will be a great, negative impact to the Japanese economy. }

\begin{center}
\textbf{~ともなると: Especially when it comes to… }
\end{center}

\par{ ~ともなると is a more emphatic version of above. It enhances the implied sense of certainty that the speaker wishes to convey. As a result from the change in tone, this form is used a lot more in speech than the above. }

\par{11. ${\overset{\textnormal{なつ}}{\text{夏}}}$ ともなると、 ${\overset{\textnormal{む}}{\text{蒸}}}$ し ${\overset{\textnormal{あつ}}{\text{暑}}}$ い ${\overset{\textnormal{きゅうしゅう}}{\text{九州}}}$ よりは ${\overset{\textnormal{ほっかいどう}}{\text{北海道}}}$ に ${\overset{\textnormal{い}}{\text{行}}}$ きたくなる。 \hfill\break
Especially when it comes to summer, one feels like going to Hokkaido rather than humid Kyushu. }

\par{12. ${\overset{\textnormal{すう}}{\text{数}}}$ ${\overset{\textnormal{か}}{\text{ヶ}}}$ ${\overset{\textnormal{げつ}}{\text{月}}}$ の ${\overset{\textnormal{かいがいりょこう}}{\text{海外旅行}}}$ ともなると、 ${\overset{\textnormal{けいかく}}{\text{計画}}}$ を ${\overset{\textnormal{た}}{\text{立}}}$ てなければならないですね。 \hfill\break
Especially when it comes to vacationing abroad for several months, you must make a plan, you know. }

\par{13. ${\overset{\textnormal{へいじつ}}{\text{平日}}}$ は ${\overset{\textnormal{ひと}}{\text{人}}}$ が ${\overset{\textnormal{すく}}{\text{少}}}$ ないですが、 ${\overset{\textnormal{きゅうじつ}}{\text{休日}}}$ ともなると、 ${\overset{\textnormal{あさ}}{\text{朝}}}$ っぱらから ${\overset{\textnormal{かんこうきゃく}}{\text{観光客}}}$ で ${\overset{\textnormal{どうろ}}{\text{道路}}}$ が ${\overset{\textnormal{じゅうたい}}{\text{渋滞}}}$ してしまいます。 \hfill\break
On weekdays there are few people out, but especially on holidays, the streets become congested starting early in the morning. }

\par{14. ${\overset{\textnormal{じゅうに}}{\text{12}}}$ ${\overset{\textnormal{がつ}}{\text{月}}}$ ともなると、 ${\overset{\textnormal{まち}}{\text{町}}}$ にはジングルベルのメロディーが ${\overset{\textnormal{あふ}}{\text{溢}}}$ れる。 \hfill\break
Definitely when it comes to December, the town is flooded with the melody of jingle bells. }

\par{15. ${\overset{\textnormal{けんぽうかいせい}}{\text{憲法改正}}}$ ともなると、 ${\overset{\textnormal{いろいろ}}{\text{色々}}}$ なアイデアが ${\overset{\textnormal{で}}{\text{出}}}$ てきて、とても ${\overset{\textnormal{おもしろ}}{\text{面白}}}$ く ${\overset{\textnormal{ぎろん}}{\text{議論}}}$ (が) ${\overset{\textnormal{でき}}{\text{出来}}}$ るようになるでしょう。 \hfill\break
Definitely when it comes to revising the constitution, all sorts of ideas would come forth, and we\textquotesingle d surely be able to have very interesting debates. }
      
\section{~となったら: If it were\dothyp{}\dothyp{}\dothyp{}\slash when it comes to\dothyp{}\dothyp{}\dothyp{}}
 
\par{ ~となったら is used more frequently in the spoken language than ~となると, but it still finds itself used most frequently in the written language. The sentences that result with this variation are typically more hypothetical in nature than those with ~となると. This is due to the use of the particle たら. In a sense, it is far more deeply tied to the translation “if it were…” }

\par{ As far as the utility of ~となったら is concerned, it too can follow nouns, verbs, or adjectives. You\textquotesingle ll see that it can be paired with both the non-past and past tense forms of a verb, which is the same as for ~となると. Choosing the past tense form, as is demonstrated in Ex. 19 and Ex. 20, causes the statement to be based on a point in time in the past and the speaker is choosing to refer back to that instance. }

\par{16. ${\overset{\textnormal{かのじょ}}{\text{彼女}}}$ が ${\overset{\textnormal{にほんはつ}}{\text{日本初}}}$ の ${\overset{\textnormal{じょせいしゅしょう}}{\text{女性首相}}}$ となったら、それは ${\overset{\textnormal{こくさいぶたい}}{\text{国際舞台}}}$ における ${\overset{\textnormal{おうべい}}{\text{欧米}}}$ トレンドの ${\overset{\textnormal{さら}}{\text{更}}}$ なる ${\overset{\textnormal{しょうり}}{\text{勝利}}}$ となるだろう。 \hfill\break
If she were to become Japan's first female prime minister, that would likely be a further win to Western trends on the international stage. }

\par{17. これが ${\overset{\textnormal{かんとうけん}}{\text{関東圏}}}$ への ${\overset{\textnormal{ちょくせつひがい}}{\text{直接被害}}}$ や ${\overset{\textnormal{つなみ}}{\text{津波}}}$ となったら、パニックになっていたかもしれない。 \hfill\break
If this were concerned with direct damage or tsunami to the Kanto region, (the public) may have been in a panic. }

\par{18. ${\overset{\textnormal{つなみ}}{\text{津波}}}$ となったら ${\overset{\textnormal{あぶ}}{\text{危}}}$ ないところで、 ${\overset{\textnormal{きょう}}{\text{今日}}}$ も ${\overset{\textnormal{ひなんかんこく}}{\text{避難勧告}}}$ が ${\overset{\textnormal{で}}{\text{出}}}$ ていた。 \hfill\break
With it being dangerous if it really were to come to a tsunami, an evacuation advisory had been called today as well. }

\par{19. ${\overset{\textnormal{あいて}}{\text{相手}}}$ が ${\overset{\textnormal{けが}}{\text{怪我}}}$ をしたとなったら、 ${\overset{\textnormal{はなし}}{\text{話}}}$ は ${\overset{\textnormal{べつ}}{\text{別}}}$ になってしまいます。 \hfill\break
If the\slash your opponent had become injured, that\textquotesingle d be a different story. }

\par{20. ${\overset{\textnormal{どぐう}}{\text{土偶}}}$ が ${\overset{\textnormal{しゅつど}}{\text{出土}}}$ したとなったら、それだけで ${\overset{\textnormal{かいたく}}{\text{開拓}}}$ の ${\overset{\textnormal{て}}{\text{手}}}$ は ${\overset{\textnormal{と}}{\text{止}}}$ められます。 \hfill\break
If dogū were to have been excavated, that alone would stop the means for development. }

\par{\textbf{Word Note }: Dogū are small humanoid\slash animal figures of prehistoric Japan. }

\par{21. LINEのグループとかで、″ ${\overset{\textnormal{たいしつ}}{\text{退室}}}$ しました″となったら ${\overset{\textnormal{き}}{\text{気}}}$ になるものですか? \hfill\break
Are you bothered when you see “left group” in groups on LINE and what not? }

\par{22. ${\overset{\textnormal{げんざいけんしゅつ}}{\text{現在検出}}}$ されている ${\overset{\textnormal{ほうしゃのう}}{\text{放射能}}}$ は、 ${\overset{\textnormal{ちかすいおせん}}{\text{地下水汚染}}}$ となったら、その ${\overset{\textnormal{どじょう}}{\text{土壌}}}$ で ${\overset{\textnormal{そだ}}{\text{育}}}$ つ ${\overset{\textnormal{のうさくもつ}}{\text{農作物}}}$ はすべて ${\overset{\textnormal{ほうしゃせいぶっしつ}}{\text{放射性物質}}}$ を ${\overset{\textnormal{と}}{\text{取}}}$ り ${\overset{\textnormal{こ}}{\text{込}}}$ むことになるのだろう。 \hfill\break
If the radiation currently being detected were to contaminate ground water, all the crops nourished by the soil would surely take in the radioactive material. }

\par{23. ${\overset{\textnormal{たいしょく}}{\text{退職}}}$ するとなったら、 ${\overset{\textnormal{だれ}}{\text{誰}}}$ しも(が) ${\overset{\textnormal{えんまんたいしゃ}}{\text{円満退社}}}$ したいと ${\overset{\textnormal{おも}}{\text{思}}}$ っています。 \hfill\break
When it comes to resigning, I think everyone wants to resign from one\textquotesingle s free will. }

\par{24. プロ ${\overset{\textnormal{やきゅう}}{\text{野球}}}$ の ${\overset{\textnormal{しあい}}{\text{試合}}}$ でも ${\overset{\textnormal{うてんちゅうし}}{\text{雨天中止}}}$ となったら、 ${\overset{\textnormal{べつ}}{\text{別}}}$ の ${\overset{\textnormal{にってい}}{\text{日程}}}$ でするんでしょうか。 \hfill\break
If even a professional baseball match were to be canceled due to rain, would it be played on a different date? }

\par{25. イギリスがEU ${\overset{\textnormal{りだつ}}{\text{離脱}}}$ となったらどうなるか。 \hfill\break
What will become of England if it leaves the EU? }

\par{26. ${\overset{\textnormal{せんぎょうしゅふ}}{\text{専業主婦}}}$ は ${\overset{\textnormal{りこん}}{\text{離婚}}}$ するとなったら、その ${\overset{\textnormal{ご}}{\text{後}}}$ の ${\overset{\textnormal{せいかつ}}{\text{生活}}}$ はどうなるんですか。 \hfill\break
If a housewife were to get divorced, what becomes of her livelihood afterward? }

\par{27. ${\overset{\textnormal{まんせき}}{\text{満席}}}$ が ${\overset{\textnormal{つづ}}{\text{続}}}$ くとなったら、それは ${\overset{\textnormal{すご}}{\text{凄}}}$ いなあと ${\overset{\textnormal{おも}}{\text{思}}}$ うんですが。 \hfill\break
I think it\textquotesingle d be pretty awesome if it were to continue being full. (Occupancy) }

\par{28. ${\overset{\textnormal{にっぽんせいふ}}{\text{日本政府}}}$ が ${\overset{\textnormal{はたん}}{\text{破綻}}}$ もしくは ${\overset{\textnormal{はたん}}{\text{破綻}}}$ の ${\overset{\textnormal{きき}}{\text{危機}}}$ となったら、 ${\overset{\textnormal{い}}{\text{生}}}$ き ${\overset{\textnormal{のこ}}{\text{残}}}$ る ${\overset{\textnormal{きぎょう}}{\text{企業}}}$ や ${\overset{\textnormal{ぎょうしゅ}}{\text{業種}}}$ は ${\overset{\textnormal{なに}}{\text{何}}}$ でしょうか? \hfill\break
If the Japanese government were to go into bankruptcy, or if there were a bankruptcy crisis, what would be the companies and industries that would survive? }

\par{29. いざアメリカに ${\overset{\textnormal{いじゅう}}{\text{移住}}}$ するとなったら、 ${\overset{\textnormal{てつづ}}{\text{手続}}}$ きが ${\overset{\textnormal{たいへん}}{\text{大変}}}$ だよ。 \hfill\break
If one were to (suddenly) now immigrate to America, the procedures would be difficult. }

\par{30. こうなったら、もうやるっきゃない。 \hfill\break
If it comes to this, I\textquotesingle ll have no choice but to do it. }

\begin{center}
\textbf{~ともなったら: Especially when it comes to\dothyp{}\dothyp{}\dothyp{} }
\end{center}

\par{ Although not as common as adding も to the other variants of this grammatical pattern, ~ともなったら is more emphatic than its counterpart without も. }

\par{31. ${\overset{\textnormal{ひとばん}}{\text{一晩}}}$ の ${\overset{\textnormal{ちゅうしゃりょうきん}}{\text{駐車料金}}}$ に ${\overset{\textnormal{くわ}}{\text{加}}}$ えてタクシー ${\overset{\textnormal{だい}}{\text{代}}}$ ともなったら、 ${\overset{\textnormal{たいへん}}{\text{大変}}}$ な ${\overset{\textnormal{きんがく}}{\text{金額}}}$ になってしまう。 \hfill\break
Especially when it comes to taxi fees in addition to a night\textquotesingle s worth of parking fees, that\textquotesingle d end up being a terrible amount of money. }

\par{32. ${\overset{\textnormal{いっこう}}{\text{一向}}}$ に ${\overset{\textnormal{すす}}{\text{進}}}$ まず ${\overset{\textnormal{た}}{\text{立}}}$ ちっぱなしともなったら、ストレスが ${\overset{\textnormal{た}}{\text{溜}}}$ まってしまう。 \hfill\break
Especially if it came to just standing on one\textquotesingle s feet for a long time and not moving forward at all, stress would build up. }

\par{33. ${\overset{\textnormal{さいばん}}{\text{裁判}}}$ ともなったら ${\overset{\textnormal{ざいさん}}{\text{財産}}}$ が ${\overset{\textnormal{お}}{\text{押}}}$ さえられるというイメージもあるかと ${\overset{\textnormal{おも}}{\text{思}}}$ います。 \hfill\break
Especially when it comes to court, you might have an image of your assets being seized. }

\par{34. ${\overset{\textnormal{しょうわじだい}}{\text{昭和時代}}}$ では、 ${\overset{\textnormal{じゅうはっ}}{\text{18}}}$ ${\overset{\textnormal{さい}}{\text{歳}}}$ にもなったら ${\overset{\textnormal{きょこん}}{\text{許婚}}}$ がいるのが ${\overset{\textnormal{あ}}{\text{当}}}$ たり ${\overset{\textnormal{まえ}}{\text{前}}}$ で、 ${\overset{\textnormal{き}}{\text{決}}}$ まっていないともなったら、 ${\overset{\textnormal{わら}}{\text{笑}}}$ いものになったらしい。 \hfill\break
In the Showa Period, it was only natural to have a fiancé even at 18, and it seems that you were laughed at if it was the case that you hadn\textquotesingle t been committed to someone yet. }

\par{ 35. お ${\overset{\textnormal{とま}}{\text{泊}}}$ り ${\overset{\textnormal{りょこう}}{\text{旅行}}}$ ともなったら、 ${\overset{\textnormal{ぜったい}}{\text{絶対}}}$ にエッチするはずだと ${\overset{\textnormal{おも}}{\text{思}}}$ う。 \hfill\break
Especially when it comes to traveling to stay overnight together, I think that you\textquotesingle d undoubtedly have sex. }
      
\section{~となれば: When\slash if it comes to\dothyp{}\dothyp{}\dothyp{}}
 
\par{ You may have wondered how the use of the particle と is tied to these expressions. You should know at this point that the particle に is typically the particle of choice for なる. However, it is important to understand that the use of the particle に can only demonstrate literal change from one state to another. }

\par{ The nuances that we have seen thus far involving comparison to other possible situations cannot be expressed with the particle に. The particle と helps establish the “when it comes to…” interpretation. In other words, we are focusing on the situation with these speech patterns. In Ex. 31, the speaker uses に to simply illustrate a change in president and what that president might do once in office. }

\par{36. ムン ${\overset{\textnormal{し}}{\text{氏}}}$ が ${\overset{\textnormal{だいとうりょう}}{\text{大統領}}}$ になれば、どんな ${\overset{\textnormal{せいさく}}{\text{政策}}}$ を ${\overset{\textnormal{だ}}{\text{出}}}$ すんだろう。 \hfill\break
When Moon (Jae-in) becomes president, I wonder what sort of policies he\textquotesingle ll issue. }

\par{ The use of ~となれば over the other variants seen thus far is done so to add more of an emotional flair to the sentence. It, like ~となったら,  is more likely to be used in the spoken language than ~となると, but because the focus on change in circumstance and relating it to other outcomes is not always something that people employ in conversation, you will not hear this used every day. However, these patterns are still used quite a bit. }

\par{37. ${\overset{\textnormal{とうきょう}}{\text{東京}}}$ オリンピックが ${\overset{\textnormal{ちゅうし}}{\text{中止}}}$ となれば ${\overset{\textnormal{さまざま}}{\text{様々}}}$ な ${\overset{\textnormal{えいきょう}}{\text{影響}}}$ が ${\overset{\textnormal{で}}{\text{出}}}$ てくるかと ${\overset{\textnormal{おも}}{\text{思}}}$ います。 \hfill\break
If the Tokyo Olympics became suspended, I think there would be various effects. }

\par{38. この ${\overset{\textnormal{じゅう}}{\text{10}}}$ ${\overset{\textnormal{にん}}{\text{人}}}$ を ${\overset{\textnormal{う}}{\text{打}}}$ ち ${\overset{\textnormal{たお}}{\text{倒}}}$ し ${\overset{\textnormal{ゆうしょう}}{\text{優勝}}}$ となれば、すごい ${\overset{\textnormal{こと}}{\text{事}}}$ ですね。 \hfill\break
It would be amazing if (he) defeated these ten people and claimed victory. }

\par{39. ${\overset{\textnormal{みずか}}{\text{自}}}$ ら ${\overset{\textnormal{ふりん}}{\text{不倫}}}$ したとなれば、 ${\overset{\textnormal{ふり}}{\text{不利}}}$ な ${\overset{\textnormal{りこん}}{\text{離婚}}}$ は ${\overset{\textnormal{しかた}}{\text{仕方}}}$ がないんでしょうか。 \hfill\break
If you yourself committed adultery, would an unfavorable divorce be inevitable? }

\par{40. あんなに ${\overset{\textnormal{くる}}{\text{苦}}}$ しんだことも、 ${\overset{\textnormal{いま}}{\text{今}}}$ となれば ${\overset{\textnormal{たい}}{\text{大}}}$ したことではなかったと ${\overset{\textnormal{かん}}{\text{感}}}$ じます。 \hfill\break
When it comes to now, I feel that having suffered all that much wasn\textquotesingle t such a big deal. }

\par{41. ${\overset{\textnormal{とちじ}}{\text{都知事}}}$ が ${\overset{\textnormal{しっきゃく}}{\text{失脚}}}$ したとなれば、 ${\overset{\textnormal{せいかい}}{\text{政界}}}$ の ${\overset{\textnormal{じんざいぶそく}}{\text{人材不足}}}$ が ${\overset{\textnormal{けねん}}{\text{懸念}}}$ されるのは ${\overset{\textnormal{とうぜん}}{\text{当然}}}$ だ。 \hfill\break
If it comes to the governor of Tokyo falls from the position, it\textquotesingle s certain that the lack of talented people in the political world will be of concern. }

\par{42. ${\overset{\textnormal{ちゅうけん}}{\text{中検}}}$ 1 ${\overset{\textnormal{きゅう}}{\text{級}}}$ を ${\overset{\textnormal{しゅとく}}{\text{取得}}}$ したとなれば、もう ${\overset{\textnormal{ちゅうごくご}}{\text{中国語}}}$ をマスターしたといってもいいでしょう。 \hfill\break
If (he) is to achieve Level 1 of the Chinese Proficiency Test, it\textquotesingle d be safe to say that he\textquotesingle s mastered Chinese. }

\par{43. ${\overset{\textnormal{いっき}}{\text{一気}}}$ に ${\overset{\textnormal{よん}}{\text{4}}}$ ${\overset{\textnormal{ほん}}{\text{本}}}$ も ${\overset{\textnormal{ばっし}}{\text{抜歯}}}$ したとなれば、 ${\overset{\textnormal{ふだん}}{\text{普段}}}$ と ${\overset{\textnormal{ちが}}{\text{違}}}$ う ${\overset{\textnormal{かお}}{\text{顔}}}$ になってしまうこともあるだろう。 \hfill\break
If you get four teeth removed in one go, you\textquotesingle ll definitely end up having a different face than normal. }

\par{44. となれば、もう ${\overset{\textnormal{あきら}}{\text{諦}}}$ めるしかないだろう。 \hfill\break
If it comes to that, there\textquotesingle d be no other choice but to just give up. }

\par{45. アメリカが ${\overset{\textnormal{きたちょうせん}}{\text{北朝鮮}}}$ を ${\overset{\textnormal{ぶりょくこうげき}}{\text{武力攻撃}}}$ するとなれば ${\overset{\textnormal{にっぽんせいふ}}{\text{日本政府}}}$ の ${\overset{\textnormal{さいだい}}{\text{最大}}}$ の ${\overset{\textnormal{かんしんじ}}{\text{関心事}}}$ は ${\overset{\textnormal{らちひがいしゃ}}{\text{拉致被害者}}}$ の ${\overset{\textnormal{きゅうしゅつ}}{\text{救出}}}$ についてでしょう。 \hfill\break
If it comes to America launched an armed attack on North Korea, the greatest concern of the Japanese government would likely be about the rescue of abductees. }

\begin{center}
\textbf{~ともなれば: Especially when\slash if it comes to… }
\end{center}

\par{ The use of the particle も adds an extra layer of assertiveness to the statement. It also goes well with the emotional flair that the use of the particle ば provides. }

\par{46. マイホームを ${\overset{\textnormal{こうにゅう}}{\text{購入}}}$ するともなれば、あれこれと ${\overset{\textnormal{かんが}}{\text{考}}}$ えることは ${\overset{\textnormal{おお}}{\text{多}}}$ くなりそうですね。 \hfill\break
Especially when it comes to buying one\textquotesingle s own home, it seems a lot of people would ponder about this and that, huh. }

\par{47. ${\overset{\textnormal{まなつび}}{\text{真夏日}}}$ ともなれば ${\overset{\textnormal{しつおん}}{\text{室温}}}$ も ${\overset{\textnormal{ふだん}}{\text{普段}}}$ より ${\overset{\textnormal{あ}}{\text{上}}}$ がります。 \hfill\break
Especially when it comes to midsummer days, indoor temperatures also rise higher than normal. }

\par{48. ${\overset{\textnormal{しょうがつ}}{\text{正月}}}$ ともなれば ${\overset{\textnormal{たはた}}{\text{田畑}}}$ は ${\overset{\textnormal{ゆき}}{\text{雪}}}$ に ${\overset{\textnormal{う}}{\text{埋}}}$ もれて ${\overset{\textnormal{しんせん}}{\text{新鮮}}}$ な ${\overset{\textnormal{やさい}}{\text{野菜}}}$ はほとんどありません。 \hfill\break
Especially when it comes to New Year, the fields are buried in snow and there are hardly any fresh vegetables. }

\par{49. ${\overset{\textnormal{かいさん}}{\text{解散}}}$ ・ ${\overset{\textnormal{そうせんきょ}}{\text{総選挙}}}$ ともなれば、 ${\overset{\textnormal{やとう}}{\text{野党}}}$ は ${\overset{\textnormal{ひっし}}{\text{必死}}}$ にあらを ${\overset{\textnormal{さが}}{\text{探}}}$ す。 \hfill\break
Especially when it comes to (parliamentary) dismissal and general elections, the opposition party searches desperately for faults. }

\par{ 50. ${\overset{\textnormal{こうきゅう}}{\text{高級}}}$ なお ${\overset{\textnormal{すしや}}{\text{寿司屋}}}$ さんに ${\overset{\textnormal{い}}{\text{行}}}$ くともなれば、 ${\overset{\textnormal{すこ}}{\text{少}}}$ しいつもとは ${\overset{\textnormal{ちが}}{\text{違}}}$ うお ${\overset{\textnormal{しゃれ}}{\text{洒落}}}$ をして ${\overset{\textnormal{きあ}}{\text{気合}}}$ いを ${\overset{\textnormal{い}}{\text{入}}}$ れたいと ${\overset{\textnormal{おも}}{\text{思}}}$ うかもしれません。 \hfill\break
Especially when it comes to going to a high grade sushi place, you probably might think to dress a little more stylish than usual and get psyched for it. }
    