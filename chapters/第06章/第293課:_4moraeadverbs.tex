    
\chapter{4 Morae Adverbs}

\begin{center}
\begin{Large}
第293課: 4 Morae Adverbs 
\end{Large}
\end{center}
 
\par{ There are a lot of adverbs that have a 促音, end in り, are onomatopoeic, and are four morae in length. Although there are definitely more adverbs that are four morae in length, this constituency deserves a name. }

\par{\textbf{品詞 Note }: If a definition is adjectival, use とした with the word. For example, がっしりとした. }
      
\section{4 Morae Adverbs}
 
\begin{ltabulary}{|P|P|}
\hline 

 Adverb & Definition(s) \\ \cline{1-2}

あっさり & Readily; flatly; easily; plain; frank \\ \cline{1-2}

うっかり & Inadvertently \\ \cline{1-2}

うっとり & Ecstatically; absentmindedly; abstractedly; vacantly \\ \cline{1-2}

おっとり & Gently; quietly; calmly \\ \cline{1-2}

がっかり & Disappointing; dejected; downcast \hfill\break
\\ \cline{1-2}

かっきり & Exactly; flat; sharp even \\ \cline{1-2}

がっくり & Heartbroken; break down; dash \hfill\break
\\ \cline{1-2}

がっしり & Stout; solid; massive; square \\ \cline{1-2}

かっちり & Tightly; exactly \\ \cline{1-2}

がっちり & Solidly built; tightly; shrewd \hfill\break
\\ \cline{1-2}

がっぷり & Firmly; latched onto \\ \cline{1-2}

がっぽり & In large quantities \\ \cline{1-2}

きっかり & Promptly; precisely; exactly \hfill\break
\\ \cline{1-2}

ぎっしり & Closely; densely \\ \cline{1-2}

きっちり & Closely; snug; tightly \\ \cline{1-2}

きっぱり & Flatly; absolutely; completely \hfill\break
\\ \cline{1-2}

くっくり & Sharply; clearly; clear-cut \hfill\break
\\ \cline{1-2}

ぐっすり & Soundly \\ \cline{1-2}

ぐったり & Completely exhausted; dead tired; limp \\ \cline{1-2}

げっそり & Being disheartened; loosing weight \\ \cline{1-2}

こっきり & Just; only \\ \cline{1-2}

こっそり & Stealthily; secretly \hfill\break
\\ \cline{1-2}

ごっそり & Completely; wholly; entirely \hfill\break
\\ \cline{1-2}

こってり & Thickly; heavily; severely; strongly \hfill\break
\\ \cline{1-2}

ざっくり & Roughly; loosely; deeply (cut) \\ \cline{1-2}

さっぱり & Refreshed; trimmed; simple; entirely; not at all \hfill\break
\\ \cline{1-2}

しっかり & Tightly; firmly; steadily \\ \cline{1-2}

しっくり & Not well \hfill\break
\\ \cline{1-2}

じっくり & Thoroughly; closely \hfill\break
\\ \cline{1-2}

しっとり & Moist; slightly wet \hfill\break
\\ \cline{1-2}

じっとり & Sticky \\ \cline{1-2}

しっぽり & Moist; delicate (affection) \hfill\break
\\ \cline{1-2}

すっかり & All; quite; completely; perfectly; entirely; utterly; right \\ \cline{1-2}

すっきり & Neat; refreshed; clean-cut; straightforward \\ \cline{1-2}

ずっしり & Heavily \\ \cline{1-2}

すっぱり & Completely; flatly \\ \cline{1-2}

すっぽり & Completely \\ \cline{1-2}

そっくり & Living; natural; lifelike; all \hfill\break
\\ \cline{1-2}

たっぷり & Sufficiently; amply; heartly; good \\ \cline{1-2}

ちゃっかり & Calculating; shrewd; smart; cheeky \hfill\break
\\ \cline{1-2}

ちょっぴり & Little; bit \\ \cline{1-2}

てっきり & Surely; beyond doubt \\ \cline{1-2}

でっぷり & Ample; beefy; stout \\ \cline{1-2}

どっかり & Heavily \\ \cline{1-2}

どっきり & Be shocked; be frightened \\ \cline{1-2}

とっくり & Carefully; well \\ \cline{1-2}

どっさり & Lot of \\ \cline{1-2}

どっしり & Massively; heavily; dignified \hfill\break
\\ \cline{1-2}

とっぷり & Completely; fully; entirely \hfill\break
\\ \cline{1-2}

どっぷり & Totally (immersed) \\ \cline{1-2}

にっこり & (Smile) sweetly \hfill\break
\\ \cline{1-2}

ねっちり & Sticky; persistently \\ \cline{1-2}

ねっとり & Sticky \\ \cline{1-2}

のっそり & Ponderous; elephantine; sluggishly \hfill\break
\\ \cline{1-2}

のっぺり & Flat \\ \cline{1-2}

はっきり & Clearly; distinctly; plainly; definitely; sharply; vividly \\ \cline{1-2}

ぱっくり & Gaping; snapping \\ \cline{1-2}

ばったり & With a clash; abruptly; unexpectedly \\ \cline{1-2}

ぱったり & Abruptly; suddenly \\ \cline{1-2}

ばっちり & Perfectly; properly \\ \cline{1-2}

ぱっちり & Open wide \\ \cline{1-2}

びっしり & Closely; densely \\ \cline{1-2}

ひっそり & Secretly \\ \cline{1-2}

ぴったり & Close; tightly; snug; exactly \\ \cline{1-2}

ぴっちり & Tightly; snugly \\ \cline{1-2}

ひょっこり & Accidentally; by chance \\ \cline{1-2}

ふっつり & Break; snapping up \\ \cline{1-2}

ぷっつり & Break; snapping up \\ \cline{1-2}

べったり & Closely; thickly; hard \\ \cline{1-2}

ぺったり & Closely; fast \\ \cline{1-2}

べっとり & Sticky; icky \\ \cline{1-2}

ぽっかり & Lightly; gaping wide \\ \cline{1-2}

ぽっきり & Only; merely \\ \cline{1-2}

ぽっくり & Suddenly \\ \cline{1-2}

ほっそり & Slender; slim; light \\ \cline{1-2}

ぽっちゃり & Plump \\ \cline{1-2}

ぼってり & Fleshy; chubby \\ \cline{1-2}

まったり & Full-bodily \\ \cline{1-2}

みっしり & Strictly; fully; closely \hfill\break
\\ \cline{1-2}

みっちり & Strictly; fully \hfill\break
\\ \cline{1-2}

めっきり & Very much; a lot; remarkably \\ \cline{1-2}

むっくり & Suddenly \\ \cline{1-2}

むっちり & Plump; chubby \\ \cline{1-2}

むっつり & Sullen; taciturn; sulky \\ \cline{1-2}

やっぱり & After all \\ \cline{1-2}

ゆっくり & Slowly; deliberately; leisurely \hfill\break
\\ \cline{1-2}

ゆったり & Easy; relaxed; comfortable; loose; expansive; full \\ \cline{1-2}

\end{ltabulary}
      
\section{Examples}
  
\par{ゆっくりと話してくれた。 \hfill\break
He slowly talked for me. }

\par{ゆったりとしたスカートを履(は)く。 \hfill\break
To put on a loose skirt. }

\par{ゆったり過ごしただけ。 \hfill\break
I just passed the day. }

\par{国会議員はあっさりやめるべきだ。 \hfill\break
The Congressman should readily resign. }

\par{みんなカメラの方を見てにっこりしましょう。 \hfill\break
Let's all smile towards the camera. }

\par{みっしり鍛えれたのかな。 \hfill\break
I wonder if he was fully trained? }

\par{もっとはっきりと発音しなさい。 \hfill\break
Pronounce this more clearly. }

\par{一晩ぐっすり眠る。 \hfill\break
To get a good night's sleep. }

\par{とっくりと話を聞いて。 \hfill\break
Listen carefully to me, you hear. }

\par{たっぷり5時間も仕事する。 \hfill\break
To work a good 5 hours. }

\par{がっかりなさったでしょう。 \hfill\break
I've disappointed you. }

\par{むっくりと起き上がった。 \hfill\break
They suddenly got up. }

\par{悪道にどっぷりとつかっているだけの男だ。 \hfill\break
He is just a man indulged in evil ways. }

\par{気分がすっきりする。 \hfill\break
To feel refreshed. }

\par{一回こっこりでやめるとは臆病だ。 \hfill\break
Quitting with one time is cowardly. \hfill\break
}

\par{ぽっかり穴が開いた。 \hfill\break
A hole gaped open. }
 
\par{のっそりと起こす。 \hfill\break
To get up sluggishly. }

\par{血のりがズボンにべっとりついてしまった。 \hfill\break
The blood clot got stuck to the pants. }

\par{ひょっこり幸子に出会ったのよ。(Feminine) \hfill\break
I unexpectedly ran into Sachiko. }

\par{彼はどっかりと座った。 \hfill\break
He sat down heavily. }

\par{さっぱりしたいいやつ \hfill\break
A good sport }

\par{あいつのなまりはさっぱり分かんねーぞぇ。(Violent) \hfill\break
I don't understand that guys dialect! }

\par{めっきり暑くなった. \hfill\break
It's become remarkably hot. }

\par{どっさりある。 \hfill\break
To have a ton. }

\par{しっかりしろ! \hfill\break
Pull yourself together! }

\par{うっかりして忘れた。 \hfill\break
It slipped my mind. }

\par{うっとり眺める。 \hfill\break
To ecstatically gaze. }

\par{うっとりさせるような声をあげる。 \hfill\break
To have an alluring voice. }

\par{日がとっぷりと暮れた。 \hfill\break
It got completely dark. }

\par{のっぺりとした目鼻立ち \hfill\break
Expressionless facial expressions }

\par{ちょっぴり皮肉じゃないか? \hfill\break
Isn't that a little ironic? }

\par{僕はほっそりとした体格です。 \hfill\break
僕は\{ほっそり・すらり\}としている。 \hfill\break
I'm slender. }
    