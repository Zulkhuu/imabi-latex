    
\chapter{Occasion}

\begin{center}
\begin{Large}
第284課: Occasion: ~うえで, ~際に, ~折に, ~際して, ~に当たって, ~に先立って, ~を前に(して), ~を控えて, ~に臨んで, \& ~に面して 
\end{Large}
\end{center}
 
\par{ There are several phrases in Japanese that relate to "on the occasion of", but never are so many phrases in Japanese completely interchangeable. They do look similar. Writing style and nuances aid in distinguishing them. Don't worry, though. Even if you get a lot of red after reading this lesson, at least your paper will be decorated. }
      
\section{~(の)うえで}
 
\par{ ~(の)うえで attaches to the 連体形 of verbs or to nouns to give a meaning of “on the occasion of”. Following it, a discussion concerning a point of concern is expected. Giving an explanation of a circumstance, it shows opinion\slash thought on an issue. When giving a meaning of ~ときに, ~際に and ~に際して become possible as well. }

\par{1. 研究を続けるうえで、中国に留学したほうがいいと判断した。 \hfill\break
I have concluded that it is best for me to study abroad in China on the basis of continuing my studies. }

\par{2. 時間を守るということは、仕事\{をする\}うえで ${\overset{\textnormal{さいていげん}}{\text{最低限}}}$ のマナーである。 \hfill\break
Being punctual in the process of work is the minimum of manner. \hfill\break
From 中級日本語文法と教え方のポイント by ${\overset{\textnormal{いちかわやすこ}}{\text{市川保子}}}$ . }

\par{3. 住宅ローンを利用したいと思っているのですが、借りる\{うえで・際(に)は・に際しては\}どのようなことに注意し たらよいでしょうか。 \hfill\break
I want to utilize a home loan, but on the occasion of borrowing, what sort of things would be good to pay attention to? }

\par{From 中級日本語文法と教え方のポイント by ${\overset{\textnormal{いちかわやすこ}}{\text{市川保子}}}$ . }
      
\section{~(の)際に}
 
\par{ ~(の)際(に)  is like above in that it means “on the occasion of”. 際 grammatically behaves like any other noun. It is a rather hard and ceremonious phrase, and it has the nuance of coping well when encountering a particular incident. }

\par{4. 就任の際には、是非式に参加してください。 \hfill\break
By all means, please participate in the ceremony on inauguration. }

\par{5. イースト菌はパンを ${\overset{\textnormal{ふく}}{\text{膨}}}$ らませる際の ${\overset{\textnormal{さようざい}}{\text{作用剤}}}$ だ。 \hfill\break
Yeast is an agent that makes bread rise. }

\par{6. 地震の際はガスの ${\overset{\textnormal{もとせん}}{\text{元栓}}}$ を閉めること。(Stern; 年齢的にも立場的にも上の人の言い方) \hfill\break
In case of an earthquake, turn off the gas at the main outlet. }

\begin{center}
~際に VS ~折に 
\end{center}

\begin{ltabulary}{|P|P|}
\hline 

~際に & Shows a limit to one extraordinary time than other situations. \\ \cline{1-2}

~折に & Shows an opportunity that happens by a stroke of luck. \\ \cline{1-2}

\end{ltabulary}

\par{ Remember that ~際に is used a lot as a formal version of ~とき, being used a lot in polite commands ~折に is very respectful and used a lot in letters and in set phrases such as 折りに触れて (on opportunity). If used in the spoken language, the speaker is most certainly old. Neither are appropriate in regular-place situations like ~とき is in the spoken language. Consider the following examples. }

\par{7. 出発の際に ${\overset{\textnormal{てんこ}}{\text{点呼}}}$ をとるから、全員、遅れないように、集合すること。 \hfill\break
Since I will take roll call on our departure, everyone fall in so as to not be late! }

\par{8. お ${\overset{\textnormal{いとま}}{\text{暇}}}$ な折にでも、お読みいただけましたら、 ${\overset{\textnormal{こうじん}}{\text{幸甚}}}$ に存じます。(手紙文) \hfill\break
I would be so obliged if you were to read this even during your spare time. }

\par{9. ひとしお ${\overset{\textnormal{ざんしょ}}{\text{残暑}}}$ の厳しき折、くれくれもご自愛くださいますように。(手紙文) \hfill\break
All the more in this heat of late summer, be sure to take good care of yourself. }

\par{\textbf{Citation Note }: These last three examples are from 日本語類義表現 使い分け辞典 by 泉原省二. }

\par{10. 一昨日、古本屋に寄った\{際・折りに\}、注文しておきましたわよ。 \hfill\break
I ordered it [when\slash on the opportunity of] [I stopped\slash stopping] by at the old book store two days ago. \hfill\break
\hfill\break
11. 見ない\{〇とき・X 際・X 折\}も、テレビを消してね。 \hfill\break
Turn off the TV even when you're not watching it. }
      
\section{~に際して}
 
\par{ ~に際して has the nuance of “in facing a certain circumstance, on the occasion of that…”. It is rather formal and is often used in the preface of a salutation message. Having said this, though, there are instances where it can be interchanged with ~際に. In such a case, you are merely switching out something that is more so just a very formal fashion of saying ~とき. }

\par{\textbf{Formality Note }: Of course, there are the variants ~に際し ~に際しまして }

\par{12. 出発に際して、一言ご挨拶申し上げます。 \hfill\break
I will give a word of salutations when we depart. }

\par{13. 入学\{に際して・の際に\} ${\overset{\textnormal{きふきん}}{\text{寄附金}}}$ が要ると聞いているんですが、本当ですか。 \hfill\break
We have been hearing that a donation is needed on admission, but is that true? }

\par{14. この度の震災に際しまして \hfill\break
On the occasion of this earthquake disaster }

\par{ \textbf{Grammar Note }: ~うえで is more “static” in showing abstract static processes than even ~に際して. You will see later in this lesson how static versus dynamic makes in comparing these phrases with ~に当たって. In this sense, ~うえで is perfect in situations where one is giving an explanation to the listener. }

\par{15a. ${\overset{\textnormal{きょうぎしょ}}{\text{協議書}}}$ を ${\overset{\textnormal{か}}{\text{交}}}$ わすに際して、 ${\overset{\textnormal{ごけいびょうどう}}{\text{互恵平等}}}$ の原則を確認しておきたい。 \hfill\break
15b. 協議書を交わすうえで、互恵平等の原則を確認しておきたい。 \hfill\break
On exchanging agreements, I want to confirm the principles of mutual impartiality. \hfill\break
From 日本語類義表現 使い分け辞典 by 泉原省二. }

\par{16. }

\par{ドアを開けると、吉田は部下を従えて、妙に媚びるような表情で立っていた。 \hfill\break
「昨日の ${\overset{\textnormal{マルヒ}}{\text{被疑者}}}$ なんですが、森井さんじゃないと、話したくないと言い張っとるんです」 \hfill\break
挨拶もそこそこに、吉田は昨日の青年について語り始めた。彼は、事情聴取に際して、是非とも勲に会いたいと言っているのだという。 \hfill\break
「冗談じゃない、俺はもう警察とは無関係の人間だ」 \hfill\break
「ところが、 ${\overset{\textnormal{マルガイ}}{\text{被害者}}}$ にも来てもらってるんですけど、そっちの方も、そうなんです。『あのおじさんになら話す』ってね」 \hfill\break
When he opened the door, Yoshida with his men was standing with a look like he was trying to oddly curry favor. \hfill\break
"About yesterday's suspect, he insisted that if it's not Mr. Morii, he wouldn't want to talk." \hfill\break
With some salutations, Yoshida began to talk about the youth from yesterday. He, on the occasion of the police interview, [the suspect] said that he by all means wanted to meet with Isao. \hfill\break
"No joke, I'm no longer a person involved with the police anymore" \hfill\break
"On the contrary, we've had the victim come too, and that person's the same, "If it's the old man, I'll talk". }

\begin{center}
\textbf{警察隠語 Police Lingo }
\end{center}

\par{ This passage from 冷たい誘惑 by 乃南アサ introduces some rather interesting police jargon. 被疑者 and 被害者 are typically read as ひぎしゃ and ひがいしゃ respectively, but the author decided to use the police terminology instead, given that one speaker is a current officer and the other is an ex-officer. Aside from these words, there are several other words used by policemen that are of benefit of you to know, especially if you like watching Japanese cop films. }

\begin{ltabulary}{|P|P|P|P|P|P|}
\hline 

警察用語 & 一般語 & 警察用語 & 一般語 & 警察用語 & 一般語 \\ \cline{1-6}

ヤサ & 家 & 害者 & 被害者 & PM & 警察官 \\ \cline{1-6}

PC & パトカー & タタキ & 強盗 & レツ & 共犯者 \\ \cline{1-6}

アカイヌ & 放火 & H & ヘロイン & シャブ & 覚醒剤 \\ \cline{1-6}

飛ぶ & 逃げる & デカ & 刑事 & ハコ & 交番 \\ \cline{1-6}

\end{ltabulary}

\par{For more words, see http:\slash \slash sumim.no-ip.com\slash wiki\slash 1167  . }
      
\section{~に当たって}
 
\par{ There is also interchangeability with ~に際して and ~に当たって in situations when A is positively received. For instance, just like the first example with ~に際して, you can also see something like the following. }

\par{17. 開会に\{あたって・あたりまして・際し・際しまして\}、一言ご挨拶申し上げます。 \hfill\break
On the occasion of opening session, I will give a word of salutations. }

\par{18. 年頭に\{あたって・際して\}の今年の抱負は、何よりも禁煙、ただ禁煙あるのみ。 \hfill\break
My ambition for this year marking the beginning of it is above all is not smoking, simply no smoking. }

\par{\textbf{Usage Warning }: A sentence like the last one would not be used in the spoken language. Due to the formality and direct tone of the speech modals, such phrases would be limited to journals or soliloquies where such statements are expected. }

\par{However, their exact meanings cause them to not be 100\% interchangeable. }

\begin{ltabulary}{|P|P|}
\hline 

Aにあたって(の) & The time right before A starts\slash Before the realization of a basis time B \\ \cline{1-2}

Aに際して(の) & The same time as A starts\slash The same time as the basis time B \\ \cline{1-2}

\end{ltabulary}

\par{ For this comparison, more grammar-study terms are used to show the full implications of these two speech modals. The first simply states that there is a time that coincides with A. This time precedes A, and B is some defined time to happen and is not realized. However, with ~に際して these time are in unison. }

\par{ Consider the examples we've seen with them. So, even when they're interchangeable, this clear distinction holds. So, consider the example when giving salutations at departure. }

\par{19. 出発に際して一言ご挨拶申し上げます。 \hfill\break
Allow me to say a few words as I\slash we depart. }

\par{ You can vision the people leaving still giving their farewells as they, say, drive off or board a plane. Now, what would you end up saying if you rephrased it with に当たって? }

\par{20. 出発に当たって一言ご挨拶申し上げます。 \hfill\break
Allow me to say a few words in lieu of my\slash our departure. }

\par{ After all, there are very similar situations using words like 開会 where they are interchangeable. So, one would think that 出発 would be OK as well. It would, but you then describe a situation more like giving farewells right before departure. However, there are instances where one or the other is only right because of this minor detail. }

\par{21. ${\overset{\textnormal{じゅけいしゃ}}{\text{受刑者}}}$ は、 ${\overset{\textnormal{しょけい}}{\text{処刑}}}$ に際し、 ${\overset{\textnormal{ぼくし}}{\text{牧師}}}$ に ${\overset{\textnormal{ざんげ}}{\text{懺悔}}}$ することが許されている。\textrightarrow  にあたり X \hfill\break
Convicts are allowed to repent to a pastor on their execution. \hfill\break
From 日本語類義表現 使い分け辞典 by 泉原省二. }

\par{\textbf{漢字 Note }: 懺 is not a 常用漢字. }

\par{22. 明日の試合出場にあたって、チーム全員はすでに ${\overset{\textnormal{いちがん}}{\text{一丸}}}$ となっている。 \hfill\break
All of the team members are in a body for the occasion of the game appearance tomorrow. \hfill\break
From 日本語類義表現 使い分け辞典 by 泉原省二. }

\par{\textbf{Formality Note }: As to be expected, ~に当たり (literary) and ~に当たりまして (very respectful) exist. }

\par{Another thing to note is that ~にあたって is dynamic whereas ~に際して is static. ~にあたって comes from 当たる, so a meaning of “coping\slash grasping with” is hidden. 際, which is also read as きわ in other instances, refers to when you have no choice but to put up with an extraordinary time of some sort. }

\par{23. 実験にあたっては、事故を起こさないように注意してください。 \hfill\break
Please be careful to not cause an accident on the experiment. \hfill\break
From 日本語類義表現 使い分け辞典 by 泉原省二. }

\par{24. 実験に際しては、事故の起こらないように注意してください。 \hfill\break
Please be careful to make sure that an accident is not caused during the experiment. \hfill\break
From 日本語類義表現 使い分け辞典 by 泉原省二. }

\par{ Notice how the implications of these sentences change. The first sounds like the speaker is giving a verbal warning before a potentially risky experiment, but the second sounds like it is a more indirect yet important advice written down for during the experiment. }

\begin{center}
\textbf{More Examples }
\end{center}

\par{25. 話の終わりに当たって、皆様方のご協力に感謝いたします。 \hfill\break
Lastly, I want to thank all of you for your cooperation. }

\par{26. 発電所の建設に当たって多くの ${\overset{\textnormal{しょうがい}}{\text{障害}}}$ に出くわす。 \hfill\break
To meet many difficulties in building a power plant. }

\par{27. この ${\overset{\textnormal{しせつ}}{\text{施設}}}$ を利用するに当たって注意すべきこと! \hfill\break
You must be careful when utilizing this facility! }

\par{28. 別れに際して泣いた。 \hfill\break
I cried at our parting. }

\par{29. 妻の死に際して \hfill\break
On the death of my wife }

\par{\textbf{Phrase Note }: ~に当たらない, on the other hand, is similar to "no" as in not hitting a certain degree. }

\par{30. ${\overset{\textnormal{きょぜつ}}{\text{拒絶}}}$ は驚くに当たらなかった。 \hfill\break
The refusal was not a surprise. }

\par{31. 驚くには当たらないと言わぬばかりに。 \hfill\break
As if to say there is no use getting alarmed. }

\begin{center}
\textbf{~に当たって VS ~うえで }
\end{center}

\par{ ~うえで is extremely indicative of the written language, which is not a characteristic of ~にあたって. For reiteration, ~うえで is used when stating a situation where “an important A, in the course from before you start it and till it ends, B may occur” and there is a “point of interest\slash necessary condition” that follows (the B). }
 
\par{ Take the first example sentence of this lesson into second consideration. }
 
\par{32. 研究を続けるうえで、中国に留学したほうがいいと判断した。 \hfill\break
I have concluded that it is best for me to study abroad in China on the basis of continuing my studies. }
 
\par{ The event A is “continuing one\textquotesingle s studies”. The B can't be just a statement of explanation. Rather, it must be a point of concern: 注意点、問題点, 重要ポイント, 必要条件, etc. }
 
\par{\textbf{Style Note }: As far as the distinction between 話し言葉 and 書き言葉 in regards to this usage of ~うえで, it\textquotesingle s not that it is a complete 0\%\slash 100\% divide. In circumstances such as speaking to a professor, more literary expressions are common as there is an academic atmosphere. This, though, blurs the divide. }

\par{33. }
 
\par{A: 先生、論文を書くうえで、どのようなことに注意したらいいでしょうか。 \hfill\break
B: まず、今までにどこまで研究 されている かをしっかり ${\overset{\textnormal{つか}}{\text{摑}}}$ む ことだ ね。 \hfill\break
From 中級日本語文法と教え方のポイント. }
 
\par{ Here, Student A asks what to pay attention to in [writing his\slash her dissertation] (A). The 問題点 of writing the dissertation is the B (necessary condition\slash important contingency). The professor responds by stating that first he\slash she needs to fully grasp what he\slash she has studied up to that point. Other important things to notice in the dialogue are underlined. }
 
\par{34. 実験する\{〇 にあたって・X\slash △ うえで\}、 ${\overset{\textnormal{きけんぶつ}}{\text{危険物}}}$ の ${\overset{\textnormal{あつか}}{\text{扱}}}$ いには、十分に注意してください。(話し言葉) \hfill\break
On the experiment, be careful enough in the treatment of hazards. \hfill\break
From 日本語類義表現 使い分け辞典 by 泉原省二. }

\par{35. 実験するうえで、 ${\overset{\textnormal{かき}}{\text{下記}}}$ の ${\overset{\textnormal{やくひん}}{\text{薬品}}}$ の取り扱いには、特に注意してください。(書き言葉) \hfill\break
On the experiment, pay special attention to the handling of the following chemicals. \hfill\break
From 日本語類義表現 使い分け辞典 by 泉原省二. }
 
\par{ Although the first sentence meets the grammatical needs of using ~うえで, it is very unnatural at best because of style conflict. When rephrased to fix this, the unnaturalness goes away. Another major grammatical difference is that ~うえで is not limited to special circumstances like ~に際してand ~に当たって are. In such case, ~うえで becomes the only option, and as a consequence, is even common in the spoken language in this regard. }
 
\par{36. 数のうえでは ${\overset{\textnormal{あっとうてき}}{\text{圧倒的}}}$ な状態だ。 \hfill\break
It's an overwhelming situation concerning the numbers. }
 
\par{37. 十分に事件を ${\overset{\textnormal{かんあん}}{\text{勘案}}}$ したうえで返事します。 \hfill\break
I will reply after sufficiently considering the matter. }
 
\par{38. 年齢を重ねていくうえで、忘れてならないのは、友への感謝であろう。 \hfill\break
As you age, what one must not forget is admiration towards friends. \hfill\break
From 日本語類義表現 使い分け辞典 by 泉原省二. }
      
\section{~に先立って}
 
\par{ Just as with ~にあたって, you can see ~に先立って in situations such as the following. }

\par{39. 開会に先立ちまして、一言ご挨拶申し上げます。 \hfill\break
In advance of the opening of session, I will give a word of salutations. \hfill\break
From 日本語類義表現 使い分け辞典 by 泉原省二. }

\par{ Aに先立ってB gives a meaning of “carrying out B in the stage before the start of A, which is important and different than the norm”. Thus, it is seen a lot in speeches and articles. It is only used in situations where there is a “temporal ranking” of some sort. There is some sort of ritual order being expressed between A and B. Thus, the time before A and B is comparatively much larger than when using ~に当たって. \hfill\break
 \hfill\break
 Given the meaning of 先立つ, to precede, it can be used to refer to the passing of someone when used in the passive. }

\par{40. 最愛の妻に先立たれた。 \hfill\break
I was died on by my beloved wife. \hfill\break
\textbf{\hfill\break
Nuance Note }: The translation has a more morbid tone that the original. }
      
\section{~を前に(して) \& ~を控えて}
 
\par{ These phrases are quite similar to each other, and they are even interchangeable when expressing a situation right before an event. However, they differ in two crucial aspects. }

\begin{ltabulary}{|P|P|}
\hline 

B~を前に(して)A & Marks the realistic moment before the beginning of an event. \\ \cline{1-2}

B~を控えてA & Marks a psychological moment before with relative disregard to time. \\ \cline{1-2}

\end{ltabulary}

\par{ Both are indicative of the written language and both describe a third person. If the speaker becomes the subject, then the speaker becomes the object of description. }

\par{\textbf{Base Note }: The 連体形 of these phrases are ~を前にした and ~を控えた respectively. }

\par{41. 筆記試験を終え、面接試験を\{まえにした・控えた\} ${\overset{\textnormal{おうぼしゃ}}{\text{応募者}}}$ は ${\overset{\textnormal{きんちょう}}{\text{緊張}}}$ した面持ちだ。 \hfill\break
Finishing the written examination, the applicants have a nervous face before the interview   examination. \hfill\break
From 日本語類義表現 使い分け辞典 by 泉原省二. }

\par{42. 期末試験を控えて学生は忙しそうだ。 \hfill\break
The students seemed busy with the final exam at hand. }

\par{ The next sentence demonstrates that 控える doesn't have to be used like above. It also has the meaning of “to hesitate\slash hold off”. Interchanging it, then, with ~を前に is then solely dependent on the logicality of the events and time. }

\par{43. 入試を一年後に\{〇 控えて・X 前に(して)\}ようやく ${\overset{\textnormal{せけんな}}{\text{世間並}}}$ みの ${\overset{\textnormal{じゅけんせい}}{\text{受験生}}}$ らしくなった。 \hfill\break
After of holding off the entrance exam, I have finally become like a normal examinee. \hfill\break
From 日本語類義表現 使い分け辞典 by 泉原省二. }

\par{ There are also instances when they are used with place, in which they are never interchangeable and therefore may even appear in the same sentence. }

\par{44. 彼の新居は、海岸をまえに、後ろに ${\overset{\textnormal{さんみゃく}}{\text{山脈}}}$ を控えた高台に位置していた。 \hfill\break
His new home was placed on top of an overlook with mountain chains close in the back and the shore before it. }
      
\section{~に臨んで \& ~に面して}
 
\par{ To finish off this lesson, consider the following speech modals: ~に臨んで, ~に面して. }

\begin{ltabulary}{|P|P|}
\hline 

Aに臨んで & On the occasion of participation in a positive\slash negative A… \\ \cline{1-2}

Aに面して & On the occasion of an extreme, negative situation A faced\dothyp{}\dothyp{}\dothyp{} \\ \cline{1-2}

\end{ltabulary}

\par{ They are basically interchangeable in emergency situations. Aに臨んで, however, shows that one is participating in an event A, whether it being present in that scene, going to such place, or forwarding oneself on. }

\par{ Oppositely, ~に面して shows that one is suddenly or passively faced with a situation. So, with crisis situations concerning things like war, failure, etc., it is often the case that the former would demonstrate a strong person in spite of the crisis and the latter would give images of a weak person in the upset. }

\par{\textbf{Base Note }: The 連体形 of these phrases are ~に臨んだ and ~に面した respectively. }

\par{45. 母親の急死に\{臨んで・面して\}、それまで仲のよかった兄たちは、 ${\overset{\textnormal{いさか}}{\text{諍}}}$ いを始めた。 \hfill\break
In facing our mother\textquotesingle s sudden death, my older brothers who had been cordial up till then began a quarrel. }

\par{46. 経済的危機に\{〇面しても、X臨んでも\}、冷静に適切な行動のとれる人を尊敬しています。 \hfill\break
I respect those that take the appropriate actions calmly even while facing economic crisis. }

\par{47. 第一試合に\{〇臨んで・X面して\}、緊張を隠しきれなかった。 \hfill\break
I couldn\textquotesingle t completely hide my nervousness in facing the first match. }

\par{48. 緊急事態に\{臨めば臨む・面すれば面する\}ほど、 ${\overset{\textnormal{ひぼん}}{\text{非凡}}}$ な力を ${\overset{\textnormal{はっき}}{\text{発揮}}}$ する者がいる。 \hfill\break
There are people that exercise extraordinary abilities the more they face state of emergencies. }

\par{49. 難に臨んで ${\overset{\textnormal{にわ}}{\text{遽}}}$ かに ${\overset{\textnormal{へい}}{\text{兵}}}$ を ${\overset{\textnormal{い}}{\text{鋳}}}$ る \hfill\break
Literally: To cast soldiers hastily in facing a crisis. }

\par{ \textbf{Phrase Note }: This is a set phrase coming from the concept of making weapons quickly after a war has started. In a practical sense, it refers to situations where one doesn't make it in time even if you hurriedly prepare after an incident has started. }

\par{ In the location sense, the difference is from the speaker\textquotesingle s position. With the former, you are looking down at a vast space from above. With the latter, you are seeing before one\textquotesingle s eyes a space expanding on the front sides. So, no matter if you interchange them, the speaker\textquotesingle s visual position and the vertical direction of the scenery is different. Note, though, that when deviating from pinpointing direction, のぞむ is spelled as 望む to signify actually looking down. }

\par{50. そのホテルは、瀬戸内海海に臨んで、高台に立っている。 \hfill\break
The hotel faces the Seto Inland Sea and stands on the height. \hfill\break
\hfill\break
51. その高層ビルは、海に面して、遊歩道に立っている。 \hfill\break
The high rise building faces the ocean and stands on the boardwalk. }

\par{52. 街を望む。 \hfill\break
To look down at the street\slash town. }
    