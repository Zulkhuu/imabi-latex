    
\chapter{Reduplication}

\begin{center}
\begin{Large}
第260課: Reduplication: Nouns 
\end{Large}
\end{center}
 
\par{ The reduplication of words ( \emph{jōgo }畳語) is a phenomenon found in Japanese in which the same morpheme (unit of meaning) is doubled to create a new yet related word. This is seen virtually across all parts of speech in Japanese to varying degrees. Before you think this is only limited to some form of slang, it must be noted that most words created by doubling something are actually quite important. }

\par{ When a noun is doubled, the resultant phrase is a plural phrase denoting a large variety of said noun. Essentially, only nouns that have been conceptualized by the Japanese as something that is both numerous and highly varied may be pluralized in this way. }

\par{i. ${\overset{\textnormal{やま}}{\text{山}}}$ \hfill\break
 \emph{Yama }\hfill\break
Mountain }

\par{ii. ${\overset{\textnormal{やまやま}}{\text{山々}}}$ \hfill\break
 \emph{Yamayama }\hfill\break
Many mountains }

\par{ As simple as this may seem, it is important to note that only a handful of these phrases are frequently used, and it isn\textquotesingle t even the case that all related words of a particular variety can be pluralized in this way. For instance, although \emph{yamayama }山々 is used, \emph{kawagawa }川々 (rivers) hardly ever used, and \emph{okaoka }岡々 (hills) is basically unheard of. }

\par{ One thing to take especial note of regarding doubling nouns is that a phonological phenomenon called \emph{rendaku }連濁 affects the pronunciation of the repeated element. \emph{Rendaku }連濁 is when the second part of a compound has its initial consonant voiced if it isn\textquotesingle t already. }

\par{iii. ${\overset{\textnormal{くに}}{\text{国}}}$ \hfill\break
 \emph{Kuni }\hfill\break
Country }

\par{iv. ${\overset{\textnormal{くにぐに}}{\text{国々}}}$ \hfill\break
 \emph{Kuni \textbf{guni }}\hfill\break
Many countries }

\par{ As far as spelling is concerned, rather than repeating the same character, the ditto character 々 is used instead. However, when a word of more than one character is repeated, the whole word is usually repeated instead of using the ditto mark. }

\par{v. ${\overset{\textnormal{しまじま}}{\text{島}}}$ \hfill\break
 \emph{Shimajima }\hfill\break
Many islands }

\par{vi. ${\overset{\textnormal{こうたいごうたい}}{\text{交代}}}$ \hfill\break
 \emph{K }\emph{ōtaig }\emph{ōtai \hfill\break
 }Shift after shift }

\par{ Unfortunately, instances of these phrases must be learned on a case by case basis, and because these phrases are so closely intertwined into the very conceptualization of core vocabulary, a good handful of the phrases that do exist have quirks that must also be addressed. For starters, we will look at the most common, straightforward instances of noun duplication. }
      
\section{Examples}
 
\par{1. ${\overset{\textnormal{さんじゅっ}}{\text{30}}}$ ${\overset{\textnormal{か}}{\text{ヶ}}}$ ${\overset{\textnormal{こくいじょう}}{\text{国以上}}}$ の ${\overset{\textnormal{ひとびと}}{\text{人々}}}$ \textbf{に }${\overset{\textnormal{いけん}}{\text{意見}}}$ を ${\overset{\textnormal{ちょうさ}}{\text{調査}}}$ しました。 \hfill\break
 \emph{Sanjukkakoku ij }\emph{ō no \textbf{hitobito }ni iken wo ch }\emph{ōsa shimashita. }\hfill\break
(I\slash we) have investigated the opinions of \textbf{(many) people }from over thirty nations. }

\par{2. ${\overset{\textnormal{にほん}}{\text{日本}}}$ の ${\overset{\textnormal{やまやま}}{\text{山々}}}$ を ${\overset{\textnormal{たの}}{\text{楽}}}$ しみましょう。 \hfill\break
 \emph{Nihon no \textbf{yamayama }wo tanoshimimash }\emph{ō. \hfill\break
 }Enjoy \textbf{the many } \textbf{mountains }of Japan. }

\par{3. アフリカの ${\overset{\textnormal{くにぐに}}{\text{国々}}}$ までもが ${\overset{\textnormal{ちゅうごく}}{\text{中国}}}$ に ${\overset{\textnormal{いそん}}{\text{依存}}}$ している。 \hfill\break
 \emph{Afurika no \textbf{kuniguni }\textbf{ }made mo ga ch }\emph{ūgoku ni ison shite iru. }\hfill\break
As far as \textbf{the nations }of Africa are dependent on China. }

\par{\textbf{Reading Note }: 依存 may alternatively be pronounced as \emph{izon }. }

\par{4. ${\overset{\textnormal{こだい}}{\text{古代}}}$ のギリシャ ${\overset{\textnormal{じん}}{\text{人}}}$ は、ゼウスをはじめ、 ${\overset{\textnormal{おお}}{\text{多}}}$ くの ${\overset{\textnormal{かみがみ}}{\text{神々}}}$ を ${\overset{\textnormal{あが}}{\text{崇}}}$ めていたといわれている。 \hfill\break
 \emph{Kodai no girishajin wa, zeusu wo hajime, }\emph{ōku no \textbf{kamigami }wo agamete ita to iwarete iru. }\hfill\break
It is said that the ancient Greeks, not only worshiped Zeus, but they also worshiped \textbf{many } \textbf{other gods }. }

\par{5. ${\overset{\textnormal{こうだい}}{\text{広大}}}$ な ${\overset{\textnormal{うちゅう}}{\text{宇宙}}}$ に ${\overset{\textnormal{ち}}{\text{散}}}$ らばる、 ${\overset{\textnormal{かぞ}}{\text{数}}}$ え ${\overset{\textnormal{き}}{\text{切}}}$ れないほどの ${\overset{\textnormal{ほしぼし}}{\text{星々}}}$ へ ${\overset{\textnormal{たびだ}}{\text{旅立}}}$ ちましょう。 \hfill\break
\textbf{ }\emph{K }\emph{ōdai na uch }\emph{ū ni chirabaru, kazoekirenai hodo no \textbf{hoshiboshi }e tabidachimash }\emph{ō. }\hfill\break
Let\textquotesingle s embark on exploring \textbf{the }countless \textbf{stars }scattered in our grand universe. }

\par{6. ここ ${\overset{\textnormal{すうじつ}}{\text{数日}}}$ 、 ${\overset{\textnormal{こうたいごうたい}}{\text{}}}$ で ${\overset{\textnormal{やす}}{\text{休}}}$ みなく ${\overset{\textnormal{さぎょう}}{\text{作業}}}$ を ${\overset{\textnormal{つづ}}{\text{続}}}$ けていました。 \hfill\break
\textbf{ }\emph{Koko s }\emph{ūjitsu, \textbf{k }}\textbf{\emph{ōtaig }}\textbf{\emph{ōtai }}\emph{de yasumi naku sagy }\emph{ō wo tsuzukete imashita. }\hfill\break
For these past few day, I have been doing work without break \textbf{shift after shift }. }

\par{7. ${\overset{\textnormal{みなみたいへいよう}}{\text{南太平洋}}}$ の ${\overset{\textnormal{しまじま}}{\text{島々}}}$ に ${\overset{\textnormal{りょこう}}{\text{旅行}}}$ してみたいと ${\overset{\textnormal{おも}}{\text{思}}}$ います。 \hfill\break
\textbf{ }\emph{Minami Taiheiy }\emph{ō no \textbf{shimajima }ni ryok }\emph{ō shite mitai to omoimasu. \hfill\break
 }I\textquotesingle d like to travel \textbf{the } \textbf{many islands }of the South Pacific. }

\par{8. ${\overset{\textnormal{われわれ}}{\text{我々}}}$ はどこから ${\overset{\textnormal{き}}{\text{来}}}$ たのか、 ${\overset{\textnormal{われわれ}}{\text{我々}}}$ は ${\overset{\textnormal{なにもの}}{\text{何者}}}$ か、 \textbf{${\overset{\textnormal{われわれ}}{\text{我々}}}$ は }どこへ ${\overset{\textnormal{い}}{\text{行}}}$ くのか。 \hfill\break
 \emph{\textbf{Wareware }wa doko kara kita no ka, \textbf{wareware }wa nanimono ka, \textbf{wareware }wa doko e iku no ka. }\hfill\break
Where did \textbf{we }come from, what are \textbf{we }, and where are \textbf{we }going? }

\par{9. ${\overset{\textnormal{あし}}{\text{足}}}$ を ${\overset{\textnormal{はこ}}{\text{運}}}$ んでくださった ${\overset{\textnormal{かたがた}}{\text{方々}}}$ 、 ${\overset{\textnormal{まこと}}{\text{誠}}}$ にありがとうございました。 \hfill\break
 \emph{Ashi wo hakonde kudasatta \textbf{katagata }, makoto ni arigat }\emph{ō gozaimashita. }\hfill\break
To \textbf{all those }who turned out, (I\slash we) sincerely thank you. }

\par{11. トルコ ${\overset{\textnormal{かくち}}{\text{各地}}}$ の ${\overset{\textnormal{まちまち}}{\text{町々}}}$ を ${\overset{\textnormal{たず}}{\text{訪}}}$ ねました。 \hfill\break
 \emph{Toruko kakuchi no \textbf{machimachi }wo tazunemashita. \hfill\break
 }I visited \textbf{the many towns }across Turkey. }

\par{12. ${\overset{\textnormal{かんきこう}}{\text{換気口}}}$ が ${\overset{\textnormal{ところどころ}}{\text{所々}}}$ に ${\overset{\textnormal{せっち}}{\text{設置}}}$ されている。 \hfill\break
 \emph{Kankik }\emph{ō ga \textbf{tokorodokoro }ni setchi sarete iru. }\hfill\break
Vents are installed \textbf{here and there }. }

\par{13. ${\overset{\textnormal{がいろじゅ}}{\text{街路樹}}}$ の ${\overset{\textnormal{きぎ}}{\text{木々}}}$ が ${\overset{\textnormal{いろ}}{\text{色}}}$ づいて ${\overset{\textnormal{きれい}}{\text{綺麗}}}$ ですね。 \hfill\break
 \emph{Gairoju no \textbf{kigi }ga irozuite kirei desu ne. }\hfill\break
The many roadside \textbf{trees }have turned colors and are lovely. }

\par{14. ${\overset{\textnormal{せかい}}{\text{世界}}}$ の ${\overset{\textnormal{すみずみ}}{\text{隅々}}}$ まで ${\overset{\textnormal{かもつ}}{\text{貨物}}}$ を ${\overset{\textnormal{はこ}}{\text{運}}}$ ぶ。 \hfill\break
 \emph{Sekai no \textbf{sumizumi }made kamotsu wo hakobu. }\hfill\break
To transport cargo to \textbf{the corners }of the world. }

\par{15. ${\overset{\textnormal{いちねん}}{\text{一年}}}$ を ${\overset{\textnormal{とお}}{\text{通}}}$ して ${\overset{\textnormal{きせつ}}{\text{季節}}}$ の ${\overset{\textnormal{はなばな}}{\text{花々}}}$ を ${\overset{\textnormal{さいばい}}{\text{栽培}}}$ しています。 \hfill\break
 \emph{Ichinen wo t }\emph{ōshite kisetsu no \textbf{hanabana }wo saibai shite imasu. }\hfill\break
(I\slash we) cultivate \textbf{the various flowers }of the seasons throughout the year. }

\par{16. ${\overset{\textnormal{かれ}}{\text{彼}}}$ らは ${\overset{\textnormal{みな}}{\text{皆}}}$ \textbf{それぞれ }の ${\overset{\textnormal{いけん}}{\text{意見}}}$ を ${\overset{\textnormal{も}}{\text{持}}}$ っています。 \hfill\break
 \emph{Karera wa mina \textbf{sorezore }no iken wo motte imasu. }\hfill\break
They all \textbf{each }have their own opinions. }

\par{17. ${\overset{\textnormal{がくせいおのおの}}{\text{学生}}}$ が ${\overset{\textnormal{たの}}{\text{楽}}}$ しみながら ${\overset{\textnormal{にほんご}}{\text{日本語}}}$ の ${\overset{\textnormal{かいわりょく}}{\text{会話力}}}$ を ${\overset{\textnormal{たか}}{\text{高}}}$ めている。 \hfill\break
 \emph{Gakusei \textbf{ono\textquotesingle ono }ga tanoshiminagara Nihongo no kaiwaryoku wo takamete iru. \hfill\break
 }\textbf{Each and every }student is increasing their conversation skills in Japanese while having fun. }

\par{\textbf{Word Note }: \emph{Ono }actually comes from an old word meaning “oneself” which is still seen in the word \emph{onore }己, which is either used to mean “oneself\slash itself” or as a derogatory “you,” of all things. }

\par{18. ${\overset{\textnormal{めいめい}}{\text{銘々}}}$ が ${\overset{\textnormal{べんとう}}{\text{弁当}}}$ を ${\overset{\textnormal{じさん}}{\text{持参}}}$ してください。 \hfill\break
 \textbf{\emph{Meimei }}\emph{ga bento wo jisan shite kudasai. }\hfill\break
\textbf{ }May \textbf{each }please bring his own bento. }

\par{\textbf{Word Note }: \emph{Ono\textquotesingle ono }各々 and \emph{meimei }銘々 are both very similar to \emph{sorezore }それぞれ. Neither, unlike \emph{sorezore }それぞれ, are particularly used in the spoken language anymore, but they both only refer to people. }

\par{19. アイドルグループの○○の ${\overset{\textnormal{めんめん}}{\text{面々}}}$ ${\overset{\textnormal{じゅうしょう}}{\text{重傷}}}$ ! \hfill\break
 \emph{Aidoru gurūpu no }\emph{marumaru }\emph{\textbf{ }no \textbf{memmen }jūshō! }\hfill\break
\textbf{Each Member }of Pop Group \textbf{\#\# }Severely Injured! }

\par{\textbf{Word Note }: \emph{Memmen }面々 is yet another word meaning “each one.” It is occasionally used in the written language; however, it is not suited for polite\slash honorific speech as the tone it gives is rather indifferent at a respectful level. It also literally means "every direction," but this meaning is obsolete and would be replaced with phrases like \emph{kaku hōmen }各方面. }

\par{20. ${\overset{\textnormal{てづく}}{\text{手作}}}$ りの ${\overset{\textnormal{しなじな}}{\text{}}}$ が ${\overset{\textnormal{そろ}}{\text{揃}}}$ っている。 \hfill\break
 \emph{Tezukuri no }\emph{shinajina }\emph{\textbf{ }ga sorotte iru. }\hfill\break
 \textbf{Various }\textbf{ }handmade \textbf{goods }\textbf{ }are lined up. }

\par{21. ${\overset{\textnormal{かぜ}}{\text{風邪}}}$ で ${\overset{\textnormal{ふしぶし}}{\text{節々}}}$ が ${\overset{\textnormal{いた}}{\text{痛}}}$ むのはなぜでしょうか。 \hfill\break
 \emph{Kaze de \textbf{fushibushi }\textbf{ }ga itamu no wa naze desh }\emph{ō }\emph{ka? \hfill\break
 }Why do all one\textquotesingle s joints \textbf{ }ache with a cold? }

\par{22. ${\overset{\textnormal{たで}}{\text{蓼}}}$ ${\overset{\textnormal{く}}{\text{食}}}$ う ${\overset{\textnormal{むし}}{\text{虫}}}$ も ${\overset{\textnormal{す}}{\text{好}}}$ \textbf{き ${\overset{\textnormal{ず}}{\text{好}}}$ \textbf{き }。 \hfill\break
 } \emph{Tade k }\emph{ū }\emph{mushi mo \textbf{sukizuki }\textbf{. }}\hfill\break
\textbf{ }Some prefer nettles\slash every man has his \textbf{taste }\textbf{. }}

\par{\textbf{Phrase Note }: This is a set expression which literally means, “There is also a matter of taste even for bugs that eat knotweed.” }

\par{23. ${\overset{\textnormal{つぎ}}{\text{次}}}$ は ${\overset{\textnormal{だれだれ}}{\text{誰々}}}$ です。 \hfill\break
 \emph{Tsugi wa \textbf{daredare }desu. }\hfill\break
Next is \textbf{so-and-so }\textbf{. }}

\par{\textbf{Word Note }: In addition to meaning "so-and-so," this word traditionally has also been used to mean “who” but in the sense of two or more people. This usage, however, has waned and hardly anyone uses it this way anymore. }

\begin{center}
 \textbf{Peculiar Examples }
\end{center}

\par{ As you can see, all the examples thus far utilize simple yet fundamental nouns in the language. The examples to follow are also very simple and important nouns, but their duplicated forms are either not as common or have something odd about them. }

\par{24. ${\overset{\textnormal{ひにく}}{\text{皮肉}}}$ なことに、 ${\overset{\textnormal{ぐんまけん}}{\text{群馬県}}}$ の ${\overset{\textnormal{むらむら}}{\text{村々}}}$ には、 ${\overset{\textnormal{しろくじじゅう}}{\text{四六時中}}}$ \textbf{ムラムラ }している ${\overset{\textnormal{そんみん}}{\text{村民}}}$ が ${\overset{\textnormal{おお}}{\text{多}}}$ いらしい。 \hfill\break
 \emph{Hiniku na koto ni, Gumma-ken no \textbf{muramura }ni wa, shirokujich }\emph{ū \textbf{muramura }shite iru sommin ga }\emph{ōi rashii. \hfill\break
 }Ironically, it seems that there are many villagers that are \textbf{horny }around the clock in the \textbf{villages }of Gunma Prefecture. }

\par{\textbf{Word Note }: Muramura 村々 would only be used in the written language because it is homophonous with the very common onomatopoeic expression \emph{muramura suru }ムラムラする, which means “to be horny.” Onomatopoeic expressions, as demonstrated with this mere example, also frequently exhibit duplication. }

\par{25. ${\overset{\textnormal{ちょうちょう}}{\text{蝶々}}}$ が ${\overset{\textnormal{むし}}{\text{虫}}}$ なのに ${\overset{\textnormal{きら}}{\text{嫌}}}$ われないのはなぜでしょうか。 \hfill\break
\textbf{ \emph{Ch }}\textbf{\emph{ōch }}\textbf{\emph{ō }}\emph{ga m \textbf{u }shi na noni kirawarenai no wa naze desh }\emph{ō ka? }\hfill\break
\textbf{ }Why is it that \textbf{butterflies }aren\textquotesingle t hated although they\textquotesingle re bugs? }

\par{\textbf{Word Note }: \emph{Ch }\emph{ōch }\emph{ō }蝶々 should just mean “(many) butterflies,” but it has ironically become detached from its literal meanings and can also just mean “butterfly.” }

\par{26. ${\overset{\textnormal{い}}{\text{行}}}$ ったことのない ${\overset{\textnormal{てらでら}}{\text{寺々}}}$ に ${\overset{\textnormal{さんぱい}}{\text{参拝}}}$ に ${\overset{\textnormal{で}}{\text{出}}}$ かける。 \hfill\break
 \emph{Itta koto no nai \textbf{teradera }\textbf{ }ni sampai ni dekakeru. }\hfill\break
To go out to pay homage to the many temples \textbf{ }one has never gone to. }

\par{27. \textbf{お }${\overset{\textnormal{てて}}{\text{手手}}}$ 、 ${\overset{\textnormal{ちょうだい}}{\text{頂戴}}}$ 。 \hfill\break
 \textbf{\emph{Otete }}\emph{, ch }\emph{ōdai. }\hfill\break
Give me your \textbf{hand(s)\slash paw(s) }. }

\par{\textbf{Word Note }: \emph{Otete }お手手 is a euphemism for “hand(s)\slash paw(s)” that is used towards children and pets. }

\par{ The following examples are indicative of when noun duplication would normally not be permissible; however, it is noteworthy that a far larger diversity of nouns can be duplicated in specialized contexts like those seen below. }

\par{28. ${\overset{\textnormal{ゆき}}{\text{雪}}}$ が ${\overset{\textnormal{き}}{\text{来}}}$ た。 ${\overset{\textnormal{たにだに}}{\text{谷々}}}$ は ${\overset{\textnormal{さんがつ}}{\text{三月}}}$ の ${\overset{\textnormal{よ}}{\text{余}}}$ も ${\overset{\textnormal{ふか}}{\text{深}}}$ く ${\overset{\textnormal{う}}{\text{埋}}}$ もれた。 \hfill\break
 \emph{Yuki ga kita. \textbf{Tanitani }\textbf{ }wa sangatsu no yo mo fukaku umoreta. \hfill\break
 }The snow came. The valleys \textbf{ }were deeply buried in it past March. \hfill\break
From 岩石の間 by 島崎藤村. }

\par{29. ${\overset{\textnormal{わたくし}}{\text{私}}}$ はほかの ${\overset{\textnormal{あな}}{\text{穴}}}$ を ${\overset{\textnormal{ちゅうい}}{\text{注意}}}$ して ${\overset{\textnormal{み}}{\text{見}}}$ た。そしてそれらの \textbf{${\overset{\textnormal{あなあな}}{\text{穴々}}}$ }が、いつの ${\overset{\textnormal{ま}}{\text{間}}}$ にか ${\overset{\textnormal{つぎつぎ}}{\text{次々}}}$ に ${\overset{\textnormal{ぬ}}{\text{塗}}}$ り ${\overset{\textnormal{かた}}{\text{固}}}$ められて ${\overset{\textnormal{い}}{\text{行}}}$ っているのを ${\overset{\textnormal{み}}{\text{見}}}$ た。 \hfill\break
 \emph{Watakushi wa hoka no ana wo ch }\emph{ū }\emph{i shite mita. Soshite sorera no \textbf{ana\textquotesingle ana }ga, itsu no ma ni ka \textbf{tsugitsugi }ni nurikatamerarete itte iru no wo mita. } \hfill\break
I looked cautiously at the \textbf{other holes }, and then I watched as \textbf{one after another }of those holes coated over before I knew it. \hfill\break
From ジガ蜂 by 島木健作. }

\par{\textbf{Word Note }: \emph{Tsugitsugi }次々 is an adverbial phrase meaning “one by one\slash one after another” by duplicating the noun \emph{tsugi }次 meaning “next.” }

\par{30. ${\overset{\textnormal{わたし}}{\text{私}}}$ を ${\overset{\textnormal{しん}}{\text{信}}}$ じる ${\overset{\textnormal{もの}}{\text{者}}}$ は、 ${\overset{\textnormal{せいしょ}}{\text{聖書}}}$ が ${\overset{\textnormal{い}}{\text{言}}}$ っているように、その ${\overset{\textnormal{ひと}}{\text{人}}}$ から ${\overset{\textnormal{い}}{\text{生}}}$ ける ${\overset{\textnormal{みず}}{\text{水}}}$ の ${\overset{\textnormal{かわがわ}}{\text{川々}}}$ が ${\overset{\textnormal{なが}}{\text{流}}}$ れ ${\overset{\textnormal{で}}{\text{出}}}$ るであろう。 \hfill\break
 \emph{Watashi wo shinjiru mono wa, seisho ga itte iru yō ni, sono hito kara ikeru mizu no \textbf{kawagawa }ga deru de arō. }\hfill\break
Whoever believes in me, as Scripture has said, \textbf{rivers }of living water will flow from him. \hfill\break
From John 7:38 }

\begin{center}
\textbf{Temporal Nouns: Nominal \& Adverbial }
\end{center}

\par{ Although we are focusing on instances of noun duplication, it is important to note that the majority of nouns that are temporal are often used as both nouns and adverbs, and this is no different when they\textquotesingle re duplicated. }

\par{31. ${\overset{\textnormal{しあわ}}{\text{幸}}}$ せな ${\overset{\textnormal{ひび}}{\text{日々}}}$ を ${\overset{\textnormal{す}}{\text{過}}}$ ごす。 \hfill\break
 \emph{Shiawase na \textbf{hibi }wo sugosu. }\hfill\break
To live out happy days . }

\par{32. ${\overset{\textnormal{てんぼうだい}}{\text{展望台}}}$ から ${\overset{\textnormal{しきおりおり}}{\text{四季}}}$ の ${\overset{\textnormal{うつく}}{\text{美}}}$ しさを ${\overset{\textnormal{たんのう}}{\text{堪能}}}$ できます。 \hfill\break
 \emph{Tembōdai kara shiki \textbf{oriori }no utsukishisa wo tan\textquotesingle nō dekimasu. \hfill\break
 }You can enjoy the beauty of the seasons \textbf{from season to season }from the observation deck. }

\par{33. ${\overset{\textnormal{しか}}{\text{鹿}}}$ も ${\overset{\textnormal{おりおり}}{\text{折々}}}$ ${\overset{\textnormal{み}}{\text{見}}}$ かけるが、カモシカは ${\overset{\textnormal{めずら}}{\text{珍}}}$ しい。 \hfill\break
 \emph{Shika mo \textbf{oriori }mikakeru ga, kamoshika wa mezurashii. }\hfill\break
Although I \textbf{occasionally }spot deer as well, wild goats are rare. }

\par{34. ${\overset{\textnormal{ときどき}}{\text{時々}}}$ 、コンロの ${\overset{\textnormal{ひ}}{\text{火}}}$ がつかない。 \hfill\break
 \textbf{ \emph{Tokidoki }}\emph{, konro no hi ga tsukanai. }\hfill\break
\textbf{Sometimes }, the gas burner doesn\textquotesingle t light. }

\par{35. ${\overset{\textnormal{あたまきん}}{\text{頭金}}}$ ゼロで ${\overset{\textnormal{つきづき}}{\text{月々}}}$ ${\overset{\textnormal{さん}}{\text{3}}}$ ${\overset{\textnormal{まんえんだい}}{\text{万円台}}}$ の ${\overset{\textnormal{しはら}}{\text{支払}}}$ いで ${\overset{\textnormal{こうにゅう}}{\text{購入}}}$ できます。 \hfill\break
 \emph{Atamakin zero de \textbf{tsukizuki }samman\textquotesingle en-dai no shiharai de kōnyū dekimasu. }\hfill\break
You can purchase with \textbf{monthly }payments in the 30,000 yen range with zero down payment. }

\par{36. ${\overset{\textnormal{ねんねん}}{\text{年々}}}$ 、 ${\overset{\textnormal{じゅよう}}{\text{需要}}}$ が ${\overset{\textnormal{ぞうか}}{\text{増加}}}$ している。 \hfill\break
\textbf{ \emph{Nen\textquotesingle nen }}\emph{, juyō ga zōka shite iru. }\hfill\break
Demand is increasing yearly \textbf{. }}

\par{37. ${\overset{\textnormal{きょうどう}}{\text{共同}}}$ で ${\overset{\textnormal{か}}{\text{買}}}$ った ${\overset{\textnormal{とち}}{\text{土地}}}$ を ${\overset{\textnormal{う}}{\text{売}}}$ ったお ${\overset{\textnormal{かね}}{\text{金}}}$ を ${\overset{\textnormal{はんはん}}{\text{半々}}}$ に ${\overset{\textnormal{わ}}{\text{分}}}$ けたことを ${\overset{\textnormal{のちのち}}{\text{後々}}}$ ${\overset{\textnormal{も}}{\text{揉}}}$ めないように ${\overset{\textnormal{しょめん}}{\text{書面}}}$ に ${\overset{\textnormal{のこ}}{\text{残}}}$ したいと ${\overset{\textnormal{おも}}{\text{思}}}$ っています。 \hfill\break
 \emph{Kyōdō de katta tochi wo utta okane wo \textbf{hanhan }\textbf{ }ni waketa koto wo \textbf{nochinochi }momenai yō ni shomen ni nokoshitai to omotte imasu. }\hfill\break
I would like to leave in writing that the money from selling land I hand jointly bought (with someone) was split in \textbf{half }\textbf{ }so that we don\textquotesingle t have a dispute in \textbf{ }\textbf{the distant future }\textbf{. }}

\par{38. その ${\overset{\textnormal{まち}}{\text{町}}}$ の ${\overset{\textnormal{ひとたち}}{\text{人達}}}$ は ${\overset{\textnormal{せんぞ}}{\text{先祖}}}$ ${\overset{\textnormal{だいだい}}{\text{}}}$ ${\overset{\textnormal{う}}{\text{受}}}$ け ${\overset{\textnormal{つ}}{\text{継}}}$ がれてきた ${\overset{\textnormal{ほうげん}}{\text{方言}}}$ を ${\overset{\textnormal{みずか}}{\text{自}}}$ ら ${\overset{\textnormal{ほうむ}}{\text{葬}}}$ り ${\overset{\textnormal{さ}}{\text{去}}}$ ったのだ。 \hfill\break
 \emph{Sono machi no hitotachi wa senzo \textbf{daidai }uketsugarete kita h }\emph{ō }\emph{gen wo mizukara hōmurisatta no da. }\hfill\break
The people of that town had buried their dialect, which had been passed down from the ancestors \textbf{generation to generation }, on their own. \hfill\break
 \hfill\break
 \textbf{Word Note }: 代々 may also be read as “ \emph{yoyo }” in far more literary fashion, utilizing the native word for “generation.” }

\par{39. ${\overset{\textnormal{まえまえ}}{\text{前々}}}$ から ${\overset{\textnormal{しゅちょう}}{\text{主張}}}$ しているように ${\overset{\textnormal{あき}}{\text{明}}}$ らかに ${\overset{\textnormal{まちが}}{\text{間違}}}$ っているんです。 \hfill\break
 \emph{ \textbf{Maemae } }\emph{kara shuch }\emph{ō }\emph{shite iru y }\emph{ō }\emph{ni akiraka ni machigatte iru n desu. }\hfill\break
Just as I have asserted from way before , it\textquotesingle s clearly mistaken\slash wrong. }

\par{40. ${\overset{\textnormal{さきざき}}{\text{先々}}}$ のことを考えると ${\overset{\textnormal{ふあん}}{\text{不安}}}$ になります。 \hfill\break
 \textbf{\emph{Sakizaki }}\emph{ }\emph{no koto wo kangaeru to fuan ni narimasu }. \hfill\break
\textbf{ }Whenever I think about what will be \textbf{way down the line }, I get anxious. }
    