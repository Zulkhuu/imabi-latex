    
\chapter{Disregard IV}

\begin{center}
\begin{Large}
第283課: Disregard IV: \{に・も\}構わず, をよそに, \& にもめげず(に) 
\end{Large}
\end{center}
 
\par{ In this fourth and final installation of expressions concerning disregard, we will cover expressions that revolve around the subject not caring about "X" when doing "Y." }
      
\section{\{に・も\}かまわず}
 
\par{ The verb 構う means “to mind\slash be concerned with.” Typically, it is used as an intransitive verb, which is why it takes に in most of the following examples. }

\par{1. ${\overset{\textnormal{わたし}}{\text{私}}}$ は ${\overset{\textnormal{ぜんぜんかま}}{\text{全然構}}}$ いません。 \hfill\break
I don\textquotesingle t mind at all. }

\par{2. ${\overset{\textnormal{わたし}}{\text{私}}}$ に ${\overset{\textnormal{かま}}{\text{構}}}$ わないでください。 \hfill\break
Don\textquotesingle t bother me. }

\par{3. ${\overset{\textnormal{たにん}}{\text{他人}}}$ のことに ${\overset{\textnormal{かま}}{\text{構}}}$ っていられない。 \hfill\break
I have too many of my own problems to deal with the problems of others. }

\par{ In the expression \{に・も\}構わず, it is used to show that one doesn\textquotesingle t care about “X” as one goes about doing “Y.” The use of も makes the verb transitive, and it is also makes the expression more emphatic overall. }

\par{4. ${\overset{\textnormal{ははおや}}{\text{母親}}}$ は ${\overset{\textnormal{あいかぎ}}{\text{合鍵}}}$ を ${\overset{\textnormal{も}}{\text{持}}}$ っているので、 ${\overset{\textnormal{じかん}}{\text{時間}}}$ に ${\overset{\textnormal{かま}}{\text{構}}}$ わず ${\overset{\textnormal{お}}{\text{押}}}$ しかけてくることもあります。 \hfill\break
My mother has a duplicate key, so she often comes intruding on me regardless of the time. }

\par{5. ${\overset{\textnormal{かれ}}{\text{彼}}}$ は ${\overset{\textnormal{あめ}}{\text{雨}}}$ に ${\overset{\textnormal{ぬ}}{\text{濡}}}$ れるのもかまわずお ${\overset{\textnormal{みせ}}{\text{店}}}$ へ ${\overset{\textnormal{ある}}{\text{歩}}}$ き ${\overset{\textnormal{つづ}}{\text{続}}}$ けた。 \hfill\break
He continued walking to the store whilst not caring about getting wet. }

\par{6. あの ${\overset{\textnormal{だんせい}}{\text{男性}}}$ は、 ${\overset{\textnormal{きんじょ}}{\text{近所}}}$ の ${\overset{\textnormal{ひと}}{\text{人}}}$ の ${\overset{\textnormal{めいわく}}{\text{迷惑}}}$ も ${\overset{\textnormal{かま}}{\text{構}}}$ わず、 ${\overset{\textnormal{まいばんおそ}}{\text{毎晩遅}}}$ くまで ${\overset{\textnormal{さわ}}{\text{騒}}}$ いでいる。 \hfill\break
That man makes racket late into the night every evening without caring about the trouble it causes his neighbors. }

\par{7. ${\overset{\textnormal{かのじょ}}{\text{彼女}}}$ は ${\overset{\textnormal{ひとめ}}{\text{人目}}}$ もかまわず ${\overset{\textnormal{な}}{\text{泣}}}$ き ${\overset{\textnormal{だ}}{\text{出}}}$ した。 \hfill\break
She burst into tears without caring about attention from others. }

\par{8. ぎっくり ${\overset{\textnormal{ごし}}{\text{腰}}}$ は ${\overset{\textnormal{ばしょ}}{\text{場所}}}$ も ${\overset{\textnormal{じかん}}{\text{時間}}}$ もかまわず ${\overset{\textnormal{おそ}}{\text{襲}}}$ ってくるものです。 \hfill\break
A strained back attacks irrespective of place and time. }

\par{9. あの ${\overset{\textnormal{みせ}}{\text{店}}}$ の ${\overset{\textnormal{じょうれんきゃく}}{\text{常連客}}}$ のほとんどは、 ${\overset{\textnormal{ねだん}}{\text{値段}}}$ も ${\overset{\textnormal{かま}}{\text{構}}}$ わず ${\overset{\textnormal{こうか}}{\text{高価}}}$ な ${\overset{\textnormal{しょうひん}}{\text{商品}}}$ を ${\overset{\textnormal{か}}{\text{買}}}$ い ${\overset{\textnormal{もと}}{\text{求}}}$ める ${\overset{\textnormal{ろうふじん}}{\text{老婦人}}}$ だ。 \hfill\break
Most of the regular customers of that shop are old women who buy expensive products without caring about price. }

\par{10. ${\overset{\textnormal{かれ}}{\text{彼}}}$ は ${\overset{\textnormal{かぞく}}{\text{家族}}}$ の ${\overset{\textnormal{しんぱい}}{\text{心配}}}$ もかまわず、 ${\overset{\textnormal{じっか}}{\text{実家}}}$ を ${\overset{\textnormal{で}}{\text{出}}}$ て ${\overset{\textnormal{じょうきょう}}{\text{上京}}}$ した。 \hfill\break
He left his parents\textquotesingle  home and went to Tokyo without caring about his family\textquotesingle s worries. }

\par{11. ${\overset{\textnormal{かのじょ}}{\text{彼女}}}$ は ${\overset{\textnormal{きゅうりょう}}{\text{給料}}}$ もかまわずに ${\overset{\textnormal{す}}{\text{好}}}$ きな ${\overset{\textnormal{しごと}}{\text{仕事}}}$ をやることにした。 \hfill\break
She decided to work that she liked without carrying about the salary. }

\par{12. ${\overset{\textnormal{なん}}{\text{何}}}$ にも ${\overset{\textnormal{かま}}{\text{構}}}$ わずに、ありのままで ${\overset{\textnormal{じぶん}}{\text{自分}}}$ になるんだ。 \hfill\break
I will be my true self without carrying about anything. }

\par{13. ${\overset{\textnormal{ところかま}}{\text{所構}}}$ わずゴミを ${\overset{\textnormal{す}}{\text{捨}}}$ てる ${\overset{\textnormal{ひと}}{\text{人}}}$ が ${\overset{\textnormal{き}}{\text{気}}}$ になります。 \hfill\break
People who throw trash away indiscriminately of place bother me. }

\par{14. ${\overset{\textnormal{わたし}}{\text{私}}}$ にかまわず、お ${\overset{\textnormal{さき}}{\text{先}}}$ にどうぞ。 \hfill\break
Please don\textquotesingle t mind me. Go on ahead. }

\par{15. ${\overset{\textnormal{かれ}}{\text{彼}}}$ らは ${\overset{\textnormal{しゅうい}}{\text{周囲}}}$ \{に・も\}かまわずキスしたりイチャイチャしたりしていた。 \hfill\break
They were kissing and making out without regard to their surroundings. }
      
\section{~をよそに}
 
\par{ The expression をよそに is used to show that one completely disregards and shoves “X” as one goes ahead and does “Y.” It is often used with words like 批判 (criticism), 反対 (opposition), 心配 (worry), etc. }

\par{16. ${\overset{\textnormal{こくみん}}{\text{国民}}}$ の ${\overset{\textnormal{はんたい}}{\text{反対}}}$ の ${\overset{\textnormal{こえ}}{\text{声}}}$ をよそに、 ${\overset{\textnormal{かいせいあん}}{\text{改正案}}}$ が ${\overset{\textnormal{かけつ}}{\text{可決}}}$ された。 \hfill\break
Despite voices of opposition from the people, the reform bill was passed. }

\par{17. ${\overset{\textnormal{けんみん}}{\text{県民}}}$ の ${\overset{\textnormal{こうぎ}}{\text{抗議}}}$ をよそに、 ${\overset{\textnormal{ぐんじきち}}{\text{軍事基地}}}$ の ${\overset{\textnormal{けんせつけいかく}}{\text{建設計画}}}$ が ${\overset{\textnormal{すす}}{\text{進}}}$ められている。 \hfill\break
Despite protest from the people of the prefecture, construction plans for the military base are underway. }

\par{18. ${\overset{\textnormal{おや}}{\text{親}}}$ の ${\overset{\textnormal{しんぱい}}{\text{心配}}}$ をよそに、 ${\overset{\textnormal{うで}}{\text{腕}}}$ が ${\overset{\textnormal{お}}{\text{折}}}$ れてギプスをしている ${\overset{\textnormal{けんたろう}}{\text{憲太郎}}}$ はまたも ${\overset{\textnormal{ともだち}}{\text{友達}}}$ と ${\overset{\textnormal{あそ}}{\text{遊}}}$ びに ${\overset{\textnormal{で}}{\text{出}}}$ かけてしまった。 \hfill\break
Kentaro, who broke his arm which is in a cast, went out again to have fun with friends despite his parents\textquotesingle  worries. }

\par{19. ${\overset{\textnormal{きょうわとう}}{\text{共和党}}}$ は、 ${\overset{\textnormal{こくみん}}{\text{国民}}}$ の ${\overset{\textnormal{ひはん}}{\text{批判}}}$ をよそに、 ${\overset{\textnormal{ことし}}{\text{今年}}}$ も ${\overset{\textnormal{ぞうぜいほうあん}}{\text{増税法案}}}$ を ${\overset{\textnormal{ていしゅつ}}{\text{提出}}}$ した。 \hfill\break
The Republican Party has submitted a tax increase bill this year as well despite criticism from the people. }

\par{20. ${\overset{\textnormal{せんせい}}{\text{先生}}}$ に ${\overset{\textnormal{どな}}{\text{怒鳴}}}$ られて ${\overset{\textnormal{みな}}{\text{皆}}}$ が ${\overset{\textnormal{きんちょう}}{\text{緊張}}}$ しているのをよそに、 ${\overset{\textnormal{かれし}}{\text{彼氏}}}$ だけぼんやりと ${\overset{\textnormal{ばか}}{\text{馬鹿}}}$ みたいに ${\overset{\textnormal{てんじょう}}{\text{天井}}}$ を ${\overset{\textnormal{む}}{\text{向}}}$ いていた。 \hfill\break
Only my boyfriend is the one looking up at the ceiling aimlessly like an idiot despite everyone being nervous from being shouted at by the teacher. }

\par{21. ${\overset{\textnormal{かれ}}{\text{彼}}}$ は ${\overset{\textnormal{りょうしん}}{\text{両親}}}$ の ${\overset{\textnormal{しんぱい}}{\text{心配}}}$ をよそに、ゲームに ${\overset{\textnormal{ぼっとう}}{\text{没頭}}}$ している。 \hfill\break
He has lost himself in gaming despite his parents\textquotesingle  worries. }

\par{22. ${\overset{\textnormal{あめ}}{\text{雨}}}$ の ${\overset{\textnormal{よほう}}{\text{予報}}}$ をよそに、 ${\overset{\textnormal{かさ}}{\text{傘}}}$ を ${\overset{\textnormal{も}}{\text{持}}}$ たずにタイミングよく ${\overset{\textnormal{すば}}{\text{素晴}}}$ らしい ${\overset{\textnormal{せいてん}}{\text{晴天}}}$ で、 ${\overset{\textnormal{かれし}}{\text{彼氏}}}$ と ${\overset{\textnormal{いっしょ}}{\text{一緒}}}$ に ${\overset{\textnormal{みずうみ}}{\text{湖}}}$ のほとりを ${\overset{\textnormal{さんぽ}}{\text{散歩}}}$ することができた。 \hfill\break
I was able to walk along the lake together with my boyfriend without bringing umbrellas under wonderful clear skies thanks to good timing despite the forecast for rain. }

\par{23. ${\overset{\textnormal{じゅうみん}}{\text{住民}}}$ の ${\overset{\textnormal{ふあん}}{\text{不安}}}$ をよそに、 ${\overset{\textnormal{ちじ}}{\text{知事}}}$ が ${\overset{\textnormal{げんぱつ}}{\text{原発}}}$ の ${\overset{\textnormal{さいかどう}}{\text{再稼働}}}$ を ${\overset{\textnormal{みと}}{\text{認}}}$ めた。 \hfill\break
Despite uneasiness that the residents have, the governor has approved the resuming operations of nuclear power. }

\par{24. ${\overset{\textnormal{かぞく}}{\text{家族}}}$ の ${\overset{\textnormal{きたい}}{\text{期待}}}$ をよそに、 ${\overset{\textnormal{おとうと}}{\text{弟}}}$ は ${\overset{\textnormal{けっきょくだいがく}}{\text{結局大学}}}$ には ${\overset{\textnormal{はい}}{\text{入}}}$ らずにアルバイト ${\overset{\textnormal{せいかつ}}{\text{生活}}}$ を ${\overset{\textnormal{つづ}}{\text{続}}}$ けている。 \hfill\break
My little brother, despite the family\textquotesingle s expectations, ended up not getting into college and continues being a part-time worker. }

\par{25. お祭りの渋滞をよそに、素通りして帰宅した。 \hfill\break
I passed by the congestion from the festival when I returned home. }

\par{26. 彼は医者の忠告をよそに、毎日お酒を飲んでいる。 \hfill\break
Despite his doctor\textquotesingle s warning, he drinks every day. }

\par{27. ${\overset{\textnormal{かくへいききんしじょうやく}}{\text{核兵器禁止条約}}}$ をよそに、 ${\overset{\textnormal{きたちょうせん}}{\text{北朝鮮}}}$ やイランなどといった ${\overset{\textnormal{くに}}{\text{国}}}$ は ${\overset{\textnormal{かくじっけん}}{\text{核実験}}}$ を ${\overset{\textnormal{おこな}}{\text{行}}}$ い ${\overset{\textnormal{つづ}}{\text{続}}}$ けている。 \hfill\break
There are nations such as North Korea and Iran that continue performing nuclear tests despite the Nuclear Weapons Convention. }

\par{28. ${\overset{\textnormal{こくみん}}{\text{国民}}}$ の ${\overset{\textnormal{ひなん}}{\text{非難}}}$ をよそに、その ${\overset{\textnormal{おしょくせいじか}}{\text{汚職政治家}}}$ はまたしても ${\overset{\textnormal{ぎせき}}{\text{議席}}}$ に ${\overset{\textnormal{かえ}}{\text{返}}}$ り ${\overset{\textnormal{ざ}}{\text{咲}}}$ いた。 \hfill\break
The corrupt politician made yet another comeback to his seat despite reproach from the people. }

\par{ The word よそ, especially when written as 他所・余所, may also mean “elsewhere.” }

\par{29. ${\overset{\textnormal{よそ}}{\text{他所}}}$ の ${\overset{\textnormal{みせ}}{\text{店}}}$ に ${\overset{\textnormal{ぎょう}}{\text{行}}}$ きな! \hfill\break
Go to another store! }

\par{30. ${\overset{\textnormal{こんや}}{\text{今夜}}}$ は ${\overset{\textnormal{よしょ}}{\text{他所}}}$ に ${\overset{\textnormal{と}}{\text{泊}}}$ まらなきゃ。 \hfill\break
I have to stay somewhere else tonight. }
      
\section{にもめげず(に)}
 
\par{ The pattern にもめげず(に) is used after nouns that relate to personal adversity that the speaker then overcomes somehow in situation Y. This phrase is typically literary, but the subject of the action is painted in a very positive light. }

\par{31. 彼は ${\overset{\textnormal{たびかさ}}{\text{度重}}}$ なる ${\overset{\textnormal{ふこう}}{\text{不幸}}}$ にもめげず、いつも ${\overset{\textnormal{まえむ}}{\text{前向}}}$ きだった。 \hfill\break
He was always positive in the face of repeated sorrows. }

\par{32. この ${\overset{\textnormal{まち}}{\text{町}}}$ の ${\overset{\textnormal{なし}}{\text{梨}}}$ は、 ${\overset{\textnormal{あくじょうけん}}{\text{悪条件}}}$ にもめげず、 ${\overset{\textnormal{りっぱ}}{\text{立派}}}$ に美味しいです。 \hfill\break
This town\textquotesingle s peaches are splendidly delicious even with unfavorable conditions. }

\par{33. ${\overset{\textnormal{かれ}}{\text{彼}}}$ はどんな ${\overset{\textnormal{こんなん}}{\text{困難}}}$ にもめげずに ${\overset{\textnormal{ゆめ}}{\text{夢}}}$ を ${\overset{\textnormal{あきら}}{\text{諦}}}$ めなかった。 \hfill\break
He didn\textquotesingle t give up on his dreams despite whatever hardship. }

\par{34. その ${\overset{\textnormal{せんしゅ}}{\text{選手}}}$ は、 ${\overset{\textnormal{くきょう}}{\text{苦境}}}$ にもめげず、 ${\overset{\textnormal{しあい}}{\text{試合}}}$ に ${\overset{\textnormal{む}}{\text{向}}}$ かってメダルを ${\overset{\textnormal{か}}{\text{勝}}}$ ち ${\overset{\textnormal{と}}{\text{取}}}$ った。 \hfill\break
The athlete thought nothing of his predicament, faced the match, and won the medal. }

\par{35. ${\overset{\textnormal{たなか}}{\text{田中}}}$ はどんな ${\overset{\textnormal{ふこう}}{\text{不幸}}}$ や ${\overset{\textnormal{ふうん}}{\text{不運}}}$ にもめげず、いつも ${\overset{\textnormal{えがお}}{\text{笑顔}}}$ を ${\overset{\textnormal{た}}{\text{絶}}}$ やさない。 \hfill\break
Tanaka never stops smiling, not discouraged by whatever sorrow or misfortune. }

\par{36. ${\overset{\textnormal{かずかず}}{\text{数々}}}$ の ${\overset{\textnormal{しれん}}{\text{試練}}}$ にもめげず、 ${\overset{\textnormal{ひろいん}}{\text{ヒロイン}}}$ が ${\overset{\textnormal{みらい}}{\text{未来}}}$ を ${\overset{\textnormal{き}}{\text{切}}}$ り ${\overset{\textnormal{ひら}}{\text{開}}}$ いていく。 \hfill\break
Not discouraged by numerous trials, the heroine opens up the future. }

\par{37. ${\overset{\textnormal{おおじしん}}{\text{大地震}}}$ にもめげずに ${\overset{\textnormal{げんき}}{\text{元気}}}$ で ${\overset{\textnormal{けなげ}}{\text{健気}}}$ な ${\overset{\textnormal{しま}}{\text{島}}}$ の ${\overset{\textnormal{ひとびと}}{\text{人々}}}$ にほっとした。 \hfill\break
I was relieved with how the people of the island were gallant and vigorous, and not discouraged by the great quake. }

\par{38. ${\overset{\textnormal{たいふう}}{\text{台風}}}$ にもめげずに ${\overset{\textnormal{おきなわ}}{\text{沖縄}}}$ から ${\overset{\textnormal{い}}{\text{行}}}$ ってきました。 \hfill\break
I\textquotesingle ve returned from Okinawa despite the typhoon. }

\par{39. 彼は ${\overset{\textnormal{びょうき}}{\text{病気}}}$ にもめげず、 ${\overset{\textnormal{たぼう}}{\text{多忙}}}$ な ${\overset{\textnormal{まいにち}}{\text{毎日}}}$ にもへこたれずに ${\overset{\textnormal{たの}}{\text{楽}}}$ しく ${\overset{\textnormal{い}}{\text{生}}}$ きている。 \hfill\break
He lives happily and not discouraged even as each day is busy despite his illness. }

\par{40. ${\overset{\textnormal{にほん}}{\text{日本}}}$ のモノづくり ${\overset{\textnormal{きぎょう}}{\text{企業}}}$ は、どんな ${\overset{\textnormal{ぎゃくふう}}{\text{逆風}}}$ にもめげず、 ${\overset{\textnormal{ちえ}}{\text{知恵}}}$ と ${\overset{\textnormal{にんたいりょく}}{\text{忍耐力}}}$ で ${\overset{\textnormal{こくふく}}{\text{克服}}}$ してきた。 \hfill\break
Japan\textquotesingle s manufacturers have overcome all sorts of adversity with wit and perseverance. }
    