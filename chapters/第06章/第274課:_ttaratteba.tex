    
\chapter{~ったら \& ~ってば}

\begin{center}
\begin{Large}
第274課: ~ったら \& ~ってば 
\end{Large}
\end{center}
 
\par{ ~ったら  and ~ってば are technically the colloquial contractions of ~といったら and ~といえば respectively. Although they can be used solely as colloquial contractions of these patterns, they have also each evolved their own unique usages that we will primarily be learning about in this lesson. }
      
\section{~ったら}
 
\par{ As stated above, ~ったら can simply be used as a very colloquial contraction of ~といったら, as demonstrated in the following example sentences. }

\par{1. ${\overset{\textnormal{おれ}}{\text{俺}}}$ が ${\overset{\textnormal{き}}{\text{来}}}$ いったら、来い! \hfill\break
When I say come, come! }

\par{2. カレーったらカレーだ! \hfill\break
Now this is curry! }

\par{3. うるさいったらうるさい! \hfill\break
God, you\textquotesingle re so annoying! }

\par{ As can be deduced from these three examples, a lot of emotion is packed into this contraction, and it can be used in both positive and negative connotations. Ex. 1 even demonstrates how it can be used in a commanding manner. }

\begin{center}
\textbf{~ったらない } \hfill\break

\end{center}

\par{ The tone that can be seen in Ex. 2 can be expanded upon in the pattern ~ったらない. This can be viewed as a simplification of ~と言ったら他にない. Essentially, “there isn\textquotesingle t anything as…as” whatever is being discussed. Or, “it is impossible” for whatever you\textquotesingle re discussing to be “any more” like how you\textquotesingle re describing it because it is already so very “x.” Lastly, as far as its usage is concerned, it can go after nouns, adjectives, or verbs. \hfill\break
 \hfill\break
4. この ${\overset{\textnormal{こいぬ}}{\text{子犬}}}$ 、 ${\overset{\textnormal{かわい}}{\text{可愛}}}$ いったらない。 \hfill\break
This puppy couldn\textquotesingle t be any cuter! }

\par{5. テキサス ${\overset{\textnormal{しゅう}}{\text{州}}}$ の ${\overset{\textnormal{あつ}}{\text{暑}}}$ さといったらない。 \hfill\break
The heat in Texas couldn\textquotesingle t get any hotter than this. }

\par{6. お ${\overset{\textnormal{ふくろ}}{\text{袋}}}$ の ${\overset{\textnormal{しんぱい}}{\text{心配}}}$ ったらなかった。 \hfill\break
You can imagine how worried my mother was. }

\par{7. ${\overset{\textnormal{にほんご}}{\text{日本語}}}$ の ${\overset{\textnormal{あいまい}}{\text{曖昧}}}$ さったらない。 \hfill\break
Japanese couldn\textquotesingle t be vaguer. }

\par{8. このイワシ、 ${\overset{\textnormal{くさ}}{\text{臭}}}$ さったらない \hfill\break
These sardines, they couldn\textquotesingle t smell any worse! }

\par{\textbf{Spelling Note }: イワシ is seldom spelled as 鰯. For those of you who notice 鰮 as an option when typing, this character is essentially not used in Japanese, but it does incidentally mean “small fish” in Chinese. }

\par{9. ${\overset{\textnormal{いま}}{\text{今}}}$ の ${\overset{\textnormal{じだい}}{\text{時代}}}$ 、 ${\overset{\textnormal{けいたい}}{\text{携帯}}}$ やパソコンなどがないなら、 ${\overset{\textnormal{ふじゆう}}{\text{不自由}}}$ といったらない。 \hfill\break
One couldn\textquotesingle t possibly be any more inconvenienced without having cellphones, PCs, etc. in this day in age. }

\par{10. みんなの ${\overset{\textnormal{まえ}}{\text{前}}}$ で ${\overset{\textnormal{ころ}}{\text{転}}}$ んで、 ${\overset{\textnormal{は}}{\text{恥}}}$ ずかしいったらない。 \hfill\break
There couldn\textquotesingle t be anything more embarrassing than tripping in front of everyone. }

\par{11. このところ、 ${\overset{\textnormal{ざんぎょうつづ}}{\text{残業続}}}$ きで ${\overset{\textnormal{つか}}{\text{疲}}}$ れると ${\overset{\textnormal{い}}{\text{言}}}$ ったらない。 \hfill\break
You can imagine how tired I\textquotesingle ve been from continued overtime recently. }

\par{12. ${\overset{\textnormal{せいじんじょせい}}{\text{成人女性}}}$ の ${\overset{\textnormal{きものすがた}}{\text{着物姿}}}$ の ${\overset{\textnormal{うつく}}{\text{美}}}$ しさったらない。 \hfill\break
There\textquotesingle s nothing as beautiful as an adult woman in kimono. }

\begin{center}
\textbf{~ったらありゃしない } \hfill\break

\end{center}

\par{ An even more emphatic variation of this is ~ったらありゃしない.  This is essentially only used with adjectives. As seen below, some variation in its appearance can be found. You can see the ったら part uncontracted as といったら, see ったら as ったりゃ, or see the りゃ part elongated as りゃあ. }

\par{13. ${\overset{\textnormal{いっ}}{\text{1}}}$ ${\overset{\textnormal{か}}{\text{ヶ}}}$ ${\overset{\textnormal{げつ}}{\text{月}}}$ もお ${\overset{\textnormal{ふろ}}{\text{風呂}}}$ に ${\overset{\textnormal{はい}}{\text{入}}}$ らないのは ${\overset{\textnormal{きたな}}{\text{汚}}}$ いといったらありゃしない。 \hfill\break
One couldn\textquotesingle t possibly be any filthier than when one hasn\textquotesingle t bathed in over a month. }

\par{14. ${\overset{\textnormal{さむ}}{\text{寒}}}$ いったりゃありゃしない! \hfill\break
It couldn\textquotesingle t be possibly colder than this! }

\par{15. ${\overset{\textnormal{うらや}}{\text{羨}}}$ ましいったりゃありゃしない。 \hfill\break
I couldn\textquotesingle t possibly be more jealous. }

\begin{center}
\textbf{Expressing Annoyance } 
\end{center}

\par{ As you can imagine, if it can be used in such a positive manner, it can be used in the exact opposite way. Whenever ったら is seen after proper nouns, you can express exasperating frustration. }

\par{16. ${\overset{\textnormal{あにき}}{\text{兄貴}}}$ ったらまだ ${\overset{\textnormal{ね}}{\text{寝}}}$ てるぞ。 \hfill\break
Ugh, my big brother…he\textquotesingle s still sleeping. }

\par{17. ${\overset{\textnormal{かんだ}}{\text{神田}}}$ さんったら、よく ${\overset{\textnormal{い}}{\text{言}}}$ うよね。 \hfill\break
Oh, Kanda-san, you say that a lot, don\textquotesingle t you. }

\par{18. ${\overset{\textnormal{まさみ}}{\text{雅美}}}$ ったら、 ${\overset{\textnormal{むちゃい}}{\text{無茶言}}}$ わないで。 \hfill\break
Ugh, Masami, don\textquotesingle t say something so absurd. }

\par{19. もう ${\overset{\textnormal{かれし}}{\text{彼氏}}}$ ったら・・・ \hfill\break
Ugh, my boyfriend… }

\par{20.トランプ ${\overset{\textnormal{だいとうりょう}}{\text{大統領}}}$ ったら、 ${\overset{\textnormal{しゅうにん}}{\text{就任}}}$ ${\overset{\textnormal{いっ}}{\text{1}}}$ ${\overset{\textnormal{か}}{\text{ヶ}}}$ ${\overset{\textnormal{げつ}}{\text{月}}}$ で、じゃんじゃん ${\overset{\textnormal{だいとうりょうれい}}{\text{大統領令}}}$ をサインしているけど、その速さが怖い! \hfill\break
Gah, President Trump has been signing executive orders like crazy in his first month since being inaugurated, but it\textquotesingle s the speed of that which is scary! }

\par{ 21. ${\overset{\textnormal{じえいたい}}{\text{自衛隊}}}$ ったら、 ${\overset{\textnormal{ほんとう}}{\text{本当}}}$ にお ${\overset{\textnormal{けん}}{\text{堅}}}$ いんだなあ。 \hfill\break
The Self Defense Force is so uptight. }
      
\section{~ってば}
 
\par{ As stated in the introduction, at its basic understanding, ~ってば can be viewed as a very colloquial contraction of ~といえば. }

\par{22. ${\overset{\textnormal{ひま}}{\text{暇}}}$ ってば ${\overset{\textnormal{ひま}}{\text{暇}}}$ でしょう。 \hfill\break
When talking about free time, it (should be) free time, you know. }

\par{23. ブドウってばフランス ${\overset{\textnormal{さん}}{\text{産}}}$ がいいなぁ。 \hfill\break
Speaking of grapes, ones made-in-France are really good. }

\par{24. この ${\overset{\textnormal{わがし}}{\text{和菓子}}}$ ってば、ぴったりだね。 \hfill\break
Speaking of this wagashi, it\textquotesingle s perfect, huh. \hfill\break
 \hfill\break
25. ${\overset{\textnormal{うみ}}{\text{海}}}$ ってば、 ${\overset{\textnormal{きのう}}{\text{昨日}}}$ までずいぶん ${\overset{\textnormal{あ}}{\text{荒}}}$ れてたね。 \hfill\break
Speaking of the sea, it was pretty rough until yesterday, huh. }

\par{ Similarly to ~ったら but with far more anger behind it, ~ってば can be used to sharply criticize. It is a great way to indicate annoyance or giving a sharp command by expressing your annoyance. }

\par{26. ${\overset{\textnormal{たつみ}}{\text{辰己}}}$ ってば! \hfill\break
Gah, Tatsumi!!! }

\par{27. ${\overset{\textnormal{ちが}}{\text{違}}}$ うってば! \hfill\break
I said it\textquotesingle s wrong! }

\par{28. ${\overset{\textnormal{はや}}{\text{早}}}$ く ${\overset{\textnormal{も}}{\text{持}}}$ ってきなさいってば! \hfill\break
I told you to hurry and bring it! }

\par{29. ${\overset{\textnormal{い}}{\text{要}}}$ らないってば、 ${\overset{\textnormal{い}}{\text{要}}}$ らない! \hfill\break
When I say it\textquotesingle s not needed, it\textquotesingle s not needed! }

\par{30. ヤダったらヤダってば! \hfill\break
When I say no, I mean no! }
    