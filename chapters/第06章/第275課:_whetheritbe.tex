    
\chapter{Whether it be\dothyp{}\dothyp{}\dothyp{}or\dothyp{}\dothyp{}\dothyp{}}

\begin{center}
\begin{Large}
第275課: Whether it be\dothyp{}\dothyp{}\dothyp{}or\dothyp{}\dothyp{}\dothyp{}: ~といい \& ~といわず 
\end{Large}
\end{center}
 
\par{ This lesson is about two very similar expressions that translate as “whether it be…or…” Though the differences between the two are subtle, understanding the finer differences between the two will be all you\textquotesingle ll have to worry about. }
      
\section{~といい}
 
\par{・AといいBといいC: In addition to A and B, there are other things that become C. }

\par{ At a basic understanding, the expression ~といい can be described in the above manner. Although everything else in question may not be 100\% like “C,” several other related things will be implied by using this pattern. This can be used in positive and negative connotations. Typically, this expression is used primarily in the written language, but it would not be totally odd to use in the spoken language as well. }

\par{1. ${\overset{\textnormal{かお}}{\text{香}}}$ りといい、 ${\overset{\textnormal{あじ}}{\text{味}}}$ といい、 ${\overset{\textnormal{けんないいち}}{\text{県内一}}}$ の ${\overset{\textnormal{おい}}{\text{美味}}}$ しさだと ${\overset{\textnormal{おも}}{\text{思}}}$ う。 \hfill\break
Whether it be its fragrance or its taste, I think that it is No. 1 in deliciousness in the prefecture. }

\par{2. ${\overset{\textnormal{にほん}}{\text{日本}}}$ は、 ${\overset{\textnormal{そうり}}{\text{総理}}}$ をはじめ、 ${\overset{\textnormal{せいじか}}{\text{政治家}}}$ といい ${\overset{\textnormal{かんりょう}}{\text{官僚}}}$ といいどうしようもないものばかりだ。 \hfill\break
As for Japan, not only the Prime Minister, but also whether it be politicians or bureaucrats, it is full of just helpless individuals. }

\par{3. ${\overset{\textnormal{ふぶき}}{\text{吹雪}}}$ の ${\overset{\textnormal{ようす}}{\text{様子}}}$ を ${\overset{\textnormal{み}}{\text{見}}}$ に ${\overset{\textnormal{もど}}{\text{戻}}}$ ってきた ${\overset{\textnormal{おっと}}{\text{夫}}}$ は、 ${\overset{\textnormal{あたま}}{\text{頭}}}$ といい、 ${\overset{\textnormal{かた}}{\text{肩}}}$ といい、 ${\overset{\textnormal{ゆき}}{\text{雪}}}$ まみれで、まるで ${\overset{\textnormal{ゆきだるま}}{\text{雪達磨}}}$ のような ${\overset{\textnormal{かっこう}}{\text{格好}}}$ でした。 \hfill\break
My husband, who returned from checking out the status of the blizzard, is covered in snow, whether it be his head or shoulders; it\textquotesingle s as if he\textquotesingle s a snowman. }

\par{4. ${\overset{\textnormal{あたま}}{\text{頭}}}$ といい、 ${\overset{\textnormal{せいかく}}{\text{性格}}}$ といい、 ${\overset{\textnormal{せんせい}}{\text{先生}}}$ のご ${\overset{\textnormal{しゅじん}}{\text{主人}}}$ は ${\overset{\textnormal{すば}}{\text{素晴}}}$ らしいです。 \hfill\break
Whether it be his brains or his personality, your husband, Sensei, is a wonderful person. }

\par{5. ${\overset{\textnormal{すし}}{\text{寿司}}}$ といい、うどんといい、 ${\overset{\textnormal{わしょく}}{\text{和食}}}$ のなんでも ${\overset{\textnormal{す}}{\text{好}}}$ きです。 \hfill\break
Whether it be sushi or udon, I like anything in Japanese cuisine. }

\par{\textbf{Orthography Note }: うどん may seldom be spelled as 饂飩. }

\par{6. ${\overset{\textnormal{せいのう}}{\text{性能}}}$ といい、デザインといい、このコーヒーメーカーが ${\overset{\textnormal{いちばんす}}{\text{一番好}}}$ きです。 \hfill\break
Whether it be its performance or its design, I like this coffee maker the best. }

\par{7. ${\overset{\textnormal{はせがわ}}{\text{長谷川}}}$ さんは、 ${\overset{\textnormal{べんきょう}}{\text{勉強}}}$ といい、スポーツといい、いつでもトップなのよ。(Feminine) \hfill\break
Hasegawa, whether it be his studies or spots,  is always at the top. }

\par{8. この ${\overset{\textnormal{まち}}{\text{街}}}$ は、 ${\overset{\textnormal{えき}}{\text{駅}}}$ からのアクセスといい、 ${\overset{\textnormal{べんり}}{\text{便利}}}$ さといい、 ${\overset{\textnormal{す}}{\text{住}}}$ むにはちょうどいいところなんですね。 \hfill\break
This town is the perfect place to live, whether it be because of its (ease of) access via }

\par{9. ${\overset{\textnormal{い}}{\text{言}}}$ わずもがな、その ${\overset{\textnormal{へや}}{\text{部屋}}}$ は ${\overset{\textnormal{かぐ}}{\text{家具}}}$ といい、カーテンといい、 ${\overset{\textnormal{じゅうたん}}{\text{絨毯}}}$ といい、 ${\overset{\textnormal{すてき}}{\text{素敵}}}$ すぎるもので ${\overset{\textnormal{あふ}}{\text{溢}}}$ れていた。 \hfill\break
It goes without saying, the room was filled with absolutely amazing things, whether it was the future, the curtains, or the carpet. }

\par{10. ${\overset{\textnormal{かお}}{\text{顔}}}$ の ${\overset{\textnormal{いろつや}}{\text{色艶}}}$ といい、 ${\overset{\textnormal{こえ}}{\text{声}}}$ の ${\overset{\textnormal{は}}{\text{張}}}$ りといい、 ${\overset{\textnormal{しまむら}}{\text{嶋村}}}$ さん、 ${\overset{\textnormal{ほんとう}}{\text{本当}}}$ にお ${\overset{\textnormal{げんき}}{\text{元気}}}$ になられましたね。 \hfill\break
From your complexion to the spring in your voice, you\textquotesingle ve really gotten better, haven\textquotesingle t you, Shimamura-san? }

\par{11. ${\overset{\textnormal{ひんしつ}}{\text{品質}}}$ といい、 ${\overset{\textnormal{ねだん}}{\text{値段}}}$ といい、アメリカ ${\overset{\textnormal{さん}}{\text{産}}}$ の ${\overset{\textnormal{ぶたにく}}{\text{豚肉}}}$ に ${\overset{\textnormal{まさ}}{\text{勝}}}$ るものはない。 \hfill\break
Whether it be its quality or its price, there isn\textquotesingle t anything that beats American pork. }

\par{12. ${\overset{\textnormal{りょうり}}{\text{料理}}}$ は、 ${\overset{\textnormal{あじ}}{\text{味}}}$ といい ${\overset{\textnormal{りょう}}{\text{量}}}$ といい ${\overset{\textnormal{まんぞく}}{\text{満足}}}$ でした。 \hfill\break
The cooking, whether it be the taste or the quantity, was satisfactory. }
      
\section{~といわず}
 
\par{・AといわずBといわずC: Starting with A and B (and others), they all become C. }

\par{ This pattern is almost identical to ~といい. The only difference is that it indicates that “A” and “B” are C 100\%. If this absolute doesn\textquotesingle t hold in reality, then ~といわず becomes ungrammatical and you should use ~といい instead. This may be used in positive or negative contexts. However, it is markedly more subjective than ~といい. }

\par{13. ${\overset{\textnormal{ふぶき}}{\text{吹雪}}}$ の ${\overset{\textnormal{ようす}}{\text{様子}}}$ を ${\overset{\textnormal{み}}{\text{見}}}$ に ${\overset{\textnormal{もど}}{\text{戻}}}$ ってきた ${\overset{\textnormal{おっと}}{\text{夫}}}$ は、 ${\overset{\textnormal{あたま}}{\text{頭}}}$ といわず、 ${\overset{\textnormal{かた}}{\text{肩}}}$ といわず、 ${\overset{\textnormal{からだぜんしん}}{\text{体全身}}}$ が ${\overset{\textnormal{ゆきだるま}}{\text{雪達磨}}}$ のようでした。 \hfill\break
My husband, who returned from checking out the status of the blizzard, whether it was his head or his shoulders, he looked like a snowman all over. }

\par{14. 彼は ${\overset{\textnormal{にゅうよくちゅう}}{\text{入浴中}}}$ といわず、 ${\overset{\textnormal{しゅうしんちゅう}}{\text{就寝中}}}$ といわず、 ${\overset{\textnormal{かたとき}}{\text{片時}}}$ もスマホを ${\overset{\textnormal{てばな}}{\text{手放}}}$ さない。 \hfill\break
Whether while bathing or sleeping, there is not even a moment he lets go over his smart phone. }

\par{15. ${\overset{\textnormal{にほんじん}}{\text{日本人}}}$ は、 ${\overset{\textnormal{こども}}{\text{子供}}}$ といわず ${\overset{\textnormal{おとな}}{\text{大人}}}$ といわず、 ${\overset{\textnormal{まんが}}{\text{漫画}}}$ やライトノベルなどをよく ${\overset{\textnormal{よ}}{\text{読}}}$ む。 \hfill\break
Japanese people, whether they be children or adults, frequently read stuff like manga and light novels. }

\par{16. ${\overset{\textnormal{わたし}}{\text{私}}}$ はアライグマに ${\overset{\textnormal{て}}{\text{手}}}$ といわず ${\overset{\textnormal{あし}}{\text{足}}}$ といわず ${\overset{\textnormal{やたら}}{\text{矢鱈}}}$ に ${\overset{\textnormal{か}}{\text{噛}}}$ まれて、 ${\overset{\textnormal{びょういん}}{\text{病院}}}$ に ${\overset{\textnormal{はこ}}{\text{運}}}$ ばれました。 \hfill\break
I was indiscriminately bitten all over, my hands, feet, etc. by a raccoon and was taken to the hospital. }

\par{17. ${\overset{\textnormal{わたし}}{\text{私}}}$ が ${\overset{\textnormal{かよ}}{\text{通}}}$ っていた ${\overset{\textnormal{ちゅうがっこう}}{\text{中学校}}}$ には、 ${\overset{\textnormal{こうちょう}}{\text{校長}}}$ といわず、 ${\overset{\textnormal{きょうとう}}{\text{教頭}}}$ といわず、まともに ${\overset{\textnormal{せいと}}{\text{生徒}}}$ のことを ${\overset{\textnormal{かんが}}{\text{考}}}$ えて ${\overset{\textnormal{しごと}}{\text{仕事}}}$ に ${\overset{\textnormal{と}}{\text{取}}}$ り ${\overset{\textnormal{く}}{\text{組}}}$ んでいる ${\overset{\textnormal{ひと}}{\text{人}}}$ は、ひとりもいなかった。 \hfill\break
There wasn\textquotesingle t a single person at the middle school I went to, whether it was the principal or the vice principal, who got the job done by earnestly considered students. }

\par{18. ${\overset{\textnormal{わたし}}{\text{私}}}$ は、 ${\overset{\textnormal{ぎゅうにく}}{\text{牛肉}}}$ といわず、 ${\overset{\textnormal{ぶたにく}}{\text{豚肉}}}$ といわず、お ${\overset{\textnormal{にく}}{\text{肉}}}$ は ${\overset{\textnormal{ぜんぜんた}}{\text{全然食}}}$ べません。 \hfill\break
I, whether it be beef or pork, do not eat meat whatsoever. }

\par{19. ${\overset{\textnormal{こうだ}}{\text{幸田}}}$ さんは ${\overset{\textnormal{にほんしゅ}}{\text{日本酒}}}$ といわずビールといわず、 ${\overset{\textnormal{さけ}}{\text{酒}}}$ であれば、 ${\overset{\textnormal{なん}}{\text{何}}}$ にでも ${\overset{\textnormal{め}}{\text{目}}}$ がない。 \hfill\break
Whether it be sake or beer, if it\textquotesingle s alcohol, he has a weakness for it. }

\par{20. ${\overset{\textnormal{め}}{\text{目}}}$ といい ${\overset{\textnormal{くちもと}}{\text{口元}}}$ といい ${\overset{\textnormal{ふくそう}}{\text{服装}}}$ といい、 ${\overset{\textnormal{ほんとう}}{\text{本当}}}$ にハンサムだと ${\overset{\textnormal{おも}}{\text{思}}}$ わない? \hfill\break
Whether it be his eyes, his mouth, or his clothes, do you think he\textquotesingle s really handsome? }

\par{21. ${\overset{\textnormal{かみ}}{\text{髪}}}$ といわず ${\overset{\textnormal{ふく}}{\text{服}}}$ といわず、とても ${\overset{\textnormal{たばこくさ}}{\text{煙草臭}}}$ くなってしまい、 ${\overset{\textnormal{じぶん}}{\text{自分}}}$ は ${\overset{\textnormal{ぜんぜんす}}{\text{全然吸}}}$ わないのに、 ${\overset{\textnormal{まと}}{\text{纏}}}$ わり ${\overset{\textnormal{つ}}{\text{付}}}$ いた ${\overset{\textnormal{にお}}{\text{臭}}}$ いで ${\overset{\textnormal{きも}}{\text{気持}}}$ ち ${\overset{\textnormal{わる}}{\text{悪}}}$ くなることもあります。 \hfill\break
Whether it be my hair or clothes, I end up reeking of tobacco, and even though I don\textquotesingle t smoke at all, the odor that follows me is often revolting. }

\par{22. ${\overset{\textnormal{こども}}{\text{子供}}}$ たちは ${\overset{\textnormal{かお}}{\text{顔}}}$ といわず ${\overset{\textnormal{かみ}}{\text{髪}}}$ といわず ${\overset{\textnormal{ふく}}{\text{服}}}$ といわず ${\overset{\textnormal{どろ}}{\text{泥}}}$ だらけで、 ${\overset{\textnormal{しろ}}{\text{白}}}$ い ${\overset{\textnormal{は}}{\text{歯}}}$ を ${\overset{\textnormal{み}}{\text{見}}}$ せてにやにや ${\overset{\textnormal{わら}}{\text{笑}}}$ っていた。 \hfill\break
The children were covered in mud, whether it be their faces, hair, or clothes, and were snickering as they showed their white teeth. }

\par{23. \{ ${\overset{\textnormal{ひる}}{\text{昼}}}$ といわず ${\overset{\textnormal{よる}}{\text{夜}}}$ といわず・ ${\overset{\textnormal{ちゅうや}}{\text{昼夜}}}$ を ${\overset{\textnormal{と}}{\text{問}}}$ わず\}、 ${\overset{\textnormal{かわせそうば}}{\text{為替相場}}}$ は ${\overset{\textnormal{じじこっこく}}{\text{時々刻々}}}$ と\{ ${\overset{\textnormal{うご}}{\text{動}}}$ いている・ ${\overset{\textnormal{か}}{\text{変}}}$ わっている\}。 \hfill\break
The exchange rate moves\slash changes all day and all night and at every moment. }

\par{24. ${\overset{\textnormal{て}}{\text{手}}}$ といわず ${\overset{\textnormal{あし}}{\text{足}}}$ といわず、 ${\overset{\textnormal{たい}}{\text{体}}}$ じゅう ${\overset{\textnormal{か}}{\text{蚊}}}$ に ${\overset{\textnormal{さ}}{\text{刺}}}$ された。 \hfill\break
Not just my hands and feet, but I was bitten by mosquitoes all over my body. }

\par{25. ${\overset{\textnormal{かれ}}{\text{彼}}}$ は、 ${\overset{\textnormal{あさ}}{\text{朝}}}$ といわず、 ${\overset{\textnormal{よる}}{\text{夜}}}$ といわず、 ${\overset{\textnormal{ひま}}{\text{暇}}}$ さえあればマンガを ${\overset{\textnormal{よ}}{\text{読}}}$ んでばかりいるみたいだ。 \hfill\break
Not just in the morning and in the evening, but he reads comics whenever he has time. }
    