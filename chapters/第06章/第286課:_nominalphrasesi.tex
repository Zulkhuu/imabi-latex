    
\chapter{Nominal Phrases I}

\begin{center}
\begin{Large}
第286課: Nominal Phrases I: 中心, 上, \& 分 
\end{Large}
\end{center}
       
\section{中心}
 
\par{ 中心 refers to the center of something, and を中心に means "centered on". }
 
\par{1. 駅を中心に商店が並んでいます。 \hfill\break
Shops are lined up and centered around the train station. \hfill\break
 \hfill\break
2. 今日の会議では海外展開を中心に議論をしましょう。 \hfill\break
Let's discuss overseas expansion as the center topic at today's meeting. \hfill\break
 \hfill\break
3. 市の中心部に住んでいます。 \hfill\break
I live in the center part of the city. \hfill\break
 \hfill\break
4. 金が議論の中心だ。 \hfill\break
Money is the heart of controversy. \hfill\break
 \hfill\break
5. 議論の中心点 \hfill\break
The central point of the discussion }
      
\section{上}
 
\par{ As a basic noun, 上 means "top\slash surface\slash above". In 上に it is used as the native version of 以上に. }

\begin{ltabulary}{|P|P|}
\hline 

1. & の上 shows something in concern with X. \\ \cline{1-2}

2. & のうえに it shows the basis for something to be established--upon. \\ \cline{1-2}

3. & した・のうえで means "after" or "as an effect of". \\ \cline{1-2}

4. & したうえに and な・のうえに mean "in addition to". \\ \cline{1-2}

5. & Aのうえに(も)A shows the seriousness of the degree of something. \\ \cline{1-2}

6. & かくなるうえは and する・したうえは mean "since it became this such". \\ \cline{1-2}

7. & の上 is after the calling name of a superior or the wife of an aristocrat. \\ \cline{1-2}

\end{ltabulary}

\par{\textbf{Usage Note }: 上 may also be used as a suffix when placed after family members and is used in respect to a superior's family. }
 
\par{\textbf{Orthography Note }: Notice the usages  below where its left in かな. }
 
\begin{center}
\textbf{Examples } 
\end{center}

\par{6. 風船が上へ上へと昇った。 \hfill\break
The balloon went up and up. }
 
\par{7. 空の上から下界を望む。 \hfill\break
To see the world from the sky. }
 
\par{8. 母上 \hfill\break
Mother (of a superior) }
 
\par{9. 坂の上に足跡が残っています。 \hfill\break
There are footprints remaining on the surface of the slope. }
 
\par{10. 数のうえでは圧倒的な状態だ。 \hfill\break
It's an overwhelming situation concerning the numbers. }
 
\par{11. 今日の繁栄は国民の努力の上に築かれているはずです。 \hfill\break
Today's prosperity should be being built upon the efforts of the people. }
 
\par{12. 十分に事件を勘案したうえで返事します。 \hfill\break
I will reply after sufficiently considering the matter. }
 
\par{13. 忙しいうえに給料まで安い。 \hfill\break
In addition to being busy, everything to my wages are low. }
 
\par{14. ご協力頂けるとは好都合のうえにも好都合でございます。 \hfill\break
Receiving your cooperation is beyond expedient. }
 
\par{15. かくなるうえは覚悟を決めよう。 \hfill\break
As it has become this, let's prepare for the worst. }
 
\par{16. この上ともよろしくお願いします。 \hfill\break
Please treat me kindly from here on. }
 
\par{17. 二条院の上。 \hfill\break
Nijouin! }
 
\par{18. 無職であるうえに、家もない。 \hfill\break
If you're unemployed, you also don't have a house. }
      
\section{分}
 
\par{ 分 is a tough character to learn how to read correctly. Although ぶん isn't all that difficult, it does present a challenge. After a number, we know that  ~分 is used in fractions. For example, 8分の6 = 6\slash 8. }
 
\par{It can also mean "amount" or "degree" as in "one's lot". It is this definition that may cause an advanced student to become confused.  In the pattern \dothyp{}\dothyp{}\dothyp{}分だけ it means "in proportion to", being based off of the latter meaning. The use of だけ is optional, but its appearance could be enough to throw you off. }
 
\begin{center}
\textbf{Examples }
\end{center}

\par{19. 彼は飲んだ分だけ、吐いちゃった。 \hfill\break
He ended up throwing up just as much as he drank. }
 
\par{20. 皿は洗ったぶんだけきれいに見えるってわけじゃないよ。 \hfill\break
It's not that the dishes look just as clean as they were washed. }
 
\par{21. 君の分はこれだけだ。 \hfill\break
Your share is just this. }
 
\par{22. 高く登った分だけ、落ちたとき痛い。 \hfill\break
The higher you climb, the harder you fall. }
 
\par{23. この分では終わらないだろう。 \hfill\break
At this rate we won't finish. }
    