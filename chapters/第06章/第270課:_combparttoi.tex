    
\chapter{と Combination Particles I}

\begin{center}
\begin{Large}
第270課: と Combination Particles I: となく \& とは 
\end{Large}
\end{center}
       
\section{となく}
  となく shows the inability to clearly decide something. It may also show a befitting uncertain amount with counter expressions. 何となく is a common phrase with it that means "somehow or another" or "vaguely". \hfill\break
 
\par{1. どことなく似ている。 \hfill\break
They somehow look alike. }
 
\par{2. なんとなく覚えてる。 \hfill\break
I vaguely remember. }

\par{3. ${\overset{\textnormal{いくど}}{\text{幾度}}}$ となく試みる。 \hfill\break
To try out countless times. }

\par{4. 何度か試みる。 \hfill\break
To try it several times. }
 
\par{\textbf{Word Note }: 幾度 is just like 何度も(何度も), しばしば, 何度も繰り返して, and 再三(再四). }
 
\par{5a. 何人となく辞めた。Countless people quit. \hfill\break
5b. 何人か(が)辞めた。Several people quit. }
 
\par{6. 何百\{となく・も\} \hfill\break
By the hundreds }
 
\par{7. あいつは何回となくチャンスがあったよ。 \hfill\break
That guy's had who knows how many times a chance. }

\par{8a. ${\overset{\textnormal{とりあつかいせつめいしょ}}{\text{取扱説明書}}}$ はそれとなくコンピューターに対する静電気の危険性を警告している。 \hfill\break
8b. 取扱説明書は、遠回しにコンピューターの静電気について注意している。 \hfill\break
Guide books warn of the danger to computers by static electricity. }
 
\par{"\dothyp{}\dothyp{}\dothyp{}となく\dothyp{}\dothyp{}\dothyp{}となく" shows a sense of extent to all of a certain thing. }
 
\par{9a. 彼らは昼となく夜となく働いた。 \hfill\break
9b. 彼らは昼夜となく働いた。 \hfill\break
9c. 彼らは昼夜関係なく働いた。 \hfill\break
They worked without day or night. }
 
\par{10a. 老人となく ${\overset{\textnormal{わこうど}}{\text{若人}}}$ となく男となく女となくあらゆる人が泣いた。 \hfill\break
10b. 老人だけでなく若人、男と女が泣き出した。 \hfill\break
Not only the old, not only the young, not only men and women, but all kinds of people cried. }
 
\begin{center}
\textbf{Somehow }
\end{center}
 
\par{There are other expressions like なんとなく. However, like with everything in Japanese, there are going to be differences. The expressions in question are なんか, なんだか, and なんとか(して). }
 
\begin{itemize}
 
\item なんか is      equivalent to など. Or, the slang      version of 何か used as an      adverb to show a feeling of obscurity.  
\item なんだか may just be      the same as 何であるか・何か in an      embedded sentence. Or, it is interchangeable with なんとなく・なぜか to      show vagueness\slash uncertainty in reason.  
\item なんとか(して) = なんらかの方法で  
\item どことなく = なんとかして.      Can't seem to show where.  
\end{itemize}
 
\par{11. 宿題なんか要らないよ。 \hfill\break
I don't need something like homework. }
 
\par{12. 何だか変だね。 \hfill\break
Isn't it strange for some reason? }
 
\par{13. 何だか妙だ。 \hfill\break
It's odd for some reason. }
 
\par{14. 何かの ${\overset{\textnormal{ひょうし}}{\text{拍子}}}$ で \hfill\break
By some chance }
 
\par{15. なんかキモい。(Slang) \hfill\break
I feel bad. }
 
\par{16. なんとかしてそこへ ${\overset{\textnormal{たど}}{\text{辿}}}$ り着きます。 \hfill\break
I'll get there somehow. \hfill\break
 \hfill\break
17. なんとかしてくれ! \hfill\break
Do something about it! }
 
\par{18. なんとかなるさ。 \hfill\break
Things'll work out. }
 
\par{19. 僕はなんとなく悲しい。 \hfill\break
I'm sad somehow or another. }
      
\section{とは}
 
\par{ とは may show surprise meaning "to think that\dothyp{}\dothyp{}\dothyp{}" as a final particle. It may also be used to in defining a word. }

\par{20. ブラックホールとは、 ${\overset{\textnormal{}}{\text{一体}}}$ どんなものなのでしょうか。 \hfill\break
What kind of thing is a black hole exactly? }
 
\par{${\overset{\textnormal{}}{\text{21. 方言}}}$ とは ${\overset{\textnormal{}}{\text{何}}}$ でしょうか。 \hfill\break
What is a dialect? }
 
\par{${\overset{\textnormal{}}{\text{22. 小川}}}$ さんとは、 ${\overset{\textnormal{}}{\text{小学校}}}$ の ${\overset{\textnormal{}}{\text{時}}}$ からの ${\overset{\textnormal{}}{\text{友だち}}}$ です。 \hfill\break
I've been friends with Ogawa since elementary school. }
 
\par{23. もう ${\overset{\textnormal{}}{\text{今年}}}$ も12 ${\overset{\textnormal{}}{\text{月}}}$ とは。 \hfill\break
To think that it's already December this year too! }
 
\par{${\overset{\textnormal{}}{\text{24. 標準語}}}$ とは20 ${\overset{\textnormal{}}{\text{世紀初頭}}}$ に ${\overset{\textnormal{}}{\text{誰}}}$ でも ${\overset{\textnormal{}}{\text{分}}}$ かる ${\overset{\textnormal{}}{\text{標準的}}}$ な ${\overset{\textnormal{}}{\text{日本語}}}$ を ${\overset{\textnormal{}}{\text{定}}}$ めるために、 ${\overset{\textnormal{}}{\text{当時}}}$ の ${\overset{\textnormal{}}{\text{東京下町弁}}}$ に ${\overset{\textnormal{}}{\text{基}}}$ づいて ${\overset{\textnormal{}}{\text{文法的}}}$ によく ${\overset{\textnormal{}}{\text{通}}}$ じる ${\overset{\textnormal{}}{\text{新方言}}}$ として ${\overset{\textnormal{}}{\text{人工的}}}$ に ${\overset{\textnormal{}}{\text{生}}}$ み ${\overset{\textnormal{}}{\text{出さ}}}$ れてしまったものなんです。 \hfill\break
Hyojungo is a grammatically sound "new dialect" artificially created in the beginning of the 20th century to make Standard Japanese for everyone to understand based on Tokyo Shitamachi Dialect of the time }
    