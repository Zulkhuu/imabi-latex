    
\chapter{Reduplication}

\begin{center}
\begin{Large}
第261課: Reduplication: Adjectives 
\end{Large}
\end{center}
 
\par{ As we have learned, reduplication is an important means of creating new words in Japanese. In this lesson, we will focus on how this affects the construction of adjectival phrases. }

\par{ Many new adjectives are created by doubling a word and following it with – \emph{shii }しい. All sorts of things can be morphed into new adjectives in this way. Everything from nouns to adjectives, adjectival nouns, verbs, prefixes, and things that would otherwise not be used as words in isolation may be doubled and followed by – \emph{shii }しい to make new adjectives. }

\par{ The words that are created in this manner are all adjectives with heightened descriptive power capable of capturing the speaker\textquotesingle s feelings and the meaning itself will always be specialized. }

\par{\textbf{Spelling Note }: As we learned when studying repetition in nouns, the second element of a reduplicated phrase is usually written as 々 when it is only one character long and written out fully if longer than one character. }

\par{i. ${\overset{\textnormal{とげとげ}}{\text{刺々}}}$ しい \hfill\break
 \emph{Togetogeshii }\hfill\break
To be thorny\slash snappy \hfill\break
 \hfill\break
ii. ${\overset{\textnormal{さ}}{\text{冴}}}$ え ${\overset{\textnormal{ざ}}{\text{冴}}}$ \textbf{え }しい \hfill\break
 \emph{Sae \textbf{zaes }hii \hfill\break
 }To be beyond vivid and refreshing }

\par{\textbf{Pronunciation Note }: The phonological phenomenon \emph{rendaku }連濁 still applies for reduplication in adjectives. As i. demonstrates, however, if the second syllable of the second element is voiced, even if the initial consonant of said second element is devoiced, it will not become voiced. }

\par{iii. ${\overset{\textnormal{しらじら}}{\text{白々}}}$ しい \hfill\break
 \emph{Shira \textbf{jira }shii \hfill\break
 }Barefaced\slash pure white }

\par{iv. くどくどしい \hfill\break
 \emph{Kudo \textbf{kudo }shii }\hfill\break
Verbose }
      
\section{Examples}
 
\begin{center}
\textbf{From Nouns }
\end{center}

\par{1. ${\overset{\textnormal{かいじょう}}{\text{会場}}}$ に ${\overset{\textnormal{ものもの}}{\text{物々}}}$ \textbf{しい }${\overset{\textnormal{ふんいき}}{\text{雰囲気}}}$ が ${\overset{\textnormal{ただよ}}{\text{漂}}}$ った。 \hfill\break
 \emph{Kaijō ni \textbf{monomonoshii }\textbf{f }un\textquotesingle iki ga tadayotta. }\hfill\break
An imposing atmosphere floated in the grounds. }

\par{2. お ${\overset{\textnormal{じょう}}{\text{嬢}}}$ さんは ${\overset{\textnormal{いぜん}}{\text{以前}}}$ にも ${\overset{\textnormal{ま}}{\text{増}}}$ して ${\overset{\textnormal{かおだ}}{\text{顔立}}}$ ちが ${\overset{\textnormal{ふくぶく}}{\text{}}}$ しくなっていました。 \hfill\break
 \emph{Ojō-san wa izen ni mo mashite kaodachi ga \textbf{fukubukushiku }\textbf{ }natte imashita. }\hfill\break
Her looks were more plump and happy-looking than ever before. }

\par{3. よく、 ${\overset{\textnormal{ずうずう}}{\text{図々}}}$ \textbf{しい }${\overset{\textnormal{ひと}}{\text{人}}}$ のことを「 ${\overset{\textnormal{つら}}{\text{面}}}$ の ${\overset{\textnormal{かわ}}{\text{皮}}}$ が ${\overset{\textnormal{あつ}}{\text{厚}}}$ い」と ${\overset{\textnormal{い}}{\text{言}}}$ います。 \hfill\break
 \emph{Yoku, \textbf{zūzūshii }hito no koto wo “tsura no kawa ga atsui” to iimasu. }\hfill\break
We often call shameless \textbf{ }people “brazen-faced.” }

\par{4. ${\overset{\textnormal{かいがい}}{\text{甲斐甲斐}}}$ \textbf{しく }${\overset{\textnormal{た}}{\text{立}}}$ ち ${\overset{\textnormal{はたら}}{\text{働}}}$ いている ${\overset{\textnormal{がいこくじん}}{\text{外国人}}}$ が ${\overset{\textnormal{おおぜい}}{\text{大勢}}}$ いました。 \hfill\break
 \textbf{\emph{Kaigaishiku }}\emph{t }\emph{achihataraite iru gaikokujin ga ōzei imashita. }\hfill\break
There were many foreigners diligently \textbf{ }at work. }

\par{5. ${\overset{\textnormal{ようちゅう}}{\text{幼虫}}}$ は \textbf{毒々しい }${\overset{\textnormal{しきさい}}{\text{色彩}}}$ を ${\overset{\textnormal{も}}{\text{持}}}$ つ。 \hfill\break
 \emph{Yōchū wa \textbf{dokudokushii }shikisai wo motsu. }\hfill\break
The larvae have a \textbf{poisonous-looking }color scheme. }

\par{6. ${\overset{\textnormal{てんいん}}{\text{店員}}}$ さんの ${\overset{\textnormal{おうたい}}{\text{応対}}}$ も ${\overset{\textnormal{かどかど}}{\text{角々}}}$ \textbf{しい }ところはない。 \hfill\break
 \emph{Ten\textquotesingle in-san no ōtai mo \textbf{kadokadoshii }tokoro wa nai. }\hfill\break
There isn\textquotesingle t anything angular about the clerk\textquotesingle s reception either. }

\par{7. そんな ${\overset{\textnormal{そらぞら}}{\text{空々}}}$ \textbf{しい }${\overset{\textnormal{うそ}}{\text{嘘}}}$ をつかなくてもいいんだよ。 \hfill\break
 \emph{Son\textquotesingle na \textbf{sorazorashii }uso wo tsukanakute mo ii n da yo. }\hfill\break
There's no need for you to make such a \textbf{false }\textbf{ }lie. }

\par{8. ${\overset{\textnormal{めめ}}{\text{女々}}}$ \textbf{しい }${\overset{\textnormal{おとこ}}{\text{男}}}$ ってどんな ${\overset{\textnormal{ひと}}{\text{人}}}$ でしょうか。 \hfill\break
 \textbf{\emph{Memeshii }}\emph{otoko tte don\textquotesingle na hito deshō ka? }\hfill\break
What kind of person is a feminine \textbf{ }man? }

\par{9. ${\overset{\textnormal{りこん}}{\text{離婚}}}$ したくないなら、あなた ${\overset{\textnormal{じしん}}{\text{自身}}}$ が ${\overset{\textnormal{おお}}{\text{雄々}}}$ \textbf{しい }${\overset{\textnormal{おとこ}}{\text{男}}}$ になる ${\overset{\textnormal{ひつよう}}{\text{必要}}}$ があります。 \hfill\break
 \emph{Rikon shitakunai nara, anata jishin ga \textbf{ōshii }\textbf{o }toko ni naru hitsuyō ga arimasu. }\hfill\break
If you don\textquotesingle t wish to divorce, you yourself must become manly . }

\par{10. この ${\overset{\textnormal{よ}}{\text{世}}}$ には、 ${\overset{\textnormal{まがまが}}{\text{禍々}}}$ \textbf{しい }オーラを ${\overset{\textnormal{まと}}{\text{纏}}}$ った ${\overset{\textnormal{ひと}}{\text{人}}}$ が ${\overset{\textnormal{おお}}{\text{多}}}$ くいます。 \hfill\break
 \emph{Kono yo ni wa, \textbf{magamagashii }\textbf{ }ōra wo matotta hito ga ōku imasu. }\hfill\break
There are many people in this world that put on an ominous aura. }

\par{\textbf{Word Note }: \emph{Maga }is an ancient word meaning “disaster.” }

\par{11. ${\overset{\textnormal{ほんとう}}{\text{本当}}}$ に ${\overset{\textnormal{こうごう}}{\text{神々}}}$ \textbf{しい }${\overset{\textnormal{じんじゃ}}{\text{神社}}}$ です。 \hfill\break
 \emph{Hontō ni \textbf{kōgōshii }jinja desu. }\hfill\break
It is a very \textbf{sublime }temple. }

\par{\textbf{Word Note }: \emph{Kōgōshii }神々しい derives from a sound change of \emph{kamigami }. This then became \emph{kamugamu }which became \emph{kaugau }, which finally resulted in \emph{kōgō }. }

\par{12. ${\overset{\textnormal{ことごと}}{\text{事々}}}$ \textbf{しく }${\overset{\textnormal{い}}{\text{言}}}$ い ${\overset{\textnormal{わけ}}{\text{訳}}}$ をする。 \hfill\break
 \textbf{\emph{Kotogoshiku }}\emph{iiwake wo suru. }\hfill\break
To \textbf{pretentiously }make an excuse. }

\par{\textbf{Usage Note }: \emph{Kotogotoshii }事々しいis particularly rare and is usually rephrased with something else. }

\par{13. ${\overset{\textnormal{かれ}}{\text{彼}}}$ はいつも ${\overset{\textnormal{しらじら}}{\text{白々}}}$ \textbf{しい }${\overset{\textnormal{うそ}}{\text{嘘}}}$ ばかりで ${\overset{\textnormal{はら}}{\text{腹}}}$ が ${\overset{\textnormal{た}}{\text{立}}}$ ちます。 \hfill\break
 \emph{Kare wa itsu mo \textbf{shirajirashii }uso bakari de hara ga tachimasu. }\hfill\break
I'm mad at how he always makes just the most glaring lies. }

\par{\textbf{Word Note }: \emph{Shira }is a form of the noun \emph{shiro }白 (white) used specifically in compounds. }

\par{14. ${\overset{\textnormal{かのじょ}}{\text{彼女}}}$ は \textbf{けばけばしく }${\overset{\textnormal{けしょう}}{\text{化粧}}}$ をしていた。 \hfill\break
 \emph{Kanojo wa \textbf{kebakebashiku }keshō wo shite ita. }\hfill\break
She \textbf{gaudily }wore her makeup. }

\par{\textbf{Spelling Note }: \emph{Kebakebashii }may also be spelled as 毳々しい. }

\par{\textbf{Word Note }: The word \emph{keba }is actually a noun meaning fuzz, which can be spelled as 毛羽 or 毳. }

\par{15. ${\overset{\textnormal{つやつや}}{\text{艶々}}}$ \textbf{しい }${\overset{\textnormal{かみ}}{\text{髪}}}$ をまっすぐに ${\overset{\textnormal{た}}{\text{垂}}}$ らす。 \hfill\break
 \textbf{\emph{Tsuyatsuyashii }}\emph{k }\emph{ami wo massugu ni tarasu. }\hfill\break
To hang down one\textquotesingle s \textbf{glossy }hair straight. }

\par{\textbf{Word Note }: \emph{Tsuya }艶 is a noun meaning “gloss\slash charm.” }

\par{16. ${\overset{\textnormal{にぎにぎ}}{\text{賑々}}}$ \textbf{しい }${\overset{\textnormal{はんかがい}}{\text{繁華街}}}$ で ${\overset{\textnormal{ゆうしょく}}{\text{夕食}}}$ する。 \hfill\break
 \textbf{\emph{ }}\textbf{ \emph{Niginigishii }}\emph{ }\emph{hankagai de yūshoku suru. }\hfill\break
To have dinner in a \textbf{lively }shopping district. }

\par{\textbf{Word Note }: The word \emph{nigi }is an ancient noun for “bustle” which has not been used in isolation in a very long time. }

\par{17. ${\overset{\textnormal{せいか}}{\text{成果}}}$ は ${\overset{\textnormal{はかばか}}{\text{捗々}}}$ \textbf{しくなかった }。 \hfill\break
 \emph{Seika wa \textbf{hakabakashikunakatta }\textbf{. }}\hfill\break
\textbf{ }The results \textbf{were not satisfactory. }}

\par{\textbf{Word Note }: The \emph{haka }comes from an old noun referring to work load\slash progress found in other words like \emph{hakaru }計る・測る・図る・量る・諮る・謀る (the nuances and spellings revolve around different senses of devising\slash measuring\slash deliberating something) and \emph{hakadoru }捗る (to make progress). }

\par{18. ドバイの ${\overset{\textnormal{にせんじゅうなな}}{\text{2017}}}$ ${\overset{\textnormal{ねん}}{\text{年}}}$ は ${\overset{\textnormal{まちぜんたい}}{\text{街全体}}}$ を ${\overset{\textnormal{て}}{\text{照}}}$ らす ${\overset{\textnormal{はなび}}{\text{花火}}}$ とともに ${\overset{\textnormal{はなばな}}{\text{華々}}}$ \textbf{しく }${\overset{\textnormal{まく}}{\text{幕}}}$ を ${\overset{\textnormal{あ}}{\text{開}}}$ けた。 \hfill\break
 \emph{Dobai no nisenjūnananen wa machi zentai wo terasu hanabi to tomo ni \textbf{hanabanashiku }\textbf{ }maku wo aketa. }\hfill\break
2017 in Dubai kicked off \textbf{splendidly }with fireworks which lit up the whole city. }

\par{\textbf{Word Note }: 華 is another form of 花 (flower). }

\par{19. ここは ${\overset{\textnormal{と}}{\text{採}}}$ れたての ${\overset{\textnormal{みずみず}}{\text{瑞々}}}$ \textbf{しい }${\overset{\textnormal{やさい}}{\text{野菜}}}$ を ${\overset{\textnormal{まいにちた}}{\text{毎日食}}}$ べるという ${\overset{\textnormal{きらく}}{\text{気楽}}}$ さを ${\overset{\textnormal{たの}}{\text{楽}}}$ しめる ${\overset{\textnormal{まち}}{\text{町}}}$ です。 \hfill\break
 \emph{Koko wa toretate no \textbf{mizumizushii }\textbf{ }yasai wo mainichi taberu to iu kirakusa wo tanoshimeru machi desu. }\hfill\break
This town here is a place where you can enjoy the ease of eating \textbf{fresh }, \textbf{ }picked vegetables every day. }

\par{\textbf{Word Note }: \emph{Mizu }瑞 is an ancient noun meaning “purity\slash luster.” }

\begin{center}
\textbf{From Adjectives\slash Adjectival Nouns }
\end{center}

\par{20. ${\overset{\textnormal{いたいた}}{\text{痛々}}}$ \textbf{しい }ニュースが ${\overset{\textnormal{あいつ}}{\text{相次}}}$ いでいる。 \hfill\break
 \textbf{\emph{Itaitashii }}\emph{ }\emph{nyūsu ga aitsuide iru. }\hfill\break
 \textbf{Painful }news keeps coming in one after another. }

\par{21. ${\overset{\textnormal{さいきん}}{\text{最近}}}$ は、 ${\overset{\textnormal{み}}{\text{見}}}$ た ${\overset{\textnormal{めねんれい}}{\text{目年齢}}}$ が ${\overset{\textnormal{わかわか}}{\text{若々}}}$ \textbf{しい }${\overset{\textnormal{じょせい}}{\text{女性}}}$ が ${\overset{\textnormal{ふ}}{\text{増}}}$ えている。 \hfill\break
 \emph{ }\emph{Saikin wa, mita me nenrei ga \textbf{wakawakashii }josei ga fuete iru. }\hfill\break
Recently, the amount of women whose apparent age is youthful is increasing. }

\par{22. ${\overset{\textnormal{ばかばか}}{\text{馬鹿馬鹿}}}$ \textbf{しい }${\overset{\textnormal{はなし}}{\text{話}}}$ はよしましょう。 \hfill\break
 \textbf{\emph{Bakabakashii }}\emph{hanashi wa yoshimashō. }\hfill\break
Let\textquotesingle s stop it with the \textbf{ludicrous }\textbf{ }talks. }

\par{23. ${\overset{\textnormal{けいすけ}}{\text{啓祐}}}$ は ${\overset{\textnormal{すうびょう}}{\text{数秒}}}$ の ${\overset{\textnormal{ちんもく}}{\text{沈黙}}}$ を ${\overset{\textnormal{やぶ}}{\text{破}}}$ って ${\overset{\textnormal{よわよわ}}{\text{弱々}}}$ \textbf{しい }${\overset{\textnormal{こえ}}{\text{声}}}$ を ${\overset{\textnormal{だ}}{\text{出}}}$ した。 \hfill\break
 \emph{K }\emph{eisuke wa sūbyō no chimmoku wo yabutte \textbf{yowayowashii }\textbf{ }koe wo dashita. }\hfill\break
Keisuke broke the several seconds of silence, speaking frailly . }

\par{24. とある ${\overset{\textnormal{こまごま}}{\text{細々}}}$ \textbf{しい }${\overset{\textnormal{もの}}{\text{物}}}$ が ${\overset{\textnormal{まと}}{\text{纏}}}$ まっている ${\overset{\textnormal{しょうひん}}{\text{商品}}}$ を ${\overset{\textnormal{こうにゅう}}{\text{購入}}}$ しました。 \hfill\break
 \emph{To aru \textbf{komagomashii }mono ga matomatte iru shōhin wo kōnyū shimashita. }\hfill\break
I purchased merchandise with certain very fine \textbf{ }parts bunched up. }

\par{25. ${\overset{\textnormal{はは}}{\text{母}}}$ はいつも ${\overset{\textnormal{にがにが}}{\text{苦々}}}$ \textbf{しい }${\overset{\textnormal{かお}}{\text{顔}}}$ をしていた。 \hfill\break
 \emph{Haha wa itsu mo \textbf{niganigashii }\textbf{ }kao wo shite ita. }\hfill\break
My mother always had an unpleasant \textbf{ }look on her face. }

\par{26. スタジアムは \textbf{${\overset{\textnormal{おもおも}}{\text{重々}}}$ しい }${\overset{\textnormal{ふんいき}}{\text{雰囲気}}}$ に ${\overset{\textnormal{つつ}}{\text{包}}}$ まれていた。 \hfill\break
 \emph{Sutajiamu wa \textbf{omo\textquotesingle omoshii }fun\textquotesingle iki ni tsutsumarete ita. }\hfill\break
A \textbf{serious }atmosphere enveloped the stadium. }

\par{27. ${\overset{\textnormal{おだのぶなが}}{\text{織田信長}}}$ は ${\overset{\textnormal{たんき}}{\text{短気}}}$ で ${\overset{\textnormal{あらあら}}{\text{荒々}}}$ \textbf{しい }${\overset{\textnormal{せいかく}}{\text{性格}}}$ を ${\overset{\textnormal{も}}{\text{持}}}$ っていたといわれる。 \hfill\break
 \emph{Oda Nobunaga wa tanki de \textbf{ara\textquotesingle arashii }\textbf{s }eikaku wo motte ita to iwareru. }\hfill\break
It is said that Oda Nobunaga had a quick temper and rough \textbf{ }personality. }

\par{28. ${\overset{\textnormal{かるがる}}{\text{軽々}}}$ \textbf{しい }${\overset{\textnormal{こうどう}}{\text{行動}}}$ を ${\overset{\textnormal{と}}{\text{取}}}$ っていると、 ${\overset{\textnormal{かんしたいしょう}}{\text{監視対象}}}$ になり、アカウントの ${\overset{\textnormal{とうけつ}}{\text{凍結}}}$ などの ${\overset{\textnormal{きび}}{\text{厳}}}$ しい ${\overset{\textnormal{しょち}}{\text{処置}}}$ になる ${\overset{\textnormal{かのうせい}}{\text{可能性}}}$ があります。 \hfill\break
 \textbf{\emph{Karugarushii }}\emph{ }\emph{kōdō wo totte iru to, kanshi taishō ni nari, akaunto no tōketsu nado no kibishii shochi ni naru kanōsei ga arimasu. }\hfill\break
If you are making careless \textbf{ }actions, you may become the target of surveillance, and there is the possibility of harsh measures such as the freezing of one\textquotesingle s account. }

\par{29. \textbf{ }\textbf{まめまめしく }${\overset{\textnormal{はたら}}{\text{働}}}$ いているのは ${\overset{\textnormal{かれ}}{\text{彼}}}$ だけだ。 \hfill\break
 \textbf{Mamemameshiku }hataraite iru no wa kare dake da. \hfill\break
The only one who is \textbf{painstakingly }working is him. }

\par{\textbf{Spelling Note }: \emph{Mamemameshii }may also be written as 忠実忠実しい. }

\par{30. ○○ ${\overset{\textnormal{せいけん}}{\text{政権}}}$ は、 ${\overset{\textnormal{みずか}}{\text{自}}}$ らの ${\overset{\textnormal{せいさく}}{\text{政策}}}$ を「○○ノミクス」だとかいって ${\overset{\textnormal{ぎょうぎょう}}{\text{仰々}}}$ \textbf{しく }${\overset{\textnormal{せんでん}}{\text{宣伝}}}$ していますが、 ${\overset{\textnormal{あたら}}{\text{新}}}$ しい ${\overset{\textnormal{なかみ}}{\text{中身}}}$ はあるんでしょうか。 \hfill\break
 \emph{Marumaru Seiken wa, mizukara no seisaku wo “marumaru-nomikusu” da toka itte \textbf{gyōgyōshiku }\textbf{ }senden shite imasu ga, atarashii nakami wa aru n deshō ka? }\hfill\break
The \#\# Administration \textbf{bombastically }propagandizes its own policies as “\#\#nomics,” but is there new substance to be had? }

\par{\textbf{Word Note }: \emph{Gyōgyōshii }仰々しい actually derives from the adjectival noun \emph{keu }希有 (rare\slash uncommon) being reduplicated and having its pronunciation and spelling altered. }

\par{31. ${\overset{\textnormal{ほし}}{\text{星}}}$ が ${\overset{\textnormal{あわあわ}}{\text{}}}$ しかった。 \hfill\break
 \emph{Hoshi ga \textbf{awa\textquotesingle awashikatta }. }\hfill\break
The stars were \textbf{faint }. }

\par{\textbf{Usage Note }: \emph{Awa\textquotesingle awashii }淡々しい is particularly rare and is usually rephrased with something else. }

\par{32. ${\overset{\textnormal{そうぞう}}{\text{騒々}}}$ \textbf{しい }${\overset{\textnormal{ばしょ}}{\text{場所}}}$ ではなかなか ${\overset{\textnormal{かいわ}}{\text{会話}}}$ が ${\overset{\textnormal{き}}{\text{聞}}}$ き ${\overset{\textnormal{と}}{\text{取}}}$ れない。 \hfill\break
 \textbf{\emph{Sōzōshii }}\emph{basho de wa nakanaka kaiwa ga kikitorenai. }\hfill\break
I can\textquotesingle t really make out conversations in noisy places. }

\par{\textbf{Word Note }: Although not used in isolation in Japanese, 騒 is being used here as an adjective for being “noisy.” }

\par{33. ${\overset{\textnormal{りり}}{\text{凛々}}}$ \textbf{しい }${\overset{\textnormal{ひと}}{\text{人}}}$ に ${\overset{\textnormal{あこが}}{\text{憧}}}$ れていました。 \hfill\break
 \textbf{\emph{Ririshii }}\emph{hito ni akogarete imashita. }\hfill\break
I was attracted by gallant \textbf{ }people. }

\par{\textbf{Word Note }: \emph{Ririshii }凛々しい derives from \emph{rinrin }凛々 (awe-inspiring\slash bitter cold). }

\begin{center}
\textbf{From Verbs }
\end{center}

\par{34. ${\overset{\textnormal{にぎ}}{\text{賑}}}$ やかで ${\overset{\textnormal{は}}{\text{晴}}}$ \textbf{れ }${\overset{\textnormal{ば}}{\text{晴}}}$ \textbf{れしい }${\overset{\textnormal{くうき}}{\text{空気}}}$ が ${\overset{\textnormal{あふ}}{\text{溢}}}$ れている。 \hfill\break
 \emph{Nigiyaka de \textbf{harebareshii }\textbf{ }kūki ga afurete iru. }\hfill\break
A lively and \textbf{splendid }\textbf{ }atmosphere is abounding. }

\par{35. ${\overset{\textnormal{まわ}}{\text{周}}}$ りに ${\overset{\textnormal{な}}{\text{馴}}}$ \textbf{れ }${\overset{\textnormal{な}}{\text{馴}}}$ \textbf{れしい }${\overset{\textnormal{ひと}}{\text{人}}}$ はいますか。 \hfill\break
 \emph{Mawari ni \textbf{narenareshii }hito wa imasu ka? }\hfill\break
Is there anyone around who is being \textbf{over-familiar }? }

\par{36. \textbf{ }\textbf{ふてぶてしい }${\overset{\textnormal{たいど}}{\text{態度}}}$ を ${\overset{\textnormal{と}}{\text{取}}}$ る ${\overset{\textnormal{ひと}}{\text{人}}}$ は ${\overset{\textnormal{わる}}{\text{悪}}}$ い ${\overset{\textnormal{いんしょう}}{\text{印象}}}$ を ${\overset{\textnormal{も}}{\text{持}}}$ たれてしまう。 \hfill\break
 \textbf{\emph{Futebuteshii }}\emph{taido wo toru hito wa warui inshō wo motarete shimau. }\hfill\break
People who take a \textbf{brazen }attitude give out bad impressions. }

\par{\textbf{Word Note }: \emph{Futebuteshii }ふてぶてしい derives from the verb \emph{futeru }ふてる, which is a dialect variation of the verb \emph{futekusareru }ふて腐れる (to become sulky), which it also gave rise to. \emph{Futebuteshii }may alternatively be written as 太々しい, 不貞不貞しい, or 不敵不敵しい. }

\par{37. ${\overset{\textnormal{わか}}{\text{若}}}$ いころの ${\overset{\textnormal{いまいま}}{\text{忌々}}}$ \textbf{しい }${\overset{\textnormal{きおく}}{\text{記憶}}}$ を ${\overset{\textnormal{おも}}{\text{思}}}$ い ${\overset{\textnormal{だ}}{\text{出}}}$ したくない。 \hfill\break
 \emph{Wakai koro no \textbf{imaimashii }kioku wo omoidashitakunai. }\hfill\break
I don't want to remember the \textbf{provoking }memories from when I was young. }

\par{\textbf{Word Note }: The \emph{ima }comes from the verb \emph{imu }忌む (to detest\slash shun). }

\par{38. ${\overset{\textnormal{ぬすびとたけだけ}}{\text{盗人}}}$ \textbf{しい }。 \hfill\break
 \emph{Nusubito \textbf{takedakeshii }. }\hfill\break
The guilty are \textbf{audacious }. }

\par{\textbf{Word Note }: \emph{Takedakeshii }猛々しい derives from the verb \emph{takeru }猛る (to rage). However, this verb form is now quite rare. }

\par{39. ${\overset{\textnormal{すがすが}}{\text{清々}}}$ \textbf{しい }${\overset{\textnormal{ひょうじょう}}{\text{表情}}}$ で ${\overset{\textnormal{きしゃかいけん}}{\text{記者会見}}}$ に ${\overset{\textnormal{のぞ}}{\text{臨}}}$ む。 \hfill\break
 \textbf{\emph{Sugasugashii }}\emph{hyōjō de kisha kaiken ni nozomu. }\hfill\break
To appear in a press conference with a \textbf{brisk }\textbf{ }expression. }

\par{\textbf{Word Note }: The \emph{suga }comes from the verb \emph{sugiru }過ぎる (to pass by). }

\begin{center}
\textbf{None of the Above }
\end{center}

\par{40. お ${\overset{\textnormal{ふたり}}{\text{二人}}}$ とも ${\overset{\textnormal{にじゅう}}{\text{20}}}$ ${\overset{\textnormal{だいぜんはん}}{\text{代前半}}}$ の ${\overset{\textnormal{ういうい}}{\text{初々}}}$ \textbf{しい }カップルです。 \hfill\break
 \emph{Ofutari tomo nijūdai zenhan no \textbf{uiuishii }kappuru desu. }\hfill\break
They are an unspoiled \textbf{ }couple both in their early twenties. }

\par{\textbf{Word Note }: \emph{Uiuishii }初々しい derives from the prefix \emph{ui }- 初, which means “first” and is seen in a handful of words such as \emph{uizan }初産 (first childbirth). }
    