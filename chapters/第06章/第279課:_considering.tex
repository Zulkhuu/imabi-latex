    
\chapter{Considering\dothyp{}\dothyp{}\dothyp{}}

\begin{center}
\begin{Large}
第279課: Considering\dothyp{}\dothyp{}\dothyp{}: わりに(は) \& にしては 
\end{Large}
\end{center}
 
\par{ In this lesson, we will learn about two related expressions used when you\textquotesingle re encountered with unexpected situations. }
      
\section{わりに(は)}
 
\par{ In the expression “A + わりに(は)+ B,” B is an unexpected outcome\slash judgment of A. The addition or omission of は is based on how much emphasis you wish to place on the contrast to one\textquotesingle s preconception. }

\par{[Formation] }

\par{ This expression is used with nouns, adjectives, adjectival nouns, and verbs. This includes past tense and negative forms of said parts of speech that can conjugate. }

\begin{ltabulary}{|P|P|}
\hline 

Noun & Noun + \{の・である\}わりに(は) \\ \cline{1-2}

Adjectives & Adj. + わりに(は) \\ \cline{1-2}

Adjectival Nouns & Adj. Noun + わりに(は) \\ \cline{1-2}

Verbs & Plain Form + わりに(は) \\ \cline{1-2}

\end{ltabulary}

\par{\textbf{Grammar Note }: The use of である instead of の with nouns is only seen in old-fashioned speech largely limited to the written language. }

\par{[Usage] }

\par{This pattern is frequently used in both positive and negative connotations, but it is best used with generalizations rather than with exact details. However, it is grammatical so long as the phrase it attaches to does not involve individual things\slash people. }

\par{ Below are example sentences with the various formation patterns from the table above. }

\par{1. カルビは、 ${\overset{\textnormal{やす}}{\text{安}}}$ いわりには ${\overset{\textnormal{おい}}{\text{美味}}}$ しい。 \hfill\break
Considering how cheap it is, galbi is delicious. }

\par{\textbf{Word Note }: Galbi is a loanword from Korean meaning ${\overset{\textnormal{ろっこつ}}{\text{肋骨}}}$ (rib). In context of cooking, it refers to meat around the rib, which if in a non-Korean cuisine context is called ばら ${\overset{\textnormal{にく}}{\text{肉}}}$ . }

\par{2. ${\overset{\textnormal{べんきょう}}{\text{勉強}}}$ しなかったわりには、よくできたなあ。 \hfill\break
Considering how you didn\textquotesingle t study, you did pretty well. }

\par{3. ${\overset{\textnormal{わたし}}{\text{私}}}$ は、 ${\overset{\textnormal{ふと}}{\text{太}}}$ ってるわりにはあんまり ${\overset{\textnormal{た}}{\text{食}}}$ べません。 \hfill\break
Considering how much I weigh, I hardly eat. }

\par{4. ${\overset{\textnormal{い}}{\text{井}}}$ ノ ${\overset{\textnormal{かわ}}{\text{川}}}$ さんは、 ${\overset{\textnormal{とし}}{\text{歳}}}$ のわりにはとても ${\overset{\textnormal{わか}}{\text{若}}}$ く ${\overset{\textnormal{み}}{\text{見}}}$ えますね。 \hfill\break
Considering his age, Mr. Inokawa sure looks young, huh. }

\par{5. うちの ${\overset{\textnormal{かれし}}{\text{彼氏}}}$ なんですが、 ${\overset{\textnormal{なんかい}}{\text{何回}}}$ も ${\overset{\textnormal{けっこん}}{\text{結婚}}}$ したくないと ${\overset{\textnormal{い}}{\text{言}}}$ っているわりには、 ${\overset{\textnormal{たわいな}}{\text{他愛無}}}$ い ${\overset{\textnormal{はなし}}{\text{話}}}$ の ${\overset{\textnormal{なか}}{\text{中}}}$ で、いつも ${\overset{\textnormal{こども}}{\text{子供}}}$ や ${\overset{\textnormal{けっこんしき}}{\text{結婚式}}}$ の ${\overset{\textnormal{はなし}}{\text{話}}}$ をしたりします。 \hfill\break
Considering how many times my boyfriend says he doesn\textquotesingle t want to get married, he always talks about kids and having a wedding in silly talk. }

\par{6. ${\overset{\textnormal{りょうきん}}{\text{料金}}}$ が ${\overset{\textnormal{やす}}{\text{安}}}$ かったわりには ${\overset{\textnormal{かんぺき}}{\text{完璧}}}$ でしたよ。 \hfill\break
Considering how cheap the fee\slash rate was, it was perfect. }

\par{7. ${\overset{\textnormal{じゅう}}{\text{10}}}$ ${\overset{\textnormal{ねんいじょうちゅうごくご}}{\text{年以上中国語}}}$ を ${\overset{\textnormal{べんきょう}}{\text{勉強}}}$ したわりには、あまり ${\overset{\textnormal{じょうず}}{\text{上手}}}$ になってない ${\overset{\textnormal{き}}{\text{気}}}$ がします。 \hfill\break
Considering how I\textquotesingle ve studied Chinese for over ten years, I feel like I haven\textquotesingle t become all that good at it. }

\par{8. ${\overset{\textnormal{せひょう}}{\text{世評}}}$ が ${\overset{\textnormal{たか}}{\text{高}}}$ かったわりには、すごくつまらなすぎて ${\overset{\textnormal{に}}{\text{2}}}$ ${\overset{\textnormal{ど}}{\text{度}}}$ と ${\overset{\textnormal{み}}{\text{観}}}$ たくもない。 \hfill\break
Considering how high its review was, it was so boring I don\textquotesingle t even want to watch it again. }

\par{9. ${\overset{\textnormal{こじんてき}}{\text{個人的}}}$ には ${\overset{\textnormal{ひかくてきがいしょく}}{\text{比較的外食}}}$ の ${\overset{\textnormal{かいすう}}{\text{回数}}}$ が ${\overset{\textnormal{おお}}{\text{多}}}$ いわりには、 ${\overset{\textnormal{たいじゅう}}{\text{体重}}}$ が ${\overset{\textnormal{ふ}}{\text{増}}}$ えてない。 \hfill\break
Considering how relatively frequent my eating out personally is, my weight isn\textquotesingle t going up. }

\par{10. この ${\overset{\textnormal{こ}}{\text{子}}}$ 、 ${\overset{\textnormal{こがたけん}}{\text{小型犬}}}$ のわりには、落ち ${\overset{\textnormal{つ}}{\text{着}}}$ いているほうです。 \hfill\break
Considering how it\textquotesingle s a small-sized dog, it\textquotesingle s laid-back. }

\par{11. よく ${\overset{\textnormal{た}}{\text{食}}}$ べるわりにあまり ${\overset{\textnormal{ふと}}{\text{太}}}$ ってない。 \hfill\break
I'm not that fat considering how often I eat. }

\par{12. ${\overset{\textnormal{のうぎょう}}{\text{農業}}}$ は、 ${\overset{\textnormal{たいへん}}{\text{大変}}}$ なわりにあまり ${\overset{\textnormal{もう}}{\text{儲}}}$ からない。 \hfill\break
Agriculture doesn\textquotesingle t profit much considering how tough it is. }

\par{13. お ${\overset{\textnormal{かね}}{\text{金}}}$ が ${\overset{\textnormal{な}}{\text{無}}}$ いわりに、よく ${\overset{\textnormal{か}}{\text{買}}}$ い ${\overset{\textnormal{もの}}{\text{物}}}$ するね。 \hfill\break
You sure shop often considering how you don\textquotesingle t have money. }

\par{14. ここは、 ${\overset{\textnormal{ちめいど}}{\text{知名度}}}$ のわりにあまり ${\overset{\textnormal{かんこうちか}}{\text{観光地化}}}$ されていない。 \hfill\break
This place hasn\textquotesingle t really become a tourist area considering its name recognition. }

\par{15. ${\overset{\textnormal{たちばな}}{\text{橘}}}$ さんは、 ${\overset{\textnormal{しこくしゅっしん}}{\text{四国出身}}}$ のわりに、あんまり ${\overset{\textnormal{なま}}{\text{訛}}}$ りを ${\overset{\textnormal{かん}}{\text{感}}}$ じませんね。 \hfill\break
Mr. Tachibana doesn\textquotesingle t have much of an accident considering he\textquotesingle s from Shikoku, no? }

\par{16. ${\overset{\textnormal{かんたん}}{\text{簡単}}}$ なわりに美味しかったね。 \hfill\break
That was delicious considering how easy it was, no? }

\par{17. ${\overset{\textnormal{めんぜいてん}}{\text{免税店}}}$ のわりに、 ${\overset{\textnormal{かわせ}}{\text{為替}}}$ レートがいいよね。 \hfill\break
The exchange rate is good considering it\textquotesingle s a duty-free shop, no? }

\begin{center}
\textbf{The Noun 割(り) } \hfill\break

\end{center}

\par{ わり comes from the noun 割(り). This literally means “ratio\slash percentage.” It can also mean 10\% and other specialized meanings. }

\par{18. ${\overset{\textnormal{さん}}{\text{3}}}$ ${\overset{\textnormal{わりび}}{\text{割引}}}$ きです。 \hfill\break
It\textquotesingle s 30\% off. }

\par{19. もっと ${\overset{\textnormal{わ}}{\text{割}}}$ りのいい ${\overset{\textnormal{しごと}}{\text{仕事}}}$ をしてほしい。 \hfill\break
I\textquotesingle d like you to do a more lucrative job. }

\par{20. おおむね ${\overset{\textnormal{いっ}}{\text{1}}}$ ${\overset{\textnormal{しゅうかん}}{\text{週間}}}$ に ${\overset{\textnormal{いっ}}{\text{1}}}$ ${\overset{\textnormal{かい}}{\text{回}}}$ の ${\overset{\textnormal{わ}}{\text{割}}}$ りで ${\overset{\textnormal{つういん}}{\text{通院}}}$ しています。 \hfill\break
I\textquotesingle m generally commuting to the hospital once a week. }

\par{21. ${\overset{\textnormal{わたし}}{\text{私}}}$ は ${\overset{\textnormal{しょうちゅう}}{\text{焼酎}}}$ をお ${\overset{\textnormal{ゆわ}}{\text{湯割}}}$ りで ${\overset{\textnormal{の}}{\text{飲}}}$ んでいます。 \hfill\break
I drink shochu mixed with hot water. }

\par{\textbf{Word Note }: Shochu is a Japanese alcoholic beverage distilled from rice, barley, sweet potatoes, buckwheat, or brown sugar. }

\par{22. 本割りとは ${\overset{\textnormal{おおずもう}}{\text{大相撲}}}$ おおずもうで ${\overset{\textnormal{はっぴょう}}{\text{発表}}}$ された ${\overset{\textnormal{とりくみおもて}}{\text{取組表}}}$ とりくみひょうによって ${\overset{\textnormal{おこな}}{\text{行}}}$ われる ${\overset{\textnormal{せいき}}{\text{正規}}}$ の ${\overset{\textnormal{とりくみ}}{\text{取組}}}$ である。 \hfill\break
”Honwari" are official matches in professional sumo wrestling held according to the matches table. }

\begin{center}
\textbf{割 (り)\{に・と\} }\hfill\break

\end{center}

\par{ As an adverb, 割(り)に, alternatively seen as 割(り)と, is used to mean “relatively\slash comparatively.” }

\par{23. ${\overset{\textnormal{わたし}}{\text{私}}}$ は ${\overset{\textnormal{かふんしょう}}{\text{花粉症}}}$ です。 ${\overset{\textnormal{しょうじょう}}{\text{症状}}}$ は ${\overset{\textnormal{わ}}{\text{割}}}$ りと ${\overset{\textnormal{おも}}{\text{重}}}$ い ${\overset{\textnormal{ほう}}{\text{方}}}$ だと ${\overset{\textnormal{おも}}{\text{思}}}$ います。 \hfill\break
I have allergies. My symptoms are relatively severe. }

\par{24. ${\overset{\textnormal{しごと}}{\text{仕事}}}$ は ${\overset{\textnormal{だいぶな}}{\text{大分慣}}}$ れたので、 ${\overset{\textnormal{わり}}{\text{割}}}$ と ${\overset{\textnormal{らく}}{\text{楽}}}$ になってきました。 \hfill\break
I\textquotesingle ve gotten used to my job a lot, so it\textquotesingle s become relatively comfortable. }

\par{25. ${\overset{\textnormal{つく}}{\text{作}}}$ って ${\overset{\textnormal{う}}{\text{売}}}$ るだけなら、 ${\overset{\textnormal{わり}}{\text{割}}}$ と ${\overset{\textnormal{かんたん}}{\text{簡単}}}$ ですよ。 \hfill\break
If you\textquotesingle re just making and selling it, then it\textquotesingle s relatively easy. }

\par{26. ${\overset{\textnormal{せいしょがくしゃ}}{\text{聖書学者}}}$ といわれる ${\overset{\textnormal{ひと}}{\text{人}}}$ は、 ${\overset{\textnormal{わり}}{\text{割}}}$ に ${\overset{\textnormal{ふる}}{\text{古}}}$ い ${\overset{\textnormal{いえがら}}{\text{家柄}}}$ の ${\overset{\textnormal{ひと}}{\text{人}}}$ が ${\overset{\textnormal{おお}}{\text{多}}}$ いようです。 \hfill\break
It seems that a lot of people who are called Bible scholars are of relatively old pedigree. }

\begin{center}
\textbf{わりにゃ(あ) } \hfill\break

\end{center}

\par{ In very casual speech, わりには may be contracted to わりにゃ(あ). }

\par{27. ${\overset{\textnormal{じょうひん}}{\text{上品}}}$ そうな ${\overset{\textnormal{かお}}{\text{顔}}}$ のわりにゃ(あ)、やる ${\overset{\textnormal{こと}}{\text{事}}}$ がえげつないなあ。 \hfill\break
Considering how elegant (her) face is, what she does is pretty vulgar. }

\par{28. ${\overset{\textnormal{みせぜんたい}}{\text{店全体}}}$ のスペースのわりにゃ(あ)、 ${\overset{\textnormal{せき}}{\text{席}}}$ が ${\overset{\textnormal{すく}}{\text{少}}}$ ない ${\overset{\textnormal{き}}{\text{気}}}$ がします。 \hfill\break
It feels like there are few seats considering the store\textquotesingle s space as a whole. }

\par{29. ${\overset{\textnormal{きたい}}{\text{期待}}}$ したわりにゃ(あ)、 ${\overset{\textnormal{たい}}{\text{大}}}$ したことでもなかったな。 \hfill\break
It really wasn\textquotesingle t all that much considering how much I was expecting it to be. }

\par{30. あいつはアホなわりにゃ(あ)、よく ${\overset{\textnormal{のうが}}{\text{能書}}}$ き ${\overset{\textnormal{た}}{\text{垂}}}$ れるよな。 \hfill\break
That guy sure likes to boast considering he\textquotesingle s an idiot. }
      
\section{にしては}
 \hfill\break
 The pattern “A + にしては + B”  is also used to describe a situation that is contrary to one\textquotesingle s expectation. It can be used in both positive and negative contexts. As far as tone is concerned, it typically isn\textquotesingle t used in harsh critiques. Any negative context is limited to disappointment\slash regret.   
\par{[Formation] }
 
\par{This pattern cannot be used with adjectives or adjectival nouns. This is due to the fact that adjectives\slash adjectival nouns by nature don\textquotesingle t have to refer to exact standards. Even when it's used with verbs, the situation it follows is always well-defined. }

\begin{ltabulary}{|P|P|}
\hline 

Nouns & Noun + にしては \\ \cline{1-2}

Verbs & Plain Form + にしては \\ \cline{1-2}

\end{ltabulary}

\par{\textbf{Translation Note }: This pattern is usually translated as “for…” or “considering…” }

\par{[Usage] }

\par{ With concrete concepts, especially those concerning quality, degree, age, price, etc., にしては is interchangeable with わりに(は). However, one defining difference is that にしては  can be used when the situation is hypothetical. However, whether the situation is concrete or hypothetical, the standard of comparison needs to be made clear. For instance, “considering…age” can be translated as ${\overset{\textnormal{ねんれい}}{\text{年齢}}}$ のわりに(は), but you can only use にしては if you replace 年齢 with something more specific like ${\overset{\textnormal{ろくじゅっ}}{\text{60}}}$ ${\overset{\textnormal{さい}}{\text{歳}}}$ (sixty years old). }

\par{ As such, にしては isn\textquotesingle t used with generalizations. The situation itself may be concrete or hypothetical in nature, but there is a standard of some sort that must be established in order for it to be grammatical. It\textquotesingle s also important to note that the phrase before it can be an individual person\slash thing, which is not the case with わりに(は). }
 
\begin{center}
\textbf{Examples }
\end{center}
 
\par{31. マックのハンバーガーの ${\overset{\textnormal{ねだん}}{\text{値段}}}$ にしては ${\overset{\textnormal{わり}}{\text{割}}}$ に ${\overset{\textnormal{あ}}{\text{合}}}$ ってるお ${\overset{\textnormal{にく}}{\text{肉}}}$ だと ${\overset{\textnormal{おも}}{\text{思}}}$ いますが、 ${\overset{\textnormal{ちが}}{\text{違}}}$ うんでしょうか。 \hfill\break
For the price of a McDonald\textquotesingle s hamburger, the meat is considerably fitting, am I wrong? }
 
\par{32. この ${\overset{\textnormal{あた}}{\text{辺}}}$ りの ${\overset{\textnormal{うなぎや}}{\text{鰻屋}}}$ にしては ${\overset{\textnormal{わり}}{\text{割}}}$ とうまいと ${\overset{\textnormal{おも}}{\text{思}}}$ います。 \hfill\break
For an eel restaurant in this area, I think it\textquotesingle s relatively good. }
 
\par{33. ${\overset{\textnormal{ほうしゅう}}{\text{報酬}}}$ にしては ${\overset{\textnormal{わり}}{\text{割}}}$ に ${\overset{\textnormal{あ}}{\text{合}}}$ わない。 \hfill\break
For a reward, it\textquotesingle s not used much. }
 
\par{34. ${\overset{\textnormal{なつ}}{\text{夏}}}$ にしては、よく ${\overset{\textnormal{あめ}}{\text{雨}}}$ が ${\overset{\textnormal{ふ}}{\text{降}}}$ りましたね。 \hfill\break
It rained a lot for summer, huh. }
 
\par{35. ${\overset{\textnormal{けんたくん}}{\text{健太君}}}$ は ${\overset{\textnormal{しょうがくせい}}{\text{小学生}}}$ にしてはよく ${\overset{\textnormal{かんじ}}{\text{漢字}}}$ を ${\overset{\textnormal{し}}{\text{知}}}$ ってるね。 \hfill\break
Kentaro sure knows Kanji well for an elementary school student, no? }
 
\par{36. ${\overset{\textnormal{いずみ}}{\text{泉}}}$ さんって、 ${\overset{\textnormal{ななじゅっ}}{\text{70}}}$ ${\overset{\textnormal{さい}}{\text{歳}}}$ にしては ${\overset{\textnormal{わか}}{\text{若}}}$ く ${\overset{\textnormal{み}}{\text{見}}}$ えますね。 \hfill\break
Mr. Izumi sure looks young for 70, doesn\textquotesingle t he? }
 
\par{37. ${\overset{\textnormal{がいこくじん}}{\text{外国人}}}$ にしては ${\overset{\textnormal{にほんご}}{\text{日本語}}}$ が ${\overset{\textnormal{じょうず}}{\text{上手}}}$ すぎてまるで ${\overset{\textnormal{にほんじん}}{\text{日本人}}}$ のようですね。 \hfill\break
Considering he\textquotesingle s a foreigner, (his) Japanese is so good it\textquotesingle s as if he\textquotesingle s Japanese, no? }
 
\par{38. ${\overset{\textnormal{ちゅうごく}}{\text{中国}}}$ にしては ${\overset{\textnormal{こうかかく}}{\text{高価格}}}$ にもかかわらず、ほとんどの ${\overset{\textnormal{どくしゃ}}{\text{読者}}}$ が ${\overset{\textnormal{ていきこうどく}}{\text{定期購読}}}$ をしている。 \hfill\break
Considering it\textquotesingle s China, despite the high price, most readers are subscribers. }
 
\par{39. ${\overset{\textnormal{ちゅうこひん}}{\text{中古品}}}$ にしては ${\overset{\textnormal{ほぞんじょうたい}}{\text{保存状態}}}$ が ${\overset{\textnormal{よ}}{\text{良}}}$ いですね。 \hfill\break
Its condition is good for a secondhand good, no? }
 
\par{40. アライグマにしては ${\overset{\textnormal{すばや}}{\text{素早}}}$ い! \hfill\break
It\textquotesingle s fast for a raccoon! }
 
\par{41. アイスクリームにしては、 ${\overset{\textnormal{ぜんぜんあま}}{\text{全然甘}}}$ くないよ。 \hfill\break
This isn\textquotesingle t sweet at all for ice cream. }
 
\par{42. 「あんたが ${\overset{\textnormal{つく}}{\text{作}}}$ ったにしては、 ${\overset{\textnormal{おい}}{\text{美味}}}$ しいよ。」「え、どういうこと? ${\overset{\textnormal{ひど}}{\text{酷}}}$ い ${\overset{\textnormal{い}}{\text{言}}}$ い ${\overset{\textnormal{かた}}{\text{方}}}$ だなあ。」 \hfill\break
“It\textquotesingle s delicious considering you made it.” “Huh, what? What an awful thing to say.” }
 
\par{43. ${\overset{\textnormal{そうじ}}{\text{掃除}}}$ したにしては、 ${\overset{\textnormal{へや}}{\text{部屋}}}$ がまだまだ ${\overset{\textnormal{きたな}}{\text{汚}}}$ いなあ。 \hfill\break
The room is still dirty considering you cleaned it. }
 
\par{44. ${\overset{\textnormal{がいこくじん}}{\text{外国人}}}$ にしてはいけないことというと、 ${\overset{\textnormal{がいこくじん}}{\text{外国人}}}$ の ${\overset{\textnormal{なま}}{\text{訛}}}$ りを ${\overset{\textnormal{まね}}{\text{真似}}}$ すること、 ${\overset{\textnormal{がいこくじん}}{\text{外国人}}}$ が ${\overset{\textnormal{いっしょ}}{\text{一緒}}}$ にいるのに ${\overset{\textnormal{にほんじん}}{\text{日本人}}}$ とばかり話したりすることなどが ${\overset{\textnormal{おも}}{\text{思}}}$ い ${\overset{\textnormal{つ}}{\text{付}}}$ きます。 \hfill\break
When talking about things that you mustn\textquotesingle t do to foreigners, mimicking a foreigners accident and only speaking to people who are Japanese when you\textquotesingle re with a foreigner come to mind. }
 
\par{45. ${\overset{\textnormal{ぼく}}{\text{僕}}}$ にしてはスペイン ${\overset{\textnormal{ご}}{\text{語}}}$ の ${\overset{\textnormal{べんきょうがんば}}{\text{勉強頑張}}}$ ってますよ。 \hfill\break
Considering how I am, I\textquotesingle m doing my best in my Spanish studies. }
 
\par{46. ちょっと ${\overset{\textnormal{はる}}{\text{春}}}$ にしては ${\overset{\textnormal{あつ}}{\text{暑}}}$ いんじゃないですか。 \hfill\break
Isn\textquotesingle t it a little hot for spring? }
 
\par{47. ${\overset{\textnormal{せんえん}}{\text{千円}}}$ にしては ${\overset{\textnormal{しつ}}{\text{質}}}$ がいいですよ。 \hfill\break
For a thousand yen, the quality is good. }
 
\par{48. ダイエットしてるにしては、よく ${\overset{\textnormal{た}}{\text{食}}}$ べるね。 \hfill\break
You sure eat a lot considering you\textquotesingle re on a diet. }
 
\par{49. 「英検はうまくいった?」「んん、どうだろう。4級にしては問題が難しかったよ。」 \hfill\break
“Did you do okay on the English proficiency test?” “Mm, I\textquotesingle m not sure. The problems were pretty difficult for Level 4.” }
 
\par{50. ${\overset{\textnormal{はじ}}{\text{初}}}$ めてにしては ${\overset{\textnormal{じょうず}}{\text{上手}}}$ だな。 \hfill\break
You\textquotesingle re pretty good for it being your first time. }
 
\begin{center}
\textbf{にしちゃ }
\end{center}

\par{  In casual contexts, にしては can be contracted to にしちゃ. }
 
\par{51. ${\overset{\textnormal{に}}{\text{2}}}$ ${\overset{\textnormal{ねん}}{\text{年}}}$ しか ${\overset{\textnormal{べんきょう}}{\text{勉強}}}$ してないにしちゃ、 ${\overset{\textnormal{えいご}}{\text{英語}}}$ うまいっすな。 \hfill\break
Considering you\textquotesingle ve only studied for two years, your English is pretty good. }
 
\par{52. あの ${\overset{\textnormal{ねだん}}{\text{値段}}}$ にしちゃ ${\overset{\textnormal{おい}}{\text{美味}}}$ しかったしね。 \hfill\break
It was also delicious for that price. }
 
\par{53. ${\overset{\textnormal{きみ}}{\text{君}}}$ にしちゃ、 ${\overset{\textnormal{めずら}}{\text{珍}}}$ しいな。 \hfill\break
That\textquotesingle s pretty rare for you. }
 
\par{54. ${\overset{\textnormal{じんせい}}{\text{人生}}}$ を ${\overset{\textnormal{むだ}}{\text{無駄}}}$ にしちゃいけないよ。 \hfill\break
You mustn\textquotesingle t waste your life. }
 
\par{\textbf{Grammar Note }: Do not confuse this にしちゃ with the にしちゃ seen as a contraction in the phrase にしてはいけない (mustn\textquotesingle t do…) }

\begin{center}
にして = であって 
\end{center}
 
\par{ In semi-archaic speech, にして can be seen used with the meaning of であって. }
 
\par{55. ${\overset{\textnormal{しょうごん}}{\text{荘厳}}}$ にして ${\overset{\textnormal{かれい}}{\text{華麗}}}$ な ${\overset{\textnormal{いっぴん}}{\text{逸品}}}$ です。 \hfill\break
It is a sublime, gorgeous article of rare beauty. }
    