    
\chapter{Disregard III}

\begin{center}
\begin{Large}
第282課: Disregard III: をものともせず(に), \{を・も\}顧みず(に), \& を押して・押し切って 
\end{Large}
\end{center}
 
\par{ This lesson revolves around expressions regarding "disregard" in the sense of not heeding adversity, risk, danger, or any negative influences and\slash or consequences to the speaker. }
      
\section{をものともせず(に)}
 
\par{ をものともせず(に) follows a noun for which the subject at hand faces some adversity to state that the subject doesn\textquotesingle t fear or even put to mind that adversity in his or her actions. If it were to be paraphrased in Japanese, it would equate to ~を全く恐れないで・~を気にもとめないで. }

\par{1. ${\overset{\textnormal{ことし}}{\text{今年}}}$ ${\overset{\textnormal{はちじゅっ}}{\text{80}}}$ ${\overset{\textnormal{さい}}{\text{歳}}}$ になる ${\overset{\textnormal{もりた}}{\text{森田}}}$ さんは、 ${\overset{\textnormal{あしこし}}{\text{足腰}}}$ の ${\overset{\textnormal{いた}}{\text{痛}}}$ みをものともせず、 ${\overset{\textnormal{わか}}{\text{若}}}$ い ${\overset{\textnormal{ひと}}{\text{人}}}$ の ${\overset{\textnormal{しどう}}{\text{指導}}}$ に ${\overset{\textnormal{はげ}}{\text{励}}}$ んでいる。 \hfill\break
Mr. Morita, who turns eighty years old this year, strives to instruct young people despite his leg and loin pain. }

\par{2. ${\overset{\textnormal{りょうしん}}{\text{両親}}}$ の ${\overset{\textnormal{もうはんたい}}{\text{猛反対}}}$ をものともせず、 ${\overset{\textnormal{かのじょ}}{\text{彼女}}}$ は ${\overset{\textnormal{がいこくじん}}{\text{外国人}}}$ の ${\overset{\textnormal{だんせい}}{\text{男性}}}$ と ${\overset{\textnormal{けっこん}}{\text{結婚}}}$ した。 \hfill\break
She married a male foreigner in the face of fierce opposition from her parents. }

\par{3. ${\overset{\textnormal{そぼ}}{\text{祖母}}}$ は ${\overset{\textnormal{つよ}}{\text{強}}}$ かった。 ${\overset{\textnormal{がん}}{\text{癌}}}$ の ${\overset{\textnormal{せんこく}}{\text{宣告}}}$ をものともせず、 ${\overset{\textnormal{さいご}}{\text{最期}}}$ まで ${\overset{\textnormal{あか}}{\text{明}}}$ るく ${\overset{\textnormal{ふるま}}{\text{振舞}}}$ った。 \hfill\break
My grandmother was strong. In the face of a cancer sentence, she was cheerful to the end. }

\par{4. この ${\overset{\textnormal{かいしゃ}}{\text{会社}}}$ は ${\overset{\textnormal{ふきょう}}{\text{不況}}}$ をものともせずに、 ${\overset{\textnormal{じゅんちょう}}{\text{順調}}}$ に ${\overset{\textnormal{う}}{\text{売}}}$ り ${\overset{\textnormal{あ}}{\text{上}}}$ げを ${\overset{\textnormal{の}}{\text{伸}}}$ ばしている。 \hfill\break
This company is steadily increasing sales in the face of a recession. }

\par{5. ${\overset{\textnormal{ばしゃ}}{\text{馬車}}}$ は、 ${\overset{\textnormal{さかみち}}{\text{坂道}}}$ をものともせず ${\overset{\textnormal{のぼ}}{\text{登}}}$ って ${\overset{\textnormal{じょうもん}}{\text{城門}}}$ の ${\overset{\textnormal{まえ}}{\text{前}}}$ で ${\overset{\textnormal{いったんと}}{\text{一旦止}}}$ まった。 \hfill\break
The carriage climbed up the hill road as if it were nothing and stopped for a moment in front of the castle gate. }

\par{6. ${\overset{\textnormal{こども}}{\text{子供}}}$ たちは ${\overset{\textnormal{きんちょう}}{\text{緊張}}}$ をものともせず ${\overset{\textnormal{うた}}{\text{歌}}}$ いきりました。 \hfill\break
The children sang in full despite nervousness. }

\par{7. ${\overset{\textnormal{きんじょ}}{\text{近所}}}$ の ${\overset{\textnormal{そとねこ}}{\text{外猫}}}$ たちは ${\overset{\textnormal{そう}}{\text{相}}}$ (も) ${\overset{\textnormal{か}}{\text{変}}}$ わらず、 ${\overset{\textnormal{おおゆき}}{\text{大雪}}}$ をものともせずに ${\overset{\textnormal{えさ}}{\text{餌}}}$ を ${\overset{\textnormal{もら}}{\text{貰}}}$ いに ${\overset{\textnormal{き}}{\text{来}}}$ ている。 \hfill\break
The outdoor cats of the neighborhood have come for food despite the heavy snow as usual. }

\par{8. その ${\overset{\textnormal{しょうねん}}{\text{少年}}}$ は、 ${\overset{\textnormal{からだ}}{\text{身体}}}$ の ${\overset{\textnormal{しょうがい}}{\text{障害}}}$ をものともせずに、 ${\overset{\textnormal{じんせい}}{\text{人生}}}$ に ${\overset{\textnormal{た}}{\text{立}}}$ ち ${\overset{\textnormal{む}}{\text{向}}}$ かった。 \hfill\break
The young lad fought against life despite his physical handicap. }

\par{9. ${\overset{\textnormal{へいし}}{\text{兵士}}}$ は、 ${\overset{\textnormal{りゅう}}{\text{竜}}}$ の ${\overset{\textnormal{は}}{\text{吐}}}$ く ${\overset{\textnormal{ほのお}}{\text{炎}}}$ をものともせず、 ${\overset{\textnormal{こうげき}}{\text{攻撃}}}$ を ${\overset{\textnormal{しか}}{\text{仕掛}}}$ けていた。 \hfill\break
The soldier was launching attacks in face of the dragon spewing fire. }

\par{10. ${\overset{\textnormal{おおあ}}{\text{大荒}}}$ れの ${\overset{\textnormal{うみ}}{\text{海}}}$ をものともせずに ${\overset{\textnormal{こうかい}}{\text{航海}}}$ を ${\overset{\textnormal{つづ}}{\text{続}}}$ けた。 \hfill\break
(We) continued the voyage despite the stormy sea. }

\par{11. その ${\overset{\textnormal{きぎょう}}{\text{企業}}}$ は ${\overset{\textnormal{ぎゃっきょう}}{\text{逆境}}}$ をものともせずに、 ${\overset{\textnormal{こうぎょうせき}}{\text{好業績}}}$ を ${\overset{\textnormal{たた}}{\text{叩}}}$ き ${\overset{\textnormal{だ}}{\text{出}}}$ している。 \hfill\break
That company is hammering out good results despite adversity. }

\par{12. その ${\overset{\textnormal{へいし}}{\text{兵士}}}$ たちは ${\overset{\textnormal{むら}}{\text{群}}}$ がる ${\overset{\textnormal{てきぐん}}{\text{敵軍}}}$ をものともせず、 ${\overset{\textnormal{つぎつぎ}}{\text{次々}}}$ と ${\overset{\textnormal{せんかん}}{\text{戦艦}}}$ を ${\overset{\textnormal{げきは}}{\text{撃破}}}$ していった。 \hfill\break
Those soldiers went on crushing enemy vessels one after another despite the swarming enemy forces. }

\par{13. ${\overset{\textnormal{けいじ}}{\text{刑事}}}$ たちは、 ${\overset{\textnormal{きけん}}{\text{危険}}}$ をものともせず、 ${\overset{\textnormal{ようぎしゃ}}{\text{容疑者}}}$ の ${\overset{\textnormal{そうさく}}{\text{捜索}}}$ を ${\overset{\textnormal{つづ}}{\text{続}}}$ けた。 \hfill\break
The detectives continued searching for the suspect despite the danger. }

\par{14. ${\overset{\textnormal{かれ}}{\text{彼}}}$ は ${\overset{\textnormal{なんびょう}}{\text{難病}}}$ をものともせず、 ${\overset{\textnormal{じぶん}}{\text{自分}}}$ の ${\overset{\textnormal{す}}{\text{好}}}$ きなことに打ち ${\overset{\textnormal{こ}}{\text{込}}}$ んでいる。 \hfill\break
He devoted to what he himself likes today despite his incurable illness. }

\par{ 15. ${\overset{\textnormal{じゅうみん}}{\text{住民}}}$ は、 ${\overset{\textnormal{たつまき}}{\text{竜巻}}}$ の ${\overset{\textnormal{ひがい}}{\text{被害}}}$ をものともせず、 ${\overset{\textnormal{たくま}}{\text{逞}}}$ しく ${\overset{\textnormal{い}}{\text{生}}}$ きている。 \hfill\break
The inhabitants live on robustly despite the damage from the tornado. }
      
\section{\{を・も\}顧みず(に)}
 
\par{ The pattern \{を・も\}顧みず(に) indicates that the subject does something without any heed to risk and\slash or danger. The use of the particle も over を is simply for emphatic purposes. }

\par{16. ${\overset{\textnormal{じぶん}}{\text{自分}}}$ の ${\overset{\textnormal{いのち}}{\text{命}}}$ (を ${\overset{\textnormal{うしな}}{\text{失}}}$ うこと)も ${\overset{\textnormal{かえり}}{\text{顧}}}$ みず、 ${\overset{\textnormal{せんじょう}}{\text{戦場}}}$ へ ${\overset{\textnormal{む}}{\text{向}}}$ かった。 \hfill\break
(He) headed for the battlefield without regard to losing his own life. }

\par{17. ${\overset{\textnormal{かれ}}{\text{彼}}}$ は ${\overset{\textnormal{きけん}}{\text{危険}}}$ を ${\overset{\textnormal{かえり}}{\text{顧}}}$ みず、 ${\overset{\textnormal{みごと}}{\text{見事}}}$ に ${\overset{\textnormal{しょくむ}}{\text{職務}}}$ を ${\overset{\textnormal{すいこう}}{\text{遂行}}}$ した。 \hfill\break
He brilliantly executed his duties heedless of danger. }

\par{18. ${\overset{\textnormal{かのじょ}}{\text{彼女}}}$ は ${\overset{\textnormal{いしゃ}}{\text{医者}}}$ の ${\overset{\textnormal{ちゅうこく}}{\text{忠告}}}$ を ${\overset{\textnormal{かえり}}{\text{顧}}}$ みず、 ${\overset{\textnormal{たばこ}}{\text{煙草}}}$ を ${\overset{\textnormal{ふ}}{\text{吹}}}$ かし ${\overset{\textnormal{つづ}}{\text{続}}}$ けた。 \hfill\break
She continued smoking heedless of her doctor\textquotesingle s advice. }

\par{19. ${\overset{\textnormal{こんご}}{\text{今後}}}$ の ${\overset{\textnormal{せいかつ}}{\text{生活}}}$ を ${\overset{\textnormal{かえり}}{\text{顧}}}$ みず、 ${\overset{\textnormal{あらい}}{\text{新井}}}$ は ${\overset{\textnormal{かいしゃ}}{\text{会社}}}$ を ${\overset{\textnormal{や}}{\text{辞}}}$ めた。 \hfill\break
Arai quit the company heedless of his life afterward. }

\par{20. ${\overset{\textnormal{み}}{\text{身}}}$ の ${\overset{\textnormal{あんぜん}}{\text{安全}}}$ も ${\overset{\textnormal{かえり}}{\text{顧}}}$ みず ${\overset{\textnormal{ほうどう}}{\text{報道}}}$ の ${\overset{\textnormal{じゆう}}{\text{自由}}}$ を ${\overset{\textnormal{うった}}{\text{訴}}}$ える ${\overset{\textnormal{せかいかくち}}{\text{世界各地}}}$ の ${\overset{\textnormal{しんぶんきしゃ}}{\text{新聞記者}}}$ たちを ${\overset{\textnormal{そんけい}}{\text{尊敬}}}$ しています。 \hfill\break
I respect newspaper journalists all around the world who advocate for freedom of the press heedless of one\textquotesingle s safety. }

\par{21. ${\overset{\textnormal{ほんがく}}{\text{本学}}}$ の ${\overset{\textnormal{がくせい}}{\text{学生}}}$ が、 ${\overset{\textnormal{じゅぎょう}}{\text{授業}}}$ を ${\overset{\textnormal{う}}{\text{受}}}$ ける ${\overset{\textnormal{ほか}}{\text{他}}}$ の ${\overset{\textnormal{がくせい}}{\text{学生}}}$ やそこで ${\overset{\textnormal{はたら}}{\text{働}}}$ く ${\overset{\textnormal{きょうしょくいん}}{\text{教職員}}}$ の ${\overset{\textnormal{めいわく}}{\text{迷惑}}}$ も ${\overset{\textnormal{かえり}}{\text{顧}}}$ みず、 ${\overset{\textnormal{とうていいっぱんしゃかい}}{\text{到底一般社会}}}$ では ${\overset{\textnormal{ゆる}}{\text{許}}}$ されない ${\overset{\textnormal{はんざいこうい}}{\text{犯罪行為}}}$ に ${\overset{\textnormal{かんよ}}{\text{関与}}}$ したことは、 ${\overset{\textnormal{まこと}}{\text{誠}}}$ に ${\overset{\textnormal{いかん}}{\text{遺憾}}}$ です。 \hfill\break
We sincerely suggest that a student at our school was involved in a criminal act that is not possibly permissible in the general public, heedless of the trouble it would cause to other students who take courses or the teaching staff members who work on the campus. }

\par{22. 彼は ${\overset{\textnormal{じゅうしょう}}{\text{重傷}}}$ の ${\overset{\textnormal{み}}{\text{身}}}$ も ${\overset{\textnormal{かえり}}{\text{顧}}}$ みず、 ${\overset{\textnormal{せんきゃく}}{\text{船客}}}$ の ${\overset{\textnormal{せわ}}{\text{世話}}}$ をしていたそうだ。 \hfill\break
It sounds that he was taking care of the passengers heedless of his own severe injuries. }

\par{23. ${\overset{\textnormal{ごしゅじん}}{\text{御主人}}}$ はわたしがいるので ${\overset{\textnormal{いえ}}{\text{家}}}$ の ${\overset{\textnormal{なか}}{\text{中}}}$ の ${\overset{\textnormal{なに}}{\text{何}}}$ をも ${\overset{\textnormal{かえり}}{\text{顧}}}$ みず、その ${\overset{\textnormal{も}}{\text{持}}}$ ち ${\overset{\textnormal{もの}}{\text{物}}}$ をみなわたしの ${\overset{\textnormal{て}}{\text{手}}}$ に ${\overset{\textnormal{ゆだ}}{\text{委}}}$ ねられました。 \hfill\break
My master does not concern himself with anything in the house; everything he owns he has entrusted to my care. }

\par{24. ${\overset{\textnormal{かれ}}{\text{彼}}}$ は ${\overset{\textnormal{じぶん}}{\text{自分}}}$ の ${\overset{\textnormal{ひりき}}{\text{非力}}}$ も ${\overset{\textnormal{かえり}}{\text{顧}}}$ みず、 ${\overset{\textnormal{かいちょう}}{\text{会長}}}$ という ${\overset{\textnormal{じゅうせき}}{\text{重責}}}$ を ${\overset{\textnormal{ひ}}{\text{引}}}$ き ${\overset{\textnormal{う}}{\text{受}}}$ けた。 \hfill\break
Heedless of his own incompetence, he took over the heavy responsibility of chairman. }

\par{25. ${\overset{\textnormal{みずか}}{\text{自}}}$ らの ${\overset{\textnormal{かちていか}}{\text{価値低下}}}$ を ${\overset{\textnormal{かえり}}{\text{顧}}}$ みぬ ${\overset{\textnormal{こうい}}{\text{行為}}}$ も、 ${\overset{\textnormal{じっしつてき}}{\text{実質的}}}$ な ${\overset{\textnormal{せいか}}{\text{成果}}}$ なし。 \hfill\break
Despite Disregarding Lowering His Own Value, No Substantial Results Gained. }

\par{26. カメラマンは ${\overset{\textnormal{みずか}}{\text{自}}}$ らの ${\overset{\textnormal{いのち}}{\text{命}}}$ も ${\overset{\textnormal{かえり}}{\text{顧}}}$ みず ${\overset{\textnormal{はげ}}{\text{激}}}$ しい ${\overset{\textnormal{せんじょう}}{\text{戦場}}}$ に ${\overset{\textnormal{た}}{\text{立}}}$ ち ${\overset{\textnormal{む}}{\text{向}}}$ かった。 \hfill\break
The cameraman faced the fierce battlefield without regard to his own life. }

\par{27. ${\overset{\textnormal{しつれい}}{\text{失礼}}}$ を ${\overset{\textnormal{かえり}}{\text{顧}}}$ みずにこのようにお ${\overset{\textnormal{ねが}}{\text{願}}}$ いしている ${\overset{\textnormal{しだい}}{\text{次第}}}$ でございます。 \hfill\break
I am taking the liberty of asking you as such. }

\par{28. ${\overset{\textnormal{しごと}}{\text{仕事}}}$ で ${\overset{\textnormal{かてい}}{\text{家庭}}}$ を ${\overset{\textnormal{かえり}}{\text{顧}}}$ みない ${\overset{\textnormal{だんせい}}{\text{男性}}}$ は ${\overset{\textnormal{ゆうせんじゅんい}}{\text{優先順位}}}$ が ${\overset{\textnormal{まちが}}{\text{間違}}}$ っているのではないか。 \hfill\break
Doesn\textquotesingle t a man who doesn\textquotesingle t blink an eye at the home due to work have his priorities messed up? }

\par{29. ${\overset{\textnormal{わたし}}{\text{私}}}$ たちは、アイダホ ${\overset{\textnormal{しゅう}}{\text{州}}}$ の ${\overset{\textnormal{いなかみち}}{\text{田舎道}}}$ を ${\overset{\textnormal{うんてん}}{\text{運転}}}$ していた ${\overset{\textnormal{さい}}{\text{際}}}$ に、 ${\overset{\textnormal{きけん}}{\text{危険}}}$ を ${\overset{\textnormal{かえり}}{\text{顧}}}$ みずに ${\overset{\textnormal{どうろ}}{\text{道路}}}$ の ${\overset{\textnormal{ま}}{\text{真}}}$ ん ${\overset{\textnormal{なか}}{\text{中}}}$ に ${\overset{\textnormal{ちんざ}}{\text{鎮座}}}$ しているアライグマたちに ${\overset{\textnormal{そうぐう}}{\text{遭遇}}}$ した。 \hfill\break
We encountered raccoons sitting in the middle of the road heedless of danger when we were driving on a country road in Idaho. }

\par{30. ${\overset{\textnormal{じぶん}}{\text{自分}}}$ を ${\overset{\textnormal{かえり}}{\text{省}}}$ みずに、 ${\overset{\textnormal{たにん}}{\text{他人}}}$ を ${\overset{\textnormal{か}}{\text{変}}}$ えようなんて ${\overset{\textnormal{ごうまん}}{\text{傲慢}}}$ なのではないか。 \hfill\break
Isn't it arrogant to change others while you don\textquotesingle t reflect on yourself? }

\par{\textbf{Spelling Note }: Although not entirely related to this grammatical pattern, かえりみる also coincides with 省みる, which means “to reflect on” and it can also be used with ず(に). }
      
\section{を押して・押し切って}
 
\par{ Putting the literal meaning of 押す—to push—aside and other literal uses, を押して can be used to mean “facing down something”, which it shares with 押し切る. Both expressions are limited to set phrases. を押して is paired with words like 熱 (fever) and 病気 (illness), but not much else. を押し切って, on the other hand, is paired with words like 反対(意見) opposition (opinion), 〇〇党 (\#\# political party), etc. }

\par{ The examples below illustrate the various usages of both 押す and 押し切る. They are not exclusively about the patterns を押して and  を押し切って. }

\par{31. ${\overset{\textnormal{さんじゅうはち}}{\text{38}}}$ ${\overset{\textnormal{ど}}{\text{℃}}}$ の ${\overset{\textnormal{ねつ}}{\text{熱}}}$ を ${\overset{\textnormal{お}}{\text{押}}}$ してでも ${\overset{\textnormal{しごと}}{\text{仕事}}}$ を全うしようとしている ${\overset{\textnormal{ぶか}}{\text{部下}}}$ に ${\overset{\textnormal{そうたい}}{\text{早退}}}$ を ${\overset{\textnormal{めい}}{\text{命}}}$ じる ${\overset{\textnormal{じょうし}}{\text{上司}}}$ は ${\overset{\textnormal{ただ}}{\text{正}}}$ しいと ${\overset{\textnormal{おも}}{\text{思}}}$ います。 \hfill\break
I think that the boss ordering his subordinate who was trying to carry out his job despite a 100℉ fever to leave early was right. }

\par{32. ホームボタンを ${\overset{\textnormal{お}}{\text{押}}}$ してロック ${\overset{\textnormal{かいじょ}}{\text{解除}}}$ 。 \hfill\break
Press home to unlock. }

\par{33. ${\overset{\textnormal{そうり}}{\text{総理}}}$ ( ${\overset{\textnormal{だいじん}}{\text{大臣}}}$ )は ${\overset{\textnormal{びょうき}}{\text{病気}}}$ を ${\overset{\textnormal{お}}{\text{押}}}$ して ${\overset{\textnormal{きしゃかいけん}}{\text{記者会見}}}$ に ${\overset{\textnormal{あらわ}}{\text{現}}}$ れた。 \hfill\break
The prime minister appeared at the press conference despite his illness. }

\par{34. ${\overset{\textnormal{しゅうい}}{\text{周囲}}}$ の ${\overset{\textnormal{はんたい}}{\text{反対}}}$ を ${\overset{\textnormal{お}}{\text{押}}}$ し ${\overset{\textnormal{き}}{\text{切}}}$ って ${\overset{\textnormal{たんしん}}{\text{単身}}}$ で ${\overset{\textnormal{かいがいせいかつ}}{\text{海外生活}}}$ を ${\overset{\textnormal{はじ}}{\text{始}}}$ めた。 \hfill\break
I faced down the opposition around me and began living abroad by myself. }

\par{35. ${\overset{\textnormal{こく}}{\text{国}}}$ は ${\overset{\textnormal{ながねん}}{\text{長年}}}$ の ${\overset{\textnormal{はんたいいけん}}{\text{反対意見}}}$ を ${\overset{\textnormal{お}}{\text{押}}}$ し ${\overset{\textnormal{き}}{\text{切}}}$ って、 ${\overset{\textnormal{かんたくじぎょう}}{\text{干拓事業}}}$ を ${\overset{\textnormal{はじ}}{\text{始}}}$ めるために ${\overset{\textnormal{いさはやわん}}{\text{諫早湾}}}$ を ${\overset{\textnormal{せ}}{\text{堰}}}$ き ${\overset{\textnormal{と}}{\text{止}}}$ めた。 \hfill\break
The country faced down the opposing views of many years and dammed up Isahaya Bay to begin land reclamation. }

\par{36. ${\overset{\textnormal{かのじょ}}{\text{彼女}}}$ は、 ${\overset{\textnormal{かれ}}{\text{彼}}}$ (の) ${\overset{\textnormal{りょうしん}}{\text{両親}}}$ の ${\overset{\textnormal{はんたい}}{\text{反対}}}$ を ${\overset{\textnormal{お}}{\text{押}}}$ し ${\overset{\textnormal{き}}{\text{切}}}$ って、 ${\overset{\textnormal{どうせい}}{\text{同棲}}}$ している ${\overset{\textnormal{かれし}}{\text{彼氏}}}$ と ${\overset{\textnormal{にゅうせき}}{\text{入籍}}}$ した。 \hfill\break
She entered the family registry of her boyfriend who she has lived together with whilst facing down his parents\textquotesingle  opposition. }

\par{37. ${\overset{\textnormal{こうめいとう}}{\text{公明党}}}$ を ${\overset{\textnormal{お}}{\text{押}}}$ し ${\overset{\textnormal{き}}{\text{切}}}$ る ${\overset{\textnormal{かたち}}{\text{形}}}$ で ${\overset{\textnormal{せいふ}}{\text{政府}}}$ ・ ${\overset{\textnormal{じみんとう}}{\text{自民党}}}$ が ${\overset{\textnormal{せいりつ}}{\text{成立}}}$ を ${\overset{\textnormal{いそ}}{\text{急}}}$ いでいる。 \hfill\break
The government—the Liberal Democratic Party—are rushing to (have the law) come into existence by facing down the New Komeito Party. }

\par{38. お ${\overset{\textnormal{きゃく}}{\text{客}}}$ さんに ${\overset{\textnormal{ねび}}{\text{値引}}}$ きを ${\overset{\textnormal{お}}{\text{押}}}$ し ${\overset{\textnormal{き}}{\text{切}}}$ られました。 \hfill\break
I was out-negotiated by the customer. }

\par{\textbf{Meaning Note }: 押し切る may also mean “to out-negotiate” as seen here. }

\par{39. ${\overset{\textnormal{わす}}{\text{忘}}}$ れないように ${\overset{\textnormal{ねん}}{\text{念}}}$ を ${\overset{\textnormal{お}}{\text{押}}}$ してください。 \hfill\break
Please remind me not to forget. }

\par{40. ${\overset{\textnormal{せなか}}{\text{背中}}}$ を ${\overset{\textnormal{お}}{\text{押}}}$ してくれてありがとう。 \hfill\break
Thank you for always supporting me. }

\par{41. ここに ${\overset{\textnormal{はんこ}}{\text{判子}}}$ を ${\overset{\textnormal{お}}{\text{押}}}$ してください。 \hfill\break
Please stamp your seal here. }

\par{42. ${\overset{\textnormal{こくみん}}{\text{国民}}}$ の ${\overset{\textnormal{はんたい}}{\text{反対}}}$ と ${\overset{\textnormal{ひなん}}{\text{非難}}}$ を ${\overset{\textnormal{お}}{\text{押}}}$ し( ${\overset{\textnormal{きっ}}{\text{切}}}$ っ)て ${\overset{\textnormal{せんそう}}{\text{戦争}}}$ を ${\overset{\textnormal{はじ}}{\text{始}}}$ めるなんて ${\overset{\textnormal{ゆる}}{\text{許}}}$ されないのだ! \hfill\break
Starting a war whilst facing down opposition and criticism of the people is impermissible! }

\par{43. 「 ${\overset{\textnormal{いけん}}{\text{違憲}}}$ 」 ${\overset{\textnormal{ひはん}}{\text{批判}}}$ を ${\overset{\textnormal{お}}{\text{押}}}$ し ${\overset{\textnormal{き}}{\text{切}}}$ って、 ${\overset{\textnormal{あべせいけん}}{\text{安倍政権}}}$ は ${\overset{\textnormal{あんぜんほしょうかんれんほう}}{\text{安全保障関連法}}}$ を ${\overset{\textnormal{せいりつ}}{\text{成立}}}$ させた。 \hfill\break
The Abe Administration established the national security bill whilst facing down critiques that doing so was “unconstitutional.” }

\par{44. ${\overset{\textnormal{ねっちゅうしょう}}{\text{熱中症}}}$ の ${\overset{\textnormal{きけん}}{\text{危険}}}$ を ${\overset{\textnormal{お}}{\text{押}}}$ して ${\overset{\textnormal{で}}{\text{出}}}$ かけてきました。 \hfill\break
I went out despite the danger of heatstroke. }

\par{45. そこを ${\overset{\textnormal{お}}{\text{押}}}$ してお ${\overset{\textnormal{ねが}}{\text{願}}}$ いします。 \hfill\break
I know I\textquotesingle m stressing that request, but please consider it. }
    