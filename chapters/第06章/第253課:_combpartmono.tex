    
\chapter{Combination Particles with もの}

\begin{center}
\begin{Large}
第253課: Combination Particles with もの 
\end{Large}
\end{center}
 
\par{ This lesson is about the particle もの and the combination particles associated with it, not about the nouns 物 and 者, although all three words share the same origin. }
      
\section{The Particle もの: Because}
 
\par{ The particle もの, colloquially as もん, may mean "because". This usage is used extensively by women and small children as an alternative to から and may be used conjunction or as a final particle. }
 
\par{1. この ${\overset{\textnormal{けしき}}{\text{景色}}}$ は ${\overset{\textnormal{なつ}}{\text{懐}}}$ かしいですね。私たちが若い時に出会ったところですもの。 \hfill\break
This scenery is nostalgic isn't it? It's because this is the place we met when we young. }
 
\par{\textbf{Sentence Note }: Scenes like above would be commonly heard among old people. }
 
\par{2. ねぇ、ねぇ、あのビデオゲーム買ってよ。どうしてもあたしほしいんだもの。 \hfill\break
Hey, hey, can you buy that video game? Cause I've got to have it. }
 
\par{The particle もの has several other usages. }

\begin{ltabulary}{|P|P|}
\hline 

1. & Similar to "will", it nominalizes a verb. \\ \cline{1-2}

2. & Used with -たい showing wish, it means "would like to do". \hfill\break
\\ \cline{1-2}

3. & Ought to. \\ \cline{1-2}

4. & "よくしたものだ" means "used to". \\ \cline{1-2}

\end{ltabulary}

\par{3. よく泳いだもんだ。 \hfill\break
I used to swim. }

\par{4. ${\overset{\textnormal{まるき}}{\text{丸木}}}$ は水に ${\overset{\textnormal{う}}{\text{浮}}}$ くものだ。 \hfill\break
A log will float in water. }

\par{5. ${\overset{\textnormal{くじら}}{\text{鯨}}}$ になりたいものです。 \hfill\break
I wish I were a whale. }

\par{6. もっとご両親は ${\overset{\textnormal{うやま}}{\text{敬}}}$ うものですよ。 \hfill\break
You should be more respectful to your parents. }

\par{7. 私、 ${\overset{\textnormal{せいじん}}{\text{成人}}}$ になったんだもの、少しぐらいお酒を飲んでもいいでしょう。 \hfill\break
I'm an adult now, so isn't it alright for me to have a little liquor to drink? }
      
\section{Combination Particles}
 
\par{ Below is a chart of combination particles with the conjunctive particle もの. }

\begin{ltabulary}{|P|P|}
\hline 

~もので & Shows reason of action for what happened \hfill\break
\\ \cline{1-2}

~ものか \hfill\break
& Extremely emphatic negation \\ \cline{1-2}

~ものなら & Should there be, if\dothyp{}\dothyp{}\dothyp{}then (volitional) \hfill\break
\\ \cline{1-2}

~ものの & Even though \\ \cline{1-2}

~ものを & Although, if only \\ \cline{1-2}

\end{ltabulary}

\par{\textbf{Usage Notes }: }

\par{1. ~ものを signifies the unsatisfactory feeling of the speaker. }

\par{2. ~ないものか shows a feeling of expectation for the realization of an idea and ~たものか shows a sense of confusion. }

\par{3. ~からいいようなものの and ~ば・といいようなものの express "though it's OK because" and "although it would be for the better if" respectively. }

\begin{center}
 \textbf{Examples }
\end{center}

\par{8. いくつかの ${\overset{\textnormal{けってん}}{\text{欠点}}}$ はある \textbf{ものの }、彼は天才です。 \hfill\break
Even though he has a few flaws, he's a genius. }

\par{9 あんなやつに負ける \textbf{もんか }い。 \hfill\break
Like I'd lose to that kind of guy! }

\par{10. ${\overset{\textnormal{つか}}{\text{掴}}}$ める \textbf{ものなら }、掴んでごらん。 \hfill\break
If you think you can grab it, go right ahead. }

\par{\textbf{Word Note }: ごらん is just like みせてもらう. }

\par{11. 負ければ ${\overset{\textnormal{す}}{\text{済}}}$ む \textbf{ものの }、そこまで自分をおとしめることができなかった。 \hfill\break
Although it would have ended better if he would have lost, he could not lower himself to such a degree. }

\par{12. あんまり ${\overset{\textnormal{うれ}}{\text{嬉}}}$ しかった \textbf{もんで }、何をしてるのかは忘れちまった。(Really casual) \hfill\break
I was so happy that I forgot what I was doing. }

\par{13. 一言だけ話せばよかった \textbf{ものを }、しゃべりすぎたので、事件に ${\overset{\textnormal{ま}}{\text{巻}}}$ き込まれてしまった。 \hfill\break
Although it would have been alright to speak just a word, I accidentally got involved into the matter because I chattered too much. }

\par{14. 聞いてくれればよかった \textbf{ものを }。 \hfill\break
If only he would have listened. }

\par{15 さっさと警察に届けりゃいい \textbf{ものを }。 \hfill\break
If only I had given it to the police promptly. \hfill\break
From 冷たい誘惑 by 乃南アサ. }

\par{\textbf{Contraction Note }: りゃ is the contraction of れば. }

\par{16. もっとうまくできない \textbf{ものか }? \hfill\break
Can't you do better? }

\par{17. この仕事は誰に ${\overset{\textnormal{たの}}{\text{頼}}}$ んだ \textbf{ものか }。 \hfill\break
To whom was this work entrusted? }

\par{18. できる \textbf{もんなら }、 ${\overset{\textnormal{か}}{\text{代}}}$ わってやりたい。 \hfill\break
If it's possible, then I'd like to switch out (for you). }

\par{19. 気づいたからいいような \textbf{ものの }、他の人がいたらどうしただろうか。 \hfill\break
Though it is all right that I noticed, what would I do if there were other people? }

\par{20. 先生に ${\overset{\textnormal{あつ}}{\text{厚}}}$ かましくも ${\overset{\textnormal{くちごた}}{\text{口答}}}$ えをしよう \textbf{ものなら }、 ${\overset{\textnormal{おおめだま}}{\text{大目玉}}}$ を ${\overset{\textnormal{く}}{\text{食}}}$ らうでしょう。 \hfill\break
Should you even have the nerve to talk back to the teacher, you'll surely get scolded severely. }

\par{28. 火事になろう \textbf{もんなら }、大変だぞ。 \hfill\break
It would be grave should there be a fire. }

\par{29. 借金で困っていた友人を、助けようと思えば助けられた \textbf{ものを }、見捨ててしまった。 \hfill\break
If only I had thought to help my friend who was struggling with debt, I could have helped her, but I accidentally ran out on her. }

\par{30. 休めばよかった \textbf{ものを }、無理をして働きすぎたので、病気になってしまったよ。 \hfill\break
Although I would have been fine if I took a rest, I unreasonably worked too much, so I ended up getting ill. }

\par{31. あたしの気持ち、君にわかってたまる \textbf{ものか }! \hfill\break
How could you ever know how I'm feeling!? }
    