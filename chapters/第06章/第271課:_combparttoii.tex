    
\chapter{と Combination Particles II}

\begin{center}
\begin{Large}
第271課: と Combination Particles II 
\end{Large}
\end{center}
 
\par{ In the next two lessons we will learn about more combination particles with と. }

\begin{ltabulary}{|P|P|P|P|}
\hline 

とも & とあって & とあれば & といっ\{た・て\} \\ \cline{1-4}

\end{ltabulary}
\hfill\break
      
\section{とも}
 
\par{ とも is an emphatic version of the case particle と. Do not confuse with the conjunctive\slash final particle とも. }
 
\par{${\overset{\textnormal{}}{\text{1. 君}}}$ とも ${\overset{\textnormal{あ}}{\text{逢}}}$ えなくなるだろう。 \hfill\break
I probably won't be able to see you. }
      
\section{とあって}
 
\par{ とあって means "because under the circumstances that\slash due to the fact that". }

\par{ ${\overset{\textnormal{}}{\text{2. 三連休}}}$ とあって、すごい ${\overset{\textnormal{}}{\text{人出}}}$ だ。 \hfill\break
There is a turnout due to the fact that it is a three-day weekend. }
 
\par{${\overset{\textnormal{}}{\text{3. 先生}}}$ の ${\overset{\textnormal{}}{\text{頼}}}$ みとあっては、お ${\overset{\textnormal{}}{\text{断}}}$ りできません。 \hfill\break
Because it is at your request, I cannot decline. }
 
\par{${\overset{\textnormal{}}{\text{4. 素晴}}}$ らしいホテルとあって、サービスは ${\overset{\textnormal{}}{\text{素晴}}}$ らしい。 \hfill\break
Since it is a wonderful hotel, the service is wonderful. }
 
\par{${\overset{\textnormal{}}{\text{5. 休日}}}$ とあって ${\overset{\textnormal{}}{\text{人出}}}$ が ${\overset{\textnormal{}}{\text{多}}}$ い。 \hfill\break
It is also that there is a lot of people out on the holidays. }
      
\section{といって・といった}
 
\par{ \{と・そう\}いった is placed after a list of nouns that are listed with particles like や, と, and とか to mean "such as". といって means "however". When not at the beginning of the sentence, it is often translated as "because" or "that\dothyp{}\dothyp{}\dothyp{}”. Though these phrases are from 言う, they are normally written in かな. }

\par{8. うちの息子は、科学や世界史といった科目が得意です。(From a mother) \hfill\break
Our son is good at subjects such as chemistry and world history. }

\par{9. 日本は、確かに経済大国となりました。\{と言って・しかし\}、国民一人一人の暮らしが豊かになったわけではあ りません。 \hfill\break
Japan has certainly become an "economic giant". However, that doesn't mean that the way of life for     each individual citizen has become wealthy. }

\par{\textbf{Sentence Note }: といって would not suitable for writing, but it would be fine in speech. }

\par{10a. 私は同意しました。だからといって、(私が) ${\overset{\textnormal{まんぞく}}{\text{満足}}}$ しているわけではありません。 \hfill\break
10b. 私は同意しましたが、満足しているわけではありません。 \hfill\break
I agreed, but that doesn't mean I am satisfied. }

\par{11a. そのように感じたからといって、誰も彼らを非難できませんよ。(Less natural ordering) \hfill\break
11b. 彼らがそのように感じたからといって、誰も非難できませんよ。 \hfill\break
Nobody can blame them that they felt that way. }

\par{12. 今はこれといってすることがない。 \hfill\break
I have nothing particular to do now. }

\par{13. 試験が終ったからといって努力を ${\overset{\textnormal{おこた}}{\text{怠}}}$ ってはいけない。 \hfill\break
You mustn't give up your efforts because the exam is over. }

\par{14. 彼女はケーキとかそういったものを焼いた。 \hfill\break
She baked cakes and stuff like that. }

\par{15. 「真理」とか「 ${\overset{\textnormal{び}}{\text{美}}}$ 」といった ${\overset{\textnormal{ちゅうしょうてき}}{\text{抽象的}}}$ な言葉 \hfill\break
Abstract words such as "truth" and "beauty }

\par{~からといって means "while it's true that\slash even though\slash just because". It normally sends a message of warning. The clause that follows should be negative in some way, either grammatically or in perspective. Don't let translation confuse you. }

\par{ A similar phrase is からに, which means "just by\slash from" and is used with the non-past. からには, on the other hand, expresses determination and is equivalent "now that". }

\par{16. ${\overset{\textnormal{てんさい}}{\text{天才}}}$ だからといって ${\overset{\textnormal{いば}}{\text{威張}}}$ るなよ。 \hfill\break
Don't brag just because you're a genius. }

\par{17. ${\overset{\textnormal{みな}}{\text{皆}}}$ そういうからには ${\overset{\textnormal{こわ}}{\text{怖}}}$ い話だよ。 \hfill\break
Just by everyone saying it, it's a scary story! }

\par{18. 彼は ${\overset{\textnormal{やくそく}}{\text{約束}}}$ したからには ${\overset{\textnormal{かなら}}{\text{必}}}$ ずやる(男だ)。 \hfill\break
He will surely do it just from promising. }

\par{\textbf{Variant Notes }: }

\par{1. からといって may be からって(ー) in colloquial speech or からとて in formal texts. }

\par{2. からには may be からは in literary language. }
      
\section{とあれば}
 
\par{ Also as とあらば, which is more poetic\slash nostalgic and old-fashioned, it shows a case established in a latter event(s) and is best translated as "if it is the case that". }
 
\par{19. お ${\overset{\textnormal{}}{\text{望}}}$ ${\overset{\textnormal{}}{\text{}}}$ みとあらば、 ${\overset{\textnormal{}}{\text{今}}}$ すぐ。(Slightly archaic) \hfill\break
(Go) right now if it is your wish. }
 
\par{${\overset{\textnormal{}}{\text{20. 必要}}}$ とあればすぐに ${\overset{\textnormal{}}{\text{伺}}}$ います。(A little old-fashioned) \hfill\break
I will come immediately if it is necessary. }
    