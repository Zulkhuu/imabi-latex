    
\chapter{抜く}

\begin{center}
\begin{Large}
第290課: 抜く: ~ぬきで(は), ~ぬきに(は), ~ぬきで, ~ぬきにして(は), \& ~ぬきの 
\end{Large}
\end{center}
 
\par{ This lesson is all about 抜く. Now, what may be so special about 抜く? It so happens to be used in ways that brings on some somewhat complicated grammar. }
      
\section{抜く}
 
\par{ 抜く means "to extract" or "to remove." It is used in compounds to provide a sense of thoroughness. In other verbs, it describes a state of extreme degree. This is related to the sense of something being thorough. }

\begin{ltabulary}{|P|P|P|P|P|P|}
\hline 

To shoot through & 射抜く & いぬく & To see through & 見抜く & みぬく \\ \cline{1-6}

To excerpt & 書き抜く & かきぬく & To gouge out & 刳り抜く & くりぬく \\ \cline{1-6}

To pierce through & 突き抜く & つきぬく & To hit through & ぶち抜く & ぶちぬく \\ \cline{1-6}

To dig through & 掘り抜く & ほりぬく & To pass (car) & 追い抜く & おいぬく \\ \cline{1-6}

To select through & 選り抜く & え・よりぬく & To win through & 勝ち抜く & かちぬく \\ \cline{1-6}

To cut through & 切り抜く & きりぬく & To know thoroughly & 知り抜く & しりぬく \\ \cline{1-6}

To persist through & 耐え抜く & たえぬく & To jump through & 出し抜く & だしぬく \\ \cline{1-6}

\end{ltabulary}
\hfill\break

\begin{center}
\textbf{Examples }
\end{center}

\par{1. 彼は辛い仕事をやり抜いた。 \hfill\break
He persevered through a harsh job. }

\par{2. 手を抜くな。 \hfill\break
Don't slack off! \hfill\break
Literally: Don't extract your hand (away). }

\par{3. 頑張り抜くのは難しい。 \hfill\break
Persisting until the end is difficult. }

\par{4. 彼は生き抜けるのか? \hfill\break
Can he make it out alive? }

\par{5. ${\overset{\textnormal{}}{\text{戦国時代}}}$ を ${\overset{\textnormal{}}{\text{生}}}$ き ${\overset{\textnormal{}}{\text{抜}}}$ く。 \hfill\break
To live through the Warring States Period. }

\par{\textbf{Word Note }: The intransitive version of 抜く is ~抜ける. It too attaches to verbs. }

\par{6. ${\overset{\textnormal{}}{\text{逆境}}}$ を ${\overset{\textnormal{}}{\text{切}}}$ り ${\overset{\textnormal{}}{\text{抜}}}$ ける。 \hfill\break
To pass through adversity. }
      
\section{~ぬきで(は), ~ぬきに(は), ~ぬきで, ~ぬきにして(は), ~ぬきの}
 
\par{ 「Xぬき\{で・に\}(は)Y(ない)」 shows that putting others aside, without X you wouldn't be able to do Y. What causes either で or に incorrect is based on whether the phrase is dynamic 動態 (で) or static 静態 (に). }

\par{7. ちょっと私ぬき\{〇 では・△ には\}イベントが始まらないんじゃなかったの?                   Without me, the event wouldn't start, no? }

\par{8. 僕ぬき\{で・△ に \}そんなこと始めるなんてできないよ。 \hfill\break
You couldn't even start that without me. }

\par{9. そろそろ食事をしますから、私ぬきで、ゲームを続けていてください。 \hfill\break
Since I'm having dinner soon, please continue the game without me. }

\par{10. 昼ご飯ぬき\{〇で・ Xに\}頑張ってるんですか。 \hfill\break
So you're trying without having had lunch? }

\par{11. 優勝なんて、あいつぬきじゃ、できっこないんだよ。 \hfill\break
(We) won't be able to win victory without that guy. }

\par{12. 朝食ぬきで働くなんて、できるわけがないよ。 \hfill\break
There's no way I can work without having breakfast. }

\par{13. 冗談\{ぬきで・はぬきにして\}、まじめに話し合いましょう。 \hfill\break
Leaving jokes aside, let's seriously talk with each other. }

\par{ As you would imagine, ~ぬきの is used in modifying other noun phrases. }

\par{14. 塩ぬきのをください。 \hfill\break
One without salt, please. }

\par{15. マッシュルームぬきの(ラーメン)をください。 \hfill\break
Ramen without mushrooms, please. }
16a. アルコール抜きのパーティー \hfill\break
16b. ノンアルコールパーティー \hfill\break
An alcohol-free party       
\section{~を抜きにして}
 
\par{ ~を抜きにして means "leaving out". This means that an X is should be left out for the realization of B. Of course, if the sentence is negative, then the opposite is true. }

\par{17. 冗談は抜きにして、そろそろ帰らないと、まずいよ。 \hfill\break
Leaving jokes aside, if you don't go home pretty soon, it won't be good. }

\par{18. 堅苦しい挨拶\{は抜きに・抜きで\}しましょう。 \hfill\break
Let's do without the ceremonious greetings. }

\par{19a. 力強い意志を抜きに(して)戦い続けることはできぬぞ。 (Old-man; 武士 talk) \hfill\break
19b. 強気でなければ戦い続けることはできぬぞ。(Old-man; 武士 talk) \hfill\break
Without a powerful will, you will not be able to continue fighting.  }
    