    
\chapter{と Combination Particles IV}

\begin{center}
\begin{Large}
第277課: と Combination Particles IV 
\end{Large}
\end{center}
 
\begin{ltabulary}{|P|P|P|}
\hline 

\emph{というのに }& \emph{という\{の・こと\}は }& \emph{とは\{いえ・いいながら・いうもの\} }\\ \cline{1-3}

\end{ltabulary}
      
\section{というのに}
 
\par{ というのに means "(al)though (that)". }

\par{1. 彼女、何日も ${\overset{\textnormal{く}}{\text{食}}}$ ってないっていうのに、健康そうに見えたな。(ちょっと乱暴) \hfill\break
While she hadn't eaten in days, she looked healthy. }

\par{\textbf{Word Note }: Remember that 食う is the vulgar form of 食べる. }

\par{2. ${\overset{\textnormal{さばく}}{\text{砂漠}}}$ はほとんど雨が降らないというのに、少数の植物は何とか生き永らえられるようだ。 \hfill\break
While deserts receive almost no rain, it looks like a few plants are somehow able to manage to live there. }

\par{${\overset{\textnormal{}}{\text{足}}}$ が ${\overset{\textnormal{}}{\text{痛}}}$ いくせに ${\overset{\textnormal{}}{\text{競走}}}$ して ${\overset{\textnormal{}}{\text{勝}}}$ ったんだ。 \hfill\break
Even though his legs hurt, he ran and won! }

\par{${\overset{\textnormal{}}{\text{彼女}}}$ はもう ${\overset{\textnormal{}}{\text{大人}}}$ のくせに、 ${\overset{\textnormal{}}{\text{母親}}}$ にまだ ${\overset{\textnormal{}}{\text{洗濯}}}$ してもらってるそうだぞ。(Masculine) \hfill\break
Even though she's already an adult, she still has her mother do the laundry! }

\par{やめろ。この ${\overset{\textnormal{}}{\text{間}}}$ も ${\overset{\textnormal{}}{\text{傷}}}$ ついたくせに。 \hfill\break
Stop it! You got hurt just the other day, remember? }

\par{\textbf{くせに VS ながら VS つつ }}

\par{\textbf{Question }: \emph{It appears that these three expressions are interchangeable? If they aren't, what is different about them? }}

\par{\textbf{Answer }: There is one usage of ながら  and つつ that is very similar to くせに and にもかかわらず. }

\par{ As we learned in Lesson 55  , the particle ながら can be used to show two opposing things happening simultaneously. For example, “despite hardship he achieved great things”. The two clauses should be similar in this matter. This is based on situations. For the most part, つつ is the same as ながら. It is not constructive as ながら is in expressions, but it is also used in patterns like つつある. }

\par{ However, くせに・にもかかわらず simply shows that one moreover takes a contrary move. So, this usage is not grounds to use these expressions interchangeable. As mentioned, there is much more to ながら and つつ, and even in contrasting they\textquotesingle re not exactly the same. }
      
\section{という\{の・こと\}は}
 
\par{ ~ということは means "that means" and ~というのは means "the reason that". Both are placed at the beginning of a sentence. In slang ~ということは may contract to ~ってことは. In extreme slang the entire phrase can be shortened to ~つうか. }

\par{10. ということは、まだ出張から帰ってないのだろう。 \hfill\break
That means he probably hasn't returned home from his business trip yet. }

\par{11. というのは、友達に ${\overset{\textnormal{だま}}{\text{騙}}}$ されたことがあるからです。 \hfill\break
The reason for that is that he has been deceived by his friends before. }

\par{12. というのも、新しい事業を始めたから。 \hfill\break
The reason for that was that he started a new business. }
      
\section{と\{はいえ・いうものの・いいながら\}}
 
\par{ ~とはいえ\slash いいながら\slash いうものの means "even though". The exact meanings are slightly different. The first emphasizes contrast with what "can be said", the second is like "while though", and the last is like "although\dothyp{}\dothyp{}\dothyp{}say". They can more or less be translated as "even though" in English. }

\par{13. ${\overset{\textnormal{つゆ・ばいう}}{\text{梅雨}}}$ とはいえ、晴れる日もある。 \hfill\break
Even though one may call it the rainy season, there are still also clear days. }

\par{\textbf{Culture Note }: The 梅雨 lasts from June to July. ばいう is a 書き言葉 reading. }

\par{14. 美味しいとはいいながら、まずいものも結構含まれている。 \hfill\break
Though (this) may be delicious, there are bad things included as well. }

\par{15. まだ若いとはいうものの彼女はかなりの経験がある。 \hfill\break
Even though she is young, she is quite experienced. }
    