    
\chapter{ほう}

\begin{center}
\begin{Large}
第266課: ほう 
\end{Large}
\end{center}
 
\par{ At its basic understanding, the word 方 means “direction” or “method.” It is used in both a physical and a temporal sense. This word exists as \emph{ほう }, a borrowing from Chinese, and as かた, the native equivalent. Both forms have survived to the present and their basic meanings are the same. To understand either, one has to also know the other. }
      
\section{かた}
 
\par{ The original word for direction in Japanese is \emph{かた }. Although no longer used in daily conversation, it is the basic meaning from which every other meaning derives. There are two b road ways humans conceptualize direction. We either talk about where we\textquotesingle re going or about the flow of time. Our lives are on both a timeline and a path. }

\par{i. 来し方 \hfill\break
The past \hfill\break
\hfill\break
\textbf{Reading Note }: This phrase is read as either こしかた or きしかた. }

\par{ This phrase literally means “the direction one has come,” but it was used both in the sense of time and direction. To this day, かた exists in temporal phrases indicating general time frame, but in the form \emph{がた }. }

\begin{ltabulary}{|P|P|P|P|P|P|}
\hline 

Dawn & 明け方(あけがた) & Evening & 夕方(ゆうがた) & Dusk & 暮れ方(くれがた) \\ \cline{1-6}

\end{ltabulary}

\par{ Direction words in Japanese tend to duly refer to people, and \emph{かた }is no different. You can see it used as a polite means of saying \emph{hito }人 (person), as well as a respectful plural phrase as \emph{がた }. }

\par{iii. ${\overset{\textnormal{だんせい}}{\text{男性}}}$ の ${\overset{\textnormal{かた}}{\text{方}}}$ のみ \emph{\hfill\break
 }Only men }

\par{iv. ${\overset{\textnormal{おくさまがた}}{\text{奥様方}}}$ の ${\overset{\textnormal{いどばたかいぎ}}{\text{井戸端会議}}}$ \hfill\break
Idle gossip among wives }

\par{ かた can also stand for the counter \emph{~ }人. However, this only works for numbers 1, 2, and 3, and the prefix お~ must precede the number. }

\par{v. お一方 (one person), お二方 (two people), お三方 (three people). }

\par{ Naturally, かた may also be used to mean “side” in reference to people. Have you ever wondered how to say “my grandfather on my mom\textquotesingle s side”? Luckily for you, かた lets you do that. In this case, it can either be read as \slash kata\slash  or \slash gata\slash . }

\par{vi. ${\overset{\textnormal{はは}}{\text{母}}}$ 方の ${\overset{\textnormal{そふ}}{\text{祖父}}}$ \hfill\break
One\textquotesingle s grandfather on one\textquotesingle s mother\textquotesingle s side }

\par{ がた may be used in letters when trying to address someone that isn\textquotesingle t the owner of the residence. For instance, say Mayuko Suzuki lives with someone by the surname of Izumi, but it is Mr. Izumi that owns the residence, yet you wish to send something to her. }

\par{vii. ${\overset{\textnormal{とうきょうとしぶやくじんぐうまえ}}{\text{東京都渋谷区神宮前}}}$ ○-○-○ ${\overset{\textnormal{いずみさまがた}}{\text{泉様方}}}$  ${\overset{\textnormal{すずきまゆこさま}}{\text{鈴木真悠子様}}}$ \hfill\break
\#-\#-\# Jingūmae, Shibuya Ward, Tokyo Metropolis \hfill\break
To Mayuko Suzuki in the care of Mr. Izumi }

\par{ If a word for “direction” can also refer to the people—which go places—and methods that those people do something, it\textquotesingle s logical to conclude that this word might also mean “how to (do something)” and by extension, doing it altogether. }

\par{viii. ${\overset{\textnormal{つか}}{\text{使}}}$ い ${\overset{\textnormal{かた}}{\text{方}}}$ \hfill\break
How to use }

\par{ix. ${\overset{\textnormal{せんかた}}{\text{詮方}}}$ ない \hfill\break
It can\textquotesingle t be helped }

\par{\textbf{Phrase Note }: Although viii. is a usage we're familiar with, ix. is difficult because of its confusing spelling and older grammar. The せん comes from する (to do) in a form that is equivalent to today\textquotesingle s しよう, which shows volition to do something. Essentially, this is an older phrase for expressing there\textquotesingle s nothing that can be done, or at least in a willful sense. Although the せん is from a verb, 詮 is used because it means “method.” }

\par{x. ${\overset{\textnormal{う}}{\text{撃}}}$ ち ${\overset{\textnormal{かた}}{\text{方}}}$ やめ \hfill\break
Cease fire! }

\par{\textbf{Phrase Note }: This phrase is a remnant of older grammar used to indicate doing. Nowadays, we use nominalizers like の or \emph{ }こと to help us do this, but at one time, かた was another viable option. Today, it\textquotesingle s relegated to old-time expressions and bureaucratic honorifics like in xi. }

\par{xi.i  ご ${\overset{\textnormal{しょうちかた}}{\text{承知方}}}$ お ${\overset{\textnormal{ねが}}{\text{願}}}$ いします。 \hfill\break
xi.ii ご ${\overset{\textnormal{しょうちお}}{\text{承知置}}}$ き ${\overset{\textnormal{くだ}}{\text{下}}}$ さい。 \emph{\hfill\break
 }Please be aware. }

\par{\textbf{Phrase Note }: In xi.i, the sentence in question would likely be written in a document of some form, perhaps an e-mail, sent among bureaucrats. xi.ii would be what a normal person would write, but the addressee cannot be someone above one\textquotesingle s own status. }

\par{ In addition to the usages above, がた also developed the ability to refer to general amount. After all, it indicates relative time in words like 明け方 (dawn). This usage, however, is no longer in general use. }

\par{xii. ${\overset{\textnormal{そうば}}{\text{相場}}}$ は、 ${\overset{\textnormal{さん}}{\text{3}}}$ ${\overset{\textnormal{わり}}{\text{割}}}$ \{ ${\overset{\textnormal{がた}}{\text{方}}}$ ・ほど\} ${\overset{\textnormal{げらく}}{\text{下落}}}$ した。 \hfill\break
The market value dropped approximately three-tenths. }

\par{\textbf{Phrase Note }: Nowadays, phrases like ほど are used for this sort of situation. }
      
\section{ほう}
 
\par{ Words are often borrowed into languages simply because the other language one is borrowing from is more prestigious. Chinese has been to Japanese what Latin has been to English. Gradually, ほう replaced かた in many of its usages for this very reason. }

\par{\textbf{Spelling Note }: Before we dive into the usages of ほう, it\textquotesingle s important to note that many Japanese learners are taught that it is improper to write it out in \emph{Hiragana }as ほう and to always write it out as 方. However, especially when both かた and ほう make sense, writers will write ほう in \emph{Hiragana }to differentiate between the two. }

\begin{center}
\textbf{General Direction }
\end{center}

\par{ If you want to say which general direction you\textquotesingle re going, you can add の方 to your destination. You can use a general direction word or an actual place. Therefore, something like 北の方 would be the same as “northward” and 大阪の方 would be the same as saying “the Ōsaka area.” To literally say “direction,” though, you would need to use the word 方向, which not surprisingly has 方 in it. }

\par{1. ${\overset{\textnormal{きゅう}}{\text{急}}}$ きょ、 ${\overset{\textnormal{おおさか}}{\text{大阪}}}$ から ${\overset{\textnormal{きゅうしゅう}}{\text{九州}}}$ の ${\overset{\textnormal{ほう}}{\text{方}}}$ へ ${\overset{\textnormal{い}}{\text{行}}}$ くことになりました。 \hfill\break
It\textquotesingle s been decided in haste that I go from Ōsaka to the Kyushu area. }

\par{\textbf{Spelling Note }: 急きょ may also be spelled as 急遽. }

\par{2. ${\overset{\textnormal{えき}}{\text{駅}}}$ の ${\overset{\textnormal{ほう}}{\text{方}}}$ へ ${\overset{\textnormal{む}}{\text{向}}}$ かいました。 \hfill\break
I went towards the train station. }

\par{3. ${\overset{\textnormal{とうきょう}}{\text{東京}}}$ の ${\overset{\textnormal{ほう}}{\text{方}}}$ に ${\overset{\textnormal{しごと}}{\text{仕事}}}$ に行っています。 \hfill\break
I go to work in the Tokyo area. }

\begin{center}
\textbf{"Which" Part of a Comparison }
\end{center}

\par{ When trying to describe someone as one type or another (comparison), we use \emph{h }\emph{ō }方. We also use it to tell “which” one we\textquotesingle re talking about. }

\par{4. ${\overset{\textnormal{ぼく}}{\text{僕}}}$ は ${\overset{\textnormal{もと}}{\text{元}}}$ から ${\overset{\textnormal{い}}{\text{胃}}}$ が ${\overset{\textnormal{よわ}}{\text{弱}}}$ いほうだ。 \hfill\break
I\textquotesingle ve always had a weak stomach. }

\par{\textbf{Sentence Note }: This sentence literally means “As for me, I\textquotesingle m the one who always has had a weak stomach.” }

\par{5. ${\overset{\textnormal{えいご}}{\text{英語}}}$ より ${\overset{\textnormal{かんこくご}}{\text{韓国語}}}$ の ${\overset{\textnormal{ほう}}{\text{方}}}$ が ${\overset{\textnormal{とくい}}{\text{得意}}}$ です。 \hfill\break
I\textquotesingle m better at Korean than English. }

\par{6. ${\overset{\textnormal{ごよう}}{\text{誤用}}}$ のほうが ${\overset{\textnormal{ひろ}}{\text{広}}}$ く ${\overset{\textnormal{つた}}{\text{伝}}}$ わっている。 \hfill\break
The misuse is what\textquotesingle s most widely circulating. }

\par{7. こちらのほうが ${\overset{\textnormal{わる}}{\text{悪}}}$ かったよ。 \hfill\break
I\textquotesingle m the one who was wrong. }

\par{8. バーガーキングのほうが、マクドナルドより ${\overset{\textnormal{す}}{\text{好}}}$ きです。 \hfill\break
I like Burger King more than McDonald's. }

\par{9. ${\overset{\textnormal{あか}}{\text{赤}}}$ ワインのほうが、ビールより ${\overset{\textnormal{からだ}}{\text{体}}}$ にいいですよ。 \hfill\break
Red wine is better for you than beer. }

\par{10. ${\overset{\textnormal{へや}}{\text{部屋}}}$ は ${\overset{\textnormal{ひろ}}{\text{広}}}$ いほうがいいじゃないですか。 \hfill\break
Isn\textquotesingle t it better that the room be wide? }

\par{11. ${\overset{\textnormal{に}}{\text{逃}}}$ げるより ${\overset{\textnormal{たたか}}{\text{戦}}}$ うほうがいいぞ。 \hfill\break
It\textquotesingle s better to fight than run away! }

\par{12. どちらの ${\overset{\textnormal{ほう}}{\text{方}}}$ が ${\overset{\textnormal{す}}{\text{好}}}$ きですか。 \hfill\break
Which do you like (better)? }

\begin{center}
\textbf{General Occupation }
\end{center}

\par{ If ほう can show general direction and “which” thing you\textquotesingle re talking about, it\textquotesingle s not that much of a stretch for it to be used to vaguely indicate what you do for a living. After all, your livelihood is carried out somewhere, and by using ほう, you are telling the person in what general field you\textquotesingle re working in. }

\par{13.うちの父は財務省の方に勤めています。 \hfill\break
My father works at the Ministry of Finance. }

\par{14. 「お仕事は?」「 ${\overset{\textnormal{きんゆう}}{\text{金融}}}$ のほうです。」 \hfill\break
“What do you do for work? “I\textquotesingle m in financing.” }

\begin{center}
\textbf{Use in Honorifics: Avoiding Directness }
\end{center}

\par{ There is a general pattern that 方 is used in generalizations. Some speakers assert that its use in indicating occupation is incorrect and unfounded, but that\textquotesingle s not the case. This usage has existed for a long time, and although it isn't grammatically necessary, that\textquotesingle s not why people use it. If you live to the west , why would you need to use 方? It would be just as easy for you to only use 西? That\textquotesingle s not, though, what often goes through the mind of a Japanese speaker. Whenever a speaker feels it\textquotesingle s important to emphasis the general direction of someplace, that person will use 方. }

\par{ The same logic works for when 方 refers to general occupation. Using it makes the statement less direct, and by doing so, also making it politer. Many speakers are taught to use 方 in this manner profusely, especially when they work at fast food restaurants and part time jobs where employees are taught how to address customers\slash clients via a manual. }

\par{ In the sentences below, every instance of 方 is grammatically unnecessary. When it\textquotesingle s used with a place, it may cause confusion as to whether the speaker is pointing out a general location or is just trying to politer. Even if this ‘can\textquotesingle  be the case, someone would have to be purposely rude or incompetent not to know how the word is intended. }

\par{15. お ${\overset{\textnormal{にもつ}}{\text{荷物}}}$ のほう、お ${\overset{\textnormal{あず}}{\text{預}}}$ かりします。 \hfill\break
I will hold onto your luggage. }

\par{16. 私のほうでやらせていただきます。 \hfill\break
I will be doing it. }

\par{17. メニューのほう、お ${\overset{\textnormal{さ}}{\text{下}}}$ げします。 \hfill\break
Allow me to take your menu. }

\par{18. お ${\overset{\textnormal{からだ}}{\text{体}}}$ のほうはどうですか。 \hfill\break
How is your body feeling? }

\par{19. カットのほう、させていただきます。 \hfill\break
I\textquotesingle ll be cutting your hair. }

\par{20. ムースのほう、お ${\overset{\textnormal{つ}}{\text{付}}}$ けします。 \hfill\break
I\textquotesingle m going to now apply muse. }

\par{21. デザートのほう、お ${\overset{\textnormal{だ}}{\text{出}}}$ ししてよろしいですか。 \hfill\break
Shall I bring out desert? }

\par{22. 私のほうから ${\overset{\textnormal{せつめい}}{\text{説明}}}$ させていただきます。 \hfill\break
I will be the one explaining. }

\par{23. 私のほうで ${\overset{\textnormal{たんとう}}{\text{担当}}}$ いたします。 \hfill\break
I will be the one leading. }

\par{24. ${\overset{\textnormal{けいやくしょ}}{\text{契約書}}}$ のほうをお ${\overset{\textnormal{も}}{\text{持}}}$ ちいたします。 \hfill\break
I will bring the contract. }

\par{25. コーヒーのほう、お ${\overset{\textnormal{も}}{\text{持}}}$ ちしました。 \hfill\break
I\textquotesingle ve brought your coffee. }

\par{26. お ${\overset{\textnormal{しょくじ}}{\text{食事}}}$ のほうをお ${\overset{\textnormal{も}}{\text{持}}}$ ちいたしました。 \hfill\break
I\textquotesingle ve brought your meal. }

\par{27. お ${\overset{\textnormal{かいけい}}{\text{会計}}}$ のほう、 ${\overset{\textnormal{せんごひゃく}}{\text{1500}}}$ ${\overset{\textnormal{えん}}{\text{円}}}$ になりました。 \hfill\break
Your bill has come out to 1500 yen. }

\par{28. こちらのほう、ご ${\overset{\textnormal{りよう}}{\text{利用}}}$ ください。 \hfill\break
Please use this (one ???). }

\par{\textbf{Sentence Note }: This instance could legitimately instead mean “please use this one.” This means that it conversely indicates which should be used rather than being vague. This is because こちらのほう and the like can be seen as set phrases, and seeing them triggers the interpretation of “which” for 方. }

\par{29. ご ${\overset{\textnormal{ちゅうもん}}{\text{注文}}}$ のほうは、お ${\overset{\textnormal{き}}{\text{決}}}$ まりですか。 \hfill\break
Have you decided on your order? }

\par{30. ${\overset{\textnormal{う}}{\text{売}}}$ り ${\overset{\textnormal{ば}}{\text{場}}}$ のほうをご ${\overset{\textnormal{あんない}}{\text{案内}}}$ します。 \hfill\break
I\textquotesingle ll show you the sales floor. }

\par{\textbf{Sentence Note }: This sentence could potentially mean “I will guide you in and around the sales area, but that is only due to the grammatical possibility of it meaning that. The chance that a speaker is actually using this phrase to mean such is slim to none. }

\par{ These examples demonstrate the wide variety of instances in which ほう is used in honorifics. As for you, the Japanese learner, get a feel for your surroundings and how the people you interact with use this word to guide your own linguistic choices.  }
    