    
\chapter{The Verbs かかる \& かける}

\begin{center}
\begin{Large}
第296課: The Verbs かかる \& かける 
\end{Large}
\end{center}
 
\par{ The verbs かかる (intransitive) and かける (transitive) are by far two of the most extensively used verbs in Japanese. Aside from the specific かかる and かける being discussed in this lesson, there are also other homophonous "かかる" and "かける" verbs. }
      
\section{The Independent Verbs かける・かかる}
 
\par{\textbf{Transitivity Note }: Note that the translations take advantage of the lack of morphologically distinct transitive and intransitive forms for most verbs in English. When necessary, notes will be provided. }

\par{\textbf{漢字 Note }: These verbs are typically written in ひらがな. 掛ける・掛かる is typically alright. When there is another spelling common for a particular nuance, this will be evident in the examples and noted. }

\par{1. To hang; sit; weigh on; attach\slash wear. }

\par{1. 出口に注意の標識をかけたので、災害があったら、安全に逃げられるだろう。 \hfill\break
Since we hung a warning sign at the exit, if there were a disaster, people could escape safely. }

\par{2. 空に美しい虹が懸かることは神の意の印の一つであるという。 \hfill\break
They say that a beautiful rainbow hanging in the sky is one of the signs of God's will. }

\par{3. ローマ帝国では罪人が十字架に架けられた。 \hfill\break
Sinners were hung on crosses in the Roman Empire. }

\par{4. 山の頂上に白雲がかかっていた。 \hfill\break
White clouds wrapped around the tops of the mountains. }

\par{5. 調査委員会に圧力が掛かった。 \hfill\break
Pressure hung over the investigation commission. }

\par{6. 心配をかけてごめんね。 \hfill\break
Sorry for making you worry. }

\par{7. 優勝がかかってるぞ。(Masculine) \hfill\break
Victory is at our reach! }

\par{8. ハンドルに手を掛ける。 \hfill\break
To sit one's hand on a handle. }

\par{9. 積荷がずしりとかかっていたから、どんどん腕が痛くなってきた。 \hfill\break
The pain was quickly getting worse because of the load weighing down on my arms. }

\par{10. 重心を失って左足に躓きかけてしまう。 \hfill\break
To end up weighing down and stumbling on one's left foot (by) losing one's balance. }

\par{11. 彼は眼鏡を掛けた。 \hfill\break
He wore glasses. }

\par{12. 試験のことなど歯牙にもかけないやつだな。 \hfill\break
He's a guy that pays no attention to stuff like the exam, isn't he? }

\par{\textbf{漢字Note }: Usually in かな. Most commonly written in 漢字 as 掛ける・掛かる. For a true physical sense of “to hang\slash be hung”, use 架ける・架かる. 懸ける・懸かる gives a sense of hanging over. }

\par{2. To do with a tool of some sort; only a usage of かける. }

\par{13. エンジンをかける。 \hfill\break
To start an engine. }

\par{14. 午前6時に目覚まし時計をかけた。 \hfill\break
I set an alarm for 6 A.M. }

\par{15. 一晩中ラジオの音がかかっていた。 \hfill\break
The sound of the radio was going on all night long. }

\par{16. CDをかける。 \hfill\break
To play a CD. }

\par{17. 袖にミシンをかける。 \hfill\break
To use a sewing machine on a sleeve. }

\par{18. 清掃員が廊下に掃除機をかけたが、生徒は一週間でまた汚した。 \hfill\break
Although the janitor vacuumed the hallway, the students made it messy again in just a week. }

\par{19. 両手に手錠がかかった。 \hfill\break
The cuffs were locked on both his hands. }

\par{20. 戸に鍵をかけたか。 \hfill\break
Did you lock the door with the key? }

\par{3. In 目に掛かる meaning "to keep in one's eyes". In honorifics, it is the humble form of 見る. And, にお目に掛かる is an honorific equivalent of お会いにする. }

\par{21. 半弓を似て目にかかる敵を射て \hfill\break
Shoot the enemy in one's eyes resembling a small bow \hfill\break
From 鷗外 }

\par{22. お初にお目にかかります。 \hfill\break
It's a pleasure to meet you. }

\par{4. Without seeing a resolution and the mind unsettled. In a transitive sense, it is closest to not forgetting anything. }

\par{23. 今日一日中、試験の結果のことが気にかかっていたんだ。 \hfill\break
The results of the exam hung in my mind all day long. }

\par{24. いつも私のことを心に懸けて下さってありがたいことです。 \hfill\break
I am grateful that you always kept me in mind. }

\par{25. 思いを懸ける。 \hfill\break
To burden with thought. }

\par{\textbf{漢字 Note }: Often written with 懸ける・懸かる. }

\par{5. To span; to straddle; wrap around. }

\par{26. 多摩川に鉄橋を架ける。 \hfill\break
To build a steel bridge over the Tama River. }

\par{27. 本州・四国間に橋が架かっている。 \hfill\break
A bridge spans between Honshu and Shikoku. }

\par{28. アオカケスが軒下に巣をかけることが多い。 \hfill\break
Blue jays often build nests in the overhangs of roofs. }

\par{29. 泥棒に縄をかけたよ。 \hfill\break
I bound the rope on the burglar! }

\par{30. 肩に襷をかける。 \hfill\break
Idiom: To exaggerate one's circle in a conversation. }

\par{\textbf{漢字 Note }: Excluding the last two examples, this usage is usually spelled with 架ける・架かる. }

\par{6. To fictionalize; perform. }

\par{31. 芝居小屋を掛ける。 \hfill\break
To fictionalize theater. }

\par{32. 公園に芝居がかかっている。 \hfill\break
Drama is being played in the park. }

\par{33. 忠臣蔵がかかっている。 \hfill\break
A play based off of the 47 ronin is being played. }

\par{7. To squirt; to pour spices on; cover; set on fire (which is covering fire over something). }

\par{34. この料理の最後の手順はフライにソースをかけることだ。 \hfill\break
The last thing in making this dish is putting sauce in the fry pan. }

\par{35. 庭木に水をかけましたか。 \hfill\break
Did you squirt water on the garden trees? }

\par{36. 埃が頭にかかった。 \hfill\break
Dust dropped onto my head. }

\par{37. クラッシュドレッドペッパーのかかったスパゲッティはおいしい! \hfill\break
Spaghetti with red pepper poured on it is delicious! }

\par{38. 日本では布団をかけて寝ますよ。 \hfill\break
In Japan, you sleep on a futon. }

\par{39. 我々は敵の城に火をかけた。 \hfill\break
We set the enemy's castle on fire. }

\par{8. To offer words. }

\par{40. 彼に後ほどお礼の電話を掛けさせていただきます。 \hfill\break
I will give him a telephone call of gratitude. }

\par{41. 声をかけたが、まだ聞こえなかったようだ。 \hfill\break
I raised my voice, but it looks like he still couldn't hear. }

\par{42. 若旦那からお座敷がかかりました。 \hfill\break
A gathering was made from the young master. }

\par{\textbf{Cultural Note }: A 座敷 was a gathering one was invited to perform as a geisha. }

\par{43. 発破をかける。 \hfill\break
To motivate someone with harsh words. }

\par{9. A skill is done on an opponent well. Deceiving and trapping are good examples. }

\par{44. 王女に魔法をかけよう! \hfill\break
Let's put a spell on the queen! }

\par{45. 睡眠術に掛かる。 \hfill\break
To be under hypnosis. }

\par{46. ぺてんにかけて金品を奪う。 \hfill\break
To trick and mug. }

\par{47. 詐欺に(引っ)かかった。 \hfill\break
I was tricked in a fraud. }

\par{48. ${\overset{\textnormal{いのしし}}{\text{猪}}}$ をかける。 \hfill\break
To trap a wild boar. }

\par{49. 逃走犯を検問にかけた。 \hfill\break
I trapped an escapee in examination. }

\par{10. To wish for divine help. }

\par{50. 合格できますようにと神様に願を懸けました。 \hfill\break
I prayed to God in order to pass. }

\par{51. 望みをかける。 \hfill\break
To wish for hope. }

\par{\textbf{漢字 Note }: 懸ける is the most common spelling. }

\par{11. To cost time, expenses, labor; to tax. }

\par{52. 完成までに二年かかったうえに、十億円かかった。 \hfill\break
It cost two years and a billion dollars by completion. }

\par{53. 出入時には保護関税がかかる。 \hfill\break
There is a protective tariff during exporting. }

\par{54. 所得税をかける。 \hfill\break
To impose an income tax. }

\par{55. 12日朝からは、復旧作業が始まっているということですが、発電所の安全確認などには時間がかかりそうだということです。 \hfill\break
Restoration efforts are said to be starting from the morning of the 12th, but the confirmation of safety of the power plants will take time. \hfill\break
From NHK following the 2011 earthquake. }

\par{12. ~にかけて(は): See Lesson 177. }

\par{13. In "鼻にかける" meaning "to boast" or in "鼻にかかる" referring to hanging in the nose. }

\par{56. 自分の日本語の能力を鼻にかける。 \hfill\break
To boast about one's Japanese skills. }

\par{57. 先生の声が風邪で鼻にかかって聞き苦しかったよ。 \hfill\break
The teacher's voice hanged in his nose due to a cold and was unpleasant to hear. }

\par{14. To come to. }

\par{58. 道路が湖畔にかかる辺りで新車が故障したみたいだ。 \hfill\break
It looks like a new car is broken-down coming near the bank passing the road on the lake shore. }

\par{59. 勝負は終盤にかかってとても緊迫した展開となりました。 \hfill\break
The match came to its final stage and became a very tense development. }

\par{15. To be a part; undertaken in; only a usage of かかる. }

\par{60. 映画の創作にこれからかかるとしたら、どんなことをするだろう。 \hfill\break
If I were to be a part in the production of a movie, what kind of things would I do? }

\par{61. 整理にかかっている。 \hfill\break
To be engaged in organizing. }

\par{62. 今すぐ準備にかかれ。 \hfill\break
Be getting prepared right now! }

\par{63. 明日の仕事にかかるまでしばらく休もう。 \hfill\break
Let's sleep for a while until we undertake in tomorrow's work. }

\par{16. To be produced due to an effect. }

\par{64. 髪にパーマをかけた。 \hfill\break
I got a perm in my hair. }

\par{65. 仕事にエンジンがかかった。 (Idiomatic) \hfill\break
To get back to normal in one's job. }

\par{66. 磨きをかけることはいいことです。 \hfill\break
Polishing up (one's skills) is always a good thing. }

\par{17. にかけて shows the "to" that you swear "to". }

\par{67. 神にかけて誓った。 \hfill\break
I swore to God. }

\par{68. 神仏に懸けて誓うのはよくないです。 \hfill\break
Swearing to God is bad. }

\par{\textbf{漢字 Note }: Either spelled as 懸ける or 賭ける, but the latter spelling is the most common. }

\par{18. To carry out something while prepared to lose; only a usage of かける. }

\par{69. 威信を賭けて戦うのは賢い決意だというわけではありません。 \hfill\break
Fighting through with one's dignity is not a wise decision. }

\par{70. 一発逆転を懸ける。 \hfill\break
To turn the tables around. }

\par{71. ポーカーに沢山の金を賭ける。 \hfill\break
To gamble a lot of money in poker. }

\par{72. 雨が降る降らないに昼飯をかけるのは子供っぽいゲームじゃないか。 \hfill\break
Isn't betting your lunch on whether it's going to rain or not a childish game? }

\par{\textbf{漢字 Note }: 賭ける is the most important spelling. }

\par{19. In "\dothyp{}\dothyp{}\dothyp{}てかかる" meaning "to cope with". }

\par{73. 敵を嘗めてかかる。 \hfill\break
To deal with an enemy putting him down. }

\par{74. 心してかかることが肝心です。 \hfill\break
Things you carefully cope with are crucial. }

\par{20. In "にかかる" following the 連用形 of a transitive verb meaning "to do\dothyp{}\dothyp{}\dothyp{}as an action of starting out". }

\par{75. 弱いと見るとすぐさま ${\overset{\textnormal{おど}}{\text{脅}}}$ しにかかるのは自然な反応ではないか。 \hfill\break
Isn't starting to threaten someone immediately when you see them as being weak a natural reaction? }

\par{76. 引き止めにかかったが、どうしても応じなかったんだ。 \hfill\break
I started to restrain (him), but I didn't get (him) to comply in the long run. }

\par{\textbf{Usage Note }: This usage is not used to show habit and is used to show a temporary action. }

\par{21. In "(よ)うとかかる", a stronger version of "(よ)うとする". }

\par{77. 彼はあの手この手で説得しようとかかってきたが、結局説得できなかったよ。 \hfill\break
He tried to persuade us this way and that, but he couldn't persuade us (about) anything well. }

\par{22. To fight against; only a usage of かかる. }

\par{78. 不良が突っかかってきた。 \hfill\break
A delinquent charged at me. }

\par{79. 者ども、掛かれ! \hfill\break
Men, fight! }

\par{23. To receive medical care or a doctor's examination. }

\par{80. すぐにお医者さんにかかりなさい。 \hfill\break
Please have a doctor see you immediately. }

\par{81. 重症ですから、すぐ医者にかかるべきですよ。 \hfill\break
Because it's a serious illness, you should see a doctor soon. }

\par{24. To sell at an auction; only a usage of かける. }

\par{82. 競売にかける。 \hfill\break
To hold auction. }

\par{25. To adopt legislation or measures; bind an agreement. }

\par{83. 事件を裁判にかけます。 \hfill\break
We will give a judgment in the case. }

\par{84. 案件を会議に懸けました。 \hfill\break
We adopted the matter in question in the meeting. }

\par{85. 毒殺事件が裁判にかけられた。 \hfill\break
A judgment was given in the poisoning case. }

\par{86. 自分の生命に保険をかけるのはとても大切なのです。 \hfill\break
To get an insurance agreement for one's own life is very important. }

\par{\textbf{漢字 Note }: 懸ける・懸かる is most appropriate. However, it is usually written in ひらがな. }

\par{26. To carry traits or tendencies; only a usage of かかる. }

\par{87. 青みのかかった緑色 \hfill\break
Blue-tinged green }

\par{88. 彼は凄みのかかった顔で睨み付けた。 \hfill\break
He glared at me with a ghastly look. }

\par{27. To dispose of oneself. }

\par{89. 自らの手にかける。 \hfill\break
To kill oneself with one's own hands. }

\par{90. 敵の手にかかる。 \hfill\break
To be disposed of by the enemy's hand. }

\par{28. "To be concerned", "be the work of", or "to concern". Only a usage of 係る. }

\par{91. 止利仏師の制作にかかった仏像でございます。 \hfill\break
This is a Buddhist statue in the image of the works of Toribusshi. }

\par{92. 賜杯の行方は大名の活躍に係っている。 \hfill\break
The whereabouts of the Emperor's cup is controlled by the great efforts of the Daimyo. }

\par{93. 試合の成否は努力いかんに係っている。 \hfill\break
The outcome of the contest depends on your effort. }

\par{94. この副詞の意味は文末の動詞に係っている。 \hfill\break
The meaning of this adverb is linked to the verb at the end of the sentence. }

\par{29. To give a ball an extraordinary rotation; only a usage of かける. }

\par{95. 打球に回転をかける。 \hfill\break
To have one\textquotesingle s ball spin. }

\par{30. To multiply; only a usage of かける. }

\par{96. 8×(かける)2は16。 \hfill\break
8 times 2 equals 16. }

\par{31. To overcharge; only a usage of かける. }

\par{97. 定価に二倍をかけて売る。 \hfill\break
To buy at two times the standard price. }

\par{32. To pay bills regularly; only a usage of かける. }

\par{98. 保険金をかけている。 \hfill\break
I'm paying insurance. }

\par{33. To crossbreed; only a usage of かける. }

\par{99. 雌馬にロバをかける。 \hfill\break
To breed a female horse with a donkey. }

\par{34. To play on words; only a usage of かける. }

\par{100. 『秋風ぞ立つ』の『秋』に『飽き』をかける。 \hfill\break
To imply "weariness" in the "autumn" of "rise autumn wind". }
      
\section{The Suffix -かかる}
 1. Same as usage 26 from earlier with the only difference being that it attaches itself to nouns instead. \hfill\break

\par{101. あいつは左がかった意見を持ってるね。 \hfill\break
That guy has a leftist opinion, doesn't he. }

\par{2. Shows the possession of color traits. }

\par{102. オレンジがかった赤い色 \hfill\break
An orange tinted red color }

\par{103. 黒みがかった紺色 \hfill\break
A black tinted navy blue }

\par{104. 娘はむらさきと白の細かい矢がすりのきものに、 ${\overset{\textnormal{えんじ}}{\text{臙脂}}}$ がかったむらさきの袴をはいていた。 \hfill\break
She was wearing a fine white and purple arrow feather kimono atop a rouge-tinged hakama. \hfill\break
From 不死 by 川端康成. }
      
\section{罹る}
 
\par{ This "かかる" verb means "to catch a disease" or "to be caught in a problem". }

\par{105. いとこは食中毒に罹った。 \hfill\break
My cousin got food poisoning. }

\par{106. 結核に罹ると、ときどき命の危険がある。 \hfill\break
When you get tuberculosis, there may be a risk to your life. }

\par{107. 放射線を浴びて、原子病に罹ってしまう。 \hfill\break
To contract radiation sickness from radiation. }
      
\section{The 連体詞 斯かる}
 
\par{ 斯かる is the classical version of このような. }
      
\section{欠ける}
 
\par{ This "かける" verb is intransitive, and its transitive form is 欠く. }

\par{1. For a part of a sold item to break of; to chip. }

\par{108. 机の縁が欠けた。 \hfill\break
The edge of the desk got chipped. }

\par{109. コーヒーカップの取っ手が欠けてしまったから、滴がじんわりと漏れてるよ。 \hfill\break
Because the coffee cup's handle accidentally got chipped, drops are slowly leaking. }

\par{2. To be missing. }

\par{110. グループのメンバーの三人が欠けている。 \hfill\break
Three people are missing out of the group. }

\par{111. この取扱説明書は5ページ欠けています。 \hfill\break
There are 5 pages missing in this user's manual. }

\par{3. To lack; to be negligent toward. }

\par{112. 攻撃は決め手に欠けたから、失敗した。 \hfill\break
Because the attack lacked a decisive move, it filled. }

\par{113. 最新記録に一秒欠ける。 \hfill\break
To lack 1 second of a record. }

\par{114. 1ドルから一セント欠ける。 \hfill\break
To lack a cent from a dollar. }

\par{4. To wane. }

\par{115. 満月が段々円形でなくなるのは常識だ。 \hfill\break
A full moon waning gradually losing in round shape is common knowledge. }
      
\section{翔ける}
 
\par{ This "かける" verb is transitive and means "to fly through the heavens". }

\par{116. 魂魄が天空を翔ける。 \hfill\break
Souls fly through the heavens. }
      
\section{駆・駈ける}
 
\par{ This "かける" verb has two usages related to each other and is transitive. It either means "to dash" or "to fly through the sky". }

\par{117. 神風が大空を駈けた。 \hfill\break
Divine wind flew through the skies. }

\par{118. 時空を駆ける。 \hfill\break
To dash through time and space. }

\par{119. 虎が熱帯林を駆けた。 \hfill\break
The tiger dashed through the tropical forest. }
      
\section{The Potential Form of Verbs}
 
\par{ Finally, to end this lesson, かける may also be the potential form of the verbs to write 書く, to scratch 掻く, and to draw 描く. }

\par{120. 漢字で書けない。 \hfill\break
I can't write in Kanji. }

\par{121. 美しい絵が描ける。 \hfill\break
To be able to draw a beautiful picture. }
    