    
\chapter{Nominal Phrases III}

\begin{center}
\begin{Large}
第288課: Nominal Phrases III 
\end{Large}
\end{center}
 
\par{ This is the last lesson about important nominal phrases. Most of them are related to each other, so that will make things easier. }

\begin{ltabulary}{|P|P|P|P|P|}
\hline 

至極 & 極み & 至り & 万一 & ゆかり \\ \cline{1-5}

\end{ltabulary}
      
\section{至極}
 
\par{ As a noun, 至極 shows an "extremity" of something. As an adverb, it means "extremely" or "exceedingly". }

\par{1. 至極便利な電気器具ですね。 \hfill\break
This is an extremely convenient electric appliance, isn't it? }

\par{2a. 滑稽至極 \hfill\break
2b. 滑稽の極み (More natural) \hfill\break
The extremity of ridicule }

\par{3a. 恐縮至極 \hfill\break
3b. 恥ずかしさの極み (More natural) \hfill\break
The extremity of shame }

\par{4. 至極当然である。 \hfill\break
To be exceedingly obvious. }
      
\section{極み}
 
\par{ 極み is a noun meaning "sublime", "extremity", and "height" and shows the utmost. This word is typically not used in speaking. }

\par{5. 絶望の極みに ${\overset{\textnormal{おちい}}{\text{陥}}}$ る。 \hfill\break
To fall into the height of despair. }

\par{6. 興奮の極み \hfill\break
The utmost of excitement }

\par{7. ここで断念するのは ${\overset{\textnormal{いかん}}{\text{遺憾}}}$ の極みだ。 \hfill\break
Abandoning here is the height of regret. }

\par{8. ${\overset{\textnormal{ぜいたく}}{\text{贅沢}}}$ の極みの生活をする。 \hfill\break
To live a lifestyle of utmost luxury. }
      
\section{至り}
 
\par{ 至り either means "the utmost" or "result". }

\par{9. ${\overset{\textnormal{ふしまつ}}{\text{不始末}}}$ は ${\overset{\textnormal{わかげ}}{\text{若気}}}$ の至りと ${\overset{\textnormal{はんせい}}{\text{反省}}}$ しております。 \hfill\break
(物事において)不始末であったことは、私の若気の至りです。 \hfill\break
Carelessness is reflecting on the stupidity of my youth. }

\par{10. ご ${\overset{\textnormal{どうけい}}{\text{同慶}}}$ の至りに存じます。 \hfill\break
I offer you mutual congratulations. }

\par{11. ${\overset{\textnormal{きょうしゅく}}{\text{恐縮}}}$ の至りです。 \hfill\break
I am sorry to the utmost. }
      
\section{万一}
 
\par{ 万一 expresses that something rarely happens, but there is a possibility of the otherwise. It may be translated as "by some chance\slash possibility". It may also be used as an adverb. }

\par{12. 彼に万一のことがあれば \hfill\break
If anything happens to him }

\par{13. 我々は万一を考えなければならぬ。 \hfill\break
We must think and prepare for the worst. }

\par{14. 彼は万一の場合に備えた。 \hfill\break
He prepared for contingencies. }

\par{15. 万一遅れたら、先に行ってくださいね。 \hfill\break
If there is the chance that you could be late, please go earlier than (planned), alright? }
      
\section{ゆかり}
 
\par{ ゆかり, in 漢字 as 縁, means "affinity", but it can also show that something or someone is related to some place. It is this usage that catch people off guard. }

\par{16. ${\overset{\textnormal{そうせき}}{\text{漱石}}}$ ゆかりの地 \hfill\break
Place in connection with Soseki }

\par{17. ${\overset{\textnormal{ぶんごう}}{\text{文豪}}}$ ゆかりの地 \hfill\break
Place in connection with a literary legend }
    