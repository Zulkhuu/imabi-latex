    
\chapter{Intransitive \& Transitive}

\begin{center}
\begin{Large}
第257課: Intransitive \& Transitive: Part 6 
\end{Large}
\end{center}
 
\par{ In this sixth lesson on verbs that can either be used intransitively or transitively, we will focus on verbs that have unique strings attached to them. There are three broad categories that we\textquotesingle ll be looking at. }

\par{1. Either the intransitive or the transitive usage is relatively new in the language. Meaning, some speakers will think it\textquotesingle s wrong to use it a certain way but many speakers still do. \hfill\break
2. The use of the verb in a transitive sense is done so to implicitly show a connection between an agent and an action. \hfill\break
3. The use of the verb in a transitive sense is done so to emphasize the agent\textquotesingle s volition in said action. }

\par{ Each verb to be looked as must be done so on an individual basis. Although they can all more or less be classified into one of these three points, they\textquotesingle re special for reasons that must be carefully considered. }
      
\section{間違う vs 間違える}
 
\par{ The verb 間違う has the basic meaning of “to be mistaken\slash incorrect.” Essentially, a certain situation and or certain results are different (than what they should be). Its transitive form is 間違える. 間違える incidentally either means “to fail\slash make a mistake (in)” or “to mistake something for something else.”  The latter meaning greatly influences how it is used to refer to mistakes. When the mistake in question was made by confusing how it should have played out, 間違える is the verb you should use. }

\par{ Now, the issue at hand is the severity of 間違える and how it affects the use of 間違う. To many speakers, 間違える is very direct. After all, it intrinsically implies that another means was actually right and what was done was wrong. Consequently, some speakers use 間違う instead to lessen the emphasis of the mistake at hand. Incidentally, speakers also tend to use 間違う more often in contexts that involve abstract issues. The more physically ascertainable the mistake is, the more likely 間違える will be used instead of 間違う. }

\par{1. ${\overset{\textnormal{きみ}}{\text{君}}}$ が ${\overset{\textnormal{まちが}}{\text{間違}}}$ っている。 \hfill\break
\emph{You }\textquotesingle re mistaken. }

\par{2. ${\overset{\textnormal{じん}}{\text{人}}}$ は ${\overset{\textnormal{なぜじんせい}}{\text{何故人生}}}$ を ${\overset{\textnormal{まちが}}{\text{間違}}}$ うのか。 \hfill\break
Why do people mess up life? }

\par{3. ${\overset{\textnormal{むすこ}}{\text{息子}}}$ の ${\overset{\textnormal{そだ}}{\text{育}}}$ て ${\overset{\textnormal{かた}}{\text{方}}}$ を ${\overset{\textnormal{まちが}}{\text{間違}}}$ ったんじゃないかと ${\overset{\textnormal{なや}}{\text{悩}}}$ んでいます。 \hfill\break
I'm worried that I might have messed up raising my son. }

\par{4. 粘着テープを貼る場所を\{間違った・間違えた\}けど、貼り直せますか。 \hfill\break
I messed up where I should stick the adhesive tape, but can I re-stick it (where it should have been)? }

\par{5. ${\overset{\textnormal{ふりこみ}}{\text{振込}}}$ の ${\overset{\textnormal{ないよう}}{\text{内容}}}$ を\{ ${\overset{\textnormal{まちが}}{\text{間違}}}$ った・ ${\overset{\textnormal{まちが}}{\text{間違}}}$ えた\}が、 ${\overset{\textnormal{と}}{\text{取}}}$ り ${\overset{\textnormal{け}}{\text{消}}}$ せますか。 \hfill\break
I messed up something in a wire transfer, but can I cancel it? }

\par{6. ワープの ${\overset{\textnormal{ばしょ}}{\text{場所}}}$ を\{ ${\overset{\textnormal{まちが}}{\text{間違}}}$ った・ ${\overset{\textnormal{まちが}}{\text{間違}}}$ えた\}! \hfill\break
We\textquotesingle ve messed up where we were supposed to warp to! }

\par{7. ${\overset{\textnormal{こせきとどけしょ}}{\text{戸籍届書}}}$ の ${\overset{\textnormal{きにゅう}}{\text{記入}}}$ を\{ ${\overset{\textnormal{まちが}}{\text{間違}}}$ った・ ${\overset{\textnormal{まちが}}{\text{間違}}}$ えた\}のですが、どうしたらいいでしょうか。 \hfill\break
I made a mistake in filling out a family registry notification form, but what should I do? }

\par{8. ${\overset{\textnormal{ぶんき}}{\text{分岐}}}$ を\{ ${\overset{\textnormal{まちが}}{\text{間違}}}$ った・ ${\overset{\textnormal{まちが}}{\text{間違}}}$ えた\} ${\overset{\textnormal{くるま}}{\text{車}}}$ が ${\overset{\textnormal{きゅう}}{\text{急}}}$ に ${\overset{\textnormal{しゃせん}}{\text{車線}}}$ (を) ${\overset{\textnormal{へんこう}}{\text{変更}}}$ して ${\overset{\textnormal{じこ}}{\text{事故}}}$ を ${\overset{\textnormal{お}}{\text{起}}}$ こしてしまった。 \hfill\break
A car which had made a mistake at a fork in the road suddenly switched lanes and caused an accident. }
      
\section{終わる vs 終える}
 
\par{ Although 終わる and 終える are intrinsically the intransitive and transitive verbs for “to end\slash finish” respectively, the difference between them is actually not this simplistic. There are in fact three primary patterns with them that you must learn. }

\begin{enumerate}

\item …が終わる: Something ends, finishes, completes, or is terminated, but any agent involved in the situation would only be implicitly implied. \hfill\break

\item …を終わる: To end something as the natural end to a means. The situation at hand has been completed. \hfill\break

\item …を終える: To purposely end something regardless if the matter at hand hasn\textquotesingle t been completed. 
\end{enumerate}

\par{ The forceful nature of the verb 終える is amplified by the use of the particle を. }

\par{10. ${\overset{\textnormal{くんれん}}{\text{訓練}}}$ が ${\overset{\textnormal{お}}{\text{終}}}$ わりました。 \hfill\break
Practice ended. }

\par{11. ${\overset{\textnormal{くんれん}}{\text{訓練}}}$ を ${\overset{\textnormal{お}}{\text{終}}}$ わりました。 \hfill\break
I finished practice. }

\par{12. ${\overset{\textnormal{くんれん}}{\text{訓練}}}$ を ${\overset{\textnormal{お}}{\text{終}}}$ えました。 \hfill\break
I ended\slash finished practice. }

\par{13. ${\overset{\textnormal{しごと}}{\text{仕事}}}$ が ${\overset{\textnormal{お}}{\text{終}}}$ わりました。 \hfill\break
Work (has) ended. \hfill\break
 \hfill\break
14. ${\overset{\textnormal{しごと}}{\text{仕事}}}$ を ${\overset{\textnormal{お}}{\text{終}}}$ わりました。 \hfill\break
I finished work. }

\par{15. ${\overset{\textnormal{しごと}}{\text{仕事}}}$ を ${\overset{\textnormal{お}}{\text{終}}}$ えました。 \hfill\break
I ended\slash finished (the) work. }

\par{16. ${\overset{\textnormal{き}}{\text{聞}}}$ き ${\overset{\textnormal{と}}{\text{取}}}$ り ${\overset{\textnormal{れんしゅう}}{\text{練習}}}$ を ${\overset{\textnormal{お}}{\text{終}}}$ わりました。 \hfill\break
I finished my listening comprehension practice. }

\par{17. うちの ${\overset{\textnormal{こども}}{\text{子供}}}$ は ${\overset{\textnormal{ことし}}{\text{今年}}}$ 、 ${\overset{\textnormal{ぎむきょういく}}{\text{義務教育}}}$ を ${\overset{\textnormal{お}}{\text{終}}}$ わりました。 \hfill\break
My child(ren) finished their compulsory education this year. }

\par{\emph{ }終える is very similar in meaning to 終わらせる. Both indicate stopping something, but 終わらせる implies that you are purposely stopping something once it has been completed. It\textquotesingle s not always the case, however, that 終える represents a reckless ends to a means. When it is used, though, it does give a more poignant, crisp end. This makes it satisfactory for when you express finishing something and you want to relish the moment that you got through the situation. }

\par{18. ${\overset{\textnormal{さいご}}{\text{最後}}}$ の ${\overset{\textnormal{けいこ}}{\text{稽古}}}$ を ${\overset{\textnormal{お}}{\text{終}}}$ えました。 \hfill\break
I finished my last practice. }

\par{19. ${\overset{\textnormal{ぶじ}}{\text{無事}}}$ に ${\overset{\textnormal{けっこんしき}}{\text{結婚式}}}$ を ${\overset{\textnormal{お}}{\text{終}}}$ えました。 \hfill\break
I\textquotesingle ve safely finished the wedding. }

\par{20. この ${\overset{\textnormal{とり}}{\text{鳥}}}$ は ${\overset{\textnormal{すだ}}{\text{巣立}}}$ ちを ${\overset{\textnormal{お}}{\text{終}}}$ えたばかりみたいですね。 \hfill\break
This bird seems to have just completed leaving the nest, huh. }

\par{21. ${\overset{\textnormal{しゅくだい}}{\text{宿題}}}$ を ${\overset{\textnormal{はや}}{\text{早}}}$ く ${\overset{\textnormal{お}}{\text{終}}}$ わらせたい。 \hfill\break
I want to quickly finish my homework. }

\par{22. ${\overset{\textnormal{かがく}}{\text{科学}}}$ は ${\overset{\textnormal{せんそう}}{\text{戦争}}}$ を ${\overset{\textnormal{しゅう}}{\text{終}}}$ わらせることができるのでしょうか。 \hfill\break
Could science bring an end to war? }

\par{23. ${\overset{\textnormal{あい}}{\text{愛}}}$ を ${\overset{\textnormal{お}}{\text{終}}}$ わらせない ${\overset{\textnormal{ほうほう}}{\text{方法}}}$ はひとつしかない。 \hfill\break
There is but one way of not bringing an end to love. }

\par{24. この ${\overset{\textnormal{しごと}}{\text{仕事}}}$ は ${\overset{\textnormal{あした}}{\text{明日}}}$ の ${\overset{\textnormal{あさ}}{\text{朝}}}$ までに ${\overset{\textnormal{お}}{\text{終}}}$ えないと、まずいですよ。 \hfill\break
If we don\textquotesingle t finish this work by tomorrow morning, things won\textquotesingle t be good. }

\par{\emph{ }終えさせる means “to make…end.” The haphazard nature of ending something before it is complete is often implied. It may also simply give the “let” nuance of the causative. These two different nuances are contrasted in the two examples below. }

\par{25. ${\overset{\textnormal{しず}}{\text{静}}}$ かに ${\overset{\textnormal{きんしんしゃ}}{\text{近親者}}}$ の ${\overset{\textnormal{し}}{\text{死}}}$ を ${\overset{\textnormal{う}}{\text{受}}}$ け ${\overset{\textnormal{い}}{\text{入}}}$ れ、 ${\overset{\textnormal{しぜん}}{\text{自然}}}$ に(かつ) ${\overset{\textnormal{おだ}}{\text{穏}}}$ やかに ${\overset{\textnormal{しょう}}{\text{生}}}$ を ${\overset{\textnormal{お}}{\text{終}}}$ えさせることが、 ${\overset{\textnormal{きんしんしゃ}}{\text{近親者}}}$ としての ${\overset{\textnormal{さいご}}{\text{最後}}}$ の ${\overset{\textnormal{つと}}{\text{務}}}$ めです。 \hfill\break
The final duty of a close relative is to calmly accept a close relative\textquotesingle s death and to naturally and gently let the individual end life. }

\par{26. ただ ${\overset{\textnormal{けいき}}{\text{刑期}}}$ を ${\overset{\textnormal{お}}{\text{終}}}$ えさせるだけなら、 ${\overset{\textnormal{なん}}{\text{何}}}$ の ${\overset{\textnormal{はんせい}}{\text{反省}}}$ もなくまた ${\overset{\textnormal{はんざい}}{\text{犯罪}}}$ を ${\overset{\textnormal{おか}}{\text{犯}}}$ すと ${\overset{\textnormal{おも}}{\text{思}}}$ う。 \hfill\break
I think that if you simply just end (the person's) sentence, (that person) will commit crime again without any self-reflection. }
      
\section{変わる vs 変える}
 
\par{ The verb for to change, as you\textquotesingle ve known it to be, is かわる and かえる for the intransitive and transitive sense respectively. Although this is true, をかわる is also possible. Before we investigate that aspect further, however, it\textquotesingle s important to understand the various nuances of かわる and かえる, especially because they have unique spellings. }

\par{・変わる・変える: For general change; the character 変 is used for showing changes in state\slash appearance\slash etc. As “state” is the broadest means of describing situation, it is what you should use if you are not sure if any of the following spellings are more appropriate. \hfill\break
・代わる・代える: For indicating substitution; the character 代 is used to show that some role\slash situation\slash standing\slash position is switched\slash being substituted for something else. \hfill\break
・替わる・替える: For indicating switching to something new; the character 替 also indicates switching, but the switching is permanent and it\textquotesingle s to something deemed new(er). \hfill\break
・換わる・換える: For exchanging; the character 換 is used to show exchanging one thing for another of the same value. It is especially common in reference to currency exchange. }

\par{27. ${\overset{\textnormal{とちゅう}}{\text{途中}}}$ で ${\overset{\textnormal{よてい}}{\text{予定}}}$ が ${\overset{\textnormal{か}}{\text{変}}}$ わりましたので ${\overset{\textnormal{のうき}}{\text{納期}}}$ を ${\overset{\textnormal{はや}}{\text{早}}}$ めてください。 \hfill\break
Plans have changed midway, so please hasten the delivery day. }

\par{28. ${\overset{\textnormal{いま}}{\text{今}}}$ までの ${\overset{\textnormal{ほうほう}}{\text{方法}}}$ を ${\overset{\textnormal{か}}{\text{変}}}$ えました。 \hfill\break
We\textquotesingle ve changed our method that we\textquotesingle ve used up until now. }

\par{29. ${\overset{\textnormal{げんち}}{\text{現地}}}$ で ${\overset{\textnormal{にほんえん}}{\text{日本円}}}$ の ${\overset{\textnormal{げんきん}}{\text{現金}}}$ をユーロに\{換・替\}えるのは、 ${\overset{\textnormal{かわせ}}{\text{為替}}}$ レートが ${\overset{\textnormal{わる}}{\text{悪}}}$ いので、 ${\overset{\textnormal{でき}}{\text{出来}}}$ るだけ ${\overset{\textnormal{さ}}{\text{避}}}$ けたほうが ${\overset{\textnormal{よ}}{\text{良}}}$ いです。 \hfill\break
When you exchange Japanese yen for euro locally, the exchange rate is bad, and so it\textquotesingle d be better that you avoid that as much as possible. }

\par{30. ${\overset{\textnormal{やまのてせん}}{\text{山手線}}}$ に ${\overset{\textnormal{の}}{\text{乗}}}$ り ${\overset{\textnormal{か}}{\text{換}}}$ えました。 \hfill\break
I transferred trains to the Yamanote Line. }

\par{31. ${\overset{\textnormal{しょうひんけん}}{\text{商品券}}}$ をお ${\overset{\textnormal{かね}}{\text{金}}}$ に ${\overset{\textnormal{か}}{\text{換}}}$ える。 \hfill\break
To exchange a gift certificate into money. }

\par{32. あの ${\overset{\textnormal{しにせ}}{\text{老舗}}}$ は ${\overset{\textnormal{だいが}}{\text{代替}}}$ わりして ${\overset{\textnormal{むすこ}}{\text{息子}}}$ さんが ${\overset{\textnormal{あと}}{\text{後}}}$ を ${\overset{\textnormal{つ}}{\text{継}}}$ いだらしいです。 \hfill\break
It appears that that old shop has been taken over by the son. }

\par{33. ${\overset{\textnormal{し}}{\text{4}}}$ ${\overset{\textnormal{がつ}}{\text{月}}}$ になって ${\overset{\textnormal{ねんど}}{\text{年度}}}$ が\{替・変\}わった。 \hfill\break
Now that it\textquotesingle s April, the fiscal year has changed. }

\par{34. ${\overset{\textnormal{じゅんばん}}{\text{順番}}}$ を\{変・替\}えました。 \hfill\break
I changed the ordering. }

\par{35. Tシャツを ${\overset{\textnormal{きが}}{\text{着替}}}$ えました。 \hfill\break
I changed T-shirts. }

\par{36. ${\overset{\textnormal{ま}}{\text{先}}}$ ずは、 ${\overset{\textnormal{しゃちょう}}{\text{社長}}}$ に ${\overset{\textnormal{か}}{\text{代}}}$ わってご ${\overset{\textnormal{あいさつもう}}{\text{挨拶申}}}$ し ${\overset{\textnormal{あ}}{\text{上}}}$ げます。 \hfill\break
First of all, I would like to say a few words on behalf of the (company) president. }

\par{37. ${\overset{\textnormal{じぶん}}{\text{自分}}}$ を ${\overset{\textnormal{ひとじち}}{\text{人質}}}$ の ${\overset{\textnormal{みが}}{\text{身代}}}$ わりとして ${\overset{\textnormal{さ}}{\text{差}}}$ し ${\overset{\textnormal{だ}}{\text{出}}}$ す。 \hfill\break
To present oneself as a sacrifice for hostage(s). }

\par{38. ${\overset{\textnormal{いま}}{\text{今}}}$ から ${\overset{\textnormal{せきゆ}}{\text{石油}}}$ に ${\overset{\textnormal{か}}{\text{代}}}$ わる ${\overset{\textnormal{しげん}}{\text{資源}}}$ に ${\overset{\textnormal{き}}{\text{切}}}$ り ${\overset{\textnormal{か}}{\text{替}}}$ えていくことが ${\overset{\textnormal{ひつよう}}{\text{必要}}}$ です。 \hfill\break
It is necessary that we switch to a resource in place of oil starting now. }

\par{39. これを以って ${\overset{\textnormal{あいさつ}}{\text{挨拶}}}$ \{に・と\} ${\overset{\textnormal{か}}{\text{代}}}$ えさせていただきます。 \hfill\break
With this, I\textquotesingle d like to give my salutations. }

\par{\textbf{Sentence Note }: 挨拶\{に・と\}代えさせていただきます literally implies that you are using whatever you said prior as the thing it attaches to. It is yet another way to add a layer of politeness in addresses and the like. }

\par{ The use of を変わる instead of を変える is meant to lessen the direct volition of the agent in the change\slash switch. }

\par{40. 早く職を変わりたい。 \hfill\break
I want to change jobs soon. }

\par{41. ○○さんに ${\overset{\textnormal{でんわ}}{\text{電話}}}$ を ${\overset{\textnormal{か}}{\text{代}}}$ わっていただけますか。 \hfill\break
May I speak to Mr. \#\#? }

\par{42. ${\overset{\textnormal{けんたろう}}{\text{健太郎}}}$ が ${\overset{\textnormal{ざせき}}{\text{座席}}}$ を ${\overset{\textnormal{か}}{\text{代}}}$ わった。 \hfill\break
Kentaro switched seats. }

\par{\textbf{Spelling Note }: As you can see, when かわる indicates a meaning closer to “to switch,” then the spelling 代わる becomes more appropriate. In Ex. 42, Kentaro simply switched seats with someone, likely with someone sitting directly next to him. }

\par{43. ${\overset{\textnormal{けんたろう}}{\text{健太郎}}}$ が ${\overset{\textnormal{ざせき}}{\text{座席}}}$ を ${\overset{\textnormal{か}}{\text{変}}}$ えた。 \hfill\break
Kentaro changed (the) seats. }

\par{\textbf{Nuance Note }: This sentence has two possible interpretations. Kentaro may have changed seating arrangements that may or may not include his own seat, or he simply switched seats but solely following his own volition to do so. For instance, this would be appropriate if he were sitting next to someone talking loudly on his cellphone and got up to sit in a seat far away from that individual. }

\par{44. ${\overset{\textnormal{せんせい}}{\text{先生}}}$ が ${\overset{\textnormal{せいと}}{\text{生徒}}}$ たちの ${\overset{\textnormal{ざせき}}{\text{座席}}}$ を\{ ${\overset{\textnormal{か}}{\text{変}}}$ わった X・ ${\overset{\textnormal{か}}{\text{変}}}$ えた\}。 \hfill\break
The teacher changed the students\textquotesingle  seats. }

\par{45.\{ ${\overset{\textnormal{てんこう}}{\text{転校}}}$ しました・ ${\overset{\textnormal{がっこう}}{\text{学校}}}$ を ${\overset{\textnormal{か}}{\text{変}}}$ わりました\}。 \hfill\break
I changed schools. }

\par{\textbf{Phrase Note }: 転校する is more common. }

\par{46. ${\overset{\textnormal{いま}}{\text{今}}}$ の ${\overset{\textnormal{しごと}}{\text{仕事}}}$ がつまらないんなら、 ${\overset{\textnormal{しごと}}{\text{仕事}}}$ を ${\overset{\textnormal{か}}{\text{変}}}$ えたらどうでしょうか。 \hfill\break
If your current job is so boring, how about changing jobs? }

\par{47. アメリカでは、 ${\overset{\textnormal{こうこうそつぎょう}}{\text{高校卒業}}}$ から ${\overset{\textnormal{よんじゅう}}{\text{40}}}$ ${\overset{\textnormal{さい}}{\text{歳}}}$ くらいまでの ${\overset{\textnormal{あいだ}}{\text{間}}}$ に ${\overset{\textnormal{ご}}{\text{5}}}$ ${\overset{\textnormal{かい}}{\text{回}}}$ くらい ${\overset{\textnormal{しごと}}{\text{仕事}}}$ を ${\overset{\textnormal{か}}{\text{変}}}$ わることも ${\overset{\textnormal{れいがい}}{\text{例外}}}$ ではありません。 \hfill\break
In America, it is not exceptional even for someone to switch jobs five or so times from the time one graduates high school to the time one turns 40. }

\par{48. ${\overset{\textnormal{し}}{\text{4}}}$ ${\overset{\textnormal{がつ}}{\text{月}}}$ から ${\overset{\textnormal{しごと}}{\text{仕事}}}$ が ${\overset{\textnormal{か}}{\text{変}}}$ わって ${\overset{\textnormal{ひろしま}}{\text{広島}}}$ から ${\overset{\textnormal{なごや}}{\text{名古屋}}}$ に ${\overset{\textnormal{てんきょ}}{\text{転居}}}$ しています。 \hfill\break
My job changed starting April, and so I\textquotesingle ve changed residence from Hiroshima to Nagoya. }

\par{49. ${\overset{\textnormal{しゅっさんご}}{\text{出産後}}}$ 、 ${\overset{\textnormal{しごと}}{\text{仕事}}}$ が ${\overset{\textnormal{か}}{\text{変}}}$ わりました。 \hfill\break
My job changed after giving birth. }

\par{\textbf{Nuance Note }: In Ex. 49, it\textquotesingle s not necessarily the case that the speaker switched jobs. It could just be that her duties at her existing job changed due to having become a mother. }

\par{50. ${\overset{\textnormal{がっき}}{\text{楽器}}}$ を ${\overset{\textnormal{か}}{\text{変}}}$ わりたかったら、 ${\overset{\textnormal{か}}{\text{変}}}$ わればいいんです。 \hfill\break
If you want to switch instruments, switch. }
    