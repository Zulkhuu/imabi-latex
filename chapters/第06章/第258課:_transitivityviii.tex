    
\chapter{Intransitive \& Transitive}

\begin{center}
\begin{Large}
第258課: Intransitive \& Transitive: Part 7 
\end{Large}
\end{center}
 
\par{ In this seventh installment about verbs that are both intransitive and transitive, we\textquotesingle ll look at another handful of verbs that deserve special attention. In this lesson, all verbs discussed are typically taught as being primarily used with に, but their usages with を mustn\textquotesingle t be overlooked. As was the case last lesson, these usages with を typically fall under the three categories below: }

\par{1. Either the intransitive or the transitive usage is relatively new in the language. Meaning, some speakers will think it\textquotesingle s wrong to use it a certain way but many speakers still do. \hfill\break
2. The use of the verb in a transitive sense is done so to implicitly show a connection between an agent and an action. \hfill\break
3. The use of the verb in a transitive sense is done so to emphasize the agent\textquotesingle s volition in said action. }
      
\section{負ける}
 
\par{ The verb 負ける is generally an intransitive verb meaning “to lose (to).” It can even mean "to break out in a rash (due to shaving, etc.). However, it has one transitive meaning that is equivalent to 安くする. In other words, in addition to meaning to succumb to defeat, it can also be mean reducing the price of something. }

\par{1. ${\overset{\textnormal{しあい}}{\text{試合}}}$ に ${\overset{\textnormal{ま}}{\text{負}}}$ けました。 \hfill\break
I lost (in) the match. }

\par{2. ${\overset{\textnormal{はだ}}{\text{肌}}}$ が ${\overset{\textnormal{よわ}}{\text{弱}}}$ いので、どうしてもかみそりに ${\overset{\textnormal{ま}}{\text{負}}}$ けて ${\overset{\textnormal{しゅっけつ}}{\text{出血}}}$ してしまいます。 \hfill\break
My skin is tender, and so no matter what, I break out from razors and end up bleeding. \hfill\break
 \hfill\break
\textbf{Spelling Note }: かみそり may also be spelled as 剃刀. }

\par{3. ${\overset{\textnormal{ねだん}}{\text{値段}}}$ を ${\overset{\textnormal{ま}}{\text{負}}}$ けて ${\overset{\textnormal{う}}{\text{売}}}$ ることの ${\overset{\textnormal{ふごうり}}{\text{不合理}}}$ に、 ${\overset{\textnormal{まった}}{\text{全}}}$ く ${\overset{\textnormal{がまん}}{\text{我慢}}}$ できません! \hfill\break
I absolutely can\textquotesingle t stand the irrationality behind lowering the price and selling it! }

\par{4. ${\overset{\textnormal{にほん}}{\text{日本}}}$ やアメリカなどでは、 ${\overset{\textnormal{ねだん}}{\text{値段}}}$ を ${\overset{\textnormal{ま}}{\text{負}}}$ けてもらうということが ${\overset{\textnormal{すく}}{\text{少}}}$ ない。 \hfill\break
There are few instances in Japan and America where you get the price down on something. }

\par{5. ${\overset{\textnormal{ちゅうかいてすうりょう}}{\text{仲介手数料}}}$ を ${\overset{\textnormal{ま}}{\text{負}}}$ けてくれない ${\overset{\textnormal{ふどうさんがいしゃ}}{\text{不動産会社}}}$ が ${\overset{\textnormal{おお}}{\text{多}}}$ い。 \hfill\break
There are many real estate companies that won\textquotesingle t lower brokerage fees. }
      
\section{当たる}
 
\par{\emph{ }当たる creates an intransitive-transitive verb pair with 当てる. Unfortunately, things get complicated due to the fact that they both have several nuances and because 当たる also happens to have transitive uses that cannot be replaced by 当てる. }

\par{当たる (intrs.): To be hit; to be equivalent to; to win (a lottery); to be stricken (by heat, food poisoning, etc.); to hit well (baseball); to feel a bite (in fishing); to be bruised (fruit); to be called upon (by a teacher); to be assigned to; to be right on the money; to lash out at. }

\par{6. ${\overset{\textnormal{いし}}{\text{石}}}$ がガラスに ${\overset{\textnormal{あ}}{\text{当}}}$ たっても ${\overset{\textnormal{わ}}{\text{割}}}$ れ ${\overset{\textnormal{づら}}{\text{辛}}}$ く、もし ${\overset{\textnormal{わ}}{\text{割}}}$ れたとしても ${\overset{\textnormal{いし}}{\text{石}}}$ がガラスを ${\overset{\textnormal{かんつう}}{\text{貫通}}}$ しない。 \hfill\break
Even if rocks hit the glass, it\textquotesingle s hard to crack. Even if it were to crack, the rock won\textquotesingle t penetrate the glass. }

\par{7. ${\overset{\textnormal{ちょうきかん}}{\text{長期間}}}$ 日が ${\overset{\textnormal{あ}}{\text{当}}}$ たらないと、うまく ${\overset{\textnormal{そだ}}{\text{育}}}$ ちません。 \hfill\break
If it isn't exposed to the sun for a long period of time, it won\textquotesingle t grow well. }

\par{8. スーツの ${\overset{\textnormal{まえ}}{\text{前}}}$ ボタンを ${\overset{\textnormal{し}}{\text{締}}}$ めないのは、 ${\overset{\textnormal{しつれい}}{\text{失礼}}}$ に ${\overset{\textnormal{あ}}{\text{当}}}$ たりますか。 \hfill\break
Would it be rude not to fasten the front button of a suit? }

\par{9. ${\overset{\textnormal{おや}}{\text{親}}}$ を ${\overset{\textnormal{わる}}{\text{悪}}}$ く ${\overset{\textnormal{おも}}{\text{思}}}$ うだけでも ${\overset{\textnormal{ばつ}}{\text{罰}}}$ が ${\overset{\textnormal{あ}}{\text{当}}}$ たります。 \hfill\break
Even by merely thinking bad about your parents, you\textquotesingle ll incur punishment. }

\par{10. ${\overset{\textnormal{あめ}}{\text{雨}}}$ の ${\overset{\textnormal{てんきよほう}}{\text{天気予報}}}$ が ${\overset{\textnormal{あ}}{\text{当}}}$ たってよかった。 \hfill\break
I\textquotesingle m glad that the rain forecast was right on. }

\par{11. ${\overset{\textnormal{つうじょう}}{\text{通常}}}$ 、 ${\overset{\textnormal{ようじん}}{\text{要人}}}$ の ${\overset{\textnormal{しんぺんけいご}}{\text{身辺警護}}}$ に ${\overset{\textnormal{あ}}{\text{当}}}$ たっている。 \hfill\break
Normally, (the individual) is assigned to the personal protection of important persons. }

\par{12. フグに\{当たったら・中ったら\}、首から ${\overset{\textnormal{した}}{\text{下}}}$ を ${\overset{\textnormal{つち}}{\text{土}}}$ に ${\overset{\textnormal{う}}{\text{埋}}}$ めろ。 \emph{\hfill\break
 }If you get poisoned by a puffer-fish, bury yourself from the neck down in dirt. }

\par{\textbf{Spelling Notes }: フグ can alternatively be spelled as 河豚. When used to mean “to be stricken,” 当たる \emph{ }can seldom be seen spelled as 中る. }

\par{13. それは\{あたってる・ ${\overset{\textnormal{いた}}{\text{傷}}}$ んでる\}だけでしょう。 \hfill\break
That\textquotesingle s just bruised, you know. }

\par{\textbf{Nuance Note }:  A lot of speakers do not understand what is meant by あたる when used in the context of indicating that food produce is bruised. This is because although it is in dictionaries, it is dialectical in nature. Some people will be confused because of how に当たる can indicate food poisoning. After all, one way of saying “food poisoning” itself is 食中り. When using 当たる to indicate bruising, there is a nuance of the bruising being caused by the produce hitting each other, likely during transport. To avoid confusion, using the verb 傷む is your best bet. }

\par{当たる (trans.): To check (by comparison); to probe into; to shave. }

\par{ This usage typically takes に as well; however, when used to mean “to check (establishments),”を is typically used. }

\par{14. イライラして ${\overset{\textnormal{だんな}}{\text{旦那}}}$ や ${\overset{\textnormal{むすこ}}{\text{息子}}}$ や ${\overset{\textnormal{いぬ}}{\text{犬}}}$ にまで ${\overset{\textnormal{あ}}{\text{当}}}$ たってしまいました。 \hfill\break
I got irritated and lashed out at my husband, son, and even my dog. }

\par{15. ${\overset{\textnormal{げんぽん}}{\text{原本}}}$ に ${\overset{\textnormal{あ}}{\text{当}}}$ たって ${\overset{\textnormal{こうせい}}{\text{校正}}}$ してほしいです。 \hfill\break
I\textquotesingle d like you to proofread by checking with the original script. }

\par{16. ${\overset{\textnormal{しょうさい}}{\text{詳細}}}$ は ${\overset{\textnormal{ほんにん}}{\text{本人}}}$ に ${\overset{\textnormal{あ}}{\text{当}}}$ たってください。 \hfill\break
For details, please see the person himself. }

\par{17. 小さな個人経営のお店を当たっています。 \hfill\break
I\textquotesingle m checking small businesses. }

\par{18. ${\overset{\textnormal{ほか}}{\text{他}}}$ を ${\overset{\textnormal{あ}}{\text{当}}}$ たってください。 \hfill\break
Please check (another store). }

\par{19. ${\overset{\textnormal{かお}}{\text{顔}}}$ を ${\overset{\textnormal{あ}}{\text{当}}}$ たりますか。 \hfill\break
Shall I shave your face? }

\par{\textbf{Word Note }: In certain lines of industry, instances of 剃る (to shave), 擂る (to grind), and other things resembling the two in pronunciation such as 鯣・スルメ (dried squid\slash cuttlefish) have そる・する replaced by 当たる.  As for スルメ, it turns into \emph{ }当たりめ. }

\par{当てる (trans.): To hit; expose; to put (on\slash against); to allot; to make a hit (in a lottery); to guess (an answer); to call on; to sit (on a cushion); to address. }

\par{20. ${\overset{\textnormal{ざぶとん}}{\text{座布団}}}$ を ${\overset{\textnormal{あ}}{\text{当}}}$ ててください。 \hfill\break
Please sit on the floor cushion. }

\par{21. ${\overset{\textnormal{わご}}{\text{和語}}}$ に ${\overset{\textnormal{かんじ}}{\text{漢字}}}$ を ${\overset{\textnormal{あ}}{\text{当}}}$ てる。 \hfill\break
To attach Kanji to native Japanese words. }

\par{22. ${\overset{\textnormal{かんようしょくぶつ}}{\text{観葉植物}}}$ を ${\overset{\textnormal{ひるま}}{\text{昼間}}}$ に ${\overset{\textnormal{おくがい}}{\text{屋外}}}$ で ${\overset{\textnormal{ひ}}{\text{日}}}$ に ${\overset{\textnormal{あ}}{\text{当}}}$ てていたら、 ${\overset{\textnormal{か}}{\text{枯}}}$ れてしまった。 \hfill\break
When I had left my decorative plant exposed to the sun outdoors in the afternoon, it died. }

\par{23. ポイントが ${\overset{\textnormal{た}}{\text{貯}}}$ まると、このポイントを ${\overset{\textnormal{りよう}}{\text{利用}}}$ して ${\overset{\textnormal{こうくうひ}}{\text{航空費}}}$ や ${\overset{\textnormal{りょひ}}{\text{旅費}}}$ に ${\overset{\textnormal{みつる}}{\text{充}}}$ てることができるようになったりします。 \hfill\break
When points build up, by using them you become able to do things like allot them to air fares and travel expenses. }

\par{\textbf{Spelling Note }: When used to mean “to allot,” あてる is spelled as 充てる. }

\par{24. ${\overset{\textnormal{こんど}}{\text{今度}}}$ は ${\overset{\textnormal{はは}}{\text{母}}}$ に ${\overset{\textnormal{あ}}{\text{宛}}}$ てて ${\overset{\textnormal{てがみ}}{\text{手紙}}}$ を ${\overset{\textnormal{か}}{\text{書}}}$ きました。 \hfill\break
I wrote a letter addressed to my mother. }

\par{\textbf{Spelling Note }: When used to mean “to address” as in a document of some sort, あてる is spelled as 宛てる. }

\par{25. ${\overset{\textnormal{じゅぎょうちゅう}}{\text{授業中}}}$ 、 ${\overset{\textnormal{あ}}{\text{当}}}$ たりたくないと ${\overset{\textnormal{おも}}{\text{思}}}$ って、 ${\overset{\textnormal{せんせい}}{\text{先生}}}$ と ${\overset{\textnormal{め}}{\text{目}}}$ を ${\overset{\textnormal{あ}}{\text{合}}}$ わせないようにしていたら、 ${\overset{\textnormal{あ}}{\text{当}}}$ てられてしまった。 \hfill\break
During class while I was trying not to lock eyes with the teacher so that I wouldn\textquotesingle t get called on, I got called upon. }

\par{26. うまく ${\overset{\textnormal{あ}}{\text{当}}}$ てた。 \hfill\break
I guessed right. }
      
\section{喜ぶ}
 
\par{ The verb 喜ぶ can incidentally be used with either に or を. The nuance is slightly different and so it is often the case that you can switch up the particles in most instances without making a grammatical error. に喜ぶ means “to be delighted\slash pleased with” and を喜ぶ means “to rejoice at\slash congratulate.” }

\par{27. ${\overset{\textnormal{あいて}}{\text{相手}}}$ の ${\overset{\textnormal{しあわ}}{\text{幸}}}$ せを ${\overset{\textnormal{いっしょ}}{\text{一緒}}}$ に ${\overset{\textnormal{よろこ}}{\text{喜}}}$ ぶと、 ${\overset{\textnormal{こうかん}}{\text{好感}}}$ が ${\overset{\textnormal{も}}{\text{持}}}$ てる。 \hfill\break
When you celebrate another person\textquotesingle s happiness together, you give a positive vibe. }

\par{28. ${\overset{\textnormal{わたしたち}}{\text{私達}}}$ は ${\overset{\textnormal{なぜたにん}}{\text{何故他人}}}$ の ${\overset{\textnormal{ふこう}}{\text{不幸}}}$ を ${\overset{\textnormal{よろこ}}{\text{喜}}}$ ぶのか。 \hfill\break
Why is it that we rejoice at other people\textquotesingle s misfortune? }

\par{29. ${\overset{\textnormal{しんさい}}{\text{震災}}}$ を ${\overset{\textnormal{よろこ}}{\text{喜}}}$ ぶようなブログなどがネット ${\overset{\textnormal{じょう}}{\text{上}}}$ に ${\overset{\textnormal{なが}}{\text{流}}}$ れているのも ${\overset{\textnormal{じじつ}}{\text{事実}}}$ である。 \hfill\break
It is also true that there are blogs and such circulating on the internet that seem to rejoice at natural disasters. }

\par{30. ${\overset{\textnormal{おい}}{\text{美味}}}$ しいデザートに ${\overset{\textnormal{よろこ}}{\text{喜}}}$ ばない ${\overset{\textnormal{じょせい}}{\text{女性}}}$ はいません。 \hfill\break
There isn\textquotesingle t a woman who isn\textquotesingle t pleased with a delicious desert. }

\par{31. ${\overset{\textnormal{おお}}{\text{大}}}$ きな ${\overset{\textnormal{せいか}}{\text{成果}}}$ に ${\overset{\textnormal{よろこ}}{\text{喜}}}$ ぶこともあれば、うまくいかず ${\overset{\textnormal{らくたん}}{\text{落胆}}}$ し、 ${\overset{\textnormal{かっとう}}{\text{葛藤}}}$ を ${\overset{\textnormal{く}}{\text{繰}}}$ り ${\overset{\textnormal{かえ}}{\text{返}}}$ した ${\overset{\textnormal{けいけん}}{\text{経験}}}$ も ${\overset{\textnormal{おお}}{\text{多}}}$ くあるでしょう。 \hfill\break
If you're ever delighted at great results, then you will surely also have experienced things not going well, getting discouraged, and then having repeated those troubles. }

\par{32. ${\overset{\textnormal{せんしゅ}}{\text{選手}}}$ たちは ${\overset{\textnormal{ひさびさ}}{\text{久々}}}$ の ${\overset{\textnormal{しょうり}}{\text{勝利}}}$ に ${\overset{\textnormal{よろこ}}{\text{喜}}}$ んでいる。 \hfill\break
The athletes are delighted about their long overdue victory. }
      
\section{怒る}
 
\par{ When used with the particle に as an intransitive verb, 怒る means “to get mad at…” As a transitive verb with the particle を, it means “to scold.” There is also the verb 叱る which also means “to scold,” but 叱る is thought to be constructive whereas 怒る is usually not constructive. }

\par{33. ${\overset{\textnormal{こども}}{\text{子供}}}$ に ${\overset{\textnormal{おこ}}{\text{怒}}}$ って、 ${\overset{\textnormal{どな}}{\text{怒鳴}}}$ ってしまいました。 \hfill\break
I got mad at my kid and accidentally shouted at him\slash her. }

\par{34. ${\overset{\textnormal{ひこうき}}{\text{飛行機}}}$ の ${\overset{\textnormal{おく}}{\text{遅}}}$ れに ${\overset{\textnormal{おこ}}{\text{怒}}}$ った ${\overset{\textnormal{じょせいきゃく}}{\text{女性客}}}$ が ${\overset{\textnormal{きゃくしつじょうむいん}}{\text{客室乗務員}}}$ に ${\overset{\textnormal{ぼうこう}}{\text{暴行}}}$ を ${\overset{\textnormal{くわ}}{\text{加}}}$ える ${\overset{\textnormal{ようす}}{\text{様子}}}$ を ${\overset{\textnormal{さつえい}}{\text{撮影}}}$ した ${\overset{\textnormal{どうが}}{\text{動画}}}$ が ${\overset{\textnormal{えんじょう}}{\text{炎上}}}$ している。 \hfill\break
A video is receiving a flood of criticisms which captures a female passenger who got angry at the plane\textquotesingle s delay assaulting a flight attendant. }

\par{35. ${\overset{\textnormal{じっさい}}{\text{実際}}}$ の ${\overset{\textnormal{じぶん}}{\text{自分}}}$ に ${\overset{\textnormal{おこ}}{\text{怒}}}$ っているのに、 ${\overset{\textnormal{つま}}{\text{妻}}}$ の ${\overset{\textnormal{ろうひくせ}}{\text{浪費癖}}}$ に ${\overset{\textnormal{おこ}}{\text{怒}}}$ っていると ${\overset{\textnormal{しん}}{\text{信}}}$ じ ${\overset{\textnormal{こ}}{\text{込}}}$ んでいる ${\overset{\textnormal{おっと}}{\text{夫}}}$ は ${\overset{\textnormal{おお}}{\text{多}}}$ い。 \hfill\break
There are many husbands who are convinced that they are mad at their wives\textquotesingle  reckless spending habits even though they are mad at their actual selves. }

\par{36. ${\overset{\textnormal{ぶか}}{\text{部下}}}$ を ${\overset{\textnormal{おこ}}{\text{怒}}}$ った ${\overset{\textnormal{かんじょう}}{\text{感情}}}$ はどれくらい ${\overset{\textnormal{けいぞく}}{\text{継続}}}$ するのか。 \hfill\break
How long do your feelings continue from having scolded an underling? }

\par{37. ${\overset{\textnormal{せいと}}{\text{生徒}}}$ を ${\overset{\textnormal{おこ}}{\text{怒}}}$ ったあと、あなたはどうしますか。 \hfill\break
What do you do after scolding a student? }

\par{38. ${\overset{\textnormal{いぬ}}{\text{犬}}}$ を\{ ${\overset{\textnormal{しか}}{\text{叱}}}$ る・ ${\overset{\textnormal{おこ}}{\text{怒}}}$ る\}ときは ${\overset{\textnormal{はな}}{\text{鼻}}}$ を ${\overset{\textnormal{たた}}{\text{叩}}}$ けばいいですか。 \hfill\break
When scolding a dog, should you hit its nose? }

\par{39. ${\overset{\textnormal{じぶん}}{\text{自分}}}$ のために ${\overset{\textnormal{おこ}}{\text{怒}}}$ る、 ${\overset{\textnormal{あいて}}{\text{相手}}}$ のために ${\overset{\textnormal{しか}}{\text{叱}}}$ る。 \hfill\break
You tell someone off for oneself; you reprimand someone for that person\textquotesingle s sake. }

\par{40. ${\overset{\textnormal{こども}}{\text{子供}}}$ を ${\overset{\textnormal{かんじょうてき}}{\text{感情的}}}$ に ${\overset{\textnormal{おこ}}{\text{怒}}}$ ってしまった。 \hfill\break
I accidentally emotionally scolded my child(ren). }
    