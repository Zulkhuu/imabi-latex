    
\chapter{The Particle に III}

\begin{center}
\begin{Large}
第300課: The Particle に III: The Conjunctive に  
\end{Large}
\end{center}
 
\par{ The conjunctive particle に is usually ignored in Japanese studies, but it is not that difficult. In this lesson, we will learn all about how exactly it is used. Some of its usages are actually really important. }
      
\section{Basic Understanding}
 
\par{ In Classical Japanese, に was used along with を and が almost interchangeably to mean “but”. Or, it would be after the 連体形 of verbs just as other case particles could. So, just as のに has two interpretations in Modern Japanese, the conjunctive に had two potential meanings. }

\par{1. 十月 ${\overset{\textnormal{みそか}}{\text{晦日}}}$ なるに、 ${\overset{\textnormal{もみじ}}{\text{紅葉}}}$ 散らで ${\overset{\textnormal{さか}}{\text{盛}}}$ りなり。 \hfill\break
Though it\textquotesingle s already the end of the tenth month (of the lunar calendar), colored leaves have not scattered; they're at peak. \hfill\break
From the ${\overset{\textnormal{さらしな}}{\text{更級}}}$ 日記. }

\par{ The first usage we will look at is how this に is used to show a prelude to an utterance. The most common example of this is the phrase ${\overset{\textnormal{よう}}{\text{要}}}$ するに, which essentially means “long story short” or “in brief”. The next most common example is 思うに, which means "presumably\slash upon thought". The similar 考えるに meaning "in considering" is not used as much. You can also say ${\overset{\textnormal{すいそく}}{\text{推測}}}$ するに (in presuming) and 想像するに (in imagining). These phrases are often used in formal situations as well as in arguments. Rarer examples show up in writing. }

\par{2. 思うに、 ${\overset{\textnormal{けんか}}{\text{喧嘩}}}$ は僕のせいだった。 \hfill\break
Upon reflection, the fight was my fault. }

\par{3. 思うに、これがAさんの ${\overset{\textnormal{ほんしょう}}{\text{本性}}}$ なのだ。 \hfill\break
Presumably, this is A-san\textquotesingle s nature. }

\par{4. ${\overset{\textnormal{ひび}}{\text{響}}}$ きから想像するに、バタフリーというポケモンは ${\overset{\textnormal{ちょう}}{\text{蝶}}}$ のような姿をしているのだろう。 \hfill\break
Imagining from its sound, the pokemon named Butterfree probably has the appearance of a butterfly. }

\par{5. 考えるには、ポケモンと一緒に戦って成長することが一番大事である。 \hfill\break
In consideration, battling and growing with one\textquotesingle s pokemon is most important. }

\par{6. この記述から想像するに、ホエルオーはスタジアムよりも大きいね。 \hfill\break
Imagining it from this description, Wailord is bigger than a stadium, right? }

\par{7. 要するに、これがあなたの言いたいことでしょう。 \hfill\break
In short, this is what you want to say, no? }

\par{8. ${\overset{\textnormal{じょうきょう}}{\text{状況}}}$ から推測するに、その位置も安定している。 \hfill\break
Judging from the conditions, that position is also stable. }

\par{9. 思うに、ポケモンは ${\overset{\textnormal{ほんのうてき}}{\text{本能的}}}$ に進化するのだ! \hfill\break
Upon thought, pokemon instinctively evolve! }

\par{10. 要するに、このゲームはポケモンという生き物を ${\overset{\textnormal{つか}}{\text{捕}}}$ まえて、育てて、対戦するゲームなんだね。 \hfill\break
In short, this game is a game you catch creatures called Pokemon and then raise and battle with them. }

\par{11. 見た目から推測するに、ピカチュウは電気タイプのポケモンだろう。 \hfill\break
Judging from appearance, Pikachu is probably an electric type pokemon. }

\par{12. 調査結果から推測するに、今回の噴火が起こった原因はヒードランの復活のためでしょう。 \hfill\break
Judging from the investigation results, the case of this last eruption is probably due to the revival of Heatran. }

\par{13. 要するに、モンスターボールでポケモンが ${\overset{\textnormal{つか}}{\text{捕}}}$ まえられる。 \hfill\break
In short, you can catch pokemon with pokeballs. }

\par{14. 私が考えるに、この ${\overset{\textnormal{あん}}{\text{案}}}$ には問題がある。 \hfill\break
In my thought, there is a problem in this plan. }

\par{15. この大きさから(想像)するに、この ${\overset{\textnormal{かい}}{\text{貝}}}$ は100年は生きてきたのだろう。 \hfill\break
Judging from the size, this shellfish has probably lived at least 100 years. }

\par{16. 要するに、注意しろということだ。 \hfill\break
Long story short, this is about paying attention. }

\par{17. 要するに、答えはノーだった。 \hfill\break
In short, the answer was no. }

\par{18. 地理的に予想するに寒そうな場所だね。 \hfill\break
Judging geographically, it seems like a cold place. }

\par{19. 想像(する)に ${\overset{\textnormal{かた}}{\text{難}}}$ くない (書き言葉) \hfill\break
Not difficult to imagine }
      
\section{Verb + に + Negative Potential}
 
\par{ The pattern “verb in 連体形 + に + same verb in negative potential” is extremely common in Japanese and can essentially be used with any verb you can put in the potential form. What this pattern means is that although you want to do X, you can\textquotesingle t for whatever reason. Usually, the reason is circumstantial. }

\par{20. 売るに売れないものだ。 \hfill\break
It\textquotesingle s something I can\textquotesingle t sell even if I wanted to. }

\par{21. 今朝からずっと雨で、行くに行けない。 \hfill\break
It\textquotesingle s been raining ever since this morning, and so I just can\textquotesingle t go. }

\par{22. 歌うに歌えない歌詞 \hfill\break
Lyrics that one just can\textquotesingle t sing }

\par{23. 行くに行けない雰囲気になった。 \hfill\break
The atmosphere became one that I couldn't go. }

\par{24. 帰るに帰れないかもなあ。 \hfill\break
Looks like I\slash we can\textquotesingle t go home… }

\par{25. トイレに行くに行けない状況になった時は、精神的にかなりきつい。 \hfill\break
It\textquotesingle s psychologically very tense whenever you\textquotesingle re in a situation you can\textquotesingle t go to the restroom. }

\par{26. 泣くに泣けない ${\overset{\textnormal{ぼん}}{\text{凡}}}$ ミスだった。 \hfill\break
It was a trivial mistake too bitter to cry about. }

\par{27. ${\overset{\textnormal{こ}}{\text{越}}}$ すに越されぬ大きな川 \hfill\break
A river impossible to traverse }
      
\section{~ようにも}
 
\par{ ~ようにも show supposition and is equivalent to "しても".  It is extremely common and is often followed with something that relates to incapability. This can also be found in the negative as ~まいにも.  This, though, is not really used anymore. }

\par{28. 起きようにも起きられなかった。 \hfill\break
I couldn\textquotesingle t get up even if I tried. }

\par{29. プールの中で歩こうにも中々 ${\overset{\textnormal{ていこう}}{\text{抵抗}}}$ で歩けない。 \hfill\break
Even if you walk inside a pool, you can\textquotesingle t quite walk even with significant resistance. }

\par{30. 歌を歌おうにも ${\overset{\textnormal{こうおん}}{\text{高音}}}$ が出にくい。 \hfill\break
Even if I try to sing the song, it\textquotesingle s hard to reach the high tones. }

\par{31. 急ごうにも急げる体でもない。 \hfill\break
I\textquotesingle m not in the physical condition to be able to hurry. }

\par{32. ${\overset{\textnormal{りれきしょ}}{\text{履歴書}}}$ を書こうにも電話番号の記入は ${\overset{\textnormal{ひっす}}{\text{必須}}}$ だから、携帯が無いと履歴書を作れない。 \hfill\break
Even if I were to write a resume, because phone number is a required field, if I don\textquotesingle t have a cellphone, I can\textquotesingle t make a resume. }

\par{33. 今から就職活動しようにも、やりたい事もないし資格とかも持っていない。 \hfill\break
Even if I were to start looking for a job now, there\textquotesingle s nothing I want to do or really have the   qualifications for. }

\par{34. ${\overset{\textnormal{じゅく}}{\text{塾}}}$ で学んだ事を復習させようにも、ほとんど時間が確保出来ない。 \hfill\break
Even if I were to have (the students) review what they've learned in cram school, I could not secure hardly any time. }

\par{35. 家出をしようにもお金がないよ。 \hfill\break
Even if run away, you don't have any money. }
    