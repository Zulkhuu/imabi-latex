    
\chapter{自発動詞}

\begin{center}
\begin{Large}
第267課: 自発動詞 
\end{Large}
\end{center}
 
\par{ Spontaneity is a cool usage of the endings ~られる and ~れる. The grammar resembles the passive, but they are semantically different. There are also unique spontaneity verbs. We'll touch on how spontaneity relates to potential phrases.  }
      
\section{Spontaneity}
 
\par{ ~られる and ~れる may show spontaneous action or situation. A spontaneous action or situation is one that occurs without the intention of the subject which is often one\textquotesingle s self. There is no will involved. This ending is typically restricted to verbs of thought, cognition, and or feeling. If mentioned, the experiencer is marked by に. The object is marked by が just like with passivization. }

\par{${\overset{\textnormal{}}{\text{1. 人}}}$ の ${\overset{\textnormal{けはい}}{\text{気配}}}$ が ${\overset{\textnormal{}}{\text{感}}}$ じられた。 \hfill\break
The presence of (the) people (there) was felt. }

\par{2. ${\overset{\textnormal{ゆ}}{\text{行}}}$ く ${\overset{\textnormal{さき}}{\text{先}}}$ が ${\overset{\textnormal{}}{\text{案}}}$ じられた。 \hfill\break
The course was considered. }

\par{${\overset{\textnormal{}}{\text{3. 波}}}$ の ${\overset{\textnormal{}}{\text{音}}}$ に ${\overset{\textnormal{おどろ}}{\text{驚}}}$ かれた。 \hfill\break
I found myself surprised by the sound of the waves. }

\par{${\overset{\textnormal{}}{\text{4. 昔}}}$ のことがふと ${\overset{\textnormal{}}{\text{思}}}$ い ${\overset{\textnormal{}}{\text{出}}}$ された。 \hfill\break
I found myself remembering about the old days all of a sudden. }

\par{5. 意気込みが感じられた。 \hfill\break
I felt ardor. }

\par{6. 空気がおいしいと感じられました。 \hfill\break
The air came to be delicious. }

\par{7. 入院した祖母の容体が案じられる。 \hfill\break
To get concerned about the condition of one's hospitalized grandmother. }

\par{8. 秋の気配が感じられる。 \hfill\break
To find oneself sensing (the coming of) autumn. }

\par{9. 僕にはどうしてもそう思われる。 \hfill\break
No matter what, all I think of is that. }

\par{10. いかにも不思議に思われた。 \hfill\break
It seemed really mysterious. }

\par{ \textbf{聞こえる \& 見える }}

\par{ The verbs 聞こえる, 見える, as well as other similar looking verbs such as 燃える and 消える also express 自発. So, we'll call these verbs 自発動詞. ${\overset{\textnormal{}}{\text{聞}}}$ こえる and ${\overset{\textnormal{}}{\text{見}}}$ える predate the potential forms ${\overset{\textnormal{}}{\text{聞}}}$ ける and ${\overset{\textnormal{}}{\text{見}}}$ られる, and because they don't actually show potential, though potential phrases do ultimately derive from spontaneity phrases due to the fact that having the potential to do something is a characteristic of you that you cannot control, we did not discuss them in the potential lesson. }

\par{${\overset{\textnormal{}}{\text{}}}$ So, if 聞こえる and 見える do not mean 聞ける and 見られる respectively, we need to see how they differ. Essentially, they show inherent ability whereas ${\overset{\textnormal{}}{\text{聞}}}$ ける and ${\overset{\textnormal{}}{\text{見}}}$ られる show that you \emph{can (if you want), indicating one's intentions can be realized }. }

\par{ ${\overset{\textnormal{}}{\text{聞}}}$ こえる describes a naturally hearing sensation. It can be defined as "to be able to hear", "to sound", "to be audible", and even "to be famous". ${\overset{\textnormal{}}{\text{見}}}$ える may mean "to be able to see". It can also mean "to come into view\slash to appear". With ように, ${\overset{\textnormal{}}{\text{見}}}$ える may show what something looks like . It may also be a very respectful form of the verb 来る. }

\par{ \textbf{Examples }}

\par{${\overset{\textnormal{}}{\text{11. 雨}}}$ (の) ${\overset{\textnormal{}}{\text{音}}}$ の ${\overset{\textnormal{}}{\text{中}}}$ でも ${\overset{\textnormal{}}{\text{彼}}}$ らの ${\overset{\textnormal{}}{\text{声}}}$ が ${\overset{\textnormal{}}{\text{聞}}}$ こえた。 \hfill\break
I could hear their voices even in the middle of the rain. }

\par{${\overset{\textnormal{}}{\text{12. 彼}}}$ は ${\overset{\textnormal{こうふん}}{\text{興奮}}}$ しているように ${\overset{\textnormal{}}{\text{見}}}$ えます。 \hfill\break
He looks like he's excited. }

\par{${\overset{\textnormal{}}{\text{13. 魚}}}$ は ${\overset{\textnormal{}}{\text{音}}}$ が ${\overset{\textnormal{}}{\text{聞}}}$ こえると ${\overset{\textnormal{}}{\text{思}}}$ う? \hfill\break
Do you think that fish can hear? }

\par{${\overset{\textnormal{}}{\text{14. 私}}}$ は ${\overset{\textnormal{}}{\text{彼}}}$ の ${\overset{\textnormal{こうえん}}{\text{講演}}}$ が ${\overset{\textnormal{}}{\text{聞}}}$ こえません。 \hfill\break
I can't hear his lecture. }

\par{15. ${\overset{\textnormal{ねんれいそうおう}}{\text{年齢相応}}}$ に ${\overset{\textnormal{}}{\text{見}}}$ える。 \hfill\break
To look one's age. }

\par{16. ${\overset{\textnormal{えいがかん}}{\text{映画館}}}$ で ${\overset{\textnormal{}}{\text{今黒沢}}}$ の ${\overset{\textnormal{}}{\text{映画}}}$ が ${\overset{\textnormal{}}{\text{見}}}$ られます。 \hfill\break
You can now see Kurosawa's movies in the theater. }

\par{17. iPhoneで ${\overset{\textnormal{てんきよほう}}{\text{天気予報}}}$ が ${\overset{\textnormal{}}{\text{聞}}}$ けます。 \hfill\break
You can hear the weather forecast on your iPhone. }

\par{${\overset{\textnormal{}}{\text{18. 昨日}}}$ は ${\overset{\textnormal{}}{\text{香具山}}}$ が ${\overset{\textnormal{}}{\text{見}}}$ えたが、 ${\overset{\textnormal{}}{\text{今日}}}$ は見えない。 \hfill\break
I could see Mt. Kaguyama yesterday, but I can't see it today. }

\par{19. ${\overset{\textnormal{}}{\text{降}}}$ り ${\overset{\textnormal{}}{\text{出}}}$ しそうに ${\overset{\textnormal{}}{\text{見}}}$ える。 \hfill\break
It appears that it's going to rain. }

\par{20. ${\overset{\textnormal{となり}}{\text{隣}}}$ のテレビの ${\overset{\textnormal{}}{\text{音}}}$ が ${\overset{\textnormal{}}{\text{聞}}}$ こえる。 \hfill\break
I can hear the neighbor's television('s sound). }

\par{${\overset{\textnormal{}}{\text{21. 新}}}$ しい ${\overset{\textnormal{}}{\text{眼鏡}}}$ をかけるとずっとよく ${\overset{\textnormal{}}{\text{見}}}$ えます。 \hfill\break
I see so much better when I wear my new glasses. }

\par{${\overset{\textnormal{}}{\text{22. 彼}}}$ は ${\overset{\textnormal{みぎめ}}{\text{右目}}}$ があまりよく ${\overset{\textnormal{}}{\text{見}}}$ えない。 \hfill\break
He can't see very well with his right eye. \hfill\break
Literally: As for him, his right eye can't see very well. }

\par{\textbf{Historical Note }: Long ago ~ゆ was used just like ~られる and ~れる and remains part of many verbs like ${\overset{\textnormal{}}{\text{燃}}}$ える and ${\overset{\textnormal{}}{\text{消}}}$ える. So, these example verbs come from 燃ゆ and 消ゆ. You may even see these old forms purposely used in songs and poetry. Their roots still end in "y" and they're intransitive and spontaneous in nature. }

\begin{center}
 \textbf{Bridging Contexts between Spontaneity and Potential }
\end{center}

\par{Another verb to note is ${\overset{\textnormal{}}{\text{思}}}$ える. It can show spontaneous thought. It can also show the ability of thought which can be seen as coming from ${\overset{\textnormal{}}{\text{思}}}$ ゆ or ${\overset{\textnormal{}}{\text{思}}}$ われる. 思える and 思われる are almost identical, but the only true difference between the two is that the former is felt to be more objective and the latter is felt to be more subjective. }

\par{ Before going to examples, note the mentioning of "potential". There are clear instances in which verbs can be interpreted as both a 自発動詞 and a 可能動詞. However, the use of a speech modal such ~てしまう or ~てくる are very important to get both meanings. }

\par{\textbf{Examples } }

\par{23. A生徒がいつのまにか書けてしまった。 \hfill\break
Student A managed to be able to write it before I knew it. }

\par{24. びっくりするくらい泣けてきたわ。 (Feminine) \hfill\break
I was surprised at how many tears (I was able to) shed. }

\par{24. 自然に笑える映画を作ってみてください。 \hfill\break
Try making a movie one would naturally laugh to. }

\par{25. 良い作品とは思えません。 \hfill\break
I cannot\slash could not think of it as a good work. }

\par{26. 地名の由来はアイヌ語から来ていると思われます。 \hfill\break
The place name's origin is thought to come from Ainu. }
    