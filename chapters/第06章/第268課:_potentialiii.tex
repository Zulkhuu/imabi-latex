    
\chapter{Potential III}

\begin{center}
\begin{Large}
第268課: Potential III 
\end{Large}
\end{center}
 
\par{ This lesson will focus on more difficult potential phrases in Japanese. Pay close attention to even the most minute differences. }
      
\section{可能}
 
\par{  可能 is a 形容動詞 meaning "possible". Impossible is ${\overset{\textnormal{ふかのう}}{\text{不可能}}}$ . }
 
\par{1. 可能ですが、本当に難しいです。 \hfill\break
It's possible, but it's really difficult. }

\par{2. ${\overset{\textnormal{じっこうかのう}}{\text{実行可能}}}$ だ。 \hfill\break
It's feasible. }

\par{3. ${\overset{\textnormal{いっしゅん}}{\text{一瞬}}}$ のうちに ${\overset{\textnormal{とうけいがく}}{\text{統計学}}}$ の ${\overset{\textnormal{もんだい}}{\text{問題}}}$ を ${\overset{\textnormal{と}}{\text{解}}}$ くのは不可能だよ。 \hfill\break
Solving a statistics problem in an instant is what's impossible! }
 
\par{4. それはほとんど不可能だ。 \hfill\break
That is all but impossible. }
 
\par{\textbf{Word Note }: 可 is in some words to mean "-able". Ex. ${\overset{\textnormal{かねん}}{\text{可燃}}}$ ごみ = combustible refuse. }

\par{5. 可否を論\{ずる・じる\}. \hfill\break
To dispute propriety. }

\par{6. 可視光線 \hfill\break
Visible ray }

\par{7. 不可視光線 \hfill\break
Unvisible ray }
      
\section{あたう  (古風な言い方)}
 
\par{ Although now quite old-fashioned, you may still come across this potential word. It's normally limited to 連体形+\{(こと)・に\}あたわず. The affirmative potential form of verbs became used in Japanese from Western influence when the first translations of Western works were attempted. So, starting from then, あたう became used as well. }

\par{\textbf{Pronunciation Note }: あたう is often written as あとう because it's pronounced as あとー. }
 
\par{8. 行くことあたわず。(Very old-fashioned) \hfill\break
I can't go. }

\par{9. 称賛おくあたわず。(Very old-fashioned) \hfill\break
I can't help but admire you. }

\par{10. あたう限り努力します。(Old-fashioned) \hfill\break
I will try to the best of my abilities. }

\par{\textbf{Usage Note }: あたうる限り is a common misuse by natives. }
      
\section{The Potential of ある  (書き言葉的)}
 
\par{ The potential form of ある is あり得(え・う)る. However, ありうる should be used only as the ${\overset{\textnormal{れんたいけい}}{\text{連体形}}}$ . This is a remnant feature of the original verb 得(う) in Classical Japanese. }

\par{11. ${\overset{\textnormal{あ}}{\text{有}}}$ り ${\overset{\textnormal{え}}{\text{得}}}$ ないことだよ。 \hfill\break
That\textquotesingle s impossible! }
 
\par{12. 彼がまだ生きているなどありえないことだ。 \hfill\break
It is impossible that he is alive. }

\par{13. うそ!信じられない!そんなの、ありえないよ。あんないい人が、人を殺すだなんて。 \hfill\break
Lie! I can't believe it! That's impossible. Such a nice person killing someone\dothyp{}\dothyp{}\dothyp{} }

\par{14. \{ ${\overset{\textnormal{あ}}{\text{有}}}$ り ${\overset{\textnormal{え}}{\text{得}}}$ る・ ${\overset{\textnormal{お}}{\text{起}}}$ こり ${\overset{\textnormal{う}}{\text{得}}}$ る\} ${\overset{\textnormal{さいがい}}{\text{災害}}}$  \hfill\break
A possible disaster }
 
\par{15. そのような ${\overset{\textnormal{じこ}}{\text{事故}}}$ は ${\overset{\textnormal{りくつ}}{\text{理屈}}}$ ではあり ${\overset{\textnormal{え}}{\text{得}}}$ るが、 ${\overset{\textnormal{じっさい}}{\text{実際}}}$ にはまず ${\overset{\textnormal{お}}{\text{起}}}$ こらない。 \hfill\break
That kind of accident is possible in reason, but above all, it won't really happen. }

\begin{center}
\textbf{~得る } 
\end{center}
 
\par{ Following from above, it is also a common practice in semi-classical format to use ${\overset{\textnormal{え}}{\text{得}}}$ る after the ${\overset{\textnormal{れんようけい}}{\text{連用形}}}$ of a verb. It can be read as うる in the 連体形, but in very old-fashioned speech, you can also see it used at the end of a sentence. With a transitive verb, we see that even this option allows を to mark the direct object. }

\par{ With a verb of non-volition, this shows possibility. When added to a verb of volition, it expresses rating or degree of ability to do something. Remember that non-volitional verbs don't have a 可能形. }
 
\par{16. このように取り得る。 \hfill\break
It can take it like this. }

\par{17. ${\overset{\textnormal{びしょう}}{\text{微笑}}}$ は ${\overset{\textnormal{きん}}{\text{禁}}}$ じえない。 \hfill\break
Smiling is irresistible. }

\par{18. この時期、 ${\overset{\textnormal{こうさ}}{\text{黄砂}}}$ が ${\overset{\textnormal{ひらい}}{\text{飛来}}}$ しうるのか。 \hfill\break
Loess flying in is possible in this time? }

\par{\textbf{Word Note }: 黄砂 is dust from the Yellow River region of China. }

\par{19. タバコは ${\overset{\textnormal{けんこう}}{\text{健康}}}$ を ${\overset{\textnormal{がい}}{\text{害}}}$ し ${\overset{\textnormal{う}}{\text{得}}}$ る。 \hfill\break
Tobacco can hurt your health. }

\par{20. 彼は死の ${\overset{\textnormal{きけん}}{\text{危険}}}$ が ${\overset{\textnormal{だっ}}{\text{脱}}}$ し ${\overset{\textnormal{え}}{\text{得}}}$ ぬ。(Old-fashioned) \hfill\break
He can't escape death. }

\par{21. 私がいることによって、彼らは裸のゲリラとしての彼らたり得ているのだ。 \hfill\break
It was because of me being there that they were able to stay being their bare guerrilla fighter \hfill\break
selves. \hfill\break
From 光の雨 by 立松和平. }

\par{\textbf{Grammar Note }: The ~たり in たり得る is a classical copular auxiliary verb. }

\par{22. \{できうる △・なしうる\}限り \hfill\break
To the best of one's ability }

\par{\textbf{Notation Note }: The triangle indicates that this is acceptable to some but not all speakers. The meaning is doubled, which should make it ungrammatical. However, you still see it used. }

\par{\textbf{Verb Note }: なす is the transitive verb pair for なる and is used just like する. It is only used in the spoken language in expressions like 財をなす(to build a fortune), 色をなす (to change one's complexion when angry), 成し遂げる (to accomplish), etc. }

\par{ In speaking of なす, なせる can't be used as the potential form. This is because it would be a look alike to the actual interpretation of "done". In Classical Japanese, there was an auxiliary equivalent to English's perfect tenses: り. This attached to the 已然形. So, you would get なせり. When used before a noun, it becomes なせる. This grammar point is only found in set phrases. }

\par{23. 神のなせる業 \hfill\break
An act of God }

\par{24. ようやく ${\overset{\textnormal{てき}}{\text{敵}}}$ の目を ${\overset{\textnormal{のが}}{\text{逃}}}$ れることを得た。 \hfill\break
I was finally able to get away from the watch of the enemy. }

\par{25. 事情を聞かざるを得ない。 \hfill\break
We will have no choice but to question. }

\par{ As you can see, the verb 得る can also function as a potential verb, which makes sense if it can do so as a supplementary verb. Here, in these rare cases it is always after を and usually after a nominalized form of a verb. In the case of ~ざる, the 連体形 of the classical negative auxiliary ~ず, case particles after the 連体形 was sufficient in Classical Japanese to use a verb phrase as a noun. So, the same principles still apply. }

\par{\textbf{History Note }: The sound change that lead to independent potential verbs for 五段 verbs is uncertain. One idea is that it is a contraction of the 連用形 of 五段 verbs plus ~得る. }

\par{Ex. 書き得る \textrightarrow  書ける }

\par{ A more plausible account, however, is that a subset of intransitive verbs that showed spontaneity (phenomena that occur naturally) promoted the generalization to all 五段 verbs. Also, historically, the endings ~られる and ~れる, though different in appearance, have stood for and continue to be used for not only potential but also passivization, spontaneity, and light honorifics. It's believed, though, that the root usage is to show spontaneity. For a potential pattern to evolve from another line of spontaneity phrases shouldn't be a surprise in light of this. }
    