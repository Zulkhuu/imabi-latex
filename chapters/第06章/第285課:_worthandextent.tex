    
\chapter{Worth \& Extent}

\begin{center}
\begin{Large}
第285課: Worth \& Extent: ~に足る, ~に依存, ~に値する, \& ~に及ぶ 
\end{Large}
\end{center}
 
\begin{ltabulary}{|P|P|P|P|}
\hline 

 ~に足る & ~に依存 \hfill\break
& ~に値する & ~に及ぶ \\ \cline{1-4}

\end{ltabulary}
      
\section{~に足る・足りる}
 
\par{ 足る and 足りる are essentially the same word. Historically, only the former existed, and the latter has only existed since around the Edo Period. Though not much time historically has passed since the Edo Period, the latter form has largely replaced the former. These two forms do not belong in the same class of verbs, and so their conjugations differ as follows: }

\begin{ltabulary}{|P|P|P|P|P|P|}
\hline 

 & 未然形+ない & 連用形+ます & 終止形 & 連体形 & 已然形+ば \\ \cline{1-6}

足りる (一段) & 足りない & 足ります & 足りる & 足りる & 足りれば \\ \cline{1-6}

足る (五段) & 足らない & 足ります & 足る & 足る & 足れば \\ \cline{1-6}

\end{ltabulary}

\par{ Both forms have the same 連用形, and in forms like 足ります・足りません, they are identical. Both verbs show that something is "sufficient\slash enough". For all intended purposes, this is a single word with fluctuation in form. This is likely because it has been a frequently used word for centuries. However, you don't necessarily hear people say the following all over Japan (though this would be said in some regions). }

\par{i. ぎりぎり5000円あれば十分足るよ。 \hfill\break
If you just barely have 5000 yen, that'll be enough. }

\par{ There are set forms that use less commonly used forms such as 足る itself. The negative form 足らない is actually not that uncommon. Interestingly enough, ~足らず exists and is used a lot even though 足りず is not used at all. Old words and grammar go together. }

\par{1. 衣食足りて ${\overset{\textnormal{れいせつ}}{\text{礼節}}}$ を知る。 (Literary set phrase) \hfill\break
Only when basic needs for living are met can people spare the effort to be polite. }

\par{2. 短い言葉で足りるところに長い言葉を使うのは ${\overset{\textnormal{ほうふ}}{\text{豊富}}}$ な知識の ${\overset{\textnormal{しるし}}{\text{印}}}$ だ。 \hfill\break
Using a long word when a short word is enough is a sign of manifold knowledge. }

\par{3. 脳たりん。 \hfill\break
You simpleton. }

\par{4. あなたの直観は、信用するに足る実績がありました。(信用するに足る is a rather common set phrase) \hfill\break
Did your intuition have any real results worth trusting? }

\par{5. 3千円必要だが半分足りない。 \hfill\break
I need 3,000 yen but I lack half of it. }

\par{6. 彼女はようやく舌足らずの矯正に成功したそうです。 \hfill\break
She has apparently finally succeeded in correcting her lisp. }

\par{7. あなた達全員に足りるだけの食糧があります。 \hfill\break
There is enough food for all of you. }

\par{8a. 言い足りない = Not say enough. \hfill\break
8b. 言う必要がない = It's not worth saying. }

\par{9. あの新作は論ずるに足りません。(Written language) \hfill\break
It's not worth arguing with the new plan. }

\par{10. 政府は取るに足りない ${\overset{\textnormal{こうもく}}{\text{項目}}}$ を予算から ${\overset{\textnormal{けず}}{\text{削}}}$ らざるを得ないでしょう。 \hfill\break
The government probably has no choice but to cut frivolous items from the budget. }
      
\section{~に依存}
 
\par{  依存 means "dependence" and is used with に as a verb to mean "to depend on". Its most correct reading is いそん. However, the amount of people who read it as いぞん is growing. This is especially the case when it is used as a verb in 依存する. }
 
\par{11. 大国の経済力に ${\overset{\textnormal{}}{\text{依存}}}$ する小国がたくさん存在しています。 \hfill\break
Many small countries that depend on the economic strength of major powers exist. }
 
\par{12. その町の経済は農業に依存している。 \hfill\break
The village's economy depends on agriculture. }
 
\par{13. 彼の ${\overset{\textnormal{えんじょ}}{\text{援助}}}$ に依存しないでください。 \hfill\break
Don't depend on his help. }
 
\par{14. 成功は方法に依存します。 \hfill\break
Success depends on the method. }
      
\section{~に値する}
 
\par{ This phrase means "to be worthy of". It is equivalent to 価値がある. }
 
\par{15. あのウェブサイトは見る\{に値する・価値がある\}。 \hfill\break
That website is worth seeing. }
 
\par{16. あの4歳の子供は大人でも見るに値する絵を描いたよ! \hfill\break
That four year old child drew a picture worthy of adult appreciation. }
      
\section{~に及ぶ}
 
\par{ ~に及ぶ is found after both nouns and the 連体形 of verbs to show that either a certain final step has at last become so. In the negative, it shows that there is no need for the said action in follows. }

\par{17a. 言うにや及ばない。(古風) \hfill\break
17b. 言うまでもない。 \hfill\break
17c. 言うには及ばない。 \hfill\break
It's not necessary to say so. }

\par{18. 彼は長ずるに及んで文才を現した。 (書き言葉) \hfill\break
He finally revealed his literary talent in excelling. }

\par{19. 犯行に及んだ。 \hfill\break
It has finally become an offense. }

\par{20. この ${\overset{\textnormal{ご}}{\text{期}}}$ に及んで何をいうのか。(ちょっと固い) \hfill\break
It's a little late to mention that. }

\par{21. 礼には及びません。 \hfill\break
Don't mention it. }

\par{22. 折り ${\overset{\textnormal{かえ}}{\text{返}}}$ し電話をかけていただくには及びません。 \hfill\break
There's no need for you to call me back. }
    