    
\chapter{Disregard II}

\begin{center}
\begin{Large}
第281課: Disregard II: によらず \& (の)いかん\{にかかわらず・を問わず・によらず\} 
\end{Large}
\end{center}
 \hfill\break
 This lesson continues on the topic of expressions that translate as "regardless of\slash irrespective of." For the most part, most of this lesson brings very little new information aside from the introduction of によらず and the use of いかん with this expression as well as those introduced in the introductory lesson on this topic.       
\section{によらず}
 
\par{ によらず comes from the verb 依る in conjunction with the auxiliary verb ず. It is inherently the negative form of によって, which is used most often to show the agent of a sentence—“by” which\slash whom something is done, or to show means. As によらず, it may either show that something is “not by…” or that “X” isn't concerned with the situation at hand. Ultimately, its main purpose is to reject dependency. In other words, the Y circumstance is not dependent on an X circumstance. }

\par{ Nouns that によらず is most likely to follow include those that deal with kind, standards, hope, efforts, goals, etc. }

\par{1. ${\overset{\textnormal{にんげん}}{\text{人間}}}$ の ${\overset{\textnormal{すく}}{\text{救}}}$ いは、 ${\overset{\textnormal{にんげん}}{\text{人間}}}$ の ${\overset{\textnormal{ししつ}}{\text{資質}}}$ や ${\overset{\textnormal{せいかく}}{\text{性格}}}$ 、 ${\overset{\textnormal{どりょく}}{\text{努力}}}$ 、 ${\overset{\textnormal{けいけん}}{\text{経験}}}$ などにはよらず、 ${\overset{\textnormal{かみさま}}{\text{神様}}}$ の ${\overset{\textnormal{あい}}{\text{愛}}}$ と ${\overset{\textnormal{ふくいん}}{\text{福音}}}$ を ${\overset{\textnormal{しん}}{\text{信}}}$ じることによって、 ${\overset{\textnormal{かみ}}{\text{神}}}$ の ${\overset{\textnormal{すく}}{\text{救}}}$ いに ${\overset{\textnormal{いた}}{\text{至}}}$ ることが出来るというものです。 \hfill\break
Man\textquotesingle s salvation does not come by a person\textquotesingle s endowments, nature, efforts, or experiences, but by believing in God\textquotesingle s love and His Gospel, one may be able to attain God\textquotesingle s saving grace. }

\par{\textbf{Sentence Note }: によらず is very common with religious expressions. This is partly because many religious texts are translated works; however, they do provide perfect context for this expression. }

\par{2. ${\overset{\textnormal{み}}{\text{見}}}$ かけによらず ${\overset{\textnormal{た}}{\text{食}}}$ べるんですね。 \hfill\break
You sure eat a lot despite your appearance, huh. }

\par{3. ${\overset{\textnormal{しごと}}{\text{仕事}}}$ の ${\overset{\textnormal{ないよう}}{\text{内容}}}$ によらず、 ${\overset{\textnormal{きゅうりょう}}{\text{給料}}}$ は ${\overset{\textnormal{いってい}}{\text{一定}}}$ している。 \hfill\break
Salaries are fixed regardless of the substance of one\textquotesingle s work. }

\par{4. ${\overset{\textnormal{ぶりょく}}{\text{武力}}}$ によらず、 ${\overset{\textnormal{けんりょく}}{\text{権力}}}$ によらず、ただ ${\overset{\textnormal{わ}}{\text{我}}}$ が ${\overset{\textnormal{れい}}{\text{霊}}}$ によって、 ${\overset{\textnormal{た}}{\text{立}}}$ ちはだかる ${\overset{\textnormal{やま}}{\text{山}}}$ は ${\overset{\textnormal{へいち}}{\text{平地}}}$ となる。 \hfill\break
Neither by force of arms nor by authority but by my spirit alone, this mountain which looms forth shall become flat land. }

\par{5. ${\overset{\textnormal{み}}{\text{見}}}$ えるところによらず、 ${\overset{\textnormal{しんこう}}{\text{信仰}}}$ によって ${\overset{\textnormal{あゆ}}{\text{歩}}}$ む。 \hfill\break
I walk not by where I can see but by faith. }

\par{6. ${\overset{\textnormal{ほっきょくせい}}{\text{北極星}}}$ は ${\overset{\textnormal{きせつ}}{\text{季節}}}$ や ${\overset{\textnormal{じかん}}{\text{時間}}}$ によらず、いつも ${\overset{\textnormal{おな}}{\text{同}}}$ じ ${\overset{\textnormal{まきた}}{\text{真北}}}$ の ${\overset{\textnormal{ほうがく}}{\text{方角}}}$ に ${\overset{\textnormal{み}}{\text{見}}}$ えます。 \hfill\break
Polaris is always seen in the same due north direction regardless of time or season. }

\par{7. ${\overset{\textnormal{きょか}}{\text{許可}}}$ によらず ${\overset{\textnormal{のうちとう}}{\text{農地等}}}$ が ${\overset{\textnormal{しゅとく}}{\text{取得}}}$ できる ${\overset{\textnormal{ばあい}}{\text{場合}}}$ があります。 \hfill\break
There are circumstances in which one can obtain agriculture land and such irrespective of license. }

\par{8. ${\overset{\textnormal{せいたい}}{\text{生体}}}$ サンプルによらず ${\overset{\textnormal{けつえき}}{\text{血液}}}$ を ${\overset{\textnormal{つか}}{\text{使}}}$ ってリアルタイムでガンの ${\overset{\textnormal{けんさ}}{\text{検査}}}$ が ${\overset{\textnormal{でき}}{\text{出来}}}$ るようになりました。 \hfill\break
It has become possible to do cancer screening in real time by using blood and not by using a biological sample. }

\par{\textbf{Spelling Note }: ガン may also be spelled as 癌. }

\par{9. ${\overset{\textnormal{ほうりつ}}{\text{法律}}}$ によらず ${\overset{\textnormal{かおにんしょうそうち}}{\text{顔認証装置}}}$ を ${\overset{\textnormal{しよう}}{\text{使用}}}$ することは ${\overset{\textnormal{あき}}{\text{明}}}$ らかに ${\overset{\textnormal{いほう}}{\text{違法}}}$ なのです。 \hfill\break
The use of face recognition apparatuses irrespective of the law is clearly unlawful. }

\par{10. ${\overset{\textnormal{なんにん}}{\text{何人}}}$ も、 ${\overset{\textnormal{ほう}}{\text{法}}}$ の ${\overset{\textnormal{てきせい}}{\text{適正}}}$ な ${\overset{\textnormal{てつづ}}{\text{手続}}}$ きによらずに、 ${\overset{\textnormal{せいめい}}{\text{生命}}}$ 、 ${\overset{\textnormal{じゆう}}{\text{自由}}}$ 、または ${\overset{\textnormal{ざいさん}}{\text{財産}}}$ を ${\overset{\textnormal{うば}}{\text{奪}}}$ われることはない。 \hfill\break
All people will never be robbed of one\textquotesingle s life, freedom, or property without due process of the law. }

\par{ Typically, whenever the noun in question is an interrogative phrase such as 誰, 何, 何事, etc. it can be interpreted as meaning “regardless of\slash no matter…” }

\par{11. ${\overset{\textnormal{なにごと}}{\text{何事}}}$ によらず、いつでも ${\overset{\textnormal{そうだん}}{\text{相談}}}$ してください。 \hfill\break
Regardless of whatever it is, please consult with me at any time. }

\par{12. ${\overset{\textnormal{おとうと}}{\text{弟}}}$ は ${\overset{\textnormal{なに}}{\text{何}}}$ によらず ${\overset{\textnormal{おくびょう}}{\text{臆病}}}$ なところがとても ${\overset{\textnormal{しんぱい}}{\text{心配}}}$ です。 \hfill\break
I am very worried about how my little brother is timid regardless of what it is. }

\par{ によらず can be seen as によらない at the end of a sentence. This grammatical situation is very indicative of the written language. The grammatical pattern itself is already quite stiff and formal. This was also the case with にかかわらない, but because that pattern is not used at the end of a sentence, によらない makes up for this. }

\par{13. ${\overset{\textnormal{さんか}}{\text{参加}}}$ の ${\overset{\textnormal{かひ}}{\text{可否}}}$ は ${\overset{\textnormal{ねんれい}}{\text{年齢}}}$ によらない。 \hfill\break
Consideration of participation is irrespective of age. }

\par{ There are quite a few common set phrases that utilize the meaning of “irrespective of…” For instance, the translation of “You can\textquotesingle t judge a book by its cover” uses this meaning of によらず. These phrases utilize this pattern at the end of the sentence. }

\par{14. ${\overset{\textnormal{ひと}}{\text{人}}}$ は ${\overset{\textnormal{み}}{\text{見}}}$ た ${\overset{\textnormal{め}}{\text{目}}}$ によらず。 \hfill\break
You can\textquotesingle t judge a book by its cover\slash a man can\textquotesingle t be judged by his appearance. }

\par{15. ${\overset{\textnormal{きっきょう}}{\text{吉凶}}}$ は ${\overset{\textnormal{ひと}}{\text{人}}}$ によりて ${\overset{\textnormal{にち}}{\text{日}}}$ によらず。 \hfill\break
Fortune comes by one\textquotesingle s action, not by the day. \hfill\break
 \hfill\break
\textbf{Sentence Note }: Ex. 15 shows how the meanings of “by” and indicating relation can blend together and are one of the same thing. Ex. 15 shows that there is always agency involved with によらず, or more specifically, the lack thereof. }
      
\section{(の)いかん\{にかかわらず・を問わず・によらず\}}
 
\par{ (の)いかん can be inserted before either にかかわらず, を問わず, によらず with hardly any change in meaning these patterns. いかん, written as 如何 in 漢字, means “circumstance” as in “how” situation X is. Interestingly enough, いかん can directly attach to nouns, making the particle の completely optional. }

\begin{center}
\textbf{~(の)いかんにかかわらず }
\end{center}

\par{ With にもかかわらず, (の)いかん typically follows words that involve multiple parts or selections. (の)いかん can only be added to nouns. These nouns, naturally, deal with circumstance. The nouns that are best with にかかわらず can be standalone nouns that refer to a single entity (ex. 契約 = Contract). It's also common with nouns created by pairing two things together like 有無 (existence or nonexistence). It is also great with nouns with indeterminate amount of outcomes such as 結果 (results) and 内容 (content). It isn\textquotesingle t, however, used with lists of three or more nouns. }

\par{16. ${\overset{\textnormal{さいひ}}{\text{採否}}}$ (の)いかんにかかわらず、 ${\overset{\textnormal{けっか}}{\text{結果}}}$ は ${\overset{\textnormal{ゆうびん}}{\text{郵便}}}$ でお ${\overset{\textnormal{し}}{\text{知}}}$ らせ ${\overset{\textnormal{いた}}{\text{致}}}$ します。 \hfill\break
We will notify you of the results via mail whether you are recruited or not. }

\par{17. ${\overset{\textnormal{ほうりつないよう}}{\text{法律内容}}}$ (の)いかんにかかわらず、 ${\overset{\textnormal{ばいしゅん}}{\text{売春}}}$ は ${\overset{\textnormal{しゅじゅ}}{\text{種々}}}$ の ${\overset{\textnormal{せっきゃくぎょう}}{\text{接客業}}}$ と ${\overset{\textnormal{むす}}{\text{結}}}$ び ${\overset{\textnormal{つ}}{\text{付}}}$ いて ${\overset{\textnormal{たよう}}{\text{多様}}}$ な ${\overset{\textnormal{けいたい}}{\text{形態}}}$ を ${\overset{\textnormal{と}}{\text{取}}}$ りつつ ${\overset{\textnormal{そんざい}}{\text{存在}}}$ している。 \hfill\break
Regardless of the legal circumstances, prostitution is connected to various service trades and exists in diverse forms. }

\par{18. ${\overset{\textnormal{けんさ}}{\text{検査}}}$ の ${\overset{\textnormal{けっか}}{\text{結果}}}$ いかんにかかわらず、 ${\overset{\textnormal{さっきゅう}}{\text{早急}}}$ にご ${\overset{\textnormal{ほうこくさ}}{\text{報告差}}}$ し ${\overset{\textnormal{あ}}{\text{上}}}$ げます。 \hfill\break
Regardless of the examination results, we will report them to you immediately. }

\par{19. ${\overset{\textnormal{こくせき}}{\text{国籍}}}$ (の)いかんにかかわらず、 ${\overset{\textnormal{ゆうしゅう}}{\text{優秀}}}$ な ${\overset{\textnormal{じんざい}}{\text{人材}}}$ を ${\overset{\textnormal{もと}}{\text{求}}}$ めております。 \hfill\break
We are seeking excellent talented people irrespective of nationality. }

\par{20. ${\overset{\textnormal{ちょうさ}}{\text{調査}}}$ の ${\overset{\textnormal{けっか}}{\text{結果}}}$ いかんにかかわらず、ご ${\overset{\textnormal{れんらく}}{\text{連絡}}}$ ください。 \hfill\break
Regardless of the results of the investigation, please contact me\slash us. }

\par{21. 履歴書等の応募書類は、結果いかんにかかわらず、ご返却いたしませんので、ご了承ください。 \hfill\break
We do not return application documents such as one\textquotesingle s resume regardless of one\textquotesingle s results, so please understand this. }

\par{22. ${\overset{\textnormal{けんさ}}{\text{検査}}}$ の ${\overset{\textnormal{けっか}}{\text{結果}}}$ いかんにかかわらず、インフルエンザのような ${\overset{\textnormal{しょうじょう}}{\text{症状}}}$ があれば、 ${\overset{\textnormal{しゅうい}}{\text{周囲}}}$ の ${\overset{\textnormal{ひと}}{\text{人}}}$ にうつす ${\overset{\textnormal{かのうせい}}{\text{可能性}}}$ があります。 \hfill\break
Regardless of the results of your examination, if you have symptoms like that of the flu, there is the possibility that you may spread (the illness) to those around you. }

\par{23. 乳牛及び肉用牛農家の全てに立入検査を実施し、疑いのある症状を発見した場合は、検査を行うとともに検査の結果いかんにかかわらず、全て焼却処分され、食肉や肉骨粉の形で流通することがないようにします。 \hfill\break
Site inspections of the entirety of farmhouses, both milk cows and beef cattle, are to be implemented; in the case that it is suspected that symptoms have been detected, all (cows) are to be incinerated along with examination regardless of the examination\textquotesingle s results, and efforts will be made to make sure that none of it is circulated in the form of meat and\slash or meat-and-bone meal. }

\par{24. ${\overset{\textnormal{こくみんとうひょう}}{\text{国民投票}}}$ の ${\overset{\textnormal{けっか}}{\text{結果}}}$ いかんにかかわらず、 ${\overset{\textnormal{ざいせいあかじ}}{\text{財政赤字}}}$ と ${\overset{\textnormal{せいちょうていたい}}{\text{成長停滞}}}$ という ${\overset{\textnormal{けいざいもんだい}}{\text{経済問題}}}$ は ${\overset{\textnormal{か}}{\text{変}}}$ わらない。 \hfill\break
Regardless of the results of a national referendum, the economic problems involving the budget deficit and slowdown in economic growth won\textquotesingle t change. }

\begin{center}
 \textbf{~(の)いかんを問わず }
\end{center}

\par{ を問わず itself is used most with nouns that refer to two or more elements. Unlike with にかかわらず, を問わず can be used with three or more nouns. This can be seen in Ex 25. }

\par{25. ${\overset{\textnormal{がくれき}}{\text{学歴}}}$ ・ ${\overset{\textnormal{ねんれい}}{\text{年齢}}}$ ・ ${\overset{\textnormal{せいべつ}}{\text{性別}}}$ ・ ${\overset{\textnormal{かこ}}{\text{過去}}}$ の ${\overset{\textnormal{じっせき}}{\text{実績}}}$ のいかんを ${\overset{\textnormal{と}}{\text{問}}}$ わず、 ${\overset{\textnormal{しょにんきゅう}}{\text{初任給}}}$ は ${\overset{\textnormal{いちりつ}}{\text{一律}}}$ です。 \hfill\break
Regardless of one\textquotesingle s education, age, sex, or past accomplishments, initial salary is at a flat rate. }

\par{26. ${\overset{\textnormal{こくせき}}{\text{国籍}}}$ のいかんを ${\overset{\textnormal{と}}{\text{問}}}$ わず、どなたでも ${\overset{\textnormal{さんか}}{\text{参加}}}$ できます。 \hfill\break
Anyone can participate regardless of nationality. }

\par{27. ${\overset{\textnormal{りゆう}}{\text{理由}}}$ のいかんを ${\overset{\textnormal{と}}{\text{問}}}$ わず、 ${\overset{\textnormal{むだんけっせき}}{\text{無断欠席}}}$ は ${\overset{\textnormal{ゆる}}{\text{許}}}$ されません。 \hfill\break
Whatever the reason, truancy is impermissible. }

\par{28. ${\overset{\textnormal{じぎょうしょ}}{\text{事業所}}}$ が ${\overset{\textnormal{しようざいりょう}}{\text{使用材料}}}$ のいかんを ${\overset{\textnormal{と}}{\text{問}}}$ わず ${\overset{\textnormal{ぶんるい}}{\text{分類}}}$ される。 \hfill\break
Enterprises are categorized irrespective of materials used. }

\par{29. ${\overset{\textnormal{そんがいはっせい}}{\text{損害発生}}}$ の ${\overset{\textnormal{げんいん}}{\text{原因}}}$ いかんを ${\overset{\textnormal{と}}{\text{問}}}$ わず、 ${\overset{\textnormal{せきにん}}{\text{責任}}}$ を ${\overset{\textnormal{いっさいお}}{\text{一切負}}}$ わない。 \hfill\break
Regardless of the cause of the breakage, we are not responsible whatsoever. }

\par{30. ${\overset{\textnormal{こっか}}{\text{国家}}}$ に ${\overset{\textnormal{ひつよう}}{\text{必要}}}$ な ${\overset{\textnormal{じぎょう}}{\text{事業}}}$ は ${\overset{\textnormal{りえき}}{\text{利益}}}$ のいかんを ${\overset{\textnormal{と}}{\text{問}}}$ わず、 ${\overset{\textnormal{じっさい}}{\text{実際}}}$ に ${\overset{\textnormal{りえき}}{\text{利益}}}$ を ${\overset{\textnormal{あ}}{\text{上}}}$ げるようにす(る)べきだ。 \hfill\break
We ought to make sure to actually increase the profits of industries that are necessary for the nation regardless of their (current) profit. }

\par{31. ${\overset{\textnormal{じじつ}}{\text{事実}}}$ いかんを ${\overset{\textnormal{と}}{\text{問}}}$ わず ${\overset{\textnormal{めいよきそん}}{\text{名誉毀損}}}$ で ${\overset{\textnormal{うった}}{\text{訴}}}$ えられる。 \hfill\break
To be sued for libel irrespective of the facts. }

\par{32. ${\overset{\textnormal{かいいん}}{\text{会員}}}$ は、 ${\overset{\textnormal{りゆう}}{\text{理由}}}$ のいかんを ${\overset{\textnormal{と}}{\text{問}}}$ わず、 ${\overset{\textnormal{ほんきてい}}{\text{本規定}}}$ に ${\overset{\textnormal{もと}}{\text{基}}}$ づく ${\overset{\textnormal{いっさい}}{\text{一切}}}$ の ${\overset{\textnormal{けんりおよ}}{\text{権利及}}}$ び ${\overset{\textnormal{ぎむ}}{\text{義務}}}$ について、これを ${\overset{\textnormal{だいさんしゃ}}{\text{第三者}}}$ に ${\overset{\textnormal{ゆずりわた}}{\text{譲渡}}}$ し、 ${\overset{\textnormal{たんぽ}}{\text{担保}}}$ の ${\overset{\textnormal{よう}}{\text{用}}}$ に ${\overset{\textnormal{きょう}}{\text{供}}}$ することはできません。 \hfill\break
A member transfers all rights and obligations founded in these provisions to the third-party beneficiary and cannot use them as collateral for whatever reason. }

\begin{center}
\textbf{~(の)いかんによらず }
\end{center}

\par{  によらず\textquotesingle s main function is to deny dependence. The words that are associated with things that people depend on, whether it be means by which they do things or whether it be things that people rely on, the purpose of によらず is to state that situation Y is not dependent on them. によらず is actually more commonly used with (の)いかん than it is without it. The use of いかん helps facilitate rejecting that a Y circumstance is dependent on an X circumstance. }

\par{33. ${\overset{\textnormal{りゆう}}{\text{理由}}}$ のいかんによらず、 ${\overset{\textnormal{さつがい}}{\text{殺害}}}$ は ${\overset{\textnormal{ゆる}}{\text{許}}}$ されないことである。 \hfill\break
Murder is not permissible for any reason whatsoever. }

\par{34. ${\overset{\textnormal{りゆう}}{\text{理由}}}$ のいかんによらず、 ${\overset{\textnormal{しけんかいしご}}{\text{試験開始後}}}$ の ${\overset{\textnormal{にゅうしつ}}{\text{入室}}}$ は ${\overset{\textnormal{みと}}{\text{認}}}$ めません。 \hfill\break
Entry once the exam has begun will not be approved for any reason whatsoever. }

\par{35. ${\overset{\textnormal{りゆう}}{\text{理由}}}$ のいかんによらず、 ${\overset{\textnormal{じっ}}{\text{10}}}$ ${\overset{\textnormal{ぷんいじょうちこく}}{\text{分以上遅刻}}}$ した ${\overset{\textnormal{ばあい}}{\text{場合}}}$ は ${\overset{\textnormal{しけん}}{\text{試験}}}$ を ${\overset{\textnormal{う}}{\text{受}}}$ けられません。 \hfill\break
You will not be able to take the exam in the event you are late ten minutes or longer whatever the reason may be. }

\par{36. ${\overset{\textnormal{りゆう}}{\text{理由}}}$ のいかんによらず、 ${\overset{\textnormal{けっきん}}{\text{欠勤}}}$ は ${\overset{\textnormal{けっきん}}{\text{欠勤}}}$ であります。 \hfill\break
An absence from work is an absence whatever the reason. }

\par{37. ジュバ ${\overset{\textnormal{し}}{\text{市}}}$ を ${\overset{\textnormal{のぞ}}{\text{除}}}$ く ${\overset{\textnormal{みなみ}}{\text{南}}}$ スーダンへの ${\overset{\textnormal{とこう}}{\text{渡航}}}$ は、 ${\overset{\textnormal{もくてき}}{\text{目的}}}$ のいかんによらず ${\overset{\textnormal{えんき}}{\text{延期}}}$ して ${\overset{\textnormal{くだ}}{\text{下}}}$ さい。 \hfill\break
Please postpone voyage to South Sudan, excluding the city of Juba, whatever your objective may be. }

\par{38. ${\overset{\textnormal{いったんじゅり}}{\text{一旦受理}}}$ した ${\overset{\textnormal{しょるい}}{\text{書類}}}$ はいかんによらず ${\overset{\textnormal{へんかん}}{\text{返還}}}$ いたしません。 \hfill\break
We will not return documentation that has been accepted whatever the reason may be. }

\par{39. ${\overset{\textnormal{りゆう}}{\text{理由}}}$ のいかんによらず、 ${\overset{\textnormal{りょうひん}}{\text{良品}}}$ の ${\overset{\textnormal{へんぴん}}{\text{返品}}}$ はお ${\overset{\textnormal{う}}{\text{受}}}$ け ${\overset{\textnormal{と}}{\text{取}}}$ りできません。 \hfill\break
We cannot accept the return of non-defective products whatever the reason may be. }

\par{40. ${\overset{\textnormal{しゅっせきけっせき}}{\text{出席欠席}}}$ のいかんによらず、 ${\overset{\textnormal{かなら}}{\text{必}}}$ ずメールで ${\overset{\textnormal{へんじ}}{\text{返事}}}$ をするようお ${\overset{\textnormal{ねが}}{\text{願}}}$ い ${\overset{\textnormal{いた}}{\text{致}}}$ します。 \hfill\break
Whether you attend or are absent, we ask that you please reply via e-mail regardless. }
    