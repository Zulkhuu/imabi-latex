    
\chapter{Reduplication}

\begin{center}
\begin{Large}
第263課: Reduplication: Adverbs 
\end{Large}
\end{center}
 
\par{ As has been the case for reduplication thus far, reduplication in adverbs is reliant upon diverse means of construction spanning several parts of speech. Some commonalities do exist, and it is these commonalities that will also emphasize the need to truly take each phrase for what it is worth on an individual basis. You will find that your control of vocabulary will be much more solid and intricate. }
      
\section{Examples}
 
\begin{center}
\textbf{From Nouns }
\end{center}

\par{ Some reduplication in adverbs results from the doubling of particular nouns. The grammatical ramifications tend to be quite unique. Some may have the particle \emph{to }と follow them, but this is never obligatory. Others need no particle at all and aid in the creation of more complex adverbial phrases. }

\par{1. それは、色々(と)事情があったんでしょう。 \hfill\break
 \emph{Sore wa, iroiro (to) jijō ga atta n deshō. \hfill\break
 }As far as that (is concerned), surely there were all sorts of circumstances. }

\par{2. 夫婦共々、より一層精進していますので、よろしくお願いします。 \hfill\break
 \emph{Fūfu tomodomo, yori issō shōjin shite imasu node, yoroshiku onegai shimasu. }\hfill\break
We, together as husband and wife, we will be even more diligent, and so we look forward to working with you. }

\par{\textbf{Word Note }: \emph{Tomo }共 is a noun meaning “both.” }

\begin{center}
\textbf{From Adverbial Nouns }
\end{center}

\par{ A very small number of reduplication in adverbs comes from adverbial nouns. The most common example is as follows. }

\par{3. 嫌々仕事をやるのなら、やらないほうがましです。 \hfill\break
 \emph{Iyaiya shigoto wo yaru no nara, yaranai hō ga mashi desu. } \hfill\break
If you\textquotesingle re going to grudgingly work, it\textquotesingle s better that you not work. }

\begin{center}
\textbf{From Adverbs }
\end{center}

\par{ A small number of reduplication in adverbs involves the doubling of adverbs. The number of such phrases in existence is quite low if one excludes those from onomatopoeia, which we\textquotesingle ll look at separately later in this lesson. Similarly to most examples of reduplication thus far, the nuance of the resultant phrase will always have a specialized meaning that will not be the same as its singular counterpart. }

\par{4. 一人で旅をしていると、いよいよ孤独感に落ち込んでしまう。 \hfill\break
 \emph{Hitori de tabi wo shite iru to, iyoiyo kodokukan ni ochikonde shimau. \hfill\break
 }As I\textquotesingle m traveling alone, I feel more and more down in a sense of isolation. }

\par{\textbf{Word Note }: \emph{Iyoiyo }いよいよ derives from the adverb \emph{iya }弥, which is known in the set phrase \emph{iya ga ue ni mo }弥が上にも (all the more). }

\par{5. ついつい言いそびれてしまいました。 \hfill\break
 \emph{Tsuitsui iisobirete shimaimashita. \hfill\break
 }I unintentionally forgot to speak of it. }

\par{6. わざわざ来てくれてありがとうございます。 \hfill\break
 \emph{Wazawaza kite kurete arigatō gozaimasu. }\hfill\break
Thank you for taking the trouble to come. }

\par{\textbf{Word Note }: \emph{Wazawaza }わざわざ derives from the adverb \emph{waza to }わざと (intentionally). It may alternatively be spelled as 態々. }

\begin{center}
\textbf{From Adjectives }
\end{center}

\par{ Reduplicated adverbial phrases resulting from the doubling of adjectival roots are frequently accompanied with the particle \emph{to }と, but whether it is obligatory or not is determined on a case-by-case basis. Sometimes, no particle intervention ever happens, while sometimes you will see the occasional \emph{ni }に as if the phrase in question is being treated as an adjectival noun in the adverbial form. These particle discrepancies are clearly defined in the examples below. }

\par{7. 近々(に)車検があります。 \hfill\break
 \emph{Chikajika (ni) shaken ga arimasu. }\hfill\break
I have a car inspection in the near future. }

\par{8. この先半年くらいで近々(に)閉めるのが確定している。 \hfill\break
 \emph{Kono saki hantoshi kurai de chikajika (ni) shimeru no ga kakutei shite iru. }\hfill\break
It's finalized that they will close business shortly about half a year from now. }

\par{9. よくよく考えてみれば当たり前の話です。 \hfill\break
 \emph{Yokuyoku kangaete mireba atarimae no hanashi desu. }\hfill\break
If you think very carefully, it\textquotesingle s quite obvious. }

\par{\textbf{Word Note }: \emph{Yokuyoku }よくよく may be used as an adverbial noun which takes either \emph{na }な or \emph{no }の when before nouns to mean “large extent,” but this is usually expressed with \emph{yohodo }よほど, which has the same grammatical limitations minus that it never takes \emph{na }な. }

\par{10. 深々と頭を下げるのは誠実さの表れ。 \hfill\break
 \emph{Fukabuka to atama wo sageru no wa seijitsusa no araware. }\hfill\break
Deeply bowing one\textquotesingle s head is an embodiment of sincerity. }

\par{11. 私は田舎で狭い畑を耕しながら細々と暮らしています。 \hfill\break
 \emph{Watashi wa inaka de semai hatake wo tagayashinagara hosoboso to kurashite imasu. }\hfill\break
I\textquotesingle m scraping by while cultivating a narrow field in the country. }

\par{12. 道端の土手に草花が青々と茂っている。 \hfill\break
 \emph{Michibata no dote ni kusabana ga aoao to shigette iru. }\hfill\break
Flowers are lushly growing thick in the embankments on the side of the road. }

\par{13. 暖炉の火が赤々と燃えている。 \hfill\break
 \emph{Danro no hi ga aka\textquotesingle aka to moete iru. \hfill\break
 }The hearth fire is burning bright red. }

\par{14. 黒々と墨で書かれている。 \hfill\break
 \emph{Kuroguro to sumi de kakarete iru. \hfill\break
 }It\textquotesingle s written deep black in India ink. }

\par{15. 間もなく夜が白々と明けてきた。 \hfill\break
 \emph{Ma mo naku yoru ga shirojiro to akete kita. }\hfill\break
The dawn grew bright in a short time. }

\par{16. 彼は自分の気持ちに薄々(と)気がついているようだ。 \hfill\break
 \emph{Kare wa jibun no kimochi ni ususu (to) ki ga tsuite iru yō da. \hfill\break
 }He seems vaguely aware of his own emotions. }

\begin{center}
\textbf{From Onomatopoeia }
\end{center}

\par{ A lot of onomatopoeic expressions are made via reduplication. In fact, an entire dictionary could be made with all the examples that could be found used in every day speech. These phrases give depth in nuance which will allow your speech to become ever more native like if used correctly. If there were ever a reason to study reduplication in detail for purposes of bettering your Japanese, placing heavy emphasis on learning and using as many of these phrases as possible will take you a long way. }

\par{ Grammatically speaking, they all have one thing in common: they may optionally take the particle \emph{to }と. The deciding factor as to when to use it or not is usually based on the cadence of the entire utterance. Essentially, whatever flows out of the mouth is right. }

\par{17. 雷がゴロゴロと鳴いている。 \hfill\break
 \emph{Kaminari ga gorogoro to naite iru. }\hfill\break
The thunder is roaring. }

\par{18. 田んぼに水が入ってカエルはゲロゲロと鳴いていました。 \hfill\break
 \emph{Tambo ni mizu ga haitte kaeru wa gerogero to naite imashita. }\hfill\break
Water got in the fields and then frogs ribbited. }

\par{\textbf{Spelling Note }: \emph{Tambo }may alternatively be spelled as 田圃. \emph{Kaeru }may alternatively be spelled as 蛙. }

\par{19. 僕はその破片をまじまじと見つめた。 \hfill\break
 \emph{Boku wa sono hahen wo majimaji to mitsumeta. \hfill\break
 }I took a long hard look at the fragments. }

\par{20. 表面が少しぶつぶつ(と)泡だってきたら醤油と味醂を加えてください。 \hfill\break
 \emph{Hyōmen ga sukoshi butsubutsu (to) awadatte kitara shōyu to mirin wo kuwaete kudasai. }\hfill\break
Once the surface starts to bubble and summer a little, add soy sauce and mirin. 7. 鶏肉500gを、適度の塩と少々のコショウを振って、こんがりとした焼き色が付くまで強火で焼いてください。 \hfill\break
 \emph{Toriniku gohyaku-guramu wo tekido no shio to shōshō no koshō wo futte kongari to shita yaki\textquotesingle iro ga tsuku made tsuyobi de yaite kudasai. }\hfill\break
Cook 500 grams of poultry by sprinkling a moderate amount of salt and a small amount of pepper and then cooking on high heat until it is beautifully browned. }

\begin{center}
\textbf{From Chinese Loans }
\end{center}

\par{ Those from Chinese loans may or may not always be stand-alone words, but what is reduplicated could be from nominal, adjectival, or adverbial phrases. For some, the addition of \emph{to }と may be optional or obligatory. This will be made visibly obvious in the examples below. }

\par{21. 少々お待ちください。 \hfill\break
 \emph{Shōshō omachi kudasai. }\hfill\break
Please wait a short while. }

\par{22. その状況については重々承知しております。 \hfill\break
 \emph{Sono jōkyō ni tsuite wa jūjū shōchi shite orimasu. \hfill\break
 }We are fully aware of the situation. }

\par{23.  段々(と)成長していく姿を見られて幸せです。 \hfill\break
 \emph{Dandan (to) seichō shite iku sugata wo mirarete shiawase desu. }\hfill\break
I'm so happy to be able to see (x) gradually grow. }

\par{24. せいぜいがんばるさ。 \hfill\break
 \emph{Seizei gambaru sa. }\hfill\break
I\textquotesingle ll keep at it as much as I can. }

\par{\textbf{Word Note }: \emph{Seizei }せいぜい is the reduplication of the noun \emph{sei }精 (energy\slash vigor). It originally has always held a positive meaning, but it now sometimes has an ironic\slash cynical nuance to it. This word may alternatively be spelled as 精々. }

\par{26. 清水さんは、ここで悠々と暮らしていたらしかった。 \hfill\break
 \emph{Shimizu-san wa, koko de yūyū to kurashite ita rashikatta. }\hfill\break
It appears that Mr. Shimizu lived here leisurely. }

\begin{center}
\textbf{From Number Phrases }
\end{center}

\par{ A small number of reduplicated phrases can be seen in number expressions such as those seen below. }

\par{27. 今のところ、五分五分かもしれません。 \hfill\break
 \emph{Ima no tokoro, gobu gobu kamoshiremasen. }\hfill\break
At present, it may be fifty-fifty. }

\par{28. 駅で降りた観光客は三々五々に散っていきました。 \hfill\break
 \emph{Eki de orita kankōkyaku wa sansangogo ni chitte ikimashita. }\hfill\break
The tourists that got off at the station scattered away in small groups. }

\par{29. いちいち指摘しなくてもいいでしょう。 \hfill\break
 \emph{Ichi\textquotesingle ichi shiteki shinakute mo ii deshō. }\hfill\break
Surely you don\textquotesingle t need to critique every single thing. }

\par{\textbf{Word Note }: If not already visually obvious, \emph{ichi\textquotesingle ichi }is reduplication of the number one, and thus, it may be alternatively written as 一々. }

\begin{center}
\textbf{Miscellaneous } 
\end{center}

\par{ If all this diversity weren't enough, there are also some examples that are truly unique. Take for example the word \emph{somosomo }そもそも below. It is the duplication of an ancient variant of \emph{sore mo }それも (that also). }

\par{30. この発想がそもそも間違っているのです。 \hfill\break
 \emph{Kono hassō ga somosomo machigatte iru no desu. }\hfill\break
This conception is what\textquotesingle s wrong in the first place. }
    