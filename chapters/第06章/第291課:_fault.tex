    
\chapter{Fault}

\begin{center}
\begin{Large}
第291課: Fault: ~せいで \& ~にかこつけて 
\end{Large}
\end{center}
 
\par{ This lesson is all about ways in Japanese to give an excuse\slash reason in ways we have yet to see thus far in the curriculum. }
      
\section{~せいで}
 
\par{ ~せいで means "since\slash because of\slash due to" in the sense of blame. If you use this pattern with something that is generally perceived as being a good thing, it will sound strange and usually unnatural. You would have to do some explaining. The way to express this in a positive fashion is by using のおかげ(さま)で, which doesn't always have to be used in a positive situation. The point is that at least it can. }
 
\begin{center}
\textbf{Examples }
\end{center}

\par{1. 妹のせいで、父に叱られた。 \hfill\break
I was scolded by my dad because of my little sister. }
 
\par{2. 俺のせいにすんな。(ちょっと失礼) \hfill\break
Don't blame me. }
 
\par{3. 津波のせいで食糧不足になった。 \hfill\break
We got into a food shortage due to the tsunami. }
 
\par{4. 病気のせいで、パーティーに行けなかった。 \hfill\break
I didn't come to the party due to an illness. }
 
\par{5. 休みなのに強風と花粉のせいで家から出れない。(Casual) \hfill\break
Although I'm on a break, due to strong winds and pollen, I can't leave the house. }
 
\par{6a. 彼が失敗したのは怠惰のせいだ。 \hfill\break
6b. 彼が失敗したのは怠けたからだ。(More natural) \hfill\break
He failed because of his laziness. }
 
\par{7. 雨のせいで、遅刻した。 \hfill\break
I arrived later due to the rain. }
 
\par{8. 彼は疲労のせいで、眠いというよりも疲れているのです。 \hfill\break
He is less sleepy than tired due to his fatigue. }
 
\begin{center}
 \textbf{~(の)せいか }
\end{center}
 
\par{ As the adverbial particle か shows uncertainty, it shows in translation when used with せい. }
 
\par{9. 天気のせいか、気分が悪い。 \hfill\break
I don't know if it's because of the weather, but I feel bad. }
 
\par{10. 山田さんは、試験ができなかったせいか今日は元気がありません。 \hfill\break
Mr. Yamada isn't well today apparently because he did bad on his exam. }

\par{11. }
 
\par{小林:ちょっと元気がないみたいけど、どうかしたの? \hfill\break
出口: うん、食中たりのせいかお腹が痛いんだ。 \hfill\break
Kobayashi: You look a little bad; what's wrong? \hfill\break
Deguchi: Yeah, I think my stomache's hurting because of food poisoning. }

\par{12. }
 
\par{中田:どうかしたの。 \hfill\break
川崎:うん、昨日の晩風邪を引いたせいか頭が痛くて。。。 \hfill\break
中田:じゃ、早く帰って寝たら? \hfill\break
Nakada: What's wrong? \hfill\break
Kawasaki: My head is hurting cause I might have gotten a cold last night. \hfill\break
Nakada: Well, how about getting home quickly and sleep? }
      
\section{~に託けて}
 
\par{ The 一段 verb 託ける means "to use as an excuse" and ~に託けて is often translated  as "under (the excuse of)". }

\par{13. ${\overset{\textnormal{しゅうきょう}}{\text{宗教}}}$ の名に ${\overset{\textnormal{かこつ}}{\text{託}}}$ けてコンキスタドーレスはアステックの帝国を ${\overset{\textnormal{せ}}{\text{攻}}}$ め滅ぼした。 \hfill\break
The Spanish Conquistadors attacked and utterly destroyed the Aztec Empire under the name of religion. }

\par{14. 彼は雪に託けて仕事をサボった。 \hfill\break
He skipped work under the excuse of snow.  }
    