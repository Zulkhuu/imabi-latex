    
\chapter{The Particles って, たって, \& だって}

\begin{center}
\begin{Large}
第255課: The Particles って, たって, \& だって 
\end{Large}
\end{center}
 
\par{ In this lesson we will learn about the particles って, たって, and だって which are necessary in improving your colloquial Japanese skills. }
      
\section{The Particle って}
 \hfill\break
 \textbf{The Case Particle }\textbf{って: Colloquial Citation } 
\par{ This is a very important feature of casual Japanese. The particle と is very frequently seen as って in spoken Japanese. Although it originally came from the contraction of という, it has since taken on と's function as a citation particle. }

\par{${\overset{\textnormal{}}{\text{1. 難}}}$ しいって ${\overset{\textnormal{}}{\text{思}}}$ った。 \hfill\break
I thought it was difficult. }
 
\par{${\overset{\textnormal{}}{\text{2. 負}}}$ けたよって ${\overset{\textnormal{}}{\text{言}}}$ われてもなあ。 \hfill\break
Even if you would have said you lost. }
 
\par{3. お ${\overset{\textnormal{}}{\text{母}}}$ さんにどうぞよろしくって伝えてくださいね。 \hfill\break
Please say nice to meet you to her mother. }
 
\par{${\overset{\textnormal{}}{\text{4. 田中}}}$ さんって ${\overset{\textnormal{}}{\text{知}}}$ ってる? \hfill\break
Do you know Mr. Tanaka? }
 
\par{${\overset{\textnormal{}}{\text{気}}}$ にしなくてもいいってことだ。 \hfill\break
It's nothing you may need to be worried about. }

\par{ "\dothyp{}\dothyp{}\dothyp{}の\dothyp{}\dothyp{}\dothyp{}の(と)" is used to list things in contrast or emphatically and "\dothyp{}\dothyp{}\dothyp{}の\dothyp{}\dothyp{}\dothyp{}ないの(って)" shows excess. As is noted, sometimes using it may be uncommon or even old-fashioned. }

\par{5. ${\overset{\textnormal{なん}}{\text{何}}}$ のかんのと ${\overset{\textnormal{もんく}}{\text{文句}}}$ をつける。 \hfill\break
To complain of what and that. }
 
\par{6. 死ぬの生きるのと ${\overset{\textnormal{おおさわ}}{\text{大騒}}}$ ぎだ。 \hfill\break
Dying and living is all turmoil. }
 
\par{7. うるさいのなんのって、 ${\overset{\textnormal{}}{\text{耳}}}$ が聞こえなくなったほどだ。 \hfill\break
It's so loud to the point I've lost my hearing. }

\par{8. ${\overset{\textnormal{むしょく}}{\text{無職}}}$ になるの、 ${\overset{\textnormal{りこん}}{\text{離婚}}}$ するの、 ${\overset{\textnormal{さんざん}}{\text{散々}}}$ な目にあったな。(Not so common) \hfill\break
Becoming unemployed and divorced, I've met a lot of terrible things hasn't it! }
 
\par{9. 寒いの暑いのと ${\overset{\textnormal{}}{\text{言}}}$ っていないで、ジョギングは ${\overset{\textnormal{まいあさ}}{\text{毎朝}}}$ しなさい。 \hfill\break
Don't complain about it being hot or cold, you must go jogging every day. }

\par{10. ${\overset{\textnormal{いた}}{\text{痛}}}$ いの ${\overset{\textnormal{}}{\text{痛}}}$ くないのと、 ${\overset{\textnormal{と}}{\text{飛}}}$ び ${\overset{\textnormal{あ}}{\text{上}}}$ がってしまったよ。(古い言い方) \hfill\break
It hurt so bad I accidentally jumped up. }
 
\par{\textbf{The Adverbial Particle }\textbf{って: Emphatic }}
 
\par{ It takes up the subject with slight exclamation and is equivalent to というのは. It's also used in repeating what someone said. }
 
\par{11. オレって、なんてバカなんだ。 \hfill\break
I'm such an idiot. }
 
\par{${\overset{\textnormal{}}{\text{12. 誰}}}$ かに ${\overset{\textnormal{}}{\text{出会}}}$ える ${\overset{\textnormal{}}{\text{時}}}$ っていつもこうだ。 \hfill\break
Meeting someone is always the same way. }
 
\par{13. したくないって、どういう ${\overset{\textnormal{}}{\text{意}}}$ ${\overset{\textnormal{}}{\text{味}}}$ だ。 \hfill\break
You don't want to do it, what do you mean? }
 
\par{${\overset{\textnormal{}}{\text{14. 誰}}}$ が\{殺人・ ${\overset{\textnormal{}}{\text{殺人者・人殺し・犯人\}}}}$ かって、あいつに ${\overset{\textnormal{}}{\text{決}}}$ まってるよ。 \hfill\break
It's settled that who the, um, murderer is that guy over there. }
 
\par{\textbf{Word Note }: ${\overset{\textnormal{}}{\text{殺人者}}}$ is a rather formal word and not used much in actual conversation, although it is common in books. ${\overset{\textnormal{}}{\text{人殺}}}$ し is a graphic word that bring images of the bloody crime scene. So, it is considered a sensitive word and often avoided. ${\overset{\textnormal{}}{\text{犯人}}}$ , although it only means "criminal", it is the most likely word to use in this situation, and most people would realize automatically that the sentence is about murder anyways. }
 
\par{\textbf{The Final Particle }\textbf{って: All of the Above }}
 
\par{15. すぐ ${\overset{\textnormal{}}{\text{帰}}}$ れってさ。 \hfill\break
He said like to hurry home. }
 
\par{${\overset{\textnormal{}}{\text{16. 間}}}$ に ${\overset{\textnormal{}}{\text{合}}}$ うだろうって。 \hfill\break
They said they would probably make it on time. }
 
\par{With a high intonation, it shows confirmation. }
 
\par{${\overset{\textnormal{}}{\text{17. 離婚}}}$ するんですって。 \hfill\break
You're getting divorced? }
 
\par{${\overset{\textnormal{}}{\text{18. 病気}}}$ だったんですってねー。 \hfill\break
It was sickness, right? }
 
\par{ In a downward intonation, it passes down an assertion strongly. It is often used to (re)assure someone like in Ex. 19. }
 
\par{${\overset{\textnormal{}}{\text{19. 大丈夫}}}$ だって。 \hfill\break
It's OK. }
 
\par{20. きっと ${\overset{\textnormal{}}{\text{合格}}}$ するって。 \hfill\break
I'll surely pass. }
 
\par{\textbf{The Conjunctive Particle }\textbf{って }}
 
\par{1. When used with ~た. }
 
\begin{itemize}
 
\item Even if you; no matter how.  
\item っていい, ってかまわない,      and って ${\overset{\textnormal{}}{\text{差}}}$ し ${\overset{\textnormal{つか}}{\text{支}}}$ えない      to show permission.  
\end{itemize}
 
\par{21a. どう ${\overset{\textnormal{}}{\text{叫}}}$ んだって、 ${\overset{\textnormal{}}{\text{聞}}}$ こえない。(Less common) \hfill\break
21b. どれほど ${\overset{\textnormal{}}{\text{叫}}}$ んだって ${\overset{\textnormal{}}{\text{聞}}}$ こえない。 \hfill\break
Even if you somehow shout, he can't hear. }
 
\par{22. どう ${\overset{\textnormal{}}{\text{見}}}$ えたって、 ${\overset{\textnormal{}}{\text{校長}}}$ は ${\overset{\textnormal{}}{\text{校長}}}$ だ。 \hfill\break
No matter how you look at it, the principal is the principal. }
 
\par{23. やったっていいじゃん。 \hfill\break
Isn't it OK to do it? }
 
\par{\textbf{Grammar Note }: When used with ん, って is often changed to て. It's also common as ってー. }
      
\section{The Particle たって}
 
\par{ For the most part, たって is ~た + って. However, there is one usage of たって that makes it unique. }

\par{The Conjunctive Particle たって }

\par{Often with a ${\overset{\textnormal{そくおん}}{\text{促音}}}$ , it is used as a colloquial variant of the phrase といっても meaning "even if you say that". There are several set expressions that accompany this. }

\begin{ltabulary}{|P|P|}
\hline 

Tense & Meaning \\ \cline{1-2}

Non-past & Even though (pronoun)\dothyp{}\dothyp{}\dothyp{} \\ \cline{1-2}

Past & Even if you (past tense verb)\dothyp{}\dothyp{}\dothyp{} \\ \cline{1-2}

Volitional & Even if (pronoun) going to\dothyp{}\dothyp{}\dothyp{} \\ \cline{1-2}

Adjectives & Even if you\dothyp{}\dothyp{}\dothyp{} \\ \cline{1-2}

\end{ltabulary}

\par{24. ハワイに行こうったって、どこも ${\overset{\textnormal{}}{\text{満員}}}$ だよ。 \hfill\break
Even if we're going to Hawaii, everywhere is full. }
 
\par{${\overset{\textnormal{}}{\text{25. 無}}}$ くったって、 ${\overset{\textnormal{}}{\text{生}}}$ きてはゆけるだろう? \hfill\break
Even if you don't have it, wouldn't you still go on living? }

\par{26. そうだと ${\overset{\textnormal{}}{\text{思}}}$ ったとしても、そうと ${\overset{\textnormal{}}{\text{決}}}$ まったわけじゃないだろ。 \hfill\break
Even if you thought that, that doesn't mean that it's supposed to still occur that way you know. }

\par{\textbf{Grammar Note }: Of course, there are some situations in which たって would be unnatural. }
      
\section{The Particle だって}
 
\par{ At times だって is merely a combination of だ and って or the voiced version of the particle だって. However, だって is also seen in unique situations. }
 
\par{\textbf{The Adverbial Particle }\textbf{だって }}
 
\par{1. It presents something with an added feeling of rebuttal in response to an assumption and is equivalent to もまた--"even (does)". }
 
\par{${\overset{\textnormal{}}{\text{27. 犬}}}$ だってそれくらい ${\overset{\textnormal{}}{\text{分}}}$ かるだろう!? \hfill\break
Even a dog should understand that!? }
 
\par{${\overset{\textnormal{}}{\text{28. 僕}}}$ だって ${\overset{\textnormal{つら}}{\text{辛}}}$ いのだ。 \hfill\break
Even I'm bitter (about it). }

\par{29. ${\overset{\textnormal{こじき}}{\text{乞食}}}$ にだって ${\overset{\textnormal{}}{\text{言}}}$ い ${\overset{\textnormal{ぶん}}{\text{分}}}$ はある。 \hfill\break
Even a beggar has one's say. }

\par{2. "AだってBだって" arranges like things with the hint that other things are applicable as well. }
 
\par{30. 学校だって、 ${\overset{\textnormal{}}{\text{大学}}}$ だって、 ${\overset{\textnormal{}}{\text{同}}}$ じところじゃないか。 \hfill\break
Isn't it the same place, whether it's school or college? }

\par{31. ペットを飼うなら犬だって猫だって同じようなもんじゃないか。 \hfill\break
If you're going to raise a pet, isn't a dog or cat the same? }

\par{32. イルカだってクジラだって哺乳類なんだよ。 \hfill\break
Whether a dolphin or a whale, it's still a mammal. }

\par{33. 子どもだって大人だって遊びたいときは思いっきり遊ぶんだよ。 \hfill\break
Whether a kid or an adult, when people want to play, they play with all their heart. }
 
\par{${\overset{\textnormal{}}{\text{34. 宝石}}}$ だって ${\overset{\textnormal{}}{\text{着物}}}$ だって ${\overset{\textnormal{}}{\text{買}}}$ う。 \hfill\break
To buy things whether it is jewelry or clothing. }
 
\par{3. With a feeling of rebuttal and attached to a phrase showing high value, it strengthens the meaning of a sentence and is equivalent to さえも meaning "even also". }
 
\par{35. うそ(だ)、 ${\overset{\textnormal{}}{\text{決定的}}}$ な ${\overset{\textnormal{}}{\text{証拠}}}$ だってあるぞ。 \hfill\break
Lie, there's even definitive evidence. }
 
\par{36. クラスにはピアニストだっているよ。 \hfill\break
There's even a pianist in the classroom. }
 
\par{4. Attached to a phrase showing the "smallest" of something and followed by a negative expression, it is used to show complete negation. }
 
\par{37. いとこにとってはこんな ${\overset{\textnormal{}}{\text{大学}}}$ は ${\overset{\textnormal{}}{\text{一日}}}$ だって ${\overset{\textnormal{がまん}}{\text{我慢}}}$ できないだろう。 \hfill\break
My cousin wouldn't last in this kind of a college for even a day. }

\par{38. ${\overset{\textnormal{わず}}{\text{僅}}}$ かだって ${\overset{\textnormal{}}{\text{許}}}$ されない。 \hfill\break
Not even a little bit will be forgiven. }
 
\par{5. Attached to an interrogative, it is used to mean "all without exception", and is equivalent to "interrogative + でも". }
 
\par{39. いつだっていいさ。 \hfill\break
Anytime is good. }
 
\par{${\overset{\textnormal{}}{\text{40. 誰}}}$ だってそんなこと\{は・を\}したく(は)ないね。 \hfill\break
Nobody wants to do a thing like that. }
 
\par{6. ${\overset{\textnormal{}}{\text{何}}}$ だって asks for a reason with a feeling of criticism--why". }
 
\par{${\overset{\textnormal{}}{\text{41. 何}}}$ だってそんなことをしたんだ。 \hfill\break
Why did you do even such a thing as that? }
 
\par{\textbf{The Final Particle }\textbf{だって }}
 
\par{1. Directly quotes the words of someone of which one thought to be unsuitable and is equivalent to だと. }
 
\par{${\overset{\textnormal{}}{\text{42. 彼}}}$ は、 ${\overset{\textnormal{}}{\text{金}}}$ を ${\overset{\textnormal{}}{\text{貸}}}$ さなかったんだって。 \hfill\break
He said, I didn't lend any money. }
 
\par{${\overset{\textnormal{}}{\text{43. 知}}}$ らなかったんだって。 \hfill\break
You didn't know? }
 
\par{${\overset{\textnormal{}}{\text{44. 遊}}}$ びたくないだって? \hfill\break
You don't want to play? }
 
\par{\textbf{Phrase Note }: This usage is a contraction of だとて. }
 
\par{2. のだって quotes what someone said. }
 
\par{45. あの ${\overset{\textnormal{}}{\text{田中}}}$ さん、 ${\overset{\textnormal{い}}{\text{行}}}$ きたくないんだって。 \hfill\break
Mr. Tanaka says he don't want to go. }
 
\par{46. あいつは ${\overset{\textnormal{}}{\text{来}}}$ るんだって。 \hfill\break
He's coming? }
 
\par{3. Asks about the information of something passed down when attached to a phrase of question or doubt. }
 
\par{${\overset{\textnormal{}}{\text{47. 何}}}$ だって? \hfill\break
What was it? }
 
\par{${\overset{\textnormal{}}{\text{48. 食費}}}$ はいくらって? \hfill\break
How much was the food expenses? }
 
\par{\textbf{Variant Note }: だって may be changed to ですって in polite context. }
    