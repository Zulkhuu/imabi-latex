    
\chapter{むしろ, かねて, さもないと, 況して\slash 況や, たとえ, 強いて, 敢えて, 一旦, \& 殊更}

\begin{center}
\begin{Large}
第297課: むしろ, かねて, さもないと, 況して\slash 況や, たとえ, 強いて, 敢えて, 一旦, \& 殊更 
\end{Large}
\end{center}
 
\par{  In this lesson we will focus on some important adverbs that deserve special attention. }

\begin{ltabulary}{|P|P|P|P|P|P|}
\hline 

むしろ & かねて & さもないと & たとえ & 一旦 & 強いて、敢えて、殊更 \\ \cline{1-6}

\end{ltabulary}
      
\section{むしろ}
 
\par{ むしろ means "rather". It can be seen in the following patterns. }

\begin{ltabulary}{|P|P|}
\hline 

B (という)より(も)むしろA & A rather than B \\ \cline{1-2}

AというかむしろB & A\dothyp{}\dothyp{}\dothyp{},no, B \\ \cline{1-2}

Bというよりもむしろ & not so much B as A \\ \cline{1-2}

\end{ltabulary}

\par{ The segments of the sentence should show parallelism. When you use むしろ with -たい, the translation becomes "prefer" or "would rather". Instead of むしろ, you may see かえって and いっそ which mean "on the contrary" and "better yet\slash still more" respectively. }
 
\par{${\overset{\textnormal{}}{\text{1. 彼女}}}$ は ${\overset{\textnormal{}}{\text{歌手}}}$ というよりもむしろ ${\overset{\textnormal{}}{\text{女優}}}$ です。 \hfill\break
She is an actress rather than a singer. }
 
\par{${\overset{\textnormal{}}{\text{2. 僕}}}$ は ${\overset{\textnormal{}}{\text{外車}}}$ よりもむしろ日本製の ${\overset{\textnormal{}}{\text{車}}}$ を ${\overset{\textnormal{}}{\text{買}}}$ うんだ。 \hfill\break
I buy Japanese-made cars rather than foreign cars. }
 
\par{3. えー、あの ${\overset{\textnormal{}}{\text{子供}}}$ は ${\overset{\textnormal{}}{\text{利口}}}$ というか、いや、むしろ ${\overset{\textnormal{}}{\text{天才}}}$ というべきだ。 \hfill\break
Eh, that kid is bright, no, a genius. }
 
\par{4. あまり ${\overset{\textnormal{}}{\text{好}}}$ きじゃなかった。それどころかむしろ ${\overset{\textnormal{にく}}{\text{憎}}}$ んでさえいた。 \hfill\break
I didn't really like it. Rather, I just detested it. }
 
\par{${\overset{\textnormal{}}{\text{5. 辛}}}$ い ${\overset{\textnormal{}}{\text{思}}}$ いをするよりはいっそ ${\overset{\textnormal{}}{\text{死}}}$ んだ ${\overset{\textnormal{}}{\text{方}}}$ がましだ。 \hfill\break
I would rather die than hold (those) painful memories. }
 
\par{${\overset{\textnormal{}}{\text{6. 海岸}}}$ へ ${\overset{\textnormal{}}{\text{行}}}$ くよりむしろ ${\overset{\textnormal{}}{\text{遊園地}}}$ へ ${\overset{\textnormal{}}{\text{行}}}$ きたい。 \hfill\break
I'd rather go to the amusement part rather than go to the beach. }

\par{7. パーティーに行くよりむしろ家にいたい。 \hfill\break
I'd rather stay inside than go to the party. }
 
\par{8. むしろこちらのほうがいいと ${\overset{\textnormal{}}{\text{思}}}$ います。 \hfill\break
I think rather that this would be good. }

\par{9. ${\overset{\textnormal{かえ}}{\text{却}}}$ って ${\overset{\textnormal{}}{\text{良}}}$ かった。 \hfill\break
It was rather good. }

\par{\textbf{Orthography Note }: むしろ can be rarely seen written in 漢字 as 寧ろ. }
      
\section{かねて}
 
\par{ かねて means "previously". かねてから means "has always been" and is used with the progressive past tense. ${\overset{\textnormal{か}}{\text{兼}}}$ ね ${\overset{\textnormal{が}}{\text{兼}}}$ ね is another adverb that may be used instead of かねて. You should not replace かねてから with かねがねから. Another equivalent phrase is ${\overset{\textnormal{}}{\text{前以}}}$ て. }
 
\par{${\overset{\textnormal{}}{\text{10. 以前}}}$ から ${\overset{\textnormal{}}{\text{痛}}}$ んでいた。 \hfill\break
It has always hurt. }
 
\par{${\overset{\textnormal{}}{\text{11. 兼}}}$ ねて(から)お ${\overset{\textnormal{}}{\text{伝}}}$ えしたように \hfill\break
As we have informed you before }
 
\par{${\overset{\textnormal{}}{\text{12. 前以}}}$ て ${\overset{\textnormal{}}{\text{備}}}$ える。 \hfill\break
To prepare beforehand. }
 
\par{${\overset{\textnormal{}}{\text{13. 次回}}}$ は ${\overset{\textnormal{}}{\text{前以}}}$ て ${\overset{\textnormal{}}{\text{電話}}}$ してくださいませんか。 \hfill\break
Could you please phone ahead next time? }
 
\par{14. あらかじめ ${\overset{\textnormal{}}{\text{聞}}}$ き ${\overset{\textnormal{}}{\text{及}}}$ んだ ${\overset{\textnormal{}}{\text{名所}}}$ を ${\overset{\textnormal{}}{\text{訪}}}$ れればよい。 \hfill\break
It would be good for you to visit the name and address you heard of before. }
 
\par{15. お ${\overset{\textnormal{}}{\text{名前}}}$ はかねて(から) ${\overset{\textnormal{}}{\text{伺}}}$ っております。(Honorific) \hfill\break
I have heard your name before. }
 
\par{16. かねがね ${\overset{\textnormal{}}{\text{望}}}$ んでいたことが ${\overset{\textnormal{}}{\text{本当}}}$ に ${\overset{\textnormal{}}{\text{実現}}}$ した。 \hfill\break
The thing that I have been hoping for actually happened! }
 
\par{17. お ${\overset{\textnormal{}}{\text{噂}}}$ はかねがね\{伺って・ ${\overset{\textnormal{}}{\text{承}}}$ って\}おりました。(Honorific) \hfill\break
I have already been told about the rumor. }
 
\par{\textbf{Orthography Note }: かねて can seldom be seen written in 漢字 as 予て. }
      
\section{さもないと}
 
\par{  さもないと in a literal sense means "if you do not do so", but it is normally simply translated as "or (else)\slash otherwise". It may also be seen as さもなくば and さもなければ. }
 
\par{${\overset{\textnormal{}}{\text{18. 出}}}$ て ${\overset{\textnormal{}}{\text{行}}}$ け。さもないと ${\overset{\textnormal{}}{\text{警察}}}$ (を) ${\overset{\textnormal{}}{\text{呼}}}$ ぶぞ。 \hfill\break
Get out. If you don't, I'll call the police! }
 
\par{${\overset{\textnormal{}}{\text{19. 7時半}}}$ までに ${\overset{\textnormal{}}{\text{出}}}$ なさい。さもなければ、 ${\overset{\textnormal{}}{\text{電車}}}$ に ${\overset{\textnormal{}}{\text{遅}}}$ れますよ。 \hfill\break
Leave (the house) by 7:30. Or else, you'll be late to your train. }
 
\par{20. そうせよ!さもなくば ${\overset{\textnormal{}}{\text{必}}}$ ずや命を ${\overset{\textnormal{}}{\text{失}}}$ おう。(Old-fashioned) \hfill\break
Do it! If not you will surely lose your life. }
 
\par{${\overset{\textnormal{}}{\text{21. 水}}}$ を ${\overset{\textnormal{}}{\text{溜}}}$ めよ。さもないと、 ${\overset{\textnormal{かんばつ}}{\text{旱魃}}}$ が ${\overset{\textnormal{}}{\text{起}}}$ こったときに、 ${\overset{\textnormal{のど}}{\text{喉}}}$ が ${\overset{\textnormal{かわ}}{\text{渇}}}$ いて ${\overset{\textnormal{}}{\text{死んでしまうよ}}}$ 。 \hfill\break
Save up water. Or else, if a drought were to occur, you could dehydrate and end up near to your death. }
      
\section{況して・況や}
 
\par{  ${\overset{\textnormal{ま}}{\text{況}}}$ して(や) and ${\overset{\textnormal{いわん}}{\text{況}}}$ や mean "much less", and they must be used with the negative. It is normally written in かな. }
 
\par{22. あいつは ${\overset{\textnormal{}}{\text{日本語}}}$ を ${\overset{\textnormal{}}{\text{読}}}$ むことすらできない。ましてや ${\overset{\textnormal{}}{\text{書いたり}}}$ ${\overset{\textnormal{}}{\text{話したりできるわけがない。}}}$  \hfill\break
He can't even read Japanese, much less write it or speak it. }
 
\par{${\overset{\textnormal{}}{\text{23. 小走}}}$ りもろくにできない。 ${\overset{\textnormal{ま}}{\text{況}}}$ して ${\overset{\textnormal{}}{\text{走}}}$ れるわけがない。 \hfill\break
She can hardly jog, much less being able to run. }
 
\par{24. 先生「今回のテストの平均点は60点でした。」 \hfill\break
Aくん「お前70点以上いった?」 \hfill\break
Bくん「まさか。平均点すら届かなかったよ。ましてや70点なんて夢のまた夢だよ」 \hfill\break
Sensei: "The average for this test was 60" \hfill\break
A-kun: "Did you get over a 70?" \hfill\break
B-kun: "No way, I couldn't even make the average, much less get a 70 which is like a hopeless dream" }

\par{${\overset{\textnormal{}}{\text{25. 敵}}}$ でも ${\overset{\textnormal{}}{\text{困}}}$ っていたら ${\overset{\textnormal{}}{\text{助}}}$ けます。ましてや ${\overset{\textnormal{}}{\text{味方}}}$ なら ${\overset{\textnormal{}}{\text{当然}}}$ です。 \hfill\break
I would help an enemy if her were to be in distress, much more a friend. }
 
\par{26. いわんや ${\overset{\textnormal{}}{\text{子供}}}$ には ${\overset{\textnormal{}}{\text{無理}}}$ だ。 \hfill\break
It\textquotesingle s useless much less with kids. }
      
\section{たとえ}
 
\par{ In 漢字 as 仮令 and also seen as たとい in older texts, たとえ is used in the following patterns to mean "no matter" and "even though\slash so". It may also be seen in "たとえ\dothyp{}\dothyp{}\dothyp{}たとしても" which means "if any\slash if I should". }

\begin{ltabulary}{|P|P|P|}
\hline 

たとえ\dothyp{}\dothyp{}\dothyp{}ても & たとえ\dothyp{}\dothyp{}\dothyp{}volitional + と(も) & たとえ\dothyp{}\dothyp{}\dothyp{}にせよ \\ \cline{1-3}

\end{ltabulary}

\par{27. たとえ ${\overset{\textnormal{}}{\text{雨}}}$ が ${\overset{\textnormal{}}{\text{降}}}$ っても ${\overset{\textnormal{}}{\text{出}}}$ かけるぞ。 \hfill\break
I'll go out even if rains. }
 
\par{28. あなたがたとえ ${\overset{\textnormal{}}{\text{何}}}$ をしようとも ${\overset{\textnormal{}}{\text{支持}}}$ しよう。 \hfill\break
I will support you no matter how you do. }
 
\par{29. たとえ ${\overset{\textnormal{}}{\text{何}}}$ があろうとやります。 \hfill\break
I'll do it no matter what. }
 
\par{${\overset{\textnormal{}}{\text{30a. 両者}}}$ の ${\overset{\textnormal{}}{\text{間}}}$ には、たとえあったとしてもごく ${\overset{\textnormal{わず}}{\text{僅}}}$ かしか、 ${\overset{\textnormal{}}{\text{相違}}}$ はありません。(Literary) \hfill\break
30b. 両者の間には、たとえあったとしてもごく僅かしか、 ${\overset{\textnormal{}}{\text{違}}}$ いはありません。(Spoken) \hfill\break
There is little if any difference between the two. }
 
\par{31. たとえ ${\overset{\textnormal{}}{\text{冗談}}}$ にせよ、 ${\overset{\textnormal{}}{\text{傷つ}}}$ けるようなことはいうべきじゃないよ。 \hfill\break
Even if it's just a joke, you should not say hurtful things. }
 
\par{32. たとえ ${\overset{\textnormal{}}{\text{殺}}}$ されても ${\overset{\textnormal{}}{\text{信念}}}$ は ${\overset{\textnormal{}}{\text{曲}}}$ げられない。 \hfill\break
Even if if I were killed, my faith could not be taken away. }
 
\par{33. たとえどんなことがあったとしても \hfill\break
No matter what }
 
\par{34. たとえどんなに ${\overset{\textnormal{}}{\text{腕}}}$ がよくても、 ${\overset{\textnormal{}}{\text{試験}}}$ に(は) ${\overset{\textnormal{}}{\text{合格}}}$ しなければならないよ。 \hfill\break
No matter how good you are, you must still pass the exam. }
 
\par{\textbf{Word Note }: たとえ may also be replaced by ${\overset{\textnormal{よ}}{\text{縦}}}$ しんば. This word, though, is rare. }
 
\par{${\overset{\textnormal{}}{\text{35a. 縦}}}$ しんば ${\overset{\textnormal{}}{\text{失敗}}}$ したとしても ${\overset{\textnormal{}}{\text{後悔}}}$ はしない。 \hfill\break
35b. たとえ失敗したとしても後悔はしない。(More natural) \hfill\break
35c. もし失敗したとしても後悔はしない。 \hfill\break
Even if I were to lose, I will have no regrets. }
      
\section{一旦}
 
\par{ 一旦 can either mean "temporarily; for the present" or "once". In the first case it shows that some action or progress is currently halted. In the second case it is used at the start of a conditional clause to show an action that is necessary to bring about a certain result. }
 
\par{36. いったん ${\overset{\textnormal{}}{\text{始}}}$ めたらやめてはいけません。 \hfill\break
Once you've begun, you mustn't quit. }
 
\par{${\overset{\textnormal{}}{\text{37. 一旦}}}$ は ${\overset{\textnormal{}}{\text{中止}}}$ と ${\overset{\textnormal{}}{\text{決}}}$ まっていました。 \hfill\break
It has been decided that we will halt temporarily. }
 
\par{${\overset{\textnormal{}}{\text{38. 一旦}}}$ の ${\overset{\textnormal{}}{\text{楽}}}$ しみ \hfill\break
A fun time }
 
\par{\textbf{Pronunciation Note }: As a noun it can be used in the sense of "once" as in "a time". The pitch when it's a noun is on いった. When it's an adverb, the pitch is on ったん. }
      
\section{強いて、敢えて、殊更}
 
\par{  ${\overset{\textnormal{し}}{\text{強}}}$ いて, ${\overset{\textnormal{あ}}{\text{敢}}}$ えて, and ${\overset{\textnormal{ことさら}}{\text{殊更}}}$ are very similar phrases that need to be addressed together. 強いて is like "by force". This makes sense because it comes from the verb 強いる which means "to force". 敢えて is "to dare\slash venture". Lastly, 殊更 is "intentionally". }
 
\par{${\overset{\textnormal{}}{\text{39. 強}}}$ いて ${\overset{\textnormal{}}{\text{言}}}$ えば \hfill\break
If I'm forced to say something }
 
\par{${\overset{\textnormal{}}{\text{40. 強}}}$ いてと ${\overset{\textnormal{}}{\text{仰}}}$ るなら \hfill\break
If you insist }
 
\par{${\overset{\textnormal{}}{\text{41. 敢}}}$ えてもう ${\overset{\textnormal{}}{\text{一度}}}$ やる覚悟があるのか? \hfill\break
You dare have the courage to try again? }
 
\par{${\overset{\textnormal{}}{\text{42. 敢}}}$ えて ${\overset{\textnormal{}}{\text{反対意見}}}$ を ${\overset{\textnormal{}}{\text{述}}}$ べた。 \hfill\break
He ventured an objection. }
 
\par{${\overset{\textnormal{}}{\text{43. 私}}}$ は ${\overset{\textnormal{}}{\text{行}}}$ くことを ${\overset{\textnormal{}}{\text{強}}}$ いられました。 \hfill\break
I was compelled to go. }
 
\par{44. \{あえて・強いて\} ${\overset{\textnormal{}}{\text{送}}}$ ってくれなくてもいいよ。 \hfill\break
You don't have to bother driving me home. }
 
\par{${\overset{\textnormal{}}{\text{45. 大統領}}}$ をあえて ${\overset{\textnormal{}}{\text{非難}}}$ するつもりはないです。 \hfill\break
I wouldn't dare criticize the president. }
 
\par{${\overset{\textnormal{}}{\text{46. 彼}}}$ は\{ ${\overset{\textnormal{}}{\text{殊更・わざと\}嫌}}}$ がらせをしていた。 \hfill\break
He was intentionally harassing others. }
    