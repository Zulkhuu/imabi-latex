    
\chapter{Nominal Phrases II}

\begin{center}
\begin{Large}
第287課: Nominal Phrases II: 相応, 代り, \& 経由 
\end{Large}
\end{center}
       
\section{~相応}
 
\par{ 相応 means "suitable\slash fit\slash just" and follows other nouns to show the fitting nature of something. Translation can sometimes be tricky. }
 
\par{1. 彼女は年齢相応に見えない。 \hfill\break
She does not look her age. }
 
\par{\textbf{Meaning Note }: The above phrase is in reference to maturity and not appearance. }
 
\par{2. 政府は収入不相応の支出をすべきではない。 \hfill\break
The government should not spend beyond its means. }
 
\par{3. それ相応の賞賛を受ける。 \hfill\break
To receive deserved praise. }
      
\section{代り}
 
\par{ 代わり means "substitute", お代わり means "seconds", and の代わりに means "instead". }
 
\par{4. 彼は鋸の代わりに剣を使ってみた。 \hfill\break
He tried to use a sword as a substitute for a saw. }
 
\par{5. 私は市長の代わりに挨拶をしました。 \hfill\break
I greeted them in replacement of the mayor. }
 
\par{6. お代わりをください。 \hfill\break
Seconds please. }
 
\par{7. 代わりに誰か来るんですか。 \hfill\break
Is anyone else coming? }
 
\par{8. 代わりに電話に出てよ。 \hfill\break
Answer the telephone for me. }
 
\par{9. お辞儀の代わりに、アメリカ人は握手をするという。 \hfill\break
They say that Americans shake hands instead of bowing. }
 
\par{10. 誰に彼の代わりができるだろうか。 \hfill\break
Who could possibly take his place? }
 
\par{11. 岩がベッド代わりになった。 \hfill\break
The rock turned into a substitute for a bed. }
 
\par{12a. お金は健康の代わりというわけではありません。 \hfill\break
12b. 健康はお金には代えられない。(Better) \hfill\break
It's not to say that money is a substitute for health. }
 
\par{13. 代わりにあなたに行ってほしい。 \hfill\break
I wish that you would go instead. }
      
\section{経由}
 
\par{ 経由 either shows how one transits or does  something "via" or "by way of" something. So, it either refers to travel  or method. }

\par{14. バンコクを経由してインドへ行く。 \hfill\break
Go to India via Bangkok. }

\par{15. パリ経由でアテネへ飛ぶ。 \hfill\break
Fly to Athens via Paris. }

\par{16. 彼女は香港経由で帰国しました。 \hfill\break
She returned to her home country via Hong Kong. }

\par{17. ベーリング海峡を経由してアジアから北アメリカへ渡ったといわれている。 \hfill\break
It is said that (they) crossed over to North America from Asia by way of the Bering Strait. }

\par{18. 議案は審議会を経由して上程される。 \hfill\break
For a measure to be introduced via the inquiry commission. }

\par{19. このバスは六本木経由の東京駅行きですか。 \hfill\break
Does this bus go to Tokyo Station via Roppongi? }

\par{\textbf{Geography Note }: Roppongi is a district of Tokyo. It houses many foreign embassies, and holds the Roppongi Hills which is a a mega-complex with office space, apartments, shops, restaurants,  cafés, movie theaters, museum, hotel, TV studio, etc. }
    