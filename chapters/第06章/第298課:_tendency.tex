    
\chapter{Tendency}

\begin{center}
\begin{Large}
第298課: Tendency: 嫌いがある, が早いか, が最後, \& そばから 
\end{Large}
\end{center}
 
\par{ In this lesson we will learn about speech modals concerning tendency. }
      
\section{嫌いがある}
 
\par{ 嫌い means "to hate"--it is treated as a 形容動詞 instead of as a verb in Japanese--and is followed by がある to show what someone or something has a \textbf{tendency of doing }. As the word suggests, this is not a good tendency. This tendency has to be somewhat bad, so it doesn't work with phrases like ぎりぎり間に合う. }

\par{1. 依存症の嫌いがあること \hfill\break
To have a tendency of dependence. }

\par{2. その生徒は遅刻する嫌いがありました。 \hfill\break
That student had a tendency to be late. }

\par{3. 彼は酒を飲みすぎる嫌いがある。 \hfill\break
He has a tendency of drinking too much sake. }

\par{4. 人には自分の聞きたくないことは耳に入れないというきらいがあるのではないか。 \hfill\break
Isn't there are tendency that what one doesn't want to hear doesn't get heard? }

\par{5. 彼女は優しいですが、無意識に人を傷付けることを言うきらいがある。 \hfill\break
She's nice but, she has a tendency to say things that hurt people without knowing it. }

\par{6. 人間は年を取ると、他の人の話を聞かなくなる嫌いがある。 \hfill\break
When humans get older, they have a tendency of not wanting to listen to other people.  }
      
\section{が早いか}
 
\par{ が早いか is best translated as "the instant". Basically the same as the noun 瞬間(instant). The situation in question \textbf{must not be of one's control }. It has also become limited to actions done by people. The actions also have to be logically related to each other. が早いか can be used with the non-past or the past tense. However, the former is the most common and some speakers may not like it with the past tense. }

\par{7. 彼女は先生を見るが早いか、逃げ出しました。 \hfill\break
She started running away the instant she saw the teacher. }

\par{8. 「飲んでみよう」と言ったが早いか、彼は全て飲んでしまった。 \hfill\break
The instant I said, "Let's try drinking it", he completely drank it all. }

\par{9. 言うが早いかやってしまった。 \hfill\break
No sooner said than done. }

\par{10. 弟は、口に押し込んだが早いか、玄関を出ていった。 (△) \hfill\break
My brother left the house just as soon as he shoved it down in his mouth. }
11. 初級者向けのIMABIの教科書は店頭に並べられたが早いか、すぐに売れていきました。 \hfill\break
The IMABI textbooks for beginners flew off the shelves the instant they were put up in front of the store. 
\par{12. ベルを聞くが早いか、生徒達は教室から飛び出していった。 \hfill\break
The instant they heard the bell, the students dashed out of the classroom. }

\par{ The verb must be one that describes something of an instant. It's not a verb that entails a longer period of time. When you see, ask, arrive, etc. that is 瞬間的. The part of the sentence that comes after is unexpected. Again, this pattern must not be used to show the speaker's wants or intentions. }
      
\section{が最後}
 
\par{ Although 最後 means "last", が最後 shows that once some starts doing something, there is no end to their action in sight. Due to this, it is very similar to 一旦(いったん) meaning "once". However, 一旦 doesn't always have the implications of nonstop. }

\par{13. 飲みだしたが最後、止められなかった。 \hfill\break
Once I started to drink it, I couldn't stop. }

\par{14. 入ったが最後二度と出てこられなくなる場所に行ったことがありますか。 \hfill\break
Have you ever gone to a place where once you entered, you became unable to leave (because it was so awesome)? }

\par{15. いったん始めたら、止めてはならない。 \hfill\break
Once you start, you must not quit. }

\par{16. いったん封を切ると、返品できません。 \hfill\break
Once you break the seal, you can't return it. }
      
\section{そばから}
 
\par{ そばから describes an event that repeatedly happens right over again, plain and simple. So, no matter what you do, right when you do it, something messes it up, and this cycle keeps repeating itself. So, it's usually a bad thing. It's usually used with the non-past form of a verb, but the past tense is also acceptable. }

\par{17. 返事を書いているそばからランスさんが次々とメールを送ってくる。まあ、日本語の勉強に非常に熱心ですね。 \hfill\break
Every moment I write a response, but each instant Lance sends a new e-mail one after another. Well, he's very earnest in his Japanese studies. }

\par{18. 妹が掃除するそばから散らかすから、もう諦めたくなりました。 \hfill\break
My little sister messes the room up (repeatedly) as soon as I clean it up, so I've become tired of doing it. }

\par{19. IMABIは漢字が多くて大変ですね。調べたそばから新しい漢字がいつも出てきます。もっとシンプルな漢字でだけ書かれていたらいいのに。 \hfill\break
It's hard that IMABI has a lot of Kanji, isn't it? The moment I look one up, new Kanji always show up. If   only it were written with just simple Kanji. }

\par{20. 習うそばから忘れてしまうので、ぜんぜん新しい単語が覚えられない。 \hfill\break
Since as soon as I learn it I forget it, I can't learn any new words. }
    