    
\chapter{Simultaneous Action II}

\begin{center}
\begin{Large}
第289課: Simultaneous Action II: The Particle つつ (VS ながら) 
\end{Large}
\end{center}
 
\par{ つつ is a somewhat old-fashioned particle that has less function in this generation, but it is still very important as it lives strong in literature. In this lesson, we will investigate how to use this particle correctly and know exactly when it is appropriate and how it differs with things interchangeable with it. }
      
\section{The Conjunctive Particle つつ}
 
\par{ Like most conjunctive particles in Japanese, the particle つつ attaches to the 連用形 and is exclusive to verbs. }

\begin{ltabulary}{|P|P|P|}
\hline 

一段活用動詞 & 見る & 見つつ \\ \cline{1-3}

五段活用動詞 & 思う & 思いつつ \\ \cline{1-3}

サ変活用動詞 & する & しつつ \\ \cline{1-3}

カ変活用動詞 & 来る & 来つつ  \\ \cline{1-3}

\end{ltabulary}

\par{ When the particle つつ is used to express simultaneity, with the latter clause being either a positive or contrary statement, it is equivalent to the particle ながら. ながら shows simultaneous action too, but it's still different. つつ cannot be attached to nominal phrases but ながら can be. }

\par{1. 夫を亡くしてから、母は ${\overset{\textnormal{ほそうで}}{\text{細腕}}}$ \{Xつつ・〇 ながら\}、私たち5人の子供を育て上げた。 \hfill\break
After losing her husband, my mother by herself raised us five kids. }

\par{ Even in formal speech, ながら is the only choice in this situation. On the one hand, when the sentence is colloquial, つつ often becomes inappropriate. }

\par{2. 何だよ。静かだと思ったら、テレビを見\{〇 ながら・X つつ\}寝ちゃって。おい、風邪引くぞ。(男性語) \hfill\break
What is this? Just when I think it\textquotesingle s quiet, you\textquotesingle re falling asleep watching TV. Hey, you\textquotesingle ll catch a cold! }

\par{3. お茶でも飲みながら、お喋りしない? ちょっと ${\overset{\textnormal{ぐち}}{\text{愚痴}}}$ を聞いてよ。 (ちょっと女性っぽい) \hfill\break
Can't we talk a bit while having some tea? Just listen to what I have to complain about. }

\par{4. なぁ、ご飯食べながら、新聞読むのか。やめてくれねーか。 ${\overset{\textnormal{ふゆかい}}{\text{不愉快}}}$ だよ。(男性語) \hfill\break
Hey, you\textquotesingle re reading a newspaper while eating? Can\textquotesingle t you quit it? It\textquotesingle s just unpleasant. }

\par{ Again, there are situations where the two are interchangeable. Differences will be elaborated on throughout this lesson. Of course, the most basic difference to remember is that つつ is indicative of the written language, and if it is used in the spoken language, it is rather formal and or old-fashioned. }

\par{5. 感謝し\{つつ・ながら\}、日々を送るがよい。 \hfill\break
It is good to live each day by being grateful. }

\par{6. 二日酔いで痛む頭を押さえつつ、トイレに入った。  (書き言葉) \hfill\break
I went to the bathroom while holding my head with a hangover. }

\par{7. 起きようと思い\{つつ・ながら\}、 ${\overset{\textnormal{うら}}{\text{恨}}}$ みの ${\overset{\textnormal{こ}}{\text{籠}}}$ もった眼差しをあげた。 \hfill\break
Thinking that I would get up, I gave a look in spite. }

\par{8. 嘘と知り\{つつ・ながら\}、まだ来ぬ人を、雨の中で待ち続けた。 \hfill\break
Knowing that that was a lie, I continued to wait in the rain for the one that had not come. }

\par{9. いやだとは言い\{つつ・ながら\}(も)、 ${\overset{\textnormal{まんざら}}{\text{満更}}}$ 嫌いでもないらしい。 \hfill\break
Although saying it\textquotesingle s bad, [he] seems to not hate it altogether. }

\par{10. 真実と知り\{つつながら\}(も)・、まだ誤った情報を伝えるのが本当に抜け目のないことだ。 \hfill\break
While knowing the truth and still telling mistaken information is a really shrewd thing. }

\par{11. 車窓の景色を眺め\{つつ・ながら\}、過ぎ去りし日々を思い返していた。 \hfill\break
As I gazed out at the scenery from the car window, I was reliving days past. }

\par{\textbf{Archaism Note }: 過ぎ去りし = 過ぎ去った. 過ぎ去りし is the 連体形 of the Classical Japanese phrase 過ぎ去りき, which uses the auxiliary verb ~き used to show recollection. 過ぎ去りし happens to be used occasionally due to its rather nostalgic feel. }

\par{ As you can see, there are situations with particles where there is a contradictory relationship in the clauses. }

\par{12. 大雨は勢いを増し\{ながら・つつ\}、堤を決壊させるまでには至っていなかった。 \hfill\break
The rain continued to amass momentum, though it didn\textquotesingle t reach the point it would break the embankment. }

\par{13. テレビを壊し\{〇 (てい)ながら・X つつ\}(も)、謝りもせず黙りこくっている。 \hfill\break
I stayed quiet without apologizing having destroyed the TV. }

\par{ In the case that ながら and つつ show a resultative relation, they behave the same, but when showing a contradictory situation, they differ. In such case, つつ still shows an ongoing situation, but ながら shows the situation after completion\slash ending of said action A. }

\par{\textbf{Particle Note }: When the particle も is added after ながら and つつ, they only show contradictory conjunction, which aids you in interpreting ながら. }

\par{\textbf{Phrasing Note }: ~ていながら more clearly demonstrates contradictory conjunction. However, semantic properties of the verbs also help. }

\par{ For instance, in the following sentences, because the verb has an internal boundary—there is an internal limit to its extent—it makes two contrasting clauses a contradictory conjunction when used with ながら. Even still, it would be even less ambiguous, again, if ~ていながら were used. }

\par{14. 彼は太(り・てい)ながら、健康なダイエットを続けている。 \hfill\break
While he has gotten fat, he continues a healthy diet. }

\par{15. 体の動きを止めながら、心を先を急いでいた。 \hfill\break
While I had stopped my body\textquotesingle s movement, my heart was hastening onward. }

\par{ However, when there is no “internal limit” in the verb, ambiguity can arise. }

\par{16. 彼女は美味しそうに食べながら、吐き気がするほどまずいと思っていた。 \hfill\break
She seemed to eat like it was delicious but thought that it was bad to the point of puking. [Both clauses at the same time] \hfill\break
While she ate it like it was delicious, she thought that it was bad to the point of puking. [Second clause is after the fact] }

\par{ Usually, in these instances, the positive interpretation is more complicated. However, this could very well be intended. Regardless, if ながらも had been used, there would be no ambiguity issue. }

\par{ つつ, with it being somewhat archaic, is not used in interrogative sentences or sentences that urge things to happen like ~ようとする. It\textquotesingle s also not used with things like ~ている. }

\par{17a. 出かけようとしながら 〇 \hfill\break
17b. 出かけようとしつつ  X \hfill\break
While I was trying to leave  }

\par{ Another thing to note is that つつ can be used in showing the course of a change that is almost instant like a takeoff but ながら can\textquotesingle t as it is generally used with time periods longer than that. Contrarily, ignoring the course of change, it can\textquotesingle t show the result of change. }

\par{18. 出発し\{〇 つつ・X ながら\}、電車はスピードを上げていった。 \hfill\break
The train sped up as it departed. }

\par{19. 出発し\{〇 ながら・X つつ\}、電車はすぐまた停まってしまった。 \hfill\break
While the train departed, it immediately stopped again. }
      
\section{つつある}
 
\par{ つつある may either show that something is trying to materialize or a certain action\slash effect is continuing. Thus, it is usually interchangeable with ている. However, you must be careful with verbs such as なる where ている is used to show a resultant state. Thus, なりつつある ≠ なっている. }

\par{20. 耳が遠くなりつつある。 \hfill\break
My hearing is going. }

\par{21. 古い伝統が消滅しつつあるというわけではないと思います。 \hfill\break
I don't think it's to say that old traditions are disappearing. }

\par{22. 冬が近づきつつある。 \hfill\break
Winter is drawing near. }

\par{23. 乱射事件は増えつつある。 \hfill\break
Shooting incidents are on the rise. }

\par{24. 黒崎一護の霊圧は弱まりつつある。 \hfill\break
Kurosaki Ichigo's spiritual pressure is waning. }

\par{25. 太陽が今昇りつつある。 \hfill\break
The sun is rising now. }

\par{26. その病気は蔓延しつつあるそうだ。 \hfill\break
It's said that the disease has been spreading. }

\par{27. 高まりつつある保護貿易主義を阻止する。 \hfill\break
To turn back the rising tide of protectionism. }

\par{28. 動物集団は減少しつつある。 \hfill\break
Animal populations are decreasing. }

\par{29. 地球が次第に暖かくなりつつあることは、温室効果の増大が原因ではない。そんな情報に操られすぎているので、真実であると思いかねる。 \hfill\break
It is not that the Earth is gradually becoming warmer because of an increase in the greenhouse effect. As such information is too manipulated, it cannot be thought of as being the truth. }

\par{30. 大気が汚染されつつあるそうです。 \hfill\break
I hear that the atmosphere is being polluted. }

\par{31. ウーロン茶の質は下がりつつあるそうです。 \hfill\break
I hear that the quality of Oolong tea is going down. }
    