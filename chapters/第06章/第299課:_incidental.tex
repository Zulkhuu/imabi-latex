    
\chapter{Incidental}

\begin{center}
\begin{Large}
第299課: Incidental: がてら, かたがた, かたわら, \& ついで 
\end{Large}
\end{center}
 
\par{ All of these items have a common theme, but don't let this confuse you! Pay attention to how these items are exactly used, and see if you can come up with a chart to differentiate them. Then, compare your chart to the one that ends this lesson. }
      
\section{がてら}
 
\par{ ~がてら either attaches to the 連用形 of a verb or to a noun. ~がてら shows that \textbf{while you're doing something, you're also taking the opportunity to do something else }. So, whatever it attaches to, there should be some sense of \textbf{movement }implied. }

\par{ ~がてら is primarily used with nouns that express an action of movement such as walking or shopping. Unlike ~ながら, ~がてら doesn't indicate that two things are simultaneous. In both cases, though, the two actions are happening in the same time period. }

\par{ ~がてら shows that things are happening in a \textbf{relatively close time span }. This is rather old-fashioned, so you probably will not see it often. However, it does show up in writing and the JLPT 1, so you still need to know what it is. }

\begin{center}
 \textbf{Examples }
\end{center}
 
\par{1. 散歩がてら友達と話し合う。 \hfill\break
While taking a walk, I talk with my friend. }
 
\par{2. 散歩がてら買い物に行く。 \hfill\break
While taking a walk, I go shopping. }
 
\par{3. 駅に行きがてら郵便局に立ち寄ります。 \hfill\break
While going to the train station, I'll drop by the post office. }
 
\par{4. 教会に行きがてら、おじさんの家に寄ってきた。 \hfill\break
While I went to church, I went to visit my uncle (on the way). }
 
\par{5. 散歩がてらちょっとピーマンを買いに行ってきます。 \hfill\break
I'll get some peppers on my walk. }
 
\par{6. 外に行きがてら、手紙を出してきてくれない? \hfill\break
Can't you send out this letter as you go outside? }
 
\par{7. 友だちを駅まで送りがてら本を返してきた。 \hfill\break
I returned a book back to my friend while I sent him to the train station. }
 
\par{8. 運動がてら犬の散歩に出かけたっけ。 \hfill\break
You said you took the dog on a walk in exercising? }

\par{\textbf{Grammar Note }: When using a noun that can be made verbs by adding する, you shouldn't attach がてら to the 連用形. So, rather than ever saying 運動しがてら, simply use 運動がてら. }
 
\par{9. 暖かい日の ${\overset{\textnormal{ひるすぎ}}{\text{午過}}}$ 食後の運動がてら水仙の水を ${\overset{\textnormal{か}}{\text{易}}}$ えてやろうと思って洗面所へ出て、水道の ${\overset{\textnormal{せん}}{\text{栓}}}$ を ${\overset{\textnormal{ねじ}}{\text{捩}}}$ っていると、その看護婦が受持の ${\overset{\textnormal{へや}}{\text{室}}}$ の茶器を洗いに来て、例の通り ${\overset{\textnormal{あいさつ}}{\text{挨拶}}}$ をしながら、しばらく自分の手にした ${\overset{\textnormal{しゅでい}}{\text{朱泥}}}$ の ${\overset{\textnormal{はち}}{\text{鉢}}}$ と、その中に盛り上げられたように ${\overset{\textnormal{ふく}}{\text{膨}}}$ れて見える ${\overset{\textnormal{たまね}}{\text{珠根}}}$ を眺めていたが、やがてその眼を自分の横顔に移して、この前御入院の時よりもうずっと御顔色が好くなりましたねと、三カ月前の自分と今の自分を比較したような批評をした。 \hfill\break
Past noon on a warm day, I thought about watering the daffodil on going to exercise and went to the washroom, and when I had twisted the water nob, the nurse had come to wash the tea utensils of the room she was in charge of, and as she gave regular greetings, she gazed at the reddish-brown pot in my hand and the bulbs that seemed swollen up like they had been piled in there, but she finally moved her glance to my profile and gave the comment, “Your complexion has really improved lately since you came in the hospital”, which seemed to be a comparison of me now and three months ago.  \hfill\break
From 変な音 by 夏目漱石. }
暖かい日の午過(ひるすぎ)食後の運動がてら水仙の水を易(か)えてやろうと思って洗面所へ出て、水道の栓(せん)を捩(ねじ)っていると、その看護婦が受持の室(へや)の茶器を洗いに来て、例の通り挨拶(あいさつ)をしながら、しばらく自分の手にした朱泥(しゅでい)の鉢(はち)と、その中に盛り上げられたように膨(ふく)れて見える珠根(たまね)を眺めていたが、やがてその眼を自分の横顔に移して、この前御入院の時よりもうずっと御顔色が好くなりましたねと、三カ月前の自分と今の自分を比較したような批評をした。 Past noon on a warm day, I thought about watering the daffodil on going to exercise and went to the washroom, and when I had twisted the water nob, the nurse had come to wash the tea utensils of the room she was in charge of, and as she gave regular greetings, she gazed at the bulbs, which looked swollen up like they had been piled in the flower pot that I held in my hand, but she finally moved her glance to my face and gave the comment, “Your complex has really improved lately since you came in the hospital”, which seemed to be a comparison of me now and three months ago. \hfill\break
Fr 
\par{ Some usages of ~がてら, given that it has passed its heyday, may be questionable. If it is a situation seemingly not completely within one's free will, it can sound unnatural. In the following case, it would be natural if the meeting is most important to you and you just had to go on a 出張 for your own comfort. }

\par{10a. イギリスへの出張がてら、地球環境会議に出席しました。 \hfill\break
10b. イギリスへの出張(の)ついでに、地球環境会議に出席しました。 \hfill\break
On the occasion of going on a business trip to England, I attended a global environment meeting. }

\par{ To reiterate, if there isn't a sense of moving, ~がてら becomes ungrammatical and you would have to use something else like ついでに (see below) or rephrase the sentence entirely. }

\par{11a. 食事がてらこの間話していた映画でも見に行こう。〇 \hfill\break
11b. 食事がてらこの間話していた映画でも見よう。X \hfill\break
Let's go see the movie we've been talking about recently while out to dinner. }

\par{12. 散歩がてら(に)行って来よう。 \hfill\break
I'll come by way of going on a walk. }

\par{\textbf{Particle Note }: You can also rarely see がてらに. This explicitly establishes a sense of direction\slash movement. }
      
\section{かたがた}
 
\par{ ~ かたがた: attaches to nouns. "Previously" or "on the occasion\slash same time as". It is mainly used with verbs related to movement such as 行く. Essentially, you do a certain thing on the occasion of tending to another prerogative. As you will see, this is typically only used in 敬語 in the form of あいさつ文. It can still be used in the sense of ~がてら by older speakers, which still makes it old-fashioned. }

\par{13. お礼かたがたご機嫌伺いをしてきましょう。 \hfill\break
Let's go come and make an inquiry on good times for their previous gratitude to us. }

\par{14. 見舞いかたがた手伝いに行った。 \hfill\break
On the occasion of inquiring on (him), I went to give some help. }

\par{15. お近くにお越しの節は、お遊びかたがた、お立ち寄りください。 \hfill\break
Please stop by when you are near while on your excursion.  }

\par{16. ご挨拶かたがたお宅に伺います。 \hfill\break
I will come to your house as I give my salutations. }

\par{17. 散歩かたがた \hfill\break
On the same time as my walk }

\par{\textbf{Usage Note }: This suffix is normally replaced with something like ついでに in the spoken language.  }

\par{\textbf{漢字 Note }: かたがた may also rarely be written as  旁(々). }
      
\section{かたわら}
 
\par{ Literally meaning "beside", 傍ら is after the 連体形 of verbs or the nominal form of a verb to show simultaneous action and is equivalent to "一方で…と同時に". 傍ら is often used with time frames that are much broader. This pattern implies that the first action is the primary action and the second is something that is also occurring in parallel with the first. }

\par{ If you were to turn this into a verbal expression, they would all turn to "連用形+ながら". かたわら is used a lot in the written language whereas ~ながら would be used in the spoken language. The former is used a lot in regard to protagonists in describing their lifestyle backgrounds in novels. If a description was given for a character to only appear once, ~ながら would be used. The pattern Xは動詞+人の+かたわらで is also literary, and it is used in describing condition, but it is only describing place of activity. In this manner, the distance between X and A (動詞+人) is like the following with the right being larger: そば \textrightarrow  隣 \textrightarrow  横 \textrightarrow  かたわら. }

\begin{center}
\textbf{Examples }
\end{center}

\par{18. 道のかたわらで \hfill\break
Beside the road }

\par{19. 彼女は高校の数学の教師の傍ら、バレーボール部の ${\overset{\textnormal{かんとく}}{\text{監督}}}$ をしています。 \hfill\break
While she is a high school mathematics teacher, she is at the same time the volleyball department manager. }

\par{20. 私はスペイン語を習う傍ら、韓国語を習うことにしました。 \hfill\break
Alongside Spanish, I have also decided to study Korean. }

\par{21a. 先生たちのかたわらで、彼らの ${\overset{\textnormal{そうだんごと}}{\text{相談事}}}$ に耳を傾けていた。X \hfill\break
21b. 先生たちのそばで、彼らの相談事に耳を傾けていた。 \hfill\break
I was listening to the side of the teachers discussing things. }

\par{22. 傍らに人なきがごとし。(Set Phrase) \hfill\break
Act outrageously as if no one is around. }
      
\section{ついで}
 
\par{  Written as 次いで, ついで is a conjunction meaning "subsequently". Written as 序で, ついで is a noun meaning "opportunity". Building on this, 序でに means "incidentally". }
 
\par{23. 序でに ${\overset{\textnormal{ひとこと}}{\text{一言}}}$ 言っておく。 \hfill\break
To incidentally give a word about something in advance. }
 
\par{24. 開会式が終わりました。次いで、 ${\overset{\textnormal{きょうぎ}}{\text{競技}}}$ に移ります。 \hfill\break
The opening ceremony ended. Subsequently\slash afterwards, we'll move to the events. }
 
\par{25. 「 ${\overset{\textnormal{いっしゅん}}{\text{一瞬}}}$ の ${\overset{\textnormal{ちり}}{\text{塵}}}$ 」という歌に次いで「ヴェロニカ」というのは僕の好きな歌だよ。 \hfill\break
After the song, "Dust of a Moment", my favorite song is called, "Velonica". }
 
\par{26. 序でがあったら、これを ${\overset{\textnormal{わた}}{\text{渡}}}$ しておいてください。 \hfill\break
If you have a chance, please send this to him. }
 
\par{ ~ついでに is somewhat interchangeable with ~がてら. However, they're not exactly the same. 「A-る・た・動作名詞(の)+ついでにB」 shows that the speaker does\slash wants to end up doing B as well in the case that A--a daily task, chore, duty, or obligation--has to be done. Using ~がてら has no such requirements. Rather, ~がてら is more like saying "Aという \textbf{せっかく }の機会だから、Bもしてみよう". }
 
\par{27a. 図書館へ本を返しに行くついでに、ジョギングする。 \hfill\break
27b. 図書館へ本を返しに行きがてら、ジョギングする。 \hfill\break
I'll go jogging while I go return the book\slash books to the library. }
 
\par{${\overset{\textnormal{}}{\text{28a. 周}}}$ りの景色を楽しむついでに、山を下りた。X \hfill\break
28b. 周りの景色を楽しみがてら、山を下りた。〇 \hfill\break
On the occasion of enjoying the surrounding scenery, I descended down the mountain. }
 
\par{ When ついでに shows ordinariness, の may be dropped with 動作名詞 in the \textbf{spoken language }. Don't do this in writing unless you're writing a dialogue . }

\par{29. 登山ついでに、いろんな写真を撮ってみたよ。 \hfill\break
I took all sorts of pictures when going mountain climbing. }

\par{ Although context is crucial, there are instances when it doesn't have to be used with a 動作名詞, but the conditions mentioned above still apply. }
 
\par{30a. お風呂がてら、 ${\overset{\textnormal{よくそう}}{\text{浴槽}}}$ を洗っておいて。X \hfill\break
30b. お風呂ついでに、浴槽を洗っておいて。 〇 \hfill\break
30c. お風呂のついでに、浴槽を洗っておいて。〇 \hfill\break
When you get a bath, be sure to wash the tub. }
 
\par{31. 居間の ${\overset{\textnormal{かべ}}{\text{壁}}}$ のついでに、家の ${\overset{\textnormal{がいへき}}{\text{外壁}}}$ ${\overset{\textnormal{ぬ}}{\text{も塗}}}$ る。 \hfill\break
To paint the outside of the house on the occasion of (painting) the living room walls. }
 
\par{32a. 彼女を見 ${\overset{\textnormal{ま}}{\text{舞}}}$ うついでに、定期 ${\overset{\textnormal{けんしん}}{\text{健診}}}$ を受けてくる。 \hfill\break
32b. 彼女を見舞いがてら、定期健診を受けてくる。 \hfill\break
I'll get my regular medical check up on the occasion that I see her. }
 
\par{ When reversed, only ついでに is possible because it's something you have an obligation for. In this situation, the order of events is unclear. It's also unclear whether the two events are happening in the same place or not. If in the same place, then the speaker would be taking the checkup in the midst of going to see about her. }
 
\par{ It's thought to be that the two events are taking place at basically same time. However, when strictly speaking, taking the latter action as the standard point of time, one can express whether or not it is before, after, or at the same time as A. On the other hand, がてら limits interpretations to just getting both down during the time you're at the same hospital\slash facility. }
 
\par{ When the non-past tense is used before ついでに, A and B can either be happening at the same time or B happens before A. When the past tense is used before ついでに, B happens after A. }
 
\par{33. 友達の家へ行くついでに、海岸で泳いだ。 \hfill\break
I swam in the ocean on the occasion of going to my friend's house. \hfill\break
Note: The swimming can be simultaneous to going to your friend's house or before. }
 
\par{34a. こっちへ来るついでに、 ${\overset{\textnormal{みやげ}}{\text{土産}}}$ を買ってきてほしいんだけど。 \hfill\break
34b. こっちへ来たついでに、土産を買ってきてほしいんだけど。X \hfill\break
 \hfill\break
 You want someone as they are to come over to get a souvenir. Why on Earth would you want a souvenir from your own place anyway? Is that even a possible usage of the word 土産? No! In reality, speakers, even when B is after A, will use the non-past with ついでに because of the previous statement that the two actions are usually deemed as happening basically at the same time. This practical interchangeability, though, is in regards to 瞬間動詞. }

\par{35. ${\overset{\textnormal{かびん}}{\text{花瓶}}}$ を壊すついでに、それらの皿も壊しといてくれないか? \hfill\break
While you destroy the vase, can't you also break those plates? }
 
\par{ Consider the following where the opposite situation holds: flipping them potentially makes ついでに ungrammatical. It just so happens that the following sentences don't have this problem as both directions are logical. The meanings of these phrases are summarized below, so don't think that all of these mean the same things. }
 
\par{36a. 実家に帰るついでに遊んできた。〇 \hfill\break
36b. 実家に帰ったついでに遊んできた。〇 \hfill\break
36c. 実家に帰りがてら、遊んできた。〇 \hfill\break
36d. 遊ぶついでに、実家に帰ってきた。〇 \hfill\break
36e. 遊んだついでに、実家に帰ってきた。△ \hfill\break
36f. 遊びがてら、実家に帰ってきた。〇 }
 
\begin{center}
\textbf{ }\textbf{がてら VS }\textbf{ついでに VS }\textbf{かたわら VS }\textbf{かたがた VS }\textbf{ながら }
\end{center}
 
\par{ Hopefully you've seen a lot of differences and overlap between these phrases. So, to summarize, here is a chart to act as your cheat sheet on how to distinguish. }

\begin{ltabulary}{|P|P|P|P|P|}
\hline 

Phrase & Characteristics of A & Importance & Time Ordering & Period \\ \cline{1-5}

A+ながら+B & Ordinary & A≦B & Same time & One scene \\ \cline{1-5}

A+かたわら+B & Ordinary & A>B & Parallel & Long time \\ \cline{1-5}

A+かたがた+B & Unordinary & A=B & Same time & Short time \\ \cline{1-5}

A+ついでに+B & Obligatory & A>B & Before or after or same & Short time \\ \cline{1-5}

A+がてら+B & Pleasure & 不明 & Same time & Short distance \\ \cline{1-5}

\end{ltabulary}
\hfill\break
\hfill\break
    