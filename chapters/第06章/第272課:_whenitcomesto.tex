    
\chapter{"When it Comes to\dothyp{}\dothyp{}\dothyp{}" I}

\begin{center}
\begin{Large}
第272課: "When it Comes to\dothyp{}\dothyp{}\dothyp{}" I: というと, といえば, \& といったら 
\end{Large}
\end{center}
 
\par{ In this lesson, we will begin coverage on expressions that can roughly all translate to “when it comes to…” to some capacity. The expressions in this lesson all involve the verb 言う in its respective conditional forms. If you have a firm grasp of these forms, the nuance differences that you are to soon be introduced to will not be news to you. }
      
\section{~というと: When\slash in speaking of\dothyp{}\dothyp{}\dothyp{}}
 
\par{ ~というと is used to indicate what is representative or deeply associated with something, and it can even describe an inevitable correlation between something (as seen in Ex. 3). This pattern is quite objective in nature and is seen most frequently in the written language. In a literal sense, you can translate it as “if one were to speak…of,” but its translation can be flexible to more easily translated as “when\slash in\slash if speaking of…” }

\par{1. ${\overset{\textnormal{にほん}}{\text{日本}}}$ の ${\overset{\textnormal{こと}}{\text{古都}}}$ というと、 ${\overset{\textnormal{へいあんきょう}}{\text{平安京}}}$ 、 ${\overset{\textnormal{なら}}{\text{奈良}}}$ でしょう。 \hfill\break
In speaking of the former capitals of Japan, they\textquotesingle re Heian-kyō and Nara. }

\par{2. イギリスと ${\overset{\textnormal{い}}{\text{言}}}$ うと、いつも ${\overset{\textnormal{くも}}{\text{曇}}}$ っているというイメージがあるかと ${\overset{\textnormal{おも}}{\text{思}}}$ います。 \hfill\break
In speaking of England, I think you might have an image of it always being cloudy. }

\par{3. ずっと ${\overset{\textnormal{せいてん}}{\text{晴天}}}$ が ${\overset{\textnormal{つづ}}{\text{続}}}$ いていたのにお ${\overset{\textnormal{まつ}}{\text{祭}}}$ りだと ${\overset{\textnormal{い}}{\text{言}}}$ うと ${\overset{\textnormal{てんき}}{\text{天気}}}$ が ${\overset{\textnormal{わる}}{\text{悪}}}$ くなるのは ${\overset{\textnormal{き}}{\text{気}}}$ のせいかな? \hfill\break
Is it just my imagination that the weather certainly gets bad whenever there\textquotesingle s a festival despite it being clear skies the whole time? }

\par{4. ${\overset{\textnormal{ご}}{\text{5}}}$ ${\overset{\textnormal{だいしんぶん}}{\text{大新聞}}}$ と ${\overset{\textnormal{い}}{\text{言}}}$ うと、 ${\overset{\textnormal{よみうり}}{\text{読売}}}$ ・ ${\overset{\textnormal{あさひ}}{\text{朝日}}}$ ・ ${\overset{\textnormal{まいにち}}{\text{毎日}}}$ ・ ${\overset{\textnormal{さんけい}}{\text{産経}}}$ ・ ${\overset{\textnormal{とうきょう}}{\text{東京}}}$ を ${\overset{\textnormal{さ}}{\text{指}}}$ すことが ${\overset{\textnormal{おお}}{\text{多}}}$ い。 \hfill\break
When speaking of the five major newspapers, the Yomiuri, Asahi, Mainichi, Sankei, and Tokyo are often referred to. }

\par{5. アメリカに ${\overset{\textnormal{ねづ}}{\text{根付}}}$ いた ${\overset{\textnormal{にほんぶんか}}{\text{日本文化}}}$ というと、 ${\overset{\textnormal{すし}}{\text{寿司}}}$ ・ ${\overset{\textnormal{い}}{\text{生}}}$ け ${\overset{\textnormal{ばな}}{\text{花}}}$ ・アニメ・ ${\overset{\textnormal{まんが}}{\text{漫画}}}$ ・ ${\overset{\textnormal{ぜん}}{\text{禅}}}$ などがあります。 \hfill\break
In speaking of Japanese culture that has taken root in America, there is sushi, ikebana, anime, manga, Zen, etc. }

\par{6. ${\overset{\textnormal{てんこう}}{\text{天候}}}$ というと「 ${\overset{\textnormal{は}}{\text{晴}}}$ れ」「 ${\overset{\textnormal{あめ}}{\text{雨}}}$ 」など ${\overset{\textnormal{おも}}{\text{思}}}$ い ${\overset{\textnormal{う}}{\text{浮}}}$ かべますが、「 ${\overset{\textnormal{なみ}}{\text{波}}}$ 」「 ${\overset{\textnormal{かぜ}}{\text{風}}}$ 」も ${\overset{\textnormal{てんこう}}{\text{天候}}}$ なんですね。 \hfill\break
We\textquotesingle re reminded of “clear weather” and “rain” in speaking of the weather, but “waves” and “the wind” are also part of the weather, you know. }

\par{7. ${\overset{\textnormal{せいじ}}{\text{政治}}}$ というと、 ${\overset{\textnormal{ふしん}}{\text{不審}}}$ なものの ${\overset{\textnormal{だいめいし}}{\text{代名詞}}}$ のように ${\overset{\textnormal{おも}}{\text{思}}}$ えます。 \hfill\break
When speaking of politics, it would certainly seem to be a classic example of something suspicious. }

\par{8. ${\overset{\textnormal{びせいぶつ}}{\text{微生物}}}$ というと、 ${\overset{\textnormal{ふけつ}}{\text{不潔}}}$ なばい ${\overset{\textnormal{きん}}{\text{菌}}}$ や ${\overset{\textnormal{びょうき}}{\text{病気}}}$ を ${\overset{\textnormal{ひ}}{\text{引}}}$ き ${\overset{\textnormal{お}}{\text{起}}}$ こす ${\overset{\textnormal{わるもの}}{\text{悪者}}}$ を ${\overset{\textnormal{れんそう}}{\text{連想}}}$ する ${\overset{\textnormal{ひと}}{\text{人}}}$ が ${\overset{\textnormal{おお}}{\text{多}}}$ いかもしれない。 \hfill\break
When speaking of microbes, there may be many people who associate them with filthy germs or bad guys that cause illness. }

\par{9. ${\overset{\textnormal{にくがん}}{\text{肉眼}}}$ で ${\overset{\textnormal{かくにん}}{\text{確認}}}$ できる ${\overset{\textnormal{たいようけい}}{\text{太陽系}}}$ の ${\overset{\textnormal{わくせい}}{\text{惑星}}}$ というと、 ${\overset{\textnormal{すいせい}}{\text{水星}}}$ 、 ${\overset{\textnormal{きんせい}}{\text{金星}}}$ 、 ${\overset{\textnormal{かせい}}{\text{火星}}}$ 、 ${\overset{\textnormal{もくせい}}{\text{木星}}}$ 、 ${\overset{\textnormal{どせい}}{\text{土星}}}$ までであり、それに ${\overset{\textnormal{ちきゅう}}{\text{地球}}}$ を ${\overset{\textnormal{くわ}}{\text{加}}}$ えて ${\overset{\textnormal{むっ}}{\text{6}}}$ つです。 \hfill\break
When speaking of the planets in the solar system that can be verified with the naked eye, there is Mercury, Venus, Mars, Jupiter, and Saturn, and when you add Earth to the mix, there are six. }

\par{10. ${\overset{\textnormal{じゅうらい}}{\text{従来}}}$ 、 ${\overset{\textnormal{きぎょう}}{\text{企業}}}$ の ${\overset{\textnormal{しゃかいてきせきにん}}{\text{社会的責任}}}$ というと、 ${\overset{\textnormal{きぎょうりんり}}{\text{企業倫理}}}$ や ${\overset{\textnormal{ほうれいじゅんしゅ}}{\text{法令順守}}}$ などが ${\overset{\textnormal{ちゅうしん}}{\text{中心}}}$ だった。 \hfill\break
Conventionally, when speaking of societal responsibility to the company, company ethics, compliance, and such were at the center. }

\par{11. ワシントン州の有名な州立大学というと、ワシントン大学ですね。 \hfill\break
In speaking of famous state universities in the State of Washington, there is Washington University. }

\par{12. ${\overset{\textnormal{かんこく}}{\text{韓国}}}$ ドラマというと、 ${\overset{\textnormal{ふじ}}{\text{不治}}}$ の ${\overset{\textnormal{やまい}}{\text{病}}}$ や ${\overset{\textnormal{こうつうじこ}}{\text{交通事故}}}$ 、 ${\overset{\textnormal{あ}}{\text{有}}}$ り ${\overset{\textnormal{え}}{\text{得}}}$ ない ${\overset{\textnormal{す}}{\text{擦}}}$ れ ${\overset{\textnormal{ちが}}{\text{違}}}$ いなどが ${\overset{\textnormal{ひんしゅつ}}{\text{頻出}}}$ する ${\overset{\textnormal{れんあい}}{\text{恋愛}}}$ ドラマを ${\overset{\textnormal{そうぞう}}{\text{想像}}}$ するという ${\overset{\textnormal{ひと}}{\text{人}}}$ が ${\overset{\textnormal{おお}}{\text{多}}}$ いだろう。 \hfill\break
When speaking of Korean dramas, there are probably many people who associate them with romance dramas which frequently feature incurable diseases, traffic accidents, impossible crossing of paths, and what not. }
\textbf{~はというと: As far as\dothyp{}\dothyp{}\dothyp{}goes } \hfill\break
 ~はというと is very similar to ~というと, but it always intrinsically illustrates a stark contrast of some sort, whether implicitly or explicitly stated. 
\par{13. ${\overset{\textnormal{にしだ}}{\text{西田}}}$ さんは ${\overset{\textnormal{さかなりょうり}}{\text{魚料理}}}$ が ${\overset{\textnormal{す}}{\text{好}}}$ きで ${\overset{\textnormal{なん}}{\text{何}}}$ でも ${\overset{\textnormal{た}}{\text{食}}}$ べますが、 ${\overset{\textnormal{にくりょうり}}{\text{肉料理}}}$ はというとかなり ${\overset{\textnormal{す}}{\text{好}}}$ き ${\overset{\textnormal{きら}}{\text{嫌}}}$ いがあるらしいです。 \hfill\break
Mr. Nishida loves fish-based dishes and will eat anything, but as far as meat dishes go, he seems to have considerable likes and dislikes. }

\par{14. ${\overset{\textnormal{わたし}}{\text{私}}}$ は ${\overset{\textnormal{りかけい}}{\text{理科系}}}$ の ${\overset{\textnormal{かもく}}{\text{科目}}}$ は ${\overset{\textnormal{とくい}}{\text{得意}}}$ なんですが、 ${\overset{\textnormal{ぶんかけい}}{\text{文科系}}}$ の ${\overset{\textnormal{かもく}}{\text{科目}}}$ はというと、 ${\overset{\textnormal{まった}}{\text{全}}}$ く ${\overset{\textnormal{にがて}}{\text{苦手}}}$ なんです。 \hfill\break
I am good at science courses, but as far as liberal arts go, I\textquotesingle m absolutely bad at them. }

\par{15. ${\overset{\textnormal{ちち}}{\text{父}}}$ も ${\overset{\textnormal{はは}}{\text{母}}}$ ものんびり ${\overset{\textnormal{す}}{\text{過}}}$ ごしていますが、 ${\overset{\textnormal{わたし}}{\text{私}}}$ はというと、 ${\overset{\textnormal{まいにち}}{\text{毎日}}}$ ただ ${\overset{\textnormal{いそが}}{\text{忙}}}$ しくて ${\overset{\textnormal{はたら}}{\text{働}}}$ いています。 \hfill\break
My father and my bother live leisurely, but I on the other hand am just busy at working every day. }

\par{16. ${\overset{\textnormal{そうしょくどうぶつ}}{\text{草食動物}}}$ はというと、 ${\overset{\textnormal{きほん}}{\text{基本}}}$ は ${\overset{\textnormal{しょくぶつ}}{\text{植物}}}$ しか ${\overset{\textnormal{た}}{\text{食}}}$ べません。 \hfill\break
As far as herbivores go, they basically only eat plants. }

\par{17. ${\overset{\textnormal{ちきゅうじょう}}{\text{地球上}}}$ で ${\overset{\textnormal{さいだい}}{\text{最大}}}$ の ${\overset{\textnormal{どうぶつ}}{\text{動物}}}$ はというと、 ${\overset{\textnormal{なんぴょうよう}}{\text{南氷洋}}}$ のシロナガスクジラです。 \hfill\break
As far as the largest animal on Earth goes, that would be the blue whale from the Antarctic Ocean. }

\par{18. ${\overset{\textnormal{にほん}}{\text{日本}}}$ のように、 ${\overset{\textnormal{そば}}{\text{蕎麦}}}$ を ${\overset{\textnormal{めん}}{\text{麺}}}$ に ${\overset{\textnormal{かこう}}{\text{加工}}}$ して ${\overset{\textnormal{た}}{\text{食}}}$ べる ${\overset{\textnormal{くに}}{\text{国}}}$ はというと、 ${\overset{\textnormal{となり}}{\text{隣}}}$ の ${\overset{\textnormal{ちゅうごく}}{\text{中国}}}$ や ${\overset{\textnormal{ちょうせん}}{\text{朝鮮}}}$ およびブータンぐらいしかない。 \hfill\break
As far as countries that manufacture buckwheat into noodles to eat like Japan go, there are only say neighboring China and Korea as well as Bhutan. }

\par{19. ${\overset{\textnormal{せかい}}{\text{世界}}}$ で ${\overset{\textnormal{もっと}}{\text{最}}}$ も ${\overset{\textnormal{はば}}{\text{幅}}}$ のある ${\overset{\textnormal{かいきょう}}{\text{海峡}}}$ はというと、カナダのバフィン ${\overset{\textnormal{とう}}{\text{島}}}$ とグリーンランドに ${\overset{\textnormal{はさ}}{\text{挟}}}$ まれたデービス ${\overset{\textnormal{かいきょう}}{\text{海峡}}}$ です。 \hfill\break
As far as the widest straight in the world goes, it would be the Davis Strait, which is sandwiched between Baffin Island and Greenland. }

\par{20. ${\overset{\textnormal{にっぽんせいふ}}{\text{日本政府}}}$ はというと、 ${\overset{\textnormal{すがかんぼうちょうかん}}{\text{菅官房長官}}}$ は ${\overset{\textnormal{ちょうさ}}{\text{調査}}}$ に ${\overset{\textnormal{ひていてき}}{\text{否定的}}}$ な ${\overset{\textnormal{けんかい}}{\text{見解}}}$ を ${\overset{\textnormal{しめ}}{\text{示}}}$ している。 \hfill\break
In speaking of the Japanese government, Chief Secretary Suga has expressed negative views on investigation (of the matter). }

\par{21. アメリカ ${\overset{\textnormal{どうじたはつ}}{\text{同時多発}}}$ テロ ${\overset{\textnormal{じけん}}{\text{事件}}}$ と ${\overset{\textnormal{い}}{\text{言}}}$ うと、あの ${\overset{\textnormal{ころ}}{\text{頃}}}$ の ${\overset{\textnormal{ちゅうとう}}{\text{中東}}}$ アジアの ${\overset{\textnormal{じょうせいあっか}}{\text{情勢悪化}}}$ を ${\overset{\textnormal{おも}}{\text{思}}}$ い ${\overset{\textnormal{だ}}{\text{出}}}$ します。 \hfill\break
In speaking of the American Simultaneous Sequential Terrorist Event (9\slash 11), I think of the worsening state of affairs in the Middle East at the time. }

\par{22. ${\overset{\textnormal{わたし}}{\text{私}}}$ は ${\overset{\textnormal{あに}}{\text{兄}}}$ が ${\overset{\textnormal{だいきら}}{\text{大嫌}}}$ いです。 ${\overset{\textnormal{なぜ}}{\text{何故}}}$ かというと、とても ${\overset{\textnormal{じこちゅうしんてき}}{\text{自己中心的}}}$ だからです。 \hfill\break
I hate my older brother. The reason why is because he is extremely self-centered. }
      
\section{~といえば: Speaking of which\dothyp{}\dothyp{}\dothyp{}}
 
\par{ When using ~といえば, a lot of enthusiasm is felt when bringing up something into conversation. It is seen a lot in conversation and can be very definitive as far as the emotional context of one\textquotesingle s statement is concerned. }
 
\par{23. 日本代表の漫画といえばナルトですよね。 \hfill\break
Speaking of a manga that represents Japan, that would be Naruto, huh. }
 
\par{24. サンフランシスコで、 ${\overset{\textnormal{いちばんゆうめい}}{\text{一番有名}}}$ な ${\overset{\textnormal{かんこうめいしょ}}{\text{観光名所}}}$ といえば、ゴールデン・ゲート・ブリッジでしょう。 \hfill\break
Speaking of the most famous tourist spot in San Francisco, that would be the Golden Gate Bridge. }
 
\par{25. ${\overset{\textnormal{なつ}}{\text{夏}}}$ の ${\overset{\textnormal{た}}{\text{食}}}$ べ ${\overset{\textnormal{もの}}{\text{物}}}$ といえば、スイカ、アイス、 ${\overset{\textnormal{そうめん}}{\text{素麺}}}$ でしょう。 \hfill\break
Speaking of summer foods, there\textquotesingle s watermelons, ice cream, and somen. }
 
\par{26. タイ ${\overset{\textnormal{りょうり}}{\text{料理}}}$ といえば、 ${\overset{\textnormal{えきまえ}}{\text{駅前}}}$ に ${\overset{\textnormal{あたら}}{\text{新}}}$ しいレストランが ${\overset{\textnormal{でき}}{\text{出来}}}$ たの ${\overset{\textnormal{し}}{\text{知}}}$ ってる? \hfill\break
Speaking of Thai cuisine, did you know there\textquotesingle s a new restaurant in front of the station? }
 
\par{27. \hfill\break
A ${\overset{\textnormal{し}}{\text{氏}}}$ : ${\overset{\textnormal{いちかわ}}{\text{市川}}}$ さん、 ${\overset{\textnormal{せんじつこうつうじこ}}{\text{先日交通事故}}}$ に ${\overset{\textnormal{あ}}{\text{遭}}}$ って ${\overset{\textnormal{にゅういん}}{\text{入院}}}$ したらしいですよ。 \hfill\break
B ${\overset{\textnormal{し}}{\text{氏}}}$ : え、そうなんですか。あ、 ${\overset{\textnormal{にゅういん}}{\text{入院}}}$ したといえば、 ${\overset{\textnormal{みかみ}}{\text{三上}}}$ さんが ${\overset{\textnormal{らいしゅうたいいん}}{\text{来週退院}}}$ だそうですよ。 \hfill\break
Person A: It seems that Mr. Ichikawa became hospitalized the other day due to getting in a traffic accident. \hfill\break
Person B: Wow, really? Oh, speaking of being put in the hospital, I hear that Mr. Mikami is to be released from the hospital next week. }
 
\par{28. ${\overset{\textnormal{こてんげいのう}}{\text{古典芸能}}}$ といえば、 ${\overset{\textnormal{かぶき}}{\text{歌舞伎}}}$ でしょう。 \hfill\break
Speaking of classical theatre, there\textquotesingle s kabuki. }
 
\par{29. ${\overset{\textnormal{くるま}}{\text{車}}}$ と ${\overset{\textnormal{い}}{\text{言}}}$ えば、 ${\overset{\textnormal{きみ}}{\text{君}}}$ はトヨタ ${\overset{\textnormal{しゃ}}{\text{車}}}$ を ${\overset{\textnormal{か}}{\text{買}}}$ ったそうだね。 \hfill\break
Speaking of cars, I hear you've bought yourself a Toyota? }
 
\par{30. リンカーンと ${\overset{\textnormal{い}}{\text{言}}}$ えば ${\overset{\textnormal{どれいせいど}}{\text{奴隷制度}}}$ を ${\overset{\textnormal{おも}}{\text{思}}}$ い ${\overset{\textnormal{だ}}{\text{出}}}$ します。 \hfill\break
In speaking of Lincoln, we think of the slavery system. }
 
\par{31. ${\overset{\textnormal{どうぶつ}}{\text{動物}}}$ と ${\overset{\textnormal{い}}{\text{言}}}$ えば、 ${\overset{\textnormal{ぼく}}{\text{僕}}}$ の ${\overset{\textnormal{いぬ}}{\text{犬}}}$ は ${\overset{\textnormal{はな}}{\text{話}}}$ せるんだ。 \hfill\break
Speaking of animals, my dog can talk. }
 
\par{32. レオナルドダヴィンチといえば、 ${\overset{\textnormal{せかいてき}}{\text{世界的}}}$ な ${\overset{\textnormal{げいじゅつか}}{\text{芸術家}}}$ だ。 \hfill\break
In speaking of Leonard da Vinci, he is as world-famous artist. }
 
\par{33. イギリスといえば、\{ ${\overset{\textnormal{こうちゃ}}{\text{紅茶}}}$ ・ロンドン・ ${\overset{\textnormal{にかいだ}}{\text{二階建}}}$ てバス・ビードルズ・イギリス ${\overset{\textnormal{おうしつ}}{\text{王室}}}$ ・ハリーポッター\}など ${\overset{\textnormal{おも}}{\text{思}}}$ い ${\overset{\textnormal{う}}{\text{浮}}}$ かべる ${\overset{\textnormal{ひと}}{\text{人}}}$ が ${\overset{\textnormal{おお}}{\text{多}}}$ いでしょう。 \hfill\break
In speaking of England, there are surely many people who think of [tea\slash London\slash two-story buses\slash the Beatles\slash the British Royal Family\slash Harry Potter]. }
 
\par{\textbf{Spelling Note }: ひげ in 漢字 may either be 髭, 髯, or 鬚. However, the first refers to beards around the mouth, the second refers to beards on the cheeks, and the third refers to beards on the chin. These respectively may be said\slash written as 口髭・髭, 頬髭・頬髯・髯, and 顎髭・顎鬚・鬚. You are not required to remember this. }
 
\par{34. メキシコの ${\overset{\textnormal{だいひょうてき}}{\text{代表的}}}$ な ${\overset{\textnormal{た}}{\text{食}}}$ べ ${\overset{\textnormal{もの}}{\text{物}}}$ といえば、タコスです。 \hfill\break
In speaking of Mexico\textquotesingle s signature food, there's the taco. }
 
\par{35. ${\overset{\textnormal{いま}}{\text{今}}}$ の ${\overset{\textnormal{わかもの}}{\text{若者}}}$ の ${\overset{\textnormal{りゅうこう}}{\text{流行}}}$ といえば、 ${\overset{\textnormal{だんせい}}{\text{男性}}}$ の ${\overset{\textnormal{あいだ}}{\text{間}}}$ ではひげが ${\overset{\textnormal{はや}}{\text{流行}}}$ っていて、 ${\overset{\textnormal{じょせい}}{\text{女性}}}$ の ${\overset{\textnormal{あいだ}}{\text{間}}}$ では ${\overset{\textnormal{なみだぶくろ}}{\text{涙袋}}}$ メイクが ${\overset{\textnormal{はや}}{\text{流行}}}$ っている。 \hfill\break
In speaking of trends among young people today, beards are trending among men and namidabukuro is trending among women. }

\par{\textbf{Word Note }: 涙袋メイク makes it look like you have the impression of tears under your eyes. }
      
\section{~といったら: When it comes to\dothyp{}\dothyp{}\dothyp{}\slash speaking of\dothyp{}\dothyp{}\dothyp{}}
 
\par{ In a literal sense, ~といったら means “were to say.” }

\par{36. ${\overset{\textnormal{こわ}}{\text{怖}}}$ くないと ${\overset{\textnormal{い}}{\text{言}}}$ ったら ${\overset{\textnormal{うそ}}{\text{嘘}}}$ になる。 \hfill\break
If I said I wasn\textquotesingle t scared, that\textquotesingle d be a lie. }

\par{37. フランス ${\overset{\textnormal{じん}}{\text{人}}}$ に「アミは ${\overset{\textnormal{ともだち}}{\text{友達}}}$ です」と ${\overset{\textnormal{い}}{\text{言}}}$ ったら ${\overset{\textnormal{こんらん}}{\text{混乱}}}$ しそうですね。 \hfill\break
If were to say “Ami is my friend” to a French person, they might get confused. }

\par{ In similar contexts seen with ~といえば, however, it is used in much the same way to give example of something intrinsically tied to whatever is the focus of conversation. It is arguably the most common form in conversation. It\textquotesingle s not so ecstatic, but it\textquotesingle s just as assertive. }

\par{38. シアトルの ${\overset{\textnormal{けしき}}{\text{景色}}}$ といったら、 ${\overset{\textnormal{くち}}{\text{口}}}$ で ${\overset{\textnormal{い}}{\text{言}}}$ い ${\overset{\textnormal{あらわ}}{\text{表}}}$ せない ${\overset{\textnormal{ほど}}{\text{程}}}$ です。 \hfill\break
When it comes to the scenery of Seattle, it\textquotesingle s beyond expression. }

\par{39. ${\overset{\textnormal{やまだ}}{\text{山田}}}$ さんの ${\overset{\textnormal{ちゅうごくごのうりょく}}{\text{中国語能力}}}$ といったら、 ${\overset{\textnormal{ちゅうごくじん}}{\text{中国人}}}$ がびっくりするほどだそうです。 \hfill\break
Speaking of Mr. Yamada\textquotesingle s Chinese proficiency, I hear it even surprises Chinese people. }

\par{40. アメリカの ${\overset{\textnormal{もとだいとうりょう}}{\text{元大統領}}}$ といったら、オバマですね。 \hfill\break
Speaking of the former president of America, that\textquotesingle d be Obama, right? }

\par{41. ${\overset{\textnormal{ひとむかしまえ}}{\text{一昔前}}}$ はお ${\overset{\textnormal{すし}}{\text{寿司}}}$ といったら ${\overset{\textnormal{こうきゅう}}{\text{高級}}}$ な ${\overset{\textnormal{た}}{\text{食}}}$ べ ${\overset{\textnormal{もの}}{\text{物}}}$ の ${\overset{\textnormal{だいひょう}}{\text{代表}}}$ でした。 \hfill\break
Long ago when speaking of sushi, it was representative of high class cuisine. }

\par{42. ${\overset{\textnormal{とうほく}}{\text{東北}}}$ で ${\overset{\textnormal{くま}}{\text{熊}}}$ といったらツキノワグマです。 \hfill\break
Speaking of bears in Tohoku, there\textquotesingle s the Asian black bear. }

\par{43. ${\overset{\textnormal{ふゆ}}{\text{冬}}}$ といったら ${\overset{\textnormal{なべりょうり}}{\text{鍋料理}}}$ ですね。 \hfill\break
When it comes to winter, it\textquotesingle s time for hot pot cooking. }

\par{44. ${\overset{\textnormal{くだものおうこく}}{\text{果物王国}}}$ といったらどこでしょうか。 \hfill\break
Speaking of the king of fruit, where would that be? }

\par{45. ${\overset{\textnormal{かぐ}}{\text{家具}}}$ といったらやっぱりイケアですね。 \hfill\break
When it comes to furniture, you definitely go with IKEA. }
    