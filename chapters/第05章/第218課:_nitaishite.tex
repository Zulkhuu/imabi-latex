    
\chapter{~に対して}

\begin{center}
\begin{Large}
第218課: ~に対して 
\end{Large}
\end{center}
 
\par{ ~に対して is something that gets abused and confused with a lot of things. One of the biggest errors with it is using it in the first place. On top of that, it is frequently confused with other phrases as ~にとって and ~について, and students often don\textquotesingle t know when to add は to make ~に対しては. This lesson will hopefully enable you to avoid such errors. }
      
\section{~に対して}
 
\par{ "Noun + ~に対して" gives a meaning of targeting something and facing\slash confronting it. }

\par{1. この頃の親は子供に(対して)甘すぎる。 \hfill\break
Parents these days are being too sweet towards their kids. }

\par{2. 政府に(対して)不満を\{言う・持つ\}な。 \hfill\break
Don't complain about the government. }

\par{3a. 大統領は記者団の質問に対して事実関係を否定した。〇 \hfill\break
3b. 大統領は記者団の質問に事実関係を否定した。X \hfill\break
The president denied all facts of the case to the press group\textquotesingle s questions. }

\par{ At the end of the sentence there is a verb or adjective that shows some sort of urging\slash pressure. It wouldn\textquotesingle t change the meaning to get rid of ~に対して in the sentences above, but it\textquotesingle s used to clearly state the object and direction of action. However, in sentences like the third one, it has to be used because it\textquotesingle s used in a sense of " \textbf{against }". }

\par{4. 今の意見に質問がありますか。今の意見に対してご質問、ご意見があったらお願いします。 \hfill\break
Do you have questions to the opinion now? If you have any questions or ideas concerning this opinion, please feel free. }

\par{5. ${\overset{\textnormal{きせいちゅう}}{\text{寄生虫}}}$ に対して、どうしたらいいでしょうか。 \hfill\break
What should we do against parasites? }

\begin{center}
\textbf{~に対しては }
\end{center}

\par{ This is still used in the spoken language, but it does have somewhat of a formal tone. ~に対しては strengthens the speaker\textquotesingle s judgment\slash feeling by emphasizing\slash contrasting. }

\par{6. 報道官は記者団に対してはまだ何も答えていません。 \hfill\break
The Press Secretary is still not answering anything to the press corps. }

\par{7. 日本人は知らない人に対しては消極的だが、親しい人に対しては積極的な態度を見せることが多い。 \hfill\break
Japanese people often show negative attitudes towards people they don\textquotesingle t know and positive attitudes to people that they are close with. }

\par{8. 危険を与えるようなものに対しては、人間側が配慮すべきであろう。 \hfill\break
The human side should consider things that cause hazard. }

\begin{center}
\textbf{文体 }
\end{center}

\par{ ~に対しまして (very polite) and ~に対し (very stiff\slash formal\slash literary) are also possible. }

\par{9. 今回の ${\overset{\textnormal{ふしょうじ}}{\text{不祥事}}}$ に対して、お ${\overset{\textnormal{わ}}{\text{詫}}}$ び申し上げます。 \hfill\break
I deeply apologize for this scandal. }

\par{10. ご家族の皆様に対し、心よりお ${\overset{\textnormal{く}}{\text{悔}}}$ やみ申し上げます。 \hfill\break
To all of those in the families, please accept my heartfelt condolences. }

\begin{center}
 \textbf{読み物: 殺害の事件 }
\end{center}

\par{3日夜 ${\overset{\textnormal{かながわけん}}{\text{神奈川県}}}$ ${\overset{\textnormal{はやまちょう}}{\text{葉山町}}}$ の住宅で9歳と生後9ヶ月の姉妹が ${\overset{\textnormal{はもの}}{\text{刃物}}}$ で ${\overset{\textnormal{さ}}{\text{刺}}}$ されて殺害された事件で ${\overset{\textnormal{さつじんみすい}}{\text{殺人未遂}}}$ の疑いで ${\overset{\textnormal{たいほ}}{\text{逮捕}}}$ された母親が、警察の調べに対し「 ${\overset{\textnormal{くだもの}}{\text{果物}}}$ ナイフで刺した」と供述しているということで、警察は動機や ${\overset{\textnormal{けいい}}{\text{経緯}}}$ について調べを進めることにしています。 \hfill\break
The mother arrested on the night of the third under the charge of attempted manslaughter of her daughters  aged 9 and 9 months old by stabbing them at home in Hayama Town, Kanagawa Prefecture says in her affidavit in the police investigation that she "stabbed (them) with a fruit knife", and the police are to investigate the motive and cause. }

\par{From NHK on October, 3, 2012. }

\par{1. How is に対して being used in this sentence? \hfill\break
2. Why would using に alone instead be wrong? }

\begin{center}
\textbf{~に対して、というより }
\end{center}

\par{ " \textbf{に対して、というより }" is a very common combination. In this case, you would not see ~に、というより. This would beat the point of making something stand out. }

\par{11. このことは日本人に対して、というよりすべての国の人に知っておいてほしいことです。 \hfill\break
Rather than to Japanese, but I want this to be known to people of all countries. }

\begin{center}
\textbf{連体形 } 
\end{center}

\par{ There are \textbf{two attributive forms }of this pattern that are generally interchangeable, ~ \textbf{に対する and ~に対しての }. }

\par{12. 結婚\{に対しての・に対する\} ${\overset{\textnormal{りょうけ}}{\text{両家}}}$ の ${\overset{\textnormal{かちかん}}{\text{価値観}}}$ の違いに不安を感じます。 \hfill\break
I feel anxiety over the differences of the two families values towards marriage. }

\par{13. ${\overset{\textnormal{おせん}}{\text{汚染}}}$ 問題に対する解決策は ${\overset{\textnormal{よぶん}}{\text{余分}}}$ の法律を可決するというわけではありませんよ。 \hfill\break
The resolution to our pollution problem is not to pass further legislation. }

\begin{center}
 \textbf{数量+~に対して: 割合 } 
\end{center}

\par{ With \textbf{number\slash counter phrases, ~に対して shows a ratio }. }

\par{14. ${\overset{\textnormal{はくさい}}{\text{白菜}}}$ 1キロに対して塩50グラム入れてください。 \hfill\break
For every kilogram of nappa, add 50 grams of salt. }

\par{15. 彼には一人の敵に対して百人もの ${\overset{\textnormal{みかた}}{\text{味方}}}$ がいた。 \hfill\break
For one enemy he had a hundred friends. }

\begin{center}
 \textbf{~に対して VS ~について・に関して }
\end{center}

\par{ Now, when is this phrase similar to ~について and ~に関して? They all have a commonality of targeting something to show concern. However, the kinds of relation they represent are never the same. ~に対して shows something in regards of confronting whereas ~について・に関して state something regarding to the thing itself. }

\par{16a. その問題に対して説明してください。X \hfill\break
16b. その問題について説明してください。〇 \hfill\break
Please explain that problem. }

\par{17. スピード ${\overset{\textnormal{いはん}}{\text{違反}}}$ に対する3万円の ${\overset{\textnormal{はんそくきん}}{\text{反則金}}}$ なんて辛いんだよな。 \hfill\break
A penalty of 30,000 yen for a traffic ticket is harsh, isn't it? }

\par{18a. その意見について反対します。X \hfill\break
18b. その意見に対して反対します。〇 \hfill\break
I am against that opinion. }

\par{19a. 昨日、水不足に関して ${\overset{\textnormal{せっすいせいげん}}{\text{節水制限}}}$ が ${\overset{\textnormal{どうにゅう}}{\text{導入}}}$ されました。X \hfill\break
19b. 昨日、水不足に対して節水制限が導入されました。〇 \hfill\break
Yesterday, water restrictions were introduced concerning the water shortage. }

\par{\textbf{Examples }}

\par{ As one can never have too many examples, the following examples will hopefully give you even more data to pull on in understanding ~に対して. }

\par{20a. 勉強しない学生に対して、 ${\overset{\textnormal{じょめい}}{\text{除名}}}$ しましょう。X \hfill\break
20b. 勉強しない学生を除名しましょう。 \hfill\break
Let\textquotesingle s drop students that don\textquotesingle t study. }

\par{21. ${\overset{\textnormal{ぐんたい}}{\text{軍隊}}}$ は反乱軍に対して ${\overset{\textnormal{かつ}}{\text{嘗}}}$ てないほどの大規模な攻撃をした。 \hfill\break
The army made the biggest ever large-scale attack against the rebel army. }

\par{\textbf{Orthography Note }: 嘗て is normally just written as かつて. }

\par{22. 高田がぐうたらであるのに対して、益田は ${\overset{\textnormal{きんべん}}{\text{勤勉}}}$ だ。 \hfill\break
In comparison to Takada being lazy, Masuda is a hard worker. }

\par{23. 2010年に円はドルに対してずっと ${\overset{\textnormal{てがた}}{\text{手堅}}}$ い動きをしていた。 \hfill\break
The yen stayed firm against the dollar in 2010. }

\par{24. 我が国は、アジア ${\overset{\textnormal{しょこく}}{\text{諸国}}}$ の人々に(対して)多大の ${\overset{\textnormal{そんがい}}{\text{損害}}}$ と ${\overset{\textnormal{くつう}}{\text{苦痛}}}$ を与えました。 \hfill\break
Our country brought great damage and pain to all of the peoples of Asia. \hfill\break
Based off of the 村山談話. }

\par{25. むしこなーずは何の虫に対して ${\overset{\textnormal{こうか}}{\text{効果}}}$ がありますか。 \hfill\break
What bugs does “big corners” have an effect on? }

\par{\textbf{Word Note }: むしこなーず is a kind of bug killer\slash repellent. }

\par{26. 私たちに(対して) ${\overset{\textnormal{えんりょ}}{\text{遠慮}}}$ しないでください。 \hfill\break
Please speak outright to us. }

\par{27. この品は値段に対して品質が悪い。 \hfill\break
As for these goods, the quality is bad for the price. }

\par{28. 日本では ${\overset{\textnormal{へいきん}}{\text{平均}}}$ して職員一人に対して百万円の ${\overset{\textnormal{けいひ}}{\text{経費}}}$ がかかります。 \hfill\break
In Japan, on average, for a single worker, the outlay costs 1 million yen. }

\par{29. 和平に対する北朝鮮の ${\overset{\textnormal{ぎわく}}{\text{疑惑}}}$ は日本の防衛を ${\overset{\textnormal{おびや}}{\text{脅}}}$ かしています。 \hfill\break
Suspicions of North Korea for peace threatens Japanese security. }

\par{30. キックに対してもパンチに対しても使うのですか? \hfill\break
Do you even use it in regards to a kick or even a punch? }

\par{31. 取扱説明書は、コンピュータに対する静電気の ${\overset{\textnormal{きけんせい}}{\text{危険性}}}$ を、間接的に警告します。 \hfill\break
Guidebooks indirectly warn of the danger to computers by static electricity. }

\par{32. 友だちの経済学に対する関心はもっぱら学問上のもののようですが。 \hfill\break
It seems that my friends\textquotesingle  interest in economics is largely academic. }

\par{\textbf{Particle Note }: が shows that the sentence is cut off, perhaps leading into something else. }

\par{33. トヨタのものづくりに対する ${\overset{\textnormal{じょうねつ}}{\text{情熱}}}$ と、日本のものづくりに対するこだわりは ${\overset{\textnormal{そうぎょう}}{\text{創業}}}$ 以来、決して変わらない。 \hfill\break
The passion towards Toyota manufacturing and dwelling on Japanese manufacturing, since our founding, will never change. \hfill\break
By 豊田章男 in April 2013. }
    