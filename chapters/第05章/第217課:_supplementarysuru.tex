    
\chapter{The Supplementary Verb する}

\begin{center}
\begin{Large}
第217課: The Supplementary Verb する 
\end{Large}
\end{center}
 
\par{ Essentially all material in this lesson should be review. What this lesson will try to do is showcase the various usages of the supplementary verb する, which by now you should be accustomed to seeing and using. }
      
\section{With Bound Particles}
 
\par{ Bound particles are often placed after a base and followed by the supplementary verb する. For example, the particles は, も, and や may follow the 連用形  and are then followed by する to greatly emphasize the verb. }

\par{1. 彼女は褒めこそすれ、非難したことはない。 \hfill\break
She does speaks highly (of others), but she never criticizes. }

\par{2. 聞きもしないで批評すんな。 \hfill\break
Don't criticize without listening. }

\par{3. 彼女は日本語が話せはするが、うまくはないね。 \hfill\break
She can speak Japanese, but she's not good. }

\par{4. 責めやしないのはいつまでも最良の決定です。 \hfill\break
Not blaming (others) is the best decision no matter what. }

\par{5. 彼は顔の前で両手を広げ、それからぱたんとあわせる。「世の中にはいっぱい納屋があって、それらがみんな僕に焼かれるのを待っているような気がするんです。海辺にぽつんと建った納屋やら、たんぼのまん中に建った納屋やら・・・・・・とにかく、いろんな納屋です。十五分もあれば ${\overset{\textnormal{きれい}}{\text{綺麗}}}$ に燃えつきちゃうんです。まるでそもそもの最初からそんなもの存在もしなかったみたいにね。誰も \textbf{悲しみゃしません }。ただ―消えちゃうんです。 ${\overset{\textnormal{﹅﹅﹅}}{\text{ぷつん}}}$ ってね」 \hfill\break
He spread out his hands in front of his face, clapped them together with a bang and then said, ”There are a lot of barns in this world, and I feel like waiting for them all to burn. Whether it is the barn standing up all alone on the seashore, the barn built in the middle of a field, just a wide variety of barns. Given fifteen minutes, one can burn down completely quite nicely. It's as if from the very beginning like those barns had no existence. No one is saddened by this. They just disappear, with a \emph{snap }". \hfill\break
From 納屋を焼く by 村上春樹. }

\par{\textbf{Grammar Note }: Notice how は may contract in the construction ”連用形+はする”. }
      
\section{In Honorifics}
 
\par{ する is a very important verb in humble speech. It is seen after the stems of verbs to show first person action. }

\par{6. ご案内します。 \hfill\break
I will guide you. }

\par{7. ご奉納します。 \hfill\break
I will offer (this). }

\par{8. お待ちします。 \hfill\break
I will wait. }
      
\section{With the Particles たり and など}
 
\par{ たり attaches to the 連用形 to primarily show back and forth action and must always be used with する. In the same fashion, など may also list actions by following the 連体形 of a verb. }

\par{9. 休日に雑誌を読むなどして過ごすのは本当につまらない。 \hfill\break
To spend the holidays doing stuff like reading magazines is really boring. }

\par{10. 彼は泣いたり笑ったりばかりしている男だ。 \hfill\break
He's only a guy that cries and laughs. }

\par{11. 遊んだり働いたりしているというのが若さの意味だ。 \hfill\break
The meaning of youth is to play and work. }

\par{12. 彼女は掃除をするなどして暮らしていた。 \hfill\break
She lived on doing things such as cleaning. }
      
\section{With the Volitional Form}
 
\par{ With the auxiliary verbs ~ う, ~よう, and ~まい + と it shows volition in trying to (not) do something. Of course, the first two auxiliaries are positive and the last is negative. }

\par{13. 嘘をつこうとしたが、つけなかった。 \hfill\break
I tried to lie, but I couldn't. }

\par{14. 負けるまいとする。 \hfill\break
To try not to lose. }

\par{15. 彼は歌おうとしたが、うまくなかった。 \hfill\break
He tried to sing, but it wasn't good. }
      
\section{~ようにする}
 
\par{ This usage is translated as "to try\dothyp{}\dothyp{}\dothyp{}". This is to try to do something for the long run, unlike above. Furthermore, the よう in this pattern is from 様: it's not the volitional ending. The volitional ending ~よう came from a sound change when the original volitional contracted from む \textrightarrow  ん and then \textrightarrow  う. With some verb classes, this resulted in the birth of ~よう, most certainly because of pronunciation ease. }

\par{16. できるだけ野菜を食べるようにする。 \hfill\break
I'll try to eat vegetables as much as possible. }

\par{17. もう二度と会わないようにした。 \hfill\break
I tried to not see him a second time. \hfill\break
\hfill\break
18. 遅れないようにしてください。 \hfill\break
Please try not be late. }

\par{\textbf{Variant Note }: ~ようとする is a more formal variant. }
      
\section{~ことにする}
 
\par{ In this speech pattern the speaker shows what he or she decided with his or her own will. Remember that this pattern can also be seen as ~こととする, which is deemed to be more punctual and formal. }

\par{19. 私は酒を飲むのをやめることとしました。 \hfill\break
I have decided to quit drinking liquor. }

\par{20. 東京に引っ越すことにした。 \hfill\break
I have decided to move to Tokyo. }

\par{21. たばこをやめることにしましたよ。 \hfill\break
I've decided to quit smoking! }
      
\section{~としたことが}
 
\par{  Being equivalent to ~ともあろうものが, it shows dismay and regret for someone's abnormal misbehavior. }

\par{22. 彼女としたことが、大変なことを言ってしまった。 \hfill\break
Of all people, she accidentally said something that bad! }
      
\section{~と[すると・したら・すれば]}
 
\par{~と[すると・したら・すれば] means "if there were". The slightly different nuances of the conditional patterns still apply. }

\par{23. 十時に出たとすればもう着いているだろう? \hfill\break
If we had left at 10 o' clock, wouldn't we already have arrived? }

\par{24. ${\overset{\textnormal{よじげん}}{\text{四次元}}}$ の ${\overset{\textnormal{せかい}}{\text{世界}}}$ があるとすれば、どのようなものだと思いますか。 \hfill\break
What do you think of what kind of a world it would be if we assumed that there was a 4 dimensional world? }

\par{25. 仮に戦争が起った としたら 、何千、何百万の人々の命が失われるだろう。 \hfill\break
Supposed that war broke out, tens, hundreds of millions of lives would be lost. }

\par{26. 温泉に行くとしたら、どこがいいのでしょうか。 \hfill\break
If we were to go to a hot spring, where would be good? }

\par{27. 雨だとすると、中止だろう。 \hfill\break
If there's rain, it'll probably be postponed. }

\par{28. 大地震が起こるとすると、どうすればいいのか。 \hfill\break
If there were to be a big earthquake, what would we do? }

\par{29. 大学とすれば、 ${\overset{\textnormal{こうぎ}}{\text{抗議}}}$ しないといけない。 \hfill\break
Assuming it's the university, we must protest. }
    