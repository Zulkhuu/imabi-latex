    
\chapter{Intransitive \& Transitive}

\begin{center}
\begin{Large}
第219課: Intransitive \& Transitive: Part 1 
\end{Large}
\end{center}
 
\par{ Previously, we learned about how the particle を can be used with intransitive verbs to mean “through.” In this lesson, we will look at verbs that don\textquotesingle t change form depending on whether they\textquotesingle re used as a transitive or an intransitive verb. These verbs in Japanese are called 自他同形動詞. }
 
\par{ One mistake that many students as well as educators make all the time is downplay the number and importance of these kinds of verbs. Japanese only has about 300 important intransitive-transitive verb pairs, and a lot of these pairs are not straightforward, and that means all other verbs can either be one or the other…or both. }
 
\par{ Each one of these verbs that can be both deserves special attention. That means we\textquotesingle ll be returning to this topic several times until we\textquotesingle ve truly gone through the mysterious quirks of Japanese transitivity. }
      
\section{閉じる, 伴う, 張る, 開く, 限る, \& 言う}
 
\begin{center}
\textbf{閉じる }
\end{center}

\par{\emph{ }閉じる  means “to close” and can be used with various things such books, the eyes, flip phones, legs, or anything that can be conceptualized as stuck\slash sealed and\slash or put back to how it was. }

\par{1. ${\overset{\textnormal{かさ}}{\text{傘}}}$ を ${\overset{\textnormal{と}}{\text{閉}}}$ じてください。 \hfill\break
Please close your umbrella. }

\par{2. ${\overset{\textnormal{ちょうりご}}{\text{調理後}}}$ も ${\overset{\textnormal{から}}{\text{殻}}}$ が ${\overset{\textnormal{と}}{\text{閉}}}$ じたままの ${\overset{\textnormal{かい}}{\text{貝}}}$ は ${\overset{\textnormal{かなら}}{\text{必}}}$ ず ${\overset{\textnormal{す}}{\text{捨}}}$ ててください。 \hfill\break
Please always throw away shellfish whose shells remain closed even after being prepared. }

\par{3. ${\overset{\textnormal{あか}}{\text{赤}}}$ ちゃんは、 ${\overset{\textnormal{にんしん}}{\text{妊娠}}}$ 8 ${\overset{\textnormal{か}}{\text{ヶ}}}$ ${\overset{\textnormal{げつ}}{\text{月}}}$ に ${\overset{\textnormal{はい}}{\text{入}}}$ ると、 ${\overset{\textnormal{きしょうじ}}{\text{起床時}}}$ には ${\overset{\textnormal{め}}{\text{目}}}$ が ${\overset{\textnormal{ひら}}{\text{開}}}$ いて、 ${\overset{\textnormal{ね}}{\text{寝}}}$ ているときには ${\overset{\textnormal{め}}{\text{目}}}$ が ${\overset{\textnormal{と}}{\text{閉}}}$ じるようになります。 \hfill\break
A baby\textquotesingle s eyes, once one has entered month eight of pregnancy, opens when it wakes up and stays closed when sleeping. }

\par{4. ${\overset{\textnormal{め}}{\text{目}}}$ を ${\overset{\textnormal{と}}{\text{閉}}}$ じてください。 \hfill\break
Please close your eyes. }

\par{ Its transitive usage is similar to the words 閉める and 閉ざす. 閉める means “to shut\slash close” and can be used with various things like doors, lids, shutters, businesses, gates,  etc. When spelled as 締める, it has various other usages such as “to fasten,” “to wear (necktie, belt, etc.), to sum up, etc. Lastly, 閉ざす is used in metaphoric expressions by personifying some emotionally packed word like 心 or 気持ち. }

\par{5. ドアを ${\overset{\textnormal{し}}{\text{閉}}}$ めてください。 \hfill\break
Please shut the door. }

\par{6. ${\overset{\textnormal{てん}}{\text{店}}}$ を ${\overset{\textnormal{し}}{\text{閉}}}$ めるときに ${\overset{\textnormal{う}}{\text{売}}}$ れ ${\overset{\textnormal{のこ}}{\text{残}}}$ った ${\overset{\textnormal{しょうひん}}{\text{商品}}}$ はどうするんですか。 \hfill\break
When closing down a store, what do you do with unsold merchandise? }

\par{7. ${\overset{\textnormal{いと}}{\text{糸}}}$ を ${\overset{\textnormal{し}}{\text{締}}}$ めてください。 \hfill\break
Please fasten the strings. }

\par{8. ${\overset{\textnormal{さ}}{\text{刺}}}$ し ${\overset{\textnormal{み}}{\text{身}}}$ を ${\overset{\textnormal{す}}{\text{酢}}}$ でしめてください。 \hfill\break
Please marinate the sashimi with vinegar. }

\par{9. ${\overset{\textnormal{きょうわとう}}{\text{共和党}}}$ は ${\overset{\textnormal{しゅうぎいん}}{\text{衆議院}}}$ の ${\overset{\textnormal{かはんすう}}{\text{過半数}}}$ を ${\overset{\textnormal{し}}{\text{占}}}$ めています。 \hfill\break
The Republican Party holds the majority in the House of Representatives. }

\par{10. \{ ${\overset{\textnormal{みずか}}{\text{自}}}$ ら・ ${\overset{\textnormal{じぶん}}{\text{自分}}}$ \}の ${\overset{\textnormal{くび}}{\text{首}}}$ を ${\overset{\textnormal{し}}{\text{絞}}}$ める。 \hfill\break
To strangle one\textquotesingle s own neck. }

\par{11. ${\overset{\textnormal{かれ}}{\text{彼}}}$ は ${\overset{\textnormal{かぞく}}{\text{家族}}}$ に ${\overset{\textnormal{こころ}}{\text{心}}}$ を ${\overset{\textnormal{と}}{\text{閉}}}$ ざしてしまっていた。 \hfill\break
He has had his heart shut off to his family. }

\begin{center}
\textbf{伴う }
\end{center}

\par{\emph{ }伴う can be used with either the particles が, \emph{ }に, or を. With the particle が, 伴う means “to accompany” in the sense “mountain climbing accompanied by danger.” With the particle に, \emph{ }伴う means “to accompany” in the sense “to be accompanied with.” With the particle を, 伴う means “to accompany” in the sense “to be accompanied by\slash to bring with…” }

\par{12. ${\overset{\textnormal{さいご}}{\text{最後}}}$ の ${\overset{\textnormal{かてい}}{\text{過程}}}$ には、 ${\overset{\textnormal{そうとう}}{\text{相当}}}$ の ${\overset{\textnormal{ぶんかてきそうしつ}}{\text{文化的喪失}}}$ が ${\overset{\textnormal{ともな}}{\text{伴}}}$ うだろう。 \hfill\break
In the final course, a considerable cultural loss will go hand in hand. }

\par{13. この ${\overset{\textnormal{ぎょうしゅ}}{\text{業種}}}$ は ${\overset{\textnormal{きけん}}{\text{危険}}}$ を ${\overset{\textnormal{ともな}}{\text{伴}}}$ う ${\overset{\textnormal{しごと}}{\text{仕事}}}$ です。 \hfill\break
This type of industry is the kind of job that brings danger with it. }

\par{14. リスクを ${\overset{\textnormal{ともな}}{\text{伴}}}$ う ${\overset{\textnormal{きんゆうしじょう}}{\text{金融市場}}}$ に ${\overset{\textnormal{じゅうじ}}{\text{従事}}}$ する。 \hfill\break
To engage oneself in the finance market which is accompanied with risk. }

\par{15. ${\overset{\textnormal{かぞく}}{\text{家族}}}$ を ${\overset{\textnormal{ともな}}{\text{伴}}}$ って ${\overset{\textnormal{い}}{\text{行}}}$ くのもお ${\overset{\textnormal{すす}}{\text{薦}}}$ めです。 \hfill\break
Going and bringing your family along is also recommended. }

\par{16. ${\overset{\textnormal{ふきょう}}{\text{不況}}}$ に ${\overset{\textnormal{ともな}}{\text{伴}}}$ って、 ${\overset{\textnormal{かいしゃ}}{\text{会社}}}$ の ${\overset{\textnormal{ぎょうせき}}{\text{業績}}}$ が ${\overset{\textnormal{あっか}}{\text{悪化}}}$ した。 \hfill\break
The company\textquotesingle s performance worsened accordingly with the recession. }

\par{17. ${\overset{\textnormal{ちけい}}{\text{地形}}}$ や ${\overset{\textnormal{きせつ}}{\text{季節}}}$ の ${\overset{\textnormal{へんか}}{\text{変化}}}$ に ${\overset{\textnormal{ともな}}{\text{伴}}}$ って ${\overset{\textnormal{きこう}}{\text{気候}}}$ がどのように ${\overset{\textnormal{へんか}}{\text{変化}}}$ するのかがわかります。 \hfill\break
We understand how the climate changes per topography and seasons. }

\begin{center}
\textbf{張る }
\end{center}

\par{ The verb \emph{ }張る has several meanings such as “to stretch\slash strain\slash etc.” among many other things. For the most part, its usages can easily be rephrased from being intransitive to transitive and vice versa. }

\par{18. ${\overset{\textnormal{じぶん}}{\text{自分}}}$ たちでテントを ${\overset{\textnormal{は}}{\text{張}}}$ りました。 \hfill\break
We pitched tents by ourselves. }

\par{19. ${\overset{\textnormal{ね}}{\text{根}}}$ が ${\overset{\textnormal{は}}{\text{張}}}$ っていてなかなか ${\overset{\textnormal{ぬ}}{\text{抜}}}$ き ${\overset{\textnormal{き}}{\text{切}}}$ れない。 \hfill\break
The roots have spread and I can\textquotesingle t seem to completely remove them. }

\par{20. フロントガラスに ${\overset{\textnormal{こおり}}{\text{氷}}}$ が ${\overset{\textnormal{は}}{\text{張}}}$ っている。 \hfill\break
Ice is forming on the windshield. }

\par{21. お ${\overset{\textnormal{ふろ}}{\text{風呂}}}$ に ${\overset{\textnormal{みず}}{\text{水}}}$ を ${\overset{\textnormal{は}}{\text{張}}}$ っておきましょう。 \hfill\break
Let\textquotesingle s fill the bath with water ahead of time. }

\par{22. ${\overset{\textnormal{き}}{\text{気}}}$ が ${\overset{\textnormal{は}}{\text{張}}}$ っていると ${\overset{\textnormal{かぜ}}{\text{風邪}}}$ を ${\overset{\textnormal{ひ}}{\text{引}}}$ かないよ。 \hfill\break
You won\textquotesingle t catch a cold when you\textquotesingle re tensed. }

\par{23. そんなに ${\overset{\textnormal{き}}{\text{気}}}$ を ${\overset{\textnormal{は}}{\text{張}}}$ って ${\overset{\textnormal{つか}}{\text{疲}}}$ れませんか。 \hfill\break
Don\textquotesingle t you get exhausted straining your nerves like that? }

\par{24. ${\overset{\textnormal{こえ}}{\text{声}}}$ を ${\overset{\textnormal{は}}{\text{張}}}$ ってください。 \hfill\break
Raise your voice. }

\par{ When used to mean “to stick\slash post,” is spelled as 貼る. }

\par{25. ${\overset{\textnormal{ふうとう}}{\text{封筒}}}$ に ${\overset{\textnormal{きって}}{\text{切手}}}$ を ${\overset{\textnormal{は}}{\text{貼}}}$ ってください。 \hfill\break
Put a stamp on the envelope. }

\begin{center}
\textbf{開く }
\end{center}

\par{ The verb 開く is both intransitive and transitive, but the subject of the sentence acts differently depending on how it\textquotesingle s used. First, let\textquotesingle s consider the following examples in English. }

\par{i. The rose buds are blooming. \hfill\break
ii. The rose bloomed. \hfill\break
iii. The school door opened on its own. \hfill\break
iv. He opened the door for me. }

\par{ In Japanese, 開く would appear in all four of these sentences. Its usages differ in the emotional state of the subject. If the usage utilizes a subject that has no willful control of itself, then changing the sentence from an intransitive one to a transitive one doesn\textquotesingle t change this fact. }

\par{26. ハスの ${\overset{\textnormal{つぼみ}}{\text{蕾}}}$ が ${\overset{\textnormal{ひら}}{\text{開}}}$ くときはバラのように ${\overset{\textnormal{きれい}}{\text{綺麗}}}$ ですね。 \hfill\break
The lotus blooming is beautiful like roses, huh. }

\par{\textbf{Spelling Note }: ハス \emph{ }may also be spelled as 蓮. バラ may also be spelled as 薔薇. }

\par{27. ${\overset{\textnormal{あお}}{\text{青}}}$ いバラが ${\overset{\textnormal{つぼみ}}{\text{蕾}}}$ を ${\overset{\textnormal{ひら}}{\text{開}}}$ いた。 \hfill\break
The blue roses bloomed. }

\par{ As you can see, 開く can mean “to bloom,” and when its buds bloom, you can describe this as a transitive sentence, with the plant having its buds flower. This, though, is an involuntary action as the act of blooming happens naturally. }

\par{ However, when the subject switches from one that has no volition over itself to one that does, the subject\textquotesingle s willfulness becomes prominent using \emph{ }を. }

\par{28. ドアが ${\overset{\textnormal{ひら}}{\text{開}}}$ いた。 \hfill\break
The door opened. }

\par{29. ${\overset{\textnormal{まど}}{\text{窓}}}$ を ${\overset{\textnormal{ひら}}{\text{開}}}$ きました。 \hfill\break
I opened the window. }

\par{ The next question that is presented here is the existence of two readings for 開く. It may either be read as あく \emph{ }or ひらく. The former is essentially only used as an intransitive verb in the sense of a gap\slash vacancy\slash etc. opening. When used to describe emptiness\slash vacancy, it is spelled as 空く. }

\par{30. ${\overset{\textnormal{あな}}{\text{穴}}}$ が\{ ${\overset{\textnormal{あ}}{\text{開}}}$ いて・ ${\overset{\textnormal{あ}}{\text{空}}}$ いて\}しまった。 \hfill\break
A hole opened. }

\par{31. ${\overset{\textnormal{てんいん}}{\text{店員}}}$ が、 ${\overset{\textnormal{あ}}{\text{空}}}$ いたカップを ${\overset{\textnormal{かたづ}}{\text{片付}}}$ けに ${\overset{\textnormal{き}}{\text{来}}}$ た。 \hfill\break
A clerk came to tidy up the empty cups. }

\par{32. いつなら ${\overset{\textnormal{あ}}{\text{空}}}$ いている? \hfill\break
When are you free? }

\par{33. ${\overset{\textnormal{しょうこ}}{\text{昌子}}}$ は ${\overset{\textnormal{いま}}{\text{今}}}$ 、 ${\overset{\textnormal{くち}}{\text{口}}}$ を ${\overset{\textnormal{ひら}}{\text{開}}}$ いて ${\overset{\textnormal{ね}}{\text{寝}}}$ ています。 \hfill\break
 \emph{Shōko wa ima, kuchi wo [hiraite\slash aite ??] nete imasu. }\hfill\break
Shoko is now sleeping with her mouth open. }

\par{ However, some speakers do use it like in Ex. 33 to indicate involuntary opening that is carried out by a clear agent. Although Shoko may be asleep, she is still the one opening her mouth when she is asleep. As sound as this reasoning may be, most speakers would still either use 開ける or ひらく. }

\par{ As for the difference between 開ける and ひらく, the former is only used to indicate the opening of a partition or exposing a space of some sort. That\textquotesingle s why it may also be used to mean “to empty” when spelled as 空ける. }

\par{34. ${\overset{\textnormal{まど}}{\text{窓}}}$ を ${\overset{\textnormal{あ}}{\text{開}}}$ けました。 \hfill\break
I opened the window. }

\par{35. ${\overset{\textnormal{なかみ}}{\text{中身}}}$ を ${\overset{\textnormal{あ}}{\text{空}}}$ けてください。 \hfill\break
Please empty out its contents. }

\par{36. ${\overset{\textnormal{てん}}{\text{店}}}$ を ${\overset{\textnormal{あ}}{\text{開}}}$ けてください。 \hfill\break
Please open the store. }

\par{ This means that business at a store has begun. The actual "opening” of the store would usually be described as 店をひらく. ひらく tends to be politer and more formal than 開ける whenever both can be used. ひらく, though, indicates two or more surfaces that are pulled apart. Think of eyelids, books, two-part doors and windows. If any such item doesn't lead to the opening of some physical space or content, then 開ける can\textquotesingle t be used. }

\par{37a. ${\overset{\textnormal{ほん}}{\text{本}}}$ を ${\overset{\textnormal{あ}}{\text{開}}}$ けてください。X \hfill\break
37b. ${\overset{\textnormal{ほん}}{\text{本}}}$ を ${\overset{\textnormal{ひら}}{\text{開}}}$ いてください。○ \hfill\break
Please open the book. }

\par{38. ${\overset{\textnormal{まぶた}}{\text{瞼}}}$ を ${\overset{\textnormal{あ}}{\text{開}}}$ けてください。 \hfill\break
Please open your eye(lids). }

\par{\textbf{Spelling Note }: 瞼 \emph{ }may be alternatively spelled as 目蓋. }

\par{39. ${\overset{\textnormal{め}}{\text{目}}}$ を ${\overset{\textnormal{ひら}}{\text{開}}}$ いてください。 \hfill\break
Please open your eye. }

\par{\textbf{Nuance Note }: In this last example, ひらく has a deeper meaning beyond the literal physical act of opening one\textquotesingle s eyes. }

\par{ Below are more examples of ひらく to showcase more of its scope of use. }

\par{40. ${\overset{\textnormal{こうざ}}{\text{口座}}}$ を ${\overset{\textnormal{ひら}}{\text{開}}}$ いてみませんか。 \hfill\break
Why not try to open a bank account? }

\par{41. ${\overset{\textnormal{ばっしご}}{\text{抜糸後}}}$ 、 ${\overset{\textnormal{きずぐち}}{\text{傷口}}}$ が ${\overset{\textnormal{あ}}{\text{開}}}$ いてしまいました。 \hfill\break
The wound opened after having my stitches removed. }

\par{42. ${\overset{\textnormal{きょうかしょ}}{\text{教科書}}}$ を ${\overset{\textnormal{ひら}}{\text{開}}}$ いてください。 \hfill\break
Please open your textbook. }

\par{43. ${\overset{\textnormal{ていきてき}}{\text{定期的}}}$ に ${\overset{\textnormal{てんじかい}}{\text{展示会}}}$ を ${\overset{\textnormal{ひら}}{\text{開}}}$ いています。 \hfill\break
We\textquotesingle re routinely holding exhibitions. }

\par{44. ${\overset{\textnormal{さかな}}{\text{魚}}}$ を ${\overset{\textnormal{ひら}}{\text{開}}}$ いて、 ${\overset{\textnormal{ほね}}{\text{骨}}}$ を ${\overset{\textnormal{と}}{\text{取}}}$ り ${\overset{\textnormal{のぞ}}{\text{除}}}$ きましょう。 \hfill\break
(Let\textquotesingle s) cut open the fish and remove the bones. }

\par{45. ${\overset{\textnormal{たいらのきよもり}}{\text{平清盛}}}$ は ${\overset{\textnormal{そう}}{\text{宋}}}$ との ${\overset{\textnormal{こっこう}}{\text{国交}}}$ を ${\overset{\textnormal{ひら}}{\text{開}}}$ いて ${\overset{\textnormal{ぼうえき}}{\text{貿易}}}$ を ${\overset{\textnormal{しんこう}}{\text{振興}}}$ した。 \hfill\break
Taira no Kiyomori opened up diplomatic relations with the Song Dynasty and promoted trade. }

\par{46. ${\overset{\textnormal{みなもとのよりとも}}{\text{源頼朝}}}$ が ${\overset{\textnormal{かまくら}}{\text{鎌倉}}}$ に ${\overset{\textnormal{ばくふ}}{\text{幕府}}}$ を ${\overset{\textnormal{ひら}}{\text{開}}}$ いた ${\overset{\textnormal{りゆう}}{\text{理由}}}$ は ${\overset{\textnormal{なに}}{\text{何}}}$ ですか。 \hfill\break
What is the reason for why Minamoto no Yoritomo opened the Bakufu Shogunate in Kamakura? }

\par{47. ${\overset{\textnormal{せっていがめん}}{\text{設定画面}}}$ を ${\overset{\textnormal{ひら}}{\text{開}}}$ いてください。 \hfill\break
Please open the settings screen. }

\par{48. ${\overset{\textnormal{さんりん}}{\text{山林}}}$ を\{ ${\overset{\textnormal{ひら}}{\text{拓}}}$ いて・ ${\overset{\textnormal{かいたく}}{\text{開拓}}}$ して\} ${\overset{\textnormal{のうち}}{\text{農地}}}$ にしました。 \hfill\break
I opened up the forest and mountain and turned it into farmland. }

\par{\textbf{Spelling Note }: In the sense of “to open up (land),” ひらく may also be spelled as 拓く. }

\begin{center}
\textbf{限る }
\end{center}

\par{ The verb 限る can be both a transitive meaning "to restrict\slash limit" and "to be restricted." Its intransitive usage is discussed at length in Lesson 226. The transitive sense is frequently used in the passive form. The intransitive form, as you will see, has no active agent. Like most other intransitive verbs, it lacks volition. This is how you can differentiate it from its transitive form, which is the opposite of this. }

\par{・Transitive Examples }

\par{49. ${\overset{\textnormal{ちゅうしゃ}}{\text{駐車}}}$ は ${\overset{\textnormal{いち}}{\text{1}}}$ ${\overset{\textnormal{じかん}}{\text{時間}}}$ に ${\overset{\textnormal{かぎ}}{\text{限}}}$ られています。 \hfill\break
Parking is limited to one hour. }

\par{50. ${\overset{\textnormal{ひがいしゃ}}{\text{被害者}}}$ を ${\overset{\textnormal{じょせい}}{\text{女性}}}$ に ${\overset{\textnormal{かぎ}}{\text{限}}}$ っている ${\overset{\textnormal{げんざい}}{\text{現在}}}$ の ${\overset{\textnormal{きてい}}{\text{規定}}}$ を ${\overset{\textnormal{みなお}}{\text{見直}}}$ し、 ${\overset{\textnormal{せいべつ}}{\text{性別}}}$ にかかわらず ${\overset{\textnormal{ひがいしゃ}}{\text{被害者}}}$ になりうる。 \hfill\break
(The government) is to re-examine the current stipulation restricting victims to women so that people may be (deemed) victims regardless of sex. }

\par{51. ${\overset{\textnormal{たいしょうしゃ}}{\text{対象者}}}$ を ${\overset{\textnormal{せいじん}}{\text{成人}}}$ に ${\overset{\textnormal{かぎ}}{\text{限}}}$ っています。 \hfill\break
We are limiting the target group to adults. }

\par{・ Intransitive Examples }

\par{52. うちの ${\overset{\textnormal{こども}}{\text{子供}}}$ に ${\overset{\textnormal{かぎ}}{\text{限}}}$ ってそんなはずがない。 \hfill\break
That could never happen to one's own child. }

\par{53. ${\overset{\textnormal{いそ}}{\text{急}}}$ いでいる ${\overset{\textnormal{とき}}{\text{時}}}$ に ${\overset{\textnormal{かぎ}}{\text{限}}}$ って ${\overset{\textnormal{しんごう}}{\text{信号}}}$ に ${\overset{\textnormal{つぎつぎ}}{\text{次々}}}$ (と) ${\overset{\textnormal{と}}{\text{止}}}$ められたことはありますか。 \hfill\break
Have you ever been stopped by lights one after another particularly when you were in a hurry? }

\begin{center}
\textbf{言う }
\end{center}

\par{ You know how the verb 言う as a transitive verb is used to mean "to say." You also know how it's used as a supplementary verb in grammar patterns such as という. As an intransitive verb, it is used to mean "to make a sound." In this sense, it is used with various onomatopoeic expressions. }

\par{\textbf{Pronunciation Note }: Remember that this verb is technically pronounced as "yū." }

\par{54. ${\overset{\textnormal{いぬ}}{\text{犬}}}$ はワンワンと ${\overset{\textnormal{い}}{\text{言}}}$ って ${\overset{\textnormal{なに}}{\text{何}}}$ か ${\overset{\textnormal{しゃべ}}{\text{喋}}}$ っているのですか。 \hfill\break
When dogs bark, are they saying something? }

\par{55. ベッドに ${\overset{\textnormal{すわ}}{\text{座}}}$ るとミシミシ(と) ${\overset{\textnormal{い}}{\text{言}}}$ う ${\overset{\textnormal{おと}}{\text{音}}}$ は ${\overset{\textnormal{となり}}{\text{隣}}}$ に ${\overset{\textnormal{き}}{\text{聞}}}$ こえるんですか。 \hfill\break
When you sit in your bed, can the creaking be heard next door? }

\par{56. プロポーズって ${\overset{\textnormal{なに}}{\text{何}}}$ を ${\overset{\textnormal{い}}{\text{言}}}$ えばいいの? \hfill\break
What should you say in a (marriage) proposal? }
    