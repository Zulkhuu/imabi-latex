    
\chapter{No Doubt that}

\begin{center}
\begin{Large}
第212課: No Doubt that: ~に違いない, ~に相違ない, \& ~に決まっている 
\end{Large}
\end{center}
 
\par{ ~に違いない and ~に相違ない are not necessarily difficult, but there are certain problems that students have. The unusual way it connects to adjective\slash verb expressions, not properly understanding what can follow them, and not understanding formality differences are the main sources of error. The lesson will be ended with the similar ~に決まっている. }
      
\section{~に違いない VS ~に相違ない}
 
\par{ First and foremost, these expressions are both translated as “there is without a doubt that…”. This is the immediate source of confusion because you can add phrases like “I think that” or “probably” in English without causing ungrammaticality. However, with these Japanese expressions, doing so does. First, consider the following defining of these terms. }

\par{\textbf{~に違いない }: Not from objective proof or logical speculation, but rather from the speaker\textquotesingle s own experience, this phrase shows one\textquotesingle s intuitive speculation\slash confidence. With such a bold move, it is without surprise that this is quite strong. It is often used in situations where it's as if one is talking to oneself in attempts to verify one\textquotesingle s own guess or deliberation. }

\par{\textbf{~に相違ない }: Like ~に違いない, it shows intuitive speculation\slash confidence, but it is more stiff and formal. This formality difference is very important to keep in mind. One reason is that speech modals more akin to the spoken language like the final particle から, which is commonly used with ~に違いない, are not typically used with ~相違ない. This also implies that there is no mistake in it; thus, it gives a more confident tone. Thus, although adverbs like きっと are common with ~に違いない, it is not with ~に相違ない due to the tone. Although perhaps of old logic, some speakers feel this is more so indicative of middle-aged\slash old men as they are more likely to use old-fashioned expressions, which would not be old-fashioned for them, and fits the traditional tone for masculine speech. }

\par{These patterns attach to phrases in the following manner. }

\begin{ltabulary}{|P|P|}
\hline 

Nouns & N(である)+~に\{違い・相違\}ない \\ \cline{1-2}

形容詞 & 形容詞 + ~に\{違い・相違\}ない \\ \cline{1-2}

形容動詞 & 形容詞(である) + ~に\{違い・相違\}ない \\ \cline{1-2}

Verbs & Verb + ~に\{違い・相違\}ない \\ \cline{1-2}

\end{ltabulary}

\par{ Notice that it does not say Adj\slash V + こと + に\{違い・相違\}ない. It is a set phrase, and like other phrases with に, the reason for why this is allowed stems from a Classical Japanese grammatical maneuver of using conjugational parts of speech as nominal phrases when in the 連体形. As this base has changed appearance for many items, it's not surprising that in Modern Japanese this technique is limited. }

\begin{center}
 \textbf{Examples }
\end{center}

\par{1. 政府の報道官の応答はどんなに(か)悔しかったに違いない。 \hfill\break
The response of the government spokesperson was in no doubt overly regrettable. }

\par{\textbf{Grammar Note }: In a more literary sense どんなに(か) may be replaced with いかばかりか. Also note that the use of か with this in the first place is hardly ever heard. }

\par{2. 違いない(その通り)、君の言う通りだ。 \hfill\break
That\textquotesingle s right. It's just as you say. }

\par{3. この計画の実行は困難に違いない。 \hfill\break
It is without a doubt that this plan's implementation is difficult. }

\par{4. ${\overset{\textnormal{どろぼう}}{\text{泥棒}}}$ が入ったに違いない。 \hfill\break
There is no doubt that a robber came in. }

\par{5. ${\overset{\textnormal{ばかもの}}{\text{馬鹿者}}}$ には違いないが、責任を取らざるをえないだろう。 \hfill\break
Although it's without a doubt that you're an idiot, there's no other way but to take responsibility. \hfill\break
By ${\overset{\textnormal{おさらぎじろう}}{\text{大仏次郎}}}$ }

\par{${\overset{\textnormal{}}{\text{}}}$ }

\par{6. 明日は ${\overset{\textnormal{くも}}{\text{曇}}}$ るに違いない。 \hfill\break
It will definitely be cloudy tomorrow. }

\par{7. 社長は天才であることに相違ない。 \hfill\break
There's no doubt that the company president is a genius. }

\par{8. 「あの女の人は誰でしょうか」「ケンさんのガールフレンドに違いありません。手を ${\overset{\textnormal{つな}}{\text{繋}}}$ いで一緒に歩いていますから」 \hfill\break
"Who is that woman?" "There's no doubt that she's Ken's girlfriend because they're holding hands walking together". }

\par{9. 「あの人は ${\overset{\textnormal{なにご}}{\text{何語}}}$ の学生でしょうか」「日本語の学生に違いありません。読んでいるものに「新しい」と書いてありますから」 \hfill\break
"What language student is that person? "That person is no doubt a Japanese student because the thing (that person) is reading has "atarashii" written on it" }

\par{10. 「あの人は日本に住んでいたんでしょうか」「そうに違いありません。日本の新聞を読んでいますから」 \hfill\break
"I wonder if that person lived in Japan" There's no doubt about it because he's reading a Japanese newspaper" }

\par{11. 「あの女の人は結婚しているんでしょうか」「そうに違いありません。 ${\overset{\textnormal{ゆびわ}}{\text{指輪}}}$ を ${\overset{\textnormal{は}}{\text{嵌}}}$ めていますから」 \hfill\break
"I wonder if that woman is married" "There's no doubt about it because she's wearing a ring" }
      
\section{~に決まっている}
 
\par{ ~に決まっている means "it is certainly\dothyp{}\dothyp{}\dothyp{}" and follows nouns, adjectives, or verbs. This speech modal shows 100\% certainty and is reflective of 話し言葉. It shows that the speaker is very confident in labeling something as so. It is also used in chastising. }

\par{12. 彼は ${\overset{\textnormal{うそ}}{\text{嘘}}}$ をついているに決まっている。 \hfill\break
He's certainly lying. }

\par{13. 夏は暑いに決まっている。 \hfill\break
It is certainly hot in summer. }

\par{14. 負けるに決まってる。(Casual) \hfill\break
(They) will certainly lose. }
    