    
\chapter{Circumstance}

\begin{center}
\begin{Large}
第215課: Circumstance: 場合 \& ~に備えて 
\end{Large}
\end{center}
 
\par{ 場合 means "circumstance" and the grammar behind it and its usage are very important. }
      
\section{場合}
 
\par{ 場合 is often in ~場合(は) to mean "in the case\slash event of\dothyp{}\dothyp{}\dothyp{}". It is a nominal phrase, so it is not limited to this grammatical construction. It is very similar to the particle ~たら in that it presents a hypothetical. As it is used to pinpoint a circumstance, it often replaces the particle in doing so. Though you see it after ~た, it is a hypothetical situation of what could happen. So, it is inappropriate if the situation that you are raising is actually of the past. }

\par{1. 火事の場合、段階を使いなさい。 \hfill\break
In the case of fire, use staircases. }

\par{2. 英語の場合はちゃんとイエスやノーを言いますが、日本語の場合は違います。 \hfill\break
In the case of English, you precisely say yes or no, but in the case of Japanese, it\textquotesingle s different. }

\par{3. 何か問題があった場合、 ${\overset{\textnormal{だれ}}{\text{誰}}}$ に ${\overset{\textnormal{れんらく}}{\text{連絡}}}$ すればいいですか。 \hfill\break
In case of any problems, who would be good to contact? }

\par{ It is very easy to use this phrase when stating a particular situation after stating the norm. After all, you are pinpointing a hypothetical that is important. }

\par{4a. 日曜や祝日は手数料は要らないが、平日の場合は手数料がかかる。 \hfill\break
4b. 日曜や祝日は手数料は要らない。ただし、平日の場合は手数料がかかる。 \hfill\break
There is no need for handling fees on Sundays and holidays, but on week days there is a charge. }

\par{\textbf{Grammar Note }: The two paraphrases are important in showing different situations where 場合 may be used, with the second option clearly being used in showing it in a two sentence statement. }

\par{\textbf{Particle Note }: The difference between ~場合, ~場合は, and ~場合に is essentially the same as with ~とき. The second option, of course, is used particularly when emphatically raising something or creating a contrast with another situation. The third option is used easily when the following clause involves an action\slash change. }

\par{5. }

\par{A: このゴムはどう使うんですか。 \hfill\break
B: 椅子がガタガタする場合に、ここに嵌めてください。 \hfill\break
A: ああ、なるほど。 \hfill\break
B: 通常の場合は、椅子の後ろのポケットに入れておいてください。 }

\par{A: How do you use this rubber? \hfill\break
B: Fit it in here in the event that the chair rattles. \hfill\break
A: Ah, I see. \hfill\break
B: In regular cases, place it the pocket behind the chair. \hfill\break
From 中級日本語文法と教え方のポイント by 市川保子. }

\par{6. 火事の場合は、エレベーターを使わないでください。 \hfill\break
In the event of a fire, please do not use the elevators. }

\par{7. 地震が起こった場合、階段を使ってください。 \hfill\break
In the case of an earthquake, use the stairs. }

\par{8. エレベーターに閉じ込められた場合、長期戦を ${\overset{\textnormal{かくご}}{\text{覚悟}}}$ して体力を ${\overset{\textnormal{しょうもう}}{\text{消耗}}}$ しないようにしてください。 \hfill\break
In the case you are trapped in an elevator, try not to waste your energy and prepare yourself for long     fight. }

\par{9. 地震が起こった場合、 ${\overset{\textnormal{あわ}}{\text{慌}}}$ てて階段を降りたり ${\overset{\textnormal{のぼ}}{\text{昇}}}$ ったりするのは危険なのでその場にしゃがんで待機しましょう。 \hfill\break
In the event of an earthquake, because it is dangerous to hastily go up and down the stairs, let\textquotesingle s crouch in that place and be on standby. }

\par{10. ${\overset{\textnormal{さいあく}}{\text{最悪}}}$ の ${\overset{\textnormal{ばあい}}{\text{場合}}}$ を覚悟する。 \hfill\break
To prepare for the worst. }

\par{11. それは極端な場合だよ。 \hfill\break
That's an extreme case. }

\par{12. 場合によりけりだ。 \hfill\break
It depends on the case. }

\par{13. 雨天の場合には、お電話ください。 \hfill\break
Call me in the case that it rains. }

\par{14. 緊急の場合は警察を呼びなさい。 \hfill\break
In the case of an emergency, call the police. }

\par{ ~ている場合 is used to criticize someone that is not dealing with a tense situation appropriately. }

\par{15. 笑ってる場合か? \hfill\break
Is this really the time to be laughing? }

\par{ Also, ~場合\{が・も\}ある, is similar to ことがある because it shows that something happens given a certain situation but doesn't always happen. }

\par{16. たまに遅刻する場合がある。 \hfill\break
There are times when I'm late. }
      
\section{~に備えて}
 
\par{ The 一段 verb 備える has several related meanings: "to provide","to furnish", "to have (attributes)". In ~に備えて, it is equivalent to either "in case of", "for", etc. and is similar to phrases like ~ために and ~しないように. }

\par{17. 大洪水に備えて ${\overset{\textnormal{ようへき}}{\text{擁壁}}}$ を造る。 \hfill\break
To build a retaining wall to prevent a big flood. }

\par{18. いつ起こるか分からない災害に備えておきましょう。 \hfill\break
Let's prepare for the disasters we can't see predict. }

\par{19. 事故に備えて車ではいつもシートベルトを締めていてください。 \hfill\break
Always keep your seat belt on in a car in case of an accident. }

\par{20. 私はオリンピックに備えて練習しています。 \hfill\break
I am training for the Olympics. }

\par{21. 台風に備えるのは重要だ。 \hfill\break
Preparing for typhoons is important. }
    