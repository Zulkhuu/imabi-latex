    
\chapter{~において・における}

\begin{center}
\begin{Large}
第231課: ~において・における 
\end{Large}
\end{center}
 
\par{  In this lesson we'll learn about the compound particle ~において and its attribute form ~における. }
      
\section{~において・おける}
 
\par{ ~において・おける can be used to mark the place of an event or field of activity. As you could imagine, ~における is the attribute form. }
 
\par{1. 国会における発言 \hfill\break
The remark in the Diet }
 
\par{2. 大会は上海において1週間にわたって開かれます。 \hfill\break
The convention will be open in Shanghai over a week\textquotesingle s time. }
 
\par{3. 来年の総会はベルリンにおいて行われます。 \hfill\break
Next year\textquotesingle s general meeting will be held in Berlin. }
 
\par{4. 会議は第一会議室において行われました。 \hfill\break
The meeting was held in the first conference room. }
 
\par{ Unlike で, however, it can be used to specify a non-physical location. It could still be paraphrased out, but lack of emphasis and this connection are as consequence. The examples below, though, aren\textquotesingle t truly quite like existential statements. Rather, they force emphasis on the setting in which the attributed item is in. }
 
\par{5. 人生におけるいくつかの過ちと選択 \hfill\break
The several mistakes and choices in human life \hfill\break
From the title of a book by ウォーリー・ラム. }
 
\par{6. それは私の人生における最良の日であった。 \hfill\break
That was the best day in my life. }
 
\par{7. 恋と戦争においては全てが正当であるというのは本当かな。 \hfill\break
I wonder if all is truly fair in love and war. }
 
\par{ When it is used to show the time of something, it is interchangeable with the particle に. Unlike に, though it does not indicate a specific time. ~における, as one would imagine, is interchangeable with の. This only means that whenever you can use ~における, you could use の instead, not the other way around. }
 
\par{\textbf{Speech Style Note }: It\textquotesingle s very formal and usually replaced by other things in 話し言葉. As such, it is rarely ever used in reference to personal activity. }
 
\par{8. 私は毎日友達の家において勉強します。X }

\par{ \textbf{Orthography Note }: It can be written in 漢字 as に[於いて・於ける]. You may often see “at” written as 於 on info cards. }
 
\par{9. 過去においての出来事 \hfill\break
Happenings in the past }
 
\par{10. 地震などの災害時においては、特に正確な情報が必要である。 \hfill\break
In times of natural disaster such as earthquakes, it\textquotesingle s important to especially have accurate  information. }
 
\par{11a. 2014年のG8のサミットはブラジルにおいて行われます。(More formal) \hfill\break
11b. 2014年のG8のサミットはブラジルで行われます。 \hfill\break
The 2014 G8 summit will be held in Brazil. }
 
\par{12. 彼は経済学において\{優れて・秀でて\}います。 \hfill\break
He is outstanding in the field of economics. }
 
\par{13. 人生に於ける使命感のある人。 \hfill\break
A person who has a sense of mission in life. }

\par{14. ${\overset{\textnormal{あぜ}}{\text{畦}}}$ は、 ${\overset{\textnormal{いなさくのうぎょう}}{\text{稲作農業}}}$ において、水田と水田の境に ${\overset{\textnormal{でいど}}{\text{泥土}}}$ を盛り上げて、水が外に漏れないようにしたものである。 \hfill\break
Causeways in rice farming are things made for water to not leak out from piling up mud on the boarders of rice patty fields. }

\par{From ウィキペディア. }
 
\par{15. 絵画においても、音楽においても、彼女より才能に恵まれた人はあまりいない。 \hfill\break
In both art and music, there aren\textquotesingle t many more gifted than her. \hfill\break
Adapted from \emph{A Dictionary of Japanese Particles }by Sue A. Kawashima. }
 
\par{16. 家族や友だち等と一緒の食卓においては、子どもの心身の成長・発達の変化を日々観察することが可能である。 \hfill\break
It is possible to observe the development changes and mind and soul growth of children daily at the dinner table together with family and friends. \hfill\break
From 保育所における食事の提供ガイドライン }
 
\par{ It may also be used in the sense of “regards to” and translated as such to be “at\slash in”. }
 
\par{17. 技術において他国に優る。 \hfill\break
To be superior to other countries in technology. }
 
\par{18. 個人の名において \hfill\break
As to individual names \hfill\break
 }

\par{19. 統計の分析の能力において彰子は非常に優れている。 \hfill\break
In regards to statistic analysis, Shoko is extremely excellent. }
 
\par{20. この点において私と彼女とは意見が食い違っています。 \hfill\break
As for this point, my opinion differs with hers. }
 
\par{21. 自分の責任において \hfill\break
At one\textquotesingle s risk }
 
\par{\textbf{Etymology Note }: 於ける comes from the verb おく's 命令形 + ~り in its 連体形. ~り is an archaic auxiliary that shows completion. }
    