    
\chapter{Circumstance}

\begin{center}
\begin{Large}
第247課: Circumstance: まま, 思いきや, \& もと 
\end{Large}
\end{center}
       
\section{まま}
 
\par{ " \textbf{As is }", with verbs it is primarily seen in the pattern ~たまま(で). It is used to show action "as is" in a certain state without any change in course or situation. However, it is not necessarily the case that the subject in question is still. However, there is a very similarly meaning particle, なり, that we will learn about later which requires the subject be still. So, this is something to keep in mind. }

\par{ It is not to say that this word is only used with the past tense, although for this meaning it most certainly is. However, in other usages you may see it used with the negative or demonstratives to show how a state is still the same without there being any change, which is of the same vein as above. }

\par{ Though not really different in meaning, there is also the pattern ままに which is used to show that one leaves something to the course of a situation or to show things are going as thought. In other words, it shows something being left to a natural course of action. You may see this after verbs in the non-past form and even in the passive. You may also see ~がままに, which is very formal and 書き言葉的. A common phrase utilizing this older grammar is 思うがままに. In more modern Japanese, this would become 思いのままに. }

\begin{center}
\textbf{Examples } 
\end{center}

\par{${\overset{\textnormal{}}{\text{1. 昔}}}$ のままの ${\overset{\textnormal{}}{\text{風景}}}$ だよね。 \hfill\break
This scenery is just as it was in the old times, isn't it? }
 
\par{2. そのままにしておいた。 \hfill\break
I left it alone the way it was. }
 
\par{${\overset{\textnormal{}}{\text{3. 何}}}$ かを ${\overset{\textnormal{なま}}{\text{生}}}$ のまま ${\overset{\textnormal{}}{\text{食}}}$ べたことがありますか。 \hfill\break
Have you ever eaten something raw? }
 
\par{${\overset{\textnormal{}}{\text{4a. 意}}}$ のままに( ${\overset{\textnormal{}}{\text{思}}}$ ったように) ${\overset{\textnormal{}}{\text{歌}}}$ う ${\overset{\textnormal{}}{\text{自由}}}$ があるよ。 \hfill\break
4b. 意のままに(思ったように)歌っていいんだよ。(More common) \hfill\break
I have the freedom to sing at will. }
 
\par{${\overset{\textnormal{}}{\text{5. 彼}}}$ は、 ${\overset{\textnormal{}}{\text{本能}}}$ のままに ${\overset{\textnormal{}}{\text{行動}}}$ しただけです。 \hfill\break
He only acted out of instinct. }
 
\par{${\overset{\textnormal{}}{\text{6. 窓}}}$ を ${\overset{\textnormal{}}{\text{開}}}$ けたままにしておいてください。 \hfill\break
Please leave the window open. }
 
\par{7. このままでは ${\overset{\textnormal{}}{\text{必}}}$ ず ${\overset{\textnormal{}}{\text{死}}}$ んでしまう。 \hfill\break
As it is now, we will surely die. }
 
\par{${\overset{\textnormal{}}{\text{8. 彼女は電気}}}$ をつけたまま ${\overset{\textnormal{}}{\text{寝}}}$ ちゃった。(砕けた) \hfill\break
She slept with the lights on. }
 
\par{9. このままお ${\overset{\textnormal{}}{\text{待}}}$ ち ${\overset{\textnormal{}}{\text{下}}}$ さい。(On the phone) \hfill\break
Please hold the line. }
 
\par{${\overset{\textnormal{}}{\text{10. 会議}}}$ がありますから、 ${\overset{\textnormal{いす}}{\text{椅子}}}$ はこのままにしておいてください。 \hfill\break
There's going to be a meeting, so please leave the chairs the way they are. }

\par{11. ${\overset{\textnormal{き}}{\text{着}}}$ の ${\overset{\textnormal{みき}}{\text{身着}}}$ のままで ${\overset{\textnormal{}}{\text{逃}}}$ げること \hfill\break
Running away with only one's clothes. }

\par{12. 自然のままで ${\overset{\textnormal{すてき}}{\text{素敵}}}$ ですわね。(Very feminine) \hfill\break
It's great keeping it natural. }
 
\par{\textbf{漢字 Note }: まま can be written in 漢字 as either 儘・侭. }

\par{\textbf{Variant\slash Pronunciation Note }: Depending on the speaker, with region being a significant factor, this may also be seen\slash pronounced as まんま. }
 
\par{ ほしいまま, a very important phrase that uses mama that is written in 漢字 as either 恣, 縦, or 擅, means "selfish". }
 
\par{${\overset{\textnormal{}}{\text{13. 権勢}}}$ を ${\overset{\textnormal{}}{\text{恣}}}$ にする。 \hfill\break
To exert one's power at will. }
 
\par{${\overset{\textnormal{}}{\text{14. 世界最高}}}$ の ${\overset{\textnormal{}}{\text{日本語}}}$ (の) ${\overset{\textnormal{}}{\text{教授}}}$ としての ${\overset{\textnormal{}}{\text{名声}}}$ をほしいままにしてるぞ! \hfill\break
I enjoy the reputation of being the greatest Japanese professor in the world! }
      
\section{思いきや}
 
\par{ ~と思いきや is equivalent to either "despite having thought" or "contrary to expectations". Its literal translation is "just as I thought\dothyp{}\dothyp{}\dothyp{}". Tense is determined by the final verb. }
 
\par{15. あっさり ${\overset{\textnormal{}}{\text{断}}}$ られると ${\overset{\textnormal{}}{\text{思}}}$ いきや、 ${\overset{\textnormal{}}{\text{彼女}}}$ は ${\overset{\textnormal{しょうだく}}{\text{承諾}}}$ してくれました。 \hfill\break
Despite having thought that she would just refuse, she consented to it. }
 
\par{16. このレストランは ${\overset{\textnormal{}}{\text{安}}}$ いと ${\overset{\textnormal{}}{\text{思}}}$ いきや、 ${\overset{\textnormal{}}{\text{会計}}}$ は5000 ${\overset{\textnormal{}}{\text{円以上}}}$ だったよ。 \hfill\break
Contrary to thinking that this restaurant was cheap, the bill was over five thousand yen! }
 
\par{${\overset{\textnormal{}}{\text{17. 彼}}}$ はもう ${\overset{\textnormal{}}{\text{帰}}}$ ってきたと ${\overset{\textnormal{}}{\text{思}}}$ いきや、 ${\overset{\textnormal{}}{\text{彼}}}$ にびっくりした。 \hfill\break
Just as I thought he had gone home, I was scared by him. }
 
\par{${\overset{\textnormal{}}{\text{18. 誰}}}$ もあの ${\overset{\textnormal{}}{\text{講座}}}$ に ${\overset{\textnormal{}}{\text{出席}}}$ しないと ${\overset{\textnormal{}}{\text{思}}}$ いきや、 ${\overset{\textnormal{}}{\text{大}}}$ ${\overset{\textnormal{}}{\text{勢}}}$ ${\overset{\textnormal{}}{\text{出}}}$ ${\overset{\textnormal{}}{\text{席}}}$ しました。 \hfill\break
Despite having thought that no one would attend that lecture, a lot of people attended. }
 
\par{${\overset{\textnormal{}}{\text{19. 日本}}}$ はどこに ${\overset{\textnormal{}}{\text{行}}}$ っても ${\overset{\textnormal{じゅうたい}}{\text{渋滞}}}$ で ${\overset{\textnormal{}}{\text{遅}}}$ くなると ${\overset{\textnormal{}}{\text{思}}}$ いきや、 ${\overset{\textnormal{}}{\text{交通}}}$ はアメリカのように ${\overset{\textnormal{}}{\text{普通}}}$ だった。 \hfill\break
Despite having thought that you would be made to slow down by congestion wherever you go in Japan, the traffic was normal like America. }
 
\par{\textbf{Definition Note }: 思いきや is equivalent to 思っていたところが. }
      
\section{もと}
 
\par{ もと may be written in 漢字 in different ways depending on how it is interpreted. You will see this word again in regards to ~をもとにして. }

\par{下・許 }

\par{When written as such, it means "under" in a physical sense. It can also refer to be under rules, forces, etc. のもとで  and のもとに are both possible, but the former refers to action\slash movement whereas the latter refers to existence\slash static situation. Both expressions are rather literary, but the latter is even more so. }

\par{20. ${\overset{\textnormal{ぎょうせい}}{\text{行政}}}$ の ${\overset{\textnormal{}}{\text{保護}}}$ の ${\overset{\textnormal{}}{\text{下}}}$ で ${\overset{\textnormal{}}{\text{税}}}$ を ${\overset{\textnormal{ちょうしゅう}}{\text{徴収}}}$ している。 \hfill\break
I'm collecting taxes under the protection of the administration. }
 
\par{${\overset{\textnormal{}}{\text{21. 法}}}$ の ${\overset{\textnormal{}}{\text{許}}}$ に \hfill\break
Under the law }

\par{22. 勇将の下に弱卒なし。 \hfill\break
There are no weak soldiers under a strong\slash brave general. }
 
\par{${\overset{\textnormal{}}{\text{23. 長男}}}$ は ${\overset{\textnormal{}}{\text{親}}}$ の ${\overset{\textnormal{}}{\text{下}}}$ を ${\overset{\textnormal{}}{\text{離}}}$ れた。 \hfill\break
The older brother left from under his parents. }
 
\par{${\overset{\textnormal{}}{\text{24. 厳}}}$ しい ${\overset{\textnormal{}}{\text{監視}}}$ の ${\overset{\textnormal{}}{\text{許}}}$ に ${\overset{\textnormal{}}{\text{置}}}$ かれるのは\{ ${\overset{\textnormal{}}{\text{大変}}}$ な・ ${\overset{\textnormal{ひど}}{\text{酷}}}$ い\}ことでしょうね。 \hfill\break
Being placed under harsh surveillance is awful, isn't it? }
 
\par{${\overset{\textnormal{}}{\text{25. 販売予想}}}$ を ${\overset{\textnormal{}}{\text{基}}}$ にして ${\overset{\textnormal{}}{\text{我々}}}$ は ${\overset{\textnormal{}}{\text{製品}}}$ の ${\overset{\textnormal{}}{\text{生産}}}$ を ${\overset{\textnormal{}}{\text{停止}}}$ することにしました。 \hfill\break
On the basis of the sales forecast, we have decided to halt the manufacturing of the product. }

\par{26. 太陽のもとで子供たちが遊んでいる。 \hfill\break
Kids are playing under the sun. }

\par{27. 街灯のもと\{に・で\}住民たちが集まっている。 \hfill\break
Residents are gathering underneath the street light(s). }
 
\par{\textbf{元・旧・故 }}
 
\par{When written as such, もと means "former\slash previous". Lastly, it may be used in two important expressions. }
 
\par{${\overset{\textnormal{}}{\text{28. 元}}}$ に ${\overset{\textnormal{}}{\text{戻}}}$ った ${\overset{\textnormal{}}{\text{方}}}$ がいい。 \hfill\break
It's best to return to the previous condition. }
 
\par{${\overset{\textnormal{}}{\text{29a. 今日}}}$ 、 ${\overset{\textnormal{もとどうりょう}}{\text{元同僚}}}$ と ${\overset{\textnormal{}}{\text{偶然出会}}}$ いましたよ。 \hfill\break
 ${\overset{\textnormal{}}{\text{29b. 今日}}}$ 、かつての ${\overset{\textnormal{}}{\text{同僚}}}$ と ${\overset{\textnormal{}}{\text{偶然会}}}$ いましたよ。 \hfill\break
I happened to meet my former colleague suddenly today. }
 
\par{${\overset{\textnormal{}}{\text{30. 元}}}$ の ${\overset{\textnormal{さや}}{\text{鞘}}}$ に ${\overset{\textnormal{}}{\text{収}}}$ まる。(Idiom) \hfill\break
To bury the hatchet. }
 ${\overset{\textnormal{}}{\text{31. 元}}}$ の ${\overset{\textnormal{もくあみ}}{\text{木阿弥}}}$ 。(Idiom) \hfill\break
Ending up right where you started. 
\par{本・元 }

\begin{ltabulary}{|P|P|}
\hline 

1. & Origin, source, root \hfill\break
\\ \cline{1-2}

2. & The basis of things, foundation. This usage may also be written in Kanji as 基. \\ \cline{1-2}

3. & The cause. This usage may also be written in Kanji as 因. \\ \cline{1-2}

4. & Funds, capital; cost price. \\ \cline{1-2}

5. & Food stock, ingredients. This usage may also be written in Kanji as 素 \\ \cline{1-2}

6. & A counter that counts the number of stumps of plants. \\ \cline{1-2}

\end{ltabulary}

\par{\textbf{Usage Notes }: }

\par{1. 本・元 may also be in 元も子もない which means "losing everything". }

\par{2. As you may have noticed, usage number 4 is the same as the third usage of 下・許. }

\begin{center}
\textbf{Examples } 
\end{center}

\par{${\overset{\textnormal{}}{\text{32. 失敗}}}$ は ${\overset{\textnormal{}}{\text{成功}}}$ の ${\overset{\textnormal{}}{\text{元}}}$ だ。 \hfill\break
Failure is the source of success. }

\par{33. ${\overset{\textnormal{さんじ}}{\text{惨事}}}$ の ${\overset{\textnormal{もと}}{\text{本}}}$ を ${\overset{\textnormal{たど}}{\text{辿}}}$ ろう。(Literary Spelling) \hfill\break
I will pursue the origin of this horrible accident. }
 
\par{${\overset{\textnormal{}}{\text{34. 木}}}$ の ${\overset{\textnormal{}}{\text{元}}}$ を ${\overset{\textnormal{}}{\text{見}}}$ つけたか。 \hfill\break
Did you find the root of the tree? }

\par{35. 風邪は万病の元。 \hfill\break
The cold is the source of all sorts of diseases. }
 
\par{${\overset{\textnormal{}}{\text{36. 酒}}}$ とタバコが ${\overset{\textnormal{}}{\text{因}}}$ で ${\overset{\textnormal{}}{\text{健康}}}$ を\{損・害\}う。 \hfill\break
You lose health due to alcohol and tobacco. }
 
\par{${\overset{\textnormal{}}{\text{37. 元}}}$ のかかる ${\overset{\textnormal{}}{\text{商売}}}$ は ${\overset{\textnormal{}}{\text{高}}}$ い。 \hfill\break
The transaction capital is high. }
 
\par{38. スープの ${\overset{\textnormal{もと}}{\text{素}}}$ はあるの。 \hfill\break
Do we have the soup stock? }
 
\par{39. 私は ${\overset{\textnormal{}}{\text{柳一本}}}$ を ${\overset{\textnormal{}}{\text{植}}}$ えた。 \hfill\break
I planted a single willow tree. }
    