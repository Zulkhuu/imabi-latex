    
\chapter{The Bases (活用形)}

\begin{center}
\begin{Large}
第201課: The Bases (活用形) 
\end{Large}
\end{center}
 
\par{ Japanese is an agglutinative language ( ${\overset{\textnormal{こうちゃくご}}{\text{膠着語}}}$ ). Agglutination in the case of Japanese refers to how endings attach to what are referred to as "bases" in chains, interwoven like DNA strands. Endings can range from auxiliaries, supplementary verbs, to even particles, but the bases they attach to are limited in number and the relationship between base and ending is not arbitrary. }
      
\section{Parts of Speech}
 
\par{ It's important to understand that the bases will differ in appearance and slightly in usage depending on the part of speech. The parts of speech that have bases are those that can conjugate. Thus, we will be investigating verbs, adjectives, and auxiliaries. In this lesson, we consider タル形容動詞 as a separate part of speech as their set of bases is a special point of interest. }

\begin{ltabulary}{|P|P|P|P|P|P|P|}
\hline 

 & 未然形 & 連用形 & 終止形 & 連体形 & 已然形 & 命令形 \\ \cline{1-7}

上一段活用動詞 & い- & い- & いる & いる- & いれ- & いろ・いよ \\ \cline{1-7}

下一段活用動詞 & え- & え- & える & える- & えれ- & えろ・えよ \\ \cline{1-7}

五段活用動詞 & -あ-・-お- & -い- & -う & -う- & -え- & -え \\ \cline{1-7}

サ変活用動詞 & さ・し・せ- & し- & する & する- & すれ- & しろ・せよ・せい \\ \cline{1-7}

カ変活用動詞 & こ- & き- & くる & くる & くれ- & こい \\ \cline{1-7}

形容詞 & かろ- & くて-・かり- & い & い-・(き-) & けれ- & かれ \\ \cline{1-7}

形容動詞 & だろ- & だっ・で・に- & だ & な- & なら- & X \\ \cline{1-7}

タル形容動詞 & (たら-) & と- & (たり) & たる- & (たれ-) & (たれ) \\ \cline{1-7}

~た & たろ- & たり & た & た & たら- & X \\ \cline{1-7}

~ず & ざら- & ず・ざり- & ず・ぬ・ん & ぬ・ざる- & ね・ざれ- & ざれ \\ \cline{1-7}

~べきだ & べから- & べく・べかり- & べし・べきだ & べき-・べかる & べけれ- & X \\ \cline{1-7}

~ます & ませ-・ましょ- & まし- & ます & ます(る) & ますれ- & ませ \\ \cline{1-7}

~なり & なら- & なり- & なり & なる- & なれ- & なれ \\ \cline{1-7}

~う・よう & X & X & う・よう & (う・よう-) & X & X \\ \cline{1-7}

~まい & X & X & まい & (まい・まじき) & X & X \\ \cline{1-7}

\end{ltabulary}

\par{ This is a lot of information condensed into one chart, so during the lesson individual examples from this chart will be discussed. }
      
\section{未然形}
 
\par{ The 未然形 literally translates into English as the "Irrealis Form". This comes from the Classical Japanese usage of making the ば hypothetical, which attached to the 未然形 for this meaning. So, you would get something like 急がば instead of 急げば. In fact, there are still several instances in Modern Japanese where this is allowed. For instance, 急がば is seen in the common set phrase 急がば回れ, which translates to "slow and steady wins the race". }

\par{ Remember that set phrases in any language are the most likely examples to find old-fashioned\slash archaic grammar. Yet, this example isn't the only example. In fact, whenever you use ~たらば or ~ならば, you are utilizing the 未然形+ば. }

\par{ The use of ~たら and ~なら in this fashion should be ungrammatical because as the chart above shows, they are the 未然形 of ~た and ~なり respectively. So, they should have an ending following them. As we should all know, though, languages evolve from errors. Now, they are deemed as particles because they have deviated from the rules. }

\begin{center}
\textbf{Examples }
\end{center}

\par{1. 星になれたならば \hfill\break
If you were able to become a star }

\par{2. できたら、見せてくれない? \hfill\break
If you were able to, can't you show me? }

\par{ As this is the base of things not having realized, it is also followed by auxiliaries of intention\slash volition. For the volition, many items have a 未然形 sound change of \slash a\slash  to \slash o\slash . }

\par{3. 行 \textbf{こう }! \hfill\break
Let's go! }

\par{4a. そうでもなか \textbf{ろう }。(ちょっと古風) \hfill\break
4b. そうでもない \textbf{でしょう }。(もっと自然) \hfill\break
It's probably not so. }

\begin{center}
 \textbf{する }
\end{center}

\par{ する\textquotesingle s original 未然形's was せ-. As it is the original, you should expect it to be used with old auxiliaries such as ~ず・ぬ・ん. These are all forms of each other, so this makes it easier to remember. Other endings that have attached to the 未然形 in Classical Japanese have all been replaced with other speech modals or changed somehow (like the volitional endings and their current relationship sound-wise with the 未然形). }

\par{5. 入会 \textbf{せず }にはいられない。 \hfill\break
I can't help but join. }

\par{6. 勉強 \textbf{せん }と。 (ちょっと古風: 方言的) \hfill\break
I have to study. }

\par{ The さ- variant in Modern Japanese is important for the forms される (passive) and させる (causative). However, in Classical Japanese you had せらる and せしむ respectively. So, the emergence of さ- comes about due the causative auxiliary さす. Essentially, the birth of させる was simultaneous with that of される. し- is paired with ~ない and the volitional ending ~よう. }

\par{7. 誘惑された人 \hfill\break
A person that got lured }

\par{8. 喧嘩を売られたのはこっちのほうだ。 \hfill\break
The one who had been picked into the fight was [him]. \hfill\break
From 混声の森  (下) by 松本清張. }

\par{9. いや、そうさせはしないのだ。 \hfill\break
No, [I] won't make [you] do such. }

\par{ The 未然形 of 五段 verbs, 形容詞, 形容動詞, and some auxiliaries ends in \slash o-\slash . }

\par{10. 新しかろうが古かろうが \hfill\break
Whether it's new or old }

\par{11. 今夜勉強しない \textbf{でしょう }。 \hfill\break
I probably won't study tonight. }

\par{12. 息子にも理屈は \textbf{あろう }。 (ちょっと古風) \hfill\break
His son must also have a reason. \hfill\break
From 混声の森 (下) by  松本清張. }

\begin{center}
 \textbf{Adjective 未然形 + Negation }
\end{center}

\par{ Though the 未然形 of adjectives has limited usage with the auxiliary ~う, which is also replaced with 終止形+だろう, the old negative auxiliary ~ず attaches to the 未然形 of adjectives. Though this would mean any instances of this combination in Modern Japanese are old-fashioned, you will come across examples in literature. }

\par{13. いささか詩人らし \textbf{からぬ }月並みな表現である。 \hfill\break
It's a commonplace expression unbecoming of a poet. }

\begin{center}
 \textbf{Potential Verbs }
\end{center}

\par{ Before Modern Japanese, there were no special conjugations for the potential such as 行ける for 行く. This came about from a very important sound change involving the auxiliary ~れる when attached to the 未然形 of 五段 verbs. }

\par{ Motivation for this lies in the fact that ~れる is primarily used to make the passive voice for 五段 and サ変 verbs. It also has another major function of making light 敬語, which is crucially important for situations such as talking to one's boss's boss or elderly people. }

\par{14. どちらに行かれますか。 \hfill\break
Where are you going? }

\par{15. シャコはカモメに食われた。 \hfill\break
The mantis shrimp was eaten by the sea gull. }

\par{ How, then, did a whole new "conjugation" form? In Classical Japanese, Japanese actually didn't express "affirmative potential". Japanese culture has always placed value in being humble in regards to oneself. So, the means of expressing potential were used when expressing negative potential. The auxiliaries ~られる and れる  were ~らる and ~る respectively. So, cannot swim would have been expressed as 泳がれず. }

\par{16. つゆまどろまれず。 \hfill\break
I couldn't doze even a little. \hfill\break
From the 更級日記. }

\par{ The 未然形 ~る was れ-, which is still the 未然形 of ~れる. Sometime during the Edo Period, "ar" in this potential pattern dropped. This renders 泳がれず as 泳げず. Though the changes in appearance of the negative and potential endings are separate events, this simple change that no doubt began as a colloquial contraction gave birth to fully independent potential verbs in the Japanese language. }

\par{ Due to Western influence with many works being translated into Japanese, the 終止形 (final form) of the potential began being used, and with the large majority of verbs now having a fully functional independent form for the potential, it was perfect timing. }

\par{17. 手紙が書けず。 \hfill\break
I can't write the letter. }

\par{ In Modern Japanese, this is now being extended to 一段 verbs. So, things like 見られる is becoming 見れる. In another 50 years or so, this may be the fully standard way to make the potential for this verb class. This entire process, though, originates from just a simple adjustment to the rules of a conjugation involving the 未然形 because having the potential to do something doesn't mean you've done it yet. }

\par{18. 何でも食べれるよ。 \hfill\break
You can eat anything. }

\par{19. 見れますか。 \hfill\break
Can you see it? }

\par{ Many people like to refer to this special development in these verbs as the 可能形. After all, it's far more deviant than the 連用形 sound changes with ~た・て. }

\par{聞く \textrightarrow  聞き- (連用形) + ~て \textrightarrow  聞きて \textrightarrow  (Drop K) 聞いて    Still using the 連用形 }

\par{聞く \textrightarrow  聞か- (未然形) + ~れる \textrightarrow  聞かれる \textrightarrow  (Drop ar) 聞ける   Still using the 未然形 }

\par{ However, this doesn't work well with the base format as seen above. Bases apply to anything that conjugates. There is no doubt that the term 可能形 refers to this potential form phenomenon, but is it of the same vein as the 未然形, 連用形, etc.? No. If it were, everything that conjugates in the language would at least have the opportunity to have it. }

\par{ Again, this is only a particular sound change of a process following the norm for a particular verb class. Furthermore, these resultant "potential verbs" have their own set of bases like any other verb. So, it is generally believed that these verbs should be treated as separate words that stem from the base verb, just as there are transitivity pairs in the language. }

\par{ Transitivity pairs didn't exist in ancient forms of Japanese. So, the emerging of a new derivative of verbs should not be surprising in a historical sense. If anything, this has allowed the language to fix a major problem in battling ambiguity with other usages of the auxiliary ~れる. }

\par{はじむ  (Old Japanese word for "to begin") \textrightarrow    Modern Japanese: はじまる (Intransitive)    はじめる  (Transitive) }

\par{いづ (Old Japanese word for "to go out")        \textrightarrow     Modern Japanse: 出る (Intransitive) \hfill\break
 出す (Transitive) \hfill\break
 出せる (Potential) }

\par{ There is a mistaken logic in Japanese texts that the 可能形 is made by affixing (attaching) る to the equivalent of the 已然形, which is what is meant by sources that call it the "E Base" because of the vowel that it ends with for all instances. However, this is nonsensical. It may be an easy way to remember how to make it for 五段 verbs. But, it completely neglects what the "E Base ≒ 已然形” is used for. It also ignores the sound change. The non-abbreviated form of the process 未然形+れる still has relevance in Modern Japanese. So, the downside that such an explanation brings is the watering down and unintended omission of important information to the student. }
      
\section{連用形}
 
\par{ The purpose of the 連用形 is to show that an action\slash process is either taking place or has already taken place. Thus, it is often followed by auxiliaries such as ~た. Conjunctive particles such as て, つつ, and ながら also follow it to indicate continuation. This is by far the most used base in Japanese conjugation. Further important usages include the 連用中止形 and making compound verbs. }

\par{20. 食卓に焜炉を \textbf{置いて }鍋を \textbf{囲み }、 \textbf{楽しく語らいながら }食事ができる。 \hfill\break
You can have a dinner while enjoying eating and talking sitting around the pot on the konro [stove] on the dining table. }

\par{ This sentence has several usages of the 連用形. You see it with て and the verb 置く, in the 連用中止形 with the verb 囲む, used as an adverb with the adjective 楽しい, and with the ながら with the verb 語る. }

\par{ One of the most difficult aspects of the 連用形 are sound changes and multiple ones for certain items. A lot has changed in regards to the base. Consider the chart below which shows how it has changed. 文語 stands for Classical Japanese and 口語 stands for Modern Japanese. }
 
\begin{ltabulary}{|P|P|P|P|P|P|P|P|P|P|}
\hline 
 
 \multicolumn{5}{|c|}{ 文語 
 }&  \multicolumn{5}{|c|}{ 口語 
 }\\ \cline{1-10} 
 
  品詞 
 &   活用の種類 
 &   例語 
 &  \multicolumn{2}{|c|}{ 語形 
 }&   活用の種類 
 &   例語 
 &  \multicolumn{2}{|c|}{ 語形 
 }&  \\ \cline{1-10} 
 
 \multirow{9}*{ 動詞 
 }&   四段活用 
 &   書く 
 &   かき 
 &   -i 
 &  \multirow{4}*{ 五段活用 
 }&  \multirow{4}*{ 書く 
 }&  \multirow{4}*{  かき かい   }&  \multirow{4}*{  -i っ\slash ん\slash い   }&  \\ \cline{1-0} \cline{1-5} \cline{6-6} \cline{7-7} \cline{8-8} \cline{9-10} 
 
  ラ行変格活用 
 &   あり 
 &   あり 
 &   -i 
 &   & \\ \cline{1-0} \cline{1-5} \cline{6-6} \cline{7-7} \cline{8-8} \cline{9-10} 
 
  ナ行変格活用 
 &   死ぬ 
 &   しに 
 &   -i 
 &   & \\ \cline{1-0} \cline{1-5} \cline{6-6} \cline{7-7} \cline{8-8} \cline{9-10} 
 
  下一段活用 
 &   蹴る 
 &   け 
 &   -e 
 &   & \\ \cline{1-0} \cline{1-10} 
 
  下二段活用 
 &   受く 
 &   うけ 
 &   -e 
 &   下一段活用 
 &   受ける 
 &   うけ 
 &   -e 
 &  \\ \cline{1-0} \cline{1-10} 
 
  上一段活用 
 &   着る 
 &   き 
 &   -i 
 &  \multirow{2}*{ 上一段活用 
 }&  \multirow{2}*{ 起きる 
 }&  \multirow{2}*{ おき 
 }&  \multirow{2}*{ -i 
 }&  \\ \cline{1-0} \cline{1-4} \cline{5-5} \cline{6-6} \cline{7-7} \cline{8-10} 
 
  上二段活用 
 &   起く 
 &   おき 
 &   -i 
  & &   & \\ \cline{1-0} \cline{1-10} 
 
  カ行変格活用 
 &   来 
 &   き 
 &   -i 
 &   カ行変格活用 
 &   来る 
 &   き 
 &   -i 
 &  \\ \cline{1-0} \cline{1-10} 
 
  サ行変格活用 
 &   す 
 &   し 
 &   -i 
 &   サ行変格活用 
 &   する 
 &   し 
 &   -i 
 &  \\ \cline{1-10} 
 
 \multirow{2}*{ 形容詞 
 }&   ク活用 
 &   なし 
 &    なく なかり   &    く かり   &  \multirow{2}*{   }&  \multirow{2}*{ ない 
 }&  \multirow{2}*{  なく なかっ   }&  \multirow{2}*{  く かっ   }&  \\ \cline{1-0} \cline{1-5} \cline{6-6} \cline{7-7} \cline{8-8} \cline{9-10} 
 
  シク活用 
 &   美し 
 &    うつくしく うつくしかり   &    しく しかり   &   & \\ \cline{1-10} 
 
 \multirow{2}*{ 形容動詞 
 }&   ナリ活用 
 &   静かなり 
 &    しずかに しずかなり   &    に なり   &  \multirow{2}*{   }&  \multirow{2}*{ 静かだ 
 }&  \multirow{2}*{  しずかに しずかだっ しずかで   }&  \multirow{2}*{  に だっ で   }&  \\ \cline{1-0} \cline{1-5} \cline{6-6} \cline{7-7} \cline{8-8} \cline{9-10} 
 
  タリ活用 
 &   堂々たり 
 &   どうどうと \hfill\break
どうどうたり 
 &   と \hfill\break
たり 
 &   & \\ \cline{1-10} 

\end{ltabulary}
  The different 連用形 are used with different endings. This should already be evident. For instance, the conjunctive particle て goes after く for 形容詞, に for 形容動詞 (which normally contracts to で), and the sound changes with 五段 verbs.  
\begin{ltabulary}{|P|P|P|P|P|P|}
\hline 

買う \textrightarrow  買って & 書く \textrightarrow  書いて & 嗅ぐ \textrightarrow  嗅いで & 勝つ \textrightarrow  勝って & 噛む \textrightarrow  嚙んで & 刈る \textrightarrow  刈って \\ \cline{1-6}

\end{ltabulary}
 
\begin{ltabulary}{|P|P|P|P|P|P|P|P|P|P|}
\hline 

\multicolumn{5}{|c|}{文語 }& \multicolumn{5}{|c|}{口語 }\\ \cline{1-10}

品詞 & 活用の種類 & 例語 & \multicolumn{2}{|c|}{語形 }& 活用の種類 & 例語 & \multicolumn{2}{|c|}{語形 }\\ \cline{1-10}

\multirow{9}*{動詞 }& 四段活用 & 書く & かき & -i & \multirow{4}*{五段活用 }& \multirow{4}*{書く }& \multirow{4}*{かき \hfill\break
かい }& \multirow{4}*{-i \hfill\break
っ\slash ん\slash い }\\ \cline{1-0} \cline{1-5} \cline{6-6} \cline{7-7} \cline{8-8} \cline{9-10}

ラ行変格活用 & あり & あり & -i \\ \cline{1-0} \cline{1-5} \cline{6-6} \cline{7-7} \cline{8-8} \cline{9-10}

ナ行変格活用 & 死ぬ & しに & -i \\ \cline{1-0} \cline{1-5} \cline{6-6} \cline{7-7} \cline{8-8} \cline{9-10}

下一段活用 & 蹴る & け & -e \\ \cline{1-0} \cline{1-10}

下二段活用 & 受く & うけ & -e & 下一段活用 & 受ける & うけ & -e \\ \cline{1-0} \cline{1-10}

上一段活用 & 着る & き & -i & \multirow{2}*{上一段活用 }& \multirow{2}*{起きる }& \multirow{2}*{おき }& \multirow{2}*{-i }\\ \cline{1-0} \cline{1-4} \cline{5-5} \cline{6-6} \cline{7-7} \cline{8-10}

上二段活用 & 起く & おき & -i  & \\ \cline{1-0} \cline{1-10}

カ行変格活用 & 来 & き & -i & カ行変格活用 & 来 & き & -i \\ \cline{1-0} \cline{1-10}

サ行変格活用 & す & し & -i & サ行変格活用 & す & し & -i \\ \cline{1-10}

\multirow{2}*{形容詞 }& ク活用 & なし & なく \hfill\break
なかり & く \hfill\break
かり & \multirow{2}*{ }& \multirow{2}*{ない }& \multirow{2}*{なく \hfill\break
なかっ }& \multirow{2}*{く \hfill\break
かっ }\\ \cline{1-0} \cline{1-5} \cline{6-6} \cline{7-7} \cline{8-8} \cline{9-10}

シク活用 & 美し & うつくしく \hfill\break
うつくしかり & しく \hfill\break
しかり \\ \cline{1-10}

\multirow{2}*{形容動詞 }& ナリ活用 & 静かなり & しずかに \hfill\break
しずかなり & に \hfill\break
なり & \multirow{2}*{ }& \multirow{2}*{静かだ }& \multirow{2}*{しずかに \hfill\break
しずかだっ \hfill\break
しずかで }& \multirow{2}*{に \hfill\break
だっ \hfill\break
で }\\ \cline{1-0} \cline{1-5} \cline{6-6} \cline{7-7} \cline{8-8} \cline{9-10}

タリ活用 & 堂々たり & どうどうと \hfill\break
どうどうたり & と \hfill\break
たり \hfill\break
\hfill\break
\\ \cline{1-10}

\end{ltabulary}

\begin{ltabulary}{|P|P|P|P|P|P|P|P|P|P|}
\hline 

\multicolumn{5}{|c|}{文語 }& \multicolumn{5}{|c|}{口語 }\\ \cline{1-10}

品詞 & 活用の種類 & 例語 & \multicolumn{2}{|c|}{語形 }& 活用の種類 & 例語 & \multicolumn{2}{|c|}{語形 }\\ \cline{1-10}

\multirow{9}*{動詞 }& 四段活用 & 書く & かき & -i & \multirow{4}*{五段活用 }& \multirow{4}*{書く }& \multirow{4}*{かき \hfill\break
かい }& \multirow{4}*{-i \hfill\break
っ\slash ん\slash い }\\ \cline{1-0} \cline{1-5} \cline{6-6} \cline{7-7} \cline{8-8} \cline{9-10}

ラ行変格活用 & あり & あり & -i \\ \cline{1-0} \cline{1-5} \cline{6-6} \cline{7-7} \cline{8-8} \cline{9-10}

ナ行変格活用 & 死ぬ & しに & -i \\ \cline{1-0} \cline{1-5} \cline{6-6} \cline{7-7} \cline{8-8} \cline{9-10}

下一段活用 & 蹴る & け & -e \\ \cline{1-0} \cline{1-10}

下二段活用 & 受く & うけ & -e & 下一段活用 & 受ける & うけ & -e \\ \cline{1-0} \cline{1-10}

上一段活用 & 着る & き & -i & \multirow{2}*{上一段活用 }& \multirow{2}*{起きる }& \multirow{2}*{おき }& \multirow{2}*{-i }\\ \cline{1-0} \cline{1-4} \cline{5-5} \cline{6-6} \cline{7-7} \cline{8-10}

上二段活用 & 起く & おき & -i  & \\ \cline{1-0} \cline{1-10}

カ行変格活用 & 来 & き & -i & カ行変格活用 & 来 & き & -i \\ \cline{1-0} \cline{1-10}

サ行変格活用 & す & し & -i & サ行変格活用 & す & し & -i \\ \cline{1-10}

\multirow{2}*{形容詞 }& ク活用 & なし & なく \hfill\break
なかり & く \hfill\break
かり & \multirow{2}*{ }& \multirow{2}*{ない }& \multirow{2}*{なく \hfill\break
なかっ }& \multirow{2}*{く \hfill\break
かっ }\\ \cline{1-0} \cline{1-5} \cline{6-6} \cline{7-7} \cline{8-8} \cline{9-10}

シク活用 & 美し & うつくしく \hfill\break
うつくしかり & しく \hfill\break
しかり \\ \cline{1-10}

\multirow{2}*{形容動詞 }& ナリ活用 & 静かなり & しずかに \hfill\break
しずかなり & に \hfill\break
なり & \multirow{2}*{ }& \multirow{2}*{静かだ }& \multirow{2}*{しずかに \hfill\break
しずかだっ \hfill\break
しずかで }& \multirow{2}*{に \hfill\break
だっ \hfill\break
で }\\ \cline{1-0} \cline{1-5} \cline{6-6} \cline{7-7} \cline{8-8} \cline{9-10}

タリ活用 & 堂々たり & どうどうと \hfill\break
どうどうたり & と \hfill\break
たり \hfill\break
\hfill\break
\\ \cline{1-10}

\end{ltabulary}
      
\section{終止形}
 
\par{ This base marks the end of a sentence. This is very important. Everything that conjugates has a 終止形. It normally expresses the non-past tense unless there is a tense item involved. What it looks like is quite straightforward as it is equivalent to the 辞書形. }

\par{21. 左に曲がる。  (左に曲がれという意味) \hfill\break
Turn left. }

\par{\textbf{Grammar Note }: Some times, the 終止形 in context may be used to instruct someone to do something. }

\par{ Recently in the history of Japanese, the 連体形 merged with the 終止形, causing the latter to look like the former for many things. For instance, the 終止形 of する used to be す, and the 連体形 was する. This also effected some auxiliaries. For instance, べし \textrightarrow  べきだ, ず \textrightarrow  ぬ. }
      
\section{連体形}
 
\par{ Though in Modern Japanese completely identical in appearance to the 終止形 for most conjugatable items, it is syntactically extremely different. The 連体形 is the attributive form, which is used in Modern Japanese to solely modify nominal phrases. }

\par{22. きれい \textbf{な }字 \hfill\break
Pretty handwriting }

\par{23. ずいぶん \textbf{ひどい }内容 \hfill\break
Quite horrible content\slash matter }

\par{24. もう宿題を \textbf{出した }んです。 \hfill\break
I've already turned in my homework. }

\par{ An issue that comes up a lot is students not realizing that verb phrases can modify sentences. There are even particle restrictions that come along with it. For instance, an embedded clause must never have the topic particle は unless there is a citation particle like と・って. Some make the mistake of finding the first が and assuming it's the subject, despite the fact that that is not always the case because of this fact. }

\par{25. }

\par{${\overset{\textnormal{ようし}}{\text{要旨}}}$ は、[ ${\overset{\textnormal{けんいち}}{\text{謙一}}}$ が長い間専務理事の ${\overset{\textnormal{いす}}{\text{椅子}}}$ に ${\overset{\textnormal{すわ}}{\text{坐}}}$ って学園の管理を ${\overset{\textnormal{ろうだん}}{\text{壟断}}}$ して \textbf{いる }]こと、[会計に不正が \textbf{ある }]こと、[それを理事会の席で大島に指摘されて謙一が ${\overset{\textnormal{ろうばい}}{\text{狼狽}}}$ \textbf{した }]こと、つづいて[彼が次の理事長の椅子を ${\overset{\textnormal{ねら}}{\text{狙}}}$ い ${\overset{\textnormal{かくさく}}{\text{画策}}}$ して \textbf{いる }]ことなどであった。 \hfill\break
The gist was that Ken'ichi has been sitting in the seat of the executive director for a long time and has been monopolizing on the management of the academy, has made a wrong in the accounting, and that that was pointed out in the board meeting by Ohshima and Ken'ichi was bewildered, and counting on, he was scheming for the seat of the next board chairman. \hfill\break
From 混声の森 (下) by 松本清張. }

\par{\textbf{Sentence Note }: The brackets indicate the long attributes phrases, and the 連体形 have been put in bold.  }
      
\section{已然形}
 
\par{ The 已然形 literally means "the already-realized form". This comes from the fact that in Classical Japanese, the particle ば would be used with it to mean "because; when". Thus, this grammar point has changed significantly. It is also used with the particle ど(も), which means although. }

\par{26. あかつきより雨降れば \hfill\break
Since it had been raining since early morning \hfill\break
From the 土佐日記. }
 
\par{27. エリスは ${\overset{\textnormal{とこ}}{\text{床}}}$ に ${\overset{\textnormal{ふ}}{\text{臥}}}$ すほどにはあらねど、小さき ${\overset{\textnormal{てつろ}}{\text{鉄炉}}}$ のほとりに椅子さし寄せて言葉少なし。 \hfill\break
Ellis wasn't to the point of being bedridden, but she lied on a chair near a small iron furnace and had little to say. \hfill\break
By ${\overset{\textnormal{おうがい}}{\text{鷗外}}}$ . }
 
\par{You also see it used in the pattern こそすれ. In Modern Japanese grammar, it is sometimes called the 仮定形 (the hypothetical form) due to how the particle ば is now used. Nevertheless, the original name generally stands. }
 
\par{28. 男女の違いこそあれ、二人は ${\overset{\textnormal{うり}}{\text{瓜}}}$ 二つだね。 \hfill\break
Of course they have their differences as man and woman, but they're like two peas in a pod. }
      
\section{命令形}
 
\par{ The 命令形 is the imperative form, which creates a command. There are many ways to make commands in Japanese, but this is the core way to. It isn't difficult to make an imperative, but it is difficult to use these phrases--with the ultimate goal to request something--like native speakers would. }

\par{\textbf{Construction Note }: Making the 命令形 is relatively easy. As we have already studied how to construct it, please refer back to the chart at the beginning of this lesson. }

\par{\textbf{Curriculum Note }: As there is already a lesson about the 命令形, every point mentioned here is meant to be summarization and not intense dissection. }

\par{ In Modern Japanese, the 命令形 is typically very stern and powerful. It is semantically impossible for it to be used with the past tense. }

\begin{center}
\textbf{Examples }
\end{center}

\par{29. こっちに来い! \hfill\break
Come here! }

\par{30. 手を挙げろ! \hfill\break
Hands up! }

\par{31. あんたのいいようにしてくれ。 (Vulgar) \hfill\break
Do as is good for you. }

\par{32. ゴミを投げるな。 \hfill\break
Don't throw trash! }

\par{33. 金を出せ! \hfill\break
Hand over money! }

\par{ One can easily see a bank robber saying 金を出せ or a soldier yelling 手を挙げろ. It's not that stern commands are never used. It's just that from a productive standpoint, you should understand that its use is limited. Experience as to when the 命令形 is used is the best solution to not sound rude. There are plenty of other ways to get your point across about wanting something done. }

\par{34. 息子は、相手の家に謝りに行けという考えに反抗している。 \hfill\break
His son was against his thought of going to apologize to his opponent's house. \hfill\break
From 混声の森 by 松本清張. }

\par{\textbf{Grammar Note }: In this instance, the command form is in quotations. Though there is a 上下関係 between the father and son, the 命令形 is used in this fashion a lot. }

\par{35. 一生懸命勉強しろといわれた。 \hfill\break
I was told to study as hard as possible.  }

\begin{center}
 \textbf{話者 VS 聴者 }
\end{center}

\par{ In studying more about the 命令形, we need to examine the relation between the 話者 (speaker) and 聴者 (listener). Inequality signs will demonstrate which side has the most power\slash influence. An equal sign would mean that they're equal in this regard. }

\begin{center}
話者 > 聴者  話者 = 聴者  話者 < 聴者 
\end{center}

\par{ In the first situation, the speaker has more authority on the choice of action than the listener. There is no place for the speaker to decide what to do. In the second situation, the speaker allows for the action, and the listener has the power to consent to it or not interfere with it. In the third section, even if the speaker makes a request for the listener to do something, the listener has the power to decide. Thus, the speaker ends up making a suggestion or some sort of advice\slash aspiration. }

\par{ With each of these situations, you can replace the 命令形 with various synonymous expressions. For instance, in the first relation, you could see ~なければならない or ~なさい. In the second situation, you could use ~ていい. In the third, ~た方がいい could be used instead. Though there are slight differences, relating the 命令形 with these patterns for the three broad situations it is used in will help you a lot in understanding it. }

\par{36. 黙って出されたものを食べろ。     (話者 >  聴者) \hfill\break
食べろ \textrightarrow  食べなさい \hfill\break
Shut up and eat what's been given to you! }

\par{37. これを飲め。 (話者 = 聴者) \hfill\break
飲め \textrightarrow  飲んで(も)いい \hfill\break
Drink this. }

\par{38. 一時間幸せになりたかったら酒を飲め,三日間幸せになりたかったら結婚しろ,一週間幸せになりたかったら豚を殺して食べろ,一生幸せになりたかったら釣りを覚えろ。 \hfill\break
命令形 \textrightarrow  ~た方がいい \hfill\break
If you want to become happy for an hour, drink; if you want to be happy for three days, get married; if you want to be happy for a week, kill a pig and eat it; if you want to be happy for a lifetime, learn how to fish.   (話者 < 聴者) \hfill\break
A Quote from 開高健. }

\par{ The notes after the example signify that the 命令形 could be paraphrased out. }
かったら酒を飲め,三日間幸せになりたかったら結婚しろ,一週間 幸せになりたかったら豚を殺して食べろ,一生幸せになりたかった ら釣りを覚えろ 
\begin{center}
\textbf{命令形 Studies }
\end{center}

\par{ Studying the 命令形 also interests Japanese grammarians. The 命令形 has been given many names. Some include 放任形 (non-interference form) and 命令法. However, is the word 命令 the most appropriate word? }

\par{ One interesting usage that is typically not mentioned is that the 命令形 is the chosen command pattern for decrees in law. Related to this is the usage of it by bosses. }

\par{39. 止まれ \hfill\break
Stop! }

\par{ This is the Japanese traffic lingo equivalent to "STOP" on stop signs. This shows an imperative and obligation to motorists and pedestrians to stop. }

\begin{center}
 \textbf{In 敬語 }
\end{center}

\par{ This has all been in regards to non-敬語 situations. There are things such as ~なさい and ~ませ that have different social implications that you need to pay close attention to. }

\par{40. いらっしゃいませ。 \hfill\break
Welcome! }

\par{ This usage, for example, is a very important set phrase that you hear all the time when entering businesses and restaurants. }

\par{41. 勉強しなさい。 \hfill\break
Study! }

\par{ This is something that you would hear a stern parent tell his child. Again, just as is the case in English, social dynamics play a crucial rule in how commands of any kind are perceived. As you have surely seen a lot of Japanese by now, this shouldn't be hard to understand. }
    