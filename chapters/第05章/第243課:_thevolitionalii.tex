    
\chapter{The Volitional II The Negative Volitional}

\begin{center}
\begin{Large}
第243課: The Volitional II: The Negative Volitional: ~まい 
\end{Large}
\end{center}
 
\par{ Just when you thought you were done, you now have to consider negative volition. }
      
\section{~まい}
 
\par{ ~まい shows \textbf{negative volition }. So, you have will for \textbf{something not to happen }. It may also be like ~ないだろう. The first meaning would be used in contexts like "I won't\dothyp{}\dothyp{}\dothyp{}" or "She vowed not to go". This can be distinguished from other contexts such as "it won't possibly rain". }

\par{ ~まい attaches to the 終止形 of 五段 verbs and auxiliary verbs, which includes ~ます. As for 一段 verbs and する and 来る, it attaches to the 未然形 or 終止形. However, the 未然形 is better. }

\begin{ltabulary}{|P|P|P|}
\hline 

一段 & 食べる + まい \textrightarrow  & 食べまい \hfill\break
食べるまい (△) \\ \cline{1-3}

五段 & 書く + まい \textrightarrow  & 書くまい \\ \cline{1-3}

~ます & 書きます + まい \textrightarrow  & 書きますまい \\ \cline{1-3}

する & する + まい \textrightarrow  & しまい (打ち消し意志) \hfill\break
するまい (普通, 打ち消し推量) \hfill\break
すまい (ちょっとかしこまった) \\ \cline{1-3}

来る & 来る + まい \textrightarrow  & 来まい \hfill\break
来るまい \\ \cline{1-3}

\end{ltabulary}

\par{\textbf{Grammar Notes }: }

\par{1. It's possible to see すまい when it is treated as a 五段 verb like with 愛す. \hfill\break
2. It is also common in casual speech and \emph{some dialects }to see よ inserted between the 未然形 and ~まい. For example, いようがいよまいが. }

\begin{center}
 \textbf{Examples }
\end{center}

\par{1a. 雪が降るまい。 \hfill\break
1b. おそらく雪は降らないだろう。(Natural) \hfill\break
It probably won't snow. }

\par{2a. 勝てまい。 \hfill\break
2b. 勝てないだろう。(Natural) \hfill\break
I doubt you'll win. }

\par{3a. あの男は負けるまい。 \hfill\break
3b. あの男は負けないだろう。(Natural) \hfill\break
That man will probably not lose. }

\par{4. それは ${\overset{\textnormal{たい}}{\text{大}}}$ した金にはなるまい。 \hfill\break
That probably won't be worth much money. }

\par{5a. ${\overset{\textnormal{がい}}{\text{害}}}$ にはなりますまい。 \hfill\break
5b. 害にはならないでしょう。(Natural) \hfill\break
It'll do you no harm. }

\par{6a. ペンギンは助かるまい。 \hfill\break
6b. ペンギンは助からないだろう。(Natural) \hfill\break
The penguins probably won't be saved. }

\par{7. 誰も信じまい。 \hfill\break
No one will probably believe it. }

\par{8a. そんな手段は認めますまい。 \hfill\break
8b. そんな手段は認められませんよ。(Natural) \hfill\break
I will not approve of such a method. }

\par{9. もう二度とそんな ${\overset{\textnormal{あやま}}{\text{過}}}$ ちは ${\overset{\textnormal{く}}{\text{繰}}}$ り返すまい! \hfill\break
I will not make such a mistake a second time! }

\par{10. 明日、行くのをやめよう。 \hfill\break
Let's not go tomorrow. }

\par{11. 絶対に ${\overset{\textnormal{らくだい}}{\text{落第}}}$ するまいぞ! \hfill\break
I will absolutely not fail! }

\par{12. もう言うまい! \hfill\break
I'll say no more! }

\par{13. いくら肉親の妹だって、姉の骨まで見たことはあるまいから、分かるもんか。 \hfill\break
Even if it was your own little sister, you'd never understand since you've surely never seen bones to that of your older sister. \hfill\break
From 死体紹介人 by 川端康成. }

\par{14. 「里子ちゃん、いらっしゃい。お雑煮のお餅を焼きましょうね。里子ちゃんも、お手つだいしてちょうだい。」などと言って、菊子は里子を台所へ呼び寄せ、信吾の寝部屋の廊下を走らせまいとするつもりらしいが、里子は聞く風もなく、ぺたぺた廊下を走りつづけた。 \hfill\break
Kikuko said something like "Satoko-chan, come here. Let's make some zoni mochi. Won't you please help?" to call Satoko to the kitchen, and although it seemed she meant on not having her run through the hall by Shingo's bedroom, Satoko paid no heed to this and continued to run down loudly through the hall. \hfill\break
From 山の音 by 川端康成. }
  \textbf{~まいとする } 
\par{ ~まいとする means "to try not to". It is normally replaced by ~ないようにする. This paraphrase can also work for when する is not the verb phrase, but in this case, because using ~まい is a little more common in the spoken language, such a paraphrase is not necessary. }

\par{15. 笑うまいとする。 \hfill\break
To try not to laugh. }

\par{16. 笑うまいとしてもつい笑ってしまうだろう。 \hfill\break
Even if you try not to laugh, you'll eventually end up laughing. }

\par{17. 負けるまいとする。 \hfill\break
To try not to lose. }

\par{18. 会うまいと決心した。 \hfill\break
I decided that I wasn't going to meet (him). }

\par{19. その時までは決して彼に会うまいと心に決めていた。 \hfill\break
Until that time, I decided in my heart not to meet him by all means. }

\par{20. 彼女は眠るまいと決心していたが、結局眠りにつけてしまった。 \hfill\break
She was resolute not to sleep, but she finally ended up falling asleep. }

\begin{center}
 \textbf{~ではあるまい }
\end{center}

\par{ ~ではあるまい shows inadequacy. ~というわけではない is now common-place. ~ではあるまいし means "it's not as if\dothyp{}\dothyp{}\dothyp{}". }

\par{21. 知らなかったわけではあるまい。 \hfill\break
It's not that they didn't know. }

\par{${\overset{\textnormal{}}{\text{22. 馬鹿}}}$ じゃあるまいし、そんなことをするな。 \hfill\break
It's not as if you're an idiot, so don't do that. }

\begin{center}
  \textbf{Volition+が・と+ Negative Volition + が・と } 
\end{center}

\par{ ~(よ)うが~まいが and ~(よ)うと~まいと mean "whether\dothyp{}\dothyp{}\dothyp{}or\dothyp{}\dothyp{}\dothyp{}". }

\par{23. 真実であろうが真実であるまいが、まだ関係はない。 \hfill\break
Whether it's true or it's not true, I still have no part in it. }

\par{24. 行こうと行くまいと僕の勝ちだ。 \hfill\break
Even if you go or don't go, it's my victory. }

\par{25. 人が来ようと来るまいとまだパーティーを ${\overset{\textnormal{ひら}}{\text{開}}}$ く。 \hfill\break
Whether people come or not, I'm still going to throw a party. }

\begin{center}
\textbf{~まいぞ }
\end{center}

\par{ ~まいぞ shows prohibition. Although ~ますまい exists, ~ますまいぞ does not. This expression is quite old-fashioned, so you may only hear old men say this or find it in literature. Tone would distinguish it from a negative volitional statement like above. }

\par{26. 行くまいぞ。 \hfill\break
You mustn't go. }

\begin{center}
 \textbf{~まじき }
\end{center}

\par{ ~まい comes from the Classical Japanese ending ~まじ (the negative equivalent of ~べし). ~まじ only attached to the 終止形, except when it was attached to the verb あり (ある). For あり, it followed its 連体形, ある. }

\par{ あるまじき happens to be retained in Modern Japanese in more literary\slash formal situations to mean "not proper to". ~まじき can also be found in other set expressions. }

\par{27. あるまじき行為だ。 \hfill\break
It is an improper act. }

\par{28. すまじきものは ${\overset{\textnormal{みやづか}}{\text{宮仕}}}$ え。(Set Phrase) \hfill\break
It is better to work for oneself than to work for someone else. }

\par{ ~まじ had every base except the 命令形. This is in stark contrast with the Modern Japanese ~まい, which only has a 終止形 and a very rarely used 連体形 (~まい). However, two older 連体形 still hold on. The original ~まじき can still be seen in set phrases, but in the early 1900s, the form ~まじい was still frequently used in literature. }

\par{29. 「早くね、早くね。」と、言うなり後向いて走り出したのは ${\overset{\textnormal{うそ}}{\text{噓}}}$ みたいにあっけなかったが、遠ざかる後姿を見送っていると、なぜまたあの娘はいつもああ ${\overset{\textnormal{しんけん}}{\text{真剣}}}$ な様子なのだろうと、この場にある \textbf{まじい }${\overset{\textnormal{ふしん}}{\text{不審}}}$ が島村の心を ${\overset{\textnormal{かす}}{\text{掠}}}$ めた。 \hfill\break
Her running off as soon as she looked back saying "hurry, hurry!" seemed all too easy like a lie, but as he looked at her retreating figure go farther away, a suspicion unfit for the scene grazed Shimamura's mind as he thought on why again she was always seemed so serious like that. \hfill\break
From 雪国 by 川端康成。 }

\par{30. ${\overset{\textnormal{しかん}}{\text{士官}}}$ の軍刀と、 ${\overset{\textnormal{はんぎょく}}{\text{半玉}}}$ の守り袋や ${\overset{\textnormal{はなかんざし}}{\text{花簪}}}$ の鈴とが、 ${\overset{\textnormal{あしびょうし}}{\text{足拍子}}}$ につれて鳴った。兵士にある \textbf{まじい }、 ${\overset{\textnormal{あいしょう}}{\text{哀傷}}}$ の歌詞でありながら、二十五前後の青年と十五六の少女との ${\overset{\textnormal{がっしょう}}{\text{合唱}}}$ であるために、進軍の歌の響きがあった。 \hfill\break
The officer's saber and the young geisha's little pouch and bells on her flowery hairpin rang along with the beating of their feet. While the lyrics were elegiac and unworthy for a soldier, there was a marching song sound for the lad around twenty five or six and the young fifteen or sixteen girl to sing in chorus to. From 童謡 by 川端康成. }

\par{31. 触れるほど顔を重ねて見るほど、能面にはある \textbf{まじい }邪道だろう。 \hfill\break
It was surely heresy unworthy of the Noh mask to even touch or try to put over one's face. \hfill\break
From 山の音 by 川端康成. }

\par{ \textbf{Tense Note }: Sentences with ~まい are always in the non-past tense because ~まじ indicates that something should or won't happen due to some experience. }
    