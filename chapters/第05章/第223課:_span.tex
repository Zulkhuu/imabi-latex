    
\chapter{Span}

\begin{center}
\begin{Large}
第223課: Span: ~にかけて \& にわたって 
\end{Large}
\end{center}
 
\par{ A common error made by students is not understanding the difference between ~から~まで and ~から~にかけて. Then, if they learn about ~にわたって, this quickly becomes confused with ~にかけて. The nouns that students chose before these expressions are also often very problematic. So, this lesson will delve into these issues so that you don\textquotesingle t end up making the same mistakes! }
      
\section{~から~まで VS ~から~にかけて}
 
\par{ First, let\textquotesingle s discuss the basic information about ~から~にかけて. This is a less concrete variant of ~から~まで. Both から and まで concretely define beginning and end points, so with the replacement of まで with ~にかけて, you make this less poignant. The main issues, then, will come from not thinking closely enough as to what\textquotesingle s practical. }

\par{1a. 秋田から大阪まで地震がありました。X \hfill\break
1b. 秋田から大阪にかけて地震がありました。〇 \hfill\break
There was an earthquake from Akita to Osaka. }

\par{2. この道路は夕方5時半から9時にかけて必ず ${\overset{\textnormal{じゅうたい}}{\text{渋滞}}}$ する。 \hfill\break
This road is always congested from 5:30 to 9 in the evening. }

\par{3. 青森県から岩手県にかけて震度5の地震が起こりました。 \hfill\break
A Shindo scale 5 earthquake occurred from Aomori to Iwate Prefecture. }

\par{4a. 午前8時から午後7時にかけて宿題をしました。X \hfill\break
4b. 午前8時から午後7時まで宿題をしました。〇 \hfill\break
I did my homework from 8 A.M to 7 P.M. }

\par{5a. オースティンでは、北側から南側にかけて車で一時間でかかる。X \hfill\break
5b. オースティンでは、北側から南側に行くのに車で一時間かかる。〇 \hfill\break
In Austin, it takes an hour by car to get from the north side to the south side. }

\par{6. テキサスからフロリダにかけて土砂降りに降った。 \hfill\break
It down-poured from Texas to Florida. }

\par{7. 週末にかけて仕事をする。 \hfill\break
To work over the weekend. }
      
\section{The Meanings of わたる}
 
\par{ Before getting to any particular usage of わたる, it\textquotesingle s important to know all what it could possibly mean. This gives insight as to what particles and kinds of words that it is to be used with. This, of course, is the primary source of confusion and error by students, so it\textquotesingle s very important to pay attention to these things. }

\begin{ltabulary}{|P|P|P|P|}
\hline 

\# & 語義 & 適当な助詞 & 書き方 \\ \cline{1-4}

1. & To cross over & を & 渡る・渉る \\ \cline{1-4}

2. & To cross over a route\slash bridge & を & 渡る \\ \cline{1-4}

3. & One goes to\slash comes from a far place & に\slash から & 渡る \\ \cline{1-4}

4. & To pass (by) through things such as the air & を & 渡る \\ \cline{1-4}

5. & To migrate through the sky & を & 渡る \\ \cline{1-4}

6. & To wander about & を & 渡る \\ \cline{1-4}

7. & To live through & を & 渡る・亘る・亙る \\ \cline{1-4}

8. & 手に渡る = To be handed down & に & 渡る \\ \cline{1-4}

 \textbf{9 }. & To cover\slash range\slash span & に & 渡る・亘る・亙る \\ \cline{1-4}

10. & To continue without being interrupted & に & 渡る・亘る・亙る \\ \cline{1-4}

\end{ltabulary}
 
\par{\textbf{Orthography Note }: This verb regardless of meaning is almost always written as 渡る. Most Japanese people might not even recognize the alternative spellings. So, keep that in mind. }
 
\par{ These meanings for the most are interrelated with each other. So, it shouldn't feel as if you're learning over ten completely different usages. Rather, it\textquotesingle s just defining with more specificity the kinds of environments that you can find the word in. }
 
\begin{center}
\textbf{Examples }
\end{center}

\par{8. 私事に\{ ${\overset{\textnormal{かか}}{\text{関}}}$ わる・ ${\overset{\textnormal{わた}}{\text{亙}}}$ る\}質問をするのは失礼ですね。 \hfill\break
Asking personal questions is rude, isn't it? }
 
\par{9. 橋を渡る。 \hfill\break
To cross a bridge. }
 
\par{10. 通りを向こう側へ渡る。 \hfill\break
To get across the street. }
 
\par{11. 彼は中国へ渡りました。 \hfill\break
He went over to China. }
 
\par{12. その土地は彼女の息子の手に渡った。 \hfill\break
The land was passed down to her son. }
 
\par{13. 世の中をうまく渡っていく。 \hfill\break
To get along living well in the world. }

\par{14. ${\overset{\textnormal{そよかぜ}}{\text{微風}}}$ が ${\overset{\textnormal{こずえ}}{\text{梢}}}$ を渡った。 \hfill\break
The breeze passed through the treetops. }

\par{15. ${\overset{\textnormal{かり}}{\text{雁}}}$ が空中を渡る。 \hfill\break
Wild geese migrate through the skies. }
      
\section{~にわたって}
 
\par{ This grammar point is the て形 of usage \#9 from the chart above, and as implied, this pattern follows a noun that expresses some quantity in regards to time, parameter, space, etc. The noun that precedesにわたって is quite limited, and the verb phrase that follows deals with continuation. }
 
\par{16a. ゴールデンウィークにわたって関東全体を観光した。X \hfill\break
16b. ゴールデンウィークのあいだ、関東全体を観光した。〇 \hfill\break
During Golden week, I went sightseeing through all of Kanto. }

\par{17a. ${\overset{\textnormal{ろうにゃくだんじょ}}{\text{老若男女}}}$ にわたって ${\overset{\textnormal{よろんちょうさ}}{\text{世論調査}}}$ が行われました。X \hfill\break
17b. 老若男女すべてに(対して)世論調査が行われました。〇 \hfill\break
A public opinion survey was done to men and women of all ages. }

\par{18. ${\overset{\textnormal{ごじゅうかた}}{\text{五十肩}}}$ かもしれませんけど、肩から腕にかけて、痛くて、重くて・・・・。 \hfill\break
It may be a stiff shoulder, but it\textquotesingle s heavy and painful from my shoulder to my arm. }
 
\par{19. 倉田博士は35年にわたって研究を続け、新療法を開発した。 \hfill\break
Professor Kurata continued to research over 35 years and developed a new treatment. }
 
\par{20. その火事は広範囲に渡って損害を与えた。 \hfill\break
The fire gave damage over a vast range. }
 
\par{21. 営業案内が全員に渡った。 \hfill\break
A pamphlet of business information was passed out to everyone. \textbf{}}
      
\section{Sentence and Word Differences: ~にかけて VS ~にわたって}
 
\par{ As errors arise from incorrect matching of pattern and word choice, we will now example the kind of sentences and words that accompany ~にかけて and ~にわたって. }

\par{ \textbf{Condition: }\textbf{~から~にかけて。( }\textbf{だから・それで~) }}

\par{22. 今晩から明朝にかけて台風が上陸します。海岸寄りの皆さんは高潮に警戒してください。 \hfill\break
The typhoon will reach landfall from this evening to tomorrow morning. Everyone along the coast please take caution with high tide. }

\par{ This could be reworded in a sentence that more explicitly shows reasoning. In this manner, it is possible to see ~から~にかけては. }

\par{23. 今晩から明朝にかけて台風が上陸するので、海岸寄りの皆さんは高潮に警戒してください。 \hfill\break
The typhoon will reach landfall from this evening to tomorrow morning, so everyone along the coast please take caution with high tide. }

\par{24. 豪雪が予測されるので、北海道から福島県にかけては注意が必要です。 \hfill\break
Since heavy snow is expected, warning is needed from Hokkaido to Fukushima Prefectures. }

\par{\textbf{Pattern Note }: The pattern ~にかけては is quite different from above. Rather than showing a temporal or spatial range, it is followed by expressions of high evaluation regarding ability, strength, etc. In this sense, it can be translated as “regarding”. }

\par{25. 彼は法律にかけては欠けている。 \hfill\break
He lacks regarding to the (knowledge of the) law. }

\par{26. ラーメン作りにかけては王さんの右に出る人はいない。 \hfill\break
In regards to ramen making, there is no one superior to Mr. Wang. }

\par{\textbf{Condition: Clause concerning the course of a condition + ~ }\textbf{にわたって + Effect }\textbf{ }}

\par{ The effect doesn't necessarily have to be bad. However, most usages of ~にわたって fit pretty well within this framework. }

\par{27. 15年にわたって遺跡の調査が行われた。その結果、このあたりは奈良時代の住居の跡だということがわかってきた。 \hfill\break
An excavation inquiry has gone on for a span of 15 years. As a result, we have come to understand that this area is the remains of Nara Period dwellings. }

\par{\textbf{Word Notes }:  }

\par{1. Words that ~ \textbf{にかけて easily follows }: Words of time\slash place such as ~日, ~時, ~県. \hfill\break
2. Words that easily \textbf{follow ~にかけて }: Verbs that describe something happening: 降る, 起こる, ある, 発生する, 等. \hfill\break
3. Words that ~ \textbf{にわたって easily follows }: Words of some parameter such as 長年, 全般, ~年, 将来, 範囲, 等. \hfill\break
4. Words that easily \textbf{follow ~にわたって }: Verbs of continuation: 続ける, 記録する, 継続する, 行(な)う, 等. }
    