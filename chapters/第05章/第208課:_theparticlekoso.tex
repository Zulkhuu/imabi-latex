    
\chapter{The Particle こそ}

\begin{center}
\begin{Large}
第208課: The Particle こそ 
\end{Large}
\end{center}
 
\par{ こそ is a rather straightforward particle to understand, but its usage is a little tricky. }
      
\section{The Fundamentals of こそ}
 
\par{ The first instances of こそ students learn about are set phrases like こちらこそ (likewise), 今度こそ (surely next time), and 今年こそ (surely next year). However, even these set phrases can get messed up. }

\par{1a. 今年こそ危機に瀕する言語の重要性が分かるようになりました。 X \hfill\break
1b. 今年 \textbf{になってはじめて }危機に瀕する言語の重要性が分かるようになりました。〇 \hfill\break
I've finally understood for the first time this year the importance of endangered languages. }

\par{2a. 彼 \textbf{こその }努力すれば、東大に行けるでしょう。X \hfill\break
2b. 彼ほどの努力をすれば、東大に行けるでしょう。〇 \hfill\break
If you strive like him, you\textquotesingle ll be able to go to Tokyo University. }

\par{ こそ strongly emphasizes what precedes it similarly to the words "certainly" and "indeed." It can be seen after nouns (especially those concerning people), time phrases, Verb+て, and the 連用形 of verbs. }

\par{3. 企業家精神こそ最もリスクが小さな道である。 \hfill\break
The entrepreneurial spirit is certainly the least risky road to take. }

\par{4. 物腰こそ ${\overset{\textnormal{いんぎん}}{\text{慇懃}}}$ だが、根は黒い。 \hfill\break
His demeanor is indeed courteous, but his true nature is mean. }

\par{5. 明日こそ鳥は羽ばたく。 \hfill\break
The birds will certainly flap tomorrow. }

\par{6. 今こそ始めましょう。 \hfill\break
Let\textquotesingle s get started now! }

\par{7. 今度こそうまくいきますよう! \hfill\break
You'll certainly do better next time! }

\par{8. まさかの時の友こそ真の友。 \hfill\break
A friend in need is a friend indeed. }

\par{9. 今年こそ日本へ行ってみよう。 \hfill\break
This year, indeed, let's try going to Japan. }

\par{10. このひどい子宮から逃げようと失敗したが、今度こそこの地獄を切り抜けて自由を手に入れよう! \hfill\break
I failed in trying to escape this damned uterus, but next time I will definitely come clear out of this hell and attain freedom! }

\par{\textbf{Sentence Note }: Someone like Stewie from Family Guy would say something like this. }

\par{11. 健全な国民が存在してこそ国家が成り立つ。 \hfill\break
A nation is born with the existence of a healthy citizen body. }

\par{12. 喜びこそすれ、怒ることはない。 \hfill\break
There\textquotesingle s nothing to be angry about when you\textquotesingle re joyous. }

\par{\textbf{Grammar Note }: ~こそすれ・こそなれ・こそであれ involve the 已然形 of the verbal element that follow こそ. }

\par{13. 褒めこそすれ、非難することはない。 \hfill\break
She does speak highly (of others), but she never criticizes. }

\par{13. 愚かな女でこそあれ、良妻賢母だなんて、とんでもない。 \hfill\break
So long as the woman is a fool, “good wife, wise mother” means nothing. }

\par{14. 毒にこそなれ薬にはならない。 \hfill\break
Do more harm than good. }

\par{ \textbf{Word Note }: When you take out こそ in にこそなれ, you get the archaic copula なり. }

\par{ Because こそ is so emphatic, sentences with it often end in ~だ, ~(よ)う, ~べきだ, etc. It is used a lot, but it is more common in the written language. Reasons for this include sentences like the last where traditional grammar holds on. In the spoken language, it tends to be used in statements by politicians and what not, and a few of the sentences above could definitely be used as slogans. }

\par{ The particle こそ can also be paired before or after を, but this is optional and quite uncommon. It may also be paired with the particles に, へ, で, と, から, in which case it will always be preceded by these particles. }

\par{15. 山田氏\{(を)こそ・こそ(を)\}知事に推薦したい。 \hfill\break
We want to recommend Mr. Yamada to the governor. }

\par{16. 日本にこそおいしい食べ物がありますよ。 \hfill\break
There is indeed delicious food in Japan. }

\par{17. 今でこそ、嘘がつけるが、あのときは、どうもつけなかった。 \hfill\break
Now I can absolutely lie, but at that time, I couldn't whatsoever. }

\par{ Unlike using が, which sounds like the speaker is picking one thing as the focus, こそ emphasizes this “focus” as the sole thing fit for the situation. It is often used in sentences where one quality is highlighted in the first clause with こそ, but then it gets negated. When this gets flipped around, the second clause with こそ shows what\textquotesingle s actually the case. }

\par{18. 「オバマ大統領の支持率が低いですね。」「いえ、違いますよ。支持率こそ伸び悩んでいますが、何かやってくれそうですよ。」 \hfill\break
“President Obama\textquotesingle s approval rating is low, isn't it?” No, you\textquotesingle re wrong. Though his approval rating may very well be lagging behind, he seems like he\textquotesingle s going to do something for us”. }

\par{19. お金は悪の元とは言うが、お金の金銭欲こそ悪の元である。 \hfill\break
They say money is the root of evil, but it\textquotesingle s the lust for money that is the root of evil. }

\par{ Using ~こそが instead of ~こそ can\textquotesingle t be easily explained. If a noun phrase being modified by こそ is the subject of the sentence, grammatically speaking, there is nothing wrong with using ~こそが. If the particle is dropped, it may be because other things in your sentence are just off for the tone to work. }

\par{20. 去勢と卵巣除去こそが犬の数の急増を防ぐのです。猫の場合は、毒を加えたミルクで十分なはずでしょう。 \hfill\break
Spaying and neutering will definitely prevent the rapid increase of the dog population. As for cats, poison laced milk should do it. }
      
\section{~からこそ \& ~ばこそ}
 
\par{ " A+からこそ+B" and "A+ばこそ+B"  are mostly interchangeable, translating roughly as "indeed it is because." Both are used when the speaker compares the situation with past experience or knowledge and can\textquotesingle t think of anything but reason A for result B. A statement of reason, with the addition of こそ, becomes a declaration of what the speaker perceives to be fact. }

\par{21. だからこそ、強い日本を作るために、憲法をなおそう! \hfill\break
Precisely because of this, in order to create a strong Japan, let us fix the constitution! }

\par{22. あなたのことを思って\{いるから・いれば\}こそ、こう言うのです。It's precisely because I\textquotesingle m thinking about you that I say this. }

\par{23. たとえ過去に犯罪を犯したとしても、人のために全力を\{尽くすから・尽くせば\}こそ、神様に報われるのです。 \hfill\break
Even if you've committed a crime in the past, it is precisely because you are giving your all to the better of people that you will be rewarded by God. }

\par{24. 愛して\{いるから・いれば\}こそ、別れなければならないのです。 \hfill\break
It is exactly because I love you that we must separate. }

\par{25. 気が\{短いからこそ・短ければこそ\}、喧嘩してはなりません。 \hfill\break
It is precisely because of your short temper that you mustn't get into an argument. }

\par{26. 充実した毎日を送れるのは、心身が健康\{だから(こそ)・であればこそ\}だ。 \hfill\break
The reason why I am able to live fulfilled each day is the fact that my mind and body are healthy. }

\par{27. その発想があったからこそ、道子は同棲生活をとりあえず清算して山岳拠点に身を投ずることができた。 \hfill\break
It was precisely because Michiko had that conception that she was for now able to throw herself to  the base in the mountains and end her cohabitation lifestyle. \hfill\break
From 光の雨 by 立松和平. }

\par{\textbf{Word Note }: 身を投ずる = 身を投じる. }

\par{\textbf{Grammar Note }: ~ばこそ has become more literary, and it is usually not used by the younger generations. Inverting the sentences can cause grammatical problems. When ~からこそだ ends the sentence, it is often the case that こそ just gets dropped. Sometimes, keeping it can make the sentence sound unnatural, especially in the spoken language. }

\par{ Examples with the format A\{から・ば\}こそB can be reworded to BのはA\{から・ば\}こそだ if the context is resultative, but in contradictory contexts, this is not possible. However, when you can reword to BのはAからこそだ, it is because A is expressing a positive cause\slash reason. Negative situations don't go. }

\begin{center}
 \textbf{~ばこそ: 一般条件 }
\end{center}

\par{ ~ばこそ may be used in showing a general condition. This is simply based on the fact that you are using the conditional particle ば in the pattern. As stated above, the context does not involve contradictory clauses. ば functions as a general condition, and the situation is in regards to knowledge, morals, or logic of some sort. The pattern becomes unnatural when the situation is already a defined condition of the past. }

\par{28. 政府のことを\{思ったからこそ 〇・思えばからこそX\}、腹を立てたんじゃないか。 \hfill\break
It's surely because you thought about the government that you got mad, no? }

\par{29. 政府のことを思えばこそ、腹を立てることも多い。 \hfill\break
Of course, when you think about the government, you often get mad. }

\par{30. 親友の友だちが励まして\{くれるから・くれればこそ\}、挫折せずに生きていけるんだ。 \hfill\break
Because I certainly have a true friend to encourage me on, I can live on without setbacks. }

\par{ "A+\{から・ば\}こそ+B" are used to express personal declaration, but when one is making a statement about objective fact regarding the laws of nature, then they become unnatural. }

\par{31. 本当に信じていればこそ、こうして頼んでいるのです。 \hfill\break
Because I truly believe in you, I am asking you this. }

\par{32. 岡田さんのことを思うからこそ、本当のことを言うべきでしょう? \hfill\break
Because I truly think about Okada-san, should I tell him\slash her the truth? }

\par{33a. 火星にも引力が\{あるからこそ・あればこそ\}、物体は地表へと落下する。X \hfill\break
33b. 火星にも引力があるから、物体は地表へ(と)落下する。〇 \hfill\break
Since there is also gravity on Mars, objects will fall to the surface. }

\par{34. 表題作の「光と影」も、まさに医家でなければもてない眼差しと、作家でなければ見抜くことができない眼差しがあったればこそと、私は確信している。 \hfill\break
Along with the title work "Light and Shadow", I'm confident that he had the eye that only a doctor could have and the eye with which he could see through things being a novelist. \hfill\break
From 光と影 by 渡辺淳一 in the 解説 by 小松伸六. }

\par{\textbf{Grammar Note }: あったれば = あれば. }
      
\section{~てこそ}
 
\par{ AてこそB is used when the speaker is evaluating or persuading the listener based on experience\slash social wisdom\slash morals\slash ethics and a positive B coming about. Because there is a condition A, a positive B naturally\slash necessarily comes about. B must be some noun (of condition), adjective, potential or passive phrase, all of which are non-volitional. B can\textquotesingle t be a phrase that shows the speaker\textquotesingle s\slash your volition. }

\par{35. 見知らぬ人を快く歓迎してこそ、誰にも熱く歓迎されるのです。 \hfill\break
As you cheerfully welcome strangers, you will be warmly welcomed by anyone. }

\par{36. 身を捨ててこそ浮ぶ瀬もあれ。(Set Phrase) \hfill\break
Literally: Just as throw your own self away, there are also rapids for your body to float in. \hfill\break
Risk all and gain all. }

\par{37. この山中に暮らしをしてこそ、完璧に平等な理想の社会を築くことができるのではないか。人間らしい叫び声を上げることができてこそ、革命家たり得るはずなのだ。 \hfill\break
Isn't it not possible to build a perfectly impartial society by living within the mountains?  It should be possible to become a revolutionist by simply giving a human shout. \hfill\break
From 光の雨 by 立松和平. }

\par{38. 原因があり条件があってこそ現象としてとらえられるのだが、条件は刻一刻と変化する。 \hfill\break
One can grasp things as phenomena so long as there is a cause and condition, but conditions change moment by moment. \hfill\break
From 光の雨 by 立松和平. }

\par{\textbf{Grammar Note }: たり得る is a combination of たり (a classical copula) and ~得る (showing potential). It is equivalent to であることができる。 }

\par{ All instances of ~てこそ may be replaced with ~てはじめて, but this doesn't mean that ~てはじめて is always interchangeable with ~てこそ. Just like ~ばこそ, ~てこそ cannot be used with some past or individual event with a defined condition. }

\par{39. 君と会ってはじめて本当の愛の意味が分かった気がする。 \hfill\break
Since meeting you, I feel like I've understood the meaning of true love for the first time. }

\par{ AてこそB is interchangeable with A\{から・ば\}こそB whenever the situation is a defined, resultative condition, the likes of which are seen with から・ので. However, if the situation is hypothetical, you cannot paraphrase ~てこそ out with just から. It should be ~てからこそ・ばこそ・ば. If it is a situation that could be defined or hypothetical, paraphrasing ~てこそ out with ~てからこそ, ~ばこそ, ~ば, or ~から is fine. }

\par{40a. 全力を尽く\{してこそ・せば・してはじめて\}、神様に報われるのです。〇 \hfill\break
40b. 全力を尽くせば、神様に報われるのです。X \hfill\break
It is precisely because of you giving your all (for something) that you will be rewarded by God. }

\par{41a. 人々がこの地球を守ってこそ、人類はサバイバルが可能だ。〇 \hfill\break
41b. 人々がこの地球を\{守ってからこそ・守ればこそ\}、人類はサバイバルが可能だ。〇人々がこの地球を守るから、人類はサバイバルが可能だ。X\slash △ \hfill\break
So long as people protect this earth, humanity has the potential to survive. }

\par{42. 一生懸命\{頑張ってこそ・頑張るからこそ・頑張ればこそ・頑張るから・頑張れば\}、成功するのだ。 \hfill\break
You'll succeed if I try all my might. Of course, by now you should realize that there are minor nuance changes in switching between the options, but at least you know what a situation looks like in which all are fine. }
    