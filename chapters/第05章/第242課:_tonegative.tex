    
\chapter{The Particle と III}

\begin{center}
\begin{Large}
第242課: The Particle と III: と + \dothyp{}\dothyp{}\dothyp{}~ない 
\end{Large}
\end{center}
 
\par{ We will look at yet another use of the case particle と. This one is quite different than all the other usages we've looked at as it behaves more like an adverb than any of the other case particles. So, keep this in mind. }
      
\section{と + \dothyp{}\dothyp{}\dothyp{}~ない}
 
\par{ と may strengthen negative expressions used with counter expressions and some other adverbs. It is especially common in the phrase 二度と. This usage stresses that things aren't going to be so. Although this usage is classified as a case particle usage, it's easy to view it as an adverbial particle as it is always paired with adverbial phrases (counters). }

\par{1. もう二度とあんなところへ行かない。 \hfill\break
I'm not going to go to a place like that ever again. }

\par{2. 3 ${\overset{\textnormal{ぷん}}{\text{分}}}$ とかからない。 \hfill\break
It won't take more than 3 minutes. }

\par{3. ${\overset{\textnormal{いのち}}{\text{命}}}$ は二つとない。 \hfill\break
Life is but once. }

\par{4. ${\overset{\textnormal{なま}}{\text{生}}}$ ものは3日ともたない。 \hfill\break
Raw foods won't last three days. }

\par{5. 一分と待てない。 \hfill\break
I can't even wait one minute. }

\par{6. ${\overset{\textnormal{へいおん}}{\text{平穏}}}$ は3日ともたない。 \hfill\break
Tranquility won't last but three days. }

\par{7. 色んな ${\overset{\textnormal{ちりょう}}{\text{治療}}}$ を受けても1週間ともたない。 \hfill\break
No matter what various treatment (I) receive, it lasts but a week. }

\par{8. 時間は一秒とかからない。 \hfill\break
Not even a second will go by. }

\par{9. この数学問題を解くのに5分とかからなかった。 \hfill\break
It didn't take more than five minutes to solve this math problem. }

\par{10. この ${\overset{\textnormal{しょくりょう}}{\text{食料}}}$ では1日ともたない。 \hfill\break
We won't even make it a day with this food. }

\par{11. この食料は1日ともたない。 \hfill\break
This food won't make it a day. }

\par{12. 同じものは二つとない。 \hfill\break
There isn't a same thing twice. }

\par{13. これは世界に2個もないものですね。 \hfill\break
There is only one in the world. }

\par{14. 雲一つとない青空を見上げる。 \hfill\break
To look up at the blue sky with not a single cloud. }

\par{15. 二ヶ月5000円とかからなかった。 \hfill\break
It didn't even cost more than ¥5000 for two months. }

\par{16. ${\overset{\textnormal{げり}}{\text{下痢}}}$ が出るまで1分と ${\overset{\textnormal{がまん}}{\text{我慢}}}$ できない。 \hfill\break
I can't even hold it one minute till diarrhea comes out. }

\par{17. こんな機会はまたとない。 \hfill\break
Such an opportunity won't come again. }

\par{18. 大失敗は二度と繰り返しません。 \hfill\break
Great failure won't repeat once more. }

\par{19. だめだ、何をやっても2日と続かない。 \hfill\break
It's useless, I can't keep going at something for two days no matter what I do. }

\par{20. この問題は手計算だと10分かかるが、パソコンに解かせれば1秒とかからない。 \hfill\break
This problem by hand takes 10 minutes, but if you get get it solved by computer, it doesn't even take a second. }

\begin{center}
 \textbf{も VS と }
\end{center}

\par{ This is very similar to the particle も, but the particle も emphasizes the greatness of something and can either be used with positive or negative sentences. This is in stark difference with this usage of と which often emphasizes the smallness and often showing intolerance about things. For instance, in 22, the speaker, despite 1 hour being reasonably short for the situation, is not wanting to be patient anymore. }

\par{21. 一時間も待った。 \hfill\break
I waited at least one hour. }

\par{22. もう一分と待たない。 \hfill\break
I won't even wait a minute. }

\par{23. 税金を一円と払わない! \hfill\break
I won't pay even a yen in taxes! }

\par{24. 税金を10万円も払えない。 \hfill\break
I can't even pay ¥100,000 in taxes. }

\par{25a. 誰ひとりと知らない。 \hfill\break
25b. 誰ひとりも知らない。 \hfill\break
Not a single person knows. \hfill\break
No one knows. }

\par{ 25a stresses the smallness of the quantity that one\slash someone either doesn't know more than essentially no one or that no one doesn't know something. 25b is the same but makes a big deal about the situation and is consequently more common than 25a. Say no one knows your new address, if you used 25a in the dialogue, you're just stating that no one knows, but if you say 25b, you're making an issue out of things and emphasizing the situation as a problem. }
    