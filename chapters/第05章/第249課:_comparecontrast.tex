    
\chapter{Compare \& Contrast}

\begin{center}
\begin{Large}
第249課: Compare \& Contrast: ~に比べて, ~に引き換え, ~に反して,~にもまして, \& ~ないまでも 
\end{Large}
\end{center}
 
\par{ The patterns in this lesson deal with compare and contrast. Although not near as similar to each other as topics in other lessons, you still need to pay attention to detail so that you don't confuse them with each other. }
      
\section{~に比べて}
 
\par{ Also ~と比べて, ~に比べると, ~に比べ, ~と比べ, and ~と比べると, ~に比べて means "compared to". It is a simple comparison. Without ~て, of course, the pattern becomes more literary. }

\par{1. 外は暑いが、それに比べて中は寒い。 \hfill\break
It's hot outside, but, moreover, in contrast, it's cold inside. }

\par{2. この本は昨日読んだのと比べると全くつまらない。 \hfill\break
This book compared to the one I read yesterday is completely boring. }

\par{3. 彼は妹と比べて若く見えました。 \hfill\break
He looked younger beside his younger sister. }
4. 今年は去年に比べ、雪の ${\overset{\textnormal{りょう}}{\text{量}}}$ が多い。 \hfill\break
In contrast to last year, there has been more snowfall this year. 
\par{5. 男子学生と女子学生の比率は、2対1だ。 \hfill\break
The ratio of male and female students is 2 to 1. }

\par{\textbf{Word Note }: The last example is shown to give other similar words with the character 比. }
      
\section{~に引き換え}
 
\par{ ~に引き換え is used to show a sharp contrast in which something is greatly better or worse than something else. This can only go after nominal phrases, so you have to nominalize verbs and adjectives if you want to use them together. }

\par{6. 健太の住んでいるマンションは新しくて、広い。それに引き換え、僕のところは古くて狭いし、駅からも遠いよ。 \hfill\break
The apartment that Kenta is living at is new and wide. By contrast, my place is old, small, and also far from the train station. }

\par{7. 同じ年の人に引き換え、彼はとても頭がよくて、たくさんの素晴らしい ${\overset{\textnormal{いぎょう}}{\text{偉業}}}$ を成し ${\overset{\textnormal{と}}{\text{遂}}}$ げました。 \hfill\break
In sharp contrast to people of his same age, he is very smart and has made a lot of wonderful achievements. }

\par{8. ${\overset{\textnormal{わくせい}}{\text{惑星}}}$ の大きさに引き換え、 ${\overset{\textnormal{めいおうせい}}{\text{冥王星}}}$ はとても小さい。 \hfill\break
In sharp contrast to the size of a planet, Pluto is very small. }
      
\section{~に反して}
 
\par{ 反する means "contrary" and に反して means "contrary to\slash against". }

\par{9. ${\overset{\textnormal{よそう}}{\text{予想}}}$ に反して成功するのはいつもいいことでしょう。 \hfill\break
Succeeding against one's expectations is always a good thing isn't it? }

\par{10. 神の意に反して人間は ${\overset{\textnormal{たが}}{\text{互}}}$ いに殺害している。 \hfill\break
Against the will of God, humans are slaying each other. }

\par{11. 我々の期待に反していた。 \hfill\break
It was contrast to our expectations. }

\par{12. ${\overset{\textnormal{こくぼうちょうかん}}{\text{国防長官}}}$ は事実に反する報告をしました。 \hfill\break
The Secretary of Defense made a report contradictory to the facts. }
      
\section{~にもまして}
 
\par{ まして is an adverbial phrase that shows something is beyond the degree of the norm, past, other things, or the present situation. If the sentence is negative, of course, you say the opposite. ~にもまして takes this back to a verbal phrase with the same purpose. }

\par{ You may also see ${\overset{\textnormal{いわん}}{\text{況}}}$ や, which is a contraction of the verb 言う, the auxiliary verb ~む, and や, which equate to something on the lines of いうまでもなく. This phrase is typically seen in negative sentences to mean “let alone”. }

\begin{center}
\textbf{Examples }
\end{center}

\par{12. あいつは日本語を読むことすらできない。まして書くのと話すなどできるものではない。 \hfill\break
He can't even read Japanese. Much less write it or speak it. }

\par{13. 彼女は小走りもろくにできない。 ${\overset{\textnormal{ま}}{\text{況}}}$ して走れるわけがない。 \hfill\break
She can hardly jog, much less being able to run. }

\par{14. 今年は去年にもまして暑さが厳しい。 \hfill\break
The heat is far more severe than even last year. }

\par{15. 彼はほとんど目が見えない。ましてや、読めるわけがないよ。 \hfill\break
He can't really see, much less being able to read. }

\par{16. ${\overset{\textnormal{てき}}{\text{敵}}}$ でも困っていたら助けます。ましてや ${\overset{\textnormal{みかた}}{\text{味方}}}$ なら当然です。 \hfill\break
I would help an enemy if her were to be in distress, much more a friend. }

\par{17. いわんや子供には無理だね。 \hfill\break
It\textquotesingle s useless much less with kids. }

\par{18. 最近は以前にもまして物覚えが悪くなった。 \hfill\break
Recently, my memory has gotten worse than it was even before. }
      
\section{~ないまでも}
 
\par{ This shows that something has not reaching past a certain level, but it is a little below. }

\par{19. 毎週とはいわないまでも、せめて月に1回は映画館に行きたい。 \hfill\break
Not that it'd have to be every week, but I'd like to at least go to the movies once a month. }

\par{20. ${\overset{\textnormal{かいせい}}{\text{快晴}}}$ とはいかないまでも、雨は降らないでほしい。 \hfill\break
Not that I won't go if it doesn't turn out to be clear weather, but I'd like it to not rain. }
    