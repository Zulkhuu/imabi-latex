    
\chapter{Can't Help III}

\begin{center}
\begin{Large}
第236課: Can't Help III: ~ないではすまない, ~ないではおかない, ~を余儀なく[される・させる], \& ~を禁じ得ない 
\end{Large}
\end{center}
 
\par{ These are even more difficult phrases dealing with not being able to help one's emotions, etc. Not only do you need to try to separate these in your mind from the phrases found in the two previous lessons about this topic, but you're going to have to pay even more attention to the differences among these patterns. }
      
\section{~\{ないで・ずに\}はすまない}
 
\par{ Also seen as ~ずにはすまない, which is more formal but can still be found used in the spoken language, is used to show that if one thinks from a societal point of view with a given circumstance, doing something is simply and certainly unavoidable. It is difficult to use, however, when you think that you have to do something from personal emotion. Given that the phrase is stiff, this should be understandable. }

\par{1. 人の心を傷つけてしまったなら、謝らずにはすまない。 \hfill\break
If you are to hurt someone\textquotesingle s heart, you can\textquotesingle t avoid apologizing. }

\par{2. 借金をせずにはすむまい。(古風で、硬い言い方) \hfill\break
You can\textquotesingle t get by without borrowing money. }

\par{3. 親に知られたら叱られないではすまないよ。 \hfill\break
If your parents found out, you won\textquotesingle t be able to avoid getting scolded. }

\par{ The non-double negative form above result in ~ないで ${\overset{\textnormal{す}}{\text{済}}}$ む, which means "to get by without\dothyp{}\dothyp{}\dothyp{}". Just as in English, this implies a good thing because you got away without doing something. 済む itself may show that something is done. }

\par{4. 済んだことだ。 \hfill\break
It's finished. }

\par{5. 済んだのですか。 \hfill\break
Is it over? }

\par{6. 金を払わないで済みました。 \hfill\break
I got by without paying for it. }

\par{7. 気づかれずに済む。 \hfill\break
To escape attention. }

\par{8. 彼が ${\overset{\textnormal{けが}}{\text{怪我}}}$ をせずに済んでよかった。 \hfill\break
I'm glad that he got by without injury. }

\par{9. ${\overset{\textnormal{じょうだん}}{\text{冗談}}}$ ですまない。 \hfill\break
To go beyond a joke. (Shows guilt) }

\par{10. 済んだことは ${\overset{\textnormal{もと}}{\text{元}}}$ に ${\overset{\textnormal{もど}}{\text{戻}}}$ らない。 \hfill\break
What's done cannot be undone. }

\par{11. 蜂を見つけたら、ゆっくり離れたほうがいい。体に止まっても振り払わずにじっとしていれば刺されずに済むよ。 \hfill\break
If you are to find a bee, it\textquotesingle s best to separate oneself from it slowly.  Even if it lands on you, you can leave without being stung if you stay still. }

\par{12. もう一つは、犯行後、同じように夜を待ち、普通の道を河内長野方面に向かって歩いたかもしれないことだ。これは車道以外に近道の旧道もあるので、ここを歩けば夜間は誰にも見られずに済む。 \hfill\break
Another possibility is that (the criminal) waited just like that for night after the crime and walked down a regular road toward Kawachi-Nagano. This could go without being seen by anyone at night walking through here because there are also old shortcuts aside from the roadway. \hfill\break
From 二重葉脈 by 松本清張. }
      
\section{~\{ないで・ずに\}はおかない}
 
\par{ Also seen as ~ずにはおかない, this pattern is used to show that one will certainly happen naturally or that you won\textquotesingle t allow something to stay not being done. It shows strong resolution. The situation where one won\textquotesingle t allow for something deals with first person. However, the first situation deals with subjects outside of first person, which includes inanimate things. }

\par{13. あの話はやはり嘘だったと、絶対に ${\overset{\textnormal{はくじょう}}{\text{白状}}}$ させないではおかないぞ。 \hfill\break
If that was indeed a lie, I will absolutely not stand without making (that person) confess. }

\par{14. 彼女が歌ったのは聞く人の心を ${\overset{\textnormal{ゆ}}{\text{揺}}}$ さぶらずにはおかない。 \hfill\break
What she had sung will surely move the hearts of those who hear. }
      
\section{~を余儀なく[される・させる]}
 
\par{${\overset{\textnormal{よぎ}}{\text{余儀}}}$ means "another way\slash problem" and 余儀ない expresses "out of one's control". So, ~を余儀なくされる, shows that one had to do something that was not of one's will and there was no other choice to get around it. You are essentially driven into a corner in a certain situation. This can also be seen in the causative as ~を余儀なくさせる. Usually the subject of sentences with ~を余儀なくされる is usually human, but the subject of sentences with ~を余儀なくさせる is usually not human but a something. }

\par{15. 悪天候のため試合の中止を余儀なくされました。 \hfill\break
Bad weather made us postpone the match. }

\par{16. アメリカの圧力が ${\overset{\textnormal{ぼうえき}}{\text{貿易}}}$ 自由化を余儀なくさせた。 \hfill\break
American pressure made free trade unavoidable. }

\par{17. 暴風雨で遠足は延期を余儀なくされた。 \hfill\break
We had to put off the field trip due to fierce rain. }

\par{18. \{余儀ない・やむを得ない\}事情で欠席しました。 \hfill\break
I was absent due to an uncontrollable situation. }

\par{${\overset{\textnormal{}}{\text{19. 辞任}}}$ を余儀なくされる。 \hfill\break
To become out of one\textquotesingle s control to resign. }
      
\section{~を禁じ得ない}
 
\par{ Although most dictionaries will just tell you this is the same as ~ないではいられない, this is not enough information to help you get a question right about it on the JLPT. Like all the other phrases in this series, this phrase describes not being able to withhold emotions, but this portrays an image of not being able to restrain emotions that have sprung up due to the circumstances. This phrase attaches to nouns that imply emotion. Sentences with it are usually first person. }

\par{20. 涙を禁じえなかった。 \hfill\break
I couldn't hold back the tears. }

\par{21. 犯人の供述を聞き、 ${\overset{\textnormal{はんこうどうき}}{\text{犯行動機}}}$ の ${\overset{\textnormal{みがって}}{\text{身勝手}}}$ さに ${\overset{\textnormal{いか}}{\text{怒}}}$ りを禁じ得なかった。 \hfill\break
I couldn't withhold my anger at the selfishness of the motive from listening to the criminal\textquotesingle s affidavit. }

\par{22. 同情の念を禁じえない。 \hfill\break
I can't hold back sympathy }

\par{23. 残された子犬を見て、涙を禁じ得なかった。 \hfill\break
I couldn't hold back the tears from looking at the puppy that was left.  }
    