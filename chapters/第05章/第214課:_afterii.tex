    
\chapter{After II}

\begin{center}
\begin{Large}
第214課: After II: ~てからというもの(は) \& ~て以来 
\end{Large}
\end{center}
 
\par{ This lesson will primarily focus on ~てからというもの(は), but it will also take time to compare it with ~てから ~て以来, which has not been discussed at this point. }
      
\section{~てからというもの(は)}
 
\par{ ~てからというもの(は): Right after something happens, there is a change, and that change persists from that point on. This pattern is not used for things in the recent past. This pattern is not used very often. So, it may be somewhat unnatural if the context doesn't sound really serious. }

\par{1. ギリシャに来てからというもの、国の家族のことを思わない日はありません。 \hfill\break
Ever since I've come to Greece, there has not been a day where I don't think of my family at home. }

\par{2. ドル安の問題は深刻である。今年度になってからというもの、ドル安(の) ${\overset{\textnormal{けいこう}}{\text{傾向}}}$ は進む一方である。 \hfill\break
The weak dollar is a serious problem. The weak dollar trend will continue onward through this fiscal year and onward. }

\par{3. このタブレット ${\overset{\textnormal{たんまつ}}{\text{端末}}}$ を使ってからというもの、 ${\overset{\textnormal{てばな}}{\text{手放}}}$ せなくなった。 \hfill\break
Ever since using this tablet, I have not been able to let go of it. }
      
\section{Comparison}
 
\par{ To begin comparing ~てから, ~てからというもの, and ~て以来 (which will be discussed for the first time), consider the following examples. }

\par{4. この夏の ${\overset{\textnormal{けんさ}}{\text{検査}}}$ で、 ${\overset{\textnormal{しんこう}}{\text{進行}}}$ ${\overset{\textnormal{すいぞうがん}}{\text{膵臓癌}}}$ と ${\overset{\textnormal{しんだん}}{\text{診断}}}$ されて\{から・からというもの・以来\}、明日にでも死ぬかのような ${\overset{\textnormal{きょうふ}}{\text{恐怖}}}$ に ${\overset{\textnormal{}}{\text{捉}}}$ われている。 \hfill\break
Ever since being diagnosed with advanced pancreatic cancer in this summer\textquotesingle s scan, I have been entrapped by the fear of maybe dying at morning\textquotesingle s wake. }

\par{5. 就職して15年、部長になって\{から・からというもの・以来\}、ほとんど毎晩のように ${\overset{\textnormal{ざんぎょう}}{\text{残業}}}$ で遅くなり、目に見えて ${\overset{\textnormal{}}{\text{顔色}}}$ が悪くなっている。 \hfill\break
Since becoming the department head after working for 15 years, he has been late almost every night it seems due to overtime, and his complexion has visibly worsened. }

\par{6. ${\overset{\textnormal{たけやま}}{\text{武山}}}$ ${\overset{\textnormal{み}}{\text{美}}}$ ${\overset{\textnormal{さこ}}{\text{砂子}}}$ 15歳のときからずっと付き合っていた ${\overset{\textnormal{こいびと}}{\text{恋人}}}$ と ${\overset{\textnormal{しつれん}}{\text{失恋}}}$ して\{から・からというもの・以来\}、食事も ${\overset{\textnormal{のど}}{\text{喉}}}$ を通らなくなって、 ${\overset{\textnormal{やつ}}{\text{窶}}}$ れ ${\overset{\textnormal{は}}{\text{果}}}$ てて、見る ${\overset{\textnormal{かげ}}{\text{影}}}$ もなくなってしまった。 \hfill\break
Misako Takeyama couldn\textquotesingle t put down food and withered down to the point of leaving a mere shadow of her former self ever since being lovelorn with her partner she had been with since the age of 15. }

\par{7. 愛犬を亡くして\{から・からというもの・以来\}、すっかり元気をなくし、人が変わったようになってしまった。 \hfill\break
Since losing (his) beloved dog, (he) has lost all vitality, and (his) entire persona changed. }

\par{ “A + て\{からというもの・以来\}+B” means that “After Situation A occurs, Situation B continues indefinitely since”. The first option ~てからというもの agrees well with native Japanese phrasing and is occasionally used in the spoken language. On the other hand, ~て以来 is more appropriate in the written language. Even so, there are still nuance differences to keep in mind. }

\begin{center}
 \textbf{Traditional Nuances of ~てからというもの \& ~て以来 }
\end{center}

\par{ First, the nuances implied in Standard Japanese (標準語) will be discussed. It is important to understand now that what is to be described is now more often than not the case if you were to talk to speakers of the younger generation. However, knowing what has traditionally been went will help you when you see these phrases used in literature. }

\par{ ~てからというもの usually has a general negative evaluation of Situation B. There is the thought in the speaker that things could be a whole lot better, and there is thus a sense of severity of seriousness involved. Contrarily, ~て以来 is the opposite and generally has Situation B be positive. Both are ungrammatical, though, if Situation B is not continuing. So, neither allows Situation B to be the instant of a change. }

\par{ In reality, not even ~てからというもの is used that much in the spoken language. It has essentially become almost as 書き言葉的 as ~て以来 despite not being a Sino-Japanese phrase. }

\par{ Though Situation B has its limitations, there is historically no problem in negating it within the same sentence afterward. In order for this to work for both, a 期間 (time period) phrase such as \#年 is attached after ~てから or ~て以来. So, you get ~てから\#年というもの or ~て以来\#年というもの. In fact, this pattern is often used without ~てから or ~て以来 for the same effect. }

\par{8. セスさんは、愛犬を亡くして\{から・以来\}3年というもの、人が変わったようになっていたが、最近、少し元気を取り戻しつつあるようだ。 \hfill\break
Seth seemed as if his personality shifted drastically since his dear dog died three years ago, but he has recently seemed to be recovering his vitality. }

\par{9. 今年は、とりわけ干ばつが長引き、この一月というもの、火事などが ${\overset{\textnormal{おお}}{\text{大}}}$ ${\overset{\textnormal{あば}}{\text{暴}}}$ れをして、各地に大きな ${\overset{\textnormal{ひがい}}{\text{被害}}}$ をもたらしている。 \hfill\break
As for this year, drought has especially prolonged, and in this month, wild fires and such have raged on and continued to bring great damage everywhere. }

\begin{center}
\textbf{Restrictions a Thing of the Past }
\end{center}

\par{ It appears, though, that the restrictions on these phrases are disappearing due to disuse, and some can\textquotesingle t really tell that they are different than ~てから. But, there should be some people who think gloom is coming if they heard the following with ~てからというもの. As said at the beginning, it is now more so emphatic with one\textquotesingle s emotional appeal in presenting a change since as being severe or serious in now either a good or bad way. }

\par{10. 先月、 ${\overset{\textnormal{けいば}}{\text{競馬}}}$ で ${\overset{\textnormal{おおあな}}{\text{大穴}}}$ を当ててからというもの、何であれ、勝ちに勝ちまくっているそうだ。 \hfill\break
Ever since winning big last month in the horse races, I hear that he\textquotesingle s raking in the wins no matter what the game is. }

\par{11. 大学生になってからというもの、毎日が楽しい。 \hfill\break
Ever since becoming a college student, everyday is fun. }

\par{12. タバコをやめてからというもの、体が元気になった。 \hfill\break
Ever since quitting smoking, my body has felt great. }

\par{13. タバコをやめてからというもの、体の調子がいい。 \hfill\break
Ever since quitting smoking, my physical condition has been good.  }
    