    
\chapter{Intransitive \& Transitive}

\begin{center}
\begin{Large}
第239課: Intransitive \& Transitive: Part V (Sino-Japanese Verbs) 
\end{Large}
\end{center}
 
\par{ In our fifth installment on verbs that do not change based on whether it is used as an intransitive or a transitive verb, we will focus on Sino-Japanese examples. }

\par{ There are plenty of verbs from Chinese that can be used either in an intransitive sense or a transitive sense because there is no morphological distinction made in Chinese. Therefore, the lack of marking transitivity found in Chinese simply carries over into Japanese. Japanese then compensates by using its backup system of particles, if you will, to help the speaker determine how the verb should be interpreted. }

\par{ This, as one might imagine, does cause issues. As you will soon see in the example sentences, many speakers frequently change する to される or to させる depending on whether they wish to make it clear that the Sino-Japanese verb in question is being used in an intransitive or transitive sense respectively. This causes grammatical ambiguity, understandably, because される and させる stand for the passive and causative forms respectively. }

\par{ Before you go on thinking that Japanese is being overly complicated, think about English for one moment. English is just as guilty as Chinese for not marking transitivity in verbal conjugations. }

\par{i. I started the movie three minutes ago. \hfill\break
ii. The movie started three minutes ago. }

\par{ If English does a poor job in marking transitivity, and if it\textquotesingle s the case that Japanese has borrowed many words from English, one might also assume that Sino-Japanese verbs are not the only foreign verbs that have this transitivity problem. \hfill\break
 \hfill\break
iii. ${\overset{\textnormal{じむしょりのうりょく}}{\text{事務処理能力}}}$ をアップする ことで、 ${\overset{\textnormal{しごと}}{\text{仕事}}}$ のすべての ${\overset{\textnormal{のうりょく}}{\text{能力}}}$ がアップする のですから、 ${\overset{\textnormal{こうりつか}}{\text{効率化}}}$ をアップさせ ましょう。 \hfill\break
By \textbf{raising }one\textquotesingle s clerical work capacity, all your job skills \textbf{will improve }, so try \textbf{upping }your efficiency. }

\par{ This example marvelously demonstrates the flux in transitivity that you will find with the verbs discussed in this lesson. Now, to learn as many of these verbs as possible, each Sino-Japanese verb taught will have a minimum of two sentences to account for its intransitive and transitive use. If nuance requires further investigation, more example sentences will be provided. }
      
\section{Common Dual Purpose Sino-Japanese Verbs}
 
\par{・変形する - To transform\slash metamorphize\slash deform }

\par{\emph{ }変形する is used both in the spoken and written language. Its intransitive and transitive usages are both very common. }

\par{1. ${\overset{\textnormal{しき}}{\text{式}}}$ を ${\overset{\textnormal{へんけい}}{\text{変形}}}$ する ${\overset{\textnormal{もんだい}}{\text{問題}}}$ です。 \hfill\break
This problem is about transforming an equation. \hfill\break
 \hfill\break
2. ${\overset{\textnormal{なんこつ}}{\text{軟骨}}}$ が ${\overset{\textnormal{ぞうしょく}}{\text{増殖}}}$ したり、 ${\overset{\textnormal{ほね}}{\text{骨}}}$ が ${\overset{\textnormal{なんか}}{\text{軟化}}}$ したりするこことで、 ${\overset{\textnormal{かんせつ}}{\text{関節}}}$ が ${\overset{\textnormal{へんけい}}{\text{変形}}}$ していきます。 \hfill\break
Joints become deformed by cartilage increasing, bones softening, etc. }

\par{・分解する – To disassemble\slash dismantle\slash decompose\slash factor\slash deblock }

\par{\emph{ }分解する is used both in the spoken language and written language. It is especially important in construction, science, and computer science. Its intransitive and transitive usages are both very common. }

\par{3. ${\overset{\textnormal{みず}}{\text{水}}}$ を ${\overset{\textnormal{でんきぶんかい}}{\text{電気分解}}}$ すると、 ${\overset{\textnormal{すいそ}}{\text{水素}}}$ と ${\overset{\textnormal{さんそ}}{\text{酸素}}}$ が ${\overset{\textnormal{はっせい}}{\text{発生}}}$ し ${\overset{\textnormal{みず}}{\text{水}}}$ が ${\overset{\textnormal{ぶんかい}}{\text{分解}}}$ します。 \hfill\break
When you electrolyze water, hydrogen and oxygen are produced, which results in the water decomposing. }

\par{4. ${\overset{\textnormal{かき}}{\text{下記}}}$ の ${\overset{\textnormal{しき}}{\text{式}}}$ を ${\overset{\textnormal{いんすうぶんかい}}{\text{因数分解}}}$ しなさい。 \hfill\break
Factor the equation(s) below. }

\par{5. ${\overset{\textnormal{しぼう}}{\text{脂肪}}}$ を ${\overset{\textnormal{ぶんかい}}{\text{分解}}}$ する ${\overset{\textnormal{こうそ}}{\text{酵素}}}$ があります。 \hfill\break
There is an enzyme that breaks down fat. }

\par{・決定する -  To decide\slash determine }

\par{\emph{ }決定する is largely literary, but it is commonly used in news reports. Its intransitive form is more common than its transitive form, but neither usage is rare by any means. }

\par{6. ${\overset{\textnormal{いいんかい}}{\text{委員会}}}$ は ${\overset{\textnormal{こんねんど}}{\text{今年度}}}$ の ${\overset{\textnormal{せいさくほうしん}}{\text{政策方針}}}$ を ${\overset{\textnormal{けってい}}{\text{決定}}}$ した。 \hfill\break
The committee decided upon the line of policy for this fiscal year. }

\par{7. ${\overset{\textnormal{ほうそうひ}}{\text{放送日}}}$ が ${\overset{\textnormal{けってい}}{\text{決定}}}$ しました。 \hfill\break
 \emph{Hōsōbi ga kettei shimashita. }\hfill\break
The air date has been determined. }

\par{\textbf{Grammar Note }: Some speakers use 決定される for the intransitive usage, but this is not grammatically necessary. }

\par{・内定する - To make a tentative decision }

\par{\emph{ }内定する is formal and literary. Its intransitive usage is the most common. }

\par{8. ${\overset{\textnormal{せんしゅう}}{\text{先週}}}$ の ${\overset{\textnormal{せんきょ}}{\text{選挙}}}$ で ${\overset{\textnormal{らくせん}}{\text{落選}}}$ した○○ ${\overset{\textnormal{し}}{\text{氏}}}$ が ${\overset{\textnormal{ないてい}}{\text{内定}}}$ したことが ${\overset{\textnormal{わ}}{\text{分}}}$ かりました。 \hfill\break
It has been discovered that Mr. \#\#, who lost in last week\textquotesingle s election, has been unofficially decided (for a certain post). }

\par{9. ${\overset{\textnormal{やくいんじんじ}}{\text{役員人事}}}$ を ${\overset{\textnormal{ないてい}}{\text{内定}}}$ しました。 \hfill\break
We have tentatively decided on officer resources. }

\par{・継続する }

\par{\emph{ }継続する is more so literary than a spoken word, but it is quite commonly used adverbially in the gerund—as 継続して –to express a continuation of a certain situation. This comes from its transitive usage. Its intransitive usage is more or less a formal synonym of \emph{ }続く. }

\par{10. 既存のドメインを ${\overset{\textnormal{けいぞく}}{\text{継続}}}$ して ${\overset{\textnormal{つか}}{\text{使}}}$ えますか。 \hfill\break
 \emph{Kison\slash kizon no domein wo keizoku shite tsukaemasu ka? }\hfill\break
Can I continue using my existing domain? }

\par{\textbf{Reading Note }: The traditional reading of 既存 is \emph{きそん }, but きぞん is becoming more and more common. In the case of this word, pronouncing it as きぞん helps distinguish it from 毀損 (defamation), which is read as きそん. }

\par{11. ${\overset{\textnormal{けいざい}}{\text{経済}}}$ の ${\overset{\textnormal{しんてん}}{\text{進展}}}$ が ${\overset{\textnormal{けいぞく}}{\text{継続}}}$ している。 \hfill\break
Economic development is continuing. }

\par{・持続する – To persist\slash last\slash sustain }

\par{ Although similar to 継続する, 持続する is used to indicate that status is persisting and being sustained whereas 継続する only describes a condition that is continuing from before. An end point to the state in question is left far more uncertain with 持続する than with 継続する. Similarly, it too is largely used in the written language, but it is also commonly used in news reports. Both its intransitive and transitive usages are commonly used. }

\par{12. ${\overset{\textnormal{へいきん}}{\text{平均}}}$ ${\overset{\textnormal{ろく}}{\text{6}}}$ ${\overset{\textnormal{じかん}}{\text{時間}}}$ ほど\{ ${\overset{\textnormal{き}}{\text{効}}}$ き ${\overset{\textnormal{め}}{\text{目}}}$ ・ ${\overset{\textnormal{やっこう}}{\text{薬効}}}$ \}が ${\overset{\textnormal{じぞく}}{\text{持続}}}$ します。 \hfill\break
The effects last for an average of approximately six hours. }

\par{13. ${\overset{\textnormal{ちょうわ}}{\text{調和}}}$ のとれた ${\overset{\textnormal{かんけい}}{\text{関係}}}$ を ${\overset{\textnormal{じぞく}}{\text{持続}}}$ することが ${\overset{\textnormal{だいいち}}{\text{第一}}}$ です。 \hfill\break
Sustaining a balanced relationship is first and foremost. }

\par{・連続する – To occur in succession }

\par{\emph{ }連続する is commonly used in both the spoken and written language. Its transitive usage is not as common, but when the verb is used as a gerund in 連続して, it can come from either its intransitive or transitive usage. }

\par{14. ${\overset{\textnormal{しゅうしょくかつどう}}{\text{就職活動}}}$ は ${\overset{\textnormal{しっぱい}}{\text{失敗}}}$ が ${\overset{\textnormal{れんぞく}}{\text{連続}}}$ するのが ${\overset{\textnormal{あ}}{\text{当}}}$ たり ${\overset{\textnormal{まえ}}{\text{前}}}$ だ。 \hfill\break
It\textquotesingle s only natural to continuously fail in job hunting. }

\par{15. ${\overset{\textnormal{こうひんしつかこう}}{\text{高品質加工}}}$ を ${\overset{\textnormal{れんぞく}}{\text{連続}}}$ することが ${\overset{\textnormal{かのう}}{\text{可能}}}$ となった。 \hfill\break
It has become possible to continually perform high quality manufacturing. }

\par{16. ${\overset{\textnormal{れんぞく}}{\text{連続}}}$ して ${\overset{\textnormal{よやく}}{\text{予約}}}$ を ${\overset{\textnormal{と}}{\text{取}}}$ ることはできますか。 \hfill\break
Is it possible to continuously make reservations? }

\par{\textbf{Nuance Note }: If there is in fact brief intervals in repeatedly doing an action, 連続で rather than \emph{ }連続して is appropriate. }

\par{・展開する – To develop\slash unfold\slash extend }

\par{ As an intransitive verb, 展開する is essentially interchangeable with 広がる, but 広がる is far more common in both the spoken and written language. In the sense of "to develop\slash unfold,” however, it can be used in an intransitive and or transitive sense. In an intransitive sense, many speakers opt to change it to 展開される. This is likely because the agent of the development is implicitly felt to be relevant. Overall, the verb is more so used in the written language, but it isn\textquotesingle t all that rare in the spoken language. }

\par{17. ${\overset{\textnormal{はる}}{\text{遙}}}$ かに ${\overset{\textnormal{さがへいや}}{\text{佐賀平野}}}$ が\{ ${\overset{\textnormal{ひろ}}{\text{広}}}$ がっている・ ${\overset{\textnormal{てんかい}}{\text{展開}}}$ している\}。 \hfill\break
The Saga Plain extends in the distance. }

\par{18. ${\overset{\textnormal{ゆういぎ}}{\text{有意義}}}$ な ${\overset{\textnormal{ぎろん}}{\text{議論}}}$ が ${\overset{\textnormal{てんかい}}{\text{展開}}}$ されました。 \hfill\break
A meaningful discussion developed (by the participants). }

\par{19. ${\overset{\textnormal{せんもんか}}{\text{専門家}}}$ たちが ${\overset{\textnormal{せいりょくてき}}{\text{精力的}}}$ な ${\overset{\textnormal{かつどう}}{\text{活動}}}$ を ${\overset{\textnormal{てんかい}}{\text{展開}}}$ した。 \hfill\break
The experts developed an energetic activity. }

\par{・移動する – To move\slash transfer\slash migrate \hfill\break
 \hfill\break
 The verb 移動する is used as a slightly formal means to simply show the movement\slash transferring\slash migration from one place to another. You will see it used in all sorts of situations including in computer science settings when you move things around. This verb is slightly more common in the written language, but it wouldn\textquotesingle t be odd to use it in the spoken language. }

\par{ Its transitive use is sometimes represented with 移動させる. However, this is not grammatically necessary. In fact, it can be grammatically confusing because it should only be the causative form as in “(X has) Y move Z…” like in Ex. 22. }

\par{20. ${\overset{\textnormal{びょういん}}{\text{病院}}}$ の ${\overset{\textnormal{しじ}}{\text{指示}}}$ で ${\overset{\textnormal{ふくしひなんじょ}}{\text{福祉避難所}}}$ に ${\overset{\textnormal{いどう}}{\text{移動}}}$ しました。 \hfill\break
I moved to a welfare shelter under the direction of the hospital. }

\par{21. ファイルやフォルダを ${\overset{\textnormal{いどう}}{\text{移動}}}$ \{させて・して\}みましょう。 \hfill\break
Try moving files and folders. }

\par{22. ${\overset{\textnormal{せいと}}{\text{生徒}}}$ たちを ${\overset{\textnormal{たかだい}}{\text{高台}}}$ に ${\overset{\textnormal{いどう}}{\text{移動}}}$ させてください。 \hfill\break
Please move the students to high ground. }

\par{・縮小する – To reduce\slash shrink\slash curtail }

\par{ The intransitive usage is most often seen as 縮小される. Although this does imply some agent doing the action, the main reasoning for why 縮小する is not simply used is because many speakers don't register it as being both intransitive and intransitive. Its transitive usage, however, is extremely common. }

\par{23. ファイルサイズを ${\overset{\textnormal{しゅくしょう}}{\text{縮小}}}$ してください。 \hfill\break
Please shrink the file size. }

\par{24. ${\overset{\textnormal{きんゆうしさん}}{\text{金融資産}}}$ の ${\overset{\textnormal{かくさ}}{\text{格差}}}$ が ${\overset{\textnormal{しゅくしょう}}{\text{縮小}}}$ した。 \hfill\break
Financial asset disparity has shrunk. }

\par{25. ツールバーが ${\overset{\textnormal{しゅくしょう}}{\text{縮小}}}$ されてしまった。 \hfill\break
The tool bar got minimized. }

\par{・拡大する – To magnify\slash enlarge\slash amplify\slash expand }

\par{ The verb 拡大する is slightly formal but still common in both the spoken and written language. Some speakers inadvertently use 拡大される when used intransitively, but unless you wish to implicitly hint at an agent, then this is not grammatically necessary. }

\par{26. ${\overset{\textnormal{しげんかかくじょうしょう}}{\text{資源価格上昇}}}$ で ${\overset{\textnormal{ゆしゅつ}}{\text{輸出}}}$ が ${\overset{\textnormal{かくだい}}{\text{拡大}}}$ している。 \hfill\break
Exports are expanding due to rises in the price of resources. }

\par{27. ${\overset{\textnormal{がめん}}{\text{画面}}}$ が ${\overset{\textnormal{じどうてき}}{\text{自動的}}}$ に ${\overset{\textnormal{かくだい}}{\text{拡大}}}$ されます。 \hfill\break
The screen automatically enlarges. }

\par{28. ${\overset{\textnormal{かっこく}}{\text{各国}}}$ が ${\overset{\textnormal{げんざい}}{\text{現在}}}$ も ${\overset{\textnormal{へいきせいさん}}{\text{兵器生産}}}$ を ${\overset{\textnormal{かくだい}}{\text{拡大}}}$ し、 ${\overset{\textnormal{かいがい}}{\text{海外}}}$ へ ${\overset{\textnormal{ぐんたい}}{\text{軍隊}}}$ を ${\overset{\textnormal{ぞうは}}{\text{増派}}}$ している。 \hfill\break
Even now, each nation is expanding its weapon production and is sending troop reinforcements overseas. }

\par{・完成する – To complete\slash accomplish }

\par{ When something has been completely accomplished, and the result is visible for all to see, you can use the verb 完成する. It is used in both the spoken and written language, and its intransitive and transitive usages are both very common. }

\par{29. ${\overset{\textnormal{つなみひなん}}{\text{津波避難}}}$ ビルが ${\overset{\textnormal{かんせい}}{\text{完成}}}$ しました。 \hfill\break
The tsunami refuge building has been completed. }

\par{30. ${\overset{\textnormal{ぼうさい}}{\text{防災}}}$ マップを ${\overset{\textnormal{かんせい}}{\text{完成}}}$ しました。 \hfill\break
I\textquotesingle ve completed the disaster prevention map. }

\par{・完了する – To complete\slash conclude }

\par{ When you conclude a task, you can use the verb 完了する. It is rather formal and both its intransitive and transitive usages are very common. }

\par{31. ${\overset{\textnormal{しゅうせい}}{\text{修正}}}$ が ${\overset{\textnormal{かんりょう}}{\text{完了}}}$ しました。 \hfill\break
Editing has been completed. }

\par{32. ${\overset{\textnormal{とうろく}}{\text{登録}}}$ を ${\overset{\textnormal{かんりょう}}{\text{完了}}}$ しました。 \hfill\break
I\textquotesingle ve completed the registration. }

\par{・終了する – To end\slash close\slash terminate }

\par{ When something ends\slash terminates, you can use the verb 終了する. It's somewhat formal and more common in the written language. Its intransitive and transitive usages are both very common. It is important to note that this verb does not imply that a task has been thoroughly completed before ending. }

\par{33. ヘドロの ${\overset{\textnormal{しゅんせつ}}{\text{浚渫}}}$ は ${\overset{\textnormal{せんきゅうひゃくきゅうじゅう}}{\text{1990}}}$ ${\overset{\textnormal{ねん}}{\text{年}}}$ に ${\overset{\textnormal{しゅうりょう}}{\text{終了}}}$ し、 ${\overset{\textnormal{ゆた}}{\text{豊}}}$ かな ${\overset{\textnormal{うみ}}{\text{海}}}$ が ${\overset{\textnormal{さいせい}}{\text{再生}}}$ した。 \hfill\break
The sludge dredging was terminated in 1990, and the rich sea restored itself. }

\par{34. ${\overset{\textnormal{ぼしゅう}}{\text{募集}}}$ を ${\overset{\textnormal{しゅうりょう}}{\text{終了}}}$ しました。 \hfill\break
We\textquotesingle ve ended recruiting\slash taking applications\slash raising (donations). }

\par{・実現する – To implement\slash materialize\slash realize }

\par{ The intransitive use of 実現する is the primary usage of this verb. As a transitive verb, many speakers are compelled to use 実現させる instead. This doesn't always necessarily mean the causative nuance of "to make\slash let someone…" is literally intended, but it will always imply a more direct involvement of the agent to make something happen. }

\par{35. ${\overset{\textnormal{きんべん}}{\text{勤勉}}}$ に ${\overset{\textnormal{はたら}}{\text{働}}}$ き、 ${\overset{\textnormal{しんぼう}}{\text{辛抱}}}$ すれば ${\overset{\textnormal{ゆめ}}{\text{夢}}}$ は ${\overset{\textnormal{じつげん}}{\text{実現}}}$ するでしょう。 \hfill\break
If you work diligently and persevere, your dreams will surely be realized. }

\par{36. ${\overset{\textnormal{じぞくてき}}{\text{持続的}}}$ な ${\overset{\textnormal{けいざいせいちょう}}{\text{経済成長}}}$ を ${\overset{\textnormal{じつげん}}{\text{実現}}}$ するためには、 ${\overset{\textnormal{げんざい}}{\text{現在}}}$ よりも ${\overset{\textnormal{だいたん}}{\text{大胆}}}$ な ${\overset{\textnormal{かわせ}}{\text{為替}}}$ ・ ${\overset{\textnormal{きんゆうかんわせいさく}}{\text{金融緩和政策}}}$ に ${\overset{\textnormal{くわ}}{\text{加}}}$ えて、 ${\overset{\textnormal{こよう}}{\text{雇用}}}$ の ${\overset{\textnormal{かくだい}}{\text{拡大}}}$ 、 ${\overset{\textnormal{ちんぎん}}{\text{賃金}}}$ の ${\overset{\textnormal{ひ}}{\text{引}}}$ き ${\overset{\textnormal{あ}}{\text{上}}}$ げなど ${\overset{\textnormal{しょうひかくだい}}{\text{消費拡大}}}$ に ${\overset{\textnormal{つな}}{\text{繋}}}$ がる ${\overset{\textnormal{せいさく}}{\text{政策}}}$ を ${\overset{\textnormal{すす}}{\text{進}}}$ める ${\overset{\textnormal{ひつよう}}{\text{必要}}}$ があります。 \hfill\break
In order to implement sustainable economic growth, in addition to far more audacious exchange and finance easing policies than now, we must forward policies that are linked to consumption expansion by such means as expanding employment, raising wages, etc. }

\par{37. ${\overset{\textnormal{いごこち}}{\text{居心地}}}$ の ${\overset{\textnormal{よ}}{\text{良}}}$ さを ${\overset{\textnormal{じつげん}}{\text{実現}}}$ \{した・させた\}モダンな ${\overset{\textnormal{じゅうたく}}{\text{住宅}}}$ です。 \hfill\break
This is a modern home that realizes coziness. }

\par{38. 息子が夢を実現\{しました・させました\}。 \hfill\break
My son has realized his dreams. }

\par{・転換する – To convert\slash divert\slash changeover\slash switch-over }

\par{ The verb 転換する is generally used to indicate changes into tendency\slash directives. So, even though the fundamentals of the matter may not change, the direction of said entity might. This word is appropriate in both the spoken and the written language. Its intransitive usage is most common. As an intransitive verb, the form 転換させる is preferred, especially when emphasis is placed on the agent. }

\par{39. ${\overset{\textnormal{かこう}}{\text{下降}}}$ トレンドが ${\overset{\textnormal{うわむ}}{\text{上向}}}$ きに ${\overset{\textnormal{てんかん}}{\text{転換}}}$ しました。 \hfill\break
The downward trend has switched upward. }

\par{40. ${\overset{\textnormal{にちぎん}}{\text{日銀}}}$ は、 ${\overset{\textnormal{きんゆうせいさく}}{\text{金融政策}}}$ を ${\overset{\textnormal{てんかん}}{\text{転換}}}$ しました。 \hfill\break
The Bank of Japan has shifted its finance policies. }

\par{41. ニクソン ${\overset{\textnormal{だいとうりょう}}{\text{大統領}}}$ は、それまでの ${\overset{\textnormal{れいせんこうぞう}}{\text{冷戦構造}}}$ を ${\overset{\textnormal{てんかん}}{\text{転換}}}$ させました。 \hfill\break
President Nixon had changed-over the structuring of the Cold War up to that time. }

\par{\emph{・ }変換する – To change\slash convert\slash transform }

\par{ The verb 変換する is a somewhat technical verb that indicates switching out\slash converting something from one thing to another. However, it cannot refer to religious conversion. That would be handled by the verb 改宗する. Its intransitive usage is rare, so much so that most speakers replace it with 変換される. Although this grammatically implicitly hints at an agent, this is not usually meant by the speaker. Rather, using the “passive form” is a means of lexicalizing a transitive verb in an intransitive means. }

\par{42. ${\overset{\textnormal{いろいろ}}{\text{色々}}}$ と ${\overset{\textnormal{そうさ}}{\text{操作}}}$ しているうちにワードの ${\overset{\textnormal{もじ}}{\text{文字}}}$ が ${\overset{\textnormal{へんかん}}{\text{変換}}}$ \{されました・しました\}。 \hfill\break
While doing all sorts of operations, the characters in Word (got) converted. \hfill\break
 \hfill\break
43. ${\overset{\textnormal{どうが}}{\text{動画}}}$ をPSP ${\overset{\textnormal{よう}}{\text{用}}}$ に ${\overset{\textnormal{へんかん}}{\text{変換}}}$ しました。 \hfill\break
I converted the video to be for the PSP. }

\par{・集中する – To concentrate\slash converge\slash centralize }

\par{ The intransitive usage of this verb is not so common and more so stilted for the written language; however, its transitive usage is very common in both the spoken and written languages. }

\par{43. ${\overset{\textnormal{せいしん}}{\text{精神}}}$ を ${\overset{\textnormal{しゅうちゅう}}{\text{集中}}}$ して ${\overset{\textnormal{どりょく}}{\text{努力}}}$ すればどんなことでも ${\overset{\textnormal{な}}{\text{成}}}$ し ${\overset{\textnormal{と}}{\text{遂}}}$ げられないことはない。 \hfill\break
If you concentrate your mind and exert yourself, there isn\textquotesingle t anything that you cannot accomplish. }

\par{44. ${\overset{\textnormal{げんざい}}{\text{現在}}}$ アクセスが ${\overset{\textnormal{しゅうちゅう}}{\text{集中}}}$ しているため、 ${\overset{\textnormal{とうこう}}{\text{投稿}}}$ ができません。 \hfill\break
Unable to post due to a current heavy traffic spike. }

\par{45. ○○ ${\overset{\textnormal{かいちょう}}{\text{会長}}}$ に ${\overset{\textnormal{ぎいん}}{\text{議員}}}$ らの ${\overset{\textnormal{しつもん}}{\text{質問}}}$ が ${\overset{\textnormal{しゅうちゅう}}{\text{集中}}}$ した。 \hfill\break
Questions from the assemblymen converged on Chairman \#\#. }

\par{・減少する – To decrease\slash decline\slash reduce }

\par{ This is a literary verb that is frequently also used in news reports. Its usually always used as an intransitive verb. In fact, even though its transitive usage is grammatically correct, it\textquotesingle s unnatural to the majority of speakers nowadays. If you\textquotesingle re compelled to use this verb in a transitive manner, the form 減少される is more natural, this is despite the fact that this could also mean “to make X decrease\slash reduce Y.” }

\par{46. メイン ${\overset{\textnormal{わん}}{\text{湾}}}$ ではタラの ${\overset{\textnormal{ぎょかくりょう}}{\text{漁獲量}}}$ が ${\overset{\textnormal{げんしょう}}{\text{減少}}}$ している。 \hfill\break
In the Gulf of Main, cod hauls are declining. }

\par{\textbf{Spelling Note }: \emph{タラ }may alternatively be spelled as 鱈. }

\par{47. ${\overset{\textnormal{たいじゅう}}{\text{体重}}}$ を\{ ${\overset{\textnormal{へ}}{\text{減}}}$ らす・ ${\overset{\textnormal{げんしょう}}{\text{減少}}}$ させる・△ ${\overset{\textnormal{げんしょう}}{\text{減少}}}$ する\}には、 ${\overset{\textnormal{げんかい}}{\text{限界}}}$ がある。 \hfill\break
There is a limit to reducing weight. }

\par{・増加する – To increase }

\par{ This verb is more common as an intransitive verb. When used as a transitive verb, some speakers opt to use 増加させる even though that can technically also be used as the verb\textquotesingle s causative form. It is literary and is frequently used in news reports. It is “to increase” as in making the quantity of something larger. }

\par{48. ${\overset{\textnormal{けつえきけんさ}}{\text{血液検査}}}$ の ${\overset{\textnormal{まえ}}{\text{前}}}$ に ${\overset{\textnormal{うんどう}}{\text{運動}}}$ することも ${\overset{\textnormal{はっけっきゅう}}{\text{白血球}}}$ が ${\overset{\textnormal{ぞうか}}{\text{増加}}}$ する ${\overset{\textnormal{げんいん}}{\text{原因}}}$ の ${\overset{\textnormal{ひと}}{\text{一}}}$ つです。 \hfill\break
Exercising before blood work is also one reason for a rise in white blood cell count. }

\par{49. この ${\overset{\textnormal{かいろ}}{\text{回路}}}$ の ${\overset{\textnormal{でんりゅうきょうきゅうりょう}}{\text{電流供給量}}}$ を ${\overset{\textnormal{ぞうか}}{\text{増加}}}$ \{する・させる\}ことができます。 \hfill\break
It is possible to increase the current supply of this circuit. }

\par{・増殖する – To increase\slash propagate }

\par{ This verb is typically used to mean “to propagate” as in organic matter. This could be procreation or the proliferation of cells. It may also refer to the increase of resources, especially assets, but this is not near as common. Although both its intransitive and transitive usages are common, as a transitive verb, it is often seen as 増殖させる. Because it is largely used in the realm of biology, the causative sense of making cells propagate, for instance, is very natural. }

\par{50. ${\overset{\textnormal{かんじゃじしん}}{\text{患者自身}}}$ の ${\overset{\textnormal{さいぼう}}{\text{細胞}}}$ を ${\overset{\textnormal{ぞうしょく}}{\text{増殖}}}$ \{して・させて\} ${\overset{\textnormal{いしょく}}{\text{移植}}}$ するという「 ${\overset{\textnormal{さいせいいりょう}}{\text{再生医療}}}$ 」が ${\overset{\textnormal{すす}}{\text{進}}}$ んでいる。 \hfill\break
“Regenerative medicine,” in which one propagates the cells of the patient himself and then transplant (said cells back into the patient), is advancing. }

\par{50. ${\overset{\textnormal{がんさいぼう}}{\text{癌細胞}}}$ は、 ${\overset{\textnormal{じょじょ}}{\text{徐々}}}$ に ${\overset{\textnormal{ぞうしょく}}{\text{増殖}}}$ し、 ${\overset{\textnormal{た}}{\text{他}}}$ の ${\overset{\textnormal{そしき}}{\text{組織}}}$ や ${\overset{\textnormal{ぞうき}}{\text{臓器}}}$ に ${\overset{\textnormal{いてん}}{\text{移転}}}$ してしまうのです。 \hfill\break
Cancer cells gradually propagate and then end up moving to other tissues and organs. }

\par{\textbf{Transitivity Note }: 移転する is another example and grammatically functions just like 移動する. 移転する can refer moving of placement\slash location or the transfer of legal rights whereas 移動する simply refers to the movement from one place to another. }

\par{・増大する – To enlarge\slash increase }

\par{ This is a literary verb largely used in an intransitive sense that refers to the increase in degree, not quantity. When used as a transitive verb, if the agent has direct involvement in the action, 増大させる is preferred. }

\par{51. ${\overset{\textnormal{いやくひん}}{\text{医薬品}}}$ の ${\overset{\textnormal{ししゅつ}}{\text{支出}}}$ が ${\overset{\textnormal{ぞうだい}}{\text{増大}}}$ している。 \hfill\break
Medical supply expenditures are increasing. }

\par{52. ${\overset{\textnormal{べいこくせいふ}}{\text{米国政府}}}$ はまたも ${\overset{\textnormal{ぼうえいよさん}}{\text{防衛予算}}}$ を ${\overset{\textnormal{ぞうだい}}{\text{増大}}}$ させることを ${\overset{\textnormal{はっぴょう}}{\text{発表}}}$ した。 \hfill\break
The U.S. government has again announced that they are to increase the defense budget. }

\par{53. ${\overset{\textnormal{おお}}{\text{多}}}$ くの ${\overset{\textnormal{くに}}{\text{国}}}$ が ${\overset{\textnormal{かがくぎじゅつ}}{\text{科学技術}}}$ (の) ${\overset{\textnormal{よさん}}{\text{予算}}}$ を ${\overset{\textnormal{ぞうだい}}{\text{増大}}}$ している。 \hfill\break
Many countries are increasing their science and technology budgets. }

\par{・固定する – fixate\slash fix }

\par{ This verb is common in both the spoken and written language. Its transitive usage is more common. When used as an intransitive verb, it is frequently seen as 固定される. This is less likely when referring to a fixed state in which no exertion was used to make it so. }

\par{54. レベルが ${\overset{\textnormal{あ}}{\text{上}}}$ がらず ${\overset{\textnormal{もと}}{\text{元}}}$ のレベルに ${\overset{\textnormal{こてい}}{\text{固定}}}$ \{して・されて\}しまうことがあります。 \hfill\break
There are times in which one level doesn\textquotesingle t go up and one is fixed to one\textquotesingle s original level. }

\par{55. ${\overset{\textnormal{たないた}}{\text{棚板}}}$ と ${\overset{\textnormal{しちゅう}}{\text{支柱}}}$ を ${\overset{\textnormal{こてい}}{\text{固定}}}$ しているネジを ${\overset{\textnormal{はず}}{\text{外}}}$ します。 \hfill\break
Remove the screws that fixate the shelf boards and props together. }

\par{\textbf{Spelling Note }: ネジ \emph{ }may alternatively be spelled as ねじ, 捩子, 螺子, 螺旋, or 捻子. }

\par{・再生する – To resuscitate\slash playback\slash etc. }

\par{ As an intransitive verb, 再生する typically refers to something restoring back to life. This can be used in a figurative sense. It may also refer to reformation of a person as well. It may also be used to refer to regeneration. This usage can be both intransitive and transitive. As a transitive verb, it can also mean to “play (back)” as in video footage. When its meanings revolving regeneration, which includes playing back sound, is used in an intransitive fashion, it\textquotesingle s typically seen as 再生される. Lastly, this verb is used in both the spoken and the written language. }

\par{56. パワーポイントで ${\overset{\textnormal{どうが}}{\text{動画}}}$ を ${\overset{\textnormal{さいせい}}{\text{再生}}}$ したいです。 \hfill\break
I want to play a video on PowerPoint. }

\par{57. Wi-Fi ${\overset{\textnormal{かんきょう}}{\text{環境}}}$ が ${\overset{\textnormal{ふあんてい}}{\text{不安定}}}$ な ${\overset{\textnormal{ばあい}}{\text{場合}}}$ 、 ${\overset{\textnormal{えいぞう}}{\text{映像}}}$ が ${\overset{\textnormal{さいせい}}{\text{再生}}}$ されないことがあります。 \hfill\break
Whenever your Wi-Fi environment is unstable, footage may not play. }

\par{58. ${\overset{\textnormal{さいせいいでんし}}{\text{再生遺伝子}}}$ を ${\overset{\textnormal{きどう}}{\text{起動}}}$ させる ${\overset{\textnormal{いんし}}{\text{因子}}}$ を ${\overset{\textnormal{はっけん}}{\text{発見}}}$ し、それを ${\overset{\textnormal{まうす}}{\text{マウス}}}$ に ${\overset{\textnormal{いしょく}}{\text{移植}}}$ されたところ、 ${\overset{\textnormal{まうす}}{\text{マウス}}}$ の ${\overset{\textnormal{そしき}}{\text{組織}}}$ が ${\overset{\textnormal{さいせい}}{\text{再生}}}$ されたそうです。 \hfill\break
I hear that they\textquotesingle ve discovered the factor that activates the gene for regeneration, and upon having it transplanting in a mouse, the mouse\textquotesingle s tissue was regenerated. }

\par{59. ${\overset{\textnormal{うしな}}{\text{失}}}$ った ${\overset{\textnormal{ぶい}}{\text{部位}}}$ を ${\overset{\textnormal{さいせい}}{\text{再生}}}$ する ${\overset{\textnormal{いでんし}}{\text{遺伝子}}}$ が ${\overset{\textnormal{そんざい}}{\text{存在}}}$ する。 \hfill\break
Genes for regeneration lost body parts exist. }

\par{60. カニの ${\overset{\textnormal{あし}}{\text{足}}}$ は ${\overset{\textnormal{なんど}}{\text{何度}}}$ も ${\overset{\textnormal{さいせい}}{\text{再生}}}$ するって ${\overset{\textnormal{ほんとう}}{\text{本当}}}$ ですか。 \hfill\break
Is it really true that crab legs regenerate many times over? }

\par{\textbf{Spelling Note }: カニ may also be spelled as 蟹. }

\par{・開始する – To begin\slash start }

\par{ This is the literary version of 始まる and 始める. It is more formal and used extensively in news reports. Some speakers use 開始される instead when used in the intransitive sense. Although this technically implicitly hints at the agent, this is not always the case. }

\par{61. ${\overset{\textnormal{へいせい}}{\text{平成}}}$ ${\overset{\textnormal{にじゅうはち}}{\text{28}}}$ ${\overset{\textnormal{ねん}}{\text{年}}}$ ${\overset{\textnormal{いち}}{\text{1}}}$ ${\overset{\textnormal{がつ}}{\text{月}}}$ から、マイナンバー ${\overset{\textnormal{せいど}}{\text{制度}}}$ が ${\overset{\textnormal{かいし}}{\text{開始}}}$ しました。 \hfill\break
The “My Number” system started in January 2016. }

\par{62. これらは全て来月から開始されます。 \hfill\break
These will all be started next month. }

\par{63. ${\overset{\textnormal{ご}}{\text{5}}}$ ${\overset{\textnormal{がつ}}{\text{月}}}$ より ${\overset{\textnormal{ひふか}}{\text{皮膚科}}}$ の ${\overset{\textnormal{しんさつ}}{\text{診察}}}$ を ${\overset{\textnormal{かいし}}{\text{開始}}}$ します。 \hfill\break
We will begin dermatology examinations starting in May. }

\par{・反転する – To Roll Over\slash Turn Around }

\par{ This verb means “to roll over\slash turn around” and is appropriate in both the written and the spoken language. When used transitively, some people prefer to use 反転させる, but this is also the verb\textquotesingle s causative form. Sometimes, using this incidentally personifies non-living agents like in Ex. 64. }

\par{64. ${\overset{\textnormal{つよ}}{\text{強}}}$ い ${\overset{\textnormal{たいふう}}{\text{台風}}}$ ${\overset{\textnormal{まる}}{\text{○}}}$ ${\overset{\textnormal{ごう}}{\text{号}}}$ は ${\overset{\textnormal{じゅうご}}{\text{15}}}$ ${\overset{\textnormal{にち}}{\text{日}}}$ 、 ${\overset{\textnormal{ぼうふういき}}{\text{暴風域}}}$ を ${\overset{\textnormal{ともな}}{\text{伴}}}$ いながら、 ${\overset{\textnormal{おきなわ}}{\text{沖縄}}}$ ・ ${\overset{\textnormal{みなみだいとうじま}}{\text{南大東島}}}$ の ${\overset{\textnormal{みなみ}}{\text{南}}}$ で ${\overset{\textnormal{しんろ}}{\text{進路}}}$ を ${\overset{\textnormal{ほくとう}}{\text{北東}}}$ に ${\overset{\textnormal{はんてん}}{\text{反転}}}$ \{した・させた\}。 \hfill\break
On the fifteenth, the strong Typhoon \#? turned its course around northeast to the south of Minamidaito Island of Okinawa along with its storm area. }

\par{ 65. アメリカに ${\overset{\textnormal{す}}{\text{住}}}$ むアフリカ ${\overset{\textnormal{けい}}{\text{系}}}$ アメリカ ${\overset{\textnormal{じん}}{\text{人}}}$ ${\overset{\textnormal{ごじゅう}}{\text{50}}}$ ${\overset{\textnormal{にん}}{\text{人}}}$ と、 ${\overset{\textnormal{みなみ}}{\text{南}}}$ アフリカ ${\overset{\textnormal{じん}}{\text{人}}}$ ${\overset{\textnormal{ごじゅう}}{\text{50}}}$ ${\overset{\textnormal{にん}}{\text{人}}}$ の ${\overset{\textnormal{しょくせいかつ}}{\text{食生活}}}$ をそっくり ${\overset{\textnormal{い}}{\text{入}}}$ れ ${\overset{\textnormal{か}}{\text{替}}}$ えてみたところ、 ${\overset{\textnormal{に}}{\text{2}}}$ ${\overset{\textnormal{しゅうかん}}{\text{週間}}}$ で ${\overset{\textnormal{ちょうない}}{\text{腸内}}}$ フローラが ${\overset{\textnormal{はんてん}}{\text{反転}}}$ し、アメリカ ${\overset{\textnormal{じん}}{\text{人}}}$ の ${\overset{\textnormal{ちょうない}}{\text{腸内}}}$ フローラは ${\overset{\textnormal{みなみ}}{\text{南}}}$ アフリカ ${\overset{\textnormal{じん}}{\text{人}}}$ と ${\overset{\textnormal{おな}}{\text{同}}}$ じ ${\overset{\textnormal{とくちょう}}{\text{特徴}}}$ を ${\overset{\textnormal{あらわ}}{\text{表}}}$ すようになり、 ${\overset{\textnormal{みなみ}}{\text{南}}}$ アフリカ ${\overset{\textnormal{じん}}{\text{人}}}$ の ${\overset{\textnormal{ちょうない}}{\text{腸内}}}$ は ${\overset{\textnormal{だいちょうがん}}{\text{大腸癌}}}$ の ${\overset{\textnormal{はっしょう}}{\text{発症}}}$ リスクが ${\overset{\textnormal{たか}}{\text{高}}}$ くなることが ${\overset{\textnormal{わ}}{\text{分}}}$ かりました。 \hfill\break
After precisely switching the dietary habits of 50 Americans of African descent living in America and 50 South Africans with each other, it was discovered that their intestinal florae (of the two groups) reversed in two weeks; the intestinal florae of the Americans expressed the same characteristics as the South Africans, and the risk of developing colorectal cancer in the intestines of the South Africans rose. }

\par{・停止する – To Suspend }

\par{ This is a slightly more literary version of 止まる and 止める meaning "to halt\slash cease\slash suspend\slash interrupt\slash ban." The suspension\slash hang-up in question is not necessarily permanent. }

\par{66. 移動が停止してしまい、ミスに繋がることがありました。 \hfill\break
Movement would halt, which sometimes led to mistakes. }

\par{67. 東京都は、卸売業者3社に対して、最大20日間、業務を停止するよう命じました。 \hfill\break
Tokyo has ordered the suspension of operations for a maximum of twenty days to three wholesalers.  }
    