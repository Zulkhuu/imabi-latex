    
\chapter{他受動詞}

\begin{center}
\begin{Large}
第229課: 他受動詞 
\end{Large}
\end{center}
 
\par{ If there is such thing as a 自受動詞, then 他受動詞 ought to exist as well. Unsurprisingly, they do. The only difference between the two is that 他受動詞 \emph{ }have a minimum requirement of three arguments. }

\par{i. [I] am being taught [karate] by [my instructor.] \hfill\break
ii. [I] am borrowing [this pencil] from [my friend]. \hfill\break
iii. [I] was entrusted with [the care of several animals] from [my neighbors]. }

\par{ Although Japanese has a high tendency of dropping things that are apparent in context such as first person and other elements of the sentence, with verbs that are 他受動詞, the three arguments of “subject,” “direct object,” and “indirect object” are always implied because they are absolutely required in the same sense that they are with their English equivalents in the examples above. \hfill\break
 \hfill\break
 As you should have already gathered, the Japanese equivalents of the verbs in the examples above along with other verbs like them make up the topic of this lesson. The verbs to be looked at are categorized neatly below. As you can see, comparing them to their passive transitive counterparts will be the primary goal of this lesson. The dichotomy for 他受動詞 behaves the same as with 自受動詞. That means if you figured out the nuances discussed in the previous lesson, then this lesson should be a breeze. }

\begin{center}
\textbf{Movement of Information }
\end{center}

\begin{center}
教わる vs 教えられる 
\end{center}

\begin{center}
\textbf{Movement of Something }
\end{center}

\begin{center}
授かる vs 授けられる 
\end{center}

\begin{center}
預かる vs 預けられる 
\end{center}

\begin{center}
借りる vs 貸される 
\end{center}

\begin{center}
\textbf{Carrying out Orders }
\end{center}

\begin{center}
言い付かる vs 言い付けられる 
\end{center}

\begin{center}
言付かる vs 言付けられる 
\end{center}

\begin{center}
\textbf{Being Caught Doing Something }
\end{center}

\begin{center}
見つかる vs 見つけられる 
\end{center}

\begin{center}
捕まる vs 捕まえられる 
\end{center}
      
\section{Movement of Information}
 
\begin{center}
\textbf{教わる vs 教えられる }
\end{center}

\par{ Contrary to expectations, 教えられる is not that frequently used in the passive form. This is because its uses in light honorifics and as the potential form of 教える are more common than its passive meaning. When the passive meaning is employed, it is usually juxtaposed with 教える in some way, or the statement itself is personally indifferent. }

\par{1a. 私は李先生\{に・から\}韓国語会話を教わっています。 \hfill\break
I\textquotesingle m learning Korean conversation from Mr. Lee. \hfill\break
1b. 私は李先生から韓国語会話を教えられていますよ(△)。 \hfill\break
I\textquotesingle m being taught Korean conversation from Mr. Lee. \hfill\break
1c. 私は李先生\{に・から\}韓国語会話を教えてもらっています(よ)。 \hfill\break
I\textquotesingle m having Mr. Lee teach me Korean conversation. }

\par{\textbf{Grammar Note }: When using the passive form as shown in 1b, the personal nature of the statement in combination with the relative lack of use of this form for such a personal comment requires that the particle から be used in order to make the sentence passable as natural. It's also important to note that when there is a clear person stated as to who is teaching you, 教えてもらう is used more than 教わる. This is so the subjective nature of the flow of learning is emphasized since 教わる \emph{ }naturally lacks a subjective feel to it because it is a 他受動詞. }

\par{2. ${\overset{\textnormal{おんしつ}}{\text{温室}}}$ での ${\overset{\textnormal{さいばい}}{\text{栽培}}}$ のやり ${\overset{\textnormal{かた}}{\text{方}}}$ を ${\overset{\textnormal{おそ}}{\text{教}}}$ わっています。 \hfill\break
I\textquotesingle m being taught how to cultivate in a greenhouse. }

\par{3. ネイティブから ${\overset{\textnormal{えいご}}{\text{英語}}}$ を ${\overset{\textnormal{きょう}}{\text{教}}}$ わらないほうがいいかもしれない。 \hfill\break
It might not be best to learn English from a native. }

\par{4. ${\overset{\textnormal{せんぱい}}{\text{先輩}}}$ から ${\overset{\textnormal{おそ}}{\text{教}}}$ わった ${\overset{\textnormal{こと}}{\text{事}}}$ がなかなか ${\overset{\textnormal{おぼ}}{\text{覚}}}$ えられません。 \hfill\break
I can\textquotesingle t seem to remember what I learned from my senpai. }

\par{5. ${\overset{\textnormal{こうしゅう}}{\text{広州}}}$ で、 ${\overset{\textnormal{ぎょうざ}}{\text{餃子}}}$ と ${\overset{\textnormal{めん}}{\text{麺}}}$ の ${\overset{\textnormal{つく}}{\text{作}}}$ り ${\overset{\textnormal{かた}}{\text{方}}}$ を ${\overset{\textnormal{おそ}}{\text{教}}}$ わりました。 \hfill\break
I learned how to make gyoza and noodles in Guangzhou. }

\par{6. ${\overset{\textnormal{かていきょうし}}{\text{家庭教師}}}$ の ${\overset{\textnormal{ほう}}{\text{方}}}$ に ${\overset{\textnormal{えいご}}{\text{英語}}}$ を ${\overset{\textnormal{おそ}}{\text{教}}}$ わっていました。 \hfill\break
I had been learning English from a private teacher. }

\par{7. ${\overset{\textnormal{かれ}}{\text{彼}}}$ は ${\overset{\textnormal{まちが}}{\text{間違}}}$ った ${\overset{\textnormal{れきし}}{\text{歴史}}}$ を ${\overset{\textnormal{おし}}{\text{教}}}$ えられているという ${\overset{\textnormal{にんしき}}{\text{認識}}}$ がない。 \hfill\break
He has no knowledge that he is being taught incorrect history. }

\par{8. ${\overset{\textnormal{おし}}{\text{教}}}$ える ${\overset{\textnormal{がわ}}{\text{側}}}$ と ${\overset{\textnormal{おし}}{\text{教}}}$ えられる ${\overset{\textnormal{がわ}}{\text{側}}}$ は、どちらの ${\overset{\textnormal{たちば}}{\text{立場}}}$ が ${\overset{\textnormal{うえ}}{\text{上}}}$ ですか。 \hfill\break
Which standpoint is higher in rank, ‘teaching\textquotesingle  or ‘being taught\textquotesingle ? }

\par{9. ${\overset{\textnormal{お}}{\text{負}}}$ うた ${\overset{\textnormal{こ}}{\text{子}}}$ に ${\overset{\textnormal{おし}}{\text{教}}}$ えられて ${\overset{\textnormal{あさせ}}{\text{浅瀬}}}$ を ${\overset{\textnormal{わた}}{\text{渡}}}$ る。 \hfill\break
A fool may give a wise man counsel. }

\par{10. ${\overset{\textnormal{おし}}{\text{教}}}$ えたり ${\overset{\textnormal{おし}}{\text{教}}}$ えられたりするときに ${\overset{\textnormal{たいせつ}}{\text{大切}}}$ なのは ${\overset{\textnormal{りかい}}{\text{理解}}}$ することです。 \hfill\break
What\textquotesingle s important when teaching or being taught is understanding. }

\par{11. ${\overset{\textnormal{かいわ}}{\text{会話}}}$ を ${\overset{\textnormal{おし}}{\text{教}}}$ えられても、 ${\overset{\textnormal{つづ}}{\text{綴}}}$ りをきちんと ${\overset{\textnormal{か}}{\text{書}}}$ けなければならない。 \hfill\break
Even if you\textquotesingle re taught conversation, you still have to be able to properly spell. }

\par{12. ${\overset{\textnormal{しごと}}{\text{仕事}}}$ で ${\overset{\textnormal{あき}}{\text{明}}}$ らかに ${\overset{\textnormal{まちが}}{\text{間違}}}$ いを ${\overset{\textnormal{おし}}{\text{教}}}$ えられていると ${\overset{\textnormal{おも}}{\text{思}}}$ う(ん)なら、きちんと ${\overset{\textnormal{しごと}}{\text{仕事}}}$ をして ${\overset{\textnormal{みかえ}}{\text{見返}}}$ してやってはどうですか。 \hfill\break
If you really think that you are being taught something clearly wrong at work, why not get back at the person by doing the job properly? }

\par{13. ${\overset{\textnormal{きょうし}}{\text{教師}}}$ から ${\overset{\textnormal{まちが}}{\text{間違}}}$ いを ${\overset{\textnormal{おし}}{\text{教}}}$ えられました。 \hfill\break
I was informed of my mistake by a teacher. }

\par{14. ${\overset{\textnormal{どうとく}}{\text{道徳}}}$ は ${\overset{\textnormal{おし}}{\text{教}}}$ えられるのか。 \hfill\break
Can morals be taught? }

\par{15. ${\overset{\textnormal{せんせい}}{\text{先生}}}$ は ${\overset{\textnormal{なに}}{\text{何}}}$ を ${\overset{\textnormal{おし}}{\text{教}}}$ えられているんですか。 \hfill\break
Teacher, what is it that you teach? }
      
\section{Movement of Something}
 
\begin{center}
\textbf{授かる vs 授けられる }
\end{center}

\par{ The verbs 授かる and 授けられる have the following meanings. }

\begin{itemize}

\item 授かる: To be awarded; to receive (a title); to be blessed with (a child). \hfill\break

\item 授ける: To grant; to award; to teach (a secret). 
\end{itemize}

\par{ Together, they create an intransitive-transitive verb pair even though both utilize the particle を. Both words are quite literary and are rarely employed in the spoken language. As for 授けられる, it is not used nearly as often as 授かる, but when it is, the context is usually religious and\slash or text translated from European languages. }

\par{\emph{ }授かる being more objective and 授けられる being more subjective is certainly applicable here, but as you will see in the examples, 授かる tends to be more natural; however, there are many instances in which another verb altogether is more common. Even so, one nuance that 授かる and \emph{ }授ける would share that another verb wouldn\textquotesingle t necessarily have is emphasizing how what is being bestowed is something worth more than what can be bought by money. Essentially, price tags aren\textquotesingle t placed on what is being given. }

\par{16. ${\overset{\textnormal{じょおうへいか}}{\text{女王陛下}}}$ から ${\overset{\textnormal{しゃくい}}{\text{爵位}}}$ を\{ ${\overset{\textnormal{あた}}{\text{与}}}$ えられる・ ${\overset{\textnormal{さず}}{\text{授}}}$ かる・ ${\overset{\textnormal{さず}}{\text{授}}}$ けられる\}。 \hfill\break
To be awarded a court rank from Her Majesty. }

\par{17. ${\overset{\textnormal{でし}}{\text{弟子}}}$ たちは、 ${\overset{\textnormal{かみさま}}{\text{神様}}}$ \{に・から\} ${\overset{\textnormal{さず}}{\text{授}}}$ けられた ${\overset{\textnormal{けんい}}{\text{権威}}}$ を ${\overset{\textnormal{も}}{\text{持}}}$ って ${\overset{\textnormal{あくりょう}}{\text{悪霊}}}$ を ${\overset{\textnormal{お}}{\text{追}}}$ い ${\overset{\textnormal{だ}}{\text{出}}}$ し、 ${\overset{\textnormal{やまい}}{\text{病}}}$ を ${\overset{\textnormal{いや}}{\text{癒}}}$ し、 ${\overset{\textnormal{かみ}}{\text{神}}}$ の ${\overset{\textnormal{くに}}{\text{国}}}$ を ${\overset{\textnormal{の}}{\text{宣}}}$ べ ${\overset{\textnormal{つた}}{\text{伝}}}$ えるために ${\overset{\textnormal{つか}}{\text{遣}}}$ わされた。 \hfill\break
The disciples, with power granted to them [by\slash from] God, were sent to cast out evil spirits, heal disease, and proclaim God\textquotesingle s kingdom. }

\par{18. ${\overset{\textnormal{かいみょう}}{\text{戒名}}}$ とは、 ${\overset{\textnormal{もともと}}{\text{元々}}}$ は ${\overset{\textnormal{ぶっきょうと}}{\text{仏教徒}}}$ となったときに ${\overset{\textnormal{さず}}{\text{授}}}$ けられる ${\overset{\textnormal{なまえ}}{\text{名前}}}$ で、 ${\overset{\textnormal{ほんらい}}{\text{本来}}}$ は ${\overset{\textnormal{せいぜん}}{\text{生前}}}$ に ${\overset{\textnormal{さず}}{\text{授}}}$ かるものです。 \hfill\break
A (posthumous) Buddhist name, was originally a name granted to someone once he became a Buddhist, and so originally, it was something awarded during one\textquotesingle s lifetime. }

\par{19. ${\overset{\textnormal{かいみょう}}{\text{戒名}}}$ は ${\overset{\textnormal{かなら}}{\text{必}}}$ ずしも ${\overset{\textnormal{そうりょ}}{\text{僧侶}}}$ から\{ ${\overset{\textnormal{さず}}{\text{授}}}$ かる・ ${\overset{\textnormal{さず}}{\text{授}}}$ けられる\}ものではない。 \hfill\break
A posthumous Buddhist name is not something you are granted by a monk. }

\par{20. ${\overset{\textnormal{にんげん}}{\text{人間}}}$ は ${\overset{\textnormal{りせい}}{\text{理性}}}$ と ${\overset{\textnormal{りょうしん}}{\text{良心}}}$ を ${\overset{\textnormal{\{}}{\text{\{}}}$ ${\overset{\textnormal{あた}}{\text{与}}}$ えられている・ ${\overset{\textnormal{さず}}{\text{授}}}$ けられている\} 。 \hfill\break
Humans are given reasoning power and a conscience. }

\par{21. ロシア ${\overset{\textnormal{こうてい}}{\text{皇帝}}}$ から ${\overset{\textnormal{くんしょう}}{\text{勲章}}}$ を\{ ${\overset{\textnormal{う}}{\text{受}}}$ ける・ ${\overset{\textnormal{さず}}{\text{授}}}$ かる・ ${\overset{\textnormal{あた}}{\text{与}}}$ えられる・ ${\overset{\textnormal{さず}}{\text{授}}}$ けられる\}。 \hfill\break
To [receive\slash be granted] a medal from the Russian emperor. }

\par{22. ${\overset{\textnormal{しといがい}}{\text{使徒以外}}}$ に ${\overset{\textnormal{かみ}}{\text{神}}}$ の ${\overset{\textnormal{くに}}{\text{国}}}$ の ${\overset{\textnormal{おうぎ}}{\text{奥義}}}$ を ${\overset{\textnormal{さず}}{\text{授}}}$ けられた ${\overset{\textnormal{ひと}}{\text{人}}}$ はいないのです。 \hfill\break
Aside from the disciples, no man has been bestowed the secrets of God\textquotesingle s kingdom. }

\par{23. この ${\overset{\textnormal{しょうごう}}{\text{称号}}}$ は ${\overset{\textnormal{おうけ}}{\text{王家}}}$ の ${\overset{\textnormal{にんしょうじょう}}{\text{認証状}}}$ で ${\overset{\textnormal{さず}}{\text{授}}}$ けられ、 ${\overset{\textnormal{とっきょじょう}}{\text{特許状}}}$ によって ${\overset{\textnormal{さず}}{\text{授}}}$ けられるのではない。 \hfill\break
This title is granted by royal charter; it is not granted by a patent. }

\par{\textbf{Grammar Note }: Ex. 23 demonstrates how 授けられる is the correct form when using the 連用中止形. However, another reason that makes it the appropriate verb choice is that the entity granting the title is bureaucratic in nature, although royal. Although homage is being paid to the royal establishment, the sentence sounds procedural and not one of glorification. }

\par{24. ${\overset{\textnormal{きょうかい}}{\text{教会}}}$ で ${\overset{\textnormal{こ}}{\text{子}}}$ どもに ${\overset{\textnormal{せんれい}}{\text{洗礼}}}$ を ${\overset{\textnormal{さず}}{\text{授}}}$ けました。 \hfill\break
I baptized the children at (the) church. }

\par{25. ${\overset{\textnormal{こども}}{\text{子供}}}$ を ${\overset{\textnormal{さず}}{\text{授}}}$ からない ${\overset{\textnormal{じんせい}}{\text{人生}}}$ を ${\overset{\textnormal{えら}}{\text{選}}}$ んできたのです。 \hfill\break
I\textquotesingle ve chosen a life of not being blessed with children. }

\begin{center}
\textbf{預かる vs 預けられる }
\end{center}

\par{ The verbs 預かる and 預ける create an intransitive-transitive verb pair for “to take care of” and “to leave in someone\textquotesingle s care.” Incidentally, both verbs utilize the particle を. However, the “who” does what isn\textquotesingle t the same. For 預かる, it is either the speaker taking care of something for someone or the speaker telling someone to look after someone. For 預ける, the speaker is leaving something in someone\textquotesingle s care. The something is often money, kids, or other things that need taking care of that you entrust with someone. }

\par{ Although 預かる typically translates as “to take care of\slash to look after,” 預けられる typically translates as “to be given custody\slash to be left…to look after\slash to be entrusted with…” It turns out that from the standpoint of English phrasing, there shouldn\textquotesingle t be much reason why you would ever confuse the two. }

\par{26. ${\overset{\textnormal{わす}}{\text{忘}}}$ れ ${\overset{\textnormal{もの}}{\text{物}}}$ を ${\overset{\textnormal{あず}}{\text{預}}}$ かっています。 \hfill\break
I\textquotesingle m looking after lost belongings. }

\par{27. ${\overset{\textnormal{さいとう}}{\text{斉藤}}}$ は ${\overset{\textnormal{もとつま}}{\text{元妻}}}$ から ${\overset{\textnormal{こども}}{\text{子供}}}$ を ${\overset{\textnormal{むり}}{\text{無理}}}$ やり ${\overset{\textnormal{あず}}{\text{預}}}$ けられて ${\overset{\textnormal{こんわく}}{\text{困惑}}}$ した。 \hfill\break
Saito was bewildered from being forcibly given custody of his kids by his ex-wife. }

\par{\textbf{Spelling Note }: 無理やり \emph{ }may also be spelled as 無理矢理. }

\par{28. ${\overset{\textnormal{おば}}{\text{叔母}}}$ にお ${\overset{\textnormal{かね}}{\text{金}}}$ を ${\overset{\textnormal{あず}}{\text{預}}}$ けました。 \hfill\break
I entrusted money to my aunt. }

\par{29. ${\overset{\textnormal{かぎ}}{\text{鍵}}}$ を ${\overset{\textnormal{あず}}{\text{預}}}$ かってください。 \hfill\break
Please hold onto the key. }

\par{30. ${\overset{\textnormal{きゅう}}{\text{急}}}$ な ${\overset{\textnormal{ようじ}}{\text{用事}}}$ が ${\overset{\textnormal{でき}}{\text{出来}}}$ たので、 ${\overset{\textnormal{ほいくじょ}}{\text{保育所}}}$ で ${\overset{\textnormal{たんきかん}}{\text{短期間}}}$ だけ ${\overset{\textnormal{こども}}{\text{子供}}}$ を ${\overset{\textnormal{あず}}{\text{預}}}$ かってもらえるでしょうか。 \hfill\break
I\textquotesingle ve got a sudden matter to attend to, so could I have my kids looked after for just a short period of time at a nursery? }

\par{31. ${\overset{\textnormal{ちい}}{\text{小}}}$ さいうちから ${\overset{\textnormal{ほいくえん}}{\text{保育園}}}$ に ${\overset{\textnormal{あず}}{\text{預}}}$ けられている ${\overset{\textnormal{こども}}{\text{子供}}}$ って ${\overset{\textnormal{かわいそう}}{\text{可哀想}}}$ じゃないですか。 \hfill\break
Aren\textquotesingle t kids who are left in the care of a nursery school from a young age pitiable? }

\par{32. ペットホテルには、 ${\overset{\textnormal{よぼうせっしゅ}}{\text{予防接種}}}$ をしていない ${\overset{\textnormal{どうぶつ}}{\text{動物}}}$ も ${\overset{\textnormal{あず}}{\text{預}}}$ けられているため、 ${\overset{\textnormal{じぶん}}{\text{自分}}}$ のペットが ${\overset{\textnormal{かんせん}}{\text{感染}}}$ してしまうかもしれないというリスクがあることをご ${\overset{\textnormal{りかい}}{\text{理解}}}$ ください。 \hfill\break
At the pet hotel, because there are also animals in its care that are not immunized, please understand that there is the risk that your pet may become infected with something. }

\par{33. 荷物を彼に預けました。 \hfill\break
I entrusted my luggage to him. }

\par{34. ${\overset{\textnormal{あず}}{\text{預}}}$ けられない ${\overset{\textnormal{てにもつ}}{\text{手荷物}}}$ はありますか。 \hfill\break
Is there any baggage that can\textquotesingle t be kept? }

\par{35. ${\overset{\textnormal{こども}}{\text{子供}}}$ を ${\overset{\textnormal{あず}}{\text{預}}}$ けられないので ${\overset{\textnormal{つ}}{\text{連}}}$ れていくつもりでいますが、 ${\overset{\textnormal{だいじょうぶ}}{\text{大丈夫}}}$ でしょうか。 \hfill\break
I\textquotesingle ve been planning to bring my children along because I can\textquotesingle t leave them somewhere, but would that be okay? }

\begin{center}
\textbf{借りる vs 貸される }
\end{center}

\par{ The verbs 借りる and 貸す create an intransitive-verb pair for "to borrow" and "to lend" respectively. Etymologically speaking, 借りる is the intransitive counterpart of 貸す. In fact, for most of its history, it\textquotesingle s actually been 借る, which is still a valid form of the verb in most of West Japan. Incidentally, both the concept of borrowing and the concept of lending involve the particle を. }

\par{ The two verbs technically become synonymous when す becomes 貸される. This is because “to be lent” is the same thing as “to borrow.” The only difference is that “to borrow” always implies that one is purposely having someone loan something to oneself whereas if one is lent something, the lender could just be providing a kind gesture without seeking any word as to whether the act is necessary in the first place. Incidentally, this nuance splicing creates problems with the frequency of use for 貸される. It turns out that its only practical application is with the set phrase 手を貸す (to lend a hand). As you will see, a foreseeable situation to use the passive form is when an opponent\slash adversary offers a hand to you, but you don\textquotesingle t necessarily like the fact that the person did so. }

\par{36. ${\overset{\textnormal{じゅうたく}}{\text{住宅}}}$ ローンを ${\overset{\textnormal{か}}{\text{借}}}$ りています。 \hfill\break
I\textquotesingle m borrowing a home loan. }

\par{37. ${\overset{\textnormal{ともだち}}{\text{友達}}}$ に ${\overset{\textnormal{か}}{\text{貸}}}$ したお ${\overset{\textnormal{かね}}{\text{金}}}$ が ${\overset{\textnormal{かえ}}{\text{返}}}$ ってこない。 \hfill\break
The money I lent my friend has yet to return. }

\par{38. ${\overset{\textnormal{て}}{\text{手}}}$ を ${\overset{\textnormal{か}}{\text{貸}}}$ されることが ${\overset{\textnormal{きら}}{\text{嫌}}}$ いな ${\overset{\textnormal{こ}}{\text{子}}}$ がいれば、 ${\overset{\textnormal{せっきょくてき}}{\text{積極的}}}$ に ${\overset{\textnormal{おとな}}{\text{大人}}}$ に ${\overset{\textnormal{て}}{\text{手}}}$ を ${\overset{\textnormal{か}}{\text{借}}}$ りる ${\overset{\textnormal{こ}}{\text{子}}}$ もいます。 \hfill\break
If there are kids that hate having a hand lent to them, there are also kids that positively borrow a hand from adults. }

\par{39. ${\overset{\textnormal{あいて}}{\text{相手}}}$ から ${\overset{\textnormal{て}}{\text{手}}}$ を ${\overset{\textnormal{か}}{\text{貸}}}$ されたら ${\overset{\textnormal{くつじょく}}{\text{屈辱}}}$ だと ${\overset{\textnormal{おも}}{\text{思}}}$ わない? \hfill\break
Do you not think it\textquotesingle s a disgrace to be lent a hand from an opponent? }

\par{40. ${\overset{\textnormal{わたし}}{\text{私}}}$ の ${\overset{\textnormal{ねんしゅう}}{\text{年収}}}$ でいくら ${\overset{\textnormal{か}}{\text{借}}}$ りられるんですか。 \hfill\break
How much can I borrow with my annual income? }
      
\section{Carrying out Orders}
 
\begin{center}
\textbf{言い付かる vs 言い付けられる \hfill\break
仰せ付かる vs 仰せ付けられる }
\end{center}

\par{ The verbs 言い付かる and 言い付ける create an intransitive-transitive verb pair meaning “to be told to\slash ordered to” and “to tell to do\slash order to do\slash to tell on someone.” }

\par{ The objective\slash subjective dichotomy between 他受動詞 and the passive forms of their transitive counterparts becomes very evident with 言い付かる and 言い付けられる. 言い付かる is rather objective and it is safe to assume that the order is legitimate and deserves being followed trough. In Ex. 43, you can see how it would be used in reference to an order from a superior. 言い付けられる, on the other hand, almost always refers to orders people essentially drag you down with. Most contexts involve the speaker being annoyed at the person making all the orders. }

\par{ As for 仰せ付かる and 仰せ付けられる, the only difference is that the element of the verbs meaning “to say” is replaced with the honorific form of 言う, 仰せる, usually seen as 仰る (a contraction of 仰せられる). However, as you may be able to deduce from what\textquotesingle s been said thus far, the concept of respecting the order of a superior and being melodramatic about being bossed around don\textquotesingle t mix together. As such, 仰せ付けられる is essentially not used. However, you might find it in obscure text from older translation works from European languages. In which case, it would come from a literal translation in which the passive is used with an explicitly stated indirect object. Overall, it is important to note that 仰せ付かる  is not really used aside from contexts such as Ex. 44. }

\par{41. ${\overset{\textnormal{ようじ}}{\text{用事}}}$ を ${\overset{\textnormal{い}}{\text{言}}}$ い ${\overset{\textnormal{つ}}{\text{付}}}$ けてこないでください。 \hfill\break
Please don\textquotesingle t come telling me to take care of something. }

\par{42. ${\overset{\textnormal{やきんあ}}{\text{夜勤明}}}$ けで ${\overset{\textnormal{かえ}}{\text{帰}}}$ ってきたばかりなのに、 ${\overset{\textnormal{いぬ}}{\text{犬}}}$ の ${\overset{\textnormal{さんぽ}}{\text{散歩}}}$ を ${\overset{\textnormal{い}}{\text{言}}}$ いつけられた。 \hfill\break
I was told to walk the dog despite having just gotten home from night shift. }

\par{43. ${\overset{\textnormal{たな}}{\text{棚}}}$ の ${\overset{\textnormal{せいり}}{\text{整理}}}$ を\{ ${\overset{\textnormal{い}}{\text{言}}}$ い ${\overset{\textnormal{つ}}{\text{付}}}$ かった・ ${\overset{\textnormal{い}}{\text{言}}}$ い ${\overset{\textnormal{つ}}{\text{付}}}$ けられた\}が、 ${\overset{\textnormal{せいりのうりょく}}{\text{整理能力}}}$ がまったくないので ${\overset{\textnormal{こま}}{\text{困}}}$ っています。 \hfill\break
I was [ordered\slash told to] arrange the shelves, but because I have absolutely no organization skills, I\textquotesingle m in a bind. }

\par{44. 大役を仰せ付かりました。 \hfill\break
I\textquotesingle ve been appointed an important task\slash major part. }

\par{45. ${\overset{\textnormal{つま}}{\text{妻}}}$ に ${\overset{\textnormal{よくそう}}{\text{浴槽}}}$ の ${\overset{\textnormal{そうじ}}{\text{掃除}}}$ を ${\overset{\textnormal{い}}{\text{言}}}$ いつけられた。 \hfill\break
I was told to scrub the tub by my wife. }

\begin{center}
\textbf{言付かる vs 言付けられる }
\end{center}

\par{ The verbs 言付かる and 言付ける have the following meanings: }

\begin{itemize}

\item 言付かる: To be entrusted with; to be asked to. \hfill\break

\item 言付ける: To have a word\slash item sent; to make an excuse. 
\end{itemize}

\par{ Both words are meant to be formal. The former may be seen in business settings in which one is entrusted with a message to relay information. Ex. 47 is exemplary of honorifics that is slightly out-of-date, not because of the word choice but because of the situation in which one implies you\textquotesingle re going to meet someone of importance outside one\textquotesingle s inside group because one is following orders. In the minds of people today, that takes away from one\textquotesingle s sense of humility towards a client. }

\par{\emph{ }言付ける, alternatively spelled as 託ける, has two rather different meanings. The first is translated in a way that makes it seem to be passive in nature, but this is only an issue with English phrasing. In actuality, it simply refers to entrusting a word\slash item to someone so that it may be passed to the right person. The second usage meaning "to make an excuse” has for the most part been entirely taken over by the verb , which incidentally is also spelled as 託ける. This coincidence likely has a reason for why the verb overall is usually not used, being substituted for an explanation of the circumstances. The passive form 言付けられる is hardly ever used. Although it can be used in completely grammatical sentences, it is simply out of style. }

\par{46. ${\overset{\textnormal{なに}}{\text{何}}}$ も ${\overset{\textnormal{でんごん}}{\text{伝言}}}$ を ${\overset{\textnormal{ことづ}}{\text{言付}}}$ かっておりません。 \hfill\break
I have not been entrusted with any messages. }

\par{47. 橋本が只今出張中の為、佐々木様によろしくお伝えくださいと言付かって参りました。 \hfill\break
Mr. Hashimoto is currently on a business trip, and so I have come to convey his best regards to you, Mr. Sasaki, per his request. }

\par{48. ${\overset{\textnormal{やまもとどの}}{\text{山本殿}}}$ を ${\overset{\textnormal{けいだい}}{\text{境内}}}$ までお ${\overset{\textnormal{つ}}{\text{連}}}$ れするよう ${\overset{\textnormal{ことづ}}{\text{言付}}}$ かって ${\overset{\textnormal{まい}}{\text{参}}}$ りました。 \hfill\break
I have come here under orders that I lead you into the compound, Lord Yamamoto. }

\par{49. ${\overset{\textnormal{よう}}{\text{用}}}$ を ${\overset{\textnormal{ことづ}}{\text{言付}}}$ かって ${\overset{\textnormal{き}}{\text{来}}}$ ました。 \hfill\break
I have been brought here under orders of service. }

\par{50. ${\overset{\textnormal{よう}}{\text{用}}}$ を ${\overset{\textnormal{ことづ}}{\text{言付}}}$ けられてやってきた。 \hfill\break
I came here under the pretext of a task. }
      
\section{Being Caught Doing Something}
 
\begin{center}
\textbf{を見つかる \& を捕まる }
\end{center}

\par{ With the addition of what one is caught “doing” 見つかる and 捕まる can be used as 他受動詞. It is important to note that although some speakers will occasionally use them anyway, the passive transitive counterpart forms are not compatible in this situation. This is because there is no intent involved with the speaker being caught doing something. The situation is spontaneous and there is no logical reason for subjectivity to be interjected. People sometimes just get caught red-handed. }

\par{51. ベッドで ${\overset{\textnormal{ね}}{\text{寝}}}$ ているところを ${\overset{\textnormal{み}}{\text{見}}}$ つかった ${\overset{\textnormal{ねこ}}{\text{猫}}}$ が ${\overset{\textnormal{おどろ}}{\text{驚}}}$ いて ${\overset{\textnormal{に}}{\text{逃}}}$ げ ${\overset{\textnormal{だ}}{\text{出}}}$ した。 \hfill\break
The cat caught sleeping on the bed was startled and ran off. }

\par{52. ${\overset{\textnormal{まち}}{\text{街}}}$ でデートしているところを ${\overset{\textnormal{み}}{\text{見}}}$ つかった。 \hfill\break
I was caught on a date in town. }

\par{53. 親に酒を飲んでいるところを見つかった。 \hfill\break
I was caught drinking by my parents. }

\par{54. ${\overset{\textnormal{かつどうか}}{\text{活動家}}}$ ${\overset{\textnormal{にじゅうご}}{\text{25}}}$ ${\overset{\textnormal{にん}}{\text{人}}}$ が ${\overset{\textnormal{ゆでん}}{\text{油田}}}$ のプラットフォームを ${\overset{\textnormal{おそ}}{\text{襲}}}$ っているところを ${\overset{\textnormal{つか}}{\text{捕}}}$ まった。 \hfill\break
Twenty-five activists were captured assaulting an oil platform. }

\par{55. ${\overset{\textnormal{かれ}}{\text{彼}}}$ らは ${\overset{\textnormal{じょせい}}{\text{女性}}}$ や ${\overset{\textnormal{なんみん}}{\text{難民}}}$ のふりをしているところを ${\overset{\textnormal{つか}}{\text{捕}}}$ まった。 \hfill\break
They were captured pretending to be women and refugees. }

\par{56. ${\overset{\textnormal{かれ}}{\text{彼}}}$ は ${\overset{\textnormal{うんてんちゅう}}{\text{運転中}}}$ に ${\overset{\textnormal{けいたい}}{\text{携帯}}}$ を ${\overset{\textnormal{しよう}}{\text{使用}}}$ しているところを ${\overset{\textnormal{つか}}{\text{捕}}}$ まった。 \hfill\break
He was caught using his cellphone while driving. }
    