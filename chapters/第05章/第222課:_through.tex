    
\chapter{Through}

\begin{center}
\begin{Large}
第222課: Through: を通じて \& を通して 
\end{Large}
\end{center}
 
\par{ ~を通して and ~を通じて without context are completely interchangeable. So, getting the small details straight will probably be the hardest thing about this lesson. }

\par{\textbf{漢字 Note }: Note that there is also ${\overset{\textnormal{かよ}}{\text{通}}}$ う and ${\overset{\textnormal{とお}}{\text{通}}}$ る. The first is used in the sense of “commute” or flowing\slash understanding such as in phrases like 学校に通う and 心が通う. 通る expresses transit. Ex. 道を通る. }
      
\section{~を通して VS ~を通じて}
 
\par{ を通しで and ~通(とお)じて are incorrect. Pronunciation is perhaps the most common mistakes about these. That should be good news. Both show a meaning equivalent to ~を経て\slash を仲介して to show a means "of through (the intermediary of)". They can also be used to mean "through" in a temporal sense. }

\par{1. 仕事を\{通じて・通して\}人脈が広がっていった。 \hfill\break
My connections spread through my job. }

\par{2. 1年を\{通じて・通して\}スキーができる。 \hfill\break
You can ski (here) throughout the year. }

\par{3. 日本語の勉強を\{通じて・通して\}日本人の考え方が少し分かってきました。 \hfill\break
Through my Japanese studies, I have come to understand the Japanese way of thinking a little. }

\par{通じる is a verb used to show the result gained by a certain action; 通す is thought to hold a volitional side. Although even in these examples either is OK, when you want to show volitional, the latter is the best choice. }

\par{ In a normal situation where this is not an exceptional circumstance, it is best not to use ~を通じて. The longer the sentence, the easier it gets to become to decide whether it fits this criterion or not. Certain speech modals also aid in the decision. }

\par{4. 私はインターネット\{〇 を通して・X を通じて\}役に立つニュースを得ました。 \hfill\break
I got useful news through the Internet. }

\par{5. 一緒に働いた経験\{〇 を通して・X を通じて\}、ふたりは生涯かわらぬ友愛をもちつづけた。 \hfill\break
Through the experience of working together, they continued to hold a friendship for life. }

\par{6. 仲人を通して縁談をすすめたほうがいいでしょう。 \hfill\break
Isn't it best to forward an engagement through a middleman? }

\par{7. ハッカーたちは、ウイルスに感染されたファイルが送りこめるインターネット\{〇 を通じて、Xを通して\}、ユーザーのシステムを破壊していく。 \hfill\break
Hackers through the internet that can send files infected with viruses go on to destroy the user\textquotesingle s system. }

\par{ Even when either is fine, there will always be slight nuance differences. }

\par{8. 社長との連絡は、すべて秘書を\{通して・通じて\}行われた。 \hfill\break
All contact with the company president was carried out through the secretary. }

\par{ The first option makes the situation far from out of the ordinary, but the second option makes it sound secretive. }

\par{ There are two broad situations that can describe most sentences with these modals. There could be an intermediary between an X and Y, or through the means of A, B is done. Consider the following information for further distinguishing. }

\par{\textbf{Aを通してB }: Through the intermediation\slash means of A, B is done simply and clearly. \hfill\break
\textbf{Aを通じてB }: A, which is an intermediate or means, is something that can\textquotesingle t be public, must be a secret, or is something unjust. }

\par{ As mentioned earlier, there is a peculiar nature about ~を通じて, and special circumstances are often mysterious, clandestine, and may involve some sort of secret means. In this sense, it contrasts the openness and simplicity of action implied by statements with ~を通して. }

\par{ Now, what about a sentence like the following? }

\par{9. この種の松林は、日本全国を\{通して・通じて\}見ることができる。 \hfill\break
You can see this kind of pine forest throughout all of Japan. }

\par{ If you get a native speaker that doesn't like one version or another, you probably need to find a different opinion. This sentence is different because there is no intermediation or means involved. In this sense they are similar to “throughout”. The decision as to which one should be used goes down to individual speaker variation pending on whether that person likes more native words over 漢語. With this, ~を通じて can be used in a stiffer and more literary fashion. }

\begin{center}
\textbf{Attribute Forms }
\end{center}

\par{ Don't you just love attribute forms? As you may have guessed, ~を通した, ~通しての, ~通じた, and ~を通じての are all possible, but they are not completely interchangeable. }

\par{ Aを\{通した・通じた\}B can\textquotesingle t be used in situations where A is a direct information source. This is because "Aを通す" and "A\{を・が\}通じる" originally meant that through the intermediation of A, two spaces are connected (examples of this will be shown later in this lesson). }

\par{ Accordingly, A both can be used as the go-between X and Y, but when A becomes the information source, "Aを通じてのB" or "Aを通してのB" must be used. }

\par{10. 秘書を\{通した・通じた・通しての・通じての\}連絡が社長に届いた。 \hfill\break
A contact through the secretary reached the company president }

\par{11. 今回、入手した情報は、その筋の人を\{〇 通しての・〇 通じての・X 通した・X 通じた\}ものです。. \hfill\break
The information I received this time is info from a person of that source. }

\par{ When showing consistency temporally or geographically, ~を通しての and ~を通じての are your only options. }

\par{12. 日本全国を\{通じての・通しての\}この種の松林は、日本の森林を特徴づけるもののひとつである。 \hfill\break
This kind of pine forest throughout Japan is one characteristic of the forests of Japan. }

\par{13. 山を\{〇 通しての・△ 通じての\}トンネル \hfill\break
A tunnel through a mountain }

\par{ It does become more natural to rephrase them out, especially since there\textquotesingle s not much of a sense of familiarity in sentences like the second to last. Also, when there is not a sense of a go-between, ~を通じた・を通した mustn't be used. }

\par{14. 彼らは一緒に働いた経験を通しての友愛を持ち続けた。 \hfill\break
They continued to a hold friendship through the experience of working together. }

\begin{center}
 \textbf{Other Meanings }
\end{center}

\par{ Remember that 通す means "pass through" and can be used in several contexts. This remains the case in ~を通して. 通じる can mean “to lead through\slash get through\slash get across”. Both are used in many idioms as well. }

\par{15. 道を開けて、車を通す。 \hfill\break
To open up the road and let cars pass through. }

\par{16. 前に化け物が立ちはだかって、僕を通そうとしなかった。 \hfill\break
A monster blocked my path in front of me, and it didn't attempt to let me pass by. }

\par{17. 空港まで政府が高速道路と新幹線を開通させた。 \hfill\break
The government created passage through a highway and bullet train line up to the airport. \hfill\break
 \hfill\break
\textbf{Word Note }: 開通する is more appropriate here, but 新幹線を通す is an appropriate phrase, but it is more general in nature and doesn't fit well with 政府. }

\par{18. 肉に串を通す。 \hfill\break
To put meet through a skewer. }

\par{19. 目を通す。 \hfill\break
To scan. }

\par{20. やりたいことを通す。 \hfill\break
To do what one wants. }

\par{21. 戻ってきてから、目を通すから。 \hfill\break
I'll look over it when I come back. }

\par{22. 針に糸を通す。 \hfill\break
To thread yard in a needle. }

\par{23. 袖に手が通せるか。 \hfill\break
Can you pass your hands through your sleeves? }

\par{24. 刺身を氷水に通さなきゃいけない。 \hfill\break
You have to pass sashimi through ice water. }

\par{25. 彼らに私の中国語は通じませんでした。 \hfill\break
I couldn't make myself understood to them in Chinese. }

\par{26. 電話が通じなくなりました。 \hfill\break
The phone went dead. }

\par{27a. バスがオースティンとサン・アントニオの間に通じている。 \hfill\break
27b. バスがオースティンとサン・アントニオの間を通している。(もっと自然) \hfill\break
The bus runs between Austin and San Antonio. }

\par{28. 肉を熱湯に通すと、殺菌できる。 \hfill\break
If you put meat through hot water, you'll sterilize it. }

\par{29. 黒いものは熱を通しやすい。 \hfill\break
It is easy for black objects to transmit heat. }

\par{30. これは水を通さない。 \hfill\break
This will not let water pass through. }

\par{31. 通してくださいませんか。 \hfill\break
Could you let me through? }

\par{32. \{火・熱\}を通す。 \hfill\break
To heat (food). }

\par{33. 我を通す。 \hfill\break
To have one's way. }

\par{34a. 風を通さないように窓を閉めてください。 \hfill\break
34b. 風が通らないように窓を閉めてください。(もっと自然) \hfill\break
In order to not ventilate air, please shut your windows. }

\par{35. 肉の中まで火を通す。 \hfill\break
To pass the meat up to the center in the fire. }

\par{36. ステーキはよく火を通すべきだ。 \hfill\break
You should cook steak well. }

\par{37. 堅を通して僕は彼女と知り合った。 \hfill\break
I got acquainted with her through Ken. }

\par{38. 政界で成功したければ、自分の建て前を明確に通さなければなりません。 \hfill\break
If you want to succeed in politics, you have to correctly put in one's official stance. }

\par{39. 理屈を通す。 \hfill\break
To correctly arrange the logic (in something). }

\par{40. 自己の信念を通すとは真に勇気があるという意味だ。 \hfill\break
Persisting in one's belief is the true definition of courage. }

\par{41. 独身で通すのはださいな。 \hfill\break
Remaining single sucks. }

\par{42a. インターネットを通じてビデオゲームを注文する。(書き言葉) \hfill\break
42b. インターネットでビデオゲームを注文する。(話し言葉) \hfill\break
To order video games via the Internet. }
 
\par{43. 全歴史を通じて、東北関東大震災は第5番目のもっとも巨大な地震です。 \hfill\break
Throughout all of history, the Tohoku-Kanto Earthquake Disaster is the 5th largest earthquake. }
 
\par{44. 一年を\{通して・通じて\} \hfill\break
Throughout the year }
 
\par{45. 週末を通じて仕上げられるのか。(ちょっと硬い) \hfill\break
Can you do it over the weekend? }

\par{46a.  ${\overset{\textnormal{めいし}}{\text{名刺}}}$ を差し出して面会を求める。 \hfill\break
46b. ${\overset{\textnormal{し}}{\text{刺}}}$ を通じて、面会を求める。(Very Rare) \hfill\break
47b. 名刺を通じて、面会を求める。(Rare) \hfill\break
To present one's card and seek a meeting. }

\par{48. ${\overset{\textnormal{きみゃく}}{\text{気脈}}}$ を通じる。 \hfill\break
To have secret connections with something\slash one. }

\par{49a. ${\overset{\textnormal{でんりゅう}}{\text{電流}}}$ を通じる。 \hfill\break
49b. 電流が通じる。(More natural) \hfill\break
To connect an electric current. }
 
\par{50. あいつは ${\overset{\textnormal{てき}}{\text{敵}}}$ に通じてる。 \hfill\break
He's colluding with the enemy! }
 
\par{51. そんなことでは社会では通じません。 \hfill\break
In such manner is not accepted in society. }

\par{52. ${\overset{\textnormal{じじょう}}{\text{事情}}}$ に通じている。 \hfill\break
I'm well-informed of the circumstances. }
 
\par{53. 僕の中国語は通じなかった。 \hfill\break
My Chinese wouldn't get across. }
 
\par{54. 電話が通じなくなった。 \hfill\break
My phone went dead. }
 
\par{55. この道は ${\overset{\textnormal{ちょうじょう}}{\text{頂上}}}$ に通じている。 \hfill\break
This road leads to the summit. }

\par{56. ${\overset{\textnormal{ていでん}}{\text{停電}}}$ で何も通じない。 \hfill\break
Nothing will go through due to the blackout. }

\par{57. 参議院\{において・で\}法案を通す。 \hfill\break
To pass a bill through the House of Councilors. }

\par{58. 立法者たちは米議会で法律を無理矢理通しすぎる。 \hfill\break
Lawmakers pass too many laws forcibly through the US Congress. }

\par{59a. 補聴器を通して話が聴けるようになった。(改まった) \hfill\break
59b. 補聴器で話が聴けるようになった。(自然) \hfill\break
I have become able to listen to conversations with a hearing aid. }

\par{60. フィルターを通して不純物を除去すればよい。 \hfill\break
It is good for you to get rid of impurities through a filter. }

\par{61. 昼夜\{を通して・を問わず\}工事を続けた。 \hfill\break
(The workers) continued to work on the construction around the clock. }

\par{62. 全幕を通して役者を見る。 \hfill\break
To see through the entire curtain and watch the actors. }

\par{63. 一生独身を通す。 \hfill\break
To live one's whole life single. }

\par{64a. 傷にオキシドールを塗れば、消毒できるようになります。(もっと自然) \hfill\break
64b. 傷をオキシドールに通して、消毒できるようになります。 \hfill\break
If you will pass your wound through hydrogen peroxide, you'll be able to disinfect it. }

\par{65a. レインコートの中まで雨が沁みてきた。(もっと自然) \hfill\break
65b. レインコートを通して雨が沁みてきた。 \hfill\break
Rain has started to soak in through my raincoat. }

\par{66a. 一年中和服で過ごす人は少ない。(自然) \hfill\break
66b. 一年中和服で通す人は少ない。(ちょっと不自然) \hfill\break
There are seldom people that wear traditional Japanese clothing throughout the year. }
 
\par{67. 気持ちが通じる。 \hfill\break
To have one's feelings get across. }

\par{68a. ${\overset{\textnormal{せこ}}{\text{世故}}}$ に ${\overset{\textnormal{つう}}{\text{通}}}$ じている。 \hfill\break
68b. 世故にたけている。 \hfill\break
I know fully about worldly affairs. }

\par{69. ${\overset{\textnormal{ひとづま}}{\text{人妻}}}$ と通じる。 \hfill\break
To commit adultery. }

\par{70. ${\overset{\textnormal{だいべん}}{\text{大便}}}$ が通じる。 \hfill\break
To have a bowel movement. }

\par{71. ${\overset{\textnormal{しぜん}}{\text{自然}}}$ であることは幸せに通じている。 \hfill\break
Being natural is tied to happiness. }
 
\par{72. 『 ${\overset{\textnormal{おく}}{\text{憶}}}$ 』は『 ${\overset{\textnormal{おく}}{\text{臆}}}$ 』に通じる。 \hfill\break
To not distinguish between 憶 and 臆. }

\par{73a. 服装を通して人の風俗を考える。 \hfill\break
73b. 服装から人に品位を考える。(もっと自然) \hfill\break
To think about the manners of people through their dress. }

\par{74a. 高校生の女子は夜通し語り合った。 \hfill\break
74b. 高校生の女子は夜を通して語り合った。(Wordier) \hfill\break
The high school girls talked to each other all through the night. }

\begin{center}
 \textbf{~によって VS ~に通して・に通じて }
\end{center}

\par{ Although mentioned in the lesson about ~によって, there is some interchangeability between ~によって, ~を通じて, and ~を通して with this usage. However, the first places emphasis on the connection between the method and effect\slash result. The latter two place emphasis on the process. }

\par{75. インターネット\{によって・を通じて・を通して\}、アルバムを販売する。 \hfill\break
To sell albums by\slash via\slash through the internet. }

\par{76. 選挙\{〇 によって・X を通じて・X を通して]委員長になる。 \hfill\break
To become the committee chairman by election. }
    