    
\chapter{Exception}

\begin{center}
\begin{Large}
第233課: Exception: ~をのぞいて, ~をおいて, \& ~ならでは 
\end{Large}
\end{center}
 
\par{ These phrases limit things as exceptions in their own unique ways. }
      
\section{~のぞいて}
 
\par{ After a noun phrase of some sort, ~除いて is used to mean “except…” and is often left in ひらがな. It is a general phrase and will differ with the emotive power of the following grammar point ~をおいて. The verb 除く can mean “to remove\slash exclude”, and “exclude” can also be expressed with the verb ${\overset{\textnormal{じょがい}}{\text{除外}}}$ する, which is of the vein of “to set aside\slash rule out”. }

\par{1. ${\overset{\textnormal{いっせき}}{\text{一隻}}}$ を除いて ${\overset{\textnormal{かんせん}}{\text{艦船}}}$ は全部 ${\overset{\textnormal{しず}}{\text{沈}}}$ んだ。 \hfill\break
Except for one vessel, all of the carriers sank. }

\par{2. きのう、暑かったことをのぞいて、僕らは楽しいときを過ごした。 \hfill\break
Yesterday, except it being hot, we had a good time. }

\par{3. 私は除いてください。 \hfill\break
Please include me out. }

\par{4. 私は報道記者と政治家を除く全ての人々が平等であると信じている。 \hfill\break
I believe it in the equality of all people except reporters and politicians. }

\par{5. 北部をのぞいて、お天気はよかったですよ。 \hfill\break
The weather was good except in the north. }

\par{6. わたしを除いて全ての人がそれを知っていたそ。 \hfill\break
Except me, it seems that everyone knew about it. }

\par{7. 女性は ${\overset{\textnormal{ちょうさたいしょう}}{\text{調査対象}}}$ から除外することになった。 \hfill\break
It was decided that women were to be taken out of the inquiry subjects. }

\par{ \textbf{Variant Note }: In more formal literary fashion, this can also be seen as ~のぞき. }

\par{8. 政府の ${\overset{\textnormal{きせいかいかくかいぎ}}{\text{規制改革会議}}}$ が ${\overset{\textnormal{ぜんめんてき}}{\text{全面的}}}$ な ${\overset{\textnormal{かいきん}}{\text{解禁}}}$ を求めている、インターネットを使った ${\overset{\textnormal{しはんやく}}{\text{市販薬}}}$ の販売について、 ${\overset{\textnormal{かんかんぼうちょうかん}}{\text{菅官房長官}}}$ や ${\overset{\textnormal{たむらこうせいろうどうだいじん}}{\text{田村厚生労働大臣}}}$ ら関係4 ${\overset{\textnormal{かくりょう}}{\text{閣僚}}}$ は、 ${\overset{\textnormal{ふくさよう}}{\text{副作用}}}$ のリスクの評価が終わっていない、ごく一部の市販薬を除き、 ${\overset{\textnormal{たいはん}}{\text{大半}}}$ を解禁する方向で ${\overset{\textnormal{ちょうせい}}{\text{調整}}}$ を進める見通しです。 }

\par{In regards to the all-round band lift of the sale of over-the-counter drugs using the internet that a government regulation reform meeting is seeking, Chief Cabinet Secretary Kan, Prime Minister Tamura of the Ministry of Health, Labor and Welfare, and four other bureaucrats in connection are forecast to further the coordination in the direction of lifting the ban in large part with exception to a small amount of over-the-counter drugs of which the evaluation of side-effect risks is not over. \hfill\break
From NHK on June 4, 2013. }

\par{\textbf{Warning Note }: Do not confuse this with the verb 覗く, which means to “to peep\slash take a look at” and has the same pronunciation. }

\par{\textbf{~をのぞいて VS ~以外 }}

\par{ This is very similar to ~以外.  One thing to notice, though, is that ~以外 is a suffix, and as seen below, the phrase is more affirmative and somewhat more formal sounding. }

\par{9. 九州以外のどこも地震が発生した。 \hfill\break
With exception of Kyushu, earthquakes occurred everywhere. }

\par{10. これ以外に方法はありませんよ。 \hfill\break
There is no method other than this. }
      
\section{~をおいて}
 
\par{ ~をおいて means "apart from". This phrase is used to highly evaluate something or someone. So, it's a good thing. There is always something like  ほかに\dothyp{}\dothyp{}\dothyp{}(い)ない after it. When you are generally saying "aside from", use ~を除いて. Like ~除いて, ~をおいて is used after noun phrases. }

\par{This phrase originally showed a meaning of "without emphasizing A as something important, one treats A as being useless and throws it out\slash places it to the side\slash separates it apart\slash removes it. In this sense, Aをおいてほかに(い)ない becomes a double negative expression, which then makes it an extremely powerful affirmative statement of A being number one. }

\par{In Japanese it is usually the case that very powerful expressions are kept in the written language, and although this is the case for をおいて, in rather formal situations, it can be used to declare one's top recommendation. It doesn't necessarily have to be in formal situations; however, as the first example sentence demonstrates clearly. }

\begin{center}
 \textbf{Examples }
\end{center}

\par{11. 君をおいて ${\overset{\textnormal{てきにんしゃ}}{\text{適任者}}}$ はいないぞ! \hfill\break
Apart from you there's no responsible person! }

\par{12. 頼れるのは彼女をおいてほかにない。 \hfill\break
There is no one to rely apart from her. }

\par{13. これをおいてほかに方法はなかった。 \hfill\break
There was no way but this. }

\par{14. それはさておき・それはさておいて \hfill\break
Aside from that }

\par{15. 九州と四国を除いてそこかしこで東日本大震災は多数の死者を出して大災害が発生する結果となりました。 \hfill\break
Apart from Kyushu and Shikoku, no matter where there are many deaths and major catastrophes have occurred in the Tohoku-Kanto Earthquake Catastrophe. }

\par{\textbf{~別 VS ~をおいて }}

\par{ The following paraphrases with 別 is used rather than ~をおいて because the situation is not one of pointing out as the best but something that should be treated separately. }

\par{16a. 震度7 は別として、釜石市には津波が押し寄せました。 \hfill\break
16b. 震度7度の地震とは別に、津波が釜石市を ${\overset{\textnormal{おそ}}{\text{襲}}}$ った。 \hfill\break
Apart from a 7 earthquake, Kamaishi City got flooded by a tsunami. }

\par{\textbf{Word Note }: The 震度 system is on a different scale than the Richter scale. }
      
\section{~ならでは}
 
\par{ ~ならでは is one of those instances where classical grammar holds on strong. The なら in this expression, like the particle なら comes from the 未然形 of the Classical Japanese copular verb なり. It is then followed by a classical usage of で, which is not related to the contraction of にて. In this case, it is the contraction of ~ずて, which is equivalent to ~なくて. は, here, is here for contrastive purposes. If the pattern were translated into something solely Modern Japanese, it would be equivalent to ~でなくては・~でなければ. }

\par{ This pattern is used to express the brilliance\slash wonderfulness of something by claiming that only it is as such. In the unabbreviated state of a sentence with it, a complementing verbal\slash adjectival phrase follows, but this can be omitted out if that phrase is being used as an attribute by replacing it with の, giving ~ならではの, which is the more common form of the pattern. }

\par{ The entire phrase, which is “AならではBない(C )” is, then, equivalent to expressions such as “AであってはじめてB” and “AだからこそB”. In the case that A is a commonplace noun, ~ならでは is of the sense of the former coming from the position of societal wisdom\slash common sense. However, when A is particular, it shows that only A can do C. }

\begin{center}
\textbf{Examples } 
\end{center}

\par{17. 君ならではできないことだよ! \hfill\break
Without you, this cannot be done! }

\par{18. それ、中国の方ならでは\{できない・の\}ものの考え方ですよ。 \hfill\break
That is only a matter of thought possibly by a person from China. }

\par{19. それって、大阪弁ならでは\{ありえない・の\}人当たりのよさですね。 \hfill\break
That is charm only that of Osaka Dialect, isn\textquotesingle t it? }

\par{20. ${\overset{\textnormal{しゃちょう}}{\text{社長}}}$ ならではの ${\overset{\textnormal{はっそう}}{\text{発想}}}$ ですね。 \hfill\break
This can only be a conception of the company president, isn't it? }

\par{21. この ${\overset{\textnormal{きょうかしょ}}{\text{教科書}}}$ ならではの ${\overset{\textnormal{とくちょう}}{\text{特徴}}}$ です。 \hfill\break
This is a characteristic only in this textbook. }

\par{22. 50年続いた老舗ならでは出せないこの味の良さ! \hfill\break
The goodness of this flavor that only an old shop for 50 years can offer! }

\par{23. この体験は沖縄ならではですよ。 \hfill\break
This experience can only be in Okinawa! }

\par{\textbf{Grammar Note }: As you can see, another exceptional thing about ~ならでは is that it can be followed by ~です with the rest of the pattern omitted. }

\par{24. さすが一流レストランのシェフならでは\{出せない・の\}味ですね。 \hfill\break
This is as expected flavor that can only be from a first rate restaurant chef. }

\par{\textbf{Phrase Note }: さすが, which is an adverb that shows something as something to be expected, is frequently used with ~ならでは. }
    