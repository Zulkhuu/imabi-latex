    
\chapter{As soon as}

\begin{center}
\begin{Large}
第244課: As soon as: ~や(いなや), ~なり, ~途端(に), ~かと思うと, \& ~次第 
\end{Large}
\end{center}
 
\par{ Though "as" itself is hard enough in Japanese, Japanese also has a lot of expressions that mean “as soon as”. This lesson will investigate how to use these expressions. Take close attention to what defines them as there are differences! }
      
\section{~や否や \& ~や}
 
\par{ Placed after the ${\overset{\textnormal{しゅうしけい}}{\text{終止形}}}$ of a verb in ~や or ~や ${\overset{\textnormal{いな}}{\text{否}}}$ や, it means "as soon as". With 否や, it is like the second action takes place before the first action can even be confirmed or done. The most used is や否や. In fact, や否や will be used 90\% of the time. }

\par{1. ${\overset{\textnormal{じしん}}{\text{地震}}}$ が ${\overset{\textnormal{はっせい}}{\text{発生}}}$ するやいなや、 ${\overset{\textnormal{きしょうちょう}}{\text{気象庁}}}$ は ${\overset{\textnormal{ただ}}{\text{直}}}$ ちに ${\overset{\textnormal{つなみけいほう}}{\text{津波警報}}}$ を出しました。 \hfill\break
As soon as the earthquake happened, the Meteorological Agency issued a tsunami warning. }
 
\par{2. 姉の顔を見るやいなや、泣き出した。 \hfill\break
I began weeping as soon as I saw my sister's face. }

\par{3a. その選手はチャンスと見るや、一気に\{攻め込んだ・攻撃をしかけた\}。 \hfill\break
3b. その選手はチャンスと見るやいなや、一気に\{攻め込んだ・攻撃をしかけた\}。 (あまり使わない言い方) \hfill\break
The player felt a chance open and attacked in one burst. }

\par{4. ${\overset{\textnormal{しゅっぱつ}}{\text{出発}}}$ するやいなや雨が ${\overset{\textnormal{ふ}}{\text{降}}}$ り出した。 \hfill\break
It started to rain as soon as I departed. }
 
\par{ ~やいなや cannot be used to show the speaker's wants or intentions. Also, don't use the past tense before ~やいなや. The action can be something that you can expect. It\textquotesingle s important to understand that the events of the second clause are actions\slash movements. }
 
\par{5. うちの猫は魚が大好きで、あげるやいなや、一気に全部食べてしまう。 \hfill\break
My cat loves fish, and the instant you give it to it, it's gone in an instant. }
      
\section{The Conjunctive Particle なり}
 
\par{ After the 連体形 of a verb, ~なり shows that something is done as soon as something else is done. So right when someone does something, they do something next in sequence to the first action. The subject is normally third person, and the subject is the same in both clauses. }

\par{6a. 宿題を済ませるなり、彼らはインターネットを使った。 \hfill\break
6b. 宿題を済ませるとすぐに、彼らはインターネットを使った。(More natural) \hfill\break
They used the Internet as soon as they finished their homework. }

\par{7. 彼は帰るなり、トイレに行った。 \hfill\break
He went to the bathroom as soon as he got home. }

\par{8. 社長は入ってくるなり、大声で怒鳴りました。 \hfill\break
As soon as the company president came in, he shouted in a big voice. }

\par{ The word comes from the なり in words like 身なり (appearance). The event in the second clause is often one that describes an action\slash condition not wanted. Although it is used some in the spoken language, it is usually reserved to writing. }

\par{ After the past tense, it shows a situation that is still in play as another action begins. It is unnatural when you move in any way to a different action. }

\par{9. 彼は靴を履いたなり、畳に上がってしまった。  △ \hfill\break
彼は靴を履いたまま、畳に上がってしまった。  〇 \hfill\break
He accidentally stepped onto the tatami mat with his shoes still. }

\par{10. 彼女は公園でベンチに座り込んだ\{なり・まま\}、眠ってしまった。 \hfill\break
She fell asleep while sitting on a park bench. }
      
\section{~途端(に)}
 
\par{ When students learn of 途端に, they immediately think of ~ときに,  especially when they learn that this phrase is used after the past tense. Understandable, ~たとき(に)and ~たとたん(に) vaguely resemble each other, especially when not written in 漢字. }

\par{ ~た ${\overset{\textnormal{とたん}}{\text{途端}}}$ (に) is, again, only used with the past tense. It marks the instant after one does something. This “something” that happens afterward is something that is \textbf{unexpected }and much of a surprise. }

\par{11. 彼がドアを開けた途端、車が爆発した。 \hfill\break
The car exploded just as he opened the door. }

\par{12. ${\overset{\textnormal{まど}}{\text{窓}}}$ を開けた途端に、犬が飛び出していった。 \hfill\break
As soon as I opened the window, the dog jumped out. }

\par{13. 立ち ${\overset{\textnormal{さ}}{\text{去}}}$ った途端に、その ${\overset{\textnormal{たてもの}}{\text{建物}}}$ が ${\overset{\textnormal{ばくはつ}}{\text{爆発}}}$ したよ! \hfill\break
As soon as he left, the building exploded! }

\par{14. ${\overset{\textnormal{どうが}}{\text{動画}}}$ を ${\overset{\textnormal{み}}{\text{観}}}$ た途端に、 ${\overset{\textnormal{ねむ}}{\text{眠}}}$ くなってしまった。 \hfill\break
I accidentally fell asleep as soon as I saw the video. }

\par{15. 余所見をした途端、転んだ。 \hfill\break
I fell down as soon as I turned away. }

\par{16. 酒を飲む と、(その)途端に 人が変わる。 \hfill\break
Once a person drinks, that person changes instantly. }

\par{\textbf{Grammar Note }: 途端 may also be used in the sense of instantly and can be seen with such statements. }

\par{ As seen in the example sentences, though, ~た途端(に)can\textquotesingle t be used when something volitional occurs. So, although a situation may be unexpected, you would need to use a phrase like ~たら、すぐに if you are using a verb of volition. }

\par{17. ベルが鳴ったら、すぐに外に出てください。 \hfill\break
When the bell rings, immediately go outside. }
      
\section{~かと思うと}
 
\par{ Perhaps because it is rarely treated separately in grammar discussions, students have a hard time knowing how to use this phrase. This describes something happening that causes a sense of surprise, similar to ~た途端(に), which follows after an event described in the first clause. Transitivity, though, is clearly different as ~た途端(に)is used with transitive expressions, but this phrase does not have that requirement. }

\par{ You cannot use this phrase for yourself. At times it may be best to use ~かと思ったら instead. Sentences of command, negation, or will appear afterwards. For some people, when the event is not of the future or going back and forth in a way that means the first action may happen again in the "future", then the use of ~かと思うと is somewhat unnatural and should be replaced with ~かと思ったら. So, for the sentences below regarding weather, you can paraphrase with the latter and have no problem. }

\begin{center}
 \textbf{Examples }
\end{center}

\par{18. 雨が降ってきたかと\{思うと ?\slash 〇・思ったら 〇\}、もう止んだ。 \hfill\break
Just when I thought it had started to rain, it stopped. }

\par{19. 空が曇ってきたと\{思うと ?\slash 〇・思ったら 〇\}、突然大雨になった。 \hfill\break
Just when I thought it got cloudy, it started to rain heavily. }

\par{20. 彼女はA君のことを好きだと言っていたかと思うと、次はB君が好きだと言い出し、何を考えているのかよく分からない。 \hfill\break
Just as I had thought she said she liked A-kun, she then stated that she liked B-kun, and so I don't know what she's thinking. }

\par{22. うちの子供は帰ってきたかと\{思うと ?\slash 〇・思ったら 〇\}、もう外に遊びに行った。 \hfill\break
The kids had already gone to play outside just when I thought they had come home. }
      
\section{~次第}
 
\par{ As a regular noun, 次第 means "course of events". It is seen after nouns a lot, especially Sino-Japanese words, but also after the 連用形 of verbs. }

\par{ When it attaches to other nouns, it often means "depending on". As such, when after the 連用形 of a verb, it may show that something is dependent on an action in question. }

\par{23. ${\overset{\textnormal{せいきゅう}}{\text{請求}}}$ があり次第だ。 \hfill\break
It's on demand. }

\par{24. 物事は状況次第。 \hfill\break
Circumstances alter cases. }

\par{25. こういう次第だ。 \hfill\break
This is how it stands. }

\par{26. 万事は君の(この)取り扱い方次第だ。 \hfill\break
It depends on the way you handle it. }

\par{27. 天気の様子次第で行くかどうか決めるんだ。 \hfill\break
I'm going to decide on whether to go or not depending on the weather condition. }

\par{28. 大統領が決心するかどうかはそれ次第だ。 \hfill\break
The president's resolution depends on this. }

\par{${\overset{\textnormal{}}{\text{29. 豊作になるかどうか}}}$ は天気次第です。 \hfill\break
An abundant harvest is dependent on the weather. }

\par{30. 結果は君次第だ。 \hfill\break
The results depend on you. }

\par{31. 昇進できるかどうかは能力次第だ。 \hfill\break
Promotion is dependent on one's ability. }

\begin{center}
 \textbf{連用形 + 次第 }
\end{center}

\par{ It may also be used with the 連用形 of a verb to show what happens "as soon as\dothyp{}\dothyp{}\dothyp{}". This is in the sense that right after something realizes, one does the next action. Thus, the second clause following must be a verb of volition in regards to the speaker. This separates it quite well from the other “as soon as” phrases in this lesson. }

\par{32. 着き次第、被災地に援助します。 \hfill\break
As soon as we arrive, we will aid in the disaster area. }

\par{33. 手当たり次第に食べ物を買い ${\overset{\textnormal{し}}{\text{占}}}$ める。 \hfill\break
To buy up food as it is made available. }

\par{34. 満員になり次第締め切ろう。 \hfill\break
Let's close up as soon as we become a full house. }

\par{35. 我々は雪がやみ次第、作業を再び始めるつもりです。 \hfill\break
We plan to resume operations as soon as it stops snowing. }

\begin{center}
 \textbf{Adverbial 次第に }
\end{center}

\par{ Lastly, 次第に is an adverb meaning "gradually", "finally", or "in order". }

\par{36. 機械は\{次第にだんだんと\} ${\overset{\textnormal{すた}}{\text{廃}}}$ れていくだろう。 \hfill\break
Machines will gradually go out of date. }

\par{37. ${\overset{\textnormal{そうおん}}{\text{騒音}}}$ が次第に消えてゆく。 \hfill\break
Noise will gradually fade away. }
    