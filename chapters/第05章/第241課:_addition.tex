    
\chapter{Addition}

\begin{center}
\begin{Large}
第241課: Addition: ~に加えて, ~にとどまらず, ~もさることながら, ~はおろか, ~はもちろん, ~はもとより, ~ともあれ 
\end{Large}
\end{center}
 
\par{ In this lesson we will learn about somewhat negative speech modals of end result. }
      
\section{~に加えて}
 
\par{ 加える means anything related to “to add”. So, ~に加えて means "in addition to". It is used a lot in enumerating negative situations. }

\par{1. 子供は指を ${\overset{\textnormal{くわ}}{\text{銜}}}$ える嫌いがある。 \hfill\break
Children have the tendency to have their fingers in their mouth. }

\par{\textbf{Orthography Note }: 銜 is rare. If anything, it would be replaced by 咥, but it's highly unlikely that you will see the word written in 漢字. }

\par{2. 彼女は日本語に加えて、英語もあまり分からないみたいね。 \hfill\break
In addition to Japanese, it looks like she barely understands English. }

\par{3. 料理に ${\overset{\textnormal{こしょう}}{\text{胡椒}}}$ は一つまみも加える必要はない。 \hfill\break
It's not necessary to add a pinch of pepper to a dish. }

\par{4. お茶に砂糖を加える。 \hfill\break
To mix sugar in the tea. }

\par{5. イギリスに加えて、フランスもドイツも多文化主義の非を鳴らしています。 \hfill\break
In addition to England, France and Germany are also denouncing multiculturalism. }

\par{7. 圧力を加える。 \hfill\break
To put pressure on. }

\par{8. そのことに加えて、彼は自分の名前を忘れたよ! \hfill\break
In addition to that, he also forgot his own name! }

\par{9. 5日のニューヨーク ${\overset{\textnormal{かぶしきしじょう}}{\text{株式市場}}}$ は、東京市場で株価が ${\overset{\textnormal{おおはば}}{\text{大幅}}}$ に値下がりしたこと \textbf{に加えて }、アメリカが景気を ${\overset{\textnormal{したざさ}}{\text{下支}}}$ えするために行っている ${\overset{\textnormal{りょうてきかんわ}}{\text{量的緩和}}}$ の ${\overset{\textnormal{しゅくしょう}}{\text{縮小}}}$ 時期など、アメリカの ${\overset{\textnormal{きんゆう}}{\text{金融}}}$ 政策に対する不安から幅広い ${\overset{\textnormal{めいがら}}{\text{銘柄}}}$ に売り注文が広がりました。 \hfill\break
As for the New York Stock Exchange on the fifth, selling orders have spread in a wide range of brands from worries towards American financial policies such as the quantitative easing curtailment period America is carrying out in order to backup the economy in addition to the large drop in prices due to the Tokyo Exchange. \hfill\break
From NHK. }
      
\section{~にとどまらず}
 
\par{ This speech modal means that X does not stop in a certain parameter, but extends farther. This pattern can be seen after nouns and the 終止形 of verbs. It comes after a phrase that shows a phenomenon or extent that is limited in some way, and then it is followed by a phrase with a larger scale implied or explicitly stated. This phrase should not be used with static expressions. }

\par{10. マスメディアによる情報というものは、今や一国にとどまらず、世界中に伝わる。 \hfill\break
Information by the media doesn't stay in one country now; it travels the world. }

\par{11. 農作物は、台風に襲われた直後にとどまらず、一年中その影響を受ける。 \hfill\break
Crops don't just receive the effects directly after being hit by a typhoon; they feel the impact throughout the year. }

\par{12. 一人の人間の明るさは、場を明るくするにとどまらず、周囲の人々に心身の活力をも与える。 \hfill\break
The brightness of a person not only brightens up the area, but also it gives energy to the hearts and minds of the surrounding people. }

\par{13. 彼女は成功にとどまらず、社会貢献に尽力した。 \hfill\break
She not only stopped at success, but she also labored as a contribution to society. }

\par{14. 更に執刀した岡山大学病院の大藤剛宏医師は手術後の記者会見で「中葉を使った移植は男の子1人を助けるにとどまらず、これまで助けることができなかった子どもたちに光が当たるという意義がある」と話しました。 \hfill\break
In addition, Dc. Oto Takahiro of Okayama University Hospital who did the surgery in a press conference afterwards said that "the transplant using the middle lobe [of the lung] doesn't just save this one boy, but it gives light to other children that have not been able to be helped till now". \hfill\break
From the NHK article 生体肺移植の男児 容体安定 by 2013年7月2日 4時15分. }
      
\section{~もさることながら}
 
\par{  ~もさることながら gives a meaning that something is so, but in addition, something else needs to be emphasized. Thus, it is equivalent to, "it goes without saying that\dothyp{}\dothyp{}\dothyp{}". It is used after nouns. }

\par{15. ${\overset{\textnormal{ちゅうかりょうり}}{\text{中華料理}}}$ は ${\overset{\textnormal{あじ}}{\text{味}}}$ もさることながら、 ${\overset{\textnormal{けんこう}}{\text{健康}}}$ にいいですよ。 \hfill\break
It goes without saying that Chinese food is tasty, but it's also good for you. }

\par{16. 彼女は人柄もさることながら、その頭の働きの良さで周囲の人をぐいぐいと引っ張っていく。 \hfill\break
Her personality goes with saying, and her great mind pulls those around her [towards her]. }
${\overset{\textnormal{ちゅうかりょうり}}{\text{中華料理}}}$ は ${\overset{\textnormal{あじ}}{\text{味}}}$ もさることながら、 ${\overset{\textnormal{けんこう}}{\text{健康}}}$ にいいですよ。 \hfill\break
It goes without saying that Chinese food is tasty, but it's also good for you. \hfill\break
      
\section{~おろか・もちろん・もとより}
 
\par{ ~はおろか・もちろん・もとより, mean "not to mention", "much less", or "let alone" and are placed after a sentence fragment. }

\par{17. 彼女は料理はもとより食器も洗う。 \hfill\break
She not only cooks, but she also washes the dishes. }

\par{18. 一ドルはおろか一セントも持ってないよ。 \hfill\break
I don't have a cent, let alone a dollar! }

\par{19. 彼は英語はもとより、スペイン語も韓国語もできます。 \hfill\break
Not to mention English, he can also speak Spanish and Korean. }

\par{20. 彼は英語はもちろん、スペイン語も韓国語もできます。 \hfill\break
Not to mention English, he can also speak Spanish and Korean. }

\par{21. 低賃金、劣悪な労働条件、一方的な首切りなどの経済・労働問題はもとより、各種の人道的問題までも引き起こしている。 \hfill\break
They are not only suffering from low wages, cases of poor working condition, and economic and labor problems such as one-sided layoffs, they are also troubled to the extent of various forms of discrimination. }

\par{22. あの子は書くのはもちろん、読むこともできない。 \hfill\break
That kid can't even read, let alone write. }

\par{23. 地方ごとに、味はもとより、料理の仕方も違ってきますね。 \hfill\break
With each region, not to mention flavor, but even cooking methods differ, don't they? }
24. 父はおろか、僕の犬でさえ反対したようだ。 \hfill\break
Not to mention my father, even my dog seemed to object. 
\par{25. 私はバイオリンはおろか、ギターも弾けない。 \hfill\break
I can't even play the guitar, much less a violet. }

\par{26. 母語はもちろん、彼は七つの他の言語も話せます。 \hfill\break
Not only his native language, but he can also speak seven other languages. }

\par{27. 彼女は話すのはおろか、足も動かせません。 \hfill\break
She can't even move her legs, let alone speak. }

\par{28. 彼は修士を得るのはもちろん、入学さえできない。 \hfill\break
He can't even enter college, let alone receive a master's degree. }

\par{\textbf{Word Notes }: もちろん may be translated as "not only\dothyp{}\dothyp{}\dothyp{}but also\dothyp{}\dothyp{}\dothyp{}" and may be an interjection meaning "of course" when by itself. もとより may be written as 元より, 固より, or 素より. }
      
\section{~ともあれ}
 
\par{ ~ともあれ is an adverbial phrase that means "never mind\slash putting aside". It is also seen in 何はともあれ meaning "in any case". }

\par{29. 中国語はともあれ、問題は日本語の勉強だぞ。 \hfill\break
Never mind Chinese, the problem is my Japanese studies! }

\par{30. 何はともあれ、後でしよう。 \hfill\break
In any case, let's do it later. }
    