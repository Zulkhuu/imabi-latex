    
\chapter{Whenever II}

\begin{center}
\begin{Large}
第216課: Whenever II: ~度に, ~都度, \& ~につけ 
\end{Large}
\end{center}
 
\par{ This lesson will discuss slightly more advanced phrases that equate to "whenever". }
      
\section{~度に}
 
\par{ ~度(毎)に means "whenever". 度 is the native word\slash counter for the number of times, making it common in set phrases such as 度を重ねる (to continue repeat itself). Grammatically speaking, it is after either nouns or the non-past form of a verb. What it follows is either a definite integer of place, time, action, or activity. This X is the trigger for Y, which without doubt will occur. Thus, whenever you say “whenever” with ~度に, you are saying that Y will 100% happen. }

\par{ Now, the ごと is important to keep in mind. In the past, it was not uncommon to see ~ごとに after verbs. Now, it causes the sentence to become unnatural. So, as an effect, ~度毎に is also typically only found with nouns. }

\par{ \textbf{Examples }}

\par{1. ${\overset{\textnormal{いくど}}{\text{幾度}}}$ も ${\overset{\textnormal{しっぱい}}{\text{失敗}}}$ を ${\overset{\textnormal{く}}{\text{繰}}}$ り返すのはつらいですね。 \hfill\break
Repeating a failure countless times is painful, isn't it? }

\par{2. ${\overset{\textnormal{なにごと}}{\text{何事}}}$ も ${\overset{\textnormal{たび}}{\text{度}}}$ を ${\overset{\textnormal{かさ}}{\text{重}}}$ ねれば ${\overset{\textnormal{かなら}}{\text{必}}}$ ずや ${\overset{\textnormal{じょうたつ}}{\text{上達}}}$ する。 \hfill\break
If you repeat anything all the time, you will certainly improve. }

\par{3. ${\overset{\textnormal{きかい}}{\text{機会}}}$ がある度に \hfill\break
Whenever you get a chance }

\par{4. ${\overset{\textnormal{おれ}}{\text{俺}}}$ はやる度に失敗しちまうんだよ。(Casual; masculine) \hfill\break
Whenever I do it, I end up failing (which is why I won't do it) }

\par{\textbf{Grammar Notes }: }

\par{1. This makes it very similar to ~につけ, which will be discussed in this lesson. However, if you interchange it for ~度に, you speak objectively and the emotional feel is lost entirely. }

\par{2. There is some similarity between ~ごとに and ~たびに. ~ごとに shows that when there is a chance, a habitual action is simply repeated. ~たびに, despite also showing same habitual action, when one encounters a \emph{particular }opportunity, it gives a sense that it is not incidental. Thus, even if the action is special but the timing is not, you can\textquotesingle t use ~たびに. }
      
\section{~都度}
 
\par{ The most appropriate translation of 都度 is "whenever". Like 度, it must be used with の when with another nominal phrase, and it may also be accompanied by に. What is neat about 都度 is that it is often seen in the pattern その都度. }

\par{都度 is less emphatic and less likely to be in more personal situations. It is very similar "each time it happens". }

\par{ \textbf{Examples }}

\par{5. ${\overset{\textnormal{とりひき}}{\text{取引}}}$ のつど ${\overset{\textnormal{ざんだからん}}{\text{残高欄}}}$ に残高 ${\overset{\textnormal{きんがく}}{\text{金額}}}$ が示されるようにしたものです。 \hfill\break
Whenever it is exchanged, the balance amount is made shown in the bank balance column. }

\par{6. 彼氏に会うその ${\overset{\textnormal{つど}}{\text{都度}}}$ 、彼女はとても ${\overset{\textnormal{うれ}}{\text{嬉}}}$ しくなるようです。 \hfill\break
It seems that she becomes very happy whenever she meets her boyfriend. }

\par{7. 夫の両親が来る時はその都度 ${\overset{\textnormal{けんか}}{\text{喧嘩}}}$ が始まる。 \hfill\break
Whenever my husband's parents come, an argument gets started (with them). }
      
\section{~につけ}
 
\par{ Following nouns or the 連体形 of verbs or adjectives, ~につけ means "each time when\slash both\dothyp{}\dothyp{}\dothyp{}and\dothyp{}\dothyp{}\dothyp{}". It is mainly used in the written language, but it is also used in formal speaking situations. What comes before it\slash what it follows is a constant that causes a change. In other words, if there is a situation X, then no matter what, along with X, Y reflexively occurs, and at the same time, a psychological change is brought about. }

\par{In the case that Y describes a past event, X is the trigger in remembering Y, and a sense of emotion suitable to that memory is expressed. This pattern is often after verbs such as 見る, 聞く, and 思う. The pattern is then often followed by phrase that captures the emotion the speaker wishes to portray. }

\par{ ~につけて is naturally more suitable for the spoken language, but when used in examples like the first one and others such as 雨につけ風につけ, ~につけて is impossible. This is due to the antithetical element. }

\par{There is also  ~につけても. This adds the particle も, which in this situation adds the function of analogy. So, aside from A, it is used to show various changes of B happening. }

\par{ \textbf{Examples }}

\par{8. 喜びにつけ悲しみにつけ \hfill\break
Both in joy and in sorrow }

\par{9. あいつを見るにつけ、兄を思い出させる。 \hfill\break
Each time I see him, he reminds me of my older brother. }

\par{10. 試験をもう一度失敗するにつけ、 ${\overset{\textnormal{ゆううつ}}{\text{憂鬱}}}$ が ${\overset{\textnormal{よみがえ}}{\text{蘇}}}$ る。 \hfill\break
Each time when I fail the exam once more, my depression comes back. }

\par{11. 何事につけても、喜んでお手伝いします。 \hfill\break
Whatever you do, we're ready to gladly help you. }

\par{12. 年のせいなんだろう、暑いにつけ、寒いにつけ、よく風邪を引くようになった。 \hfill\break
I wonder if it's because of my age; both when hot and when cold, I have become more prone to having colds. }

\par{13. この写真を見るにつけ、いつも思い出すのは、あの決して美しくはなかった青春時代のことである。(Written) \hfill\break
Each time I see this photo, I always remember my completely ugly youth. }
    