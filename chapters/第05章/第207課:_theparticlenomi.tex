    
\chapter{The Particles のみ}

\begin{center}
\begin{Large}
第207課: The Particles のみ  
\end{Large}
\end{center}
 
\par{ The particle のみ is very similar to だけ. One of the main things to keep in mind is that it is not のに. }
      
\section{The Adverbial Particle のみ}
 
\par{ のみ has fallen into more so literary usage due to the rise of だけ. のみ has four distinct usages. }

\begin{itemize}

\item Limits something to a certain amount. 
\item のみだ shows that at \emph{that }much, there is no other situation. 
\item のみか means "not just that". 
\item のみならず means "not only". 
\end{itemize}

\begin{center}
 \textbf{Examples }
\end{center}

\par{1. この ${\overset{\textnormal{}}{\text{村}}}$ のみが ${\overset{\textnormal{}}{\text{水害}}}$ に\{免れた・ ${\overset{\textnormal{あ}}{\text{遭}}}$ わなかった\}。 \hfill\break
Only this village escaped the damages from the flood. }
 
\par{${\overset{\textnormal{}}{\text{2. 神}}}$ のみぞ ${\overset{\textnormal{}}{\text{知}}}$ る。 \hfill\break
Only God knows. }
 
\par{3. 我々はただ ${\overset{\textnormal{にんたい}}{\text{忍耐}}}$ があるのみだ。 \hfill\break
We only have perseverance. }

\par{${\overset{\textnormal{}}{\text{4. 入場}}}$ は ${\overset{\textnormal{}}{\text{招待者}}}$ のみ可 \hfill\break
Admission by invitation only }
 
\par{${\overset{\textnormal{}}{\text{5. 老兵}}}$ は ${\overset{\textnormal{}}{\text{死}}}$ なずただ ${\overset{\textnormal{}}{\text{消}}}$ え ${\overset{\textnormal{}}{\text{去}}}$ るのみ。 \hfill\break
Old soldiers never die, they just fade away. }
 
\par{${\overset{\textnormal{}}{\text{6. 富士山}}}$ はただ ${\overset{\textnormal{}}{\text{高}}}$ いのみならず、 ${\overset{\textnormal{}}{\text{美}}}$ しいのです。 \hfill\break
Mount Fuji is not just tall, but it is also beautiful. }

\par{7. ${\overset{\textnormal{けんじゅつ}}{\text{剣術}}}$ のみか ${\overset{\textnormal{せこ}}{\text{世故}}}$ にも ${\overset{\textnormal{た}}{\text{長}}}$ け \hfill\break
Not only know about fencing, but also know about worldly affairs. \hfill\break
From 池波正太郎 }
 
\par{8a. この ${\overset{\textnormal{}}{\text{水族館}}}$ には、 ${\overset{\textnormal{}}{\text{深海魚}}}$ のみならず、 ${\overset{\textnormal{}}{\text{熱帯魚}}}$ もいます。(A little old-fashioned) \hfill\break
8b. この ${\overset{\textnormal{}}{\text{水族館}}}$ には、 ${\overset{\textnormal{}}{\text{深海魚}}}$ だけでなく、 ${\overset{\textnormal{}}{\text{熱帯魚}}}$ もいます。(More natural) \hfill\break
In this aquarium, there are not only deep-sea fish but also tropical fish. }
 
\par{\textbf{Grammar Note }: Using だけでなく is much more common because, again, のみ is very literary. }

\par{9. ${\overset{\textnormal{きんきゅう}}{\text{緊急}}}$ の ${\overset{\textnormal{}}{\text{場合}}}$ にのみ ${\overset{\textnormal{}}{\text{連絡}}}$ せよ。 \hfill\break
Call me in only in the case of an emergency. }

\par{10. ${\overset{\textnormal{みずぶそく}}{\text{水不足}}}$ はいよいよ ${\overset{\textnormal{}}{\text{深刻}}}$ だ。 ${\overset{\textnormal{}}{\text{後}}}$ はただ ${\overset{\textnormal{}}{\text{雨}}}$ が ${\overset{\textnormal{}}{\text{降}}}$ るのを ${\overset{\textnormal{}}{\text{待}}}$ つのみだ。 \hfill\break
The water shortage is really serious. The only thing to do is wait for it to rain. }
    