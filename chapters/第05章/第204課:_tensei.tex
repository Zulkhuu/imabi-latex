    
\chapter{Tense I}

\begin{center}
\begin{Large}
第204課: Tense I: The Auxiliary Verb -TA 
\end{Large}
\end{center}
 
\par{ In its basic understanding, the auxiliary verb -TA stands for the past tense, especially when it is in a mono-clausal sentence (a sentence with a single clause) in isolation from other grammatical circumstances. This can be demonstrated with the following examples. }

\par{1a. ${\overset{\textnormal{わたし}}{\text{私}}}$ はきのう、 ${\overset{\textnormal{とざん}}{\text{登山}}}$ をしました。〇 \hfill\break
1b. ${\overset{\textnormal{わたし}}{\text{私}}}$ はきのう、 ${\overset{\textnormal{とざん}}{\text{登山}}}$ をします。X \hfill\break
I mountain climbed yesterday. }

\par{2a. ${\overset{\textnormal{こま}}{\text{細}}}$ かいことは ${\overset{\textnormal{あすはっぴょう}}{\text{明日発表}}}$ します。〇 \hfill\break
2b. ${\overset{\textnormal{こま}}{\text{細}}}$ かいことは ${\overset{\textnormal{あすはっぴょう}}{\text{明日発表}}}$ しました。X \hfill\break
I will announce the fine details tomorrow. }

\par{ These sentences demonstrate a basic understanding of what non-past and past tense are in Japanese, but they ignore the other varied nuances and situations that both forms can represent. This is because mono-clausal sentences like these only make up a small percentage of the complexity that can be found in the language. }

\par{ Whenever you read a book or listen to speakers talk, you will notice that -TA and -RU\slash U (the morphemes used in the ‘non-past\textquotesingle  form) alternate from one to the other, and you\textquotesingle ll also notice that they often deviate far away from the concept of tense-agreement found in English. One grammatical circumstance that negates the notion that these endings are fixated to one tense is the creation of conditionals. }

\par{3. ドアを ${\overset{\textnormal{あ}}{\text{開}}}$ けたら、 ${\overset{\textnormal{かなら}}{\text{必}}}$ ず ${\overset{\textnormal{し}}{\text{閉}}}$ めてください。 \hfill\break
If you open the door, please make sure to close it. \hfill\break
 \hfill\break
4. ドアを ${\overset{\textnormal{あ}}{\text{開}}}$ けると、コウモリが ${\overset{\textnormal{はい}}{\text{入}}}$ ってきた。 \hfill\break
When I opened the door, a bat came inside. }

\par{ Seeing as how the tense in both the dependent clauses of 3 and 4 are the same in English, the fact that there is this dichotomy between -TA and -RU\slash U is perplexing. Never will you see 開けるら, and never will you see 開けたと. One way to explain this is that Japanese makes a distinction between foreground and background circumstances. When something happens\slash is so in the background, it usually or must take -RU\slash U. This means that for 4, the door opening is a pretext for the bat having entered the home. The door is presumably open when the bat enters, and so that clause takes -RU. The bat entering is at the foreground of the sentence and thus takes -TA. In 3, the action of having opened the door is at the foreground when you get to closing it back shut. Therefore, it takes -TA. These examples demonstrate a need to analyze these endings far more closely. }

\par{ First, we will study the individual usages of -TA. Then, in the next lesson we\textquotesingle ll study the individual usages of -RU\slash U. By doing so, you'll see how they both heavily reflect the speaker\textquotesingle s intent rather than follow concrete rules, which inevitably means that context must be looked at. Instead of having the decision between the two be mechanical based on a handful of criteria, deciding between the two often relies on a feel for the situation--pragmatics. As we study these forms, think about the dynamics that could change how you interpret them--time, voice (narrator versus self), state, etc. }

\par{\textbf{Notation Note }: For this lesson and the one that follows, the forms -RU\slash U, -TA, -TE IRU, and -TE ITA will be Romanized as such due to the technical nature of the discussion at hand. Although this lesson focuses on -TA, it also makes note usages of these other forms due to their similarities. }

\par{\textbf{Terminology Note }: }

\par{1. A morpheme is a meaningful unit of language that cannot be further divided. \hfill\break
2. The copula is also treated as being grammatically equal to -RU\slash U. All references to -RU\slash U, therefore, will also apply to it. The non-past forms of adjectives also count as being in the -RU\slash U form. }
      
\section{Etymology of -TA}
 
\par{ The auxiliary verb -TA comes from the conjunctive particle てcombined with the existential verb あり (Modern ある). At this time, it took the form たり and served to either show completion or continuation. Meaning, it was equivalent to either てしまった or てある・ている. Both these two meanings survive in their respective capacities in Modern Japanese. The large reason for why -TA has taken on so many meanings was that there were several other endings for aspect that collapsed into -TA over time. }

\par{ -TA attaches to the 連用形 of conjugatable parts of speech. The bases of -TA—both classical and modern—are shown below: }

\begin{ltabulary}{|P|P|P|}
\hline 

The Bases 活用形 & Classical 古文 & Modern 現代語 \\ \cline{1-3}

Irrealis 未然形 & たら & たろ \\ \cline{1-3}

Continuative 連用形 & たり & たり \\ \cline{1-3}

Terminal 終止形 & たり & た \\ \cline{1-3}

Attributive 連体形 & たる & た \\ \cline{1-3}

Realis 已然形 \hfill\break
Hypothetical 仮定形 & たれ & たら \\ \cline{1-3}

Imperative 命令形 & たれ & た \\ \cline{1-3}

\end{ltabulary}

\par{ As implied by the chart, the particle たり is the continuative form ( ${\overset{\textnormal{れんようけい}}{\text{連用形}}}$ ) of -TA. Also, the particle たら is treated as the hypothetical form ( ${\overset{\textnormal{かていけい}}{\text{仮定形}}}$ ) of -TA. Additionally, note how -TA has an imperative form. This will be one of the many usages discussed. }
      
\section{Usages of -TA}
 
\par{1 . The most basic meaning of -TA is to demonstrate a situation that has happened ( \textbf{past tense) }or has been completed ( \textbf{perfect tense }) in the past. The nature of the act\slash event must be assessed by the context at hand to determine whether -TA denotes a past and\slash or perfect tense interpretation. }

\par{1a. In English, the \textbf{past tense }is made by adding –(e)d to the base form of the verb. Its function is to talk about the past. The “ \textbf{past simple }” form is simply the verb + -(e)d and nothing more. The “ \textbf{past continuous\slash progressive }” form, which denotes a continuing action which began in the past, involves the pattern “was + verb + -ing.” }

\par{i. I worked. \textrightarrow  ${\overset{\textnormal{わたし}}{\text{私}}}$ は ${\overset{\textnormal{はたら}}{\text{働}}}$ きました。(Past Simple) \hfill\break
ii. I was working. \textrightarrow  ${\overset{\textnormal{わたし}}{\text{私}}}$ は ${\overset{\textnormal{はたら}}{\text{働}}}$ いていました。 (Past Continuous) }

\par{ Both -TA and -TE ITA can indicate past events\slash states whose duration was long. However, -TA does not explicitly denote duration of a past event. }

\par{5. ${\overset{\textnormal{せいと}}{\text{生徒}}}$ たちが ${\overset{\textnormal{じっけん}}{\text{実験}}}$ を ${\overset{\textnormal{おこな}}{\text{行}}}$ いました。 (Past Simple) \hfill\break
The students \textbf{conducted }an experiment. \hfill\break
 \hfill\break
6. ${\overset{\textnormal{おおあめこうずいけいほう}}{\text{大雨洪水警報}}}$ が ${\overset{\textnormal{はっぴょう}}{\text{発表}}}$ された。(Past Simple + Passive) \hfill\break
A storm-flood warning \textbf{was announced }. }

\par{7. ${\overset{\textnormal{わたし}}{\text{私}}}$ も ${\overset{\textnormal{はな}}{\text{話}}}$ していました。 (Past Continuous\slash Progressive) \hfill\break
I \textbf{was }also \textbf{speaking }. }

\par{8a. ${\overset{\textnormal{かれ}}{\text{彼}}}$ が ${\overset{\textnormal{せんそう}}{\text{戦争}}}$ に ${\overset{\textnormal{たい}}{\text{対}}}$ する ${\overset{\textnormal{おも}}{\text{思}}}$ いを ${\overset{\textnormal{かた}}{\text{語}}}$ った。 \hfill\break
8b. 彼が戦争に対する思いを語っていました。 \hfill\break
He \textbf{spoke of }his thoughts towards war. (9a) \hfill\break
He \textbf{was speaking of }his thoughts towards war. (9b) }

\par{1b. The perfect tenses are created by using the appropriate tense of the auxiliary verb “to have”—or “got”—plus the past participle of a verb. }

\par{iii.  I have worked.  \textrightarrow  ${\overset{\textnormal{はたら}}{\text{働}}}$ いたことがあります。(Present Perfect) \hfill\break
iv. I had worked. \textrightarrow  (かつて) ${\overset{\textnormal{はたら}}{\text{働}}}$ いていました。(Past Perfect\slash Pluperfect) \hfill\break
v. I will have worked. \textrightarrow  それまでに ${\overset{\textnormal{はたら}}{\text{働}}}$ いたでしょう。(Future Perfect) }

\par{ The \textbf{present perfect }is used to denote a past event that has occurred and been completed in the past yet has consequences in the present. In iii., the present consequence is that one has a work history. Similarly, the \textbf{present progressive perfect }is used to denote an action that has occurred up to the present and may continue. }

\par{9. アメリカ ${\overset{\textnormal{せいふ}}{\text{政府}}}$ が ${\overset{\textnormal{きたちょうせん}}{\text{北朝鮮}}}$ に ${\overset{\textnormal{たい}}{\text{対}}}$ して ${\overset{\textnormal{ちょうはつこうい}}{\text{挑発行為}}}$ をやめるよう ${\overset{\textnormal{あらた}}{\text{改}}}$ めて求めた。(Present Perfect) \hfill\break
The American government \textbf{has }once again \textbf{requested }that North Korea stop its provocations. }

\par{10. ${\overset{\textnormal{うけとりにん}}{\text{受取人}}}$ が ${\overset{\textnormal{すで}}{\text{既}}}$ に ${\overset{\textnormal{しぼう}}{\text{死亡}}}$ していた ${\overset{\textnormal{ばあい}}{\text{場合}}}$ (Present Perfect) \hfill\break
In the case the recipient \textbf{has already died\slash is already deceased }}

\par{11. 彼はその ${\overset{\textnormal{ひみつ}}{\text{秘密}}}$ をずっと ${\overset{\textnormal{かく}}{\text{隠}}}$ していた! (Present Progressive Perfect) \hfill\break
He \textbf{has been hiding }that secret the whole time! }

\par{12 うまく ${\overset{\textnormal{せいちょう}}{\text{成長}}}$ したね。 \hfill\break
You \textbf{\textquotesingle ve }sure \textbf{grown }, huh. }

\par{13. ${\overset{\textnormal{かのじょ}}{\text{彼女}}}$ は ${\overset{\textnormal{せ}}{\text{背}}}$ が ${\overset{\textnormal{たか}}{\text{高}}}$ くなった。 (Simple Past\slash Present Perfect) \hfill\break
She \textbf{became tall }\slash  \textbf{has become tall }. }

\par{14.いつの ${\overset{\textnormal{ま}}{\text{間}}}$ にか ${\overset{\textnormal{こえ}}{\text{声}}}$ が ${\overset{\textnormal{ひく}}{\text{低}}}$ くなった。(Present Perfect) \hfill\break
My voice \textbf{got deep }before I knew it. }

\par{ -TE IRU may also express the present perfect if the state the action brings about is ongoing. This, though, can and is most often expressed with -TA when modifying nouns. }

\par{15. ${\overset{\textnormal{ていでん}}{\text{停電}}}$ が ${\overset{\textnormal{ひろ}}{\text{広}}}$ い ${\overset{\textnormal{はんい}}{\text{範囲}}}$ で ${\overset{\textnormal{はっせい}}{\text{発生}}}$ している。 \hfill\break
A blackout \textbf{has occurred }over a wide extent ( \textbf{and is still ongoing }). }

\par{16. ${\overset{\textnormal{ふしぎ}}{\text{不思議}}}$ な ${\overset{\textnormal{かたち}}{\text{形}}}$ をした ${\overset{\textnormal{たてもの}}{\text{建物}}}$ が ${\overset{\textnormal{なら}}{\text{並}}}$ んでいる。 \hfill\break
Buildings \textbf{in }mysterious shapes \textbf{are lined up }. }

\par{ The \textbf{past perfect\slash pluperfect }is used to denote a past event that had already been completed prior to a point in time which is referred to. In iv., the point in time when the speaker had worked is prior to “now” or an unspecified “then.” Although iv. utilizes -TE ITA, the past perfect can still be reworded to use -TA like in Ex. 17. }

\par{17. ${\overset{\textnormal{かこ}}{\text{過去}}}$ に ${\overset{\textnormal{いっかい}}{\text{一回}}}$ は ${\overset{\textnormal{はたら}}{\text{働}}}$ いたことがあります。 \hfill\break
I \textbf{had worked }once in the past. }

\par{18. ${\overset{\textnormal{の}}{\text{乗}}}$ っていた3 ${\overset{\textnormal{にん}}{\text{人}}}$ が ${\overset{\textnormal{しぼう}}{\text{死亡}}}$ した。(Pluperfect\slash Past Continuous) \hfill\break
The three people who \textbf{were\slash had been riding }died. }

\par{19. ${\overset{\textnormal{あんぜん}}{\text{安全}}}$ バーの ${\overset{\textnormal{てつせい}}{\text{鉄製}}}$ の ${\overset{\textnormal{ぼう}}{\text{棒}}}$ が ${\overset{\textnormal{ねもとふきん}}{\text{根元付近}}}$ で ${\overset{\textnormal{お}}{\text{折}}}$ れていたことが ${\overset{\textnormal{わ}}{\text{分}}}$ かりました。 (Pluperfect\slash Present Perfect) \hfill\break
It has been discovered that the metal rod near the base of the safety bar \textbf{had\slash has been broken }. }

\par{\textbf{Sentence Note }: Because it is likely that the rod is still not fixed as it is piece of evidence, the present perfect interpretation is also valid. }

\par{ The \textbf{future perfect }is a verb form used to describe an event that is expected\slash planned to happen before a future point in time. This is usually denoted by -TA + でしょう, but the form is not limited to this for the future perfect (Ex. 20). The \textbf{future progressive perfect }also exists with -TE IRU, which indicates an event that is expected\slash planned to happen in the future whose effects are forecast to continue even after the event realizes. }

\par{20. ${\overset{\textnormal{けんた}}{\text{健太}}}$ が ${\overset{\textnormal{けっこん}}{\text{結婚}}}$ した ${\overset{\textnormal{じぶん}}{\text{自分}}}$ を ${\overset{\textnormal{そうぞう}}{\text{想像}}}$ してみた。 \hfill\break
Kenta imagined himself as a \textbf{married }man. }

\par{2. Utilizing the past tense (in English translation), -TA may denote a \textbf{situation that occurred under certain circumstances in the past }. The circumstance may in fact be that it was in the past that one “ \textbf{would\slash used to }” do something. With states, this would be interpreted as a repetitive situation in the past. With actions, this would be interpreted as habitual repetition. }

\par{21. あの ${\overset{\textnormal{ころ}}{\text{頃}}}$ は ${\overset{\textnormal{しあわ}}{\text{幸}}}$ せだった。 \hfill\break
I used to be happy in those days. }

\par{22. あの ${\overset{\textnormal{ころ}}{\text{頃}}}$ は ${\overset{\textnormal{よ}}{\text{良}}}$ かった。 \hfill\break
Those were the good old days. }

\par{23. かつてあの ${\overset{\textnormal{みせ}}{\text{店}}}$ に ${\overset{\textnormal{い}}{\text{行}}}$ きました。 \hfill\break
I once would\slash used to go to that store. }

\par{24a. ${\overset{\textnormal{わか}}{\text{若}}}$ い ${\overset{\textnormal{ころ}}{\text{頃}}}$ 、 ${\overset{\textnormal{まち}}{\text{町}}}$ の ${\overset{\textnormal{ぼち}}{\text{墓地}}}$ をよく ${\overset{\textnormal{ある}}{\text{歩}}}$ いたものだ。 \hfill\break
24b. ${\overset{\textnormal{わか}}{\text{若}}}$ い ${\overset{\textnormal{ころ}}{\text{頃}}}$ 、 ${\overset{\textnormal{まち}}{\text{町}}}$ の ${\overset{\textnormal{ぼち}}{\text{墓地}}}$ をよく ${\overset{\textnormal{ある}}{\text{歩}}}$ いたものだった。 \hfill\break
I frequently used to walk the town\textquotesingle s cemetery when I was young. (24a) \hfill\break
He\slash she used to walk the town\textquotesingle s cemetery when he\slash she was young. (24b) }

\par{\textbf{Grammar Note }: “Used to” is frequently expressed with a verb in the -TA form while preceded by the adverb よく and followed by ものだ. The addition of ものだ gives a matter-of-fact nuance to the statement. The matter-of-fact being so in one\textquotesingle s interpretation of the past calls for the copula to be in the non-past form, but when switching from first person to third person, ものだった is used. This is because another person\textquotesingle s past situation cannot be stated at the same emotional level as would be the case with ものだ. }

\par{25. あの ${\overset{\textnormal{ころ}}{\text{頃}}}$ は、よく ${\overset{\textnormal{み}}{\text{見}}}$ に ${\overset{\textnormal{い}}{\text{行}}}$ っていました。 \hfill\break
Those days, I used to always go to see it. }

\par{\textbf{Grammar Note }: The use of ものだ is not a requirement to bring about the meaning of “used to” as has been shown to be the case in Ex. 22. In this example, -TE ITA is demonstrated to also hold this function without the intervention of ものだ. The difference is that the “matter-of-fact” nuance is lost. Translation-wise, the use of -TE ITA in conjunction with よく brings about a non-literal interpretation of “always.” }

\par{26. ${\overset{\textnormal{がっこう}}{\text{学校}}}$ から ${\overset{\textnormal{かえ}}{\text{帰}}}$ るとすぐに ${\overset{\textnormal{しゅくだい}}{\text{宿題}}}$ を ${\overset{\textnormal{す}}{\text{済}}}$ ませた。 \hfill\break
I would immediately finish my homework when I got home from school. }

\par{27. ${\overset{\textnormal{は}}{\text{歯}}}$ を ${\overset{\textnormal{みが}}{\text{磨}}}$ くとよく ${\overset{\textnormal{しゅっけつ}}{\text{出血}}}$ しました。 \hfill\break
I would often bleed when I brushed my teeth. }

\par{28. ジムに ${\overset{\textnormal{い}}{\text{行}}}$ くとよく ${\overset{\textnormal{こえ}}{\text{声}}}$ をかけられました。 \hfill\break
I would often be called out by someone when I went to the gym. }

\par{29. ${\overset{\textnormal{か}}{\text{買}}}$ い ${\overset{\textnormal{もの}}{\text{物}}}$ デートに ${\overset{\textnormal{い}}{\text{行}}}$ くと、よく ${\overset{\textnormal{かのじょ}}{\text{彼女}}}$ と ${\overset{\textnormal{けんか}}{\text{喧嘩}}}$ してしまいました。 \hfill\break
I would often end up getting into arguments with my girlfriend when(ever) we went on shopping dates. }

\par{3. -TA may express a \textbf{sudden recalling of a future event\slash plan }that the speaker has already recognized as being definite. Many of the realizations we have in our daily lives involve things we already know about. Those facts may just be floating in the back of our minds until for some reason we remember about them and freak out. In the bigger picture, -TA can be viewed as denoting the point in time at which the speaker recognizes something to have already happened (Usages 1-2, 6-7) or recognize as being the case (Usages 3-5). }

\par{30. あ、 ${\overset{\textnormal{あしたしけん}}{\text{明日試験}}}$ があった! \hfill\break
Ah, I have an exam tomorrow! }

\par{31. ${\overset{\textnormal{あしたしごと}}{\text{明日仕事}}}$ だった。。。 \hfill\break
I have work tomorrow… }

\par{32. ${\overset{\textnormal{きょう}}{\text{今日}}}$ は ${\overset{\textnormal{しめきりび}}{\text{締切日}}}$ だった! \hfill\break
Today\textquotesingle s the deadline! }

\par{33. あ、 ${\overset{\textnormal{わす}}{\text{忘}}}$ れてた! \hfill\break
Oops, I\textquotesingle d forgotten! }

\par{4. In the same vein, -TA may also show \textbf{sudden discovery }of a certain state that has been and is still so. This form of discovery also implies that the speaker thinks he\slash she should have known about the discovery. In this sense, discovery is being treated semantically like remembering. This usage is also referred to as “Discovery Present” and is closely tied to present progressive perfect form. }

\par{34. ここだった! \hfill\break
It\textquotesingle s here! \hfill\break
So it was here! }

\par{35. え、 ${\overset{\textnormal{とうきょう}}{\text{東京}}}$ にいたの? \hfill\break
What, you\textquotesingle re in Tokyo? }

\par{36. あ、 ${\overset{\textnormal{ひ}}{\text{引}}}$ き ${\overset{\textnormal{だ}}{\text{出}}}$ しに\{あった・入ってた\}! \hfill\break
Ah, it\textquotesingle s in the drawer! \hfill\break
Ah, it was in the drawer! }

\par{37. あ、そこにいたの! \hfill\break
There you are! }

\par{38. そう、ここにいたよ。 \hfill\break
Yep, I\textquotesingle m here. \hfill\break
Yep, I was here along. }

\par{\textbf{Contrast Note }: If -RU\slash U were switched out in these sentences, they would still be grammatical. However, they would lose the indirect report feel because there would no longer be an implied admission that the speaker ought to have known. }

\par{5. Another usage of -TA is to request for \textbf{confirmation }of a fact from the listener. It may also be used in a rhetorical sense of confirmation. This usage is frequently employed in ${\overset{\textnormal{けいご}}{\text{敬語}}}$ to add another layer of formality when asking questions. It\textquotesingle s important to note that the use of a past tense marker for seeking confirmation can also be seen in English. }

\par{39. ${\overset{\textnormal{しつれい}}{\text{失礼}}}$ ですが、どちら ${\overset{\textnormal{さま}}{\text{様}}}$ でしたか。 \hfill\break
I apologize for asking, but who were you again? }

\par{40. お ${\overset{\textnormal{なまえ}}{\text{名前}}}$ はなんと仰いましたか。 \hfill\break
What was your name? }

\par{41. ${\overset{\textnormal{はんこ}}{\text{判子}}}$ をお ${\overset{\textnormal{も}}{\text{持}}}$ ちでしたね。 \hfill\break
You had your seal, correct? }

\par{42. ${\overset{\textnormal{どうせき}}{\text{同席}}}$ された ${\overset{\textnormal{かた}}{\text{方}}}$ はどなたでしたか。 \hfill\break
Who was it that you sat with? }

\par{43. ${\overset{\textnormal{かのじょ}}{\text{彼女}}}$ 、 ${\overset{\textnormal{なんさい}}{\text{何歳}}}$ だ(った)っけ。 \hfill\break
How old was she? }

\par{44. ${\overset{\textnormal{つぎ}}{\text{次}}}$ は ${\overset{\textnormal{わたし}}{\text{私}}}$ の ${\overset{\textnormal{ばん}}{\text{番}}}$ でしたか。 \hfill\break
Next turn was mine, was it. }

\par{45. ${\overset{\textnormal{きょう}}{\text{今日}}}$ は ${\overset{\textnormal{げつようび}}{\text{月曜日}}}$ でしたね。 \hfill\break
Today\textquotesingle s Monday, huh. }

\par{6. -TA may also be used to denote a \textbf{proclamation, assertion, and or realization of a situation that has not realized }; however, the situation is treated as if it has. This usage cannot be used with time expressions because it already has an implied sense of urgency for the statement to be taken seriously. }

\par{46. よし、 ${\overset{\textnormal{か}}{\text{買}}}$ った! \hfill\break
Alright, I\textquotesingle m buying it! }

\par{47. もうやめた。 \hfill\break
I quit. }

\par{48. もう ${\overset{\textnormal{あきら}}{\text{諦}}}$ めた! \hfill\break
I give up! }

\par{49. じゃあ、 ${\overset{\textnormal{たの}}{\text{頼}}}$ みましたよ。 \hfill\break
Alright, well, I\textquotesingle m counting on you. }

\par{50. よし、これで ${\overset{\textnormal{か}}{\text{勝}}}$ った! \hfill\break
Alright, I\textquotesingle ll win with this! }

\par{7. -TA may be used to express an \textbf{urgent\slash immediate command }. This is the imperative usage of -TA mentioned earlier. How this works grammatically is that the speaker presses the listener so much to do something immediately that it\textquotesingle s as if it\textquotesingle s already been completed. Though not literally possible, this reflects the hyperbolic origin of this grammar point. }

\par{51. どいた、どいた! \hfill\break
Step back, step back! }

\par{52. 子供は帰った、帰った! \hfill\break
Kids go home, go home! }

\par{53. ちょっと ${\overset{\textnormal{ま}}{\text{待}}}$ った! \hfill\break
Wait right there! }

\par{54. さあ、どんどん ${\overset{\textnormal{ある}}{\text{歩}}}$ いた ${\overset{\textnormal{ある}}{\text{歩}}}$ いた! \hfill\break
Well, get to walking! }

\par{55. ${\overset{\textnormal{すわ}}{\text{座}}}$ った、 ${\overset{\textnormal{すわ}}{\text{座}}}$ った! \hfill\break
Sit, sit! }
    