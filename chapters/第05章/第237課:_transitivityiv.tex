    
\chapter{Intransitive \& Transitive}

\begin{center}
\begin{Large}
第237課: Intransitive \& Transitive: Part 3 
\end{Large}
\end{center}
 
\par{ In this third lesson on verbs with both intransitive and transitive usages, we\textquotesingle ll continue to uncover peculiarities in Japanese at the individual word basis. }
      
\section{する, 増す, 働く, 引く}
 
\begin{center}
\textbf{する } 
\end{center}

\par{\emph{ }する is the most important verb in the Japanese language as we have already learned due to how many usages it has and how important those usages are to the entirety of the language. Unsurprisingly, its usages can be classified as either being intransitive or transitive in nature. }

\par{ As an intransitive verb, する can demonstrate a sense being sensed, in some state (often with onomatopoeia), show the worth of something (as in price), or the elapse of time (with time phrases). }

\par{1. アテモヤは ${\overset{\textnormal{ざとう}}{\text{砂糖}}}$ で ${\overset{\textnormal{につ}}{\text{煮詰}}}$ めたリンゴのような ${\overset{\textnormal{あじ}}{\text{味}}}$ がします。 \hfill\break
The atemoya has a flavor like that of an apple boiled down in sugar. }

\par{\textbf{Spelling Note }: \emph{Ringo }is seldom spelled as 林檎. }

\par{2. ${\overset{\textnormal{め}}{\text{目}}}$ がぐるぐると ${\overset{\textnormal{まわ}}{\text{回}}}$ るような ${\overset{\textnormal{めまい}}{\text{眩暈}}}$ がした。 \hfill\break
I got dizzy as if my eyes were spinning. }

\par{3. ${\overset{\textnormal{みみな}}{\text{耳鳴}}}$ りがするとお ${\overset{\textnormal{ば}}{\text{化}}}$ けが ${\overset{\textnormal{がわ}}{\text{側}}}$ にいるって ${\overset{\textnormal{ほんとう}}{\text{本当}}}$ ですか。 \hfill\break
Is it true that a ghost is next to you when your ears ring? }

\par{4. ${\overset{\textnormal{はいご}}{\text{背後}}}$ に ${\overset{\textnormal{きょうれつ}}{\text{強烈}}}$ な ${\overset{\textnormal{ばくはつおん}}{\text{爆発音}}}$ がした。 \hfill\break
There was the sound of an intense explosion in the background. }

\par{5. しばらくしてから ${\overset{\textnormal{い}}{\text{行}}}$ きましょう。 \hfill\break
Let\textquotesingle s go after a little while. }

\par{\textbf{Spelling Note }: しばらく is sometimes written as 暫く. }

\par{6. このネックレスはいくらしたの? \hfill\break
How much was this necklace? }

\par{\textbf{Spelling Note }: いくら \emph{ }is seldom spelled as 幾ら. }

\par{7. ${\overset{\textnormal{せま}}{\text{狭}}}$ くて ${\overset{\textnormal{いき}}{\text{息}}}$ が ${\overset{\textnormal{つ}}{\text{詰}}}$ まるような ${\overset{\textnormal{かん}}{\text{感}}}$ じする。 \hfill\break
I feel cramped. }

\par{8. ${\overset{\textnormal{ひや}}{\text{日焼}}}$ けしちゃった。 \hfill\break
I got sunburned. }

\par{ As a transitive verb, its primary meaning is “to do.” Aside from its complex grammatical usages that happen to be transitive, it can be used to show occupation, mean “to play (a game\slash sport\slash etc.),” “to wear (an accessory)", or even “to be\dothyp{}\dothyp{}\dothyp{}(shaped)\slash to have a…(face)\slash etc.” when describing appearances. }

\par{9. ${\overset{\textnormal{なに}}{\text{何}}}$ をしたらいいでしょうか。 \hfill\break
What should I do? }

\par{10. ここは、 ${\overset{\textnormal{だいだい}}{\text{代々}}}$ パン ${\overset{\textnormal{や}}{\text{屋}}}$ をしている ${\overset{\textnormal{しにせ}}{\text{老舗}}}$ です。 \hfill\break
This here is an old bakery past down for generations. }

\par{11. ${\overset{\textnormal{てぶくろ}}{\text{手袋}}}$ をしなさい。 \hfill\break
Please wear your cloves. }

\par{12. ${\overset{\textnormal{かれ}}{\text{彼}}}$ は ${\overset{\textnormal{なが}}{\text{長}}}$ い ${\overset{\textnormal{かみ}}{\text{髪}}}$ をしている。 \hfill\break
He has long hair. }

\par{13. ${\overset{\textnormal{か}}{\text{買}}}$ い ${\overset{\textnormal{もの}}{\text{物}}}$ でもしようか。 \hfill\break
How about we shop or something? }

\par{14. ${\overset{\textnormal{ぼく}}{\text{僕}}}$ の ${\overset{\textnormal{つと}}{\text{勤}}}$ めている ${\overset{\textnormal{かいしゃ}}{\text{会社}}}$ にパチンコをする ${\overset{\textnormal{ひと}}{\text{人}}}$ が ${\overset{\textnormal{すうにん}}{\text{数人}}}$ いました。 \hfill\break
There were several people who play pachinko at the company where I work. }

\par{15. ${\overset{\textnormal{わくせい}}{\text{惑星}}}$ ってどんな ${\overset{\textnormal{かたち}}{\text{形}}}$ をしているんですか。 \hfill\break
What sort of shape do planets have? }

\par{16. ${\overset{\textnormal{へん}}{\text{変}}}$ な ${\overset{\textnormal{かお}}{\text{顔}}}$ しないでよ。 \hfill\break
Don\textquotesingle t make weird faces. }

\par{\textbf{Spelling Note }: 為る may very well be the Kanji for “to do,” but it is no longer used in regular writing. If, though, you feel compelled to know how in its truly transitive sense of “to do” is spelled in Kanji, then this is how. }

\begin{center}
\textbf{増す }
\end{center}

\par{\emph{ }増す is a literary verb that means “to increase.” In this regard, it is very similar to the intransitive\slash transitive verb pair 増える and \emph{ }増やす. 増える, unlike 増す, is commonly used in both the written and spoken language. It, though, can have emotion attached to it whereas 増す is only used in an objective sense. However, unlike 増す, it cannot be used to express (dramatic) increase in degree (See Exs. 17, 18, and 21). }

\par{17. プミポン ${\overset{\textnormal{こくおう}}{\text{国王}}}$ の ${\overset{\textnormal{そうしつ}}{\text{喪失}}}$ で、タイの ${\overset{\textnormal{しょうらい}}{\text{将来}}}$ への ${\overset{\textnormal{ふあん}}{\text{不安}}}$ が ${\overset{\textnormal{ま}}{\text{増}}}$ している。 \hfill\break
With the loss of King Bhumibol, suspense over Thailand\textquotesingle s future is massing. }

\par{18. ${\overset{\textnormal{きょくうせいとう}}{\text{極右政党}}}$ が ${\overset{\textnormal{いきお}}{\text{勢}}}$ いを ${\overset{\textnormal{ま}}{\text{増}}}$ している。 \hfill\break
The far right political party is gathering strength. }

\par{19. ${\overset{\textnormal{にほん}}{\text{日本}}}$ では ${\overset{\textnormal{こうれいしゃ}}{\text{高齢者}}}$ の ${\overset{\textnormal{じんこう}}{\text{人口}}}$ が ${\overset{\textnormal{ま}}{\text{増}}}$ している。 \hfill\break
In Japan, the elderly population is increasing. }

\par{20. ${\overset{\textnormal{ぜんじつ}}{\text{前日}}}$ の ${\overset{\textnormal{おおあめ}}{\text{大雨}}}$ で ${\overset{\textnormal{かわ}}{\text{川}}}$ の ${\overset{\textnormal{みずかさ}}{\text{水嵩}}}$ が ${\overset{\textnormal{ま}}{\text{増}}}$ して、 ${\overset{\textnormal{かわ}}{\text{川}}}$ が ${\overset{\textnormal{はんらん}}{\text{氾濫}}}$ した。 \hfill\break
In the heavy rain the other day, the river\textquotesingle s banks enlarged, causing the river to inundate. }

\par{21. ${\overset{\textnormal{じしんは}}{\text{地震波}}}$ は、 ${\overset{\textnormal{ふか}}{\text{深}}}$ さと ${\overset{\textnormal{とも}}{\text{共}}}$ に ${\overset{\textnormal{そくど}}{\text{速度}}}$ を ${\overset{\textnormal{ま}}{\text{増}}}$ している。 \hfill\break
The seismic waves are increasing in depth as well as speed. }

\par{22. ${\overset{\textnormal{けんきゅうじん}}{\text{研究人}}}$ の ${\overset{\textnormal{にんずう}}{\text{人数}}}$ が ${\overset{\textnormal{ふ}}{\text{増}}}$ えた。 \hfill\break
The number of researchers has increased. }

\par{23. ${\overset{\textnormal{こうつうじこ}}{\text{交通事故}}}$ の ${\overset{\textnormal{かず}}{\text{数}}}$ が ${\overset{\textnormal{ふ}}{\text{増}}}$ えているのは ${\overset{\textnormal{なぜ}}{\text{何故}}}$ だろうか。 \hfill\break
Why is that the number of traffic accidents is increasing? }

\par{24. ${\overset{\textnormal{ごうとう}}{\text{強盗}}}$ が ${\overset{\textnormal{ふ}}{\text{増}}}$ えてきているため、 ${\overset{\textnormal{ほうせきてん}}{\text{宝石店}}}$ などでは ${\overset{\textnormal{げんじゅう}}{\text{厳重}}}$ な ${\overset{\textnormal{ぼうはんたいさく}}{\text{防犯対策}}}$ が ${\overset{\textnormal{ひつよう}}{\text{必要}}}$ です。 \hfill\break
Because robberies have risen, strong crime prevention measures are necessary at places such as jewelry stores. }

\par{ 増やす is used in the sense of “to increase the number of (resources).” When used in the sense of “to increase (fortune\slash animals\slash plants)” as in promulgation, it is often spelled as 殖やす. When this meaning is used in an intransitive sense, 殖える can be used. }

\par{25. ${\overset{\textnormal{そうぞくざいさん}}{\text{相続財産}}}$ が\{ ${\overset{\textnormal{ふ}}{\text{増}}}$ えて・ ${\overset{\textnormal{ふ}}{\text{殖}}}$ えて\}いきます。 \hfill\break
Your inheritance will increase. }

\par{26. ${\overset{\textnormal{しょうひん}}{\text{商品}}}$ の ${\overset{\textnormal{しゅるい}}{\text{種類}}}$ を ${\overset{\textnormal{ふ}}{\text{増}}}$ やすことで、カナダでの ${\overset{\textnormal{う}}{\text{売}}}$ り ${\overset{\textnormal{あ}}{\text{上}}}$ げを ${\overset{\textnormal{の}}{\text{伸}}}$ ばしたいと ${\overset{\textnormal{おも}}{\text{思}}}$ います。 \hfill\break
By increasing the variety of merchandise, I would like to expand sales in Canada. }

\par{27. ${\overset{\textnormal{せいぶつ}}{\text{生物}}}$ の ${\overset{\textnormal{しゅるい}}{\text{種類}}}$ を ${\overset{\textnormal{ふ}}{\text{殖}}}$ やしていきたいと ${\overset{\textnormal{おも}}{\text{思}}}$ います。 \hfill\break
I would like to increase the diversity of living things. }

\par{28. ${\overset{\textnormal{ふどうさん}}{\text{不動産}}}$ を ${\overset{\textnormal{かつよう}}{\text{活用}}}$ して ${\overset{\textnormal{ざいさん}}{\text{財産}}}$ を ${\overset{\textnormal{ふ}}{\text{殖}}}$ やす。 \hfill\break
To increase assets by utilizing real estate. }

\begin{center}
\textbf{働く }
\end{center}

\par{ As an intransitive verb, 働く means “to work” or “to function.” As a transitive verb, it means “to perpetrate.” }

\par{29. ${\overset{\textnormal{ちゅうごく}}{\text{中国}}}$ では、 ${\overset{\textnormal{でかせ}}{\text{出稼}}}$ ぎ ${\overset{\textnormal{ろうどうしゃ}}{\text{労働者}}}$ の ${\overset{\textnormal{おお}}{\text{多}}}$ くは、 ${\overset{\textnormal{こうじょう}}{\text{工場}}}$ などで ${\overset{\textnormal{はたら}}{\text{働}}}$ いているようです。 \hfill\break
In China, a lot of migrant workers seem to work at places like factories. }

\par{30. ${\overset{\textnormal{じょうし}}{\text{上司}}}$ が ${\overset{\textnormal{ふせい}}{\text{不正}}}$ を ${\overset{\textnormal{はたら}}{\text{働}}}$ いていることにたまたま ${\overset{\textnormal{き}}{\text{気}}}$ づいてしまいました。 \hfill\break
I\textquotesingle ve incidentally noticed that my boss is committing fraud. }

\par{\textbf{Spelling Note }: たまたま is seldom spelled as 偶々. }

\par{31. ${\overset{\textnormal{あくじ}}{\text{悪事}}}$ を ${\overset{\textnormal{はたら}}{\text{働}}}$ いても ${\overset{\textnormal{なに}}{\text{何}}}$ も ${\overset{\textnormal{かん}}{\text{感}}}$ じないという ${\overset{\textnormal{ひと}}{\text{人}}}$ は ${\overset{\textnormal{すく}}{\text{少}}}$ ない。 \hfill\break
There are few people who don\textquotesingle t feel anything from having committed an evil deed. }

\par{32. ${\overset{\textnormal{むすこ}}{\text{息子}}}$ が ${\overset{\textnormal{ぬす}}{\text{盗}}}$ みを ${\overset{\textnormal{はたら}}{\text{働}}}$ いているとは ${\overset{\textnormal{かんが}}{\text{考}}}$ えたくなかった。 \hfill\break
I didn't want to think that my son was committing robberies. }

\begin{center}
\textbf{引く }
\end{center}

\par{ As a transitive verb, ひく can mean a variety of things with just as many ways to spell it. 引く just happens to be the most basic way to spell it. As an intransitive verb, it simply means “to ebb\slash fade.” }

\par{33. くじを ${\overset{\textnormal{ひ}}{\text{引}}}$ いてみました。 \hfill\break
I tried drawing a lot. }

\par{\textbf{Spelling Note }: くじ may also be spelled as 籤. }

\par{34. ${\overset{\textnormal{さんかしゃ}}{\text{参加者}}}$ の ${\overset{\textnormal{め}}{\text{目}}}$ を ${\overset{\textnormal{ひ}}{\text{惹}}}$ いていました。 \hfill\break
It had been drawing the participants\textquotesingle  attention. }

\par{\textbf{Spelling Note }: When used to mean “to attract\slash captivate,” ひく is usually written as 惹く. }

\par{35. ${\overset{\textnormal{せんじつ}}{\text{先日}}}$ 、 ${\overset{\textnormal{かぜ}}{\text{風邪}}}$ を引きました。 \hfill\break
I caught a cold the other day. }

\par{36. \{まっすぐに ${\overset{\textnormal{せん}}{\text{線}}}$ ・ ${\overset{\textnormal{ちょくせん}}{\text{直線}}}$ \}を ${\overset{\textnormal{ひ}}{\text{引}}}$ いてください。 \hfill\break
Please draw a straight line. }

\par{\textbf{Spelling Note }: まっすぐ \emph{ }may alternatively be spelled as 真っ直ぐ. }

\par{37. ${\overset{\textnormal{じしょ}}{\text{辞書}}}$ を ${\overset{\textnormal{ひ}}{\text{引}}}$ いてください。 \hfill\break
Please consult a dictionary. }

\par{38. ${\overset{\textnormal{さん}}{\text{3}}}$ から ${\overset{\textnormal{に}}{\text{2}}}$ を ${\overset{\textnormal{ひ}}{\text{引}}}$ くと、 ${\overset{\textnormal{いち}}{\text{1}}}$ になります。 \hfill\break
When you subtract 2 from 3, you get 1. }

\par{39. カードを ${\overset{\textnormal{ひ}}{\text{引}}}$ いてください。 \hfill\break
Please draw a card. }

\par{40. ${\overset{\textnormal{ねつ}}{\text{熱}}}$ が ${\overset{\textnormal{ひ}}{\text{引}}}$ いてから2、 ${\overset{\textnormal{みっ}}{\text{3}}}$ ${\overset{\textnormal{か}}{\text{日}}}$ は ${\overset{\textnormal{がいしゅつ}}{\text{外出}}}$ を ${\overset{\textnormal{ひか}}{\text{控}}}$ えてください。 \hfill\break
Refrain from going out for two to three days after the fever has receded. }

\par{41. 潮が ${\overset{\textnormal{ひ}}{\text{引}}}$ いたら ${\overset{\textnormal{ある}}{\text{歩}}}$ いて ${\overset{\textnormal{わた}}{\text{渡}}}$ れます。 \hfill\break
You can walk across once the tide has ebbed away. }

\par{42. ${\overset{\textnormal{そんし}}{\text{孫氏}}}$ は ${\overset{\textnormal{ながねん}}{\text{長年}}}$ 、 ${\overset{\textnormal{せいじ}}{\text{政治}}}$ の ${\overset{\textnormal{おもてぶたい}}{\text{表舞台}}}$ を\{ ${\overset{\textnormal{いんたい}}{\text{引退}}}$ している・ ${\overset{\textnormal{しりぞ}}{\text{退}}}$ いている\slash  ${\overset{\textnormal{ひ}}{\text{退}}}$ いている\}。 \hfill\break
Mr. Sun has retired from the center stage of politics for many years. }

\par{\textbf{Spelling Note }: When used to mean “to draw back\slash retire,” ひく is often written as 退く but it becomes indistinguishable from the verb しりぞく, which is far more common and used for the same purpose. }

\par{43. ${\overset{\textnormal{ひ}}{\text{挽}}}$ き ${\overset{\textnormal{にく}}{\text{肉}}}$ を ${\overset{\textnormal{じぶん}}{\text{自分}}}$ で ${\overset{\textnormal{ひ}}{\text{挽}}}$ きたいのですが、どんな ${\overset{\textnormal{にく}}{\text{肉}}}$ を ${\overset{\textnormal{ようい}}{\text{用意}}}$ すればいいでしょうか。 \hfill\break
I\textquotesingle d like to mince ground meet by myself, but what sort of meat should I prepare? }

\par{\textbf{Spelling Note }: When used to mean “to saw\slash mince,” ひく \emph{ }is usually written as 挽く. }

\par{44. コーヒー ${\overset{\textnormal{まめ}}{\text{豆}}}$ を\{ ${\overset{\textnormal{ひ}}{\text{挽}}}$ く・ ${\overset{\textnormal{ひ}}{\text{碾}}}$ く\}と、いい ${\overset{\textnormal{かお}}{\text{香}}}$ りがします。 \hfill\break
Coffee grains have a good scent when you grind them. }

\par{\textbf{Spelling Note }: When used to mean “to grind\slash mill,” ひく \emph{ }is often written as 挽く. It may also be traditionally written as 碾く. }

\par{45. ${\overset{\textnormal{まっちゃ}}{\text{抹茶}}}$ はなぜ ${\overset{\textnormal{いしうす}}{\text{石臼}}}$ で\{ ${\overset{\textnormal{ひ}}{\text{挽}}}$ く・ ${\overset{\textnormal{ひ}}{\text{碾}}}$ く\}んですか。 \hfill\break
Why is it that you grind matcha in a stone mortar? }

\par{\textbf{Culture Note }: 抹茶 is powdered green tea. }

\par{46. ${\overset{\textnormal{わたし}}{\text{私}}}$ が ${\overset{\textnormal{じてんしゃ}}{\text{自転車}}}$ に ${\overset{\textnormal{の}}{\text{乗}}}$ って ${\overset{\textnormal{ちょくしん}}{\text{直進}}}$ していたところ、 ${\overset{\textnormal{まえ}}{\text{前}}}$ からきた ${\overset{\textnormal{させつ}}{\text{左折}}}$ する ${\overset{\textnormal{くるま}}{\text{車}}}$ に ${\overset{\textnormal{ひ}}{\text{轢}}}$ かれました。 \hfill\break
As I was riding straight ahead on my bicycle, I was knocked down by a car turning left which had come from ahead. }

\par{\textbf{Spelling Note }: When used to mean “to run over (with a vehicle)", ひく is usually spelled as 轢く. }

\par{47. ピアノを ${\overset{\textnormal{ひ}}{\text{弾}}}$ けますか。 \hfill\break
Can you play the piano? }

\par{\textbf{Spelling Note }: When used to mean “to play (a string instrument)", ひくis spelled as 弾く. }

\par{\textbf{Spelling Notes }: ひく may seldom be spelled as 曳く with a nuance of “to tow.” This is especially the case with towing boats, which may be expressed alternatively with the verb 曳航する. When used to mean “to pull\slash drag ahead,” ひく may seldom be spelled as 牽く. In this sense of “traction\slash hauling,” the verb 牽引する would be far more common. }
    