    
\chapter{The Causative II}

\begin{center}
\begin{Large}
第226課: The Causative II: The Corruption of Transitive Verbs  
\end{Large}
\end{center}
 
\par{ Causative conjugations in Japanese are complicated. Historically, many transitive verbs are derived by adding す to a stem, creating naturally causative-like expressions. So, 燃やす describes the event of someone putting something on fire. The thing doesn't just spontaneously ignite. If it were to, you would use the intransitive 燃える. Yet, the causative forms for both 燃える and 燃やす involve using a form ultimately coming from the same source as す that makes so many transitive verbs. }

\begin{ltabulary}{|P|P|P|P|P|}
\hline 

Verb Root & Intransitive & Intransitive Causative & Transitive & Transitive Causative \\ \cline{1-5}

moy- &  \emph{moeru }\hfill\break
to burn &  \emph{moesaseru }\hfill\break
to make X burn &  \emph{moyasu }\hfill\break
to burn X &  \emph{moyasaseru }\hfill\break
to make someone burn X \\ \cline{1-5}

\end{ltabulary}

\par{ One would hope that this could be straightforward for every verb, but reality tells otherwise. There is also the short causative 燃えさす for 燃えさせる, but this form is usually △・X for most speakers. If the verb in the base form were a 五段 verb not ending in す, then it would be OK to anyone. Due to the fact that せる comes from す, the す in more and more transitive verbs is starting to become せる. This lesson will explain the ins and outs of this problem. }
      
\section{Corrupted Transitive Verb Forms}
 
\begin{center}
 \textbf{Intransitive \textrightarrow  Transitive and or Intransitive Causative }
\end{center}

\par{ Historically, the causative endings were さす and す. The す ending transitives and causative phrases is the same word, ignoring conjugation differences. Odd vowel changes may happen when attaching す to make transitives, but they all semantically resemble causative phrases. }

\begin{ltabulary}{|P|P|P|P|}
\hline 

To burn \textrightarrow  To burn it & 燃える \textrightarrow  燃やす & To boil \textrightarrow  To boil it & 沸く \textrightarrow  沸かす \\ \cline{1-4}

To get off \textrightarrow  To drop off & 降りる \textrightarrow  降ろす & To melt \textrightarrow  To melt it & 溶く \textrightarrow  溶かす \\ \cline{1-4}

\end{ltabulary}

\par{  On the left side, we have verbs with transitive forms and intransitive causative forms. That means 燃えさせる・燃やす and 降りさせる・降ろす are all words with \textbf{significant difference }in meaning. On the right side, we have verbs with no intransitive causative form. These forms, though, look and act like intransitive causative phrases. Verbs like on the left are able to have both forms with the aid of さ insertion. What then would need to happen in form for verbs on the right to have a \emph{causative }form? }

\par{ If ~せる comes from す, then ~せる should be able to be used just like ~させる to create intransitive causative phrases. Consider 澄む (to be transparent). 澄む's transitive form is 澄ます, which is causative in nature. A causative form of 澄む would be identical in meaning to a transitive form. So, 澄ませる would be the potential form of 澄ます. Nevertheless, because 澄ます is causative in nature, the す is seen by more speakers as the short causative and changing it to 澄ませる for the causative. Even so, no one would ever have a problem telling when 澄ませる is meant to be the causative or potential. }

\par{ So, which transitive verbs ending in す are vulnerable to changing to end in せる? Certainly, any intransitive verb with no distinct transitive and intransitive causative form is vulnerable. As we look at individual cases of this change, the majority fall under this situation. Even so, -asu is becoming more stigmatic of dialectical speech. Thus, verbs with intransitive causative and transitive verbs are now starting to have the transitive form ending in せる. 沸かす and 溶かす are more vulnerable to change than 燃やす, but the verb 甘やかす (to spoil), which you'll see later as well, has already changed for some speakers to 甘やかせる. This is despite the fact that 甘えさせる exists. This means that one day speakers may use 燃やせる to mean "to burn X". }

\begin{center}
 \textbf{Comparing Old and Modern Conjugations }
\end{center}

\par{ To tell whether an intransitive verb has both a causative and transitive form, let's compare the modern and old conjugations for the causative. させる and せる come from さす and す. They did not conjugate exactly like す(る) as in the present or the past, but they were still similar due to them sharing the same origin. The chart below show shows the old causative conjugations and the modern ones (short and long). As you can see, although the short causative is a retention of the old, some change has occurred. }

\begin{ltabulary}{|P|P|P|P|P|P|P|}
\hline 

 & 未然形 & 連用形 & 終止形 & 連体形 & 已然形 & 命令形 \\ \cline{1-7}

Old さす & させ & させ & さす & さする & さすれ & させよ \\ \cline{1-7}

Old す & せ & せ & す & する & すれ & せよ \\ \cline{1-7}

Modern させる & させ & させ & させる & させる & させれ & させろ \\ \cline{1-7}

Modern せる & せ & せ & せる & せる & せれ & せろ \\ \cline{1-7}

Modern さす & X & (し) & (さす) & X & X & X \\ \cline{1-7}

Modern す & さ & し & す & す & せ & (せ) \\ \cline{1-7}

\end{ltabulary}

\par{ The modern さす is almost nonexistent in Standard Japanese. Its usage is usually dialectical. As for the modern す, it is still really productive. Its 連体形 and 已然形 are not as common, but they're usually not felt as being wrong. The 命令形 is questionable to some speakers, so it will be left in parentheses. Aside from the short causative-passive usage of the 未然形 with the modern す, the modern long causative forms are the most used (and at times the only ones you can\slash should use). }

\par{ As you can see, さす has almost died out. Any instance of it is dialectical. The modern す is also starting to die. Signs of this are evident in the chart. One base is already questionable, and as stated above, aside from the short-causative passive, you're likely to use the modern せる. If the language is hating on -asu, then it's no wonder transitive verbs ending in this are starting to end in -aseru. }

\begin{center}
\textbf{Case-by-Case Study }
\end{center}

\par{ So far, we have seen what verbs are being most affected, and we've even looked at the causative conjugations of the past and present and found yet another source of the problem. Now, we are going to look at individual examples to see how well they fit into the framework described thus far. There is a lot of individual variation, so it is almost impossible to fully express all the details of the options. }

\begin{center}
\textbf{When 下一段 is Most Common }
\end{center}

\par{ Transitive verbs which have shifted from す \textrightarrow  せる are most certainly the trigger for all the change we see in other verbs. }

\begin{itemize}

\item 合う means "to fit\slash go well" and its transitive form is 合わす. However, this has almost entirely been replaced by 合わせる. The only time 合わす gets used is in the causative-passive 
\item 任す means "to entrust" and its intransitive form has long been dead. It too has largely died in favor of 任せる. The only time it gets used more frequently is in the causative-passive. 
\item 飽かす is from the old form of 飽きる, 飽く, meaning "to bore\slash tire of". 飽かす may mean "to use lavishly", but it also is used in a causative sense. However, the forms 飽きさせる and 飽かせる are more common. 飽かせる is sometimes felt as being wrong, but this is a minority opinion. 
\end{itemize}

\begin{center}
\textbf{When Both 下一段 \& 五段 are Most Common }
\end{center}

\par{ These next verbs are undergoing change to only ending in せる, but the traditional form is still holding on strong. Unlike the previous verbs, form preference is more hectic. }

\begin{itemize}

\item 済む means "to finish" or "to be over" and its transitive form is 済ます. This has largely been replaced by 済ませる, especially in Tokyo. 済ました is becoming rare and being treated as dialectical, but it's still used by a lot of people. The same goes for 済まさない versus 済ませない. The passive is still usually seen as 済まされる. Yet, 済ませれば is more common 済ませば. \hfill\break
Standard usage still defines 済ませる as being the potential form of 済ます. 
\item 泣く means "to cry" and its transitive form is 泣かす. At an emotional level, it is not as forceful as the causative form 泣かせる. 泣かせる is not deemed to mean "to be able to make someone cry" and has largely replaced 泣かす. Both 泣かす and 泣かせる are treated as causative phrases. 泣かす is only more common in the causative-passive. Some speakers even think 泣かした is dialectical, but there are still conjugations where they're both more or less just as common. For instance, 泣かさない and 泣かせない are both common. Yet, 泣かせれば is more common than 泣かせば. The potential would either be 泣かすことができる, 泣かせることができる, or 泣かせられる. 
\item 浮く means "to float" and its transitive form is 浮かす. 浮かせる is also very common, but it can still be the potential form of 浮かす. Just like the rest, 浮かす is most common in the causative-passive and the frequency of other conjugations is the same. 
\item 尖る means "to become sharp" and its transitive form is 尖らす, which may also be 尖らせる. Conjugations are more likely to be used with the latter, but there is flux to be had among speakers. 
\item 寝る means to "sleep" and its transitive form is  寝かす, which may also be 寝かせる. Both forms are used frequently, and there is no clear consensus on how the two differ. 
\end{itemize}
\textbf{When 五段 is Most Common }  This group of verbs have せる forms that are acceptable, but the original す form is most common. This is unlike 済ませる, which although common, is not always deemed correct. 
\begin{itemize}

\item 驚く means "to be surprised" and its transitive form is 驚かす. 驚かせる is also acceptable. Most conjugations, though, are more natural with 驚かす. 
\item 輝く means "to shine\slash glow" and its transitive form is 輝かす. 輝かせる is more common in the base form. Its form are slowly becoming more common than 輝かす but still has a way to go. 
\item 鳴る means "to ring" and its transitive form is 鳴らす. 鳴らせる, however, sounds causative and is not used the same as 鳴らす. However, 鳴らす is more common overall. 
\item 反る means "to bend" and its transitive form is 反らす. 反らせる is now an acceptable form, but a lot of its conjugations are typically interpreted as being in the potential. This is the same for 逸らす・逸らせる (to evade). 
\item One of the meanings of 巡る is "to surround" and its transitive for is 巡らす. 巡らせる is now acceptable and becoming more common; however, it still retains its potential meaning. 
\end{itemize}
\textbf{Verbs Just Starting to Undergo す \textrightarrow  せる }  All of the verbs above have undergone す \textrightarrow  せる successfully with variation in retention of the original form. In the following verbs, the new せる ending form is wrong to a lot of speakers, but they are becoming more common. If there are alternate forms, the most common will be in bold.  
\begin{ltabulary}{|P|P|P|P|P|}
\hline 

Intransitive & Intransitive Causative & Transitive & Transitive Causative & New Transitive \\ \cline{1-5}

甘える & 甘えさせる & 甘やかす & 甘やかさせる & 甘やかせる \\ \cline{1-5}

 &  & ひけらかす & ひけらかさせる & ひけらかせる \\ \cline{1-5}

尽きる & 尽きさせる &  \textbf{尽くす }・尽かす &  \textbf{尽くさせる }・尽かさせる & 尽かせる \\ \cline{1-5}

\end{ltabulary}
\textbf{Verbs which Haven't Undergone す \textrightarrow  せる }
\par{ Knowing what form to use is not an easy task as a learner. In the end, the best thing that you should do is listening to what Japanese speakers are using. If various forms exist, they're likely going to be different. For instance, 甘えさせる and 甘やかす exist, but they're vastly different. The first is a good thing and the second is a bad thing. }

\par{ Intransitive verbs with causative-like transitive forms do exist, but the causative should still come from the transitive version of the verb. These include verbs like 鳴る and 散る, but they're in the minority. For instance, 照る means "to shine" and 照らす means "to brighten", but the causative is 照らさせる. You cannot find 照らせる to mean 照らす. It is the potential as expected. }

\par{ Over time, more verbs will switch over to having their transitive forms end in せる. This lesson was about transitive verbs having corrupted forms due to the causative. This confusion arose due to common etymology, and there is various degrees of acceptance of variant forms. This is what you can expect in a wider change affecting the entire language. Sadly, this will only be truly easy centuries later when this change has fully taken effect. }

\begin{center}
\textbf{Examples }
\end{center}

\par{ Forms are listed from most to least common. }

\par{1. 食事を\{済ませる・済ます\}。 \hfill\break
To finish supper. }

\par{2. 知識を\{ひけらかす 〇・ひけらかせる △\}。 \hfill\break
To show off one's knowledge. }

\par{3. 子供を\{甘やかす 〇・甘やかせる △\}。 \hfill\break
To spoil one's\slash a children. }

\par{4. 聴衆を\{飽きさせない・飽かさない・飽かせない\}。 \hfill\break
It doesn't bore the crowd. }
    