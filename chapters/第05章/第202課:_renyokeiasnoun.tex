    
\chapter{連用形 \textrightarrow  Noun}

\begin{center}
\begin{Large}
第202課: 連用形 \textrightarrow  Noun 
\end{Large}
\end{center}
 
\par{ Verbs in the 連用形 can be very noun-like. Whether the result is actually a true-noun, a limited noun, or rarely like a noun shouldn't distract us from noting how verbs in this form can clearly deviate from a truly verbal meaning. }

\par{ First, let\textquotesingle s assume that a lot of verbs can at least be noun-like in some contexts. We can see at least five different types of resultant noun phrases. }

\begin{enumerate}

\item  Those that are so completely independent from the verb that they are separate words in the lexicon (vocabulary of the language). 
\item Those that are independent in the sense that they can stand alone as a noun in a sentence without the aid of context or additional words. \hfill\break

\item Those that may stand alone as a noun but need contextual support through phrasing to make sense. 
\item Those that need to be in a compound to function properly as a noun phrase. 
\item Those that are found in constructions like 連用形+に+Motion verb in which the noun-like phrase is too verbal to be considered a true noun phrase. 
\end{enumerate}
      
\section{Type 1}
 
\par{ Our example verb for this category is 光 (light). Although its meaning of radiance in 目の光 gets closer to its verb roots, if we were to say 〇が光り〇, we would treat 光り as a verb phrase in the 連用形. This difference between hikari and hikar-i suggests that the first comes from the second, but it has lost its literal connection to the verb 光る. The second clearly has not and is merely a form of the latter. In spelling, we distinguish the two for this verb, but that won\textquotesingle t always be a guarantee. }

\par{ If the verb in which the noun comes from has disappeared, then it becomes even clearer how independent the noun is from its verbal root. Some examples including the following. }

\par{${\overset{\textnormal{きり}}{\text{霧}}}$ (mist): Comes from the 連用形 of the old verb 霧る meaning “to mist” }

\par{${\overset{\textnormal{すもう}}{\text{相撲}}}$ (sumo wrestling): Comes from the 連用形 of the old verb 争(すま)ふ meaning to fight. }

\par{${\overset{\textnormal{かぶき}}{\text{歌舞伎}}}$ (Kabuki theatre): Comes from the 連用形 of the old verb かぶく meaning “to tilt (one\textquotesingle s head)”. }

\par{ Another interesting example is the verb 歌う. We find that there are two nouns derived from it: 歌 and 謡い. The 漢字 have the same reading here, but the words うた and うたい have different meanings. The first can mean song or even poem (though when meant to mean poem it is often spelled as 詩). You may even see the character 唄 to refer to a song akin to a lullaby. The latter, 謡い, refers specifically to Noh song. We see that the latter clearly comes from the 連用形. At first glance, it seems that 歌 is from the root of the verb. In reality, it is probably from the contraction of うたい. }
      
\section{Type 2}
 
\par{ This is a less drastic form of Type 1. For 連用形 of this type, it can be used like any other noun grammatically. It needs no context to be understood in isolation as a noun phrase with a specific meaning. This does not mean that the 連用形 can be used verbally. So, we could say that 光 may belong here. }

\par{ Many verbs regarding psychological and emotional state are represented here as well as human cognitive activities, speech acts, and actions regarding expression. }

\par{1. ほとんどの ${\overset{\textnormal{ぞうき}}{\text{臓器}}}$ が日ごろ \textbf{休み }なく働き続けています。 (休む) \hfill\break
Almost all of your organs continue to work nonstop on a normal basis. }

\par{2. ${\overset{\textnormal{つか}}{\text{}}}$ \textbf{れ }を感じたときは、自分の体が「休んで!」と言っているサインです。 (疲れる) \hfill\break
When you've felt fatigue, that's a sign from your body telling you "to take a rest!". }

\par{3. 薬と死の ${\overset{\textnormal{にお}}{\text{}}}$ \textbf{い }がする場所から ${\overset{\textnormal{のが}}{\text{逃}}}$ れる。   (匂う) \hfill\break
To escape from a place with the smell of medicine and death. }

\par{4. ISISの ${\overset{\textnormal{ねら}}{\text{狙}}}$ \textbf{い }はシリアとイラクの ${\overset{\textnormal{こっきょう}}{\text{国境}}}$ ${\overset{\textnormal{ちたい}}{\text{地帯}}}$ にイスラム国家を ${\overset{\textnormal{じゅりつ}}{\text{樹立}}}$ するところにあります。 (狙う) \hfill\break
The aim of ISIS is to establish an Islamic State in the border region between Syria and Iraq. }

\par{5. エジプト ${\overset{\textnormal{しみん}}{\text{市民}}}$ の ${\overset{\textnormal{のぞ}}{\text{望}}}$ \textbf{み }は ${\overset{\textnormal{かな}}{\text{叶}}}$ うのか。 (望む) \hfill\break
Will the hopes of the Egyptian people be fulfilled? }

\par{6. 新しい ${\overset{\textnormal{}}{\text{知ら}}}$ \textbf{せ }を伝える。 (知らせる) \hfill\break
To inform on a new notice. }

\par{7. ${\overset{\textnormal{まじょ}}{\text{魔女}}}$ は ${\overset{\textnormal{もりながけ}}{\text{森永家}}}$ ${\overset{\textnormal{ぜんいん}}{\text{全員}}}$ に ${\overset{\textnormal{のろ}}{\text{呪}}}$ \textbf{い }をかけた。 (呪う) \hfill\break
The witch put a curse on all of the Morinaga family. }

\par{8. 川の \textbf{流れ }に ${\overset{\textnormal{そ}}{\text{沿}}}$ って泳ぐ。 (流れる) \hfill\break
To swim along the current of the river. }

\par{9. ${\overset{\textnormal{なんどく}}{\text{難読}}}$ ${\overset{\textnormal{かんじ}}{\text{漢字}}}$ の \textbf{読み }を覚える ${\overset{\textnormal{ほうほう}}{\text{方法}}}$ を教えてくれませんか。 (読む) \hfill\break
Could you teach me methods to remember the readings of difficult to read Kanji? }

\par{10. どうしても何時間も ${\overset{\textnormal{いか}}{\text{怒}}}$ \textbf{り }が ${\overset{\textnormal{おさ}}{\text{収}}}$ まらない。 (怒る) \hfill\break
(My\slash X's) anger won't simmer down for hours no matter what. }

\par{ Even though these are clearly nouns, they still individually can have odd restrictions. }

\par{11a. きょうは仕事で疲れたなあ。〇 \hfill\break
11b. きょうの仕事は疲れたなあ。〇 ・△ \hfill\break
11c. きょうの \textbf{働き }は \textbf{疲れ }だったなあ。 X  (働く \& 疲れる) }

\par{ In this example, we see a major restriction on 働き and 疲れ. Even when you replace them with the correct words, the grammar is still off. This demonstrates how semantic and syntactic restrictions work in unison to form natural utterances like 11a. Take into consideration Ex. 12, which demonstrates the most natural and common use of 行き as a suffix. }

\par{12. ${\overset{\textnormal{にほん}}{\text{日本}}}$ ${\overset{\textnormal{ゆき}}{\text{}}}$ の ${\overset{\textnormal{ひこうき}}{\text{飛行機}}}$ に乗る。  (行く = ゆく) \hfill\break
To ride a plain bound for Japan. }
      
\section{Type 3}
 
\par{ Many verbs need context to be used independently. The point here is at least they can be stand-alone nouns. They just need help to make any sense. Consider Ex. 13-14. }

\par{13a. \textbf{泳ぎ }は体にいいですね。X  (泳ぐ) \hfill\break
13b. 泳ぐことは体にいいですね。〇 \hfill\break
Swimming is good for you. }

\par{14. やっぱり魚は \textbf{泳ぎ }が速いね。 (泳ぐ) \hfill\break
Fish are definitely fast at swimming. }

\par{ Ex. 13 demonstrates how if you give the wrong sort of help, you get a bad sentence. Many examples of this are found in set idiomatic phrases. Some require a XはYが structure to appear whereas others simply want a subject attribute. The requirements to have these sort of nouns work varies a lot, which is why the mastery of this type is going to very difficult. You have to in a sense get used to how Japanese phrases things, which is not an easy task. }

\par{15a. \textbf{高まり }を感じる。X    (高まる) \hfill\break
15b. 心の \textbf{高まり }を感じる。〇 \hfill\break
To feel an emotional high. }

\par{16. 彼は \textbf{${\overset{\textnormal{}}{\text{分}}}$ かり }が ${\overset{\textnormal{おそ}}{\text{遅}}}$ い。  (分かる) \hfill\break
He's a slow learner. }

\par{17. ${\overset{\textnormal{がめん}}{\text{画面}}}$ の ${\overset{\textnormal{うつ}}{\text{映}}}$ \textbf{り }が悪いですね。  (映る) \hfill\break
The quality of the screen is bad, isn't it? }

\par{18. エンジンの \textbf{かかり }が ${\overset{\textnormal{おそ}}{\text{遅}}}$ い。  (かかる) \hfill\break
The engine starts up slowly. }

\par{19. ${\overset{\textnormal{ほうちょう}}{\text{庖丁}}}$ がなまって \textbf{切れ }が悪くなった。  (切れる) \hfill\break
The kitchen knife has gotten dull and doesn't cut well. }

\par{20. 神経の ${\overset{\textnormal{つた}}{\text{伝}}}$ \textbf{わり }を ${\overset{\textnormal{しゃだん}}{\text{遮断}}}$ することで痛みを ${\overset{\textnormal{やわ}}{\text{和}}}$ らげる ${\overset{\textnormal{ちりょう}}{\text{治療}}}$ です。  (伝わる) \hfill\break
This is a treatment to alleviate pain by circumventing neural transmissions. }

\par{21. ${\overset{\textnormal{ないぞう}}{\text{内臓}}}$ は ${\overset{\textnormal{くさ}}{\text{腐}}}$ \textbf{り }が早い。 (腐る) \hfill\break
Innards rot quickly. }

\par{21. エアコンの ${\overset{\textnormal{き}}{\text{効}}}$ \textbf{き }が悪い。 (効く) \hfill\break
The air conditioning doesn't work well. }

\par{22. 今年はピーマンの ${\overset{\textnormal{でき}}{\text{出来}}}$ \textbf{が }イマイチですよ。 (出来る) \hfill\break
The pepper turnout this year is not that good. }

\par{23. ${\overset{\textnormal{きつえん}}{\text{喫煙}}}$ などは、 ${\overset{\textnormal{ち}}{\text{血}}}$ の ${\overset{\textnormal{めぐ}}{\text{巡}}}$ \textbf{り }を悪くする ${\overset{\textnormal{おも}}{\text{主}}}$ な ${\overset{\textnormal{げんいん}}{\text{原因}}}$ である。 (巡る) \hfill\break
Smoking and the like is a major factor in worsening blood circulation. }

\par{24. 細長く ${\overset{\textnormal{ねば}}{\text{粘}}}$ \textbf{り }のない ${\overset{\textnormal{こめ}}{\text{米}}}$ に ${\overset{\textnormal{なじ}}{\text{馴染}}}$ む。  (粘る) \hfill\break
To get to non-sticky thin rice. }

\par{25. ${\overset{\textnormal{えだ}}{\text{枝}}}$ が ${\overset{\textnormal{しめ}}{\text{湿}}}$ っていて、 ${\overset{\textnormal{た}}{\text{焚}}}$ き ${\overset{\textnormal{び}}{\text{火}}}$ の ${\overset{\textnormal{も}}{\text{燃}}}$ \textbf{え }が悪い。  (燃える) \hfill\break
The branches are damp, and so the bonfire's flame is bad. }
      
\section{Type 4}
 
\par{ The next group requires compounding. Do not think, though, that words can\textquotesingle t be in more than one category. Rather, we must truly investigate individual meanings of a word to determine what type it belongs to. For example, ${\overset{\textnormal{だし}}{\text{出汁}}}$ (soup stock) comes from 出す and would be an example of Type 1. Aside from this, 出し needs to be in a compound like ゴミ出し to work as a noun. }

\par{ Because of this required compounding, you do sometimes get sequential voicing. }

\par{26. プロでも ${\overset{\textnormal{ゆきお}}{\text{雪}}}$ \textbf{ろし }は ${\overset{\textnormal{こんなん}}{\text{困難}}}$ です。  (下ろす) \hfill\break
Removing snow is even difficult for professionals. }

\par{27. ${\overset{\textnormal{えど}}{\text{江戸}}}$ ${\overset{\textnormal{じだい}}{\text{時代}}}$ の根 \textbf{付 }を ${\overset{\textnormal{そうぞく}}{\text{相続}}}$ する。  (付ける) \hfill\break
To inherit netsuke from the Edo Period. }

\par{\textbf{Item Note }: Netsuke are small miniature carvings that are placed at the end of cords hanging from a pouch of some sort. }

\par{28. ${\overset{\textnormal{もくぞう}}{\text{木造}}}$ 2階 \textbf{建て }が ${\overset{\textnormal{ぜんしょう}}{\text{全焼}}}$ しました。  (建てる) \hfill\break
The wooden, two-story structure completely burned up. }

\par{29. 2014年の ${\overset{\textnormal{つゆい}}{\text{梅雨}}}$ りは6月8日と言われています。  (入る = いる) \hfill\break
They say that the start of 2014's rainy season was June, 8th.  }

\par{30. ${\overset{\textnormal{すうじ}}{\text{数字}}}$ ${\overset{\textnormal{あ}}{\text{}}}$ わせ ${\overset{\textnormal{じょう}}{\text{錠}}}$ を使うのが ${\overset{\textnormal{べんり}}{\text{便利}}}$ です。  (合わせる) \hfill\break
Using a number combination lock is convenient. }

\par{31. 昼 \textbf{寝 }のあとに ${\overset{\textnormal{ずつう}}{\text{頭痛}}}$ がします。  (寝る) \hfill\break
I have headaches after an afternoon nap. }

\par{32. ${\overset{\textnormal{しっさく}}{\text{失策}}}$ の ${\overset{\textnormal{あなう}}{\text{穴}}}$ めを国民に求める。  (埋める) \hfill\break
To seek the patching of a failed policy from the people. }

\par{33. ${\overset{\textnormal{さけ}}{\text{鮭}}}$ と ${\overset{\textnormal{やさい}}{\text{野菜}}}$ をごちゃ ${\overset{\textnormal{}}{\text{}}}$ \textbf{ぜ }にして 焼きます。  (混ぜる) \hfill\break
We mix salmon and vegetables together and cook them. }

\par{34. 結婚の話が立ち ${\overset{\textnormal{ぎ}}{\text{消}}}$ \textbf{え }になった。  (消える) \hfill\break
The talk of marriage fizzled away. }

\par{35. 人々の行き \textbf{来 }が ${\overset{\textnormal{にぎ}}{\text{賑}}}$ やかだった。  (来る) \hfill\break
The coming and going of people was lively. }

\begin{center}
\textbf{連用形 \textrightarrow  Noun \textrightarrow  する Verb } 
\end{center}

\par{ At times, a noun made in this way may attach to another noun, but to be used verbally again, it needs to become a verb by adding する, which will be introduced shortly. As we haven't gone over how to conjugate this verb yet, we'll just look at some examples of this phenomenon. Consider the following words. }

\begin{ltabulary}{|P|P|P|P|P|P|}
\hline 

Phrase & Reading & Meaning & Phrase & Reading & Meaning \\ \cline{1-6}

 目隠しする &  めかくしする &  To blindfold &  タグ付けする &  たぐつけする &  To tag \\ \cline{1-6}

 粗探しする &  あらさがしする &  To nitpick &  拾い読みする &  ひろいよみする &  To browse \\ \cline{1-6}

\end{ltabulary}

\par{\textbf{Part of Speech Note }: The あら in 粗探し is actually from the stem of the adjective 粗い (rough\slash coarse). }
      
\section{Type 5: 連用形+に+Motion Verb}
 
\par{ In the grammar pattern 連用形+に+Motion Verb plus a few other instances we\textquotesingle ll see later on in Japanese grammar, verbs that do or don\textquotesingle t fall in the categories above are able to be used like nouns while simultaneously holding onto their verbal meaning. }

\par{ If you want to say things like “I am going to watch the movie”, you must use this grammar pattern. The Japanese for this is 私は映画を見に行きます. The grammar is very parallel to English. In both languages, we would never call the verb a noun, per say. However, it certainly behaves like one in this situation. }

\par{ If the verb is a する verb made by putting する after a Sino-Japanese word, then you don\textquotesingle t even need し. So, say you want to say, I\textquotesingle m going to observe at Kyoto University. You could say 京都大学を見学しに行きます. However, you could just say 京都大学を見学に行きます. Either way is grammatically fine. }

\begin{center}
\textbf{Examples } 
\end{center}

\par{36. 私は一人でイチゴを \textbf{探し }に 歩きました。 \hfill\break
I walked to search for strawberries by myself. }

\par{37. ${\overset{\textnormal{あわ}}{\text{慌}}}$ てて ${\overset{\textnormal{にもつ}}{\text{荷物}}}$ を \textbf{取り }に ${\overset{\textnormal{もど}}{\text{戻}}}$ った。 \hfill\break
I rushed to return to get my luggage. }

\par{38. 彼は \textbf{遊び }に 来た。 \hfill\break
He came to play. }

\par{39. \textbf{ ${\overset{\textnormal{そうだん}}{\text{相談}}}$ (し) }に 来る。 \hfill\break
To come to discuss. }

\par{40. \textbf{買い }${\overset{\textnormal{もの}}{\text{}}}$ に 行く。 \hfill\break
To go shopping. }

\par{\textbf{Sentence Note }: 買い物をする is the verb phrase! }

\par{\textbf{Phrase Note }: ~物 attaches to the ${\overset{\textnormal{れんようけい}}{\text{連用形}}}$ of verbs of type 4 to refer to the things that you do the action with. So, 飲み物 = drink, 食べ物 = food, and 読み物 = reading (material). }

\par{41. ${\overset{\textnormal{ぼく}}{\text{僕}}}$ らは ${\overset{\textnormal{えいが}}{\text{映画}}}$ を \textbf{見 }に 来たよ。 \hfill\break
We came to see a movie. }

\par{42. ${\overset{\textnormal{ばん}}{\text{晩}}}$ ご ${\overset{\textnormal{はん}}{\text{飯}}}$ を \textbf{食べ }に 来ませんか。 \hfill\break
Will you come to eat dinner? }

\par{43. ${\overset{\textnormal{つま}}{\text{妻}}}$ はハンドバッグを \textbf{${\overset{\textnormal{}}{\text{取}}}$ り }に 帰った。 \hfill\break
My wife went home to go get her handbag. }
    