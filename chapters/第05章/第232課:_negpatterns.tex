    
\chapter{More Negative Patterns}

\begin{center}
\begin{Large}
第232課: More Negative Patterns: ~ないことには, ~なし・なき, ~なしで(は), ~なしに(は), ~なくして(は), ~ことなく, \& 甲斐もなく 
\end{Large}
\end{center}
 
\par{ This lesson will introduce you to more patterns that utilize ~ない or derivations of it. }
      
\section{~ないことには(~ない)}
 
\par{ AないことにはB means "if\dothyp{}\dothyp{}\dothyp{}don't\slash in not\dothyp{}\dothyp{}\dothyp{}; unless". So, if you don't do A, you get stuck with B. B could be positive or negative depending on the outcome that you wish to express. It can be used with verbs, adjectives, or the copula, but don't forget how to properly use ない! This pattern must be only used in declarative sentences. }

\begin{ltabulary}{|P|P|P|}
\hline 

品詞 & 接続 & 例 \\ \cline{1-3}

名詞 & N+でないことには & 人でないことには \\ \cline{1-3}

形容詞 & Adj \textrightarrow  く-連用形 + ないことには & 高くないことには \\ \cline{1-3}

形容動詞 & Adj + でないことには & 精密でないことには \\ \cline{1-3}

動詞 & Adj \textrightarrow  未然形 + ないことには & しないことには \\ \cline{1-3}

\end{ltabulary}

\par{1. 泳がないことには、海岸に行けなくなった。 \hfill\break
In not swimming, I became unable to go to the beach. }

\par{2. 急がないことには、間に合わなくなる。 \hfill\break
In not hurrying, you won't make it in time. }

\par{3. 勝たないことには、 ${\overset{\textnormal{しあい}}{\text{試合}}}$ に負ける。 \hfill\break
If you don't win, you lose in the tournament. }

\par{4. がんばらないことには、今度の韓国語の試験に合格できないよ。 \hfill\break
Unless you try hard, you won't pass this next Korean exam. }

\par{\textbf{Grammar Note }: Do not confuse this with \dothyp{}\dothyp{}\dothyp{}ないことには where は is simply emphatic. The phrase discussed in this section is conjunctive in nature whereas this isn't. }

\par{5. 自分に関わりのないことには口を出すなよ。 \hfill\break
Don't interfere in things that don't concern you! }
      
\section{~なし}
 
\par{  ${\overset{\textnormal{な}}{\text{無}}}$ し means  "without"  and is after nominal phrases. If なし for some reason is placed before a noun, it becomes 無き. There is more behind this form of ない. The problem lies with how to use ~なしに(は), ~なしで(は), and ~なくして(は). These phrases show that if something were not to happen, a certain event won't be. These phrases show necessity, and there is a negative phrase in the next clause. }

\par{6. コーヒーなしに過ごせない。 \hfill\break
I can't go on without coffee. }

\par{7. 涙なしに語ることができない。(書き言葉的) \hfill\break
I can't tell it without tears coming. }

\par{8. 国民の理解と協力なしには実行できない。(書き言葉) \hfill\break
We can't act forth without the understanding and cooperation of the citizens. }

\par{9. 政府開発援助なしではこの革新的な医学研究は続けることができない。 \hfill\break
Without government development aid, this groundbreaking medical research can't continue. }

\par{10. しっかりと準備することなくしては成功できぬ。 (硬い書き言葉) \hfill\break
Without properly preparing, you can't succeed. }

\par{11. 先生方のご指導なくして私の大学合格は有り得ませんでした。(とても丁寧; Useful expression) \hfill\break
Without the guidance of my teachers, my college success wouldn't have been possible. }

\par{12. ${\overset{\textnormal{せいじか}}{\text{政治家}}}$ はテレプロンプター(原稿)なしで話せない。 \hfill\break
Politicians can't talk without a teleprompter. }

\par{13. ${\overset{\textnormal{ひゃくがい}}{\text{百害}}}$ あって ${\overset{\textnormal{いちり}}{\text{一利}}}$ なし。(Set phrase\slash proverb) \hfill\break
To do no good but a lot of harm. }

\par{14. 「憲法を改正することなくして、軍事大国になれない」と考える人が増えているそうだ。 \hfill\break
It sounds that the number of people who think that without revising the constitution, [Japan] won't be able to become a military power is growing. }

\par{\textbf{Grammar Note }: Also seen as ~ずして(は), ~なくして(は) is a more old-fashioned\slash archaic equivalent of ~なしに(は), but it can be used in plus and or negative situations. }

\begin{center}
\textbf{~なしで(は) VS ~なしに(は) }
\end{center}

\par{ Although both present a premise X, which makes Y possible, they are slightly different. }

\par{1. ~なしで(は) is used with a noun that shows a general condition, and the listener is told that there must be X. Thus, the speaker can use this when pointing out a ban or the need for evasion. }

\par{2. ~なしに(は) can't be used directly towards the listener to get that person working. When these expressions are interchangeable, the former shows an individual cause whereas the latter shows a more obvious situation that is deemed to be evident and necessary. Thus, it is more suited for the written language. }

\par{15. スペイン語?無理いうなや。辞書なし\{で・に\}は、分からんわ。(ちょっと関西弁っぽい) \hfill\break
Spanish? Don't be unreasonable. I can't understand it without a dictionary. }

\par{16. 客商売だから、ネクタイなし\{〇 で・X に\}は、困るんだけども。(話し言葉) \hfill\break
Since it's business (client-oriented), it'd be troubling without a tie. }

\par{17. 届け出なし\{〇 に・△ で\}外泊するのは、キャンプの規則に違反することになります。 \hfill\break
Sleeping outdoors is now a violation of camp rules without notification. }

\par{18. ノックなしに(は)、ドアを開けないでください。 \hfill\break
Don't open the door without knocking. }

\par{\textbf{Grammar Note }: は in the last situation would make it fit well, but the message was probably written on the door rather than spoken. Had the sentence been spoken, with ノックしないで being more common, ~なしに would exemplary demonstrate consideration to the listener. }

\begin{center}
 \textbf{~ないことには~ない VS ~ことなしに(は)~ない VS ~ことなく }
\end{center}

\par{  XないことにはYない shows that for the realization of Y, X is necessary. X can be a positive or negative situation. However, ~ことなしに(は)~ない and ~ことなく can only be used with verbs whereas the former pattern is also seen with nouns and adjectives. Furthermore, these latter expressions show that "for Y to realize, X is unavoidable". So, it's not positive. }

\par{19a. ${\overset{\textnormal{どりょく}}{\text{努力}}}$ \{することなしには・しないことには\}、成功はできない。 \hfill\break
19b. 成功するためには、努力を ${\overset{\textnormal{さ}}{\text{避}}}$ けていることはできない。 \hfill\break
You cannot succeed unless you put in effort. }

\par{20. 「憲法を変え\{・ることなしには・ないことには\}、軍事大国になれないと考える人が増えている」といわれますが、本当かどうかよく分かりませんね。 \hfill\break
It's said that the number of people that think that without changing the constitution, [Japan] won't be able to come a military power is growing, but I don't really know if that's true. }

\par{21. 優しい人でないことには、お金があっても、結婚できませんよ。 \hfill\break
You can't get married even if you have money if you're not a nice person. }

\par{ Other things to keep in mind is that both ~ことなしに(は)~ない and ~ことなく are more 書き言葉的; however, the former is even more ${\overset{\textnormal{かた}}{\text{硬}}}$ い because it has the old なし in it. You do still see ~こと(も)なく in the spoken language, though, in which it has interchangeability with 連用形+もしないで. }

\par{22a. 多少のリスクを負わないことには、ビジネスなんかできないよ。とうてい避けられないもんだから。〇 \hfill\break
22b. 多少のリスクを負う\{ことなく・ことなしに\}、ビジネスなんかできないよ。とうてい避けられないもんだから。X \hfill\break
Without taking some risk, you can't do business. That's because you can't avoid it no matter what. }
23. 石に ${\overset{\textnormal{つまづ}}{\text{躓}}}$ いて ${\overset{\textnormal{ころ}}{\text{転}}}$ んだ子供が\{泣くこともなく・泣きもしないで\}、 ${\overset{\textnormal{えがお}}{\text{笑顔}}}$ で立ち上がったなんてびっくりしたよ。 \hfill\break
I was surprised that the kid who tripped and fell on the rocks got up with a smile without even crying. 
\par{24. しょうことなく、メザキさんと並んで、いくら行っても太くもならないし細くもならない道を、長く歩いた。 \hfill\break
With it not being helped, I went along with Mr. Mezaki, and we walked a long ways down the road, which no matter how much you go neither widens nor thins. \hfill\break
From 溺レる by 川上弘美. }

\par{\textbf{Phrase Note }: しょうことなく = しょうことなしに = やむを得ず. }
      
\section{甲斐(かい)もなく}
 
\par{   ${\overset{\textnormal{かい}}{\text{甲斐}}}$ means "avail" and ${\overset{\textnormal{かい}}{\text{甲斐}}}$ もなく, thus, means "to no avail\slash in vain". Just as in English, it is not really casual. }

\par{25. 私たちは ${\overset{\textnormal{れんしゅう}}{\text{練習}}}$ した ${\overset{\textnormal{かい}}{\text{甲斐}}}$ もなく負けてしまった。 \hfill\break
We lost despite practicing. }

\par{26. わざわざフランスから出かけてきた甲斐がありました。 \hfill\break
It was worth our while to have come all the way from France. }

\par{27. 勉強した甲斐もなく落ちてしまいました。 \hfill\break
I ended up failing to no avail of studying. }

\par{28. ${\overset{\textnormal{はたら}}{\text{働}}}$ き ${\overset{\textnormal{がい}}{\text{甲斐}}}$ \hfill\break
 Just the value of working }

\par{29. ${\overset{\textnormal{としがい}}{\text{年甲斐}}}$ もなく \hfill\break
Unbecoming to one's age }
    