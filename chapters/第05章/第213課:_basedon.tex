    
\chapter{Based on}

\begin{center}
\begin{Large}
第213課: Based on: に基づいて、を踏まえて, をもとにして, に沿って, に即して, \& に則って 
\end{Large}
\end{center}
 
\par{ These phrases are definitely very similar to each other, but as you will see during the compare and contrast sections of this lesson, there are important differences to keep in mind. }
      
\section{~に基づいて}
 
\par{ ~に基づいて means "on the basis of\slash based off of\slash based on". Its attribute form may either be ~に基づいての or ~に基づいた. Broadly speaking, there are two main usages of Xに基づいてY. The first is "without deterring from a standard\slash criterion\slash norm\slash rule\slash law X, one carries out\slash executes an action Y". The other, which is found primarily in the written language, is "having A as a basis\slash foundation\slash modal\slash example\slash pattern\slash reference, one does\slash makes a decision B". }
 
\par{1. 事実に基づいた ${\overset{\textnormal{きじゅつ}}{\text{記述}}}$ をする。 \hfill\break
Describe on the basis of reality. }
 
\par{2. 彼女の議論は確かに ${\overset{\textnormal{しょうこ}}{\text{証拠}}}$ に ${\overset{\textnormal{もと}}{\text{基}}}$ づいていなかった。 \hfill\break
Her argument was certainly not on the basis of evidence. }
 
\par{3. 明らかなことに基づいた議論の ${\overset{\textnormal{よち}}{\text{余地}}}$ のない事実。 \hfill\break
Unarguable facts based on obvious things. }

\par{4. このマニュアルに基づいて操作します。 \hfill\break
I will operate it based on this manual. }

\par{5. 規則に基づいて処理する。 \hfill\break
To process according to regulations. }
 
\par{6. さっきの例文に基づいた、もっと複雑なものです。 \hfill\break
These are more complicated ones based off of the previous example sentences. }
 
\par{7. 証拠に基づいて、 ${\overset{\textnormal{かがいしゃ}}{\text{加害者}}}$ に ${\overset{\textnormal{しけい}}{\text{死刑}}}$ を言い ${\overset{\textnormal{わた}}{\text{渡}}}$ した。 \hfill\break
In line with the evidence, the assailant was sentenced to death. }

\par{\textbf{Variant Note }: ~に基づいて may also be ~に基づき in stiffer writing. }

\par{\textbf{Grammar Note }: For the attribute form, ~に基づいての must only be used when what it precedes is a standard of some sort. If it is not, you must use ~に基づいた instead. }
      
\section{を踏まえて}
 
\par{ The  一段 verb 踏まえる means "to be based on". So, ~を踏まえて means "on the basis of" just like ~に基づいて. It can also be seen as ~を踏まえ in the written language. Its attribute form may either be ~を踏まえての or ~を踏まえた. As its original literal meaning is "to tread on" in a defensive posture, it is used after words that can show a basis\slash foundation\slash precedent for something. Again, it cannot follow material for reference as a standard. Rather, it must follow an extant basis. This pattern is usually seen in the written language and finds itself in news reports all the time. }
 
\par{8. 自分の経験を踏まえてこういう。 \hfill\break
To say such based on one's own experience. }
 
\par{9. これまでの状況を踏まえて得た最終結論です。 \hfill\break
This is the last argument gotten based on the conditions up to now. }
 
\par{10. 報告を\{踏まえた・踏まえての\}処理 \hfill\break
Processing based on the report }
 
\par{\textbf{Translation Note }: 踏まえる may also mean "to tread". }
 
\par{11. 両足で大地を\{踏まえて・踏んで\}立って! \hfill\break
Tread on the ground with both your legs and stand! }
      
\section{~をもとにして}
 
\par{ ~をもとにして is used to show that one bases off of the good qualities of something. It does not necessarily have to be 100\% congruent with the full truth in the circumstances. It shows willful change, and it is usually shortened in the spoken language to ~をもとに. When this pattern happens to be used as an attribute, it becomes ~をもとに \textbf{した }. }

\par{12. 私は ${\overset{\textnormal{おやゆず}}{\text{親譲}}}$ りの財産をもとにして ${\overset{\textnormal{とみ}}{\text{富}}}$ を作りました。 \hfill\break
I built my wealth on the fortune from my father. }

\par{13. ${\overset{\textnormal{うそ}}{\text{嘘}}}$ をもとにして ${\overset{\textnormal{おこな}}{\text{行}}}$ う。 \hfill\break
To act upon a lie. }

\par{14. 事実を基にして書かれた劇だ。 \hfill\break
It is a play written on the basis of the truth. }
 
\par{\textbf{漢字 Note }: Although this pattern is used both in the spoken and written language, when もと is written in 漢字, you have the options 基, 本, 素, 原, 源, 元, and 下. Thus, the meaning of this expression can be further refined. The first refers to a standard. The second refers to the foundation of something. The third refers to the subject matter. The fourth refers to raw materials. The fifth refers to a source. The sixth refers to the original way. The seventh refers to being under an influence. However, most natives cannot distinguish this well, and this is only knowledge relevant for when reading through literature. }
      
\section{~に沿って \& More}
 
\begin{center}
 \textbf{~に沿って }
\end{center}

\par{ ~に沿って is used in a physical or cause and effect sense. It can also be seen as ~に沿い is stiff writing, but even here it is considerably rare. The attribute forms ~に沿った and ~に沿っての are slightly different. The former can be used when A fits nicely with B, but there can be some deviance. However, there can be no such deviance with the latter. }

\par{15. ${\overset{\textnormal{じせい}}{\text{時勢}}}$ に沿って生きる。 \hfill\break
To live in consonance of the times. }
 
\par{16. 土手に沿って、 ${\overset{\textnormal{こけ}}{\text{苔}}}$ が生えている。 \hfill\break
Moss is growing along the embankment. }
 
\par{17. 事実に沿って、話しをしてくださいますか。 \hfill\break
Could you please talk along the facts (of the case)? }
 
\par{18. 湖に\{〇 沿った・X 沿っての\}公園を歩きました。 \hfill\break
I walked through the park that's alongside the lake. }
 
\par{19. 事業計画に\{沿った・沿っての\} ${\overset{\textnormal{じっしじょうきょう}}{\text{実施状況}}}$  \hfill\break
The status of implementation along the business plan }
 
\par{20. 僕は川に沿って歩いた。 \hfill\break
I walked along the river. }
 
\par{\textbf{Variant Note }: ~に沿って may be replaced by the suffixes ~ ${\overset{\textnormal{ぞ}}{\text{沿}}}$ い and ~ ${\overset{\textnormal{づた}}{\text{伝}}}$ い in the physical sense. }

\par{21. ${\overset{\textnormal{のき}}{\text{軒}}}$ づたいに ${\overset{\textnormal{そうこ}}{\text{倉庫}}}$ の裏に廻って行った。 \hfill\break
Along the eaves, I went around the back of the storehouse. }
 
\par{22. 川伝いに走り回る。 \hfill\break
To run around along the river. }

\par{23. ${\overset{\textnormal{こはん}}{\text{湖畔}}}$ 沿いの道路を歩く。 \hfill\break
To walk through the street alongside the lake shore. }

\par{24. 流れに沿うてやがて広野に出ると、 ${\overset{\textnormal{ちょうじょう}}{\text{頂上}}}$ は面白く切り ${\overset{\textnormal{きざ}}{\text{刻}}}$ んだようで、そこからゆるやかに美しい ${\overset{\textnormal{しゃせん}}{\text{斜線}}}$ が遠い ${\overset{\textnormal{すそ}}{\text{裾}}}$ まで伸びている山の ${\overset{\textnormal{は}}{\text{端}}}$ に月が色づいた。 ${\overset{\textnormal{のずえ}}{\text{野末}}}$ にただ一つの眺めである、その山の ${\overset{\textnormal{まった}}{\text{全}}}$ き姿を、淡い夕映えの空がくっきりと ${\overset{\textnormal{ふかはなだいろ}}{\text{濃深縹色}}}$ に描き出した。 \hfill\break
Along the stream, when we at last entered the wide, the moon changed colors in the edge of the mountain, which had a gently beautiful slant with the peak interestingly cutting it that stretched to the far base of the mountain. The light sunset sky sharply drew the perfect view of the mountain for the single view at the corner of the field a light yet deep indigo. \hfill\break
From 雪国 by 川端康成. }

\par{\textbf{漢字 Note }: The spellings 添う・副う exist. The first is seen with the sense of "addition" and the latter is seen with the sense of "expectation\slash satisfaction". }

\par{25. 身に添った悲しみ \hfill\break
The sadness on oneself }

\par{26. 花が期待に副って見事に咲いたよ。 \hfill\break
The flowers bloomed splendidly to my expectations. }

\begin{center}
\textbf{\hfill\break
~に即して }
\end{center}

\par{ ~に即して is reserved to writing and often very stiff. If used in the spoken language, it is very formal or official sounding. It is frequently used to show that something is based on\slash along things. In even more formal writing, it can be seen as ~に即し. }
 
\par{27. 時代に即した教育 \hfill\break
Education according to the times }
 
\par{28. 事実に即したドラマ \hfill\break
A drama in accordance with the facts }
 
\par{29. 法律に即した判断 \hfill\break
A decision in accordance with the law }
 
\par{30. 理念に即した行動 \hfill\break
Actions in accordance to an ideal }
 
\par{\textbf{Spelling Note }: As for spelling そくして, when the situation is based off of a given fact or observation you use 即して. However, if the situation is based on rules you use 則して, following the meanings of the 漢字 in question. There is a tendency to only use the former. }

\par{31. 前例に則して処理する。 \hfill\break
To process according to precedent. }

\par{\textbf{Attribute Note }: The attribute forms available depends on the spelling. If 即して, you can only use に即した. If 則して, you can use either ~に則しての or ~に則した. Although with ~に即して one can't feel any sense of excess or deviance, with ~に則して you can. Thus, the differences in possible attribute forms arise. It's opposite to that of ~に沿って's attribute forms, so be careful. }

\begin{center}
 \textbf{~にのっとって }
\end{center}

\par{ ~に則って is from the て形 of 則る, which is used to show that one protects some sort of tradition\slash rule set up in the past. This part of the meaning comes from an old verb のる in combination with 採る. This のり can still be found in many words written as 法: 御法 (humble law), 内法 (inside measure), etc. This is why 則る can also, but rarely, be seen written as 法る. It is no surprise that these two concepts are found in the word 法則 (law; rule). }

\par{ Whether this pattern is used in the spoken language or not is debatable, but it is very literary\slash rather stiff written speech modal. It can be used in official situations and in technical terms in relation to rules\slash costumes. It can also be seen as ~に則り, and the attribute form can be seen as ~に則っての or ~に則った without any restriction. }

\par{32. 遺言状に則って、遺産相続がなされます。 \hfill\break
Inheritance will be carried out according to the will. }

\par{33. 憲法に則り、最高裁判所の判断が示される。 \hfill\break
The Supreme Court's decision will be presented in accordance with the Constitution. }

\par{34. 仕来りに\{則った・則っての\}儀式 \hfill\break
A ceremony that is in accordance with tradition }
      
\section{Points of Confusion}
 
\par{ Although this section is not meant to exhaustively nitpick things, by the end of this section you will certainly have a better understanding of the differences between these phrases taught. Note that only those that truly seem synonymous will be addressed here. }

\par{35a. これまでの経験に基づいて、これからも頑張りたいと思う。X \hfill\break
35b. これまでの経験を元にして、これからも頑張りたいと思う。〇 }

\par{ If you paid any attention to the kind of words and situations above with ~に基づいて, you should have noticed that they are all in the same vein in regards to the basis of something in regards to fact or circumstances. The problem with the first sentence, though, will lead us into a serious issue: what are the differences between ~に基づいて, ~をもとにして, ~に沿って, ~に即して, and ~に則って. }
 
\par{1. ~に基づいて, ~に基いて being a less common spelling, shows that an action or circumstance is taking place based on fundamentals, matter at hand, or in combination with some sort of proof. }

\par{36. 判決に\{基づいて・基づき\}、刑が執行される。 \hfill\break
The verdict will be carried out according to the punishment. }
 
\par{37. 人は法に基づいて裁かれる。 \hfill\break
People are judged according\slash based on the law. }
 
\par{\textbf{Usage Note }: This pattern is often used after words like 事実, 証拠, 経験, 規則, 情報, etc. Words that often follow this pattern include 判断する, 行動する, 決める, 作る, 裁く, 言い渡す, 下す, 等. }
 
\par{2. ~をもとにして, on the other hand, shows a meaning of “ \textbf{basing on }fundamentals or matter at hand, while one capitalizes\slash utilizes on it or while one utilizes a certain part…”. The orientation of the expressions, thus, are quite different. }
 
\par{38. 資料を基にする。 \hfill\break
To base it on the materials. }
 
\par{39. 人の噂をもとにして判断してはいけないよ。 \hfill\break
I mustn't judge something based on people\textquotesingle s rumors! }
 
\par{40. これまでの経験をもとにして、頑張りたいと思う。 \hfill\break
I wish to try hard based on my experience up to now. }
 
\par{\textbf{Usage Note }: This pattern is often used after words like データ, 情報, 事件, 話, 噂, 等. Words that often follow this pattern include 頑張る, 書く, 作る, 対処する, 等. }
 
\par{3. ~に沿って, like the above two, all share the feature of being based on the matter at hand or fundamentals of something, but the peculiar part about this is that the meaning of “without there being a physical distance” is included. }
 
\par{41. 道に沿って花が植えられている。 \hfill\break
Flowers are planted along the street. }
 
\par{42. 塀にそって進んでくれ。(Vulgar command) \hfill\break
Move forward along the wall! }
 
\par{43. 川に沿って歩きなさい。 \hfill\break
Walk along the river. }
 
\par{44. 私の考えにそってやってほしい。 \hfill\break
I would like for it to be done along my ideas. }
 
\par{\textbf{Usage Note }: This pattern is often used after words like 道路, 道, 川, 壁, 歩道, 考え, 意向, 等. Words that often follow this pattern include 行く, 歩く, する, やる, 進む, 等. }
 
\par{Without getting into the other two remaining patterns, there are still times when all three structures appear to work. However, there will always be nuance differences based on the lines outlined above. }
 
\par{45. 事実\{ に 基づいて・をもとにして・にそって\}作ってある。 \hfill\break
It was created and based on facts. }
 
\par{Nuances differences still exist. The first sounds purely factual. The second sounds like it was based on fact but not entirely. The third sounds like it was closely aligned with the facts. In more complex sentences, these differences can be large enough for ungrammaticality calls if violated. }
 
\par{46. 取材\{〇 に基づいて・〇 をもとにして・X にそって\}書かれている。 \hfill\break
It is written based on collected data. }
 
\par{\textbf{Grammar Note }:  ~にそって is bad because collected data could be conflicting. If this were known not to be the case, the unnaturalness would go away. Contextual environment and what kind of word(s) you\textquotesingle re using a pattern with help you put things together within grammatical restrictions. }

\par{47. その図面\{〇 に基づいて・X をもと(にして)・X にそって\}、配線工事をお願いします。 \hfill\break
Please do the wiring work according to the blueprint. }

\par{48. この判例\{に基づいて・をもと(にして)\}裁判での闘い方を考えてみよう。 \hfill\break
Let's try to consider the ways of fighting in the trial based on this judicial precedent. }
 
\par{49. 高校で得た経験に基づいて頑張りたい。X  \textrightarrow  をもとにして \hfill\break
I will try hard based on experience I got from high school. }

\par{50. 過去の経験に基づいて判断した。〇 \hfill\break
I judged based on past experience. }
 
\par{\textbf{~に即して・に則って }}
 
\par{ Besides getting the reading wrong for the last one, which is ~にのっとって, the latter is often attached to words concerning standards\slash norms. Experiences are personal, so using it with such expressions would be unnatural. On the other hand, ~に即して also has the same meaning of “being based on\slash following”, just not with the particular restraint as ~に則って. If the spelling of the first is changed to ~に則して, it can have a meaning of showing something is done along rules\slash laws. }
 
\par{51. 伝統に\{則って・即して・基づいて\}行われている。 \hfill\break
It is being done according to tradition. }
 
\par{52. 不法入国者は法律に則して強制送還される。 \hfill\break
Illegal immigrants will be forcefully repatriated according to the law. }
    