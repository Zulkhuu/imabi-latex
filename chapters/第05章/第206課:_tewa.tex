    
\chapter{~ては}

\begin{center}
\begin{Large}
第206課: ~ては 
\end{Large}
\end{center}
 
\par{ In this lesson, we will learn about the grammar point ~ては, which is a combination of the conjunctive particle て and the bound particle は, used here in its contrastive role. Firstly, as a very brief reminder, below is a quick summation of how to conjugate with ~ては. Because this grammar pattern only concerns verbs, we\textquotesingle ll only need to worry about when~ては may become ~では. This occurs for \emph{Godan }verbs which end in ぶ・む・ぬ. }

\begin{ltabulary}{|P|P|P|}
\hline 

 \emph{Ichidan }Verb & 食べる + ては \textrightarrow  & 食べては \\ \cline{1-3}

 \emph{Godan }Verb & 立つ + ては \textrightarrow  \hfill\break
死ぬ + ては \textrightarrow  & 立っては \hfill\break
死んでは \\ \cline{1-3}

する & する + ては \textrightarrow  & しては \\ \cline{1-3}

くる & くる + ては \textrightarrow  & きては \\ \cline{1-3}

だ & だ + ては \textrightarrow  & では \\ \cline{1-3}

\end{ltabulary}

\par{\textbf{Curriculum Note }: This lesson does not cover how the contrastive は can go after the gerund use of the particle て as seen in phrases like ~について, ~に関して, ~に対して, etc. This is to be discussed later in IMABI. }
      
\section{The Usages of ~ては}
 
\par{\textbf{Usage 1: Trouble Causing Hypotheticals }}

\par{ The first usage of ~ては we will look at is how it is used to express situations that bring about anxiety, misgivings, uneasiness, fear, and\slash or inconvenience. }

\par{1. ${\overset{\textnormal{がっこう}}{\text{学校}}}$ の ${\overset{\textnormal{たいいく}}{\text{体育}}}$ の ${\overset{\textnormal{うんどう}}{\text{運動}}}$ で ${\overset{\textnormal{ぜんそくほっさ}}{\text{喘息発作}}}$ が ${\overset{\textnormal{で}}{\text{出}}}$ ては ${\overset{\textnormal{こま}}{\text{困}}}$ ります。 \hfill\break
I\textquotesingle m troubled with asthma attacks happening due to exercise at school in gymnastics. }

\par{2. ${\overset{\textnormal{ろうじん}}{\text{老人}}}$ に ${\overset{\textnormal{み}}{\text{見}}}$ られては ${\overset{\textnormal{こま}}{\text{困}}}$ ります。 \hfill\break
I\textquotesingle m embarrassed when I\textquotesingle m looked at by old people. }

\par{3. ${\overset{\textnormal{ごかい}}{\text{誤解}}}$ があっては ${\overset{\textnormal{こま}}{\text{困}}}$ りますので。 \hfill\break
Since it\textquotesingle d be worrisome if there is a misunderstanding. }

\par{4. ${\overset{\textnormal{きず}}{\text{傷}}}$ が ${\overset{\textnormal{のこ}}{\text{残}}}$ っては ${\overset{\textnormal{たいへん}}{\text{大変}}}$ だ。 \hfill\break
Things will be difficult if there are any wounds\slash nicks. }

\par{5. ${\overset{\textnormal{しか}}{\text{鹿}}}$ にやられては ${\overset{\textnormal{もと}}{\text{元}}}$ も ${\overset{\textnormal{こ}}{\text{子}}}$ もないので、 ${\overset{\textnormal{しんりん}}{\text{森林}}}$ の ${\overset{\textnormal{しゅうい}}{\text{周囲}}}$ を ${\overset{\textnormal{でんきさく}}{\text{電気柵}}}$ で ${\overset{\textnormal{かこ}}{\text{囲}}}$ っています。 \hfill\break
Since it\textquotesingle d all be for naught if it were ruined by deer, we have the surroundings of the forest enclosed in an electric fence. }

\begin{center}
~ては(いけない・ならない・だめ) 
\end{center}

\par{ We have actually already learned about this usage of ~ては when we learned about “must” and “must not” conditional phrases. As review, we will go over the basic combinations for these conditional phrases once more. \hfill\break
 \hfill\break
}
【Must Not】  
\par{・~てはいけない: This is used to tell someone he\slash she mustn\textquotesingle t do something. It isn\textquotesingle t simply used just to forcibly prohibit things. It could simply imply that the act in question is not favorable and that it will not be approved of by the speaker. This pattern is not typically used towards those who are higher in social status. \hfill\break
\hfill\break
・~てはならない: This is used to prohibit something with a sense of duty and responsibility. Whereas the phrase above is most frequently used to prohibit and\slash or disapprove of the action(s) of individuals, this phrase is most frequently used to objectively state things that ought not be allowed by society at large. Because of this, it is frequently used in law and other important, official documents. \hfill\break
\hfill\break
・~てはだめだ: This phrase is a more colloquial, softer variant of ~てはいけない. }

\par{6. お ${\overset{\textnormal{ぼん}}{\text{盆}}}$ に ${\overset{\textnormal{つち}}{\text{土}}}$ を ${\overset{\textnormal{ほ}}{\text{掘}}}$ ってはいけない。 \hfill\break
One mustn\textquotesingle t dig up dirt during the Festival of the Dead. }

\par{7. ${\overset{\textnormal{わたし}}{\text{私}}}$ たちが ${\overset{\textnormal{せきにん}}{\text{責任}}}$ を ${\overset{\textnormal{ほうき}}{\text{放棄}}}$ してはならない。 \hfill\break
We must not abdicate our responsibility. }

\par{8. ${\overset{\textnormal{どうぶつ}}{\text{動物}}}$ を ${\overset{\textnormal{ころ}}{\text{殺}}}$ してはだめなのはなぜなんだろう。 \hfill\break
Why is it that it\textquotesingle s bad to kill animals? }

\par{9. ${\overset{\textnormal{み}}{\text{見}}}$ た ${\overset{\textnormal{め}}{\text{目}}}$ で ${\overset{\textnormal{はんだん}}{\text{判断}}}$ してはいけません。 \hfill\break
One mustn\textquotesingle t judge based on appearance. }

\par{10. もう ${\overset{\textnormal{にど}}{\text{二度}}}$ と ${\overset{\textnormal{せんそう}}{\text{戦争}}}$ を ${\overset{\textnormal{お}}{\text{起}}}$ こしてはなりません。 \hfill\break
We mustn\textquotesingle t start a war ever again. }

\begin{center}
【Must】 
\end{center}

\par{・~なくてはいけない: The use of this pattern indicates that the listener ought to do something, not just because the speaker is demanding such action, but that not doing whatever it is will be unbeneficial\slash unfavorable for the speaker and\slash or listener. This is often used to make statements regarding common sense, morality, societal common wisdom, or current trends. Typically, the sentence is not interpreted as first-person unless a first-person pronoun is explicitly used. \hfill\break
\hfill\break
・~なくてはならない: This pattern is used for very affirmative commands out of a sense of duty, but this “must” pattern is directed more so toward individual responsibilities rather than societal ones. This pattern is also preferred in formal writing over the phrase above. }

\par{11. なんで ${\overset{\textnormal{べんきょう}}{\text{勉強}}}$ (を)しなくてはいけないの? \hfill\break
Why do you have to study? }

\par{\textbf{Sentence Note }: The “you” in the sentence is the indirect “you” and not necessarily literally second-person. This is also the case for Ex. 12 }

\par{12. ${\overset{\textnormal{おやし}}{\text{親知}}}$ らずは ${\overset{\textnormal{ぜったい}}{\text{絶対}}}$ に ${\overset{\textnormal{ぬ}}{\text{抜}}}$ かなくてはいけないんですか。 \hfill\break
Do you have to always pull wisdom teeth? }

\par{13. ${\overset{\textnormal{けいざい}}{\text{経済}}}$ も ${\overset{\textnormal{こくみん}}{\text{国民}}}$ ひとりひとりも ${\overset{\textnormal{つね}}{\text{常}}}$ に ${\overset{\textnormal{せいちょう}}{\text{成長}}}$ を ${\overset{\textnormal{めざ}}{\text{目指}}}$ さなくてはいけないのです。 \hfill\break
Not just the economy but also each and every citizen must constantly aim at growth. }

\par{14. ${\overset{\textnormal{けいたいでんわ}}{\text{携帯電話}}}$ は、 ${\overset{\textnormal{わたし}}{\text{私}}}$ にとってなくてはならない ${\overset{\textnormal{ひつじゅひん}}{\text{必需品}}}$ だ。 \hfill\break
A cellphone is a necessity that one can\textquotesingle t be without to me. }

\par{15. ${\overset{\textnormal{たいふう}}{\text{台風}}}$ でも ${\overset{\textnormal{しゅっきん}}{\text{出勤}}}$ しなくてはならない ${\overset{\textnormal{かいしゃ}}{\text{会社}}}$ の ${\overset{\textnormal{たいせい}}{\text{体制}}}$ をどう思いますか。 \hfill\break
Would do you think about company systems that mandate (workers) be present even during typhoons? }

\par{\textbf{Usage 2: Condition for Strong Emotional Response }}

\par{ This usage of ~ては is used to express that an action\slash state that has come about is the reason for a strong emotional response, whether that response be a rebuke, retort, or astonishment. }

\par{16. そこまで ${\overset{\textnormal{い}}{\text{言}}}$ われては ${\overset{\textnormal{はんろん}}{\text{反論}}}$ しないわけにはいかない。 \hfill\break
Being talked about to that degree, I have no choice but object. }

\par{17. そこまでからかわれ ${\overset{\textnormal{ばか}}{\text{馬鹿}}}$ にされては ${\overset{\textnormal{がまん}}{\text{我慢}}}$ (が)なりません。 \hfill\break
I can\textquotesingle t stand being so ridiculed and made a fool of. }

\par{18. ${\overset{\textnormal{すじ}}{\text{筋}}}$ の ${\overset{\textnormal{とお}}{\text{通}}}$ らないことを ${\overset{\textnormal{へいぜん}}{\text{平然}}}$ とやられては ${\overset{\textnormal{だま}}{\text{黙}}}$ っていられない。 \hfill\break
I can\textquotesingle t stay silent having something illogical so calmly be done to me. }

\par{19. ${\overset{\textnormal{いのち}}{\text{命}}}$ まで ${\overset{\textnormal{きけん}}{\text{危険}}}$ に ${\overset{\textnormal{さら}}{\text{晒}}}$ されては ${\overset{\textnormal{だま}}{\text{黙}}}$ っていられない。 \hfill\break
I can\textquotesingle t stay silent as even my life is put in danger. }

\par{20. ${\overset{\textnormal{ぬ}}{\text{濡}}}$ れ ${\overset{\textnormal{ぎぬ}}{\text{衣}}}$ を ${\overset{\textnormal{き}}{\text{着}}}$ せられては ${\overset{\textnormal{だま}}{\text{黙}}}$ ってはいられない。 \hfill\break
I can\textquotesingle t stay silent when I\textquotesingle m falsely accused. }

\par{\textbf{Usage 3: Repeated Action\slash Effect }}

\par{ Similar to the particle たり, ~ては is most frequently used to express the repetition (of a series of) actions. This is usage is more naturally emphatic than たり due to the presence of the contrastive\slash emphatic は. It is most frequently used in the written language and song lyrics as it adds a layer of expressive capability that isn\textquotesingle t necessarily indicative of standard conversation. }

\par{ Grammatically speaking, the second verbal element of the pattern V+ては+V needs to be in the 連用中止形. This is the form of a verb that can at times be used as nouns. Incidentally, this pattern can be treated as a complex nominal phrase as an effect (See Ex. 28). }

\par{21. ${\overset{\textnormal{か}}{\text{書}}}$ いては ${\overset{\textnormal{け}}{\text{消}}}$ し、 ${\overset{\textnormal{か}}{\text{書}}}$ いては ${\overset{\textnormal{け}}{\text{消}}}$ し、なんとかレポートを ${\overset{\textnormal{か}}{\text{書}}}$ き ${\overset{\textnormal{あ}}{\text{上}}}$ げた。 \hfill\break
I wrote and erased, wrote and erased in writing up the report. }

\par{22. ${\overset{\textnormal{はし}}{\text{走}}}$ っては ${\overset{\textnormal{やす}}{\text{休}}}$ み、 ${\overset{\textnormal{はし}}{\text{走}}}$ っては ${\overset{\textnormal{やす}}{\text{休}}}$ み、 ${\overset{\textnormal{すす}}{\text{進}}}$ み ${\overset{\textnormal{つづ}}{\text{続}}}$ けた。 \hfill\break
From running to resting, I continued forward. }

\par{23. ${\overset{\textnormal{め}}{\text{目}}}$ を ${\overset{\textnormal{と}}{\text{閉}}}$ じ、 ${\overset{\textnormal{いき}}{\text{息}}}$ を ${\overset{\textnormal{す}}{\text{吸}}}$ っては ${\overset{\textnormal{は}}{\text{吐}}}$ く。 \hfill\break
Eyes closed, I inhale and exhale. }

\par{24. ${\overset{\textnormal{じんせい}}{\text{人生}}}$ とは ${\overset{\textnormal{なみ}}{\text{波}}}$ のように ${\overset{\textnormal{よ}}{\text{寄}}}$ せては ${\overset{\textnormal{かえ}}{\text{返}}}$ しているものである。 \hfill\break
Life breaks and retreats like waves. }

\par{25. あの ${\overset{\textnormal{ねこ}}{\text{猫}}}$ は、 ${\overset{\textnormal{し}}{\text{死}}}$ んでは ${\overset{\textnormal{い}}{\text{生}}}$ き ${\overset{\textnormal{かえ}}{\text{返}}}$ り、 ${\overset{\textnormal{い}}{\text{生}}}$ き ${\overset{\textnormal{かえ}}{\text{返}}}$ っては ${\overset{\textnormal{し}}{\text{死}}}$ に、まるで ${\overset{\textnormal{ふじみ}}{\text{不死身}}}$ だ。 \hfill\break
That cat constantly dies and comes back to life; it\textquotesingle s as if it\textquotesingle s immortal. }

\par{26. これまで ${\overset{\textnormal{けっこん}}{\text{結婚}}}$ しては ${\overset{\textnormal{りこん}}{\text{離婚}}}$ と ${\overset{\textnormal{さいこん}}{\text{再婚}}}$ を ${\overset{\textnormal{かさ}}{\text{重}}}$ ねた ${\overset{\textnormal{ふじこ}}{\text{不二子}}}$ は、なんと ${\overset{\textnormal{よん}}{\text{4}}}$ ${\overset{\textnormal{ど}}{\text{度}}}$ もデキ ${\overset{\textnormal{こん}}{\text{婚}}}$ ! \hfill\break
Fujiko, who has up till now repeatedly been married, divorced, and then remarried, has had four shotgun weddings! }

\par{27. ${\overset{\textnormal{さまざま}}{\text{様々}}}$ な ${\overset{\textnormal{けしき}}{\text{景色}}}$ が現れては ${\overset{\textnormal{き}}{\text{消}}}$ えていった。 \hfill\break
Various sceneries appeared and went away. }

\par{28. ${\overset{\textnormal{うつじょうたい}}{\text{鬱状態}}}$ になると ${\overset{\textnormal{お}}{\text{落}}}$ ち ${\overset{\textnormal{つ}}{\text{着}}}$ きがなくなり、ずっと ${\overset{\textnormal{へや}}{\text{部屋}}}$ を ${\overset{\textnormal{ある}}{\text{歩}}}$ き ${\overset{\textnormal{まわ}}{\text{回}}}$ ったり、 ${\overset{\textnormal{た}}{\text{立}}}$ っては ${\overset{\textnormal{すわ}}{\text{座}}}$ りを ${\overset{\textnormal{く}}{\text{繰}}}$ り ${\overset{\textnormal{かえ}}{\text{返}}}$ したり(する)などが ${\overset{\textnormal{み}}{\text{見}}}$ られます。 \hfill\break
When someone becomes depressed, one will see behaviors such as loss of composure, walking constantly back and forth in rooms, and repeatedly standing up and sitting down. }

\par{29. ${\overset{\textnormal{ふく}}{\text{服}}}$ を ${\overset{\textnormal{えら}}{\text{選}}}$ ぶときしっくりこなくて、 ${\overset{\textnormal{ぬ}}{\text{脱}}}$ いでは ${\overset{\textnormal{き}}{\text{着}}}$ てを ${\overset{\textnormal{なんかい}}{\text{何回}}}$ も ${\overset{\textnormal{く}}{\text{繰}}}$ り ${\overset{\textnormal{かえ}}{\text{返}}}$ してしまいます。 \hfill\break
When I pick out clothes, I can\textquotesingle t get it together and I end up repeatedly taking clothes off and putting them back on many times over. }

\par{\textbf{Grammar Note }: Verbs that end up being one-mora long when put in the 連用中止形 usually manifest in the て form when the “V + ては + V” is used as a noun. }

\par{30. ${\overset{\textnormal{ゆうしょく}}{\text{夕食}}}$ が ${\overset{\textnormal{おそ}}{\text{遅}}}$ いうえにたくさん ${\overset{\textnormal{た}}{\text{食}}}$ べては ${\overset{\textnormal{ふと}}{\text{太}}}$ るのは ${\overset{\textnormal{あ}}{\text{当}}}$ たり ${\overset{\textnormal{まえ}}{\text{前}}}$ だ。 \hfill\break
On top of dinner being late, it\textquotesingle s only natural to be gaining weight each time one eats. }

\par{\textbf{Usage 4: Infallible Repeat }}

\par{ The purpose of this usage of ~ては is to explain how something always happens under the condition that it marks. Think of this as an amalgam of the three usages above being intended simultaneously. }

\par{31. ${\overset{\textnormal{た}}{\text{他}}}$ の ${\overset{\textnormal{ひと}}{\text{人}}}$ と ${\overset{\textnormal{おな}}{\text{同}}}$ じようなことをしていては、いつまでも ${\overset{\textnormal{せいこう}}{\text{成功}}}$ しない。 \hfill\break
You will not succeed forever by repeatedly doing the same things as other people. }

\par{32. ${\overset{\textnormal{みな}}{\text{皆}}}$ のように ${\overset{\textnormal{あそ}}{\text{遊}}}$ んでは ${\overset{\textnormal{なに}}{\text{何}}}$ もプラスにはならないよ。 \hfill\break
There will be no plus to messing around like everyone else. }

\par{33. コソコソしていては ${\overset{\textnormal{なにごと}}{\text{何事}}}$ もうまくいかない。 \hfill\break
Nothing will go well from constantly being sneaky about things. }

\par{34. ${\overset{\textnormal{じぶん}}{\text{自分}}}$ の ${\overset{\textnormal{からだ}}{\text{身体}}}$ を ${\overset{\textnormal{ひてい}}{\text{否定}}}$ していてはダイエットは ${\overset{\textnormal{せいこう}}{\text{成功}}}$ できない。 \hfill\break
You cannot succeed with a diet by constantly denying what your body (is trying to tell you). }

\par{35. ${\overset{\textnormal{せ}}{\text{急}}}$ いては ${\overset{\textnormal{こと}}{\text{事}}}$ を ${\overset{\textnormal{そん}}{\text{損}}}$ じる。 \hfill\break
Haste makes waste. }

\par{\textbf{Usage 5: ~てはみる }}

\par{ The purpose of ~てはみる is to express that although one will make an attempt at doing something, one doesn\textquotesingle t have the confidence and\slash or doesn\textquotesingle t expect a good result. }

\par{36. ${\overset{\textnormal{かんが}}{\text{考}}}$ えてはみるよ。 \hfill\break
I\textquotesingle ll think about it (but I\textquotesingle m not so sure I\textquotesingle ll be okay with it). }

\par{37. やってはみるけど、うまくいくかどうかは ${\overset{\textnormal{わ}}{\text{分}}}$ からない。 \hfill\break
I\textquotesingle ll definitely try, but I don\textquotesingle t know whether it\textquotesingle ll go well. }

\par{38. ${\overset{\textnormal{いちおうまいかい}}{\text{一応毎回}}}$ クリックしてはみるものの、 ${\overset{\textnormal{いっかい}}{\text{一回}}}$ も ${\overset{\textnormal{あ}}{\text{当}}}$ たったことありません。 \hfill\break
I at any rate try clicking it every time, but I have yet to win even once. }

\par{39. ${\overset{\textnormal{しょくじ}}{\text{食事}}}$ だって ${\overset{\textnormal{き}}{\text{気}}}$ を ${\overset{\textnormal{つ}}{\text{付}}}$ けてるし、 ${\overset{\textnormal{いろいろ}}{\text{色々}}}$ と ${\overset{\textnormal{ちょうせん}}{\text{挑戦}}}$ してはみるけれど、どんなに ${\overset{\textnormal{がんば}}{\text{頑張}}}$ ってもヤセない。 \hfill\break
I pay attention to what my meals are and I try all sorts of challenges, but not matter how much I try, I don\textquotesingle t get slimmer. }

\par{40. ${\overset{\textnormal{えいたんご}}{\text{英単語}}}$ を ${\overset{\textnormal{おぼ}}{\text{覚}}}$ えてはみるけど、いつも ${\overset{\textnormal{おぼ}}{\text{覚}}}$ えられない、 ${\overset{\textnormal{おぼ}}{\text{覚}}}$ えられる ${\overset{\textnormal{き}}{\text{気}}}$ がしない。 \hfill\break
I\textquotesingle ll try committing English vocabulary to memory, but I always can\textquotesingle t remember, or I\textquotesingle m always not in the mood to be able to remember them. }

\par{\textbf{Usage 6: ~てはどうか }}

\par{ By using the pattern ~てはどうか, you can suggest that someone do something. There is an implication that the suggestion hasn\textquotesingle t been tried yet by the speaker, thus the use of the contrastive は. In more formal speech, this is expressed as ~てはいかがですか. }

\par{41. ${\overset{\textnormal{まほう}}{\text{魔法}}}$ の ${\overset{\textnormal{せかい}}{\text{世界}}}$ に ${\overset{\textnormal{き}}{\text{来}}}$ てはどう? \hfill\break
How about coming to the world of magic? }

\par{42. ${\overset{\textnormal{て}}{\text{手}}}$ を ${\overset{\textnormal{か}}{\text{貸}}}$ すから ${\overset{\textnormal{くるまいす}}{\text{車椅子}}}$ に ${\overset{\textnormal{の}}{\text{乗}}}$ ってはどうか。 \hfill\break
How about if you use the while chair if I lend you my hands? }

\par{43. ${\overset{\textnormal{す}}{\text{捨}}}$ てる ${\overset{\textnormal{まえ}}{\text{前}}}$ に ${\overset{\textnormal{りよう}}{\text{利用}}}$ してはどうですか。 \hfill\break
How about using it before throwing it away? }

\par{44. ${\overset{\textnormal{ひしょち}}{\text{避暑地}}}$ へ ${\overset{\textnormal{あし}}{\text{足}}}$ を ${\overset{\textnormal{の}}{\text{延}}}$ ばしてはいかがですか。 \hfill\break
How would you like going to relax at a summer resort? }

\par{45. ご ${\overset{\textnormal{らん}}{\text{覧}}}$ になってはいかがですか。 \hfill\break
How would you like seeing it? }

\par{\textbf{Usage 7: ~てはいる }}

\par{ The contrastive marker は may be inserted inside ~ている to imply that one is doing something, or that what is in question is indeed the case, but that other actions\slash states are not being undertaken\slash happening. }

\par{46.ここ ${\overset{\textnormal{ご}}{\text{5}}}$ ${\overset{\textnormal{ねん}}{\text{年}}}$ は ${\overset{\textnormal{おな}}{\text{同}}}$ じ ${\overset{\textnormal{おとこ}}{\text{男}}}$ と ${\overset{\textnormal{どうせい}}{\text{同棲}}}$ してはいるが、 ${\overset{\textnormal{なに}}{\text{何}}}$ も ${\overset{\textnormal{きず}}{\text{築}}}$ いてはいない。 \hfill\break
Although I\textquotesingle ve been living with the same man for the last five years, we haven\textquotesingle t built anything (together). }

\par{47. お ${\overset{\textnormal{とう}}{\text{父}}}$ さんは ${\overset{\textnormal{な}}{\text{亡}}}$ くなってはいるけれど、その ${\overset{\textnormal{そんざい}}{\text{存在}}}$ は ${\overset{\textnormal{かぞく}}{\text{家族}}}$ のなかにずっとあるものです。 \hfill\break
Although our father is no longer with us, his being remains forever within our family. }

\par{48. ${\overset{\textnormal{かいしゃ}}{\text{会社}}}$ としてはそういう ${\overset{\textnormal{いと}}{\text{意図}}}$ でやってはいるけど、その ${\overset{\textnormal{いと}}{\text{意図}}}$ どおりに ${\overset{\textnormal{つた}}{\text{伝}}}$ わっているかはまた ${\overset{\textnormal{べつもんだい}}{\text{別問題}}}$ なわけで、それはしょうがないのではないか。 \hfill\break
As a company, it is doing so with that intent, but whether it\textquotesingle s being transmitted as intended is a separate problem, but }

\par{49. トランプ ${\overset{\textnormal{だいとうりょう}}{\text{大統領}}}$ が ${\overset{\textnormal{ていあん}}{\text{提案}}}$ した ${\overset{\textnormal{よさん}}{\text{予算}}}$ は ${\overset{\textnormal{きょくたん}}{\text{極端}}}$ に ${\overset{\textnormal{さくげん}}{\text{削減}}}$ されてはいるが、 ${\overset{\textnormal{ざいせいししゅつ}}{\text{財政支出}}}$ は ${\overset{\textnormal{どうし}}{\text{同氏}}}$ が ${\overset{\textnormal{だいとうりょうせん}}{\text{大統領選}}}$ の ${\overset{\textnormal{せんきょかつどうちゅう}}{\text{選挙活動中}}}$ にこだわった ${\overset{\textnormal{もんだい}}{\text{問題}}}$ ではなかった。 \hfill\break
Although the budget presented by President Trump is exclusively reduced, government spending was not a problem that he fussed over during his campaigning in the presidential election. \hfill\break
 \hfill\break
50. ${\overset{\textnormal{さまざま}}{\text{様々}}}$ な ${\overset{\textnormal{ぶたい}}{\text{舞台}}}$ で ${\overset{\textnormal{かくはいぜつ}}{\text{核廃絶}}}$ が ${\overset{\textnormal{ぎろん}}{\text{議論}}}$ されてはいるが、 ${\overset{\textnormal{かっこく}}{\text{各国}}}$ の ${\overset{\textnormal{おもわく}}{\text{思惑}}}$ から ${\overset{\textnormal{かなら}}{\text{必}}}$ ずしも ${\overset{\textnormal{しんてん}}{\text{進展}}}$ しているとは ${\overset{\textnormal{い}}{\text{言}}}$ い ${\overset{\textnormal{がた}}{\text{難}}}$ い。 \hfill\break
Although total abolition of nuclear weapons is discussed in various settings, it is difficult to say that it is necessarily progressing due to each country\textquotesingle s ulterior motives. }

\par{\textbf{Usage 8: ~ではない }}

\par{ As the last use of ~ては to be discussed in this lesson, we return to a grammar point that was first introduced in Lesson 9. Now that you understand how the particle は is exclusively treated as a contrast marker after the particle て, it is only natural to conclude that some degree of contrast is implied with ~ではない, which is indeed the case. }

\par{51. ${\overset{\textnormal{じったい}}{\text{実態}}}$ は ${\overset{\textnormal{かなら}}{\text{必}}}$ ずしもそうではない。 \hfill\break
The reality is not always so. }

\par{52. イチジクは果物ではない。 \hfill\break
Figs are not fruit. }

\par{53. しばしば、この ${\overset{\textnormal{そしき}}{\text{組織}}}$ は ${\overset{\textnormal{けっか}}{\text{結果}}}$ ではなく ${\overset{\textnormal{かてい}}{\text{過程}}}$ や ${\overset{\textnormal{かんりょうしゅぎ}}{\text{官僚主義}}}$ に ${\overset{\textnormal{き}}{\text{気}}}$ をとられている。 \hfill\break
Too often the focus of this organization has not been on results, but on bureaucracy and process. }

\par{54. どの ${\overset{\textnormal{くに}}{\text{国}}}$ も ${\overset{\textnormal{ぐんじてき}}{\text{軍事的}}}$ 、 ${\overset{\textnormal{ざいせいてき}}{\text{財政的}}}$ な ${\overset{\textnormal{ふたん}}{\text{負担}}}$ の ${\overset{\textnormal{ふきんこう}}{\text{不均衡}}}$ に ${\overset{\textnormal{た}}{\text{耐}}}$ える ${\overset{\textnormal{ひつよう}}{\text{必要}}}$ があるべきではない。 \hfill\break
No nation should have to bear a disproportionate share of the burden, militarily, or financially. }

\par{55. ブラジルのテメル ${\overset{\textnormal{だいとうりょう}}{\text{大統領}}}$ は ${\overset{\textnormal{じゅうく}}{\text{19}}}$ ${\overset{\textnormal{にちごぜん}}{\text{日午前}}}$ 、 ${\overset{\textnormal{こくれんそうかい}}{\text{国連総会}}}$ で、 ${\overset{\textnormal{きたちょうせん}}{\text{北朝鮮}}}$ を ${\overset{\textnormal{つよ}}{\text{強}}}$ く ${\overset{\textnormal{ひはん}}{\text{批判}}}$ するとともに、 ${\overset{\textnormal{ぐんじりょく}}{\text{軍事力}}}$ に ${\overset{\textnormal{たよ}}{\text{頼}}}$ るのではなく、 ${\overset{\textnormal{かっこく}}{\text{各国}}}$ が ${\overset{\textnormal{きょうりょく}}{\text{協力}}}$ して ${\overset{\textnormal{へいわてき}}{\text{平和的}}}$ な ${\overset{\textnormal{かいけつほうほう}}{\text{解決方法}}}$ を ${\overset{\textnormal{さぐ}}{\text{探}}}$ るべきだと ${\overset{\textnormal{うった}}{\text{訴}}}$ えました。 \hfill\break
President Temer of Brazil on the morning of the 19 th at the United Nations General Assembly, along with strongly criticizing North Korea, urged that each nation not rely on military force but rather search for peaceful solutions through mutual collaboration. }

\begin{center}
\textbf{Contractions }
\end{center}

\par{ For Usages 1, 2, 3, 4, and 8, you may find ては contracted to ちゃ(あ) ・じゃ(あ) . There isn't any difference in meaning whether the vowel \slash a\slash  is elongated. }

\par{56. ${\overset{\textnormal{わる}}{\text{悪}}}$ いことはしちゃいけないよ。 \hfill\break
You mustn\textquotesingle t do anything bad. }

\par{57. ${\overset{\textnormal{さいしょ}}{\text{最初}}}$ から ${\overset{\textnormal{あたら}}{\text{新}}}$ しい ${\overset{\textnormal{つと}}{\text{勤}}}$ めに ${\overset{\textnormal{おく}}{\text{遅}}}$ れちゃあ ${\overset{\textnormal{たいへん}}{\text{大変}}}$ だろう? \hfill\break
Wouldn\textquotesingle t it be terrible if you were late to your new job from the very beginning? }

\par{58. ${\overset{\textnormal{いど}}{\text{挑}}}$ むって ${\overset{\textnormal{い}}{\text{言}}}$ われちゃあ、 ${\overset{\textnormal{おれ}}{\text{俺}}}$ らがやんなきゃなあ。 \hfill\break
We gotta do it if we\textquotesingle re told to have a throw down. }

\par{59. ${\overset{\textnormal{じめつ}}{\text{自滅}}}$ してちゃ ${\overset{\textnormal{か}}{\text{勝}}}$ てるわけがない! \hfill\break
You can\textquotesingle t possibly when by constantly ruining yourself! }

\par{60. ${\overset{\textnormal{やさ}}{\text{優}}}$ しいだけじゃ ${\overset{\textnormal{こま}}{\text{困}}}$ るわ。 \hfill\break
It\textquotesingle d just be trouble if (he\slash she) were just nice. }

\par{61. いや、 ${\overset{\textnormal{よ}}{\text{呼}}}$ んじゃあまずい。 \hfill\break
No, now\textquotesingle s not the time to call for (him\slash her). }

\par{62. ワガママで ${\overset{\textnormal{じぶんほんい}}{\text{自分本位}}}$ な ${\overset{\textnormal{おとこ}}{\text{男}}}$ じゃ(あ) ${\overset{\textnormal{こま}}{\text{困}}}$ る。 \hfill\break
I\textquotesingle d be in a rut with a man who\textquotesingle s selfish and egotistic. }

\par{63. いつまでも ${\overset{\textnormal{な}}{\text{泣}}}$ いてちゃいけないね。 \hfill\break
You mustn\textquotesingle t cry forever. }

\par{64. ${\overset{\textnormal{だま}}{\text{黙}}}$ ってちゃ ${\overset{\textnormal{こま}}{\text{困}}}$ るんだよ。 \hfill\break
[I\textquotesingle m\slash we\textquotesingle re] going to go through a bunch of trouble with you staying quiet. }

\par{ 65. ぐずぐずしてちゃ ${\overset{\textnormal{だめ}}{\text{駄目}}}$ だ! \hfill\break
You can\textquotesingle t just be dawdling! }
    