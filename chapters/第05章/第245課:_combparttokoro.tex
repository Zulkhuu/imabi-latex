    
\chapter{Combination Particles with ところ}

\begin{center}
\begin{Large}
第245課: Combination Particles with ところ 
\end{Large}
\end{center}
 
\par{ While learning about the particle ところ, you'll be introduced to combination particles. We've seen many instances where 2(+) particles can be used together, but we haven't really dealt with instances where one or more used together make a new phrase. }

\par{Combination Particles : A particle phrase made up of either more than one particle or a particle(s)  with another part of speech . }
      
\section{ところ}
 
\par{ The noun 所 means "place". There are four usages of ところ. These usages can be used in many situations. It can be used to show place, moment of time, situation, or the substance of a matter. More specific things that it can show include the following. }

\begin{itemize}

\item Show a specific space or spot. 
\item Show a particular place, part, or position. 
\item Show a point, address, region, etc. 
\item Shows the current time in phrases like このところ and 今日の所. 
\item Show a particular time. 
\item With a meaning of content\slash outline. 
\item 早いところ = Promptly 
\end{itemize}

\par{ When seen after a verb, tense is very important for correct interpretation. With the \textbf{non-past }tense, ところ means "just about to\dothyp{}\dothyp{}\dothyp{}". When used with the \textbf{past }tense of a verb, it shows what you "just did".  This can only be used to indicate the time when a certain action has literally just been done . If you want to say that you just recently did something, you should use ~たばかりです. }

\par{ When used with the \textbf{progressive }, it shows that you are now doing a certain action. The "just" may also be emphasized with adverbs such as ちょうど. This is simply an extension of translation of its usage to show time. The tense just shows the time factor. }
 
\par{1. 帰るところだ。 \hfill\break
I'm about to go home. }
 
\par{2. 学校に行くところでした。 \hfill\break
I was about to go to school. }
 
\par{3. 事故を起こすところだったよ。 \hfill\break
I was about to cause an accident. }
 
\par{4. 忘れるところでした。 \hfill\break
I almost forgot. }
 
\par{5. よく晴れるところもあります。 \hfill\break
There are also places where it is quite clear. }
 
\par{6. 雨の降るところもありそうです。 \hfill\break
There also seems to be places where it's raining. }
 
\par{7. 陽が沈むところです。 \hfill\break
The sun is about to set. }
 
\par{8. 会社を出たところです。 \hfill\break
I just left the company. }
 
\par{9. 彼の言ったところが分かりませんでした。 \hfill\break
I didn't understand what \emph{he }said. }
 
\par{10. この国の議会では、B党の占めるところが大きい。 \hfill\break
In this\slash our country's parliament, B party is the majority. }
 
\par{11. 彼女の理論は、仲間の調査によるところが大きい。 \hfill\break
Her theory mainly relies on the research by his friends. }
 
\par{12. 人は建物を余すところなく建てている。 \hfill\break
People are building buildings exhaustively. }
 
\par{13. もう ${\overset{\textnormal{いっぽ}}{\text{一歩}}}$ といったところですが。 \hfill\break
You're just one more step away, but\dothyp{}\dothyp{}\dothyp{} }
 
\par{14. 私たちは田舎に新築の家をちょうど買ったところです。 \hfill\break
We have just bought a new house in the countryside. }
 
\par{\textbf{Grammar Note }: You can also see ところ after speech modals like ~ようとする ~てしまった. It works the same way. }
      
\section{Combination Particles with ところ}
 
\par{ The meaning of a combination will reflect that of its individual parts put together. However, for the conjunctive particles that are created with ところ, this may not be that obvious. These combinations below may also not be used as conjunctive particles and can be more literally interpreted in different situations. Don't let punctuation get the best of you. }

\begin{ltabulary}{|P|P|P|}
\hline 

 & Meaning & Usage \\ \cline{1-3}

ところへ & Just when & Shows coincidence of events after -ている or -た. \\ \cline{1-3}

ところが & But\slash while & Shows unexpected bad outcome. \\ \cline{1-3}

ところで & Even if (X) were to\dothyp{}\dothyp{}\dothyp{} & Poses an undesirable hypothetical situation. \\ \cline{1-3}

ところを & Although usually & Shows something unexpected is happening \\ \cline{1-3}

\end{ltabulary}

\par{\textbf{Grammar Notes }: }

\par{1. ところ may often be shortened to just とこ. }

\par{2. ところへ may also be ところに. }

\par{3. The particle ところ either shows a sequence or contradiction depending on context. The last usage is synonymous with ところが. These are conjunction usages of ところ. }

\begin{center}
 \textbf{Examples }
\end{center}

\par{15. 科学の宿題をし始めようとしているところへ、僕の友だちが遊びにきた。 \hfill\break
Just when I was about to start doing my science homework, my friends came to play. }

\par{16. ${\overset{\textnormal{にわか}}{\text{俄}}}$ ${\overset{\textnormal{あめ}}{\text{雨}}}$ がやんだところへ、 ${\overset{\textnormal{たいふう}}{\text{台風}}}$ が町を ${\overset{\textnormal{おそ}}{\text{襲}}}$ った。 \hfill\break
Just when the shower quit, a typhoon hit the town. }

\par{17. 彼女は僕が好きらしいんだよね。ところが、彼女の方は僕が知ってるなんて全く気がついてないよ。 \hfill\break
She (seems) to love me you know. But, she doesn't realize that I know it at all. }

\par{\textbf{Grammar Note }: As you can see, ところが may also be used as a conjunction at the beginning of a sentence to mean "even so\slash however\slash but". }

\par{18. 半分したところで、彼はもう済ませたよ。 \hfill\break
I've finished half, but he's already finished! }

\par{19. いつもなら6時に起きるところを、今朝は ${\overset{\textnormal{ねぼう}}{\text{寝坊}}}$ して、学校に遅刻してしまった。 \hfill\break
Although I always wake up at six, I overslept this morning and became late to school. }
    