    
\chapter{Intransitive \& Transitive}

\begin{center}
\begin{Large}
第220課: Intransitive \& Transitive: Part II 
\end{Large}
\end{center}
 
\par{ In this second lesson on verbs with both intransitive and transitive usages, we\textquotesingle ll continue to uncover peculiarities in Japanese at the individual word basis. }
      
\section{巻く, 運ぶ, 吹く, 催す, 結ぶ, \& 頼る}
 
\begin{center}
\textbf{巻く } 
\end{center}

\par{\emph{ }巻く can be used to “to wind\slash coil\slash etc.” as an intransitive or a transitive verb. Its intransitive usage is not that common, and it is usually rephrased out of the sentence, often with 巻き付く. }

\par{1. ${\overset{\textnormal{おも}}{\text{思}}}$ いがけないほど ${\overset{\textnormal{うず}}{\text{渦}}}$ (が) ${\overset{\textnormal{ま}}{\text{巻}}}$ いている。 \hfill\break
The whirlpool is swirling beyond expectation. \hfill\break
It is whirling beyond expectation. (Without が). }

\par{2. ${\overset{\textnormal{じょうくう}}{\text{上空}}}$ の ${\overset{\textnormal{くも}}{\text{雲}}}$ が ${\overset{\textnormal{うず}}{\text{渦}}}$ (を) ${\overset{\textnormal{ま}}{\text{巻}}}$ いている。 \hfill\break
The clouds in the sky above are swirling in a whirlpool. }

\par{3. ${\overset{\textnormal{くすり}}{\text{薬}}}$ を ${\overset{\textnormal{ぬ}}{\text{塗}}}$ って ${\overset{\textnormal{ほうたい}}{\text{包帯}}}$ を ${\overset{\textnormal{ま}}{\text{巻}}}$ いてください。 \hfill\break
Please apply the medicine and wind a bandage (around the wound). }

\par{4. くるりと ${\overset{\textnormal{ま}}{\text{巻}}}$ いた ${\overset{\textnormal{はり}}{\text{針}}}$ のような ${\overset{\textnormal{なが}}{\text{長}}}$ い ${\overset{\textnormal{くち}}{\text{口}}}$ が ${\overset{\textnormal{とくちょう}}{\text{特徴}}}$ です。 \hfill\break
It\textquotesingle s long and completely wrapped up mouth, which is akin to a needle, is its trait. }

\par{5. ${\overset{\textnormal{へび}}{\text{蛇}}}$ がとぐろを ${\overset{\textnormal{ま}}{\text{巻}}}$ いている。 \hfill\break
The snake has coiled itself up. }

\par{\textbf{Spelling Note }: とぐろ \emph{ }may also be spelled as 蜷局. }

\par{6. ${\overset{\textnormal{ほそ}}{\text{細}}}$ いツルが ${\overset{\textnormal{ま}}{\text{巻}}}$ (き ${\overset{\textnormal{つ}}{\text{付}}}$ )いている。 \hfill\break
The slender vines are twined around. }

\par{\textbf{Spelling Note }: ツル may also be spelled as 蔓. }

\par{7. ${\overset{\textnormal{しろ}}{\text{城}}}$ が ${\overset{\textnormal{しろ}}{\text{白}}}$ い ${\overset{\textnormal{けむり}}{\text{煙}}}$ に\{ ${\overset{\textnormal{つつ}}{\text{包}}}$ まれている・ ${\overset{\textnormal{かこ}}{\text{囲}}}$ まれている・ ${\overset{\textnormal{ま}}{\text{巻}}}$ かれている\}。 \hfill\break
The castle is enveloped by white smoke. }

\begin{center}
\textbf{蒔く, 播く, \& 撒く }
\end{center}

\par{ There are three more まく that need to be addressed. All three are solely transitive verbs. }

\par{蒔く: Used to mean “to plant\slash sow\slash seed.” \hfill\break
播く: Interchangeable with 蒔く. \hfill\break
撒く: Used to mean “to scatter.” It may also be used in a figurative sense such as in “to spread (rumors).” }

\par{8. ${\overset{\textnormal{やさい}}{\text{野菜}}}$ の ${\overset{\textnormal{たね}}{\text{種}}}$ を\{ ${\overset{\textnormal{ま}}{\text{蒔}}}$ ・ ${\overset{\textnormal{ま}}{\text{播}}}$ いた\}のに、 ${\overset{\textnormal{め}}{\text{芽}}}$ が ${\overset{\textnormal{で}}{\text{出}}}$ ません。 \hfill\break
Even though I\textquotesingle ve sowed the vegetable seeds, they haven\textquotesingle t sprouted. }

\par{9. ベランダや ${\overset{\textnormal{げんかんさき}}{\text{玄関先}}}$ だけに ${\overset{\textnormal{まめ}}{\text{豆}}}$ を ${\overset{\textnormal{ま}}{\text{撒}}}$ いたとしても、きちんと ${\overset{\textnormal{かいしゅう}}{\text{回収}}}$ することは ${\overset{\textnormal{ひつよう}}{\text{必要}}}$ です。 \hfill\break
Even if you\textquotesingle ve only scattered beans on your veranda and at your front door, it is necessary that you properly retrieve them. }

\par{\textbf{Cultural Note }: This is a reference to cleanup efforts after having scattered beans as part of commemorating 節分 (the last day of winter in the traditional Japanese calendar). }

\par{10. ${\overset{\textnormal{わる}}{\text{悪}}}$ い ${\overset{\textnormal{うわさ}}{\text{噂}}}$ を ${\overset{\textnormal{ま}}{\text{撒}}}$ き ${\overset{\textnormal{ち}}{\text{散}}}$ らしている ${\overset{\textnormal{ひと}}{\text{人}}}$ も、あまり ${\overset{\textnormal{しんよう}}{\text{信用}}}$ されていないんでしょう。 \hfill\break
The people spreading awful rumors are also likely not all that trusted either. }

\par{11. うちの ${\overset{\textnormal{ねこ}}{\text{猫}}}$ はほぼ ${\overset{\textnormal{まいにち}}{\text{毎日}}}$ 、 ${\overset{\textnormal{あそ}}{\text{遊}}}$ びでトイレ(の) ${\overset{\textnormal{すな}}{\text{砂}}}$ を ${\overset{\textnormal{ま}}{\text{撒}}}$ き ${\overset{\textnormal{ち}}{\text{散}}}$ らしてしまいます。 \hfill\break
My cat scatters the sand in its litter box out of play almost every day. }

\begin{center}
\textbf{運ぶ }
\end{center}

\par{ As an intransitive verb, 運ぶ means “to proceed\slash to go (well),” but it is far more commonly used as a transitive verb to mean “to carry\slash transport.” }

\par{12. ${\overset{\textnormal{しょくぶつ}}{\text{植物}}}$ の ${\overset{\textnormal{おお}}{\text{多}}}$ くは、 ${\overset{\textnormal{たね}}{\text{種}}}$ を ${\overset{\textnormal{つく}}{\text{作}}}$ るために ${\overset{\textnormal{かふん}}{\text{花粉}}}$ を ${\overset{\textnormal{かぜ}}{\text{風}}}$ で ${\overset{\textnormal{はこ}}{\text{運}}}$ ばなければなりません。 \hfill\break
A lot of plants must carry their pollen via the wind to create their seeds. }

\par{13. ${\overset{\textnormal{ものごと}}{\text{物事}}}$ がうまく\{いっている・ ${\overset{\textnormal{はこ}}{\text{運}}}$ んでいる\}ときにも ${\overset{\textnormal{ゆだん}}{\text{油断}}}$ (を)してはいけない。 \hfill\break
You also mustn\textquotesingle t be careless when things are going well. }

\par{14. ${\overset{\textnormal{せいか}}{\text{成果}}}$ が ${\overset{\textnormal{で}}{\text{出}}}$ ているからといって、 ${\overset{\textnormal{じんせい}}{\text{人生}}}$ がうまく\{いっている・ ${\overset{\textnormal{はこ}}{\text{運}}}$ んでいる\}とは ${\overset{\textnormal{かぎ}}{\text{限}}}$ らない。 \hfill\break
Just because one is making results, it isn\textquotesingle t necessarily the case that life is going well. }

\par{15. ${\overset{\textnormal{すべ}}{\text{全}}}$ てがうまく\{いっている・ ${\overset{\textnormal{はこ}}{\text{運}}}$ んでいる\}わけではありませんが、 ${\overset{\textnormal{おお}}{\text{大}}}$ きな ${\overset{\textnormal{しっぱい}}{\text{失敗}}}$ もしていません。 \hfill\break
It\textquotesingle s not the case that everything is going well, but I\textquotesingle m also not making any great failures. }

\par{ As the following example sentence demonstrates, 運ぶ can also be used to mean “to carry out.” However, the verb 進める is far more common in this regard. }

\par{16. ${\overset{\textnormal{たいとう}}{\text{対等}}}$ に ${\overset{\textnormal{こうしょう}}{\text{交渉}}}$ を\{ ${\overset{\textnormal{すす}}{\text{進}}}$ める・ ${\overset{\textnormal{はこ}}{\text{運}}}$ ぶ\}ことは ${\overset{\textnormal{こんなん}}{\text{困難}}}$ です。 \hfill\break
Carrying out negotiations equally is difficult. }

\par{17. ${\overset{\textnormal{ほんじつ}}{\text{本日}}}$ の ${\overset{\textnormal{さぎょう}}{\text{作業}}}$ をもちまして ${\overset{\textnormal{いったんちゅうだん}}{\text{一旦中断}}}$ の ${\overset{\textnormal{はこ}}{\text{運}}}$ びとさせていただきます。 \hfill\break
We will temporarily suspend progress as of today\textquotesingle s work. }

\par{\textbf{Grammar Note }: As a noun, 運び far more frequently is used to mean “progress” than its verbal form 運ぶ. }

\begin{center}
\textbf{\emph{Fuku }吹く }
\end{center}

\par{ The verb 吹く can be used as both an intransitive and a transitive verb, but as a transitive verb, it is rather restricted. For one, the subject never acts out of its own volition. This is just like what was the case with \emph{ひら }く and the meaning “to bloom.” }

\par{18. ${\overset{\textnormal{へや}}{\text{部屋}}}$ の ${\overset{\textnormal{すみ}}{\text{隅}}}$ から ${\overset{\textnormal{そよかぜ}}{\text{微風}}}$ が ${\overset{\textnormal{ふ}}{\text{吹}}}$ いてきた。 \hfill\break
A breeze blew in from the corner of the room. }

\par{\textbf{Reading Note }: In literature, 微風 may seldom be read as びふう. }

\par{19. ${\overset{\textnormal{よる}}{\text{夜}}}$ に ${\overset{\textnormal{くちぶえ}}{\text{口笛}}}$ を ${\overset{\textnormal{ふ}}{\text{吹}}}$ いてはいけない。 \hfill\break
You mustn\textquotesingle t whistle at night. }

\par{20. カモメは、 ${\overset{\textnormal{うみ}}{\text{海}}}$ から ${\overset{\textnormal{ふ}}{\text{吹}}}$ く ${\overset{\textnormal{じょうしょうきりゅう}}{\text{上昇気流}}}$ に ${\overset{\textnormal{の}}{\text{乗}}}$ って ${\overset{\textnormal{と}}{\text{飛}}}$ びます。 \hfill\break
Seagulls fly by riding the updrafts that blow from the sea. }

\par{\textbf{Spelling Note }: カモメ may also be spelled as 鴎・鷗. The former variant is an unofficial abbreviation that has become widely used. }

\par{ When used to mean “to bud,” 吹くcan never be used in the passive. This is largely due to the absence of volition plants have in budding. It's also important to note that the verb \emph{ }芽吹く, which has 吹く in it, typically replaces 吹く for this nuance. }

\par{21a. ${\overset{\textnormal{きぎ}}{\text{木々}}}$ が ${\overset{\textnormal{め}}{\text{芽}}}$ を ${\overset{\textnormal{ふ}}{\text{吹}}}$ き ${\overset{\textnormal{はじ}}{\text{始}}}$ めた。 \hfill\break
21b. ${\overset{\textnormal{きぎ}}{\text{木々}}}$ が ${\overset{\textnormal{めぶ}}{\text{芽吹}}}$ き ${\overset{\textnormal{はじ}}{\text{始}}}$ めた。 \hfill\break
21c. 芽が ${\overset{\textnormal{きぎ}}{\text{木々}}}$ によって ${\overset{\textnormal{ふ}}{\text{吹}}}$ かれた。X \hfill\break
The trees have begun to bud. }

\par{22. ${\overset{\textnormal{とつぜん}}{\text{突然}}}$ 、 ${\overset{\textnormal{なべ}}{\text{鍋}}}$ が ${\overset{\textnormal{ふ}}{\text{噴}}}$ き ${\overset{\textnormal{こぼ}}{\text{零}}}$ れて、 ${\overset{\textnormal{ねっとう}}{\text{熱湯}}}$ が ${\overset{\textnormal{あし}}{\text{足}}}$ に ${\overset{\textnormal{か}}{\text{掛}}}$ かって ${\overset{\textnormal{やけど}}{\text{火傷}}}$ をしてしまいました。 \hfill\break
Suddenly, the stew boiled over and my leg got burned by boiling water. }

\par{\textbf{Word Note }: Just as in English, a pot for stew can be used to refer to the stew itself. }

\par{ There are two other verbs that are also ふく. Both are purely transitive and do not have the same grammatical constraints as 吹く・噴く above. }

\par{23. ${\overset{\textnormal{て}}{\text{手}}}$ を ${\overset{\textnormal{ぬ}}{\text{濡}}}$ らしたハンカチで ${\overset{\textnormal{ふ}}{\text{拭}}}$ く。 \hfill\break
To wipe one\textquotesingle s hand(s) with a wettened handkerchief. }

\par{24. 犬が顔を拭かれて怒ってしまった。 \hfill\break
The dog got angry from having its face wiped. }

\par{25. ${\overset{\textnormal{かわら}}{\text{瓦}}}$ で ${\overset{\textnormal{やね}}{\text{屋根}}}$ を ${\overset{\textnormal{ふ}}{\text{葺}}}$ きました。 \hfill\break
I thatched the roof with tile. }

\par{26. ${\overset{\textnormal{くさ}}{\text{草}}}$ で ${\overset{\textnormal{ふ}}{\text{葺}}}$ かれた ${\overset{\textnormal{ぶぶん}}{\text{部分}}}$ が ${\overset{\textnormal{ふ}}{\text{吹}}}$ き ${\overset{\textnormal{と}}{\text{飛}}}$ ばされた。 \hfill\break
The parts thatched with grass were blown off. }

\begin{center}
\textbf{催す }
\end{center}

\par{ When you open up a dictionary, the first meaning of 催す that you will find is “to hold (a ceremony).” However, instances like 宴を催す (to hold a banquet) are rare and literary. Although the verb itself is usually only used in the written language, its most important meaning is “to feel (a physical sensation).” In that sense, it can be used as either an intransitive or a transitive verb, and although が and を appear seemingly interchangeable, the use of が is disappearing. }

\par{27. ${\overset{\textnormal{すこ}}{\text{少}}}$ し ${\overset{\textnormal{ねむけ}}{\text{眠気}}}$ \{が・を\} ${\overset{\textnormal{もよお}}{\text{催}}}$ してから、 ${\overset{\textnormal{あんていざい}}{\text{安定剤}}}$ を ${\overset{\textnormal{の}}{\text{飲}}}$ む。 \hfill\break
To take a stabilizer after showing signs of some drowsiness. }

\par{28. ビールで ${\overset{\textnormal{にょうい}}{\text{尿意}}}$ \{が・を\} ${\overset{\textnormal{もよお}}{\text{催}}}$ すのはアルコールに ${\overset{\textnormal{りにょうさよう}}{\text{利尿作用}}}$ があるためです。 \hfill\break
Having the urge to urinate from bear is due to the diuretic effect of alcohol. }

\par{29. ${\overset{\textnormal{いしき}}{\text{意識}}}$ が ${\overset{\textnormal{もうろう}}{\text{朦朧}}}$ とし、 ${\overset{\textnormal{は}}{\text{吐}}}$ き ${\overset{\textnormal{け}}{\text{気}}}$ \{が・を\} ${\overset{\textnormal{もよお}}{\text{催}}}$ す。 \hfill\break
To feel nauseated while in a hazy state. }

\par{30. ${\overset{\textnormal{かんちょう}}{\text{浣腸}}}$ (を)して ${\overset{\textnormal{べんい}}{\text{便意}}}$ \{が・を\} ${\overset{\textnormal{もよお}}{\text{催}}}$ す。 \hfill\break
To feel a bowel movement from taking an enema. }

\begin{center}
\textbf{結ぶ }
\end{center}

\par{\emph{ }結ぶ is usually used as a transitive verb meaning “to tie\slash link.” As an intransitive verb, it can be used to mean “to bear (fruit)” or for dew to coagulate, but other verbs typically replace it. }

\par{31. ${\overset{\textnormal{くさ}}{\text{草}}}$ の ${\overset{\textnormal{は}}{\text{葉}}}$ に ${\overset{\textnormal{つゆ}}{\text{露}}}$ が\{ ${\overset{\textnormal{つ}}{\text{付}}}$ いている・ ${\overset{\textnormal{ふちゃく}}{\text{付着}}}$ している・ ${\overset{\textnormal{むす}}{\text{結}}}$ んでいる\}。 \hfill\break
Dew has attached\slash condensed onto the grass leaves. }

\par{32. ${\overset{\textnormal{どりょく}}{\text{努力}}}$ が\{ ${\overset{\textnormal{み}}{\text{実}}}$ を ${\overset{\textnormal{むす}}{\text{結}}}$ ぶ・ ${\overset{\textnormal{みの}}{\text{実}}}$ る\}と ${\overset{\textnormal{しん}}{\text{信}}}$ じている。 \hfill\break
I believe our efforts will bear fruit. }

\par{33. ${\overset{\textnormal{どりょく}}{\text{努力}}}$ の ${\overset{\textnormal{み}}{\text{実}}}$ が ${\overset{\textnormal{むす}}{\text{結}}}$ ぶことを ${\overset{\textnormal{ねが}}{\text{願}}}$ っています。 \hfill\break
I wish that the fruits of our efforts will manifest. }

\par{34. マカオと ${\overset{\textnormal{ほんこん}}{\text{香港}}}$ (と)を ${\overset{\textnormal{むす}}{\text{結}}}$ ぶ ${\overset{\textnormal{はし}}{\text{橋}}}$ の ${\overset{\textnormal{こうじ}}{\text{工事}}}$ が ${\overset{\textnormal{すす}}{\text{進}}}$ んでいる。 \hfill\break
Construction on a bridge connecting Macao and Hong Kong is making progress. }

\par{\textbf{Spelling Note }: Traditionally, マカオ \emph{ }is spelled as 澳門. }

\par{35. ${\overset{\textnormal{じょうやく}}{\text{条約}}}$ を\{ ${\overset{\textnormal{むす}}{\text{結}}}$ ぶ・ ${\overset{\textnormal{ていけつ}}{\text{締結}}}$ する\}ことに ${\overset{\textnormal{せいこう}}{\text{成功}}}$ する。 \hfill\break
To succeed in entering a treaty. }

\begin{center}
\textbf{頼る }
\end{center}

\par{ The verb 頼る has three different nuances depending on how it is used. }

\par{・In "(X を) Yに頼る," it shows dependency meaning “to rely on Y (for X).” \hfill\break
・In “Yを頼る,” it shows from whom\slash what you get help from. Essentially, you are purposely using connections. \hfill\break
・In "Yを頼りにする,” it shows with whom\slash what one depends on out of trust. }

\par{36. ${\overset{\textnormal{にほん}}{\text{日本}}}$ が ${\overset{\textnormal{いま}}{\text{今}}}$ 、 ${\overset{\textnormal{ひゃく}}{\text{100}}}$ パーセント ${\overset{\textnormal{ゆにゅう}}{\text{輸入}}}$ に ${\overset{\textnormal{たよ}}{\text{頼}}}$ っている ${\overset{\textnormal{た}}{\text{食}}}$ べ ${\overset{\textnormal{もの}}{\text{物}}}$ ってなんですか? \hfill\break
What foods does Japan now 100\% rely on imports for? }

\par{37. ${\overset{\textnormal{いしゃ}}{\text{医者}}}$ と ${\overset{\textnormal{くすり}}{\text{薬}}}$ に ${\overset{\textnormal{たよ}}{\text{頼}}}$ るのを ${\overset{\textnormal{や}}{\text{止}}}$ めませんか。 \hfill\break
Why not stop relying upon doctors and medicine? }

\par{38. スマホに ${\overset{\textnormal{たよ}}{\text{頼}}}$ ると、 ${\overset{\textnormal{しゅうちゅうりょく}}{\text{集中力}}}$ が ${\overset{\textnormal{お}}{\text{落}}}$ ちる。 \hfill\break
If you rely on your smart phone, your concentration will drop. }

\par{39. ${\overset{\textnormal{じゅようよそく}}{\text{需要予測}}}$ に ${\overset{\textnormal{たよ}}{\text{頼}}}$ りすぎるべきではない。 \hfill\break
You mustn't over-rely on demand forecasts. }

\par{40. ${\overset{\textnormal{けいざい}}{\text{経済}}}$ を ${\overset{\textnormal{しげん}}{\text{資源}}}$ の ${\overset{\textnormal{ゆしゅつ}}{\text{輸出}}}$ に ${\overset{\textnormal{たよ}}{\text{頼}}}$ る ${\overset{\textnormal{くに}}{\text{国}}}$ が ${\overset{\textnormal{おお}}{\text{多}}}$ くあります。 \hfill\break
There are many countries who rely on the importing of resources for their economies. }

\par{41. ${\overset{\textnormal{しりょう}}{\text{飼料}}}$ として ${\overset{\textnormal{つか}}{\text{使}}}$ われる ${\overset{\textnormal{こくもつ}}{\text{穀物}}}$ の ${\overset{\textnormal{おお}}{\text{多}}}$ くを ${\overset{\textnormal{かいがい}}{\text{海外}}}$ からの ${\overset{\textnormal{ゆにゅう}}{\text{輸入}}}$ に ${\overset{\textnormal{たよ}}{\text{頼}}}$ っている。 \hfill\break
We rely on the imports from foreign countries for a lot of the grain that is used for feed. }

\par{42. ${\overset{\textnormal{しようりょう}}{\text{使用量}}}$ のほとんどを ${\overset{\textnormal{ちゅうごく}}{\text{中国}}}$ に ${\overset{\textnormal{たよ}}{\text{頼}}}$ っている。 \hfill\break
We\textquotesingle re relying on China for most of the amount used, has no choice but to accept. }

\par{43. ${\overset{\textnormal{ちじん}}{\text{知人}}}$ を ${\overset{\textnormal{たよ}}{\text{頼}}}$ って ${\overset{\textnormal{とべい}}{\text{渡米}}}$ しました。 \hfill\break
I relied on an acquaintance to travel to America. }

\par{44. ${\overset{\textnormal{いちどちず}}{\text{一度地図}}}$ を ${\overset{\textnormal{たよ}}{\text{頼}}}$ って ${\overset{\textnormal{おこな}}{\text{行}}}$ ってみました。 \hfill\break
I tried going once by depending on a map. }

\par{45. ${\overset{\textnormal{つて}}{\text{伝手}}}$ を ${\overset{\textnormal{たよ}}{\text{頼}}}$ ってソマリアの ${\overset{\textnormal{かいぞく}}{\text{海賊}}}$ たちに ${\overset{\textnormal{あ}}{\text{会}}}$ いに ${\overset{\textnormal{い}}{\text{行}}}$ きました。 \hfill\break
I used connections to go meet the Somalian pirates. }

\par{46. ${\overset{\textnormal{みな}}{\text{皆}}}$ さんは ${\overset{\textnormal{なに}}{\text{何}}}$ を ${\overset{\textnormal{たよ}}{\text{頼}}}$ りに ${\overset{\textnormal{い}}{\text{生}}}$ きていらっしゃいますか。 \hfill\break
What does everyone rely on to live? }

\par{47. ${\overset{\textnormal{ちず}}{\text{地図}}}$ を ${\overset{\textnormal{たよ}}{\text{頼}}}$ りに ${\overset{\textnormal{もよ}}{\text{最寄}}}$ りの ${\overset{\textnormal{えき}}{\text{駅}}}$ に ${\overset{\textnormal{む}}{\text{向}}}$ かいました。 \hfill\break
I headed toward the nearest train station, relying on a map. }

\par{48. いくら ${\overset{\textnormal{けっこん}}{\text{結婚}}}$ できなくても ${\overset{\textnormal{こんかつ}}{\text{婚活}}}$ ビジネスを ${\overset{\textnormal{たよ}}{\text{頼}}}$ るのはやめた ${\overset{\textnormal{ほう}}{\text{方}}}$ がいい。 \hfill\break
No matter how much trouble you have in getting married, it\textquotesingle s better to stop relying on marriage hunting businesses. }

\par{49. ${\overset{\textnormal{めいい}}{\text{名医}}}$ を ${\overset{\textnormal{たよ}}{\text{頼}}}$ って ${\overset{\textnormal{びょういん}}{\text{病院}}}$ を ${\overset{\textnormal{えら}}{\text{選}}}$ ぶという ${\overset{\textnormal{ひと}}{\text{人}}}$ は ${\overset{\textnormal{すく}}{\text{少}}}$ なくありません。 \emph{\hfill\break
 }There are far from few individuals who chose hospitals by recoursing to noted physicians. }

\par{ 50. ${\overset{\textnormal{げんざいおや}}{\text{現在親}}}$ を ${\overset{\textnormal{たよ}}{\text{頼}}}$ って ${\overset{\textnormal{せいかつ}}{\text{生活}}}$ しています。 \hfill\break
Currently, I am living by relying on my parents. }
    