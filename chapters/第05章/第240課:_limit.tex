    
\chapter{Limit}

\begin{center}
\begin{Large}
第240課: Limit: ~限り, ~を限りに, ~限りでは, ~に限って, ~に限らない, \& ~とは限らない 
\end{Large}
\end{center}
 
\par{ 限り is a rather complicated noun. Although it means "limit" and understanding it is no problem for most students, similar looking structures with it cause problems.  }
      
\section{限り}
 
\par{  ~かぎり shows up in a lot of expressions, but what comes before it must be taken into account. Although particles may be a pain to get used to, you should still love them. }
 
\par{1. ~かぎり: This shows a parameter of a certain condition. Before ~かぎり, you may see nouns, adjectives, and verbs. }

\par{1. ${\overset{\textnormal{ほしょう}}{\text{保証}}}$ の限りではない。 \hfill\break
It is not in the warranty. }
 
\par{2. 目の ${\overset{\textnormal{とど}}{\text{届}}}$ く限り、晴れ ${\overset{\textnormal{わた}}{\text{渡}}}$ っている。 \hfill\break
The sky is clear as far as the eye can see. }
 
\par{3. 可能な限り、薬を飲まないようにしています。 \hfill\break
I'm trying as best I can to not take the medicine. }
 
\par{4. できる限りのことをするのも無理なのだ。 \hfill\break
Doing as much as possible is also no good. }

\par{5. ${\overset{\textnormal{ほうりつ}}{\text{法律}}}$ の ${\overset{\textnormal{およ}}{\text{及}}}$ ぶ限り ${\overset{\textnormal{げんみつ}}{\text{厳密}}}$ に ${\overset{\textnormal{しょ}}{\text{処}}}$ すべきだ。 \hfill\break
It should be strictly dealt with to the full extent of the law. }
 
\par{6. 今年も能力の限り、お役に立てるよう働くことを ${\overset{\textnormal{ちか}}{\text{誓}}}$ います。 \hfill\break
I vow to work to be of benefit [to X] to the best of my ability this year as well. }
 
\par{7. 外は ${\overset{\textnormal{ごうせつ}}{\text{豪雪}}}$ でも、部屋のなかにいるかぎりは ${\overset{\textnormal{まなつび}}{\text{真夏日}}}$ のようだ。 \hfill\break
Even if there is heavy snowfall outside, for as long as you're inside the room, it's like a hot summer day. }

\begin{center}
\textbf{読み物: A Passage from 心 by 夏目漱石 } 
\end{center}

\par{8. 次の日私は先生の ${\overset{\textnormal{あと}}{\text{後}}}$ につづいて海へ飛び ${\overset{\textnormal{こ}}{\text{込}}}$ んだ。そうして先生といっしょの ${\overset{\textnormal{ほうがく}}{\text{方角}}}$ に泳いで行った。二 ${\overset{\textnormal{ちょう}}{\text{丁}}}$ ほど ${\overset{\textnormal{おき}}{\text{沖}}}$ へ出ると、先生は後ろを ${\overset{\textnormal{ふ}}{\text{振}}}$ り返って私に話し ${\overset{\textnormal{か}}{\text{掛}}}$ けた。広い ${\overset{\textnormal{あお}}{\text{蒼}}}$ い海の表面に ${\overset{\textnormal{う}}{\text{浮}}}$ いているものは、その近所に私ら二人より ${\overset{\textnormal{ほか}}{\text{外}}}$ になかった。そうして強い太陽の光が、 ${\overset{\textnormal{め}}{\text{眼}}}$ の届く \textbf{限り }水と山とを照らしていた。私は自由と ${\overset{\textnormal{かんき}}{\text{歓喜}}}$ に ${\overset{\textnormal{み}}{\text{充}}}$ ちた筋肉を動かして海の中で ${\overset{\textnormal{おど}}{\text{躍}}}$ り ${\overset{\textnormal{くる}}{\text{狂}}}$ った。先生はまたぱたりと手足の運動を ${\overset{\textnormal{や}}{\text{已}}}$ めて ${\overset{\textnormal{あおむ}}{\text{仰向}}}$ けになったまま ${\overset{\textnormal{なみ}}{\text{浪}}}$ の上に寝た。私もその ${\overset{\textnormal{まね}}{\text{真似}}}$ をした。青空の色がぎらぎらと眼を ${\overset{\textnormal{い}}{\text{射}}}$ るように ${\overset{\textnormal{つうれつ}}{\text{痛烈}}}$ な色を私の顔に ${\overset{\textnormal{な}}{\text{投}}}$ げ ${\overset{\textnormal{つ}}{\text{付}}}$ けた。「 ${\overset{\textnormal{ゆかい}}{\text{愉快}}}$ ですね」と私は大きな声を出した。 }

\par{\textbf{漢字 Notes }: }

\par{1. 蒼い is a very literary spelling for blue. And, you may also encounter 碧い in similar contexts. 蒼い is a rather dull "blue", and it is often used in contexts where one "is blue in the face". 碧い is used when "blue" may have anywhere from a light to dark green hue to it. \hfill\break
2. ~掛ける is usually spelled as ~かける in everyday writing. \hfill\break
3. Rather than using 満ちる, 充ちる has the added nuance of inundation. 満ちる is over all the most important spelling due to the fact that 充 does not have the reading み(ちる) listed in the 常用漢字表. \hfill\break
4. 躍る differs from 踊る in the sense that the latter is "dance" in a rhythmic sense whereas the former is just jumping up and down. \hfill\break
5. 已む specifically describes a phenomenon, action, or condition that has continued for some time stopping completely. It is generally replaced with 止む except in literal situations like this when the writer wishes to make this nuance clear. \hfill\break
6. 浪 specifically refers to ripples in water or water rising up. }

\par{ The next day, I followed Sensei into the ocean. I then swam in the same direction as him. After going over a hundred meters offshore, Sensei turned around to talk to me. It was only us floating atop the surface of the wide, blue sea. The strong sunlight made the water and mountains glow for as far as one could see. I madly danced around in the waves with my muscles full of joy and freedom. Sensei, again, suddenly froze his arms and legs and lay flat on his back asleep on the waves, and so I mimicked him. The blue sky cast a scathing light into my eyes as if to pierce them out.  "How pleasant!", I yelled out. }

\par{Questions: }

\par{1. Find an example where a Japanese item is naturalized for the English translation. }

\par{2. Why might liberties in translation be needed with this passage? }

\begin{center}
 ~ない限り, which is an application of above, can be translated as "until" in contexts like below. 
\end{center}
 
\par{9. 君が考えを ${\overset{\textnormal{あらた}}{\text{改}}}$ めないかぎり、うまくいかない。 \hfill\break
Until you rethink yourself, things won't go well. }
 
\par{10. あいつは、あまりにも疲れていないかぎり、夜遅くまでIMABIを ${\overset{\textnormal{へんしゅう}}{\text{編集}}}$ するそうだ。 \hfill\break
That guy seems to edit IMABI up into the night unless he's too tired. }
 
\par{11. ${\overset{\textnormal{おすい}}{\text{汚水}}}$ を飲まないかぎり大丈夫ですよ。 \hfill\break
So long as you don't drink filthy water, you should be fine. }
 
\par{ After adjectives at the end of a sentence, ~かぎりだ shows the speaker's emotions with what is being said not being simply about the nature of things, but the matter at hand doesn't extend beyond a certain limit. }
 
\par{12. 誰からも連絡がなく、 ${\overset{\textnormal{さび}}{\text{寂}}}$ しいかぎりだ。 \hfill\break
It's just very lonely that I don't get contact from anyone. }
 
\par{ Another thing that this form can do that the two below cannot is be followed with a volitional expression. }
 
\par{13. 生きている限り歌おう。 \hfill\break
I'll sing until I die. }
 
\par{ As a suffix after temporal phrases in the form of ~かぎりで, it express deadline. }
 
\par{14. 来月末限りで辞任するつもりだ。 \hfill\break
I plan to resign at the end of next month. }
 
\par{2. ~を限りに, when used with time phrases, explicitly states when will be the last time that one does something. With non-temporal phrases, it shows some sort of physical limitation. }
 
\par{15. 彼は声の限りに ${\overset{\textnormal{さけ}}{\text{叫}}}$ んだ。 \hfill\break
He screamed to the top of list lungs. }
 
\par{16. 今日を限りに酒は飲みません。 \hfill\break
I will not drink alcohol from this day forward! }
 
\par{17. これを限りにお前とは ${\overset{\textnormal{えん}}{\text{縁}}}$ を切る。 \hfill\break
From here on I break my ties with you. }
 
\par{18. 今年の年末を限りに期限が切れる。 \hfill\break
The term\slash period\slash deadline will expire at the end of this year. }
 
\par{3. ~限りでは: Verbs of cognition such as 知る, 見る, 聞く, 読む, etc. come before, and the following clause deals with a judgment. In totality, this construction means "as far as\dothyp{}\dothyp{}\dothyp{}". }
 
\par{19. 知られる限りでは ${\overset{\textnormal{いぜんゆくえふめい}}{\text{依然行方不明}}}$ 。  (Headline) \hfill\break
[X] is\slash are still missing as far as is known. }
 
\par{20. 私の知っている限りでは、彼はそんなことをする人ではありません。 \hfill\break
As far as I know, he is not the kind of person to do something like that. }
 
\par{4. ~に限って: This raises a time, person, or thing to mean "insofar\slash unless". This is used a lot with とき. }
 
\par{21. あたしが出かけるときにかぎって、雨が降るのよ。(女性語) \hfill\break
It doesn't rain unless I go out [somewhere]. }

\par{22. ${\overset{\textnormal{わ}}{\text{我}}}$ が ${\overset{\textnormal{しゃ}}{\text{社}}}$ の社員に限って、そのような不正はするはずがない。 \hfill\break
Insofar as our company's workers, we have no reason to do such illegality. }
 
\par{23. 読みたいときに限って、アクセスできない。 \hfill\break
I can't access it insofar as to when I want to read it. }
 
\par{ As one would expect, this phrase has a negative form, ~に限らない.  This means that something is not yet decided. Of course, in formal\slash written situations, you can see it as ~に限らず. }

\par{24. ${\overset{\textnormal{なま}}{\text{怠}}}$ けたがるのは、子供に限ったことじゃない。 \hfill\break
Wanting to slack off is not limited to just kids. }
 
\par{25. 対象はアメリカ人に限りません。 \hfill\break
Our target is not limited to Americans. }
 
\par{26. 韓国語を勉強している学生は、東アジア専攻に限りません。 \hfill\break
Students who study Korean are not limited to East Asian majors. }
 
\par{5. ~とは限らない may resemble ~に限らない, and the best of students mix them up. However, because of the particle と, no part of speech limitations exists. ~とは限らない means "not necessarily". This phrase is frequently used to speak against stereotypes. }
 
\par{27. 日本人がみんな勉強ばかりしているとは限らない。 \hfill\break
It's not necessarily the case that all Japanese people do nothing but study. }
 
\par{28. アメリカ人は、必ずしも日本人より背が高いとは限らない。 \hfill\break
Americans are not necessarily taller than Japanese people. }
 
\par{29. テキサスの人がみんな必ずバーベキューばかり食べるとは限らない。 \hfill\break
All Texans don't necessarily eat just barbecue. }

\par{30. ${\overset{\textnormal{ねん}}{\text{念}}}$ を入れたからっていい ${\overset{\textnormal{しあ}}{\text{仕上}}}$ がりになるとは限らない。 \hfill\break
Because we paid attention to detail, it's (now) a good finish. }
    