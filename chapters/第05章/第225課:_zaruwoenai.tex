    
\chapter{~ざるを得ない \& やむを得ない}

\begin{center}
\begin{Large}
第225課: ~ざるを得ない \& やむを得ない 
\end{Large}
\end{center}
 
\par{ The を in these phrases is not the case particle you are used to. Here, it is a conjunctive particle, which is a usage from Classical Japanese that remains in these expressions. Pay attention to what phrases these phrases are similar and the things that make them different. }

\par{\textbf{漢字 Note }: The やむ in やむを得ない can be written in 漢字 as 止む or 已む. }
      
\section{~ざるを得ない}
 
\par{ ~ざる is the ${\overset{\textnormal{れんたいけい}}{\text{連体形}}}$ of the classical\slash old-fashioned negative auxiliary verb ~ず; thus, this pattern attaches to the ${\overset{\textnormal{みぜんけい}}{\text{未然形}}}$ . ${\overset{\textnormal{え}}{\text{得}}}$ ない is the negative form of 得る. Put together, ~ざるを ${\overset{\textnormal{え}}{\text{得}}}$ ない means "there is no other choice but to". As for する and 来る, you must use the せ- and こ- 未然形 respectively because we are using an older ending. So, \textbf{never say }\textbf{しざる }. }

\begin{ltabulary}{|P|P|}
\hline 

する \textrightarrow  & せざる \\ \cline{1-2}

来る \textrightarrow  & 来ざる \\ \cline{1-2}

\end{ltabulary}

\par{ Common questions in regards to this pattern include what the subject of the sentence is, when it is appropriate to use the pattern, and how the pattern differs from the “must phrases”. }

\par{ This phrase has a negative tone, and it should not be used in situations where such implications would be inappropriate. The action is something that the agent does not truly want to do. There is also some outside force that is making the agent act. }

\par{Although this is typically more common in the written language, it is still occasionally used in the spoken language. In this case, it is often followed by things like ~だろう・でしょう, ~(ん)じゃないか, ~と思う, 等. }

\begin{center}
 \textbf{Examples }
\end{center}

\par{1. ...ということを ${\overset{\textnormal{みと}}{\text{認}}}$ めざるを得ません。 \hfill\break
We have no other choice but to recognize… }

\par{2. ${\overset{\textnormal{あくてんこう}}{\text{悪天候}}}$ のため、 ${\overset{\textnormal{われわれ}}{\text{我々}}}$ はピクニックを ${\overset{\textnormal{ちゅうし}}{\text{中止}}}$ せざるを得なかったです。 \hfill\break
There was no other choice but to call off the picnic due to the weather. }

\par{3. このコンピューターが壊れたら、新しいのを買わざるをえないよ。 \hfill\break
If this computer breaks, there is no other choice but to buy a new one. }

\par{4. ${\overset{\textnormal{はいしゃ}}{\text{歯医者}}}$ さんに行かざるを得ない。 \hfill\break
There's no other choice but to go to the dentist. }

\par{5. 僕が引き受けざるを得ないじゃないか。 \hfill\break
There\textquotesingle s no other way but me having to undertake it, no? }

\par{6. 海外旅行に行くと、自分がアメリカ人であることを意識せざるをえない。 \hfill\break
When you travel overseas, you have no other choice but to be conscious of the fact that you are an       American. }

\par{7. 生徒が試験に失敗をすれば、先生は責任を取らざるをえない。 \hfill\break
If the student fails an exam, the student has no choice but to take responsibility. }

\par{8. 社長からの命令なので、やらざるを得ません。 \hfill\break
Since this is an order from the company president, I have no choice but to do it. }

\par{9a. この問題はもう一度検討しざるを得ない。X \hfill\break
9b. この問題はもう一度検討せざるを得ない。〇 \hfill\break
We have no choice but to examine this problem one more time. }
      
\section{やむを得ない}
 
\par{  ${\overset{\textnormal{や}}{\text{止}}}$ むを得ない shows that something is inevitable. We don't attach it to anything like the phrase above because 止む is a verb--the intransitive form of やめる. }

\par{10. それは ${\overset{\textnormal{おそ}}{\text{恐}}}$ らく止むを得なかったのだろう。 \hfill\break
It was perhaps inevitable. }

\par{11. 止むを得ない理由の ${\overset{\textnormal{ちこく}}{\text{遅刻}}}$ だから。 \hfill\break
It's because of inevitable delays. }
    