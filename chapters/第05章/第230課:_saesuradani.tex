    
\chapter{The Particles さえ, すら, \& だに}

\begin{center}
\begin{Large}
第230課: The Particles さえ, すら, \& だに 
\end{Large}
\end{center}
 
\par{ The particles さえ, すら, and だに are often interchangeable, but pay close attention to detail. There is some history involved, so try not to use these words anachronistically. Although they're interchangeable with each other, there are subtle differences. }
      
\section{The Adverbial Particle すら}
 
\par{ すら is the original particle for minimal example--"even\dothyp{}\dothyp{}\dothyp{}not to mention..". Now it has evolved to show minimal expectation, which is showing an extreme (usually negative) example. So, it is equal to "(not) even". It can be seen as ですら. すら is faded into literature and somehow survived to the present. Now, だに is ironically rarer than すら. }

\par{ Using this with other particles is very tricky as there is historical and personal variation. For instance, using it with を isn't wrong. However, many people just use すら. For when you do want to use it with を, をすら and すらを are both possible, but the latter is extremely old-fashioned. Even so, it still pops up. }

\par{1. なにしろ、警視庁みずから一般人の犯歴データを流していたことが問題になったばかりなのに、公の裁判所の正式決定すらをはねつけたんですからね。 \hfill\break
Anyhow, although the Metropolitan Police themselves having been leaking out general people's criminal record data has just become a problem, it's because they even rejected a public court's official decision. \hfill\break
From 明日はどっちだ! by 岡留安則. }
      
\section{The Adverbial Particle さえ}
 
\par{ さえ comes from the verb 添ふ (to add), which has become 添える in Modern Japanese. So, the original meaning of this particle was  "in addition to". Eventually, さえ began to be used to show minimal example. Both usages survive today, but the latter can be alternatively expressed with \textbf{でさえ }, as is demonstrated in the example below. It is to note that both ですら and でさえ are only used after nouns! This is because the case particle で is in these expressions. }

\par{2. 今どき、男の子だったら、小学生でさえ知ってるよ。 \hfill\break
These days even boys in elementary school know. }

\par{ The meaning of X(で)さえY is this. Some X matter and some Y action\slash happening\slash condition are usually not connected in any one, yet when you bind them together, you emphasize a situation that is not the norm by any means. This is what minimal example means. }

\par{3. 虎でさえ彼を傷つけることはできない。 \hfill\break
Not even a tiger can hurt him. }
4. 彼は食費\{さえ・すら\}惜しんだ。 \hfill\break
He even begrudged taking the money for food. 
\par{ It's sometimes hard to tell which meaning is meant. Minimal example is often used in negative expressions, but when this is not the case, consider the context. "Even a kid can understand it" is minimum example. "In addition to us, even the dogs understand" shows addition. It doesn't help that both English and Japanese are vague on this. }

\par{5. 雪どころか雨さえ降らなかった。 \hfill\break
Not even rain fell, let alone snow. }

\par{6 初心者にさえできることだ。 \hfill\break
It's something even a beginner can do. }

\par{7a. 子供\{(で)\{さえ・すら\}・でも\}(そのこと)知ってる(ことだ)よ。〇 \hfill\break
7b. 子供もさえそんなこと知ってるよ。X \hfill\break
Even kids know that. }

\par{8. ただでさえ電車が遅れているのに、寄り道をしようだなんて。 \hfill\break
Even under normal circumstances, hanging out along the way even when the train is late is just\dothyp{}\dothyp{}\dothyp{} }

\par{9a. ぼくは独りだから、君さえぼくの心の頼りだよ。X \hfill\break
9b. ぼくは独りだから、君だけがぼくの心の頼りだよ。〇 \hfill\break
Because I'm all alone, you're the only thing my heart relies on. }

\par{\textbf{Particle Note }: }

\par{1. さえ can be after other particles like に. をさえ is also possible, but を is usually dropped. In literature, however, をさえ is more common. }

\par{10. 秘密にしていたのに、三歳の子供にさえ気づかれてしまった。 \hfill\break
Even though I was keeping it a secret, it was even found out by a three year old. }

\par{11. 自分はまだ生きているのだという実感をさえ持つことができたのだった。 \hfill\break
I was able to have the realization that I myself was still alive. \hfill\break
From 光の雨 by 立松和平. }

\par{2. There is also さえも and さえもが. Both are more emphatic than さえ and show minimal example, but が implies new information and or surprise. If you understand も and が well enough, this shouldn't be hard to visualize. }

\par{12. 死さえもが玉井にとっては安楽なのだ。 \hfill\break
Even death was something which would be bring ease to Tamai. \hfill\break
From 光の雨 by 立松和平. }
13. 子どもばっかりじゃなく、大人さえも悲鳴をあげたよ。 \hfill\break
The children and even the adults screamed. 
\begin{center}
 \textbf{Important Patterns }
\end{center}

\par{ Both さえ and すら can be after the 連用形 of verbs similar to は, や. etc. You can also use them after て. This creates a phrase that puts emphasis on the action rather than the noun. すら, again, is increasingly becoming rarer, so you will probably only see it with nouns. Here is a chart using the pattern さえ…ば (if only you~) to illustrate this grammar. }

\begin{ltabulary}{|P|P|P|}
\hline 

Noun + さえ + Verb + ば & 薬さえ飲めば & If you only drank this medicine (and no other) \\ \cline{1-3}

Verb (連用形) + さえ + すれば & 薬を飲みさえすれば & If you only drink this medicine (and do nothing else) \\ \cline{1-3}

Verb + て + さえ + いれば & 薬を飲んでさえいれば & If you are drinking this medicine (and doing nothing else) \\ \cline{1-3}

\end{ltabulary}

\par{\textbf{Grammar Note }: Verb+ている+さえすれば  \textrightarrow  Verb+ていさえすれば. In this case, it doesn't matter that 連用形 of いる is left like that. Remember, this is not like ~ており、・・・. }

\par{14. 食べ((すぎ)さえし)なければ、もっとやせ(られる・るようになるよ)。 \hfill\break
If you would only not eat, you would be able to lose more weight. }

\par{15. 幸せでありさえすればかまわないんだ。 \hfill\break
I don't care so long as you are happy. }

\par{16. 知ってさえいればな~。(Colloquial Spelling) \hfill\break
If only I had known! }

\par{17. 時間さえあれば、手伝うんだが。 \hfill\break
If only I had time, I'd help. }

\par{18. 生きていさえすれば、何も要らない。 \hfill\break
If you just live on, you won't need anything. }

\par{19. 勉強さえすれば、試験はできるでしょう。 \hfill\break
If you just study, you should be able to do the exam. }

\par{20. 一生懸命頑張りさえすれば、何でも好きなことができる。 \hfill\break
If you just try with all your hardest, you should be able to do anything you like. }

\par{21. 分からない漢字は、調べさえすれば分かりますよ。 \hfill\break
If you simply look up Kanji you don't understand, you'll understand them. }

\par{22. 日本語は、日本人と毎日話そうとしてさえいれば話せるようになります。 \hfill\break
If you just try talking to Japanese people everyday, you'll become able to speak in Japanese. }

\par{23. 病気は、この漢方薬さえ飲めば、よくなります。 \hfill\break
If you drink just this herbal medicine, your sickness should become better. }

\par{24. 和書さえ読めれば、幸せです。 \hfill\break
If I read just Japanese books, I'm happy. }

\par{25. これさえあれば、後は何も要らない。 \hfill\break
If we only had this, we'd need nothing later. }

\par{\textbf{Nuance Note }: Again, these particles emphasize what they directly follow, so although there is some levity on where to place さえ in a sentence, be aware of this. }

\par{\textbf{Speech Style Note }: Though さえ may be used in the spoken language, it is more common in the written language. }

\par{26. しかしこの進学については、思い出すさえ忌々しい事情がある。 \hfill\break
But, remembering about going onto university brought on annoying circumstances by just remembering. \hfill\break
From 金閣寺 by 三島由紀夫. }

\par{\textbf{Grammar Note }: さえ also rarely follows the 連体形 of verbs. This is essentially old-fashioned. }
      
\section{The Adverbial Particle だに}
 
\par{ だに, not to be confused with the noun ダニ (tick), is used in both negative and positive sentences. It raises the most probable thing and negates it and anything aside from it. Originally, it showed minimal desire or want. In this sense it means "at the least". This led to it being used to show minimal example just like さえ and すら. This is equivalent to だけで. It is the rarest of the three. }

\par{27a. 夢にだに思ったことはない。(かなり古風) \hfill\break
27b. 夢でさえ思ったこともない。(もっと一般的) \hfill\break
I have not thought of something in just a dream. }

\par{28. ネオンは錯綜し、吞み屋の大きな赤い提灯は微動だもしなかった。 \hfill\break
The neon lights complicated things, and the bar's big, red lantern didn't even sway a bit. \hfill\break
From スタア by 三島由紀夫. }

\par{\textbf{Grammar Note }: だも is a contraction of だにも. }
    