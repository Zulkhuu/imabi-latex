    
\chapter{Intransitive Verbs Translated in the Passive Voice}

\begin{center}
\begin{Large}
第227課: Intransitive Verbs Translated in the Passive Voice 
\end{Large}
\end{center}
 
\par{ There are many times when an intransitive verb is technically translated with an English passive expression. For instance, 決まる, the first verb we\textquotesingle ll be looking at, can either be translated as “to be decided\slash settled.”  However, just because it\textquotesingle s translated as this doesn\textquotesingle t mean that there isn\textquotesingle t a transitive passive equivalent with a different nuance. In this lesson, a handful of these kinds of verbs will be looked at carefully so that you may get the sense of when to use which. }
      
\section{決まる vs 決められる}
 
\par{ The verbs 決まる and 決める create an intransitive-transitive verb pair meaning “to be decided\slash to decide.” However, 決められる (to be decided) also exists. This means that one must truly look at the meanings of both 決まる and 決める carefully to make tales of how 決まる and 決められる may differ. }

\par{ Firstly, it\textquotesingle s important to understand that although they share basic meanings, there are plenty of instances that only one or the other may be used. }

\par{\emph{ }決まる indicates a matter that is fixated\slash settled upon as a natural conclusion. The agent of the action, even if one exists, is not emphasized at all. 決める is the opposite of this. The agent is emphasized and its presence is felt in any form that it takes. }

\par{1. ${\overset{\textnormal{さいばん}}{\text{裁判}}}$ で ${\overset{\textnormal{ゆうざい}}{\text{有罪}}}$ が ${\overset{\textnormal{き}}{\text{決}}}$ まった ${\overset{\textnormal{こっかいぎいん}}{\text{国会議員}}}$ は ${\overset{\textnormal{ぎいん}}{\text{議員}}}$ を ${\overset{\textnormal{や}}{\text{辞}}}$ めさせられます。 \hfill\break
The Diet member whose guilty verdict was decided in court will be made to resign his seat. }

\par{2. どうにも ${\overset{\textnormal{はら}}{\text{腹}}}$ が ${\overset{\textnormal{き}}{\text{決}}}$ まらない。 \hfill\break
I can\textquotesingle t make up my mind. }

\par{\textbf{Idiom Note }: In Japanese, “to make up one\textquotesingle s mind” is 腹を決める. Essentially, the Japanese use the “gut” as the reference point for gut decisions. The reason why 決める is switched to 決まる in Ex. 2 is to emphasis the lack of control the speaker has in making up his own mind. This incapability goes well with the lack of volition that 決まる has in the outcome of things. }

\par{3. ${\overset{\textnormal{まいばん}}{\text{毎晩}}}$ 、 ${\overset{\textnormal{き}}{\text{決}}}$ まった ${\overset{\textnormal{じかん}}{\text{時間}}}$ に ${\overset{\textnormal{にょうい}}{\text{尿意}}}$ で ${\overset{\textnormal{お}}{\text{起}}}$ きます。 \hfill\break
Every evening, I wake up at a fixed time to go pee. }

\par{4. ${\overset{\textnormal{あらかじ}}{\text{予}}}$ め ${\overset{\textnormal{こうし}}{\text{講師}}}$ と ${\overset{\textnormal{こうしょう}}{\text{交渉}}}$ して ${\overset{\textnormal{にちじ}}{\text{日時}}}$ を ${\overset{\textnormal{き}}{\text{決}}}$ めてください。 \hfill\break
Negotiate with your lecturer beforehand and decide upon a date and time. }

\par{5. ピンクのスーツで ${\overset{\textnormal{き}}{\text{決}}}$ めましょう。 \hfill\break
Dress up nicely with a pink suit. }

\par{\textbf{Meaning Note }: 決める can be used to mean “to dress up (nicely).” 決まる may also be used to indicate that one\textquotesingle s attire\slash appearance is good looking. }

\par{6. スクイズを ${\overset{\textnormal{き}}{\text{決}}}$ め、 ${\overset{\textnormal{さん}}{\text{3}}}$ ${\overset{\textnormal{てん}}{\text{点}}}$ を ${\overset{\textnormal{だっしゅ}}{\text{奪取}}}$ した。 \hfill\break
(The player) successfully carried out a squeeze and took back three points. }

\par{ One pattern that 決まる is only used in is に決まっている. This is a highly subjective statement used to indicate that whatever it follows is undoubtedly so without a shadow of a doubt. This attaches directly after nouns, adjectives, and verbs. }

\par{7. ${\overset{\textnormal{はなし}}{\text{話}}}$ が ${\overset{\textnormal{むじゅん}}{\text{矛盾}}}$ だらけで ${\overset{\textnormal{ほんにん}}{\text{本人}}}$ が ${\overset{\textnormal{はんにん}}{\text{犯人}}}$ に ${\overset{\textnormal{き}}{\text{決}}}$ まっているでしょう。 \hfill\break
(His) story is littered with contradictions; it\textquotesingle s without a doubt that he himself is the criminal. }

\par{8. ${\overset{\textnormal{たいふう}}{\text{台風}}}$ が ${\overset{\textnormal{く}}{\text{来}}}$ れば、 ${\overset{\textnormal{でんしゃ}}{\text{電車}}}$ は ${\overset{\textnormal{おく}}{\text{遅}}}$ れるに ${\overset{\textnormal{き}}{\text{決}}}$ まっている。 \hfill\break
When a typhoon comes, it\textquotesingle s a guarantee that the trains will be late. }

\par{\emph{ }決める, when used in に決める, is very similar to にする. The use of 決める over する is used to emphasize the decision aspect. }

\par{9. ${\overset{\textnormal{あおじる}}{\text{青汁}}}$ を ${\overset{\textnormal{の}}{\text{飲}}}$ むことに ${\overset{\textnormal{き}}{\text{決}}}$ めました。 \hfill\break
I have decided to drink aojiru. }

\par{\textbf{Word Note }: 青汁 is a Japanese drink made from green leafy vegetables. An American equivalent would be something like V8. }

\par{10a. マサチューセッツ ${\overset{\textnormal{しゅう}}{\text{州}}}$ へ ${\overset{\textnormal{りゅうがく}}{\text{留学}}}$ することに\{ ${\overset{\textnormal{き}}{\text{決}}}$ めた・した\}。 \hfill\break
I\textquotesingle ve decided to study abroad at Massachusetts. \hfill\break
10b. マサチューセッツ ${\overset{\textnormal{しゅう}}{\text{州}}}$ に ${\overset{\textnormal{りゅうがく}}{\text{留学}}}$ することを ${\overset{\textnormal{き}}{\text{決}}}$ めた。 \hfill\break
I\textquotesingle ve decided to study abroad at Massachusetts. }

\par{\textbf{Grammar Note }: Although it's not wrong to have two に phrases in one sentence, whenever it can be helped, one of the に usually has to go. In 10a. the particle へ is used, but in 10b. the first に is kept and 決める is preceded by を. The nuance difference of using を instead of に indicates that the decider had at least more than one option and after some thought chose. The use of に does not imply a decision process from multiple options. Incidentally, the use of the particle を can emphasize satisfaction to the decision (See Ex. 11 below). }

\par{11. ${\overset{\textnormal{おおいた}}{\text{大分}}}$ に ${\overset{\textnormal{のこ}}{\text{残}}}$ ることを ${\overset{\textnormal{き}}{\text{決}}}$ めました。 \hfill\break
I\textquotesingle ve decided to stay in Ōita. }

\par{ When に決めている is after a noun, it shows what is always decided upon. }

\par{12. ${\overset{\textnormal{かぞく}}{\text{家族}}}$ が ${\overset{\textnormal{あつ}}{\text{集}}}$ まるときはドライブイン ${\overset{\textnormal{とり}}{\text{鳥}}}$ に ${\overset{\textnormal{き}}{\text{決}}}$ めています。 \hfill\break
When the family gathers, we make a habit of going to Drivein-Tori. }

\par{13. ${\overset{\textnormal{いちにち}}{\text{一日}}}$ の ${\overset{\textnormal{こづか}}{\text{小遣}}}$ いは ${\overset{\textnormal{ちゅうしょくこ}}{\text{昼食込}}}$ みで1000 ${\overset{\textnormal{えん}}{\text{円}}}$ に ${\overset{\textnormal{き}}{\text{決}}}$ めています。 \hfill\break
Daily allowance is set to 1000 yen including lunch. }

\par{ The difference between using こと or よう before に決める is rather small. With こと, you demonstrate what you\textquotesingle ve decided to do. With よう, you demonstrate what you\textquotesingle re determined to try to do. }

\par{14. ${\overset{\textnormal{いっしょ}}{\text{一緒}}}$ にプレイしない\{こと・よう\}に ${\overset{\textnormal{き}}{\text{決}}}$ めている。 \hfill\break
[We\textquotesingle ve settled\slash we\textquotesingle re determined] not to play together. }

\par{ In the realm of sports, 決める can indicate a successful move. In the realm of martial arts, it can mean “to immobilize” with certain techniques. As for  決まる, its use in sports indicates that a match is settled. }

\par{15. ${\overset{\textnormal{かんぬき}}{\text{閂}}}$ に ${\overset{\textnormal{き}}{\text{決}}}$ めているように ${\overset{\textnormal{み}}{\text{見}}}$ えるが、 ${\overset{\textnormal{かんぜん}}{\text{完全}}}$ に ${\overset{\textnormal{き}}{\text{決}}}$ まっていない。 \hfill\break
It appears that (he) is immobilized from the overhook, but (the match) is not entirely settled. }

\par{ As for 決められる, one thing that must be noted is that sometimes it is simply the potential form of 決める. }

\par{16. ${\overset{\textnormal{ひとり}}{\text{一人}}}$ で ${\overset{\textnormal{き}}{\text{決}}}$ められずにいたのですが、やっと ${\overset{\textnormal{き}}{\text{決}}}$ められました。 \hfill\break
I had been unable to decide on my own, but I\textquotesingle ve finally been able to decide. }

\par{\emph{ }決められる always implies an agent even if it isn\textquotesingle t explicitly expressed. In Ex. 17, the ‘indirect passive\textquotesingle  is used. This is because the speaker is upset that what was to become of his\slash her dental treatment was decided by the doctor without full consent. In the rest of the examples that follow, \emph{ }決められる is used as a normal passive verb with the agent either expressed or readily obvious. }

\par{17. ${\overset{\textnormal{は}}{\text{歯}}}$ の ${\overset{\textnormal{ちりょうほうしん}}{\text{治療方針}}}$ を ${\overset{\textnormal{かって}}{\text{勝手}}}$ に ${\overset{\textnormal{き}}{\text{決}}}$ められました。 \hfill\break
My dental treatment plan was arbitrarily decided. }

\par{18. ${\overset{\textnormal{せいれき}}{\text{西暦}}}$ ${\overset{\textnormal{さんびゃくにじゅうご}}{\text{325}}}$ ${\overset{\textnormal{ねん}}{\text{年}}}$ には、 ${\overset{\textnormal{だいいち}}{\text{第一}}}$ ${\overset{\textnormal{にかいあ}}{\text{ニカイア}}}$ ${\overset{\textnormal{こうかいぎ}}{\text{公会議}}}$ が ${\overset{\textnormal{おこな}}{\text{行}}}$ われ、「 ${\overset{\textnormal{しゅんぶん}}{\text{春分}}}$ の ${\overset{\textnormal{ひ}}{\text{日}}}$ 」が ${\overset{\textnormal{さん}}{\text{3}}}$ ${\overset{\textnormal{がつ}}{\text{月}}}$ ${\overset{\textnormal{にじゅういち}}{\text{21}}}$ ${\overset{\textnormal{にち}}{\text{日}}}$ に ${\overset{\textnormal{き}}{\text{決}}}$ められた。 \hfill\break
In the year 325 AD, the First Council of Nicaea was convened, and the Vernal Equinox was decided upon to be March 21 st . }

\par{19. ${\overset{\textnormal{さんかこく}}{\text{参加国}}}$ の ${\overset{\textnormal{とうひょう}}{\text{投票}}}$ により、グリニッジ ${\overset{\textnormal{てんもんだい}}{\text{天文台}}}$ を ${\overset{\textnormal{とお}}{\text{通}}}$ る ${\overset{\textnormal{しごせん}}{\text{子午線}}}$ を ${\overset{\textnormal{けいど}}{\text{経度}}}$ ${\overset{\textnormal{れい}}{\text{0}}}$ ${\overset{\textnormal{ど}}{\text{度}}}$ とすることが ${\overset{\textnormal{き}}{\text{決}}}$ められました。 \hfill\break
By vote from participating countries, the prime meridian was decided upon to be the merdian that goes through the Greenwich Observatory. }

\par{20. ${\overset{\textnormal{せんきゅうひゃくはちじゅうに}}{\text{1982}}}$ ${\overset{\textnormal{ねん}}{\text{年}}}$ に「 ${\overset{\textnormal{かいようほう}}{\text{海洋法}}}$ に ${\overset{\textnormal{かん}}{\text{関}}}$ する ${\overset{\textnormal{こくさいれんごうじょうやく}}{\text{国際連合条約}}}$ 」が ${\overset{\textnormal{つく}}{\text{作}}}$ られ、 ${\overset{\textnormal{りょうかい}}{\text{領海}}}$ は ${\overset{\textnormal{じゅうに}}{\text{12}}}$ ${\overset{\textnormal{かいりいない}}{\text{海里以内}}}$ とすることが ${\overset{\textnormal{き}}{\text{決}}}$ められました。 \hfill\break
The United Nations Convention on the Law of the Sea was created in 1982, and the extent of national waters was set to be within 12 nautical miles. }
      
\section{定まる vs 定められる}
 
\par{\emph{ }定まる and 定める are more formal equivalents to 決まる and 決める. For instance, 定まる can also indicate that a decision has been settled. However, 定まる implies that a certain situation is maintained by said decision. When 定められる is used, as was the case for 決められる, the agent is either explicitly stated or obvious and there is volition behind the action. There is no volition implied with 定まる. }

\par{21. ${\overset{\textnormal{うんめい}}{\text{運命}}}$ が ${\overset{\textnormal{さだ}}{\text{定}}}$ まっているわけではない。 \hfill\break
It is not the case that one\textquotesingle s destiny is fixed. }

\par{22. ${\overset{\textnormal{みんしんとうせいけん}}{\text{民進党政権}}}$ の ${\overset{\textnormal{しょうてん}}{\text{焦点}}}$ が ${\overset{\textnormal{さだ}}{\text{定}}}$ まった。 \hfill\break
The focus of the Democratic Party of Japan has been set. }

\par{23. ${\overset{\textnormal{かんむてんのう}}{\text{桓武天皇}}}$ はなぜ ${\overset{\textnormal{きょうと}}{\text{京都}}}$ に ${\overset{\textnormal{みやこ}}{\text{都}}}$ を ${\overset{\textnormal{さだ}}{\text{定}}}$ めたのでしょうか。 \hfill\break
Why is that Emperor Kanmu set the capital to Kyoto? }

\par{24. ${\overset{\textnormal{かくしゅう}}{\text{各州}}}$ および ${\overset{\textnormal{し}}{\text{市}}}$ などの ${\overset{\textnormal{ぜいむとうきょく}}{\text{税務当局}}}$ が ${\overset{\textnormal{どくじ}}{\text{独自}}}$ の ${\overset{\textnormal{ぜいせい}}{\text{税制}}}$ を ${\overset{\textnormal{さだ}}{\text{定}}}$ めている。 \hfill\break
Each state and city tax authority sets its own system of taxation. }

\par{25. その ${\overset{\textnormal{ひとみ}}{\text{瞳}}}$ は、まるで ${\overset{\textnormal{えもの}}{\text{獲物}}}$ に ${\overset{\textnormal{ねら}}{\text{狙}}}$ いを ${\overset{\textnormal{さだ}}{\text{定}}}$ めた ${\overset{\textnormal{とら}}{\text{虎}}}$ の ${\overset{\textnormal{め}}{\text{目}}}$ だ。 \hfill\break
Those eyes were like that of a tiger\textquotesingle s locked onto its prey. }

\par{ In the following examples, 定まる would not be used instead. 定められる happens to be the verb of choice in technical circumstances. This is likely because the agent needs to be unambiguous. }

\par{26. ${\overset{\textnormal{ほうりつ}}{\text{法律}}}$ で ${\overset{\textnormal{さだ}}{\text{定}}}$ められた ${\overset{\textnormal{きんむじかん}}{\text{勤務時間}}}$ の ${\overset{\textnormal{じょうげん}}{\text{上限}}}$ を ${\overset{\textnormal{し}}{\text{知}}}$ りたいのです。 \hfill\break
I want to know the upper limit on working hours set by law. }

\par{27. 「 ${\overset{\textnormal{しょくひんひょうじきじゅん}}{\text{食品表示基準}}}$ 」で ${\overset{\textnormal{さだ}}{\text{定}}}$ められた ${\overset{\textnormal{めいしょう}}{\text{名称}}}$ が ${\overset{\textnormal{きさい}}{\text{記載}}}$ されている。 \hfill\break
The names set by the “Food Product Labeling Standards” are listed. }
      
\section{焦げる vs 焦がされる}
 
\par{\emph{ }焦げる is a verb meaning “to be burned” by some sort of fire or heat. Its transitive equivalent is 焦がす. The transitive version, however, can be used in non-literal expressions, widening its range of usage. 焦がされる gets used to mean “to get burned” by the fault of someone or something. The agent, then, is emphasized with it. With 焦げる, there is no ‘culprit\textquotesingle  involved. }

\par{28. ${\overset{\textnormal{りょうり}}{\text{料理}}}$ をしていて、うっかり ${\overset{\textnormal{なべ}}{\text{鍋}}}$ やフライパンを ${\overset{\textnormal{しょう}}{\text{焦}}}$ がしてしまったことはありませんか。 \hfill\break
Have you ever unintentionally burned at pot or frying pan while cooking? }

\par{29. アイロンでワイシャツが ${\overset{\textnormal{こ}}{\text{焦}}}$ げてしまった。 \hfill\break
The white shirt got burnt by the iron. }

\par{30. ライターで ${\overset{\textnormal{かみ}}{\text{髪}}}$ を ${\overset{\textnormal{こ}}{\text{焦}}}$ がされてしまったのですが、 ${\overset{\textnormal{かみ}}{\text{髪}}}$ の ${\overset{\textnormal{け}}{\text{毛}}}$ が ${\overset{\textnormal{こ}}{\text{焦}}}$ げてしまったとき、どのようなケアをすればいいんでしょうか。 \hfill\break
My hair got burnt by someone with a lighter, and so what sort of care should you do when your hair gets burned? }

\par{31. 旦那様のご経験に食材を火に入れすぎて焦がされてしまったことはございませんか。 \hfill\break
In your experiences with your husband, have you ever had a situation where he heated the ingredients for too long and burned them? }

\par{\textbf{Grammar Note }: Ex. 31 is a perfect example demonstrating a bridging context for how one can see the interconnection between light honorifics and the passive. In this case, 焦がされる should be interpreted as the light honorific form of 焦がす, but interpreting as the passive does not really alter the meaning of the sentence. }

\par{32. ${\overset{\textnormal{たいよう}}{\text{太陽}}}$ に ${\overset{\textnormal{こ}}{\text{焦}}}$ がされて ${\overset{\textnormal{ほて}}{\text{火照}}}$ った。 \hfill\break
I was flushed from the sun. }
      
\section{かかる vs かけられる}
 
\par{ The verb かかる and \emph{ }かける as we know have many usages. The most basic meaning of the two is “to be hung” and “to hang” respectively. The former has no volition entailed in it. However, かける always does. Thus, when you use \emph{ }かけられる, the agent of the action is always implied if it isn\textquotesingle t explicitly stated. }

\par{\textbf{Orthography Note }: As the sentences below demonstrate, かかる and かける have various spellings depending on nuance. }

\par{33. ${\overset{\textnormal{じゅう}}{\text{10}}}$ ${\overset{\textnormal{まんえん}}{\text{万円}}}$ の ${\overset{\textnormal{しょうきん}}{\text{賞金}}}$ が ${\overset{\textnormal{か}}{\text{懸}}}$ かっている。 \hfill\break
There is a prize for 100,000 yen. }

\par{34. テーブルの ${\overset{\textnormal{うえ}}{\text{上}}}$ の ${\overset{\textnormal{かべ}}{\text{壁}}}$ に ${\overset{\textnormal{か}}{\text{掛}}}$ けられた ${\overset{\textnormal{かがみ}}{\text{鏡}}}$ が、 ${\overset{\textnormal{かのじょ}}{\text{彼女}}}$ の目を ${\overset{\textnormal{ひ}}{\text{惹}}}$ いた。 \hfill\break
The mirror hung up on the wall above the table caught her eye. }

\par{35. ${\overset{\textnormal{たにん}}{\text{他人}}}$ からかけられた ${\overset{\textnormal{のろ}}{\text{呪}}}$ いを ${\overset{\textnormal{は}}{\text{跳}}}$ ね ${\overset{\textnormal{かえ}}{\text{返}}}$ すにはどうすればいいでしょうか。 \hfill\break
What should I do to repel a spell cast on me by someone? }

\par{36. キリストは、 ${\overset{\textnormal{なぜ}}{\text{何故}}}$ 、 ${\overset{\textnormal{じゅうじか}}{\text{十字架}}}$ に ${\overset{\textnormal{か}}{\text{架}}}$ けられたのでしょうか。 \hfill\break
Why was it that Christ was put on the cross? }

\par{37. ${\overset{\textnormal{そら}}{\text{空}}}$ に ${\overset{\textnormal{みごと}}{\text{見事}}}$ な ${\overset{\textnormal{にじ}}{\text{虹}}}$ が ${\overset{\textnormal{か}}{\text{架}}}$ かっている。 \hfill\break
There is a magnificent rainbow suspended in the sky. }

\par{38. ホテルには ${\overset{\textnormal{やけい}}{\text{夜景}}}$ を ${\overset{\textnormal{か}}{\text{描}}}$ いた ${\overset{\textnormal{かい}}{\text{絵}}}$ が ${\overset{\textnormal{か}}{\text{掛}}}$ かっている。 \hfill\break
There is a picture of the nightscape hung up in the hotel. }
      
\section{伝わる vs 伝えられる}
 
\par{\emph{ }伝わる is an intransitive verb meaning “to be handed down\slash transmitted\slash circulated” and its transitive counterpart is 伝える, which is typically translated as “to convey\slash transmit\slash communicate\slash propagate.” 伝わる does not imply personal volition, and so whenever there is an agent with volition involved, when used in a passive sense, 伝えられる becomes your choice. }

\par{39. ${\overset{\textnormal{い}}{\text{言}}}$ いたいことが ${\overset{\textnormal{つた}}{\text{伝}}}$ わらない。 \hfill\break
I can\textquotesingle t get across what I want to say. }

\par{40. ${\overset{\textnormal{わ}}{\text{我}}}$ が ${\overset{\textnormal{や}}{\text{家}}}$ に ${\overset{\textnormal{だいだいつた}}{\text{代々伝}}}$ わるレシピをご ${\overset{\textnormal{しょうかい}}{\text{紹介}}}$ します。 \hfill\break
I will introduce a recipe passed down from generation to generation in my family. }

\par{41. ${\overset{\textnormal{うえ}}{\text{上}}}$ からの ${\overset{\textnormal{おと}}{\text{音}}}$ は ${\overset{\textnormal{かべ}}{\text{壁}}}$ を ${\overset{\textnormal{つた}}{\text{伝}}}$ わって ${\overset{\textnormal{き}}{\text{聞}}}$ こえてきます。 \hfill\break
Noise from above can be heard through the walls. }

\par{42. ダイヤモンドは ${\overset{\textnormal{ねつ}}{\text{熱}}}$ をよく ${\overset{\textnormal{つた}}{\text{伝}}}$ えると ${\overset{\textnormal{い}}{\text{言}}}$ われます。 \hfill\break
It\textquotesingle s say that diamonds conduct heat well. }

\par{43. ${\overset{\textnormal{ぶっきょう}}{\text{仏教}}}$ は ${\overset{\textnormal{ちょうせんじん}}{\text{朝鮮人}}}$ によって ${\overset{\textnormal{にほん}}{\text{日本}}}$ に ${\overset{\textnormal{つた}}{\text{伝}}}$ えられた。 \hfill\break
Buddhism was propagated to Japan by Koreans. }

\par{44. ${\overset{\textnormal{ちゅうごくたいりく}}{\text{中国大陸}}}$ や ${\overset{\textnormal{ちょうせんはんとう}}{\text{朝鮮半島}}}$ から ${\overset{\textnormal{いじゅう}}{\text{移住}}}$ した ${\overset{\textnormal{ひとびと}}{\text{人々}}}$ によって、 ${\overset{\textnormal{にほんれっとう}}{\text{日本列島}}}$ に ${\overset{\textnormal{いなさく}}{\text{稲作}}}$ が ${\overset{\textnormal{つた}}{\text{伝}}}$ えられました。 \hfill\break
Rice cultivation was propagated to the Japanese Archipelago by people who had migrated there from Mainland China and the Korean Peninsula. }
      
\section{集まる vs 集められる}
 
\par{ To “gather\slash assemble” in the intransitive sense is 集まる. In the transitive sense, it\textquotesingle s 集める. Because the intransitive sense can also be translated in English as “to be gathered,” some may confuse it with the passive of the transitive form, 集められる. However, as continues to be the case for all the other examples in this lesson, \emph{ }集められる has an agent, and the action involved is done by the volition of said agent. 集まる has no volition entailed. }

\par{45. ${\overset{\textnormal{しょくたく}}{\text{食卓}}}$ にはいつも、 ${\overset{\textnormal{せかいじゅう}}{\text{世界中}}}$ から ${\overset{\textnormal{さまざま}}{\text{様々}}}$ な ${\overset{\textnormal{しょくざい}}{\text{食材}}}$ が ${\overset{\textnormal{あつ}}{\text{集}}}$ まっています。 \hfill\break
There are always various ingredients from all over the world gathered on the dinner table. }

\par{46. ${\overset{\textnormal{あべないかく}}{\text{安倍内閣}}}$ に ${\overset{\textnormal{しじ}}{\text{支持}}}$ が ${\overset{\textnormal{あつ}}{\text{集}}}$ まっている ${\overset{\textnormal{じょうきょう}}{\text{状況}}}$ だ。 \hfill\break
The circumstance is that support is gathering for Abe\textquotesingle s cabinet. }

\par{47. あの ${\overset{\textnormal{ひと}}{\text{人}}}$ の ${\overset{\textnormal{まわ}}{\text{周}}}$ りには ${\overset{\textnormal{つね}}{\text{常}}}$ に ${\overset{\textnormal{ひと}}{\text{人}}}$ が ${\overset{\textnormal{あつ}}{\text{集}}}$ まっている。 \hfill\break
There are always people gathered around that person. }

\par{48. ${\overset{\textnormal{こきゃく}}{\text{顧客}}}$ からの ${\overset{\textnormal{くじょう}}{\text{苦情}}}$ がたくさん ${\overset{\textnormal{あつ}}{\text{集}}}$ められている。 \hfill\break
Tons of complaints are being gathered from customers. }

\par{49. ${\overset{\textnormal{けんけつ}}{\text{献血}}}$ によって ${\overset{\textnormal{あつ}}{\text{集}}}$ められた ${\overset{\textnormal{けつえき}}{\text{血液}}}$ はどのようなルートで ${\overset{\textnormal{かんじゃ}}{\text{患者}}}$ に ${\overset{\textnormal{ゆけつ}}{\text{輸血}}}$ されるんですか。 \hfill\break
Through what sort of routes do the blood that is gathered by donations transfused into patients? }
      
\section{収まる vs 収められる}
 
\par{ The verb 収まる can be translated as “to settle into\slash be settled into\slash installed\slash in one\textquotesingle s place.” Essentially, it refers to things being in place and settled. That\textquotesingle s why it can even refer to a weapon being sheathed. When used to refer to payment having been paid, it\textquotesingle s spelled as 納まる. Conversely, 収める・納める handles the transitive twist of these meanings. As is the case with all the other verbs, \emph{ }収まる・納まる has no volition. Therefore, 収められる・納められる is the correct passive expression when there is a willful agent involved. }

\par{50. ${\overset{\textnormal{まんぷく}}{\text{満腹}}}$ なのに ${\overset{\textnormal{い}}{\text{胃}}}$ に ${\overset{\textnormal{おさ}}{\text{収}}}$ まっていく。 \hfill\break
Even despite being full, it smoothly settles in one\textquotesingle s stomach. }

\par{51. ${\overset{\textnormal{に}}{\text{2}}}$ ${\overset{\textnormal{ねんいない}}{\text{年以内}}}$ に ${\overset{\textnormal{きゅうじゅうきゅう}}{\text{99}}}$ ${\overset{\textnormal{てん}}{\text{.}}}$ ${\overset{\textnormal{に}}{\text{2}}}$ ${\overset{\textnormal{パーセント}}{\text{%}}}$ の ${\overset{\textnormal{おさ}}{\text{納}}}$ めるべき ${\overset{\textnormal{ぜいがく}}{\text{税額}}}$ が ${\overset{\textnormal{こっこ}}{\text{国庫}}}$ に ${\overset{\textnormal{おさめ}}{\text{納}}}$ まっているということになります。 \hfill\break
99.2\% of the taxes that you ought to pay within two years is paid to the national treasury. }

\par{52. ${\overset{\textnormal{かたな}}{\text{刀}}}$ を ${\overset{\textnormal{さや}}{\text{鞘}}}$ に ${\overset{\textnormal{おさ}}{\text{収}}}$ める。 \hfill\break
To sheathe a sword. }

\par{53. 中世の重要な資料や遺産などが収められている。 \hfill\break
Important medieval materials and heritage items are dedicated (here). }

\par{54. ${\overset{\textnormal{かべ}}{\text{壁}}}$ の ${\overset{\textnormal{あな}}{\text{穴}}}$ の ${\overset{\textnormal{なか}}{\text{中}}}$ に ${\overset{\textnormal{ぶつぞう}}{\text{仏像}}}$ が ${\overset{\textnormal{おさ}}{\text{納}}}$ められています。 \hfill\break
Buddha statues are installed in the holes of the walls. }
      
\section{混ざる・混じる vs 混ぜられる}
 
\par{ まざる and まじる both mean “to be blended\slash mixed.” まざる is closer to “to be blended” whereasまじる is closer to “to be mixed.” The use of the character 混 is to emphasize things being mixed but technically separate whereas the character 交 is used to emphasize that things are blended together as one. Both まざる and まじる have zero volition. It is \emph{ }まぜられる that you need to use for when things are mixed together with an active agent. }

\par{55. この ${\overset{\textnormal{きじ}}{\text{生地}}}$ にナイロンが\{混・交\}ざっている。 \hfill\break
Nylon is blended into this fabric. }

\par{56. ${\overset{\textnormal{からみ}}{\text{辛味}}}$ の ${\overset{\textnormal{すく}}{\text{少}}}$ ない ${\overset{\textnormal{やさい}}{\text{野菜}}}$ カレーを ${\overset{\textnormal{ま}}{\text{混}}}$ ぜればマイルドな ${\overset{\textnormal{あじ}}{\text{味}}}$ わいになります。 \hfill\break
If you mix in a vegetable curry, which has little spice, it becomes a mild flavor. }

\par{57. ${\overset{\textnormal{たん}}{\text{痰}}}$ に ${\overset{\textnormal{ち}}{\text{血}}}$ が ${\overset{\textnormal{ま}}{\text{混}}}$ じっている。 \hfill\break
Blood is mixed in the phlegm. }

\par{58. ${\overset{\textnormal{だいしょう}}{\text{大小}}}$ の ${\overset{\textnormal{たてもの}}{\text{建物}}}$ が ${\overset{\textnormal{い}}{\text{入}}}$ り\{混・交\}じっている。 \hfill\break
Buildings big and small are mixed together. }

\par{59. ${\overset{\textnormal{きゅうしょくしつ}}{\text{給食室}}}$ の ${\overset{\textnormal{じゃぐち}}{\text{蛇口}}}$ から ${\overset{\textnormal{で}}{\text{出}}}$ た ${\overset{\textnormal{すいどうすい}}{\text{水道水}}}$ に ${\overset{\textnormal{くろ}}{\text{黒}}}$ い ${\overset{\textnormal{いぶつ}}{\text{異物}}}$ が\{ ${\overset{\textnormal{こんにゅう}}{\text{混入}}}$ して・ ${\overset{\textnormal{ま}}{\text{混}}}$ じって・ ${\overset{\textnormal{ざつ}}{\text{雑}}}$ じって・\}いた。 \hfill\break
A black foreign substance was mixed in the tap-water from the faucet in the lunch room. }

\par{\textbf{Spelling Note }: When used for saying that a foreign substance is mixed in with something, the spelling  雑ざるmay occasionally be used. }

\par{60. ${\overset{\textnormal{あぶら}}{\text{油}}}$ で ${\overset{\textnormal{ま}}{\text{混}}}$ ぜられた ${\overset{\textnormal{え}}{\text{絵}}}$ の ${\overset{\textnormal{ぐ}}{\text{具}}}$ で ${\overset{\textnormal{え}}{\text{絵}}}$ を ${\overset{\textnormal{か}}{\text{描}}}$ く。 \hfill\break
To draw a picture with coloring materials mixed with oil. }

\par{ Of course, 混ぜられる can also be the potential form of 混ぜる. }

\par{61. ジュースにも ${\overset{\textnormal{ま}}{\text{混}}}$ ぜられるからいいですね。 \hfill\break
It\textquotesingle s good because you can also mix it in juice. }
      
\section{育つ vs 育てられる}
 
\par{ The verb 育つ means “to grow up.” Because it can also be translated as “to be raised\slash brought up,” some confuse it with 育てられる. However, there is no active agent with 育つ. To express an active agent in the passive sense, 育てられる, the passive form of the transitive form, needs to be used. Of course, there is also the fact that 育てられる may also be used as the potential form of \emph{ }育てる. }

\par{62. ${\overset{\textnormal{ぼく}}{\text{僕}}}$ が ${\overset{\textnormal{そだ}}{\text{育}}}$ った ${\overset{\textnormal{まち}}{\text{町}}}$ に ${\overset{\textnormal{すこ}}{\text{少}}}$ しでも ${\overset{\textnormal{なに}}{\text{何}}}$ か ${\overset{\textnormal{こうけん}}{\text{貢献}}}$ したいなと ${\overset{\textnormal{おも}}{\text{思}}}$ っています。 \hfill\break
I\textquotesingle d like to donate something back, even if it\textquotesingle s just a little, to the town I grew up in. }

\par{63. ${\overset{\textnormal{わたし}}{\text{私}}}$ は ${\overset{\textnormal{とうきょう}}{\text{東京}}}$ で ${\overset{\textnormal{う}}{\text{生}}}$ まれて、 ${\overset{\textnormal{おおさか}}{\text{大阪}}}$ で ${\overset{\textnormal{そだ}}{\text{育}}}$ ちました。 \hfill\break
I was born in Tokyo and raised in Ōsaka. }

\par{64. ${\overset{\textnormal{どうぶつ}}{\text{動物}}}$ に ${\overset{\textnormal{そだ}}{\text{育}}}$ てられた ${\overset{\textnormal{ひと}}{\text{人}}}$ って ${\overset{\textnormal{ほんとう}}{\text{本当}}}$ にいるの? \hfill\break
Is it really true that there are people who were raised by animals? }

\par{ 65. ${\overset{\textnormal{さる}}{\text{猿}}}$ の ${\overset{\textnormal{いちぞく}}{\text{一族}}}$ に ${\overset{\textnormal{そだ}}{\text{育}}}$ てられました。 \hfill\break
I was raised by a family of monkeys. }
    