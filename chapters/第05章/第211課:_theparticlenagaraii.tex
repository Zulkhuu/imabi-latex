    
\chapter{The Particle ながら II}

\begin{center}
\begin{Large}
第211課: The Particle ながら II: ながら(も)\& ながらに(して) 
\end{Large}
\end{center}
 
\par{ In this second installment concerning the particle ながら, we learn about its second most common usage: showing contradiction. }
      
\section{Contradiction}
 
\par{ The particle ながら, when used to show contradiction, follows the same restraints on clauses as when it shows simultaneous action. This is because the contradiction involves the same subject and because the two parts of the contradiction still happen in the same time span. }

\par{ Unlike its primary usage to show non-contradictory simultaneous action, this use of ながら can attach to pretty much anything. You can find it directly after nouns, adjectival nouns, adjectives, adverbs, and verbs. }

\begin{ltabulary}{|P|P|P|}
\hline 

 & Affirmative & Negative \\ \cline{1-3}

Nouns & N (であり)+ ながら(も) & Nではない + ながら(も) \\ \cline{1-3}

Adjectival Nouns & Adj. N(であり)+ながら(も) & Adj. Nではない+ながら(も) \\ \cline{1-3}

Adjectives & Adj. +ながら(も) & Drop \slash i\slash  + くない+ながら(も) \\ \cline{1-3}

Adverbs & Adv. + ながら(も) &  \\ \cline{1-3}

Verbs & Stem + ながら(も) & V +ない \textrightarrow  ず・ぬ + ながら(も) \\ \cline{1-3}

\end{ltabulary}

\par{  The productivity of all these combinations will not be equal in the real world. ながら(も) is most frequently used with verbs, and for everything else it may be follow, creative license is required. Due to the fact that this pattern is more so employed in the written language, you will need to explore this grammar on a case-by-case basis. There are many set phrases that utilize this, some of which are very important. For instance, 残念だ  means “to be unfortunate,” and when used with ながら as 残念ながら, it\textquotesingle s used to mean “unfortunately…” }

\par{ The addition of the particle も in this expression is only used for emphatic purposes. As such , normally, ながら and ながらも are interchangeable. }

\par{1. ${\overset{\textnormal{かのじょ}}{\text{彼女}}}$ はあんなにいろいろ ${\overset{\textnormal{くろう}}{\text{苦労}}}$ しながらも、それを ${\overset{\textnormal{く}}{\text{苦}}}$ にしていない。 \hfill\break
Even though she's going through that many troubles, she isn't worrying about them. }

\par{2. ${\overset{\textnormal{ゆざわ}}{\text{湯沢}}}$ さんは ${\overset{\textnormal{だいきぎょう}}{\text{大企業}}}$ の ${\overset{\textnormal{しゃちょう}}{\text{社長}}}$ (であり)ながら(も)、 ${\overset{\textnormal{しず}}{\text{静}}}$ かな ${\overset{\textnormal{お}}{\text{落}}}$ ち ${\overset{\textnormal{つ}}{\text{着}}}$ いた ${\overset{\textnormal{ものごし}}{\text{物腰}}}$ が ${\overset{\textnormal{いんしょうてき}}{\text{印象的}}}$ でした。 \hfill\break
Though Mr. Yuzawa is the president of a large corporation, his quiet and calm demeanor is impressive. }

\par{3. ${\overset{\textnormal{まゆこ}}{\text{真由子}}}$ は ${\overset{\textnormal{しょしんしゃ}}{\text{初心者}}}$ (であり)ながら(も)、 ${\overset{\textnormal{せんざい}}{\text{潜在}}}$ ${\overset{\textnormal{のうりょく}}{\text{能力}}}$ はかなりあると ${\overset{\textnormal{おも}}{\text{思}}}$ います。 \hfill\break
Even though Mayuko is a beginner, I think that her skills are considerably good. }

\par{4. しかしながら、 ${\overset{\textnormal{よさんあん}}{\text{予算案}}}$ はまだ ${\overset{\textnormal{しゅうせい}}{\text{修正}}}$ の ${\overset{\textnormal{よち}}{\text{余地}}}$ がある。 \hfill\break
Nevertheless, the budget still has plenty of room for improvement. }

\par{5. すでに ${\overset{\textnormal{いちがつ}}{\text{一月}}}$ を ${\overset{\textnormal{す}}{\text{過}}}$ ごしてしまったのですが、 ${\overset{\textnormal{おそ}}{\text{遅}}}$ まきながら ${\overset{\textnormal{はつもうで}}{\text{初詣}}}$ に ${\overset{\textnormal{で}}{\text{出}}}$ かけました。 \hfill\break
I\textquotesingle ve already let January pass, but I belatedly went out to my first shrine visit of the New Year. }

\par{6. ${\overset{\textnormal{さぬき}}{\text{讃岐}}}$ うどんとは ${\overset{\textnormal{い}}{\text{言}}}$ いながらも、ほとんどは ${\overset{\textnormal{がいこくさん}}{\text{外国産}}}$ の ${\overset{\textnormal{こむぎ}}{\text{小麦}}}$ で ${\overset{\textnormal{つく}}{\text{作}}}$ られている。 \hfill\break
Although we call it “Sanuki Udon,” most of it is made with foreign manufactured wheat. }

\par{7. ${\overset{\textnormal{かのじょ}}{\text{彼女}}}$ はダイエットしていると ${\overset{\textnormal{い}}{\text{言}}}$ いながら(も)ケーキばかり ${\overset{\textnormal{た}}{\text{食}}}$ べている。 \hfill\break
Although she says that she\textquotesingle s on a diet, all she eats is cake. }

\par{8. ${\overset{\textnormal{は}}{\text{恥}}}$ ずかしながら、 ${\overset{\textnormal{じこしょうかい}}{\text{自己紹介}}}$ です。 \hfill\break
As embarrassing as this is, this is my self-introduction. }

\par{\textbf{Form Note }:  Note that with the adjective 恥ずかしい, the final \slash i\slash  is dropped. }

\par{9.\{ ${\overset{\textnormal{われ}}{\text{我}}}$ ・ ${\overset{\textnormal{じぶん}}{\text{自分}}}$ \}ながら ${\overset{\textnormal{なさ}}{\text{情}}}$ けない。 \hfill\break
This is deplorable even if I do say so myself. }

\par{10. ゾンビーは ${\overset{\textnormal{し}}{\text{死}}}$ んでいながらも、 ${\overset{\textnormal{せいぜん}}{\text{生前}}}$ と ${\overset{\textnormal{おな}}{\text{同}}}$ じ ${\overset{\textnormal{こうどう}}{\text{行動}}}$ を ${\overset{\textnormal{く}}{\text{繰}}}$ り ${\overset{\textnormal{かえ}}{\text{返}}}$ します。 \hfill\break
Zombies, though dead, repeat the same actions as when they were alive. }

\par{11. ゆっくりながら(も)、 ${\overset{\textnormal{かんせい}}{\text{完成}}}$ に ${\overset{\textnormal{ちか}}{\text{近}}}$ づいています。 \hfill\break
Although slowly, I\textquotesingle m approaching completion. }

\par{12. ${\overset{\textnormal{ざんねん}}{\text{残念}}}$ ながら、 ${\overset{\textnormal{きろく}}{\text{記録}}}$ は ${\overset{\textnormal{こうしん}}{\text{更新}}}$ できませんでした。 \hfill\break
Unfortunately, I was unable to break the record. }

\par{13. ${\overset{\textnormal{けんたろうくん}}{\text{憲太郎君}}}$ は ${\overset{\textnormal{しっぱい}}{\text{失敗}}}$ するとわかりながら(も)、そのままやってしまう ${\overset{\textnormal{せいかく}}{\text{性格}}}$ だ。 \hfill\break
Kentaro has the kind of personality of doing something as is even if he understands that he\textquotesingle ll fail. }

\par{14. ${\overset{\textnormal{かみ}}{\text{神}}}$ が ${\overset{\textnormal{ゆる}}{\text{赦}}}$ さない ${\overset{\textnormal{つみ}}{\text{罪}}}$ だと ${\overset{\textnormal{し}}{\text{知}}}$ っていながらも、 ${\overset{\textnormal{こい}}{\text{故意}}}$ に ${\overset{\textnormal{つづ}}{\text{続}}}$ けていく。 \hfill\break
To continue purposely doing something whilst knowing it is a sin God does not condone. }

\par{15. ${\overset{\textnormal{しゃいん}}{\text{社員}}}$ が ${\overset{\textnormal{ふせい}}{\text{不正}}}$ を ${\overset{\textnormal{し}}{\text{知}}}$ りながら ${\overset{\textnormal{かんゆう}}{\text{勧誘}}}$ していたことがわかりました。 \hfill\break
It has been discovered that company employees were soliciting whilst knowing of the illegality. }

\par{16. ${\overset{\textnormal{せま}}{\text{狭}}}$ いながらも、 ${\overset{\textnormal{じぶん}}{\text{自分}}}$ のアパートを ${\overset{\textnormal{て}}{\text{手}}}$ に ${\overset{\textnormal{い}}{\text{入}}}$ れることができました。 \hfill\break
While small, I\textquotesingle ve obtained my own apartment. }

\par{17. ${\overset{\textnormal{まず}}{\text{貧}}}$ しいながらも、 ${\overset{\textnormal{おだ}}{\text{穏}}}$ やかに ${\overset{\textnormal{く}}{\text{暮}}}$ らすことができます。 \hfill\break
Once can live calmly live whilst being poor. }

\par{18. ${\overset{\textnormal{およ}}{\text{及}}}$ ばずながら、 ${\overset{\textnormal{いっしょうけんめい}}{\text{一生懸命}}}$ やります。 \hfill\break
I\textquotesingle ll do it to the best of my ability, as poor as that may be. }

\par{19. ${\overset{\textnormal{かんぜん}}{\text{完全}}}$ ではないながらも、 ${\overset{\textnormal{えいご}}{\text{英語}}}$ の ${\overset{\textnormal{つづ}}{\text{綴}}}$ りと ${\overset{\textnormal{おと}}{\text{音}}}$ にはある ${\overset{\textnormal{ていどきそく}}{\text{程度規則}}}$ があります。 \hfill\break
Although not absolute, there are rules to some degree to the spelling and sounds of English. }

\begin{center}
 \textbf{もさることながら }
\end{center}

\par{\emph{ }もさることながら is a set phrase that follows nouns that states that the one quality is of course true, but a second quality is also just as so. }

\par{20. ${\overset{\textnormal{ちゅうかりょうり}}{\text{中華料理}}}$ は ${\overset{\textnormal{あじ}}{\text{味}}}$ もさることながら、 ${\overset{\textnormal{けんこう}}{\text{健康}}}$ にいいですよ。 \hfill\break
It goes without saying that Chinese food is tasty, but it's also good for you. }

\par{21. ${\overset{\textnormal{ふじさん}}{\text{富士山}}}$ は ${\overset{\textnormal{もみじ}}{\text{紅葉}}}$ もさることながら、 ${\overset{\textnormal{ふゆげしき}}{\text{冬景色}}}$ も ${\overset{\textnormal{きれい}}{\text{綺麗}}}$ です。 \hfill\break
The autumn leaves of Mt. Fuji go without saying, but its winter-scape is also pretty. }

\par{ 22. この ${\overset{\textnormal{くるま}}{\text{車}}}$ は ${\overset{\textnormal{ねんぴ}}{\text{燃費}}}$ の ${\overset{\textnormal{よ}}{\text{良}}}$ さもさることながら、ネット ${\overset{\textnormal{じょう}}{\text{上}}}$ での ${\overset{\textnormal{ひょうばん}}{\text{評判}}}$ も ${\overset{\textnormal{たか}}{\text{高}}}$ かったです。 \hfill\break
The gas mileage of this car goes without saying, but its internet review was high as well. }
      
\section{ながらに(して)}
 
\par{ In the patterns ながらに(して)or ながら(の), the particle ながら may also mean “as” as in staying “as is” in a certain condition.  Just as is the case with the past usage, instances of this meaning ought to be learned on a case-by-case basis. }

\par{23. すべての ${\overset{\textnormal{ひと}}{\text{人}}}$ は ${\overset{\textnormal{う}}{\text{生}}}$ まれながらに ${\overset{\textnormal{びょうどう}}{\text{平等}}}$ である。 \hfill\break
All men are born equal. }

\par{24. ${\overset{\textnormal{たなか}}{\text{田中}}}$ さんは ${\overset{\textnormal{いえ}}{\text{家}}}$ に ${\overset{\textnormal{い}}{\text{居}}}$ ながらにして、 ${\overset{\textnormal{つき}}{\text{月}}}$ に ${\overset{\textnormal{ななじゅう}}{\text{70}}}$ ${\overset{\textnormal{まんえん}}{\text{万円}}}$ くらいの ${\overset{\textnormal{しゅうにゅう}}{\text{収入}}}$ を ${\overset{\textnormal{え}}{\text{得}}}$ ている。 \hfill\break
Mr. Tanaka earns an income of approximately 700,000 yen a month whilst staying at home. }

\par{25. ${\overset{\textnormal{そんみん}}{\text{村民}}}$ は ${\overset{\textnormal{むかし}}{\text{昔}}}$ ながらの ${\overset{\textnormal{でんとう}}{\text{伝統}}}$ を ${\overset{\textnormal{まも}}{\text{守}}}$ り ${\overset{\textnormal{つづ}}{\text{続}}}$ けている。 \hfill\break
The villages continue protecting traditions as they were long ago. }

\par{26. ${\overset{\textnormal{かれ}}{\text{彼}}}$ は ${\overset{\textnormal{こども}}{\text{子供}}}$ ながらにしっかりしている。 \hfill\break
He\textquotesingle s quite level-headed despite being a kid. }

\par{27. ${\overset{\textnormal{なつこ}}{\text{夏子}}}$ は、 ${\overset{\textnormal{じぶん}}{\text{自分}}}$ の ${\overset{\textnormal{め}}{\text{目}}}$ の ${\overset{\textnormal{まえ}}{\text{前}}}$ で ${\overset{\textnormal{お}}{\text{起}}}$ こった ${\overset{\textnormal{ひさん}}{\text{悲惨}}}$ な ${\overset{\textnormal{できごと}}{\text{出来事}}}$ を ${\overset{\textnormal{なみだ}}{\text{涙}}}$ ながらに ${\overset{\textnormal{かた}}{\text{語}}}$ りました。 \hfill\break
Natsuko tearfully spoke of the tragic events that transpired in front of her eyes. }

\par{28. いつもながら、 ${\overset{\textnormal{かれし}}{\text{彼氏}}}$ の ${\overset{\textnormal{りょうり}}{\text{料理}}}$ はとてもうまい。 \hfill\break
My boyfriend\textquotesingle s cooking is very delicious as always. }

\par{29. この ${\overset{\textnormal{かいしゃ}}{\text{会社}}}$ は ${\overset{\textnormal{むかし}}{\text{昔}}}$ ながらの ${\overset{\textnormal{せいほう}}{\text{製法}}}$ で ${\overset{\textnormal{とうふ}}{\text{豆腐}}}$ を ${\overset{\textnormal{つく}}{\text{作}}}$ っている。 \hfill\break
This company makes tofu with traditional methods. }

\par{30. ${\overset{\textnormal{にんげん}}{\text{人間}}}$ には、 ${\overset{\textnormal{う}}{\text{生}}}$ まれながらにして ${\overset{\textnormal{ひんぷ}}{\text{貧富}}}$ の ${\overset{\textnormal{さ}}{\text{差}}}$ がある。 \hfill\break
Humans are born with a disparity of wealth. }
    