    
\chapter{Tense II}

\begin{center}
\begin{Large}
第205課: Tense II: The Morpheme -RU\slash U 
\end{Large}
\end{center}
 
\par{ The morpheme -RU\slash U is one of the most fundamental morphemes in the Japanese language. Although it manifests differently depending on the part of speech, its functions remain depending on the semantic and syntactic conditions of the sentence. }

\par{ In its basic understanding, the -RU\slash U form is thought of as being the non-past tense marker of Japanese. Although it is not limited to this interpretation, it can account for any usage in which it corresponds to either the present tense or future tense of English. }

\par{ In this lesson, we will study the various usages of -RU\slash U form as well as learn how some of these usages overlap with -TA and -TE IRU. Before continuing, if you have not read through the previous lesson on -TA, please go back and read it first. }
      
\section{-RU\slash U Morphology}
 
\par{ In Japanese, the -RU\slash U form is typically simply referred to as the ル ${\overset{\textnormal{けい}}{\text{形}}}$ . It is a linguistic yet colloquial term for the terminal form ( ${\overset{\textnormal{しゅうしけい}}{\text{終止形}}}$ ) \slash attributive form ( ${\overset{\textnormal{れんたいけい}}{\text{連体形}}}$ ). Because the latter terms are meant only to indicate morphology in relation to word order, we\textquotesingle ll stick to calling this the “-RU\slash U form” for the purposes of this discussion. }

\par{ After the root of any conjugatable speech is a morpheme that forms the -RU\slash U form. For \emph{Ichidan }verbs ( ${\overset{\textnormal{いちだんどうし}}{\text{一段動詞}}}$ ), the morpheme is most easily identified because it happens to be - \emph{ru }which attaches to the stem of these verbs. For \emph{Godan }Verbs ( ${\overset{\textnormal{ごだんどうし}}{\text{五段動詞}}}$ ), the morpheme manifests as - \emph{u }, which attaches to the consonant-ending stems that these verbs have. For adjectives, the morpheme manifests as - \emph{i }, - \emph{shii }, or - \emph{jii }. For the copula, because it is a contraction of である, it is appropriate to treat \emph{da }as a whole to be the morpheme. }
      
\section{-RU\slash U: Present, Future, \& Past}
 
\begin{center}
\textbf{PRESENT TENSE }
\end{center}

\par{ In English, the \textbf{present tense }expresses an action that is ongoing or habitually performed, or a state that currently and\slash or generally exists. -RU\slash U also expresses all these functions. However, it\textquotesingle s first important to understand how flexible even the concept of “now” can be in both English and Japanese. Take for example the following sentences: }

\par{i. My boyfriend just got home now. \textrightarrow  彼氏が今帰ったところだ。 \hfill\break
ii. My boyfriend is coming home now. \textrightarrow  彼氏が今帰っているところだ。 \hfill\break
iii. My boyfriend will come home now. \textrightarrow  彼氏が今帰る(ところだ)ね。 \hfill\break
iv. My boyfriend is home now. \textrightarrow  彼氏がもう帰っている。 }

\par{ We associate “now” with the present, but as these examples demonstrate, it is not fixated solely to the present tense. This is analogous to how the -RU\slash U form functions. These examples also scratch at the surface of how complex the endings -RU\slash U, -TA, and -TE IRU can be. For instance, -TE IRU can both denote the \textbf{present continuous\slash progressive }form like in ii. and the \textbf{present progressive perfect }form like in iv. In iii., the act of “coming home” is perceived to just be starting, which is in contrast to the continued ongoing state implied in ii. }

\par{1. The most fundamental usage of the -RU\slash U form is to show \textbf{present state }. Existential verbs, adjectives, and adjectival nouns are quintessential here. Present states may very well be ongoing. For instance, in Ex. 5, the existence of people who want to quit their jobs is a present state. The act of wanting to quit is an ongoing state that is marked with -TE IRU. This emphasizes duration of a continued ongoing state. When the “continued” aspect of an ongoing action is not implied, however, -RU\slash U should be used instead. }

\par{1. ${\overset{\textnormal{にわ}}{\text{庭}}}$ には ${\overset{\textnormal{き}}{\text{木}}}$ が ${\overset{\textnormal{さん}}{\text{3}}}$ ${\overset{\textnormal{ほん}}{\text{本}}}$ ある。 \hfill\break
There are three trees in the (court)yard. }

\par{2. ${\overset{\textnormal{ほっかいどう}}{\text{北海道}}}$ を ${\overset{\textnormal{せいそくち}}{\text{生息地}}}$ にする ${\overset{\textnormal{かめ}}{\text{亀}}}$ もいる。 \hfill\break
There \textbf{are }also turtles who \textbf{have }Hokkaido as their habitat. }

\par{3. ${\overset{\textnormal{あや}}{\text{彩}}}$ ちゃんの ${\overset{\textnormal{ふく}}{\text{服}}}$ 、めっちゃかわいい。 \hfill\break
Aya-chan\textquotesingle s clothes, they \textbf{\textquotesingle re }so \textbf{cute }. }

\par{4. これが ${\overset{\textnormal{さいこう}}{\text{最高}}}$ (だ)! \hfill\break
This \textbf{is }the \textbf{best }! }

\par{\textbf{Grammar Note }: Even when the copula is omitted, the -RU\slash U morpheme is still present grammatically. }

\par{5. ${\overset{\textnormal{しごと}}{\text{仕事}}}$ を ${\overset{\textnormal{や}}{\text{辞}}}$ めたがっている人がたくさんいる。 \hfill\break
There are lots of people who want to quit their jobs. }

\par{6. ${\overset{\textnormal{なん}}{\text{何}}}$ でもケチをつけたがる ${\overset{\textnormal{ひと}}{\text{人}}}$ が ${\overset{\textnormal{いだ}}{\text{抱}}}$ く ${\overset{\textnormal{しんり}}{\text{心理}}}$ とは ${\overset{\textnormal{なに}}{\text{何}}}$ か。 \hfill\break
What is the mentality that people harbor who want to find fault in everything? }

\par{7. ${\overset{\textnormal{もうれつ}}{\text{猛烈}}}$ な ${\overset{\textnormal{かぜ}}{\text{風}}}$ が ${\overset{\textnormal{ふ}}{\text{吹}}}$ いたと ${\overset{\textnormal{み}}{\text{見}}}$ られ\{ます・ています\}。 \hfill\break
It is believed that there was fierce wind. }

\par{2. In the same vein as Usage 1, the -RU\slash U form may also denote a \textbf{present psychological state }. This is frequently employed with verbal\slash adjectival expressions of emotion. }

\par{8. ${\overset{\textnormal{ほんとう}}{\text{本当}}}$ にむかつくわ。 \hfill\break
It really ticks me off. }

\par{9. ${\overset{\textnormal{はら}}{\text{腹}}}$ が ${\overset{\textnormal{た}}{\text{立}}}$ つ。 \hfill\break
This makes me mad. }

\par{10. ${\overset{\textnormal{とりはだ}}{\text{鳥肌}}}$ が ${\overset{\textnormal{た}}{\text{立}}}$ つ。 \hfill\break
This gives me goosebumps. }

\par{11. ${\overset{\textnormal{きも}}{\text{気持}}}$ ち ${\overset{\textnormal{わる}}{\text{悪}}}$ い。 \hfill\break
This is disgusting\slash unpleasant. }

\par{12. ${\overset{\textnormal{こわ}}{\text{怖}}}$ い。 \hfill\break
I\textquotesingle m scared. }

\par{3. Some verbs are used in \textbf{utterances whose implications are instantaneous }with said utterance. These instances create what is known as the “utterance present ( ${\overset{\textnormal{はつげんげんざい}}{\text{発言現在}}}$ ). }

\par{13. ${\overset{\textnormal{ちか}}{\text{誓}}}$ います。 \hfill\break
I vow. }

\par{14. ご ${\overset{\textnormal{めいふく}}{\text{冥福}}}$ をお ${\overset{\textnormal{いの}}{\text{祈}}}$ りします。 \hfill\break
I pray for his\slash her\slash their soul(s). }

\par{15. ${\overset{\textnormal{ふたり}}{\text{二人}}}$ の ${\overset{\textnormal{しあわ}}{\text{幸}}}$ せを ${\overset{\textnormal{ねが}}{\text{願}}}$ います。 \hfill\break
I wish for the two\textquotesingle s happiness. }

\par{16. ${\overset{\textnormal{やくそく}}{\text{約束}}}$ する! \hfill\break
I promise! }

\par{4. Sometimes, as if we\textquotesingle re narrating to ourselves, or perhaps when we are narrating, the -RU\slash U form is used similarly to that of an \textbf{infinitive }to describe what is happening\slash is to happen in front of the speaker\textquotesingle s\slash one\textquotesingle s eyes. Of course, if the action can\textquotesingle t literally be seen with the eyes, this discrepancy doesn\textquotesingle t stop this usage from being valid. }

\par{Grammar Note: An \textbf{infinitive }is the basic form of verb which has no inflection binding it to a particular subject and\slash or tense, and this too is a function of the -RU\slash U form. }

\par{17. ${\overset{\textnormal{あめ}}{\text{雨}}}$ が ${\overset{\textnormal{ふ}}{\text{降}}}$ る。 \hfill\break
Rain falls (in front of my eyes). }

\par{18. ${\overset{\textnormal{かんこくりょうり}}{\text{韓国料理}}}$ を ${\overset{\textnormal{た}}{\text{食}}}$ べ ${\overset{\textnormal{つ}}{\text{尽}}}$ くす。 \hfill\break
Consuming Korean cuisine. }

\par{19. ${\overset{\textnormal{やいば}}{\text{刃}}}$ が ${\overset{\textnormal{さ}}{\text{刺}}}$ さる。 \hfill\break
The blade pierces. }

\par{20. ${\overset{\textnormal{いき}}{\text{息}}}$ が ${\overset{\textnormal{と}}{\text{止}}}$ まる。 \hfill\break
My breath stops. }

\par{5. \textbf{Habitual repetition }is another facet of a person\textquotesingle s current state. -TE IRU can similarly be used to denote what one “always does,” but it must be used with adverbs of frequency to establish this meaning. This is so that it can show present habitual action rather than an ongoing action. Even so, it doesn\textquotesingle t denote an inherent habitualness. The -RU\slash U form need not have such adverbs for this meaning to be had. However, it conversely becomes far more open ended in interpretation without words like “always” or “every day.” It could be interpreted as future intent without context guiding the listener to think habitual action. }

\par{ Whenever the speaker feels a need to emphasize his current habit, especially when criticized for not doing something, the use of -TE IRU becomes imperative. Habitual statements with -RU\slash U are most suitable in neutral situations where there is no need to emphasize one\textquotesingle s current habit(s). }

\par{\textbf{Grammar Note }: This usage may also be used in the second and third person in question form. }

\par{21. いつも ${\overset{\textnormal{れいじ}}{\text{零時}}}$ に ${\overset{\textnormal{ね}}{\text{寝}}}$ ます。 \hfill\break
I always go to sleep at midnight. }

\par{22. ${\overset{\textnormal{わたし}}{\text{私}}}$ は ${\overset{\textnormal{しょくご}}{\text{食後}}}$ に ${\overset{\textnormal{は}}{\text{歯}}}$ を ${\overset{\textnormal{みが}}{\text{磨}}}$ きます。 \hfill\break
I brush my teeth after eating. \hfill\break
I will brush my teeth after eating (from now on). }

\par{23. ${\overset{\textnormal{まいあさ}}{\text{毎朝}}}$ シャワーを ${\overset{\textnormal{あ}}{\text{浴}}}$ びます。 \hfill\break
I take a shower every morning. }

\par{24. ${\overset{\textnormal{まいしゅうきょうかい}}{\text{毎週教会}}}$ に ${\overset{\textnormal{い}}{\text{行}}}$ きますか。 \hfill\break
Do you go to church every week? }

\par{25a. ${\overset{\textnormal{わたし}}{\text{私}}}$ は ${\overset{\textnormal{まいにちこうえん}}{\text{毎日公園}}}$ を ${\overset{\textnormal{さんぽ}}{\text{散歩}}}$ します。 \hfill\break
25b. ${\overset{\textnormal{わたし}}{\text{私}}}$ は ${\overset{\textnormal{まいにちこうえん}}{\text{毎日公園}}}$ を ${\overset{\textnormal{さんぽ}}{\text{散歩}}}$ しています。 \hfill\break
I walk the park every day. (25a) \hfill\break
I\textquotesingle m walking the park every day. (25b) }

\par{26. ${\overset{\textnormal{きみ}}{\text{君}}}$ がいると、いつも ${\overset{\textnormal{わら}}{\text{笑}}}$ えてる。 \hfill\break
I\textquotesingle m always able to laugh when you\textquotesingle re here (with me). }

\par{27. ${\overset{\textnormal{ひと}}{\text{人}}}$ に ${\overset{\textnormal{はな}}{\text{話}}}$ している ${\overset{\textnormal{とちゅう}}{\text{途中}}}$ で、 ${\overset{\textnormal{なに}}{\text{何}}}$ をしたかったのか ${\overset{\textnormal{わす}}{\text{忘}}}$ れてしまうことはありませんか。 \hfill\break
While talking to someone, do you ever forget what you wanted to talk about? }

\par{6. Yet another nuance that falls under the umbrella of current state is denoting a \textbf{characteristic and\slash or general truth }. However, a general truth need not literally be a current state. It could be a situation that regularly occurs under certain conditions. }

\par{28. カモメは ${\overset{\textnormal{おも}}{\text{主}}}$ に ${\overset{\textnormal{みずべ}}{\text{水辺}}}$ に ${\overset{\textnormal{す}}{\text{棲}}}$ みます。 \hfill\break
Seagulls mainly live at waterfronts. }

\par{29. ${\overset{\textnormal{ちゅうごくご}}{\text{中国語}}}$ って ${\overset{\textnormal{むずか}}{\text{難}}}$ しい。 \hfill\break
Chinese is difficult. }

\par{30. ${\overset{\textnormal{きみ}}{\text{君}}}$ はよく ${\overset{\textnormal{しゃべ}}{\text{喋}}}$ るね。 \hfill\break
You sure talk a lot. }

\par{31. ${\overset{\textnormal{けむし}}{\text{毛虫}}}$ が ${\overset{\textnormal{ちょう}}{\text{蝶}}}$ に ${\overset{\textnormal{か}}{\text{変}}}$ わります。 \hfill\break
Caterpillars become butterflies. }

\par{32. ${\overset{\textnormal{いっぱんじん}}{\text{一般人}}}$ は ${\overset{\textnormal{ひと}}{\text{人}}}$ を ${\overset{\textnormal{ころ}}{\text{殺}}}$ すと ${\overset{\textnormal{たいほ}}{\text{逮捕}}}$ されます。 \hfill\break
When ordinary person murders someone, he gets arrested. }

\par{33. ${\overset{\textnormal{にほん}}{\text{日本}}}$ は ${\overset{\textnormal{じしん}}{\text{地震}}}$ が ${\overset{\textnormal{おお}}{\text{多}}}$ いところだ。 \hfill\break
Japan is a place where there are many earthquakes. }

\par{34. ${\overset{\textnormal{じしん}}{\text{地震}}}$ はプレートが ${\overset{\textnormal{もと}}{\text{元}}}$ に ${\overset{\textnormal{もど}}{\text{戻}}}$ ろうとする ${\overset{\textnormal{とき}}{\text{時}}}$ に起こります。 \hfill\break
Earthquakes occur when plates try to return to their original positions. }

\begin{center}
\textbf{FUTURE TENSE }
\end{center}

\par{7. The future tense in English denotes \textbf{an action\slash state that has not yet happened }. Even when it is used by itself with no other modal changes, -RU\slash U can indicate something that you are rather certain will occur in the future. }

\par{35. もうすぐ ${\overset{\textnormal{しょうとうじかん}}{\text{消灯時間}}}$ だ。 \hfill\break
It\textquotesingle s almost lights-out. }

\par{36. ${\overset{\textnormal{でんき}}{\text{電気}}}$ が ${\overset{\textnormal{き}}{\text{消}}}$ えるね。 \hfill\break
The lights will go off, ok? }

\par{37. ${\overset{\textnormal{あした}}{\text{明日}}}$ は ${\overset{\textnormal{きゅうじつ}}{\text{休日}}}$ だ。 \hfill\break
Tomorrow is a holiday. }

\par{38. ${\overset{\textnormal{かえ}}{\text{帰}}}$ る ${\overset{\textnormal{とちゅう}}{\text{途中}}}$ で ${\overset{\textnormal{ぎんこう}}{\text{銀行}}}$ に ${\overset{\textnormal{よ}}{\text{寄}}}$ ってください。 \hfill\break
Please stop by the bank while you\textquotesingle re coming home. }

\par{39. きっと ${\overset{\textnormal{ごうかく}}{\text{合格}}}$ するよ。 \hfill\break
You\textquotesingle ll definitely pass. }

\par{8. -RU\slash U may show \textbf{first-person intention }and\slash or plan when used in the future tense. It must, though, be paired with a verb of volition (意志動詞). -RU\slash U may also simply provide information about what will happen in the future depending on the situation. In English, the -ing form or “going to…” pattern are frequently used for this. }

\par{\textbf{Grammar Note }: This usage may also be used in second person and third person in the form of a question. It may also be used in the affirmative in third person, but the -RU\slash U form must be paired with a modal change that incorporates a less direct tone. Lastly, when this usage is used in the affirmative in second person, it creates a command (See Usage 11). }

\par{40. ${\overset{\textnormal{あす}}{\text{明日}}}$ テキサスに ${\overset{\textnormal{た}}{\text{発}}}$ ちます。 \hfill\break
I head out to Texas tomorrow. }

\par{41. ${\overset{\textnormal{こんや}}{\text{今夜}}}$ は ${\overset{\textnormal{ばんごはん}}{\text{晩御飯}}}$ を ${\overset{\textnormal{がいしょく}}{\text{外食}}}$ にする。 \hfill\break
Tonight, I\textquotesingle m going to eat out for dinner. }

\par{42. ${\overset{\textnormal{かいしゃ}}{\text{会社}}}$ を ${\overset{\textnormal{や}}{\text{辞}}}$ めます。 \hfill\break
I\textquotesingle m quitting\slash going to quit my job (at the company). }

\par{43. ${\overset{\textnormal{しよう}}{\text{私用}}}$ で ${\overset{\textnormal{きゅうか}}{\text{休暇}}}$ を ${\overset{\textnormal{と}}{\text{取}}}$ ります。 \hfill\break
I\textquotesingle m taking\slash going to take }

\par{44. ${\overset{\textnormal{わたし}}{\text{私}}}$ は ${\overset{\textnormal{あす}}{\text{明日}}}$ から ${\overset{\textnormal{いっ}}{\text{1}}}$ ${\overset{\textnormal{しゅうかんきょうと}}{\text{週間京都}}}$ へ ${\overset{\textnormal{い}}{\text{行}}}$ きます。 \hfill\break
As of tomorrow, I will be going to Kyoto for a week. }

\par{45. はい、 ${\overset{\textnormal{わたし}}{\text{私}}}$ が ${\overset{\textnormal{い}}{\text{行}}}$ きます。 \hfill\break
Yes, I am the one going. }

\par{46. はい、 ${\overset{\textnormal{しょうひぜい}}{\text{消費税}}}$ は ${\overset{\textnormal{らいげつ}}{\text{来月}}}$ から ${\overset{\textnormal{にー}}{\text{2}}}$ ${\overset{\textnormal{パーセント}}{\text{\%}}}$ ほど ${\overset{\textnormal{ひ}}{\text{引}}}$ き ${\overset{\textnormal{あ}}{\text{上}}}$ げられることになります。 \hfill\break
Yes, as of next month, the consumers tax will be risen approximately two percent. }

\par{47. ${\overset{\textnormal{なんにん}}{\text{何人}}}$ ${\overset{\textnormal{き}}{\text{来}}}$ ますか。 \hfill\break
How many people are coming? }

\par{48. この ${\overset{\textnormal{くるま}}{\text{車}}}$ がいくらなら ${\overset{\textnormal{か}}{\text{買}}}$ いますか。 \hfill\break
How much would you buy this car for? }

\par{49. ${\overset{\textnormal{たいふう}}{\text{台風}}}$ ${\overset{\textnormal{さん}}{\text{3}}}$ ${\overset{\textnormal{ごう}}{\text{号}}}$ は、 ${\overset{\textnormal{きゅうしゅう}}{\text{九州}}}$ の ${\overset{\textnormal{なんぶ}}{\text{南部}}}$ に ${\overset{\textnormal{せっきん}}{\text{接近}}}$ していて、 ${\overset{\textnormal{ま}}{\text{間}}}$ もなく ${\overset{\textnormal{じょうりく}}{\text{上陸}}}$ する ${\overset{\textnormal{みこ}}{\text{見込}}}$ みです。 \hfill\break
Typhoon No. 3 is approached the southern portion of Kyushu, and it is forecast to make landfall any moment. }

\par{9.With a rising intonation, -RU\slash U indicates \textbf{surprise about a future event }. In second person, it can show disbelief, rebuke, or scoffing toward a statement the speaker deems improbable. However, it is not limited to these sorts of negative nuances in second person. You can express surprise about a future event in first and second person. }

\par{\textbf{Grammar Note }: -RU\slash U may also indicate surprise about a present state. Unlike the -TA form, it doesn't imply that the speaker should have known, and it doesn\textquotesingle t indicate the reality at hand as having been recognized in the past. }

\par{47. え、私が行く? \hfill\break
What, I\textquotesingle m going? }

\par{48. え、山田君がやる?冗談でしょう? \hfill\break
What, you\textquotesingle re going to do it, Yamada-kun? You\textquotesingle re joking, right? }

\par{49. あ、やってくれる?ありがとう! \hfill\break
Oh, you\textquotesingle re going to do it? Thank you! }

\par{50. ${\overset{\textnormal{せんそう}}{\text{戦争}}}$ が ${\overset{\textnormal{お}}{\text{終}}}$ わる?それは ${\overset{\textnormal{おそ}}{\text{恐}}}$ らく ${\overset{\textnormal{あ}}{\text{有}}}$ り ${\overset{\textnormal{え}}{\text{得}}}$ ないだろう。 \hfill\break
The war\textquotesingle s going to end? That\textquotesingle s probably impossible. }

\par{51. ${\overset{\textnormal{ことしじゅう}}{\text{今年中}}}$ に ${\overset{\textnormal{しゅとちょっかじしん}}{\text{首都直下地震}}}$ が ${\overset{\textnormal{お}}{\text{起}}}$ きます? \hfill\break
There\textquotesingle s going to be an earthquake directly hitting the Tokyo area within this year? }

\par{52. あ、そうだ! ${\overset{\textnormal{きょうごご}}{\text{今日午後}}}$ ${\overset{\textnormal{さん}}{\text{3}}}$ ${\overset{\textnormal{じはん}}{\text{時半}}}$ に ${\overset{\textnormal{かいぎ}}{\text{会議}}}$ がある。 \hfill\break
Ah, that\textquotesingle s right! There\textquotesingle s a meeting at 3:30 PM today. }

\par{53. あ、あの ${\overset{\textnormal{ひと}}{\text{人}}}$ は ${\overset{\textnormal{すずき}}{\text{鈴木}}}$ さんだ。 \hfill\break
Ah, that person is Mr. Suzuki. }

\begin{center}
\textbf{INSTRUCTION }
\end{center}

\par{10. The -RU\slash U form may also be used to show step-by-step instructions. This is frequently used in recipes. The instructions are not necessarily directed at one particular person; however, commands can be made by using the -RU\slash U form (Usage 11). }

\par{54. ${\overset{\textnormal{あぶら}}{\text{油}}}$ をフライパンに ${\overset{\textnormal{い}}{\text{入}}}$ れて、 ${\overset{\textnormal{ひゃくななじゅうご}}{\text{175}}}$ ${\overset{\textnormal{ど}}{\text{度}}}$ くらいになるまでに ${\overset{\textnormal{ひ}}{\text{火}}}$ を ${\overset{\textnormal{つ}}{\text{付}}}$ ける。 \hfill\break
Add cooking oil to the frying pan and heat until it is about 175 degrees. }

\par{55. ${\overset{\textnormal{なべ}}{\text{鍋}}}$ に ${\overset{\textnormal{ゆ}}{\text{湯}}}$ を ${\overset{\textnormal{わ}}{\text{沸}}}$ かし、 ${\overset{\textnormal{とうふ}}{\text{豆腐}}}$ を ${\overset{\textnormal{くず}}{\text{崩}}}$ し ${\overset{\textnormal{い}}{\text{入}}}$ れて、 ${\overset{\textnormal{ざる}}{\text{笊}}}$ にあげて ${\overset{\textnormal{みず}}{\text{水}}}$ を ${\overset{\textnormal{き}}{\text{切}}}$ る。 \hfill\break
Boil water in the pot, break up the tofu, and drain in a strainer. }

\par{56. ${\overset{\textnormal{ちょうみりょう}}{\text{調味料}}}$ と ${\overset{\textnormal{たま}}{\text{玉}}}$ ネギを ${\overset{\textnormal{くわ}}{\text{加}}}$ えて ${\overset{\textnormal{いた}}{\text{炒}}}$ め、 ${\overset{\textnormal{と}}{\text{溶}}}$ き ${\overset{\textnormal{たまご}}{\text{卵}}}$ を ${\overset{\textnormal{くわ}}{\text{加}}}$ えて ${\overset{\textnormal{かる}}{\text{軽}}}$ く ${\overset{\textnormal{ま}}{\text{混}}}$ ぜ、 ${\overset{\textnormal{ひ}}{\text{火}}}$ を ${\overset{\textnormal{と}}{\text{止}}}$ める。 \hfill\break
Sauté upon adding spices and onion, add a beaten egg and lightly mix, and then turn off the heat. }

\par{57. ${\overset{\textnormal{ぎゅうにく}}{\text{牛肉}}}$ は ${\overset{\textnormal{た}}{\text{食}}}$ べやすい ${\overset{\textnormal{おお}}{\text{大}}}$ きさに ${\overset{\textnormal{き}}{\text{切}}}$ り、 ${\overset{\textnormal{かたくりこ}}{\text{片栗粉}}}$ をまぶす。 \hfill\break
As for the beef, cut it into easy to eat sizes and then smear the beef with potato starch. }

\par{58. スキレットにごま ${\overset{\textnormal{あぶら}}{\text{油}}}$ を ${\overset{\textnormal{うす}}{\text{薄}}}$ くひいて ${\overset{\textnormal{ひ}}{\text{火}}}$ にかけ、ご ${\overset{\textnormal{はん}}{\text{飯}}}$ を ${\overset{\textnormal{ひろ}}{\text{広}}}$ げて ${\overset{\textnormal{の}}{\text{載}}}$ せる。 \hfill\break
Lightly cover the skillet with sesame oil, add heat, and spread the rice on top. }

\begin{center}
\textbf{COMMAND }
\end{center}

\par{11. When -RU\slash U is used to make a command, it does not have the same time constraint that -TA has. Although it implies that the listener better get to it--which is why it is often used by teachers, parents, or people with a clear higher status over someone else—it is not the case that it has to happen immediately for it to be grammatical. }

\par{59. さっさと ${\overset{\textnormal{かたづ}}{\text{片付}}}$ ける。 \hfill\break
Get to cleaning up. }

\par{60. ${\overset{\textnormal{じゅう}}{\text{10}}}$ ${\overset{\textnormal{びょうご}}{\text{秒後}}}$ に ${\overset{\textnormal{はし}}{\text{走}}}$ る。 \hfill\break
Run in ten seconds. }

\par{ This sentence would likely be said by a coach and\slash or someone who would be instructing you to do something. This sentence demonstrates how the “instruction” meaning of -RU\slash U derives from its sense of command. The pragmatic difference is that the “instructions” given in the -RU\slash U form are very frequently in polite speech. Even so, an instructor\slash knowledgeable person instructing is intrinsically higher in position than the listener learning from said individual. }

\par{61. すぐに ${\overset{\textnormal{た}}{\text{食}}}$ べる! \hfill\break
Eat it now! }

\par{\textbf{Sentence Note }: Ex. 61 would most likely be said by a semi-strict parent. }

\par{62. ${\overset{\textnormal{われわれ}}{\text{我々}}}$ は ${\overset{\textnormal{しゅっぱつ}}{\text{出発}}}$ する。 \hfill\break
We\textquotesingle re departing. }

\par{\textbf{Sentence Note }: The use of first person plural allows for a rather subtle yet explicit means of getting others to act alongside oneself. The person saying this would be the leader of the group. }

\par{63. ${\overset{\textnormal{きみ}}{\text{君}}}$ は ${\overset{\textnormal{あす}}{\text{明日}}}$ から ${\overset{\textnormal{おおさか}}{\text{大阪}}}$ へ ${\overset{\textnormal{しゅっちょう}}{\text{出張}}}$ \{する・だ\}。 \hfill\break
You will be going to Osaka on business starting tomorrow. }

\par{64. このお ${\overset{\textnormal{きゃく}}{\text{客}}}$ さんは ${\overset{\textnormal{きみ}}{\text{君}}}$ が ${\overset{\textnormal{せったい}}{\text{接待}}}$ \{する・だ\}。 \hfill\break
You\textquotesingle ll be entertaining this patron. }

\par{65. ${\overset{\textnormal{いま}}{\text{今}}}$ だ! \hfill\break
Now! }

\par{\textbf{Sentence Note }: Ex. 65 can be used without imposing a sense of social hierarchy. However, the person stating it would still be taking the initiative to get others to act. }

\par{65. ${\overset{\textnormal{か}}{\text{書}}}$ くんだ! \hfill\break
Write! }

\par{\textbf{Grammar Note }: Though not exactly the same, it is important to note that the -RU\slash U form is used with のだ・んだ instead of the -TA form. This is due to the time restriction placed on -TA form imperatives. }

\begin{center}
\textbf{PAST TENSE }
\end{center}

\par{12. -RU\slash U need not always refer to non-past time. There are instances where it does refer to a past event. If -TA were used, it would indicate that the speaker perceives the situation to be remote, but if -RU\slash U were used, it would mean that the past situation is perceived as if it were directly before the speaker. -TA suggests a detached, objective attitude on the part of the speaker toward the situation, but -RU\slash U suggests the speaker's subjective and psychological involvement with the situation. }

\par{ When both -RU\slash U and -TA are present together, the -RU\slash U event\slash state must either be clearly completed\slash established before the -TA event\slash state. For instance, in Ex. 67, Mr. Hirota had good-looking teeth before ever showing them when he smiled. Ex. 67 also demonstrates how this facet of the -RU\slash U form also affects choosing between -TE IRU and -TE ITA. The latter would show definitive completion of a once ongoing event, which is not logical to posit in Ex. 67. }

\par{66. ひどいことを ${\overset{\textnormal{い}}{\text{言}}}$ うね。 \hfill\break
What a horrible thing to say. }

\par{67. ${\overset{\textnormal{やまだ}}{\text{山田}}}$ さんは ${\overset{\textnormal{は}}{\text{歯}}}$ を ${\overset{\textnormal{だ}}{\text{出}}}$ して ${\overset{\textnormal{わら}}{\text{笑}}}$ った。 ${\overset{\textnormal{わり}}{\text{割}}}$ と ${\overset{\textnormal{きれい}}{\text{綺麗}}}$ な ${\overset{\textnormal{は}}{\text{歯}}}$ を ${\overset{\textnormal{も}}{\text{持}}}$ っている。 \hfill\break
Mr. Hirota smiled showing his teeth. He had rather good-looking teeth. }

\par{68. ${\overset{\textnormal{にょうぼう}}{\text{女房}}}$ がどうしても ${\overset{\textnormal{けっこん}}{\text{結婚}}}$ してほしいって ${\overset{\textnormal{な}}{\text{泣}}}$ いて ${\overset{\textnormal{たの}}{\text{頼}}}$ むから、 ${\overset{\textnormal{しかた}}{\text{仕方}}}$ なく ${\overset{\textnormal{けっこん}}{\text{結婚}}}$ してやったんだよ。 \hfill\break
My wife had cried and begged that I marry her, and so I reluctantly did. }

\par{69. ${\overset{\textnormal{たまご}}{\text{卵}}}$ がないと ${\overset{\textnormal{もんく}}{\text{文句}}}$ を ${\overset{\textnormal{い}}{\text{言}}}$ うから、もう ${\overset{\textnormal{いっかい}}{\text{一回}}}$ ${\overset{\textnormal{か}}{\text{買}}}$ いに ${\overset{\textnormal{い}}{\text{行}}}$ ったの。 \hfill\break
You complained about there being no eggs, so I went out again to go buy some. }

\par{70. ${\overset{\textnormal{ち}}{\text{血}}}$ も ${\overset{\textnormal{なみだ}}{\text{涙}}}$ もないことを ${\overset{\textnormal{い}}{\text{言}}}$ うから、バチが ${\overset{\textnormal{あ}}{\text{当}}}$ たったんだ。 \hfill\break
You got what you deserved for saying something so heartless. }
    