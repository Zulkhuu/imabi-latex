    
\chapter{~によって}

\begin{center}
\begin{Large}
第224課: ~によって 
\end{Large}
\end{center}
 
\par{ When we learned about the agent marker に, we learned how several functions of に are interrelated. From its basic definition of showing where something exists, に by extension marks the existence of an agent. As we learned in that lesson and the lessons that followed, に marks the agent of stative-transitive, passive sentences, and causative predicates. }

\par{ At the conclusion of our discussion on how に functions as an agent, we learned that に can in fact follow verbal expressions. By doing so, the verb becomes grammatically nominalized, but because the resultant noun is still verbal in nature, the “agent” becomes interpreted as a reason\slash purpose\slash cause for the main predicate of the sentence. Thus, the purpose-marker is born. }

\par{ In Lesson 115, we briefly discussed how ~によって can be used to mark the agent. This is the combination of the agent-marker に and よる, which has various nuances including “to be caused by\slash to depend on\slash to be based on.” At its basic understanding, it helps fully establish that something is an agent, especially when other usages of に are present. }
      
\section{~によって・により}
 
\par{1. 原因 (Cause)・理由 (Reason): The first usage of ~によって is to objectively and indifferently present cause\slash reason. This speech pattern, thus, is not used so much in general conversation due to its lack of emotion. It is, however, perfectly appropriate for speaking in a technical manner. }

\par{ In English, this usage translates as “due to.” Whenever it is used to directly modify a noun, you use ~による (Exs. 3-6). }

\par{\textbf{Spelling Note }: The Kanji spelling for this nuance is に因って. }

\par{1. ${\overset{\textnormal{ふみきり}}{\text{踏切}}}$ での ${\overset{\textnormal{おお}}{\text{大}}}$ きな ${\overset{\textnormal{しょうとつじこ}}{\text{衝突事故}}}$ によって、 ${\overset{\textnormal{うんこう}}{\text{運行}}}$ に ${\overset{\textnormal{にじゅう}}{\text{20}}}$ ${\overset{\textnormal{じかん}}{\text{時間}}}$ の ${\overset{\textnormal{おく}}{\text{遅}}}$ れが ${\overset{\textnormal{で}}{\text{出}}}$ た。 \hfill\break
Due to a large collision at the railroad crossing, a 20-hour delay came out in operations.  }

\par{2. このビルは ${\overset{\textnormal{おおじしん}}{\text{大地震}}}$ \{によって・で\}ばらばらに ${\overset{\textnormal{はかい}}{\text{破壊}}}$ された。 \hfill\break
This building was destroyed to pieces [by\slash in] the big earthquake. }

\par{3. ${\overset{\textnormal{かしつ}}{\text{過失}}}$ による ${\overset{\textnormal{じんしんじこ}}{\text{人身事故}}}$ \hfill\break
Traffic accident caused by a blunder  }

\par{4. ${\overset{\textnormal{おおじしん}}{\text{大地震}}}$ による ${\overset{\textnormal{かおくとうかい}}{\text{家屋倒壊}}}$ は、すでに ${\overset{\textnormal{いちまんこ}}{\text{一万戸}}}$ を ${\overset{\textnormal{こ}}{\text{超}}}$ えているといわれる。 \hfill\break
It's said that the number of homes that collapsed due to the big earthquake has already surpassed 10,000 homes. }

\par{5. ${\overset{\textnormal{ふくしまげんぱつじこ}}{\text{福島原発事故}}}$ による ${\overset{\textnormal{ほうしゃのうかんきょうおせん}}{\text{放射能環境汚染}}}$ \hfill\break
Environmental radioactivity pollution caused by the Fukushima Nuclear Accident  }

\par{6. ${\overset{\textnormal{うし}}{\text{牛}}}$ インフルエンザによる ${\overset{\textnormal{しぼうしゃ}}{\text{死亡者}}}$ が ${\overset{\textnormal{ぞくしゅつ}}{\text{続出}}}$ しています。 \hfill\break
Deaths are occurring one after another due to cow influenza.  }

\par{2. 手段 (Method): によって can also be used in the sense of “by” to express method. This usage has considerable overlap with the particle で, but because によって deals heavily in showing agents of happenstance, when method is being used in a way that is not a speaker literally using a tool\slash method, then only によって would be appropriate. }

\par{ When modifying nouns directly with this meaning, ~によって takes either the form ~による or ~によっての. }

\par{7. インターネットにより ${\overset{\textnormal{せかい}}{\text{世界}}}$ のニュースを ${\overset{\textnormal{し}}{\text{知}}}$ る。 \hfill\break
To know the world news by the Internet.  }

\par{8. インターネット\{によって・で\}、 ${\overset{\textnormal{にんげん}}{\text{人間}}}$ は ${\overset{\textnormal{い}}{\text{居}}}$ ながらにして、 ${\overset{\textnormal{せかい}}{\text{世界}}}$ を ${\overset{\textnormal{がいかん}}{\text{概観}}}$ できる。 \hfill\break
Through\slash with the internet, people can stay seated as they can survey the world.  }

\par{9. ${\overset{\textnormal{とうひょう}}{\text{投票}}}$ \{によって・で\} ${\overset{\textnormal{き}}{\text{決}}}$ めよう。 \hfill\break
Let's decide by vote.  }

\par{10. ${\overset{\textnormal{たいてい}}{\text{大抵}}}$ の ${\overset{\textnormal{せいこう}}{\text{成功}}}$ は ${\overset{\textnormal{ふだん}}{\text{不断}}}$ の ${\overset{\textnormal{どりょく}}{\text{努力}}}$ によって ${\overset{\textnormal{え}}{\text{得}}}$ られる。 \hfill\break
Most success is gained by ceaseless effort. }

\begin{center}
~によって VS ~を通じて・通して 
\end{center}

\par{ There is some interchangeability between ~によって, ~を通じて, and ~を通して with Usage 2. ~によって places emphasis on the connection between the method and effect\slash result, which is in line with how the agent-maker に behaves. The latter two phrases, on the other hand, place emphasis on the process.  }

\par{12. インターネット\{によって・を ${\overset{\textnormal{つう}}{\text{通}}}$ じて・を ${\overset{\textnormal{とお}}{\text{通}}}$ して\}、アルバムを ${\overset{\textnormal{はんばい}}{\text{販売}}}$ する。 \hfill\break
To sell albums by\slash via\slash through the Internet.  }

\par{13. ${\overset{\textnormal{せんきょ}}{\text{選挙}}}$ \{〇 によって・X を ${\overset{\textnormal{つう}}{\text{通}}}$ じて・X を ${\overset{\textnormal{とお}}{\text{通}}}$ して\} ${\overset{\textnormal{いいんちょう}}{\text{委員長}}}$ になる。 \hfill\break
To become the committee chairman by election. }

\par{3. 受身の動作主 (Agent of a Passive Structure): As we have already learned about, ~によって can be used with the passive form of a verb to show the agent (doer). If a sentence with a receiver (indirect object) is passivized, the agent must be marked by ~によって. Additionally, ~による is the only acceptable attribute form for this usage. }

\par{14. ${\overset{\textnormal{たなか}}{\text{田中}}}$ さん\{〇 によって・X に\} ${\overset{\textnormal{おかだ}}{\text{岡田}}}$ さんに ${\overset{\textnormal{てがみ}}{\text{手紙}}}$ が ${\overset{\textnormal{か}}{\text{書}}}$ かれた。 \hfill\break
A letter was written to Mr. Okada by Mr. Tanaka. }

\par{15. その ${\overset{\textnormal{しごと}}{\text{仕事}}}$ は ${\overset{\textnormal{かとう}}{\text{加藤}}}$ さんによってなされました。 \hfill\break
The work was done by Mr. Kato.  }

\par{16. ライトアップ\{によって・で\} ${\overset{\textnormal{むげんてき}}{\text{夢幻的}}}$ な ${\overset{\textnormal{こうけい}}{\text{光景}}}$ が ${\overset{\textnormal{う}}{\text{生}}}$ み ${\overset{\textnormal{だ}}{\text{出}}}$ された。 \hfill\break
A dreamy scenery was created by\slash with a light-up.  }

\par{17. ${\overset{\textnormal{まつもとせいちょう}}{\text{松本清張}}}$ による ${\overset{\textnormal{しょうせつ}}{\text{小説}}}$ \hfill\break
A novel by Matsumoto Seichō }

\par{4. 根拠 (Proof)・拠り所 (Basis): Following a rule\slash law\slash example\slash precedent\slash previous example\slash etc., it is used a lot to clearly show proof\slash basis. In this usage, ~による is the only acceptable attribute form. This should only be used when something clear is understood. It should not be used for mere hearsay.  }

\par{18. ${\overset{\textnormal{しょうげん}}{\text{証言}}}$ による ${\overset{\textnormal{しんじつ}}{\text{真実}}}$ \hfill\break
Truth according to testimony  }

\par{19. ${\overset{\textnormal{ぜんれい}}{\text{前例}}}$ \{による・に ${\overset{\textnormal{もと}}{\text{基}}}$ づく\} ${\overset{\textnormal{はんけつ}}{\text{判決}}}$ が ${\overset{\textnormal{くだ}}{\text{下}}}$ されることになる。 \hfill\break
A judgment is to be made based on precedent. }

\par{5. 場合 (Circumstance)・相応変化 (Reasonable Change): This usage is perhaps the most common in the spoken language, equating to “depending on…” Both ~による and ~によっての are correct attribute forms, although the latter is more common in the spoken language.  }

\par{20. ${\overset{\textnormal{てんこう}}{\text{天候}}}$ によっては ${\overset{\textnormal{ちゅうし}}{\text{中止}}}$ もありえるでしょう。 \hfill\break
Depending on the weather conditions, (the match) may possibly be canceled. }

\par{21. ${\overset{\textnormal{とき}}{\text{時}}}$ と ${\overset{\textnormal{ばあい}}{\text{場合}}}$ に\{よっての・よる\} ${\overset{\textnormal{たいしょ}}{\text{対処}}}$ \hfill\break
Approach which depends on the time and circumstance  }

\par{22. お ${\overset{\textnormal{かね}}{\text{金}}}$ によって、 ${\overset{\textnormal{ひと}}{\text{人}}}$ は ${\overset{\textnormal{か}}{\text{変}}}$ わるものです。 \hfill\break
People change depending on money. }

\par{23. この ${\overset{\textnormal{しごと}}{\text{仕事}}}$ は ${\overset{\textnormal{はんばいすう}}{\text{販売数}}}$ によって ${\overset{\textnormal{きゅうりょう}}{\text{給料}}}$ が ${\overset{\textnormal{か}}{\text{変}}}$ わります。 \hfill\break
As for this job, one's salary changes depending on sales. }

\par{24. ${\overset{\textnormal{いたまえちょう}}{\text{板前長}}}$ の ${\overset{\textnormal{うで}}{\text{腕}}}$ によって、 ${\overset{\textnormal{すし}}{\text{寿司}}}$ の ${\overset{\textnormal{あじ}}{\text{味}}}$ の ${\overset{\textnormal{よ}}{\text{良}}}$ し悪しが ${\overset{\textnormal{さゆう}}{\text{左右}}}$ されます。 \hfill\break
The quality of the flavor of sushi is dependent on the skill of the chef.  }

\par{25. ${\overset{\textnormal{ふく}}{\text{含}}}$ まれる ${\overset{\textnormal{ざいりょう}}{\text{材料}}}$ によって、いろいろな ${\overset{\textnormal{あじ}}{\text{味}}}$ のラーメンがあります。 \hfill\break
There are many flavors of ramen based on the ingredients put in it.  }

\par{26. ${\overset{\textnormal{ひと}}{\text{人}}}$ によって ${\overset{\textnormal{かんが}}{\text{考}}}$ えが ${\overset{\textnormal{ちが}}{\text{違}}}$ います。 \hfill\break
Opinions vary from person to person. }

\par{27. ${\overset{\textnormal{あいさつことば}}{\text{挨拶言葉}}}$ はその ${\overset{\textnormal{ひ}}{\text{日}}}$ の ${\overset{\textnormal{てんき}}{\text{天気}}}$ によっていろいろ ${\overset{\textnormal{い}}{\text{言}}}$ えばよい。 \hfill\break
It's best to say that there are various greetings based on the weather of the day. }

\par{28. ${\overset{\textnormal{ようび}}{\text{曜日}}}$ によって ${\overset{\textnormal{じゅぎょう}}{\text{授業}}}$ が ${\overset{\textnormal{か}}{\text{変}}}$ わります。 \hfill\break
Classes change depending on the day of the week. }

\par{29. ${\overset{\textnormal{ひと}}{\text{人}}}$ によって ${\overset{\textnormal{す}}{\text{好}}}$ きなものと ${\overset{\textnormal{きら}}{\text{嫌}}}$ いなものは ${\overset{\textnormal{こと}}{\text{異}}}$ なります。 \hfill\break
One's likes and dislikes differ from person to person. }

\par{30. 「でも、すごいのよ。 ${\overset{\textnormal{にかい}}{\text{二階}}}$ の ${\overset{\textnormal{ひと}}{\text{人}}}$ が ${\overset{\textnormal{ある}}{\text{歩}}}$ くと、 ${\overset{\textnormal{てんじょう}}{\text{天井}}}$ がみしみしって ${\overset{\textnormal{な}}{\text{鳴}}}$ るの。 ${\overset{\textnormal{まどわく}}{\text{窓枠}}}$ も ${\overset{\textnormal{き}}{\text{木}}}$ だし、 ${\overset{\textnormal{かざむ}}{\text{風向}}}$ きによってはかたかた ${\overset{\textnormal{な}}{\text{鳴}}}$ って、すきま ${\overset{\textnormal{かぜ}}{\text{風}}}$ が ${\overset{\textnormal{はい}}{\text{入}}}$ ってくるみたい。ドアなんか、鍵なんかなくても体当たりすれば開きそうよ」 \hfill\break
“But, it\textquotesingle s dreadful! When the people on the second-floor walk, the ceiling creaks. The window frame is made of wood, which makes it clatter in the wind, and it\textquotesingle s like a draft is coming in. The door, it seems like it\textquotesingle ll open if you charge at it even if there wasn't a lock." From 冷たい誘惑 by 乃南アサ. }
その ${\overset{\textnormal{}}{\text{仕事}}}$ は ${\overset{\textnormal{かとう}}{\text{加藤}}}$ さんによってなされました。 \hfill\break
The work was done by Mr. Kato. \hfill\break
      
\section{~によらず}
 
\par{ ~ず is an old negative auxiliary, and when it is paired with ~による, the resulting phrase gives a meaning of "in any situation everything\slash despite\slash regardless". }
 
\par{31. ${\overset{\textnormal{なにごと}}{\text{何事}}}$ によらず ${\overset{\textnormal{ちゅうい}}{\text{注意}}}$ が ${\overset{\textnormal{かんじん}}{\text{肝心}}}$ だ。 \hfill\break
No matter what, attention is crucial. }
 
\par{32. ${\overset{\textnormal{だれ}}{\text{誰}}}$ によらず ${\overset{\textnormal{ひと}}{\text{人}}}$ の ${\overset{\textnormal{ふしまつ}}{\text{不始末}}}$ の ${\overset{\textnormal{しりぬぐ}}{\text{尻拭}}}$ いなどしたくはないよ。 \hfill\break
No matter who it is, I don't want to reap the harvests off other people's misconduct. }
      
\section{~によると・によったら・によれば}
 
\begin{center}
\textbf{~によると } \hfill\break

\end{center}

\par{ Also seen as ~によりますと in extremely polite\slash formal speech, the most common usage of this is to mean "according to". }

\par{33. ${\overset{\textnormal{しんぶん}}{\text{新聞}}}$ によると ${\overset{\textnormal{たいふう}}{\text{台風}}}$ が ${\overset{\textnormal{せっきんちゅう}}{\text{接近中}}}$ だった。 \hfill\break
According to the newspaper, the typhoon was approaching. }

\par{34. ( ${\overset{\textnormal{てんき}}{\text{天気}}}$ ) ${\overset{\textnormal{よほう}}{\text{予報}}}$ によりますと、 ${\overset{\textnormal{あした}}{\text{明日}}}$ は ${\overset{\textnormal{は}}{\text{晴}}}$ れるそうです。 \hfill\break
According to the weather report, it'll clear up tomorrow. }

\par{35. ニュースによると、 ${\overset{\textnormal{あさ}}{\text{朝}}}$ は ${\overset{\textnormal{は}}{\text{晴}}}$ れだそうだ。 \hfill\break
According to the news, morning will be clear skies. }

\par{36. ${\overset{\textnormal{でんとう}}{\text{電灯}}}$ の ${\overset{\textnormal{はつめい}}{\text{発明}}}$ はエジソンによるとされている。 \hfill\break
The invention of electric light is credited to Edison. }

\begin{center}
\textbf{~によれば }
\end{center}

\par{ A more direct and personal means of saying "according to" is ~によれば. }

\par{37. }

\par{\hfill\break
お ${\overset{\textnormal{かあ}}{\text{母}}}$ さま ${\overset{\textnormal{がた}}{\text{方}}}$ のために }
 
\par{${\overset{\textnormal{こじき}}{\text{古事記}}}$ 、 ${\overset{\textnormal{にほんしょき}}{\text{日本書紀}}}$ ( ${\overset{\textnormal{きき}}{\text{記紀}}}$ )によれば、 ${\overset{\textnormal{いざなぎのみこと}}{\text{伊弉諾尊}}}$ ( ${\overset{\textnormal{だんしん}}{\text{男神}}}$ )、 ${\overset{\textnormal{いざなみみこと}}{\text{伊弉冉尊}}}$ ( ${\overset{\textnormal{めがみ}}{\text{女神}}}$ ) ${\overset{\textnormal{ふたはしら}}{\text{二柱}}}$ の ${\overset{\textnormal{おおがみ}}{\text{大神}}}$ による ${\overset{\textnormal{くにう}}{\text{国生}}}$ みの ${\overset{\textnormal{しんわ}}{\text{神話}}}$ は、 ${\overset{\textnormal{にほん}}{\text{日本}}}$ の ${\overset{\textnormal{こくどたんじょうものがたり}}{\text{国土誕生物語}}}$ です。 ${\overset{\textnormal{わたし}}{\text{私}}}$ たち ${\overset{\textnormal{にほんじん}}{\text{日本人}}}$ の ${\overset{\textnormal{そせん}}{\text{祖先}}}$ は、 ${\overset{\textnormal{にほん}}{\text{日本}}}$ の ${\overset{\textnormal{こくど}}{\text{国土}}}$ は ${\overset{\textnormal{かみさま}}{\text{神様}}}$ がお ${\overset{\textnormal{う}}{\text{生}}}$ みになり、 ${\overset{\textnormal{す}}{\text{住}}}$ みやすく ${\overset{\textnormal{よ}}{\text{良}}}$ い ${\overset{\textnormal{くに}}{\text{国}}}$ に ${\overset{\textnormal{つく}}{\text{作}}}$ りあげられたものと ${\overset{\textnormal{しん}}{\text{信}}}$ じていました。ですから、 ${\overset{\textnormal{にほん}}{\text{日本}}}$ の ${\overset{\textnormal{こくど}}{\text{国土}}}$ に ${\overset{\textnormal{す}}{\text{住}}}$ む ${\overset{\textnormal{にほんじん}}{\text{日本人}}}$ は、すべて ${\overset{\textnormal{かみさま}}{\text{神様}}}$ の ${\overset{\textnormal{こ}}{\text{子}}}$ としてかたく ${\overset{\textnormal{むす}}{\text{結}}}$ ばれている ${\overset{\textnormal{いしき}}{\text{意識}}}$ が非常に強かったのです。 \hfill\break
For Mothers }

\par{According to the Kojiki and Nihon Shoki (Kiki), the tale of the birth of our nation by the two-great kami Izanagi (male kami) and Izanami (female kami), is the creation story of Japan. Our Japanese ancestors that the kami created the Japanese land to make a great and easy to live country. That is why Japanese people living in the land of Japan felt a strong and tight binding connection to the land as all being children of the kami. }

\par{\textbf{Citation Note }: From かみさまのおはなし. }

\begin{center}
\textbf{~によったら }
\end{center}

\par{ ~によったら would directly translate as "if it were according to." It is most common in the set phrase ことによったら, meaning "perhaps." }

\par{38. \{ことによると・ことによったら\} ${\overset{\textnormal{れっしゃ}}{\text{列車}}}$ が ${\overset{\textnormal{ていし}}{\text{停止}}}$ するかもしれません。 \hfill\break
Perhaps the train might halt. }
    