    
\chapter{わけではない, わけがない, わけにはいかない}

\begin{center}
\begin{Large}
第210課: わけではない, わけがない, わけにはいかない 
\end{Large}
\end{center}
 
\par{ In the previous lesson, we learned all about the phrase わけだ and its relation with similar phrases. What we did not do is cover what happens when we use this in the negative. This is because there are three possible options that are all different from each other in their own unique ways. These phrases are as follows: }

\begin{itemize}

\item わけではない 
\item わけがない 
\item わけにはいかない 
\end{itemize}

\par{ Because the situations they are respectively used in are quite different, it will be very important to hone down on these differences in order to properly use them. }
      
\section{わけではない}
 
\par{\emph{ }わけではない is, simply put, the basic negation of わけだ. Therefore, it may be used in all five usages mentioned in the previous lesson. Therefore, context will be necessary to show whether it used in negating reason, cause\slash effect, acknowledge, or (re)stating fact. It is simply translated as “it is not that…” However, despite showing negation, it can be viewed as not being a 100\% denial. }

\par{1. ${\overset{\textnormal{ぜったい}}{\text{絶対}}}$ に ${\overset{\textnormal{はんたい}}{\text{反対}}}$ (だ)というわけではないんですが、もうすこし ${\overset{\textnormal{かんが}}{\text{考}}}$ えてみたいんです。 \hfill\break
It\textquotesingle s not that I\textquotesingle m absolutely against it, but I would like to think about it a little more. }

\par{2. 「 ${\overset{\textnormal{ぼく}}{\text{僕}}}$ は ${\overset{\textnormal{なん}}{\text{何}}}$ とも ${\overset{\textnormal{い}}{\text{言}}}$ えないんですけれど、 ${\overset{\textnormal{ねもと}}{\text{根本}}}$ さんも ${\overset{\textnormal{さんせい}}{\text{賛成}}}$ なんでしょうか」「いや、うちもその ${\overset{\textnormal{いけん}}{\text{意見}}}$ には ${\overset{\textnormal{ぜんめんてき}}{\text{全面的}}}$ に ${\overset{\textnormal{さんせい}}{\text{賛成}}}$ (だ)というわけじゃありません」 \hfill\break
“I can\textquotesingle t really say anything, but are you too in support of it, Mr. Nemoto?” ”No, it\textquotesingle s not that I\textquotesingle m wholeheartedly in support of the opinion either.” }

\par{3. ${\overset{\textnormal{かれし}}{\text{彼氏}}}$ の ${\overset{\textnormal{かんが}}{\text{考}}}$ えてることがすべて ${\overset{\textnormal{わ}}{\text{分}}}$ かるわけじゃないが、おおよそは ${\overset{\textnormal{わ}}{\text{分}}}$ かるよ。 \hfill\break
It\textquotesingle s not that I understand everything my boyfriend is thinking, but I generally do. }

\par{4. 「ベジタリアンということは、 ${\overset{\textnormal{にく}}{\text{肉}}}$ や ${\overset{\textnormal{さかな}}{\text{魚}}}$ は ${\overset{\textnormal{ぜんぜんた}}{\text{全然食}}}$ べないということですか。」「いろいろなベジタリアンがいますが、 ${\overset{\textnormal{わたし}}{\text{私}}}$ の ${\overset{\textnormal{ばあい}}{\text{場合}}}$ は、 ${\overset{\textnormal{ぜんぜん}}{\text{全然}}}$ ${\overset{\textnormal{た}}{\text{食}}}$ べないというわけではありません。 ${\overset{\textnormal{さかな}}{\text{魚}}}$ は ${\overset{\textnormal{た}}{\text{食}}}$ べることがありますから。つまり、 ${\overset{\textnormal{ぎょさいしょくしゅぎしゃ}}{\text{魚菜食主義者}}}$ なんです。ペスクタリアンとも ${\overset{\textnormal{い}}{\text{言}}}$ いますよ。」 \hfill\break
“Is being a vegetarian not eating any meat or fish?” “There are many kinds of vegetarians, but as for myself, it\textquotesingle s not that I don\textquotesingle t completely eat (meat). That\textquotesingle s because I eat fish. In other words, I follow pescaterianism. You can also call me a pescatarian.” }

\par{5. お ${\overset{\textnormal{さけ}}{\text{酒}}}$ はあまり ${\overset{\textnormal{の}}{\text{飲}}}$ みませんが、 ${\overset{\textnormal{の}}{\text{飲}}}$ めないわけじゃないですよ。 \hfill\break
I don\textquotesingle t really drink alcohol, but it\textquotesingle s not that I can\textquotesingle t drink it. }

\par{6. ${\overset{\textnormal{すべ}}{\text{全}}}$ てが ${\overset{\textnormal{わたし}}{\text{私}}}$ の ${\overset{\textnormal{せきにん}}{\text{責任}}}$ (だ)というわけではありません。 \hfill\break
That doesn't mean it's entirely my fault. }

\par{7. そこまで ${\overset{\textnormal{ほんき}}{\text{本気}}}$ で ${\overset{\textnormal{えいご}}{\text{英語}}}$ を ${\overset{\textnormal{べんきょう}}{\text{勉強}}}$ したいわけではない。 \hfill\break
It\textquotesingle s not that I want to seriously study English to that extent. }

\par{8. ${\overset{\textnormal{つゆ}}{\text{梅雨}}}$ の ${\overset{\textnormal{ころ}}{\text{頃}}}$ はよく ${\overset{\textnormal{ふ}}{\text{降}}}$ りますが、 ${\overset{\textnormal{まいにちふ}}{\text{毎日降}}}$ るわけではありません。 \hfill\break
In the rainy season it frequently rains, but it\textquotesingle s not the case that it rains every day. }

\par{9. ${\overset{\textnormal{し}}{\text{死}}}$ ぬ ${\overset{\textnormal{き}}{\text{気}}}$ でやっても ${\overset{\textnormal{ほんとう}}{\text{本当}}}$ に ${\overset{\textnormal{し}}{\text{死}}}$ ぬわけじゃないよ。 \hfill\break
Even if you did it with the will to die, it\textquotesingle s not that you will really die. }

\par{10. まだ ${\overset{\textnormal{けっこん}}{\text{結婚}}}$ していないですが、 ${\overset{\textnormal{けっこん}}{\text{結婚}}}$ したくないわけでもないんです。 \hfill\break
I\textquotesingle m not married yet, but it\textquotesingle s not that I don\textquotesingle t want to get married. }

\par{11. ホタテのこういうことを ${\overset{\textnormal{し}}{\text{知}}}$ りたかったわけじゃないのよ。 \hfill\break
It\textquotesingle s not the case at all that I wanted to know this about scallops. }

\par{12. 「この ${\overset{\textnormal{かんじゃ}}{\text{患者}}}$ は ${\overset{\textnormal{ぜんぜんたす}}{\text{全然助}}}$ からないんですか。」「いいえ、 ${\overset{\textnormal{ぜんぜんたす}}{\text{全然助}}}$ からないというわけではありませんが、 ${\overset{\textnormal{むずか}}{\text{難}}}$ しいでしょう。」 \hfill\break
“Can this patient not be saved?” “No, well it\textquotesingle s not the case that he can\textquotesingle t be helped at all, but it\textquotesingle ll be difficult.” }

\par{13. ${\overset{\textnormal{こい}}{\text{恋}}}$ を ${\overset{\textnormal{し}}{\text{知}}}$ らないわけじゃないし、 ${\overset{\textnormal{に}}{\text{逃}}}$ げてるわけでもない。 \hfill\break
It\textquotesingle s not that I don\textquotesingle t know love, and it\textquotesingle s also not even that I\textquotesingle m running away. }

\par{14.「 ${\overset{\textnormal{さとしくん}}{\text{智君}}}$ のこと、 ${\overset{\textnormal{す}}{\text{好}}}$ きじゃないの?」「 ${\overset{\textnormal{す}}{\text{好}}}$ きじゃないってわけじゃないけど…」 \hfill\break
“Do you not like Satoshi?” “It\textquotesingle s not that I don\textquotesingle t like him, but…” }

\par{15. ${\overset{\textnormal{わる}}{\text{悪}}}$ いことをしたわけじゃない。 \hfill\break
It\textquotesingle s not that I did something bad. }
      
\section{わけがない}
 Wake ga nai わけがないis used when you are negative the commonsense reasoning behind something being right\slash correct\slash impossible. In its own way, it is very similar to phrases arienai ありえないand shinjirarenai 信じられない. It can be translated as “there\textquotesingle s no way that…” 16. ビッグフートが存在(そんざい)するわけがない。 Biggu fūto ga sonzai suru wake ga nai. There\textquotesingle s no way that Big Foot exists. 17. 六月(ろくがつ)に雪(ゆき)が降(ふ)るわけがない。 Rokugatsu ni yuki ga furu wake ga nai. There\textquotesingle s no way that it will\slash can snow in June. 18. ペンギンは絶滅(ぜつめつ)するわけがないでしょう。 Pengin wa zetsumetsu suru wake ga nai deshō. Surely there\textquotesingle s no way that penguins will go extinct. 19. 偶然(ぐうぜん)なわけがないでしょう。 Gūzen na wake ga nai deshō. Surely there\textquotesingle s no way it was a coincidence. 20. 英語(えいご)はおろか、中国語(ちゅうごくご)も学(まな)べるわけないでしょう。 Eigo wa oroka, chūgokugo mo manaberu wake ni wa nai deshō. English is one thing, but there\textquotesingle s no way that I could learn Chinese, no? 21. 黙(だま)って見送(みおく)ることなど出来(でき)るわけがないでしょう。 Damatte miokuru koto nado dekiru wake ga nai deshō. There\textquotesingle s no way I can just stay silent and see (him) off, you know. 22. 皆(みんな)に信用(しんよう)されるわけがないでしょう。 Min\textquotesingle na ni shin\textquotesingle yō sareru wake ga nai deshō. There\textquotesingle s definitely no way that (he) would be trusted by everyone, you know. 23. 病院(びょういん)を退院(たいいん)して1(いち)日(にち)も過(す)ぎていないんだから、旅行(りょこう)に行(い)けるわけなんてないよ。 Byōin wo tai\textquotesingle in shite ichinichi mo sugite inai n da kara, ryokō ni ikeru wake nante nai yo. It hasn\textquotesingle t even been a day since (he) left the hospital, so there\textquotesingle s absolutely no way he\textquotesingle d go travelling. 24. 稚魚(ちぎょ)はたとえば小(ちい)さい池(いけ)だったら、全部駆除(ぜんぶくじょ)できるのかもしれないけど、琵琶湖(びわこ)みたいなところでは完全(かんぜん)に駆除(くじょ)できるわけがないでしょう。 Chigyo wa tatoeba chiisai ike dattara, zembu kujo dekiru no ka mo shirenai kedo, biwa-ko mitai na tokoro de wa kanzen ni kujo dekiru wake ga nai deshō. If say it were a small pond, you could exterminate all the juvenile fish, but if it were a large place like Lake Biwa, there\textquotesingle s no way that you could possibly completely exterminate them. 25. いくら頭(あたま)がいい人(ひと)でも、一年間(いちねんかん)で広東語(かんとんご)がマスターできるわけがありません。 Ikura atama ga ii hito demo, ichinenkan de kantongo ga masutā dekiru wake ga arimasen. No matter how smart someone is, there\textquotesingle s no way one could master Cantonese in year.  \emph{ }わけがない is used when you are negating the commonsense reasoning behind something being right\slash correct\slash impossible. In its own way, it is very similar to phrases ありえない and 信じられない. It can be translated as “there\textquotesingle s no way that…”  
\par{16. ビッグフートが ${\overset{\textnormal{そんざい}}{\text{存在}}}$ するわけがない。 \hfill\break
There\textquotesingle s no way that Big Foot exists. }
 
\par{17. ${\overset{\textnormal{ろくがつ}}{\text{六月}}}$ に ${\overset{\textnormal{ゆき}}{\text{雪}}}$ が ${\overset{\textnormal{ふ}}{\text{降}}}$ るわけがない。 \hfill\break
There\textquotesingle s no way that it will\slash can snow in June. }
 
\par{18. ペンギンは ${\overset{\textnormal{ぜつめつ}}{\text{絶滅}}}$ するわけがないでしょう。 \hfill\break
Surely there\textquotesingle s no way that penguins will go extinct. }
 
\par{19. ${\overset{\textnormal{ぐうぜん}}{\text{偶然}}}$ なわけがないでしょう。 \hfill\break
Surely there\textquotesingle s no way it was a coincidence. }
 
\par{20. ${\overset{\textnormal{えいご}}{\text{英語}}}$ はおろか、 ${\overset{\textnormal{ちゅうごくご}}{\text{中国語}}}$ も ${\overset{\textnormal{まな}}{\text{学}}}$ べるわけないでしょう。 \hfill\break
English is one thing, but there\textquotesingle s no way that I could learn Chinese, no? }
 
\par{21. ${\overset{\textnormal{だま}}{\text{黙}}}$ って ${\overset{\textnormal{みおく}}{\text{見送}}}$ ることなど ${\overset{\textnormal{でき}}{\text{出来}}}$ るわけがないでしょう。 \hfill\break
There\textquotesingle s no way I can just stay silent and see (him) off, you know. }
 
\par{22. ${\overset{\textnormal{みんな}}{\text{皆}}}$ に ${\overset{\textnormal{しんよう}}{\text{信用}}}$ されるわけがないでしょう。 \hfill\break
There\textquotesingle s definitely no way that (he) would be trusted by everyone, you know. }
 
\par{23. ${\overset{\textnormal{びょういん}}{\text{病院}}}$ を ${\overset{\textnormal{たいいん}}{\text{退院}}}$ して ${\overset{\textnormal{いち}}{\text{1}}}$ ${\overset{\textnormal{にち}}{\text{日}}}$ も ${\overset{\textnormal{す}}{\text{過}}}$ ぎていないんだから、 ${\overset{\textnormal{りょこう}}{\text{旅行}}}$ に ${\overset{\textnormal{い}}{\text{行}}}$ けるわけなんてないよ。 \hfill\break
It hasn\textquotesingle t even been a day since (he) left the hospital, so there\textquotesingle s absolutely no way he\textquotesingle d go traveling. }
 
\par{24. ${\overset{\textnormal{ちぎょ}}{\text{稚魚}}}$ はたとえば ${\overset{\textnormal{ちい}}{\text{小}}}$ さい ${\overset{\textnormal{いけ}}{\text{池}}}$ だったら、 ${\overset{\textnormal{ぜんぶくじょ}}{\text{全部駆除}}}$ できるのかもしれないけど、 ${\overset{\textnormal{びわこ}}{\text{琵琶湖}}}$ みたいなところでは ${\overset{\textnormal{かんぜん}}{\text{完全}}}$ に ${\overset{\textnormal{くじょ}}{\text{駆除}}}$ できるわけがないでしょう。 \hfill\break
If say it were a small pond, you could exterminate all the juvenile fish, but if it were a large place like Lake Biwa, there\textquotesingle s no way that you could possibly completely exterminate them. }
 
\par{25. いくら ${\overset{\textnormal{あたま}}{\text{頭}}}$ がいい ${\overset{\textnormal{ひと}}{\text{人}}}$ でも、 ${\overset{\textnormal{いちねんかん}}{\text{一年間}}}$ で ${\overset{\textnormal{かんとんご}}{\text{広東語}}}$ がマスターできるわけがありません。 \hfill\break
No matter how smart someone is, there\textquotesingle s no way one could master Cantonese in year. }
      
\section{わけにはいかない}
 
\par{ When stating that an action is not reasonable\slash proper for obvious reasons and cannot be done as an effect, you use わけにはいかない. It can be translated as “there\textquotesingle s no way\dothyp{}\dothyp{}\dothyp{}can…\slash \dothyp{}\dothyp{}\dothyp{}cannot afford to…” }

\par{26. まだまだ ${\overset{\textnormal{しゅうりょうほうこくだ}}{\text{終了報告出}}}$ してないから、 ${\overset{\textnormal{かえ}}{\text{帰}}}$ るわけにはいかないだろう。 \hfill\break
I still haven\textquotesingle t done the end of the day report, so there\textquotesingle s no way I can go home. }

\par{27. ${\overset{\textnormal{あいまい}}{\text{曖昧}}}$ な ${\overset{\textnormal{たいど}}{\text{態度}}}$ を ${\overset{\textnormal{と}}{\text{取}}}$ るわけにはいきません。 \hfill\break
You cannot afford to take a vague attitude. }

\par{28. ${\overset{\textnormal{かぜ}}{\text{風邪}}}$ を ${\overset{\textnormal{ひ}}{\text{引}}}$ いてしまったんですが、 ${\overset{\textnormal{きょう}}{\text{今日}}}$ は ${\overset{\textnormal{だいじ}}{\text{大事}}}$ な ${\overset{\textnormal{かいぎ}}{\text{会議}}}$ があるので、 ${\overset{\textnormal{やす}}{\text{休}}}$ むわけにはいきません。 \hfill\break
I caught a cold, but because I have an important meeting today, I can\textquotesingle t afford to take the day off. }

\par{29. ${\overset{\textnormal{くるま}}{\text{車}}}$ で ${\overset{\textnormal{き}}{\text{来}}}$ ちゃったから、お ${\overset{\textnormal{さけの}}{\text{酒飲}}}$ むわけにはいかん。 \hfill\break
I accidentally came by car, so there\textquotesingle s no way I can drink. }

\par{30. ${\overset{\textnormal{かれし}}{\text{彼氏}}}$ が ${\overset{\textnormal{つく}}{\text{作}}}$ った ${\overset{\textnormal{りょうり}}{\text{料理}}}$ は ${\overset{\textnormal{から}}{\text{辛}}}$ すぎても、 ${\overset{\textnormal{た}}{\text{食}}}$ べないわけにはいかない。 \hfill\break
Even if my boyfriend\textquotesingle s cooking is too spicy, there\textquotesingle s no way I can just not eat it. }

\par{31. ${\overset{\textnormal{やす}}{\text{休}}}$ みだからといって、 ${\overset{\textnormal{べんきょう}}{\text{勉強}}}$ しないわけではない。 \hfill\break
I can\textquotesingle t afford to not be studying just because we\textquotesingle re on holiday. }

\par{32. ${\overset{\textnormal{ぎふぼ}}{\text{義父母}}}$ が ${\overset{\textnormal{たず}}{\text{訪}}}$ ねてくるから、 ${\overset{\textnormal{そうじ}}{\text{掃除}}}$ しないわけにはいかないですね。 \hfill\break
My in-laws are coming to visit, so I can\textquotesingle t afford not to clean. }

\par{33. ${\overset{\textnormal{やまぐち}}{\text{山口}}}$ さんも ${\overset{\textnormal{いっしょ}}{\text{一緒}}}$ に ${\overset{\textnormal{こ}}{\text{来}}}$ てくださるわけにはいきませんか。 \hfill\break
Is there no way that you can\textquotesingle t come with us, Mr. Yamaguchi? }

\par{34. ${\overset{\textnormal{ふゆ}}{\text{冬}}}$ だからって、 ${\overset{\textnormal{まいにちかみ}}{\text{毎日髪}}}$ を ${\overset{\textnormal{あら}}{\text{洗}}}$ わないわけにはいかない。 \hfill\break
I can\textquotesingle t afford to not wash my hair every day just because it\textquotesingle s winter. }

\par{35. ${\overset{\textnormal{にほんごきょういく}}{\text{日本語教育}}}$ は ${\overset{\textnormal{りそう}}{\text{理想}}}$ ばかり ${\overset{\textnormal{い}}{\text{言}}}$ っても ${\overset{\textnormal{げんじつ}}{\text{現実}}}$ を ${\overset{\textnormal{むし}}{\text{無視}}}$ するわけにはいきません。 \hfill\break
Even if Japanese education only speaks of ideals, we can\textquotesingle t afford to ignore reality. }

\par{36. ${\overset{\textnormal{ま}}{\text{負}}}$ けるわけにはいかないよ。 \hfill\break
I can\textquotesingle t afford to lose! }

\par{37. ${\overset{\textnormal{つか}}{\text{疲}}}$ れたからってやめるわけにはいかない。 \hfill\break
You can't afford to quit just because you got tired. }

\par{38. そうかといって、 ${\overset{\textnormal{しん}}{\text{信}}}$ じるわけにはいかない。 \hfill\break
Even so, you can\textquotesingle t afford to believe (that). }

\par{39. 「 ${\overset{\textnormal{あす}}{\text{明日}}}$ まで ${\overset{\textnormal{ま}}{\text{待}}}$ っていただけないでしょうか」「ええ、でも、 ${\overset{\textnormal{いそ}}{\text{急}}}$ いでいるので、 ${\overset{\textnormal{なが}}{\text{長}}}$ い ${\overset{\textnormal{あいだま}}{\text{間待}}}$ つわけにはいかないんです」 \emph{\hfill\break
}"Could you wait until tomorrow?" "Yes, but, since I'm in a hurry, I can't afford to wait long." \hfill\break
}

\par{40. ${\overset{\textnormal{ことわ}}{\text{断}}}$ るわけにはいかないので、 ${\overset{\textnormal{めいれい}}{\text{命令}}}$ に ${\overset{\textnormal{したが}}{\text{従}}}$ おう。 \hfill\break
I cannot afford to refuse, and so I\textquotesingle ll follow the order. }
    