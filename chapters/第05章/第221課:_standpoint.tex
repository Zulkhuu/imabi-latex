    
\chapter{Standpoint}

\begin{center}
\begin{Large}
第221課: Standpoint: として(は) VS にとって(は) 
\end{Large}
\end{center}
 
\par{ This lesson is about relatively similar speech modals that concern describing standpoint or point of view. Although their English translations cause great confusion to learners, don't let this beat you. }
      
\section{~として(は)}
 
\par{1. The first uses of ~として that we will study are after nominal phrases. This is something really important to keep in mind because other usages of this will go after other things. So, don't get things confused in your mind. }

\par{ The first thing that として is most known for is when it shows "qualification\slash title". In other words, it functions like the as in "to carry out one's duty \textbf{as }an officer". It can also be used to show what you have something as. This is somewhat causative in nature, but in this sense it is closer to utility. So, just as you can say, "to use a sharp stick \textbf{as }a knife", you can say 鋭い枝をナイフとして使う. Notice, though, that "as" in English can be used in many more situations than what has just been defined for として. }

\par{ "(Aは)B としては" is typically just a contrastive usage of として, but it can have nuances more similar to "among" in the sense of ~の場合には. It could also emphasis a side of judgment. For instance, if you were to say 私としては, you are emphasizing your side of opinion. }

\par{\textbf{Particle Note }: When used to modify another noun phrase, which is possible as well in English, you must use ~としての. }

\par{1. ジョン ${\overset{\textnormal{まんじろう}}{\text{万次郎}}}$ は、日本人として始めて、アメリカで教育を受けた。 \hfill\break
Jon Manjiro was the first person as a Japanese to receive an education in America. }

\par{2. 石炭を燃料として使う。 \hfill\break
To use coal as fuel. }

\par{3. 私一己としては \hfill\break
For my own part\slash personally }

\par{4. 私としては、彼の意見に反対です。 \hfill\break
As for me, I don't agree with his opinion. }

\par{5. 最高傑作として作る。X   \textrightarrow   最高傑作を作る。 \hfill\break
To create as a masterpiece. \hfill\break
\hfill\break
6. 最高傑作としての条件 〇 \hfill\break
Conditions as a best work masterpiece }

\par{7. くそあいつは教師としても無能だぞ。(Vulgar) \hfill\break
That guy is also incompetent as a teacher. }

\par{\textbf{Usage Note }: This usage is deemed to be case. }

\par{2. Shows complete negation with a word showing unit. }

\par{8. 僕は、一日として彼女のことを思わない日はない。 \hfill\break
There's not even a single day that I don't think about her. }

\par{9. 一人として賛成しません。 \hfill\break
Not one person can agree. }

\par{\textbf{Usage Note }: This usage is deemed to be adverbial. }

\par{3. After the 終止形 it means "in assuming that" or "with the reason of". }

\par{10. 話は後でするとして、まず食事にしましょう。 \hfill\break
Let's leave talking for later and eat first. }

\par{\textbf{Usage Note }: This usage is deemed to be conjunctive. }

\par{\textbf{Grammar Note }: として is a grammaticalized object that should be studied separately from とする. Though its usages derive from it, it has become its own grammar point in its own right. Thus, it should be differentiated from a として from とする used at the end of a dependent clause, in which て would still have a conjunctive function. The として is also not the same as the として found in the 連用形 of タル形容動詞. }

\par{11. 堂堂としている。 \hfill\break
It's magnificent. (Grand; impressive) }

\par{12. われわれはその実験の結果を基礎として、学会雑誌に発表することになりました。 \hfill\break
We had the results of that experiment as our basis and published (the results) in a scientific journal. }

\begin{center}
 \textbf{In Relation to }\textbf{~とする }
\end{center}

\par{ Up to now, there have been several mentions of ~とする. How exactly does this fit with this section? Furthermore, how are you supposed to know what you're looking at? "AをBとする" is "to make\slash have A be B" a certain way where と shows the content\slash substance of a result. ~とする can mean "to presume; to think that; etc. }

\par{13. その結果を基礎としている。 \hfill\break
To be having the results as the basis. }

\par{14. 業としている。 \hfill\break
To be pursuing a vocation. }

\par{15. 石炭を燃料とする。 \hfill\break
To have coal be (used as) fuel. }

\par{16. 今は配慮を必要としている。 \hfill\break
We need forethought now. }

\par{17. 一時として目が離せないよ。 \hfill\break
My eyes can leave a single moment. }

\par{18. 長官として発言する。 \hfill\break
To speak as a general. }

\par{19. 初めて会う人やよく知らない人には敬語を使うのが礼儀だとされていますよ。 \hfill\break
It is deemed proper etiquette to use Keigo towards people such as those you first meet or don't know well. }

\par{20. 人を殺すのが犯罪(だ)とされていない者はいるのか。 \hfill\break
Is there a person that doesn't deem killing a person a crime? }

\begin{center}
\textbf{~としても }
\end{center}

\par{~としても is after nominal phrases to mean "even the position of\dothyp{}\dothyp{}\dothyp{}" or "even if\dothyp{}\dothyp{}\dothyp{}were to\dothyp{}\dothyp{}\dothyp{}". }

\par{21. 私としても、この事柄は重要ではないとよく分かります。 \hfill\break
As for me, too, I understand well that this matter is not important. }

\par{22. 新車を買うとしても、今度もやっぱ(り)黒いのにしよう。 \hfill\break
Even if we were to buy a new vehicle, let's surely get the black one this time. }

\par{23. 行けるとしても、少し遅れてしまうかも。 \hfill\break
Even if we were able to go, we would end up being a little late. }

\par{24. 生まれ変わったとしても、またそんなことをするのか。 \hfill\break
Even if I were born again, would such a thing happen again? }
      
\section{~にとって}
 
\par{ ~にとって, equivalent to "(の立場・たちば)からみて", means "to\slash for" and shows someone's perspective on something. ~にとって may be with は to make a \textbf{slightly contrasting }image or with の attributively. It may be ~にとりまして in polite speech and ~にとっちゃ(あ) in slang. }

\par{25. 彼にとっちゃその方がええんだよ。(砕けた) \hfill\break
For him, that way's better. }

\par{26. 初心者にとってはあのクラスは難しい。 \hfill\break
That class is difficult to beginners. }

\par{27. 人間にとっての ${\overset{\textnormal{ことがら}}{\text{事柄}}}$ \hfill\break
A circumstance to humans }

\par{28. 韓国語を知っている人にとって、日本語は難しくないはずです。 \hfill\break
For people who know Korean, Japanese isn't supposed to be difficult. }

\par{29. ${\overset{\textnormal{かくばくだん}}{\text{核爆弾}}}$ は世界平和にとっての重大な ${\overset{\textnormal{きょうい}}{\text{脅威}}}$ です。 \hfill\break
Nuclear weapons are great threats to world peace. }
      
\section{~として VS ~にとって}
 
\par{ Many students often overextend the usages of these two speech modals. Common mistakes include confusing these patterns with topicalization and "although phrases". As a quick exercise, consider the following sentences and what you should do to fix them. }

\par{30a. 彼女は学生として、学校へ行かず、遊んでばかりいるよ。X \hfill\break
30b. 外国人として大変なのは食べ物だと思います。X \hfill\break
30c. 僕にとってその色はあんまり好きじゃない。X }

\par{ Hopefully you found something wrong with these sentences. The first order of business is to where these speech modals literally come from. ~として comes from the case particle と + the verb する; ~にとって comes from the case particle に + the verb 取る. Once put together, they mark their own unique case function. Thus, some people often call them "compound case particles". }

\par{ "Noun + ~として" and "Noun + ~にとって" both show position\slash point of view, but in the case of the former, it is often used to state an action that one does\slash did as qualification and\slash or from a position\slash standpoint. This may sound confusing, but consider the following. }

\par{31. クラスの代表として委員会に出席しました。(Qualification\slash position) \hfill\break
I attended the committee meeting as the class representative. }

\par{32. あなたの医者として、食事の量を減らすよう忠告します。 \hfill\break
As your doctor, I advise you to decrease the amount you eat. }

\par{ On the other hand, ~にとって shows an evaluation or value judgment from the\slash a standpoint (of someone). }

\par{33. 私にとりまして、出張は大変思い出深かったです。 \hfill\break
For me, the business trip was very profoundly memorable. }
 
\par{\textbf{Usage Note }: ~にとりまして is normally used when talking to someone much older than you. So, the above sentence would most likely be 私にとって、出張は大変思い出深かったです. }

\par{34. これは私にとって忘れられない思い出です。 \hfill\break
This is a memory to me that I can't forget. }

\par{35. これはあたしにとってとっても大切なものなんだわ。(Feminine; Emphatic) \hfill\break
This is something very important to me. }

\begin{center}
 \textbf{With は }
\end{center}

\par{ は is added to them when raising a circumstance as the topic, emphasis, or show a contrast. }

\par{36. 私としては、その考えには賛成いたしかねます。 \hfill\break
As for me, I can't agree to that idea. }

\par{37. あんたにとっちゃ朝飯前かもしれんが、俺にとっちゃ大変な仕事なんだよ。(砕けた) \hfill\break
To you it may be trivial, but to me it's a big job. }

\begin{center}
 \textbf{Contrast }
\end{center}

\par{38. 人\{として・にとって]大切なことは何か。 }

\par{ With ~として, the sentence suggests that what's being important should be done by people whereas ~にとって simply states what's so, or in this case simply stating a question. }

\par{39. }

\par{Aさん: A銀行がB銀行と ${\overset{\textnormal{がっぺい}}{\text{合併}}}$ するそうですよ。 \hfill\break
Bさん: へえ。でも、A銀行\{としては・にとっては\}、そう悪いことではないんじゃないですか。 \hfill\break
Aさん: でも、B銀行としては黙って見ているわけにはいかないんじゃないですか。 }

\par{40. }
 
\par{Aさん: ${\overset{\textnormal{もんぶしょう}}{\text{文部省}}}$ としてはどう考えていますか。 \hfill\break
Bさん: 文部省としましては、今回の ${\overset{\textnormal{しょち}}{\text{処置}}}$ はやむを得ないものと考えております。(謙譲語) }
 
\par{Exercise: Translate the last two passages! }

\par{ These patterns also differ greatly in what kinds of sentences they're used in. }

\begin{ltabulary}{|P|P|}
\hline 

 & ~ \textbf{として }\\ \cline{1-2}

Circumstance & There is a sentence that shows that something happens, and the sentence after shows an opinion or stance concerning it. \\ \cline{1-2}

Opinion & Used to present one's opinion and then show an opposite position. \\ \cline{1-2}

\end{ltabulary}

\par{41. 市から感謝状と記念品をいただきました。これは我が家の家宝として大切にしたいです。 \hfill\break
I received a letter of thanks and mementos from the city. I would like to treat this as a family treasure in our home. }

\par{42. 義務教育の段階で外国語を教えた方がいい。外国語が話せるのは社会人として絶対に必要な教育である。 \hfill\break
Foreign language should be taught at the level of mandatory education. Being able to speak a foreign language is absolutely necessary education for a person of society. }

\begin{ltabulary}{|P|P|}
\hline 

 & ~ \textbf{にとって }\\ \cline{1-2}

~にとって~の\slash ことは~だ\slash である & A very emphatic derivative of the basic form. \\ \cline{1-2}

In topic sentence & After introducing the topic, a concrete explanation follows. \\ \cline{1-2}

\end{ltabulary}

\par{43. 今僕にとって唯一の楽しみになっているのは日本語の勉強です。 \hfill\break
What has now become my sole enjoyment is my Japanese studies. }

\par{44. 昔の人々の生活にとって、お寺は欠かせないものだった。 \hfill\break
The temple was a necessary thing to the lives of the ancient. }

\begin{center}
 \textbf{Words Used with Them }
\end{center}

\par{1. XとしてY }

\par{Words that follow Y: }

\par{Verbs that describe action: 言う、許す、行動する. \hfill\break
Verbs\slash adjectives that describe condition: 恥ずかしい、有名だ、知られている. }
 
\par{45. 彼は弁護士としては無能です。 \hfill\break
He is no good as a lawyer. }

\par{2. XにとってY \hfill\break
\hfill\break
Words that follow Y: }

\par{Potential expressions: 可能だ、忘れられない. \hfill\break
Adjectives of evaluation: 大切だ、難しい、大変だ. \hfill\break
Adjective + Noun }

\par{46. 世界の経済にとって、 ${\overset{\textnormal{きんゆう}}{\text{金融}}}$ を引き ${\overset{\textnormal{し}}{\text{締}}}$ めるのは ${\overset{\textnormal{この}}{\text{好}}}$ ましくない事態だ。 \hfill\break
For the world economy, tightening the money market is not a desirable situation. }

\par{あ }

\par{なたの ${\overset{\textnormal{}}{\text{医者}}}$ として、 ${\overset{\textnormal{}}{\text{食事}}}$ の ${\overset{\textnormal{}}{\text{量}}}$ を ${\overset{\textnormal{}}{\text{減}}}$ らすよう ${\overset{\textnormal{}}{\text{忠告}}}$ します。 \hfill\break
As your doctor, I advise you to decrease the amount you eat. }

\par{Aさん: A銀行がB銀行と合併するそうですよ。 \hfill\break
Bさん: へえ。でも、A銀行\{としては・にとっては\}、そう悪いことではないんじゃないですか。 \hfill\break
Aさん: でも、B銀行としては黙って見ているわけにはいかないんじゃないですか。 }

\par{Aさん: 文部省としてはどう考えていますか。 \hfill\break
Bさん: 文部省としましては、今回の処置はやむを得ないものと考えております。(謙譲語) }

\par{Exercise: Translate the last two passages! }
    