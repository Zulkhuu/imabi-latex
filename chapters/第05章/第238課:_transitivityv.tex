    
\chapter{Intransitive \& Transitive}

\begin{center}
\begin{Large}
第238課: Intransitive \& Transitive: Part 4 
\end{Large}
\end{center}
 
\par{ In this fourth lesson on verbs with both intransitive and transitive usages, we\textquotesingle ll continue to uncover peculiarities in Japanese at the individual word basis. }
      
\section{持つ, 馳せる, 跳ねる, はだける, 生じる, 踊る, 寄せる, 誤る, 笑う,つとめる}
 
\begin{center}
\textbf{持つ }
\end{center}

\par{ As a transitive verb, \emph{ }持つ means “to hold\slash possess\slash have.” As an intransitive verb, it means to keep (as in perishable goods) or “to be durable (as in the body).” As an intransitive verb, it is usually spelled as もつ. }

\par{1. ${\overset{\textnormal{だれ}}{\text{誰}}}$ でも ${\overset{\textnormal{じぶん}}{\text{自分}}}$ の ${\overset{\textnormal{こうどう}}{\text{行動}}}$ に ${\overset{\textnormal{せきにん}}{\text{責任}}}$ を ${\overset{\textnormal{も}}{\text{持}}}$ っている。 \hfill\break
Everyone holds responsible for his own actions. }

\par{2. ブランドの ${\overset{\textnormal{さいふ}}{\text{財布}}}$ を ${\overset{\textnormal{も}}{\text{持}}}$ っています。 \hfill\break
I have a brand wallet. }

\par{3. このままでは ${\overset{\textnormal{からだ}}{\text{体}}}$ がもたない。 \hfill\break
At this rate, my body won\textquotesingle t last. }

\par{4. お ${\overset{\textnormal{みそしる}}{\text{味噌汁}}}$ を ${\overset{\textnormal{つく}}{\text{作}}}$ ったら、 ${\overset{\textnormal{なんにち}}{\text{何日}}}$ くらいもちますか。 \hfill\break
Once you've made miso soup, about how many days is it good for? }

\begin{center}
\textbf{馳せる }
\end{center}

\par{\emph{ }馳せる has almost entirely disappeared from Modern Japanese, but its grammar is interesting. In the physical sense, it either means “to hurry\slash run to…” or “to ride…fast.” Nowadays, the verb is usually limited to set phrases like 思いを馳せる (to give more than a passing thought to…). }

\par{5. ${\overset{\textnormal{かれ}}{\text{彼}}}$ は ${\overset{\textnormal{よりとも}}{\text{頼朝}}}$ のもとへ(と) ${\overset{\textnormal{は}}{\text{馳}}}$ せようとした。 \hfill\break
He ran for Yoritomo\textquotesingle s side. }

\par{6. ${\overset{\textnormal{しろ}}{\text{城}}}$ の ${\overset{\textnormal{ほう}}{\text{方}}}$ から ${\overset{\textnormal{うま}}{\text{馬}}}$ が\{ ${\overset{\textnormal{はし}}{\text{走}}}$ り ${\overset{\textnormal{よ}}{\text{寄}}}$ って・ ${\overset{\textnormal{か}}{\text{駆}}}$ け ${\overset{\textnormal{よ}}{\text{寄}}}$ って・ ${\overset{\textnormal{は}}{\text{馳}}}$ せて\} ${\overset{\textnormal{き}}{\text{来}}}$ た。 \hfill\break
Horses came running from the direction of the castle. }

\par{7. ${\overset{\textnormal{むしゃ}}{\text{武者}}}$ が ${\overset{\textnormal{うま}}{\text{馬}}}$ を\{ ${\overset{\textnormal{はし}}{\text{走}}}$ らせながら・ ${\overset{\textnormal{は}}{\text{馳}}}$ せながら\} ${\overset{\textnormal{や}}{\text{矢}}}$ を ${\overset{\textnormal{い}}{\text{射}}}$ た。 \hfill\break
The warriors shot arrows as they raced their horses. }

\par{8. ${\overset{\textnormal{べいこく}}{\text{米国}}}$ の ${\overset{\textnormal{く}}{\text{暮}}}$ らしに ${\overset{\textnormal{おも}}{\text{思}}}$ いを\{ ${\overset{\textnormal{めぐ}}{\text{巡}}}$ らして・ ${\overset{\textnormal{は}}{\text{馳}}}$ せて\}います。 \hfill\break
I\textquotesingle ve been thinking nostalgically upon my living in America. }

\begin{center}
\textbf{跳ねる }
\end{center}

\par{\emph{ Haneru }as an intransitive verb means “to jump\slash leap\slash splash,” and as a transitive verb it means “to splash\slash hit (with a car)\slash reject” among other things. Traditionally, the intransitive form is spelled as 跳ねる and the transitive form is spelled as 撥ねる.  Usually, though, 跳ねる or はねる will work. }

\par{ Grammatically speaking, the intransitive form cannot be used in certain forms such as the passive. In such instances, the transitive form must be used. }

\par{9. ${\overset{\textnormal{あ}}{\text{揚}}}$ げ ${\overset{\textnormal{もの}}{\text{物}}}$ をしていて ${\overset{\textnormal{あぶら}}{\text{油}}}$ が ${\overset{\textnormal{は}}{\text{跳}}}$ ねて ${\overset{\textnormal{め}}{\text{目}}}$ に ${\overset{\textnormal{はい}}{\text{入}}}$ った。 \hfill\break
I was deep-frying food when oil splashed up and got in my eyes. }

\par{10. ${\overset{\textnormal{かいめん}}{\text{海面}}}$ には、 ${\overset{\textnormal{さかな}}{\text{魚}}}$ が ${\overset{\textnormal{は}}{\text{跳}}}$ ねています。 \hfill\break
Fish are leaping up from the sea. }

\par{11. あの ${\overset{\textnormal{くるま}}{\text{車}}}$ は、 ${\overset{\textnormal{みずたま}}{\text{水溜}}}$ りの ${\overset{\textnormal{どろ}}{\text{泥}}}$ を ${\overset{\textnormal{ほこうしゃ}}{\text{歩行者}}}$ に\{ ${\overset{\textnormal{は}}{\text{跳}}}$ ねた・ ${\overset{\textnormal{は}}{\text{撥}}}$ ねた\}。 \hfill\break
That car splashed mud from the muddle over the pedestrian(s). }

\par{12. ${\overset{\textnormal{どろ}}{\text{泥}}}$ が ${\overset{\textnormal{わたし}}{\text{私}}}$ の ${\overset{\textnormal{は}}{\text{晴}}}$ れ ${\overset{\textnormal{ぎ}}{\text{着}}}$ に ${\overset{\textnormal{は}}{\text{跳}}}$ ねてしまった。 \hfill\break
Mud splashed onto my best clothes. }

\par{13. ${\overset{\textnormal{くるま}}{\text{車}}}$ に ${\overset{\textnormal{みず}}{\text{水}}}$ をはねられて ${\overset{\textnormal{ぬ}}{\text{濡}}}$ れてしまった。 \hfill\break
I got wet from being splattered with water by a car. }

\par{14. ${\overset{\textnormal{そうこうちゅう}}{\text{走行中}}}$ に ${\overset{\textnormal{まるた}}{\text{丸太}}}$ のようなものをはねてしまった。 \hfill\break
I ran over a log of some sort while driving. }

\par{15. ${\overset{\textnormal{けいびいん}}{\text{警備員}}}$ の ${\overset{\textnormal{だんせい}}{\text{男性}}}$ がはねられて ${\overset{\textnormal{しぼう}}{\text{死亡}}}$ しました。 \hfill\break
A male security officer passed away from being ran over. }

\par{16. ${\overset{\textnormal{けんさ}}{\text{検査}}}$ で ${\overset{\textnormal{ふりょう}}{\text{不良}}}$ 品をはね(のけ)る。 \hfill\break
To exclude defective products in inspection. }

\par{\textbf{Spelling Note }: \emph{Hanenokeru }may be spelled as 撥ね除ける. }

\par{ In addition to the meanings mentioned above, the transitive 撥ねる may also be used to mean “to point up\slash add a hook.” This is typically in reference to things like mustaches or the hooks on characters. }

\par{17. 「干」という ${\overset{\textnormal{かんじ}}{\text{漢字}}}$ を ${\overset{\textnormal{は}}{\text{撥}}}$ ねて ${\overset{\textnormal{か}}{\text{書}}}$ くと、「于」という ${\overset{\textnormal{べつじ}}{\text{別字}}}$ になります。 \hfill\break
The Kanji "干" when written with a hook becomes “于,” a separate character. }

\par{ The transitive 撥ねる also has the meaning “to make nasal.” This is in reference to sound changes in Japanese that result in sounds being turned into ん. }

\par{18. 「 ${\overset{\textnormal{し}}{\text{死}}}$ にて」は「 ${\overset{\textnormal{し}}{\text{死}}}$ んで」と ${\overset{\textnormal{は}}{\text{撥}}}$ ねます。 \hfill\break
We nasalize “shinite” as “shinde.” }

\par{ The transitive \emph{haneru }also has the meaning of “to behead.” Although typically spelled as はねる, its traditional spelling is 刎ねる. }

\par{19. ${\overset{\textnormal{くび}}{\text{首}}}$ を ${\overset{\textnormal{は}}{\text{刎}}}$ ねろ! \hfill\break
Behead him! }

\par{\textbf{Word Note }: \emph{Kubi }may refer to the head along with the neck. This comes from the fact that the neck is the point of severing in a beheading. Historically, 頸 should be the character for neck because 首 refers to the head in Chinese. In anatomy, the head is often referred to as 頭部(とうぶ) while the neck is referred to as 頸部(けいぶ). }

\begin{center}
\textbf{はだける }
\end{center}

\par{\emph{ }開ける is unique in that it traditionally creates an intransitive\slash transitive verb pair with 開かる. From appearances alone, 開かる should be the intransitive form and 開ける should be the transitive form, but now, 開ける  can be used as both to mean “to open (one\textquotesingle s clothes) to expose (one\textquotesingle s chest).” Although not limited to the chest, it can be used to indicate clothing no longer covering some part of the body. }

\par{20. ${\overset{\textnormal{あし}}{\text{足}}}$ を ${\overset{\textnormal{うご}}{\text{動}}}$ かしても、 ${\overset{\textnormal{すそ}}{\text{裾}}}$ が\{ ${\overset{\textnormal{はだ}}{\text{開}}}$ ける・ ${\overset{\textnormal{はだ}}{\text{開}}}$ かる\} ${\overset{\textnormal{しんぱい}}{\text{心配}}}$ などありません。 \hfill\break
Even if you move your legs, there\textquotesingle s no worry of your cuffs being exposed. }

\par{21. ${\overset{\textnormal{ちゃくよう}}{\text{着用}}}$ が ${\overset{\textnormal{こんなん}}{\text{困難}}}$ で、 ${\overset{\textnormal{むね}}{\text{胸}}}$ が ${\overset{\textnormal{はだ}}{\text{開}}}$ ける ${\overset{\textnormal{おそ}}{\text{恐}}}$ れもある。 \hfill\break
Wearing is difficult, and there is also the fear of your chest becoming exposed. }

\par{22. ${\overset{\textnormal{かれ}}{\text{彼}}}$ はシャツのボタンを ${\overset{\textnormal{はず}}{\text{外}}}$ し、 ${\overset{\textnormal{きんにくしつ}}{\text{筋肉質}}}$ の ${\overset{\textnormal{むね}}{\text{胸}}}$ を ${\overset{\textnormal{はだ}}{\text{開}}}$ けた。 \hfill\break
He undid the buttons of his shirt and exposed his muscular chest. }

\par{ One meaning that \emph{ }開かる doesn\textquotesingle t share with 開ける  is “to obstruct\slash block (the way),” and in this sense, it is usually seen in the compound verb 立ちはだかる. }

\par{23. ${\overset{\textnormal{め}}{\text{目}}}$ の ${\overset{\textnormal{まえ}}{\text{前}}}$ に ${\overset{\textnormal{おお}}{\text{大}}}$ きな ${\overset{\textnormal{かべ}}{\text{壁}}}$ が ${\overset{\textnormal{た}}{\text{立}}}$ ちはだかっている。 \hfill\break
A large wall stands in the way in front of my eyes. }

\begin{center}
\textbf{生じる }
\end{center}

\par{ The verb 生じる means “to happen\slash occur\slash germinate.” For the most part, it is usually used as an intransitive verb. However, it can technically also be used as a transitive verb. This is possible when the subject of the verb can be viewed as the agent. Yet, many speakers don\textquotesingle t like the verb being used as a transitive verb if it\textquotesingle s not used in the causative form 生じさせる. This is why, as the example sentences demonstrate, there will always be a way to phrase out the transitive 生じる. }

\par{24. ${\overset{\textnormal{ひょうめんちか}}{\text{表面近}}}$ くの ${\overset{\textnormal{さいぼう}}{\text{細胞}}}$ から ${\overset{\textnormal{め}}{\text{芽}}}$ が\{ ${\overset{\textnormal{で}}{\text{出}}}$ る・ ${\overset{\textnormal{しょう}}{\text{生}}}$ じる\}ことが分かりました。 \hfill\break
We discovered that buds sprout from the cells close to the surface. }

\par{25. その ${\overset{\textnormal{ちが}}{\text{違}}}$ いによって、 ${\overset{\textnormal{ぼうえき}}{\text{貿易}}}$ から ${\overset{\textnormal{りえき}}{\text{利益}}}$ が ${\overset{\textnormal{しょう}}{\text{生}}}$ じる。 \hfill\break
Based on that difference, profit results from trade. }

\par{26a. ${\overset{\textnormal{とうふ}}{\text{豆腐}}}$ にカビが\{ ${\overset{\textnormal{は}}{\text{生}}}$ えた・ ${\overset{\textnormal{しょう}}{\text{生}}}$ じた\}。 \hfill\break
26b. ${\overset{\textnormal{とうふ}}{\text{豆腐}}}$ がカビを ${\overset{\textnormal{しょう}}{\text{生}}}$ じた。 \hfill\break
Mold grew on the tofu. }

\par{\textbf{Spelling Note }: \emph{Kabi }may also be spelled as 黴. }

\par{27. ${\overset{\textnormal{ふりえき}}{\text{不利益}}}$ を\{ ${\overset{\textnormal{かぶ}}{\text{被}}}$ る・ ${\overset{\textnormal{しょう}}{\text{生}}}$ じる\} ${\overset{\textnormal{かのうせい}}{\text{可能性}}}$ が ${\overset{\textnormal{たか}}{\text{高}}}$ い。 \hfill\break
There is a high probability of suffering a loss. }

\par{28. ${\overset{\textnormal{たしょう}}{\text{多少}}}$ の ${\overset{\textnormal{こんらん}}{\text{混乱}}}$ を\{ ${\overset{\textnormal{まね}}{\text{招}}}$ く・ ${\overset{\textnormal{しょう}}{\text{生}}}$ じる\} ${\overset{\textnormal{ことば}}{\text{言葉}}}$ の ${\overset{\textnormal{ひと}}{\text{一}}}$ つです。 \hfill\break
This is one (of several) words that causes some confusion. }

\par{29. ${\overset{\textnormal{こきゅう}}{\text{呼吸}}}$ に ${\overset{\textnormal{もんだい}}{\text{問題}}}$ \{が・を\} ${\overset{\textnormal{しょう}}{\text{生}}}$ じる ${\overset{\textnormal{しっかん}}{\text{疾患}}}$ では、 ${\overset{\textnormal{こきゅうしょうがい}}{\text{呼吸障害}}}$ だけが ${\overset{\textnormal{もんだい}}{\text{問題}}}$ になることは ${\overset{\textnormal{すく}}{\text{少}}}$ ない。 \hfill\break
In ailments that cause problems in one\textquotesingle s respiration, there are few instances in which respiratory impairment is the only problem at hand. }

\par{30. ${\overset{\textnormal{ふくさよう}}{\text{副作用}}}$ を ${\overset{\textnormal{しょう}}{\text{生}}}$ じ(させ)ることなく ${\overset{\textnormal{りょうこう}}{\text{良好}}}$ な ${\overset{\textnormal{すいみん}}{\text{睡眠}}}$ を ${\overset{\textnormal{え}}{\text{得}}}$ ることができます。 \hfill\break
We will be ale to get satisfactory sleep without causing any side effects. }

\par{31. ${\overset{\textnormal{めんえききのう}}{\text{免疫機能}}}$ に ${\overset{\textnormal{ししょう}}{\text{支障}}}$ を\{ ${\overset{\textnormal{き}}{\text{来}}}$ たす・ ${\overset{\textnormal{しょう}}{\text{生}}}$ じ(させ)る\} ${\overset{\textnormal{じゅうとく}}{\text{重篤}}}$ な ${\overset{\textnormal{しっぺい}}{\text{疾病}}}$ に ${\overset{\textnormal{かか}}{\text{罹}}}$ ってしまう。 \hfill\break
To suffer from a severe illness that creates an impediment to one\textquotesingle s immune system. }

\par{\textbf{Word Notes }: There are several words for “illness.” Of these include 病気, 病い, 疾病, 疾患, and 患い. }

\par{疾病 is a clinical terminology for “illness.” 疾患 refers to ailments that bring about physical and or mental symptoms. 病気 is the more general term for “illness” used most commonly in the spoken language and in more subjective situations. \emph{ }病い is the native word for “sickness,” but it takes on a personal tone to an ailment. Whereas 疾患 can refer to a medical state of function failure, 胸の病い would refer to personal suffering in the chest. The native equivalent of 疾患 is 患い and is even more emphatic than 病い, but it is more so used to refer to suffering of the heart. However, it is rarely used outside of literature. }

\begin{center}
\textbf{踊る }
\end{center}

\par{\emph{ }踊る can be used to mean “to dance” in an intransitive or transitive sense. When used to mean “to pound\slash throb\slash jump,” it\textquotesingle s spelled as 躍る. }

\par{32. ワルツを ${\overset{\textnormal{おど}}{\text{踊}}}$ りましょう。 \hfill\break
Let\textquotesingle s dance the waltz. }

\par{33. ${\overset{\textnormal{こころ}}{\text{心}}}$ が ${\overset{\textnormal{おど}}{\text{躍}}}$ っている。 \hfill\break
My heart is throbbing. }

\begin{center}
\textbf{寄せる }
\end{center}

\par{ As an intransitive verb, 寄せる means “to surge (as in waves).” As a transitive verb, it means “to come\slash bring near.” }

\par{34. ${\overset{\textnormal{おき}}{\text{沖}}}$ に ${\overset{\textnormal{なみ}}{\text{波}}}$ が ${\overset{\textnormal{よ}}{\text{寄}}}$ せている。 \hfill\break
Waves are surging in the open sea. }

\par{35. ${\overset{\textnormal{かれ}}{\text{彼}}}$ は ${\overset{\textnormal{みみもと}}{\text{耳元}}}$ に ${\overset{\textnormal{くち}}{\text{口}}}$ を ${\overset{\textnormal{よ}}{\text{寄}}}$ せてそっと ${\overset{\textnormal{ささや}}{\text{囁}}}$ いた。 \hfill\break
He brought his mouth near to my ears and softly whispered. }

\par{36. いつも ${\overset{\textnormal{みけん}}{\text{眉間}}}$ に ${\overset{\textnormal{しわ}}{\text{皺}}}$ を ${\overset{\textnormal{よ}}{\text{寄}}}$ せている ${\overset{\textnormal{ひと}}{\text{人}}}$ といつも ${\overset{\textnormal{えがお}}{\text{笑顔}}}$ の ${\overset{\textnormal{ひと}}{\text{人}}}$ はどちらが ${\overset{\textnormal{す}}{\text{好}}}$ きですか。 \hfill\break
Which do you like, people who are always furrowing their brows or people who always have a smile on their face? }

\par{37. ${\overset{\textnormal{きょう}}{\text{今日}}}$ も、マムシたちが ${\overset{\textnormal{くさやぶ}}{\text{草藪}}}$ に ${\overset{\textnormal{み}}{\text{身}}}$ を ${\overset{\textnormal{よ}}{\text{寄}}}$ せていた。 \hfill\break
The pit vipers were living under the clump of bushes today as well. }

\par{\textbf{Spelling Note }: \emph{Mamushi }may also be spelled as 蝮. }

\begin{center}
\textbf{誤る }
\end{center}

\par{ Traditionally, 誤る was the intransitive version of 過つ, both revolving around expressing failure\slash mistake. Nowadays, 過つ is hardly used aside from its noun form 過ち (fault\slash indiscretion), and 誤る exists both as an intransitive and a transitive verb, but mostly a transitive verb aside from when 誤った is used similarly to 間違った (mistaken) before nouns. }

\par{38. どこで ${\overset{\textnormal{みち}}{\text{道}}}$ を ${\overset{\textnormal{あやま}}{\text{誤}}}$ ってしまったのだろうか。 \hfill\break
Where have I gone wrong? }

\par{39. ${\overset{\textnormal{そうさ}}{\text{操作}}}$ を ${\overset{\textnormal{あやま}}{\text{誤}}}$ って ${\overset{\textnormal{あいてさき}}{\text{相手先}}}$ の ${\overset{\textnormal{でんわばんごう}}{\text{電話番号}}}$ を ${\overset{\textnormal{ひと}}{\text{一}}}$ つ ${\overset{\textnormal{さくじょ}}{\text{削除}}}$ してしまった。 \hfill\break
I made a mistake in handling (my phone) and accidentally deleted one of my contact\textquotesingle s phone numbers. }

\par{40. あなたは ${\overset{\textnormal{はり}}{\text{鍼}}}$ について ${\overset{\textnormal{あやま}}{\text{誤}}}$ った ${\overset{\textnormal{にんしき}}{\text{認識}}}$ をしていませんか。 \hfill\break
Do you not have a mistaken perception about acupuncture? }

\begin{center}
\textbf{笑う }
\end{center}

\par{ As an intransitive verb, 笑う means “to laugh,” but as a transitive verb it means “to laugh at\slash make fun of.” As a transitive verb, it can alternatively be spelled as 嗤う. }

\par{41. ${\overset{\textnormal{いぬ}}{\text{犬}}}$ も ${\overset{\textnormal{わら}}{\text{笑}}}$ うんでしょうか。 \hfill\break
Do dogs also laugh? }

\par{42. ${\overset{\textnormal{かれ}}{\text{彼}}}$ の ${\overset{\textnormal{うれ}}{\text{嬉}}}$ しそうに ${\overset{\textnormal{わら}}{\text{笑}}}$ っている ${\overset{\textnormal{すがた}}{\text{姿}}}$ を ${\overset{\textnormal{そうぞう}}{\text{想像}}}$ してみた。 \hfill\break
I tried imagining the look of him happily laughing. }

\par{43. ${\overset{\textnormal{いちえん}}{\text{一円}}}$ を\{ ${\overset{\textnormal{わら}}{\text{笑}}}$ う・ ${\overset{\textnormal{わら}}{\text{嗤}}}$ う\} ${\overset{\textnormal{もの}}{\text{者}}}$ は ${\overset{\textnormal{いちえん}}{\text{一円}}}$ に ${\overset{\textnormal{な}}{\text{泣}}}$ く。 \hfill\break
He who makes fun of one yen will cry at one yen. }

\par{44. ${\overset{\textnormal{めくそはなくそ}}{\text{目糞鼻糞}}}$ を\{ ${\overset{\textnormal{わら}}{\text{笑}}}$ う・ ${\overset{\textnormal{わら}}{\text{嗤}}}$ う\}。 \hfill\break
The pot calls the kettle black. }

\par{45. ${\overset{\textnormal{なぜわら}}{\text{何故笑}}}$ ってはいけない ${\overset{\textnormal{ばめん}}{\text{場面}}}$ で ${\overset{\textnormal{わら}}{\text{笑}}}$ ってしまうんだろうか。 \hfill\break
We do (I\slash we) laugh in scenes where we ought not to laugh? }

\begin{center}
\textbf{つとめる }
\end{center}

\par{\emph{ Tsutomeru }has both intransitive and transitive nuances. They are conveniently spelled differently. }

\par{Intransitive Nuances: 勤める, 努める \hfill\break
Transitive Nuance: 務める }

\par{46. ${\overset{\textnormal{おおてがいしゃ}}{\text{大手会社}}}$ に ${\overset{\textnormal{つと}}{\text{勤}}}$ めています。 \hfill\break
I work at a major company. }

\par{47. ${\overset{\textnormal{じつげん}}{\text{実現}}}$ に ${\overset{\textnormal{つと}}{\text{努}}}$ めています。 \hfill\break
I\textquotesingle m striving to realize it. }

\par{48. ${\overset{\textnormal{だいりにん}}{\text{代理人}}}$ を ${\overset{\textnormal{つと}}{\text{務}}}$ めています。 \hfill\break
I\textquotesingle m serving as a proxy\slash agent\slash representative. }

\par{ There is also an intransitive 勤まる・務まる, which is used to mean “to be fit for (job\slash post).” In the case of a typical job, the former spelling is used. In the case of a typical post, the latter spelling is used. }

\par{49. 私に務まるだろうか。 \hfill\break
Am I even fit (for the post)? }

\par{50. とても ${\overset{\textnormal{つと}}{\text{勤}}}$ まりそうもない。 \hfill\break
I\textquotesingle m far from fit (for the job). }
    