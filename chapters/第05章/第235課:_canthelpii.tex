    
\chapter{Can't Help II}

\begin{center}
\begin{Large}
第235課: Can't Help II: ~てならない, ~てやまない, ~てしかたがない, ~にたえない, \& ~てたまらない 
\end{Large}
\end{center}
 
\par{ This lesson will continue coverage on phrases regarding "not being able to restrain" one's emotions or feeling. The phrases here with exception to ~にたえない, which is brought up because of ~てこたらない, all use the て form to attach to verbs\slash adjectives, but be careful about what kind of words they can semantically be used with. }
      
\section{~てならない}
 
\par{ ~てならない is found primarily in the written language, though this doesn't mean you can\textquotesingle t hear it be used in the spoken language, and its purpose is to show that a certain feeling can\textquotesingle t be helped being felt\slash thought. If the verb, though, does not refer to emotion or something spontaneous, it cannot be used. This pattern, also, does not attach to the negative form. }

\par{${\overset{\textnormal{}}{\text{1. 不思議}}}$ に思えてならない。 \hfill\break
I can't help but think it\textquotesingle s strange. }

\par{2. 悪化するように思えてならない。 \hfill\break
I can't help but think that it\textquotesingle s going to get worse. }

\par{3. もう10年も ${\overset{\textnormal{ふるさと}}{\text{故郷}}}$ に帰っていないので、両親に会いたくてならない。 \hfill\break
Since it's already been 10 years since I haven't been home at my hometown, I can\textquotesingle t help but want to meet my parents. }

\par{\textbf{漢字 Note }: 故郷 may also be read as こきょう. }

\par{4. ${\overset{\textnormal{ひじ}}{\text{肘}}}$ が痛くてならない。 \hfill\break
I can't help but think about my hurting elbow. }

\par{5. ${\overset{\textnormal{たいくつ}}{\text{退屈}}}$ な日本語の授業に出ていると、眠くてならない。 \hfill\break
I can't help but sleep when I'm at my boring Japanese class. }

\par{6. 毎日 ${\overset{\textnormal{さび}}{\text{寂}}}$ しくてならない。 \hfill\break
I can\textquotesingle t help but be sad every day. }

\par{7. 日本語能力試験が心配でならない。 \hfill\break
I can't help but worry about the JLPT. }

\par{8. 失敗するような気がしてなりません。 \hfill\break
I can't help but have the feeling that I'm going to fail. }

\par{\textbf{Warning Note }: Do not confuse this with the must\slash must not pattern ~てはならない. }
      
\section{~てやまない}
 
\par{ ~てやまない is a somewhat ${\overset{\textnormal{かた}}{\text{硬}}}$ い phrase that is used to show that one will continue to be holding a strong feeling. You just won\textquotesingle t stop. It is used with verbs concerning emotion, but it is not used with emotional verbs that describe temporary states. The grammatical person is usually first person. Understandably, it is not appropriate to attach this phrase to the negative form. }

\par{9. この写真に ${\overset{\textnormal{うつ}}{\text{写}}}$ っているのは私が愛してやまない犬だ。 \hfill\break
The thing in this picture is a dog that I shall always love. }

\par{10. 親は子供の将来を期待してやまないものだ。 \hfill\break
Parents never stop hoping for their children;s future. }

\par{11. 愛犬が死んで悲しくてやまない。 \hfill\break
I will continue to be sad over my beloved dog dying. }

\par{12. 皆さんの幸せを願ってやみません。 \hfill\break
I will continue to wish for everyone's happiness. }

\par{13. ご ${\overset{\textnormal{かつやく}}{\text{活躍}}}$ を願ってやまない。 \hfill\break
I can't stop wishing for you to flourish. }
      
\section{~てしかたがない}
 
\par{ ~てしかたがない is very similar to ~てならない, but since it originally referred to there not being a way to withstand or conquer, it gives off a feeling that one can\textquotesingle t stand something while thinking at the same time that there is nothing else that could be done. Therefore, it is also very common in the spoken language. This is unlike ~てならない which is often felt to be quite old-fashioned. }

\par{ ~てならない and ~てしかたがない are different when the verb doesn\textquotesingle t express something spontaneous or emotional, in which case you use ~てしかたがない because nothing is predicated in its definition that it must be used with a verb of emotion. However, this does not mean that ~てしかたがない can\textquotesingle t be used with verbs of spontaneity and or emotion because it still can. }

\par{\textbf{Variant Note }: ~しようがない and ~てしょうがない are more casual versions with the latter being the most casual. }

\begin{center}
\textbf{Examples }
\end{center}

\par{14. もう残念でしかたがない。 \hfill\break
It\textquotesingle s no use that it\textquotesingle s already regrettable. }

\par{15. いくら着込んでも寒くてしょうねーな。 \hfill\break
No matter how much I wear, I\textquotesingle m still cold! }

\par{16. 彼が言うと、僕を非難しているように聞こえて\{ならない・しかたがない\}。 \hfill\break
Whenever he says (something), I can't help but hear it that he's criticizing me. }

\par{17. 朝っぱらからくそ犬めが鳴くんで、うるさくしょうがねーよ。(Vulgar) \hfill\break
Since the damn dog barks from early in the morning, it's annoying as hell. }

\par{18. ${\overset{\textnormal{ちゅうせん}}{\text{抽選}}}$ にもれたなんて、残念でしかたないね。 \hfill\break
It can't be helped that I'm disappointed that I didn't get drawn. }
      
\section{~にたえる・~にたえない}
 
\par{ 耐える means " to endure". ~に耐える is seen after nouns and verbs of personal attention to show the worth of something. With する verbs, する is dropped. For example, it is used in contexts such as being worth to read, hear, criticize, applaud, value, etc. }

\par{ ~に耐えない, though, has two usages. When used with the 連体形, it means "cannot stand\slash endure to\dothyp{}\dothyp{}\dothyp{}". When used \textbf{after a noun }, it shows a strong emotional reaction in which one can\textquotesingle t hold back such emotions. This may sound very similar to the first usage, but this usage is strictly used with nouns. }

\par{\textbf{Spelling Note }: The phrase can also be spelled as に堪えない. }

\par{\textbf{Style\slash Word Choice Note }: It is somewhat of a stiff phrase and used with a limited number of nouns such as 感謝, 感激, ${\overset{\textnormal{ざんき}}{\text{慙愧}}}$ (shame), 等. }

\begin{center}
\textbf{Examples } 
\end{center}

\par{19. 感謝に耐えません。 \hfill\break
I can't help but thank you. }

\par{20. 見るに耐えないOLだぞ。 \hfill\break
She's a butt-ugly office lady. }

\par{21. ${\overset{\textnormal{ざんき}}{\text{慙愧}}}$ (の念)に堪えない。 \hfill\break
To be deeply ashamed of oneself. }

\par{22. お忙しいところを多くの方にお集まりいただき、感激に堪えません。(謙譲語) \hfill\break
I am overwhelmed with emotion from receiving all of you gathering while you are all busy. }

\par{23. 気の毒で同情に堪えない。 \hfill\break
I am deeply sorry for you and sympathize with you. }

\par{24. ${\overset{\textnormal{まこと}}{\text{誠}}}$ に ${\overset{\textnormal{がいたん}}{\text{慨嘆}}}$ に堪えない。 \hfill\break
It is truly a matter for great regret. }

\par{25. 父は聞くに耐えない ${\overset{\textnormal{た}}{\text{歌}}}$ いぶりをする。 \hfill\break
My dad has an awful manner of singing. }

\par{26. 鑑賞に耐えるものだ。 \hfill\break
It is something worth appreciating. }

\par{\textbf{Word Note }: Again, this phrase is used with limited words such as 見る、聞く、批判、 ${\overset{\textnormal{かんしょう}}{\text{鑑賞}}}$ 、議論、etc. }
      
\section{~てたまらない}
 
\par{ Although usually left in かな, if you do see this pattern written in 漢字, don\textquotesingle t confuse the reading for the one in ~(に)堪えない. They're not the same, of course. This is a very common phrase in the spoken language, unlike a lot of the phrases in this lesson, and it is a strong speech style meant to show that one can\textquotesingle t resist a certain emotion, sense, or want. }

\par{ ~てたまらない can't be used with spontaneous verbs. This is contrast with things like ~てならない and ~てしかたがない. On the other hand, if you are using an adjective that shows an objective degree, ~てたまらない can be used but ~てならない・~てしかたがない can\textquotesingle t. }

\begin{center}
\textbf{Examples } 
\end{center}

\par{26. 日本へ帰りたくてたまらない。 \hfill\break
I can't stand but want to go back home to Japan. }

\par{27. 隣の車のマフラー音がうるさくて、たまらないよ。 \hfill\break
The muffler of the car next to me is annoying, and I can't stand it. }

\par{28. 隣の駐車場の車の音がうるさくて、たまりません。困ってますよ。 \hfill\break
The car noise from the parking lot next to me is annoying, and I can't stand it. I\textquotesingle m at a loss. }

\par{29. お兄さんは毎日退屈で堪らないそうです。 \hfill\break
I hear that your older brother can\textquotesingle t stand him being bored every day? }
    