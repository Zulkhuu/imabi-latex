    
\chapter{As (change)}

\begin{center}
\begin{Large}
第250課: As (change): ~につれて, ~に従って, ~に伴って, ~とともに, ~に応じて, \& ~に応えて 
\end{Large}
\end{center}
 
\par{ ~につれて, ~に従って, ~に伴って, and ~とともに all have meanings of showing the transition\slash change of one situation as another transition\slash change is under way. We will also see how ~に応じて・応えてrelate to them. }
      
\section{~につれて}
 
\par{ ~につれて is either after a noun or a verb in the 連体形 to express "as one thing changes, another thing changes as well". These changes are simultaneous. Although it can be after nouns or verbs, it is after verbs more often. ~につれて is often used in situations where there is a proportional relationship with the two events in the respective clauses. }

\par{ As such, the first clause acts as the trigger, proportionately raising the degree of the change\slash transition of the latter clause. Common translations include "as", "together with", and "in accordance to\slash with". }

\par{\textbf{Variant Note }: It can be seen in more formal\slash written language as ~につれ. }

\begin{center}
\textbf{Examples } 
\end{center}

\par{1. 歌は世につれ世は歌につれ。 \hfill\break
Song changes by generation, and generations change by song. }

\par{2. 昭和から平成へと時代が変わるにつれて ${\overset{\textnormal{ぐんか}}{\text{軍歌}}}$ は歌われなくなりました。 \hfill\break
With the changing of periods from the Showa to the Heisei, military songs became unsung. }

\par{3. 時間が経つにつれて忘れっぽくなる。 \hfill\break
As time passes, I become forgetful. }

\par{4. 都心化(が進む)につれて自分の地域への関心が ${\overset{\textnormal{うす}}{\text{薄}}}$ れる。 \hfill\break
As urbanization progresses, the concern towards one's region fades. }

\par{5. 夜が ${\overset{\textnormal{ふ}}{\text{更}}}$ けるにつれて ${\overset{\textnormal{あらし}}{\text{嵐}}}$ は激しくなった。 \hfill\break
The storm became fierce as the night grew late. }

\par{6. 季節の変化につれて気温が ${\overset{\textnormal{}}{\text{変わる。}}}$  \hfill\break
The temperatures change in accordance to the seasons. }
7. 冬になるにつれて、山が白くなってくる。 \hfill\break
The mountains become white as winter depends. 
\par{8. イナゴは稲に付く害虫なので駆除しなくてはなりません。昔の人はそこからイナゴを食べることを思いつきました。当時は、食べることを前提に捕まえていたので、殺虫剤は使っていませんでした。しかし戦後、 農薬が ${\overset{\textnormal{}}{\text{普及し、それにつ}}}$ れてイナゴは ${\overset{\textnormal{}}{\text{激減}}}$ してしまいました。現在では、殺虫剤を使ったイナゴの駆除も行われていますが、安全な無農薬栽培やオーガニック栽培をしている農家では一匹ずつ捕まえているそうです。大変な作業ですね。 \hfill\break
Locusts are harmful insects which stick to rice plants and have to be exterminated. People in the past thought of eating locusts from this. At the time, they wouldn't use pesticides with eating them as a premise. However, after the war, agrochemicals spread, and with this locusts were decreased in number. Presently, there is extermination involving pesticides, but at farms which use safe non-agriculture chemical and organic cultivation, locusts are caught by hand one by one. That's definitely a tough job. }
      
\section{~に従って}
 
\par{ ~に従って can be used in a hierarchical sense to mean "according to". As should be expected, ~に従い is more formal and indicative of the written language. Both に従っての and に従った are appropriate attribute forms. }

\par{9. 命令に従って、行動しろ。 \hfill\break
Act according to the command. }

\par{10. 私の合図に従って行動してください。 \hfill\break
Please act in according to my signals. }

\par{ It may also be used like ~につれて. In the first usage above, it attaches to nouns. As for this usage, it attaches primarily to verbs. It, however, simply states that the two actions described in the respective clauses are happening in parallel. }

\par{11. 南に行くに従い、気温がどんどん高くなる。(Somewhat literary) \hfill\break
The temperature gets higher as you go south. }

\par{12. 人口が増えるにしたがって、住宅問題が起こってくる。 \hfill\break
As the population increases, housing problems will come to occur. }

\par{13. 人口が増えるにしたがって、犯罪が多発する。 \hfill\break
As the population increases, crime will frequently occur. }

\par{ The contrast in nuance between ~につれて and ~にしたがって can definitely be seen in the following example. }

\par{14. スタジアムで場内の歓声が高まる\{につれて・にしたがって\}。実況アナウンサーの声も大きくなっていった。 \hfill\break
As the cheers in the stadium intensified, the commentator\textquotesingle s voice got ever louder. }

\par{ With ~につれて, the suspense is captured with the first clause being the trigger for the events of the second. With ~にしたがって, the sentence doesn't become ungrammatical, but it loses the emotive drive and merely states that the two events described in the respective clauses are happening in parallel. }
      
\section{~に伴って}
 
\par{ 伴う means "to accompany" and ~に伴って means "associated\slash accompanied with" or "as". Just like ~につれて, it may show that something is happening together with a change\slash transition of a certain situation. There is a temporal context of “taking A as the opportunity, therefore…B”.  It is also often the case that it infers that the thing in the second clause continues. }

\par{ What precedes it is either a noun or verb. In the case of a verb, unlike the other options, the particle の may be optionally used after the verb for nominalization. To be clear, this is not required and is not done in the first place for the other options. Lastly, ~に伴って is more literary. }

\par{\textbf{Variant Note }: It may also be seen in writing as ~に伴い. }

\par{15. ${\overset{\textnormal{おせん}}{\text{汚染}}}$ に伴う問題について話しましょう。 \hfill\break
Let's talk about the problems accompanied with pollution. }

\par{16. 人口の増加に伴って、たくさんの住宅が建てられる。 \hfill\break
Many houses are constructed in association to the increase in population. }

\par{\textbf{Particle Note }: 伴う is used with を when used transitively. }
      
\section{~とともに}
 
\par{ ~とともに also behaves like ~につれて to show that something happens together with a change\slash transition of a certain situation. However, unlike ~につれて, the entire sentence doesn\textquotesingle t necessarily have to show a state of progress. In other cases it can have meanings equivalent to 同時に and 一緒に. As for its usage similar to ~につれて, it\textquotesingle s also the case that the words it follows must show change in situation. It cannot be words pertaining to action, which ~につれ and ~にしたがって can be. }

\par{17. 人口の増加とともに、犯罪が多発する。 \hfill\break
Along with the increase in population, crimes will frequently occur. }

\par{18. 子供が卒業するとともに、 ${\overset{\textnormal{ふぼかい}}{\text{父母会}}}$ も ${\overset{\textnormal{かいさん}}{\text{解散}}}$ しました。 \hfill\break
The parents' association also dissolved at the same time the kids graduated. }

\par{19. 皆さんとともに ${\overset{\textnormal{かいてき}}{\text{快適}}}$ な道づくりを進めます。 \hfill\break
With everyone we will go forward with a pleasant road construction. }

\par{\textbf{漢字 Note }: This pattern may also be written in 漢字 as ~と共に. }
      
\section{~に応じて}
 
\par{ ~に応じて means "accordingly" or "as to". The first meaning is important in this lesson as it ties with the other patterns. When it is used with a verb that shows change, it describes the corresponding\slash dealing with a certain change. In this instance, it is possible to switch it out with ~につれて, ~にしたがって, ~にともなって, and ~とともに. }

\par{ In its other meaning, it is equivalent to ~に応えて (as to\slash to satisfy). Although this, too, sounds like it is of the vein of the other phrases, it\textquotesingle s not. Although in such situations it may seem that ~に従って is possible, it would be considerably more negative. }

\par{20a. 住民の要求に応じて、説明会を開くことになりました。〇 \hfill\break
20b. 住民の要求に従って、説明会を開くことになりました。△ \hfill\break
20c. 住民の要求\{にともなって・とともに\}、説明会を開くことになりました。? \hfill\break
20d. 住民の要求につれて、説明会を開くことになりました。X \hfill\break
In meeting the wants of the citizens, it has been decided that we open an information seminar. }

\par{ It has the attribute forms ~に応じての and ~に応じた. When に応じて comes before a noun, の is necessary after 応じて. Or, you can use に応じた. The 一段 conjugating verb 応じる either means "reply" or "to comply". }

\par{21. 質問に応じて、彼女は問題の解答を ${\overset{\textnormal{くわ}}{\text{詳}}}$ しく説明しました。 \hfill\break
As to answer the question, she explained the answer to the problem well. }

\par{22. 命令に応じた服従。 \hfill\break
Obedience in accordance to command. }

\par{23. 料金が ${\overset{\textnormal{きょり}}{\text{距離}}}$ と時間に応じて決まります。 \hfill\break
The fee is determined according to distance and time. }

\par{24. 「うん」と応じる。 \hfill\break
To respond with a "yes". }

\par{25. 相手の ${\overset{\textnormal{いらい}}{\text{依頼}}}$ に応じましたか。 \hfill\break
Have you responded to your partner's request? }

\par{26. 車の速さに応じてガソリンの消費量が変わる。 \hfill\break
Fuel use will change depending upon the car's speed. }
      
\section{~に応えて}
 
\par{ The 一段 verb 応える may be used to mean "to come through\slash satisfy\slash answer" or "to tell\slash go to one's heart". The latter usage shows some sort of burden. }

\par{27. 要求に ${\overset{\textnormal{こた}}{\text{応}}}$ えて、社員の給料が上がりました。 \hfill\break
To meet demand, the employee's salaries went up. }

\par{28. 国民に応えて、法律は ${\overset{\textnormal{はいし}}{\text{廃止}}}$ されました。 \hfill\break
The law was repealed in response to the citizens. }

\par{29. 期待に応える。 \hfill\break
To come through with expectations. }

\par{30. ${\overset{\textnormal{ふかざけ}}{\text{深酒}}}$ は身体に応えるのです。 \hfill\break
Hard drinking takes its toll on the body. }

\par{31. その当時供給は需要に\{応えて・満たして\}いなかった。 \hfill\break
In those days, the supply didn't meet the demand. }
    