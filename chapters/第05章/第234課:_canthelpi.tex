    
\chapter{Can't Help I}

\begin{center}
\begin{Large}
第234課: Can't Help I: ~ないではいられない 
\end{Large}
\end{center}
 
\par{ Double negative expressions seem to cause confusion for Japanese learners. Especially if these learners come from language backgrounds that include languages where these kind of expressions still yield negative meaning, it\textquotesingle s not surprising that phrases like ~ないではいられない would cause problems. }

\par{ It\textquotesingle s also not helpful that some textbooks call certain adverbs “negative” when it\textquotesingle s just a matter of semantics. In this lesson, you will learn about ~ないではいられない by knowing not only what it means, but also what it is similar to and how it is different. }
      
\section{~ないではいられない}
 
\par{ At a basic understanding, ~ないではいられない means "cannot help but\dothyp{}\dothyp{}\dothyp{}". This translation, however, makes it seem a whole lot like ~てしまう.It also looks vaguely similar to ~なければならない and happens to also mean something similar. Consider the following errors. }

\par{1a. 期末試験があるので、今晩勉強しないではいられない。X \hfill\break
1b. 期末試験があるので、今晩勉強しなければならない。〇 \hfill\break
Intended: Since I have a final exam, I have to study tonight. }

\par{2a. その赤ちゃんの笑顔を見ると、笑わないではいられなくなる。X \hfill\break
2b. その赤ちゃんの笑顔を見ると、自然と笑ってしまう。〇 \hfill\break
Intended: When(ever) I see that baby\textquotesingle s smiling face, I cannot help but laugh. }

\par{ If you couldn't figure out why the first wording was wrong, we'll reexamine them later as to why they\textquotesingle re wrong after we go through some facts about ~ないではいられない. }

\begin{center}
 \textbf{Conjugation and Variants }
\end{center}

\par{ This pattern can also be seen as ~ずにはいられない, which is more literary. ~ず is an old negative ending. Like other negative endings, it attaches to the 未然形. As it is old, it attaches to the old 未然形 of する, せ-. Thus, you get せずにはいられない, and not しずにはいられない. }

\begin{ltabulary}{|P|P|P|P|}
\hline 

Class & 例 & ~ないではいられない & ~ずにはいられない (書き言葉的) \\ \cline{1-4}

一段 & 感じる & 感じないではいられない & 感じずにはいられない \hfill\break
感ぜずにはいられない (古風) \\ \cline{1-4}

一段 & 食べる & 食べないではいられない & 食べずにはいられない \\ \cline{1-4}

五段 & 泣く & 泣かないではいられない & 泣かずにはいられない \\ \cline{1-4}

五段 & 思う & 思わないではいられない & 思わずにはいられない \\ \cline{1-4}

カ変 & 来る & 来ないではいられない & 来ずにはいられない \\ \cline{1-4}

サ変 & する & しないではいられない & せずにはいられない \\ \cline{1-4}

\end{ltabulary}

\begin{center}
 \textbf{Defining }
\end{center}

\par{ Now, what does ~ずにはいられない really mean? Consider the following. }

\par{\textbf{~ずにはいられない }: This expression shows the speaker's feeling of ending up doing something without being able to restrain one\textquotesingle s willpower. So, it cannot be used with things that are deemed to be spontaneous. The situation must be one where you should have willful control over, but you succumb to something. This, though, doesn't have to always be used in a negative fashion, because it\textquotesingle s often the case that the situation is good. }

\par{3. 会社でいやなことがあって、飲ま\{ないで・ずに\}はいられなかったんだ。 \hfill\break
Bad things happened at the company, and so I couldn't help but drink. }

\par{4. ダイエット中でも食べずにはいられない。 \hfill\break
I can't help but eat even during a diet. }

\par{5. 僕の彼女に勧められれば、買わないではいられないよ。 \hfill\break
If I\textquotesingle m recommended to by my girlfriend, I can't help but buy it. }

\begin{center}
 \textbf{More Notes }
\end{center}

\par{ This phrase, as the last example shows, is often used with conditionals. Remember that when talking about a third person, speech modals like ~だろう and ~ようだ become necessary. This pattern is also frequently used after clauses that establish a reason as for why “one cannot help but…”. Although earlier it was noted that this does not necessarily have to be used in a negative light, it still can be. }

\par{6. その話を聞いたら、いくらやさしい彼女でも怒らずにはいられないだろう。 \hfill\break
Even such a nice person like her wouldn't probably be able to help but be angry if she heard that story. }

\par{7. そのニュースに対して疑問を抱かずにはいられませんでした。 \hfill\break
I couldn't help but hold doubts in regards to that news. }

\par{ This expression is frequently used with verbs of action, emotion, and thought. As this pattern indicates a slipping of willful control to letting go in doing such action, it is frequently used with adverbs such as どうしても, なぜか, つい, etc. }

\par{8. 僕は黙ってたほうがええと思ったが、どうしても一言言わないではいられんかった。(Casual) \hfill\break
I thought that it would be best to just stay quiet, but I couldn't help but say something. }

\par{ This pattern is similar to ~てしまう in that both show a sense of accidentally doing something, but if something is felt to spontaneously\slash physiologically occur, ~てしまう remains grammatical while ~ないではいられない does not. }

\par{9a. 嬉しくて買わないではいられなかった。X \hfill\break
9b. 嬉しくて買ってしまった。 \hfill\break
I was happy and ended up\slash accidentally bought it. }

\par{ Mixing this phrase up with ~なければならない is not acceptable. The literal translation of such a mistake doesn't even make much sense in English, that is unless you\textquotesingle re purposely being humorous, which in that case humor often breaks grammatical rules regardless of language. }

\par{10. 笑わないではいられません。 \hfill\break
I can't help but laugh. }

\par{11. おかしくて笑わないではいられない。 \hfill\break
I couldn't help but laugh because it was funny. }

\par{\textbf{Sentence Note }: Although the first sentence in this lesson with 笑う was marked wrong, with context that implies the willful effort of trying to withhold laughter, this pattern can be used. Remember that with this pattern you can\textquotesingle t restrain oneself from doing something, so there has to be some sense that you gave up trying. }

\begin{center}
 \textbf{More Examples }
\end{center}

\par{12. 会わないではいられなかった。 \hfill\break
I couldn't help but see her. }

\par{13. 泣かないではいられませんでした。 \hfill\break
I couldn't help but cry. }

\par{14. お酒を飲まないではいられない。 \hfill\break
I can't help but drink sake. }

\par{15. サービスが悪いと、一言文句を言わないではいられない。 \hfill\break
When the service is bad, I can\textquotesingle t help but complain a bit.  }

\par{16. 単純な質問を見ると、答えずにはいられないよ。 \hfill\break
When I see a simple question, I can\textquotesingle t help but answer it. }

\par{17. 何かできることを手伝わないではいられなかった。 \hfill\break
I couldn't help but do what I could. }

\par{18. この本を読み始めたら、終りまで読まないではいられませんでした。 \hfill\break
When I started reading this book, I couldn't help but read it until the end. }

\par{19. 私は困っ\{た・ている\}人を見ると、声をかけないではいられません。 \hfill\break
When I see someone in distress, I can\textquotesingle t help but call to them. }

\par{20. 私は困っ\{た・ている\}人を見ると、助けないではいられません。 \hfill\break
When I see someone in distress, I can\textquotesingle t help but help them. }

\par{21. ${\overset{\textnormal{とうせん}}{\text{当選}}}$ したからって ${\overset{\textnormal{よろこ}}{\text{喜}}}$ んではいられない。 \hfill\break
I can\textquotesingle t be so happy just because I got elected. }

\par{\textbf{Grammar Note }: When you do not make this phrase a double negative, you change the translation as seen in the example above to “can\textquotesingle t be..”. }

\par{\textbf{Curriculum Note }: We will return to similar expressions dealing with restraint of various sorts later in IMABI. }

\begin{center}
 \textbf{~ざるを得ない VS ~ないではいられない }
\end{center}

\par{ ~ざるを得ない is very similar, but the speaker has more control in the fact that they feel an obligation to do something, but obligation itself is not all there has to be for someone to actually do something. Thus, the speaker still has some control, howbeit very little. }

\par{22. 酔っ払いに注意しないではいられなかった。 \hfill\break
I couldn't help but pay attention to the drunks. }

\par{23. 酔っ払いに注意せざるを得なかった。 \hfill\break
I had no choice but to pay attention to the drunks. }
    