    
\chapter{以 Adverbial Nouns}

\begin{center}
\begin{Large}
第246課: 以 Adverbial Nouns  
\end{Large}
\end{center}
 
\par{ This lesson is about a handful of important phrases which start with the character 以. Adverbial nouns can follow other nouns without particles and can easily just be used as adverbs and after the て形. }

\begin{ltabulary}{|P|P|P|P|}
\hline 

以外 & 以内 & 以降 & 以後 \\ \cline{1-4}

以前 & 以来 & 以上 & 以下 \\ \cline{1-4}

以遠 & 以往 & 以東 & 以北 \\ \cline{1-4}

以南 & 以西 &  &  \\ \cline{1-4}

\end{ltabulary}
\hfill\break
      
\section{以外}
 
\par{1. Attached to a nominal phrase meaning "except". }

\par{1. 九州と四国以外で地震が発生した。 \hfill\break
With exception of Kyushu and Shikoku, earthquakes had occurred everywhere. }

\par{2. これ以外に方法はありませんよ。 \hfill\break
There is no method other than this. }

\par{3. 従業員以外の方の入室は\{お ${\overset{\textnormal{ことわ}}{\text{断}}}$ りします・ ${\overset{\textnormal{きん}}{\text{禁}}}$ じられています\}。 \hfill\break
Entrance except for employees is prohibited. }

\par{2. 以外の ${\overset{\textnormal{なに}}{\text{何}}}$ ものでもない means "\dothyp{}\dothyp{}\dothyp{} is entirely\dothyp{}\dothyp{}\dothyp{}". }

\par{4. ${\overset{\textnormal{ぐち}}{\text{愚痴}}}$ 以外の何ものでもない。 \hfill\break
It is complete stupidity. }

\par{5. ${\overset{\textnormal{こうきしん}}{\text{好奇心}}}$ 以外の何ものでもありませんね。 \hfill\break
It's entirely curiosity isn't it? }

\par{3. ”Apart from". }

\par{6a. ${\overset{\textnormal{じょうへき}}{\text{城壁}}}$ 以外の地へは ${\overset{\textnormal{ふ}}{\text{踏}}}$ み込めないなら、 ${\overset{\textnormal{たいほう}}{\text{大砲}}}$ を ${\overset{\textnormal{う}}{\text{撃}}}$ つがよい。(Bookish) \hfill\break
6b. 城壁以外から踏み込めないなら、大砲を撃つ(の)がよい。(More natural) \hfill\break
If you cannot break in apart from the wall, it would be good to fire the cannon. }
      
\section{以内}
 
\par{  The opposite of 以外, 以内 means "inside of\slash within". }

\par{7a. ${\overset{\textnormal{きょうかいせん}}{\text{境界線}}}$ 以内に止まる。 \hfill\break
7b. 境界線の中で止まる。(More natural) \hfill\break
To stop within the boundary line. }
 
\par{8. 二百字以内で ${\overset{\textnormal{まと}}{\text{纏}}}$ められるのか。(Harsh) \hfill\break
Can you summarize it within 200 characters? }
 
\par{9. 6時間以内に着くはずです。 \hfill\break
It is supposed to arrive within 6 hours. }
      
\section{以降}
 
\par{ 以降 means "thereafter". }

\par{10. 20世紀以降、日本は新技術において進んでいます。 \hfill\break
From the 20th century thereafter, Japan has been advancing in new technology. }

\par{11. 面会人は午後10時以降本病院に留まることはできません。 \hfill\break
Visitors can not remain in the hospital on and after 10 p.m. }

\par{12. 8時以降は店を閉じています。 \hfill\break
We will close the shop after 8 o' clock. }
      
\section{以後}
 
\par{  Following after nouns and the て形 of a verb, 以後 means "thereafter\slash since then". 以後 is quite \textbf{objective }while 以降 is not. 以後 may also mean "from now on". }
 
\par{13a. 以後、十分気をつけます。 \hfill\break
13b. 以後はもっと注意します。 \hfill\break
I'll be more careful from now on. }
 
\par{14. その店舗は7時\{以後は・以降は・から\}開店しています。 \hfill\break
The shop will be open after 7 o' clock. }
 
\par{15. 彼女は ${\overset{\textnormal{せんせんしゅう}}{\text{先々週}}}$ 風邪を引いて、以後ずっと寝込んでいます。 \hfill\break
She caught a cold two weeks ago and has been in bed ever since. }
 
\par{16. それ以後ずっと \hfill\break
Ever after }

\par{17a. ${\overset{\textnormal{たいしょくいご}}{\text{退職以後}}}$ は ${\overset{\textnormal{ゆうゆうじてき}}{\text{悠々自適}}}$ 。 \hfill\break
17b. 退職後は悠々自適。(More natural) \hfill\break
After retirement is leisure with dignity. }
      
\section{以前}
 
\par{ The opposite of 以降, 以前 means "before" or "ago". }

\par{18. 以前どこかであなたに出会ったことがあります。 \hfill\break
I believe I've seen you somewhere before. }

\par{19. 4人とも以前は旧「Enron」に勤めていた。 \hfill\break
All four formerly worked at Enron. }

\par{20. 以前よりずっとよく見えます。 \hfill\break
It looks much better than it was before. }

\par{21. 以前ほど魚を食べていない。 \hfill\break
We're eating not as much fish. }

\par{22. 俺は以前にもまして勉強しようと決めたのだよ。 \hfill\break
I decided to study harder still more than ever before. }
      
\section{以来}
 
\par{  以来 means "from\slash since". In a slightly old fashioned sense, it may mean "henceforth". It is used when the situation is continuing action or state since a certain point in time. Other situations are taken care of by ~てから. Also, this phrase is interchangeable with the even more less frequently used ~てこのかた. }
 
\par{23a. 以来君と旅行するのはご ${\overset{\textnormal{めん}}{\text{免}}}$ だ。(Original from text; old-fashioned) \hfill\break
23b. 今後、君と旅行するのはご免だ。(More natural) \hfill\break
I decline from traveling with you henceforth. \hfill\break
From ${\overset{\textnormal{そうせき}}{\text{漱石}}}$ . }
 
\par{24a. 以来、気をつけたまえ。(Old-fashioned) \hfill\break
24b. 以後、気をつけたまえ。(Slightly old-fashioned; Masculine) \hfill\break
From henceforth, be careful. }
 
\par{25. 彼は大学を出て以来会社に勤めています。 \hfill\break
He has been working at the company since he left college. }
 
\par{26a. 子供の ${\overset{\textnormal{ころ}}{\text{頃}}}$ 以来知っている。 \hfill\break
26b. 子供の頃から知っている。(More natural) \hfill\break
To know since childhood. }

\par{27. 佐伯に関しては、このマンションに越して以来、何の音沙汰もなかったから、徐々に緊張が解け始めているが、拳銃のことは、日を追うに連れて祥子の心に重くのしかかってきていた。 \hfill\break
In regards to Sahaku, there hadn't been any news since moving into this apartment, so Sachiko's tensions gradually began to wind down, but as for the handgun, as she pursued the days, it came to press heavily on her heart. \hfill\break
From 冷たい誘惑 by 乃南アサ. }

\par{\textbf{Word Note }: 音沙汰 literally has "sound" and "incident" in it. Although 沙汰 in this case is used to mean "update", the word is most likely used here to continue a not so favorable tone in regards to 佐伯. Here are some more phrases with 沙汰. }

\begin{ltabulary}{|P|P|P|P|P|P|}
\hline 

警察沙汰 & Police case & 取り沙汰 & Idle talk & 裁判沙汰 & Litigation \\ \cline{1-6}

表沙汰 & Creating publicity & 色恋沙汰 & Love affair & 狂気の沙汰 & Madness \\ \cline{1-6}

\end{ltabulary}
      
\section{以上}
 
\par{ 1. "More than". }

\par{28. このグループには12歳以上の子供が ${\overset{\textnormal{ふく}}{\text{含}}}$ まれている。 \hfill\break
Children twelve and over are included in this group. }

\par{29. iPadは400ドル以上する。 \hfill\break
The iPad costs 400 dollars and over. }

\par{30. 3ヶ月以上大阪に ${\overset{\textnormal{たいざい}}{\text{滞在}}}$ するつもりだ。 \hfill\break
I plan to stay in Osaka for over three months. }

\par{31. 予想以上 \hfill\break
Beyond expectation }

\par{32. 期待以上でした。 \hfill\break
It exceeded our hopes. }

\par{33 . これ以上の ${\overset{\textnormal{めんどう}}{\text{面倒}}}$ には ${\overset{\textnormal{た}}{\text{耐}}}$ えられない。 \hfill\break
I can't take it anymore. }

\par{2. Above-mentioned; foregoing; herein-before. }

\par{34. 以上の説明でご理解を ${\overset{\textnormal{いただ}}{\text{頂}}}$ きたい。 \hfill\break
I would like to receive some understanding with the above explanation. }

\par{35. 以上は方法を説明したものだ。 \hfill\break
The above-mentioned explains how to do it. }

\par{3. That is the end; the end; this is all. }

\par{36. 以上で私の報告は終わります。 \hfill\break
Let me finish my report with this. }

\par{37a. 死亡者3千名、行方不明者1万5千名、以上1万8千名。 \hfill\break
37b. 死亡者3千名、行方不明者1万5千名、計1万8千名。 \hfill\break
Fatalities 3000 people, missing people 15,000 people, and 18,000 people in total. }

\par{4. Following the 連体形 of a verb, it means "seeing that", "over", or "since". }

\par{38. 前回の売り上げ以上の成績を上げる。 \hfill\break
To raise improvements over the previous sales. }

\par{39. 昨年以上に売り上げは ${\overset{\textnormal{の}}{\text{伸}}}$ ばせない。 \hfill\break
We can't boost our sales above as they were last year. }

\par{40. (私を)信頼できない以上、犬の世話をさせるべきじゃない。 \hfill\break
Since you don't trust me, you should not let me take care of your dog. }
      
\section{以下}
 
\par{ 以下 means "below(-mentioned)\slash following". }

\par{41. 以下次号。 \hfill\break
To be continued. }

\par{42. 結果は以下の通り。 \hfill\break
The results are as follows. }

\par{43a. 氷点以下となったため、寒さで体温の下がる低体温症で亡くなった方もいました。(Rare) \hfill\break
43b. 氷点下となりましたから、寒さで体温の下がる低体温症で亡くなった方もいました。 \hfill\break
Because it has become below freezing, there are those that have died from hypothermia due to loss of body temperature from the cold. }

\par{44. 小数点以下を切り捨てる。 \hfill\break
To round off the figures below\slash after the decimal point. }

\par{45. 社長以下八名が出席しました。 \hfill\break
With the president, eight people in total were present.  }
      
\section{以遠 \& 以往}
 
\par{ 以遠 means "beyond" and 以往 means "hereafter\slash formerly". }

\par{46. この電車は大阪以遠は各駅停車となります。 \hfill\break
This train stops at Osaka and all stations beyond. }

\par{47a. 終戦以往百年。(Literary; old-fashioned) \hfill\break
47b. 終戦後百年。(Natural) \hfill\break
47c. 終戦以来百年。(Natural) \hfill\break
100 years hereafter the war. }
      
\section{以東、以北、以南、以西}
 
\par{ These words, 以東, 以北, 以南, and 以西 mean "east of", "north of", "south of", and "west of" respectively. However, the area is everything below. So, consider that when you use these words. }

\par{48. 東京以東 \hfill\break
East of Tokyo }

\par{49. 東京の東は千葉県です。 \hfill\break
To the east of Tokyo is Chiba Prefecture. }

\par{50. ミシシッピ川以西はグレート・プレーンズです。 \hfill\break
West of the Mississippi River are the Great Plains. }

\par{51a. 東京以北には福島第一原発発電所があります。X \hfill\break
51b. 東京の北には福島第一原発発電所があります。〇 \hfill\break
51c. 福島第一原発発電所は東京の北です。〇 \hfill\break
North of Tokyo is the Fukushima Number One Nuclear Electric Power Plant. }

\par{52a. 仙台以南は東京です。X \hfill\break
52b. 仙台の南に東京はあります。〇 \hfill\break
52c. 東京は仙台の南です。〇 \hfill\break
South of Sendai is Tokyo. }
    