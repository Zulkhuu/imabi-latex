    
\chapter{Slang る Verbs}

\begin{center}
\begin{Large}
第333課: Slang る Verbs 
\end{Large}
\end{center}
 
\par{ Although most of these words come from the last \textbf{forty years }or so, it has been a trend in Japanese to create new verbs by just attaching る to nouns, contractions of nouns, or contractions of other phrases, to create new verbs. They tend to break several rules in the process, but that's what makes them ever more interesting. }

\par{ To clarify what this lesson includes, it is an exhaustive list of these words that have been made in the last 100 years or so. As such, not all of these are used. These are all generally treated as slang, so it is still best to use the expressions on the far right column. There is also differences in the frequency of these phrases depending on region. }
      
\section{The Words}
 
\par{ The most peculiar verbs of all are those that have る attached to them in slang. Below is a rather exhaustive list of these words. They are all supposed to in theory be conjugated as 一段 verbs, because they are slang involving adding る attached to a stem,  if you find a non-一段 conjugation for any of them, don't be surprised. In fact, a lot of these are 五段 verbs. }

\par{ In fact, the conjugation power of many of these verbs is not to the full extant as a normal verb. That is because of their recent origin. }

\begin{ltabulary}{|P|P|P|P|}
\hline 

事故る & じこる & To get in an accident & Slang for 事故に遭う \\ \cline{1-4}

退治る & たいじる & To exterminate & Slang for 退治する \\ \cline{1-4}

皮肉る & ひにくる & To speak sarcastically & Slang for 皮肉をいう \\ \cline{1-4}

愚痴る & ぐちる & To complain & Slang for 愚痴をいう \\ \cline{1-4}

駄句る & だくる & To make horrible poems & Slang for つまらない句を作る \\ \cline{1-4}

ラグる & らぐる & To lag & Slang for ラグがある \\ \cline{1-4}

サボる & さぼる & To be truant\slash play hooky & Slang for 仕事を怠ける \\ \cline{1-4}

ミスる & ミスる & To make a miss & Slang for ミスを犯す \\ \cline{1-4}

写メる & しゃめる & To take a picture with cellphone and e-mail it to someone. & Slang for 写メールを撮る \\ \cline{1-4}

ポシャる & ぽしゃる & To fizzle out & Slang from シャッポを脱ぐ \\ \cline{1-4}

パニクる & ぱにくる & To get in a panic & Slang for 頭がパニックになる \\ \cline{1-4}

寝ぼる & ねぼる & To oversleep & Slang for 寝坊する \\ \cline{1-4}

デコる & でこる & To decorate & Slang for 飾りつけをする \\ \cline{1-4}

きょどる & きょどる & To act suspiciously & Slang for 挙動不審な行動を取る \\ \cline{1-4}

デマる & でもる & To say baseless lies & Slang for 根拠のないう噂話をする \\ \cline{1-4}

ハモる & はもる & To create harmony & Slang for ハーモニーを作る \\ \cline{1-4}

ファブる & ふぁぶる & To use febreze & Slang for ファブリーズを吹き付ける \\ \cline{1-4}

ピヨる & ぴよる & To be knocked out and see stars & Slang for ふらついている活動不能状態になる \\ \cline{1-4}

脂る & やにる & To smoke & Slang for タバコを吸う \\ \cline{1-4}

ヤグる & やぐる & To be witnessed cheating & Slang for 浮気現場を目撃される \\ \cline{1-4}

ラリる & らりる & To be high on a thinner & Slang for シンナーで酔ったようなふらふら状態になる \\ \cline{1-4}

ミソる & みそる & To reach one's thirties & Slang for 三十代になる \\ \cline{1-4}

ロンゲる & ろんげる & For a guy to grow his hair out & Slang for 男性が長髪にする \\ \cline{1-4}

はぶる & はぶる & To leave out & Slang for 仲間外れにする \\ \cline{1-4}

ハミる & はみる & To leave out & Slang for 仲間外れにする \\ \cline{1-4}

拉致る & らちる & To abduct & Slang for 拉致する \\ \cline{1-4}

モナる & もなる & To commit adultery & Slang for 不倫をする \\ \cline{1-4}

ルンペる & るんぺる & To become a vagrant & Slang for 浮浪者になる \\ \cline{1-4}

ペリる & ぺりる & To have a shameless attitude & Slang for 図々しい態度を取る \\ \cline{1-4}

ボコる & ぼこる & To gang up on someone & Slang for 袋叩きにする \\ \cline{1-4}

ブチる & ぶちる & To break a promise & Slang for 約束を破る \\ \cline{1-4}

凸る & でこる & To charge at & Slang for 突撃する \\ \cline{1-4}

凸る & とつる & To charge at & Slang for 突撃する \\ \cline{1-4}

ビビる & びびる & To get cold feet & Slang for おじけづく \\ \cline{1-4}

チャリる & ちゃりる & To ride a bicycle & Slang for 自転車に乗る \\ \cline{1-4}

フィニる & ふぃにる & To finish something & Slang for 物事を終わらせる \\ \cline{1-4}

ポチる & ぽちる & To click purchase and buy & Slang for 購入ボタンを押して買う \\ \cline{1-4}

ポニョる & ぽにょる & To get fat; watch "Ponyo on the Cliff" and act like Ponyo & Slang for 太る; 「崖の上のポニョ」を見てからポニョのような行動をする \\ \cline{1-4}

バビる & ばびる & To be really surprised & Slang for 非常に驚く \\ \cline{1-4}

オケる & おける & To go do karaoke & Slang for カラオケに行く \\ \cline{1-4}

パシャる & ぱしゃる & To take a photograph & Slang for 写真を撮る \\ \cline{1-4}

テクる & てくる & To have a technique & Slang for テクニックを持っている \\ \cline{1-4}

デリバる & でりばる & To take delivery & Slang for 出前を取る \\ \cline{1-4}

デリる & でりる & To delete; take delivery & Slang for 削除する; 出前を取る \\ \cline{1-4}

バクる & ばくる & To catch a criminal; to steal an idea & Slang for 犯人を捕まえる; アイデアを盗む \\ \cline{1-4}

テンパる & てんぱる & To be rushed; to lose flexibility & Slang for 焦る; 余裕がない状態になる \\ \cline{1-4}

直る & ちょくる & To confess right when you meet & Slang for 会ってすぐ告白する \\ \cline{1-4}

ギコる & ぎこる & To be addicted to the Internet & Slang for ネット中毒になる \\ \cline{1-4}

チョメる & ちょめる & To have sex & Slang for 性交する・男女が仲よくする \\ \cline{1-4}

コスる & こする & To cosplay & Slang for コスチューム・プレイをする \\ \cline{1-4}

赤る & せきる & To exchange info on an infrared system & Slang for 赤外通信で情報交換する \\ \cline{1-4}

タヒる & たひる & To die; be extremely tired & Slang for 死ぬ; 非常に疲れている \\ \cline{1-4}

ダビる & だびる & To give a copy to the media & Slang for メディアへコピーする \\ \cline{1-4}

即る & そくる & To have sex with a girl you just picked up & Slang for ナンパした女性と会ったその日に性交に及ぶ \\ \cline{1-4}

ダブる & だぶる & To double; repeat grade level & Slang for 重複する; 留年する \\ \cline{1-4}

チクる & ちくる & To tell-tale & Slang for 告げ口(を)する \\ \cline{1-4}

だきょる & だきょる & To compromise & Slang for 妥協する \\ \cline{1-4}

駄弁る & だべる & To chat endlessly & Slang for 駄弁を弄する \\ \cline{1-4}

ザビる & ざびる & To be a person with thin hair & Slang for 髪が薄い人である \\ \cline{1-4}

ジャズる & じゃずる & To enjoy jazz & Slang for ジャズを楽しむ \\ \cline{1-4}

ジモる & じもる & To go to one's hometown for fun & Slang for 地元で遊ぶ \\ \cline{1-4}

シケる & しける & For the economy to be bad & Slang for 景気が悪い \\ \cline{1-4}

ビニる & びにる & To go to a convenience store & Slang for コンビニエンスストアに行く \\ \cline{1-4}

サンダる & さんだる & To go to a convenience store late at night & Slang for 深夜コンビニエンスストアに行く \\ \cline{1-4}

腰る & こしる & To chicken & Slang for 腰が引ける \\ \cline{1-4}

チキる & ちきる & To chicken & Slang for 怯える。 \\ \cline{1-4}

サクる & さくる & To delete; come home quickly; sacrifice & Slang for 削除する; さくっと帰る; 生贄にする \\ \cline{1-4}

ぐれる & ぐれる & To be delinquent & Slang for あてが外れる \\ \cline{1-4}

キャピる & きゃぴる & To be flippant like a girl & Slang for 若い女性のようにはじける \\ \cline{1-4}

ゲトる & げとる & To get what one wants & Slang for ほしいものを手に入れる \\ \cline{1-4}

ケミる & けみる & To have good chemistry & Slang for ハーモニーを織り成す \\ \cline{1-4}

呉儀る & ごぎる & To cancel a commitment right before it & Slang for 直前に約束をキャンセルする \\ \cline{1-4}

雑魚る & ざこる & To lose one's honor\slash charm & 名誉・魅力がなくなる。 \\ \cline{1-4}

グダる & ぐだる & To be exhausted & Slang for グダグダだ \\ \cline{1-4}

ダグる & だぐる & To be dead tired & Slang for ぐったりする \\ \cline{1-4}

パクる & ぱくる & To pick up ladies & Slang for ナンパする \\ \cline{1-4}

タクる & たくる & To ride a taxi & Slang for タクシーに乗る \\ \cline{1-4}

ハイカる & はいかる & To be in Western fashion & Slang for 西洋風の格好をする \\ \cline{1-4}

ガスる & がする & To be misty; to go to Gasto & Slang for 霧がかかる; ガスとで飲食する \\ \cline{1-4}

亀る & かめる & To be late & Slang for 遅れる \\ \cline{1-4}

過疎る & かそる & To depopulate; hair to thin out & Slang for 過疎化する; 髪の毛が薄くなる \\ \cline{1-4}

トラバる & とらばる & To trackback & Slang for ブログのトラックバックをする \\ \cline{1-4}

バグる & ばぐる & For bugs to break out & Slang for 虫が発生する \\ \cline{1-4}

ウニる & うにる & For ones head to be all confused & Slang for 頭が混乱になる \\ \cline{1-4}

ブログる & ぶろぐる & To blog & Slang for ブログを利用する \\ \cline{1-4}

小田急る & おだきゅうる & To ride the Odakyuu Line & Slang for 小田急線に乗る \\ \cline{1-4}

イモる & いもる & To be frightened & Slang for おじけづく \\ \cline{1-4}

エダる & えだる & To work with no rest & Slang for 不眠不休で働く \\ \cline{1-4}

アムる & あむる & To copy Amuro Namie fashion; to sell Amway products & Slang for 安室奈美恵のファッションを真似する; アムウェイ商品を売る \\ \cline{1-4}

ふぁぼる & ふぁぼる & To favorite something on Twitter & Slang for Twitterのつぶやきをお気に入りに登録する \\ \cline{1-4}

アサヒる & あさひる & To fabricate & Slang for 捏造する \\ \cline{1-4}

ヤフる & やふる & To surf on Yahoo & Slang for ヤフーで検索する \\ \cline{1-4}

ビる & びる & To surf on MSN live search & Slang for MSNライブサーチで検索する \\ \cline{1-4}

MSNる & えむえすえぬる & To surf on MSN live search & Slang for MSNライブサーチで検索する \\ \cline{1-4}

ライブサーチる & らいぶさーちる & To surf on MSN live search & Slang for MSNライブサーチで検索する \\ \cline{1-4}

リバる & りばる & To become free; to puke from drinking & Slang for 解放する; 飲みすぎて嘔吐する \\ \cline{1-4}

アジる & あじる & To agitate & Slang for 扇動する \\ \cline{1-4}

無視る & むしる & To ignore & Slang for 無視する \\ \cline{1-4}

ディスる & でぃする & To disrespect & Slang for 軽蔑する・罵る・貶す \\ \cline{1-4}

告る & こくる & To confess & Slang for 告白する \\ \cline{1-4}

チャーる & ちゃーる & To go to a coffee shop & Slang for 喫茶店に行く \\ \cline{1-4}

ラーメる & らーめる & To go eat ramen & Slang for ラーメンを食べに行く \\ \cline{1-4}

ググる & ぐぐる & To google & Slang for グーグルで検索する \\ \cline{1-4}

テロる & てろる & To terrorize & Slang for 暴力に訴えてことを成す \\ \cline{1-4}

マクる & まくる & To go to Mc. Donalds & Slang for マクドナルドに行く \\ \cline{1-4}

マクドる & まくどる & To go to Mc. Donalds (West Japan) & Slang for マクドナルドに行く \\ \cline{1-4}

マヨる & まよる & To put mayonnaise (on) & Slang for マヨネーズをかける \\ \cline{1-4}

ファミる & ふぁみる & To go a family restaurant & Slang for ファミリーレストランに行く \\ \cline{1-4}

サダハる & さだはる & To keep on at it even in one's fifties & Slang for 中年になってもがんばる \\ \cline{1-4}

小林る & こばやしる & To do something refreshing & Slang for すかっとしたことをする \\ \cline{1-4}

アピる & あぴる & To appeal to (コギャル語) & Slang for アピールする \\ \cline{1-4}

ネグる & ねぐる & To neglect & Slang for ネグレクトする \\ \cline{1-4}

江川る & えがわる & To sacrifice others without hesitation & Slang for 犠牲にしてはばからない \\ \cline{1-4}

田淵る & たぶちる & To be chubby & Slang for 太り気味 \\ \cline{1-4}

野暮る & やぼる & To do something thoughtless & Slang for 野暮なことをする \\ \cline{1-4}

スタバる & すたばる & To go to Starbucks & Slang for スターバックスに行く \\ \cline{1-4}

カンコる & かんこる & To be desolate & Slang for 人が少なく寂れている \\ \cline{1-4}

\end{ltabulary}
\hfill\break
\textbf{Word Notes }:  1. Also, as for Mc. Donalds, it is abbreviated as マック in East Japan and  マクド in West Japan. 2. Some slang words are inspired by the names of people such as is the case with 田淵る. 田淵 was the name of a fat character in a manga in the 80s.     