    
\chapter{Funeral}

\begin{center}
\begin{Large}
第309課: Funeral 
\end{Large}
\end{center}
 
\par{ Burial customs in Japan are very serious and complex. In this lesson, you will learn a lot about what is appropriate and what is not. }
      
\section{What to Do}
 
\par{ When a person dies,  ${\overset{\textnormal{こうでん}}{\text{香典}}}$  is usually given as a funeral gift of money to the family at the vigil or funeral. This is given in an envelope called a  ${\overset{\textnormal{のしぶくろ}}{\text{熨斗袋}}}$ . This is a general term for envelopes for giving money, so you'll see more specific terms later in this section. This envelope looks different depending on religion, so you will need to be careful and ask for what is appropriate. For instance, if there is a picture of a lotus on it, it should only be used for those who follow Buddhism. What is put in the inscription is not an easy question, and it depends heavily on the religious beliefs of the deceased. It also doesn't help that there isn't unilateral decision on what to do. }

\par{ \textbf{What is 香典? }}

\par{ In 香典, there is this meaning of 持ちつ持たれつ・ ${\overset{\textnormal{そうごふじょ}}{\text{相互扶助}}}$  (give-and-take). The amount of money you give is based on your relation to the person. The most common amounts are 5000円, 3000円, and 10,000円, but the latter may be avoided as it is deemed unlucky by many to give an amount with an initial even digit. If you are unable to give much money, you should write  ${\overset{\textnormal{きくいちりん}}{\text{菊一輪}}}$ . The money should also be new bills found at a bank or 郵便局. What to put on the top of the paper envelope (表書き) is no easy matter. }

\par{\textbf{Grammar Note }: This つ is an old particle equivalent to たり. }

\par{\textbf{The 中包み }}

\par{ In the inside, the front side of the card (中包み) should have the amount to be given in old-style characters (大字). So, 壱, 弐, 参, 四, 五・伍, 六, 七, 八, 九, 拾, 百, 千, and 萬 should be used instead. 円 should be spelled as 圓. The name and address of the individual should be written on the back side. This is to keep the sadness of the death to be underneath. }

\par{ \textbf{御香典 }}

\par{ The first thing to note is that ご・お should be written in 御 to show more respect. 御香典 may be safe if you don't know the religion. It may be read as either おこうでん or ごこうでん, but the first is the most common, and the decision may also deal with specific sect. Regardless of reading, this is not appropriate for Shinto followers. So,  ${\overset{\textnormal{ごれいぜん}}{\text{御霊前}}}$  is felt by many others as the most appropriate heading. }

\par{\textbf{For Shinto Followers }}

\par{If the person is a Shinto believer,  ${\overset{\textnormal{ごしんぜん}}{\text{御神前}}}$  or  ${\overset{\textnormal{おんたまぐしりょう}}{\text{御玉串料}}}$  may be used.  ${\overset{\textnormal{おさかきりょう}}{\text{御榊料}}}$ ,  ${\overset{\textnormal{ごしんせんりょう}}{\text{御神饌料}}}$ , and  ${\overset{\textnormal{おそなえものりょう}}{\text{御供物料}}}$  may also be acceptable.  ${\overset{\textnormal{はつほりょう}}{\text{初穂料}}}$  is written in an envelope addressed to the Shinto shrine.  ${\overset{\textnormal{ぬさりょう}}{\text{幣料}}}$  may be used on the 金包み with the money for the お ${\overset{\textnormal{はら}}{\text{祓}}}$ い (purification) costs. }

\par{ \textbf{For Christians }}

\par{ For Christians,  ${\overset{\textnormal{おはなりょう}}{\text{御花料}}}$  or  ${\overset{\textnormal{おん}}{\text{御}}}$ ミサ料・御弥撒料 (for Catholics) may be used. Using 漢字, of course, makes things more formal. You may also see  ${\overset{\textnormal{おんはなわりょう}}{\text{御花環料}}}$ . For design-less  ${\overset{\textnormal{ぶしゅうぎぶくろ}}{\text{不祝儀袋}}}$ , which equates to 熨斗袋 in the sense that it is an envelope for a misfortunate occasion,  ご霊前 may be acceptable. }

\par{\textbf{For Buddhists }}

\par{ 御霊前 can be possibly used even if a person is Buddhist, Shinto follower, or Christian. However, if the person is of the  ${\overset{\textnormal{じょうどしんしゅう}}{\text{浄土真宗}}}$  sect, you should use 御仏前 instead. If you can't find this out, 御霊前 is acceptable. So, it can be used if you don't know the religious of the person. You can also use this for 御供物 (flowers). }

\par{ 御霊前・御香典 can be used on the vigil or funeral, but ${\overset{\textnormal{ごぶつぜん}}{\text{御仏前}}}$ ・御佛前 should be used after the Buddhist service on the 49th day the person has died.  ${\overset{\textnormal{ごこうりょう}}{\text{御香料}}}$ ・ ${\overset{\textnormal{ごこうげりょう}}{\text{御香華料}}}$  is also acceptable for this occasion. However, as mentioned earlier, 御仏前・御佛前 is acceptable for all the above in the 浄土真宗 Buddhism. 佛 is the old form of 仏, which makes it more formal. }

\par{ \textbf{Contacting the Family }}

\par{ If the person is a loved one of your family or a close family, you should contact the family as soon as you can. If it is an acquaintance of some sort, you should wait until after the vigil to give your condolences. When you do receive contact, make sure to not only give your condolences but also find out the time and place of the vigil and funeral as well as the person's religion. If you already gave your 香典 at the vigil, you should only write your name in the register at the funeral. }

\par{ \textbf{弔電 }}

\par{ If you unable to attend either or if it is business related, you should send a  ${\overset{\textnormal{ちょうでん}}{\text{弔電}}}$ , This is literally a "condolence telegram", but there ways of doing this through phone and fax. The 表書き for this should be  ${\overset{\textnormal{おとむらいりょう}}{\text{御弔料}}}$  along with the name of the company\slash organization. However, you should consult others for this as well as practices vary in acceptability. In it, you should have  ${\overset{\textnormal{こ}}{\text{故}}}$ 〇〇〇〇様ご遺族様 in it. The prefix 故 means “late” as in “the late Mr. Smith”. 遺族 means “the family of the deceased”. 様 is attached to show respect. }

\par{ \textbf{Not Knowing the Person }}

\par{ It is also appropriate to  ${\overset{\textnormal{おくやみ}}{\text{御悔}}}$  if the person is not someone you are close or related to. However, using one of the more appropriate, religious focused headings is more respectful. }

\par{\textbf{Thanks to the お寺・僧侶 }}

\par{There are also phrases to show thanks to the temple\slash monks. }

\par{${\overset{\textnormal{おふせ}}{\text{御布施}}}$ : Used to give thanks to the temple and or monks for a Buddhist funeral. }

\par{御経料: This is the same as 御布施. }

\par{${\overset{\textnormal{どきょうおんれい}}{\text{読経御礼}}}$ : It is used in the same sense as 御経料. おれい may be used a lot instead, but おんれい is more formal. Don\textquotesingle t pronounce 読経 as どっきょう. }

\par{${\overset{\textnormal{かいみょうりょう}}{\text{戒名料}}}$ : This is used for showing thanks for the posthumous Buddhist name being given to the deceased. }

\par{${\overset{\textnormal{ごえこうりょう}}{\text{御回向料}}}$ : This is used for showing thanks for reading out a sutra at the funeral. }

\par{${\overset{\textnormal{おくるまりょう}}{\text{御車料}}}$ : This is used for travel expenses. }

\par{${\overset{\textnormal{おぜんりょう}}{\text{御膳料}}}$ : Used for food costs. It may also be 御食事料. \hfill\break
${\overset{\textnormal{ごそくいりょう}}{\text{御足衣料}}}$ : This may also be used for travel expenses.  }

\par{ \textbf{配り物 }}

\par{ If you were to be in the position of presenting a gift for a funeral offering, there are a few phrases for this. }

\par{${\overset{\textnormal{こころざし}}{\text{志}}}$ : Used in Buddhist and Shinto styles. This is used as a return gift for 香典 or  ${\overset{\textnormal{ほうよう}}{\text{法要}}}$  (Buddhist service). }

\par{${\overset{\textnormal{そくよう}}{\text{粗供養}}}$ : Present from a 法要. }

\par{\textbf{Final Note }: As you can see, funeral arrangements in Japanese culture is very complex. However, this also shows us just how complex お~ and ご~ phrases may be. As this section illustrates, you may even see おん~. }
    