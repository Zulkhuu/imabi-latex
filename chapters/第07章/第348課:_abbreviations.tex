    
\chapter{Abbreviations}

\begin{center}
\begin{Large}
第348課: Abbreviations 
\end{Large}
\end{center}
 
\par{ Abbreviations (略語) are very handy when words get lengthy. In fact, the Japanese term for abbreviation is an abbreviation. The non-abbreviated term is 省略語. }
      
\section{略語}
 
\par{ The following chart shows the most common methods of abbreviation in Japanese. }

\begin{ltabulary}{|P|P|}
\hline 

Phrases & Words \\ \cline{1-2}

①First characters of each part only \hfill\break
& ⑤First part only \hfill\break
\\ \cline{1-2}

②First part only \hfill\break
& ⑥Last part only \hfill\break
\\ \cline{1-2}

③Last part only \hfill\break
& ⑦No ん \hfill\break
\\ \cline{1-2}

④First and last part only \hfill\break
&  \\ \cline{1-2}

\end{ltabulary}

\par{ Some abbreviations date back hundreds of years. For example, the 盆(ぼん) in the Obon Festival is a contraction of 盂蘭盆会(うらぼんえ). }

\par{ When exposed to the West, new phrases were often abbreviated. Examples include ハンカチ from handkerchief and 五輪, which referred to the Berlin Olympics, for the Olympics. }

\par{ The most commonly abbreviated words are 外来語 and 熟語. Contractions in conjugation and native words exist too. Titles in news reports have extended use of abbreviations. Of course, you will sometimes see regional variation such as the contraction of McDonald's (マックドナルド): マック (East Japan) and マクド (West Japan). }

\begin{ltabulary}{|P|P|P|P|P|}
\hline 

Abbreviation & Kana & From & Type & Meaning \\ \cline{1-5}

全銀協会 & ぜんぎんきょうかい & 全国銀行協会 & ① & Japan's Bank Association \\ \cline{1-5}

米 & べい & 米国 & ⑤ & America \\ \cline{1-5}

特急 & とっきゅう & 特別急行 & ① & Express train \\ \cline{1-5}

パソコン & ぱそこん & パソナルコンピューター & ① & Personal computer \\ \cline{1-5}

小学 & しょうがく & 小学校 & ⑤ & Elementary school \\ \cline{1-5}

東電 & とうでん & 東京電力 & ① & Tokyo Electric; TEPCO \hfill\break
\\ \cline{1-5}

ケータイ & けーたい & 携帯電話 & ② & Cellphone \\ \cline{1-5}

ブログ & ぶろぐ & ウェブログ & ③ & Blog \\ \cline{1-5}

エンターテイメント & えんたーていめんと & エンターテインメント & ⑦ & Entertainment \\ \cline{1-5}

チョコ & ちょこ & チョコレート & ⑤ & Chocolate \\ \cline{1-5}

アイス & あいす & アイスコーヒー & ② & Ice Coffee \\ \cline{1-5}

レイコー & れいこー & アイスコーヒー & ① & Ice Coffee (Kansai Variant) \\ \cline{1-5}

京葉 & けいよう & 東京千葉 & ③ & Tokyo-Chiba\slash Keiyou Line \hfill\break
\\ \cline{1-5}

高校 & こうこう & 高等学校 & ④ & High school \\ \cline{1-5}

入管 & にゅうかん & 入国管理局 & ① & Immigration office \\ \cline{1-5}

東大 & とうだい & 東京大学 & ① & Tokyo University \\ \cline{1-5}

日経 & にっけい & 日本経済新聞 & ① & Japan Economic Times \\ \cline{1-5}

ファミレス & ふぁみれす & ファミリレストラン & ① & Family restaurant \\ \cline{1-5}

ラブホ & らぶほ & ラブホテル & ② & Love hotel \\ \cline{1-5}

テレビ & てれび & テレビジョン & ② & Television \\ \cline{1-5}

OL & おーえーる & Office Lady & ① & Office lady \\ \cline{1-5}

国連 & こくれん & 国際連合 & ① & United Nations \\ \cline{1-5}

ソ連 & それん & ソビエト連合 & ① & Soviet Union \\ \cline{1-5}

安保理 & あんぽり & 安全保障理事会 & ① & Security Council \\ \cline{1-5}

取引所 & とりひきしょ & 証券取引所 & ③ & Stock exchange \\ \cline{1-5}

リハビリ & りはびり & リハビリテーション & ③ & Rehabilitation \\ \cline{1-5}

ブル & ぶる & ブルドーザー \hfill\break
ブルドッグ \hfill\break
ブルジョワ & ⑤ & Bulldozer; bulldog; bourgeoisie \hfill\break
\\ \cline{1-5}

セクハラ & せくはら & セクシャルハラスメント & ① & Sexual harassment \\ \cline{1-5}

\end{ltabulary}
\hfill\break
 Aside from these sort of abbreviations, there are odd abbreviations of specific phrases. For instance, the phrase 半端なことではない (not half-way) is often abbreviate to 半端ない in casual speech. However, if you don't know both, the abbreviated form is difficult to understand because grammatical items were dropped.     