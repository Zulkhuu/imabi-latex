    
\chapter{The Date}

\begin{center}
\begin{Large}
第312課: The Date: Traditional Calendar System 
\end{Large}
\end{center}
 
\par{ We have already learned well before how to make the date in Japanese. Unlike in American English, in Japanese, the data begins with the year, which is then followed by the month and then the day of the month. You can then go further and add the name of the day of the week after this in various ways. }

\par{ What we have not addressed so far is the fact that Japanese has another way of representing the date. We are used to using the Western Calendar, but the Japanese have their own calendar system. Also, Japan used to follow the lunar calendar. So, this lesson will address both of these things. }
      
\section{Western Calendar}
 
\par{ For review, look at the following dates written in various fashions. }

\par{1. 2014年3月11日 \hfill\break
March 11, 2014 }

\par{2. 2200年10月13日 \hfill\break
October 13, 2200 }

\par{3. 1745年4月9日 \hfill\break
April 9, 1745 }

\par{4. 一九五四年一一月二五日 \hfill\break
November 25, 1954 }

\par{5. 13年4月3日発売予定! \hfill\break
Release 4\slash 3\slash 13! }

\par{\textbf{Abbreviation Note }: Just as in English, the year may be abbreviated likewise. As such, 95年 almost always means 1995. Also, as we are in the 21st century, 13年 = 2013年. }

\par{6. 四百五拾六年壱拾弐月参日 \hfill\break
December 3rd, 456 \hfill\break
\hfill\break
\textbf{Spelling Note }: This last example uses 大字 numeral variants, which are often used in official documents as well as on money. This convention is closer to how numbers are used in Chinese. Although December would still be read off as じゅうにがつ, it is spelled as if it were read as いちじゅうにがつ. You will hardly ever encounter this convention, but it is important to know that it exists. }

\par{\textbf{Orthography Note }: Japanese numerals are written in various fonts in printed material. You will notice that people will have their own preferences. Some may use English fonts like in Ex. 1 or  some will use a Japanese font as in Ex. 2-5. }
      
\section{Traditional Calendar}
 
\par{  There are two calendar systems in use today in Japan. The most common is the Western Calendar ( ${\overset{\textnormal{せいれき}}{\text{西暦}}}$ ) but the traditional calendar based on the era ( ${\overset{\textnormal{ねんごう}}{\text{年号}}}$ ) of a political time period ( ${\overset{\textnormal{かいげん}}{\text{改元}}}$ ) is used heavily in formal\slash official situations. Each period in the traditional calendar starts with ${\overset{\textnormal{がんねん}}{\text{元年}}}$ and then continues as you would expect. The current era is the ${\overset{\textnormal{へいせいじだい}}{\text{平成時代}}}$ , which will end presumably when the current sitting emperor passes away. }

\par{7. ${\overset{\textnormal{しょうわ}}{\text{昭和}}}$ 47年 \hfill\break
1972 }

\par{8. 平成元年 \hfill\break
1989 }

\par{ Era names extend all the way to the beginning of Imperial reign in Japan, beginning with 神武天皇. For the majority of these first eras, the name of that era matches the traditional name of the emperor in question, which is usually a posthumous name coined with Buddhist influence. }

\par{ Later on, a tradition of coining a new term to commemorate an era takes shape. This practice continues to this day. Many speakers don't know era names older than the 明治時代, which began in 1868. Although speakers are exposed to many era names throughout the course of their education, knowledge about them is rather scattered. As such, it is not imperative of you to remember any of them but perhaps the last four. }

\par{ The chart below, although quite long and extensive, shows all era names from the beginning (660 B.C.) to the present. Again, you don't have to memorize its contents. Use this as a reference whenever you come across a date you can't comprehend. }

\par{\textbf{Chart Note }: Although the chart indicates what year each era begins, each era technically begins once the reign is switched to said emperor of that era. As such, you will need to do research regarding the month and day the transition would have taken place to create an accurate date for the years in which a new era begins. }

\begin{ltabulary}{|P|P|P|P|P|P|}
\hline 

元号 & 読み & 元年(西暦) & 元号 & 読み & 元年(西暦) \\ \cline{1-6}

神武天皇 & じんむてんのう & 前600年 & 綏靖天皇 & すいぜいてんのう & 前581年 \\ \cline{1-6}

安寧天皇 & あんねいてんのう & 前548年 & 懿徳天皇 & いとくてんのう & 前510年 \\ \cline{1-6}

孝昭天皇 & こうしょうてんのう & 前475年 & 考安天皇 & こうあんてんのう & 前392年 \\ \cline{1-6}

孝霊天皇 & こうれいてんのう & 前290年 & 孝元天皇 & こうげんてんのう & 前214年 \\ \cline{1-6}

開化天皇 & かいかてんのう & 前157年 & 崇神天皇 & すじんてんのう & 前97年 \\ \cline{1-6}

垂仁天皇 & すいにんてんのう & 前29年 & 景行天皇 & けいこうてんのう & 71年 \\ \cline{1-6}

成務天皇 & せいむてんのう & 131年 & 仲哀天皇 & ちゅうあいてんのう & 192年 \\ \cline{1-6}

神功皇后 & じんぐうこうごう & 201年 & 応神天皇 & おうじんてんのう & 270年 \\ \cline{1-6}

仁徳天皇 & にんとくてんのう & 313年 & 履中天皇 & りちゅうてんのう & 400年 \\ \cline{1-6}

反正天皇 & はんぜいてんのう & 406年 & 允恭天皇 & いんぎょうてんのう & 412年 \\ \cline{1-6}

安康天皇 & あんこうてんのう & 454年 & 雄略天皇 & ゆうりゃくてんのう & 457年 \\ \cline{1-6}

静寧天皇 & せいねいてんのう & 480年 & 顕宗天皇 & けんぞうてんのう & 485年 \\ \cline{1-6}

仁賢天皇 & にんけんてんのう & 488年 & 武烈天皇 & ぶれつてんのう & 499年 \\ \cline{1-6}

継体天皇 & けいたいてんのう & 507年 & 安閑天皇 & あんかんてんのう & 534年 \\ \cline{1-6}

宣化天皇 & せんかてんのう & 536年 & 欽明天皇 & きんめいてんのう & 540年 \\ \cline{1-6}

敏達天皇 & びだつてんのう & 572年 & 用明天皇 & てんようてんのう & 586年 \\ \cline{1-6}

崇峻天皇 & すしゅんてんのう & 588年 & 推古天皇 & すいこてんのう & 593年 \\ \cline{1-6}

舒明天皇 \hfill\break
& じょめいてんのう & 629年 & 皇極天皇 & こうぎょくてんのう & 642年 \\ \cline{1-6}

大化 & たいか & 645年 & 白雉 & はくち & 650年 \\ \cline{1-6}

斉明天皇 & さいめいてんのう & 655年 & 天智天皇 & てんじてんのう & 662年 \\ \cline{1-6}

天武天皇 & てんむてんのう & 672年 & 朱鳥 & しゅちょう & 686年 \\ \cline{1-6}

持統天皇 & じとうてんのう & 687年 & 文武天皇 & もんむてんのう & 697年 \\ \cline{1-6}

大宝 & たいほう & 701年 & 慶雲 & けいうん & 704年 \\ \cline{1-6}

和銅 & わどう & 708年 & 霊亀 & れいき & 715年 \\ \cline{1-6}

養老 & ようろう & 717年 & 神亀 & じんき & 724年 \\ \cline{1-6}

天平 & てんぴょう & 729年 & 天平感宝 & てんぴょうかんぽう & 749年 \\ \cline{1-6}

天平勝宝 & てんぴょうしょうほう & 749年 & 天平宝字 \hfill\break
& てんぴょうほうじ & 757年 \\ \cline{1-6}

天平神護 & てんぴょうじんご & 765年 & 神護景雲 & じんごけいうん & 767年 \\ \cline{1-6}

宝亀 & ほうき & 770年 & 天応 & てんおう & 781年 \\ \cline{1-6}

建暦 & けんりゃく & 782年 & 大同 & たいどう & 806年 \\ \cline{1-6}

弘仁 & こうにん & 810年 & 天長 & てんちょう & 824年 \\ \cline{1-6}

承和 & じょうわ & 834年 & 嘉祥 & かしょう & 848年 \\ \cline{1-6}

仁寿 & にんじゅ & 851年 & 斉衡 & さいこう & 854年 \\ \cline{1-6}

天安 & てんあん & 857年 & 貞観 & じょうかん & 859年 \\ \cline{1-6}

元慶 & がんぎょう & 877年 & 仁和 & にんな & 885年 \\ \cline{1-6}

寛平 & かんぴょう & 889年 & 昌泰 & しょうたい & 898年 \\ \cline{1-6}

延喜 & えんぎ & 901年 & 延長 & えんちょう & 923年 \\ \cline{1-6}

承平 & じょうへい & 931年 & 天慶 & てんぎょう & 938年 \\ \cline{1-6}

天暦 & てんりゃく & 947年 & 天徳 & てんとく & 957年 \\ \cline{1-6}

応和 & おうわ & 961年 & 康保 & こうほう & 964年 \\ \cline{1-6}

安和 & あんな & 968年 & 天禄 & てんろく & 970年 \\ \cline{1-6}

天延 & てんえん & 973年 & 貞元 & じょうげん & 976年 \\ \cline{1-6}

天元 & てんげん & 978年 & 永観 & えいかん & 983年 \\ \cline{1-6}

寛和 & かんな & 985年 & 永延 & えいえん & 987年 \\ \cline{1-6}

永祚 & えいそ & 989年 & 正暦 & しょうりゃく & 990年 \\ \cline{1-6}

長徳 & ちょうとく & 995年 & 長保 & ちょうほう & 999年 \\ \cline{1-6}

寛弘 & かんこう & 1004年 & 長和 & ちょうわ & 1012年 \\ \cline{1-6}

寛仁 & かんにん \hfill\break
& 1017年 \hfill\break
& 治安 \hfill\break
& じあん \hfill\break
& 1021年 \hfill\break
\\ \cline{1-6}

万寿 \hfill\break
& まんじゅ & 1024年 & 長元 & ちょうげん & 1028年 \\ \cline{1-6}

長暦 & ちょうりゃく & 1037年 & 長久 & ちょうきゅう & 1040年 \\ \cline{1-6}

寛徳 & かんとく & 1044年 & 永承 & えいしょう & 1046年 \\ \cline{1-6}

天喜 & てんき & 1053年 & 康平 & こうへい & 1058年 \\ \cline{1-6}

治暦 & じりゃく & 1065年 & 延久 & えんきゅう & 1069年 \\ \cline{1-6}

承保 & じょうほう & 1074年 & 永保 & えいほ & 1077年 \\ \cline{1-6}

応徳 & おうとく & 1081年 & 寛治 & かんじ & 1084年 \\ \cline{1-6}

嘉保 & かほう & 1087年 & 永長 & えいちょう & 1094年 \\ \cline{1-6}

承徳 & じょうとく & 1096年 & 康和 & こうわ & 1097年 \\ \cline{1-6}

長治 & ちょうじ & 1099年 & 嘉承 & かしょう & 1104年 \\ \cline{1-6}

天仁 & てんにん & 1106年 & 天永 & てんえい & 1108年 \\ \cline{1-6}

永久 & えいきゅう & 1110年 & 元永 & げんえい & 1113年 \\ \cline{1-6}

保安 & ほうあん & 1118年 & 天治 & てんじ & 1120年 \\ \cline{1-6}

大治 & だいじ & 1124年 & 天承 & てんじょう & 1126年 \\ \cline{1-6}

長承 & ちょうじょう & 1131年 & 保延 & ほうえん & 1132年 \\ \cline{1-6}

永治 & えいじ & 1135年 & 康治 & こうじ & 1141年 \\ \cline{1-6}

天養 & てんよう & 1142年 & 久安 & きゅうあん & 1144年 \\ \cline{1-6}

仁平 & にんぺい & 1145年 & 久寿 & きゅうじゅ & 1151年 \\ \cline{1-6}

保元 & ほうげん & 1154年 & 平治 & へいじ & 1156年 \\ \cline{1-6}

永暦 & えいりゃく & 1160年 & 応保 & おうほ & 1161年 \\ \cline{1-6}

長寛 & ちょうかん & 1163年 & 永万 & えいまん & 1165年 \\ \cline{1-6}

仁安 \hfill\break
& になん \hfill\break
& 1166年 \hfill\break
& 嘉応 \hfill\break
& かおう \hfill\break
& 1169年 \hfill\break
\\ \cline{1-6}

承安 \hfill\break
& しょうあん & 1171年 & 安元 & あんげん & 1175年 \\ \cline{1-6}

治承 & じしょう & 1177年 & 養和 & ようわ & 1181年 \\ \cline{1-6}

寿永 & じゅえい & 1182年 & 元暦 & げんりゃく & 1184年 \\ \cline{1-6}

文治 & ぶんじ & 1185年 & 建久 & けんきゅう & 1190年 \\ \cline{1-6}

正治 & しょうじ & 1199年 & 建仁 & けんにん & 1201年 \\ \cline{1-6}

元久 & げんきゅう & 1204年 & 建永 & けんえい & 1206年 \\ \cline{1-6}

承元 & じょうげん & 1207年 & 建暦 & けんりゃく & 1211年 \\ \cline{1-6}

建保 & けんぽう & 1213年 & 承久 & しょうきゅう & 1219年 \\ \cline{1-6}

貞応 & じょうおう & 1222年 & 元仁 & げんにん & 1224年 \\ \cline{1-6}

嘉禄 & かろく & 1225年 & 安貞 & あんてい & 1227年 \\ \cline{1-6}

寛喜 & かんき & 1229年 & 貞永 & じょうえい & 1232年 \\ \cline{1-6}

天福 & てんぷく & 1233年 & 文暦 & ぶんりゃく & 1234年 \\ \cline{1-6}

嘉禎 & かてい & 1235年 & 暦仁 & りゃくにん & 1238年 \\ \cline{1-6}

延応 & えんおう & 1239年 & 仁治 & にんじ & 1240年 \\ \cline{1-6}

寛元 & かんげん & 1243年 & 宝治 & ほうじ & 1247年 \\ \cline{1-6}

建長 & けんちょう & 1249年 & 康元 & こうげん & 1256年 \\ \cline{1-6}

正嘉 & しょうか & 1257年 & 正元 & しょうげん & 1259年 \\ \cline{1-6}

文応 & ぶんおう & 1260年 & 弘長 & こうちょう & 1261年 \\ \cline{1-6}

文永 & ぶんえい & 1264年 & 建治 & けんじ & 1275年 \\ \cline{1-6}

弘安 & こうあん & 1278年 & 正応 & しょうおう & 1288年 \\ \cline{1-6}

永仁 & えいにん & 1293年 & 正安 & しょうあん & 1299年 \\ \cline{1-6}

乾元 & けんげん & 1302年 & 嘉元 & かげん & 1303年 \\ \cline{1-6}

徳治 & とくじ & 1306年 & 延慶 & えんぎょう & 1308年 \\ \cline{1-6}

応長 & おうちょう & 1311年 & 正和 & しょうわ & 1312年 \\ \cline{1-6}

文保 \hfill\break
& ぶんぽう \hfill\break
& 1317年 \hfill\break
& 元応 \hfill\break
& げんおう \hfill\break
& 1319年 \hfill\break
\\ \cline{1-6}

元亨 \hfill\break
& げんこう・げんきょう & 1321年 & 正中 & しょうちゅう & 1324年 \\ \cline{1-6}

嘉暦 & かりゃく & 1326年 & 元徳 & げんとく & 1329年 \\ \cline{1-6}

元弘 & げんこう & 1331年 & 建武 & けんむ & 1334年 \\ \cline{1-6}

延元 & えんげん & 1336年 & 興国 & こうこく & 1340年 \\ \cline{1-6}

正平 & しょうへい & 1346年 & 建徳 & けんとく & 1370年 \\ \cline{1-6}

文中 & ぶんちゅう & 1372年 & 天授 & てんじゅ & 1375年 \\ \cline{1-6}

弘和 & こうわ & 1381年 & 元中 & げんちゅう & 1384年 \\ \cline{1-6}

明徳 & めいとく & 1390年 & 応永 & おうえい & 1394年 \\ \cline{1-6}

正長 & しょうちょう & 1428年 & 永享 & えいきょう & 1429年 \\ \cline{1-6}

嘉吉 & かきつ & 1441年 & 文安 & ぶんあん & 1444年 \\ \cline{1-6}

宝徳 & ほうとく & 1449年 & 享徳 & きょうとく & 1452年 \\ \cline{1-6}

康正 & こうしょう & 1455年 & 長禄 & ちょうろく & 1457年 \\ \cline{1-6}

寛正 & かんしょう & 1460年 & 文正 & ぶんしょう & 1466年 \\ \cline{1-6}

応仁 & おうにん & 1467年 & 文明 & ぶんめい & 1469年 \\ \cline{1-6}

長享 & ちょうきょう & 1487年 & 延徳 & えんとく & 1489年 \\ \cline{1-6}

明応 & めいおう & 1492年 & 文亀 & ぶんき & 1501年 \\ \cline{1-6}

永正 & えいしょう & 1504年 & 大永 & だいえい・たいえい & 1521年 \\ \cline{1-6}

享禄 \hfill\break
& きょうろく \hfill\break
& 1528年 \hfill\break
& 天文 \hfill\break
& てんぶん \hfill\break
& 1532年 \hfill\break
\\ \cline{1-6}

弘治 \hfill\break
& こうじ & 1555年 & 永禄 & えいろく & 1558年 \\ \cline{1-6}

元亀 & げんき & 1570年 & 天正 & てんしょう & 1573年 \\ \cline{1-6}

文禄 & ぶんろく & 1592年 & 慶長 & けいちょう & 1596年 \\ \cline{1-6}

元和 & げんな & 1615年 & 寛永 & かんえい & 1624年 \\ \cline{1-6}

正保 & しょうほう & 1644年 & 慶安 & けいあん & 1648年 \\ \cline{1-6}

承応 & じょうおう & 1652年 & 明暦 & めいれき & 1655年 \\ \cline{1-6}

万治 & まんじ & 1658年 & 寛文 & かんぶん & 1661年 \\ \cline{1-6}

延宝 & えんぽう & 1673年 & 天和 & てんな & 1681年 \\ \cline{1-6}

貞享 & じょうきょう & 1684年 & 元禄 & げんろく & 1688年 \\ \cline{1-6}

宝永 & ほうえい & 1704年 & 正徳 & しょうとく & 1711年 \\ \cline{1-6}

享保 & かんぽほ \hfill\break
& 1716年 & 元文 & げんぶん & 1736年 \\ \cline{1-6}

寛保 & きょう & 1741年 & 延享 & えんきょう & 1744年 \\ \cline{1-6}

寛延 & かんえん & 1748年 & 宝暦 & ほうれき & 1751年 \\ \cline{1-6}

明和 & めいわ & 1764年 & 安永 & あんえい & 1772年 \\ \cline{1-6}

天明 & てんめい & 1781年 & 寛政 & かんせい & 1789年 \\ \cline{1-6}

享和 & きょうわ & 1801年 & 文化 & ぶんか & 1804年 \\ \cline{1-6}

文政 & ぶんせい & 1818年 & 天保 & てんぽう & 1830年 \\ \cline{1-6}

弘化 & こうか & 1844年 & 嘉永 & かえい & 1848年 \\ \cline{1-6}

安政 & あんせい & 1854年 & 万延 & まんえん & 1860年 \\ \cline{1-6}

文久 & ぶんきゅう & 1861年 & 元治 & げんじ & 1864年 \\ \cline{1-6}

慶応 & けいおう & 1865年 \hfill\break
& 明治 \hfill\break
& めいじ \hfill\break
& 1868年 \hfill\break
\\ \cline{1-6}

大正 \hfill\break
& たいしょう \hfill\break
& 1912年 \hfill\break
& 昭和 \hfill\break
& しょうわ \hfill\break
& 1926年 \\ \cline{1-6}

平成 & へいせい & 1989年 &  &  &  \\ \cline{1-6}

\end{ltabulary}
        
\section{Lunar Months}
 
\par{ The original Japanese calendar was based off the Chinese lunar calendar, which starts 3 to 7 weeks later than the Western calendar. The traditional names of these months are still popular in poetry, names, etc. You \textbf{don't have to memorize them or how to write them. } }

\begin{ltabulary}{|P|P|P|P|}
\hline 

First month \hfill\break
& 睦月 & むつき & The month of affection \\ \cline{1-4}

Second month \hfill\break
& 如月・ 衣更着 & きさらぎ・きぬさらぎ & The month to change clothes \\ \cline{1-4}

Third month \hfill\break
& 弥生 & やよい & The month of new life \\ \cline{1-4}

Fourth month \hfill\break
& 卯月 & うづき & The month full of flowers \\ \cline{1-4}

Fifth month \hfill\break
& 皐月・早苗月 & さつき & The month to plant rice \\ \cline{1-4}

Sixth month \hfill\break
& 水無月 & みなつき & The month of water. \\ \cline{1-4}

Seventh month \hfill\break
& 文月 & ふみづき & The month of books \\ \cline{1-4}

Eighth month \hfill\break
& 葉月 & はづき & The month of leaves \\ \cline{1-4}

Ninth month \hfill\break
& 長月 & ながつき & The long month \\ \cline{1-4}

Tenth month \hfill\break
& 神無月 & かんなづき & The month of the gods \\ \cline{1-4}

Eleventh month \hfill\break
& 霜月 & しもつき & The month of frost \\ \cline{1-4}

Twelfth month & 師走 & しわす & The month where priests run \\ \cline{1-4}

\end{ltabulary}
    