    
\chapter{The Particle つ}

\begin{center}
\begin{Large}
第344課: The Particle つ 
\end{Large}
\end{center}
 
\par{ The particle つ has been greatly reduced to hardly nothing in Modern Japanese, but it holds on it certain cases that this lesson will discuss. }
      
\section{The Case Particle つ}
 
\par{ The \textbf{attributive }case particle つ follows nominals to indicate the possessive and survives in few words such as 目毛・睫(まつげ) "eyelash". If you pay attention really closely to how this is actually used, you'll see that this particle seems to have a locative function as well. }

\par{1. 天津神 \hfill\break
Heavenly gods }

\par{2. ${\overset{\textnormal{たき}}{\text{滝}}}$ つ ${\overset{\textnormal{せ}}{\text{瀬}}}$ とは川の流れの激しいところだ。 \hfill\break
A takitsuse is a place where the current of a river is violent. }

\par{3. 天津日嗣 \hfill\break
Imperial throne }

\par{4. ${\overset{\textnormal{あま}}{\text{天}}}$ つ風 \hfill\break
Winds of heaven }

\par{5. ${\overset{\textnormal{あきついり}}{\text{秋入梅}}}$ \hfill\break
 Long fall rain }

\par{6a. 夏の ${\overset{\textnormal{すえ}}{\text{末}}}$ つ ${\overset{\textnormal{かた}}{\text{方}}}$ \hfill\break
6b. 夏の終わり (Modern) \hfill\break
The end of summer }

\par{7. 上枝・秀つ枝   (A 雅語) \hfill\break
Upper branch }

\par{\textbf{Word Note }: This last example is read as ほつえ. The opposite is 下枝, which although is normally read as したえだ should be read as しずえ as the true antonym. The ず is ultimately the particle つ. }

\par{8. おとつい \hfill\break
The day before yesterday }

\par{\textbf{Word Note }: In more standard Japanese, the last word becomes おととい. おとつい comes from a sound change of the Classical Japanese phrase ${\overset{\textnormal{をち}}{\text{遠}}}$ つ ${\overset{\textnormal{ひ}}{\text{日}}}$ . }

\par{9. しかも ${\overset{\textnormal{あまつひ}}{\text{天日}}}$ のごとく若くかがやかしく、悩みと憂いが兆しかけた眉は ${\overset{\textnormal{りり}}{\text{凛々}}}$ しさを加え、 ${\overset{\textnormal{とよあしはらのなかつくに}}{\text{豊葦原中国}}}$ にまた見ること叶わぬような美しい ${\overset{\textnormal{わかもの}}{\text{壮夫}}}$ であった。 \hfill\break
Moreover, he was a beautiful, splendid man that she could never see with the likes of Toyoashihara no Nakatsukuni, though he was young and bright like the heavenly sun in addition to the chivalry in his eyebrows which foretold sorrow and grief. \hfill\break
From 獅子・孔雀 by 三島由紀夫. }

\par{\textbf{Word Note }: 中国 is another example of this particle although it's in a name. }

\par{10. どのような ${\overset{\textnormal{まがつひ}}{\text{禍津日}}}$ も、わたくし共の恋の樹を枯らす力は ${\overset{\textnormal{も}}{\text{有}}}$ ちませぬ。 \hfill\break
No calamitous day could have the power to cut down our tree of love. \hfill\break
From 獅子・孔雀 by 三島由紀夫. }
      
\section{The Conjunctive Particle つ}
 
\par{ The conjunctive particle つ shows two repetitive actions. It is normally only seen in set expressions. }

\par{11a. 彼は行きつ戻りつしながら待った。 \hfill\break
11b. 彼は行き来しながら待った。(More modern) \hfill\break
He waited while walking to and fro. \hfill\break
}

\par{12. 矯めつ眇めつ眺める。 \hfill\break
To take a good look at something. \hfill\break
}

\par{13a. 彼はとつおいつ思案する嫌いがある。 \hfill\break
13b. 彼はあれやこれや思案する嫌いがある。(More common) \hfill\break
He has the tendency to ponder over this and that. }

\par{14. 夢のなかではいつも父と追いつ追われつして、親雄が殺されるか父が殺されるかした。 \hfill\break
In Chikao's dreams it was always a cat chase with his father, and it was either him or his father being killed. \hfill\break
From 獅子 by 三島由紀夫. }

\par{15. 浮きつ沈みつ \hfill\break
Floating up and down }

\par{16. 御米は依然として、のつそつ床の中で動いていた。 \hfill\break
Oyone was still stretching and crouching in her bed. \hfill\break
From 門 by 夏目漱石. }

\par{\textbf{Derivation Note }: のつそっつ is a contraction of のっつそっつ, which is a contraction of のりつ反りつ. }

\par{17. 組んず解れつ \hfill\break
Locked in a grapple }

\par{\textbf{Derivation Note }: The previous expression comes from a sound change of 組みつ解れつ. }

\begin{center}
\textbf{The Similar ~み~み }
\end{center}

\par{ ~み can also be seen in the pattern "連用形 of verb + ~み + verb with ~ず + ~み" to show two actions or conditions were being repeated in alternation. This, too, is an \textbf{archaism }. }
 
\par{18. 降りみ降らずみ (古語) \hfill\break
From raining to not raining }

\par{ ~み can also be seen in the pattern "連用形 of verb + ~み + verb with ~ず + ~み" to show two actions or conditions were being repeated in alternation. This, too, is an  \textbf{archaism }. }

\par{24. 降りみ降らずみ (古語) \hfill\break
From raining to not raining }
    