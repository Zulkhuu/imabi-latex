    
\chapter{Miss out}

\begin{center}
\begin{Large}
第316課: Miss out: ~残す, ~漏らす, ~損なう, ~そびれる, ~損じる, \& ~逃す 
\end{Large}
\end{center}
 
\par{ In this lesson, we will learn about even more compound verb endings. As the title indicates, all of these endings are in reference to missing out on doing something. }
      
\section{~残す}
 
\par{ ${\overset{\textnormal{のこ}}{\text{残}}}$ す means "to save\slash leave". In compound verbs it builds upon this meaning to show that "something important was not done before leaving". 残る, the intransitive form, can similarly appear with some words to imply a sense of "being left" in a certain state. }

\par{1. その仕事はやり残したんです。 \hfill\break
I left the work undone. }

\par{2. ソーダを残しておいて。 \hfill\break
Save me some soda. }

\par{3. 動物の数え方は食べ残る部分からきたと思われる。 \hfill\break
It is believed that the method of counting animals comes from the parts left over after eating (them). }

\par{4.  食べ残りを排水口に捨てない方がいいと聞いたんですが、本当ですか。 \hfill\break
I heard that it is best not to put scraps in the disposal, but is this true? }

\par{5. お客さんが食べ残しを持ち帰って体調を崩した場合、飲食店はどのような責任があるのですか。 \hfill\break
In the case a customer brings home left overs and gets ill, does the food establishment have any responsibility? }

\par{\textbf{Word Note }: 食べ残り and 食べ残し both exist, but the former refers to scraps that aren't suitable for eating later. The latter is "leftovers". As a society, Japanese people don't like for there to be leftovers. Businesses are weary of customers getting sick from leftovers and customers feel a responsibility not to order more than they can eat. Nonetheless, more places in Japan are beginning to add guidelines as to how to make sure leftovers can be taken home safely. }

\par{6. 食べ残したご飯を翌日に食べるようにすればよいのでは? \hfill\break
Wouldn't it be good to try to eat left over meals the next day? }

\par{7. 食べ残して捨てるのはもったいないです。 \hfill\break
Throwing away what you have left over from eating is wasteful. }

\par{8. 飲み残したワインを味を落とさないで保存する方法はありますか。 \hfill\break
How do you preserve left over wine without losing its taste? }

\par{9. 耐乏生活に生き残る。 \hfill\break
To live through austerity. }
      
\section{~漏らす}
 
\par{  ${\overset{\textnormal{も}}{\text{漏}}}$ らす means "to leak" and is in compound verbs to show that one ends up not doing something important due to carelessness. }

\par{10. 必要なことを言い漏らす。 \hfill\break
To end without saying something necessary. }

\par{11. うっかりして ${\overset{\textnormal{ねんがっぴ}}{\text{年月日}}}$ を書き漏らした。 \hfill\break
I carelessly forgot to write down the date. }

\par{12. 大事なところを聞き漏らした。 \hfill\break
I failed to listen to a crucial part. }

\par{\textbf{Contrast Note }: The last word is not exactly the same as 聞き逃す, which implies that you didn't listen to the entire thing. }
      
\section{~損なう・~そびれる・~損じる・~逃す}
 
\par{  ${\overset{\textnormal{そこ}}{\text{損}}}$ なう means "to spoil\slash ruin" and what happens is always the subject's fault. In compounds it shows what one \textbf{missed out on doing }, and it may be interchangeable with the suffix ~そびれる. Another similar ending is ~ ${\overset{\textnormal{そん}}{\text{損}}}$ じる which may show that one misses out on or fails in doing something. ~ ${\overset{\textnormal{のが}}{\text{逃}}}$ す is also possible. }

\par{\textbf{漢字 Note }: 損なう is often left in かな. }
 
\par{13. ボールを ${\overset{\textnormal{と}}{\text{捕}}}$ り ${\overset{\textnormal{そこ}}{\text{損}}}$ なう。 \hfill\break
To miss the ball. }
 
\par{${\overset{\textnormal{}}{\text{14a. 彼}}}$ の ${\overset{\textnormal{}}{\text{言}}}$ ったことを ${\overset{\textnormal{}}{\text{聞}}}$ き ${\overset{\textnormal{}}{\text{損}}}$ なった。 \hfill\break
14b. 彼の言ったことを聞きそびれた。(More natural) \hfill\break
I missed out on what he said. }
 
\par{\textbf{Variant Note }: 聞きそびれた would have been a better choice in the sentence above. This is because 聞きそこなう can also mean "mishear". }
 
\par{${\overset{\textnormal{}}{\text{15a. 健康}}}$ を ${\overset{\textnormal{そこな}}{\text{害}}}$ った。 \hfill\break
15b. 健康を損なった。(Normal spelling) \hfill\break
15c. 健康を害した。(Alternative phrase) \hfill\break
He ruined his health. }

\par{16. ${\overset{\textnormal{せ}}{\text{急}}}$ いては ${\overset{\textnormal{こと}}{\text{事}}}$ を ${\overset{\textnormal{しそん}}{\text{仕損}}}$ じる。(ことわざ) \hfill\break
Haste makes waste. \hfill\break
Literally: Rushing fails things. }
 
\par{\textbf{Orthography Note }: 仕損じる means "to blunder". It is often spelled as し損じる, but it can also be seldom spelled as 為損じる.  為る is actually how you would spell する as in "to do" in 漢字. The normal spelling 仕損じる is actually ${\overset{\textnormal{}}{\text{当}}}$ て ${\overset{\textnormal{}}{\text{字}}}$ . }
 
\par{${\overset{\textnormal{}}{\text{17. 手紙}}}$ を ${\overset{\textnormal{}}{\text{書}}}$ き ${\overset{\textnormal{}}{\text{損}}}$ じた。 \hfill\break
I made a mistake in writing the letter. }
 
\par{${\overset{\textnormal{}}{\text{18. 言}}}$ い ${\overset{\textnormal{}}{\text{損}}}$ ずる。 (Old-fashioned) \hfill\break
To miss out saying. }
 
\par{\textbf{Word Note }: The following is normally 言い ${\overset{\textnormal{}}{\text{損}}}$ なう or 言い ${\overset{\textnormal{}}{\text{損}}}$ ねる. }
 
\par{\textbf{Variant Note }: Remember that ~じる and ~ずる are variants of each other and that both are voiced forms of する. ~ ずる is typically rarer and more formal. }
 
\par{${\overset{\textnormal{}}{\text{19a. 遊}}}$ ぶ ${\overset{\textnormal{}}{\text{機会}}}$ を\{ ${\overset{\textnormal{}}{\text{逸}}}$ した・逃してしまった\}のは ${\overset{\textnormal{}}{\text{残念}}}$ です。(Most natural) \hfill\break
19b. 遊びそこねたのは残念です。(OK) \hfill\break
It's a shame that he missed out on playing. }
 
\par{${\overset{\textnormal{}}{\text{20. 死}}}$ に ${\overset{\textnormal{}}{\text{損}}}$ なって生き恥をさらす。 \hfill\break
To fail to die and live on with one's shame exposed. }

\par{21. ${\overset{\textnormal{ねぼう}}{\text{寝坊}}}$ したとき、学校に来る前に何をしそびれましたか。 \hfill\break
What did you not do when you came to school after having overslept? }

\par{22. ${\overset{\textnormal{えんぜつ}}{\text{演説}}}$ を ${\overset{\textnormal{}}{\text{聞}}}$ き ${\overset{\textnormal{のが}}{\text{逃}}}$ す。 \hfill\break
To fail to listen to the speech. }
 
\par{${\overset{\textnormal{}}{\text{23. 見逃}}}$ してしまうよ。 \hfill\break
You\textquotesingle re missing it\slash you're going to miss it! }
 
\par{\textbf{Orthography Note }: For the meaning of "ruin", 損なう may also be written as 害う. However, because this is not a reading in the 常用漢字表, unversed speakers will not know how it is read and assume it is an error. As such, it is best to only recognize it for when it does show up in literature. }

\par{\textbf{Speech Style Note }: ~損ずる is the same as ~損じる etymologically, but the latter is slightly less literary but still unlikely to be used in the spoken language. }
    