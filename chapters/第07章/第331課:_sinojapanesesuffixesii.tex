    
\chapter{Sino-Japanese Suffixes II}

\begin{center}
\begin{Large}
第331課: Sino-Japanese Suffixes II 
\end{Large}
\end{center}
 
\par{ Adding more suffixes to your vocabulary will greatly help your vocabulary skills. }
      
\section{More Suffixes}
 
\par{-城(じょう) = Castle }

\par{1. 姫路城は兵庫県にあります。 \hfill\break
Himeji Castle is in Hyogo Prefecture. }

\par{-状(じょう) = State; condition }

\par{2. 液状 \hfill\break
Liquid state }

\par{-上(じょう) = "from\dothyp{}\dothyp{}\dothyp{}point of view". At times it may also mean "on" as in on something. }

\par{3a. iPad上でFlash Playerが作動できません。 \hfill\break
3b. iPadではFlash Playerが作動できません。 \hfill\break
Flash Player doesn't work on the iPad. }

\par{4. 立場上言えない。 \hfill\break
I cannot say in relation to the situation. }

\par{5. 歴史上 \hfill\break
From a historical point }

\par{6. ${\overset{\textnormal{いんせき}}{\text{隕石}}}$ が落ちたせいで地球上の色んな動物が絶滅した。 \hfill\break
Many animals have gone extinct on the Earth due to meteorites. }

\par{7. 健康上の理由で、彼が退職するという旨の連絡をしました。 \hfill\break
On reasons in terms of health, he gave a message to the effect that he'll resign. }

\par{8. 法律上 \hfill\break
In the eyes of the law }

\par{9. 実際(上) \hfill\break
In practice }

\par{10a. 構文上 \hfill\break
10b. 文の構成上 \hfill\break
As a matter of sentence structure }

\par{-性(せい) = -Ness }

\par{11. 安全性 \hfill\break
Safeness }

\par{-製(せい) = Made in }

\par{12. 日本製の車は安全ですよ。 \hfill\break
Japanese made cars are safe. }

\par{-制(せい) = System (of order) }

\par{13. 私は交替制で働いています。 \hfill\break
I'm working in a shift. }

\par{-世(せい) = Era; epoch }

\par{14. エリサベス2世はアイルランドをご訪問なさいました。 \hfill\break
Elizabeth II visited Ireland. }

\par{15. 洪積世にある大陸の形は現在とほぼ同じである。 \hfill\break
The shape of the continents were about the same in the Diluvial Epoch. }

\par{-星(せい) = Star }

\par{16. 南十字星 \hfill\break
The 4 brightest stars of the Southern Cross. \hfill\break
}

\par{-前(ぜん) = Before. Stiff and literary. }

\begin{ltabulary}{|P|P|P|P|}
\hline 

紀元前 & B.C. & 使用前に & Before usage \\ \cline{1-4}

\end{ltabulary}

\par{-然(ぜん) = -like. It is very uncommon. }

\par{17. 学生然 \hfill\break
Student-like }

\par{-千万(せんばん) = Exceeding }

\begin{ltabulary}{|P|P|P|P|}
\hline 

無礼千万 & Extremely rude & 笑止千万 & Quite absurd \\ \cline{1-4}

\end{ltabulary}

\par{-層(そう) = Layer; stratum }

\par{18. 溶岩層を見たことがありますか。 \hfill\break
Have you seen a lava bed? }

\par{-著(ちょ) = Written by }

\par{19. この本は宮部みゆき著です。 \hfill\break
This book is written by Miyabe Miyuki. }

\par{-庁(ちょう) = Office; agency }

\begin{ltabulary}{|P|P|P|P|}
\hline 

官公庁 & Public offices & 金融庁 & Financial Service Agency \\ \cline{1-4}

\end{ltabulary}

\par{-調(ちょう) = Meter; style }

\par{20. 七五調 \hfill\break
Seven-five meter }

\par{-艇(てい) = Boat }

\par{21. 魚雷艇 \hfill\break
Torpedo boat }

\par{-的 = -ic. Makes things 形容動詞. Whether to use it or not is a hard question. For instance, 基本の and 基本的な are both possible, but the latter is more abstract. Whether something should be stated more concretely and abstractly is going to differ in context and among speakers. What makes this more confusing is that the absence of な with 的 is not always certain, but it usually happens in jargon and or long formal phrases. It is also important to note that 的の is old-fashioned, and if you read anything earlier than World War II, you will encounter it. }

\par{22. 水深は比較的に浅い。 \hfill\break
The water level is relatively shallow. }

\par{23. 私的に \hfill\break
Personally }

\par{\textbf{Speech Note }: The above phrase is not proper, but it is becoming increasingly more common. }

\par{24. 経済的な自動車 \hfill\break
An economical vehicle }

\par{25. 「行くわ」というと、女性的に聞こえる。 \hfill\break
If you say "iku wa", it sounds feminine. }

\par{-都(と) = Capital prefecture }

\par{26. 東京都 \hfill\break
Tokyo Prefecture }

\par{-党(とう) = Political party }

\par{27. 民主党の調査 \hfill\break
A Japanese Democratic Party investigation }

\par{-島(とう) = Island }

\par{28. 硫黄島(いおうとう) \hfill\break
Iwo Jima }

\par{-塔(とう) = Tower }

\par{29. テレビ塔 \hfill\break
Television tower }

\par{-堂(どう) = In the names of important buildings such as temples, shrines, the legislature, etc. }

\par{30. 国会議事堂は立法府の建物だ。 \hfill\break
The Diet Building is the building of the legislature. }

\par{-道(どう) = In Hokkaido Prefecture }

\par{31. 北海道 \hfill\break
Hokkaido Prefecture }

\par{-派(は) = Faction }

\par{32. 正統派 \hfill\break
Orthodox faction }

\par{-版(ばん) = Edition }

\par{33. パリ版 \hfill\break
Paris edition }

\par{-犯(はん) = Crime }

\par{34. 殺人犯 \hfill\break
Homicide }

\par{-判(はん) = Product size \hfill\break
\hfill\break
35. 大判(おおはん) \hfill\break
Large size }

\par{-班(はん) = Group; crew }

\par{36. カメラ班 \hfill\break
Camera crew }

\par{-藩(はん) = Clan }

\par{37. 竜野藩 \hfill\break
Tatsuno Clan }

\par{-費(ひ) = Bill\slash expenses }

\begin{ltabulary}{|P|P|P|P|P|P|}
\hline 

医療費 & Medical bill & 材料費 & Material cost & 生活費 & Living costs \\ \cline{1-6}

\end{ltabulary}

\par{-妃(ひ) = Queen }

\par{38. 皇太子妃 \hfill\break
Crowned princess }

\par{-票(ひょう) = Vote }

\par{39. ${\overset{\textnormal{ふどう}}{\text{浮動}}}$ 票じゃないか。 \hfill\break
Isn't it a floating vote? }

\par{40. 反対票を投ずる。 \hfill\break
To give a no vote. }

\par{\textbf{Word Note }: 投ずる is normally 投じる. }

\par{-表(ひょう) = List; table }

\begin{ltabulary}{|P|P|P|P|P|P|}
\hline 

予定表 & Schedule & 通知表 & Report card & 周期表 & Periodic table \\ \cline{1-6}

視力検査表 & Eye chart & 九九表 & Multiplication table & 時刻表 & Time table \\ \cline{1-6}

\end{ltabulary}

\par{-病(びょう) = Sickness }

\begin{ltabulary}{|P|P|P|P|P|P|}
\hline 

高山病 & Altitude sickness & 風土病 & Endemic disease & 放射線病 & Radiation sickness \\ \cline{1-6}

心臓病 & Heart disease & 肺病 & Lung disease & 接触伝染病 & Contagious disease \\ \cline{1-6}

\end{ltabulary}

\par{- 府(ふ) = Metropolitan prefecture (Osaka and Kyoto) }

\par{41. 大阪府 \hfill\break
Osaka Prefecture }

\par{-風(ふう) = Style }

\par{42. フランス風の料理が大好きですよ。 \hfill\break
I love French food. }

\par{-風情(ふぜい) = Worthless person }

\par{43. 私風情にはもったいないでしょう。 \hfill\break
A worthless person like me is unworthy of this, right? }

\par{-補(ほ) = Assistant }

\par{44. 書記官補 \hfill\break
Assistant secretary }

\par{-法(ほう) = Law }

\begin{ltabulary}{|P|P|P|P|P|P|}
\hline 

労働法 & Labor law & 国際法 & International law & 交通法 & Traffic law \\ \cline{1-6}

行政法 & Administrative law & 禁酒法 & Prohibition & 連邦法 & Federal law \\ \cline{1-6}

\end{ltabulary}

\par{-魔(ま) = Freak }

\par{45. インターネット魔 \hfill\break
Internet freak }

\par{-放題(ほうだい) = "a state left to continue as is" or "\dothyp{}\dothyp{}\dothyp{}as much as you want". }

\par{46. 2000円で食べ放題です。 \hfill\break
Eat as much as you want for 2,000 yen. }

\par{${\overset{\textnormal{}}{\text{47. 荒}}}$ れ放題の庭だね。 \hfill\break
That's an overgrown garden with weeds, isn't it? }

\par{-余(よ) = Over }

\par{48. 日本に二ヶ月余、滞在するつもりです。 \hfill\break
I plan to stay in Japan for over 2 months. }

\par{-楼(ろう) = Tower }

\par{${\overset{\textnormal{}}{\text{49. 摩天}}}$ 楼 \hfill\break
Skyscraper }

\par{-領(りょう) = Territory }

\par{50. アメリカ領 \hfill\break
American territory }

\par{-料(りょう) = Fee }

\begin{ltabulary}{|P|P|P|P|P|P|}
\hline 

原稿料 & Contribution fee & サービス料 & Service fee & 通行料 & Toll \\ \cline{1-6}

診察料 & Doctor fee & 保険料 & Insurance premium & 授業料 & Tuition fee \\ \cline{1-6}

\end{ltabulary}

\par{-量(りょう) = Amount }

\par{51. 読書量が多い。 \hfill\break
To read a lot. }

\par{52. 降雨量 \hfill\break
Rainfall }

\par{-力(りょく) = Power }

\par{53. 原子力委員会 \hfill\break
The Atomic Energy Commission }

\par{\textbf{Word Note }: -会 is another suffix meaning "meeting; group; commission". }

\par{-類(るい) = Kind }

\begin{ltabulary}{|P|P|P|P|}
\hline 

菌類 & Fungus & 地衣類 & Lichens \\ \cline{1-4}

\end{ltabulary}
 \hfill\break
    