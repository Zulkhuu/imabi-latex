    
\chapter{Etymology}

\begin{center}
\begin{Large}
第349課: Etymology 
\end{Large}
\end{center}
 
\par{ Most words in Japanese come from one or more of three sources. }
      
\section{大和言葉(やまとことば)}
 
\par{ Native words define the language itself. Regardless of the percentage, native words are used extensively. They're very similar looking throughout the Japonic language family. }

\begin{ltabulary}{|P|P|P|P|}
\hline 

Meaning & Standard & Okinawan & Amami \\ \cline{1-4}

Chest & M üne & N'ni & Munï \\ \cline{1-4}

Blood \hfill\break
&  Tɕi &  Tɕi ː &  Tʃi ː \\ \cline{1-4}

Fog & Ki ɾi &  Tɕiɾi & Ki ɺ̠ i \\ \cline{1-4}

Rain & Äme &  ʔ a mi & ʔamï \\ \cline{1-4}

\end{ltabulary}

\par{\textbf{IPA\slash Curriculum Note }: Tʃ is more similar to an English "ch". ɺ̠ is a retroflex lateral flap. }

\par{Native words suggest that Japanese is an isolate. Some words, though, are similar to Korean. In fact, many deemed native are actually ancient borrowings from Korean and elsewhere. Words like 馬 and 梅 came from China before 漢字. カササギ (magpie) is from Old Korean. クズリ comes from Nivkh. Lots of words come from Ainu. When the 弥生 (ancestors of the Japanese people) came to Japan, they met people called 蝦夷(えみし・えぞ). There are no records about these languages, but their descendants are the Ainu. In some Northern Japanese dialects such as 気仙, significant vocabulary comes from them. Many words considered native may be borrowed words from extinct 蝦夷 languages. }

\par{As for Modern Japanese native words, most are polysyllabic and CV (consonant-vowel). Japanese grammar for the most part doesn't intermingle items of different etymologies; however, some exceptions exist. For example, する follows Sino-Japanese words to make them verbs. な is added to Sino-Japanese abstract nouns to make them adjectives, and る is often added to words to create colloquial verbs. }

\par{\textbf{Word Note }: Surnames for the most part are native in origin, but given names are a different story. As for writing, native words are generally written in 漢字 or ひらがな. }
      
\section{漢語}
 
\par{  漢語 (Sino-Japanese words) were borrowed from Chinese readings from several stages of Chinese. Nearly 70\% of all Japanese words are 漢語, but they make up only 20\% of the most frequently used words. 音読み is the basis of deciding whether a word is of Chinese origin; however, be cautious of 当て字. There are rare exceptions where a 訓読み is actually of Chinese origin: 馬 and 梅. }

\par{\textbf{和製漢語 }}

\par{和製漢語 literally means "Japanese-made Chinese words". At the start of modern innovation, many terms were created with 音読み to describe new things. }

\begin{ltabulary}{|P|P|P|}
\hline 

Environment & 自然 & しぜん \\ \cline{1-3}

Telephone & 電話 & でんわ \\ \cline{1-3}

Science & 科学 & かがく \\ \cline{1-3}

Dermis & 皮膚 & ひふ \\ \cline{1-3}

\end{ltabulary}

\par{Some Sino-Japanese words made in Japan even correspond to unique Japanese items. }

\begin{ltabulary}{|P|P|P|P|P|P|}
\hline 

Manga & 漫画 & まんが & Judo & 柔道 & じゅうどう \\ \cline{1-6}

Tokyo & 東京 & とうきょう & Japanese paper \hfill\break
& 和紙 & わし \\ \cline{1-6}

Public bath \hfill\break
& 銭湯 & せんとう & Go (game) \hfill\break
& 碁 & ご \\ \cline{1-6}

Powdered green tea \hfill\break
& 抹茶 & まっちゃ & Kendo & 剣道 & けんどう \\ \cline{1-6}

Geisha & 芸者 & げいしゃ & Daimyo & 大名 & だいみょう \\ \cline{1-6}

Dojo & 道場 & どうじょう & Haiku & 俳句 & はいく \\ \cline{1-6}

Shintoism & 神道 & しんとう & Shogun & 将軍 & しょうぐん \\ \cline{1-6}

\end{ltabulary}

\par{Many Sino-Japanese words are the product of putting characters in a Chinese word order and using 音読み to make a more technical term. Of course, there are always the words that are half Sino-Japanese and half native-- 湯桶読み and 重箱読み . }

\begin{ltabulary}{|P|P|P|P|P|}
\hline 

Meaning & Native &  & Sino-Japanese &  \\ \cline{1-5}

Reply & 返り事 & かえりごと & 返事 & へんじ \\ \cline{1-5}

To get angry \hfill\break
& 腹が立つ & はらがたつ & 立腹 & りっぷく \\ \cline{1-5}

Ninja & 忍び者 & しのびもの & 忍者 & にんじゃ \\ \cline{1-5}

Fire breakout \hfill\break
& 火が出る & ひがでる & 出火 & しゅっか \\ \cline{1-5}

\end{ltabulary}
      
\section{外来語}
 
\par{ Foreign loanwords are from modern languages. Preference to use these words is somewhat debatable.  Some have no native equivalent. Some have  been borrowed because the possible Japanese term is too complicated. For the most part, these words are nouns, but with the aid of grammatical items such as する, な, and る, they may be turned into verbs, adjectives, etc. }

\par{At times words have mixed etymology. Foreign expressions in Japanese are often abbreviated. This is so common with English terms that it creates " 和製英語  " or Japanese-made English words. These words have diverged from their original appearance, meaning, or both. At times 外来語 are  made in Japan to describe a unique thing and are then brought back to  the original language as a new phrase. The best example of this is  anime. }

\par{Common languages where 外来語 may arrive include French, Spanish, Portuguese, German, Dutch, and several primarily Indo-European languages. There are even some 外来語 from modern Chinese dialects. }

\par{外来語 are primarily written in カタカナ. There are some older loanwords, though, that may be written in 漢字. More traditional approximations are common. So, many speakers will still call a smart phone a スマートホン rather than a スマートフォン. But, there is a movement in electronic terminology to prefer newer combinations. Nowadays, though, you are more likely to hear the abbreviated form スマホ. }

\par{To some speakers, loan words are undesirable. In the realm of technology, words have been coined with normal Japanese processes and borrowing. For example, function keys used to be called 機能キー. Now they're called ファンクションキー. Many speakers would rather use 機能 rather than ファンクション. }

\par{\textbf{Common 外来語 }}

\begin{ltabulary}{|P|P|P|P|}
\hline 

外来語 & Origin & Meaning & Original Language \\ \cline{1-4}

アベック & Avec & Romantic couple \hfill\break
& French \\ \cline{1-4}

アイドル & Idol & Pop star \hfill\break
& English \\ \cline{1-4}

アイスクリーム & Ice cream & Ice cream & English \\ \cline{1-4}

アイゼン & Steigeisen & Climbing irons \hfill\break
& German \\ \cline{1-4}

アニメ & Animation & Anime & English \\ \cline{1-4}

アンケート & Enquête & Questionnaire & French \\ \cline{1-4}

アンニュイ & Ennui & Ennui & French \\ \cline{1-4}

アップ & Upgrade & Upgrade & English \\ \cline{1-4}

アロエ & Aloë & Aloe & Dutch \\ \cline{1-4}

アルバイト & Arbeit & Part-time job \hfill\break
& German \\ \cline{1-4}

アルコール & Alcohol\slash álcool & Alcohol & Dutch\slash Portuguese \\ \cline{1-4}

バイク & Bike & Motorcycle & English \\ \cline{1-4}

バスケ & Basketball & Basketball & English \\ \cline{1-4}

バター & Butter & Butter & English \\ \cline{1-4}

ビル & Building & Modern steel building \hfill\break
& English \\ \cline{1-4}

ビール・麦酒 & Bier & Beer & Dutch \\ \cline{1-4}

ボンベ & Bombe & Steel canister & German \\ \cline{1-4}

ボールペン & Ballpoint pen & Ballpoint pen \hfill\break
& English \\ \cline{1-4}

ボタン & Botão & Button & Portuguese \\ \cline{1-4}

ブランコ & Balanço & Swing & Portuguese \\ \cline{1-4}

チンキ & Tinktuur & Tincture & Dutch \\ \cline{1-4}

ダブル & Double & Double & English \\ \cline{1-4}

デパート & Department store \hfill\break
& Department store \hfill\break
& English \\ \cline{1-4}

ドンマイ & Don't mind \hfill\break
& Don't worry about it \hfill\break
& English \\ \cline{1-4}

ドライバー & Driver & Screwdriver & English \\ \cline{1-4}

ドラマ & Drama & TV drama \hfill\break
& English \\ \cline{1-4}

エアコン & Air conditioning \hfill\break
& Air conditioning \hfill\break
& English \\ \cline{1-4}

エキス & Extract & Extract & Dutch \\ \cline{1-4}

エネルギ & Energie & Energy & German \\ \cline{1-4}

エレベター & Elevator & Elevator & English \\ \cline{1-4}

エロ & Eros & Erotic & English \\ \cline{1-4}

エステ & Esthétique & Beauty salon \hfill\break
& French \\ \cline{1-4}

ファンファーレ & Fanfare & Music fanfare & German \\ \cline{1-4}

フロント & Front desk \hfill\break
& Front desk \hfill\break
& English \\ \cline{1-4}

ガラス & Glas & Glass & Dutch \\ \cline{1-4}

ガソリンスタンド & Gasoline stand \hfill\break
& Gas station \hfill\break
& English \\ \cline{1-4}

ガーゼ & Gaze & Gauze & German \\ \cline{1-4}

ゲーセン & Game center & Video arcade \hfill\break
& English \\ \cline{1-4}

ゴム & Gom & Rubber & Dutch \\ \cline{1-4}

グラス & Glass & Drinking glass \hfill\break
& English \\ \cline{1-4}

グロ & Grotesque & Grotesque & English \\ \cline{1-4}

ハンカチ & Handkerchief & Handkerchief & English \\ \cline{1-4}

ハンスト & Hunger strike \hfill\break
& Hunger strike \hfill\break
& English \\ \cline{1-4}

ホルモン & Hormon & Hormone & German \\ \cline{1-4}

ホース & Hoos & Hose & Dutch \\ \cline{1-4}

ホーム & Platform & Railway platform \hfill\break
& English \\ \cline{1-4}

イエス & Jesus & Jesus & Portuguese \\ \cline{1-4}

イクラ & икра & Salmon roe \hfill\break
& Russian \\ \cline{1-4}

イメージ & Image & Image & English \\ \cline{1-4}

インフレ & Inflation & Inflation & English \\ \cline{1-4}

イラスト & Illustration & Illustration & English \\ \cline{1-4}

イヤホン & Earphone & Earphone & English \\ \cline{1-4}

ジュース & Juice & Juice & English \\ \cline{1-4}

カメラ & Camera & Camera & English \\ \cline{1-4}

合羽 & Capa de chuva \hfill\break
& Rain coat \hfill\break
& Portuguese \\ \cline{1-4}

カラン & Kraan & Faucet & Dutch \\ \cline{1-4}

カラオケ & 空 + orchestra \hfill\break
& Karaoke & Japanese and English \hfill\break
\\ \cline{1-4}

コーヒー・珈琲 & Koffie & Coffee & Dutch \\ \cline{1-4}

コップ & Copo & A glass \hfill\break
& Portuguese \\ \cline{1-4}

クラブ・倶楽部 & Club & Club & English \\ \cline{1-4}

マージャン・麻雀 & 麻雀 & Mahjong & Mandarin Chinese \hfill\break
\\ \cline{1-4}

マンション & Mansion & Condominium block \hfill\break
& English \\ \cline{1-4}

マスコミ & Mass communication \hfill\break
& The media \hfill\break
& English \\ \cline{1-4}

メール & Mail & E-mail & English \\ \cline{1-4}

ミルク & Milk & Milk & English \\ \cline{1-4}

モテル & Motel & Motel & English \\ \cline{1-4}

ノルマ & норма & Quota & Russian \\ \cline{1-4}

ノート & Note & Notebook & English \\ \cline{1-4}

OL & Office lady \hfill\break
& Female office worker \hfill\break
& English \hfill\break
\\ \cline{1-4}

パン & Pão\slash pan & Bread & Portuguese\slash Spanish \\ \cline{1-4}

ペスト & Pest & Black Plague & German \\ \cline{1-4}

ピエロ & Pierrot & Clown & French \\ \cline{1-4}

プロ & Professional & Professional & English \\ \cline{1-4}

プロレス & Professional wrestling & Professional wrestling \hfill\break
& English \hfill\break
\\ \cline{1-4}

ライバル & Rival & Rival enemy & Englis \\ \cline{1-4}

ラッコ & Rakko & Sea otter \hfill\break
& Ainu \\ \cline{1-4}

レントゲン & Röntgen & X-ray & German \\ \cline{1-4}

リモコン & Remote control \hfill\break
& Remote control \hfill\break
& English \\ \cline{1-4}

リンク & Link & Link & English \\ \cline{1-4}

ロードショー & Road show \hfill\break
& Premiere & English \\ \cline{1-4}

ロマン & Roman & Novel & French \\ \cline{1-4}

リュックサック & Rucksack & Backpack & German \\ \cline{1-4}

サービス & Service & Service & English \\ \cline{1-4}

サボる & Sabotage & To slack off \hfill\break
& French \\ \cline{1-4}

サンドイッチ & Sandwich & Sandwich & English \\ \cline{1-4}

サラダ & Salada & Salad & Portuguese \\ \cline{1-4}

セレブ & Celebrity & Rich person & English \\ \cline{1-4}

刹那 & Ksana & Moment & Sanskrit \\ \cline{1-4}

ソフト & Software & Software & English \\ \cline{1-4}

ストーブ & Stove & Space heater \hfill\break
& English \\ \cline{1-4}

タバコ・煙草 & Tabaco & Tobacco & Portuguese \\ \cline{1-4}

テーマ & Thema & Theme & German \\ \cline{1-4}

テレビ & Television & Television & English \\ \cline{1-4}

トイレ & Toilet & Toilet & English \\ \cline{1-4}

トナカイ & Tunakkay & Reindeer & Ainu \\ \cline{1-4}

トランプ & Trump & Playing cards \hfill\break
& English \\ \cline{1-4}

ワンピース & One piece \hfill\break
& Single piece dress \hfill\break
& English \\ \cline{1-4}

ワープロ & Word processor \hfill\break
& Word processor \hfill\break
& English \\ \cline{1-4}

ヨード & Jod & Iodine & German \\ \cline{1-4}

ズボン & Jupon & Pants & French \\ \cline{1-4}

\end{ltabulary}

\par{マンションを ${\overset{\textnormal{}}{\text{探}}}$ す。 \hfill\break
To search for an apartment. }
 
\par{\textbf{Vocab Note }: There are two words for apartment, アパート and マンション. An アパート is more like a 1~2 story building with a relatively cheaper rent whereas a マンション would be at least 3 stories. }
 
\par{ペンキを ${\overset{\textnormal{ぬ}}{\text{塗}}}$ る。 \hfill\break
To paint. }
 
\par{\textbf{Noun Note }: 塗る is paired with ペンキ. 塗る can also mean "to varnish; color; rub". }
 
\par{\hfill\break
「パン」という語はポルタゴル語から来た。 \hfill\break
The word "pan" came from Portuguese. }

\begin{center}
 \textbf{Chinese Loan Words }
\end{center}

\par{\textbf{}  Newer readings from Chinese keep coming in from modern Chinese languages. Many words involve food and culture items. }

\par{皮蛋(ピータン)はガチョウの卵を灰で包んだ保存食である。 \hfill\break
Pi dan is a preserved food of goose egg wrapped in ash. }

\par{搾菜(ザーサイ)は中国料理の漬物です。 \hfill\break
Szechuan pickles is a pickled food in Chinese cuisine. }

\par{ギョーザはとてもおいしいですね。 \hfill\break
Gyoza is really delicious, isn't it? }

\begin{ltabulary}{|P|P|P|}
\hline 

Loan & 漢字 &  \\ \cline{1-3}

メンマ & 麺媽 & Bamboo shoots boiled, sliced, fermented, dried in salt, and soaked 
\\ \cline{1-3}

ラー油 & 辣油 & Chinese red chili oil \\ \cline{1-3}

リーチ & 立直 & Declaring that one is one tile away from winning in mahjong \\ \cline{1-3}

ワンタン & 雲呑 & Wonton \\ \cline{1-3}

\end{ltabulary}
     