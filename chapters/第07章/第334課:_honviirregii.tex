    
\chapter{Honorific Speech VI}

\begin{center}
\begin{Large}
第334課: Honorific Speech VI: Irregular Verbs II 
\end{Large}
\end{center}
 
\par{ There are several exceptional honorific verbs that have more than one usage. This lesson will focus on such verbs. This lesson covers mostly any meaning that you may find in a dictionary. Dictionaries aren't always up-to-date with things, but usages you'll find here go from samurai-like old-fashioned speech to very common speech. So, there is definitely a lot you can learn about these words. It's just that not all of the information will be practical in conversing in 敬語. }
      
\section{参る}
 
\par{\textbf{The Intransitive Verb 参る \hfill\break
}}

\par{1. The humble verb form of 行く and 来る, the speaker is generally oneself or someone in one's in-group. 参る is humble, but so is 伺う. The difference between the two is that 伺う should be used when going to the place of the addressee. }

\par{1. 今すぐ参ります。 \hfill\break
I will come at once. }

\par{2. 喜んで参ります。 \hfill\break
I would be glad to come. }

\par{2. A polite version of 行く and 来る. Notice how it is extended to items. }

\par{3. 絶好の機会が参りました。 \hfill\break
The ideal chance has come. \hfill\break
}

\par{4. 電車が参ります。 \hfill\break
The train will come. \hfill\break
}

\par{5. 時間があれば参ります。 \hfill\break
If I have time, I will come. \hfill\break
}

\par{3. In the 命令形 primarily in the speech of samurai, it is a boastful equivalent of 行く or 来る. Or, it may be a solemn variant of 行く or 来る where it is also acceptable in honorific speech as 参られる. }

\par{6. 近う参れ。(Classical) \hfill\break
Come close. }

\par{4. To make homage to a shrine. }

\par{7. (お)墓にお参りをする。 \hfill\break
To visit a grave. }

\par{8. (お)寺へお参りに行きます。 \hfill\break
I am going to go worship at the temple. \hfill\break
}

\par{5. To lose or surrender. }

\par{9. まいった。  (Not so common anymore) \hfill\break
I'm beat. \hfill\break
}

\par{10. そう簡単には参らぬぞ。(Old-fashioned) \hfill\break
I won't surrender so easily. \hfill\break
}

\par{11. 参ったか。  (Anime-like; somewhat old) \hfill\break
Do you give up? }

\par{6. To be bewildered. }

\par{12. 彼女の行儀の悪さには参りました。 \hfill\break
I was embarrassed at her awful manners. \hfill\break
 }

\par{7. To be worn down from hardship in mind and body. }

\par{13. まいった体 \hfill\break
A worn down body }

\par{14. こう忙しくて神経がまいった。 \hfill\break
I was busy like this and my nerves wore out. }

\par{8. To die. }

\par{15. とうとうあいつも参った。(Old-fashioned) \hfill\break
Even he finally died. \hfill\break
 }

\par{9. To have one's heart stolen by the opposite sex. }

\par{16. 彼女の美しさにすっかり参ったのだな。 \hfill\break
I was completely had by her beauty, wasn't I? \hfill\break
}

\par{10. To suffer from. }

\par{17. 風邪でまいっている。 (ちょっと古い言い方) \hfill\break
風で苦しい。(More natural) \hfill\break
I'm suffering from a cold. \hfill\break
}

\par{18. 俺ん足痛ぇまいっとるで。(Dialectical) \hfill\break
I'm suffering from a hurt leg. \hfill\break
}

\par{\textbf{The Transitive Verb 参る }\hfill\break
}

\par{It is used at the end of a formal letter to a give a meaning of giving the letter to a superior. Or it is used to mean "to service" or "to present to". These meanings are very rare, old-fashioned, and warrior-like phrases. So, they are essentially \textbf{no longer used }. }

\par{19. 御神酒を参る。    (文語的) \hfill\break
To serve sacred wine. }

\par{20. 母上様参る    (手紙の脇付にする、古めかしい語) \hfill\break
Letter presented to thy mother \hfill\break
}

\par{\textbf{The Supplementary Verb 参る }}

\par{1. "\dothyp{}\dothyp{}\dothyp{}ていく・くる" in humble speech. }

\par{21. さまざまな体験をしてまいりました。 \hfill\break
I have come to experience various things. \hfill\break
}

\par{2. "\dothyp{}\dothyp{}\dothyp{}ていく・くる" in polite speech. }

\par{22. 雨が降ってまいりました。 \hfill\break
The rain came. }

\par{23. 日増しに暑くなってまいりました。 \hfill\break
It has been becoming hotter by the day. \hfill\break
}

\par{3. In samurai speech in the 命令形 as a boastful word in "\dothyp{}\dothyp{}\dothyp{}ていく・くる" or as a solemn expression. }

\par{24. 一度こちらへ連れて参れ。 \hfill\break
Come lead to here once. }
      
\section{伺う}
 
\par{ 伺う is either the humble form of the verbs "to listen\slash ask", "to talk to guests" or "to visit\slash to come". Remember that 伺う is more polite than 参る and should only be used in contexts referring to visiting people and not things. }

\par{25. よろしかったら明日伺います。 \hfill\break
I will come tomorrow if it is alright with you. }

\par{26. 二、三の質問を伺ってもよろしいですか。 \hfill\break
May I ask you a couple of questions? }

\par{27. 近々渡米される由伺いました。 \hfill\break
I heard that you're going to America soon. }

\par{28. 早速お話をお伺いしましょう。 \hfill\break
I shall listen to your story promptly. }

\par{27. 暑中お伺い申し上げます。 \hfill\break
I am inquiring about your health in the summer season. }

\par{28. 伝言を伺いましょうか。 \hfill\break
Shall I take a message? }

\par{29. 御用をお伺いしていますか。 \hfill\break
Are you being waited on? }

\par{30. お宅へお伺いします。 \hfill\break
I'm going to come to your house. }
      
\section{仰る}
 
\par{ 仰る, a contraction of 仰せられる, is the respectful form of the verb いう and may also show what someone's name is. 仰られる is to be avoided, but 言われる is too direct for most honorific situations. 仰せられる is the most polite form of this word. 仰る is sometimes used by females in the 命令形, 仰い. }

\par{31. 早く仰い。(やや女性っぽくてあまり使われていない) \hfill\break
Hurry and say it. }

\par{32. 礼ですが、お名前は何と仰いますか。 \hfill\break
Excuse me, but what is your name? }

\par{33. 先生は何と仰っているのですか。 \hfill\break
What is the teacher saying? }

\par{34. 仰る通りだと存じております。 \hfill\break
I think exactly what you say. }

\par{35. 総理大臣はこう仰せられました。 \hfill\break
The Prime Minister said so. }
      
\section{申す}
 
\par{ 申す is the humble form of 言う and the more polite version is 申し上(あ)げる. However, 申し上げる is not normally used when you are giving respect to those involved in something. }

\par{36. 鈴木と申します。 \hfill\break
I am Suzuki. }
37. 初めまして、私は春木と申します。 \hfill\break
Nice to meet you, my name is Haruki. 
\par{38. 皆様に心からお礼を申し上げます。 \hfill\break
I give gratitude to all of you from my heart. }

\par{39. 申し上げておきますが \hfill\break
For your information }

\par{40. ここは銀座と申します。 (Might hear in some play) \hfill\break
This is called Ginza. }

\par{ 申し上げる may be a supplementary verb after a noun with the prefixes お- or ご- to make verbs humble. 申す may also do this, but it isn't as humble. }

\par{41. お待ちしております。(一番自然) \hfill\break
お待ち申しております。 (あまり使われていない) \hfill\break
I am waiting. }

\par{42. お詫び申し上げます。 \hfill\break
I must apologize. }

\par{43. お悔やみ申し上げます。 \hfill\break
Please accept my condolences. }

\par{44a. ご相伴申します。  (やや古い言い方) \hfill\break
44b. お相伴申します。  (古い言い方) \hfill\break
44c. お相伴与ります。  (やや珍しい言い方) \hfill\break
I shall be partaking. }

\par{ It is also possible to see 申す used in boastful and neutral yet polite context. These sentences are old-fashioned and should not be applied into your speech. }

\par{45. 名を申せ。(Old-fashioned) \hfill\break
Say your name. }

\par{46. そう申しておらん。(Old) \hfill\break
I'm not saying such\slash that.  }

\par{ It is also possible in older texts (古文) to see it as 申される. }

\par{47. おぬし、今何と申されたか。(Classical) \hfill\break
What have you said? }
      
\section{いただく}
 
\par{  頂く, aside from honorifics, may mean "to place on top of". It can also be used figuratively to show a title being placed upon someone, but this isn't really common. }

\par{48. 彼は頭に霜を頂いた。 \hfill\break
His head is covered in frost. \hfill\break
Idiomatic Expression: Refers to people who with age begin to have white hair. }

\par{49. 星を頂いて帰る。 \hfill\break
To return home with stars on top. \hfill\break
Idiomatic Expression: Particularly used to refer when people come home from work in the night. }

\par{50a. 王冠を頂く。(Figurative) \hfill\break
50b. 王冠を被る。(Normal) \hfill\break
To wear a crown. }

\par{51. 白く雪を頂いた山頂を眺める。 \hfill\break
To gaze at mountain peaks covered in snow. }

\par{52. この展望台から、白く雪を頂いた大山が一望できます。 \hfill\break
You can see all of the snow-covered peak of Mt. Daisen from this observatory. }

\par{53. 遠くに雪を頂いた富士山が見えました。 \hfill\break
I was able to see snow-covered Mt. Fuji from afar. }

\par{54. 〇氏を名誉会長に頂く。 (あまり言わない) \hfill\break
To have someone as the honorary president. }

\par{\textbf{もらう }}

\par{いただく shows  favor. It may be used with お- and ご-,  which gives an added sense of regular   politeness. いただく may also be used to show that one receives something of profitable value without effort. -ていただく is used to show favor and is seen in the patterns ~ていただく, お- verb in 連用形 + いただく, or ご- 漢語 noun + いただく. }
 
\par{55. お暇を頂く。 \hfill\break
To receive free time. }

\par{56. ご高配を頂く。 \hfill\break
To receive good offices. }

\par{57. 手伝っていただけませんか。 \hfill\break
Could you please help me? }

\par{58. 私の事はご心配頂かなくて結構です。 \hfill\break
It is alright for you to not trouble over (me). }

\par{ 賜る, however, is much more polite than いただく. However, it is also rarer and sometimes old-fashioned. }

\par{59. 結構なお品を賜り、ありがとうございます。 \hfill\break
I am so grateful you bestowed us this fine good. }

\par{ 頂戴する is possible in honorifics in limited situations. }

\par{60. ご意見を頂戴し、ありがとうございます。 \hfill\break
We are thankful for your opinions. }

\par{~ させていただく is used  to show that one is to be allowed the favor to do something. }

\par{61. 閉会をさせていただきます。 \hfill\break
Allow me to have the privilege of closing (this event). }

\par{\textbf{食べる, 飲む, \& 風呂に入る } }

\par{いただく is often used as the humble form of 食べる and 飲む. Being used to mean 風呂に入る is not so common, but it is still possible. }

\par{62. いただきましょうか。 \hfill\break
Shall we eat\slash drink? }

\par{63. では、お先に頂きます。 \hfill\break
Alright, then I'll be first (to eat\slash drink\slash get in tub). }
      
\section{致す}
 
\par{ Of course, not all of these meanings are common or used frequently. 致す is typically used to mean する in humble contexts, but it also shows up in many set expressions. }

\par{64. 何に致しましょうか。 \hfill\break
What can I do for you. }

\par{65. お手伝いいたしましょうか。 \hfill\break
Shall I help you? }

\par{66. 私の不徳の致すところです。 \hfill\break
It is entirely my fault. }

\par{\textbf{Sentence Note }: Ex. 66 is not meant to be something you would just randomly say. Rather, this is very ceremonious and likely to be used in a very formal situation where elegant use of honorifics is expected of you. }

\par{67. 祖国に思いを致す。 \hfill\break
To sympathize with one's native country. }

\par{68. 力を致す。 \hfill\break
To exert one's strength to the maximum. }

\par{69. 謹んでお詫び致します。 \hfill\break
謹んでお詫び申し上げます。(Even more common) \hfill\break
I humbly apologize. }

\par{70. 何をぐずぐずいたしておるのか。(古風で尊大な言い方) \hfill\break
What are you complaining about? }

\par{71. あと十分致しますと別府に着きます。 \hfill\break
We will arrive in Beppu in another ten minutes. }

\par{72. よろしくお願い致します。 \hfill\break
Please treat me well. }

\par{73. 危うきを見て命を致す。 (Proverb\slash saying) \hfill\break
To give one's life in a crisis. }

\par{74. 参上いたせ。  (古風で尊大な言い方) \hfill\break
Call on him! }

\par{\textbf{Sentence Note }: Such a usage of this verb as seen in Ex. 74 for boastful purposes is essentially 武家言葉. }
      
\section{差し上げる}
 
\par{  差し上げる is the honorific form of the giving verbs あげる and やる and is used as a supplementary verb as well. It also has another meaning of "to lift". Although in honorifics it is humble, there are other aspects to Japanese culture that might actually make it rude. This is the case in Ex. 77a below. }

\par{75. 高々と差し上げる。 \hfill\break
To lift up very high. }

\par{76. お茶も差し上げずに失礼いたしました。 \hfill\break
I have been very rude for not having given you tea. }

\par{77a. 何か飲み物でも差し上げましょうか。  (ちょっと失礼) \hfill\break
77b. 何かお飲物はいかがですか。 \hfill\break
77c. 何かお飲みになりますか。 \hfill\break
Can I get you anything to drink? }

\par{78. 先着100名様にプレゼントを差し上げます。 \hfill\break
We will give a prize to the first 100 people. }

\par{79. ご案内して差し上げなさい。 \hfill\break
Please guide him. }
      
\section{給う}
 
\par{ 給う, also pronounced and spelled as たもう, is an  old honorific verb meaning "to give". As a supplementary verb, it  attaches to the 連用形 to show respect to superiors. In the 命令形, though, it creates a harsh command. }

\par{80. 神様の与えたもうた試練 \hfill\break
A trial sent by God }

\par{81. 止め給え。 (Older guy to inferior\slash purposely old-fashioned) \hfill\break
Stop it! }

\par{82. 黙ってくれたまえ。 (Older guy to inferior\slash purposely old-fashioned) \hfill\break
Shut up! }

\par{83. お褒めのお言葉を給う。(古風) \hfill\break
To give words of praise. }
      
\section{召す}
 
\par{  召す  is the honorific\slash respectful form of many different verbs as the chart shows. }

\begin{ltabulary}{|P|P|}
\hline 

Transitive & 買う; 呼び寄せる; 風呂を使う; 食う・飲む; 着る・履く; 年をとる; 風邪を引く \\ \cline{1-2}

Intransitive & 乗る; 気に入る \\ \cline{1-2}

\end{ltabulary}

\begin{center}
 \textbf{Examples }
\end{center}

\par{84. お気に召しましたか。 \hfill\break
Are you pleased with it? }

\par{85. お風呂を召す。 \hfill\break
To use the bathtub. \hfill\break
}
 
\par{86. 天国へ召される。 \hfill\break
To go to heaven\slash to be called to heaven. \hfill\break
}
 
\par{\textbf{Word Note }: 召される is a slightly more polite version of 召す and may replace なさる. }

\par{87. 花を召しませ。 \hfill\break
Buy flowers. }

\par{88. 浴衣をお召しになる。 \hfill\break
To wear a yukata. }

\par{\textbf{Cultural Note }: A 浴衣 is a thin kimono worn in the summertime. }

\par{89. ご酒はお召しになりますか。 \hfill\break
Will you be drinking alcohol? \hfill\break
}

\par{\textbf{Word Note }: Notice how お酒 is replaced by ご酒(しゅ). }

\par{90. 彼は風邪を召されました。 \hfill\break
He caught a cold. }

\par{91. 馬にお召しになる。 \hfill\break
To ride a horse. }
    