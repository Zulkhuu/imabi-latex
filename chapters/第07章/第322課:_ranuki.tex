    
\chapter{ら抜き言葉}

\begin{center}
\begin{Large}
第322課: ら抜き言葉  
\end{Large}
\end{center}
 
\par{ ら抜き is t he dropping of ら in the auxiliary verb ~られる when used to show potential. Although it is deemed improper by many speakers, it is important to point out that the independent potential verb forms of 五段 did not exist as ‘standard\textquotesingle  Japanese 150 years ago. This lesson will be about how ら抜き is being used in the current speech of a growing majority. }
      
\section{ら抜き言葉って?}
 
\par{ ら抜き, again, is the dropping of ら in the auxiliary  ~られる. The ease of conjugation is quite obvious. ~れる attaches the exact same way as ~られる does. }

\begin{ltabulary}{|P|P|P|P|}
\hline 

Verb Class & Base & Conjugation Example & 例文 \\ \cline{1-4}

 上一段活用動詞 & 未然形 &  見る+れる \textrightarrow  見れる &  テレビ\{が・を\}見れる \\ \cline{1-4}

 下一段活用動詞 & 未然形 &  食べる +れる \textrightarrow  食べれる &  ピザ\{が・を\}食べれる  \\ \cline{1-4}

\end{ltabulary}

\par{ Grammar is a living entity defined by current speakers. It is not surprising that this innovation came about because ~られる has three other usages! It is also used to make passives, phrases of spontaneity, and light honorific phrases. These things are complicated enough to require separate lessons. }

\par{ One could say that the grammar is just too different for each usage to ever confuse them. For instance, in addition to ら抜き being bad, others will also point out that the use of を with potential phrases is traditionally incorrect. Thus, if people were using traditional, correct grammar, one would never confuse the potential in particular with any of the other three usages because of the use of が to mark the object. There is a reason for this particle choice, but we will treat が and を here as being interchangeable as the number of speakers who do not distinguish them at all in this situation is increasing. }

\begin{center}
 \textbf{Mistaking Words? }
\end{center}

\par{ Because ~れる only has the potential meaning, there is assurance that you are only purveying the potential meaning. Furthermore, if the phenomenon is so well known that speakers refer to words with it as ら抜き言葉, there is no chance that someone won\textquotesingle t understand you. That is not say that there are no examples of once separate phrases now being homophonous. For example, 切れる (to be able to cut) and 着れる (to be able to wear) may not even be distinguished by intonation in some reasons. This problem, though, can be viewed as very rare. For the sake of bringing this sort of problem up, there is also 寝れる VS 練れる, and 送れる VS 遅れる.  }

\par{1. 古い服を着れた! \hfill\break
I was able to wear my old clothes! }

\par{2. ハサミでよく切れた! \hfill\break
I was able to cut it well with scissors! }

\par{3. あの人は、ある程度練れた文章を書けました。 \hfill\break
That person was able to write to a certain degree a well-rounded composition. }

\par{4. 寝れない夜を過ごしてる。 \hfill\break
I'm spending sleepless nights. }

\par{5. とりあえず何とか完成して送れた。 \hfill\break
For the time being, I somehow finished it and was able to send it.  }

\par{6. ひどい交通渋滞で遅れた。 \hfill\break
I was late due to horrible congestion. }

\par{\textbf{Word Form Note }: Ex. 5 and 6 show us that this talk of confusion is void. 送れる is not ら抜き and is the proper potential form of 送る. No speaker is ever going to confuse it with 遅れる in context. }

\par{7. 大雨が降ってこんなとこに居れるわけない。 \hfill\break
There's no way I'll be able to stay here in this rain. }

\par{8. サクランボを容器に入れる。 \hfill\break
To put cherries in a container. }

\par{ \textbf{More Info on ら抜き言葉 }}

\par{ In formal writing, ら抜き言葉 are almost non-existent, but in day-to-day writing, it is appearing more frequently. However, the fact that people typically don\textquotesingle t like characters in books featuring these words is rather interesting. }

\par{ With the percentage of the population which doesn't use ら抜き言葉 at all being a small minority, and with the existence of politicians and the like using it frequently on the air, one can easily imagine this solidifying as “new correct Japanese” in the near future. Change is often hated when in progress. So, for this reason, an actual news anchor or journalist will never use it unless quoting the words of an interviewee. Subtitles and tel-ops often reinsert ら if missing in speech, which is perhaps the media\textquotesingle s way of reversing the change, though they are clearly losing the battle with such tactics. It is also a fact that ら抜き言葉 are readily type-able, which will only facilitate the spread of this change. }

\begin{center}
 \textbf{Treating it as Dialectical? }
\end{center}

\par{ What about the dialects in which ら抜き is the standard? They are probably under influence from regions which say it\textquotesingle s backward, but because it is the standard in these regions and it is becoming standard elsewhere, it\textquotesingle s hard to imagine any of these regions reverting to the current ‘standard\textquotesingle . }

\begin{center}
 \textbf{Frequency of Use }
\end{center}

\par{ Some verbs will be more commonly used with ~れる than others. For instance, in casual speech, one may easily say 見れる, 来れる, 食べれる, しゃべれる and the like. It may, though, be the case that you are not prone to extending this to any verb including compounds (for instance, saying 出かけれる). }

\par{9. いつまで続けれるのか。 \hfill\break
How long can\dothyp{}\dothyp{}\dothyp{}continue? }

\par{10. なかなか準備し始めれないのだ。 \hfill\break
I can't seem to start preparing. }

\par{ \textbf{Speech Style }}

\par{ We've touched upon register just now. Register refers to speech style in social context. Say you are the boss of a company and wish to enforce ‘proper language\textquotesingle  habits on your employees. This is a very understandable situation in modern Japan, and in honorifics, new colloquialisms are not easily accepted, and in the case of honorifics, potential phrases are avoided to begin with. Some people go so far as to say ~れます instead of ~られます in casual, polite speech is an abomination. Continue to see more examples of this as years go by. Some claim that people are beginning to use this in honorifics, but this is hard to believe. It is more likely the person intended to use the potential form to his boss, which is a マナー違反 to begin with. }

\par{11. 何時に来れますか。 \hfill\break
When can you come? }

\begin{center}
\textbf{Relation with other Auxiliaries }
\end{center}

\par{ It is important to note that ら抜き would not be used in 共通語 with older (usages of) auxiliaries. }

\par{12. 見られよう \textrightarrow  見れよう X  }

\par{ In speaking of other auxiliaries, people very frequently use the ~よう conjugation to correct people. So, you may find people saying, if you use ~よう then you should use ~られる. Another example is using the 命令形. If you drop る from the potential and don\textquotesingle t get a valid command form, then you can\textquotesingle t use ~れる. As 五段 verbs would only work, 見れる is invalid because 見れ is invalid. This opinion is cute, but there are dialects which allow for the imperatives of 一段 verbs to end in れ. So, for certain regions of Japan, this analogy would not work. }

\begin{center}
 \textbf{Odd Dialectical Phenomena }
\end{center}

\par{ In some dialects such as 広島弁, corruptions such as 行けれる can be found. This resembles the slang phenomenon by a minority of speakers to take ~られる and attach it to the 連用形 of the 短縮形 (shortened form) of the potential. Ex. 出す \textrightarrow  出せられる. This, though, like 行けれる are deemed to be errors by most Japanese speakers. }

\begin{center}
\textbf{More Examples }
\end{center}

\par{13. 早く ${\overset{\textnormal{お}}{\text{起}}}$ きれねーよ。(Vulgar) \hfill\break
I can't wake up early! }

\par{14. お金払わないと、得れないもんだ。 \hfill\break
It's something you can't get without paying money. }

\par{15. ハイしか答えれないんだ。 \hfill\break
All I can response with is "ハイ”. }

\par{16. こうして食べると野菜もたくさん食べれるね。 \hfill\break
If you eat this way, you can eat a lot of vegetables. }

\par{17. いつもよりたくさん食べれた。 \hfill\break
I was able to eat a lot more than usual. }

\par{18. (Aさんは)きのうの雨で熱が出て来れないようだ。 \hfill\break
(A-san) doesn't seem to be able to come because he got a fever from the rain yesterday. }

\par{19. 出かけれそうにない。 \hfill\break
It doesn't seem I have a chance at going out. }

\par{20. うまくボールを蹴れない。 \hfill\break
I can't kick the ball well. }

\par{21. あと一歩で教室から出れるところで、ドアが突然閉まった。 \hfill\break
Just when I was one step away from being able to leave the classroom, the door suddenly shut. }
    