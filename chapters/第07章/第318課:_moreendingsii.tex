    
\chapter{Endings VII}

\begin{center}
\begin{Large}
第318課: Endings VII: ~返る・す, ~捲る, ~殴る, \& ~腐る, \& ~通す 
\end{Large}
\end{center}
 
\par{ Some of these endings are a little difficult, so be sure to study hard. }
      
\section{~返る \& ~返す}
 
\par{  返る means "to return" and is used in compounds to mean "to become completely\slash extremely". The transitive form of 返す is used after the 連用形 of verbs to show that one does something again starting from the very beginning. }

\par{1. 観客はすぐに静まり返った。 \hfill\break
The audience automatically fell over. }

\par{2. ${\overset{\textnormal{さ}}{\text{冴}}}$ え返った光を見よ。 (ちょっと古風) \hfill\break
Look at the light that has become completely clear. }

\par{3. ${\overset{\textnormal{しょげかえ}}{\text{悄気返}}}$ った彼女は自殺してしまった。 \hfill\break
She committed suicide completely disheartened. }
 
\par{4. 思い返して ${\overset{\textnormal{だんねん}}{\text{断念}}}$ した。 \hfill\break
I reconsidered it and abandoned the plan. }
 
\par{5. もう一度繰り返して下さいませんか。(とても丁寧) \hfill\break
Could you repeat it one more time? }

\par{6. 彼は ${\overset{\textnormal{む}}{\text{噎}}}$ せ返り続けた。 \hfill\break
He continued to sob convulsively. }
 
\par{\textbf{漢字 Note }: 噎 and 悄 are 表外字. }
      
\section{~殴る \& ~腐る}
 
\par{ ~殴る is used to show that one is doing something rather violently. ~腐る can be used to show disgust in what someone is doing. }

\par{7. 記事を書き殴る。 \hfill\break
To rapidly write down an article. }

\par{8. ぶち殴る。 \hfill\break
To beat hard. }

\par{9. つまらんことばっかり言い腐ってるやつだな、あんた! (Vulgar) \hfill\break
You're just a guy that just says useless things! }
      
\section{~捲(まく)る}
 
\par{ 捲る means "to roll up". In compounds it shows something recklessly done without an end in sight. }

\par{10. 高校生の時にマンガを読みまくりました。 \hfill\break
When I was a high school student, I would do nothing but read manga. }

\par{11. 最近パソコンを使いまくってるよ。 \hfill\break
I've recently been doing nothing but using the computer. }

\par{12. コーラを飲みまくるのは健康のために悪いですよ。 \hfill\break
Drinking nothing but cola is bad for your health! }

\par{13. しゃべりまくるのはむかつくよね。 \hfill\break
Talking on and on is annoying, isn't it? }

\par{14. 食べまくるのは人間の性質の短所である。 \hfill\break
Eating greedily is a fault in human nature. }

\par{15. 電車の中で高校生がしゃべりまくっているのを聞いていた。 \hfill\break
I listened to the high school students talked their jaws off on the train. }

\par{16. 部屋は荒れまくり。 \hfill\break
~My room is just a mess. }
      
\section{~通す}
 
\par{ 通す, following the 連用形 of a verb,  it means  "to continue to do\dothyp{}\dothyp{}\dothyp{}until the end". Below is a chart of some  common  verbs utilizing it. \hfill\break
}

\begin{ltabulary}{|P|P|}
\hline 

To persist in & 押し通す \\ \cline{1-2}

To continue to read until the end & 読み通す \\ \cline{1-2}

To carry through & 遣り通す \\ \cline{1-2}

To sing until the end & 歌い通す \\ \cline{1-2}

\end{ltabulary}

\par{\textbf{Spelling Note }:   透 may be used instead of 通 when showing the   transmitting of something such as light. In a more physical sense, 徹 may be used. }
    