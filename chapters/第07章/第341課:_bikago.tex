    
\chapter{美化語}

\begin{center}
\begin{Large}
第341課: 美化語 
\end{Large}
\end{center}
 
\par{  ${\overset{\textnormal{びかご}}{\text{美化語}}}$ is the embellishment of words. 美化 means "beautification". 美化語, at its basic understanding, is making one's speech sound more refined. 美化語 elevates one's speech to a well-mannered style. Grammatically, it is neither ${\overset{\textnormal{ていねいご}}{\text{丁寧語}}}$ "polite speech" nor 敬語. However, it applies to 敬語 but is usually classified as 丁寧語. Several 美化語 phrases have come from traditional women's speech. }
 
\par{The honorific お- is generally added to words regardless of origin. Even foreign loanwords such as ズボン are used with お- in 美化語. Words like 勉強 can either be seen with お- or ご-. This is sure to cause more confusion. For most words, though, the general お- and ご- distinctions are maintained. }
      
\section{お- VS ご-}
 
\par{ お- is used with native words and 和製語 (Sino-Japanese words made in Japan such as telephone). ご- is used only with Sino-Japanese words not used with お-. み-, おみ-, and おんみ- are used in words with religious or imperial importance. }

\begin{ltabulary}{|P|P|P|P|}
\hline 

お庭 & Garden & お天気 & Weather \hfill\break
\\ \cline{1-4}

お飲み物 & Drinks & お菓子 & Sweets \\ \cline{1-4}

お食事 & Meal & お店 & Store \\ \cline{1-4}

お料理 & Cooking & ご祝儀 & Congratulatory gift \\ \cline{1-4}

ご機嫌 & Mood & ご挨拶 & Greeting \hfill\break
\\ \cline{1-4}

ご結婚 & Marriage & ご連絡 & Contact \hfill\break
\\ \cline{1-4}

\end{ltabulary}

\par{When an honorific prefix is used, it may greatly alter the word. }

\begin{ltabulary}{|P|P|}
\hline 

飯(めし)⇒ご飯(はん) & Rice; meal \\ \cline{1-2}

汁(しる)⇒お汁(つゆ) & Soup; broth; dipping sauce* \\ \cline{1-2}

水⇒お冷 & Cold water \\ \cline{1-2}

うまい⇒おいしい & Delicious \\ \cline{1-2}

便所⇒お手洗い & Lavatory \\ \cline{1-2}

\end{ltabulary}

\par{*: This definition is specific to お汁. }

\par{Note: Other forms of 美化語 include anything that euphemizes a phrase. Another word for this is 雅語. Since they are hard to pinpoint, they will not be discussed. Dictionaries will label 雅語 with 雅. }
      
\section{The Relationship between 美化語 and 丁寧語}
 
\par{  丁寧語 utilizes more polite nouns rather than honorific prefixes. In women's speech, however, honorific prefixes are commonly attached to everyday words. In more respectful language, this gender restriction disappears and sentence patterns conform. }
 
\par{1. お ${\overset{\textnormal{}}{\text{水}}}$ を ${\overset{\textnormal{}}{\text{飲}}}$ んでください。(美化語) \hfill\break
Please drink water. }
 
\par{2a. お ${\overset{\textnormal{}}{\text{肉}}}$ をお ${\overset{\textnormal{}}{\text{食}}}$ べにならないほうがよろしいと ${\overset{\textnormal{}}{\text{思}}}$ いますわ。(美化語) \hfill\break
2b. お肉をお食べにならないほうがよろしいかと思われます。(敬語) \hfill\break
2c. お肉をお食べにならないほうがよろしいでしょう。(敬語) \hfill\break
I think that it best not to eat meat. }
 
\par{3. ご協力を頂くために \hfill\break
In order to receive your (honorable) cooperation (敬語) }
 
\par{If you overuse 美化語, you may sound sarcastic and pretentious. Some words like おトイレ, おコーヒー, おタバコ may be seen in other settings and in the speech of older women. No single description can make a good generalization of when and when not to do this. }
 
\par{It is rare to see loan words take honorifics, but when they do they normally take お-, but some like おビール and おソース are more common. Also, it is imperative to use these suffixes in reference to things during tea ceremonies. }

\par{\textbf{美化語 from Women's Speech }}

\par{Lots of words with お- used to only be used by women but are now used by everyone. }

\begin{ltabulary}{|P|P|P|P|}
\hline 

おかか & Finely chopped katsuobushi. &  お欠き & Thin, dried, baked mochi \hfill\break
\\ \cline{1-4}

お菜(かず) & Side dish & おから & Tofu residue \\ \cline{1-4}

お強(こわ) &  Glutinous rice steamed with red beans & おじや & Rice gruel \\ \cline{1-4}

おつけ & Soups & おでん & Oden \\ \cline{1-4}

お腹(なか) & Stomach & おなら & Fart \\ \cline{1-4}

お萩(はぎ) & Rice ball coated with red beans & おまる & Potty \\ \cline{1-4}

\end{ltabulary}
      
\section{The Necessity of 美化語}
 
\par{Some words are almost always accompanied with honorific prefixes while some are always used with them. To find out whether a noun has to have an honorific prefix, you should first see if a word can be looked up in a dictionary without the prefix. If a word is 美化語-sensitive, most dictionaries will tell you. }

\begin{ltabulary}{|P|P|P|P|P|}
\hline 

Definition & Word & 仮名 & 美化語 & Always or Usually? \\ \cline{1-5}

Hot water & 湯 & ゆ & お湯 & Usually \\ \cline{1-5}

Tea & 茶 & お茶 & お茶 & Usually \\ \cline{1-5}

Rice\slash meal & 飯 & はん & ご飯 & Always \\ \cline{1-5}

Ghost & 化け & ばけ & お化け & Always \\ \cline{1-5}

Flattery & 世辞 & せじ & お世辞 & Always \\ \cline{1-5}

Stomach & 腹 & はら & お腹* & Always \\ \cline{1-5}

\end{ltabulary}

\par{*: This word must have お- so it can be read as おなか. }

\par{4. お ${\overset{\textnormal{}}{\text{湯}}}$ がこのボイラーから ${\overset{\textnormal{}}{\text{出}}}$ ます。 \hfill\break
The hot water comes out with this boiler. }
 
\par{5. その古い(お) ${\overset{\textnormal{しろ}}{\text{城}}}$ にはお ${\overset{\textnormal{ば}}{\text{化}}}$ けが ${\overset{\textnormal{}}{\text{出}}}$ るらしい。 \hfill\break
It sounds like ghosts appear in that old castle. }
 
\par{6. お ${\overset{\textnormal{せじ}}{\text{世辞}}}$ を ${\overset{\textnormal{}}{\text{言}}}$ う。 \hfill\break
To give a compliment. }

\par{7. その(お)知らせを受け取らなかった。 \hfill\break
I didn't receive that notice. }
 
\par{8. お ${\overset{\textnormal{せち}}{\text{節}}}$ \hfill\break
Osechi (a food served during New Year's) }
 
\par{9. お ${\overset{\textnormal{せっかい}}{\text{節介}}}$ はよせ。 \hfill\break
Mind your own business. }
 
\par{10. お腹が ${\overset{\textnormal{}}{\text{痛}}}$ い。 \hfill\break
My stomach hurts. }

\par{11. ${\overset{\textnormal{はら}}{\text{腹}}}$ に ${\overset{\textnormal{いちげき}}{\text{一撃}}}$ を ${\overset{\textnormal{く}}{\text{食}}}$ らう。 \hfill\break
To receive a blow in the stomach. }

\par{12. ${\overset{\textnormal{にゅうよくご}}{\text{入浴後}}}$ はお ${\overset{\textnormal{ゆ}}{\text{湯}}}$ を ${\overset{\textnormal{なが}}{\text{流}}}$ して ${\overset{\textnormal{す}}{\text{捨}}}$ てておいてください。 \hfill\break
You must run the (hot) water off after taking a bath. }
      
\section{美化語 and Change of Meaning}
 
\par{ At times when an honorific prefix is attached to a word there is a change in meaning. There is usually always a reasonable relationship between the two. }

\begin{ltabulary}{|P|P|P|P|}
\hline 

Meaning without お &  & Meaning with お &  \\ \cline{1-4}

Eight & 八つ & Snack \hfill\break
& おやつ \\ \cline{1-4}

Child & 子 & God's son \hfill\break
& み子 \\ \cline{1-4}

Master & 主人 & Someone's husband \hfill\break
& ご主人 \\ \cline{1-4}

Family & 家族 & Someone's family \hfill\break
& ご家族 \\ \cline{1-4}

Fishing & 釣り & Change (coins) \hfill\break
& お釣り \\ \cline{1-4}

To squeeze & 絞る & Hand towel \hfill\break
& お絞り \\ \cline{1-4}

\end{ltabulary}

\par{\textbf{Word Note }: お ${\overset{\textnormal{しぼ}}{\text{絞}}}$ り is an example of where an honorific prefix is added to 連用形 of a verb to refer to an object in relationship to the verb. }
 
\par{13. 8つある。 \hfill\break
To have eight things. }
 
\par{14. おやつを ${\overset{\textnormal{も}}{\text{持}}}$ ってきた? \hfill\break
Did you bring a snack? }

\par{15. バナナはおやつに入りますか。 \hfill\break
Are bananas in the snacks? }

\par{\textbf{Culture Note }: This is a cliche question that came about from a period when most schools did not allow snacks on field trips, and if they did, there was a 300 yen budget. }

\par{16. ${\overset{\textnormal{つ}}{\text{釣}}}$ りに ${\overset{\textnormal{}}{\text{行}}}$ きたい。 \hfill\break
I like to go fishing. }
 
\par{17. お ${\overset{\textnormal{}}{\text{釣}}}$ りは ${\overset{\textnormal{けっこう}}{\text{結構}}}$ です。 \hfill\break
Keep the change. }
 
\par{${\overset{\textnormal{}}{\text{18a. 彼女}}}$ は ${\overset{\textnormal{しゅじん}}{\text{主人}}}$ づらをしている。 \hfill\break
 ${\overset{\textnormal{}}{\text{18b. 彼女}}}$ は ${\overset{\textnormal{えら}}{\text{偉}}}$ ぶっている。(More natural) \hfill\break
18c. 彼女は偉そうに ${\overset{\textnormal{ふるま}}{\text{振舞}}}$ っている。(More natural) \hfill\break
She acts like the boss. }
 
\par{19. ご ${\overset{\textnormal{しゅじん}}{\text{主人}}}$ はお ${\overset{\textnormal{げんき}}{\text{元気}}}$ ですか。 \hfill\break
Is your husband all right? }
    