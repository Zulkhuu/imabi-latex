    
\chapter{Final Endings}

\begin{center}
\begin{Large}
第320課: Final Endings: ~こける, ~さす, ~倦む\slash 倦ねる, ~逸れる, ~成す, \& ~古す 
\end{Large}
\end{center}
 
\par{ These will be the final set of endings that you'll have to go through. }
      
\section{~こける}
 
\par{ こける means "sink\slash collapse in". ~こける means that something continues on for a long time. }

\par{1. 眠りこける。 \hfill\break
To sleep deeply. }

\par{2. 笑いこける。 \hfill\break
To laugh heartily. }
      
\section{~さす}
 
\par{ ~さす either shows that you stop in the middle of something that you have just started or that something is being delayed that has been going on. }
3. 彼は演説の途中で言いさしてしまった。 He accidentally broke off in the middle of his speech. \hfill\break
\hfill\break
4a. 読みさす嫌いがある。 4b. 読むのを途中でやめる嫌いがある。 I have a habit of leaving books half-read. \hfill\break
\hfill\break
5. 歌詞を忘れたから歌いさしなきゃいけなかった。 Since I forgot the lyrics, I had to stop in the middle of singing. 
\par{\textbf{Word Note }: ~さす comes from 止す. }

\begin{center}
 \textbf{途中 }
\end{center}

\par{ 途中 means "during" and is a noun. It can refer to being on the way to somewhere or something not yet completed and is in a state of progress. In the above sentences, it was followed by being stopped mid-way. }

\par{6. 買い物に行く途中で・・・ \hfill\break
In the middle of going to go shopping\dothyp{}\dothyp{}\dothyp{} }

\par{7. 演奏を途中でやめる。 \hfill\break
To end a performance midway. }

\par{8. 来る途中で事故がありました。 \hfill\break
There was an accident on my way. }
      
\section{~倦む・倦ねる}
 
\par{ ~倦む・倦ねる follows the 連用形 and shows that something is too much for somebody or that one is tired of something. The verb and the resultant verbs are intransitive. }

\par{9. 考え倦んだ結果、私は辞職します。 \hfill\break
As my plans haven't shaped out, I will resign. }

\par{10. 待ちあぐむ。 \hfill\break
I can no longer wait. }

\par{11. 攻めあぐねる。 \hfill\break
To lose the fighting initiative. }
      
\section{~逸れる}
 
\par{ 逸れる means "to stray" and -逸れる shows that "one has missed out on something". It is also more emphatic as ~っぱぐれる. }

\par{12. 逸れた羊を捜し求めていた。 \hfill\break
I searched for the stray sheep. \hfill\break
 }

\par{13. 取りはぐれた宝石 \hfill\break
Jewelry missed taking \hfill\break
}

\par{14. 僕はデザートを食いっぱぐれちゃった。 \hfill\break
I ended up missing out on desert. \hfill\break
}

\par{15. 行きはぐれる。 \hfill\break
To miss out on going. }
      
\section{~成す}
 
\par{ 成す has a few related meanings, of these are "to form", "to do", "to accomplish", "to establish", "to give birth to", etc. In compound verbs it shows a meaning of intention and deliberate action. }

\par{16. 織り成した人生模様 \hfill\break
Interwoven facets of life }

\par{17. 思いなしか。 \hfill\break
Maybe it's my imagination? \hfill\break
}

\par{18. 我々は彼を指導者と見なした。 \hfill\break
We looked at him as our leader. }
      
\section{~古す}
 
\par{ 古す means "to wear out" and is attached to the 連用形 to create compound verbs that depict that "time and time again, X loses its newness". Two common verbs that you will see this ending used with are 着る "to wear" and 言う "to say". }

\par{19. 着古して大分擦り切れてた。 \hfill\break
It was worn-out and quite threadbare. }

\par{20. あのおやじー、いつも使い古した言葉をいって使い古した帽子を被ってんのさ。(砕けた) \hfill\break
That old guy's always saying worn-out phrases and is always wearing worn-out hats. }

\par{21. 着古しの上着だな。(Masculine) \hfill\break
That's an old worn-out jacket, isn't it? }

\par{22. ちょっと言い古されてるが、光陰矢のごとしは本当だよ! \hfill\break
It's a little cliche, but time really does fly like an arrow. }

\par{\textbf{Usage Note }: 言い古す is always seen in the 受身形. }
    