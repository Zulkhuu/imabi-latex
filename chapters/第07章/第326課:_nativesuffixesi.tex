    
\chapter{Native Suffixes I}

\begin{center}
\begin{Large}
第326課: Native Suffixes I: Nominal 
\end{Large}
\end{center}
 
\par{ The next couple of lessons will be about suffixes. }
      
\section{Native Nominal Suffixes 1-25}
 
\par{1. 荒し (あらし) means "troll". }

\par{1. 墓荒し \hfill\break
An internet troll }

\par{2. 顔 (がお)shows the "face" or "look" of something. }

\begin{ltabulary}{|P|P|P|P|P|P|}
\hline 

笑い顔 & Smiley face & 得意顔 & Triumphant look & 訳知り顔 & Know-it-all look \\ \cline{1-6}

\end{ltabulary}

\par{3. 掛かり (がかり) attaches to nouns to show that something is dependent on or resembles something. With verbs it means "with these course of events". }

\par{2. 行きがかり上、そうなってしまった。 \hfill\break
After going to this course of events, it became so. }

\par{3. 一人掛かりでやった仕事だ。 \hfill\break
It's only a one-man job. }

\par{4. ${\overset{\textnormal{しばい}}{\text{芝居}}}$ 掛かりな声 \hfill\break
A voice resembling theatre }

\par{5. ${\overset{\textnormal{おやが}}{\text{親掛}}}$ かりの ${\overset{\textnormal{くらい}}{\text{位}}}$ 。 \hfill\break
A position dependent on one's parents. }

\par{6. 神がかり的な能力 \hfill\break
Exceptional ability }

\par{4. 掛け (かけ) attaches to nouns and the 連用形 of verbs. With verbs it shows that something is in the middle of happening\slash being done. With nouns, it means "rack". \hfill\break
 \hfill\break
7. タオル掛け \hfill\break
Towel rack }

\par{8. 壊れかけ \hfill\break
In the middle of collapsing }

\par{5. 掛け (がけ) with nouns means "to put on". With intransitive verbs it shows something is in the middle of doing something. With  一人(ひとり), 二人(ふたり), etc., it shows how many people can sit. With numbers it means "tenths". }

\par{9. ${\overset{\textnormal{たすき}}{\text{襷}}}$ 掛け \hfill\break
Tucking up sleeves }

\par{10. 帰りがけに店に ${\overset{\textnormal{よ}}{\text{寄}}}$ ってきた。 \hfill\break
I came by the store on the way home. }

\par{11. 10人掛けのソファを見たところだよ! \hfill\break
I just saw a sofa that could fit 10 people! }

\par{12. 定価の6掛けで古本を買いました。 \hfill\break
I bought used books 40\% off. }

\par{13. 7人掛けシート \hfill\break
Seven person seat }

\par{\textbf{Culture Note }: In the majority of trains in Japan, seats often sit seven people and three seats separated by doors on both sides of the train is the norm. }

\par{6. がし attaches to the 命令形 of verbs to create phrases that push an opinion. }

\begin{ltabulary}{|P|P|P|P|}
\hline 

これ見よがし。 & Out of display. & 聞こえよがしに言う。 & Talk at someone. \\ \cline{1-4}

\end{ltabulary}

\par{7. ~方(かた) \textrightarrow  Lesson 81. }

\par{8. 柄 (がら) attaches to nouns to show a suitable condition. }

\par{14. 土地柄 \hfill\break
Local color }

\par{15. ${\overset{\textnormal{じせつがら}}{\text{時節柄}}}$ ご ${\overset{\textnormal{じあい}}{\text{自愛}}}$ 下さい。 \hfill\break
Please take care of yourself during these times. }

\par{16. ${\overset{\textnormal{しょうばいがら}}{\text{商売柄}}}$ 朝が早い。 \hfill\break
The morning is early for the nature of this business. }

\par{9. 絡み (がらみ) is used after words expressing age or price to show "about" or "around". And, it's after nouns to show there is a close connection with something. }

\par{17. ${\overset{\textnormal{こうせい}}{\text{厚生}}}$ がらみの国際的な問題です。 \hfill\break
It's an international problem concerning welfare. }

\par{18. バッテリーがらみの問題 \hfill\break
Problems concerning batteries }

\par{19. 1万円がらみの品物だ。 (ほとんど使われていない) \hfill\break
These are goods around 10,000 yen in worth. }

\par{10. 刻み (きざみ) is equivalent to "-毎に" and "-置きに" and means "every\dothyp{}\dothyp{}\dothyp{}". }

\par{20. 五センチ刻み。 \hfill\break
Every 5 centimeters. }

\par{21. 百円刻みに ${\overset{\textnormal{りょうきん}}{\text{料金}}}$ が上がります。 \hfill\break
The fee goes up every 100 yen. }

\par{11. きって attaches to nouns and shows what is the most\dothyp{}\dothyp{}\dothyp{}in something. }

\par{22. 彼は ${\overset{\textnormal{わ}}{\text{我}}}$ が校きっての ${\overset{\textnormal{しゅうさい}}{\text{秀才}}}$ です。 \hfill\break
He is the brightest student in our school. }

\par{23. 当代きっての名優。 \hfill\break
Great actor of our time. }

\par{12. 競 (くら) attaches to the 連用形 of verbs to show some sort of contest. }

\begin{ltabulary}{|P|P|P|P|}
\hline 

駆けっ競 & Footrace & 睨(にら)め競 & Staring contest \\ \cline{1-4}

\end{ltabulary}

\par{13. ぐるみ shows that the subject is completely included. }

\par{24. 家族ぐるみでその ${\overset{\textnormal{なんもん}}{\text{難問}}}$ に取り ${\overset{\textnormal{く}}{\text{組}}}$ みました。 \hfill\break
The entire family dealt with the difficult problem. }

\par{25. 村ぐるみ = 村人が全員で \hfill\break
Forming a crowd }

\par{14. くんだり, which comes from the word 下り, attaches to names or words of place to show a far distance from a certain reference point. This is not widely used. }

\par{26. 青森くんだりまで行ってしまった。 \hfill\break
I ended up going as far as Aomori. }

\par{27. こんな田舎くんだりまで、よく来てくださいました。 \hfill\break
I'm very grateful that you have come all this way to the country for us. }

\par{15. 気 (け) attaches to nouns and the stems of adjectives\slash verbs to show sensing of a certain condition. }

\par{28. ${\overset{\textnormal{いやけ}}{\text{嫌気}}}$ がさしてるぜ。 \hfill\break
I'm tired of it. }

\par{29. ${\overset{\textnormal{さむけ}}{\text{寒気}}}$ がする。 \hfill\break
To feel chilly. }

\par{30. ${\overset{\textnormal{どくけ}}{\text{毒気}}}$ に当てられる。 \hfill\break
To be overwhelmed. }

\par{31. ${\overset{\textnormal{ひとけ}}{\text{人気}}}$ のない町。 \hfill\break
An abandoned town. }

\par{16. 子 (こ) shows a child of a particular nature. With either nouns or the 連用形 of verbs, it shows a person of a certain occupation. When attached to a place, 子 shows  birthplace. This suffix also helps create female names. }

\par{32. 明治っ子 \hfill\break
A person of the Meiji Period }

\par{33. 恵美子 \hfill\break
Emiko }

\par{34. 彼は売れっ子の作家ですね。 \hfill\break
He's a popular writer, isn't he? }

\par{17. -っこ is a colloquial version of すること. It's often attached to verbs in the potential form and then followed by ない to \textbf{explosively show incapability. }It may also show competition with the 連用形 of verbs, smallness with onomatopoeia, and make slang variants out of common nouns. }

\par{35. あいつには分かりっこないな。 \hfill\break
He'll never understand. }

\par{36. にゃんこ \hfill\break
Kitty cat }

\par{37. ${\overset{\textnormal{な}}{\text{慣}}}$ れっこになる。 \hfill\break
To become used to. }

\par{38. 僕は ${\overset{\textnormal{か}}{\text{駆}}}$ けっこで一番になった。 \hfill\break
I got first in a race. }

\par{39. ${\overset{\textnormal{にら}}{\text{睨}}}$ めっこしよう! \hfill\break
Let's have a staring contest! }

\par{40. ほら、台風が学校をぺしゃんこにしたんだ。 \hfill\break
Look, the typhoon leveled the school! }

\par{41. 彼には日本語が話せっこない。 \hfill\break
He'll never be able to speak Japanese. }

\par{42. 歌えっこない! \hfill\break
I can't sing! }

\par{43. お腹がマジでぺしゃんこだぞ。 \hfill\break
I'm so hungry! }

\par{\textbf{Sentence Note }: The last sentence would be most likely used by someone really thin. }

\begin{ltabulary}{|P|P|P|P|P|}
\hline 

Corner & 隅 \textrightarrow  隅っこ & Edge & 端 \textrightarrow  端っこ & Root 根 \textrightarrow  根っこ \\ \cline{1-5}

\end{ltabulary}

\par{18. ごかし = one is actually planning one's own interests while pretending the other. }

\par{44. おためごかし \hfill\break
Self-aggrandizement under pretense of aiding another }

\par{45. 親切ごかし \hfill\break
Acting nice while actually playing the devil's advocate. }

\par{19. -越し means "over" and may be used in a sense of doing something over some sort of distance such as one's shoulders or in a time sense where it shows a condition continuing throughout a given time period. }

\par{46. 窓越しに話しかける。 \hfill\break
To talk out the window. }

\par{47. ${\overset{\textnormal{かたご}}{\text{肩越}}}$ しに ${\overset{\textnormal{のぞ}}{\text{覗}}}$ き ${\overset{\textnormal{こ}}{\text{込}}}$ む。 \hfill\break
To look over one's shoulders. }

\par{48. 私たちは10年\{越し・来\}の付き合いです。 \hfill\break
We've known each other for over 10 years. }
 
\par{20. ごっこ = game of mimicry. }

\begin{ltabulary}{|P|P|P|P|P|P|}
\hline 

鬼ごっこ & Tag & 電車ごっこ & Make-believe train & お医者さんごっこ & Playing doctor \\ \cline{1-6}

\end{ltabulary}

\par{21. ごと attaches to nouns showing inclusion in "all". }

\par{49. リンゴを ${\overset{\textnormal{かわ}}{\text{皮}}}$ ごと食べる。 \hfill\break
Eat the apple with all of the skin. }

\par{50. ${\overset{\textnormal{ふく}}{\text{服}}}$ ごと放り込む。 \hfill\break
To shovel off all clothes. }

\par{51. ${\overset{\textnormal{かせ}}{\text{枷}}}$ ごと ${\overset{\textnormal{は}}{\text{果}}}$ てない空へただ飛び立つ。 \hfill\break
All of the shackles will just fly away into the endless sky. }

\par{22. 沙汰(ざた)  is often after nouns and the 連用形 to mean "affair" or "gossip". }

\begin{ltabulary}{|P|P|P|P|P|P|}
\hline 

色恋沙汰 & Love affair & 取り沙汰 & Current gossip & 警察沙汰 & Police case \\ \cline{1-6}

\end{ltabulary}

\par{23. 様(ざま) can attach to nouns to show a certain direction. After the 連用形 of a verb, it either shows the course of a condition or action or give a sense of "just as\dothyp{}\dothyp{}\dothyp{}". }

\par{52. ${\overset{\textnormal{ひきょう}}{\text{卑怯}}}$ な生き様 \hfill\break
A cowardly way of living }

\par{53. 授業が続け様に3つあります。 \hfill\break
I have three classes back to back. }

\par{54. ${\overset{\textnormal{さか}}{\text{逆}}}$ さまの夢へと堕ちる。 (Unvoiced example) \hfill\break
To fall in a backwards dream. }

\par{55. ${\overset{\textnormal{きぎ}}{\text{木々}}}$ が風で ${\overset{\textnormal{よこざま}}{\text{横様}}}$ に ${\overset{\textnormal{たお}}{\text{倒}}}$ れた。 \hfill\break
The trees fell down sideways by the wind. }

\par{\textbf{Reading Note }: 横様 = よこざま・よこさま }

\par{56. ${\overset{\textnormal{ふ}}{\text{振}}}$ り ${\overset{\textnormal{む}}{\text{向}}}$ き ${\overset{\textnormal{ざま}}{\text{様}}}$ に彼をにらみつけた。 \hfill\break
I turned and glared at him. }

\par{57. }

\par{「あっ。 ${\overset{\textnormal{こちょう}}{\text{小蝶}}}$ さんが帰る、小蝶さんが帰る」 \hfill\break
と、 ${\overset{\textnormal{かなや}}{\text{金彌}}}$ はぱっと起き上がって、 ${\overset{\textnormal{よろこ}}{\text{喜}}}$ びいっぱい ${\overset{\textnormal{さけ}}{\text{叫}}}$ びながら部屋を飛び出しざま、 \hfill\break
「小蝶さん」 \hfill\break
 ${\overset{\textnormal{にぎや}}{\text{賑}}}$ かな笑い声に続いて、 \hfill\break
 \hfill\break
雨、雨、降れ降れ \hfill\break
母さんが \hfill\break
 ${\overset{\textnormal{じゃ}}{\text{蛇}}}$ の ${\overset{\textnormal{め}}{\text{目}}}$ でお迎え \hfill\break
うれしいな \hfill\break
ピッチピッチ \hfill\break
チャップチャップ \hfill\break
ランランラン \hfill\break
 \hfill\break
 ${\overset{\textnormal{ようき}}{\text{陽気}}}$ に調子づいた ${\overset{\textnormal{おさな}}{\text{幼}}}$ い歌声が、玄関の方へ遠ざかって行った。 \hfill\break
\hfill\break
"Ah! Kocho-san's coming home, Kocho-san's coming home!", Kanami yelled full of joy as she sprang up, and just as she dashed out of the room yelling "Ko-chan", she continued with a lively laughter singing: }

\par{Rain fall, rain fall! \hfill\break
Mom has come to see me with her umbrella \hfill\break
How happy am I! \hfill\break
Splishy splishy splashy splashy \hfill\break
Lalala }

\par{Her warm, young, melodious singing voice went all the water to the entrance. \hfill\break
From 童謡 by 川端康成. }

\par{\textbf{Passage Note }: That song is the first part of a famous 童謡 (children's song) called あめふり. The onomatopoeia is rather unique to the song itself, so it is rather difficult to adequately translate them into English. }

\par{24. しな attaches to the 連用形 of verbs to mean "at time of doing\dothyp{}\dothyp{}\dothyp{}". }

\par{58. 寝しなに酒を飲むと、 ${\overset{\textnormal{ふつかよ}}{\text{二日酔}}}$ いがかかる。 \hfill\break
If you drink sake when you're going to sleep, you'll get a hangover. }

\par{59. 出しなに電話がかかってきた。 \hfill\break
I got a call on my phone when I was leaving. }

\par{60. 泣きしなに転んでしまった。 \hfill\break
I fell at the time I was crying. }

\par{61. 隣りの男は横浜を出しなに薄目をあけただけで、だらしなく居眠り続けた。 \hfill\break
The man right by just halfway opened his eyes leaving Yokohama, and he continued to carelessly   doze. \hfill\break
From 山の音 by 川端康成. }

\par{25. 凌ぎ(しのぎ)= "to tide over" and is idiomatic in translation. It follows nominal phrases. }

\par{62. ${\overset{\textnormal{たいくつ}}{\text{退屈}}}$ しのぎ \hfill\break
To kill time }

\par{63. その場しのぎの手段で切り抜ける。 \hfill\break
To get over at half measures. }
    