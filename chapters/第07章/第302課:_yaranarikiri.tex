    
\chapter{The Particles やら, なり, \& きり}

\begin{center}
\begin{Large}
第302課: The Particles やら, なり, \& きり 
\end{Large}
\end{center}
 
\par{ In this lesson we will learn about the adverbial particles やら, なり, and きり. Some of them have usages that have different classifications. So, always be careful. }
      
\section{The Particle やら}
 
\par{The Adverbial Particle やら }

\par{  This particle is rather interesting. Some phrases such as どうやら shown below in sentences are used all of the time. However, when used to list things vaguely (usage 2), it is not so common and generally mainly found in books. Despite this, it is not impossible to hear someone say it every now and then because it is more emphatic than typical means of expression. }

\begin{itemize}

\item Shows uncertainty 
\item やら lists things very vaguely and emphatically. May be rude as it should not be used in reference to someone above you. 
\item とやら shows obscurity without any particularity. 
\end{itemize}

\begin{center}
\textbf{Examples }
\end{center}

\par{${\overset{\textnormal{}}{\text{1a. 雨}}}$ が ${\overset{\textnormal{}}{\text{降}}}$ るのやら ${\overset{\textnormal{}}{\text{降}}}$ らないのやらよく ${\overset{\textnormal{}}{\text{分}}}$ からない。 \hfill\break
1b. 雨が降るのか降らないのかよく分からない。(もっと一般的な言い方) \hfill\break
I don't know whether it's going to rain or not rain. }
 
\par{2. どうやら ${\overset{\textnormal{るす}}{\text{留守}}}$ のようです。 \hfill\break
It looks like they're somehow not at home. }

\par{3. えっと、英之とやらだったかな。(ちょっと失礼な言い方) \hfill\break
Uh, it was a person called Hideyuki. }

\par{4. どうやら一荒れ来そうだ。 \hfill\break
It looks like we're going to have a storm. }

\par{5. モウリさんはどうやら置いてきたものが多いらしかった。わたしは、一つか二つしか、置いてきたものはない。その一つ二つでさえ持て余しているのだから、モウリさんの疲れかた をや 、だろう。 \hfill\break
Mouri somehow seemed to have come with a lot of things. As for me, I've only came with one or two things. Since even those one or two things are hard to deal with, I guess you could say that's her way of fatigue. \hfill\break
From 溺レる by 川上弘美. }

\par{\textbf{Particle Note }: をや is a combination of the case particle を and the adverbial particle や. It is more common in things like 漢文訓読. It is often seen with いわんや (Let alone)、~において~をや. The phrase itself is a rhetorical device used to add a sense of lament. So, with the example, the author chose it to give the impression that the speaker might get fatigued thinking about his partner's things. }

\par{The Final Particle やら }

\par{Questions oneself with a sense of uncertainty. This is \textbf{rarely }used nowadays. The variant patterns in the first example apply for the second as well, but they are meant to show you how this has been absorbed by other speech patterns. }
 
\par{${\overset{\textnormal{}}{\text{6a. 彼は勝}}}$ った(の)やら。     (May sound old to some) \hfill\break
6b. 彼は勝った(の)かな(あ)   (一般的な言い方) \hfill\break
6c. 彼は勝ったのかしら。        (Feminine) \hfill\break
6d. 彼は勝ったんだろうか。     (Somewhat masculine) \hfill\break
I wonder if he won. }
 
\par{${\overset{\textnormal{}}{\text{7. 行}}}$ ったやら。   (古風) \hfill\break
I wonder if he went. }

\par{8. 私もどれほど ${\overset{\textnormal{あんど}}{\text{安堵}}}$ しましたことやら。  (Literary) \hfill\break
Oh how I was relieved too. \hfill\break
From 光の雨 by 立松和平. }
      
\section{The Particle なり}
 
\par{The Adverbial Particle なり }

\par{1. In "AなりBなり" meaning "either\dothyp{}\dothyp{}\dothyp{}or\dothyp{}\dothyp{}\dothyp{}". B may be a question word hinting at other options. This is normally replaced by other things. In this case, though, it's not necessarily old-fashioned. }
 
\par{${\overset{\textnormal{}}{\text{9a. 俺}}}$ なりあんたやりが ${\overset{\textnormal{}}{\text{行}}}$ かなきゃ。(Masculine) \hfill\break
9b. 俺かあんたが行かなきゃ。(Masculine) \hfill\break
Either I or you have to go. }

\par{10. ${\overset{\textnormal{だい}}{\text{大}}}$ なり ${\overset{\textnormal{しょう}}{\text{小}}}$ なり ${\overset{\textnormal{}}{\text{欠点}}}$ はある。 \hfill\break
There is either a fault in being big or in being small. }
 
\par{\textbf{Sentence Note }: Ex. 10 is more common as it is a common expression. }
 
\par{11a. フランスなりどこなりへ ${\overset{\textnormal{}}{\text{引}}}$ っ ${\overset{\textnormal{}}{\text{越}}}$ したらいい。 \hfill\break
11b. フランスかどこかに引っ越したらいい。(Better choice) \hfill\break
It would be nice to move to France or somewhere. }

\par{2. A more literary variant of ~なければ~ほど. }

\par{12. 「なんで君はある金全部を使ってしまうの」とユキヲは言うのだが、どうやって金の始末をうまくすればいいのか、私には見当もつかないのだ。 \textbf{少なければ少ないなりに }、多ければ多いなりに、私はいつでもぴたりと使いきってしまう。体が、あり金にあわせて ${\overset{\textnormal{の}}{\text{伸}}}$ びたり ${\overset{\textnormal{ちぢ}}{\text{縮}}}$ んだりするような感じだった。 \hfill\break
Yukio asks "why do you end up using all the money?", but I can't guess on how to dispose of the money well. Whether it be less or more, I always completely use it up. I felt as if my body stretched and contracted with [how much] money I had. \hfill\break
From 溺レる by 川上弘美. }
 
\par{3. Shows an example with the implication of the possibility of other options. }

\par{13. }
 
\par{Aくん「どこか海外に行きたいかな。」 \hfill\break
Bくん「なんか西洋の空気を感じたいな。」 \hfill\break
Aくん「じゃ、ヨーロッパへなり行きましょう。」 \hfill\break
A: I kind of want to go overseas. \hfill\break
B: I'd like to feel the West and what not. \hfill\break
A: Let's go to somewhere like Europe. }
 
\par{${\overset{\textnormal{}}{\text{14. 運動}}}$ するなりしてみてはどうか。 \hfill\break
How about trying to drive or something? }

\begin{center}
 \textbf{なりと }
\end{center}

\par{ This is a very interesting combination particle that has several usages. In its first usage, it sets up something as a bare minimal condition, making it essentially a humble でも. This usage, though, has become quite old-fashioned. Some exceptions to this include 何なりと which is used frequently. }

\par{15. 何なりとお申し付けを \hfill\break
Whatever I instruct\dothyp{}\dothyp{}\dothyp{} }

\par{16. よろしかったらお茶なりと ${\overset{\textnormal{め}}{\text{召}}}$ し上がってください。 \hfill\break
If it's all right, please have something like tea. }

\par{17. どうなりと好きにせよ! (Rude , old-style command) \hfill\break
Do what you'd like! }

\par{\textbf{Speech Note }: Using something humble like this in a command results in a rather rude command. }

\par{~なりと may also be seldom seen abbreviated as ~なと. However, this is out of use in mainstream Japanese and is viewed to be dialectical. }

\par{18. 「まことに申し訳ございません。お腹が減ったでしょうから、どうぞ、これなと召し上がって下さい。いずれあとから主人がご挨拶に参ります」 \hfill\break
"I sincerely apologize. I'm sure you are hungry, so please have this. The owner will come by to greet you sometime later." \hfill\break
From 翳った旋舞 by 松本清張. }

\par{ In other contexts, it can show that one may choose voluntarily, but this doesn't always have to be in nice contexts. It can also be used in the pattern AなりとBなりと and function just like なり above. It may be seen as なと, which is somewhat dialectical and very old-fashioned, and なりとも. }

\par{19. 死ぬなり(と)生きるなり(と)勝手にしろ。 \hfill\break
Whether you die or live, do whatever you want. }

\par{20. 願わくは一言なりとも知らせてほしゅうございます。(すごく古い言い方) \hfill\break
I wish to be informed at least one word. }

\par{The Conjunctive Particle なり }

\par{After the 連体形 of a verb, it shows that something is done as soon as something else is done. So right when someone does something, they do something next in sequence to the first action. The subject is normally third person, and the subject is the same in both clauses. The second part has a little sense of unexpected strong will. }

\par{21a. 宿題を済ませるなり、すぐに彼らはインターネットを使った。 \hfill\break
21b. 宿題を済ませるとすぐに彼らはインターネットを使った。(More natural) \hfill\break
They used the Internet as soon as they finished their homework. }

\par{22. 彼は帰るなり、トイレに行った。 \hfill\break
He went to the bathroom as soon as he got home. }

\par{23. 社長は入ってくるなり、大声で ${\overset{\textnormal{どな}}{\text{怒鳴}}}$ りました。 \hfill\break
As soon as the company president came in, he shouted in a big voice. }

\par{ After the past tense, it shows a situation that is still in play as another action begins. There shouldn't be any movement involved by the individual. }
 
\par{24. 机の椅子に座ったなり、眠ってしまった。 \hfill\break
I feel sleep in my desk chair. }

\par{\textbf{The Suffix ~なり }}

\par{~なり may attach to nouns to give a meaning of "like". It may attach itself to nominal phrases or the ${\overset{\textnormal{}}{\text{連体形}}}$ of adjectives to show the means by which something is done like. It may also be after the ${\overset{\textnormal{}}{\text{連用形}}}$ of verbs to be like the phrase ~まま (as is). }
 
\par{${\overset{\textnormal{}}{\text{25. 彼は親}}}$ の ${\overset{\textnormal{}}{\text{言}}}$ いなりになった。 \hfill\break
He did as his parents told (and never changed). }
 
\par{${\overset{\textnormal{}}{\text{26. 道}}}$ なりに ${\overset{\textnormal{}}{\text{行}}}$ く。 \hfill\break
To go along the road. }
 
\par{${\overset{\textnormal{}}{\text{27. 私}}}$ なりに ${\overset{\textnormal{}}{\text{努力}}}$ します。 \hfill\break
I will put effort into it a way that suits me. }
 
\par{${\overset{\textnormal{}}{\text{28. 体}}}$ を ${\overset{\textnormal{}}{\text{弓}}}$ なりに ${\overset{\textnormal{そ}}{\text{反}}}$ らす。 \hfill\break
To lean one's body back in an arch-like shape. }
      
\section{The Adverbial Particle きり}
 
\par{ きり is generally used to limit things. In this sense, it is just like だけ. The sense of limitation is stronger. Consider this. だけ comes from 丈. But, きり comes from 切り. With this alone, you can see how they would have slightly different nuances. }

\par{ きり may also follow ~た followed by a negative expression to refer to something that has not surpassed a limit meaning "after doing\dothyp{}\dothyp{}\dothyp{}". It may also be in ${\overset{\textnormal{まる}}{\text{丸}}}$ っきり which means "at all" in a \textbf{negative potential expression }. }

\par{\textbf{Variant Notes }: }

\par{1. きり may be seen as っきり a lot in the spoken language. \hfill\break
2. ぎり is an old-fashioned variant. }

\par{\textbf{Etymology Note }: This particle comes from the noun 切り, which means "limit" in this instance. }

\begin{center}
 \textbf{Examples }
\end{center}
 
\par{${\overset{\textnormal{}}{\text{29. 二人}}}$ きりで ${\overset{\textnormal{}}{\text{話しましょう}}}$ 。 \hfill\break
Let's talk just the two of us. }
 
\par{30. たった ${\overset{\textnormal{}}{\text{二人}}}$ きりで ${\overset{\textnormal{}}{\text{国}}}$ を ${\overset{\textnormal{せいふく}}{\text{征服}}}$ した。 \hfill\break
The two of them conquered the country. }
 
\par{${\overset{\textnormal{}}{\text{31. 寝}}}$ たきりになる。 \hfill\break
To become confined to continuously lying down. }
 
\par{${\overset{\textnormal{}}{\text{32. \{座}}}$ りっきりで・座ったまま(で)・座りっぱなしで\} ${\overset{\textnormal{}}{\text{仕事}}}$ を ${\overset{\textnormal{}}{\text{続}}}$ ける ${\overset{\textnormal{}}{\text{母}}}$ はいつも ${\overset{\textnormal{}}{\text{肩}}}$ が ${\overset{\textnormal{こ}}{\text{凝}}}$ っている。 \hfill\break
My mother, who continuously sits while working, is stiff in the shoulders. }
 
\par{${\overset{\textnormal{}}{\text{33. 中国}}}$ に ${\overset{\textnormal{}}{\text{行}}}$ ったきり、彼の ${\overset{\textnormal{ゆくえ}}{\text{行方}}}$ が分からない。 \hfill\break
After going to China, we don't know his whereabouts. }
 
\par{34. まるっきり ${\overset{\textnormal{}}{\text{飲}}}$ めない。 \hfill\break
I can't drink at all. }
 
\par{35. もうこれ(っ)きりだよ。 \hfill\break
Make this the last time. }
 
\par{${\overset{\textnormal{}}{\text{36. 俺}}}$ の ${\overset{\textnormal{}}{\text{手元}}}$ にあるのは1 ${\overset{\textnormal{}}{\text{円}}}$ これきりだぞ。(男性語) \hfill\break
This is my very last yen. \hfill\break
Literally: The last thing in my hands is just this yen. }

\par{37. 二階の彼女の―そして彼のともなる部屋は、通りに向いた窓際に、安い ${\overset{\textnormal{ぬ}}{\text{塗}}}$ り ${\overset{\textnormal{ぐすり}}{\text{薬}}}$ の ${\overset{\textnormal{にお}}{\text{匂}}}$ いがする ${\overset{\textnormal{そまつ}}{\text{粗末}}}$ な机があるきりで、その外は灰色の ${\overset{\textnormal{かべ}}{\text{壁}}}$ と押入れの ${\overset{\textnormal{ふすま}}{\text{襖}}}$ だけだった。 \hfill\break
Near the window in her room on the second floor--the room he and she had together--facing the street, there was just a crude desk with the smell of cheap ointment, and aside from that there was only a gray wall and the closet fusuma. \hfill\break
From 死体紹介人 by 川端康成. }

\par{37. 「父や母は忙しいからたまにきり来ないだろうがね。よかったら君も泊りがけに来たまえ」  (ちょっと古風) \hfill\break
"But my mom and dad don't come but occasionally because they're busy. If it's alright, come stay over". \hfill\break
From 友情 by 武者小路実篤. }

\par{\textbf{Grammar Notes }: }

\par{1. たまにきり is an instance where this particle can be seen with an adverb, but in today's speech, the phrase would be replaced with たまにしか. \hfill\break
2. ~たまえ is now old-fashioned and normally used by middle aged men and older in various situations, of which none are polite to say. However, given that the book in which this sentence is from was published in 1920, the users would be of the same generation. This is a clear instance of a generational shift. }
    