    
\chapter{ク語法}

\begin{center}
\begin{Large}
第336課: ク語法 
\end{Large}
\end{center}
 
\par{ In Japanese grammar, there is an odd grammatical pattern involving ク. Presumably unrelated to く in adjectival conjugation, the く in this lesson is found in more archaic expressions. All of these expressions grammatically speaking are peculiar, which makes this a very interesting topic to address. }
      
\section{What is this ク?}
 
\par{ It is thought that the く to be discussed in this lesson comes from the archaic noun あく, which may be found in Modern Japanese slightly altered in the verb 憧れる, which may still show up in old-fashioned and or dialect speech as あくがれる. It is believed to have been a dummy noun in the same way こと is today. With that being the case, you would expect verbs to be in the 連体形 when used with it. However, because the language then hated sequential vowels so much, there would be various sound changes to reduce the sequence into one vowel. }

\begin{ltabulary}{|P|P|P|P|}
\hline 

いう + あく \textrightarrow  いわく & する + あく \textrightarrow  すらく & 安き + あく \textrightarrow  安けく & ぬ + あく \textrightarrow  なく \\ \cline{1-4}

\end{ltabulary}

\par{ There was also an ending for recollection, ~き, that could be used with あく. When in the 連体形, ~き became ~し. However, because it is believed that the i vowel was actually different--transcribed as ï--the combination of the two resulted in しく. This あく is also supposed to be the く in いづく (where?). }
      
\section{Usage}
 
\par{ As you can imagine its usage in Modern Japanese is rather limited. In fact, it's been a somewhat fossilized state ever since the Heian Period, which was a long time ago. And, you get mistakes in usage such as the following. These mistakes were made early on be speakers and have been passed down as such. All of the words coined from misuse minus 恐るらくは are used in limited contexts. 恐らく is actually somehow the current form. }

\begin{ltabulary}{|P|P|P|P|}
\hline 

Misuse & Should be & Misuse & Should be \\ \cline{1-4}

惜しむらくは (Sadly) & 惜しまくは & 望むらくは (What I wish for) & 望まくは \\ \cline{1-4}

疑うらくは (In doubting) & 疑わくは & 恐るらくは (Perhaps) & 恐らくは \\ \cline{1-4}

\end{ltabulary}

\par{ Aside from these phrases from misuse, there are other examples used in Modern Japanese. Below is a somewhat exhaustive list of these phrases. }

\begin{ltabulary}{|P|P|P|P|P|P|}
\hline 

曰く & In saying; reason & Literary & 体たらく & Predicament & Not just 書き言葉 \\ \cline{1-6}

老いらく & Old age & Literary & すべからく & By all means & Literary \\ \cline{1-6}

願わく\{は・ば\} & I pray that & Literary & 思惑 & Speculation; expectation & Not just 書き言葉 \\ \cline{1-6}

\end{ltabulary}

\par{\textbf{漢字 Notes }: }

\par{1. すべからく may be written as 須(ら)く. This is a combination of すべき in its 連体形 すべかる + あく. \hfill\break
2. 思惑\textquotesingle s more traditional spelling is 思わく as 惑 in this spelling is 当て字. \hfill\break
3. 願わくは may also rarely be spelled as 希わくは in literature. }

\begin{center}
 \textbf{Examples }
\end{center}

\par{ This section will purposely avoid uses of ク語法 no longer relevant to Modern Japanese. Remember that aside from the occasional set phrase and 恐らく that this grammar pattern is very 書き言葉的. }

\par{1. 古人曰く勝て兜の緒を締めよ。 \hfill\break
The ancient say defeat and fasten the chords of your armor. }

\par{\textbf{Idiom Note }: 兜の緒を締めよ also has the meaning of not letting your guard down. }

\par{2. 法はすべからく守るべし。 \hfill\break
The law ought to be protected. }

\par{3. 老いらくの恋 \hfill\break
An old man's love }

\par{4. 世間の思惑通りには生きていけない。 \hfill\break
I mustn't live as the world expects me to. }

\par{5. 恐らくあの猫が隣の川に溺れたと思う。 \hfill\break
I think that that cat perhaps drowned in the river next to us. }

\par{6. なんという体たらくだ! \hfill\break
What a mess you've gotten yourself into! }

\par{7. 願わくは一流企業に就職したい。 \hfill\break
I pray I get hired into a first class company. }

\par{8. 願わくは京都に居を構えたい。 \hfill\break
I wish to set up residence in Kyoto. }

\par{9. 願わくは御名を尊まれんことを。 \hfill\break
I pray that your name be honored. }

\par{\textbf{Word Note }: 願わくは is typically used in prayer like expressions, and due to the mistaken form 願わくば, it may sometimes have a meaning closer to 願いが叶うならば. }

\par{ As is evident in the modern examples, it has lost its productivity as a grammatical pattern for a long time, but you should get a sense at least of what role it once had (presumably in a larger range of expressions). }
    