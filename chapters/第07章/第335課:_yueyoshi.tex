    
\chapter{故, 所以・由縁, 謂れ, \& 由}

\begin{center}
\begin{Large}
第335課: 故, 所以・由縁, 謂れ, \& 由 
\end{Large}
\end{center}
 
\par{ Reason, as we have learned, is expressed with various grammar points such as the particles から and ので, nouns such as 理由 and 訳, as well as several other phrases like ~ため. Each means of expressing reasons presents the learner even more ways to nuance one\textquotesingle s speech to better convey reason. In this lesson, we will learn about even more phrases whose basic understandings are tied to the notion of reason. These words are 故, 所以・由縁, 謂れ, and 由. }
      
\section{故}
 
\par{ At its most basic understanding, 故 means “reason.” }
 
\par{1. ${\overset{\textnormal{ゆえ}}{\text{故}}}$ あって ${\overset{\textnormal{どうこう}}{\text{同行}}}$ することになった。 \hfill\break
It has been decided for a certain reason that we accompany the other. }
 
\par{2. ${\overset{\textnormal{ゆえ}}{\text{故}}}$ ない ${\overset{\textnormal{ぶじょく}}{\text{侮辱}}}$ はせん。 \hfill\break
I shan\textquotesingle t make a senseless insult. }
 
\par{3. そう ${\overset{\textnormal{ひはん}}{\text{批判}}}$ されるのも ${\overset{\textnormal{ゆえな}}{\text{故無}}}$ しとしない。 \hfill\break
Being criticized as such is also not without reason. }
 
\par{\textbf{Phrase Note }: 故無しとしない can be paraphrased as それなりの理由がある. }
 
\par{4. ${\overset{\textnormal{ち}}{\text{地}}}$ は ${\overset{\textnormal{なん}}{\text{何}}}$ の ${\overset{\textnormal{ゆえ}}{\text{故}}}$ を以って ${\overset{\textnormal{とうなん}}{\text{東南}}}$ に ${\overset{\textnormal{かたむ}}{\text{傾}}}$ くや。 \hfill\break
Wherefore doth the Earth tilteth to the southeast? }
 
\par{ However, this is not its only meaning. Depending on the expression, it may be synonymous to the word 由緒 meaning “history” as in having a connection to something. }
 
\par{5. ${\overset{\textnormal{ゆえ}}{\text{故}}}$ ある ${\overset{\textnormal{いひん}}{\text{遺品}}}$ を ${\overset{\textnormal{す}}{\text{捨}}}$ てられまい。 \hfill\break
I cannot possibly discard items full of history to me that were left to me. }
 
\par{ It may also be translatable as “circumstance(s)” and “appearance.” }
 
\par{6. その ${\overset{\textnormal{ゆえ}}{\text{故}}}$ ありげな ${\overset{\textnormal{うつく}}{\text{美}}}$ しい ${\overset{\textnormal{すがた}}{\text{姿}}}$ 、この ${\overset{\textnormal{よ}}{\text{世}}}$ のものではないようにも ${\overset{\textnormal{おも}}{\text{思}}}$ える。 \hfill\break
That suggestive, beautiful appearance, it seems as if it\textquotesingle s not of this world. }
 
\begin{center}
\textbf{~ゆえに }
\end{center}
 
\par{ The grammar pattern ~ゆえに, which is created with the noun 故 from above and the purpose-marking に, is used to express either a positive or negative cause which brings about a typically negative yet atypical result. }
 
\par{ This pattern is quite old-fashioned, but it is occasionally still found. What is most complicated about it is how it connects with other parts of speech. }

\begin{ltabulary}{|P|P|}
\hline 

Nouns & N + ゆえ(に) \hfill\break
N + がゆえに \hfill\break
N + のゆえに \hfill\break
N + である(が)ゆえに \\ \cline{1-2}

形容詞 & Adj + (が)ゆえに \\ \cline{1-2}

形容動詞 & Adj. N + である(が)ゆえに \hfill\break
Adj. N + なゆえに \\ \cline{1-2}

Verbs & V + (が)ゆえに \\ \cline{1-2}

\end{ltabulary}
 
\par{ It is possible to see (が)ゆえに follow conjugatable parts of speech in both the non-past and the past tense. It may even be seen after the auxiliary verb ます. However, it is mostly seen after nouns where it exhibits the most variety in appearance. }
 
\par{ It is often the case that the speaker relates to the situation somehow. Even if the sentence is in the third person, the speaker still relates to the subject of the sentence. This pattern is also used in academic papers as an objective marker of reason. In an academic setting, the effect that follows does not have to be limited to a negative circumstance, but outside academic settings, this is a requirement. }
 
\par{7. ${\overset{\textnormal{みじゅくもの}}{\text{未熟者}}}$ ゆえお ${\overset{\textnormal{ゆる}}{\text{許}}}$ しください。 \hfill\break
Please forgive me for I am a novice. }
 
\par{\textbf{Sentence Note }: The result of forgiving the speaker for being a novice may not seem like a negative circumstance, but the possibility of not being forgiven is a negative outcome, and it is this direness being expressed by the speaker that consists an unfavorable situation. }
 
\par{8. ${\overset{\textnormal{おんな}}{\text{女}}}$ であるがゆえに ${\overset{\textnormal{う}}{\text{受}}}$ ける ${\overset{\textnormal{さべつ}}{\text{差別}}}$ の ${\overset{\textnormal{こうぞう}}{\text{構造}}}$ を ${\overset{\textnormal{し}}{\text{知}}}$ れば、もっと ${\overset{\textnormal{らく}}{\text{楽}}}$ になると ${\overset{\textnormal{おも}}{\text{思}}}$ い ${\overset{\textnormal{こ}}{\text{込}}}$ んでいる ${\overset{\textnormal{じょせい}}{\text{女性}}}$ たちがいる。 \hfill\break
There are women who are (incorrectly) convinced that things will get easier so long as they know about the framework of discrimination that they receive for being women. }
 
\par{9. ${\overset{\textnormal{おんな}}{\text{女}}}$ であるゆえに ${\overset{\textnormal{ちちおや}}{\text{父親}}}$ の ${\overset{\textnormal{ざいさん}}{\text{財産}}}$ を ${\overset{\textnormal{そうぞく}}{\text{相続}}}$ できず ${\overset{\textnormal{まず}}{\text{貧}}}$ しい ${\overset{\textnormal{く}}{\text{暮}}}$ らしを ${\overset{\textnormal{し}}{\text{強}}}$ いられる。 \hfill\break
She is compelled to live in poverty, unable to inherit her father\textquotesingle s fortune for being a woman. }
 
\par{10. その ${\overset{\textnormal{びぼう}}{\text{美貌}}}$ と ${\overset{\textnormal{わか}}{\text{若}}}$ さのゆえに、 ${\overset{\textnormal{あやま}}{\text{誤}}}$ ったイメージをもたれているように ${\overset{\textnormal{おも}}{\text{思}}}$ われる。 \hfill\break
Because of her youth and beauty, there is an incorrect image of her. }
 
\par{11. ${\overset{\textnormal{わか}}{\text{若}}}$ いがゆえに ${\overset{\textnormal{あくせい}}{\text{悪性}}}$ の ${\overset{\textnormal{びょうき}}{\text{病気}}}$ であるとは ${\overset{\textnormal{おも}}{\text{思}}}$ いもよらなかった。 \hfill\break
Due to my being young, the thought of having a bad disease was inconceivable. }
 
\par{12. ${\overset{\textnormal{おとこ}}{\text{男}}}$ ゆえに ${\overset{\textnormal{てき}}{\text{敵}}}$ が ${\overset{\textnormal{おお}}{\text{多}}}$ い。 \hfill\break
My enemies I do have for being a man. }
 
\par{13. ${\overset{\textnormal{あい}}{\text{愛}}}$ ゆえに ${\overset{\textnormal{ひと}}{\text{人}}}$ は ${\overset{\textnormal{くる}}{\text{苦}}}$ しまねばならぬ。 \hfill\break
For love people must suffer. }
 
\par{14. こういう ${\overset{\textnormal{しず}}{\text{静}}}$ かなイケメンは、 ${\overset{\textnormal{しず}}{\text{静}}}$ かであるがゆえに、 ${\overset{\textnormal{おお}}{\text{多}}}$ くの ${\overset{\textnormal{じょせい}}{\text{女性}}}$ が ${\overset{\textnormal{みお}}{\text{見落}}}$ としている。 \hfill\break
It is this kind of quiet good-looking guy that many women overlook due to his being quiet. }
 
\par{15. ${\overset{\textnormal{い}}{\text{生}}}$ き ${\overset{\textnormal{いそ}}{\text{急}}}$ ぐゆえに" ${\overset{\textnormal{し}}{\text{死}}}$ "が ${\overset{\textnormal{せっきん}}{\text{接近}}}$ していることに ${\overset{\textnormal{き}}{\text{気}}}$ づかない。 \hfill\break
It is because we live fast that we do not notice “death” approaching. }
 
\par{16. アライグマは、 ${\overset{\textnormal{やせいどうぶつ}}{\text{野生動物}}}$ であるがゆえに、 ${\overset{\textnormal{ほかく}}{\text{捕獲}}}$ や ${\overset{\textnormal{じこ}}{\text{事故}}}$ を ${\overset{\textnormal{まぬが}}{\text{免}}}$ れ、 ${\overset{\textnormal{せいじゅう}}{\text{成獣}}}$ になれる ${\overset{\textnormal{かくりつ}}{\text{確率}}}$ は ${\overset{\textnormal{きわ}}{\text{極}}}$ めて ${\overset{\textnormal{ひく}}{\text{低}}}$ い。 \hfill\break
Raccoons, because they are wild animals, have an extremely low probability of evading capture and accidents and becoming adults. }
 
\par{17. ${\overset{\textnormal{りこん}}{\text{離婚}}}$ が ${\overset{\textnormal{ふ}}{\text{増}}}$ えたがゆえに ${\overset{\textnormal{さいこん}}{\text{再婚}}}$ も ${\overset{\textnormal{ふ}}{\text{増}}}$ えているという ${\overset{\textnormal{げんじつ}}{\text{現実}}}$ はあるが、だからといって ${\overset{\textnormal{だれ}}{\text{誰}}}$ もが ${\overset{\textnormal{かんたん}}{\text{簡単}}}$ に ${\overset{\textnormal{つぎ}}{\text{次}}}$ の ${\overset{\textnormal{けっこん}}{\text{結婚}}}$ を ${\overset{\textnormal{き}}{\text{決}}}$ められるわけではない。 \hfill\break
Although there is the reality that remarrying is on the rise due to divorce having risen, it is not the case nonetheless that anyone can easily decide one\textquotesingle s next marriage. }
 
\par{18. それだけに、 ${\overset{\textnormal{そうめい}}{\text{聡明}}}$ なゆえに ${\overset{\textnormal{なに}}{\text{何}}}$ かしらの ${\overset{\textnormal{いわかん}}{\text{違和感}}}$ を ${\overset{\textnormal{も}}{\text{持}}}$ ちつつ ${\overset{\textnormal{まいにち}}{\text{毎日}}}$ を ${\overset{\textnormal{す}}{\text{過}}}$ ごされているのでしょう。 \hfill\break
For that reason alone, because you are wise, I\textquotesingle m sure you spend each feeling out of place somehow. }
 
\par{19. ${\overset{\textnormal{にほん}}{\text{日本}}}$ は、 ${\overset{\textnormal{しまぐに}}{\text{島国}}}$ (の)ゆえに ${\overset{\textnormal{ほか}}{\text{他}}}$ から ${\overset{\textnormal{おお}}{\text{大}}}$ きな ${\overset{\textnormal{しんりゃく}}{\text{侵略}}}$ も ${\overset{\textnormal{う}}{\text{受}}}$ けず、 ${\overset{\textnormal{したが}}{\text{従}}}$ って ${\overset{\textnormal{ほろ}}{\text{滅}}}$ びるかどうかという ${\overset{\textnormal{しんこく}}{\text{深刻}}}$ な ${\overset{\textnormal{しれん}}{\text{試練}}}$ にも ${\overset{\textnormal{あ}}{\text{遭}}}$ わずに ${\overset{\textnormal{い}}{\text{生}}}$ きてこられた ${\overset{\textnormal{こく}}{\text{国}}}$ である。 \hfill\break
Japan, as a result of being an island nation, is a country that hasn\textquotesingle t ever sustained a large invasion from the outside, and as a result has lived without facing the serious tribulation of perishing. }
 
\par{20. ${\overset{\textnormal{う}}{\text{受}}}$ け ${\overset{\textnormal{つ}}{\text{継}}}$ いだ ${\overset{\textnormal{つみ}}{\text{罪}}}$ のゆえに ${\overset{\textnormal{ひと}}{\text{人}}}$ の ${\overset{\textnormal{こころ}}{\text{心}}}$ は ${\overset{\textnormal{よわ}}{\text{弱}}}$ く ${\overset{\textnormal{しんこう}}{\text{信仰}}}$ が ${\overset{\textnormal{か}}{\text{欠}}}$ けてしまう。 \hfill\break
It is because of the sin man has inherited that the heart is weak and also why one\textquotesingle s faith is lacking. }
 
\par{21. ${\overset{\textnormal{でんせつ}}{\text{伝説}}}$ の ${\overset{\textnormal{まじゅう}}{\text{魔獣}}}$ (である)がゆえに、 ${\overset{\textnormal{せいぎょ}}{\text{制御}}}$ できぬのである。 \hfill\break
One cannot control it for it is a legendary beast. }
 
\par{22. ${\overset{\textnormal{せいぎ}}{\text{正義}}}$ と ${\overset{\textnormal{あい}}{\text{愛}}}$ がゆえに ${\overset{\textnormal{たたか}}{\text{闘}}}$ う。 \hfill\break
To fight for justice and love. }
 
\par{23. ${\overset{\textnormal{いと}}{\text{糸}}}$ が ${\overset{\textnormal{うつく}}{\text{美}}}$ しいゆえに、その ${\overset{\textnormal{ししゅう}}{\text{刺繍}}}$ は ${\overset{\textnormal{うつく}}{\text{美}}}$ しい。 \hfill\break
The embroidery is beautiful by virtue of the thread being beauty. }
 
\par{24. ${\overset{\textnormal{びしょうじょ}}{\text{美少女}}}$ は、 ${\overset{\textnormal{うつく}}{\text{美}}}$ しいがゆえに、 ${\overset{\textnormal{ひさん}}{\text{悲惨}}}$ な ${\overset{\textnormal{おも}}{\text{思}}}$ いをすることがある。 \hfill\break
Beautiful girls often experience tragedy owing to their beauty. }
 
\par{25. ${\overset{\textnormal{うつく}}{\text{美}}}$ しさゆえに ${\overset{\textnormal{あい}}{\text{愛}}}$ するのなら、 ${\overset{\textnormal{わたし}}{\text{私}}}$ を ${\overset{\textnormal{あい}}{\text{愛}}}$ さないでおくれよ。 \hfill\break
If you are to love me by reason of my beauty, please do not live me. }
 
\begin{center}
\textbf{事故 }
\end{center}
 
\par{ It is interesting to know that 事故 at one time had a native equivalent, which was literally ことゆえ. ことゆえ is a combination of 事 and 故, and the resultant combination brings about a nuance of “accident” or potential difficulties in the circumstances at hand. }
 
\par{26. ${\overset{\textnormal{な}}{\text{慣}}}$ れぬことゆえ、お ${\overset{\textnormal{けいこがんば}}{\text{稽古頑張}}}$ って ${\overset{\textnormal{くだ}}{\text{下}}}$ さい。 \hfill\break
As you are yet accustomed, please do your best in training. }
 
\par{27. ${\overset{\textnormal{なにぶんこども}}{\text{何分子供}}}$ のことゆえ、お ${\overset{\textnormal{ゆる}}{\text{赦}}}$ しください。 \hfill\break
Please forgive him as he is but a child. }
 
\par{28. ${\overset{\textnormal{ぜんせ}}{\text{前世}}}$ に ${\overset{\textnormal{い}}{\text{行}}}$ った悪しき ${\overset{\textnormal{こと}}{\text{事}}}$ ゆえに、そうした ${\overset{\textnormal{くなん}}{\text{苦難}}}$ を ${\overset{\textnormal{けいけん}}{\text{経験}}}$ するに ${\overset{\textnormal{あたい}}{\text{値}}}$ することになるのである。 \hfill\break
One becomes deserving of experiencing such hardship due to bad deeds one committed in previous lives. }
 
\begin{center}
\textbf{故に }
\end{center}
 
\par{ It is often the case that 故に is used as a sentence-initial conjunction when the previous clause ends in a verb or adjective. When this is done, 故に is translatable as “therefore.” A more emphatic version of this is それゆえ, which also translates as “therefore.” }
 
\par{29. ${\overset{\textnormal{われおも}}{\text{我思}}}$ う。 ${\overset{\textnormal{ゆえ}}{\text{故}}}$ に、 ${\overset{\textnormal{わが}}{\text{我}}}$ あり。 \hfill\break
I think. Therefore, I am. }
 
\par{30. ${\overset{\textnormal{にっぽん}}{\text{日本}}}$ の ${\overset{\textnormal{ふうけい}}{\text{風景}}}$ は ${\overset{\textnormal{うつく}}{\text{美}}}$ しい。ゆえに、 ${\overset{\textnormal{にっぽん}}{\text{日本}}}$ の ${\overset{\textnormal{うた}}{\text{歌}}}$ は ${\overset{\textnormal{うつく}}{\text{美}}}$ しい。 \hfill\break
Japanese landscape is beautiful. Therefore, Japanese songs are beautiful. }
 
\par{31. それゆえ、 ${\overset{\textnormal{りれきしょうえ}}{\text{履歴書上}}}$ に、これまでの ${\overset{\textnormal{しごと}}{\text{仕事}}}$ に ${\overset{\textnormal{かんけい}}{\text{関係}}}$ するとは ${\overset{\textnormal{おも}}{\text{思}}}$ えない ${\overset{\textnormal{みゃくらく}}{\text{脈絡}}}$ のない ${\overset{\textnormal{しかく}}{\text{資格}}}$ や ${\overset{\textnormal{めんきょ}}{\text{免許}}}$ が ${\overset{\textnormal{れっきょ}}{\text{列挙}}}$ されていたら、プラスになるどころか、マイナスにすらなりかねないのです。 \hfill\break
Therefore, having a list of qualifications and licenses with seemingly no logical connection to the jobs you\textquotesingle ve had up to that point on your resume is likely to be a negative rather than a positive. }
 
\par{32. それゆえ ${\overset{\textnormal{おとこ}}{\text{男}}}$ はその ${\overset{\textnormal{ふぼ}}{\text{父母}}}$ を ${\overset{\textnormal{はな}}{\text{離}}}$ れ、 ${\overset{\textnormal{つま}}{\text{妻}}}$ と ${\overset{\textnormal{むす}}{\text{結}}}$ び ${\overset{\textnormal{あ}}{\text{合}}}$ い、 ${\overset{\textnormal{ふたり}}{\text{二人}}}$ は ${\overset{\textnormal{いったい}}{\text{一体}}}$ となるのである。 \hfill\break
Therefore, shall a man leave his father and mother, and shall cleave unto his wife: and they shall be one flesh. }
 
\begin{center}
\textbf{何故 }
\end{center}
 
\par{ Although no longer used in the spoken language, 何ゆえ is in fact a way to say “why.” It is akin to the somewhat old-fashioned English expression “wherefore.” }
 
\par{33. ${\overset{\textnormal{せかい}}{\text{世界}}}$ は ${\overset{\textnormal{なに}}{\text{何}}}$ ゆえ ${\overset{\textnormal{かく}}{\text{核}}}$ の ${\overset{\textnormal{ほのお}}{\text{炎}}}$ に ${\overset{\textnormal{つつ}}{\text{包}}}$ まれたのか \hfill\break
Why was it that the world was enveloped in the flames of nuclear weaponry? }
 
\par{34. ${\overset{\textnormal{なに}}{\text{何}}}$ ゆえ ${\overset{\textnormal{しょうせつ}}{\text{小説}}}$ を ${\overset{\textnormal{か}}{\text{書}}}$ くのか。 \hfill\break
Wherefore does one write novels? }
 
\par{35. ${\overset{\textnormal{いごじゅうぶん}}{\text{以後充分}}}$ に ${\overset{\textnormal{き}}{\text{気}}}$ をつけますゆえ、お ${\overset{\textnormal{ゆる}}{\text{許}}}$ し ${\overset{\textnormal{くだ}}{\text{下}}}$ さいまし。 \hfill\break
As I will pay attention moving forward, I ask that you please forgive me. }
 
\par{\textbf{Sentence Note }: This sentence would have been heard in the late 1800s and early 1900s but would be viewed as extremely old-fashioned. }
 
\par{36. これを ${\overset{\textnormal{きかい}}{\text{機会}}}$ に、 ${\overset{\textnormal{やくしゃいんいちどうあら}}{\text{役社員一同新}}}$ たな ${\overset{\textnormal{きも}}{\text{気持}}}$ ちで ${\overset{\textnormal{しゃぎょうはってん}}{\text{社業発展}}}$ に ${\overset{\textnormal{ぜんりょく}}{\text{全力}}}$ を ${\overset{\textnormal{つ}}{\text{尽}}}$ くす ${\overset{\textnormal{しょぞん}}{\text{所存}}}$ でございますゆえ、 ${\overset{\textnormal{なにとぞ}}{\text{何卒}}}$ ご ${\overset{\textnormal{こうしょう}}{\text{高承}}}$ の ${\overset{\textnormal{うえ}}{\text{上}}}$ 、より ${\overset{\textnormal{いっそう}}{\text{一層}}}$ のご ${\overset{\textnormal{こうぎ}}{\text{厚誼}}}$ を ${\overset{\textnormal{たまわ}}{\text{賜}}}$ りますようお ${\overset{\textnormal{ねが}}{\text{願}}}$ い申し上げます。 \hfill\break
With this opportunity, we intend to do our utmost in the company\textquotesingle s development with a new sense of unity with all executive staff and works. As such, upon your kind affirmation, we ask that you bestow us with more of your kindness and support. }
 
\par{\textbf{Sentence Note }: Ex. 36, as well as Ex. 37, are examples of honorific speech utilizing ゆえ in a way that is still used. Indicative of the written language, the particle に is omitted. }
 
\par{37. ${\overset{\textnormal{ねんねんさまざま}}{\text{年々様々}}}$ な ${\overset{\textnormal{しょくざい}}{\text{食材}}}$ の ${\overset{\textnormal{ねだん}}{\text{値段}}}$ が ${\overset{\textnormal{こうとう}}{\text{高騰}}}$ する ${\overset{\textnormal{じょうきょう}}{\text{状況}}}$ が ${\overset{\textnormal{つづ}}{\text{続}}}$ いておりますゆえ、 ${\overset{\textnormal{ぜんたいてき}}{\text{全体的}}}$ な ${\overset{\textnormal{かかく}}{\text{価格}}}$ の ${\overset{\textnormal{みなお}}{\text{見直}}}$ しをせざるをえない ${\overset{\textnormal{じょうきょう}}{\text{状況}}}$ となっております。 \hfill\break
Due to the continued state of the prices of various foodstuffs steeply rising year by year, we are compelled to do an overall price revision. }
 
\begin{center}
\textbf{~がために }\hfill\break

\end{center}

\par{ ~がために is very similar to ~(が)ゆえに. As we learned in the previous lesson, it too is old-fashioned and limited almost entirely to the written language, and it similarly expresses an atypical reason\slash cause that brings about an atypical result, but neither the cause nor the result have to be negative in connotation in any way. Although this ought to make it more objective, because ~ために is already objective, the use of the particle が makes the grammar pattern emphatic and consequently subjective. With ~がために, it is never the case that the speaker is speaking from experience or sense of sympathy, nor is it used with the first person. Due to its subjectivity, it is also not seen in academic settings. }
 
\par{38. ${\overset{\textnormal{とうせん}}{\text{当選}}}$ したいがために ${\overset{\textnormal{かんばん}}{\text{看板}}}$ を ${\overset{\textnormal{ぬ}}{\text{塗}}}$ り ${\overset{\textnormal{か}}{\text{替}}}$ える ${\overset{\textnormal{ひと}}{\text{人}}}$ は ${\overset{\textnormal{しんよう}}{\text{信用}}}$ できない。 \hfill\break
I cannot credit people who change their policies to get elected. }
 
\par{39. ${\overset{\textnormal{わか}}{\text{若}}}$ いがために、 ${\overset{\textnormal{しゃかいじん}}{\text{社会人}}}$ としての ${\overset{\textnormal{けいけん}}{\text{経験}}}$ が ${\overset{\textnormal{あさ}}{\text{浅}}}$ くなりがちです。 \hfill\break
Because they are young, there is a tendency that their experiences as working adults is shallow.  }
      
\section{所以・由縁}
 
\par{ Both 所以 and 由縁 are read as ゆえん, which is a contraction of ゆえなり, which is Classical Japanese for “is the reason.” Each respective spelling conjures up different nuances of the noun 故 as an effect. 所以 is used to mean “reason\slash cause” whereas 由縁 is used to mean “origin\slash connection\slash history” and is synonymous with other words like 由緒 and 由来. Neither words are particularly used in the spoken language. }

\par{40. ${\overset{\textnormal{ゆえん}}{\text{所以}}}$ など ${\overset{\textnormal{ゆくえし}}{\text{行方知}}}$ らず。 \hfill\break
Cause is nowhere to be found. }

\par{41. ${\overset{\textnormal{しょくにん}}{\text{職人}}}$ が「 ${\overset{\textnormal{ぎじゅつ}}{\text{技術}}}$ は ${\overset{\textnormal{み}}{\text{見}}}$ て ${\overset{\textnormal{ぬす}}{\text{盗}}}$ め」と ${\overset{\textnormal{い}}{\text{言}}}$ う ${\overset{\textnormal{ゆえん}}{\text{所以}}}$ だろう。 \hfill\break
The worker would say to look at a technique and steal it. }

\par{42. ${\overset{\textnormal{し}}{\text{死}}}$ する ${\overset{\textnormal{ゆえん}}{\text{所以}}}$ は ${\overset{\textnormal{すなわ}}{\text{即}}}$ ち ${\overset{\textnormal{しょう}}{\text{生}}}$ ずる ${\overset{\textnormal{ゆえん}}{\text{所以}}}$ なり。 \hfill\break
The reason for dying, in other words, is the reason for living. }

\par{\textbf{Grammar Note }: なり is the basic copula verb of Classical Japanese. }

\par{43. ${\overset{\textnormal{ひと}}{\text{人}}}$ の ${\overset{\textnormal{ひと}}{\text{人}}}$ たる ${\overset{\textnormal{ゆえん}}{\text{所以}}}$ を ${\overset{\textnormal{まな}}{\text{学}}}$ ぶ。 \hfill\break
To study the reason for people being people. }

\par{\textbf{Grammar Note }: The particle の here is equivalent to the particle が,  being used to mark the subject of a subordinate clause. Also, the auxiliary verb たる is a classical copula verb that is still seen in old-fashioned expressions such as Ex. 43. }

\par{44. ${\overset{\textnormal{ちめい}}{\text{地名}}}$ の ${\overset{\textnormal{ゆえん}}{\text{由縁}}}$ を ${\overset{\textnormal{たず}}{\text{尋}}}$ ねる。 \hfill\break
To ask about what\textquotesingle s related to the place name. }

\par{45. ${\overset{\textnormal{ちめい}}{\text{地名}}}$ の ${\overset{\textnormal{ゆえん}}{\text{由縁}}}$ を ${\overset{\textnormal{し}}{\text{知}}}$ ることで、 ${\overset{\textnormal{み}}{\text{身}}}$ を ${\overset{\textnormal{まも}}{\text{守}}}$ る ${\overset{\textnormal{だいいっぽ}}{\text{第一歩}}}$ になるかもしれない。 \hfill\break
By knowing the affinity of place names, we might make the first step to protecting ourselves. }
      
\section{謂れ}
 
\par{ 謂れ comes from the 未然形 of 言われる, the passive form of the verb 言う. It is used to mean either “reason\slash cause” or “history\slash origin.” It is seen frequently in the expression ~いわれはない meaning “there is no reason for.” }

\par{46. ${\overset{\textnormal{だれ}}{\text{誰}}}$ もこんな ${\overset{\textnormal{め}}{\text{目}}}$ に ${\overset{\textnormal{あ}}{\text{遭}}}$ う謂れはない。 \hfill\break
There is no reason for anyone to go through such suffering. }

\par{47. ${\overset{\textnormal{わたくし}}{\text{私}}}$ には、こんな ${\overset{\textnormal{うんめい}}{\text{運命}}}$ を ${\overset{\textnormal{ひ}}{\text{引}}}$ き ${\overset{\textnormal{じゅ}}{\text{受}}}$ けなければならないいわれはない。 \hfill\break
There is no reason for why I must accept such a fate as this. }

\par{48. ${\overset{\textnormal{いわ}}{\text{謂}}}$ れのない ${\overset{\textnormal{うわさ}}{\text{噂}}}$ や ${\overset{\textnormal{あくしつ}}{\text{悪質}}}$ な ${\overset{\textnormal{か}}{\text{書}}}$ き ${\overset{\textnormal{こ}}{\text{込}}}$ みによって ${\overset{\textnormal{ひぼうちゅうしょう}}{\text{誹謗中傷}}}$ される。 \hfill\break
To be slandered by baseless rumors and malicious posts. }

\par{49. ${\overset{\textnormal{しゅうへん}}{\text{周辺}}}$ に ${\overset{\textnormal{てんざい}}{\text{点在}}}$ している ${\overset{\textnormal{しせき}}{\text{史跡}}}$ や謂れのある ${\overset{\textnormal{とち}}{\text{土地}}}$ を ${\overset{\textnormal{めぐ}}{\text{巡}}}$ りました。 \hfill\break
I went around to see historical landmarks and places with history that dot the area. }
      
\section{由}
 
\par{ 由 is a multifaceted word whose basic meaning is also “reason\slash cause.” However, it may also mean “method” such as in the set phrase in Ex. 53 and “piece of information,” which is seen in Ex. 51 and Ex. 52, which are examples of how it is still used in respectful speech. Otherwise, the word is truly relegated to set expressions. }
 
\par{50. ${\overset{\textnormal{よし}}{\text{由}}}$ ありげな ${\overset{\textnormal{もんどう}}{\text{問答}}}$ を ${\overset{\textnormal{な}}{\text{投}}}$ げかける。 \hfill\break
To throw meaningful dialogue. }
 
\par{51. この ${\overset{\textnormal{よし}}{\text{由}}}$ 、お ${\overset{\textnormal{つた}}{\text{伝}}}$ えください。 \hfill\break
Please convey this (to the individual). }
 
\par{52. お ${\overset{\textnormal{げんき}}{\text{元気}}}$ との ${\overset{\textnormal{よし}}{\text{由}}}$ 、 ${\overset{\textnormal{なに}}{\text{何}}}$ よりです。 \hfill\break
I am glad to hear that you are doing well. \hfill\break
 \hfill\break
53. ${\overset{\textnormal{し}}{\text{知}}}$ る ${\overset{\textnormal{よし}}{\text{由}}}$ もない。 \hfill\break
There is no way of knowing. }
    