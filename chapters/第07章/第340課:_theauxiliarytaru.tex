    
\chapter{The Auxiliary ~たる}

\begin{center}
\begin{Large}
第340課: The Auxiliary ~たる: たるや、たるもの、たりとも 
\end{Large}
\end{center}
 
\par{ This pesky classical copular auxiliary is still occasionally used in Modern Japanese, and as you would expect, the title really does say all. First, we will discuss the so-called タル形容動詞 that are made with it, see how the true copular role it has manages to get used, and then focus on grammar topics relevant for high level Japanese proficiency tests such as the JLPT N1. }
      
\section{タル形容動詞}
 
\par{ This defunct class of adjectival verbs have the limited bases と-連用形 and the たる-連体形. However, with more antiquated grammar, you can see the other bases. So, for completeness, the full base set is given below. }

\begin{ltabulary}{|P|P|P|P|P|P|}
\hline 

未然形 & 連用形 & 終止形 & 連体形 & 已然形 & 命令形 \\ \cline{1-6}

たら- & たり-・と & たり & たる & たれ- & たれ \\ \cline{1-6}

\end{ltabulary}

\par{ Like 形容動詞, the と-連用形 is used adverbially. The exact number of タル形容動詞 still used today is uncertain, but they are in decline. Their attributive base is also typically replaced with とした. However, in more formal writing, these adjectives pop up everywhere. Depending on the adjective, they may have acquired other legitimate attributive forms. For instance, you can use 主な and 主たる (principal\slash main). }

\begin{center}
 \textbf{Examples }
\end{center}

\par{1a. 主たる理由はこれです。(古風) \hfill\break
1b. 主な理由はこれです。(もっと自然) \hfill\break
The main reason is this. }
 
\par{2. 全然たる狂人 \hfill\break
An absolute maniac }
 
\par{3. 然したる相違もない。 \hfill\break
There isn't a special\slash particular difference. }
 
\par{4. \{名立たる・有名な\}観光地 \hfill\break
A famous tourist spot }

\par{5. ${\overset{\textnormal{めんぜん}}{\text{面前}}}$ での ${\overset{\textnormal{ちょうしょう}}{\text{嘲笑}}}$ は ${\overset{\textnormal{ぶじょく}}{\text{侮辱}}}$ の ${\overset{\textnormal{さい}}{\text{最}}}$ たるものだ。 \hfill\break
Scorn in one's presence is the extremity of insult. }
 
\par{6a. 最たる例 (古風) \hfill\break
6b. 最も ${\overset{\textnormal{けんちょ}}{\text{顕著}}}$ な例  (もっと自然) \hfill\break
Prime example }

\par{7. ${\overset{\textnormal{どうどう}}{\text{堂々}}}$ \{たる・とした\} ${\overset{\textnormal{すがた}}{\text{姿}}}$ \hfill\break
 A magnificent figure }

\par{8. ${\overset{\textnormal{だんこ}}{\text{断固}}}$ \{たる・とした\}決意 \hfill\break
Resolute determination }

\par{9. ${\overset{\textnormal{たんたん}}{\text{淡々}}}$ \{たる・とした\} ${\overset{\textnormal{くちょう}}{\text{口調}}}$ で話す。 \hfill\break
To speak in a cool tone. }

\par{10. ${\overset{\textnormal{ばくぜん}}{\text{漠然}}}$ \{たる・とした\}不安 \hfill\break
Vague anxiety }

\par{11. ${\overset{\textnormal{びょう}}{\text{眇}}}$ たる小会社 \hfill\break
Insignificant firm }

\par{12. ${\overset{\textnormal{じゅんぜん}}{\text{純然}}}$ たる銀行 \hfill\break
Pure and simple bank }

\par{13. ${\overset{\textnormal{げん}}{\text{厳}}}$ たる事実 \hfill\break
Harsh fact }

\par{14. ${\overset{\textnormal{ゆうゆう}}{\text{悠々}}}$ たる ${\overset{\textnormal{おおぞら}}{\text{大空}}}$ \hfill\break
 Endless sky }

\par{15. ${\overset{\textnormal{うつぼつ}}{\text{鬱勃}}}$ たる ${\overset{\textnormal{やしん}}{\text{野心}}}$ \hfill\break
 Irresistible ambition }
 
\par{\textbf{Base Note }: Though they have no modern 終止形, when the need arises, the old たり-終止形 is replaced with the たる-連体形 followed by だ, giving たるだ. }
 
\par{16. 行政改革は号令のみで旧態依然たるだ。 \hfill\break
The administrative reform is merely an order, and the (system) is still as the old state. }
 
\par{\textbf{Archaism Note }: There are still instances where the auxiliary copula verb たり is still used in Modern Japanese. As it is more emphatic, it often serves a role in formal yet serious situations. One instance is 日本よ国家たれ! (Japan, be a nation!) and its derivations. }
      
\section{たるや}
 
\par{  This is a usage of above that is rarely seen but is used to emphatically emphasis a topic, and it is derived from the auxiliary たる and the adverbial\slash bound particle や. }

\par{17. その技術たるや、ものすごい。 \hfill\break
The technique is terrific! }
      
\section{たるもの(は)}
 
\par{ This pattern follows nouns, as you would expect, to show that a certain thing is suitable in situations such as having a certain responsibility or when one is outstanding for a certain role. Remember that what follows should show what form, role, or status something should be in to be what X or what たるもの follows. }

\par{18. 社会人たるもの、挨拶や時間を守ることなどができて当然でしょう。 \hfill\break
It's only natural for a person of society to maintain salutations and time. }

\par{19. 紳士たるもの、強くなければならない。 \hfill\break
A gentleman must be strong. }

\par{20. 政治家たるものは、失言があってはならない。 \hfill\break
Politicians must not make gaffes. }

\par{21. 教師たるものは、言動に気をつけなければならない。 \hfill\break
Teachers must pay attention to their behavior. }

\par{22. 大統領たるものは、 ${\overset{\textnormal{えいり}}{\text{鋭利}}}$ な頭脳を持たなければならない。 \hfill\break
A president must hold a sharp mind. }
      
\section{たりとも}
 
\par{ The conjunctive particle たりとも is equivalent to "not even" and is interchangeable with と言えども and ではあっても. It is really on the combination of the auxiliary たる + the conjunctive particle とも, which is why it's mentioned here. This pattern should follow a counter phrase that a small or indeterminate number. However, even if the number is one, if the counter phrase implies something big like a year or a ton, then this is not applicable. }

\par{23. 一字たりとも許さぬぞ。 \hfill\break
I won't allow even a single letter (being wrong). }

\par{24. 1日たりとも忘れたことはありません。 \hfill\break
I haven't forgotten even a single day. }

\par{25. 一年たりとも X }

\par{26. ${\overset{\textnormal{いっしゅん}}{\text{一瞬}}}$ たりとも気を ${\overset{\textnormal{ぬ}}{\text{抜}}}$ いてはいけないよ。 \hfill\break
You mustn't lose focus for even a moment. }

\par{27. これからは1日たりとも日本語の練習を ${\overset{\textnormal{なま}}{\text{怠}}}$ けてはいけない! \hfill\break
You mustn't slack off your Japanese practice for even a day from now on! }

\par{28. 1円たりとも無駄にしてほしくありません。 \hfill\break
I don't wish for even a single yen to be wasted. }

\par{29. 1分たりとも遅れないようにしてください。 \hfill\break
Please make it toward you are not late for even one minute. }

\par{30. 何人たりとも \hfill\break
Not one at all }

\par{\textbf{Reading Note }: 何人 in the last expression can be read as なにびと, なんびと, or なんぴと. }
    