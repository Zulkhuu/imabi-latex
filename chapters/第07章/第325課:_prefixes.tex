    
\chapter{Prefixes}

\begin{center}
\begin{Large}
第325課: Prefixes 
\end{Large}
\end{center}
 
\par{ Prefixes, 接頭語, are key to understanding hundreds of phrases in Japanese. If you are unable to find a word in a dictionary, it is probably because the word you are looking for has a prefix or a suffix. }
      
\section{What is a Prefix?}
 A prefix attaches to the front of words to add a given meaning. Different from suffixes, a prefix will never change the part of speech of a word. In terms of etymology, there are three kinds of nouns in Japanese: native, Sino-Japanese, and loan word. Likewise, this also reflects in prefixes. However, nouns and prefixes are not limited to each other based on etymology.       
\section{Prefixes Attached to Nouns \& Adjectives}
 
\par{ Although there are some that may have a few possible purposes, prefixes are generally one phrase definitions or are taught as such. The chart below illustrates the most common and easy prefixes you will see and should know in Japanese. Definitions of example words should show whether the word is a noun or an adjective. }

\begin{ltabulary}{|P|P|P|P|}
\hline 

Prefix & 仮名 & Meaning(s) & Examples \\ \cline{1-4}

亜 & あ & 
\par{To apply to; sub-; semi- }

\par{-ous acid }
& 亜熱帯(あねったい) Subtropics \hfill\break
\\ \cline{1-4}

相 & あい & 
\par{Together }

\par{Adds a formal nuance to a word. }
& 相等(あいひと)しい Equal to each other \hfill\break
\\ \cline{1-4}

空き & あき & 
\par{Empty }
& 空き缶(あきかん) Empty can \hfill\break
空き瓶(あきびん) Empty bottle \\ \cline{1-4}

幾 & いく & 
\par{How many? }
& 幾久(いくひさ)しく Forever \hfill\break
\\ \cline{1-4}

薄 & うす & 
\par{Light }
& 薄明(うすあ)かり  Dim light \hfill\break
\\ \cline{1-4}

小 & お & 
\par{Small; thin }
& 小川(おがわ) Brook \hfill\break
\\ \cline{1-4}

御 & お & See Lesson 127 &  \\ \cline{1-4}

大 & おお & 
\par{Big; wide; a lot }

\par{Substantially; utmost }

\par{Respectfully shows praise }
& 大違(おおちが)い Completely different \hfill\break
\\ \cline{1-4}

御 & お(ん)み & See Lesson 146 &  \\ \cline{1-4}

御 & おん & See Lesson 146 &  \\ \cline{1-4}

過 & か & 
\par{Excessive }
& 過保護(かほご) Over-protection \hfill\break
\\ \cline{1-4}

各 & かく & 
\par{Each }
& 各国(かっこく) Every nation \hfill\break
\\ \cline{1-4}

金 & きん & 
\par{Placed before a money amount }
& 金1000円(きんせんえん) Cash 1000 yen \\ \cline{1-4}

小 & こ & 
\par{Small }

\par{Shows scorn or hate }

\par{Around }
& 小雨(こさめ) Shower \hfill\break
小賢(こざか)しい Smart-alecky 
\\ \cline{1-4}

御 & ご & See Lesson 146 &  \\ \cline{1-4}

再 & さい & 
\par{Re- }
& 再選挙(さいせんきょ) Reelection \hfill\break
\\ \cline{1-4}

最 & さい & 
\par{-est }
& 最高級(さいこうきゅう) Top-class \hfill\break
\\ \cline{1-4}

純 & じゅん & 
\par{Pure }
& 純愛(じゅんあい) Pure love \hfill\break
\\ \cline{1-4}

準 & じゅん & 
\par{Semi- }
& 準決勝(じゅんけっしょう) Semifinal \hfill\break
\\ \cline{1-4}

諸 & しょ & 
\par{Every; all }
& 諸国 (しょこく) Every nation \hfill\break
\\ \cline{1-4}

新 & しん & 
\par{New }
& 新世界 (しんせかい) New world \\ \cline{1-4}

全 & ぜん & 
\par{All; whole }
& 全責任(ぜんせきにん) Full responsibility \hfill\break
\\ \cline{1-4}

総 & そう & 
\par{All }
& 総利益 (そうりえき) Gross profit \hfill\break
\\ \cline{1-4}

第 & だい & 
\par{Number\dothyp{}\dothyp{}\dothyp{} }
& 第3(だいさん) Number 3 \hfill\break
\\ \cline{1-4}

大 & だい & 
\par{A degree, scale, or situation is severe }

\par{Excellent; high in rank }
& 大首都圏(だいしゅとけん) Greater metropolitan area \\ \cline{1-4}

築 & ちく & 
\par{Years since construction }
& 築5年 Five years since construction \\ \cline{1-4}

超 & ちょう & 
\par{Over- }
& 超大国 (ちょうたいこく) Superpower \hfill\break
\\ \cline{1-4}

当 & とう & 
\par{This }

\par{Our \hfill\break
}
& 当問題 (とうもんだい) This issue \\ \cline{1-4}

共 & とも & 
\par{Together }
& 共働き(ともばたらき) Both working \hfill\break
\\ \cline{1-4}

初 & はつ & 
\par{A first }

\par{First in the year }
& 初霜(はつしも) First frost of the year \hfill\break
\\ \cline{1-4}

反 & はん & 
\par{Anti- }
& 反政府(はんせいふ) Anti-government \\ \cline{1-4}

半 & はん & 
\par{Half; incomplete; small }
& 半ズボン(はんずぼん) Shorts \hfill\break
\\ \cline{1-4}

非 & ひ & 
\par{Un-; non-; an- }
& 非公式(ひこうしき) Informal \hfill\break
\\ \cline{1-4}

一 & ひと & 
\par{One; a little \hfill\break
}

\par{Shows a past time }
& 一時(ひととき) A time \\ \cline{1-4}

不 & ふ & 
\par{Un-; non- }
& 不必要 (ふひつよう) Unnecessary \\ \cline{1-4}

無・不 & ぶ & 
\par{Un-; non- }
& 無意識 (むいしき) Unconscious \hfill\break
\\ \cline{1-4}

不可 & ふか & 
\par{Can't do }
& 不可思議(ふかしぎ) Mysterious \\ \cline{1-4}

古 & ふる & 
\par{Old }
& 古本(ふるほん) Used book \hfill\break
古新聞(ふるしんぶん) Used newspaper \\ \cline{1-4}

本 & ほん & 
\par{This }

\par{Real, true; main \hfill\break
}
& 本店(ほんてん) Home office \\ \cline{1-4}

真(ん・っ) & ま(ん・っ) & 
\par{Really, genuine; right in the }

\par{Relating animals }
& 真ん丸 (まんまる) Perfect circle \\ \cline{1-4}

毎 & まい & 
\par{Each time }
& 毎年(まいとし) Every year \\ \cline{1-4}

豆 & まめ & 
\par{Small }
& 豆鉄砲(まめてっぽう) Popgun \\ \cline{1-4}

丸 & まる & 
\par{Whole; all }
& 丸キロ(まるきろ) Whole kilogram \hfill\break
\\ \cline{1-4}

御 & み & See Lesson 146 &  \\ \cline{1-4}

未 & み & 
\par{Not yet; un- }
& 未完成 (みかんせい) Incomplete \\ \cline{1-4}

明 & みょう & 
\par{The next }
& 明後日 (みょうごにち) The day after tomorrow \\ \cline{1-4}

無 & む & 
\par{Un-; non- }
& 無意味(むいみ) Meaningless \\ \cline{1-4}

物 & もの & 
\par{Whole thing feeling a certain way }
& 物悲(ものがな)しい Melancholy \hfill\break
\\ \cline{1-4}

諸 & もろ & 
\par{All; both, together \hfill\break
}
& 諸刃(もろは) Double-edged \\ \cline{1-4}

手 & て & 
\par{Strengthens meaning of an adjective. }
& 手厚(てあつ)い Cordial \hfill\break
手強(てごわ)い Formidable \hfill\break
手緩(てぬる)い Lenient \hfill\break
\\ \cline{1-4}

悪 & わる & 
\par{Unpleasant }

\par{To go overboard }
& 悪足掻(わるあが)き Useless efforts \\ \cline{1-4}

\end{ltabulary}
      
\section{Prefixes with Verbs}
 
\par{ Unlike for nouns, there aren't that many prefixes that attach to verbs. Of the ones that do exist, there are often variants. Below is a chart that illustrates basically all the prefixes you'll see before a verb in Japanese--there's just not that many of them. }

\par{Note: あい- from above may also be used with verbs. }

\begin{ltabulary}{|P|P|P|P|}
\hline 

Prefix & 仮名 & Meaning(s) & Examples \hfill\break
\\ \cline{1-4}

打ち & うち & 
\par{A little }
& 打ち明ける (うちあける) To confide \\ \cline{1-4}

押し & おし & 
\par{To do strongly; to do purposelessly }
& 押し上げる (おしあげる) To boost \hfill\break
\\ \cline{1-4}

押っ & おっ & 
\par{At once; with vigor }

\par{Strong expression of chasing }

\par{Completely }
& 押っつける (おっつける) To force \\ \cline{1-4}

掻き & かき & 
\par{To do all at once }
& 掻き集める (かきあつめる) To scramble \hfill\break
\\ \cline{1-4}

差し・っ & さし・っ & 
\par{Completely; positively \hfill\break
}
& 差し伸べる (さしのべる) To reach out \\ \cline{1-4}

すっ & すっ & 
\par{Strengthens meaning }
& すっ飛ぶ (すっとぶ) To rush away \hfill\break
\\ \cline{1-4}

立ち & たち & 
\par{Strengthens meaning or emphasis }
& 立ち向かう (たちむかう) To stand up to \\ \cline{1-4}

突き・っ・ん & つき・っ・ん & 
\par{Strengthens the meaning of an action }
& 突き刺す (つきさす) To stab \\ \cline{1-4}

取り・っ & とり・っ & 
\par{Sets up and strengthens one's emphasis }
& 取り戻す (とりもどす) To recover; get back \hfill\break
\\ \cline{1-4}

引き・っ・ん & ひき・っ・ん & 
\par{Puts more stress on something }
& 引き留める (ひきとめる) To keep back \\ \cline{1-4}

ぶち・ぶっ・ぶん & ぶち・ぶっ・ぶん & 
\par{Strengthens the meaning of an action }
& ぶち抜く (ぶちぬく) To break in \hfill\break
\\ \cline{1-4}

吹っ & ふっ & 
\par{To do well vigorously }
& 吹っ飛ぶ (ふっとぶ) To blow off \\ \cline{1-4}

罷り & まかり & 
\par{To strengthen one's stress on something }
& 罷り通る (まかりとおる) To be justified \\ \cline{1-4}

\end{ltabulary}
      
\section{Prefixes for Loanwords}
 
\par{ With the influx of foreign expressions, so have the amount of foreign prefixes. Below is a list of some of the most commonly seen prefixes for loan words in Japanese. }

\begin{ltabulary}{|P|P|P|}
\hline 

Prefix & Meaning(s) & Examples \\ \cline{1-3}

アンチ & Anti- & アンチテーゼ Antithesis \\ \cline{1-3}

インター & Inter- & インターセプト Intercept \\ \cline{1-3}

ウルトラ & Ultra- & ウルトラマリン Ultramarine \\ \cline{1-3}

エレクトロ & Electro- & エレクトロニクス Electronics \\ \cline{1-3}

オート & Auto- & オートショー Auto-show \\ \cline{1-3}

グッド & Good & グッドデザイン Good design \\ \cline{1-3}

サンタ & Saint & サンタクロース Santa Claus \\ \cline{1-3}

スーパー & Super- & スーパーマーケット Supermarket \\ \cline{1-3}

セミ & Semi- & セミプロ Semipro \\ \cline{1-3}

トリプル & Tri- & トリプルプレー Triple play \\ \cline{1-3}

ネオ & Neo- & ネオコン Neo-con \\ \cline{1-3}

ノン & Non- & ノンストップ Nonstop \\ \cline{1-3}

パン & Pan- & パンアメリカニズム Pan-americanism \\ \cline{1-3}

プレ & Pre- & プレオリンピック Pre-olympics \\ \cline{1-3}

ポスト & Post- & ポストモダニズム Post-modernism \\ \cline{1-3}

マクロ & Macro- & マクロウィルス Macro virus \\ \cline{1-3}

マイクロ & Micro- & マイクロチップ Microchip \\ \cline{1-3}

マルチ & Multi- & マルチタスキング Multitasking \\ \cline{1-3}

ミニ 
& Mini- & ミニカー Minicar \\ \cline{1-3}

ギガ & Giga- & ギガバイト Gigabyte \\ \cline{1-3}

キロ & Kiro- & キログラム Kilogram \\ \cline{1-3}

デカ & Deca- & デカグラム Decagram \\ \cline{1-3}

デシ & Deci- & デシミリ Decimeters \\ \cline{1-3}

テラ & Tera- & テラバイト Terabyte \\ \cline{1-3}

ナノ & Nano- & ナノメートル Nanometer \\ \cline{1-3}

ピコ & Pico- & ピコメートル Picometer \\ \cline{1-3}

フェムト & Femto- & フェムトメートル Femtometer \\ \cline{1-3}

ヘクト & Hecto- & ヘクトメートル Hectometer \\ \cline{1-3}

\end{ltabulary}
    