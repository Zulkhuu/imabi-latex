    
\chapter{Native Suffixes III}

\begin{center}
\begin{Large}
第328課: Native Suffixes III: Adjectival \& Adverbial 
\end{Large}
\end{center}
       
\section{Adjectival and Adverbial}
 
\par{\textbf{~勝ち }: A 形容動詞 conjugating suffix that shows that "something is prone to happen". Often with a negative tone, it's associated with bad\slash unwanted tendencies. Though it does have 漢字, it is almost always written in ひらがな. }

\par{1. 彼女の小説は退屈になりがちだ。 \hfill\break
Her novels tend to become boring. }

\par{2. 彼は\{ ${\overset{\textnormal{}}{\text{怠惰}}}$ に・怠けて\}走りがちだ。 \hfill\break
He tends to laziness. }

\par{3. 学生にありがちな間違いじゃない(か)? \hfill\break
Isn't that an error tended by students? }

\par{4. あいつは職務を ${\overset{\textnormal{おこた}}{\text{怠}}}$ りがちだ。 \hfill\break
That guy tends to neglect his duties. }

\par{A similar phrase is ともすると meaning "apt to". It may also be seen as ともすれば. }

\par{5. 不注意な政治家はともすると間違いを犯したことにさえ気づかないでしょう。 \hfill\break
A careless politician is apt to not even be aware that he has made a mistake. }

\par{6. ともすれば怠けぐせが出る。 \hfill\break
To be apt to have a lazy tendency. }

\par{\textbf{~がましい }: attaches to nouns or a verb's 連用形 to show a certain tendency felt vividly. }

\par{7. 婚礼の晴れがましい場に臨む。 \hfill\break
To attend at a grand wedding ceremony place. }

\par{8. 差し出がましいようですが\dothyp{}\dothyp{}\dothyp{} \hfill\break
I hope it is not too presumptuous of me\dothyp{}\dothyp{}\dothyp{} }

\par{9. おこがましい自称 \hfill\break
An impudent purport }

\par{10. ${\overset{\textnormal{うら}}{\text{恨}}}$ みがましい \hfill\break
Reproachful }

\par{11. 言い訳がましい。 \hfill\break
Sounds like an excuse. }

\par{12. 校長はとても押し付けがましいですね。 \hfill\break
The principal is very pushy, isn't he? }

\par{13. 恩着せがましいことを言うな。 \hfill\break
Don't say such condescending things. }

\par{\textbf{~がわしい }: shows that something or an action has a certain tendency. It is quite rare. }

\par{14. ${\overset{\textnormal{みだ}}{\text{濫}}}$ りがわしい話じゃないか。 \hfill\break
Isn't that such a morally corrupt argument? }

\par{\textbf{~くさい }: attaches to nominal phrases or the stem of adjectives to either show that something has a particular smell or a certain feeling to it. }

\par{15. 古臭いな! \hfill\break
That's old-fashioned! }

\par{${\overset{\textnormal{}}{\text{16. 彼は酷}}}$ くニンニク臭かった。 \hfill\break
He smelled like garlic badly. }

\par{17. ガス臭いところだね。 \hfill\break
This is a gassy smelling place isn't it? }

\par{18. ${\overset{\textnormal{いんき}}{\text{陰気}}}$ 臭い小説 \hfill\break
A gloomy novel }

\par{\textbf{~ぐましい }: shows a condition that is starting to appear that attaches to nouns. It is basically exclusive to the noun 涙 "tear". }

\par{19. 涙ぐましい努力をしよう。 \hfill\break
Let's make a painstaking effort to do this. }

\par{\textbf{~こい }: attaches to nouns and either shows that there is a lot of something included or that a certain nature or situation is extreme. It is often accompanied with a 促音. }

\par{20. 脂っこいものを食べない方がいいですよ。 \hfill\break
It's best not to eat fatty foods. }

\par{21. 彼は本当にしつこい要求をしてばかりいるセールスマンです。 \hfill\break
He is salesman that really just makes persistent demands. }

\par{22. このシロップは粘っこすぎる。 \hfill\break
This syrup is too sticky. }

\par{\textbf{~たらしい }: attaches to nouns and the stems of adjectives to give a strong sense of displeasure. A 促音 is often present when used. }

\par{23. 憎たらしそうにすんな。 \hfill\break
Don't be hateful. }

\par{24. 嫌みったらしい文句 \hfill\break
A very obscene phrase }

\par{25. 長たらしい演説だった。 \hfill\break
It was a lengthy speech. }

\par{\textbf{~ぼったい }: attaches to adjective stems or 連用形 of verbs. Something is quite so. }

\par{26. 寝不足で ${\overset{\textnormal{まぶた}}{\text{瞼}}}$ が本間で ${\overset{\textnormal{は}}{\text{腫}}}$ れぼったいさ。 \hfill\break
My eyes are like really puffed up from lack of sleep. }

\par{27. 厚ぼったい唇あるね。 \hfill\break
You have some thick lips don't you? }

\par{\textbf{~めかしい }: "something looks like\dothyp{}\dothyp{}\dothyp{}". Attached to nouns and adjective stems. }

\par{28. 予言めかしい。 \hfill\break
Sounds like a prediction. }

\par{29. 艶かしい声 \hfill\break
An amorous voice }

\par{30. 古めかしい服装を着る。 \hfill\break
To wear old-fashioned clothes. }

\par{\textbf{~か(な)・~やか(な)・~らか(な) }: Attach to make adjectives that show a nature. There is a lot of variation as to what gets these and what doesn't. For instance, though ${\overset{\textnormal{かろ}}{\text{軽}}}$ やか (non-serious) is a standard example, the forms 軽らか and 軽ろか (much rarer) show up in literature. }

\par{31. 仄かな光 \hfill\break
Dim light }

\par{32. それは定かな事実ではない。 \hfill\break
That is not a definite fact. }

\par{33. 老人にお辞儀を ${\overset{\textnormal{つつ}}{\text{慎}}}$ ましやかにすべきだ。 \hfill\break
You should bow reservedly to the elderly. }

\par{34. 静かな川の流れが美しい。 \hfill\break
The gentle water flow is beautiful. }

\par{35. しめやかな ${\overset{\textnormal{そうぎ}}{\text{葬儀}}}$ を行う。 \hfill\break
To hold a solemn funeral service. \hfill\break
 \hfill\break
36. 清らかな心身 \hfill\break
Pure body and mind }

\par{37. 晴れやかで生気がみなぎった顔。 \hfill\break
A face bright and full of life. }

\par{38. 大らかな性格を持つ。 \hfill\break
To have a broad-minded nature. }

\par{39. 彼女は高らかに歌い続けた。 \hfill\break
She continued to sing her heart out. }

\par{40.海は ${\overset{\textnormal{おだ}}{\text{穏}}}$ やかです。 \hfill\break
The sea is calm. }

\par{41a. 華やかな装い \hfill\break
41b. 派手やかな装い   (ちょっと古風) \hfill\break
Gorgeous outfit\slash apparel }

\par{42. ${\overset{\textnormal{つづま}}{\text{約}}}$ やかな文章 \hfill\break
A concise composition }

\par{43. その目は再び若若しくかがやき、その ${\overset{\textnormal{ほお}}{\text{頰}}}$ には ${\overset{\textnormal{にお}}{\text{匂}}}$ いやかなためらうような ${\overset{\textnormal{くれな}}{\text{紅}}}$ いが ${\overset{\textnormal{さ}}{\text{射}}}$ した。 \hfill\break
Those eyes youthfully glistened and lustrous rouge, as if to hesitant, shined on her cheeks. \hfill\break
From ${\overset{\textnormal{かるのみこ}}{\text{軽王子}}}$ と ${\overset{\textnormal{そとおりひめ}}{\text{衣通姫}}}$ by 三島由紀夫. }

\par{44. 川の水音のみはとりわけ近く、めずらかな夜の小鳥の ${\overset{\textnormal{さえず}}{\text{囀}}}$ りかときかれたのだ。 \hfill\break
The sound of the river was close among other things, and it sounded like it was the rare, night song of a small bird. \hfill\break
From 軽王子と衣通姫 by 三島由紀夫. }

\par{\textbf{Word Note }: めずらか is a rare version of 珍しい, attaching か instead of しい to the root. }

\begin{center}
\textbf{Adverbial } 
\end{center}

\par{\textbf{~すがら }: attaches to nouns. It may show that something is amid or that there's nothing else but X. }

\par{45. 雨は夜もすがら降り続いた。 \hfill\break
The rain continued to rain throughout the night. }

\par{46. 彼は京都までの道すがら思い出話した。 \hfill\break
He reminisced about being along the way Kyoto. }

\par{47. 身すがら = 身一つ \hfill\break
Just oneself }

\par{48. 手すがら \hfill\break
Nothing else but one's hand }
    