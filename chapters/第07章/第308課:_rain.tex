    
\chapter{Rain}

\begin{center}
\begin{Large}
第308課: Rain  
\end{Large}
\end{center}
 
\par{ There isn't anything hard about 雨, but it is actually one of many words potentially from the Austronesian language family (which formed in Taiwan and spread throughout the Philippines, Indonesia, the Pacific and elsewhere). A growing number of Japanese scholars believe these speakers had contact with Japanese speaking populations in the ancient period. 雨 is just one of the resultant words from this probable cultural exchange. }

\par{ When words evolve, they often create offshoots with related meanings. Before you start thinking this is just like Inuit having tons of words for snow, remember that English is just as guilty. First, we'll have fun figuring out where 雨 came from. Then, we'll look at all the words for kinds of rain as well as sounds for rain. In the end, you'll have enough rain words at your disposal to make rainy poetry\dothyp{}\dothyp{}\dothyp{} }
      
\section{The Etymology of 雨}
 
\par{ The Kanji 雨 has four readings: ウ, あめ, あま-, -さめ. The first is from Chinese, so we won't need it until later. The last three, however, are clearly related to each other, but the lack of "s" in the first two poses a problem to naive natives or learners that think native words have changed little. }

\par{ To those that think Japanese and Korean are related to each other, 雨 really causes problems because the Korean word for rain is \slash pi\slash . How can \slash pi\slash  possibly have any connection to something in Japanese? This, too, will be looked into. }

\begin{center}
\textbf{sebu\slash (n)sepu: The Beginning of 雨 } 
\end{center}

\par{ Early constructed Austronesian words for rain include sebu and (n)sepu.This word in Austronesian had several related words, all of which appear to have contributed to Japanese. It appears that sebu made it to Japanese as sopo and sabu. It appears that there was free variation between b and p in the ancient period, and given that similar words are being borrowed, vowels can easily shift around. }

\begin{center}
 sopo \textrightarrow  \textbf{そぼ }降る, \textbf{どぼん }, \textbf{ぽと }ぽと, \textbf{ほと }ばしる, \textbf{ほと }びる 
\end{center}

\par{ In Poem 3883 of the 万葉集, the oldest work in Japanese, the phrase 小雨そほ降る appears. そほ is sopo. The pronunciation of o-sounds is contested. Some posit that Japanese had two o-vowels similar to Modern Korean. Sopo would eventually become sofo and then be found as sobo in the phrase そぼ降る (to drizzle), basically just as it is seen 2,000 years in the past. }

\par{ The s in sopo became replaced with the t. A direct derivative of topo is どぼん, which means "plop".  The the two syllables would then flip to produce poto. This poto survives in ぽとぽと, which is an onomatopoeia that describes the trickling down of heavy tears. It also survives in the verbs ほとばしる (to gush out) and ほとびる (to swell from absorbed moisture). }

\begin{center}
sabu  \textrightarrow  \textbf{ざぶ }ざぶ, \textbf{さば }さば, \textbf{ざぶん }, \textbf{さめ }ざめ, - \textbf{さめ }, \textbf{あめ }, \textbf{あま }- 
\end{center}

\par{ The form sabu can be seen today in words like ざぶざぶ (onomatopoeia for gushing), さばさば (candid), さっぱり (refreshing). ざぶん (splash\slash plop), however, is a clear descendant of sabu. The ba in "saba" would be interchanged with ma. This should not be surprising because b and m remain interchangeable in many words to this day like つぶる・つむる (to close one's eyes). }

\par{ With time, the さめざめ in さめざめ(と)泣く (crying sorrowfully) was born. Thus, we finally have -same. Drop the s and you get 雨! あま- would then be a given derivative due to the noun compounding rules established by that time. }

\begin{center}
 \textbf{The Korean \slash pi\slash  } 
\end{center}

\par{ I f we want to look for a pre-sebu word for rain in Japanese of Korean origin, we would need to find something that looks like \slash pi\slash . It turns out that this is very easy to do. 樋 is a word for gutter and water pump, and it so happens to have the readings とひ and ひ. If we take the と to be related to 戸 and focus on the ひ, we see a resemblance to \slash pi\slash . In fact, if we consider ancient pronunciation, they're the same. The readings とゆ and とよ came much later, so they pose no problem to this analysis. }

\begin{center}
 \textbf{Conclusions }
\end{center}

\par{ Is anything of this fact without a shadow of a doubt? No, but that's not the main point that you should take from this. Rather, if you as a learner had gathered up these words in your vocabulary and pondered on them, you'd probably find similar connections. Finding patterns in words that are similar that you know will only help you retain what things mean and make sense of the language as a whole. }

\par{\textbf{Citation Note }: It would do no justice to not mention 川本崇雄, whose brilliant research led to these discoveries. His book 南から来た日本語 is a truly fascinating book that any person who is interested in the origins of Japanese words should read. }
      
\section{雨に関する言葉}
 
\par{ It appears that Japanese is now the new target for having too many words for the same thing: 雨. Is that really the case? Sure, there are definitely many words derived from or with 雨, but so what? When you type 雨 into an online dictionary like jisho.org, you will find tons of words with 雨. Not all are commonly used words. Some are very formal. If a rain word doesn't have 雨 in it, you probably won't think to look elsewhere to find other such words. }

\par{ Sure, it rains a lot in Japan. It also rains a lot in other parts of the world. By no means should you ignore the relationship between language and geography, but the overwhelming majority of "rain" related words come from Chinese. Yes, the language does have many words for rain, but it's no more than other languages spoken in areas with similar climates. }

\par{\textbf{Speech Styles Note }: "Written language" simply refers to literary contexts. Some words may be used in the spoken language, but they may be more formal or typical of a news report. The descriptions will try to account for all of this, but each word will have its unique but ever changing use. }

\par{\textbf{Reading Note }: Remember that ウ is the Sino-Japanese reading for 雨. It is typically an indicator that a word is formal. It is used constructively to make many rain related words, some of which may be commonly used in the spoken language like 雨量(うりょう) for rainfall. }

\begin{center}
 \textbf{The Many 'Rain' Words }
\end{center}

\par{\textbf{雨 }: Not having 雨 in a list of 'rain' words would be stupid. It's used in all sorts of settings in speaking and writing. It is the basic word for rain in Japanese. For origin, reread the start of this lesson. }

\par{\textbf{小雨(こさめ) }: This refers to a little rain falling or rain falling that is very thin. It is very common in the spoken language, but it has the rare reading こあめ. I would advice not to use this, but its existence does bring to mind the struggle for the initial s's existence. In the written language, this word can also be read as しょうう. しょうう may also be spelled as 少雨. This can be translated as either "light\slash slight rain", "drizzle", or even "sprinkle". }

\par{\textbf{晴一時小雨(はれいちじこさめ) }: This is a brief 小雨. }

\par{\textbf{大雨(おおあめ) }: This is rain that falls in large quantities. It is the opposite of 小雨, and it is also very common in the spoken language. In the written language, it may also be read as たいう. This may be translated as "downpour" or "heavy rain". }

\par{\textbf{小降り(こぶり) }: This word refers to rain or snow falling with weak intensity. It is appropriate to relate this to "rain letting up". This word is a commonly used word in the spoken language. }

\par{\textbf{大降り(おおぶり) }: This is the antonym of 小降り. It refers to rain coming down with great intensity, and it can be translated as "downfall", "heavy rain", "torrential rain", but remember that it is referring to intensity rather than quantity. Though large amounts of rain come with great intensity, the angle the word takes is different than with 大雨. This word is also commonly used in the spoken language. }

\par{\textbf{小粒(こつぶ)の雨 }: This is small drop rain. This word is very common in the spoken language. }

\par{\textbf{大粒(おおつぶ)の雨 }: This is big drop rain. This word is very common in the spoken language. }

\par{\textbf{土砂降り(どしゃぶり) }: Usually followed by the copula although the colloquial verb form 土砂降る does exist, this is a very natural way of saying "downpour". The rain in a 土砂降り is of 大粒 size. }

\par{\textbf{豪雨(ごうう) }: This is a slightly more technical word for "downpour", which is why it is found in more technical words like 集中豪雨 (localized torrential flooding). This word refers to large amounts of rain coming down with great intensity. It's like a combination of 大雨 and 大降り. }

\par{\textbf{雨脚・雨足(あまあし) }: This very common word means "passing shower". The word may uncommonly be read as あめあし. }

\par{\textbf{涙雨(なみだあめ) }: This can be used to refer to rain akin to sad tears or it can refer to a light amount of rain falling. }

\par{\textbf{弱雨(じゃくう) }: This is a made-up word that no one uses, and it's meant to compliment the opposite, 強雨, which does exist. In the spoken language, you would just say \textbf{弱い雨 }, which is used in meteorology as well. }

\par{\textbf{強雨(きょうう) }: This would be \textbf{強い雨 }in the spoken language, but this word does exist in technical situations. The wind related term 強風 is used in the spoken language whereas 強雨, which is not that congruent, but what do you expect? }

\par{\textbf{微雨(びう) }: This is a literary word for 細か\{な・い\}雨. This is light, drizzly rain. }

\par{\textbf{細雨(さいう) }: This is a literary word for 細か\{な・い\}雨. Though also referring to a drizzle, it may also refer to misty rain. However, this is expressed in the spoken language as 霧雨. }

\par{\textbf{(小)糠雨((こ)ぬかあめ) }: This is rain where the rain drops are incredibly thin and misty. This is acceptable in the spoken language, though it isn't something that you would casually drop in a conversation. 小- is used as an intensifier. }

\par{\textbf{霧雨(きりさめ) }: This is the spoken language word for "misty rain". }

\par{\textbf{煙雨(えんう) }: This is a literary word that comes from a comparison of smoke to misty rain. }

\par{\textbf{俄雨(にわかあめ) }: A combination of にわか (sudden\slash abrupt) and 雨, 俄雨 refers to a rain that suddenly pours and then just ends. This is somewhat specific, but it doesn't prevent it from being used in the spoken language. }

\par{\textbf{夕立(ゆうだち) }: This is a shower that lasts into the evening that starts in the latter part of the afternoon in the summer. The rain comes down hard and suddenly, and lightning typically accompanies it. }

\par{\textbf{急雨(きゅうう) }: This is a rare word for 俄雨. 急な雨 would be used in the spoken language to replace this. }

\par{\textbf{白雨(はくう) }: This is a very literary term for the same thing as 俄雨. }

\par{\textbf{驟雨(しゅうう) }: This is an insanely rare, literary word for the same thing as 俄雨. }

\par{\textbf{長雨(ながあめ) }: This is a commonly spoken word meaning "long rain" in the sense that rain continues on for several days. }

\par{\textbf{霖(ながめ) }: Coming from a contraction of ながあめ, this word can often be found in things like poetry. }

\par{\textbf{霖雨(りんう) }: This is rather literary term that means the same thing as 長雨. }

\par{\textbf{陰霖(いんりん) }: This is a very rare word for 長雨. }

\par{\textbf{陰雨(いんう) }: This is a literary word referring to gloomy rain. This can be expressed in the spoken language with 陰気な雨. }

\par{\textbf{淫雨(いんう) }: This is a literary word that refers to long lasting rains, but bad consequences are implied to crops. }

\par{\textbf{時雨(しぐれ) }: This is a somewhat poetic word that refers to the scattered rains that start in late autumn and end in early winter. It is seen in a lot of compound expressions such as 時雨空, 秋時雨, 初時雨, 村時雨, etc. It even has the verb form 時雨れる, although 時雨が降る is far more common. }

\par{\textbf{梅雨(つゆ・ばいう) }: The first reading is native and the second reading is Sino-Japanese, but both are extremely common words in both the written and spoken languages. Though the latter may be slightly less frequent, both words refer to the rainy season which lasts in Japan from June to July. ばいう may also be rarely spelled as 黴雨. ばいう does get used more in technical phrases such as 梅雨前線 (rainy season front). Some years, the rainy season may be dry, 空梅雨(からつゆ). Entering the rainy season is called 梅雨入り・入梅(つゆいり). A dry spell in the rainy season is called a 梅雨晴れ(つゆばれ). The end of the season is called the 梅雨明け(つゆあけ). }

\par{\textbf{雨季(うき) }: This is another commonly used word for "rainy season", but the main difference is that this is not Japan-specific. When stressing on a particular part of the year, this word may alternatively be spelled as 雨期. }

\par{\textbf{五月雨(さみだれ) }: Also rarely read as さつきあめ, this somewhat common word refers to continuous long rains in the fifth lunar month. This is the same as 梅雨, but the basis of meaning is slightly different. }

\par{\textbf{春雨(はるさめ) }: In relation to rain, this word refers to the thin rain that quietly falls in spring. This word is common in the spoken language, but it can also be read as しゅんう in the written language. Aside from rain, it can also mean "thin bean starch noodles". This is clearly analogous to the shape of "spring rains". }

\par{\textbf{春霖(しゅんりん) }: This is a rare literary word for 春雨. }

\par{\textbf{菜種梅雨(なたねづゆ) }: This is a rare, literary, yet native word for 春霖. }

\par{\textbf{夏雨(かう) }: This is very rare and practically only found in the technical phrase 温暖夏雨気候 (temperate rainy summer climate). なつさめ is a viable phrase, but it is not particular common, and not all speakers will agree that it is a word. }

\par{\textbf{秋雨(あきさめ) }: Read as あきさめ in the spoken language and as しゅうう in the written language, this word refers to autumn rain. This especially refers to long autumnal rains, and it is not appropriate to refer to 俄雨 in this season. The word itself in the spoken language is common, especially in autumn\dothyp{}\dothyp{}\dothyp{} }

\par{\textbf{秋霖(しゅうりん) }: This is a rare literary word for 秋雨. }

\par{\textbf{冬雨(ふゆさめ) }: This is not a word to many speakers, but it does sometimes gets used. People assume that if you can say はるさめ and あきさめ, then there shouldn't be any problem with ふゆさめ. As it is a contested word, use something else for school assignments. }

\par{\textbf{春の雨; 夏の雨; 秋の雨: 冬の雨 }: Aside from just specific terms like above, you could just use these phrases to just mention rain in a certain season. }

\par{\textbf{氷雨(ひさめ) }: This means "freezing rain", referring to really cold rain, but t can also mean "hail". Hail, though, is typically the next two words. }

\par{\textbf{雹(ひょう) }: Equivalent to "hail(stone)", 雹 fall down from cumulonimbus clouds in a thunderstorm. The diameter of one is at least five millimeters. }

\par{\textbf{霰(あられ) }: These is regular, small-sized hail as the result of mist crystallizing in the air. This is far more common in the spoken language than 雹. Both words are typically spelled in ひらがな. }

\par{\textbf{霙(みぞれ) }: This word means "sleet" and is a rather common word, although it is typically spelled in ひらがな. }

\par{\textbf{雨混(ま)じりの雪 }: Rain-mixed snow. This phrase and the next one are both common expressions. }

\par{\textbf{雪混じりの雨 }: Snow-mixed rain. The major precipitation is opposite of the previous expression. }

\par{\textbf{雨後雪(あめのちゆき) }: Often seen in weather reports, this means "snow after rain”. }

\par{\textbf{雨下(うか) }: A rare written language word for raining, which is typically 雨が降る in the spoken language. }

\par{\textbf{風雨(ふうう) }: Literally "wind and rain", this is actually a rather common word. However, it's a tad bit too formal to be seen in colloquial conversations. }

\par{\textbf{暴風雨(ぼうふうう) }: This is a technical yet still frequently used word for "storm". }

\par{\textbf{雨模様(あまもよう) }: Also read as あめもよう, this is the most common way to refer to "signs of rain". }

\par{\textbf{雨催い(あまもよい) }: This is an old-fashioned way of saying the previous thing, and it can be alternatively read as あめもよい. }

\par{\textbf{雨気(うき) }: This is the Sino-Japanese word for "signs of rain", and as you can guess, it's very formal and basically not used. This can also be read as あまけ, which is more common. }

\par{\textbf{雨氷(うひょう) }: This is rain that is colder than 0℃, and it quickly hardens when it hits the surface of something. Though translated as "freezing rain", it is quite technical and practical only used in meteorological formats. }

\par{\textbf{横降り(よこぶり) }: This word is a commonly spoken word meaning "driving rain". However, it can also refer to snow. The point is that it refers to rain being made to fall slanted due to gale winds. }

\par{\textbf{吹き降り(ふきぶり) }: This word is a commonly spoken word that refers to strong wind accompanied with heavy rain, thus making it synonymous with 横降り. In this case, however, the emphasis is on the wind itself, not the direction of the rain. }

\par{\textbf{天泣(てんきゅう) }: Though more natural to just say 雲がないのに雨が降る, this is not as esoteric as some of these other words. }

\par{\textbf{晴後雨(はれのちあめ) }: Often seen during news weather reports, this means "rain after being clear". }

\par{\textbf{雷雨(らいう) }: This word means "thunderstorm" and is an extremely common word. }

\par{\textbf{降雨(こうう) }: This is "precipitation" and is just as formal as the English word. The spoken word for this is 雨降り. }

\par{\textbf{人工(降)雨(じんこう(こう)う) }: Artificial precipitation. }

\par{\textbf{放射能雨(ほうしゃのうう) }: More naturally read as ほうしゃのうあめ, this word means "radioactive rain". Highly polluted\slash radioactive rain can also be called 黒い雨, which is used in the spoken language. }

\par{\textbf{酸性雨(さんせいう) }: ”Acid rain". }

\par{\textbf{流星雨(りゅうせいう) }: ”Meteor shower". }

\par{\textbf{雨露(うろ) }: This word means "rain and dew", but it is usually reading as あめつゆ for the spoken language, which can also be more naturally worded as 雨と露. }

\par{\textbf{涼雨(りょうう) }: 涼しい雨 in the spoken language, this word refers to lovely rain in the summer that cools everything down. }

\par{\textbf{冷雨(れいう) }: This word means "cold rain", but it is a formal word and would be replaced with 冷たい雨 in the spoken language. }

\par{\textbf{寒雨(かんう) }: This is a literary word for "wintry rain". This could also be expressed as 寒々とした雨, but this is still not totally colloquial. }

\par{\textbf{夜雨(やう) }: If used in the spoken language it would be 夜の雨, and to be more poetic, you could read 夜雨 as よさめ and 夜の雨  as よのあめ. These words simply mean "night rain". A "rainy night" would be 雨夜(あまよ). }

\par{\textbf{緑雨(りょくう) }: This word refers to rain that falls in the season when new greens grow. However, this is relative to what part of the world you live in, and although you would assume this would be used for around April and May, the word is just uncommon and literary. }

\par{\textbf{甘雨(かんう) }: Homophonous yet even rarer than 寒雨, this word refers to rain that graciously falls to replenish greens. }

\par{\textbf{慈雨・滋雨(じう) }: This is a slightly more common word for the same thing as 甘雨, though it would be more appropriate in the spoken language to say something like 草木を育てる雨. Notice how it is often more common to use the definition of a more technical word in the spoken language. This phrase is actually short for 干天慈雨. }

\par{\textbf{喜雨(きう) }: Joyful, pleasant rain that can help with the crops. This is just as rare and literary as the other words for the same thing. }

\par{\textbf{恵雨(けいう) }: This word also refers to blessed rain for crops, but it's not used in the spoken language either. This would naturally be expressed as 恵みの雨. }

\par{\textbf{十雨(じゅうう) }: This means "refreshing rain once in ten days", but this is never used. The similar four character idiom 十風五雨・五風十雨 (halcyon weather) is more common. Just saying 十日ごとに雨が降る will be fine. }

\par{\textbf{凍雨(とうう) }: This somewhat technical word may either refer to rain as cold as ice or ice pellets. However, both concepts in the spoken language would most likely be expressed as 凍るほど冷たい雨・氷のように冷たい雨 and 氷の粒 respectively. }

\par{\textbf{晴雨(せいう) }: This is a technical word for clear and rainy weather. }

\par{\textbf{弾雨(だんう) }: This is a shower of bullets. It is a literary word, so you will more than likely hear the definition used in the spoken language instead. This is a contraction of 弾丸飛雨(だんがんひう). }

\par{\textbf{法雨(ほうう) }: ”Shower of Dharma", also known as のりのあめ. This is not a common word. }

\par{\textbf{快雨(かいう) }: This is a rare word for rain that makes oneself feel refreshed. }

\par{\textbf{穀雨(こくう) }: This literally means "grain rain", and it is a solar term for ~April 20th. }

\par{\textbf{祈雨(きう) }: This is a very rare word for praying for rain. 雨乞い is used in the spoken language. }

\par{\textbf{止雨(しう) }:  This is a very rare word for praying for rain to stop. }

\par{\textbf{樹雨(きさめ) }:  This is a rather easy word describing raindrops that fall of the leaves and branches of trees. This is not, though, a common word. }

\par{\textbf{私雨(わたくしあめ) }: An obscure and hardly used term used to refer to rain that mainly hits a certain area. }

\par{\textbf{遣(や)らずの雨 }: This is rain that seems to be preventing you or someone from leaving something. This is an interesting phrase, but it isn't really common. }

\par{\textbf{黒風白雨(こくふうはくう) }: This is an uncommon four character idiom that means "sudden rain shower in a dust storm". }

\par{\textbf{篠突く雨(しのつくあめ) }: This is a rather literary word referring to heavy rain that looks as if it is bamboo bound together and dropped. Something like ざあざあ降る would replace it in the spoken language. }

\begin{center}
\textbf{Common Rain Associated Words } 
\end{center}

\par{ The table below lists many rain associated words that you can add to your vocabulary. Most of these words are quite practical and can be used in everyday language. }

\begin{ltabulary}{|P|P|P|P|P|P|P|P|P|}
\hline 

雨戸 & あまど & Sliding storm shutter & 雨具 & あまぐ & Rain gear & 雨着 & あまぎ & Raincoat \\ \cline{1-9}

雨天 & うてん & Rainy weather & 雨滴 & うてき & Raindrop & 雫・滴 & しずく & Raindrop \\ \cline{1-9}

雨林 & うりん & Rain forest & 雨雲 & あまぐも & Rain cloud & 雨水 & あまみず & Rainwater \\ \cline{1-9}

雨水 & うすい & Rain water & 雨間 & あまあい & Break in rain & 雨音 & あまおと & Sound of rain \\ \cline{1-9}

雨傘 & あまがさ & Umbrella & 雨靴 & あまぐつ & Overshoes & 雨空 & あまぞら & Rainy sky \\ \cline{1-9}

雨垂れ & あまだれ & Raindrop & 雨粒 & あまつぶ & Raindrop & 雨樋 & あまどい & Rain gutter \\ \cline{1-9}

雨漏り & あまもり & Rain leak & 雨風 & あめかぜ & Rain \& wind & 雨上り & あめあがり & Rain letup \\ \cline{1-9}

雨続き & あめつづき & Rain continuation & 雨合羽 & あまがっぱ & Rain poncho & 雨後 & うご & After rain \\ \cline{1-9}

\end{ltabulary}

\begin{center}
 \textbf{Rain Onomatopoeia } 
\end{center}

\par{ What about onomatopoeia for rain? Along with the ones you got from earlier, plenty exist to describe the exact sound of rain\slash water\slash tears falling. This list isn't exhaustive, but after all of these rain words, you probably can't stand any more. }

\begin{ltabulary}{|P|P|P|P|}
\hline 

ぽつぽつ & Rain falling down just a little & ぼつぼつ & Same as ぽつぽつ \\ \cline{1-4}

しとしと & Rain quietly falling & ざあざあ & Rain pouring \\ \cline{1-4}

ざんざん & The same as ざあざあ & ぽとぽと & Rain falling in big drops \\ \cline{1-4}

しょぼしょぼ & Drizzling & ぐずぐず & Drizzling \\ \cline{1-4}

ぱらぱら & For rain to spatter & さっと & Suddenly \\ \cline{1-4}

\end{ltabulary}
\hfill\break
    