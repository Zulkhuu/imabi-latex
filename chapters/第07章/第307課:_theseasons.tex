    
\chapter{The Seasons}

\begin{center}
\begin{Large}
第307課: The Seasons 
\end{Large}
\end{center}
 
\par{ The seasons in Japan occur at the same time as the rest of the Northern Hemisphere. In this lesson we will learn about events and vocab concerning the seasons. The four seasons, 四季, are recited as "春夏秋冬" in Japanese. A season (季節) has a native and a formal term. }

\begin{ltabulary}{|P|P|P|P|P|P|}
\hline 

Spring & 春、春季 & はる、しゅんき & Fall & 秋、秋季 & あき、しゅき \\ \cline{1-6}

Summer & 夏、夏季 & なつ、かき & Winter & 冬、冬季 & ふゆ、とうき \\ \cline{1-6}

\end{ltabulary}
      
\section{Spring}
 
\par{  The Japanese proverb 我が世の春を謳歌(おうか)する means "enjoy the height of one's prosperity", referring to the lovely attitude towards spring. In literature spring is often emphasized with the epithets 梓弓 and 弓を張る. }

\begin{ltabulary}{|P|P|P|}
\hline 

Event & Date & Description \\ \cline{1-3}

花見 & 4月 & Every year thousands go look at cherry blossoms (桜). \\ \cline{1-3}

雛祭り & 3月3日 & Households are decorated with 雛人形. \\ \cline{1-3}

卒園式 \hfill\break
卒業式 \hfill\break
人事異動 & 3月 & (Kindergarten) graduations and personnel changes \\ \cline{1-3}

入園式 \hfill\break
入学式 \hfill\break
入社式 & 4月 & (Kindergarten) enrollment and new employees' ceremonies \\ \cline{1-3}

潅仏会(かんぶつえ) & 4月8日 & The birthday of Buddha. \\ \cline{1-3}

イースター & 4月 &  \\ \cline{1-3}

母の日 & First 日曜日 of 5月 & Mother's Day \\ \cline{1-3}

\end{ltabulary}
      
\section{Summer}
 
\begin{ltabulary}{|P|P|P|}
\hline 

 Event & Date & Description \\ \cline{1-3}

天神祭り & 6~7月 & The 天満宮(てんまんぐう) and the 天神社(てんじんじゃ) shrines celebrate for a month to commemorate the death of 菅原道真(すがわらのみちざね). Signature items includes お迎(むか)え人形, どんどこ船, 龍踊り, and the ギャル御輿(みこし). 
\\ \cline{1-3}

七夕 & 7月 & 七夕(たなばた) is based on the myth that the two stars, 織姫 (Vega) and 牽牛・けんぎゅう (Altair), meet each other but once a year. \\ \cline{1-3}

お盆 & 7月15日 & Honors the return of the dead with dances 盆踊り and 大太鼓 drums. \\ \cline{1-3}

夏休み &  & Summer vacation. \\ \cline{1-3}

花火大会 &  & Fireworks festivals. \\ \cline{1-3}

\end{ltabulary}
      
\section{Autumn}
 
\par{ As the proverb 秋の日は釣瓶落とし says, fall sets as soon as a bucket drops into a well. There are many proverbs and words in relationship to fall. Below are some of the most important of these phrases. }

\begin{ltabulary}{|P|P|}
\hline 

高秋 & An autumn with a perfectly blue sky. \\ \cline{1-2}

素秋 & Equivalent to autumn being perfectly golden. \\ \cline{1-2}

白秋 & Equivalent to autumn being perfectly golden. \\ \cline{1-2}

白帝 & The god that controls autumn. \\ \cline{1-2}

金秋 & A golden autumn. \\ \cline{1-2}

三秋 & The three parts of fall. \\ \cline{1-2}

九秋 & The three months of fall. \\ \cline{1-2}

天高く馬肥ゆる秋 & Autumn with the sky clear and blue, and horses growing stout. \\ \cline{1-2}

一日三秋 & Waiting with each moment feeling like an eternity. \\ \cline{1-2}

一日千秋 & Waiting with each moment feeling like an eternity. \\ \cline{1-2}

物言えば唇寒し秋の風 & Silence is golden. \\ \cline{1-2}

女心と秋の空 & A woman's heart and autumn weather. \\ \cline{1-2}

秋の扇 & A woman who has lost a man's affection. \\ \cline{1-2}

秋の鹿は笛に寄る & People may bring about their demise for love. \\ \cline{1-2}

春秋に富む & To be young and have a bright future. \\ \cline{1-2}

\end{ltabulary}
      
\section{Winter}
 
\par{ Winter is a wonderful time. Winter starts on the winter solstice (冬至). 暖冬 means "warm winter" and 厳冬 "means "cold winter". People throw 雪玉 (snowballs) and make 雪だるま (snowmen). As people wait for 真冬日, the day that the temperature drops below 0, many others wait for the holidays and events like 冬の祭り. }

\begin{ltabulary}{|P|P|P|}
\hline 

Event & Date & Description \\ \cline{1-3}

クリスマス & 12月25日 & Christmas usually loses religious value. \\ \cline{1-3}

お歳暮 & 12月 & Giving particular gifts to those that showed you kindness in the year. \\ \cline{1-3}

忘年会 & 年末 & End of the year party to  literally "forget" the year. \\ \cline{1-3}

年越し & 12月31日 & New  Year's Eve. \\ \cline{1-3}

新年会 & 1月 & A party held for the beginning of the new year. \\ \cline{1-3}

節分 & \hfill\break
& The last day of winter is commemorated by bean scattering. \\ \cline{1-3}

\end{ltabulary}
      
\section{National Holidays}
 
\par{ Japanese national holidays used to all have names based off of Shintoism and Buddhism. But, after World War II, the terms became secularized. Holidays were established in Japan by the Public Holiday Law, 国民の祝日に関する法律. The practice of a Monday becoming a holiday because Sunday is one is called 振替休日. A holiday made when a holiday is before and after a day is called a 国民の休日. }

\begin{ltabulary}{|P|P|P|P|}
\hline 

元日 & New Year's Day & 1月1日 & The beginning of the year! \\ \cline{1-4}

成人の日 & Coming of Age Day & 2nd 月曜日 of 1月 & Congratulate those that are now 20. \hfill\break
\\ \cline{1-4}

建国記念の日 & Foundation Day & 2月11日 & Japan's birthday \\ \cline{1-4}

春分の日 & Vernal Equinox Day & ~3月20日 & Admire nature. \\ \cline{1-4}

昭和の日 & Showa Day & 4月29日 & Emperor Showa's birthday \\ \cline{1-4}

憲法記念日 & Constitution Memorial Day & 5月3日 & Constitution enactment celebration \\ \cline{1-4}

みどりの日 & Greenery Day & 5月4日 & Be grateful for Mother Earth! \\ \cline{1-4}

こどもの日 & Children's Day & 5月5日 & A fabulous idea \hfill\break
\\ \cline{1-4}

海の日 & Marine Day & 3rd 月曜日 of 七月 & Be grateful to the ocean. \\ \cline{1-4}

敬老の日 & Respect-for-the-Aged Day & 3rd 月曜日 of 九月 & Respect old people! \\ \cline{1-4}

秋分の日 & Autumnal Equinox Day & ~9月23日 &  \\ \cline{1-4}

体育の日 & Health and Sports Day & 2nd 月曜日 of 十月 & Get a workout! \\ \cline{1-4}

文化の日 & Culture Day & 10月3日 & Marks announcement of  constitution. 
\\ \cline{1-4}

天皇誕生日 & The Emperor's Birthday & 12月23日 & A good excuse for a holiday \\ \cline{1-4}

\end{ltabulary}
      
\section{五節句}
 
\par{ The are five 節句 (seasonal festivals) observed in Japan, and their dates are based off of the lunar calendar. The dates are now confused because the Western calendar is in standard use. }

\begin{ltabulary}{|P|P|P|P|P|}
\hline 

Number & Name of Day & Name of Event & Traditional Date & New Date \\ \cline{1-5}

1 & 人日 (じんじつ) \hfill\break
& 七草の節句 & 7th day of 1st month \hfill\break
& January 7 \\ \cline{1-5}

2 & 上巳 (じょうし) & 桃の節句 & 3rd day of 3rd month \hfill\break
& March 3 \\ \cline{1-5}

3 & 端午 (たんご) \hfill\break
& 端午の節句 & 5th day of 5th month \hfill\break
& May 5 \\ \cline{1-5}

4 & 七夕 (たなばた) \hfill\break
& 七夕の節句 & 7th day of 7th month \hfill\break
& July 7 \\ \cline{1-5}

5 & 重陽 (ちょうよう) \hfill\break
& 菊の節句 & 9th day of 9th month \hfill\break
& September 9 \\ \cline{1-5}

\end{ltabulary}
      
\section{正月}
 
\par{ The Japanese New Year used to start at the same time as it does in China. Now, it happens on January 1st. The New Year is called 正月. January 1st is 元旦, and New Year's Eve is 大晦日. }

\par{People send 年賀状 (New Year cards), except to those that have had a lost loved one in the year. For those that have lost someone, they send out a 喪中葉書き saying not to send 年賀状. What is written in a 年賀状? You could see some of the following. }

\begin{ltabulary}{|P|}
\hline 

今年もよろしくお願いします \\

明けましておめでとうございます \\

謹賀新年 \\

\end{ltabulary}

\par{On December 31 the つらがね is rung 108 times to ward off evil spirits. There are a lot of things that are only eaten at New Years.  餅 is a rice paddy cake that a lot of people choke on and die each year. 餅 becomes a decoration called 鏡餅. It consists of two 餅 with a 橙 (bitter orange). Some more お節料理 (New Year dishes) include: }

\begin{ltabulary}{|P|P|}
\hline 

蒲鉾(かまぼこ) & Fish cakes \\ \cline{1-2}

昆布 & (Boiled) seaweed \\ \cline{1-2}

黒豆 & Sweetened black soybeans \\ \cline{1-2}

お雑煮 & Soup with 餅 and other ingredients \\ \cline{1-2}

七草粥(ななくさがゆ) & Seven-herb rice soup \\ \cline{1-2}

金平牛蒡(きんぴらごぼう) & Simmered burdock root \\ \cline{1-2}

\end{ltabulary}

\par{Kids receiving money is called お年玉. People play games such as the card game カルタ, kite flying (凧揚(たこあ)げ), etc. This last game involves blindfolding and decorating  someone with paper face parts. カルタ involves figuring out what line of poetry comes from which poem of the  百人一首, a Classical Japanese compilation. }

\par{To mark the first event of something in the new year, use 初-. }

\begin{ltabulary}{|P|P|P|}
\hline 



初詣 & はつもうで & Going to the shrine for the first time of the year. \hfill\break
\\ \cline{1-3}

初日の出 & はつひので & First sunrise of the year. \hfill\break
\\ \cline{1-3}

初夢 & はつゆめ & First dream of the year \\ \cline{1-3}

初顔合わせ & はつかおあわせ & First meeting of the year \\ \cline{1-3}

\end{ltabulary}
    