    
\chapter{Numbers X}

\begin{center}
\begin{Large}
第315課: Numbers X: Measurements 
\end{Large}
\end{center}
 
\par{ The commonality that America has with Japan is that there are two competing systems of measurement with one being far more prevalent than the other. The majority of measurements that you will see in Japan will be in the universal metric system. However, there are situations in which native measure words are used instead. Rarer ones, as you can imagine, still live on in literature, set phrases, and other parts of life. }

\par{ This lesson's primary purpose is to showcase these systems. Whether individual words are used or not will still be mentioned, but the entire lists will be given for those who wish to be pedantic.  }
      
\section{SI Units}
   Even if you are an American, the following terms should be very familiar to if you have studied the sciences. The only non-Western word below is 秒. However, there are times in Japanese when you do use セコンド・セカンド. These words are typically used in longer loan phrases.  
\begin{ltabulary}{|P|P|P|}
\hline 

English & Quantity & Japanese \\ \cline{1-3}

Meter & Distance & メートル \\ \cline{1-3}

Gram & Mass & グラム \\ \cline{1-3}

Second & Time & 秒 \\ \cline{1-3}

Ampere & Electric current & アンペア \\ \cline{1-3}

Kelvin & Heat & ケルビン \\ \cline{1-3}

Candela & Luminous intensity & カンデラ \\ \cline{1-3}

Mole & Amount & モル \\ \cline{1-3}

\end{ltabulary}
  Expressing things like 時速 and 風速 complicate things. I f you driving 60 km\slash h, you could certainly say 60キロメートル・パー・アワー. In practicality, just キロ could be used. This would be ambiguous with the actual distance you have traveled, but this is no different than in English. You could avoid English loans by saying 60キロ毎時間, but this is far from being colloquial. If you were dealing with seconds, you could use 秒, セコンド, or セカンド. \slash  could be read as パー or 毎. However, if you choose 毎, you should use 秒. If you use パー, you should use セカンド・セカンド. What if you had m・s. If you see this, you would read it is as メートル、セカンド or メートル、セコンド.   風速 is more difficult. The typical unit for this is m\slash s. This is read as メーター・パー・セカンド. A more technical reading of second in this field is セック. This comes from the English abbreviation sec. Now, because reading units (単位) is not universally standardized, the things mentioned above applies. For 風量 (air flow\slash volume), you have words like 立米 for m³. So m³\slash h = リュウベイ・パー・アワー. In actual application, you are to read things how your colleagues and or boss does. Because English and Japanese terms both exist for all of these things, this is only natural to happen.   The most important thing to not confuse is what is meant by average comments such as 風速50メートル and 時速50キロ(メートル). If you just get a number for any of these situations, you need to understand the base unit for these concepts. Thus, it should be a given that people are referred to m\slash s in regards to wind speed but km\slash h when referring to the speed of one's car. \textbf{\hfill\break
More Units \hfill\break
\hfill\break
}
\begin{ltabulary}{|P|P|P|P|P|P|P|P|}
\hline 

面積 & Area & 長さ & Length & 重さ & Weight & 体積・容積 & Volume\slash Capacity \\ \cline{1-8}

平方センチメートル & cm 2 & ミリ[メートル] & mm & ミリグラム & mg & 立方センチメートル & cm 3 \\ \cline{1-8}

平方メートル & m 2 & センチ[メートル] & cm & グラム & g & 立方メートル & m 3 \\ \cline{1-8}

平方キロメートル & km 2 & メートル & m & キロ[グラム] & kg & ミリリットル & ml \\ \cline{1-8}

 &  & キロ[メートル] & km & トン & Ton & シーシー & cc \\ \cline{1-8}

 &  &  &  &  &  & リットル & ℓ = Liter \\ \cline{1-8}

\end{ltabulary}
\textbf{Prefixes of Magnitude }
\begin{ltabulary}{|P|P|P|P|P|P|P|P|}
\hline 

10 -1 & d & Deci- & デシ & 10 1  & da & Deca- & デカ  \\ \cline{1-8}

10 −2 & c & Centi- & センチ & 10 2 & h & Hecto- & ヘクト \\ \cline{1-8}

10 −3 & m & Milli- & ミリ & 10 3  & k & Kilo- & キロ \\ \cline{1-8}

10 −6 & μ & Micro- & マイクロ & 10 6 & M & Mega- & メガ \\ \cline{1-8}

10 −9 & n & Nano- & ナノ & 10 9 & G & Giga- & ギガ \\ \cline{1-8}

10 −12 & p & Pico- & ピコ & 10 12 & T & Tera- & テラ \\ \cline{1-8}

10 −15 & f & Femto- & フェムト & 10 15 & P & Peta- & ぺタ \\ \cline{1-8}

10 −18 & a & Atto- & アト & 10 18 & E & Exa- & エクサ \\ \cline{1-8}

10 −21 & z & Zepto- & ゼプト & 10 21 & Z & Zetta- & ゼタ \\ \cline{1-8}

10 −24 & y & Yocto- & ヨクト & 10 24 & Y & Yotta- & ヨタ \\ \cline{1-8}

\end{ltabulary}
 
\begin{center}
 \textbf{Derived Units }
\end{center}

\begin{ltabulary}{|P|P|P|P|}
\hline 

Unit & Japanese & Symbol & Quantity \\ \cline{1-4}

Hertz & ヘルツ & Hz & Frequency \\ \cline{1-4}

Radian & ラジアン & rad  & Angle \\ \cline{1-4}

Newton & ニュートン & N & Force \\ \cline{1-4}

Pascal & パスカル & Pa & Pressure \\ \cline{1-4}

Joule & ジュール & J & Energy \\ \cline{1-4}

Volt & ボルト & V & Voltage \\ \cline{1-4}

Ohm & オーム &  Ω & Impedance \\ \cline{1-4}

Watt & ワット & W & Power \\ \cline{1-4}

Coulomb & クーロン & C & Electric charge \\ \cline{1-4}

Farad & ファラド & F & Electric capacitance \\ \cline{1-4}

Lumen & ルーメン & lm & Luminous flux \\ \cline{1-4}

Celsius & セルシウス & ℃ & Temperature \\ \cline{1-4}

Siemens & ジーメンス & S & Electrical conductance \\ \cline{1-4}

Tesla & テスラ & T & Strength of magnetic fields \\ \cline{1-4}

Henry & ヘンリー & H & Inductance \\ \cline{1-4}

Lux & ルクス & lx & Luminance \\ \cline{1-4}

Weber & ウェーバ & Wb & Magnetic flux \\ \cline{1-4}

Sievert & シーベルト & Sv & Equivalent dose \\ \cline{1-4}

Becquerel & ベクレル & Bq & Radioactivity \\ \cline{1-4}

Katal & カタール & kat \hfill\break
& Catalytic activity  \\ \cline{1-4}

Gray & グレイ & Gy & Absorbed dose \\ \cline{1-4}

\end{ltabulary}
 
\par{\textbf{Usage Note }:  シーベルト and ベクレル have become frequently used in the news ever since the Fukushima Nuclear Power Plant Disaster in 2011. }

\begin{ltabulary}{|P|P|P|}
\hline 

English & Quantity & Japanese \\ \cline{1-3}

Meter & Distance & メートル \\ \cline{1-3}

Gram & Mass & グラム \\ \cline{1-3}

Second & Time & 秒(びょう) \\ \cline{1-3}

Ampere & Electric current & アンペア \\ \cline{1-3}

Kelvin & Heat & ケルビン \\ \cline{1-3}

Candela & Luminous intensity \hfill\break
& カンデラ \\ \cline{1-3}

Mole & Amount & モル \\ \cline{1-3}

\end{ltabulary}
 
\par{Prefixes of Magnitude  }

\begin{ltabulary}{|P|P|P|P|P|P|P|P|}
\hline 

10 −1 & d & Deci- & デシ & 10 1 & da & Deca- & デカ \\ \cline{1-8}

10 −2 & c & Centi- & センチ & 10 2 & h & Hecto- & ヘクト \\ \cline{1-8}

10 −3 & m & Milli- & ミリ & 10 3 & k & Kilo- & キロ \\ \cline{1-8}

10 −6 & μ & Micro- & マイクロ & 10 6 & M & Mega- & メガ \\ \cline{1-8}

10 −9 & n & Nano- & ナノ & 10 9 & G & Giga- & ギガ \\ \cline{1-8}

10 −12 & p & Pico- & ピコ & 10 12 & T & Tera- & テラ \\ \cline{1-8}

10 −15 & f & Femto- & フェムト & 10 15 & P & Peta- & ぺタ \\ \cline{1-8}

10 −18 & a & Atto- & アト & 10 18 & E & Exa- & エクサ \\ \cline{1-8}

10 −21 & z & Zepto- & ゼプト & 10 21 & Z & Zetta- & ゼタ \\ \cline{1-8}

10 −24 & y & Yocto- & ヨクト & 10 24 & Y & Yotta- & ヨタ \\ \cline{1-8}

\end{ltabulary}

\par{Derived Units }

\begin{ltabulary}{|P|P|P|P|}
\hline 

Unit &  Japanese & Symbol & Quantity \\ \cline{1-4}

Hertz &  ヘルツ & Hz & Frequency \\ \cline{1-4}

Radian &  ラジアン & rad & Angle \\ \cline{1-4}

Newton &  ニュートン & N & Force \\ \cline{1-4}

Pascal &  パスカル & Pa & Pressure \\ \cline{1-4}

Joule &  ジュール & J & Energy \\ \cline{1-4}

Volt &  ボルト & V & Voltage \\ \cline{1-4}

Ohm &  オーム & Ω & Impedance \\ \cline{1-4}

Watt &  ワット & W & Power \\ \cline{1-4}

Coulomb &  クーロン & C & Electric charge \\ \cline{1-4}

Farad &  ファラド & F & Electric capacitance \\ \cline{1-4}

Lumen &  ルーメン & lm & Luminous flux \\ \cline{1-4}

Celsius &  セルシウス & °C & Temperature \\ \cline{1-4}

Siemens &  ジーメンス & S & Electrical conductance \\ \cline{1-4}

Tesla &  テスラ & T & Strength of magnetic fields \\ \cline{1-4}

Henry &  ヘンリー & H & Inductance \\ \cline{1-4}

Lux &  ルクス & lx & Luminance \\ \cline{1-4}

Weber &  ウェーバ & Wb & Magnetic flux \\ \cline{1-4}

Sievert &  シーベルト & Sv & Equivalent dose \\ \cline{1-4}

Becquerel &  ベクレル & Bq & Radioactivity \\ \cline{1-4}

Katal &  カタール & kat & Catalytic activity \\ \cline{1-4}

Gray &  グレイ & Gy & Absorbed dose \\ \cline{1-4}

\end{ltabulary}

\par{Note: シーベルト and ベクレル have become frequently used in the news ever since the Fukushima Nuclear Power Plant Disaster in 2011. }

\begin{ltabulary}{|P|P|P|}
\hline 

English & Quantity & Japanese \\ \cline{1-3}

Meter & Distance & メートル \\ \cline{1-3}

Gram & Mass & グラム \\ \cline{1-3}

Second & Time & 秒(びょう) \\ \cline{1-3}

Ampere & Electric current & アンペア \\ \cline{1-3}

Kelvin & Heat & ケルビン \\ \cline{1-3}

Candela & Luminous intensity \hfill\break
& カンデラ \\ \cline{1-3}

Mole & Amount & モル \\ \cline{1-3}

\end{ltabulary}
 
\par{Prefixes of Magnitude  }

\begin{ltabulary}{|P|P|P|P|P|P|P|P|}
\hline 

10 −1 & d & Deci- & デシ & 10 1 & da & Deca- & デカ \\ \cline{1-8}

10 −2 & c & Centi- & センチ & 10 2 & h & Hecto- & ヘクト \\ \cline{1-8}

10 −3 & m & Milli- & ミリ & 10 3 & k & Kilo- & キロ \\ \cline{1-8}

10 −6 & μ & Micro- & マイクロ & 10 6 & M & Mega- & メガ \\ \cline{1-8}

10 −9 & n & Nano- & ナノ & 10 9 & G & Giga- & ギガ \\ \cline{1-8}

10 −12 & p & Pico- & ピコ & 10 12 & T & Tera- & テラ \\ \cline{1-8}

10 −15 & f & Femto- & フェムト & 10 15 & P & Peta- & ぺタ \\ \cline{1-8}

10 −18 & a & Atto- & アト & 10 18 & E & Exa- & エクサ \\ \cline{1-8}

10 −21 & z & Zepto- & ゼプト & 10 21 & Z & Zetta- & ゼタ \\ \cline{1-8}

10 −24 & y & Yocto- & ヨクト & 10 24 & Y & Yotta- & ヨタ \\ \cline{1-8}

\end{ltabulary}

\par{Derived Units }

\begin{ltabulary}{|P|P|P|P|}
\hline 

Unit &  Japanese & Symbol & Quantity \\ \cline{1-4}

Hertz &  ヘルツ & Hz & Frequency \\ \cline{1-4}

Radian &  ラジアン & rad & Angle \\ \cline{1-4}

Newton &  ニュートン & N & Force \\ \cline{1-4}

Pascal &  パスカル & Pa & Pressure \\ \cline{1-4}

Joule &  ジュール & J & Energy \\ \cline{1-4}

Volt &  ボルト & V & Voltage \\ \cline{1-4}

Ohm &  オーム & Ω & Impedance \\ \cline{1-4}

Watt &  ワット & W & Power \\ \cline{1-4}

Coulomb &  クーロン & C & Electric charge \\ \cline{1-4}

Farad &  ファラド & F & Electric capacitance \\ \cline{1-4}

Lumen &  ルーメン & lm & Luminous flux \\ \cline{1-4}

Celsius &  セルシウス & °C & Temperature \\ \cline{1-4}

Siemens &  ジーメンス & S & Electrical conductance \\ \cline{1-4}

Tesla &  テスラ & T & Strength of magnetic fields \\ \cline{1-4}

Henry &  ヘンリー & H & Inductance \\ \cline{1-4}

Lux &  ルクス & lx & Luminance \\ \cline{1-4}

Weber &  ウェーバ & Wb & Magnetic flux \\ \cline{1-4}

Sievert &  シーベルト & Sv & Equivalent dose \\ \cline{1-4}

Becquerel &  ベクレル & Bq & Radioactivity \\ \cline{1-4}

Katal &  カタール & kat & Catalytic activity \\ \cline{1-4}

Gray &  グレイ & Gy & Absorbed dose \\ \cline{1-4}

\end{ltabulary}

\par{Note: シーベルト and ベクレル have become frequently used in the news ever since the Fukushima Nuclear Power Plant Disaster in 201 }
      
\section{Japanese Units}
 
\par{ Before the advent of standardized units, the Japanese used a system of units called the 尺貫法. Remember that these are all 助数詞 (counters)! }

\begin{center}
\textbf{Length: The }\textbf{尺 }
\end{center}
 
\par{The 尺 is the basis of the 尺貫法. The 曲尺 was used in carpentry, the 鯨尺 (25\% larger than the 曲尺) in clothing, and the 呉服尺 (1.2x larger) in Japanese traditional dress. }

\begin{ltabulary}{|P|P|P|P|}
\hline 

Unit & Reading & 尺 Equivalent & Meters \\ \cline{1-4}

毛 & もう & 1\slash 10,000 & .00003030 \\ \cline{1-4}

厘 & りん & 1\slash 1,000 & .0003030 \\ \cline{1-4}

分 & ぶ & 1\slash 100 & .003030 \\ \cline{1-4}

寸 & すん & 1\slash 10 & .03030 \\ \cline{1-4}

尺 & しゃく & 1 & .3030 \\ \cline{1-4}

間 & けん & 6 & 1.818 \\ \cline{1-4}

広 & ひろ & 6 & 1.818 \\ \cline{1-4}

丈 & じょう & 10 & 3.030 \\ \cline{1-4}

町 & ちょう & 360 & 109.1 \\ \cline{1-4}

里 & り & 12,960 & 3297 \\ \cline{1-4}

\end{ltabulary}

\par{\textbf{Word Notes }: }

\par{1. 広 is used for \textbf{depth }. }

\par{2. 里 previously stood for 600 meters. }
 
\par{\textbf{Idiom Note }: 一寸先は闇, literally "an inch ahead is darkness" is equivalent to "Who knows what tomorrow will bring?". }
 
\begin{center}
\textbf{Area: The }\textbf{坪 }
\end{center}

\begin{ltabulary}{|P|P|P|P|}
\hline 

Unit & Reading & 坪 Equivalent & Square Meters \\ \cline{1-4}

勺 & しゃく & 1\slash 100 & .03306 \\ \cline{1-4}

合 & ごう & 1\slash 10 & .3306 \\ \cline{1-4}

畳 & じょう & 1\slash 2 & 1.653 \\ \cline{1-4}

坪 & つぼ & 1 & 3.306 \\ \cline{1-4}

歩 & ぶ & 1 & 3.306 \\ \cline{1-4}

畝 & せ & 30 & 99.17 \\ \cline{1-4}

段・反 & たん・ & 300 & 991.17 \\ \cline{1-4}

町 & ちょう & 3000 & 9917 \\ \cline{1-4}

方里 & ほうり & 1555.2 & 15423 \\ \cline{1-4}

\end{ltabulary}

\par{\textbf{Word Notes }: }

\par{1. The 歩 is used in agriculture whereas the 坪 is used in construction . For units larger than 1, they are used for large areas such as forests. }

\par{2. 歩 is often added to 畝, 反, and 町 when reading out measurements. For example, 6反8畝歩. }

\begin{center}
\textbf{Volume: The }\textbf{升 }
\end{center}

\begin{ltabulary}{|P|P|P|P|}
\hline 

Unit & Reading & 升 Equivalent & Liters \\ \cline{1-4}

才 & さい & 1\slash 1000 & .001804 \\ \cline{1-4}

勺 & しゃく & 1\slash 100 & .01804 \\ \cline{1-4}

合 & ごう & 1\slash 10 & .1804 \\ \cline{1-4}

升 & しょう & 1 & 1.804 \\ \cline{1-4}

斗 & と & 10 & 18.04 \\ \cline{1-4}

石 & こく & 100 & 180.4 \\ \cline{1-4}

\end{ltabulary}
\textbf{Mass: The 匁 } 
\begin{ltabulary}{|P|P|P|P|}
\hline 

Unit & Reading & 匁 Equivalent & Grams \\ \cline{1-4}

分 & ぶ & 1\slash 10 & .375 \\ \cline{1-4}

匁・文目 & もんめ & 1 & 3.75 \\ \cline{1-4}

百目 & ひゃくめ & 100 & 375 \\ \cline{1-4}

斤(目) & きん(め) & 160 & 600 \\ \cline{1-4}

貫(目) & かん(め) & 1000 & 3750 \\ \cline{1-4}

\end{ltabulary}
\hfill\break
1. 氷を五十八貫目使ったわ、うちのねえさんにずいぶん厄介かけたわ。 \hfill\break
I used 58 kanme of ice. I sure put a lot of trouble on my sister. \hfill\break
From 童謡 by 川端康成.      