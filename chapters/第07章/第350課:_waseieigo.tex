    
\chapter{和製英語}

\begin{center}
\begin{Large}
第350課: 和製英語 
\end{Large}
\end{center}
  和製英語 is Japanese-made English. This means that these are going to be expressions that are fully or partly derived from English but may not have the same exact meaning or form as the original English phrase.       
\section{和製英語}
 
\par{ 和製英語 are constantly appearing. Many things that get Katakanized often end up being very long, and so they tend to get shortened. Once the abbreviation sticks, a new word is born. There are even ways of making many of these verbs and adjectives. }

\par{ググる = To google  マクる = To go to McDonalds  サボる = To slack off }

\par{ One weird instance that causes learners some confusion is on how to say apartment. There are two words for apartment, アパート and マンション. An アパート is more like a 1~2 story building with a relatively cheaper rent whereas a マンション would be at least 3 stories. }

\par{  Some Japanese speakers, perhaps more so true of older speakers, view these words as unsightly as they are thought of as being words that shouldn't exist because they're corruptions of actual loans. Others think the exact opposite and consider it the beauty of the evolution of language regardless of whether English speakers still understand the phrases or not. }

\par{ The chart below lists a lot of common 和製英語. ? indicates what an average native speaker of English would think what the word means having not known the origin of the phrase. }

\begin{ltabulary}{|P|P|P|}
\hline 

Phrase & To English speaker\dothyp{}\dothyp{}\dothyp{} & Meaning \\ \cline{1-3}

フリーサイズ & Free size & One size fits all \\ \cline{1-3}

パソコン & Pass kon & Personal computer \\ \cline{1-3}

モータープール & Motor pool & Parking lot \\ \cline{1-3}

ルポ & Rupo & Reportage \\ \cline{1-3}

ドタキャン & Dotacan & Last-minute cancellation \\ \cline{1-3}

ナイーブ & Naive & Innocent; sensitive \\ \cline{1-3}

エンゲージリング & Engage ring & Engagement ring \\ \cline{1-3}

インキー & In key & Having one's keys locked in one's car \\ \cline{1-3}

ホーム & Home & Railway platform \\ \cline{1-3}

ゲーセン & Gay sin & Game center \\ \cline{1-3}

ワイシャツ & Why shots & T shirt \\ \cline{1-3}

バックミラー & Back mirror & Rearview mirror \\ \cline{1-3}

フリーダイヤル & Free dailer & Toll free number \\ \cline{1-3}

ジェットコースター & Jet coaster & Roller coaster \\ \cline{1-3}

OL & OL & Female office worker \\ \cline{1-3}

フリートーク & Free talk & Free conversation \\ \cline{1-3}

フロントガラス & Front glass & Windshield \\ \cline{1-3}

ベッドタウン & Bed town & Commuter town \\ \cline{1-3}

プレイガイド & Play guide & Ticket agency \\ \cline{1-3}

ドライバー & Driver & Screwdriver; Driver (golf) \\ \cline{1-3}

チアガール & Cheer girl & Cheerleader \\ \cline{1-3}

ジーパン & G pan & Jeans and pants \\ \cline{1-3}

ナイター & Nighter & Night game \\ \cline{1-3}

ライブハウス & Live house & Live music club \\ \cline{1-3}

ランニングホームラン & Running home run & Inside-the-park home run \\ \cline{1-3}

フェミニスト & Feminist & Man who indulges women; feminist \\ \cline{1-3}

サンドバッグ & Sandbag & Punching bag \\ \cline{1-3}

ダンプカー & Dump car & Dumper \\ \cline{1-3}

スキンシップ & Skinship & Close physical contact \\ \cline{1-3}

サラリーマン & Salaryman & Salaryman\slash Salary worker \\ \cline{1-3}

モガ & Moga? & Modern girl \\ \cline{1-3}

モボ & Mobo? & Modern boy \\ \cline{1-3}

デッドボール & Dead ball & Hitting a pitcher with the ball \\ \cline{1-3}

ボールペン & Bowl pen & Ballpoint pen \\ \cline{1-3}

アニソン & A Nissan & Anime theme song \\ \cline{1-3}

ロードショー & Road show & Premiere \\ \cline{1-3}

ソフト & Soft & Software \\ \cline{1-3}

トレーナー & Trainer & Trainer; sweatshirt \\ \cline{1-3}

アメフト & A meth toe & American football \\ \cline{1-3}

ワープロ & Wah pro & Word processor \\ \cline{1-3}

マジックテープ & Magic tape & Velcro \\ \cline{1-3}

キャッチホン & Catch hone & Call waiting (from catch phone) \\ \cline{1-3}

ハンスト & Han stow? & Hungry strike \\ \cline{1-3}

マスコミ & Mass kami & The media \\ \cline{1-3}

コンセント & Consent & Electric outlet \\ \cline{1-3}

プロレス & Proless & Professional wrestling \\ \cline{1-3}

セクハラ・セクシャルハラスメント & Sexual harassment & Sexual harassment \\ \cline{1-3}

シュークリーム & Shoe cream & Cream puff (Actually from French) \\ \cline{1-3}

ガソリンスタンド & Gasoline stand & Gasoline station \\ \cline{1-3}

コマーシャルソング & Commercial song & Jingle \\ \cline{1-3}

ノーブランド & No brand & Generic \\ \cline{1-3}

ドクターコース & Doctor course & Doctor's course (PhD) \\ \cline{1-3}

着メロ & Receive-mellow? & Ringtone (メロ as in melody メロディー) \\ \cline{1-3}

フリーター & Freeter? & Freeloader \\ \cline{1-3}

エアコン & Air con & Air conditioner \\ \cline{1-3}

スマホ & Smart ho & Smartphone (Abbreviation of \\ \cline{1-3}

セレブ & Celeb & Celebrity \\ \cline{1-3}

エンスト & End stow & Engine stop \\ \cline{1-3}

カーナビ & Car navy & Car navigator \\ \cline{1-3}

バナリパ & Banaripah & Banana Republic (not everyone would get this) \\ \cline{1-3}

ワンピ & One pee & One piece \\ \cline{1-3}

ロン毛 & ?? & Long hair \\ \cline{1-3}

パワハラ & Power hara & Power harassment (abuse of authority) \\ \cline{1-3}

シネコン & Shin akon & Cinema complex \\ \cline{1-3}

\end{ltabulary}

\begin{center}
\textbf{Those Needing More Explanation }
\end{center}

\par{ There are many 和製英語 that deserve special attention. }

\par{モーニングサービス: Morning service can be found at Japanese coffee shops where you can find very cheep (500 yen or so) meals containing eggs, toast, and coffee or maybe other things. Similar phrase include バリューセット for value meals in Western chains such as Mc. Donalds. }

\par{ターミナルホテル: Hotel that is cheap and very convenient. }

\par{ワンルームマンション: A very small room in a high-rise building around 6~8帖. }

\par{シルバーシート: Seats that should be given to elderly people. It may also be called 優先席. }

\par{ツーショット: A picture shot of two people. }

\par{ペアルック: Look-alike. }

\par{セロ(ハン)テープ: Scotch tape. It is not saran wrap. }

\par{ガムテープ: Duct tape. Although it sounds like it's tape made out of gum, it isn't. }

\par{スズランテープ: Used to often tie up magazines. It may also be used tie up cardboard boxes. }

\par{ナイター: This is not a nighter. It is a nightgame. Party time! }

\par{パスケース: What you put your boarder pass to something in. Looks like a small wallet. }

\par{ミスド: Mr. Donuts. }

\par{シュークリーム: This is not shoe cream. It is actually a cream puff. }

\par{ファッキン: This is not fucking. It is going to Fast Kitchen. }

\par{メロドラマ: This is not melodramatic. It's a soap opera. }

\par{スキンシップ: This is physical contact, especially between mother and child. }

\par{ドーナツ化現象: This is suburbanization. }

\par{コンデンスミルク: This is sweetened condensed milk (加糖練乳). In speaking of milk, Japanese call evaporated milk エバミルク (無糖練乳). }

\par{写メール: 写メ for short, is the practice of taking a picture with your cellphone and e-mailing it to someone. }

\begin{center}
 \textbf{Examples (More to Come) }
\end{center}

\par{\textbf{マイカー }で通勤する。 \hfill\break
I commute with my own car. }

\par{\textbf{脱サラ(リーマン) }をして、事業を興す人が急増しています。 \hfill\break
The number of people leaving salary jobs and running their own business is sharply rising. }
    