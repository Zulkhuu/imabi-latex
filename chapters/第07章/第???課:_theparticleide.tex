    
\chapter*{The Particle いで}

\begin{center}
\begin{Large}
第???課: The Particle いで 
\end{Large}
\end{center}
 
\par{ In this lesson, we will direct attention to a unique grammar that, although not common, is a demonstration of how older grammar can evolve and survive—even if it becomes dialectical. }

\par{ As indicated by the title, the grammar pattern to be discussed is ~いで. This derives from the conjunctive particle で, which although largely not used in Modern Japanese aside from set expressions, is equivalent in meaning and usage to ~ないで and  ~ずに. We will first learn about the etymological background that links this ~で with ~いで. Then, we\textquotesingle ll learn about how ~いで remains used today. }
      
\section{The Origin of ~で \& ~いで}
 
\par{ The conjunctive particle で is the contraction of the old 連用形 of the negative auxiliary ず, に-, followed by the conjunctive particle て. It may also potentially be the contraction of the ず-連用形 of the negative auxiliary ず in conjunction with て (~ずて), but the etymology would still be the same. This grammar began being used in Early Middle Japanese (794-1185 A.D.). As the particle derives from a negative auxiliary, it attaches to the 未然形 of a verb. }

\par{1. この ${\overset{\textnormal{うら}}{\text{恨}}}$ み、 ${\overset{\textnormal{は}}{\text{晴}}}$ らさ \textbf{で }おくべきか。 \hfill\break
How could I not dispel this resentment? \hfill\break
 \hfill\break
\textbf{Grammar Notes }: \hfill\break
1. A Modern Japanese rendition of this would be 「この恨み、晴らさないでおけようか。いや、晴らさなければならない」. \hfill\break
2. The べきか of this example is equivalent to できるのだろうか. }

\par{2. いかで ${\overset{\textnormal{つき}}{\text{月}}}$ を ${\overset{\textnormal{み}}{\text{見}}}$ ではあらむ。 \hfill\break
How is it that one can stand not to look at the moon? \hfill\break
 \hfill\break
\textbf{Grammar Note }: A Modern Japanese rendition of this would be 「どうして月を見ないでいられようか、いやいられない」. }

\par{ The pronunciation of the conjunctive particle で was actually \slash nde\slash . It is this nasal initial \slash n\slash  that became contracted into a nasal \slash i\slash , thus rendering the grammar as いで. This grammar began being used in the Muromachi Period (1336-1573 A.D.). Its usage is identical to its predecessor. It too attaches to the 未然形 of a verb. }

\par{3. ${\overset{\textnormal{われ}}{\text{我}}}$ とても ${\overset{\textnormal{おんな}}{\text{女}}}$ の ${\overset{\textnormal{み}}{\text{身}}}$ 、 ${\overset{\textnormal{はら}}{\text{腹}}}$ が ${\overset{\textnormal{た}}{\text{立}}}$ たいであるものか。 \hfill\break
I after all am a woman; how could I not get mad? \hfill\break
From ${\overset{\textnormal{ねびき}}{\text{寿}}}$ の ${\overset{\textnormal{かどまつ}}{\text{門松}}}$ by ${\overset{\textnormal{ちかまつもんざえもん}}{\text{近松門左衛門}}}$ . }

\par{\textbf{Grammar Note }: A Modern Japanese rendition of this would be 「私だって女の身、腹が立たないでいられるものか」. }

\par{4. ${\overset{\textnormal{わたし}}{\text{私}}}$ や ${\overset{\textnormal{こども}}{\text{子供}}}$ は ${\overset{\textnormal{なにき}}{\text{何着}}}$ いでも、とかく ${\overset{\textnormal{おとこ}}{\text{男}}}$ は ${\overset{\textnormal{せけん}}{\text{世間}}}$ が ${\overset{\textnormal{だいじ}}{\text{大事}}}$ 。 \hfill\break
Even if neither I nor the children have anything to wear, the world will remain important to the men.  \hfill\break
From ${\overset{\textnormal{しんじゅうてんのあみじま}}{\text{心中天網島}}}$ by ${\overset{\textnormal{ちかまつもんざえもん}}{\text{近松門左衛門}}}$ . }
      
\section{Use in Modern Japanese}
 
\par{ ~いで remained used throughout Japanese up into the Edo Period (1603-1868). In the literature of this era, it was incredibly common in 落語 (professional storytelling). As Ex. 4 demonstrated, it was also possible to see ~いでも (= ~なくても・ないでも). It was also possible to see ~いでは (= ~ないでは・ずには). These two grammar points also carried over into Early Modern Japanese. In Modern Japanese, all facets of ~いで are no longer common. Speakers that do use this grammar tend to either be older and\slash or from West Japan where it has held on the most. }

\par{5. ${\overset{\textnormal{しあい}}{\text{試合}}}$ をせいでは ${\overset{\textnormal{とお}}{\text{通}}}$ されぬ。 \hfill\break
One will not be let through without playing a match. }

\par{6. そんな ${\overset{\textnormal{じゅみょう}}{\text{寿命}}}$ を ${\overset{\textnormal{ちぢ}}{\text{縮}}}$ めるようなゲームをやらいでもいいよ。 \hfill\break
You don\textquotesingle t need to play such a life shortening game like that. }

\par{7. そんなことも ${\overset{\textnormal{し}}{\text{知}}}$ らいでよう ${\overset{\textnormal{い}}{\text{言}}}$ うわ。 \hfill\break
You sure talk for someone who doesn\textquotesingle t even know that! }

\par{\textbf{Dialect Note }: よう = よく in dialects of West Japan. }

\par{8. ${\overset{\textnormal{なんべん}}{\text{何遍}}}$ も ${\overset{\textnormal{い}}{\text{言}}}$ わいでも ${\overset{\textnormal{き}}{\text{聞}}}$ こえてるよ。 \hfill\break
I\textquotesingle m able to hear you without you saying it over and over. }

\begin{center}
\textbf{~いでか }
\end{center}

\par{ As illustrated by Exs. 1-3, it is without a doubt that this grammar is most commonly used in expressing “how could\slash would I not…” This usage is rendered as ~いでか in today\textquotesingle s speech. This is generally deemed to be 関西弁. However, because it existed in Early Modern Japanese, it is more accurate to say that it only remains used in West Japan, where the dialect happens to be 関西弁. This grammar can occasionally be seen in literature, including manga. Even in West Japan, though, it is losing currency. }

\par{9. 「ほう、わかるか?」「わからいでか!」 \hfill\break
“Oh, you understand?” “How would I not understand!” }

\par{10. ${\overset{\textnormal{わ}}{\text{分}}}$ からいでか、お ${\overset{\textnormal{まえ}}{\text{前}}}$ は ${\overset{\textnormal{わか}}{\text{判}}}$ り ${\overset{\textnormal{やす}}{\text{易}}}$ すぎなんだよ。 \hfill\break
How would I know understand, you\textquotesingle re too easy to get. }

\par{11. 「やるのか?」「やらいでか!」 \hfill\break
“Are you really going to?” “The hell I am!” }

\par{12. ${\overset{\textnormal{し}}{\text{死}}}$ ぬ ${\overset{\textnormal{まえ}}{\text{前}}}$ に ${\overset{\textnormal{い}}{\text{言}}}$ いたいことを ${\overset{\textnormal{い}}{\text{言}}}$ わいでか。 \hfill\break
How would I not say what I want to say before dying? }

\par{13. ${\overset{\textnormal{おこ}}{\text{怒}}}$ らいでか! \hfill\break
You bet I\textquotesingle m angry! How could I not be? }

\par{14. ${\overset{\textnormal{きづ}}{\text{気付}}}$ かいでか。 \hfill\break
How could I not notice? }

\par{15. 「 ${\overset{\textnormal{かれ}}{\text{彼}}}$ が ${\overset{\textnormal{うらぎ}}{\text{裏切}}}$ り ${\overset{\textnormal{もの}}{\text{者}}}$ だと ${\overset{\textnormal{き}}{\text{気}}}$ が ${\overset{\textnormal{つ}}{\text{付}}}$ いたんだい?」「わからいでか!ていうか、 ${\overset{\textnormal{たいちょう}}{\text{隊長}}}$ ですら ${\overset{\textnormal{し}}{\text{知}}}$ ってて ${\overset{\textnormal{くち}}{\text{口}}}$ を ${\overset{\textnormal{つぐ}}{\text{噤}}}$ んでたんだ」 \hfill\break
“So you realized that he was a traitor?” “How would I not know!? Or, should I say, even the captain knew but just kept his mouth shut.” }
    