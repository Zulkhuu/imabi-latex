    
\chapter{Native Suffixes II}

\begin{center}
\begin{Large}
第327課: Native Suffixes II: Nominal II 
\end{Large}
\end{center}
 
\par{ After this lesson, we will transition to native endings that result in non-nominal phrases. }
      
\section{Native Nominal Suffixes 26-53}
 
\par{26. ~ ${\overset{\textnormal{ず}}{\text{好}}}$ き follows nouns to describe people that like a certain think a lot or a character that is liked a lot by people. }
 
\par{1. 夫は外出好きです。 \hfill\break
My husband likes to go out. }
 
\par{2. 「あの人は、よく外国旅行をしますね」「よほど ${\overset{\textnormal{ぼうけんず}}{\text{冒険好}}}$ きなんでしょう」 \hfill\break
"That person often travels overseas" "He must be quite adventurous" }
 
\par{3. 人好きの ${\overset{\textnormal{せいしつ}}{\text{性質}}}$ \hfill\break
An amiable personality }
 
\par{27. ~ずじまい means that something wasn't done and something accidentally ended up happening--しないで終わってしまった. As you can see, it comes from the combination of the old negative auxiliary ず and 終い (end) voiced. }

\par{4. 結局彼と会わずじまいだった。 \hfill\break
I ended up not meeting with him. }
 
\par{28. ~尽く attaches to nouns and the stems of adjectives. It either shows that something relies entirely on something, something has the sole purpose of something, or what something is based off of. Most of the words used with this are uncommon and are becoming restricted to formal writing. }
 
\par{5. 納得ずくで解約する。 (経済) \hfill\break
To dissolve with mutual consent. }
 
\par{6. 腕ずくで強い相手を倒す。 \hfill\break
To beat a strong enemy by might. }
 
\par{${\overset{\textnormal{}}{\text{7a. 算盤}}}$ ずくで戦う。 \hfill\break
7b. 計算ずくで戦う。 (より自然) \hfill\break
To fight calculatively. }
 
\par{${\overset{\textnormal{}}{\text{8. 欲得}}}$ ずくで引き受けた仕事ではない。 \hfill\break
This is not a job I took on just for gain. }

\par{9. 欲得ずくの結婚をする。 \hfill\break
To marry for the money. }
 
\par{10. 力ずくで家に押し入る。 \hfill\break
To enter a house by force. }

\par{11. 腕ずくで捩じ伏せる。 \hfill\break
To hold down someone's arms with brute force. }
 
\par{29. ~ ${\overset{\textnormal{ぜ}}{\text{攻}}}$ め means "offense\slash attack" and attaches to nouns to show \textbf{bombardment }. In doing so, 攻め becomes voiced as ぜめ. }
 
\par{12. 守りから攻めに転じる一手は賢いのか。 \hfill\break
Is switching to the offensive from the defensive a wise move? }

\par{13. 次の一手で守りから攻めに転じる! \hfill\break
I'll switch to offensive from defensive with this next move! }

\par{14. ${\overset{\textnormal{ひょうろうぜ}}{\text{兵糧攻}}}$ めで敵を ${\overset{\textnormal{こうふく}}{\text{降伏}}}$ させる。 \hfill\break
To force surrender with starvation. }
 
\par{15. 質問攻めにする。 \hfill\break
To bombard with questions. }
 
\par{16. 我々は広告攻めに ${\overset{\textnormal{あ}}{\text{遭}}}$ っている。 \hfill\break
We are bombarded with advertisements. }
 
\par{17. 誰もが毎日電子メールに\{やられています・ ${\overset{\textnormal{へきえき}}{\text{辟易}}}$ しています\}。 \hfill\break
Everyone is being overwhelmed by e-mails every day. }
 
\par{30. ~だてら shows that something is unfitting of normal quality. This is old-fashioned and essentially not used. }
 
\par{18. 女だてらってよく言われますが、男だてらって言いませんね。 \hfill\break
"Unladylike" is used frequently, but we don't say "unmanly-like". \hfill\break
\hfill\break
19. 親だてらにいらぬ世話を焼く。(古い言い方) \hfill\break
To meddle unlike that of a parent. }
 
\par{31. ~ ${\overset{\textnormal{た}}{\text{垂}}}$ れ is a pejorative equivalent to "-ass" and attaches to nouns or adjectives. }
 
\par{20. しみったれ \hfill\break
Stingy a**. }
 
\par{21. 馬鹿垂れ \hfill\break
Dumba** \hfill\break
 \hfill\break
22. くそったれ \hfill\break
S\%*\#head }
 
\par{32. ~付(つ・づ)け  may either be attached to the 連用形 of a verb to show what one always does or attached to a day as "づけ" to show a schedule. }
 
\par{23. こちらは行きつけの図書館です。 \hfill\break
This is the library that I always go to. }
 
\par{${\overset{\textnormal{}}{\text{24. 六日}}}$ ${\overset{\textnormal{}}{\text{付}}}$ けの ${\overset{\textnormal{はつれい}}{\text{発令}}}$ \hfill\break
Proclamation for the sixth }
 
\par{${\overset{\textnormal{}}{\text{25. 掛}}}$ かり ${\overset{\textnormal{つ}}{\text{付}}}$ けの医者 \hfill\break
Family doctor }
 
\par{33. ~ ${\overset{\textnormal{づ}}{\text{詰}}}$ め shows that an action or condition is continuing. Attach to the 連用形 of verbs. }
 
\par{26. 歩き詰めでくたくただ。 \hfill\break
I was exhausted due to continuous walking. \hfill\break
 \hfill\break
27. 立ち詰めで働いていた。 \hfill\break
I have been working while continuing to stand up. }
 
\par{34. ~ ${\overset{\textnormal{づ}}{\text{連}}}$ れ shows a company of people. }
 
\par{28. 旅の道連れ \hfill\break
A traveling companion. }
 
\par{29. 三人連れでいればいい。 \hfill\break
It's good to be in groups of threes. }
 
\par{30. 家族連れでハイキングに行く。 \hfill\break
To go hiking with one's family in tow. }
 
\par{35. ~手 is attached to the 連用形 of verbs and denotes the action doer. }
 
\par{31. あいつはくどい話し手だな。 \hfill\break
He's a windy speaker, isn't he? }
 
\par{32. 彼はいつも ${\overset{\textnormal{ちゅういぶか}}{\text{注意深}}}$ い聞き手です。 \hfill\break
He is always an alert listener. }
 
\par{33. 書き手 \hfill\break
Writer }
 
\par{36. ~ ${\overset{\textnormal{どお}}{\text{通}}}$ し, like ~詰め, shows that something continues as such. It too attaches to the 連用形 of verbs. \hfill\break
 \hfill\break
34. 雪は ${\overset{\textnormal{よどお}}{\text{夜通}}}$ し降り続いた。 \hfill\break
It snowed all night long. \hfill\break
 \hfill\break
35. 立ち通しでした。 \hfill\break
I was kept standing. }
 
\par{37. ~ ${\overset{\textnormal{どお}}{\text{通}}}$ り follows nouns meaning "Street\slash Avenue" or "in accordance with". Do not confuse with ~通し. }
 
\par{36. 彼女の住所はウェスト・パーク通り6番です。 \hfill\break
Her address is 6 West Park Street. }
 
\par{37. サンセット大通り \hfill\break
Sunset Boulevard }
 
\par{38. 自分の思い通りにする。 \hfill\break
To have it one's own way. }
 
\par{${\overset{\textnormal{}}{\text{39. 求刑}}}$ の ${\overset{\textnormal{けっか}}{\text{結果}}}$ だが、 ${\overset{\textnormal{よそうどお}}{\text{予想通}}}$ りだ。 \hfill\break
These are the results of the prosecution; it was just as expected. }
 
\par{${\overset{\textnormal{}}{\text{40. 概}}}$ ね ${\overset{\textnormal{へいじょうどお}}{\text{平常通}}}$ りです。 \hfill\break
Things are basically just as normal. }
 
\par{38. ~並み means "average\slash ordinary" As a suffix, it may show a line of similar things such as houses, everything in the same condition, or something of a same level or kind. }
 
\par{41. ウィキペディアはブリタニカ並みに正確だと思います。 \hfill\break
I think that Wikipedia is as accurate as Britannica. }
 
\par{${\overset{\textnormal{}}{\text{42. 軒並}}}$ みが ${\overset{\textnormal{かいめつ}}{\text{壊滅}}}$ した。 \hfill\break
The row of houses were completely destroyed. }
 
\par{43. 世間並み \hfill\break
Ordinary }
 
\par{44. 山並み(の ${\overset{\textnormal{りんかく}}{\text{輪郭}}}$ )が見える。 \hfill\break
To be able to see the outline of the row of mountains. }
 
\par{45. 七月並みの気温です。 \hfill\break
This is average July temperatures. }

\par{46. この種の顔写真は、誰でもひとし なみ に指名手配犯のように見せてしまう。 \hfill\break
This kind of face photo ends up showing anyone equally like a wanted criminal. \hfill\break
From 火車 by 宮部みゆき. }

\par{\textbf{Word Note }: ひとしなみ, spelled in 漢字 as 等し並, is a 形容動詞 made with the combination of 並み and the stem of 等しい }
 
\par{39. ~抜く as a suffix in the form ~抜き means "without". }
 
\par{47. 塩抜きのポテトチップを買えるかな。 \hfill\break
I wonder if you can buy potato chips without salt. }
 
\par{48. 砂糖抜きでお願いします。 \hfill\break
Without sugar please. }
 
\par{49. 昼飯抜きで仕事をすると、健康を ${\overset{\textnormal{そこ}}{\text{損}}}$ なってしまう。 \hfill\break
If you work without lunch, you'll end up losing your health. }

\par{50. ${\overset{\textnormal{し}}{\text{染}}}$ み ${\overset{\textnormal{}}{\text{抜}}}$ きをする。 \hfill\break
To remove stains. }
 
\par{40. ~ ${\overset{\textnormal{ば}}{\text{栄}}}$ え attaches to the 連用形 of verbs and is used to show a heard or shown advantage. It is limited to the verbs 見る and 聞く. }
 
\par{51. 音楽は聞き栄えがしなかった。 \hfill\break
The music was not pleasing to the ears. }
 
\par{${\overset{\textnormal{}}{\text{52. 彼女は一向}}}$ に見栄えがしないよ。 \hfill\break
She's not anything to look at at all. }
 
\par{53. あのドレスは見栄えがしました。 \hfill\break
That dress was much to look at. }
 
\par{41. ~ ${\overset{\textnormal{ばな}}{\text{端}}}$ attaches to the 連用形 of verbs and means "about to start". Its usage is very limited. The first example is a very common phrase with it, but other than that, it is extremely rare. Use on with things you find it used with. }
 
\par{54. 出端 \hfill\break
Moment of departure }
 
\par{${\overset{\textnormal{}}{\text{55. 寝入}}}$ りばなに弟が泣き出した。 \hfill\break
My little brother began to cry as I was falling asleep. }
 
\par{42. ~っ ${\overset{\textnormal{ぱな}}{\text{放}}}$ し comes from 放す meaning "to unchain". ~っ放し strengthens this meaning by showing how "you leave something\dothyp{}\dothyp{}\dothyp{}". It's most associated with frequent careless forgetting. }
 
\par{56. テレビをつけっぱなしにするな! \hfill\break
Don't leave the TV on! }
 
\par{57. 私は昨夜テレビをつけっぱなしにして寝てしまった。 \hfill\break
Last night I fell asleep leaving the television on. }
 
\par{58. 部屋を出るときは ${\overset{\textnormal{でんとう}}{\text{電灯}}}$ をつけっぱなしにしないようにしてくださいね。 \hfill\break
Try not to leave the lights on when you leave the room, OK? }
 
\par{43. ~ ${\overset{\textnormal{ば}}{\text{張}}}$ り attaches to personal nouns and means "reminiscent of". }
 
\par{${\overset{\textnormal{}}{\text{59. 雷電張}}}$ りの ${\overset{\textnormal{そらもよう}}{\text{空模様}}}$ だった。 \hfill\break
It was weather reminiscent of thunder and lightning. }
 
\par{60. セザンヌ張りの ${\overset{\textnormal{ふうけいが}}{\text{風景画}}}$ 。 \hfill\break
Landscape paintings reminiscent of Cezanne. }
 
\par{61. シェイクスピア張りの文体。 \hfill\break
Writing style reminiscent of Shakespeare. }
 
\par{44. ~ ${\overset{\textnormal{び}}{\text{日}}}$ means day. It is sometimes unvoiced. }

\begin{ltabulary}{|P|P|P|P|P|P|}
\hline 

収集日 & しゅうしゅうび & Collection day & 締切日 & しめきりび & Closing day \\ \cline{1-6}

忌み日 & いみび & Unlucky day (astrology) & 冬日 & ふゆび & Winter day \\ \cline{1-6}

春日 & はるひ・はるび & Spring day & 誕生日 & たんじょうび & Birthday \\ \cline{1-6}

\end{ltabulary}

\par{45. ~ ${\overset{\textnormal{べ}}{\text{辺}}}$ is attached to geographical nouns to mean "-side". }
 
\par{62. 水辺に ${\overset{\textnormal{なら}}{\text{並}}}$ んでいる町。 \hfill\break
Towns lining the waterside. }
 
\par{63. 海辺で休日を過ごしたよ。 \hfill\break
I spent the holiday at the seaside. }

\par{64. ${\overset{\textnormal{あらし}}{\text{嵐}}}$ がその ${\overset{\textnormal{おおぶね}}{\text{大船}}}$ を岸辺へと押しやった。 \hfill\break
The storm impelled the big ship to the shoreline. }
 
\par{65. 山辺 \hfill\break
Mountainside }

\par{\textbf{Historical Note }: It's interesting to note that some scholars believe this ending is actually an ancient borrowing from Ainu. }

\par{46. ~っぺ is used after names to show familiarity, but it can also be used in a hurtful way. It all depends on context. If you're called a 田舎っぺ, that's not a nice thing. These sorts of nuances don't translate well into English, so keep in mind what this stands for. }

\par{66. 花っぺ (Nickname for Hanako) \hfill\break
Hana-ppe }

\par{67. 太郎っぺ \hfill\break
Taro }
 
\par{47. ~(っ)ぽ(っ)ち attaches to demonstrative pronouns or numbers meaning "merely but". This ending is very casual, which explains the speech styles of the example sentences. }
 
\par{68. 1ドルぽっちでは買えねー。(Casual; vulgar) \hfill\break
I can't buy anything with just a dollar. }
 
\par{69. これっぽっちでは足りないもんだ。 \hfill\break
This is just merely lacking. }
 
\par{48. ~ ${\overset{\textnormal{まえ}}{\text{前}}}$ attaches to words of number or person to show portion or suitable amount. This suffix oddly does not follow any exception to any counter rules. }
 
\par{70. 一人前になる。 \hfill\break
To come to age. }
 
\par{71. 二人前を ${\overset{\textnormal{たい}}{\text{平}}}$ らげる。 \hfill\break
To consume food enough for two people. }
 
\par{49. ~向け: See Lesson 151. }

\par{50. ~目 has three usages. It is seen in the pattern ”Counter phrase + 目” to create ordinal expressions (\#th). You also see it after the 連用形 of verbs to show the point or place of partition in something. The last usage is just as important. In its third usage it is after either after the stem of adjectives or the 連用形 of verbs to show tendency\slash disposition\slash degree. Of course, given that there are two usages when after the 連用形 of verbs, you will have to consider the meaning of the verbs themselves to differentiate the usages. }

\par{72. 今年の15冊目を読みきりました。 \hfill\break
I've completely read through my 15th book of the year. }

\par{73. このレストランでまったく切れ目のない列ができています。 \hfill\break
There is a line with no gaps at all. }

\par{${\overset{\textnormal{}}{\text{74. 縫}}}$ い目が ${\overset{\textnormal{ほころ}}{\text{綻}}}$ びる。 \hfill\break
For seams to split. }
75. 落ち目になる。 \hfill\break
For one's fortune to decline. 
\par{76. 袂はやや長めであった。 \hfill\break
(Her) sleeves were quite lengthy. \hfill\break
From 不死 by 川端康成. }

\par{77. 若い女の答えはあいまいだったが、白粉の濃いめの顔を赤らめもしないで、もううしろを見せるとフォウムへ出て行った。 \hfill\break
The young woman's answer was vague, but she left to the platform and had already shown her backside doing so without ever having her strongly powdered face blush. \hfill\break
From 山の音 by 川端康成. }

\par{\textbf{Exception Note }: 濃い + 目 should result in 濃目, but this is very rare. Instead, 濃い目 is used. }
 
\par{51. ~もの, either 物 for things or 者 for people, attaches to the stem of nouns to show things or people "to\dothyp{}\dothyp{}\dothyp{}". }
 
\par{78. 動物に食べ物を与えないでくださいませんか。 \hfill\break
Could you please not give food to the animals? }
 
\par{79. 何か飲み物がほしい。 \hfill\break
I want something to drink. }
 
\par{80. 乗り物に ${\overset{\textnormal{よ}}{\text{酔}}}$ う。 \hfill\break
To have travel sickness. }
 
\par{81. この本はためになる読み物です。 \hfill\break
This book is good reading. }
 
\par{82. 生き物一つもいない。 \hfill\break
There is not a living thing. }
 
\par{83. 壊れ物! \hfill\break
Fragile objects! }
 
\par{84. 彼女は本当に働き者だよ。 \hfill\break
She is a really hard worker. }
 
\par{85. 彼はまったく ${\overset{\textnormal{なま}}{\text{怠}}}$ け者だぞ。 \hfill\break
He is completely lazy. }
 
\par{86. 聞き物はない。 \hfill\break
To not have something worth listening to. }

\par{52. ~ ${\overset{\textnormal{や}}{\text{屋}}}$ may be added to a type of trade to show a kind of shop. In the same vein, it may also be the end of an alias. As an extension of showing one's trade, it can pinpoint people with certain dispositions like being sarcastic, shy, or what not. In this last case, it is often with the auxiliary ~がる. }

\begin{ltabulary}{|P|P|P|P|}
\hline 

電気屋 & Electric store & 恥ずかしがり屋 & Shy person \\ \cline{1-4}

皮肉屋 & Sarcastic person & 鈴の屋 & Suzunoya (alias) \\ \cline{1-4}

技術屋 & Engineer & 寿司屋 & Sushi bar \\ \cline{1-4}

\end{ltabulary}
 
\par{53. ~や is a suffix that attaches to nouns of person to show intimacy. }
 
\par{87. じいや \hfill\break
Gramps }
 
\par{88. 坊やが天使のような表情を持った。 \hfill\break
The little boy possessed an angelic expression. }
 
\par{89. ねえや \hfill\break
Sis' }

\par{30. 尽く (ずく) attaches to nouns and the stems of adjectives. It either shows that something relies entirely on something, something has the sole purpose of something, or what something is based off of. }

\par{納得ずく \hfill\break
With mutual consent }

\par{腕尽くで相手をぶっ倒す。 \hfill\break
To beat an enemy by might. }

\par{欲得ずくで付き合う。 \hfill\break
To go steady with having a mercenary attitude. }

\par{力ずくで押し入らない方がよほどよい。 \hfill\break
It is much better to not enter by force. }

\par{31. 尽くめ (ずくめ) \textrightarrow  Lesson 109  \hfill\break
32. 達 (たち) \textrightarrow  Lesson 5  }

\par{33. だてら shows that something is unfitting of normal quality. }

\par{女だてら \hfill\break
Unladylike }

\par{親だてらにいらぬ世話を焼く。 \hfill\break
To meddle unlike that of a parent. }

\par{34. だらけ \textrightarrow  Lesson 109  }

\par{35. 垂れ (たれ) is a pejorative equivalent to "-ass" and attaches to nouns or adjectives. }

\par{しみったれ \hfill\break
Stingy a**. }

\par{馬鹿垂れ \hfill\break
Dumba** \hfill\break
\hfill\break
クソッタレ \hfill\break
S\%*\#head }

\par{36. ちゃん \textrightarrow  Lesson 70  \hfill\break
}

\par{37. 付け (つけ・づけ) may either be attached to the 連用形 of a verb  to show what one always does or attached to a day as "-づけ" to show a schedule. }

\par{行きつけの図書館はこちらです。 \hfill\break
This is the library that I always go to. }

\par{六日付(づ)けの発令。 \hfill\break
Proclamation for the sixth. }

\par{掛かり付(つ)けの医者。 \hfill\break
The family doctor. }

\par{38. 詰め (づめ) shows that an action or condition is continuing and attaches to the 連用形 of verbs. }

\par{歩き詰めでくたくただ。 \hfill\break
I was exhausted due to continuous walking. \hfill\break
\hfill\break
立ち詰めで働いていた。 \hfill\break
I have been working while continuing to stand up. }

\par{39. 連れ (づれ) shows a company of people. }

\par{旅の道連れ \hfill\break
A traveling companion. }

\par{三人連れでいればいい。 \hfill\break
It's good to be in groups of threes. }

\par{家族連れでハイキングに行く。 \hfill\break
To go hiking with one's family in tow. }

\par{40. 手 (て) is attached to the 連用形 of verbs and denotes the action doer. }

\par{くどい話し手だな。 \hfill\break
He's a windy speaker, isn't he? }

\par{彼はいつも注意深い聞き手です。 \hfill\break
He is always an alert listener. }

\par{書き手 \hfill\break
Writer }
41. 通し (どおし) is just like -詰め and shows that something continues as such. It also follows the 連用形 of verbs. \hfill\break
\hfill\break
雪が夜通し降り続いた。 \hfill\break
It snowed all night long. \hfill\break
\hfill\break
立ち通しでした。 \hfill\break
I was kept standing. 
\par{42. 通り (どおり) follows nouns and either means "Street\slash Avenue" or "following\slash in accordance with". }

\par{彼女の住所はウェスト・パーク通り6番です。 \hfill\break
Her address is 6 West Park Street. }

\par{サンセット大通り \hfill\break
Sunset Boulevard }

\par{自分の思い通りにする。 \hfill\break
To have it one's own way. }

\par{43. 殿 (どの) \textrightarrow  Lesson 70  \hfill\break
44. 共 (ども) \textrightarrow  Lesson 70  \hfill\break
}

\par{45. 栄え (ばえ) attaches to the 連用形 of verbs and is used to show a heard or shown advantage. It is limited to the verbs 見る and 聞く. Below are some examples. }

\par{音楽は聞き栄えがしなかった。 \hfill\break
The music was not pleasing to the ears. }

\par{一向に見栄えがしないよ。 \hfill\break
She's not anything to look at at all. }

\par{あのドレスは見栄えがしました。 \hfill\break
That dress was much to look at. }

\par{46. 端 (ばな) attaches to the 連用形 of verbs and means "about to start". }

\par{出端 \hfill\break
Moment of departure }

\par{寝入りばなに弟が泣き出した。 \hfill\break
My little brother began to cry as I was falling asleep. }

\par{47. 張り (ばり) attaches to personal nouns and means "reminiscent of". }

\par{雷電張りの空模様だった。 \hfill\break
It was weather reminiscent of thunder and lightning. }

\par{セザンヌ張りの風景画。 \hfill\break
Landscape paintings reminiscent of Cezanne. }

\par{シェイクスピア張りの文体。 \hfill\break
Writing style reminiscent of Shakespeare. }

\par{48. 辺 (べ) is attached to geographical nouns to mean "-side". }

\par{水辺に並んでいる町。 \hfill\break
Towns lining the waterside. }

\par{海辺で休日を過ごしたよ。 \hfill\break
I spent the holiday at the seaside. }

\par{水辺に欸乃(あいだい)を聞く。 \hfill\break
To hear a fisherman's song on the waterside. }

\par{嵐(あらし)がその大船を岸辺へと押しやった。 \hfill\break
The storm impelled the big ship to the shoreline. }

\par{49. (っ)ぽ(っ)ち attaches to demonstrative pronouns or numbers meaning "merely but". }

\par{1ドルぽっちでは買えねー。 \hfill\break
I can't buy anything with just a dollar. }

\par{これっぽっちでは足りないもんだ。 \hfill\break
This is just merely lacking. }

\par{50. 前 (まえ) attaches to words of number or person to show portion or suitable amount. This suffix oddly does not follow any exception to any counter rules. }

\par{一人前になる。 \hfill\break
To come to age. }

\par{二人前を平らげる。 \hfill\break
To consume food enough for two people. }

\par{51. み \textrightarrow  Lesson 112  }

\par{52. みどろ is a less common alternative to the suffix -まみれ to show things being covered in something. }

\par{血みどろの遺体 \hfill\break
A bloody corpse }

\par{汗みどろの労働者 \hfill\break
A sweat covered laborer }

\par{53. -もの, either 物 for things or 者 for people, attaches to the stem of nouns to show things or people "to\dothyp{}\dothyp{}\dothyp{}". }

\par{動物に食べ物を与えないでくださいませんか。 \hfill\break
Could you please not give food to the animals. }

\par{何か飲み物がほしい。 \hfill\break
I want something to drink. }

\par{着物 \hfill\break
Clothes; kimono }

\par{銀座へ買い物に行った。 \hfill\break
I went shopping in Ginza. }

\par{クリスマスの贈り物 \hfill\break
Christmas presents }

\par{乗り物に酔(よ)う。 \hfill\break
To have travel sickness. }

\par{この本はためになる読み物です。 \hfill\break
This book is good reading. }

\par{生き物一つもいない。 \hfill\break
There is not a living thing. }

\par{建物はでっかいよ! \hfill\break
The building is huge! }

\par{壊れ物! \hfill\break
Fragile objects! }

\par{彼女は本当に働き者だよ。 \hfill\break
She is a really hard worker. }

\par{まったく怠け者だぞ。 \hfill\break
He is completely lazy. }

\par{聞き物はない。 \hfill\break
To not have something worth listening to. }

\par{54. 屋 (や)may be added to a type of trade to show a kind of shop , describe a certain type of person often with -がる , or show an alias name. }

\par{鈴の屋 \hfill\break
Suzunoya (alias) }

\par{皮肉屋 \hfill\break
Sarcastic person }

\par{技術屋 \hfill\break
Engineer }

\par{地上げ屋 \hfill\break
Land shark\slash speculator }

\par{電気屋 \hfill\break
Electric appliance store; electrician }

\par{55. や is a suffix that attaches to nouns of person to show intimacy. }

\par{ばあや \hfill\break
Ma-ma }

\par{じいや \hfill\break
Gramps }

\par{坊やが天使のような表情を持った。 \hfill\break
The little boy possessed an angelic expression. }

\par{ねえや \hfill\break
Sis' }

\par{56. 等(ら) \textrightarrow  Lesson 5  \hfill\break
}

\par{Additional Suffixes to be Seen in Lesson 107  ~  109  }

\par{The suffixes -勝(が)ち, -っ放(ぱな)し, -っこ, -放題, -がてら, -方(かた), -並(な)み, -だらけ, -まみれ, -尽(ず)くめ, -めく, -気(げ), and -気味(ぎみ) are discussed in those lessons. }

\par{30. 尽く (ずく) attaches to nouns and the stems of adjectives. It either shows that something relies entirely on something, something has the sole purpose of something, or what something is based off of. }

\par{納得ずく \hfill\break
With mutual consent }

\par{腕尽くで相手をぶっ倒す。 \hfill\break
To beat an enemy by might. }

\par{欲得ずくで付き合う。 \hfill\break
To go steady with having a mercenary attitude. }

\par{力ずくで押し入らない方がよほどよい。 \hfill\break
It is much better to not enter by force. }

\par{31. 尽くめ (ずくめ) \textrightarrow  Lesson 109  \hfill\break
32. 達 (たち) \textrightarrow  Lesson 5  }

\par{33. だてら shows that something is unfitting of normal quality. }

\par{女だてら \hfill\break
Unladylike }

\par{親だてらにいらぬ世話を焼く。 \hfill\break
To meddle unlike that of a parent. }

\par{34. だらけ \textrightarrow  Lesson 109  }

\par{35. 垂れ (たれ) is a pejorative equivalent to "-ass" and attaches to nouns or adjectives. }

\par{しみったれ \hfill\break
Stingy a**. }

\par{馬鹿垂れ \hfill\break
Dumba** \hfill\break
\hfill\break
クソッタレ \hfill\break
S\%*\#head }

\par{36. ちゃん \textrightarrow  Lesson 70  \hfill\break
}

\par{37. 付け (つけ・づけ) may either be attached to the 連用形 of a verb  to show what one always does or attached to a day as "-づけ" to show a schedule. }

\par{行きつけの図書館はこちらです。 \hfill\break
This is the library that I always go to. }

\par{六日付(づ)けの発令。 \hfill\break
Proclamation for the sixth. }

\par{掛かり付(つ)けの医者。 \hfill\break
The family doctor. }

\par{38. 詰め (づめ) shows that an action or condition is continuing and attaches to the 連用形 of verbs. }

\par{歩き詰めでくたくただ。 \hfill\break
I was exhausted due to continuous walking. \hfill\break
\hfill\break
立ち詰めで働いていた。 \hfill\break
I have been working while continuing to stand up. }

\par{39. 連れ (づれ) shows a company of people. }

\par{旅の道連れ \hfill\break
A traveling companion. }

\par{三人連れでいればいい。 \hfill\break
It's good to be in groups of threes. }

\par{家族連れでハイキングに行く。 \hfill\break
To go hiking with one's family in tow. }

\par{40. 手 (て) is attached to the 連用形 of verbs and denotes the action doer. }

\par{くどい話し手だな。 \hfill\break
He's a windy speaker, isn't he? }

\par{彼はいつも注意深い聞き手です。 \hfill\break
He is always an alert listener. }

\par{書き手 \hfill\break
Writer }
41. 通し (どおし) is just like -詰め and shows that something continues as such. It also follows the 連用形 of verbs. \hfill\break
\hfill\break
雪が夜通し降り続いた。 \hfill\break
It snowed all night long. \hfill\break
\hfill\break
立ち通しでした。 \hfill\break
I was kept standing. 
\par{42. 通り (どおり) follows nouns and either means "Street\slash Avenue" or "following\slash in accordance with". }

\par{彼女の住所はウェスト・パーク通り6番です。 \hfill\break
Her address is 6 West Park Street. }

\par{サンセット大通り \hfill\break
Sunset Boulevard }

\par{自分の思い通りにする。 \hfill\break
To have it one's own way. }

\par{43. 殿 (どの) \textrightarrow  Lesson 70  \hfill\break
44. 共 (ども) \textrightarrow  Lesson 70  \hfill\break
}

\par{45. 栄え (ばえ) attaches to the 連用形 of verbs and is used to show a heard or shown advantage. It is limited to the verbs 見る and 聞く. Below are some examples. }

\par{音楽は聞き栄えがしなかった。 \hfill\break
The music was not pleasing to the ears. }

\par{一向に見栄えがしないよ。 \hfill\break
She's not anything to look at at all. }

\par{あのドレスは見栄えがしました。 \hfill\break
That dress was much to look at. }

\par{46. 端 (ばな) attaches to the 連用形 of verbs and means "about to start". }

\par{出端 \hfill\break
Moment of departure }

\par{寝入りばなに弟が泣き出した。 \hfill\break
My little brother began to cry as I was falling asleep. }

\par{47. 張り (ばり) attaches to personal nouns and means "reminiscent of". }

\par{雷電張りの空模様だった。 \hfill\break
It was weather reminiscent of thunder and lightning. }

\par{セザンヌ張りの風景画。 \hfill\break
Landscape paintings reminiscent of Cezanne. }

\par{シェイクスピア張りの文体。 \hfill\break
Writing style reminiscent of Shakespeare. }

\par{48. 辺 (べ) is attached to geographical nouns to mean "-side". }

\par{水辺に並んでいる町。 \hfill\break
Towns lining the waterside. }

\par{海辺で休日を過ごしたよ。 \hfill\break
I spent the holiday at the seaside. }

\par{水辺に欸乃(あいだい)を聞く。 \hfill\break
To hear a fisherman's song on the waterside. }

\par{嵐(あらし)がその大船を岸辺へと押しやった。 \hfill\break
The storm impelled the big ship to the shoreline. }

\par{49. (っ)ぽ(っ)ち attaches to demonstrative pronouns or numbers meaning "merely but". }

\par{1ドルぽっちでは買えねー。 \hfill\break
I can't buy anything with just a dollar. }

\par{これっぽっちでは足りないもんだ。 \hfill\break
This is just merely lacking. }

\par{50. 前 (まえ) attaches to words of number or person to show portion or suitable amount. This suffix oddly does not follow any exception to any counter rules. }

\par{一人前になる。 \hfill\break
To come to age. }

\par{二人前を平らげる。 \hfill\break
To consume food enough for two people. }

\par{51. み \textrightarrow  Lesson 112  }

\par{52. みどろ is a less common alternative to the suffix -まみれ to show things being covered in something. }

\par{血みどろの遺体 \hfill\break
A bloody corpse }

\par{汗みどろの労働者 \hfill\break
A sweat covered laborer }

\par{53. -もの, either 物 for things or 者 for people, attaches to the stem of nouns to show things or people "to\dothyp{}\dothyp{}\dothyp{}". }

\par{動物に食べ物を与えないでくださいませんか。 \hfill\break
Could you please not give food to the animals. }

\par{何か飲み物がほしい。 \hfill\break
I want something to drink. }

\par{着物 \hfill\break
Clothes; kimono }

\par{銀座へ買い物に行った。 \hfill\break
I went shopping in Ginza. }

\par{クリスマスの贈り物 \hfill\break
Christmas presents }

\par{乗り物に酔(よ)う。 \hfill\break
To have travel sickness. }

\par{この本はためになる読み物です。 \hfill\break
This book is good reading. }

\par{生き物一つもいない。 \hfill\break
There is not a living thing. }

\par{建物はでっかいよ! \hfill\break
The building is huge! }

\par{壊れ物! \hfill\break
Fragile objects! }

\par{彼女は本当に働き者だよ。 \hfill\break
She is a really hard worker. }

\par{まったく怠け者だぞ。 \hfill\break
He is completely lazy. }

\par{聞き物はない。 \hfill\break
To not have something worth listening to. }

\par{54. 屋 (や)may be added to a type of trade to show a kind of shop , describe a certain type of person often with -がる , or show an alias name. }

\par{鈴の屋 \hfill\break
Suzunoya (alias) }

\par{皮肉屋 \hfill\break
Sarcastic person }

\par{技術屋 \hfill\break
Engineer }

\par{地上げ屋 \hfill\break
Land shark\slash speculator }

\par{電気屋 \hfill\break
Electric appliance store; electrician }

\par{55. や is a suffix that attaches to nouns of person to show intimacy. }

\par{ばあや \hfill\break
Ma-ma }

\par{じいや \hfill\break
Gramps }

\par{坊やが天使のような表情を持った。 \hfill\break
The little boy possessed an angelic expression. }

\par{ねえや \hfill\break
Sis' }

\par{56. 等(ら) \textrightarrow  Lesson 5  \hfill\break
}

\par{Additional Suffixes to be Seen in Lesson 107  ~  109  }

\par{The suffixes -勝(が)ち, -っ放(ぱな)し, -っこ, -放題, -がてら, -方(かた), -並(な)み, -だらけ, -まみれ, -尽(ず)くめ, -めく, -気(げ), and -気味(ぎみ) are discussed in those lessons. }

\par{30. 尽く (ずく) attaches to nouns and the stems of adjectives. It either shows that something relies entirely on something, something has the sole purpose of something, or what something is based off of. }

\par{納得ずく \hfill\break
With mutual consent }

\par{腕尽くで相手をぶっ倒す。 \hfill\break
To beat an enemy by might. }

\par{欲得ずくで付き合う。 \hfill\break
To go steady with having a mercenary attitude. }

\par{力ずくで押し入らない方がよほどよい。 \hfill\break
It is much better to not enter by force. }

\par{31. 尽くめ (ずくめ) \textrightarrow  Lesson 109  \hfill\break
32. 達 (たち) \textrightarrow  Lesson 5  }

\par{33. だてら shows that something is unfitting of normal quality. }

\par{女だてら \hfill\break
Unladylike }

\par{親だてらにいらぬ世話を焼く。 \hfill\break
To meddle unlike that of a parent. }

\par{34. だらけ \textrightarrow  Lesson 109  }

\par{35. 垂れ (たれ) is a pejorative equivalent to "-ass" and attaches to nouns or adjectives. }

\par{しみったれ \hfill\break
Stingy a**. }

\par{馬鹿垂れ \hfill\break
Dumba** \hfill\break
\hfill\break
クソッタレ \hfill\break
S\%*\#head }

\par{36. ちゃん \textrightarrow  Lesson 70  \hfill\break
}

\par{37. 付け (つけ・づけ) may either be attached to the 連用形 of a verb  to show what one always does or attached to a day as "-づけ" to show a schedule. }

\par{行きつけの図書館はこちらです。 \hfill\break
This is the library that I always go to. }

\par{六日付(づ)けの発令。 \hfill\break
Proclamation for the sixth. }

\par{掛かり付(つ)けの医者。 \hfill\break
The family doctor. }

\par{38. 詰め (づめ) shows that an action or condition is continuing and attaches to the 連用形 of verbs. }

\par{歩き詰めでくたくただ。 \hfill\break
I was exhausted due to continuous walking. \hfill\break
\hfill\break
立ち詰めで働いていた。 \hfill\break
I have been working while continuing to stand up. }

\par{39. 連れ (づれ) shows a company of people. }

\par{旅の道連れ \hfill\break
A traveling companion. }

\par{三人連れでいればいい。 \hfill\break
It's good to be in groups of threes. }

\par{家族連れでハイキングに行く。 \hfill\break
To go hiking with one's family in tow. }

\par{40. 手 (て) is attached to the 連用形 of verbs and denotes the action doer. }

\par{くどい話し手だな。 \hfill\break
He's a windy speaker, isn't he? }

\par{彼はいつも注意深い聞き手です。 \hfill\break
He is always an alert listener. }

\par{書き手 \hfill\break
Writer }
41. 通し (どおし) is just like -詰め and shows that something continues as such. It also follows the 連用形 of verbs. \hfill\break
\hfill\break
雪が夜通し降り続いた。 \hfill\break
It snowed all night long. \hfill\break
\hfill\break
立ち通しでした。 \hfill\break
I was kept standing. 
\par{42. 通り (どおり) follows nouns and either means "Street\slash Avenue" or "following\slash in accordance with". }

\par{彼女の住所はウェスト・パーク通り6番です。 \hfill\break
Her address is 6 West Park Street. }

\par{サンセット大通り \hfill\break
Sunset Boulevard }

\par{自分の思い通りにする。 \hfill\break
To have it one's own way. }

\par{43. 殿 (どの) \textrightarrow  Lesson 70  \hfill\break
44. 共 (ども) \textrightarrow  Lesson 70  \hfill\break
}

\par{45. 栄え (ばえ) attaches to the 連用形 of verbs and is used to show a heard or shown advantage. It is limited to the verbs 見る and 聞く. Below are some examples. }

\par{音楽は聞き栄えがしなかった。 \hfill\break
The music was not pleasing to the ears. }

\par{一向に見栄えがしないよ。 \hfill\break
She's not anything to look at at all. }

\par{あのドレスは見栄えがしました。 \hfill\break
That dress was much to look at. }

\par{46. 端 (ばな) attaches to the 連用形 of verbs and means "about to start". }

\par{出端 \hfill\break
Moment of departure }

\par{寝入りばなに弟が泣き出した。 \hfill\break
My little brother began to cry as I was falling asleep. }

\par{47. 張り (ばり) attaches to personal nouns and means "reminiscent of". }

\par{雷電張りの空模様だった。 \hfill\break
It was weather reminiscent of thunder and lightning. }

\par{セザンヌ張りの風景画。 \hfill\break
Landscape paintings reminiscent of Cezanne. }

\par{シェイクスピア張りの文体。 \hfill\break
Writing style reminiscent of Shakespeare. }

\par{48. 辺 (べ) is attached to geographical nouns to mean "-side". }

\par{水辺に並んでいる町。 \hfill\break
Towns lining the waterside. }

\par{海辺で休日を過ごしたよ。 \hfill\break
I spent the holiday at the seaside. }

\par{水辺に欸乃(あいだい)を聞く。 \hfill\break
To hear a fisherman's song on the waterside. }

\par{嵐(あらし)がその大船を岸辺へと押しやった。 \hfill\break
The storm impelled the big ship to the shoreline. }

\par{49. (っ)ぽ(っ)ち attaches to demonstrative pronouns or numbers meaning "merely but". }

\par{1ドルぽっちでは買えねー。 \hfill\break
I can't buy anything with just a dollar. }

\par{これっぽっちでは足りないもんだ。 \hfill\break
This is just merely lacking. }

\par{50. 前 (まえ) attaches to words of number or person to show portion or suitable amount. This suffix oddly does not follow any exception to any counter rules. }

\par{一人前になる。 \hfill\break
To come to age. }

\par{二人前を平らげる。 \hfill\break
To consume food enough for two people. }

\par{51. み \textrightarrow  Lesson 112  }

\par{52. みどろ is a less common alternative to the suffix -まみれ to show things being covered in something. }

\par{血みどろの遺体 \hfill\break
A bloody corpse }

\par{汗みどろの労働者 \hfill\break
A sweat covered laborer }

\par{53. -もの, either 物 for things or 者 for people, attaches to the stem of nouns to show things or people "to\dothyp{}\dothyp{}\dothyp{}". }

\par{動物に食べ物を与えないでくださいませんか。 \hfill\break
Could you please not give food to the animals. }

\par{何か飲み物がほしい。 \hfill\break
I want something to drink. }

\par{着物 \hfill\break
Clothes; kimono }

\par{銀座へ買い物に行った。 \hfill\break
I went shopping in Ginza. }

\par{クリスマスの贈り物 \hfill\break
Christmas presents }

\par{乗り物に酔(よ)う。 \hfill\break
To have travel sickness. }

\par{この本はためになる読み物です。 \hfill\break
This book is good reading. }

\par{生き物一つもいない。 \hfill\break
There is not a living thing. }

\par{建物はでっかいよ! \hfill\break
The building is huge! }

\par{壊れ物! \hfill\break
Fragile objects! }

\par{彼女は本当に働き者だよ。 \hfill\break
She is a really hard worker. }

\par{まったく怠け者だぞ。 \hfill\break
He is completely lazy. }

\par{聞き物はない。 \hfill\break
To not have something worth listening to. }

\par{54. 屋 (や)may be added to a type of trade to show a kind of shop , describe a certain type of person often with -がる , or show an alias name. }

\par{鈴の屋 \hfill\break
Suzunoya (alias) }

\par{皮肉屋 \hfill\break
Sarcastic person }

\par{技術屋 \hfill\break
Engineer }

\par{地上げ屋 \hfill\break
Land shark\slash speculator }

\par{電気屋 \hfill\break
Electric appliance store; electrician }

\par{55. や is a suffix that attaches to nouns of person to show intimacy. }

\par{ばあや \hfill\break
Ma-ma }

\par{じいや \hfill\break
Gramps }

\par{坊やが天使のような表情を持った。 \hfill\break
The little boy possessed an angelic expression. }

\par{ねえや \hfill\break
Sis' }

\par{56. 等(ら) \textrightarrow  Lesson 5  \hfill\break
}

\par{Additional Suffixes to be Seen in Lesson 107  ~  109  }

\par{The suffixes -勝(が)ち, -っ放(ぱな)し, -っこ, -放題, -がてら, -方(かた), -並(な)み, -だらけ, -まみれ, -尽(ず)くめ, -めく, -気(げ), and -気味(ぎみ) are discussed in those lessons. }

\par{------------------------------------------------------------------------------- }

\par{Next Lesson \textrightarrow  第104課 : Native Suffixes II: Adjectival and Adverbial  }

\par{------------------------------------------------------------------------------- }

\par{Next Lesson \textrightarrow  第104課 : Native Suffixes II: Adjectival and Adverbial  }
    