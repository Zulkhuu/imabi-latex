    
\chapter{Opportunity}

\begin{center}
\begin{Large}
第306課: Opportunity: きっかけ \& 契機 
\end{Large}
\end{center}
 
\par{ This lesson is the second of five lessons that focus on nominal phrases that deserve special attention. }
      
\section{きっかけ}
 
\par{  As a noun, きかっけに means "trigger" or "prompt". In the speech modal きっかけに it means "take the opportunity to\dothyp{}\dothyp{}\dothyp{}". }
 
\par{1. 日本語を勉強し始めたきかっけけは何ですか。 \hfill\break
What was it that triggered you to begin studying Japanese? }
 
\par{2. この事件をきかっけに、反対の声を上げる。 \hfill\break
To raise opposition in taking advantage of the case. }
 
\par{3. 事故をきかっけに、家族の ${\overset{\textnormal{けっそく}}{\text{結束}}}$ を ${\overset{\textnormal{かた}}{\text{固}}}$ める。 \hfill\break
Take the accident to bring the family back together. }
      
\section{契機}
 
\par{ 契機 means "opportunity" and is in the speech modal を契機に meaning "taking the opportunity". }

\par{4. それを契機に、新しい商売を始める。 \hfill\break
Taking the opportunity and starting a new business. }

\par{5. 言論弾圧を契機に暴動が起こる事がバーレーンに生じています。 \hfill\break
Violence is occurring in taking the opportunity of suppressing speech in Bahrain. }
    