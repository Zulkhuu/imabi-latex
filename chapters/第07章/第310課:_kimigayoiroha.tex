    
\chapter{The 君が代 \& いろは}

\begin{center}
\begin{Large}
第310課: The 君が代 \& いろは 
\end{Large}
\end{center}
 
\par{ 君が代, which is usually translated as "His Majesty's Reign", is the 国歌 (national anthem) of the great nation of Japan. The song first started out as a 和歌 from the Heian Period (平安時代). After the 明治維新 (Meiji Restoration) in 1880, it was turned into a song and then eventually treated as the national anthem, but this only became official in 1999 despite most Japanese people already thinking it was. The anthem happens to be the shortest in the world with only 32 characters. }
      
\section{君が代}
 
\par{\textbf{Lyrics }: }

\begin{center}
君が代は \hfill\break
千代に八代に \hfill\break
さざれ石の \hfill\break
巌となりて \hfill\break
苔の生すまで 
\end{center}

\par{\textbf{かな Version }: }

\begin{center}
きみがよは \hfill\break
ちよにはちよに \hfill\break
さざれいしの \hfill\break
いわおとなりて \hfill\break
こけのむすまで 
\end{center}

\par{\textbf{Standard English Translation }: }

\begin{center}
May your reign \hfill\break
Continue for a thousand, eight thousand generations, \hfill\break
Until the pebbles \hfill\break
Grow into boulders \hfill\break
Lush with moss 
\end{center}

\par{ The song originated from a 短歌 in the 古今和歌集. The song describes how the Imperial Household will continue forever. One main difference between the current version and the original was that 君が代 was わが君, reflecting how it used to be a song sung on formal occasions to wish for a long life. Many disapprove of the strong implications of revering the Emperor as figure of worship. Others view it as a piece of history and that the politics is a separate matter. }
      
\section{The Grammar}
 
\par{ Given that there is quite a bit of old grammar in such a small song, it's important to note those differences. First, let's look at the lyrics again. }

\begin{center}
君が代は \hfill\break
千代に八代に \hfill\break
さざれ石の \hfill\break
巌となりて \hfill\break
苔の生すまで 
\end{center}

\par{ The first issue is 君. This word at the time the song was written could refer to someone like the Emperor, but it could also be used to anyone of high status. In fact, Genji, the main character in the famous story 源氏物語, is often referred as 光の君 in it. }

\par{ The case particles が and の used to be completely interchangeable in Classical Japanese. This is explains essentially most of the differences. Another is that the て形 contractions are not seen at this time, and if they were, it would not have been indicative of literary language just yet. Even to this day when things are purposely made classical, such contractions are almost always completely avoided. }
      
\section{いろは (伊呂波)}
 
\par{ This is often called the Japanese alphabet song, despite that かな are syllabaries. The いろは first appeared in 万葉仮名 but brilliantly uses each basic sound only once. The song was attributed to 空海, who founded the 真言 sect of Buddhism. However, due to a rather interesting hidden message in the poem, it's likely that it was actually written after his death. }

\begin{center}
 \textbf{万葉仮名 Version }
\end{center}
  以呂波耳本部止 千利奴流乎和加  餘多連曽津祢那 \hfill\break
 良牟有為能於久 \hfill\break
 耶万計不己衣天 \hfill\break
 阿佐伎喩女美之 \hfill\break
 恵比毛勢須  \textbf{かな Version }
\begin{center}
いろはにほへと \hfill\break
ちりぬるを \hfill\break
わかよたれそ \hfill\break
つねならむ \hfill\break
うゐのおくやま \hfill\break
けふこえて \hfill\break
あさきゆめみし \hfill\break
ゑひもせす 
\end{center}

\par{  As you can see, there is no marking of voiced sounds. The pronunciation of words has also changed just as much as the grammar. ん was not standard as a specific character for N' until Modern times, so it shouldn't be expected to be in here. It also has the obsolete ゐ and ゑ as these sounds were used in Japanese at the time. However, once the current 五十音 ordering, which is based off the Sanskrit ordering of sounds, was adopted in the Meiji Restoration, this became out of use for most situations to list the sounds. }

\par{いろはにほへど = 色は匂えど  Although the scent lingers \hfill\break
ちりぬるを    = 散りぬるを   It (the flowers) will eventually blossom \hfill\break
わかよたれそ  = 我が世誰ぞ  Who in our world \hfill\break
常ならむ     = 常ならむ    Never changes? \hfill\break
うゐのおくやま = 有為の奥山  The deep mountains of vanity \hfill\break
けふこえて   = 今日超えて   Today we cross \hfill\break
あさきゆめみし = 浅き夢見じ  And we shall not see superficial dreams \hfill\break
ゑひもせす   = 酔ひもせず  Nor be drunk   }

\par{  The particle ど follows the 已然形 and has the same function as けれど, which is a derivative of it with an auxiliary. -ぬる is the 連体形 of the intransitive perfective auxiliary verb -ぬ, which is no longer used in Modern Japanese and followed the 連用形 of verbs. }

\par{ The ぞ is a 係助詞 here, which is particle class with rather difficult usage. Essentially, it would be at the end of the sentence in Modern Japanese, ignoring the fact that the particle in question is only archaically used in Modern Japanese to be used in conjunction with the old volitional ending -む to make a rhetorical question. }

\par{ けふ means "today" and it used to be pronounced as \slash kepu\slash . This would eventually become \slash keu\slash , which eventually become a diphthong and turn into \slash kyo:\slash , which it is today in Standard Modern Japanese. The word is a combination of け, the original word for today and ふ, which is the 終止形 of the Classical form of the verb 経る (to pass time). }

\par{ ~じ is a negative auxiliary that means ~ないだろう. Thus, it naturally follows the 未然形. Lastly, もせず = もしないで. }

\begin{center}
 \textbf{五十音順 }
\end{center}

\par{ For many things in life, though, the 五十音順 is used instead. But, have you ever wondered how you would arrange 濁音 and 拗音?  If you had たかた and たかだ, you would list them in that order, but if you had かわさきじろう (川崎次郎) and かわざきいちろう (河崎一郎), you would start with 河崎一郎. This is because you have to think of the order of the 五十音. い comes before し・じ. You look at 濁音 last. As for 拗音, there is no single convention, but non-拗音 are usually first. }
    