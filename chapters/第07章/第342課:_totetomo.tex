    
\chapter{The Particles とて \& とも}

\begin{center}
\begin{Large}
第342課: The Particles とて \& とも 
\end{Large}
\end{center}
 
\par{ The particle とて is the ancestor of ても and って. Then, we'll end with とも. }
      
\section{The Case Particle とて}
 
\par{  The case particle とて is equivalent to と言って・思って, and  こととて means "with\dothyp{}\dothyp{}\dothyp{}being the reason". It is basically the older equivalent of ので・から.It is \textbf{rather archaic }. So, many speakers will use other things instead. In fact, it's even hard to find a Japanese person that even knows anything about とて. It is, though, preserved in dramas and what not, things that purposely retain older language. }
 
\par{1. 子供のこととてお許し下さい。 (文語) \hfill\break
Please forgive me with this being a childish thing. }
 
\par{2. 先を急ぐとて旅立った。 \hfill\break
When I thought to hurry ahead, I set off on my journey. }
 
\par{3. 猿とて竜を殺そうとすれば自分が死ぬのは予想できるだろう。(More emphatic) \hfill\break
Even a monkey would know that you would surely die if you tried to kill the dragon. }
 
\par{4a. 不注意のこととて事故に ${\overset{\textnormal{あ}}{\text{遭}}}$ った。(Old) \hfill\break
4b. 注意不足だったせいで事故に遭った。(Natural) \hfill\break
4c. 不注意ゆえ事故に遭った。(Literary) \hfill\break
I had an accident because of my carelessness. }

\par{5a. ${\overset{\textnormal{ねん}}{\text{念}}}$ を入れたこととてよい仕上がりだ。 \hfill\break
5b. 念を入れただけあって、よい仕上がりだ。 \hfill\break
Because we paid attention to detail, it's (now) a good finish. }
 
\par{6a. 山に登るとて出かけた。 \hfill\break
6b. 山に登ると言って出かけた。(Natural) \hfill\break
When I said to climb at the mountain, he went out to. }
      
\section{The Adverbial Particle とて}
 
\par{ とて is equivalent to だって and であっても and means "even (though)". }

\par{7. 犬とてそんなことは分かる。 \hfill\break
Even dogs understand such a thing. }

\par{8. この事件とて例外ではありません。 \hfill\break
Even this case is not exceptional. }

\par{9. どのような事件とても例外にはあらず。 \hfill\break
There is no exception no matter what the situation. \hfill\break
}

\par{10. 彼のためとて(何でも)やるというわけではない。 \hfill\break
Even if it was for him, it's not to say that I would. \hfill\break
}

\par{11. 彼女らとてやりたくてやったわけではない。 \hfill\break
Even though they wanted to do it, it's not to say that they did it. }
      
\section{The Conjunctive Particle とて}
 
\par{  The conjunctive とて may attach to the 終止形 to mean "even if". The combinations -たとて and -だとて become -たって and -だって respectfully in the spoken language. からとて, which is normally replaced with \textbf{からといって or からって }, means "just because". }

\par{12. ${\overset{\textnormal{つか}}{\text{疲}}}$ れたからとて ${\overset{\textnormal{あきら}}{\text{諦}}}$ めるわけにはいかぬ! \hfill\break
You just can't quit because you're tired! }

\par{13. ${\overset{\textnormal{なぐ}}{\text{殴}}}$ られたとてかまわない。 \hfill\break
It's fine even if it's struck. }

\par{14. たとえ反対されたとて、やり ${\overset{\textnormal{ぬ}}{\text{抜}}}$ くぞ。 \hfill\break
Even if you were to protest, you'd still carry through. }

\par{15. 本当かなあ、本当にノー・プロブレムかなあと心配だったのだが、さりとて他に道もないし、とにかくこれからウ       ルデレまで行ってみるしかない。 \hfill\break
I worried whether it were so, that there really was no problem; nevertheless, there is no other road, and we had no choice but to try to go from now to Uludere anyway. \hfill\break
From 雨天炎天―ギリシャ・トルコ辺境紀行― by 村上春樹. }

\par{\textbf{Grammar Note }: さり is the 終止形 of an ancient copular structure, which accounts for its odd appearance in comparison to modern structures. さりとて = そうだからといって. }
      
\section{The Conjunctive\slash Final Particle とも}
 
\begin{ltabulary}{|P|P|}
\hline 

 Phrase & Meaning \\ \cline{1-2}

どんなに\dothyp{}\dothyp{}\dothyp{}とも & No matter\dothyp{}\dothyp{}\dothyp{} \\ \cline{1-2}

どれだけ\dothyp{}\dothyp{}\dothyp{}とも & No matter\dothyp{}\dothyp{}\dothyp{} \\ \cline{1-2}

Adverb + とも & At the least\slash most \\ \cline{1-2}

Repetition of とも & Whether \\ \cline{1-2}

とも & 語尾 emphasizing a strong statement \\ \cline{1-2}

たとえ + vol. +とも & Even if \\ \cline{1-2}

\dothyp{}\dothyp{}\dothyp{}なくともいい・かまわない \hfill\break
& Don't mind \hfill\break
\\ \cline{1-2}

Value\slash degree + とも & At the very \hfill\break
\\ \cline{1-2}

\end{ltabulary}

\par{\textbf{Classification Note }: When used as a 語尾 it is classified as a final particle. }

\par{16. たとえ雨が降ろうともフットボールをする。 \hfill\break
I play soccer even if it's raining. }

\par{17. どれだけ時間がかかろうとも、支持します。 \hfill\break
I will support them no matter how long it takes. }

\par{18. 遅くとも3時までに来なさい。(From superior to inferior) \hfill\break
Come here by three at the latest. }

\par{19. 少なくとも10人は要ります。 \hfill\break
We need ten people at the least. \hfill\break
}

\par{20. 理解されなくともまったくかまわない。 \hfill\break
I really don't care if its not understood. \hfill\break
}

\par{21. 来るとも来ないとも分からない。 \hfill\break
I don't know whether he's coming or not coming. }

\par{22. そうだとも! \hfill\break
Absolutely! }

\par{23. 特に英語を勉強しなくとも、楽勝だろうね。 \hfill\break
Particularly English, i's probably a piece of cake for you even if you don't study. }
    