    
\chapter{Numbers IX}

\begin{center}
\begin{Large}
第314課: Numbers IX: Decimal, Fraction, 大字, \& Large Numbers 
\end{Large}
\end{center}
 
\par{ Way back when numbers were first discussed, you learned the basic information about numbers in Japanese. However, the Sino-Japanese numbering system can go much farther. This lesson will also consider things not mentioned before such as decimal fractions. }
      
\section{Decimal}
 
\par{ Decimals, ${\overset{\textnormal{しょうすう}}{\text{小数}}}$ , aren't difficult to make at all in Japanese. First, you need a number before the decimal point, which is simply read out as 点. The actual name for decimal point is 小数点. If you have zero before the decimal point, then you should say ゼロ・れい・まる. If you have a full number before the decimal point, you should say it correctly. So, if there is 300 before the decimal point, say さんびゃく. After the decimal point, you can simply read out the numbers. So, 4.56757 = よん てん ご ろく なな ご なな. }

\begin{center}
 \textbf{Decimal Fractions }
\end{center}

\par{ As you can imagine, this is not the traditional way of dealing with decimal values. Decimal fractions, which break up things by negative powers and referred to in simple English as tenths, hundredths, etc. place, are represented in Japanese with a set of decimal fraction units that stand for each place from the decimal point and back. }

\par{ This seems easy enough, right? Sadly, Japanese doesn't make this 100\% straight-forward. There are two different decimal fraction systems in use in Japan today. They are not completely interchangeable and are used in their respective situations. }

\begin{center}
 \textbf{First Decimal Fraction System }
\end{center}

\par{ The first decimal fraction system has many units, but practical use ends at 10-2. Even 厘 is not that common. This system is used in batting averages, idioms, etc. Given how important baseball and set expressions are, remembering four small things isn't that bad. The chart below, though, lists all of the possible units in this system for completeness. }

\begin{ltabulary}{|P|P|P|P|P|P|P|P|P|}
\hline 

10-1 & 分 & ぶ & 10-2 & 厘 & りん & 10-3 & 毛 & もう \\ \cline{1-9}

10-4 & 糸 & し & 10-5 & 忽 & こつ & 10-6 & 微 & び \\ \cline{1-9}

10-7 & 繊 & せん & 10-8 & 沙 & しゃ & 10-9 & 塵 & じん \\ \cline{1-9}

10-10 & 埃 & あい & 10-11 & 渺 & びょう & 10-12 & 漠 & ばく \\ \cline{1-9}

10-13 & 模糊 & もこ & 10-14 & 逡巡 & しゅんじゅん & 10-14 & 須臾 & しゅゆ・すゆ \\ \cline{1-9}

10-16 & 瞬息 & しゅんそく & 10-17 & 指弾 & しだん & 10-17 & 刹那 & せつな \\ \cline{1-9}

10-19 & 六徳 & りっとく & 10-20 & 虚 & きょ & 10-20 & 空 & くう \\ \cline{1-9}

10-22 & 清 & せい & 10-23 & 浄 & じょう &  &  &  \\ \cline{1-9}

\end{ltabulary}

\par{1. 七分袖 \hfill\break
Three-quarter sleeves\slash Sleeves 7\slash 10 the size of a long sleeve. }

\begin{center}
\textbf{Second Decimal Fraction System }
\end{center}

\par{ The second system is not all that different. This one is far shorter and is only used up until 10-5. This is mainly due to the fact that it is used for showing discounts, and prices don't need to be exact as say a scientific finding. This system introduces a different unit for 10-1 and pushes the others from the first system down a power. Realistically, you would never need to go past 分. }

\begin{ltabulary}{|P|P|P|P|P|P|P|P|P|P|}
\hline 

10-1 & 割(わり) & 10-2 & 分 & 10-3 & 厘 & 10-4 & 毛 & 10-5 & 糸 \\ \cline{1-10}

\end{ltabulary}

\par{2. 五分五分の確率 \hfill\break
Even odds }

\par{3. ${\overset{\textnormal{くぶ}}{\text{九分}}}$ ${\overset{\textnormal{くりん}}{\text{九厘}}}$ \hfill\break
 Basically 99\% chance, this is used to mean that something is basically so. }

\par{4. あの ${\overset{\textnormal{しょうりょこう}}{\text{小旅行}}}$ での ${\overset{\textnormal{い}}{\text{往}}}$ きの汽車のなかの気楽な対話は、その \textbf{八九分 }を隣りのお ${\overset{\textnormal{しゃべ}}{\text{喋}}}$ りや小さな妹たちに ${\overset{\textnormal{お}}{\text{負}}}$ うていたものだとわかった。 \hfill\break
I realized that our carefree dialogue on that brief train journey was largely due to the chatterbox behind us and the two little sisters. }

\par{\textbf{Reading Note }: 八九分 is read as はっくぶ and literally means 80, 90\%, but it is used to mean "largely". }

\par{5. 九分九厘勝てる。 \hfill\break
We'll almost certainly be able to win. }

\par{\textbf{Reading Note }: 9割 is always read as きゅうわり.  Also, no one says よわり. It's よんわり. }

\begin{center}
\textbf{Percentage }
\end{center}

\par{ Hopefully you know that .5 = 50\%. So, how exactly do you express percentage in Japanese? Thankfully, it's the same as in English. You just have a number and パー(セント)・%. }

\par{6. 4.5\% \hfill\break
よん てん ご パー(セント) }

\begin{center}
 \textbf{Fractions }
\end{center}

\par{  For a fraction ( ${\overset{\textnormal{ぶんすう}}{\text{分数}}}$ ) take a number, add " ${\overset{\textnormal{}}{\text{分}}}$ の" and follow it with another number. This creates phrases such as "三分の二" meaning "two of three parts" or "two thirds". }

\par{7. ケーキの四分の一 \hfill\break
A quarter\slash fourth of the cake. }

\par{8. 五分の六 \hfill\break
Six fifths }

\par{\textbf{Vocabulary Note }: Improper fraction in Japanese is " ${\overset{\textnormal{かぶんすう}}{\text{仮分数}}}$ ". }

\par{\textbf{Reading Note }: 1\slash 4th is always read as よんぶんのいち. }
      
\section{大字}
 
\par{ 大字 are formal 漢字 for numbers created to prevent counterfeiting. That sounds like a noble thing to do, but for those that don't like complicated for simple things, they can be annoying additional characters for things you already know. Thankfully, most have become obsolete, so you will only be introduced to the ones pertaining to Japan today.  }

\par{ The only ones that you are responsible for knowing are  1 (壱), 2  (弐), 3 (参), 10 (拾), and 10,000 (萬). The latter is not actually used on currency anymore in Japan, but it does have unofficial usage. }
      
\section{Astronomical Numbers}
 
\par{ Japanese, with the aid of Sino-Japanese numerals, can be used to count all of the particles of sand on the Earth with greater success. Again, in Japanese big numbers are grouped into powers of four. We stopped in the first lesson concerning numbers at 10\^{}12 (trillion), 兆. }

\par{ Now, you'll get to see the limit of this system, which is 10\^{}68 until some genius decides to add to this insane list. By no means should you remember any of this as even I only remember a few of these on the spot, but it is some more frivolous and trivial knowledge for fellow Japanese nerds that want to know more about something than the average Japanese speaker. There is some good out of this seemingly nonsensical information. You get to an uncommon reading of 京. You also get to see a weird usage of 不可思議. }

\par{ なゆた is actually a rather fun word that happens to be used practically to refer to an endless quantity. It sounds fitting, but at look at this system, it is somewhat a misnomer. Well, it's not as bad as the native equivalent of such a concept which treats ${\overset{\textnormal{ち}}{\text{千}}}$ and ${\overset{\textnormal{よろず}}{\text{万}}}$ as numerous. }

\par{ If you never remember 秭 , don't feel bad. If you tried to use this in practical use--how it would be practical is beyond reason--people will probably think you meant 一助 or that you just made an error. }

\begin{ltabulary}{|P|P|P|P|P|P|}
\hline 

10\^{}16 & 一京 & いっけい & 10\^{}20 & 一垓 & いちがい \\ \cline{1-6}

10\^{}24 & 一秭 & いちじょ・いちし & 10\^{}28 & 一穣 & いちじょう \\ \cline{1-6}

10\^{}32 & 一溝 & いっこう & 10\^{}36 & 一潤 & いっかん \\ \cline{1-6}

10\^{}40 & 一正 & いっせい & 10\^{}44 & 一載 & いっさい \\ \cline{1-6}

10\^{}48 & 一極 & いちごく & 10\^{}52 & 一恒河沙 & いちこうがしゃ \\ \cline{1-6}

10\^{}56 & 一阿僧祇 & いちあそうぎ & 10\^{}60 & 一那由[他・多] & いちなゆた \\ \cline{1-6}

10\^{}64 & 一不可思議 & いちふかしぎ & 10\^{}68 & 一無量大数 & いちむりょうたいすう \\ \cline{1-6}

\end{ltabulary}
   Though this may never be relevant to you, it's interesting to know that the values of these numbers haven't been standardized for very long. Historically in English, the same thing can be said for word like "billion" and "trillion", which refer to 10\^{}9 and 10\^{}12 respectively in American English.   兆 is a unit that is used in Japan, Taiwan, Korea, and China, but it is in China that the value is different. Currently in China, it refers to 10\^{}6, which is half of the current value of 10\^{}12 in the rest of Asia. The other systems are confusing to understand, so the chart below should suffice if you were to ever find them being used, which would mean you're reading really old stuff about math. \hfill\break
\hfill\break
From Wikipedia: 
\begin{ltabulary}{|P|P|P|P|P|P|P|P|}
\hline 

下数 万進(現在) 万万進 上数 

10 5  & 億  & 10 8  & 一億 & 10 8 & 一億 & 10 8 & 一億 \\ \cline{1-8}

10 6  & 兆 & \multicolumn{2}{|c|}{- }& \multicolumn{2}{|c|}{- }& \multicolumn{2}{|c|}{- }\\ \cline{1-8}

10 7  & 京  & 10 11  & 千億 & 10 15 & 千万億 & 10 15 & 千万億 \\ \cline{1-8}

\multicolumn{2}{|c|}{}& 10 12  & 一兆 & 10 16 & 一兆 & 10 16 & 一兆 \\ \cline{1-8}

10 13  & 十兆 & 10 17  & 十兆 & 10 17 & 十兆 \\ \cline{1-8}

10 14  & 百兆 & 10 18  & 百兆 & 10 18 & 百兆 \\ \cline{1-8}

10 15  & 千兆 & 10 19  & 千兆 & 10 19 & 千兆 \\ \cline{1-8}

10 16  & 一京 & 10 20  & 一万兆 & 10 20 & 一万兆 \\ \cline{1-8}

\multicolumn{2}{|c|}{}& 10 21  & 十万兆 & 10 21 & 十万兆 \\ \cline{1-8}

10 22  & 百万兆 & 10 22 & 百万兆 \\ \cline{1-8}

10 23  & 千万兆 & 10 23 & 千万兆 \\ \cline{1-8}

10 24  & 一京 & 10 24 & 一億兆 \\ \cline{1-8}

\multicolumn{2}{|c|}{}& 10 25  & 十億兆 \\ \cline{1-8}

10 26  & 百億兆 \\ \cline{1-8}

10 27  & 千億兆 \\ \cline{1-8}

10 28  & 一万億兆 \\ \cline{1-8}

10 29  & 十万億兆 \\ \cline{1-8}

10 30  & 百万億兆 \\ \cline{1-8}

10 31  & 千万億兆 \\ \cline{1-8}

10 32  & 一京  \\ \cline{1-8}

\end{ltabulary}
\hfill\break
\hfill\break
    