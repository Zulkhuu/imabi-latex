    
\chapter{Punctuation}

\begin{center}
\begin{Large}
第346課: Punctuation  
\end{Large}
\end{center}
 
\par{ Text (文章) is made of sentences (文), which make paragraphs (文節). Words aren't spaced, but overall, punctuation is very similar to English. Unlike English, though, there aren't understood guidelines. Punctuation is hardly taught. Most points in this lesson come from the Ministry of Education's suggestions from 1946. To best demonstrate the contents of it, examples provided in it will also be shown here with revised spellings. }

\par{\textbf{Terminology Notes }: Punctuation marks = 約物(やくもの); Punctuation = 句読法(くとうほう) }
      
\section{句点・まる}
 
\par{ 。 marks the end of a statement\slash sentence including inversions even in parentheses. However, if it is very simple item, it is omitted. The decision is subjective, but things like 「気をつけて」 don't need 。. You may also see it after parentheses at the end of a sentence. }
 
\par{1. 大雨の影響で三重県内では桑名市で少なくとも7棟の住宅が、菰野町で1棟が床上まで水につかりました。 \hfill\break
Due to the effect of heavy rain there are 7 houses at least in Kuwana City inside Mie Prefecture and one house in Komono Town where they are submerged in water above floor level. }
 
\par{2. 来た、来た、あの猫が。 \hfill\break
It came, it came, that cat. }
 
\par{3. 「どっちへ。」「葛飾まで。」 \hfill\break
"To where?" "To Katsushika" }
 
\par{4. おいしい、これ。 \hfill\break
This is delicious. }
 
\par{5. このことは、すでに第四章で説明した(六十七頁参照) \hfill\break
This is already explained in Chapter 4 (See Pg. 67) }

\begin{center}
 \textbf{The }\textbf{ピリオド }
\end{center}
 
\par{The ピリオド is used in horizontal writing to separate the parts of a date. Its actual use as a period is extremely, extremely rare because there is 。. }
 
\par{6. 99.4.5 \hfill\break
March 5, 1999 }
      
\section{読点(とうてん) \& コンマ}
 
\par{ The 読点, 、, is used at the writer's discretion for a number of things. Its main usage is to show a pause. It may also show the separation of numbers. As for numbers, there is the native breakup by powers of four, and then there is the Western way by powers of three. Both are used with the Western way in formal, official business. In horizontal text, the comma may rarely be seen as ,. }

\par{ Above is what you'd need if you quickly want to know information about the Japanese comma. There are actually other guidelines out there, but as mentioned before, many Japanese never study them. Nevertheless, knowing the following information will aid you in your writing. }

\par{7. 私は、野球が大好きです。 \hfill\break
I love baseball. }

\par{8. 二、三〇〇円 \hfill\break
Two, three hundred yen }

\par{9. 三五六, 五六七, 三五二 \hfill\break
356,567,352 }

\par{10. 出た,出た,月が。 \hfill\break
It's out, it's out, the moon. }

\par{11. 聞いたか、僕のいうことを? \hfill\break
Did you hear\dothyp{}\dothyp{}\dothyp{}what I said? }

\par{ As a primary general rule, a comma is placed after a pause in a sentence, with the most obvious place being after dependent clauses. Even if a pause takes the 終止形, 、 is used instead of 。 if the sense of the sentence continues. However, based on other balances in the sentence, 。 may prevail. }

\par{12a. 父も喜び、母も喜んだ。 \hfill\break
12b. 父も喜んだ、母も喜んだ。 \hfill\break
My father and my mother were joyous. }

\par{13. この真心が天に通じ、人の心をも動かしたのであろう。彼の事業はようやく村人の間に理解されはじめました。 \hfill\break
This devotion lead to heaven, and definitely moved the hearts of people. His work was finally beginning to be understood among the villagers. }

\par{ There is also something like a half-period called the シロテン. Its use has been proposed and promoted, and in the event of leaving the above sentence as, }

\par{14. この真心が天に通じ、人の心をも動かしたのであろう。彼の事業は・・・ }

\par{ it would be perfect rather than the period. The use of シロテン is very limited, but you may see it depending on how much literature you read.   is what it looks like. It is essentially a hollowed out comma. Typing it is not an option yet. However, if there were more people willing to use it, that could be changed. }

\par{ The secondary principal to using the comma is placing it before and after adverbial phrases. However, intonation may cause some the comma to be deleted. }

\par{15. 昨夜、帰宅以来、お尋ねの件について(、)当時の日誌を調べてみましたところ、やはり(、)そのとき申し上げた通りでありました。(敬語) \hfill\break
Last night, after returning home, in regards to matter I was asked, just when I looked through my diary at the time, it was as to be expected just as I had said then. }

\par{ A comma is meant in Japanese to flow with the context. The context should be easy to process, and so the most important thing to gather is that its use is appropriately controlled with realistic utterance in mind.  As such, just as in English, the comma can clarify intended meaning. With that said, its use with conjunctions, interjections, and aizuchi are quite the same. }

\par{16. 坊や、おいで。 \hfill\break
Boy, come here. }

\par{17. また、私は・・・・ \hfill\break
I, again,\dothyp{}\dothyp{}\dothyp{} \hfill\break
 \hfill\break
18. 私は(、)反対です。 \hfill\break
I'm against it. }

\par{ When there is more than one attribute, the first attribute should be marked with a comma. }

\par{19. くじゃくは、長い、美しい尾を扇のように広げました。 \hfill\break
The peacock expanded its long, beautiful tail like a fan. }

\par{ Of course, let's not forget the comma's use in listing. It's also often placed before the start of 「」. It's also after と、 when text follows rather than a citation verb. Even in listing, though, "unneeded" commas are commonly deleted. This choice is seen with particles like と and も. }

\par{20a. 父も母も兄も姉も \hfill\break
20b. 父も、母も、兄も、姉も \hfill\break
Also my father, mother, older brother and older sister }

\par{\textbf{Terminology Note }: 読(点) is 、 and コンマ is , . }
      
\section{感嘆符 \& 疑問符}
 
\par{ Also referred to with the loanword, エスクラメーションマーク, a ! is used to show great exclamation. }

\par{21. 一体全体どうなってるんだ! \hfill\break
What on earth! }

\begin{center}
\textbf{The }\textbf{疑問符(ぎもんふ) }
\end{center}

\par{A 疑問符 or クエスチョンマーク,?, is used to mark a question emphatically. The question is normally created by a rise in intonation rather than with the particle か. The mark is usually called ハテナ. }

\par{22. えっ、どういうこと? \hfill\break
Huh, what do you mean? }
      
\section{点線}
 
\par{ 点々 is a line of usually three~five dots that shows the dodging\slash evasion\slash significant pause during a conversation. When lengthened to a whole chain, it is then called a 点線, "a line of dots". This is used frequently to show that there is an omission of text afterwards. This is just like in English. This can be seen in connecting chapter titles and page numbers in tables of contents like below. }

\par{23. えっと・・・何? \hfill\break
Uh\dothyp{}\dothyp{}\dothyp{}.what? }

\par{24. 「それからね・・・・・・いやいや、もうなんにも申し上げますまい」 (ちょっと古風) \hfill\break
"Then,\dothyp{}\dothyp{}\dothyp{}\dothyp{}\dothyp{}\dothyp{}.no, no, I already don't want to say anything any more". }

\par{25. 第百章\dothyp{}\dothyp{}\dothyp{}\dothyp{}\dothyp{}\dothyp{}\dothyp{}\dothyp{}\dothyp{}\dothyp{}\dothyp{}\dothyp{}\dothyp{}\dothyp{}\dothyp{}\dothyp{}\dothyp{}\dothyp{}\dothyp{}\dothyp{}\dothyp{}\dothyp{}\dothyp{}\dothyp{}\dothyp{}\dothyp{}\dothyp{}\dothyp{}\dothyp{}\dothyp{}\dothyp{}\dothyp{}\dothyp{}\dothyp{}\dothyp{}\dothyp{}..567ページ \hfill\break
Chapter 100\dothyp{}\dothyp{}\dothyp{}\dothyp{}\dothyp{}\dothyp{}\dothyp{}\dothyp{}\dothyp{}\dothyp{}\dothyp{}\dothyp{}\dothyp{}\dothyp{}\dothyp{}\dothyp{}\dothyp{}\dothyp{}\dothyp{}\dothyp{}\dothyp{}\dothyp{}\dothyp{}\dothyp{}\dothyp{}\dothyp{}\dothyp{}..Page 567 }
      
\section{中線}
 
\par{ The 中線, ―, which is often longer, is essentially the Japanese version of a dash. As the exact 中線 seen so much in Japanese literature is often not available in typing, it is usually replaced with two dashes (ダッシュ). As far as terminology is concerned, 中線 = ダッシュ. This is only a typographical difference. }

\par{ The 中線, which is the official term for this punctuation mark, is often used just like a 点線 to show dodging\slash pause in speech. At the end, then, it would show a lingering effect. }

\par{26. 「それはね、――いや、もうやめしましょう。」 \hfill\break
"That,\dothyp{}\dothyp{}\dothyp{}.no, let's just quit already". }

\par{27. 「まあ、ほんとうにおかわいそうに――。」 \hfill\break
"Well, [subject] really is pitiful". }

\par{ In the place of 「」, it may separate text from the main body of the sentence, but it is more detached. }

\par{27. これではならない――といって起ちあがったのが彼であった。 \hfill\break
The one saying that \emph{this won't work }and got up was him. }

\par{ It may also be used to show spatial\slash temporal distance. In this sense, it can also function as から~まで. }

\par{28. 五分―十分―十五分 \hfill\break
Five minutes, ten minutes, fifteen minutes }

\par{27. 上野―新橋 \hfill\break
Ueno to Shinbashi }

\par{ It can also be used in the sense of すなわち. }

\par{28. 心持―心理学の用語によれば情緒とか気分とか状態意識とかいうのであるが、 \hfill\break
Feeling--according to psychological terminology is things like emotion, mood, and situation awareness, but\dothyp{}\dothyp{}\dothyp{} }

\par{It may set things aside for explanatory emphasis, mean "from\dothyp{}\dothyp{}\dothyp{}to\dothyp{}\dothyp{}\dothyp{}", and set numbers or names apart in Japanese addresses. It should be written vertically in vertical text. }

\par{\textbf{Caution Note }: This has nothing to do with the 長音符 used with かな to show vowel elongation. There is no correlation, and the 長音符 and 中線 are visibly different. ー (長音符) VS ― (中線・ダッシュ). If you have trouble seeing this, zoom in and notice the curve on the left-hand part of the 長音符. Also, the 長音符 has a different origin. The dash comes from Western orthography, but the 長音符 comes from the right-hand side of 引. The mark was used well before the 明治時代 (Meiji Period). }
      
\section{中点・中黒}
 
\par{ The 中黒 may \emph{juxtapose }similar items, act as a decimal, show the date, juncture foreign compound words, and separate titles, names, and positions. It may also be called 中ぽち, 中ぽつ, and 黒丸(くろまる). }

\par{29. 一四・七 \hfill\break
14.7 }

\par{30. ヒラリー・クリントン \hfill\break
Hillary Clinton }

\par{31. 平成一三・五・二六 \hfill\break
May 26, Heisei 13 (2001) }

\par{32. 安心と信頼・品質をあなたへ \hfill\break
Peace of mind and trust\slash quality to you }

\par{33. 全銀協会長・永易克典 \hfill\break
Japanese Banker's Association Head, Katsunori Nagayasu }

\par{34. 東京・大阪 \hfill\break
Tokyo-Osaka }

\par{35. 大阪・京都・神戸 \hfill\break
Osaka-Kyoto-Kobe }
      
\section{脇点 \& 脇線}
 
\par{ The 脇線 is a vertical line that directs the reader's attention a certain part of a phrase or word or the entirety of such. This comes from English influence and becomes useful when the punctuation mark traditionally used, which will be discussed next, cannot be typed easily. }

\par{36. そう考え られる 。 }

\begin{center}
 \textbf{The }\textbf{脇点(わきてん) }
\end{center}

\par{The 脇点 is a comma-like italic punctuation mark in Japanese vertical text that is used to direct some sort of special attention to what it's beside. It can also appear like a period. It may also go by the names 圏点 and 傍点. If you especially read material that is written vertically, you will see it everywhere. }
      
\section{スペース}
 
\par{  The space may be used in かな-only text to separate phrases to prevent confusion. It is seen before the beginning of a new paragraph, similar to a tab. Spaces are also left after non-Japanese punctuation marks such as !. Spaces may also be between names and things in an address or title. }

\par{38. スペースは かなのみでの ぶんに つかわれているよ。 \hfill\break
Spaces are only used in Kana-only sentences. }

\par{39. 何? 分かりません。 \hfill\break
What? I don't understand. }

\par{40. 大和銀行 大阪支店 \hfill\break
Yamato Bank, Osaka Branch }

\par{41. 藤原 恵子 \hfill\break
Keiko Fujiwara }
      
\section{Brackets}
 
\begin{center}
\textbf{The }\textbf{鉤括弧(かぎかっこう) and }\textbf{二重(ふたえ)括弧 }
\end{center}

\par{The 鉤括弧 is the true Japanese quotation mark and is seen as 「」 in horizontal texts and is rotated 90° in vertical texts. When quotations are within a quotations, you must use double quotation marks, 『』. }

\par{42. 「こんにちは」 }

\par{43. 国歌「君が代」 \hfill\break
National Anthem "Kimigayo" }

\par{44. 「さっきお出かけの途中、『なにかめずらしい本はないか。』とお立ち寄りくださいました。」  (尊敬語) \hfill\break
"Just a while ago while heading out, [he] stopped by and [asked] "are there any rare books?". }

\par{\textbf{Typological Note }: You may also come across〝 〟, which have no particular name and are limited to the use of just using quotations around important phrases. }

\par{ The ” “ (引用符) is typically used as quotation marks for short phrases--"so-called".They are also sometimes referred to as ノノカギ. }

\par{45. これは有名な”東京タワー”です。 \hfill\break
This is the so-called famous Tokyo Tower. }

\begin{center}
\textbf{The }\textbf{丸括弧(まるかっこ)・パーレン・小括弧(しょうかっこ) }
\end{center}

\par{() may be used to show readings of characters (although ふりがな is more prevalent). It may also show annotation. In editing, it is used to enclose instructions\slash signatures. in vertical writing, horizontal ones enclose numbers\slash letters for sections. }

\par{46. 広日本文典(明治三十年刊) \hfill\break
 \emph{Kounihon Grammar }(Published Meiji Period Year 30) }

\par{47. (その一)(第二回)(承前)(続き)(完)(終)(未完)(続く)(山田) \hfill\break
(The first)(second time)(continued from)(continuation)(completion)(end)(incomplete)(to continue)(Yamada) }

\par{48. (一) \hfill\break
(イ) \hfill\break
(a) }

\par{\textbf{Math Note }: In the realm of math, it may show coordinates in geometry, show the argument of a function in programming, show a math matrix, etc. }

\par{\textbf{Reading Note }: 丸括弧 is also read as まるがっこ. }

\begin{center}
\textbf{Other Brackets }
\end{center}

\begin{ltabulary}{|P|P|P|}
\hline 

Bracket & Name(s) & Description \\ \cline{1-3}

[] & 角括弧 & Its use in literature is almost nonexistent, but it is used in specialized fields in the same way as in English. For instance, it may close a mathematical interval, be seen in chemical formulas, in phonetic transcription, coding, etc. \\ \cline{1-3}

〈〉 & 山括弧 & Used in quantum physics for brackets. You may also see this mark doubled to show readings. \\ \cline{1-3}

〔〕 &  亀甲括弧・亀の子括弧 & Encases abbreviations of some sort and may also be doubled \\ \cline{1-3}

\{\} &  波括弧・ブレース・カーリブラケット・カール・中括弧 & Encloses words or lines considered to be together. \\ \cline{1-3}

【】 〖〗 & 隅付き括弧 & When white, used to mark 常用漢字 not taught by 6th grade. When black, it encases labeling and may mark 教育漢字. ALso called 隅付きパーレン, 太亀甲, and 黒亀甲. \\ \cline{1-3}

\end{ltabulary}
\hfill\break
      
\section{ツナギ}
 
\par{ The ツナギ is a =, resembling an equal sign, and is used in the way a hyphen is in English at the end of a line when a word continues onto the next page. This is only seen with Kana text, and even this has fallen out of use. The ツナギテン, ―, is used in the sense of から~まで. }

\par{49. }

\par{サルハ トウトウ ジブ= \hfill\break
ンガ ワルカッタト ア= \hfill\break
ヤマリマシタ。 \hfill\break
The money finally apologized and said that he was the one at fault. }
      
\section{波ダッシュ・波形・波線・にょろ}
 
\begin{itemize}
 
\item From\dothyp{}\dothyp{}\dothyp{}to\dothyp{}\dothyp{}\dothyp{}  
\item Separates a title from a subtitle on the same line.  
\item Replaces dashes  
\item Indicates origin.  
\item Can show long vowels emphatically for comic effect.  
\item Can suggest that music is playing:  
\end{itemize}

\begin{center}
\textbf{Examples } 
\end{center}

\par{50. あ〜〜〜 \hfill\break
Ahhhhhh }

\par{51. ♬ 〜 \hfill\break
Suggests music }

\par{52. フランス〜 \hfill\break
From France }

\par{53. 5時~7時 }

\par{54. 〜概要〜 \hfill\break
-Outline- }

\par{55. 東京〜大阪 \hfill\break
From Tokyo to Osaka }
      
\section{コロン \& セミコロン}
 
\par{ Colons are used in showing time when using Arabic numerals.  A semicolon can be used in place of a ナカテン. This is rare and seen in horizontal text }

\par{56. 3:46 }

\par{57. 静岡;浜松;名古屋;大阪,京都,神戸;岡山;広島を  }
      
\section{Ditto Marks}
 
\par{ Ditto marks are very important in 漢字 and かな orthography. Ditto mark in Japanese may be: }

\begin{ltabulary}{|P|P|P|P|P|P|}
\hline 

踊り字 & おどりじ & 繰り返し符号 & くりかえしきごう & 重ね字 & かさねじ \\ \cline{1-6}

送り字 & おくりじ & 揺すり字 & ゆすりじ & 重字 & じゅうじ \\ \cline{1-6}

重点 & じゅうてん & 畳字 & じょうじ &  &  \\ \cline{1-6}

\end{ltabulary}

\par{Along with this, there are quite a few ditto marks in Japanese. }

\begin{ltabulary}{|P|P|P|P|}
\hline 

々・仝 & どうのじてん & Doubles a 漢字 or compound & 時々、部分々々 \\ \cline{1-4}

ヽ & いちのじてん & Doubles a カタカナ; makes sound basic. & ハヽ \\ \cline{1-4}

ヾ & いちのじてん & Doubles a カタカナ and makes it voiced. & タヾ \\ \cline{1-4}

ゝ & いちのじてん & Doubles a ひらがな; makes sound basic. & つゝ \\ \cline{1-4}

ゞ & いちのじてん & Doubles a ひらがな and makes it voiced. & すゞき \\ \cline{1-4}

〻 & ふたのじてん & Doubles the reading of a preceding 漢字. & 各〻 \\ \cline{1-4}

〳〵 & くのじてん & Looks like くin vertical texts. Doubles previous phrase. \hfill\break
& 見る〳〵 \\ \cline{1-4}

〴〵 & くのじてん & Looks like ぐin vertical texts. Doubles previous phrase. \hfill\break
&  離 \hfill\break
れ \hfill\break
離 \hfill\break
れ \textrightarrow  離 \hfill\break
れ \hfill\break
〴 \hfill\break
〵 \\ \cline{1-4}

\end{ltabulary}

\par{\textbf{Spelling Note }: The second くのじてん voices the first sound of the phrase it doubles. }
      
\section{Miscellaneous Marks}
 
\begin{ltabulary}{|P|P|P|}
\hline 

 Mark & Name & Description \\ \cline{1-3}

〓 & ダブルハイフン・下駄記号 & Sometimes used instead of a ・ but typically symbolizes non-existence Kanji glyphs. \\ \cline{1-3}

※ & 米印・星 & Japanese version of an asterisk and is used to show an annotation. \\ \cline{1-3}

* & アステリスク & English-like version of the asterisk. \\ \cline{1-3}

〆 & しめ & Closes a letter. \\ \cline{1-3}

♪♫♬♩ & 音符 & Musical notes that suggest melody. \\ \cline{1-3}

\textrightarrow ←↑↓ & 矢印 & Whatever arrows do. \\ \cline{1-3}

♡ ♥ & ハートマーク & Love \\ \cline{1-3}

\end{ltabulary}
 
\begin{center}
 \textbf{The }\textbf{庵点(いおりてん) }
\end{center}
  
\par{ The 庵点 is used to mark the beginning of a song and is most known for showing the start of a person's line in Noh Theatre. }
    