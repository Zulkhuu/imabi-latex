    
\chapter*{~ために III}

\begin{center}
\begin{Large}
第???課: ~ために III: ~んがために \& ~がために 
\end{Large}
\end{center}
 
\par{ In this lesson, we will return our focuses to the phrase ~ために once more. This time, we will learn how this phrase can be preceded by the particle が instead of the usual particle の when after nominal phrases. In doing so, we will look at three different patterns. }

\begin{enumerate}

\item N + ~がために \hfill\break

\item V + ~んがために 
\item V + ~がために 
\end{enumerate}
 As hinted at by the introduction above, が is instrumental in having what precedes it be a nominal phrase. If it is already a noun, no further analysis is required. If the phrase is verbal, then that verb is deemed to be nominalized to allow ~ために to follow.       
\section{Archaic Use of が: 連体修飾格}
 
\par{ In the past, the particles が and の shared far more interchangeability than they do today. It was once possible to use が to mark an attribute to the same degree の does today. Nowadays, this is limited to particular words (mostly place names) and grammatical structures. As が\textquotesingle s use with ~ために in this fashion is a given, below are a handful of examples found in words still used commonly today. }

\par{i. ${\overset{\textnormal{わ}}{\text{我}}}$ が ${\overset{\textnormal{や}}{\text{家}}}$ \hfill\break
One\textquotesingle s home\slash family }

\par{ii. ${\overset{\textnormal{じゆう}}{\text{自由}}}$ が ${\overset{\textnormal{おか}}{\text{丘}}}$ \hfill\break
Jiyūgaoka }

\par{iii. ${\overset{\textnormal{おに}}{\text{鬼}}}$ ${\overset{\textnormal{が}}{\text{ヶ}}}$ ${\overset{\textnormal{しま}}{\text{島}}}$ \hfill\break
Onigashima (Island of Ogres) }

\par{\textbf{Terminology Note }: As indicated by the title of this section, this use of ga in linguistic terms is referred to as the 連体修飾格 (adnominal modifier case). }
      
\section{Noun + ~がために}
 
\par{ When ~がために directly follows a noun, it is essentially the same as ~のために. The only difference is that ~がために is literary and is never used in the spoken language. Other aspects of a sentence also tend to be old-fashioned when it is used. Depending on what kind of noun precedes it, ~がために can either mark an objective or a cause\slash reason. }

\par{1. ${\overset{\textnormal{た}}{\text{誰}}}$ が ${\overset{\textnormal{ため}}{\text{為}}}$ に ${\overset{\textnormal{かぶ}}{\text{株}}}$ は ${\overset{\textnormal{あ}}{\text{上}}}$ がる。 \hfill\break
For whom do stocks rise? }

\par{\textbf{Word Note }: 誰が為 should not be read as だれがため. 誰 was originally た, and this pronunciation is preserved in this expression. }

\par{2. ${\overset{\textnormal{なに}}{\text{何}}}$ がために ${\overset{\textnormal{い}}{\text{生}}}$ きているのかを ${\overset{\textnormal{し}}{\text{知}}}$ らずに ${\overset{\textnormal{もうもくてき}}{\text{盲目的}}}$ な ${\overset{\textnormal{ひおく}}{\text{日送}}}$ りをしていた。 \hfill\break
I was blindly living my days not knowing what I was living for. }

\par{3. ${\overset{\textnormal{とお}}{\text{遠}}}$ く ${\overset{\textnormal{けいし}}{\text{京師}}}$ を ${\overset{\textnormal{はな}}{\text{離}}}$ れていたので、 ${\overset{\textnormal{げんき}}{\text{玄機}}}$ がために ${\overset{\textnormal{ちから}}{\text{力}}}$ をいたすことができなかった。 \hfill\break
Since (he) was far away from the capital, he was unable to render assistance to Genki. \hfill\break
From ${\overset{\textnormal{ぎょげんき}}{\text{魚玄機}}}$ by ${\overset{\textnormal{もりおうがい}}{\text{森鷗外}}}$ . }

\par{\textbf{Word Note }: 京師 is an outdated term that can refer to old capitals of countries which use(d) Chinese characters. Thus, in context it can refer to cities such as Beijing, Xi\textquotesingle an, Seoul, Kyoto, etc. }

\par{4. ${\overset{\textnormal{かれ}}{\text{彼}}}$ は ${\overset{\textnormal{ちち}}{\text{父}}}$ の ${\overset{\textnormal{まず}}{\text{貧}}}$ しきがために、 ${\overset{\textnormal{じゅうぶん}}{\text{充分}}}$ なる ${\overset{\textnormal{きょういく}}{\text{教育}}}$ を ${\overset{\textnormal{う}}{\text{受}}}$ けなかった。 \hfill\break
Due to his father being poor, he did not receive a proper education. }

\par{\textbf{Grammar Notes }: \hfill\break
1. 貧しき is the original 連体形 of the adjective 貧しい. With ~がために following, the phrase 父の貧しき is essentially nominalized, and the use of the particle の is simply for marking the subject of a subordinate clause. \hfill\break
2. なる is the original 連体形 of the adjectival noun 充分だ. }

\par{5. ${\overset{\textnormal{な}}{\text{亡}}}$ き ${\overset{\textnormal{もの}}{\text{者}}}$ の ${\overset{\textnormal{ため}}{\text{為}}}$ の ${\overset{\textnormal{ぶつじ}}{\text{仏事}}}$ ではなく、まさに ${\overset{\textnormal{わ}}{\text{我}}}$ が ${\overset{\textnormal{ため}}{\text{為}}}$ の ${\overset{\textnormal{ぶつじ}}{\text{仏事}}}$ であった。 \hfill\break
Instead of it being a memorial service for the deceased individual, it was like it was a memorial service for myself. }

\par{6. ${\overset{\textnormal{とうしょ}}{\text{当初}}}$ の ${\overset{\textnormal{かんがえ}}{\text{考}}}$ には、 ${\overset{\textnormal{わ}}{\text{我}}}$ が ${\overset{\textnormal{にほんこく}}{\text{日本国}}}$ の ${\overset{\textnormal{ふぶんふめい}}{\text{不文不明}}}$ なるは ${\overset{\textnormal{きょういく}}{\text{教育}}}$ のあまねからざるがためのみ、 ${\overset{\textnormal{きょういく}}{\text{教育}}}$ さえ ${\overset{\textnormal{いきとど}}{\text{行届}}}$ けば ${\overset{\textnormal{ぶんめいふきょう}}{\text{文明富強}}}$ は ${\overset{\textnormal{ひ}}{\text{日}}}$ を ${\overset{\textnormal{き}}{\text{期}}}$ していたすべし、との ${\overset{\textnormal{きょうさん}}{\text{胸算}}}$ にてありしが、さて ${\overset{\textnormal{こんにち}}{\text{今日}}}$ にいたりて ${\overset{\textnormal{じっさい}}{\text{実際}}}$ の ${\overset{\textnormal{もよう}}{\text{模様}}}$ を ${\overset{\textnormal{み}}{\text{見}}}$ るに、 ${\overset{\textnormal{きょういく}}{\text{教育}}}$ はなかなかよく ${\overset{\textnormal{い}}{\text{行}}}$ きとどきて ${\overset{\textnormal{じ}}{\text{字}}}$ を ${\overset{\textnormal{し}}{\text{知}}}$ る ${\overset{\textnormal{もの}}{\text{者}}}$ も ${\overset{\textnormal{おお}}{\text{多}}}$ く、 ${\overset{\textnormal{いちげいいちのう}}{\text{一芸一能}}}$ に ${\overset{\textnormal{たっ}}{\text{達}}}$ したる ${\overset{\textnormal{せんもん}}{\text{専門}}}$ の ${\overset{\textnormal{がくしゃ}}{\text{学者}}}$ も ${\overset{\textnormal{すく}}{\text{少}}}$ なからずして、まずもって ${\overset{\textnormal{ぜんねん}}{\text{前年}}}$ の ${\overset{\textnormal{しょもう}}{\text{所望}}}$ はやや ${\overset{\textnormal{たっ}}{\text{達}}}$ したる ${\overset{\textnormal{すがた}}{\text{姿}}}$ なれども、これがために ${\overset{\textnormal{くに}}{\text{国}}}$ の ${\overset{\textnormal{ぶんめいふきょう}}{\text{文明富強}}}$ をいたしたるの ${\overset{\textnormal{しょうこ}}{\text{証拠}}}$ とては、 ${\overset{\textnormal{はなは}}{\text{甚}}}$ だ ${\overset{\textnormal{すく}}{\text{少}}}$ なきが ${\overset{\textnormal{ごと}}{\text{如}}}$ し。 \hfill\break
My initial thought was that the illiterate and ignorant state of our nation Japan was solely due to education not being widespread and that so long as education were scrupulous, we would someday attain a rich and powerful civilization; however, looking at our actual current state, though education has come along well with many having learned to read and there being far from few expert scholars gifted in their particular skills, and though this is the rather accomplished state of what I had previously desired in the first place, it all seems far too little as the evidence for our nation having accomplished forging a rich and powerful civilization. }

\par{\textbf{Grammar Notes }: \hfill\break
1. The particle は is seen directly after the 連体形 of the auxiliary verb ~なり—なる. In Classical grammar, it is possible for は to directly follow the 連体形 of a verbal phrase. The Modern Japanese equivalent of ~なるは would be ~であることは. Thus, the verbal phrase is nominalized, allowing ~がため(に) to follow. \hfill\break
2. The particle に is deleted in ~がために due to the presence of the particle のみ. \hfill\break
3. あまねからざる is the 連体形 of the negative form of the Classical adjective あまねし 【遍し/普し】, which did not survive into the modern language. It translates into Modern Japanese as 広く行き渡っている. \hfill\break
4. The auxiliary verb ~べし has various meanings, but in this passage, it is equivalent to できるだろう. \hfill\break
5. ~にてありしが = ~であったが. \hfill\break
6. The ~たる in 達したる and いたしたる is the 連体形 of ~たり, the original 終止形 of the auxiliary verb ~た. \hfill\break
7. なれども = であるけれども. \hfill\break
8. Modern Japanese equivalents of ~が如し include ~と同じだ, ~のようだ, and ~の通りだ. Grammatically speaking, the が used in this expression is the same as in ~がため(に). \hfill\break
9. The Modern Japanese equivalent of ~とて is ~と言っては. }
      
\section{Verb + ~んがために}
 
\par{ In the grammar pattern ~んがために, ~ん is the contracted form of the Classical auxiliary verb ~む, which is the predecessor of the auxiliary verbs of volition ~よう and ~う. Here, ~ん is in the 連体形, which means the verbal phrase it is a part of functions as a noun. As such, ~がために attaches normally. Although unnatural, a literal translation into Modern Japanese of this pattern would be ~ようとすることのために. }

\par{ As is the case with ~よう and ~う, ~ん follows the 未然形 of verbs. To visualize how this pattern connects to each kind of verb, below is a conjugation chart. }

\begin{ltabulary}{|P|P|P|}
\hline 

\slash eru\slash - \emph{Ichidan }Verb & 得る + んがために \textrightarrow  & 得んがために (to gain) \\ \cline{1-3}

\slash iru\slash - \emph{Ichidan }Verb & 見る + んがために \textrightarrow  & 見んがために (to look) \\ \cline{1-3}

\slash u\slash - \emph{Godan }Verb & 救う + んがために \textrightarrow  & 救わんがために (to save) \\ \cline{1-3}

\slash ku\slash - \emph{Godan }Verb & 築く + んがために \textrightarrow  & 築かんがために (to build) \\ \cline{1-3}

\slash gu\slash - \emph{Godan }Verb & 稼ぐ + んがために \textrightarrow  & 稼がんがために (to earn) \\ \cline{1-3}

\slash su\slash - \emph{Godan }Verb & 示す + んがために \textrightarrow  & 示さんがために (to show) \\ \cline{1-3}

\slash tsu\slash - \emph{Godan }Verb & 保つ + んがために \textrightarrow  & 保たんがために (to maintain) \\ \cline{1-3}

\slash nu\slash - \emph{Godan }Verb & 死ぬ + んがために \textrightarrow  & 死なんために (to die) \\ \cline{1-3}

\slash mu\slash - \emph{Godan }Verb & 読む + んがために \textrightarrow  & 読まんがために (to read) \\ \cline{1-3}

\slash ru\slash - \emph{Godan }Verb & 成る + んがために \textrightarrow  & 成らんがために (to become) \\ \cline{1-3}

 \emph{Suru }(Verb) & する + んがために \textrightarrow  & せんがために (to do) \\ \cline{1-3}

 \emph{Kuru }& くる + んがために \textrightarrow  & こんがために (to come) \\ \cline{1-3}

\end{ltabulary}

\par{\textbf{Conjugation Note }: 来んがために is essentially not used. In fact, only one search result appears for it in Google. Ex. 20 is an adaptation of this search result. It is worth noting that although this is by no means ungrammatical or necessarily unnatural to say, it is perplexing how this verb in particular is not seemingly used despite it being necessarily mentioned in conjugation notes for this very pattern—both for native learner and foreign learner use. }

\par{ The basic sentence pattern used for this expression is “A + Vんがために + B.” Whereas “A + Vために + B” either expresses a certain reason\slash objective A to bring about a certain outcome B,” this pattern emphasizes that B is the only means by which objective A can be accomplished. Thus, it is highly nuanced. }

\par{ However, it is most often not the case that the speaker would have thought long and hard about the situation to come to the conclusion that is being expressed. Rather, the situation being described is very much “last resort” in feel, and the circumstance the experiencer is facing is mostly a very difficult and\slash or abnormal dilemma. }

\par{ This, though, doesn\textquotesingle t mean that the situation overall must be dire or necessarily pressing. There is an almost spontaneous realization on the part of the experiencer that A must be done for B to happen. In other words, the action described in A is usually impulsive and not indicative of planning. A can also be described as a breakthrough action that disrupts the status quo. }

\par{ As was the case for the uses of ~ために we learned early on in our studies, the particle に can be seen omitted as is often the case for this particle in the written language. This expression can also be seen as ~んがための when used to modify another nominal phrase (Ex. 7). }

\par{7. ${\overset{\textnormal{じぶん}}{\text{自分}}}$ の ${\overset{\textnormal{りえき}}{\text{利益}}}$ を ${\overset{\textnormal{え}}{\text{得}}}$ んがための ${\overset{\textnormal{はつげん}}{\text{発言}}}$ では、 ${\overset{\textnormal{ひと}}{\text{人}}}$ の ${\overset{\textnormal{こころ}}{\text{心}}}$ を ${\overset{\textnormal{うご}}{\text{動}}}$ かせない。 \hfill\break
You can't move people's hearts with words only for one's personal interest. }

\par{8. ${\overset{\textnormal{かれ}}{\text{彼}}}$ は ${\overset{\textnormal{ゆめ}}{\text{夢}}}$ を ${\overset{\textnormal{じつげん}}{\text{実現}}}$ させんがため、 ${\overset{\textnormal{じょうきょう}}{\text{上京}}}$ した。 \hfill\break
He moved to Tokyo to fulfill his dreams. }

\par{\textbf{Sentence Note }: The act of fulfilling his dreams is not what\textquotesingle s impulsive or last minute about the situation. Rather, it is implied that there must have been a sudden breakthrough on the will of the subject to finally break the status quo and pack his things and head for Tokyo. }

\par{9. ${\overset{\textnormal{まやくはんたい}}{\text{麻薬反対}}}$ という ${\overset{\textnormal{おも}}{\text{思}}}$ いを ${\overset{\textnormal{しめ}}{\text{示}}}$ さんがために、 ${\overset{\textnormal{かれ}}{\text{彼}}}$ はあらゆる ${\overset{\textnormal{ほうほう}}{\text{方法}}}$ を ${\overset{\textnormal{こころ}}{\text{試}}}$ みた。 \hfill\break
He tried out every method to show his thoughts for being against drugs. }

\par{10. ${\overset{\textnormal{わ}}{\text{我}}}$ が ${\overset{\textnormal{こ}}{\text{子}}}$ の ${\overset{\textnormal{むざい}}{\text{無罪}}}$ を ${\overset{\textnormal{しょうめい}}{\text{証明}}}$ せんがために、 ${\overset{\textnormal{ひっし}}{\text{必死}}}$ で ${\overset{\textnormal{しょうこ}}{\text{証拠}}}$ を ${\overset{\textnormal{さが}}{\text{探}}}$ そうと ${\overset{\textnormal{ちか}}{\text{誓}}}$ った。 \hfill\break
I vowed to search desperately for proof to prove his own child's innocence. }

\par{11. ${\overset{\textnormal{かれ}}{\text{彼}}}$ は ${\overset{\textnormal{こども}}{\text{子供}}}$ を ${\overset{\textnormal{すく}}{\text{救}}}$ わんがために、 ${\overset{\textnormal{いのち}}{\text{命}}}$ を ${\overset{\textnormal{お}}{\text{落}}}$ とした。 \hfill\break
He lost his life trying to save the child. }

\par{12. ${\overset{\textnormal{かれ}}{\text{彼}}}$ は、 ${\overset{\textnormal{かんけん}}{\text{漢検}}}$ ${\overset{\textnormal{いっ}}{\text{1}}}$ ${\overset{\textnormal{きゅう}}{\text{級}}}$ に ${\overset{\textnormal{ごうかく}}{\text{合格}}}$ せんがために、 ${\overset{\textnormal{まいにちひっし}}{\text{毎日必死}}}$ で ${\overset{\textnormal{がんば}}{\text{頑張}}}$ った。 \hfill\break
He did his best every day desperately to pass the Kanken Level 1. }

\par{13. ${\overset{\textnormal{こっかい}}{\text{国会}}}$ で ${\overset{\textnormal{ほうあん}}{\text{法案}}}$ を ${\overset{\textnormal{とお}}{\text{通}}}$ さんがため、 ${\overset{\textnormal{しゅしょう}}{\text{首相}}}$ は ${\overset{\textnormal{ねまわ}}{\text{根回}}}$ し ${\overset{\textnormal{こうさく}}{\text{工作}}}$ を ${\overset{\textnormal{かいし}}{\text{開始}}}$ した。 \hfill\break
The Prime Minister began to pull strings to get the bill through the Diet. }

\par{14. ${\overset{\textnormal{しゃちょう}}{\text{社長}}}$ にならんがために、 ${\overset{\textnormal{いろいろ}}{\text{色々}}}$ な ${\overset{\textnormal{うらこうさく}}{\text{裏工作}}}$ をしていたといわれる。 \hfill\break
It is said that (he) did all sorts of maneuvering behind the scenes to become the company president. }

\par{15. ただ ${\overset{\textnormal{いちりゅうだいがく}}{\text{一流大学}}}$ に ${\overset{\textnormal{はい}}{\text{入}}}$ らんがために ${\overset{\textnormal{べんきょう}}{\text{勉強}}}$ している ${\overset{\textnormal{ひと}}{\text{人}}}$ が ${\overset{\textnormal{おお}}{\text{多}}}$ い。 \hfill\break
There are a lot of people that are studying to just get into a first-rate university. }

\par{16. ${\overset{\textnormal{みんしゅう}}{\text{民衆}}}$ を ${\overset{\textnormal{きょういく}}{\text{教育}}}$ せんがために、 ${\overset{\textnormal{おお}}{\text{多}}}$ くの ${\overset{\textnormal{がっこう}}{\text{学校}}}$ は ${\overset{\textnormal{た}}{\text{建}}}$ てられたのである。 \hfill\break
Many schools were built to educate the masses. }

\par{17. 支配者が権力を保たんがために、強硬手段を取った。 \hfill\break
The ruler took strong measures to maintain power. }

\par{18. ${\overset{\textnormal{われわれ}}{\text{我々}}}$ は、 ${\overset{\textnormal{せかい}}{\text{世界}}}$ の ${\overset{\textnormal{しょうらい}}{\text{将来}}}$ に ${\overset{\textnormal{へいわ}}{\text{平和}}}$ と ${\overset{\textnormal{はんえい}}{\text{繁栄}}}$ を ${\overset{\textnormal{きず}}{\text{築}}}$ かんがために ${\overset{\textnormal{どりょく}}{\text{努力}}}$ している。 \hfill\break
We are endeavoring to build peace and prosperity to the future of the world. }

\par{19. ある ${\overset{\textnormal{とき}}{\text{時}}}$ は、その ${\overset{\textnormal{こどもたち}}{\text{子供達}}}$ の ${\overset{\textnormal{も}}{\text{持}}}$ つ ${\overset{\textnormal{けってん}}{\text{欠点}}}$ を ${\overset{\textnormal{ただ}}{\text{正}}}$ しく ${\overset{\textnormal{なお}}{\text{直}}}$ さんがために、また ${\overset{\textnormal{た}}{\text{足}}}$ らざるものを ${\overset{\textnormal{おぎな}}{\text{補}}}$ わんがために ${\overset{\textnormal{はなし}}{\text{話}}}$ の ${\overset{\textnormal{だいざい}}{\text{題材}}}$ を ${\overset{\textnormal{えら}}{\text{選}}}$ ぶこともあれば、その ${\overset{\textnormal{こころもち}}{\text{心持}}}$ で ${\overset{\textnormal{かた}}{\text{語}}}$ られるでありましょう。 \hfill\break
Other times, if there were moments in which to choose the theme of the story to properly correct the shortcomings of the children or to supplement what they lacked, I would surely be able to narrate with that disposition. \hfill\break
From ${\overset{\textnormal{どうわ}}{\text{童話}}}$ を ${\overset{\textnormal{か}}{\text{書}}}$ く ${\overset{\textnormal{とき}}{\text{時}}}$ の ${\overset{\textnormal{こころ}}{\text{心}}}$  ${\overset{\textnormal{by}}{\text{by}}}$  ${\overset{\textnormal{おがわみめい}}{\text{小川未明}}}$ . }

\par{20. ${\overset{\textnormal{まる}}{\text{〇}}}$ の ${\overset{\textnormal{いんじ}}{\text{韻字}}}$ を ${\overset{\textnormal{も}}{\text{持}}}$ って ${\overset{\textnormal{こ}}{\text{来}}}$ んがために、 ${\overset{\textnormal{むり}}{\text{無理}}}$ にこじつけたもので ${\overset{\textnormal{しな}}{\text{品}}}$ を ${\overset{\textnormal{お}}{\text{落}}}$ としたようです。 \hfill\break
It seems I lowered the quality (of the poem) by unreasonably straining it to bring about the rhyming word \#. }

\par{\textbf{Word Note }: 韻字 are characters placed at the end of a stanza in Chinese prose to add rhyme. }

\par{21. ${\overset{\textnormal{かんぶん}}{\text{漢文}}}$ における「 ${\overset{\textnormal{くとうほう}}{\text{句読法}}}$ 」は、 ${\overset{\textnormal{ちゅうごくぶん}}{\text{中国文}}}$ を ${\overset{\textnormal{にほんごしき}}{\text{日本語式}}}$ に ${\overset{\textnormal{よ}}{\text{読}}}$ まんがために ${\overset{\textnormal{にほん}}{\text{日本}}}$ で ${\overset{\textnormal{こうあん}}{\text{考案}}}$ されたものである。 \hfill\break
The “punctuation rules” found in \emph{Kanbun }were devised in Japan to read Chinese works in a Japanese style. }

\par{\textbf{Culture Note }: \emph{Kanbun }is Classical Chinese literature primarily written by Japanese people. }

\par{22. ${\overset{\textnormal{せいかつひ}}{\text{生活費}}}$ を ${\overset{\textnormal{かせ}}{\text{稼}}}$ がんがために ${\overset{\textnormal{いっしょうけんめい}}{\text{一生懸命}}}$ (に) ${\overset{\textnormal{はたら}}{\text{働}}}$ いている。 \hfill\break
I am working my utmost to earn my living expenses. }

\par{23. ${\overset{\textnormal{すなわ}}{\text{即}}}$ ち、イエス ${\overset{\textnormal{さま}}{\text{様}}}$ は ${\overset{\textnormal{し}}{\text{死}}}$ なんがために ${\overset{\textnormal{し}}{\text{死}}}$ に ${\overset{\textnormal{う}}{\text{得}}}$ べき ${\overset{\textnormal{じょうたい}}{\text{状態}}}$ を ${\overset{\textnormal{と}}{\text{取}}}$ り、 ${\overset{\textnormal{われ}}{\text{我}}}$ らと ${\overset{\textnormal{おな}}{\text{同}}}$ じ ${\overset{\textnormal{ちにく}}{\text{血肉}}}$ を ${\overset{\textnormal{そな}}{\text{具}}}$ えたのである。 \hfill\break
In other words, Jesus took a state in which death was possible to die, therefore possessing flesh and blood just like us. }
 
\par{\textbf{Sentence Note }: It is not implied that Jesus took human form impulsively to die. However, atonement of humanity\textquotesingle s sin via death was a predicate to Jesus\textquotesingle  descent, and taking a form in which it was possible to die was the only means available, which is why 死なんがために is used. }

\par{24. ${\overset{\textnormal{かみさま}}{\text{神様}}}$ は、 ${\overset{\textnormal{おのれ}}{\text{己}}}$ を ${\overset{\textnormal{み}}{\text{見}}}$ んがために ${\overset{\textnormal{あい}}{\text{愛}}}$ を ${\overset{\textnormal{そうぞう}}{\text{創造}}}$ せざるを ${\overset{\textnormal{え}}{\text{得}}}$ なかった。 \hfill\break
God had no choice but to create love to assess Himself. }
 
\par{\textbf{Sentence Note }: Similarly to Ex. 23, it is not necessarily the case that God\textquotesingle s action in B is a result of A being an impulsive drive. Although this may be true, A should be interpreted more so as B being the only result that could fulfill objective A. }
      
\section{Verb + ~がために}
 
\par{ ~がために may also be seen directly after verbal expressions in either the non-past or past tense. This is also old-fashioned and almost entirely limited to the written language. Its purpose is to express an atypical reason\slash cause that brings about an atypical result. It intrinsically does not imply either the reason\slash cause or the result is a good or bad thing, but it is very subjective in nature due to が functioning as an intensifier. This is because of the exhaustive-listing function of が. In other words, this is a very literary yet emphatic version of the cause marking ~ために. }

\par{ Grammaticality speaking, the verb phrase that precedes が is essentially nominalized. Although the 連体形 of a verb alone is enough to modify ~ために, the presence が renders it as a noun. }

\par{25. ${\overset{\textnormal{ゆうし}}{\text{融資}}}$ を ${\overset{\textnormal{う}}{\text{受}}}$ けてしまったがために ${\overset{\textnormal{さ}}{\text{差}}}$ し ${\overset{\textnormal{お}}{\text{押}}}$ さえが ${\overset{\textnormal{もくぜん}}{\text{目前}}}$ に ${\overset{\textnormal{せま}}{\text{迫}}}$ ってきた。 \hfill\break
The seizure (of my assets) came close at hand due to the financing I had received. }

\par{26. ${\overset{\textnormal{かのじょ}}{\text{彼女}}}$ は、 ${\overset{\textnormal{ふともも}}{\text{太腿}}}$ に ${\overset{\textnormal{しぼう}}{\text{脂肪}}}$ が ${\overset{\textnormal{つ}}{\text{付}}}$ いてしまったがために、 ${\overset{\textnormal{き}}{\text{着}}}$ たいものが ${\overset{\textnormal{き}}{\text{着}}}$ られないという ${\overset{\textnormal{せつ}}{\text{切}}}$ ない ${\overset{\textnormal{けいけん}}{\text{経験}}}$ を ${\overset{\textnormal{あじ}}{\text{味}}}$ わった。 \hfill\break
She tasted the painful experience of not being able to wear what she wanted because she had put on fat in her thighs. }

\par{27. ${\overset{\textnormal{ひと}}{\text{人}}}$ に ${\overset{\textnormal{な}}{\text{馴}}}$ れてしまったがために、 ${\overset{\textnormal{しゅうい}}{\text{周囲}}}$ を ${\overset{\textnormal{けいかい}}{\text{警戒}}}$ しなくなって ${\overset{\textnormal{ほしょくしゃ}}{\text{捕食者}}}$ に ${\overset{\textnormal{た}}{\text{食}}}$ べられてしまったり、 ${\overset{\textnormal{はな}}{\text{放}}}$ しても ${\overset{\textnormal{かえ}}{\text{帰}}}$ ってきてしまったりするという ${\overset{\textnormal{とり}}{\text{鳥}}}$ も ${\overset{\textnormal{すく}}{\text{少}}}$ なくない。 \hfill\break
There are also far from few birds who get eaten by predators because they are no longer wary of their surroundings or who end up returning even after being released because they became too tame with people. }

\par{28. ${\overset{\textnormal{かれ}}{\text{彼}}}$ は ${\overset{\textnormal{さつじんはん}}{\text{殺人犯}}}$ の ${\overset{\textnormal{むすこ}}{\text{息子}}}$ に ${\overset{\textnormal{う}}{\text{生}}}$ まれてしまったがために ${\overset{\textnormal{ひかげ}}{\text{日陰}}}$ の ${\overset{\textnormal{じんせい}}{\text{人生}}}$ を ${\overset{\textnormal{あゆ}}{\text{歩}}}$ まざるを ${\overset{\textnormal{え}}{\text{得}}}$ なかった。 \hfill\break
He had no choice but to go through life in the shadows due to being born as the son of a murderer. }

\par{ ~がために can also be seen as an emphatic yet literary version of the objective marking ~ために. This usage is less common than the one above, so much so that it is sometimes perceived as a mistake. Using it broadly as a replacement of ~ために, however, would be a misuse of this pattern. Ultimately, the tone given off by this use of ~がために is somewhat “matter-of-fact” but in a very composed and authoritative manner. }

\par{29. この ${\overset{\textnormal{うみ}}{\text{海}}}$ が ${\overset{\textnormal{み}}{\text{見}}}$ たいがために、 ${\overset{\textnormal{まいあさお}}{\text{毎朝起}}}$ きる ${\overset{\textnormal{き}}{\text{気}}}$ がするのだった。 \hfill\break
(She) felt the urge to wake up in the morning to see the ocean. \hfill\break
From 光の雨 by ${\overset{\textnormal{たてまつわへい}}{\text{立松和平}}}$ . }

\par{30. ${\overset{\textnormal{わたし}}{\text{私}}}$ が、 ${\overset{\textnormal{なん}}{\text{何}}}$ か ${\overset{\textnormal{こどもたち}}{\text{子供達}}}$ に ${\overset{\textnormal{むか}}{\text{向}}}$ ってお ${\overset{\textnormal{はなし}}{\text{話}}}$ をするとしたら、まず、それがどんな ${\overset{\textnormal{こどもたち}}{\text{子供達}}}$ であるかを ${\overset{\textnormal{し}}{\text{知}}}$ ろうとするでしょう。 ${\overset{\textnormal{つぎ}}{\text{次}}}$ に、いくつ ${\overset{\textnormal{くらい}}{\text{位}}}$ であるかを ${\overset{\textnormal{み}}{\text{見}}}$ ます。それによって ${\overset{\textnormal{はなし}}{\text{話}}}$ を ${\overset{\textnormal{えら}}{\text{選}}}$ び、よく ${\overset{\textnormal{わか}}{\text{分}}}$ るようにしたいがためです。 \hfill\break
If I were to narrate something to children, I would first try to know what kind of children they are. Then, I would see how old there are. I\textquotesingle d choose my story based on that so that they can understand well. \hfill\break
From ${\overset{\textnormal{どうわ}}{\text{童話}}}$ を ${\overset{\textnormal{か}}{\text{書}}}$ く ${\overset{\textnormal{とき}}{\text{時}}}$ の ${\overset{\textnormal{こころ}}{\text{心}}}$  ${\overset{\textnormal{by}}{\text{by}}}$  ${\overset{\textnormal{おがわみめい}}{\text{小川未明}}}$ . }

\par{31. ${\overset{\textnormal{ゆえ}}{\text{故}}}$ に ${\overset{\textnormal{こ}}{\text{子}}}$ を ${\overset{\textnormal{おし}}{\text{教}}}$ えるがためには ${\overset{\textnormal{ろう}}{\text{労}}}$ を ${\overset{\textnormal{はばか}}{\text{憚}}}$ るべからず、 ${\overset{\textnormal{ざい}}{\text{財}}}$ を ${\overset{\textnormal{いと}}{\text{愛}}}$ しむべからず。 \hfill\break
Therefore, one mustn\textquotesingle t have scruples about the work or be attached to ones riches in order to teach a child. \hfill\break
From ${\overset{\textnormal{きょういく}}{\text{教育}}}$ の ${\overset{\textnormal{こと}}{\text{事}}}$ by ${\overset{\textnormal{ふくざわゆきち}}{\text{福沢諭吉}}}$ . }

\par{32. その ${\overset{\textnormal{せいねん}}{\text{青年}}}$ は毎日、「 ${\overset{\textnormal{にほんご}}{\text{日本語}}}$ の ${\overset{\textnormal{ほん}}{\text{本}}}$ 」を ${\overset{\textnormal{よ}}{\text{読}}}$ みたいがために ${\overset{\textnormal{あおぞらぶんこ}}{\text{青空文庫}}}$ を ${\overset{\textnormal{りよう}}{\text{利用}}}$ している。 \hfill\break
That young man uses Aozora Bunko every day to read “Japanese books.” }

\par{\textbf{Sentence Note }: The ~がために used in this sentence can be seen as marking both a reason and an objective. }
    