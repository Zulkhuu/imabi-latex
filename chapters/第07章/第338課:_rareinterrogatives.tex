    
\chapter{Interrogatives IV}

\begin{center}
\begin{Large}
第338課: Interrogatives IV: Rare Words 
\end{Large}
\end{center}
 
\par{ Interrogatives themselves are not so hard to understand or use. What you may not know is that there are even more less used interrogatives in Japan aside from the simple 誰, 何, いつ, どこ, and 何故. }
      
\section{Rare Interrogatives}
 
\begin{center}
\textbf{なんらか } 
\end{center}

\par{${\overset{\textnormal{なん}}{\text{何}}}$ らか, also ${\overset{\textnormal{なにがし}}{\text{何某}}}$ か, means "any" or "some". Both of these words are quantity words. So, they describe indefinite amounts. 何らか is the most common. }

\par{1. ${\overset{\textnormal{なにがし}}{\text{何某}}}$ かの ${\overset{\textnormal{かね}}{\text{金}}}$ \hfill\break
 A certain sum of money }

\par{2. 何らかの ${\overset{\textnormal{えいきょう}}{\text{影響}}}$ が出る。 \hfill\break
To have some effect come out. }

\par{3. あなたの ${\overset{\textnormal{もう}}{\text{申}}}$ し立てに何らかの ${\overset{\textnormal{こんきょ}}{\text{根拠}}}$ がありますか。 \hfill\break
Do you have any evidence for your statement? }

\par{4. 何らかの ${\overset{\textnormal{てん}}{\text{点}}}$ で \hfill\break
In some way or other }

\par{ なにがし (某・何某) is a somewhat old-fashioned indefinite pronoun meaning "a certain". It equates in meaning to the Sino-Japanese prefix 某 read as ぼう. 某‐ can be used after proper nouns, place and time words. It is very similar to ある・或る. ある can be used to refer to something when one has no certainty of the actual matter\slash thing. When one uses 某‐, though, one knows exactly what the subject (matter) is. You just think it is best not to say it. It also sometimes gives the impression one is hiding information, especially when you overuse it. It goes well with more stiff 書き言葉. However, there are exceptions. }

\begin{ltabulary}{|P|P|P|P|}
\hline 

ある国 \textrightarrow  某国 & ある日 \textrightarrow  某日 & あるところ \textrightarrow  某所 & ある雑誌 \textrightarrow  某誌 \\ \cline{1-4}

ある会社 \textrightarrow  某会社 & ある先生 \textrightarrow  某教師 & ある省 \textrightarrow  某省 & ある月 \textrightarrow  某月 \\ \cline{1-4}

あるテレビ局 \textrightarrow  某テレビ局 & ある人 \textrightarrow  某氏 & あるタレント \textrightarrow  某タレント & ある高校 \textrightarrow  某高校 \\ \cline{1-4}

\end{ltabulary}

\par{5. どこぞのなにがし \hfill\break
A certain someone from somewhere }

\par{ The ら in 何ら is the same as in 彼ら. So, this is essentially the plural form of what. It is used as a more emphatic way of saying なにも. It may be used adverbially and nominally. It is not as common as なにも. }

\par{6. 何ら ${\overset{\textnormal{ふあん}}{\text{不安}}}$ はないよ。 \hfill\break
I have no fear. }

\par{7. 何ら ${\overset{\textnormal{うたが}}{\text{疑}}}$ いはない。 \hfill\break
I have no doubt. }

\par{8. 何らの利益もないよ。 \hfill\break
There's no profit\slash benefit whatsoever. }

\begin{center}
 \textbf{いずれ }
\end{center}

\par{ いずれ, an alternative to どれ, is best translated as either "either", "both", or "sooner or later". These usages are distinguished easily from each other. If used as an adverb, it means, "sooner or later". If used with a particle such as も, it's like どれも to mean "either\slash both". }

\par{9. ${\overset{\textnormal{いず}}{\text{何}}}$ れ分かるよ。 \hfill\break
You'll understand sooner or later! }

\par{10. いずれ ${\overset{\textnormal{あらた}}{\text{改}}}$ めて ${\overset{\textnormal{うかが}}{\text{伺}}}$ います。 \hfill\break
I will come again in the near future. }

\par{11. いずれ ${\overset{\textnormal{おと}}{\text{劣}}}$ らぬ (Set phrase) \hfill\break
Equally competent }

\par{\textbf{Interrogative Note }: Do not get the impression that these are all of the possible interrogatives in Japanese. You will discover more, though more likely rarer, interrogatives as you continue to study Japanese. The examples below are just to show you what more is out there. You don't have to remember this stuff. }

\begin{center}
\textbf{いずくんぞ }
\end{center}

\par{ いずくんぞ, formally and formerly as いづくんぞ, means どうして. The following characters have been used to write it, but only the last is ever seen today: 悪・安・寧・焉. }

\par{12. ${\overset{\textnormal{いま}}{\text{未}}}$ だ ${\overset{\textnormal{せい}}{\text{生}}}$ を知らず、焉んぞ死を知らんや。 \hfill\break
How would you know death if you've yet to know life? }

\par{13. ${\overset{\textnormal{こうし}}{\text{孔子}}}$ が『未だ人に ${\overset{\textnormal{つか}}{\text{事}}}$ うること ${\overset{\textnormal{あた}}{\text{能}}}$ わず、焉んぞ ${\overset{\textnormal{よ}}{\text{能}}}$ く ${\overset{\textnormal{き}}{\text{鬼}}}$ に事えん。』と ${\overset{\textnormal{い}}{\text{曰}}}$ わく。 \hfill\break
Confucius said, "How could you possibly serve the divine well if you can't even serve man well?" }

\par{14. ${\overset{\textnormal{えんじゃく}}{\text{燕雀}}}$ 安んぞ ${\overset{\textnormal{こうこく}}{\text{鴻鵠}}}$ の ${\overset{\textnormal{こころざし}}{\text{志}}}$ を知らんや。 \hfill\break
How is that sparrows and swallows have greater will power than large birds? }

\par{15. ${\overset{\textnormal{にわとり}}{\text{鶏}}}$ を ${\overset{\textnormal{さ}}{\text{割}}}$ くに焉んぞ ${\overset{\textnormal{ぎゅうとう}}{\text{牛刀}}}$ を用いん。 \hfill\break
You don't need a butcher's knife to kill a chicken. }

\par{16. ${\overset{\textnormal{おうこう}}{\text{王候}}}$ ${\overset{\textnormal{しょうしょう}}{\text{将相}}}$ 寧んぞ ${\overset{\textnormal{しゅ}}{\text{種}}}$ あらんや。 \hfill\break
Why would nobles and shogun have seeds? (Any person can achieve through effort and luck) }

\par{17. いずくんぞ知らん。(Very old-fashioned) \hfill\break
How would you know? }

\begin{center}
 \textbf{那辺・奈辺 }
\end{center}

\par{18. こ ${\overset{\textnormal{こ}}{\text{ヽ}}}$ に ${\overset{\textnormal{おい}}{\text{於}}}$ て ${\overset{\textnormal{こ}}{\text{此}}}$ の問題の重大なることや、その ${\overset{\textnormal{かいけつ}}{\text{解決}}}$ の \textbf{${\overset{\textnormal{なへん}}{\text{那邊}}}$ }にあるかを ${\overset{\textnormal{さけ}}{\text{叫}}}$ んで、 ${\overset{\textnormal{こくみん}}{\text{國民}}}$ の注意を ${\overset{\textnormal{かんき}}{\text{喚起}}}$ する ${\overset{\textnormal{せんかくしゃ}}{\text{先覺者}}}$ の ${\overset{\textnormal{ふんぱつどりょく}}{\text{奮發努力}}}$ が ${\overset{\textnormal{はなは}}{\text{甚}}}$ だ ${\overset{\textnormal{のぞ}}{\text{望}}}$ ましい ${\overset{\textnormal{しだい}}{\text{次第}}}$ であります。 \hfill\break
This is dependent on the much needed strenuous efforts of a pioneer who will alert the public's attention and strongly ask what the most important thing to this problem is and where the solution is. }

\par{From ${\overset{\textnormal{こくごこくじ}}{\text{國語國字}}}$ 問題 by ${\overset{\textnormal{ふくながきょうすけ}}{\text{福永恭助}}}$ . }

\par{\textbf{Sentence Notes }: }

\par{1. 那辺 is a rare interrogative equivalent to どこ・どの辺. }

\par{2. The example was written before simplification took place. Notice how many characters look different. }

\begin{center}
 \textbf{如何 }
\end{center}

\par{ 如何 is sometimes read as いかん. This is a contraction of いかに. Both words are equivalent in meaning to どんなに. いかん is seen in formal situations, but its usage has continued to go down. いかに is used even less, but it survives in set phrases and in common usage in things like いかにも, which is used to show agreement in the same way as 全く and なるほど. }

\par{19. 人間、いかに生くべきか。 \hfill\break
How should man live? }

\par{20. この件は如何致しましょうか。 \hfill\break
What should we do with this case? }

\begin{center}
\textbf{Interesting Dialect Phrases }
\end{center}

\par{ If we include dialectical interrogative phrases, the number of such phrases in Japanese goes up a lot. Some phrases are localized to very small areas, but they're still interesting to look at. Below are phrases from 福島弁. Of these なして, also happens to be used in many other places in Japan. The rest are quite localized to the 東北 Region. }

\par{21. え、なして? \hfill\break
Wait, why? }

\par{22. なして泣いてるの? \hfill\break
Why are you crying? }

\par{ The following phrases and translations are taken from the Wikipedia page on Fukushima Dialect. }

\begin{itemize}
 
\item なじょうだ、なじょした、なじょんした? \textrightarrow  どうした、どんなだ (県) 
\begin{itemize}
 
\item なじょうする、なじょする、なじょんする?\textrightarrow どうする?  
\item なじょんかして\textrightarrow どうにかして  
\item なじょ、なじょう?(う、の部分を伸ばす感じで発音する)\textrightarrow  どう  
\item なじょして、なじょんした?、なじょんして? \textrightarrow  どうした?どうやって?  
\item なじょな \textrightarrow  どんな (北、中、南、会)  
\item なじょうだが \textrightarrow  どうだか (浜、北、南、会)  
\item なじょったい \textrightarrow  どうだかね (浜、北、中)  
\item なじょーにしっぺい \textrightarrow  どうしたらよかろう (北、中、南)  
\item なじょーも \textrightarrow  なんとも (会) 
\end{itemize}

\end{itemize}
     