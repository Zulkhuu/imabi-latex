    
\chapter{連体詞}

\begin{center}
\begin{Large}
第303課: 連体詞 
\end{Large}
\end{center}
 
\par{ 連体詞 are simply adjectival words that are stuck in the 連体形 or are modifier phrases treated as a single unit. }
 
\par{There are also 連体詞 that come from two defunct classes of 形容動詞: ナル形容動詞 and タル形容動詞, which still possess functional 連用形, に and と respectively. They can be viewed as adjectives with just dormant bases. In some expressions, their former past as completely functional adjectives can be more obvious. These words, though, will be discussed later on. }
      
\section{連体詞}
 
\par{  Below is a chart of most of the 連体詞 in Japanese followed by example sentences. This list is not exhaustive, and the defunct adjectival classes are discussed later in this lesson. }

\begin{ltabulary}{|P|P|P|P|}
\hline 

Big & 大きな & Every & あらゆる \\ \cline{1-4}

Funny & おかしな & Mere & ほんの \\ \cline{1-4}

Small & 小さな & Eternal \hfill\break
& 長・永の \hfill\break
\\ \cline{1-4}

New & 新たな & A certain & 或る \\ \cline{1-4}

Considerable & 大した & Pretending & 素知らぬ \\ \cline{1-4}

Unthinkable & とんだ & Unexpected & ひょんな \\ \cline{1-4}

Remaining & 残んの & This & こな \\ \cline{1-4}

Odd & 異な \hfill\break
& Reckless & 大それた \\ \cline{1-4}

Considerate & 心有る & Improper \hfill\break
& あらぬ \hfill\break
\\ \cline{1-4}

Following & 明くる & Unknown & 見知らぬ \\ \cline{1-4}

Untiring & 飽くなき & Various & 色んな \\ \cline{1-4}

Last & 去んぬる & Only (thing) & ものの \\ \cline{1-4}

Reasonable & 無理からぬ & Useless & 言われぬ \\ \cline{1-4}

Past & 往にし & Strange & 奇しき \\ \cline{1-4}

\end{ltabulary}

\par{\textbf{Usage Notes }: }

\par{1. All adjectival こそあど words are 連体詞. Many 連体詞 are created from classical grammatical items. }

\par{2. Many of these are uncommon and rather archaic. }

\par{3. 無理からぬ comes from the からぬ from よからぬ, the classical negative attribute form of よい, attached to 無理. }

\begin{center}
 \textbf{Examples }
\end{center}

\par{1. 色んな文書を ${\overset{\textnormal{そろ}}{\text{揃}}}$ える。 \hfill\break
To arrange various documents. }

\par{2. 大きな ${\overset{\textnormal{きれつ}}{\text{亀裂}}}$ が入っている。 \hfill\break
There is a large crack (in the wall). }

\par{3. 無理からぬ話ですね。 \hfill\break
It's a reasonable story, isn't it? }

\par{4. ${\overset{\textnormal{く}}{\text{奇}}}$ しき出来事 \hfill\break
A strange incident }

\par{5. ${\overset{\textnormal{あ}}{\text{明}}}$ くる日 \hfill\break
The following day }

\par{6. ものの見事に \hfill\break
With great success }

\par{7. ${\overset{\textnormal{そ}}{\text{素}}}$ 知らぬ顔 \hfill\break
A pretending face }

\par{8. あらゆる努力をして新たな計画を ${\overset{\textnormal{ね}}{\text{練}}}$ った。 \hfill\break
With every effort he created a new plan. }

\par{9. ひょんなことから明くる日の出来事が大きな問題になった。 \hfill\break
From sheer chance the following day became a serious problem. }

\par{10a. ${\overset{\textnormal{なが}}{\text{永}}}$ の別れとなった。 \hfill\break
10b. 永遠の別れとなった。(More spoken and also more common) \hfill\break
It became an eternal separation. }

\par{11. ${\overset{\textnormal{あ}}{\text{飽}}}$ くなき欲望がありますね。 \hfill\break
You sure have a persistent desire don't you? }

\par{12. ${\overset{\textnormal{のこ}}{\text{残}}}$ んのお月を見よ。  (雅語) \hfill\break
Look at the remaining moon. }

\par{13. ウォルマートにはありとあらゆる品物が ${\overset{\textnormal{なら}}{\text{並}}}$ んでいます。 \hfill\break
Everything that you can think of is at Walmart. }

\par{14. 心ある市民の ${\overset{\textnormal{いか}}{\text{怒}}}$ りを買うなんて ${\overset{\textnormal{あくぎょう}}{\text{悪行}}}$ だ。 \hfill\break
Buying the hatred of sensible citizens is an evil act. }

\par{15. 新たな ${\overset{\textnormal{ぼうけん}}{\text{冒険}}}$ が始まる。 \hfill\break
A new adventure begins. }

\par{16. 彼はあらゆる努力を尽くした。 \hfill\break
He made every effort. }

\par{17. ぼんやりとあらぬ方を見やる。 \hfill\break
To vaguely gaze at an improper way. }
面前でのちょうしょうは侮辱の最たるものだ。 \hfill\break
Scorn in one's presence is the extremity of insult. \hfill\break
    