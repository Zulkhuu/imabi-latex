    
\chapter{Native Suffixes IV}

\begin{center}
\begin{Large}
第329課: Native Suffixes IV: Verbal  
\end{Large}
\end{center}
 
\par{ Native verbal suffix endings attach to nouns to create new verbal phrases. Some originate from independent verbs while others are purely constructive endings. }

\begin{ltabulary}{|P|P|P|}
\hline 

~がかる & ~がる & ~ぐむ \\ \cline{1-3}

~さびる & ~染みる & ~立つ \\ \cline{1-3}

~つく & ~付く & ~付ける \\ \cline{1-3}

~ばむ & ~張る & ~びる \\ \cline{1-3}

~ぶる & ~めかす & ~めく \\ \cline{1-3}

~やく & ~やぐ &  \\ \cline{1-3}

\end{ltabulary}
      
\section{Verbal Suffixes}
 
\par{~ぐむ: something is appearing which had been within something. So, it is used with many physical attributes. }
 
\par{${\overset{\textnormal{}}{\text{1.柳}}}$ の木々が\{ ${\overset{\textnormal{め}}{\text{芽}}}$ ぐみ・芽吹き\}、降り積もった雪が解けてきた。 \hfill\break
The willows and all the trees began to sprout as the snow piles melted. }
 
\par{${\overset{\textnormal{}}{\text{2. 氷}}}$ ${\overset{\textnormal{と}}{\text{解}}}$ け ${\overset{\textnormal{さ}}{\text{去}}}$ り ${\overset{\textnormal{あし}}{\text{葦}}}$ は ${\overset{\textnormal{つの}}{\text{角}}}$ ぐむ。 \hfill\break
Ice melts as the reeds sprout forth. }

\par{3. ${\overset{\textnormal{あざわら}}{\text{嘲笑}}}$ われて涙ぐんだ。 \hfill\break
I was moved to tears from being scorned at. }
 
\par{4. 君の目が涙ぐんでいるようです。 \hfill\break
Your eyes are watery. }
 
\par{~さびる shows that something behaves like something. It is very rare. }

\par{5. ${\overset{\textnormal{かみ・かん}}{\text{神}}}$ さびた ${\overset{\textnormal{やしろ}}{\text{社}}}$ \hfill\break
 A temple of age and divinity }

\par{6. ${\overset{\textnormal{おきな}}{\text{翁}}}$ さびる \hfill\break
Behaving as an old man }
 
\par{~染(じ)みる: often in a negative fashion, either shows that something is stained with something or something is quite felt as such. }
 
\par{7. 汗染みた。 \hfill\break
It got a sweat stain. }
 
\par{8. 子供じいた真似をするな。 \hfill\break
Don't act childish. }
 
\par{9. 彼の顔は何となく標本じみて見えた。 \hfill\break
His face for some reason or another appeared to be like a specimen. }
 
\par{10. 年寄り染みた \hfill\break
Characteristic of aging }
 
\par{11. 油染みた \hfill\break
Oil-stained }
 
\par{12. 二階が ${\overset{\textnormal{しんさつしつ}}{\text{診察室}}}$ に待合室、下は ${\overset{\textnormal{ぎこうしつ}}{\text{技工室}}}$ のあの ${\overset{\textnormal{かなづち}}{\text{金槌}}}$ や、 ${\overset{\textnormal{きん}}{\text{金}}}$ をのばすローラーや、ガラスの炎や、エンジンなんかの、神経質な工場じみた音に加えて、小さい子供が四人もあり、おまけに ${\overset{\textnormal{でんしゃどおり}}{\text{電車通}}}$ だった。 \hfill\break
To the second floor there was an examination room and a waiting room, and below, in addition to the iron hammer, metal roller, glass fire, engine, and what not neurotic factory-like noises, there were four small kids, and to make matters worse, it was a street with a tram. \hfill\break
From 死体紹介人 by 川端康成. }
 
\par{~(だ)つ: characteristics of something in terms of condition and state. }

\par{13. ${\overset{\textnormal{やおちょう}}{\text{八百長}}}$ レースだと観衆が殺気立った。 \hfill\break
The spectators seethed in rage that it was a rigged race. }
 
\par{${\overset{\textnormal{}}{\text{14a. 鳥肌}}}$ 立つ。 \hfill\break
14b. 鳥肌が立つ。(Natural) \hfill\break
To have goosebumps. }
 
\par{~つく: attaches to onomatopoeic words to show a certain kind of condition. }
 
\par{15. 外国でまごついたことがある? \hfill\break
Have you gotten confused at a foreign country? }
 
\par{16. むかついてくる。 \hfill\break
To get ticked off. }
 
\par{17. 画面がちらつく。 \hfill\break
For the screen to flicker. }
 
\par{18. 俺の決心がぐらついちまった。(Used by somewhat older men) \hfill\break
My determination shuddered. }
 
\par{19. 小雨がまだぱらついてる。 \hfill\break
Rain is still sprinkling. }
 
\par{~付(づ)く: shows inseparability or that you cannot part from a certain thing or condition. }
 
\par{20. インターネットづいてんのよ。(砕けた女性語) \hfill\break
You're really hooked to the internet. }
 
\par{21. 色づいてくのが分かったんだ。(Casual) \hfill\break
I knew things were going to start turning. }
 
\par{22. 調子付いている。 \hfill\break
Things are going well. }

\par{23. ${\overset{\textnormal{おじけ}}{\text{怖気}}}$ づいて逃げると!? \hfill\break
Running away for fear!? }
 
\par{24. 木が根付く。 \hfill\break
For trees to take root. }

\par{~付(づ)ける grants a certain condition. }
 
\par{25. 納税者に義務付ける。 \hfill\break
To obligate the taxpayers. }
 
\par{26. もう一度やってみろと力づけました。 \hfill\break
I encouraged him to try it again once more. }
 
\par{27. 我々は基準として位置づけております。 \hfill\break
We are positioning it as standard. }
 
\par{~ばむ shows that a certain condition appears or holds a certain quality. }
 
\par{28. この黄ばんだ紙は本当に古くなってきたな。 \hfill\break
This yellow paper has really become old, hasn't it? }

\par{29. ${\overset{\textnormal{けしき}}{\text{気色}}}$ ばむ。 \hfill\break
To miff one's anger. }
 
\par{30. むしむしして汗ばむ。 \hfill\break
To feel warm and sweaty. }
 
\par{31. 俺の ${\overset{\textnormal{わき}}{\text{腋}}}$ の下がじっとり汗染みた。 \hfill\break
My armpits are damp and sweaty. }
 
\par{~張る: persists something and shows that a tendency is even more remarkable. }
 
\par{32. あまり ${\overset{\textnormal{よくば}}{\text{欲張}}}$ らないで。 \hfill\break
Don't be so greedy. }
 
\par{33. そんなに ${\overset{\textnormal{かくば}}{\text{角張}}}$ っては ${\overset{\textnormal{きゅうくつ}}{\text{窮屈}}}$ だよ。 \hfill\break
Being so ceremonious is quite constraining. }
 
\par{34. 四角張らずに \hfill\break
Without formality }
 
\par{35. 格式ばる。 \hfill\break
To conform to formalities. }
 
\par{36. あまり気張るな。 \hfill\break
Don't strain yourself. }
 
\par{~びる: is attached to nouns or 形容詞 to show that something is in some sort of condition. }
 
\par{37. 古びた。 \hfill\break
Worn-out. }
 
\par{38. 大人びて見える。 \hfill\break
To look adult-like. }
 
\par{39. ありゃ ${\overset{\textnormal{ひな}}{\text{鄙}}}$ びた ${\overset{\textnormal{とうじ}}{\text{湯治}}}$ ${\overset{\textnormal{ば}}{\text{場}}}$ や。(Dialect; old people) \hfill\break
That over there's a beat-down health spa resort. }
 
\par{~ぶる: attaches to nouns and the stems of adjectives. "Pretending\slash acting as such". }
 
\par{${\overset{\textnormal{}}{\text{40. 偉}}}$ ぶる。 \hfill\break
To swagger. }
 
\par{41. もったいぶるなよ。 \hfill\break
Stop making it such a big deal. }
 
\par{42. 学者ぶる。 \hfill\break
To pretend to be a scholar. }
 
\par{~めかす: is the verbal form of ~めかしい. It shows that you make something out to be\dothyp{}\dothyp{}\dothyp{} }
 
\par{${\overset{\textnormal{}}{\text{43. 冗談}}}$ めかす。 \hfill\break
To be (half) in joke. }

\par{45. ${\overset{\textnormal{きらめ}}{\text{煌}}}$ かせたあの夜空。 \hfill\break
The brightened night sky. }

\par{46. ${\overset{\textnormal{ほの}}{\text{仄}}}$ めかしただけだ。 \hfill\break
I only suggested. }
 
\par{~やく: attaches to onomatopoeic phrases to show that one takes on a certain behavior. }
 
\par{47. 耳元で ${\overset{\textnormal{ささや}}{\text{囁}}}$ く。 \hfill\break
To whisper in the ears. }
 
\par{48. 僕は彼にそっと ${\overset{\textnormal{つぶや}}{\text{呟}}}$ いた。 \hfill\break
I gently muttered it to him. }
 
\par{~やぐ: attaches to nouns and the stems of adjectives to show that something takes on or behaves to fit a certain appearance. }
 
\par{${\overset{\textnormal{}}{\text{49. 華}}}$ やいだ雰囲気。 \hfill\break
A cheerful atmosphere. }
 
\par{50. 若やいだ声ですね。 \hfill\break
You have a young sounding voice, don't you? }
 
\par{${\overset{\textnormal{}}{\text{51. 物々}}}$ しく ${\overset{\textnormal{}}{\text{鮮}}}$ やいで \hfill\break
Brilliantly and ostentatiously }
    