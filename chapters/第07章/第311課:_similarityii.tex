    
\chapter{Similarity II}

\begin{center}
\begin{Large}
第311課: Similarity II: ~ごとく, ~さながら, \& ~しかず 
\end{Large}
\end{center}
 
\par{ This lesson introduces more difficult\slash less common phrases for showing similarity. }
      
\section{~ごとく}
 
\par{ You may also seldom see ${\overset{\textnormal{ごと}}{\text{如}}}$ . ごと is 如くwhen used adverbially, 如き when used adjectivally in the 連体形, and 如し when used adjectivally in the 終止形. ごと is largely replaced by 如くだ. This is old-fashioned and often used for poetic reasons or in set expressions. }

\par{1. 月の如く \hfill\break
\textbf{Like }the moon }

\par{2. 天使の如く \hfill\break
\textbf{Like }an angel }

\par{3. 人生はあたかも春の夜の夢の如しということではないと思うんだ。 \hfill\break
I don't think that human life is \textbf{as if }it was like a dream of a spring night. }

\par{4. \textbf{我々ごとき }が坐っていてアンケートと本で調べた(だけでもないが)点に致命的弱味がある。 \hfill\break
There is a fatal weak point in \textbf{the like of us }who sat and researched questionnaires and books (though that isn't all we did). \hfill\break
From 大分方言 by 高田一彦. }

\par{\textbf{Grammar Note }: Notice how ごとき is used. It is not connected to 我々 with の, and it is treated as a nominal phrase. The reason for this is that in older forms of Japanese, the 連体形 of adjectives could take on case particles such as が to be used as nominal phrases. The result here is the very formal, written expression for "the like of us". }

\par{5. 売れ ${\overset{\textnormal{ゆ}}{\text{行}}}$ き ${\overset{\textnormal{はなは}}{\text{甚}}}$ だ ${\overset{\textnormal{ふ}}{\text{振}}}$ るわざるがごとし。(Classical) \hfill\break
It \textbf{appears }that sales are really not flourishing. }

\par{\textbf{Word Notes }: 売れ行き comes from the combination of 売れる (to be sold) and 行く. \hfill\break
~ざるがごとし is Classical Japanese for ~ないようだ. }
      
\section{~さながら}
 
\par{ This is an infrequently used word that means "just like". It is rather literary and is not really used in the spoken language. Even its Kanji spelling is not particularly known. However, it is a topic on the JLPT NI, so it is something that you need to take note of. }

\par{ When さながら  attaches to a noun, it shows that something is just as something else. In this sense, it is similar to 同然. In other contexts, it is an adverb meaning まるで. This, again, would not be something used in 話し言葉. }

\par{6. 本当の戦争さながら \hfill\break
Just as a real war }

\par{7. あの人の姿はさながら死骸のようだった。 \hfill\break
That person's figure was just like that of a dead body. }

\par{8. 王 ${\overset{\textnormal{さなが}}{\text{宛}}}$ らの ${\overset{\textnormal{}}{\text{大歓迎}}}$ \hfill\break
A royal welcome }
      
\section{~しかず}
 
\par{ しかず, frequently spelled as 如かず, comes from an old verb no longer used anymore to show that something is the same thing as and or does not exceed a certain point. This, though, is not in a negative sense but in a good meaning. A similar phrase is ~には及ばない. Though 如かず is still used, it's normally used in set phrases. }

\par{9. 百聞は一見に如かず。 \hfill\break
Seeing is believing. }

\par{10. 尽く書を信ずれば則ち書なきに如かず。 \hfill\break
If you believe what is in a book fully without hesitation, it is as if there is no book at all. }

\par{\textbf{Set Phrase Note }: This is a proverb from Mencius lecturing people that if they believed everything in the Book of History, 書経, without having any critical judgment on it, it would be best if the book didn't exist at all. }
    