    
\chapter{Sino-Japanese Suffixes I}

\begin{center}
\begin{Large}
第330課: Sino-Japanese Suffixes I 
\end{Large}
\end{center}
 
\par{  Not all of Sino-Japanese suffixes will be mentioned in this lesson. Although this may look like it's a ton of information, you have actually seen the majority of these endings already. }
      
\section{The Suffixes}
 
\par{-員(いん) is an employee of some sort. }

\par{1. 参議院議員 \hfill\break
Member of the House of Councilors }

\par{2. 審判員 \hfill\break
Umpire }

\par{-下(か) = "under the(\dothyp{}\dothyp{}\dothyp{}of)" }

\par{3. 独裁者が首都を戒厳令下に置いた。 \hfill\break
The dictator put the capital under martial law. }

\par{${\overset{\textnormal{}}{\text{4. 香港}}}$ は ${\overset{\textnormal{}}{\text{中国}}}$ の ${\overset{\textnormal{}}{\text{支配下}}}$ にある。 \hfill\break
Hong Kong is under the control of China. }
-外(がい) = Outside. This is used in more specialized vocabulary. 
\par{5. そちらは私の権限外でございますす。 \hfill\break
That is outside my authority. }

\par{6. 普通外 \hfill\break
Out of the ordinary }

\par{-街(がい) = Street }

\par{7. ロサンゼルスの日本人街 \hfill\break
Japanese street in Los Angeles }

\par{-加減(かげん) = Tendency }

\par{8. 湯加減はどう? \hfill\break
How's the bath water? }

\par{9. 酒のいい飲み加減 \hfill\break
Sake at the best temperature }

\par{-格好(かっこう) = Around\dothyp{}\dothyp{}\dothyp{}years old }

\par{10. 30格好の指導者 \hfill\break
A leader around 30 years old }

\par{-間(かん) = Between. It is used with location words like cities and nations. }

\par{11. 東京・大阪間の本線はどちらでしょうか。 \hfill\break
Where is the main line between Tokyo and Osaka? }

\par{12. 国際間の緊張を緩和することは不可能だ。 \hfill\break
Lessening of international tensions is impossible. }

\par{13. 林間学校 \hfill\break
Open-air school }

\par{-感(かん) = Sense }

\par{14. 劣等感との和解は簡単にはかなわない。 \hfill\break
A compromise with inferiority is simply unbearable. }

\par{15. あの男は清潔感がある。 \hfill\break
That man has a sense of cleanliness. }

\par{-巻(かん) = Volume (of a book) }

\par{16. ブリタニカの第10巻 \hfill\break
The 10th volume of the Britannica }

\par{-観(かん) = Point of view }

\par{17. 日本人観 \hfill\break
The Japanese point of view }

\par{-癌(がん) Cancer }

\par{18. 乳癌 \hfill\break
Breast cancer }

\par{-儀 = In regards to\slash as for (with pronouns, people's names, or nouns associated with them). }

\par{19. 私儀(わたくしぎ) \hfill\break
As for me }

\par{\textbf{Word Note }: This is a very humble expression that refers to oneself used in the written language. }

\par{-気(ぎ) = -like nature }

\par{20. 商売気)のない実業家は存在するのでしょう。 \hfill\break
I wonder a non-profit motivated businessman exists? }

\par{-級(きゅう) = Level; grade; degree; class }

\par{21. 第一級の博士 \hfill\break
A first class professor }

\par{-強(きょう) = A little more than }

\par{22. 昨日の余震は震度五強だった。 \hfill\break
Yesterday's aftershocks were a little more than 5 (degrees magnitude). }

\par{-狂(きょう) = Fan(atic) }

\par{23. 大の相撲狂じゃない。 \hfill\break
I am no big sumo fanatic. }

\par{-県(けん) = Prefecture (Excluding Hokkaido, Kyoto, Osaka, and Tokyo) }

\par{24. 兵庫県は日本の43県の一つである。 \hfill\break
Hyogo Prefecture is one of Japan's 43 ken-prefectures. }

\par{-圏(けん) = Bloc (sphere) }

\par{25. 木星の大気圏の外観を見ると大赤班が確認できる。 \hfill\break
If you look at the surface of the atmosphere of Jupiter, you can confirm a Great Red Spot. }

\par{26. 関東一円が暴風圏に入る見込みです。 \hfill\break
It is projected that all of the Kanto district will enter the storm zone. }

\par{\textbf{Word Note }: This 一円 means "the whole area", not "one yen". }

\par{-券(けん) = Ticket }

\begin{ltabulary}{|P|P|P|P|P|P|}
\hline 

搭乗券 & Boarding pass & 入場券 & Admission ticket & 航空券 & Plane ticket \\ \cline{1-6}

乗車券 & Railroad ticket & 会員券 & Membership card & 割引券 & Coupon \\ \cline{1-6}

\end{ltabulary}

\par{-権(けん) = Right }

\par{27. 選挙権 \hfill\break
The right to vote }

\par{-腱(けん) = Tendon }

\par{28. アキレス腱は足にある。 \hfill\break
The Achilles' tendon is in the leg. }
-家(け) = Name of a family \hfill\break
29. 徳川家は有名なのです。 \hfill\break
The Tokugawa Family is famous.  -系(けい) = Lineage; system \hfill\break
30. アフリカ系移民によってアメリカの文化はよほど豊かになりましたよね。 \hfill\break
By African immigrants American culture has greatly become enriched,hasn't it? \hfill\break
\hfill\break
31. 太陽系は9つの惑星と彗星などが含まれる。 \hfill\break
The solar system includes things such as the nine planets and smaller satellites.  32. 小型の動物は循環系を持たないものが多い。 \hfill\break
There are a lot of small sized animals that do not possess a circulatory system.  -御 = Adds a sense of reverence. This is old-fashioned and usually limited to the written language and or purposely semi-archaic settings.  33. 姪御さん \hfill\break
Someone's niece 
\par{34. 殿御は女性から男性に対して敬っていう語である。 \hfill\break
Gentleman is a phrase said to respect males from women. }

\par{\textbf{Word Note }: 殿御 is old-fashioned. }

\par{35. 「親御」とは他人の父母の敬称でございます。 \hfill\break
"Oyago" is a respectful title of someone's father and mother. }

\par{-湖(こ) = Lake }

\par{36. 琵琶湖は日本で最大の面積と貯水量を誇り様々な種類が生息している。 \hfill\break
Lake Biwa boasts the largest area and water capacity in Japan and various species inhabit it. }

\par{-公(こう) = Title that either shows respect, affection, or scorn. }

\begin{ltabulary}{|P|P|P|P|P|P|P|P|}
\hline 

菅公 & Great Mr. Kan & 熊公 & Mr. Bear & 先公 & Teacher (derogatory) & ポリ公 & Cop (derogatory) \\ \cline{1-8}

ワン公 & Mr. Doggy & エテ公 & Mr. Monkey &  &  &  &  \\ \cline{1-8}

\end{ltabulary}

\par{-工(こう) = Person of a particular trade }
37. 父は配管工をしています。 \hfill\break
My dad is a plumber. 
\par{-号(ごう)= Number\slash edition }

\par{38. 台風2号は那覇を襲ったということです。 \hfill\break
Typhoon No. 2 hit Naha. }

\par{39. コロコロの6月号は来週に発行されるはずです。 \hfill\break
The June issue of CoroCoro is supposed to be issued next week. }

\par{-座(ざ) = Constellation; name of theater }

\par{40. 歌舞伎座は2013年に再び開かれる予定です。 \hfill\break
The Kabukiza Theater will be reopened in 2013. }

\par{41. こぐま座は小北斗七星とも呼ばれる。 \hfill\break
Ursa Minor is also called the Little Dipper. }

\par{-山(さん・ざん) = Mount; mountain }

\par{42. 富士山は活火山(かっかざん)である。 \hfill\break
Mount Fuji is an active volcano. }

\par{-酸(さん) = Acid }

\par{43. 蟻酸を扱う際に注意しなければいけないのだ。 \hfill\break
You must be careful when you are dealing with formic acid. }

\par{-三昧(ざんまい) ="luxury" and is used in idiomatic expressions. It is usually used in positive contexts. }

\par{44. ${\overset{\textnormal{ぜいたく}}{\text{贅沢}}}$ 三昧 \hfill\break
Living in luxury }
 
\par{45. 読書三昧でした。 \hfill\break
It was a reader's paradise. }
 
\par{${\overset{\textnormal{}}{\text{46a. 刃物}}}$ 三昧に及ぶ。(不自然) \hfill\break
46b. 刃物沙汰に及ぶ。(自然) \hfill\break
To amount to a knife fight. }

\par{\textbf{Phrase Note }: The first option is rather strange. This is likely due to a mistranslation from a Western text into Japanese. However, this phrase may be seen used creatively in contexts where there is a luxury to be had with an array of 刃物. }

\par{-市(し) = City }

\par{47. 釜石市が津波で破壊されてしまった。 \hfill\break
Kamaishi City was destroyed by a tsunami. }

\par{-誌(し) = Magazine }

\par{48. 週刊誌 \hfill\break
Weekly magazine }

\par{-死(し) = Death }

\par{49. 自然死 \hfill\break
Natural death }

\par{-師(し) = Skilled worker; Reverend }

\par{50. 美容師に髪を結ってもらう。 \hfill\break
She has her hairdresser do her hair. }

\par{-視(し) = View }

\par{51. 採算を度外視する。 \hfill\break
To neglect profit. }

\par{-弱(じゃく) = A little less than. The opposite of 強. }

\par{52. 地震は六弱の強さでした。 \hfill\break
The earthquake was a strength of a little less than magnitude 6 (on the Shindo scale). }

\par{-衆(しゅう) = The masses }

\par{53. 旦那衆 \hfill\break
Gents }

\par{-相(しょう) = Minister }

\par{54. 外(務)相 \hfill\break
Foreign minister }

\par{-賞(しょう) = Prize }

\par{55. ノーベル賞を与える。 \hfill\break
To give the Nobel Prize. }

\par{-省(しょう) = Ministry }

\par{56. 厚生労働省は日本の行政機関であります。 \hfill\break
The Ministry of Health, Labor, and Welfare is a governmental body of Japan. }

\par{-症(しょう) = Disease }

\par{57. 感染症 \hfill\break
Infectious disease }

\par{-商(しょう) = Dealer }

\par{58. 美術商 \hfill\break
Art dealer }

\par{-州(しゅう) = State; province }

\par{59. アラスカ(州)は寒すぎるじゃん? \hfill\break
Isn't Alaska too cold? }

\par{-宗(しゅう) = Sect }

\par{60. 天台宗は大乗仏教の宗派の一つで、法華経に基づいた。 \hfill\break
The Tendai sect is a faction of Mahayana Buddhism and is based off of the Lotus Sutra. }
    