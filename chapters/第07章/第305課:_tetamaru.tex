    
\chapter{~てたまる}

\begin{center}
\begin{Large}
第305課: ~てたまる 
\end{Large}
\end{center}
 
\par{ The 五段 verb ${\overset{\textnormal{たま}}{\text{堪}}}$ る means "to endure". In this lesson, we will see it used after the て form to show the speaker's utter intolerance for something. }
      
\section{~てたまる}
 
\par{ ~て堪る is followed most frequently by either か or ものか to show that one would never let something happen. The speaker is showing a strong resolution of not having something be in such a state\slash condition. As you can imagine, this isn't used in polite speech. }

\par{ This is synonymous with 断じて~など(し)ない. 断じて ≒ 決して. ~ものか shows a strong sense of negation, and it can also be seen in more casual speech as ~もんか. It literally creates a strong negative rhetorical question. }

\par{\textbf{Examples }}

\par{1. ${\overset{\textnormal{らくだい}}{\text{落第}}}$ などして ${\overset{\textnormal{}}{\text{堪}}}$ るものか。 \hfill\break
I would never fail! }

\par{${\overset{\textnormal{}}{\text{2. 負}}}$ けてたまるか。 \hfill\break
I would never be defeated! }

\par{${\overset{\textnormal{}}{\text{3. 死}}}$ んでたまるものか。 \hfill\break
I would never die!  }

\par{4a. そんなことを断じて許せない。 \hfill\break
4b. そんなことを許してたまるものか。 \hfill\break
I would never allow this. }

\par{\textbf{Sentence Note }: 4a shows the speaker's incapability of allowing such. 4b shows a strong negative emotional reaction of allowing such, and the latter as with all of the other sentences with ~てたまる(もの)か  invoke emotions such as anger. So, although the sentence with 断じて could be reworded to be even polite, that's impossible with ~てたまるものか. It's a phrase that should be limited to one's ウチ. }

\par{ \textbf{~てたまらない }}

\par{ たまらない can be translated as "to die for", "ache", "kill for", etc. You literally can't stand it, and as you see below, this is often used in a positive sense. When it's not, it should be obvious by what's being used. For instance, in the first example, the speakers really want to go, and you get the sense that they just can't stand waiting any longer. This phrase is sometimes seen in polite speech as Ex. 4 illustrates. }

\par{ \textbf{Examples }}

\par{${\overset{\textnormal{}}{\text{5. 私たち}}}$ は ${\overset{\textnormal{}}{\text{行}}}$ きたくてたまらないのです。 \hfill\break
We are eager to go. }

\par{6. あまりの ${\overset{\textnormal{あつ}}{\text{暑}}}$ さに ${\overset{\textnormal{ひとやす}}{\text{一休}}}$ みしたくてたまらない。 \hfill\break
I'm dying for a break from the heat. }

\par{${\overset{\textnormal{}}{\text{7. 一杯飲}}}$ みたくてたまらなかった。 \hfill\break
I couldn't stand not having a drink. }

\par{8. ${\overset{\textnormal{こんや}}{\text{今夜}}}$ の ${\overset{\textnormal{}}{\text{彼女}}}$ はたまらなく ${\overset{\textnormal{}}{\text{美}}}$ しい。 \hfill\break
She is irresistibly beautiful tonight. }

\par{9. ${\overset{\textnormal{はら}}{\text{腹}}}$ が ${\overset{\textnormal{}}{\text{立}}}$ ってたまらない。 \hfill\break
To get furiously mad. }

\par{10. ${\overset{\textnormal{りょこう}}{\text{旅行}}}$ したくてたまらないよ。 \hfill\break
I'm aching to travel. }

\par{11. 負けて悔やしくてたまらない。 \hfill\break
I can't stand but regret losing. }

\par{12. 負けてたまらない。 \hfill\break
I can't stand losing. }

\par{\textbf{Grammar Note }: Although the subject of the last sentence could be different, both sentences show the speaker's\slash subject's not standing the fact of losing. The positive\slash negative associations of this pattern clearly fall on semantic lines of what you're using in combination with it. }
    