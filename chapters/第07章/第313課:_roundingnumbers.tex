    
\chapter{Numbers VIII}

\begin{center}
\begin{Large}
第313課: Numbers VIII: 概数 \& Reading the 九九 
\end{Large}
\end{center}
 
\par{ Have you ever wondered how you would say things like “I\textquotesingle ve been gone for three, four days”? These round number expressions are called 概数, and this lesson will teach you how to read them. So, it\textquotesingle s important to mention this separately. In reading out, though, you would usually hear #から# or #か# for many situations. }

\par{ After learning about how to read 概数, we will learn how to read the multiplication table in Japanese, which is called 九九. }
      
\section{概数}
 
\par{ The chart below shows how to create 概数 in Japanese, but then the majority of this lesson will be about the difficulty of reading these phrases when counters are put into the mix. }

\begin{ltabulary}{|P|P|P|P|P|P|}
\hline 

One and two… & いちに & Two and three… & にさん & Three and four… & さんし \\ \cline{1-6}

Four and five…. & しご & Five and six… & ごろく & Six and seven… & ろくしち \\ \cline{1-6}

Seven and eight… & しちはち & Eight and nine… & はっく・はちく & Nine and ten… & くじゅう \\ \cline{1-6}

\end{ltabulary}

\begin{center}
 \textbf{The Inconsistency of Reading 概数 Phrases }
\end{center}

\par{ Just as in English, these phrases would be pronounced with the expected word breaks. Unlike English, however, there is no single pronunciation for the majority of these phrases. The information below will address many of the most important things to know, but listening to what people actually say around you is the best way to acquire these phrases. }

\par{ When dealing with larger numbers, you would not attach いちに to 十 or 百. When さんし is used before 十, 百, 千, 万, 億, or 兆, it is usually realized as さんよ(ん). In fact, さんしひゃく and さんしせん are almost completely got. }

\par{ しご is oddly enough never changed to よんご. ろくしち to ろくなな will probably become more common as time goes by, but it is currently not widespread. }

\par{ 8~9 is はっく for the majority of people. はちくis usually realized as はちきゅう with はっきゅう coming at a far second in the following cases. 8,000~9,000 can be はちくせん or はっくせん or even はっきゅうせん, but it is usuallyはちきゅうせん. As for 80~90, most people say はちきゅうじゅう but はっきゅうじゅう, はちくじゅう, and はっくじゅう are minority readings. As for 800~900, most people say はちきゅうひゃく. はっきゅうひゃく, はちくひゃく, and はっくひゃく are used by a very small percentage of the population. }

\par{ What about something like 3~4回? さんよんかい. For 3~4人 and 3~4時間, it is common practice to read them respectively as さんよにん and さんよじかん. Ultimately, さんし has disappeared with counters. 3~4枚 is usually さんよんまい. Another potential reason for this is that さんしまい would sound like 三姉妹. }

\par{ What about 6~7回? ろくしちかい or ろくななかい are both OK, but just as has been said about, replacing しち with なな has not spread this far as し \textrightarrow  よん has in this same environment. So, people overwhelmingly use ろくしちかい. You also use ろくしち in other phrases such as 6~7冊 and 6~7人. }

\par{ For 8~9回, it\textquotesingle s most common to hear はちきゅうかい instead of はっきゅかい, はちくかい, or はっくかい. In 18~19の少年, it\textquotesingle s common to say じゅうはちく. There\textquotesingle s also the set phrase 十中八九 read as じゅうちゅうはっく meaning “8 or 9 cases out of 10”.  For 8~9時間, はちくじかん is most common. Likewise, 8~9人 is usually はちくにん. Replacing さんし with さんよん・さんよ is not so problematic, but replacing はちく with はちきゅう in phrases in which はちく is most common does pose a problem. As for 80~90円 and 800~900円, many people would read them out by using か or から. }

\par{ Other things to note include what happens with日 and 概数. It has so many exceptions , but you can still say いちににち, にさんにち, しごにち, and ごろくにち. 3~4日 is さんよっか. This could also be read as さんよんにち by a small amount of people, but さんしにち because people would think死日. There\textquotesingle s also ついたち、ふつか and いちにちふつか for 1~2日. This is often rephrased to ${\overset{\textnormal{いちりょうじつ}}{\text{一両日}}}$ . }

\par{ Then, 1~2人  read as いちににん is uncommon and usually ひとりかふたり. 2~3人 read as にさんにん, though, is very common. }
      
\section{九九}
 
\par{ Oddly enough, 九九 refers to the multiplication chart and is read as くく. In simplified multiplication speech, 被乗数 (the thing being multiplied) and the 乗数 (multiplier) have special pronunciations for certain numbers. In Japanese the thing being multiplied gets stated before the multiplier. }

\begin{ltabulary}{|P|P|P|}
\hline 

For the multiplicand “one” & いち \textrightarrow  いん & いんいちがいち 1 x 1 = 1 \\ \cline{1-3}

For the multiplier “eight” & は \textrightarrow  は・ぱ & ごはしじゅう 5 x 8 = 40; さんぱにじゅうし 3 x 4 = 24 \\ \cline{1-3}

For the multiplier “two” & に \textrightarrow  にん & ににんがし 2 x 2 = 4 \\ \cline{1-3}

Look at what happens to 3 &  & さざんがく 3 x 3 = 9; さぶろくじゅうはち 3 x 6 = 18 \\ \cline{1-3}

\end{ltabulary}

\par{ It\textquotesingle s also possible to have 促音 being inserted or shown up as a result of sound change. }

\begin{ltabulary}{|P|P|P|}
\hline 

ろっくごじゅうし 6 x 9 = 54 & はっくしちじゅうに 8 x 9 = 72 & ごっくしじゅうご 5 x 9 = 45 \\ \cline{1-3}

\end{ltabulary}

\par{ Let\textquotesingle s look at a more complete list. Though you could say things in a more Western fashion as いち かける いち イコール いち, the following manner is perhaps the most important. This is especially the case in reference to the 九九の表. }

\par{Below are all of the multiplication table combinations and their special readings. }

\par{1の段 \hfill\break
1×1=1 いんいちがいち \hfill\break
1×2=2 いんにがに \hfill\break
1×3=3 いんさんがさん \hfill\break
1×4=4 いんしがし \hfill\break
1×5=5 いんごがご \hfill\break
1×6=6 いんろくがろく \hfill\break
1×7=7 いんしちがしち \hfill\break
1×8=8 いんはちがはち \hfill\break
1×9=9 いんくがく \hfill\break
 \hfill\break
2の段 \hfill\break
2×1=2 にいちがに \hfill\break
2×2=4 ににんがし \hfill\break
2×3=6 にさんがろく \hfill\break
2×4=8 にしがはち \hfill\break
2×5=10 にごじゅう \hfill\break
2×6=12 にろくじゅうに \hfill\break
2×7=14 にしちじゅうし \hfill\break
2×8=16 にはちじゅうろく \hfill\break
2×9=18 にくじゅうはち \hfill\break
 \hfill\break
3の段 \hfill\break
3×1=3 さんいちがさん \hfill\break
3×2=6 さんにがろく \hfill\break
3×3=9 さざんがきゅう \hfill\break
3×4=12 さんしじゅうに \hfill\break
3×5=15 さんごじゅうご \hfill\break
3×6=18 さぶろくじゅうはち \hfill\break
3×7=21 さんしちにじゅういち \hfill\break
3×8=24 さんぱにじゅうし \hfill\break
3×9=27 さんくにじゅうしち }

\par{4の段 }

\par{4×1=4  しいちがし \hfill\break
4×2=8  しにがはち \hfill\break
4×3=12 しさんじゅうに \hfill\break
4×4=16 ししじゅうろく \hfill\break
4×5=20 しごにじゅう \hfill\break
4×6=24 しろくにじゅうし \hfill\break
4×7=28 ししちにじゅうはち \hfill\break
4×8=32 しはさんじゅうに \hfill\break
4×9=36 しくさんじゅうろく }

\par{5の段 }

\par{5×1=5  ごいちがご \hfill\break
5×2=10 ごにじゅう \hfill\break
5×3=15 ごさんじゅうご \hfill\break
5×4=20 ごしにじゅう \hfill\break
5×5=25 ごごにじゅうご \hfill\break
5×6=30 ごろくさんじゅう \hfill\break
5×7=35 ごしちさんじゅうご \hfill\break
5×8=40 ごはしじゅう \hfill\break
5×9=45 ごっくしじゅうご }

\par{6の段 }

\par{6×1=6  ろくいちがろく \hfill\break
6×2=12 ろくにじゅうに \hfill\break
6×3=18 ろくさんじゅうはち \hfill\break
6×4=24 ろくしにじゅうし \hfill\break
6×5=30 ろくごさんじゅう \hfill\break
6×6=36 ろくろくさんじゅうろく \hfill\break
6×7=42 ろくしちしじゅうに \hfill\break
6×8=48 ろくはしじゅうはち \hfill\break
6×9=54 ろっくごじゅうし }

\par{7の段 }

\par{7×1=7  しちいちがしち \hfill\break
7×2=14 しちにじゅうし \hfill\break
7×3=21 しちさんにじゅういち \hfill\break
7×4=28 しちしにじゅうはち \hfill\break
7×5=35 しちごさんじゅうご \hfill\break
7×6=42 しちろくしじゅうに \hfill\break
7×7=49 しちしちしじゅうく \hfill\break
7×8=56 しちはごじゅうろく \hfill\break
7×9=63 しちくろくじゅうさん }

\par{8の段 }

\par{8×1=8  はちいちがはち \hfill\break
8×2=16 はちにじゅうろく \hfill\break
8×3=24 はちさんにじゅうし \hfill\break
8×4=32 はちしさんじゅうに \hfill\break
8×5=40 はちごしじゅう \hfill\break
8×6=48 はちろくしじゅうはち \hfill\break
8×7=56 はちしちごじゅうろく \hfill\break
8×8=64 はっぱろくじゅうし \hfill\break
8×9=72 はっくしちじゅうに }

\par{9の段 }

\par{9×1=9  くいちがく \hfill\break
9×2=18 くにじゅうはち \hfill\break
9×3=27 くさんにじゅうしち \hfill\break
9×4=36 くしさんじゅうろく \hfill\break
9×5=45 くごしじゅうご \hfill\break
9×6=54 くろくごじゅうし \hfill\break
9×7=63 くしちろくじゅうさん \hfill\break
9×8=72 くはしちじゅうに \hfill\break
9×9=81 くくはちじゅういち }

\par{ Using various numbers and or including が seems to be rather random, but one could say that about similar conventions in other languages including English. If you look more closely, you can see that there is regularity. For instance, がis no longer used when the product is 10(+). 8 as a multiplier doesn't change when after 1 or 2. にん is only used after に. The other oddities are in the 3の段. }

\par{ Not all speakers may pronounce these all the same way, but these would be the "correct" answers you would find in a textbook, and for the most part, educated individuals will pronounce them as such. }
    