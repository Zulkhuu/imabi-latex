    
\chapter{思しい \& 思わしい}

\begin{center}
\begin{Large}
第301課: 思しい \& 思わしい 
\end{Large}
\end{center}
 
\par{ This lesson will be about the words 思しい and 思わしい. These two words both translate as "to appear to be." They are largely used in literature while only seldom used in conversation, and they are arguably etymologically the same word. }
      
\section{使い方・使い分け}
 
\begin{center}
 \textbf{Etymological Confusion }
\end{center}

\par{ The root \slash obo-\slash  is the same thing as \slash omo-\slash . Meaning, 思う, 覚える, and the like are etymologically related. Japanese once had \slash mb\slash  as a phoneme. As for 思しき, m any sources claim that it comes from the verb 思す. This verb comes from 思ほす, which comes from 思はす (思わす). 思しい comes from attaching \slash –shi\slash  to this. This is similar to other words like 願わしい (desirable) from 願う. 思しい is clearly etymologically the same as 思わしい. The only difference is that the former form shows sign of extra sound change than the latter. }

\begin{center}
 \textbf{Definitions }
\end{center}

\par{ First, we will look at several dictionary endings for the two to see if we can deduce any important information. The first dictionary to look at is Edition 5 of the 広辞苑 . }

\par{\textbf{思しい・覚しい }}

\par{①    (…と)思われる。そのように見受けられる。 \hfill\break
②    こうありたいと思われる。 }

\par{\textbf{思わしい }}

\par{(多く打消の語を伴って)好ましく思われる。気にいる。 }

\par{ Google\textquotesingle s dictionary says the following. }

\par{\textbf{▽思しい/覚しい }}

\par{①    (「…とおぼしい」「とおぼしき」の形で)…と思われる。…のように見える。 \hfill\break
②    こうありたいと望まれる。希望している。 }

\par{\textbf{思わしい }}

\par{①    好ましい状態である。よく思われる。現代では、多く否定の表現を伴って用いられる。 \hfill\break
②    そう思われるようすである。 }

\begin{center}
 \textbf{Qualification of Usage }
\end{center}

\par{ Some speakers try saying that 思しい is only for showing appearance and that 思わしい is for showing what is hoped for. Yet, these entries show that both meanings exist for both. So, we will learn about how they are practically used and grammatical differences that they have to differentiate them. }

\par{ 思しい is going to almost always be used in an attribute phrase and is most often in the form of 思しき in the written language, but it is often 思しい in more spoken language contexts. The form 思しい is not really common, but it does show up periodically in literature. 思わしい doesn't have this restriction, but it\textquotesingle s mainly going to be in the negative form. }

\par{ Using 思しい to mean こうありたいと思われる is not used anymore and is only found in older\slash classical texts. It is listed in dictionaries because of its historical existence. This meaning, though, has been taken over by 思わしくない. In 国語 tests, such questions regarding this usage come up, and you would not get points for choosing 思しい. This is because current language usage does not include this definition. Also, 望まれる or the definition こうありたいと思われる may be more common overall. }

\par{ As for the meaning of showing resemblance, many prescriptive people say that 思しい should be the only one to mean this. Although overall it is what most often appears rather than 思わしい, many speakers use this word to show resemblance. This is reflected in the dictionary definitions above. Thus, this is when textbook definitions and actual usage in society don't match up. }

\begin{center}
\textbf{Examples }
\end{center}

\par{ Now it\textquotesingle s time for example sentences to get a sense of the somewhat literary\slash formal\slash 書き言葉的 style that these phrases are used in and what sorts of phrases fit well with them.  }

\par{1. 犯人と思しき人物を見失う。 \hfill\break
To lose sight of a person thought to be a criminal. }

\par{2. 一見、画家と思しき人  (Uncommon usage) \hfill\break
A person who in one look seems to be an artist }

\par{3. 私の町でも、英語指導助手と思しき人たちを見かける。 \hfill\break
I also come across people in my town who appear to be assistant English teachers. }

\par{4. おととい浅草へ行ったら、団体旅行の外国人と思しき人たちが大勢歩いていた。 \hfill\break
Two days ago when I went to Asakusa, there were a lot of people walking around who looked to be foreigners in a group vacation. }

\par{5. この町では外国人と思しき人を見かけません。 \hfill\break
I don't come across foreign-looking individuals in this town. }

\par{6. 逃走中と思しき女性 \hfill\break
A woman who appears to be on the run }

\par{7. 危険物と思しき物を見たら、すぐに警察に知らせてください。 \hfill\break
If you find something that appears to be a hazardous material, immediately contact the police. }

\par{8. 最近、景気が悪くて、外国人労働者と思しき人が多くなった。 \hfill\break
The economic has been bad recently, and the number of people who appear to be foreign workers has increased. }

\par{9. この手紙には暗号と思しき文字が書かれている。 \hfill\break
There is lettering in this letter that appears to be coded. }

\par{10a. 線路と車輪から出ると思わしい音  ?? \hfill\break
10b. 線路と車輪の間から出ていると思われる音 \hfill\break
10c. 線路と車輪の擦れる音 \hfill\break
A sound resembling what comes from between the tracks and the car wheels (of a train) }

\par{\textbf{Sentence Note }: Because 思わしい is used 90\% with the negative, this sentence may sound strange to many Japanese speakers. 10b is a more natural version of 10a that means the same thing, but 10c is a more practical sentence which just lacks the sense of "resembling". }

\par{11. 病状が思わしくない。 \hfill\break
The disease's condition is not satisfactory. }

\par{12. オリンピックの成績が思わしくなかった。 \hfill\break
The Olympics' results were not satisfactory. }

\par{13. やはり手首の状態が思わしくなかったですね。 \hfill\break
My wrist's condition was just not satisfactory, you know. }

\par{14. 近頃体調があまり思わしくない。 \hfill\break
I haven't been so well recently. }

\par{15. ${\overset{\textnormal{ぜつめつ}}{\text{絶滅}}}$ ${\overset{\textnormal{きぐ}}{\text{危惧}}}$ ${\overset{\textnormal{しゅ}}{\text{種}}}$ と思しき動物もちらほら見かける。 \hfill\break
I even glimpse animals that appear to be endangered species here and there. }

\par{16. ${\overset{\textnormal{せきさく}}{\text{脊索}}}$ ${\overset{\textnormal{どうぶつ}}{\text{動物}}}$ と思しき化石 \hfill\break
Fossils that resemble chordates }

\par{17. 人類の祖先と思しき生物が地球上に登場したのは2000万前といわれている。 \hfill\break
It's said that creatures resembling the ancestors of mankind appeared on the earth 20 million years ago. }

\par{18. 人類の ${\overset{\textnormal{そせん}}{\text{祖先}}}$ と思わしき化石がアフリカの各地で次々と発見されている。 \hfill\break
Fossils of human species thought to be those of man's ancestors are continuously being found all  across Africa. }

\par{19. 寝起きと思しく乱れた髪  ?    (書き言葉) \hfill\break
寝起きと思しき乱れた髪 \hfill\break
Missed up hair like one has woken up from bed }

\par{\textbf{Grammar Note }: Using 思しい in the 連用形 as 思しく is rather rare, but it is still possible. }

\par{20. テストの結果が\{〇 思わしくなかった・△\slash X 思しくなかった\}。 \hfill\break
My test results were not satisfactory. }

\par{21. 犯人と思しい人相の人   (やや書き言葉的) \hfill\break
A person with the looks of a criminal \hfill\break
From the 広辞苑第5版の397ページ }

\par{22. 観光客と思しい人たちが結構乗っています。   (Not conversational) \hfill\break
There are quite a few people riding who appear to be tourists. }

\par{ The following are most certainly possible in the spoken language. After all, と思しき is still occasionally used in the spoken language. It's just not used a whole lot. }

\par{23. Aさんと思しき人物を駅で見た。 \hfill\break
I saw a person that looked like A-san at the train station. }

\par{24. 犬の餌と思しき物体が落ちている。 \hfill\break
Something what appears to be dog food has fallen. }

\par{25. 犯人と思しき人物を見た。 \hfill\break
I saw a person that appeared to be a criminal. }
    