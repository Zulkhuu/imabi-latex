    
\chapter{The Particle して}

\begin{center}
\begin{Large}
第343課: The Particle して 
\end{Large}
\end{center}
 
\par{  The particle して has lost its steam, but it is still important to know how to use it. }
      
\section{The Case Particle して}
 1. Shows that everyone is doing\slash being something. 
\par{1. みなして考えよう。 \hfill\break
Let's all think together. }

\par{2. 二人して行事を阻止した。 \hfill\break
The two together obstructed the event. }

\par{3. 親子四人して出かけてハワイに行った。 \hfill\break
The family of four left and went to Hawaii. }

\par{4. 兄弟二人して天才だよ。 \hfill\break
The two siblings are geniuses. \hfill\break
}

\par{2. Used with the causative to \textbf{mark the performer }of the action. It is frequently used with the auxiliary verb -しめる. It may be used with を. }

\par{5. 私をして言わしめれば \hfill\break
If you ask me }

\par{6. 師匠を(して)降参させたとは大した腕前だ。 \hfill\break
Making the master surrender is a considerable ability. \hfill\break
}

\par{3. Shows the method by which something \textbf{is done "with" }. }

\par{7a. 薬して治す。(Classical) \hfill\break
7b. 薬で治す。(Modern) \hfill\break
Heal with medicine. }

\par{8a. 御衣して耳を塞ぎ給ひつ (Classical) \hfill\break
8b. 彼女は職服で耳を覆った。(Modern) \hfill\break
She covered her ears with the robe. \hfill\break
From the 源氏物語 }
      
\section{The Conjunctive Particle して}
 
\par{1. Following the 連用形 of an adjective or ~ ず , the conjunctive particle して, while affirmative, raises something similar or opposite. }

\par{9. 荘厳にして優美な曲ですね。 \hfill\break
It's a song that is solemn at the same time it is elegant isn't it? }

\par{10. 簡にして要を得る。 \hfill\break
To be brief and to the point. }

\par{2. After the 連用形 of an adjective or ~ず, it is equivalent to "being". }

\par{11. 若くして名を成した。 \hfill\break
I made a name for myself being young. }

\par{12. 彼女なくしては生きられない。 \hfill\break
I can't live without her. }

\par{13. 努力せずして成功はないぞ。 \hfill\break
There is no success without putting effort. }
      
\section{にして}
 
\par{ にして is equivalent to で or なのに meaning "while". にしては is slightly more emphatic. にしても is like "even if (you) were to". にしても may also be seen in more forceful contexts as にしろ・せよ. In casual speech, it may be にしたって. }

\par{14. 一瞬にして燃え尽きた。 \hfill\break
It burned away \textbf{at }an instant. }

\par{15. いずれにしろ、すぐ戻ってくる。 \hfill\break
I'll be back soon \textbf{whether or not }. }

\par{16. 食べないにしても、もっとやせるようになるというわけではない。 \hfill\break
\textbf{Even if you were to }not eat, that doesn't mean you'll be able to get thinner. }

\par{17. 夏にしては、よく雨が降りました。 \hfill\break
It's rained often \textbf{for }summer. }

\par{18. リーダーにしてからがこの始末だ。 \hfill\break
\textbf{Even if you're }the leader, this is the end result. \hfill\break
\hfill\break
\textbf{Grammar Note }: にしてからが is a slightly more contrasting variant of にして. }
      
\section{からして}
 
\par{1. Strengthens the meaning of a starting point. }

\par{19. 彼は子供の時分からして優しかった。 \hfill\break
He was nice \textbf{since }he was a child. }

\par{20. 若者の時からして \hfill\break
\textbf{Since }the time\dothyp{}\dothyp{}\dothyp{}was a youth }

\par{21. こんな初歩的なことからして理解できない。 \hfill\break
You can't even understand \textbf{(from) }such an elementary thing. }

\par{22. 言うことからして生意気だよ。 \hfill\break
You saying, \textbf{for starters }, is impudent. \hfill\break
\hfill\break
\textbf{Grammar Note }: よりして may also be used in this way. \hfill\break
}

\par{2. Shows continued basis of a decision and is equivalent to からみて and からいって--"based on\slash seeing from". }

\par{${\overset{\textnormal{}}{\text{23. 彼女}}}$ は ${\overset{\textnormal{}}{\text{顔}}}$ つきからして、 ${\overset{\textnormal{}}{\text{本当}}}$ に ${\overset{\textnormal{}}{\text{美}}}$ しいですね。 \hfill\break
Starting with her look, she's really beautiful, isn't she? }

\par{${\overset{\textnormal{}}{\text{24. 状況}}}$ から ${\overset{\textnormal{}}{\text{見}}}$ て、 ${\overset{\textnormal{けいざいきき}}{\text{経済危機}}}$ だ。 \hfill\break
Seeing from the situation, it's an economic crisis. }

\par{25. 今の状況からして節電の必要がない。 \hfill\break
\textbf{Based on }the current situation, it is not necessary to have brown-outs. }

\par{26. あなたの口ぶりからして諦める気があるようだ。 \hfill\break
It appears you have intentions of giving up \textbf{judging from }your talk. }

\par{27. 彼はひらがなからして ${\overset{\textnormal{}}{\text{読}}}$ めない。 \hfill\break
He can't even read Hiragana. }

\par{28. ${\overset{\textnormal{たいど}}{\text{態度}}}$ からして、あいつは ${\overset{\textnormal{しっぱい}}{\text{失敗}}}$ したのだろう。 \hfill\break
Based on his attitude, he probably failed. }

\par{29. どの ${\overset{\textnormal{てん}}{\text{点}}}$ からみても ${\overset{\textnormal{てんさい}}{\text{天才}}}$ だよ。 \hfill\break
No matter what you look at, he's a genius. }

\par{30. ${\overset{\textnormal{しけん}}{\text{私見}}}$ から ${\overset{\textnormal{}}{\text{言}}}$ って \hfill\break
From my personal opinion }

\par{3. Frequently followed by して, から may also strengthen the sense of "much less" when raising the most fundamental thing. In other words, it emphasizes the "not to mention" aspect of something. }
 
\par{31. 先生から(して)そんな ${\overset{\textnormal{ふくそう}}{\text{服装}}}$ では ${\overset{\textnormal{こま}}{\text{困}}}$ ります。 \hfill\break
Even for a teacher, that kind of attire is bothersome. }
 
\par{32. 学生からそんな ${\overset{\textnormal{しせい}}{\text{姿勢}}}$ はだめだよ。 \hfill\break
That kind of attitude is bad even from students. }
      
\section{なくして}
 
\par{ なくして means "if it wasn't for"  and is seen after \textbf{nouns }. }

\par{33. 愛なくして、人間は存在できないだろう。 \hfill\break
\textbf{If it wasn't for }love, humans would probably not exist. }

\par{34. 苦労なくして儲けなし。 \hfill\break
\textbf{No }pain no gain. }
    