    
\chapter{Romanization Systems}

\begin{center}
\begin{Large}
第347課: Romanization Systems 
\end{Large}
\end{center}
 
\par{ There have been several systems used to transliterate Japanese with Latin characters. }
      
\section{Kunrei-skiki Romanization}
 
\par{  The Kunrei-shiki was promulgated by the Ministry of Education and was repealed and then given stipulated variants for international relationships and then decreed again in 1954. New かな extensions are hard to apply with this system. So, ambiguity is a problem. For example, ティ and チ are both rendered as "ti". }

\begin{ltabulary}{|P|P|P|P|P|P|P|P|P|}
\hline 

 & ア・ A & イ・ I & ウ・う U & エ・え E & オ・お O & YA ャ & YU ュ & YO ョ \\ \cline{1-9}

K & カ・か KA & キ・き KI & ク・く KU & ケ・け KE & コ・こ KO & キャ・きゃ KYA \hfill\break
& キュ・きゅ KYU \hfill\break
& キョ・きょ KYO \hfill\break
\\ \cline{1-9}

S & サ・さ SA & シ・し SI & ス・す SU & セ・せ SE & ソ・そ SO & シャ・しゃ SYA \hfill\break
& シュ・しゅ SYU \hfill\break
& ショ・しょ SYO \hfill\break
\\ \cline{1-9}

T & タ・た TA & チ・ち TI & ツ・つ TU & テ・て TE & ト・と TO & チャ・ちゃ TYA \hfill\break
& チュ・ちゅ TYU \hfill\break
& チョ・ちょ TYO \hfill\break
\\ \cline{1-9}

N & ナ・な NA & ニ・に NI & ヌ・ぬ NU & ネ・ね NE & ノ・の NO & ニャ・にゃ NYA \hfill\break
& ニュ・にゅ NYU \hfill\break
& ニョ・にょ NYO \hfill\break
\\ \cline{1-9}

H & ハ・は HA & ヒ・ひ HI & フ・ふ HU & ヘ・へ HE & ホ・ほ HO & ヒャ・ひゃ HYA \hfill\break
& ヒュ・ひゅ HYU \hfill\break
& ヒョ・ひょ HYO \hfill\break
\\ \cline{1-9}

M & マ・ま MA & ミ・み MI & ム・む MU & メ・め ME & モ・も MO & ミャ・みゃ MYA \hfill\break
& ミュ・みゅMYU \hfill\break
& ミョ・みょ MYO \hfill\break
\\ \cline{1-9}

Y & ヤ・や YA & (I) & ユ・ゆ YU & (E) & ヨ・よ YO & \hfill\break
& \hfill\break
& \hfill\break
\\ \cline{1-9}

R & ラ・ら RA & リ・り RI & ル・る RU & レ・れ RE & ロ・ろ RO & リャ・りゃ RYA \hfill\break
& リュ・りゅ RYU \hfill\break
& リョ・りょ RYO \hfill\break
\\ \cline{1-9}

W & ワ・わ WA & ヰ・ゐ I & (U) & ヱ・ゑ E & ヲ・を O & \hfill\break
& \hfill\break
& \hfill\break
\\ \cline{1-9}

 &  &  &  &  & ン・ん N(') &  &  &  \\ \cline{1-9}

G & ガ・が GA & ギ・ぎ GI & グ・ぐ GU & ゲ・げ GE & ゴ・ご GO & ギャ・ぎゃ GYA \hfill\break
& ギュ・ぎゅ GYU \hfill\break
& ギョ・ぎょ GYO \hfill\break
\\ \cline{1-9}

Z & ザ・ざ ZA & ジ・じ ZI & ズ。ず ZU & ゼ・ぜ ZE & ゾ・ぞ ZO & ジャ・じゃ ZYA \hfill\break
& ジュ・じゅ ZYU \hfill\break
& ジュ・じょ ZYO \hfill\break
\\ \cline{1-9}

D & ダ・だ DA & ヂ・ぢ ZI & ヅ・づ ZU & デ・で DE & ド・ど DO & ヂャ・ぢゃ ZYA \hfill\break
& ヂュ・ぢゅ ZYU \hfill\break
& ヂョ・ぢょ ZYO \hfill\break
\\ \cline{1-9}

B & バ・ば BA & ビ ・び BI & ブ・ぶ BU & ベ・べ BE & ボ・ぼ BO & ビャ・びゃ BYA \hfill\break
& ビュ・びゅ BYU \hfill\break
& ビョ・びょ BYO \hfill\break
\\ \cline{1-9}

P & パ・ぱ PA & ピ・ぴ PI & プ・ぷ PU & ペ・ぺ PE & ポ・ぽ PO & ピャ・ぴゃ PYA \hfill\break
& ピュ・ぴゅ PYU \hfill\break
& ピョ・ぴょ PYO \\ \cline{1-9}

\end{ltabulary}

\par{\textbf{Other Orthographic Rules }}

\begin{itemize}

\item The particles ヘ, ハ・は, and ヲ・を are "e", "wa", and "o" respectively. 
\item Long vowels are represented with a circumflex mark (ˆ). 
\item ン・ん is n' before y. \hfill\break

\item No exceptions to spelling double consonants. 
\end{itemize}
\textbf{Exceptions for International Purposes }\hfill\break
\hfill\break

\begin{ltabulary}{|P|P|P|P|}
\hline 

A & I & U & O \\ \cline{1-4}

くゎ KWA \hfill\break
&  &  &  \\ \cline{1-4}

ぐゎ GWA \hfill\break
&  &  &  \\ \cline{1-4}

 &  & つ TSU \hfill\break
&  \\ \cline{1-4}

 &  & ふ FU \hfill\break
&  \\ \cline{1-4}

しゃ SHA \hfill\break
& し SHI \hfill\break
& しゅ SHU \hfill\break
& しょ SHO \hfill\break
\\ \cline{1-4}

じゃ JA \hfill\break
& じ JI \hfill\break
& じゅ JU \hfill\break
& じょ JO \hfill\break
\\ \cline{1-4}

ぢゃ DYA \hfill\break
& ぢ DI \hfill\break
& ぢゅ DYU \hfill\break
& ぢょ DYO \hfill\break
\\ \cline{1-4}

 &  & づ DU \hfill\break
&  \\ \cline{1-4}

\end{ltabulary}
      
\section{Hepburn Romanization}
 
\par{ The Hepburn Romanization system, ヘボン式ローマ字, is credited to James Curtis Hepburn who transcribed the Japanese language into English letters. There are three editions. }

\par{Note: Rules for long vowels, long consonants, particles, and n' will be discussed during comparison. As for extended かな, the representation is the same as seen in IMABI Romanization. }

\par{\textbf{The First Edition }}

\begin{ltabulary}{|P|P|P|P|P|P|P|P|P|}
\hline 



 & ア・ A & イ・ I & ウ・う U & エ・え E & オ・お O & YA ャ & YU ュ & YO ョ \\ \cline{1-9}

K & カ・か KA & キ・き KI & ク・く KU & ケ・け KE & コ・こ KO & キャ・きゃ KYA \hfill\break
& キュ・きゅ KYU \hfill\break
& キョ・きょ KYO \hfill\break
\\ \cline{1-9}

S & サ・さ SA & シ・し SHI & ス・す SZ \hfill\break
& セ・せ SE & ソ・そ SO & シャ・しゃ SYA \hfill\break
& シュ・しゅ SYU \hfill\break
& ショ・しょ SYO \hfill\break
\\ \cline{1-9}

T & タ・た TA & チ・ち CHI & ツ・つ TSZ & テ・て TE & ト・と TO & チャ・ちゃ CHA \hfill\break
& チュ・ちゅ CHU \hfill\break
& チョ・ちょ CHO \hfill\break
\\ \cline{1-9}

N & ナ・な NA & ニ・に NI & ヌ・ぬ NU & ネ・ね NE & ノ・の NO & ニャ・にゃ NYA \hfill\break
& ニュ・にゅ NYU \hfill\break
& ニョ・にょ NYO \hfill\break
\\ \cline{1-9}

H & ハ・は HA & ヒ・ひ HI & フ・ふ FU & ヘ・へ HE & ホ・ほ HO & ヒャ・ひゃ HYA \hfill\break
& ヒュ・ひゅ HYU \hfill\break
& ヒョ・ひょ HYO \hfill\break
\\ \cline{1-9}

M & マ・ま MA & ミ・み MI & ム・む MU & メ・め ME & モ・も MO & ミャ・みゃ MYA \hfill\break
& ミュ・みゅMYU \hfill\break
& ミョ・みょ MYO \hfill\break
\\ \cline{1-9}

Y & ヤ・や YA &  & ユ・ゆ YU &  & ヨ・よ YO & \hfill\break
& \hfill\break
& \hfill\break
\\ \cline{1-9}

R & ラ・ら RA & リ・り RI & ル・る RU & レ・れ RE & ロ・ろ RO & リャ・りゃ RYA \hfill\break
& リュ・りゅ RYU \hfill\break
& リョ・りょ RYO \hfill\break
\\ \cline{1-9}

W & ワ・わ WA & ヰ・ゐ WI & \hfill\break
& ヱ・ゑ WE & ヲ・を WO & \hfill\break
& \hfill\break
& \hfill\break
\\ \cline{1-9}

 &  &  &  &  & ン・ん N(-), M &  &  &  \\ \cline{1-9}

G & ガ・が GA & ギ・ぎ GI & グ・ぐ GU & ゲ・げ GE & ゴ・ご GO & ギャ・ぎゃ GYA \hfill\break
& ギュ・ぎゅ GYU \hfill\break
& ギョ・ぎょ GYO \hfill\break
\\ \cline{1-9}

Z & ザ・ざ ZA & ジ・じ JI & ズ。ず DZ & ゼ・ぜ ZE & ゾ・ぞ ZO & ジャ・じゃ JA \hfill\break
& ジュ・じゅ JU \hfill\break
& ジュ・じょ JO \hfill\break
\\ \cline{1-9}

D & ダ・だ DA & ヂ・ぢ JI & ヅ・づ DZ & デ・で DE & ド・ど DO & ヂャ・ぢゃ JA \hfill\break
& ヂュ・ぢゅ JU \hfill\break
& ヂョ・ぢょ JO \hfill\break
\\ \cline{1-9}

B & バ・ば BA & ビ ・び BI & ブ・ぶ BU & ベ・べ BE & ボ・ぼ BO & ビャ・びゃ BYA \hfill\break
& ビュ・びゅ BYU \hfill\break
& ビョ・びょ BYO \hfill\break
\\ \cline{1-9}

P & パ・ぱ PA & ピ・ぴ PI & プ・ぷ PU & ペ・ぺ PE & ポ・ぽ PO & ピャ・ぴゃ PYA \hfill\break
& ピュ・ぴゅ PYU \hfill\break
& ピョ・ぴょ PYO \\ \cline{1-9}

\end{ltabulary}

\par{\textbf{The Second Edition }}

\begin{ltabulary}{|P|P|P|P|P|P|P|P|P|}
\hline 



 & ア・ A & イ・ I & ウ・う U & エ・え YE & オ・お O & YA ャ & YU ュ & YO ョ \\ \cline{1-9}

K & カ・か KA & キ・き KI & ク・く KU & ケ・け KE & コ・こ KO & キャ・きゃ KIYA \hfill\break
& キュ・きゅ KIU \hfill\break
& キョ・きょ KIYO \hfill\break
\\ \cline{1-9}

S & サ・さ SA & シ・し SI & ス・す SU & セ・せ SE & ソ・そ SO & シャ・しゃ SHA \hfill\break
& シュ・しゅ SHU \hfill\break
& ショ・しょ SHO \hfill\break
\\ \cline{1-9}

T & タ・た TA & チ・ち CHI & ツ・つ TSU & テ・て TE & ト・と TO & チャ・ちゃ CHA \hfill\break
& チュ・ちゅ CHU \hfill\break
& チョ・ちょ CHO \hfill\break
\\ \cline{1-9}

N & ナ・な NA & ニ・に NI & ヌ・ぬ NU & ネ・ね NE & ノ・の NO & ニャ・にゃ NYA \hfill\break
& ニュ・にゅ NYU \hfill\break
& ニョ・にょ NYO \hfill\break
\\ \cline{1-9}

H & ハ・は HA & ヒ・ひ HI & フ・ふ FU & ヘ・へ HE & ホ・ほ HO & ヒャ・ひゃ HYA \hfill\break
& ヒュ・ひゅ HYU \hfill\break
& ヒョ・ひょ HYO \hfill\break
\\ \cline{1-9}

M & マ・ま MA & ミ・み MI & ム・む MU & メ・め ME & モ・も MO & ミャ・みゃ MYA \hfill\break
& ミュ・みゅMYU \hfill\break
& ミョ・みょ MYO \hfill\break
\\ \cline{1-9}

Y & ヤ・や YA & \hfill\break
& ユ・ゆ YU &  & ヨ・よ YO & \hfill\break
& \hfill\break
& \hfill\break
\\ \cline{1-9}

R & ラ・ら RA & リ・り RI & ル・る RU & レ・れ RE & ロ・ろ RO & リャ・りゃ RYA \hfill\break
& リュ・りゅ RYU \hfill\break
& リョ・りょ RYO \hfill\break
\\ \cline{1-9}

W & ワ・わ WA & ヰ・ゐ WI &  & ヱ・ゑ YE & ヲ・を WO & \hfill\break
& \hfill\break
& \hfill\break
\\ \cline{1-9}

 &  &  &  &  & ン・ん N(-), M \hfill\break
&  &  &  \\ \cline{1-9}

G & ガ・が GA & ギ・ぎ GI & グ・ぐ GU & ゲ・げ GE & ゴ・ご GO & ギャ・ぎゃ GYA \hfill\break
& ギュ・ぎゅ GYU \hfill\break
& ギョ・ぎょ GYO \hfill\break
\\ \cline{1-9}

Z & ザ・ざ ZA & ジ・じ ZI & ズ。ず DZU & ゼ・ぜ ZE & ゾ・ぞ ZO & ジャ・じゃ JA \hfill\break
& ジュ・じゅ JU \hfill\break
& ジュ・じょ JO \hfill\break
\\ \cline{1-9}

D & ダ・だ DA & ヂ・ぢ ZI & ヅ・づ DZU & デ・で DE & ド・ど DO & ヂャ・ぢゃ JA \hfill\break
& ヂュ・ぢゅ JU \hfill\break
& ヂョ・ぢょ JO \hfill\break
\\ \cline{1-9}

B & バ・ば BA & ビ ・び BI & ブ・ぶ BU & ベ・べ BE & ボ・ぼ BO & ビャ・びゃ BYA \hfill\break
& ビュ・びゅ BYU \hfill\break
& ビョ・びょ BYO \hfill\break
\\ \cline{1-9}

P & パ・ぱ PA & ピ・ぴ PI & プ・ぷ PU & ペ・ぺ PE & ポ・ぽ PO & ピャ・ぴゃ PYA \hfill\break
& ピュ・ぴゅ PYU \hfill\break
& ピョ・ぴょ PYO \\ \cline{1-9}

\end{ltabulary}

\par{Edits }

\begin{itemize}

\item エ・え and ヱ・ゑ were transcribed as ye. 
\item ズ・ず and ヅ・づ were both changed to dzu. 
\item ス・す was not transcribed as sz. 
\item ツ・つ was not transcribed as tsz. 
\item キャ・きゃ, キュ・きゅ, and キョ・きょ were transcribed as kiya, kiu, and kiyo respectively. \hfill\break

\item くわ was transcribed as kwa. \textbf{\hfill\break
}
\end{itemize}

\par{\textbf{Third Edition: 1887 (Traditional Hepburn\slash Hyoujun-shiki Roomaji 標準式ローマ字) }}

\begin{ltabulary}{|P|P|P|P|P|P|P|P|P|}
\hline 



 & ア・ A & イ・ I & ウ・う U & エ・え E & オ・お O & YA ャ & YU ュ & YO ョ \\ \cline{1-9}

K & カ・か KA & キ・き KI & ク・く KU & ケ・け KE & コ・こ KO & キャ・きゃ KYA \hfill\break
& キュ・きゅ KYU \hfill\break
& キョ・きょ KYO \hfill\break
\\ \cline{1-9}

S & サ・さ SA & シ・し SHI & ス・す SU & セ・せ SE & ソ・そ SO & シャ・しゃ SHA \hfill\break
& シュ・しゅ SHU \hfill\break
& ショ・しょ SHO \hfill\break
\\ \cline{1-9}

T & タ・た TA & チ・ち CHI & ツ・つ TSU & テ・て TE & ト・と TO & チャ・ちゃ CHA \hfill\break
& チュ・ちゅ CHU \hfill\break
& チョ・ちょ CHO \hfill\break
\\ \cline{1-9}

N & ナ・な NA & ニ・に NI & ヌ・ぬ NU & ネ・ね NE & ノ・の NO & ニャ・にゃ NYA \hfill\break
& ニュ・にゅ NYU \hfill\break
& ニョ・にょ NYO \hfill\break
\\ \cline{1-9}

H & ハ・は HA & ヒ・ひ HI & フ・ふ FU & ヘ・へ HE & ホ・ほ HO & ヒャ・ひゃ HYA \hfill\break
& ヒュ・ひゅ HYU \hfill\break
& ヒョ・ひょ HYO \hfill\break
\\ \cline{1-9}

M & マ・ま MA & ミ・み MI & ム・む MU & メ・め ME & モ・も MO & ミャ・みゃ MYA \hfill\break
& ミュ・みゅMYU \hfill\break
& ミョ・みょ MYO \hfill\break
\\ \cline{1-9}

Y & ヤ・や YA &  & ユ・ゆ YU &  & ヨ・よ YO & \hfill\break
& \hfill\break
& \hfill\break
\\ \cline{1-9}

R & ラ・ら RA & リ・り RI & ル・る RU & レ・れ RE & ロ・ろ RO & リャ・りゃ RYA \hfill\break
& リュ・りゅ RYU \hfill\break
& リョ・りょ RYO \hfill\break
\\ \cline{1-9}

W & ワ・わ WA & ヰ・ゐ (W)I &  & ヱ・ゑ (W)E & ヲ・を (W)O & \hfill\break
& \hfill\break
& \hfill\break
\\ \cline{1-9}

 &  &  &  &  & ン・ん N(-), M \hfill\break
&  &  &  \\ \cline{1-9}

G & ガ・が GA & ギ・ぎ GI & グ・ぐ GU & ゲ・げ GE & ゴ・ご GO & ギャ・ぎゃ GYA \hfill\break
& ギュ・ぎゅ GYU \hfill\break
& ギョ・ぎょ GYO \hfill\break
\\ \cline{1-9}

Z & ザ・ざ ZA & ジ・じ JI & ズ。ず ZU & ゼ・ぜ ZE & ゾ・ぞ ZO & ジャ・じゃ JA \hfill\break
& ジュ・じゅ JU \hfill\break
& ジュ・じょ JO \hfill\break
\\ \cline{1-9}

D & ダ・だ DA & ヂ・ぢ JI & ヅ・づ ZU & デ・で DE & ド・ど DO & ヂャ・ぢゃ JA \hfill\break
& ヂュ・ぢゅ JU \hfill\break
& ヂョ・ぢょ JO \hfill\break
\\ \cline{1-9}

B & バ・ば BA & ビ ・び BI & ブ・ぶ BU & ベ・べ BE & ボ・ぼ BO & ビャ・びゃ BYA \hfill\break
& ビュ・びゅ BYU \hfill\break
& ビョ・びょ BYO \hfill\break
\\ \cline{1-9}

P & パ・ぱ PA & ピ・ぴ PI & プ・ぷ PU & ペ・ぺ PE & ポ・ぽ PO & ピャ・ぴゃ PYA \hfill\break
& ピュ・ぴゅ PYU \hfill\break
& ピョ・ぴょ PYO \\ \cline{1-9}

\end{ltabulary}

\par{Edits }

\begin{itemize}

\item エ・え and ヱ・ゑ are e and (w)e respectively. 
\item The w in wi, we, and wo is made optional. 
\item 四つ仮名 characters become standardized. \hfill\break

\item Ky-sounds are no longer exceptions. 
\item Kw and gw is only recognized in historical contexts. 
\end{itemize}

\par{\textbf{Modern Hepburn }}

\begin{ltabulary}{|P|P|P|P|P|P|P|P|P|}
\hline 



 & ア・ A & イ・ I & ウ・う U & エ・え E & オ・お O & YA ャ & YU ュ & YO ョ \\ \cline{1-9}

K & カ・か KA & キ・き KI & ク・く KU & ケ・け KE & コ・こ KO & キャ・きゃ KYA \hfill\break
& キュ・きゅ KYU \hfill\break
& キョ・きょ KYO \hfill\break
\\ \cline{1-9}

S & サ・さ SA & シ・し SHI & ス・す SU & セ・せ SE & ソ・そ SO & シャ・しゃ SHA \hfill\break
& シュ・しゅ SHU \hfill\break
& ショ・しょ SHO \hfill\break
\\ \cline{1-9}

T & タ・た TA & チ・ち TI & ツ・つ TU & テ・て TE & ト・と TO & チャ・ちゃ CHA \hfill\break
& チュ・ちゅ CHU \hfill\break
& チョ・ちょ CHO \hfill\break
\\ \cline{1-9}

N & ナ・な NA & ニ・に NI & ヌ・ぬ NU & ネ・ね NE & ノ・の NO & ニャ・にゃ NYA \hfill\break
& ニュ・にゅ NYU \hfill\break
& ニョ・にょ NYO \hfill\break
\\ \cline{1-9}

H & ハ・は HA & ヒ・ひ HI & フ・ふ FU & ヘ・へ HE & ホ・ほ HO & ヒャ・ひゃ HYA \hfill\break
& ヒュ・ひゅ HYU \hfill\break
& ヒョ・ひょ HYO \hfill\break
\\ \cline{1-9}

M & マ・ま MA & ミ・み MI & ム・む MU & メ・め ME & モ・も MO & ミャ・みゃ MYA \hfill\break
& ミュ・みゅMYU \hfill\break
& ミョ・みょ MYO \hfill\break
\\ \cline{1-9}

Y & ヤ・や YA &  & ユ・ゆ YU &  & ヨ・よ YO & \hfill\break
& \hfill\break
& \hfill\break
\\ \cline{1-9}

R & ラ・ら RA & リ・り RI & ル・る RU & レ・れ RE & ロ・ろ RO & リャ・りゃ RYA \hfill\break
& リュ・りゅ RYU \hfill\break
& リョ・りょ RYO \hfill\break
\\ \cline{1-9}

W & ワ・わ WA & ヰ・ゐ (W)I &  & ヱ・ゑ (W)E & ヲ・を (W)O & \hfill\break
& \hfill\break
& \hfill\break
\\ \cline{1-9}

 &  &  &  &  & ン・ん N(') &  &  &  \\ \cline{1-9}

G & ガ・が GA & ギ・ぎ GI & グ・ぐ GU & ゲ・げ GE & ゴ・ご GO & ギャ・ぎゃ GYA \hfill\break
& ギュ・ぎゅ GYU \hfill\break
& ギョ・ぎょ GYO \hfill\break
\\ \cline{1-9}

Z & ザ・ざ ZA & ジ・じ JI & ズ。ず ZU & ゼ・ぜ ZE & ゾ・ぞ ZO & ジャ・じゃ JA \hfill\break
& ジュ・じゅ JU \hfill\break
& ジュ・じょ JO \hfill\break
\\ \cline{1-9}

D & ダ・だ DA & ヂ・ぢ JI & ヅ・づ ZU & デ・で DE & ド・ど DO & ヂャ・ぢゃ JA \hfill\break
& ヂュ・ぢゅ JU \hfill\break
& ヂョ・ぢょ JO \hfill\break
\\ \cline{1-9}

B & バ・ば BA & ビ ・び BI & ブ・ぶ BU & ベ・べ BE & ボ・ぼ BO & ビャ・びゃ BYA \hfill\break
& ビュ・びゅ BYU \hfill\break
& ビョ・びょ BYO \hfill\break
\\ \cline{1-9}

P & パ・ぱ PA & ピ・ぴ PI & プ・ぷ PU & ペ・ぺ PE & ポ・ぽ PO & ピャ・ぴゃ PYA \hfill\break
& ピュ・ぴゅ PYU \hfill\break
& ピョ・ぴょ PYO \\ \cline{1-9}

\end{ltabulary}

\par{Edits }

\begin{itemize}

\item (W)i and (w)e are essentially i and e respectfully. \hfill\break

\item (W)o is based essentially on personal preference. 
\item ン・ん is no longer ever n- or m. 
\end{itemize}

\par{\textbf{Compare and Contrast }}

\par{Changes also occurred in respect to long vowels and long consonants and with n' over time. Because there are no differences between these topics in the three original editions of Hepburn, this discussion will be about Traditional vs. Modern. }

\par{\textbf{Long Vowels }}

\begin{ltabulary}{|P|P|P|}
\hline 

 & Traditional & Modern \\ \cline{1-3}

A & Aa &  Ā \\ \cline{1-3}

I & Ii & Ii \\ \cline{1-3}

U & Uu &  Ū \\ \cline{1-3}

E+E & E\slash ee &  Ē \\ \cline{1-3}

E+I & Ei & Ei \\ \cline{1-3}

O+O & Oo &  Ō \\ \cline{1-3}

O+U & Ou &  Ō \\ \cline{1-3}

\end{ltabulary}

\par{Note: For loanwords all vowels are elongated with macrons. Also, common variations exist for O+U. Such variants include "oh", "o", "ou", "oo", and "ô". }

\par{\textbf{Particles }}

\begin{ltabulary}{|P|P|P|}
\hline 

 & Traditional & Modern \\ \cline{1-3}

へ & He & E \\ \cline{1-3}

は & Wa & Wa \\ \cline{1-3}

を & Wo & (W)o \\ \cline{1-3}

\end{ltabulary}

\par{\textbf{Double Consonants } }

\par{There is no difference between any edition of Hepburn in respect to double consonants. Double consonants are represented normally with sh, ts, and ch doubled as "ssh", "tts", and "tch". }

\par{\textbf{The Uvular Nasal Consonant }}

\par{ン・ん in Traditional Hepburn is transcribed as m when before a m, b, or p sound. When before an n-sound or a vowel, it is separated from these sounds with a hyphen. In Modern Hepburn, m is replaced with n and an apostrophe is used instead of a hyphen when separated n' from n-sounds and vowels. }
      
\section{Nihon-shiki Romanization}
 
\par{ The Nihon-shiki was created by Aikitsu Tanakadate 愛橘田中館 and is the basis for the Kunrei-shiki and is essentially the same minus a few exceptions. These differences are: }

\begin{itemize}

\item Yotsugana differences are maintained. \hfill\break

\item W-sounds are distinguished completely. 
\item Kw and gw are not changed to k and g. 
\end{itemize}

\begin{ltabulary}{|P|P|P|P|P|P|P|P|P|}
\hline 

 & ア・ A & イ・ I & ウ・う U & エ・え E & オ・お O & YA ャ & YU ュ & YO ョ \\ \cline{1-9}

K & カ・か KA & キ・き KI & ク・く KU & ケ・け KE & コ・こ KO & キャ・きゃ KYA \hfill\break
& キュ・きゅ KYU \hfill\break
& キョ・きょ KYO \hfill\break
\\ \cline{1-9}

S & サ・さ SA & シ・し SI & ス・す SU & セ・せ SE & ソ・そ SO & シャ・しゃ SYA \hfill\break
& シュ・しゅ SYU \hfill\break
& ショ・しょ SYO \hfill\break
\\ \cline{1-9}

T & タ・た TA & チ・ち TI & ツ・つ TU & テ・て TE & ト・と TO & チャ・ちゃ TYA \hfill\break
& チュ・ちゅ TYU \hfill\break
& チョ・ちょ TYO \hfill\break
\\ \cline{1-9}

N & ナ・な NA & ニ・に NI & ヌ・ぬ NU & ネ・ね NE & ノ・の NO & ニャ・にゃ NYA \hfill\break
& ニュ・にゅ NYU \hfill\break
& ニョ・にょ NYO \hfill\break
\\ \cline{1-9}

H & ハ・は HA & ヒ・ひ HI & フ・ふ HU & ヘ・へ HE & ホ・ほ HO & ヒャ・ひゃ HYA \hfill\break
& ヒュ・ひゅ HYU \hfill\break
& ヒョ・ひょ HYO \hfill\break
\\ \cline{1-9}

M & マ・ま MA & ミ・み MI & ム・む MU & メ・め ME & モ・も MO & ミャ・みゃ MYA \hfill\break
& ミュ・みゅMYU \hfill\break
& ミョ・みょ MYO \hfill\break
\\ \cline{1-9}

Y & ヤ・や YA &  & ユ・ゆ YU &  & ヨ・よ YO & \hfill\break
& \hfill\break
& \hfill\break
\\ \cline{1-9}

R & ラ・ら RA & リ・り RI & ル・る RU & レ・れ RE & ロ・ろ RO & リャ・りゃ RYA \hfill\break
& リュ・りゅ RYU \hfill\break
& リョ・りょ RYO \hfill\break
\\ \cline{1-9}

W & ワ・わ WA & ヰ・ゐ WI &  & ヱ・ゑ WE & ヲ・を WO & \hfill\break
& \hfill\break
& \hfill\break
\\ \cline{1-9}

 &  &  &  &  & ン・ん N(') &  &  &  \\ \cline{1-9}

G & ガ・が GA & ギ・ぎ GI & グ・ぐ GU & ゲ・げ GE & ゴ・ご GO & ギャ・ぎゃ GYA \hfill\break
& ギュ・ぎゅ GYU \hfill\break
& ギョ・ぎょ GYO \hfill\break
\\ \cline{1-9}

Z & ザ・ざ ZA & ジ・じ ZI & ズ。ず ZU & ゼ・ぜ ZE & ゾ・ぞ ZO & ジャ・じゃ ZYA \hfill\break
& ジュ・じゅ ZYU \hfill\break
& ジュ・じょ ZYO \hfill\break
\\ \cline{1-9}

D & ダ・だ DA & ヂ・ぢ DI & ヅ・づ DU & デ・で DE & ド・ど DO & ヂャ・ぢゃ DYA \hfill\break
& ヂュ・ぢゅ DYU \hfill\break
& ヂョ・ぢょ DYO \hfill\break
\\ \cline{1-9}

B & バ・ば BA & ビ ・び BI & ブ・ぶ BU & ベ・べ BE & ボ・ぼ BO & ビャ・びゃ BYA \hfill\break
& ビュ・びゅ BYU \hfill\break
& ビョ・びょ BYO \hfill\break
\\ \cline{1-9}

P & パ・ぱ PA & ピ・ぴ PI & プ・ぷ PU & ペ・ぺ PE & ポ・ぽ PO & ピャ・ぴゃ PYA \hfill\break
& ピュ・ぴゅ PYU \hfill\break
& ピョ・ぴょ PYO \\ \cline{1-9}

KW & クヮ・くゎ KWA \hfill\break
&  &  &  &  &  &  &  \\ \cline{1-9}

GW & グヮ・グヮ GWA \hfill\break
&  &  &  &  &  &  &  \\ \cline{1-9}

\end{ltabulary}
      
\section{JSL Romanization}
 
\par{  The JSL Romanization created by Eleanor Jorden and introduced in her work \emph{Japanese: The Spoken Language }is heavily based off of the Kunrei-shiki Romanization System but has some important differences. }

\begin{enumerate}

\item All long vowels are shown with doubling. "Ou" is only seen when they are separate sounds. \hfill\break

\item ŋ is distinguished as ḡ. 
\item N' is written as n̄. 
\item An acute accent is used to denote the first high-pitch mora in a word. 
\item A grave accent shows the last high-pitch mora in a word. 
\item A circumflex shows the only high-pitch mora in a word. 
\end{enumerate}
    