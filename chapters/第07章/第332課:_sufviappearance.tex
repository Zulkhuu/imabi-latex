    
\chapter{Suffixes VI Appearance}

\begin{center}
\begin{Large}
第332課: Suffixes VI: Appearance: ~だらけ, ~まみれ, ~ずくめ, ~めく, ~げ, ~みどろ, \& ~気味 
\end{Large}
\end{center}
 
\par{ There are several suffixes in Japanese that represent states of appearance. }
      
\section{~だらけ}
 
\par{  ~だらけ attaches to nouns to describe how something is \textbf{riddled with }something. ~だらけ may be used with both physical and psychological items. There is an extent to how you can use this though. Things like ${\overset{\textnormal{こんらん}}{\text{混乱}}}$ だらけ would be wrong. This is because it should be used with things that are either physical or dealing with (emotional) phenomenon and not a "state". Rather than 混乱だらけ, consider phrases like すごく ${\overset{\textnormal{ろうばい}}{\text{狼狽}}}$ している or とても ${\overset{\textnormal{どうよう}}{\text{動揺}}}$ している. }

\par{1. 彼女は(口を開けば)\{不平・不満\}\{ばっかり言ってる・だらけだ\}な。 \hfill\break
She's full of complaints isn't she? }
 
\par{2. パソコンは ${\overset{\textnormal{ほこり}}{\text{埃}}}$ \{かぶっていた・だらけだった\}。 \hfill\break
The desktop was covered in dust. }

\par{3. ${\overset{\textnormal{こうかい}}{\text{後悔}}}$ だらけ \hfill\break
Covered in regret }

\par{4. ${\overset{\textnormal{きず}}{\text{傷}}}$ だらけ \hfill\break
Covered in injuries }
 
\par{${\overset{\textnormal{}}{\text{5.  血}}}$ だらけになった ${\overset{\textnormal{さつじんはん}}{\text{殺人犯}}}$ は ${\overset{\textnormal{に}}{\text{逃}}}$ げ ${\overset{\textnormal{ば}}{\text{場}}}$ がなくなって ${\overset{\textnormal{じさつ}}{\text{自殺}}}$ したそうだ。 \hfill\break
They say that since the murderer covered in blood didn't have a place to run away, he committed suicide. }
      
\section{~塗れ}
 
\par{  ~まみれ comes from ${\overset{\textnormal{まみ}}{\text{塗}}}$ れる which means "to be covered with". Similar to ~だらけ ~まみれ shows how something is \textbf{completely covered }in something. Unlike ~だらけ, ~まみれ is \textbf{solely used with physical objects }. However, there is a trend to use ~まみれ with things that have a "physical persona". This, though, is metaphoric. Creating metaphors in any language causes rules to become more flexible. }

\par{6. ${\overset{\textnormal{いたい}}{\text{遺体}}}$ は ${\overset{\textnormal{}}{\text{血塗}}}$ れだったということです。 \hfill\break
It is said that the corpse was completely covered in blood. }
 
\par{7. メキシコ ${\overset{\textnormal{わん}}{\text{湾}}}$ は\{辺り一面\} ${\overset{\textnormal{じゅうゆ}}{\text{重油}}}$ \{埋め尽くされました・まみれになりました\}。 \hfill\break
The Gulf of Mexico became completely covered in heavy oil. }
 
\par{${\overset{\textnormal{}}{\text{8. 彼}}}$ はただの ${\overset{\textnormal{ざせつまみ}}{\text{挫折塗}}}$ れのやつだけだ。 \hfill\break
He's just completely covered in failure. }
      
\section{~尽くめ}
 
\par{  ~ずくめ shows that something is "completely\dothyp{}\dothyp{}\dothyp{}". It is normally either attached to nouns of colors or こと. There is, though, a condition that whatever it attaches to that the context is in relation to human intent. }

\par{ A similar ending is ~ずくし. It means "all sorts of". Here, the human intent condition is non-existent. In writing, using づ instead of ず is OK.  This is just a conflict between Old and New orthographies that has just been left to the individual to decide. }
 
\begin{center}
\textbf{Examples } 
\end{center}

\par{${\overset{\textnormal{}}{\text{9. 人}}}$ が ${\overset{\textnormal{}}{\text{黒尽}}}$ くめでした。 \hfill\break
The person was completely black. }
 
\par{10. いい ${\overset{\textnormal{}}{\text{事}}}$ づくしの ${\overset{\textnormal{}}{\text{人}}}$ \hfill\break
A person full of good luck }
 
\par{${\overset{\textnormal{}}{\text{11a. 白}}}$ づくめで ${\overset{\textnormal{}}{\text{抗議}}}$ してる ${\overset{\textnormal{}}{\text{人}}}$ は ${\overset{\textnormal{せっとうざい}}{\text{窃盗罪}}}$ で ${\overset{\textnormal{うった}}{\text{訴}}}$ えられた ${\overset{\textnormal{}}{\text{人達}}}$ だ。 \hfill\break
 ${\overset{\textnormal{}}{\text{11b. 抗議}}}$ をしている、 ${\overset{\textnormal{}}{\text{白}}}$ づくめの ${\overset{\textnormal{}}{\text{人達}}}$ は、 ${\overset{\textnormal{}}{\text{窃盗罪}}}$ で ${\overset{\textnormal{うった}}{\text{訴}}}$ えられた。 \hfill\break
The people protesting dressed in white were accused of theft. }
      
\section{~めく}
 
\par{ ~めく shows what something looks like or what condition something becomes, and it may follow nouns, adverbs, or the stem of adjectives and verbs. The closest translation for ~めく is "to show signs of". Although all tenses are possible, ~めく is \textbf{normally only used in the non-past or past tense }. Depending on the verb, there are may be more options. For instance, うごめく can be seen in things like the progressive. }

\begin{ltabulary}{|P|P|P|P|P|P|}
\hline 

To be astir & ざわめく & To flash & 閃く & To be mysterious & 謎めく \\ \cline{1-6}

To glimmer & 仄めく & To color & 色めく & To be spring-like & 春めく \\ \cline{1-6}

To wriggle & うごめく & To flourish & 時めく & To be autumn-like & 秋めく \\ \cline{1-6}

\end{ltabulary}

\par{\textbf{Grammar Note }: When attached to verbs, the new verb is often seen contracted and then written differently. ~めく is most often used with nouns and 形容動詞. }
 
\par{${\overset{\textnormal{}}{\text{12. 教室}}}$ は ${\overset{\textnormal{こうふん}}{\text{興奮}}}$ で ${\overset{\textnormal{ざわ}}{\text{騒}}}$ めいた。 \hfill\break
The classroom stirred with excitement. }
 
\par{${\overset{\textnormal{}}{\text{13. 庭}}}$ の ${\overset{\textnormal{かえで}}{\text{楓}}}$ が ${\overset{\textnormal{}}{\text{色}}}$ めくのは ${\overset{\textnormal{}}{\text{本当}}}$ に ${\overset{\textnormal{}}{\text{美}}}$ しいですよね。 \hfill\break
The changing in colors of the maple trees in the garden is really beautiful, isn't it? }
 
\par{14. \{うわぁ・ああ\}、\{もぞもぞ動いてる・ ${\overset{\textnormal{うごめ}}{\text{蠢}}}$ いてる\} ${\overset{\textnormal{}}{\text{虫}}}$ (がいる)! \hfill\break
Ah, a wriggling worm! }
 
\par{${\overset{\textnormal{}}{\text{15. 彼}}}$ は ${\overset{\textnormal{}}{\text{今}}}$ を ${\overset{\textnormal{}}{\text{時}}}$ めく ${\overset{\textnormal{}}{\text{歌手}}}$ ・ ${\overset{\textnormal{}}{\text{有名人}}}$ です。 \hfill\break
He is one of the most prosperous singers\slash famous people of the time. }

\par{16. ${\overset{\textnormal{なぞ}}{\text{謎}}}$ めいた ${\overset{\textnormal{ようす}}{\text{様子}}}$ をしている。 \hfill\break
To have an air of mystery. }
 
\par{${\overset{\textnormal{}}{\text{17. 謎}}}$ めいた ${\overset{\textnormal{ようそう}}{\text{様相}}}$ を ${\overset{\textnormal{てい}}{\text{呈}}}$ する。 \hfill\break
To be in a situation covered in mystery. }
 
\par{${\overset{\textnormal{}}{\text{18. 冬}}}$ めいてきた。 \hfill\break
It became winter-like. }

\par{19. 陽光に川面が煌めく。 \hfill\break
The river surface glistens in the sunlight. }

\par{20. \textbf{どよめいて }、人々が飛びのく。一人の背中が飛鳥にぶつかった。はね飛ばされて、壁に肩をぶつける。さらにはね返って \textbf{よろめき }、あやうく転びかけたときだった。 \hfill\break
Making a stir, people jumped back. A person's back hit Asuka. Being sent flying, she hit her shoulder on the wall. In addition, it was when she had bounced back, staggered, and dangerously started to tumble. \hfill\break
From 野生の風 by 村山由佳. }
      
\section{~みどろ}
 
\par{ ~みどろ is a less common alternative to the suffix ~まみれ to show things being covered in something. }
 
\par{21. 血みどろの遺体 \hfill\break
A bloody corpse }
 
\par{22. 汗みどろの労働者 \hfill\break
A sweat covered laborer }
      
\section{~気}
 
\par{  ~げ conjugates as a ${\overset{\textnormal{けいようどうし}}{\text{形容動詞}}}$ . Used to show a sense of true appearance, it describes something with intense fervor and emotion. Lastly, ~げ may be attached to the stem of adjectives and to nouns. }
 
\par{${\overset{\textnormal{}}{\text{23. 悲}}}$ しげに ${\overset{\textnormal{}}{\text{泣}}}$ く。 \hfill\break
Cry while being sad all over. }
 
\par{24. 均衡予算を黙殺するのは大人げない。 }

\par{Ignoring the balanced budget proposal is childish. }
 
\par{${\overset{\textnormal{}}{\text{25. 彼}}}$ は ${\overset{\textnormal{}}{\text{何}}}$ か ${\overset{\textnormal{}}{\text{言}}}$ いたげな ${\overset{\textnormal{}}{\text{顔}}}$ をしていた。 \hfill\break
He had the face of wanting to say something. }

\par{26. 「今も新しい戦争が僕らを追っかけて来ているのかもしれないし、僕らのなかの前の戦争が、亡霊のように僕らを追っかけているかもしれないんです。」と修一は憎さげに、「お父さまこそ、あの娘がちょっと変わっているから、ひそかに魅力を感じて、妙な考えをくどくどもってまわってらっしゃる。女がほかの女とどこかちょっとちがっているだけで、男はつかまるのですからね。」 \hfill\break
"A new word may have come its way to chase after us even now, but the war of the past within ourselves may be chasing after us like a departed spirit", and Shuichi continued odiously with "And you, of course, Father inwardly feel a mystique since she's changed a little, and you're going about stuttering long-winded, strange thoughts. That's man is just something that it's caught up with the small differences one woman has with other women." \hfill\break
From 山の音 by 川端康成. }

\par{27. 菊子は不審げに信吾を見て、そして顔を赤らめた。 \hfill\break
Kikuko looked strangely at Shingo and then blushed. \hfill\break
From 山の音 by 川端康成. }
 
\par{\textbf{Grammar Note }: ~げ is almost never used with verbs . ~げ is closest to ~そうだ, but ~そうだ lacks the emotional emphasis. As emotional emphasis is very important, it is used with nouns that we aspire to be and adjectives that express our emotions. }

\par{ The reading け and ぎ also exist, and they confusingly are used similarly to the げ reading. Luckily, the latter reading is not written as 気. The reading け is seen a lot in phrases like 飾り気 (affectation), 色気 (sex appeal), and 寒気 (feeling chilly). 人気 has two readings. When it is read as にんき, it means "popularity", and when it is read as ひとけ, it is the antonym of deserted, but it is usually used in this sense in the phrase 人気がない.  }

\par{ The reading ぎ of 気 is rather restricted to nouns that can have perceptually suitable characteristics. Words such as  商売気 (mercenary mind), 男気 (chivalry), and 女気 (woman care) come to mind, and the only other examples are all literary words unlikely to be used in conversation, or at the very least appear rarely in literature. Examples of such words include 謀反気 (rebelling spirit), 芝居気 (theatrical spirit), 芸気 (theatrical spirit), and 回り気 (worrying and doubtful spirit). }

\par{28. お前さんも余程回り気の人だね。 \hfill\break
You're quite the worry wort too, aren't you.  \hfill\break
From 深川女房 by 小栗風葉. }

\par{ Interestingly enough, the most common examples of ぎ and け are replaced with っけ. Examples of this include人っ気, 飾りっ気, and 商売っ気. Even words that typically only use け such as 真っ黒け and 真っ白け may have っ insertion in actual conversation. Even more interesting is the fact that the phrases 真っ赤っ赤 (bright red) and 真っ黄っ黄 (bright yellow) exist. The first one shows up dictionaries and is used widely, but the latter is rather dialectical. The interesting point here is how the forms seem to mimic the use of っけ for emphasis. }
      
\section{~気味}
 
\par{ ~ ${\overset{\textnormal{ぎみ}}{\text{気味}}}$ is seen after either a noun or the ${\overset{\textnormal{れんようけい}}{\text{連用形}}}$ of a verb. It shows a visible indication or slight feeling and is often used in a negative context. Common translations include "sign of", "on the\dothyp{}\dothyp{}\dothyp{}side", and "slight". }
 
\par{29. 彼は ${\overset{\textnormal{}}{\text{疲}}}$ れ ${\overset{\textnormal{}}{\text{気味}}}$ だ。 \hfill\break
He has the signs of fatigue. }
 
\par{${\overset{\textnormal{}}{\text{30. 彼}}}$ は ${\overset{\textnormal{}}{\text{太}}}$ り ${\overset{\textnormal{}}{\text{気味}}}$ だ。 \hfill\break
He's a little on the heavy side. }
 
\par{${\overset{\textnormal{}}{\text{31. 今朝}}}$ はちょっと ${\overset{\textnormal{}}{\text{風邪気味}}}$ だ。 \hfill\break
I have a slight feeling of a cold \emph{this morning }. }
 
\par{\textbf{Word Note }: ~味 comes from 気味, which means "sensation" or "tendency". }
    