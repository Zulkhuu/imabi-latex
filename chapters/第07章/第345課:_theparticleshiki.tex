    
\chapter{The Particle しき}

\begin{center}
\begin{Large}
第345課: The Particle しき 
\end{Large}
\end{center}
 
\par{ The particle しき greatly resembles だけ, but it is no longer used a lot and is very limited in usage.  }
      
\section{The Adverbial Particle しき}
 
\par{ しき is seen after これ, それ,  and あれ to express that something is "only about\dothyp{}\dothyp{}\dothyp{}much". The particle しき derives from 式 meaning "style". しき is synonymous to the pattern たかが…くらい. }

\par{1. これしきだ。 \hfill\break
It's just this much. }

\par{2. これしきの問題で弱音を吐くな。 \hfill\break
Don't whine about such an insignificant problem. }

\par{3. たかがそれくらいのもんだから、心配するな。 \hfill\break
Don't worry cause it's only just that much. }

\par{4. ほんのあれしきのことで驚いてはいけぬぞ。 \hfill\break
You mustn't be astonished at such a mere thing as that. }

\par{\textbf{Orthography Note }: You may write しき in 当て字 as 式. }
    