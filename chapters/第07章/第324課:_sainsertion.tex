    
\chapter{さ入れ言葉}

\begin{center}
\begin{Large}
第324課: さ入れ言葉 
\end{Large}
\end{center}
 
\par{ There are few instances in Japanese grammar where さ gets inserted for apparently no reason. There are three different instances of this in Japanese. }
      
\section{In the Causative}
 
\par{ One application of さ入れ言葉 is found within the causative conjugations. The causative conjugation is believed to be derived from the auxiliary す, which is etymologically the same as する attaching to the 未然形. When used with 一段動詞 and 来る, さ would be inserted, which gives us ~さす・~させる today. }

\begin{ltabulary}{|P|P|P|P|}
\hline 

 & 五段動詞 (飲む) &  & 一段動詞  (分ける) \\ \cline{1-4}

語幹 & nom- & 語幹 & wake- \\ \cline{1-4}

終止形 & nom-u & 終止形 & wake-ru \\ \cline{1-4}

使役形 & nom-ase- & 使役形 & wake-sase- \\ \cline{1-4}

\end{ltabulary}

\par{ Here, it is clear that the inclusion of さ in the right-hand column is a completely natural and correct. Now, consider what\textquotesingle s happening now (at least for a small subset of people). }

\begin{ltabulary}{|P|P|P|P|}
\hline 

 & 五段動詞 (読む) &  & 一段動詞 (食べる) \\ \cline{1-4}

使役形 & ? yom-a \textbf{sa }se- & 使役形 & * tabe- \textbf{sa }sase- \\ \cline{1-4}

\end{ltabulary}

\par{ This is what is now meant by さ入れ言葉 because an unnecessary さ is being inserted. This phenomenon only affects 五段動詞, as you can imagine if it\textquotesingle s already occurred in 一段動詞 in the past. Therefore some people have proposed that this phenomenon has come about from an analogical leveling with the causative element “sase” found with 一段動詞. }

\par{ A few things have come to light about the use of さ入れ言葉. }

\par{1.    It is a language change that has begun to increase in usage in recent years. }

\par{2.    Its use is spreading and will continue to be used more frequently. }

\par{3.    There is a tendency for it to appear in ceremonious speech. The more formal the situation, the more likely it is to be used. It does not appear in idioms or nominalized expressions. }

\par{4.    Sometimes, there are combinations where a combination may be considered a correct usage of the causative or an instance of さ入れ言葉. For example, consider the word 飛ばす. If treated as a transitive verb pair of 飛ぶ, 飛ばさせる is the logical causative form. If you analyze it as the causative form of 飛ぶ, then this form would be an example of さ入れ言葉. This all depends on whether there is an object for the verb or not. In the following example, it is a clear case of さ入れ言葉. }

\par{Ex. ちょっと時間の関係で飛ばさしていただきまして… \hfill\break
I would like to skip through with a small concern of time. }

\par{5.    It possesses nuances not necessarily found in the “correct usage” of the causative. }

\par{6.    It is becoming grammaticalized with ~させていただく. It is used with this pattern more than anything else. It is marginally seen with ~てもらう, ~た, ~ない, ~る. However, every other conjugation is not really attested. Thus, it may become the case that さ入れ言葉 will only occur with ~させて\{いただく・もらう\} within a few decades and potentially become standard. Nevertheless, its usage as a long way to go before it is used equally with the current standard form. }

\par{7.    It is only used with verbs that imply speaker self-control. }

\par{8.    It does not appear when double さ would occur. }

\par{9.    It hardly ever shows up with verbs at or above five morae. This is understandable because if this change is believed to be in the beginning stage, you would expect it to pop up in short but very frequently used verbs. }

\par{10.  Overall, women use this slightly more than men, which is evidence that this phenomenon is its beginning stage as women are the pioneers of language change. It is important to note, though, that there is a tendency for women to use this more the more formal of a situation they find themselves in, and the opposite trend can be said for men, which is probably due to a prescriptive grammar take on things }

\par{11.  People born before 1940s don\textquotesingle t really use this. People born after this time are progressively using it more. Nevertheless, the percentage of people who use さ入れ言葉 is still quite small. }

\par{12.  This pattern is spreading via the 関東地方, but whether it is being spread through people after relocation or it\textquotesingle s something being picked up through other means is uncertain. }

\par{13.  It is most frequent in spontaneous speech. Meaning, if there is any preparation involved in the delivery of a message, it is more likely not going to be used. Thus, there is no trend suggesting that this pattern has started to seep in 書き言葉. This is further evidence to suggest that it is overall in the beginning stage of diffusion. }

\par{14.  This pattern is still viewed as an error that should be fixed. However, because it is used to elevate politeness, when it is used, the context requires more elaborate honorific speech. This explains the demographic demonstrations mentioned thus far. }
      
\section{With ~すぎる}
 
\par{ With the adjective ない, さ must be inserted. This gives なさすぎる. ~すぎる should attach directly to the stem of ~ない. The inclusion of さ to the auxiliary ~ない to make it more flowing is catching on among some young people, but it is not deemed to be correct Japanese. }

\par{1. 彼は何もできな(さ)すぎる。 \hfill\break
He can't do anything. }

\par{ \hfill\break
2. 彼は世間をあまりにも知らな(さ)すぎる。 \hfill\break
He knows too little of the world. }

\par{ Using さ入れ with 少ない or 汚い because they naturally end in \slash nai\slash  is not correct. This misuse has not caught on. This is also the case for any other words that happen to end in \slash nai\slash . }
      
\section{With ~そうだ}
 
\par{ With ~そうだ meaning "to seem", it is obligatory that さ be inserted when used with the adjectives よい and ない. Nowadays, it is common to use さ with ない even when it is used as an auxiliary, which is traditionally not correct. }

\par{ Though percentages of speakers who add さ to these additional words is small, there are natives who attach it to other adjectives only two morae long. For instance, 濃い + そう \textrightarrow  濃そうだ is standard, but a very small minority uses 濃さそうだ. This is no doubt dialectical, but it's fascinating that it exists. }

\par{ Another example is 憂い. This adjective is normally replaced by 物憂い. What's interesting is that さ is often seen inserted when 物憂い is used with ~そうだ. Now, because of the limited use of 憂い, 憂さそうだ would be hard to attest. So, it is safe to say it practically does not exist. 酸い, now usually 酸っぱい except in set phrase or some dialects, would also fit into this category, but 酸っぱさそうだ is not valid, and it is unclear whether 酸さそうだ is. }
 
\par{\textbf{Classroom Usage }: You should only use さ with 良い and 無い. Although it is linguistically correct to say other instantiations exist, your teachers seek what is thought to be true Standard Japanese. Thus, it is best to avoid things that may be given an X on test. }
    