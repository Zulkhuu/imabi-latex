    
\chapter{The Particles ども, \& だの}

\begin{center}
\begin{Large}
第304課: The Particles ども, \& だの 
\end{Large}
\end{center}
 
\par{ In this lesson, we will learn about the conjunctive particles ども and だの. }
      
\section{The Particle ど(も)}
 
\par{ ど(も) is attached to the 已然形 of verbs to mean "even though". As for だ, use either であれど(も) or なれど(も). This will be very old-fashioned with some exceptions. }
 
\par{1. \textbf{行けども、行けども }、砂ばかりだった。   (普段) \hfill\break
 \textbf{Even though we kept going and going }, there was only sand. }
 
\par{2. しかし、結果は一そう悪く、 \textbf{待てど暮らせど }何の返事も無く、自分はその ${\overset{\textnormal{しょうそう}}{\text{焦燥}}}$ と不安のために、かえって薬の ${\overset{\textnormal{りょう}}{\text{量}}}$ をふやしてしまった。 \hfill\break
But, the result was even worse, and \textbf{having waited and wanted }and not received a response of any sort, I ended up increasing my dosage all the more out of impatience and anxiety. \hfill\break
From 人間失格 by 太宰治. }

\par{3. 自ら ${\overset{\textnormal{かえり}}{\text{顧}}}$ みて ${\overset{\textnormal{なお}}{\text{縮}}}$ くんば ${\overset{\textnormal{}}{\text{千万人}}}$ \textbf{と ${\overset{\textnormal{いえど}}{\text{雖}}}$ も }${\overset{\textnormal{われ}}{\text{我}}}$ ${\overset{\textnormal{ゆ}}{\text{往}}}$ かん。 \hfill\break
Upon reflecting on my actions, I continue to go forward \textbf{regardless of }what the masses \textbf{may say }. \hfill\break
Famous, frequently quoted statement made by 吉田松陰. }

\par{4. 同じ秘書経験者 \textbf{といえども }、菅と安倍を分けるものはまさしくそれだった。 \hfill\break
\textbf{Despite }both being experienced secretaries, it was that very point of commonality which separated Suga and Abe from each other. }

\begin{center}
\textbf{The Particle たりとも }
\end{center}
 
\par{ The conjunctive particle たりとも is closely related etymologically to であっても. It is composed of another copula verb of older-style Japanese and the particle とも. In actual practice, its usage is restricted to set phrases. }

\par{5. なにより、自らの「同志」の一人が政権に参画したのだ。しかも彼は事務方の官房副長官という、過去 \textbf{一度たりとも }民間人の就いたことのない重要ポストに座っている。 \hfill\break
More than anything else, there was one person who took part in planning in the administration with the same sentiments as [the Prime Minister]. Furthermore, he was sitting in a high-level post, the administrative Deputy Chief Cabinet Secretary, for which no civilian had been appointed to \textbf{even once }in the past. }

\par{6. \textbf{一瞬たりとも }おろかにはできぬ。 \hfill\break
You can't even neglect it \textbf{even for a moment }. }

\par{7. 2001年、同じ場所で演説をした前首相の小泉純一郎は、街宣車から降りた途端、 \textbf{一歩たりとも }動けなくなった。 \hfill\break
The moment Former Prime Minister Koizumi Jun'ichiro stepped out from the propaganda vehicle onto the same spot he had given a speech in 2001, he was unable to make \textbf{even a step }. }
      
\section{The Particle だの}
 
\par{ The particle だの, similar to other particles such as と, や, and とか, enumerates parallel things. For the most part, its usage implies that the sentence is not completely enumerating all interrelated\slash parallel things, and in doing so, it usually has negative connotations. It is more correct, though, to say that rather than labeling these sentences as having a negative nuance, we can say that it is typically used for certain things that the speaker doesn't wish for and or have some sort of psychological distance. }
 
\par{The pattern ~だの~だの is used with two or more nouns, adjectives, or verbs, and other case particles may be found after the final instance of だの. If you can only think of one thing and wish to use ~だの, ~だのなんだの would have to be used to avoid sounding unnatural. It's also important to note that dropping any だの is ungrammatical. }

\begin{center}
 \textbf{Examples }
\end{center}
 
\par{8. 大学に入ると、教科書だの定期券だのを買わなければならない。 \hfill\break
When you enter college, you have to buy textbooks, a commuter pass, and so on. }
 
\par{9. 十歳の息子にアイフォンだのパソコンだのとうるさくせがまれて困ってます。 \hfill\break
I'm troubled over the fact that I'm being pestered by my 10 year old son for an iPhone, computer, and who knows what all. }
 
\par{10. 書類だの本だのが ${\overset{\textnormal{さんらん}}{\text{散乱}}}$ している。 \hfill\break
Things such as documents and books are scattered. }
 
\par{11. 目のつけどころが素晴らしいだの文章の構成が ${\overset{\textnormal{すぐ}}{\text{優}}}$ れているだのと、自分でも信じられないような ${\overset{\textnormal{ほ}}{\text{褒}}}$ め言葉が並 んでいるのはわけがわからない。 \hfill\break
These accolades lined up that even I myself can't believe like the focal points being wonderful and the   sentence structure being superb are incomprehensible. }
 
\par{12. まずいだのきらいだのと、あんたは文句ばっかり言ってるんだよ。(Masculine; vulgar) \hfill\break
All you're ever complaining about is how bad it tastes and how much you hate it. }

\par{13. ${\overset{\textnormal{つら}}{\text{辛}}}$ いだの面白くないだの、文句ばっか言うのよ!(Feminine) \hfill\break
All you ever complain about is stuff like it being miserable or uninteresting. }
 
\par{14. その国では、 ${\overset{\textnormal{しんどう}}{\text{神童}}}$ だの天才だのと呼ばれる才能ある子供たちは、 ${\overset{\textnormal{ようしょうき}}{\text{幼少期}}}$ から ${\overset{\textnormal{すで}}{\text{既}}}$ にエリート教育を ${\overset{\textnormal{さず}}{\text{授}}}$ けられ、成長するに伴いその能力を ${\overset{\textnormal{ぞんぶん}}{\text{存分}}}$ に開花させていくらしい。 \hfill\break
In that country, children with talent that are considered prodigies or geniuses are already given elite education from early childhood, and along with growth, their potential is fully opened to their content. \hfill\break
From 日本教育121号 in an article by 鈴木智美. }

\par{15. ${\overset{\textnormal{じっしひ}}{\text{実施費}}}$ が高すぎるだの、非実用的だのといわれますが、それでも私はX議長の ${\overset{\textnormal{せいさく}}{\text{政策}}}$ を支持しております。 \hfill\break
Though it is said that the implementation costs are too high or that it is not practical, I am still in support of Chairman X's policy. }
 
\par{16. 大学に入ると、教科書だの定期券だの必要なものを買わなければならない。 \hfill\break
When you enter college, you have to important things such as textbooks and a commuter pass. \hfill\break
 \hfill\break
 \textbf{Grammar Note }: Notice that 教科書 and 定期券 are examples of 必要なもの. The three phrases are treated as being of the same grammatical "case\slash function", and given their relationship, no additional particle is needed. }
 
\par{17. 最近の中学生は、iPhoneだの ${\overset{\textnormal{たんまつ}}{\text{端末}}}$ タブレットなどを欲しがるそうだ。 \hfill\break
Junior high students these days apparently wish for iPhones and other tablet devices. }
 
\par{\textbf{Particle Note }: The pattern ~だの~だの may change to ~だの~など under the condition that the pair of words are in the same "category". As categorization of things widely varies from person to person, there is variable opinion on grammatical acceptance of this change, which is even more apparent with more complex examples than this. However, the default pattern is always correct so long as you are using だの correctly in the first place. }
    