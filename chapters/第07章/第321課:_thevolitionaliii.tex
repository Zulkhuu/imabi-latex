    
\chapter{The Adjectival Volitional and Older Volitional Forms}

\begin{center}
\begin{Large}
第321課: The Adjectival Volitional and Older Volitional Forms 
\end{Large}
\end{center}
 
\par{ Now that you have learned a lot about the volitional and negative volitional form with verbs, you need to learn about how to make an adjectival volitional phrase and know what it can mean. Furthermore, there are old-fashioned ways of saying things that you'll still encounter and need to know. }
      
\section{Adjectival Volition}
 
\par{ The auxiliary ~う can also attach to the 未然形 of adjectives. Now, we haven't ever had to use this so far. So, we will first look at what this conjugation looks like. }

\begin{ltabulary}{|P|P|P|P|P|P|}
\hline 

Class & Construction & Example & Class & Construction & Example \\ \cline{1-6}

形容詞 & から+う \textrightarrow  かろう & 新しかろう & 形容動詞 & だら+う \textrightarrow  だろう & 簡単だろう \\ \cline{1-6}

\end{ltabulary}

\par{\textbf{Pronunciation Note }: Remember that ”ろう" is pronounced as "ろー". }

\par{ For 形容詞 this pattern has become longer but easier to use. So, instead of 新しかろう, you use 新しいだろう. Why the original pattern became old-fashioned is not certain, but だろう・でしょう are far more prevalent now. And, using them instead makes the conjugation the same for 形容詞 and 形容動詞. }

\par{ Now, rather than meaning "let's\dothyp{}\dothyp{}\dothyp{}", which is verbal by nature, the adjectival volition's main meaning is to show guess. This is equivalent to "probably", but according to context, it can be closer to guess about the future rather than present state. }

\par{1. 安かろう悪かろう。(Set Phrase) \hfill\break
Cheap is rubbish\slash You get what you pay for. }

\par{2. 早かろう悪かろう。 (Set Phrase) \hfill\break
Finishing early surely means a worse off result. }

\par{3. 一人きりは寂しかろう。(Old-fashioned) \hfill\break
It must be lonely being all alone. }

\par{4. あの紙が黄色かろう。(Old-fashioned) \hfill\break
That paper \{is\slash will\} probably be yellow. }

\par{5. 水面の下を泳ぐ魚たちは悲しかろう。 (Old-fashioned) \hfill\break
The fish swimming beneath the water's surface are surely sad. }

\par{6. 喜ばしかったでしょうに、お旅は残念でした。(Somewhat vague) \hfill\break
Even though one would have thought it to be delightful, the trip was regrettable. }
 
\par{7. 遠かったでしょうに、歩いてきたのですか。 \hfill\break
Although one would think it to have been far, you came here by walking? }

\par{ If you want to say something like "let's have it easy", you would need to use ~\{く・に\}する+Volitional. So, you would need the adjective in its 連用形 to be used as a 副詞 and then used with a verb. After all, "volition", the will to do something, naturally requires a verbal element. So, why not use する? }

\par{8. それを簡潔にしましょう。 \hfill\break
Let's make this simple. }

\par{9. 試験を合格することをもっと難しくしましょうか。 \hfill\break
How about we make it harder to pass the exams. }

\par{10. このぶす女郎、クアグマイアにきれいにしてもらおうか。(Vulgar) \hfill\break
How 'bout we have Quagmire clean up this ugly whore. }

\par{\textbf{Personal Note to Disregard }: Yes, watching Family Guy gave me inspiration to make this sentence. }

\par{ The one time that the traditional adjectival volitional form for 形容詞 is used often in Modern Japanese aside from set phrases is in the pattern ~かろう\{が・と\}~かろう\{が・と\}. This is equivalent to "whether\dothyp{}\dothyp{}\dothyp{}or\dothyp{}\dothyp{}\dothyp{}", and the adjectival phrases that you choose should be reasonably contrastive, and antonymous phrases are your best options. }

\par{11. 休日であろうが、なかろうが、仕事をしなければならない。 \hfill\break
A holiday or not, you have to do your job. }

\par{12. 彼女が実の犯人であろうとなかろうと、中国で有罪の判決が言い渡されたから、 もうすぐ処刑されるだろう。 \hfill\break
Whether or not she's the true criminal or not, because she's been found guilty in China, she will probably be executed soon. }

\par{13. 寒かろうが暑かろうが、試合を中止しないよ。 \hfill\break
Whether it's hot or cold, I won't postpone the game. }

\par{14. 古かろうが新しかろうが、そんなもの買ったとしても、あてにならないよ。 \hfill\break
Whether it's old or new, even if you were to buy something like that, it won't do you any good. }

\par{15. 母さんが夜なべをして、手袋、編んでくれた。木枯らし吹いちゃ、冷たかろうに、せっせと編んだんだよ。 \hfill\break
Mother knitted mittens for me on her night ship. Even though it was supposed to be cold with the wintry breeze blowing, she diligently knitted it. \hfill\break
From a song entitled 母さんの歌. Original lyrics have been changed to more standard Japanese. }

\par{16. 美しかろうが醜かろうが、人間は人間だ。 \hfill\break
Whether beautiful or ugly, a human is a human. }

\par{17. 黒かろうが白かろうが、蝶を俺のトカゲのエサとしか思えない。 \hfill\break
Whether it's white or black, I can only think of butterflies as food for my lizard. }

\par{ As ~ない conjugates as a 形容詞, ~なかろう = ~ない\{だろう・でしょう\} for verbs as well. }

\par{18. 富山氏は当選しなかろう。(Old-fashioned). \hfill\break
Mr. Toyama will probably not be elected. }

\par{${\overset{\textnormal{}}{\text{喜}}}$ ばしかったでしょうに、お ${\overset{\textnormal{}}{\text{旅}}}$ は ${\overset{\textnormal{}}{\text{残念}}}$ でした。 \hfill\break
Even though one would have thought it to be delightful, the trip was regrettable. }

\par{${\overset{\textnormal{}}{\text{遠}}}$ かったでしょうに、 ${\overset{\textnormal{}}{\text{歩}}}$ いてきたのですか。 \hfill\break
Although one would think it to have been far, you came here by walking? }
      
\section{~たろう}
 
\par{ In polite speech (and thus more rare) as ~ましたろう, this is equivalent to ~ただろう. Of course, because だろう has been becoming more vulgar, ~たでしょう is usually the better equivalent. }

\par{19a. 新しくなかったろう。(Old-fashioned) \hfill\break
19b. 新しくなかったでしょう。(Natural) \hfill\break
It probably wasn't new. }

\par{20. よろしかったろう。(Old-fashioned) \hfill\break
It was surely good. }

\par{21. もう少し注意したなら、成功したろうに。\textrightarrow  もう少し注意したなら、成功しただろうに。 (もっと自然) \hfill\break
If you would have payed just a little more attention, you would have succeeded. }
      
\section{であろう}
 
\par{ When you use less contracted forms, you are naturally being more polite\slash formal. When you use であろう instead of だろう, you are most likely writing down your statement. Even so, this is rather formal, and in this case, it makes the statement sound more definitive and objective. Thus, it isn't surprising that it has been used a lot in the Bible. }

\par{22. 努力の甲斐があれば ${\overset{\textnormal{せいこう}}{\text{成功}}}$ するであろう。 \hfill\break
Success will surely reward you for availing efforts. }

\par{23. ${\overset{\textnormal{かなら}}{\text{必}}}$ ずや ${\overset{\textnormal{}}{\text{成功}}}$ するであろう。 \hfill\break
You will absolutely succeed. }

\par{24. エジプトでは ${\overset{\textnormal{はんせいふは}}{\text{反政府派}}}$ のデモ ${\overset{\textnormal{たい}}{\text{隊}}}$ は ${\overset{\textnormal{ぎゃっきょう}}{\text{逆境}}}$ にあっても、 ${\overset{\textnormal{くじ}}{\text{挫}}}$ けないであろう。 \hfill\break
Even in adversity, the anti-government demonstrators in Egypt will surely not crumble. }

\par{25. すると、ペテロが答えた、「悔い改めなさい。そして、あなたがたひとりびとりが罪のゆるしを得るために、イエ ス・キリストの名によって、バプテスマを受けなさい。そうすれば、あなたがたは聖霊の賜物を受けるであろう。       …」 \hfill\break
Then, Peter said unto them, "Repent and be baptized everyone of you in the name of Jesus Christ for the remission of your sins, and you shall receive the gift of the Holy Ghost. \hfill\break
使徒行伝  第2章38節  Acts 2:38 }

\par{ ~でありましょう is the polite form and is not necessarily literary. This is because でしょう may not cut it for being formal enough in some situations, and this would be your next alternative before using something like でございましょう・でいらっしゃいましょう (both of which are possible). Like with everything in this lesson, this is \textbf{not used anymore }. }

\par{26. あちらの方をご存じでありましょうか。 \textrightarrow   あちらの方をご存じですか。 (Natural) \hfill\break
Do you know that individual? }

\par{27. 極めて良心的な店でありましょう。  \textrightarrow  極めて良心的な店でしょう。   (Natural) \hfill\break
My, isn't it an extremely upright store. }

\par{ The negative で(は)なかろう is the negative form, and it is literary with the same nuances as its positive counterpart. The polite form of this, でありますまい, is extremely old-fashioned and is essentially only preserved in the most elegant 敬語 and literature from by-gone eras. }

\par{28. 純粋な月影石でなかろう。  \textrightarrow  純粋な月影石ではないでしょう。 (Modern) \hfill\break
It is probably not a genuine moon rock. }

\par{29. テロでなかろうがデモは規制する。 \hfill\break
Though it may not be terror, we'll control demonstrations. \hfill\break
Infamous poor words of 石破茂. }
    