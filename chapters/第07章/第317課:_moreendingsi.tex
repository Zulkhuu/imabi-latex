    
\chapter{More Endings I}

\begin{center}
\begin{Large}
第317課: More Endings I: ~果てる ~付ける・く, ~立てる・つ, ~尽くす,\& ~こなす 
\end{Large}
\end{center}
       
\section{~果てる}
 
\par{ 果てる means "to come to an end" and is intransitive. Due to its meaning, it may also be a euphemism for "to die". When attached to the 連用形 of a verb, it shows that something has been done completely or up to a given limit. It is often used in a negative fashion. }

\par{1. ${\overset{\textnormal{せい}}{\text{精}}}$ も ${\overset{\textnormal{ね}}{\text{根}}}$ も尽き ${\overset{\textnormal{は}}{\text{果}}}$ てる。 \hfill\break
To use up all of one's energy. }

\par{2. その ${\overset{\textnormal{きゅうでん}}{\text{宮殿}}}$ は ${\overset{\textnormal{あ}}{\text{荒}}}$ れ果てている。 \hfill\break
The palace is falling into ruin. }

\par{3. ${\overset{\textnormal{しょち}}{\text{処置}}}$ に困り果てる。 \hfill\break
To be completely perplexed in measures (being taken). }

\par{4. 変わり果てた ${\overset{\textnormal{すがた}}{\text{姿}}}$ だな。 \hfill\break
It's truly a completely different appearance\dothyp{}\dothyp{}\dothyp{} }

\par{5. 忙しくて ${\overset{\textnormal{つか}}{\text{疲}}}$ れ果ててしまいました。 \hfill\break
I was so busy that I ended up exhausting myself. }

\par{6a. ${\overset{\textnormal{じがい}}{\text{自害}}}$ して果てる。 \hfill\break
6b. \{自害・自殺\}する。(More Natural) \hfill\break
To commit suicide. }

\par{\textbf{Similar Ending Note }: ~果せる・遂せる・おおせる is similar to ~果てる in spelling, but it actually means "to successfully do". Its usage is rather limited and most commonly seen in literature with verbs such as 死ぬ, 盗む, 生きる, 逃げる, 隠す, etc. As you can see, the connotations are usually quite negative. }

\par{7. ネズミのように逃げ遂せる(の)か。 \hfill\break
Are you going to successfully run away like a rat? }
      
\section{~付ける・く}
 
\par{ ~付ける derives from its usages as an independent verb on the lines of "attaching".  It either shows something out of habit like いつも\dothyp{}\dothyp{}\dothyp{}する and しなれる or something very firmly or aggressively. The intransitive form is ~付く. }

\par{8. 真っ先に駆けつけよ! \hfill\break
Run straight forth! }

\par{9. 怒鳴りつけるのは失礼じゃん。 \hfill\break
Isn't shouting at someone rude? }

\par{10. 彼女の言葉は頭に焼きつけられた。 \hfill\break
Her words were burned into my mind. \hfill\break
 \hfill\break
11. 記憶に焼きついた。 \hfill\break
It\textquotesingle s branded into my memory. }

\par{12. 彼はふと押し付けた。 \hfill\break
He suddenly pushed it on me. \hfill\break
 \hfill\break
13. この靴を履きつけている。 \hfill\break
I'm always wearing these shoes. }

\par{14. 銀行は貸し金を貸し付けまくりました。 \hfill\break
The banks overly lent out loans in a fury. }
      
\section{~尽くす}
 
\par{  ${\overset{\textnormal{つ}}{\text{尽}}}$ くす either means "to administer" or "to work with all one's might". An important set phrase is ${\overset{\textnormal{ひつぜつ}}{\text{筆舌}}}$ に ${\overset{\textnormal{つ}}{\text{尽}}}$ くしがたい" meaning "indescribable". ~尽くす builds upon the latter usage to mean "to do\dothyp{}\dothyp{}\dothyp{}exhaustively". ~尽くす follows the 連用形. }

\par{15. お医者さんは ${\overset{\textnormal{ひなんみん}}{\text{避難民}}}$ のために尽くしています。 \hfill\break
The doctor is administering to the refugees. }

\par{16. 彼女は ${\overset{\textnormal{まわ}}{\text{周}}}$ りを走り尽くし続けた。 \hfill\break
She continued to exhaustively run through the neighborhood. }

\par{17. ${\overset{\textnormal{さいぜん}}{\text{最善}}}$ を尽くす。 \hfill\break
To do one's best. }

\par{18a. ${\overset{\textnormal{こうずい}}{\text{洪水}}}$ は ${\overset{\textnormal{まちぜんたい}}{\text{町全体}}}$ をなめ ${\overset{\textnormal{つ}}{\text{尽}}}$ くした。? \hfill\break
18b. 洪水は町全体を ${\overset{\textnormal{おそ}}{\text{襲}}}$ った。〇 \hfill\break
The flood wiped the entire town away. \hfill\break
Literally: The flood licked (away) the town completely. }

\par{\textbf{Sentence Note }: There is one problem with 18a. なめ尽くす is used with ${\overset{\textnormal{ほのお}}{\text{炎}}}$ ・ ${\overset{\textnormal{かじ}}{\text{火事}}}$ . For example, 山火事はふもとの町をなめ尽くした = The mountain fire licked away the town at the foot of the mountain. }

\begin{center}
\textbf{全て VS }\textbf{全部 VS }\textbf{全体 VS }\textbf{全員 VS }\textbf{総員 VS }\textbf{皆 }
\end{center}

\par{Since the last sentence used 全体, this is a perfect time to contrast these six very similar words. }

\begin{ltabulary}{|P|P|}
\hline 

 全て &  All (of something); everything \\ \cline{1-2}

全部 &  All as in altogether \\ \cline{1-2}

全体 &  The entirety of something \\ \cline{1-2}

全員 &  All members; everyone \\ \cline{1-2}

総員 &  Same as 全員 but more technical \\ \cline{1-2}

皆 &  Everyone; everything \\ \cline{1-2}

\end{ltabulary}

\par{ The differences are minor, but they can make a difference. 全部 suggests that there are individual parts. 全体 refers to all of a perimeter. }
      
\section{~立てる・つ}
 
\par{ 立てる and 立つ are the transitive and intransitive verb pairs for to stand. ~立てる shows that one does an action energetically. ~立つ shows that an emotion of some sort rises. }

\par{19. 僕の心が\{浮き立った・ときめいた\}。 \hfill\break
My heart was enlivened. }

\par{20. ${\overset{\textnormal{せいえん}}{\text{声援}}}$ に勇み立つ。 \hfill\break
To be encouraged by support. }
 
\par{21. 責め立てても無理だよ。 \hfill\break
It's useless even if you reproach him. }

\par{22. ${\overset{\textnormal{ほのお}}{\text{炎}}}$ が燃え立つ。 \hfill\break
For flames to blaze. }
 
\par{23. 記者を質問で攻め立てた。 \hfill\break
I bombarded the reporter with questions. }
      
\section{~こなす}
 
\par{ こなす may be used to mean "to sell out", "to handle easily", "to break to pieces\slash digest", or "to use a learned skill at will", and ~こなす means "to do\dothyp{}\dothyp{}\dothyp{}completely\slash well". }
 
\par{24. 馬を乗りこなす。 \hfill\break
To ride a horse well. }
 
\par{25. 彼女は舞踏会に着こなした。 \hfill\break
She was dressed stylishly at the ball. }
    