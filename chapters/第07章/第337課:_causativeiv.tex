    
\chapter{~しめる \& ~させ給う}

\begin{center}
\begin{Large}
第337課: ~しめる \& ~させ給う 
\end{Large}
\end{center}
 
\par{ ~しめる is an old-fashioned causative ending that is still used infrequently in Modern Japanese. Aside from this, the causative endings once were used to show extreme politeness, and this usage is preserved in old-fashioned speech. This lesson will be very short. So, feel grateful. }
      
\section{~しめる}
 
\par{ The auxiliary verb ~しめる is \textbf{limited }to only a handful of verbs which include する, なす, いう, ある, 知る, and じる\slash ずる-する verbs. This is not an exhaustive list, but due to its archaic nature, its use is very limited, and it is suggested that you only attempt to use it with verbs you see, how you see it used, and in formal writing. }

\par{1. ${\overset{\textnormal{かんきょう}}{\text{環境}}}$ を変化せしめる ${\overset{\textnormal{よういん}}{\text{要因}}}$ は ${\overset{\textnormal{おんしつこうか}}{\text{温室効果}}}$ ではない。(書き言葉) \hfill\break
The primary factor of changing the environment is not because of the greenhouse effect. }

\par{2. それは、国民の然らしめるところだ。(ちょっと硬い書き言葉) \hfill\break
That is a part where the citizens bring about it. }

\par{\textbf{Variant Note }: 然らしめる = そうさせる. }

\par{3. 日本語の素晴らしさを世界に知らしめよう。(ちょっと古風) \hfill\break
Let\textquotesingle s inform the world of the awesomeness of Japanese. }

\par{ に言わせ(れ)ば・に言わしめれば to mean "if you ask\dothyp{}\dothyp{}\dothyp{}". As expected, the latter is more literary. }

\par{4a. 私に言わせれば、その計画はいい考えじゃないと思います。 \hfill\break
4b. 私に言わしめれば、その計画はよい考えではないと思っております。(謙譲語: ちょっと古風) \hfill\break
If you ask me, I think that plan isn't a good idea. }

\par{ 知らしめる is occasionally used in Modern Japanese as a strongly nuanced way of saying 知らせる・認知させる. }

\par{5. ほら君はのまれている 無責任な言葉を知らしめているだけさ。 \hfill\break
Look at you overwhelmed; I'm just making irresponsible remark known. \hfill\break
From DIV作の「線路」 }

\par{  Another important combination is たらしめる, which is actually equivalent to にする and is the combination of the 未然形 of a classical yet very emphatic copula verb たり and ~しめる. This たり should not be confused with the particle たり, which has a very different origin. You will see this again in future conjugations and grammar points. }

\par{6. 我が社を世界のトップたらしめる。(Formal; literary) \hfill\break
We will make our company at the top of the world. }
 \textbf{読み物: ${\overset{\textnormal{むらやまだんわ}}{\text{村山談話}}}$ \textbf{の ${\overset{\textnormal{ばっすい}}{\text{抜粋}}}$ }}
\par{7. わが国は、遠くない過去の ${\overset{\textnormal{いちじき}}{\text{一時期}}}$ 、 ${\overset{\textnormal{こくさく}}{\text{国策}}}$ を ${\overset{\textnormal{あやま}}{\text{誤}}}$ り、 ${\overset{\textnormal{せんそう}}{\text{戦争}}}$ への道を ${\overset{\textnormal{あゆ}}{\text{歩}}}$ んで国民を ${\overset{\textnormal{そんぼう}}{\text{存亡}}}$ の ${\overset{\textnormal{きき}}{\text{危機}}}$ に ${\overset{\textnormal{おちい}}{\text{陥}}}$ れ、 ${\overset{\textnormal{しょくみんちしはい}}{\text{植民地支配}}}$ と ${\overset{\textnormal{しんりゃく}}{\text{侵略}}}$ によって、多くの ${\overset{\textnormal{くにぐに}}{\text{国々}}}$ 、とりわけアジア ${\overset{\textnormal{しょこく}}{\text{諸国}}}$ の人々に ${\overset{\textnormal{たい}}{\text{対}}}$ して ${\overset{\textnormal{ただい}}{\text{多大}}}$ の ${\overset{\textnormal{そんがい}}{\text{損害}}}$ と ${\overset{\textnormal{くつう}}{\text{苦痛}}}$ を与えました。私は、 ${\overset{\textnormal{みらい}}{\text{未来}}}$ に ${\overset{\textnormal{あやま}}{\text{誤}}}$ ち ${\overset{\textnormal{な}}{\text{無}}}$ から しめん とするが ${\overset{\textnormal{ゆえ}}{\text{故}}}$ に、 ${\overset{\textnormal{うたが}}{\text{疑}}}$ うべくもないこの ${\overset{\textnormal{れきし}}{\text{歴史}}}$ の事実を ${\overset{\textnormal{けんきょ}}{\text{謙虚}}}$ に受け ${\overset{\textnormal{と}}{\text{止}}}$ め、ここにあらためて ${\overset{\textnormal{つうせつ}}{\text{痛切}}}$ な ${\overset{\textnormal{はんせい}}{\text{反省}}}$ の意を表し、心からのお ${\overset{\textnormal{わ}}{\text{詫}}}$ びの気持ちを ${\overset{\textnormal{ひょうめい}}{\text{表明}}}$ いたします。また、この歴史がもたらした ${\overset{\textnormal{ないがい}}{\text{内外}}}$ すべての ${\overset{\textnormal{ぎせいしゃ}}{\text{犠牲者}}}$ に深い ${\overset{\textnormal{あいとう}}{\text{哀悼}}}$ の ${\overset{\textnormal{ねん}}{\text{念}}}$ を ${\overset{\textnormal{ささ}}{\text{捧}}}$ げます。 }

\par{ Our country caused great pain and damage to many countries, particularly to all peoples of Asia, by colonial rule and invasion, driving citizens to a state of life and death crisis by mistaking national policy and walking down a road of war in a period of the not so distant past. In order to rid such mistakes in the future, I humbly accept these irrefutable facts of history, express here again bitter remorse and my heartfelt apology. Again, please accept my sincere and deep condolences to all victims this history brought. }

\par{ This excerpt from an apology for past human violations by Japan in World War II by ex. Prime Minister Murayama provides insight to well-crafted writing. The passage begins with the Classical attribute marker function of が. 無からしめんとする is a combination 無い, -しめる, the Classical volitional auxiliary verb -む, and とする. Together, the phrase is equivalent to なくさせようとする, which is translated in the passage with が故に as "in order to rid". }

\par{ \textbf{Formal, Semi-Classical Text }}

\par{ Below is an excerpt from a piece by 川端康成. This is a report on the mental health of a man who has been charged with the murder of two women. Here below is part of the report, and notice how せしめる used to be せしむる in the 連体形. }

\par{8. 「 ${\overset{\textnormal{ちのうけんさ}}{\text{智能検査}}}$ の成績は ${\overset{\textnormal{きわ}}{\text{極}}}$ めて ${\overset{\textnormal{ゆうりょう}}{\text{優良}}}$ を示している。一般知識、計算能力、論理選択、正文、 ${\overset{\textnormal{じゅうてん}}{\text{充填}}}$ 、構文、定義について行いたる成績は、全然標準点と同一である。 }

\par{記憶力は自己の ${\overset{\textnormal{けいれき}}{\text{経歴}}}$ につき誤りなく、記名力の試験は表に示す ${\overset{\textnormal{ごと}}{\text{如}}}$ く、対語試験に ${\overset{\textnormal{おい}}{\text{於}}}$ て全て正当数を得、 ${\overset{\textnormal{こと}}{\text{殊}}}$ に無関係対語試験に於てさえ完全に再生することを得るは、 ${\overset{\textnormal{けだ}}{\text{蓋}}}$ し優良なるものと言わなければならない。 \hfill\break
故に記憶、記名の ${\overset{\textnormal{しょうぎ}}{\text{障礙}}}$ は認められず、従って記憶を ${\overset{\textnormal{げんたい}}{\text{減退}}}$ \textbf{せしむる }精神的原因を有しないことを証明するのである。\dothyp{}\dothyp{}\dothyp{}」 \hfill\break
"He has shown great excellence in the I.Q test. In his assessments carried out in general knowledge, numeric ability, logical choice, official text, loading artillery, syntax, and defining, he completely meets the standard marks on all these things. \hfill\break
There is no mistake in his memory about his own record, and his signature test was as like it appeared on the front, and he got all the right numbers in the antonyms test, and given that he completely play backed his performance on the rather unrelated antonyms test, one would have to say that he is indeed superb. \hfill\break
Thereupon, no handicaps can be found in his memory or inscription, so we testify that he doesn't have a mental cause letting down his memory\dothyp{}\dothyp{}\dothyp{}." \hfill\break
From 散りぬるを by 川端康成. }

\par{\textbf{Word Note }: Official text (正文) would have been written in 漢文 at the time. }

\par{\textbf{Grammar Notes }: }

\par{1. As you should expect, 得るは would be 得るのは in more modern Japanese. At one time, the 連体形 could allow verbs to be used as nominals just as in the context above. }

\par{2. Notice also that the 連体形 of ~た used to be たる rather than just た. }
      
\section{Extreme Politeness}
 
\par{ ~させる, ~せる, and ~しめる may show extreme politeness. This usage is \textbf{very rare }, and it is seen frequently in \textbf{classical }texts. They are often followed by - ${\overset{\textnormal{たま}}{\text{給}}}$ う. In Classical Japanese, 給う is spelled as 給ふ. This is not something you have to remember, but \textbf{old }expressions\slash patterns like this can easily be found in set phrases such as the second example. }

\par{9. ${\overset{\textnormal{だいとうりょう}}{\text{大統領}}}$ も ${\overset{\textnormal{さわ}}{\text{騒}}}$ が \textbf{せ ${\overset{\textnormal{たま}}{\text{給}}}$ う }。(とても古風) \hfill\break
The president was also in a panic. }

\par{10. み心のままになさ \textbf{しめ給へ }。(Classical) \hfill\break
Do at your heart's desire. }

\par{${\overset{\textnormal{}}{\text{11. 神}}}$ の ${\overset{\textnormal{}}{\text{与}}}$ えたもうた ${\overset{\textnormal{}}{\text{試練}}}$ 。 \hfill\break
A trial from God。 }

\par{\textbf{Sound Change Note }: ~たまひたり \textrightarrow  ~たまうた \textrightarrow  ~たもうた. }

\par{12. ${\overset{\textnormal{おとど}}{\text{大臣}}}$ の ${\overset{\textnormal{おんさと}}{\text{御里}}}$ に ${\overset{\textnormal{げんじ}}{\text{源氏}}}$ の ${\overset{\textnormal{きみ}}{\text{君}}}$ まかで \textbf{させ ${\overset{\textnormal{たま}}{\text{給}}}$ ふ }。(Classical) \hfill\break
Lord Genji left for the home of the minister. \hfill\break
From the 源氏物語. }
    