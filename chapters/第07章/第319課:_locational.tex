    
\chapter{Locational}

\begin{center}
\begin{Large}
第319課: Locational: ~渡る\slash 渡す, ~出る, ~上がる\slash 上げる, ~入る\slash 入れる, ~下がる\slash 下げる, ~おりる\slash おろす, ~落ちる\slash 落とす, \& ~回る\slash 回す 
\end{Large}
\end{center}
 
\par{ This lesson is about the compound verb endings of direction. Not everything is common, but it is important for you to understand things that you may come across. }
      
\section{~渡る}
 
\par{ One of the most unique things about the verb 渡る is its usage in に渡せられる・渡らせ給う, which are rare, literary alternative ways to express ある and いる in honorific  speech. }
1. あの人は美男に渡らせられる。(Literary and old-fashioned) \hfill\break
He is a beautiful man.  
\par{In compound verbs after the  連用形, ~渡る is used to show the extent of a perimeter over a large  area. When used in a transitive sense for the same purpose, it is replaced with its transitive form 渡す. }
 
\par{2. 空が晴れ渡った。 \hfill\break
The sky is clear. }

\par{3. ${\overset{\textnormal{しろ}}{\text{城}}}$ を明け渡すな。 \hfill\break
Do not surrender the castle. }

\par{4. ${\overset{\textnormal{あくま}}{\text{悪魔}}}$ に ${\overset{\textnormal{たましい}}{\text{魂}}}$ を売り渡すのは ${\overset{\textnormal{おろ}}{\text{愚}}}$ かことだ。 \hfill\break
Selling your soul to the devil is a foolish thing to do. }

\par{5. ${\overset{\textnormal{はんじ}}{\text{判事}}}$ は判決をそろそろ言い渡すだろう。 \hfill\break
The judge will probably pass a sentence soon. }
 
\par{6. インターネット中に知れ渡っている。 \hfill\break
To be well-known throughout the internet. }

\par{7. ${\overset{\textnormal{しゅくほう}}{\text{祝砲}}}$ が\{鳴り渡る・ ${\overset{\textnormal{とどろ}}{\text{轟}}}$ く\}。 \hfill\break
For gun salutes to resound. }
      
\section{~出る}
 
\par{ ~出る expresses that something is going out and is, with few exceptions, intransitive. Unlike ~出す which expresses great emphasis that doesn't have to infer direction, ~出る always does. }

\par{8. ${\overset{\textnormal{いたい}}{\text{遺体}}}$ から ${\overset{\textnormal{つ}}{\text{突}}}$ き出ている ${\overset{\textnormal{やいば}}{\text{刃}}}$ \hfill\break
 A sword protruding from a body }

\par{9. ${\overset{\textnormal{あふ}}{\text{溢}}}$ れ出る血が土に ${\overset{\textnormal{た}}{\text{溜}}}$ まった。 \hfill\break
The flowing blood pooled on the ground. }
 
\par{10. 涙が ${\overset{\textnormal{わ}}{\text{湧}}}$ き出た。 \hfill\break
Tears welled up. }
 
\par{11. 彼は外へ走り出た。 \hfill\break
He ran outside. }

\par{12. ${\overset{\textnormal{かいてい}}{\text{海底}}}$ から ${\overset{\textnormal{せきゆ}}{\text{石油}}}$ が ${\overset{\textnormal{う}}{\text{浮}}}$ き出た。 \hfill\break
Oil stood out from the sea floor. }
 
\par{13. 目の ${\overset{\textnormal{たま}}{\text{玉}}}$ が飛び出るような値段だよ。 \hfill\break
It's a price that can make your eyeballs pop out! }
      
\section{~上がる \& -上げる}
 
\par{ 上がる and 上げる may be used in compound verbs to show that things are being lifted\slash raised. As we have learned, 上がる is intransitive and 上げる is transitive. ~上がる and ~上げる may also show the completion of something. }

\par{14. 浮き上がっているように見えませんか。 \hfill\break
Does it not appear that they're standing out? (As in being visible) }

\par{15. そんなにでかすぎる岩持ち上げれるのかい?(Rude) \hfill\break
Can you even lift such a humongous rock? \hfill\break
}

\par{16. スーツケースを引き上げる。 \hfill\break
To lug suitcases up. \hfill\break
}

\par{17. 彼はボールを投げ上げた。 \hfill\break
He threw the ball up into the air. \hfill\break
}

\par{18a. メモに書き上がる。(More formal) \hfill\break
18b. メモに書いてある。 \hfill\break
To be written up in a note. }

\par{19. ${\overset{\textnormal{しょうひん}}{\text{商品}}}$ の値段を ${\overset{\textnormal{お}}{\text{押}}}$ し上げる。 \hfill\break
To boost product prices. }

\par{20. ${\overset{\textnormal{しゃめん}}{\text{斜面}}}$ を ${\overset{\textnormal{は}}{\text{這}}}$ い上がる。 \hfill\break
To crawl up a slope. }

\par{21. ${\overset{\textnormal{ぶき}}{\text{武器}}}$ を取って立ち上がる。 \hfill\break
To take up arms. }
 
\par{22. やれやれ、新たな難問が持ち上がったのー。(Old person) \hfill\break
Oh my, a new problem has come up. }
      
\section{~入る \& ~入れる}
 
\par{  ~入る and ~入れる are the intransitive and transitive verbal pairs for "to enter". In compound verbs they refer to something completely done or done deeply or ardently. }

\par{23. ${\overset{\textnormal{みかた}}{\text{味方}}}$ に引き入れる。 \hfill\break
To win over. }
 
\par{24. 恐れ入りました。 \hfill\break
I am much obliged. }
 
\par{25. 痛み入る。 \hfill\break
To be greatly obliged. }

\par{26. ${\overset{\textnormal{みごと}}{\text{見事}}}$ な能力に感じ入る。 \hfill\break
To be greatly impressed by astounding abilities. }

\par{27. ${\overset{\textnormal{は}}{\text{恥}}}$ じ入りました。 \hfill\break
I was ashamed. }

\par{28. ${\overset{\textnormal{た}}{\text{絶}}}$ え入るような顔 \hfill\break
A dead looking face }
 
\par{29.  私たちは買い入れるペースを ${\overset{\textnormal{きゅうげき}}{\text{急激}}}$ に上げています。 \hfill\break
We are suddenly raising the pace of purchasing. }
 
\par{30. 赤ちゃんが寝入っているうちに両親は買い物に行った。 \hfill\break
The parents went shopping while the baby was in a deep slumber. }
 
\par{31. 立ち入ったことを聞くようだが。 \hfill\break
If I'm not being too inquisitive. }
 
\par{32. 立ち入り禁止 \hfill\break
Keep Out }

\par{33. ${\overset{\textnormal{くうらん}}{\text{空欄}}}$ に書き入れる。 \hfill\break
To fill in blank spaces. }
 
\par{34. 意見を聞き入れる。 \hfill\break
To accede an opinion. }
 
\par{35. 新しい ${\overset{\textnormal{ほうしん}}{\text{方針}}}$ を取り入れるつもりはない。 \hfill\break
There is no plan to bring in a new policy. }
      
\section{~下がる \& ~下げる}
 
\par{ 下がる and 下げる are the verbal pairs for "to hang down\slash fall" and are intransitive and transitive respectively. The nuances of the verbs that are created with ~下がる and ~下げる are negative. }

\par{36. あいつは引き下がるだろう。 \hfill\break
He will probably back down. }
 
\par{37. 彼らは\{ ${\overset{\textnormal{そがん}}{\text{訴願}}}$ ・ ${\overset{\textnormal{ふふく}}{\text{不服}}}$ 申し立て・ ${\overset{\textnormal{いぎ}}{\text{異議}}}$ 申し立て\}を引き下げた。 \hfill\break
They pulled down the petition. }

\par{38. ${\overset{\textnormal{た}}{\text{垂}}}$ れ下がってる ${\overset{\textnormal{つらら}}{\text{氷柱}}}$ は明日落ちるだろうな。 \hfill\break
The hanging icicles will probably fall tomorrow. }
 
\par{39a. 言ったことを取り下げるはずだ。? \hfill\break
39b. 言ったことを\{取り消す ${\overset{\textnormal{てっかい}}{\text{撤回}}}$ する\}はずだ。〇 \hfill\break
You should retract what you said. }
 
\par{\textbf{Word Note }: 取り下げる should be used in reference to things like documents\slash written statements. }
      
\section{~おりる \& ~おろす}
 
\par{  Meaning "to go down\slash drop", おりる and おろす are used in compound expressions to show a sense of going down, but a negative emphasis is not always the case. おりる is typically written as 降りる and おろす is typically written as 下ろす. }

\par{40. ${\overset{\textnormal{ぶたい}}{\text{舞台}}}$ から引き下ろす。 \hfill\break
To pluck someone done from a stage. }
 
\par{41. 東京を見下ろす。 \hfill\break
To look down at Tokyo. \hfill\break
 }

\par{\textbf{Word Note }: 見下ろす can be used literally and metaphorically. }
 
\par{42. 私の友達は小説を書き下ろしました。 \hfill\break
My friend wrote a new novel. }
 
\par{\textbf{Nuance Note }: This suggests that your friend is a professional novelist. }

\par{43a. ${\overset{\textnormal{さんちょう}}{\text{山頂}}}$ から ${\overset{\textnormal{か}}{\text{駆}}}$ け降りる。 \hfill\break
43b. 山を駆け下りる。 \hfill\break
To run down the peak of a mountain. }

\par{44. ${\overset{\textnormal{ま}}{\text{舞}}}$ い降りてきた。 \hfill\break
It came flying down. }
 
\par{45. 強風が吹き下ろしていた。 \hfill\break
A strong wind was blowing down. }
      
\section{~落ちる \& ~落とす}
 
\par{ 落ちる and 落とす are the verbal pairs for "to fall\slash drop" and are intransitive and transitive respectfully. These verbs show a downward sense of direction in a compound verb. }

\par{46. この世に生れ落ちなくてよかったのに。 \hfill\break
If only I had not been born in this world. }
 
\par{47. 焼け落ちるのにほんの五分もかからなかったぞ。 \hfill\break
It only took a mere five minutes to burn down. }
 
\par{48. 俺の皿から ${\overset{\textnormal{すべ}}{\text{滑}}}$ り落ちちまった。(Mature male; vulgar) \hfill\break
It slipped from my plate. }
 
\par{49. 彼が水を ${\overset{\textnormal{はら}}{\text{払}}}$ い落とした。 \hfill\break
He shook off the water. }
 
\par{50. \{見落とされた・見過ごされた\}のかな。 \hfill\break
I wonder if it was overlooked. }
 
\par{51. 銀行口座から引き落とす。 \hfill\break
To charge from one's bank account. }

\par{52. ${\overset{\textnormal{くだもの}}{\text{果物}}}$ を振り落とそう。 \hfill\break
Let's shake some fruit off. }
 
\par{53. 家を ${\overset{\textnormal{がけ}}{\text{崖}}}$ から ${\overset{\textnormal{つ}}{\text{突}}}$ き落とすのは ${\overset{\textnormal{あぶ}}{\text{危}}}$ ないよ。 \hfill\break
Pushing a house off of a cliff is dangerous. }
      
\section{~回る \& ~回す}
 
\par{ 回る and 回す are the verbal pairs for "to turn\slash rotate" and are intransitive and transitive respectively. When used in compounds, they show movement while doing something. ~回る can be seen attached to both intransitive and transitive verbs, but ~回す is limited to transitive verbs. It is to be expected that a nuance difference is present when both ~回る and ~回す can be used. You will see this more clearly in the examples below. }

\par{54. あちこちを駆け回る。 \hfill\break
To run around here and there. }
 
\par{55. 仕事を探し回るつもりがあるのか。 \hfill\break
Do you have any intentions of looking about for a job? }
 
\par{56. 自分の周りを見回した方がよいです。(ちょっと古風; Formal) \hfill\break
It is best to look around one's surroundings. }
 
\par{57. 私は新車を ${\overset{\textnormal{ちゅうしゃじょう}}{\text{駐車場}}}$ に乗り回したいです。 \hfill\break
I want to ride around a new car in a parking lot. }
 
\par{${\overset{\textnormal{}}{\text{58. 警備員}}}$ は ${\overset{\textnormal{りんばん}}{\text{輪番}}}$ で見回ります。 \hfill\break
The guards patrol in rotations. }
 
\par{\textbf{Nuance Note }: This infers that there are a lot of guards patrolling in rotation. You may consider using 交代で instead of  輪番で. }
 
\par{${\overset{\textnormal{}}{\text{59. 犯人}}}$ を ${\overset{\textnormal{お}}{\text{追}}}$ い回す。 \hfill\break
To chase about a criminal. }
    