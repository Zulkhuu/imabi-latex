    
\chapter{擬声語 IV}

\begin{center}
\begin{Large}
第339課: 擬声語 IV: From Chinese 
\end{Large}
\end{center}
 
\par{ There are some onomatopoeia that have entered Japanese through Chinese. They are typically rare, but they feature the mouth radical. The reason why they deserve special attention is that you may not realize at first hand that they're onomatopoeic. }
      
\section{Onomatopoeia from Chinese}
   In Chinese the large majority of onomatopoeic expressions are made up of 漢字 with the mouth radical 口. Individual characters are often repeated in the same way こそ is repeated in こそこそ. There are also many instances when the characters may be different, but the readings are still very similar. At other times, there isn't any apparent symmetry. However, if we think of English expressions like "kaboom", this doesn't have to be a requirement for something to be onomatopoeia.  
\par{ Furthermore, we must understand that things Sino-Japanese tend to be literary. Because native onomatopoeia is so natural of spoken speech, we can expect Chinese onomatopoeic expressions to enter Japanese as rather technical words and may not quite be used or viewed as onomatopoeia. Some readily are viewed as such as seen in Ex. 1 and 2. }

\par{ Nevertheless, it is certain that you will hardly find most of these spoken. These words are important, though, in higher class literature in which even today new Sino-Japanese expressions may be imported. For you as the advanced learner, running into new characters in this section may very well be the most beneficial and important thing to this lesson. }
 
\begin{center}
\textbf{Examples } 
\end{center}

\par{${\overset{\textnormal{ここ}}{\text{1. 呱呱}}}$ の ${\overset{\textnormal{こえ}}{\text{声}}}$ をあげる。 \hfill\break
To make one's first cry. }
 
\par{2. \{ ${\overset{\textnormal{かか}}{\text{呵呵}}}$ と・ ${\overset{\textnormal{おおごえ}}{\text{大声}}}$ で\} ${\overset{\textnormal{わら}}{\text{笑}}}$ い ${\overset{\textnormal{つづ}}{\text{続}}}$ ける。 \hfill\break
To continue to laugh loudly. }
 
\par{3. ${\overset{\textnormal{りゅうりょう}}{\text{嚠喨}}}$ たる音色 \hfill\break
Clear and resounding tone\slash sound }
 
\par{4. 高評 \textbf{${\overset{\textnormal{さくさく}}{\text{嘖嘖}}}$ }\hfill\break
The resounding of high reputation }
 
\par{5. ${\overset{\textnormal{きき}}{\text{嘻嘻}}}$ として遊ぶ。 \hfill\break
To play cheerfully. }
 
\par{6. ${\overset{\textnormal{りょうりょう}}{\text{喨々}}}$ たる響き \hfill\break
A reverberating echo }
 
\par{7. ${\overset{\textnormal{なんなん}}{\text{喃々}}}$ と ${\overset{\textnormal{だべん}}{\text{駄弁}}}$ を ${\overset{\textnormal{ろう}}{\text{弄}}}$ する。 \hfill\break
To babble and speak rubbish. }
 
\par{8. ${\overset{\textnormal{きこくしゅうしゅう}}{\text{鬼哭啾啾}}}$  \hfill\break
The groaning of a dead spirit }
 
\par{9. ${\overset{\textnormal{とつとつ}}{\text{吶々}}}$ \hfill\break
Faltering }
 
\par{ Next are examples that really seem to have lost any sense of being onomatopoeic in Japanese. But if you think about some of them, you can still picture how they may relate to sound. For instance, 丁寧 (polite), which you will see has another spelling resembling the rest of these words, itself emulates the 'sound' of polite speech. }
 
\par{10. ${\overset{\textnormal{そしゃく}}{\text{咀嚼}}}$ \hfill\break
Mastication }
 
\par{11. ${\overset{\textnormal{ほうこう}}{\text{咆哮}}}$ をあげる。 \hfill\break
To let out a roar. }

\par{12. 叮嚀・丁寧 \hfill\break
Polite }
 
\par{13. ${\overset{\textnormal{かしゃく}}{\text{呵責}}}$ ・呵嘖 \hfill\break
Accusation }
 
\par{14. ${\overset{\textnormal{たんか}}{\text{啖呵}}}$ を切る。 \hfill\break
To declare defiantly. }
 
\par{15. ${\overset{\textnormal{あうん}}{\text{阿吽}}}$ の呼吸 \hfill\break
The harmonizing, mentally and physically, of two parties engaged in an activity }

\par{\textbf{Etymology Note }: 阿吽 in the last example actually ultimately derives from the interjection Ohm from Sanskrit.  }
    