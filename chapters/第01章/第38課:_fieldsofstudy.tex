    
\chapter{Fields of Study}

\begin{center}
\begin{Large}
第38課: Fields of Study 
\end{Large}
\end{center}
 
\par{ This lesson will be about how to express fields of study in Japanese. We will be utilizing this opportunity to take a break from pure grammar studies to expand upon vocabulary in an easy and productive way. }

\par{\textbf{Curriculum Note }: This lesson as recently been truncated to simply be about fields of study. As such, it is subject to its own remodeling in the near future. }
      
\section{Fields of Study 専門(せんもん)}
 
\par{ Many fields of study end with the suffix - \emph{gaku }学, which equates to "the study of."  It also corresponds to the English suffix -ology. Of course, not all fields of study end in this. Below are some of the most common fields of studies: }

\par{
\begin{ltabulary}{|P|P|P|P|P|P|}
\hline 

医学 & いがく & Medical science & 科学 & かがく & Science \\ \cline{1-6}

化学 & かがく & Chemistry & 数学 & すうがく & Mathematics \\ \cline{1-6}

物理学 & ぶつりがく & Physics & 音楽 & おんがく & Music \\ \cline{1-6}

建築学 & けんちくがく & Architecture & 工学 & こうがく & Engineering \\ \cline{1-6}

電気工学 & でんきこうがく & Electrical engineering & 土木工学 & どぼくこうがく & Civil engineering \\ \cline{1-6}

機械工学 & きかいこうがく & Mechanical engineering & 遺伝子工学 & いでんしこうがく & Genetic engineering \\ \cline{1-6}

電子工学 & でんしこうがく & Electronics & 哲学 & てつがく & Philosophy \\ \cline{1-6}

言語学 & げんごがく & Linguistics & 社会学 & しゃかいがく & Sociology \\ \cline{1-6}

法律学 & ほうりつがく & Law & 経済学 & けいざいがく & Economics \\ \cline{1-6}

生物学 & せいぶつがく & Biology & 地学 & ちがく & Geology \\ \cline{1-6}

教育学 & きょういくがく & Education & 体育学 & たいいくがく & Physical education \\ \cline{1-6}

農学 & のうがく & Agriculture & 心理学 & しんりがく & Psychology \\ \cline{1-6}

美術 & びじゅつ & Fine arts & 芸術 & げいじゅつ & Art \\ \cline{1-6}

経営学 & けいえいがく & Business administration & 文学 & ぶんがく & Literature \\ \cline{1-6}

天文学 & てんもんがく & Astronomy & 宗教学 & しゅうきょうがく & Theology \\ \cline{1-6}

薬学 & やくがく & Pharmachology & 生化学 & せいかがく & Biochemstry \\ \cline{1-6}

環境科学 & かんきょうかがく & Environmental science & コンピューター科学 & コンピューターかがく & Computer science \\ \cline{1-6}

地理学 & ちりがく & Geography & 国際関係学 & こくさいかんけいがく & International relations \\ \cline{1-6}

\end{ltabulary}
}

\par{\textbf{Word Note }: Science is ${\overset{\textnormal{かがく}}{\text{科学}}}$ . In reference to courses, it's ${\overset{\textnormal{りか}}{\text{理科}}}$ . "Science department" is ${\overset{\textnormal{りがくぶ}}{\text{理学部}}}$ . "Science course" is ${\overset{\textnormal{りかけい}}{\text{理科系}}}$ . 理系, an abbreviation, is used in expressions such as 理系の ${\overset{\textnormal{がくせい}}{\text{学生}}}$ . 理学部の学生 is a "science major". }
    