    
\chapter*{Expressions of Apology}

\begin{center}
\begin{Large}
第??課: Expressions of Apology 
\end{Large}
\end{center}
 \hfill\break
 Apologizing to others is one of the most sensitive things that we do on a daily basis. Without proper grace and etiquette as well as sensitivity to the matter at hand, apologies can be interpreted as insincere. As such, more so than even the words that go into an apology, the way one conducts oneself is what's most important. Nevertheless, Japanese places a lot of intricacy into how one apologies.   In the phrases introduced in this lesson, a lot of complexity will be had in differences among speech styles. This inevitably means confronting grammar that hasn't been fully introduced up to this point. For the purpose of this lesson, though, focus on the phrases that center around apologizing. 
\par{\textbf{Grammar Note }: You will notice the prefix \emph{o\slash go }- お・ご in front of many phrases discussed in this lesson. This prefix is an honorific marker which helps make what it attaches to be more respectful. Much later on, we will learn how to use this constructively. }
 \textbf{Notation Note }: High pitch morae are marked in bold and pitch falls are denoted by ↓.        
\section{Apologizing: O-wabi お詫び}
 
\par{ The basic translation of “I\textquotesingle m sorry” in Japanese is \emph{sumimasen }すみません. This word can also be used to mean “excuse me.” When the apology is for something that occurred in the past, you use \emph{sumimasendeshita }すみませんでした. Many speakers drop the first \slash m\slash  in the phrase, resulting in \emph{suimasen }すいません. In plain speech, this can be seen as \emph{sumanai }すまない or \emph{suman }すまん. To organize its conjugations together, we get the following. }

\begin{ltabulary}{|P|P|}
\hline 

Plain Non-Past & Polite Non-Past \\ \cline{1-2}

すまない・すまん \hfill\break
\emph{Sumanai\slash suman }& すみません・すいません \hfill\break
\emph{Sumimasen\slash suimasen }\\ \cline{1-2}

Plain Past & Polite Past \\ \cline{1-2}

すまなかった・すまんかった \hfill\break
\emph{Sumanakatta\slash sumankatta }& すみませんでした・すいませんでした \hfill\break
\emph{Sumimasendeshita\slash suimasendeshita }\\ \cline{1-2}

\end{ltabulary}

\par{\textbf{Etymology Note }: This word comes from the negative form of the verb \emph{sumu }済む. In this expression, the meaning of “to feel at ease” is at play he re. Essentially, the speaker is guilty for what\textquotesingle s going on. }

\par{\textbf{Intonation Notes }: \hfill\break
1. す \textbf{ま }ない. \hfill\break
2. す \textbf{みませ }ん. }

\par{1. すみません、すいません。 \hfill\break
 \emph{Sumimasen, suimasen. \hfill\break
 }Sorry, sorry. \hfill\break
Excuse me, excuse me. }

\par{2. すみません、 ${\overset{\textnormal{とお}}{\text{通}}}$ してください。 \hfill\break
 \emph{Sumimasen, t }\emph{ōshite kudasai. \hfill\break
 }Excuse me, could you let me through? }

\par{3. ${\overset{\textnormal{まちが}}{\text{間違}}}$ えてしまってすみません。 \hfill\break
 \emph{Machigaete shimatte sumimasen. \hfill\break
 }I\textquotesingle m sorry for messing up. }

\par{4. ${\overset{\textnormal{こんらん}}{\text{混乱}}}$ させて(しまって)すみません。 \hfill\break
 \emph{Konran sasete (shimatte) sumimasen. \hfill\break
 }I\textquotesingle m sorry for confusing you. }

\par{5. すみません、お ${\overset{\textnormal{じかん}}{\text{時間}}}$ よろしいですか。 \hfill\break
 \emph{Sumimasen, o-jikan yoroshii desu ka? \hfill\break
 }Excuse me, but could I take a little bit of your time? }

\par{6. すいません、お ${\overset{\textnormal{かんじょう}}{\text{勘定}}}$ ! \hfill\break
 \emph{Suimasen, o-kanj }\emph{ō! \hfill\break
 }Excuse me! Check, please! }

\par{7. ${\overset{\textnormal{たいへん}}{\text{大変}}}$ すみませんでした。 \hfill\break
 \emph{Taihen sumimasendeshita. \hfill\break
 }I\textquotesingle m terribly sorry for what had happened. }

\par{8. あ、\{すまない・すまん\}。 \hfill\break
 \emph{A, [sumanai\slash suman]. \hfill\break
 }Oh, sorry. }

\begin{center}
\textbf{\emph{Shitsurei }失礼 }
\end{center}

\par{ The next phrase involving apologies to look at is the word \emph{shitsurei }失礼, a noun\slash adjectival noun meaning "impoliteness\slash discourtesy." By itself, it can be used to mean “excuse me” when leaving, but it is usually seen as \emph{shitsurei shimasu }失礼します in this regard. In the past tense as \emph{shitsurei shimashita }失礼しました, the phrase is used a lot for apologizing for what one has done. To summarize its conjugations, they would be organized as such. }

\begin{ltabulary}{|P|P|P|}
\hline 

Plain Non-Past & Polite Non-Past & Humble Non-Past \\ \cline{1-3}

 \emph{Shitsurei (suru) }失礼(する) &  \emph{Shitsurei shimasu }失礼します &  \emph{Shitsurei itashimasu }失礼いたします \\ \cline{1-3}

Plain Past & Polite Past & Humble Past \\ \cline{1-3}

 \emph{Shitsurei shita }失礼した &  \emph{Shitsurei shimashita }失礼しました &  \emph{Shitsurei itashimashita }失礼いたしました \\ \cline{1-3}

\end{ltabulary}

\par{\textbf{Grammar Note }: We will look at its use in the non-past tense more closely later in this lesson. }

\par{\textbf{Intonation Note }: The intonation of \emph{shitsurei }失礼 is し \textbf{つ }れい. }

\par{9a. ${\overset{\textnormal{ちょうぶんしつれい}}{\text{長文失礼}}}$ しました。 \hfill\break
9b. ${\overset{\textnormal{ちょうぶん}}{\text{長文}}}$ すみません。 \hfill\break
 \emph{Ch }\emph{ōbun shitsurei shimashita. \hfill\break
Ch }\emph{ōbun sumimasen. }\hfill\break
I apologize for the long message. }

\par{10. ${\overset{\textnormal{しつれい}}{\text{失礼}}}$ しました。 ${\overset{\textnormal{もう}}{\text{申}}}$ し ${\overset{\textnormal{わけ}}{\text{訳}}}$ ありません。 \hfill\break
 \emph{Shitsurei shimashita. M }\emph{ō }\emph{shiwake arimasen. \hfill\break
}I was rude. I\textquotesingle m terribly sorry. }

\begin{center}
\textbf{\emph{Mōshiwake arimasen }申し訳ありません }
\end{center}

\par{ In Ex. 10, yet another expression for “sorry” was used: \emph{mōshiwake arimasen }申し訳ありません. This literally means “have no excuse.” It is more formal than \emph{sumimasen }すみません, but it can in fact be altered to be used in any speech style. }

\begin{ltabulary}{|P|P|P|P|}
\hline 

Plain & Casual Polite & Polite & Humble \\ \cline{1-4}

申し訳ない \hfill\break
\emph{Mōshiwake nai }& 申し訳ないです \hfill\break
\emph{Mōshiwake nai desu }& 申し訳ありません \hfill\break
 \emph{Mōshiwake arimasen }& 申し訳ございません \hfill\break
\emph{Mōshiwake gozaimase }\emph{n }\\ \cline{1-4}

\end{ltabulary}

\par{\textbf{Grammar Note }: Alternatively, there also exists \emph{mōshiwake naku zonjimasu }申し訳なく存じます, which is the most formal one can make this expression. This adds a sense of “feeling apologetic” on top of actually making an apology. }

\par{ Of course, these expressions all have past tense forms that are used whenever one\textquotesingle s impolite act occurred in the past. }

\begin{ltabulary}{|P|P|P|P|}
\hline 

Plain & Casual Polite & Polite & Humble \\ \cline{1-4}

申し訳なかった \hfill\break
\emph{Mōshiwake nakatta }& 申し訳なかったです \hfill\break
\emph{Mōshiwake nakatta desu }& 申し訳ありませんでした \hfill\break
\emph{Mōshiwake arimasendeshit }\emph{a }& 申し訳ございませんでした \hfill\break
\emph{Mōshiwake gozaimasendeshita }\\ \cline{1-4}

\end{ltabulary}

\par{11. お ${\overset{\textnormal{いそが}}{\text{忙}}}$ しいところ、 ${\overset{\textnormal{たいへんもう}}{\text{大変申}}}$ し ${\overset{\textnormal{わけ}}{\text{訳}}}$ ございません。 \hfill\break
\emph{O-isogashii tokoro, taihen m }\emph{ōshiwake gozaimasen. \hfill\break
}I deeply apologize for this while you\textquotesingle re busy. }

\par{12. ${\overset{\textnormal{せんじつ}}{\text{先日}}}$ の ${\overset{\textnormal{ぼうねんかい}}{\text{忘年会}}}$ では、 ${\overset{\textnormal{よ}}{\text{酔}}}$ いに ${\overset{\textnormal{まか}}{\text{任}}}$ せて ${\overset{\textnormal{たいへん}}{\text{大変}}}$ な(ご) ${\overset{\textnormal{ぶれい}}{\text{無礼}}}$ をして、 ${\overset{\textnormal{もう}}{\text{申}}}$ し ${\overset{\textnormal{わけ}}{\text{訳}}}$ ありませんでした。 \hfill\break
 \emph{Senjitsu no b }\emph{ōnenkai de wa, yoi ni maka }\emph{sete taihen na (go-)burei wo shite, m }\emph{ōshiwake arimasendeshita. \hfill\break
 }At the year-end party the other day, I let alcohol get the best of me, in which I was very out-of-place, and I deeply apologize. \hfill\break
 \textbf{\hfill\break
}\textbf{Grammar Note }: \emph{Burei }無礼 is a noun\slash adjectival noun which means “impoliteness” just like \emph{shitsurei }失礼. Here, it\textquotesingle s used in its own verbal construct in \emph{go-burei wo suru }ご無礼をする. Some feel that the use of the honorific prefix \emph{go }- ご is inappropriate in this construct, and so many speakers omit it from this phrase. }

\par{13. ${\overset{\textnormal{たびかさ}}{\text{度重}}}$ なる ${\overset{\textnormal{しつれい}}{\text{失礼}}}$ 、 ${\overset{\textnormal{たいへんもう}}{\text{大変申}}}$ し ${\overset{\textnormal{わけ}}{\text{訳}}}$ ございませんでした。 \hfill\break
 \emph{Tabikasanaru shitsurei, taihen m }\emph{ōshiwake gozaimasendeshita. \hfill\break
 }I am terribly sorry for how I\textquotesingle ve repeatedly been discourteous. }

\par{14. お ${\overset{\textnormal{きゃくさま}}{\text{客様}}}$ には ${\overset{\textnormal{たいへん}}{\text{大変}}}$ ご ${\overset{\textnormal{めいわく}}{\text{迷惑}}}$ をお ${\overset{\textnormal{か}}{\text{掛}}}$ けして ${\overset{\textnormal{もう}}{\text{申}}}$ し ${\overset{\textnormal{わけ}}{\text{訳}}}$ ございません。 \hfill\break
 \emph{O-kyaku-sama ni wa taihen go-meiwaku wo o-kake-shite moshiwake gozaimasen. \hfill\break
 }We are terribly sorry for the trouble we\textquotesingle ve placed on customers. }

\par{\textbf{Sentence Note }: Even though the offense to customers would have been done in the past, the use of the non-past tense instead emphasizes the speaker\textquotesingle s current sense of guilt. }

\par{15. いつも ${\overset{\textnormal{なに}}{\text{何}}}$ かとご ${\overset{\textnormal{むり}}{\text{無理}}}$ をお ${\overset{\textnormal{ねが}}{\text{願}}}$ いし、 ${\overset{\textnormal{もう}}{\text{申}}}$ し ${\overset{\textnormal{わけ}}{\text{訳}}}$ なく ${\overset{\textnormal{ぞん}}{\text{存}}}$ じます。 \hfill\break
 \emph{Itsumo nanika to go-muri wo o-negai shi, m }\emph{ōshiwake naku zonjimasu. \hfill\break
 }I am really sorry that I keep asking you to do me favors. }

\par{\textbf{Sentence Note }: The (adjectival) noun \emph{muri }無理 used here to mean “favors” literally means “unreasonable.” }

\begin{center}
\textbf{\emph{O-wabi }お詫び } \hfill\break

\end{center}

\par{ In very formal situations, speakers will also use a form of the phrase \emph{o-wabi (wo) shimasu }お詫び(を)します. \emph{O-wabi }お詫び means "apology,"  and it can either be treated as a noun or a \emph{suru }-verb. Below are its most important forms along with example sentences. }

\begin{ltabulary}{|P|P|P|}
\hline 

Humble & More Humble & Most Humble \\ \cline{1-3}

お詫び(を)します \hfill\break
 \emph{O-wabi shimasu }& お詫び(を)致します \hfill\break
 \emph{O-wabi wo itashimasu }& お詫び(を)申し上げます \hfill\break
 \emph{O-wabi (wo) mōshiagemasu }\\ \cline{1-3}

\end{ltabulary}

\par{16. ${\overset{\textnormal{かさ}}{\text{重}}}$ ね ${\overset{\textnormal{がさ}}{\text{重}}}$ ね、お ${\overset{\textnormal{わ}}{\text{詫}}}$ び(を) ${\overset{\textnormal{もう}}{\text{申}}}$ し ${\overset{\textnormal{あ}}{\text{上}}}$ げます。 \hfill\break
\emph{Kasanegasane, o-wabi (wo) mōshiagemasu. }\hfill\break
I sincerely apologize again. }

\par{17. ${\overset{\textnormal{きのう}}{\text{昨日}}}$ の ${\overset{\textnormal{しつれい}}{\text{失礼}}}$ をお ${\overset{\textnormal{わ}}{\text{詫}}}$ びします。 \hfill\break
\emph{Kinō no shitsurei wo o-wabi shimasu. }\hfill\break
I apologize for my rude behavior yesterday. }

\par{18. ${\overset{\textnormal{かんり}}{\text{管理}}}$ の ${\overset{\textnormal{ふてぎわ}}{\text{不手際}}}$ をお ${\overset{\textnormal{わ}}{\text{詫}}}$ び ${\overset{\textnormal{いた}}{\text{致}}}$ します。 \hfill\break
\emph{Kanri no futegiwa wo o-wabi itashimasu. }\hfill\break
I apologize for managerial awkwardness. }

\par{19. ${\overset{\textnormal{いくえ}}{\text{幾重}}}$ にもお ${\overset{\textnormal{わ}}{\text{詫}}}$ びを ${\overset{\textnormal{いた}}{\text{致}}}$ します。 \hfill\break
\emph{Ikue ni mo o-wabi wo itashimasu. }\hfill\break
I cannot apologize enough. }

\par{\textbf{Phrase Note }: \emph{Ikue ni mo }幾重にも literally means "repeatedly." }

\par{20. どれほど ${\overset{\textnormal{こころ}}{\text{心}}}$ の ${\overset{\textnormal{なか}}{\text{中}}}$ にお ${\overset{\textnormal{わ}}{\text{詫}}}$ びの ${\overset{\textnormal{きも}}{\text{気持}}}$ ちがあっても、それを ${\overset{\textnormal{かたち}}{\text{形}}}$ にしなければ、 ${\overset{\textnormal{あいて}}{\text{相手}}}$ には ${\overset{\textnormal{つた}}{\text{伝}}}$ わらない。 \hfill\break
\emph{Dore hodo kokoro no naka ni o-wabi no kimochi ga atte mo, sore wo katachi ni shinakereba, aite ni wa tsutawaranai. }\hfill\break
No matter how many apologetic feelings you have inside, if you don't have it take form, then it will not come across to the other person. }

\par{ In addition to the phrases above, there are other verbs that mean "to apologize" that must be looked at. These verbs are as follows. }

\begin{ltabulary}{|P|P|}
\hline 

 \emph{Ayamaru }謝る & This is the basic verb of describing the act of apologizing. \\ \cline{1-2}

 \emph{Wabiru }詫びる & Synonymous with above but limited in usage. \\ \cline{1-2}

 \emph{Shazai suru }謝罪する & Formal\slash refined version of \emph{ayamaru }謝る. \\ \cline{1-2}

 \emph{Chinsha suru }陳謝する & Formal variant of \emph{shazai suru }謝罪する used especially in writing. \\ \cline{1-2}

\end{ltabulary}

\par{\hfill\break
21. 健太郎は ${\overset{\textnormal{がいけん}}{\text{外見}}}$ では ${\overset{\textnormal{あやま}}{\text{謝}}}$ っているが、\{ ${\overset{\textnormal{あやま}}{\text{謝}}}$ る・お ${\overset{\textnormal{わ}}{\text{詫}}}$ びの\} ${\overset{\textnormal{きも}}{\text{気持}}}$ ちが ${\overset{\textnormal{いっさいかん}}{\text{一切感}}}$ じられない。 \hfill\break
\emph{Kentar }\emph{ō }\emph{wa gaiken de wa ayamatte iru ga, [ayamaru\slash o-wabi no] kimochi ga issai kanjirarenai. }\hfill\break
Kentaro may be outwardly apologizing, but I feel absolutely no feeling of remorse. }

\par{22. ${\overset{\textnormal{せんせい}}{\text{先生}}}$ が ${\overset{\textnormal{あやま}}{\text{謝}}}$ ってくれません。 \hfill\break
\emph{Sensei ga ayamatte kuremasen. }\hfill\break
My teacher won't apologize. }

\par{23. お ${\overset{\textnormal{なまえ}}{\text{名前}}}$ を ${\overset{\textnormal{か}}{\text{書}}}$ き ${\overset{\textnormal{まちが}}{\text{間違}}}$ えたことを ${\overset{\textnormal{ちんしゃ}}{\text{陳謝}}}$ \{します・いたします\}。 \hfill\break
\emph{O-namae wo kakimachigaeta koto wo chinsha [shimasu\slash itashimasu]. }\hfill\break
I\slash we apologize for misspelling your name. }

\par{24. ${\overset{\textnormal{こんかい}}{\text{今回}}}$ の ${\overset{\textnormal{けん}}{\text{件}}}$ を ${\overset{\textnormal{げんしゅく}}{\text{厳粛}}}$ に ${\overset{\textnormal{う}}{\text{受}}}$ け ${\overset{\textnormal{と}}{\text{止}}}$ め、 ${\overset{\textnormal{ちんしゃ}}{\text{陳謝}}}$ いたします。 \hfill\break
\emph{Konkai no ken wo genshuku ni uketome, chinsha itashimasu. }\hfill\break
We are solemnly coming to grips with this case and apologize (for what has happened). }

\par{25. これまでに ${\overset{\textnormal{だれ}}{\text{誰}}}$ かに ${\overset{\textnormal{しゃざい}}{\text{謝罪}}}$ (を)したことはありますか。 \hfill\break
\emph{Kore made ni dareka ni shazai (wo) shita koto wa arimasu ka? }\hfill\break
Is there anyone you have apologized to up to now? }

\par{26. ${\overset{\textnormal{ぶれい}}{\text{無礼}}}$ を ${\overset{\textnormal{どげざ}}{\text{土下座}}}$ して ${\overset{\textnormal{わ}}{\text{詫}}}$ びる。 \hfill\break
\emph{Burei wo dogeza shite wabiru. }\hfill\break
To kneel down on the ground and apologize for an offense. }

\begin{center}
\textbf{\emph{Gomen-nasai }ごめんなさい }
\end{center}

\par{ The next phrase to learn about is \emph{gomen-nasai }ごめんなさい. It is generally used towards people you\textquotesingle re familiar with. Knowing the person and not necessarily being above or below the person in social status are key points to using this phrase properly. Casually, it\textquotesingle s shortened to \emph{gomen }ごめん. }

\par{\textbf{Spelling Note }: This phrase is occasionally spelled as ご免なさい or 御免なさい. }

\par{\textbf{Intonation Note }: The intonation of this phrase is ご \textbf{めんなさ }い. }

\par{27. ${\overset{\textnormal{ほんとう}}{\text{本当}}}$ にごめんなさい。 \hfill\break
 \emph{Hont }\emph{ō ni gomen-nasai. \hfill\break
 }I\textquotesingle m really sorry. }

\par{28. ${\overset{\textnormal{ごかい}}{\text{誤解}}}$ があったら、ごめんなさい。 \hfill\break
 \emph{Gokai ga attara, gomen-nasai. \hfill\break
 }I\textquotesingle m sorry if there was a misunderstanding. }

\par{29. あ、ごめん。 ${\overset{\textnormal{だいじょうぶ}}{\text{大丈夫}}}$ ? \hfill\break
 \emph{A, gomen. Daij }\emph{ōbu? \hfill\break
 }Oh, sorry. Are you alright? }

\begin{center}
\textbf{"My Bad" }\hfill\break

\end{center}

\par{ Another means of saying “sorry” is by using the adjective \emph{warui }悪い. This is done in casual conversation among friends. In this situation, it is frequently pronounced as \emph{warii }わりぃ. }

\par{30. あ、わりぃ。 \hfill\break
 \emph{A, warii. \hfill\break
 }Oh, my bad. }

\begin{center}
 \textbf{Sorry to Impose }
\end{center}

\par{ The phrases \emph{kyōshuku [desu\slash de gozaimasu] }恐縮\{です・でございます\} and \emph{osoreirimasu }恐れ入ります are used interchangeably to mean “I\textquotesingle m sorry to impose.” They may also be used in the sense of “feel obliged” when the context is one where the speaker is imposing by accepting favor\slash consideration. }

\par{31. ご ${\overset{\textnormal{たぼう}}{\text{多忙}}}$ のところ、 ${\overset{\textnormal{きょうしゅく}}{\text{恐縮}}}$ です。 \hfill\break
 \emph{Go-tab }\emph{ō no tokoro, ky }\emph{ōshuku desu. \hfill\break
 }I\textquotesingle m sorry to impose when you\textquotesingle re very busy. }

\par{32. お ${\overset{\textnormal{はなしちゅう}}{\text{話中}}}$ 、 ${\overset{\textnormal{たいへんきょうしゅく}}{\text{大変恐縮}}}$ でございます。 \hfill\break
 \emph{O-hanashi-ch }\emph{ū, taihen ky }\emph{ōshuku de gozaimasu. \hfill\break
 }I\textquotesingle m terribly sorry to impose while you\textquotesingle re talking. }

\par{33. お ${\overset{\textnormal{きづか}}{\text{気遣}}}$ いいただき、 ${\overset{\textnormal{まこと}}{\text{誠}}}$ に ${\overset{\textnormal{おそ}}{\text{恐}}}$ れ ${\overset{\textnormal{い}}{\text{入}}}$ ります。 \hfill\break
 \emph{O-kizukai itadaki, makoto ni osoreirimasu. \hfill\break
 }I feel truly obliged that you were concerned. }

\par{34. ご ${\overset{\textnormal{ぶれい}}{\text{無礼}}}$ \{しました・いたしました\}。 \hfill\break
 \emph{Go-burei [shimashita\slash itashimashita]. } \hfill\break
I apologize (for my rudeness). }

\begin{center}
\textbf{Entering and Parting }
\end{center}

\par{ Whenever one is entering a room or leaving a room, the phrase \emph{shitsurei shimasu }失礼します can be heard. For the former situation, it is common whenever one is clearly having to interrupt or disturb someone. For the latter situation, it is always used when leaving someone. It is also commonplace to hear when hanging up on the phone. }

\par{35. ${\overset{\textnormal{しつれい}}{\text{失礼}}}$ します、お ${\overset{\textnormal{てすき}}{\text{手隙}}}$ ですか。 \hfill\break
 \emph{Shitsurei shimasu, o-tesuki desu ka? \hfill\break
 }Excuse me, are you free? }

\par{36. ${\overset{\textnormal{しっけい}}{\text{失敬}}}$ します。 \hfill\break
 \emph{Shikkei shimasu. \hfill\break
 }Excuse me. }

\par{\textbf{Sentence Note }: Alternatively, \emph{shikkei shimasu }失敬します can be heard used instead by superiors when parting with colleagues. \emph{Shikkei }失敬 is synonymous with \emph{shitsurei }失礼, but due to difference in cadence, it isn\textquotesingle t as widely used. It is deemed dialectical by some and is especially used in Nagoya. }

\par{ When entering someone's home, room, office, or entryway, speakers will say \emph{o-jama shimasu }お邪魔します to that person. The noun \emph{jama }邪魔 means hindrance, implying that one\textquotesingle s presence can be perceived as intruding on that person\textquotesingle s turf. It can be used in the past tense whenever one feels it\textquotesingle s necessary to leave after having clearly inconvenienced the other person. Or, it can also be seen in the progressive form, especially by those in cleaning services when workers are busy tidying up your space despite you having arrived. }

\par{37. お ${\overset{\textnormal{じゃま}}{\text{邪魔}}}$ します。 \hfill\break
 \emph{O-jama shimasu. \hfill\break
 }Excuse me for disturbing\slash interrupting you. }

\par{38.  お ${\overset{\textnormal{じゃま}}{\text{邪魔}}}$ しております。 \hfill\break
 \emph{O-jama shite orimasu. \hfill\break
 }I apologize for being in the way. }

\par{39. お ${\overset{\textnormal{じゃま}}{\text{邪魔}}}$ しました。 \hfill\break
 \emph{O-jama shimashita. \hfill\break
 }I\textquotesingle m sorry for having disturbed you. }

\begin{center}
\textbf{\emph{Gomen-kudasai }ごめんください  }
\end{center}

\par{ When entering someone's place without that person having not come to great you, it is customary to say \emph{gomen-kudasai }ごめんください. An even more formal form of this is \emph{gomen-kudasaimase }ごめんくださいませ, but this form is actually more commonly used as a means of hanging up in the customary service industry as a far more polite version of \emph{shitsurei shimasu }失礼します. }

\par{40. ごめんください。 ${\overset{\textnormal{たなか}}{\text{田中}}}$ さん、いらっしゃいますか。 \hfill\break
 \emph{Gomen-kudasai. Tanaka-san, irasshaimasu ka? \hfill\break
 }May I come in? Are you there, Tanaka-san? }

\par{41. ごめんくださいませ。 \hfill\break
 \emph{Gomen-kudasaimase. \hfill\break
 }I\textquotesingle m hanging up now.\slash May I come in? }

\begin{center}
 \textbf{\emph{Go-kigen yō }御機嫌よう }
\end{center}

\par{ There is one last phrase to cover. Some speakers will use \emph{go-kigen yō }御機嫌ようwhen both crossing paths with people and when parting with people. In English, it is akin to “how do you do?” and “adieu.” This phrase is mostly used by women. }

\par{42. ${\overset{\textnormal{ごきげん}}{\text{御機嫌}}}$ よう。 \hfill\break
 \emph{Go-kigen y }\emph{ō. \hfill\break
 }How do you?\slash Adieu. }

\begin{center}
\textbf{Condolences }
\end{center}

\par{ Giving condolences in Japanese is a very sensitive topic. The standard phrase for saying “my condolences” that even some Westerners know is \emph{o-ki no doku ni }お気の毒に. However, it is a phrase that shouldn\textquotesingle t be used directly to the person involved, or at least not as is, because it will otherwise be taken to be sarcastic or apathetic. The phrase literally means “poison to one\textquotesingle s heart,” and so although it is meant to show empathy towards those that are going through misfortune and\slash or suffering, using it must be done so with the utmost sincerity. }

\par{ Because it is such a problem using this phrase as is, many speakers opt for \emph{o-ki no doku-sama [desu\slash deshita] }お気の毒様\{です・でした\}. There is hardly any difference between the non-past and past tense forms in many circumstances. Although the past tense is the appropriate form when the misfortune has happened in the past, the non-past tense is best when you wish to make further commentary with those involved. }

\par{43. この ${\overset{\textnormal{たび}}{\text{度}}}$ は ${\overset{\textnormal{まこと}}{\text{誠}}}$ にお ${\overset{\textnormal{き}}{\text{気}}}$ の ${\overset{\textnormal{どくさま}}{\text{毒様}}}$ です。 \hfill\break
 \emph{Kono tabi wa makoto ni o-ki no doku-sama desu. \hfill\break
 }Please accept my sympathy at this time. }

\par{\textbf{Sentence Note }: This sentence would most likely be used to people who are close and\slash or related to someone that has gone through a great misfortune and\slash or death. }

\par{44. ${\overset{\textnormal{じこ}}{\text{事故}}}$ に ${\overset{\textnormal{あ}}{\text{遭}}}$ われたとは、お ${\overset{\textnormal{き}}{\text{気}}}$ の ${\overset{\textnormal{どくさま}}{\text{毒様}}}$ でした。 \hfill\break
 \emph{Jiko ni awareta to wa, o-ki no doku-sama deshita. \hfill\break
 }I am terribly sorry to hear that you were in an accident. }

\par{45. お ${\overset{\textnormal{く}}{\text{悔}}}$ やみ ${\overset{\textnormal{もう}}{\text{申}}}$ し ${\overset{\textnormal{あ}}{\text{上}}}$ げます。 \hfill\break
\emph{O-kuyami moshiagemasu. }\hfill\break
My deepest sympathy. }

\par{\textbf{Sentence Note }: This phrase is used especially at funerals to the deceased person\textquotesingle s relatives. }

\par{ Another important phrase used toward people who have gone through a terrible loss or misfortune including the loss of a loved one is \emph{go-shūshō-sama [desu\slash de gozaimasu\slash deshita\slash de gozaimashita] }ご愁傷様\{です・でございます・でした・でございました\}. The use of the past tense is typically used most often when this is all the speaker can think of that's appropriate to say, whereas the use of the non-past tense is best used when the speaker feels compelled to speak more about the matter. Having said all this, it is still very important that you handle the matter with grace and utmost respect so that the listener will not perceive your words to be insincere or sarcastic. }

\par{46. ご ${\overset{\textnormal{しゅうしょうさま}}{\text{愁傷様}}}$ でした。 \hfill\break
 \emph{Go-shūshō-sama deshita. \hfill\break
 }My deepest sympathy. }

\par{47. この ${\overset{\textnormal{たび}}{\text{旅}}}$ は ${\overset{\textnormal{まこと}}{\text{誠}}}$ にご ${\overset{\textnormal{しゅうしょうさま}}{\text{愁傷様}}}$ でございます。 \hfill\break
 \emph{Kono tabi wa makoto ni go-shūshō-sama de gozaimasu. \hfill\break
 }My truest and deepest sympathy goes out to you at this time. }
    