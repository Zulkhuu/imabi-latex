    
\chapter{Numbers I}

\begin{center}
\begin{Large}
第27課: Numbers I: Sino-Japanese Numbers 
\end{Large}
\end{center}
 
\par{ Sino-Japanese numbers come from Chinese and are the most important kind of numbers in Japanese. Number(s) in Japanese is ${\overset{\textnormal{かず}}{\text{数}}}$ . A similar word is ${\overset{\textnormal{すうじ}}{\text{数字}}}$ . 数字 refers to units and or the characters that write out units. So, 数 doesn't always mean the same thing as 数字. }
      
\section{Sino-Japanese Numbers}
 
\begin{center}
 \textbf{1~10 }
\end{center}

\par{ Though we are dealing with numbers that come from Chinese, some such as those in bold are actually native words treated as Sino-Japanese numbers. 4 is usually \textbf{よん }because し is homophonous with ${\overset{\textnormal{し}}{\text{死}}}$ (death). 9 is usually きゅう because く is homophonous with ${\overset{\textnormal{く}}{\text{苦}}}$ (suffering). し and く are hardly ever in big numbers. 7 is often \textbf{なな }, and you don't hear しち after the tens as an initial digit. }

\begin{ltabulary}{|P|P|P|P|P|P|P|P|P|P|}
\hline 

1 & 一(いち) & 2 & 二(に) & 3 & 三(さん) & 4 & 四( \textbf{よん }・し) & 5 & 五(ご) \\ \cline{1-10}

6 & 六(ろく) & 7 & 七( \textbf{なな }・しち) & 8 & 八(はち) & 9 & 九(きゅう・く) & 10 & 十(じゅう) \\ \cline{1-10}

\end{ltabulary}

\par{\textbf{Zero Note }: Zero can be ゼロ, れい (零), or 〇(まる). まる means "circle" and is like "o" for zero. ゼロ is written as 0 in numbers. れい is Sino-Japanese and is also common. }

\begin{center}
 \textbf{11~100,000 }
\end{center}

\par{  Because this system is broken up by powers of four, big numbers are easy. 11 is "ten one". 200 is "two hundred". 10,000 is the fourth power, so the unit changes. To raise the power further, you add 十, 百, and 千 in that order. This doesn't stop where the chart stops, but it's all you need to know for now. }

\begin{ltabulary}{|P|P|P|P|P|P|}
\hline 

11 & 十一(じゅういち) & 12 & 十二(じゅうに) & 20 & 二十(にじゅう) \\ \cline{1-6}

30 & 三十(さんじゅう) & 100 & 百(ひゃく) & 300 & 三百( \textbf{さんびゃく }) \\ \cline{1-6}

600 & 六百( \textbf{ろっぴゃく }) & 800 & 八百( \textbf{はっぴゃく }) & 1,000 & (一)千((いっ)せん) \\ \cline{1-6}

3,000 & 三千( \textbf{さんぜん }) & 4,000 & 四千(よんせん) & 7,000 & ななせん \\ \cline{1-6}

8,000 & 八千( \textbf{はっせん }) & 10,000 & 一万(いちまん) & 100,000 & 十万(じゅうまん) \\ \cline{1-6}

\end{ltabulary}

\par{\textbf{Number Notes }: }

\par{1. いっせん really isn't ever used unless a number precedes it. So, if you want to say 11,000, you should say いちまんいっせん. }

\par{2. Due to certain phonetic rules, 300, 600, 800, 3,000, and 8,000 seem irregular. }
\textbf{}
\par{\textbf{Practice (1) }: }

\par{Part I: Write out the following numbers in 漢字. }

\par{1. 3,004  2. 684  3. 10,563  4. 24 \hfill\break
5. 795  6. 18,974  7. 100,674  8. 201,345 }

\par{Part II: Write out the following numbers in Arabic numerals. }

\par{1. 五百六十七  2. 四十五万六千七百八十九  3. 三十五   4. 八千九百五十九 \hfill\break
5. 二百十  6. 四十七  7. 十万二百四十五  }

\par{Part III: Write out the following numbers in かな. }

\par{1. 四十  2. 七千六百三十八  3. 二千三百四十六  4. 三百五十三  5. 百二  6. 八百九十三 }
      
\section{Spelling Numbers}
 
\par{ The most common way to write numbers is using 0, 1, 2, etc. 漢字 are still used in vertical texts, set phrases, and what not but not in the realm of math. 漢字 spellings can be used like Arabic numerals, which is common in pricing and literature. Units like 万 can be in 漢字 like in 2万2千 (22,000), which is common in news reports. }

\begin{ltabulary}{|P|P|P|P|}
\hline 

Arabic Numerals & Traditional & Kanji like 0, 1, 2, 3\dothyp{}\dothyp{}\dothyp{} & Abbreviated 1, 2, 3\dothyp{}\dothyp{}\dothyp{} \\ \cline{1-4}

203 & 二百三 & 二〇三 & 203 \\ \cline{1-4}

5,843 & 五千八百四十三 & 五八四三 & 5843 \\ \cline{1-4}

9,630,000 & 九百六十三万 & 九六三〇〇〇〇 & 963万 \\ \cline{1-4}

\end{ltabulary}

\par{\textbf{Practice (2) }: Write the following numbers in a different way. }

\par{1. 100  2. 6万7370  3. 七〇四三〇〇 }

\par{Note: Add マイナス(-) to the front to make negative numbers. So, -5 would be マイナスご. }
      
\section{Phone Number}
 
\par{\textbf{ }For phone numbers ( ${\overset{\textnormal{でんわばんごう}}{\text{電話番号}}}$ ) simply read out the numbers. One mora numbers may become two morae long. So, に \textrightarrow  にぃ. 4, 7, and 9 are normally よん, なな, and きゅう respectively. For zero, you can hear ゼロ, れい, or まる. まる is especially preferred by clerks. In conversion, ゼロ is the most common. On TV, you may often hear れい especially in formal situations like NHK news announcements. の stands for a dash, but not everyone includes it. This is just like how English speakers optionally say space or dash. }

\par{「電話番号は ${\overset{\textnormal{}}{\text{何番}}}$ ですか。」「3056-7735です。」 \hfill\break
"What is your phone number?" "It's 3056-7735". }
 
\par{You could also say XXXX番のXXXX. 番 is the \#. , 2 and 10 may be ふた and とお respectively with -番. In fact, ひゃくとおばん, \#110, is the Japanese equivalent of 911. \#119 is used to call firefighters, etc. }

\par{\textbf{Practice (3) }: Write out the following made-up phone numbers in  かな. }

\par{1. 977-947-4773  2. 6734-8539    3. 36-463-7574  4. 64-64-647       5. 6766-7744 }

\par{--------------------------------------------------------------------- }

\par{Next Lesson \textrightarrow  第7課: Counters  }
      
\section{Keys}
 
\par{Practice (1) }

\par{Part I }

\par{1. 三千四       2. 六百八十四      3. 一万五百六十三      4. 二十四                             5. 七百九十五 \hfill\break
6. 一万八千九百七十四              7. 十万六百七十四      8. 二十万(一)千三百四十五 }

\par{Part II }

\par{1. 567      2. 456,789     3. 35     4. 8,959    5. 210   6. 47   7. 100,245 }

\par{Part III }

\par{1. よんじゅう    2. ななせんろっぴゃくさんじゅうはち  3. にせんさんびゃくよんじゅうろく 4. さんびゃくごじゅうさん \hfill\break
5. ひゃくに  6. はっぴゃくきゅうじゅうさん }

\par{Practice (2) }

\par{1. 一〇〇、百、1百   2. 六七三七〇、六万七千三百七十、67,370  3. 704,300、七十万四千三百、70万4千3百 }

\par{Practice (3) }

\par{1. きゅう なな なな (の) きゅう よん なな (の) よん なな なな さん \hfill\break
2. ろく なな さん よん (の) はち ご さん きゅう \hfill\break
3. さん ろく (の) よん ろく さん (の) なな ご なな よん \hfill\break
4. ろく よん (の) ろく よん (の) ろく よん なな \hfill\break
5. ろく なな ろく ろく (の) なな なな よん よん }
    