    
\chapter{The Particle Ka か III}

\begin{center}
\begin{Large}
第44課: The Particle Ka か III: Indirect Question 
\end{Large}
\end{center}
 
\par{ Subordinate clauses ( \emph{jūzokusetsu }従属節) are the opposite of independent clauses ( \emph{dokuritsusetsu }独立節). In English, independent clauses stand alone as sentences with at least a subject and a verb, and dependent clauses are composed of a subject and verb but do not form a complete thought. Thus, the former is used as a sentence and may also be called the “main clause ( \emph{shusetsu }主節)” but the latter is not. In the English sentences, below, the clause put in [] is labeled as either being an independent or a dependent clause. }

\begin{enumerate}

\item [I went to the park.] – Independent \hfill\break

\item The one [who went to the park]… - Dependent \hfill\break

\item [That song is amazing.]  - Independent \hfill\break

\item Which song was it [(that) I thought was amazing]? – Dependent \hfill\break

\item I forgot [what we did the other day.] - Dependent 
\item I know [what you did the other day.] - Dependent 
\end{enumerate}

\par{ In English, dependent clauses are usually marked with the words "what," “which,” “whose,” “where,” “who,” “whom,” and “that.” English doesn\textquotesingle t particularly distinguish between using these words based on whether a question is being embedded or not. In other words, “what” doesn\textquotesingle t look different in v. or vi. despite that the meaning is not the same. }
      
\section{Vocabulary List}
 
\par{\textbf{Nouns }}

\par{・独立節 \emph{Dokuritsusetsu }– Independent clause }

\par{・従属節 \emph{J }\emph{ūzokusetsu }– Dependent clause }

\par{・主節 \emph{Shusetsu }– Main clause }

\par{・山 \emph{Yama }– Mountain }

\par{・サメ \emph{Same }– Shark }

\par{・魚 \emph{Sakana\slash Uo }– Fish }

\par{・宿題 \emph{Shukudai }– Homework }

\par{・生物 \emph{Seibutsu }– Creature }

\par{・妻 \emph{Tsuma }– Wife }

\par{・虫 \emph{Mushi }– Bug(s) }

\par{・お風呂 \emph{Ofuro }– Bath }

\par{・(お)店 \emph{(O)mise }– Store\slash shop\slash restaurant }

\par{・素麺 \emph{S }\emph{ōmen }– Fine white noodles }

\par{・夫 \emph{Otto }– Husband }

\par{・旦那 \emph{Dan\textquotesingle na }– Husband (informal) }

\par{・警察 \emph{Keisatsu }– Police }

\par{・犯行 \emph{Hank }\emph{ō }– Crime }

\par{・現場 \emph{Gemba }– Scene }

\par{・状態 \emph{J }\emph{ōtai }– Situation }

\par{・方法 \emph{H }\emph{ōh }\emph{ō }– Method }

\par{・時点 \emph{Jiten }– Point in time }

\par{・端末 \emph{Tammatsu }- Device }

\par{・今週 \emph{Konsh }\emph{ū }– This week }

\par{・来週 \emph{Raish }\emph{ū }– Next week }

\par{・水曜 \emph{Suiy }\emph{ō }- Wednesday }

\par{・もの \emph{Mono }– Thing }

\par{・存在 \emph{Sonzai }- Existence }

\par{・意義 \emph{Igi }– Meaning\slash significance }

\par{・次 \emph{Tsugi }– Next }

\par{・相手 \emph{Aite }– Partner\slash other party }

\par{・生鮮食料品 \emph{Seisen shokury }\emph{ōhin }– Fresh foods }

\par{・インターネット \emph{Int }\emph{ānetto }– Internet }

\par{・情報 \emph{J }\emph{ōh }\emph{ō }– Information }

\par{・子 \emph{Ko }– Child }

\par{・本 \emph{Hon }– Book }

\par{・粘土 \emph{Nendo }- Clay }

\par{・日 \emph{Hi }– Day\slash sun }

\par{・世界 \emph{Sekai }– World }

\par{・世 \emph{Yo }– World\slash society }

\par{・男 \emph{Otoko }- Man }

\par{・女 \emph{On\textquotesingle na }- Woman }

\par{・辺 \emph{Hen }– Area\slash neighborhood }

\par{・主人公 \emph{Shujink }\emph{ō }– Protagonist }

\par{・寿司 \emph{Sushi }– Sushi }

\par{・刺し身 \emph{Sashimi }– Sashimi }

\par{・水道水 \emph{Suid }\emph{ōsui }– Tap water }

\par{・お茶 \emph{Ocha }– Tea }

\par{・水 \emph{Mizu }– Water }

\par{・ミルク \emph{Miruku }– Milk }

\par{・公園 \emph{K }\emph{ōen }– Park }

\par{・デパート \emph{Dep }\emph{āto }– Department store }

\par{・人 \emph{Hito }– Person }

\par{\textbf{Pronouns }}

\par{・私 \emph{Wata(ku)shi }– I }

\par{・僕 \emph{Boku }–  I (male) }

\par{・彼女 \emph{Kanojo }– She }

\par{\textbf{Proper Nouns }}

\par{・健太君 \emph{Kenta-kun }– Kenta }

\par{・カイロ \emph{Kairo }– Cairo }

\par{・七夕 \emph{Tanabata }– Star Festival }

\par{・織姫 \emph{Orihime }– Vega }

\par{・彦星 \emph{Hikoboshi }- Altair }

\par{・山口さん \emph{Yamaguchi-san }– Mr\slash M(r)s. Yamaguchi }

\par{・札幌 \emph{Sapporo }– Sapporo }

\par{・東京 \emph{T }\emph{ōky }\emph{ō }– Tokyo }

\par{\textbf{Adjectives }}

\par{・大きい \emph{Ōkii }– Large }

\par{・美味しい \emph{Oishii }– Delicious }

\par{・不味い \emph{Mazui }– Nasty\slash disgusting }

\par{・難しい \emph{Muzukashii }– Difficult\slash hard }

\par{・古い \emph{Furui }– Old }

\par{・熱い \emph{Atsui }– Hot (general) }

\par{・暑い \emph{Atsui }– Hot (weather) }

\par{・寒い \emph{Samui }– Cold (weather) }

\par{・少ない \emph{Sukunai }– Few }

\par{・多い \emph{Ōi }– Many\slash a lot }

\par{・安い \emph{Yasui }- Cheap }

\par{・高い \emph{Takai }– Expensive\slash high }

\par{・眠い \emph{Nemui }– Sleepy }

\par{・健康にいい \emph{Kenk }\emph{ō ni ii }– Healthy }

\par{\textbf{N }\textbf{umber Phrases }}

\par{・ひとつ \emph{Hitotsu }– One (thing) }

\par{\textbf{I dioms }}

\par{・しょうがない \emph{Sh }\emph{ō ga nai }– It can\textquotesingle t be helped }
 
\par{\textbf{ Adjectival Nouns }}

\par{・巨大だ \emph{Kyodai da }- To be enormous }

\par{・嫌いだ \emph{Kirai  da }– To hate }

\par{・幸せだ \emph{Shiawase da }– To be happy }

\par{・ゲイだ \emph{Gei da }– To be gay }

\par{・静かだ \emph{Shizuka da }– To be quiet }

\par{・綺麗だ \emph{Kirei da }– To be pretty }

\par{・生の \emph{Nama no }– Raw\slash natural }

\par{\textbf{Adverbs }}

\par{・きのう \emph{Kin }\emph{ō }– Yesterday }

\par{・やはり \emph{Yahari }– As expected }

\par{・さっぱり \emph{Sappari }– Not in the least\slash completely (neg.) }

\par{・十分に \emph{J }\emph{ūbun ni }– Enough }

\par{・本当に \emph{Hont }\emph{ō ni }– Really }

\par{・凄く \emph{Sugoku }– Very\slash immensely }

\par{・今日 \emph{Ky }\emph{ō }– Today }

\par{・大体 \emph{Daitai }– Generally\slash for the most part }

\par{・全部 \emph{Zembu }– All }

\par{・明日 \emph{Ashita }– Tomorrow }

\par{・ほとんど \emph{Hotondo }– Hardly }

\par{・きっと \emph{Kitto }– Surely }

\par{・何だか \emph{Nandaka }– A little\slash rather\slash somewhat }

\par{・最高に \emph{Saik }\emph{ō ni }– The most }

\par{・もう \emph{M }\emph{ō }– Already\slash yet }

\par{・一体 \emph{Ittai }– The heck }

\par{・一番 \emph{Ichiban }– Most\slash best }

\par{・みんな \emph{Min\textquotesingle na }– Everyone }

\par{\textbf{\emph{(ru) Ichidan }Verbs }}

\par{・確かめる \emph{Tashikameru }– To check\slash make sure (trans.) }

\par{・寝る \emph{Neru }– To sleep (intr.) }

\par{・起きる \emph{Okiru }– To get up\slash occur (intr.) }

\par{・調べる \emph{Shiraberu }– To investigate (trans.) }

\par{・教える \emph{Oshieru }– To teach\slash tell (trans.) }

\par{・いる \emph{Iru }– To be (live animate objects) (intr.) }

\par{\textbf{\emph{(u) Godan }Verbs }}

\par{・登る \emph{Noboru }– To climb }

\par{・飲み込む \emph{Nomikomu }– To swallow\slash gulp up }

\par{・分かる \emph{Wakaru }– To become clear\slash be known\slash understand (intr.) }

\par{・知る \emph{Shiru }– To know\slash recognize (trans.) }

\par{・歌う \emph{Utau }– To sing (trans.) }

\par{・届く \emph{Todoku }– To arrive (intr.) }

\par{・言う \emph{Iu }– To say (trans.) }

\par{・盗む \emph{Nusumu }– To steal (trans.) }

\par{・混む \emph{Komu }– To be crowded (intr.) }

\par{・許す \emph{Yurusu }– To forgive (trans.) }

\par{・作る \emph{Tsukuru }– To make (trans.) }

\par{・買う \emph{Kau }– To buy (trans.) }

\par{・成る \emph{Naru }– To become (intr.) }

\par{・飲む \emph{Nomu }– To drink\slash swallow\slash take (medicine) }

\par{・行く \emph{Iku }– To go (intr.) }

\par{・死ぬ \emph{Shinu }– To die (intr.) }

\par{\textbf{Irregular Verbs }}

\par{・来る \emph{Kuru }– To come (intr.) }

\par{\textbf{\emph{suru }Verbs }}

\par{・調査する \emph{Ch }\emph{ōsa suru }– To investigate (trans.) }

\par{・発送する \emph{Hass }\emph{ō suru }– To ship (trans.) }

\par{・到着する \emph{T }\emph{ōchaku suru }– To arrive (intr.) }

\par{・運用する \emph{Un\textquotesingle y }\emph{ō suru }– To operate \slash make use of }

\par{・注目する \emph{Ch }\emph{ūmoku suru }– To notice\slash give attention to }

\par{・注文する \emph{Ch }\emph{ūmon suru }– To order }

\par{・沸騰する \emph{Futt }\emph{ō suru }– To boil }

\par{\textbf{Question Words }}

\par{・誰 \emph{Dare }– Who }

\par{・誰か \emph{Dareka }- Someone }

\par{・何 \emph{Nani\slash nan }– What }

\par{・何か \emph{Nanika }– Something }

\par{・いつ \emph{Itsu }– When }

\par{・いつか \emph{Itsuka }– Sometime }

\par{・どこ \emph{Doko }– Where }

\par{・どこか \emph{Dokoka }– Somewhere }

\par{・どう \emph{D }\emph{ō }– How }

\par{・どうか \emph{D }\emph{ōka }– Shomehow }

\par{\textbf{Demonstratives }}

\par{・この \emph{Kono }– This (adj.) }

\par{・あの \emph{Ano }– That over there (adj.) }

\par{・どれ \emph{Dore }– Which (noun) }

\par{・どの \emph{Dono }– Which (adj.) }

\par{・ここ \emph{Koko }– Here }

\par{・あいつ \emph{Aitsu }– That guy }

\par{・どんな \emph{Don\textquotesingle na }– What kind }

\par{・ある \emph{Aru }– A certain }
      
\section{Subordinate Clause: When not a question}
 
\par{ When you are creating a subordinate clause but not to embed a question, you simply place the verbal phrase like you would an adjectival phrase directly before the noun you wish to modify. This is exactly opposite of English, which depending on the length of the phrase, usually places complex verbal qualifiers after the noun in question. }

\par{1. ${\overset{\textnormal{けんたくん}}{\text{健太君}}}$ はきのう、 ${\overset{\textnormal{やま}}{\text{山}}}$ を ${\overset{\textnormal{のぼ}}{\text{登}}}$ りました。 \hfill\break
\emph{Kenta-kun wa kinō, yama wo noborimashita. \hfill\break
 }Yesterday, Kenta climbed a mountain. }

\par{2. きのう ${\overset{\textnormal{やま}}{\text{山}}}$ を ${\overset{\textnormal{のぼ}}{\text{登}}}$ ったのは、 ${\overset{\textnormal{けんたくん}}{\text{健太君}}}$ でした。 \hfill\break
\emph{Kinō yama wo nobotta no wa , Kenta-kun deshita. }\hfill\break
The one who climbed the mountain yesterday was Kenta. }

\par{ In Ex. 2, the particle \emph{no }の is used to change \emph{kinō yama wo nobotta }きのう山を登った into a nominal phrase exactly. It is essentially the “one” in the sentence with no intervening “who” being necessary. This should not be new information at this point as we've seen plenty of sentences in which whole phrases, whether they end in adjectives or verbs, modify nouns. Now you just know a little more about what\textquotesingle s going on. }

\par{3. ${\overset{\textnormal{きょだい}}{\text{巨大}}}$ な ${\overset{\textnormal{さかな}}{\text{魚}}}$ があのサメを ${\overset{\textnormal{の}}{\text{飲}}}$ み ${\overset{\textnormal{こ}}{\text{込}}}$ んだ。 \hfill\break
\emph{Kyodai na sakana ga ano same wo nomikonda. } \hfill\break
An enormous fish swallowed that shark. }

\par{4. あのサメを ${\overset{\textnormal{の}}{\text{飲}}}$ み ${\overset{\textnormal{こ}}{\text{込}}}$ んだ、この ${\overset{\textnormal{さかな}}{\text{魚}}}$ はやはり ${\overset{\textnormal{おお}}{\text{大}}}$ きいですね。 \hfill\break
\emph{Ano same wo nomikonda, kono sakana wa yahari ōkii desu ne. }\hfill\break
This fish, which swallowed that shark, is big as expected, isn\textquotesingle t it. }
      
\section{Embedded\slash Indirect Questions}
 
\par{ When a subordinate clause is an embedded question, the particle \emph{ka }か intervenes. Embedded sentences, as a rule, don\textquotesingle t include the topic of the sentence. This is because the topic should be outside, in the main clause. As such, you will expect to see \emph{ga }が in dependent clauses. If for some reason you do see \emph{wa }は, that\textquotesingle s a clue that you are not actually in the embedded clause, and if you are, it\textquotesingle s not being used to mark the topic. }

\par{5. カイロは(、)どこの ${\overset{\textnormal{くに}}{\text{国}}}$ にあるか ${\overset{\textnormal{し}}{\text{知}}}$ っていますか。 \hfill\break
\emph{Kairo wa(,) doko no kuni ni aru ka shitte imasu ka? } \hfill\break
Do you know which country Cairo is at? \hfill\break
Literally: As for Cairo, do you know which country it is at? }

\par{6. どの ${\overset{\textnormal{しゅくだい}}{\text{宿題}}}$ が ${\overset{\textnormal{むずか}}{\text{難}}}$ しかったか ${\overset{\textnormal{おし}}{\text{教}}}$ えてください。 \hfill\break
\emph{Dono shukudai ga muzukashikatta ka oshiete kudasai. }\hfill\break
Please tell me which homework was difficult. }

\par{7. どれがおいしいかまずいかさっぱりわかりません。 \hfill\break
\emph{Dore ga oishii ka mazui ka sappari wakarimasen. }\hfill\break
I have no idea which is delicious and which is nasty. }

\par{8. この ${\overset{\textnormal{せいぶつ}}{\text{生物}}}$ は ${\overset{\textnormal{なに}}{\text{何}}}$ か ${\overset{\textnormal{おし}}{\text{教}}}$ えてください。 \hfill\break
\emph{Kono seibutsu wa nani ka oshiete kudasai. \hfill\break
}Please tell me what creature this is. }

\par{9. ${\overset{\textnormal{たなばた}}{\text{七夕}}}$ はいつか ${\overset{\textnormal{し}}{\text{知}}}$ っていますか。 \hfill\break
Tanabata wa itsu ka shitte imasu ka? \hfill\break
Do you know when Tanabata is? }

\par{\textbf{Culture Note }: \emph{Tanabata }七夕, also known as the Star Festival, is a Japanese festival that celebrates the meeting of the deities \emph{Orihime }織姫 and \emph{Hikoboshi }彦星 represented by the stars Vega and Altair respectively. }

\begin{center}
 \textbf{\emph{Ka dō ka }かどうか }
\end{center}

\par{ To express "whether (…)or not,” you can substitute what “not” would stand for with \emph{dō ka }どうか. This gives rise to the phrase \emph{ka dō ka }かどうか. }

\par{10. ${\overset{\textnormal{つま}}{\text{妻}}}$ が ${\overset{\textnormal{ね}}{\text{寝}}}$ ているか ${\overset{\textnormal{お}}{\text{起}}}$ きているか ${\overset{\textnormal{わ}}{\text{分}}}$ からない。 \hfill\break
\emph{Tsuma ga nete iru ka okite iru ka wakaranai. }\hfill\break
I don't know whether my wife is asleep or awake? }

\par{11. ${\overset{\textnormal{つま}}{\text{妻}}}$ が ${\overset{\textnormal{ね}}{\text{寝}}}$ いているかどうか ${\overset{\textnormal{わ}}{\text{分}}}$ からない。 \hfill\break
\emph{Tsuma ga nete iru ka dō ka wakaranai. }\hfill\break
I don\textquotesingle t know whether or not my wife is asleep. }

\par{12. ( ${\overset{\textnormal{わたし}}{\text{私}}}$ は、) ${\overset{\textnormal{かのじょ}}{\text{彼女}}}$ が ${\overset{\textnormal{むし}}{\text{虫}}}$ が ${\overset{\textnormal{きら}}{\text{嫌}}}$ いかどうかわかりません。 \hfill\break
\emph{(Watashi wa,) kanojo ga mushi ga kirai ka dō ka wakarimasen. }\hfill\break
I don't know whether or not she hates bugs. }

\par{13. ${\overset{\textnormal{おっと}}{\text{夫}}}$ が ${\overset{\textnormal{ほんとう}}{\text{本当}}}$ に ${\overset{\textnormal{しあわ}}{\text{幸}}}$ せかどうか ${\overset{\textnormal{わ}}{\text{分}}}$ かりません。 \emph{\hfill\break
Otto ga hontō ni shiawase ka dō ka wakarimasen. \hfill\break
}I really don\textquotesingle t know whether or not my husband is really happy. }

\par{ Without the \emph{dō ka }どうか, you create a sentence that asks “if” something is so. }

\par{14. お ${\overset{\textnormal{ふろ}}{\text{風呂}}}$ が ${\overset{\textnormal{じゅうぶん}}{\text{十分}}}$ に ${\overset{\textnormal{あつ}}{\text{熱}}}$ いかどうか ${\overset{\textnormal{たし}}{\text{確}}}$ かめてください。 \hfill\break
\emph{Ofuro ga jūbun ni atsui ka dō ka tashikamete kudasai. \hfill\break
}Please check whether or not the bath is hot enough. }

\par{15. お ${\overset{\textnormal{ふろ}}{\text{風呂}}}$ が ${\overset{\textnormal{じゅうぶん}}{\text{十分}}}$ に ${\overset{\textnormal{あつ}}{\text{熱}}}$ いか ${\overset{\textnormal{たし}}{\text{確}}}$ かめてください。 \hfill\break
\emph{Ofuro ga jūbun ni atsui ka tashikamete kudasai. }\hfill\break
Please check if the bath is hot enough. }

\par{16. きょうは、あの ${\overset{\textnormal{みせ}}{\text{店}}}$ の ${\overset{\textnormal{そうめん}}{\text{素麺}}}$ が ${\overset{\textnormal{ほんとう}}{\text{本当}}}$ にまずいか ${\overset{\textnormal{ちょうさ}}{\text{調査}}}$ します。 \hfill\break
\emph{Kyō wa, ano mise no sōmen ga hontō ni mazui ka chōsa shimasu. }\hfill\break
Today, I will investigate if that place\textquotesingle s somen really is disgusting. }

\begin{center}
 \textbf{Indirect Questions are Noun-Like }
\end{center}

\par{ In Japanese, whenever there is a quantifier that follows \emph{ka }か, it will refer to the noun phrase embedded in the indirect question. }

\par{17. ${\overset{\textnormal{やま}}{\text{山}}}$ ${\overset{\textnormal{ぐち}}{\text{口}}}$ さんは ${\overset{\textnormal{だれ}}{\text{誰}}}$ が ${\overset{\textnormal{うた}}{\text{歌}}}$ ったか ${\overset{\textnormal{だい}}{\text{大}}}$ ${\overset{\textnormal{たい}}{\text{体}}}$ ${\overset{\textnormal{し}}{\text{知}}}$ っているでしょう。 \hfill\break
\emph{Yamaguchi-san wa dare ga utatta ka daitai shitte iru deshō. }\hfill\break
Mr\slash M(r)s. Yamaguchi probably knows for the most part who sang. }

\par{\textbf{Word Note }: \emph{Daitai }大体 corresponds to the quantity of people \emph{dare }誰 refers to. }

\par{18. ${\overset{\textnormal{わたし}}{\text{私}}}$ は ${\overset{\textnormal{なに}}{\text{何}}}$ が ${\overset{\textnormal{とど}}{\text{届}}}$ いたか ${\overset{\textnormal{ぜんぶし}}{\text{全部知}}}$ っています。 \hfill\break
\emph{Watashi wa nani ga todoita ka zembu shitte imasu. }\hfill\break
I know all of what has arrived. }

\par{\textbf{Word Note }: \emph{Zembu }全部 corresponds to the quantity of things \emph{nani }何refers to. }

\par{ This demonstrates that embedded questions end up functioning like nouns thanks to か. This can be further demonstrated by how other particles can follow it. }

\par{19. ${\overset{\textnormal{けいさつ}}{\text{警察}}}$ は ${\overset{\textnormal{だれ}}{\text{誰}}}$ が ${\overset{\textnormal{はんこうげんば}}{\text{犯行現場}}}$ にいたか(を) ${\overset{\textnormal{しら}}{\text{調}}}$ べています。 \hfill\break
\emph{Keisatsu wa dare ga hankō gemba ni ita ka wo shirabete imasu. \hfill\break
}The police are investigating who was at the crime scene. }

\par{20. どんな ${\overset{\textnormal{じょうたい}}{\text{状態}}}$ になるかに ${\overset{\textnormal{ちゅうもく}}{\text{注目}}}$ してください。 \hfill\break
\emph{Don\textquotesingle na jōtai ni naru ka ni chūmoku shite kudasai. \hfill\break
}Pay attention to what sort of situation it becomes. }

\par{21. ${\overset{\textnormal{ふる}}{\text{古}}}$ いかどうかを ${\overset{\textnormal{しら}}{\text{調}}}$ べる ${\overset{\textnormal{ほうほう}}{\text{方法}}}$ \hfill\break
\emph{Furui ka dō ka wo shiraberu hōhō \hfill\break
}Method of investigating whether or not it\textquotesingle s old }

\par{22. ないものはしょうがないですが、 ${\overset{\textnormal{あす}}{\text{明日}}}$ の ${\overset{\textnormal{じてん}}{\text{時点}}}$ で ${\overset{\textnormal{はっそう}}{\text{発送}}}$ する ${\overset{\textnormal{たんまつ}}{\text{端末}}}$ は ${\overset{\textnormal{らいしゅうすいよう}}{\text{来週水曜}}}$ までに ${\overset{\textnormal{さっぽろ}}{\text{札幌}}}$ に ${\overset{\textnormal{とうちゃく}}{\text{到着}}}$ するかが ${\overset{\textnormal{し}}{\text{知}}}$ りたいです。 \hfill\break
 \emph{Nai mono wa shōga nai desu ga, asu no jiten de hassō suru tammatsu wa raishū suiyō made ni Sapporo ni tōchaku suru ka ga shiritai desu. }\hfill\break
The ones you don\textquotesingle t have can\textquotesingle t be helped, but I want to know if the devices that you will ship as of tomorrow will make it by Wednesday next week to Sapporo. }

\par{\textbf{Grammar Note }: The ending \emph{-tai }~たい is an ending that denotes personal want to do something, and it conjugates, interestingly enough, as an adjective. We will learn more about it later in IMABI. }

\begin{center}
\textbf{\emph{No ka }のか }
\end{center}

\par{ In our last discussion on \emph{ka }か, we learned that the \emph{no }の in \emph{no ka }のか adds weight to the question one is making. In other words, it makes your concern serious. Now, what determines the overall civility of your question is all based on what else is added to the sentence, but putting that aside, this same principle applies to embedded questions as well. }

\par{23. ${\overset{\textnormal{そんざいいぎ}}{\text{存在意義}}}$ は ${\overset{\textnormal{なん}}{\text{何}}}$ なのか ${\overset{\textnormal{おし}}{\text{教}}}$ えてください。 \hfill\break
\emph{Sonzai igi wa nan na no ka oshiete kudasai. }\hfill\break
Please tell me what exactly the meaning of life is. }

\par{24. ${\overset{\textnormal{なに}}{\text{何}}}$ が ${\overset{\textnormal{なに}}{\text{何}}}$ かはほとんどわかりません。 \hfill\break
\emph{Nani ga nani ka wa hotondo wakarimasen. }\hfill\break
I hardly understand what is what. }

\par{25. ${\overset{\textnormal{なに}}{\text{何}}}$ を ${\overset{\textnormal{い}}{\text{言}}}$ っているのかわからない。 \hfill\break
\emph{Nani wo itte iru no ka wakaranai. }\hfill\break
I don\textquotesingle t understand what you\textquotesingle re saying. }

\par{ When you use \emph{ka }か all on its own, the level of uncertainty you are portraying is more or less at about 20\%. You aren\textquotesingle t sure, so you\textquotesingle re asking a question, but your level of uncertainty is neither pressing nor significant in the least. This helps explain the tone of all the examples of it thus far in this lesson. }

\par{26. ${\overset{\textnormal{つぎ}}{\text{次}}}$ の ${\overset{\textnormal{あいて}}{\text{相手}}}$ は ${\overset{\textnormal{だれ}}{\text{誰}}}$ かわかりません。 \hfill\break
\emph{Tsugi no aite wa dare ka wakarimasen. } \hfill\break
I don\textquotesingle t know who my next opponent is. }

\par{27. ${\overset{\textnormal{つぎ}}{\text{次}}}$ の ${\overset{\textnormal{あいて}}{\text{相手}}}$ は ${\overset{\textnormal{だれ}}{\text{誰}}}$ なのかわかりません。 \hfill\break
\emph{Tsugi no aite wa dare na no ka wakarimasen. }\hfill\break
I really don\textquotesingle t know who my next opponent is. }

\par{ The use of \emph{no ka }のか raises the level of one\textquotesingle s uncertainty to 50\%. When the particle \emph{ka }か is directly after a question word, it first and foremost shows that something is uncertain. As such, in Ex. 26 …the fact that one has an opponent is just what is uncertain about the situation. With the change to \emph{no ka }のか, your attention is shifting toward suspicion as to who will really be one\textquotesingle s opponent. }

\par{28. ${\overset{\textnormal{だれ}}{\text{誰}}}$ が ${\overset{\textnormal{ぬす}}{\text{盗}}}$ んだのか ${\overset{\textnormal{し}}{\text{知}}}$ りませんか。 \hfill\break
\emph{Dare ga nusunda no ka shirimasen ka? }\hfill\break
Would you happen to know who stole it? }

\par{29. ここは ${\overset{\textnormal{せいせんしょくりょうひん}}{\text{生鮮食料品}}}$ が ${\overset{\textnormal{やす}}{\text{安}}}$ いのか、すごく ${\overset{\textnormal{こ}}{\text{混}}}$ んでいます。 \hfill\break
\emph{Koko wa seisen shokuryōhin ga yasui no ka, sugoku konde imasu. }\hfill\break
Whether it\textquotesingle s because fresh produce is cheap, but it is really crowded here. }

\par{30. ${\overset{\textnormal{うんよう}}{\text{運用}}}$ している ${\overset{\textnormal{ひと}}{\text{人}}}$ が ${\overset{\textnormal{すく}}{\text{少}}}$ ないのか、インターネットに ${\overset{\textnormal{じょうほう}}{\text{情報}}}$ がほとんどありません。 \hfill\break
\emph{Un\textquotesingle yō shite iru hito ga sukunai no ka, intānetto ni jōhō ga hotondo arimasen. \hfill\break
}Whether it\textquotesingle s because there are few people running it, there is hardly any information online. }

\begin{center}
\textbf{Question Word + \emph{ka }か }
\end{center}

\par{ In speaking of \emph{ka }か attaching to questions to show uncertainty, consider the follow phrases that are created when “question + \emph{ka }か” is not used at the end of the sentence. }
・ \emph{Dareka }誰か = Someone ・ \emph{Nanika }何か =  Something ・ \emph{Itsuka }いつか  = Sometime ・ \emph{Dokoka }どこか  =  Somewhere ・ \emph{Dōka }どうか = Somehow or other 
\par{ Their parts of speech are just like the question words that make them up. \emph{Dareka }誰か, \emph{nanika }何か, and \emph{dokoka }どこか can either be used as nouns or adverbs. This means that \emph{ga }が and \emph{wo }を will always be optional if applicable. \emph{Itsuka }いつか  and \emph{dōka }どうか on the other hand are only used as adverbs and never take these particles. }

\par{31. いつかきっと、 ${\overset{\textnormal{せかい}}{\text{世界}}}$ が ${\overset{\textnormal{ひと}}{\text{一}}}$ つになる ${\overset{\textnormal{ひ}}{\text{日}}}$ が ${\overset{\textnormal{く}}{\text{来}}}$ る。 \hfill\break
\emph{Itsuka kitto, sekai ga hitotsu ni naru hi ga kuru. }\hfill\break
Someday surely, the day in which the world becomes one will arrive. }

\par{32. あの ${\overset{\textnormal{こ}}{\text{子}}}$ は ${\overset{\textnormal{ねんど}}{\text{粘土}}}$ で ${\overset{\textnormal{なに}}{\text{何}}}$ か(を) ${\overset{\textnormal{つく}}{\text{作}}}$ った。 \hfill\break
\emph{Ano ko wa nendo de nanika (wo) tsukutta. }\hfill\break
That child made something with clay. }

\par{33. ${\overset{\textnormal{こんしゅう}}{\text{今週}}}$ は、 ${\overset{\textnormal{だんな}}{\text{旦那}}}$ がある ${\overset{\textnormal{ほん}}{\text{本}}}$ を ${\overset{\textnormal{か}}{\text{買}}}$ いにどこかへ ${\overset{\textnormal{い}}{\text{行}}}$ った。 \hfill\break
\emph{Konshū wa, dan\textquotesingle na ga aru hon wo kai ni dokoka e itta. }\hfill\break
This week, my husband went somewhere to buy a certain book. }

\par{34. どうか ${\overset{\textnormal{ゆる}}{\text{許}}}$ してください。 \hfill\break
\emph{Dōka yurushite kudasai. \hfill\break
}Please forgive me. }

\begin{center}
\textbf{\emph{Da ka }だか } 
\end{center}

\par{ Say you are even more doubtful and or suspicious about the situation. In which case, rather than using \emph{ka }か or \emph{no ka }のか, you can actually use \emph{da ka }だか. This is when it becomes appropriate to have the particle \emph{ka }か follow the copula \emph{da }だ. The level of one\textquotesingle s uncertainty with this is about 80\%. }

\par{ \emph{Da ka }だか attaches to nouns and adjectival nouns with no modification done to either side. With other parts of speech, you will need to insert \emph{n }\slash  \emph{no }ん・の between it and a verb\slash adjective, giving \emph{n da ka }んだか・ \emph{no da ka }のだか. }

\par{35. ${\overset{\textnormal{つぎ}}{\text{次}}}$ の ${\overset{\textnormal{あいて}}{\text{相手}}}$ は ${\overset{\textnormal{だれ}}{\text{誰}}}$ だかわかりません。 \hfill\break
\emph{Tsugi no aite wa dare da ka wakarimasen. }\hfill\break
I have no clue who my next opponent is. }

\par{36. ${\overset{\textnormal{ぼく}}{\text{僕}}}$ は、あいつが ${\overset{\textnormal{ほんとう}}{\text{本当}}}$ に ${\overset{\textnormal{おんな}}{\text{女}}}$ だかわからないんですよ。 \hfill\break
\emph{Boku wa, aitsu ga hontō ni on\textquotesingle na da ka wakaranai n desu yo. }\hfill\break
I really have no clue if that guy is a woman. }

\par{37. ${\overset{\textnormal{あいて}}{\text{相手}}}$ がゲイだかわからない。 \hfill\break
\emph{Aite ga gei da ka wakaranai. }\hfill\break
I sure don't know if my partner\slash the person (I'm dealing with) is gay. }

\par{ 80\% uncertainty is not a guarantee. In the example below, the question posed is simply drawing the listener in to elicit a response for the speaker to then say yea or nay. }

\par{38. この ${\overset{\textnormal{よ}}{\text{世}}}$ で ${\overset{\textnormal{さいこう}}{\text{最高}}}$ に ${\overset{\textnormal{きれい}}{\text{綺麗}}}$ なものは ${\overset{\textnormal{なん}}{\text{何}}}$ だか ${\overset{\textnormal{し}}{\text{知}}}$ ってます? \hfill\break
\emph{Kono yo de saikō ni kirei na mono wa nan da ka shittemasu? }\hfill\break
Do you what the prettiest thing in the world is? }

\par{ \emph{Nandaka }何だか can also be used as an adverb meaning "a little\slash somewhat\slash rather." }

\par{39. この ${\overset{\textnormal{へん}}{\text{辺}}}$ は ${\overset{\textnormal{なん}}{\text{何}}}$ だか ${\overset{\textnormal{しず}}{\text{静}}}$ かですね。 \hfill\break
\emph{Kono hen wa nandaka shizuka desu ne. }\hfill\break
This area is rather quiet, isn't it? }

\par{40. ${\overset{\textnormal{なん}}{\text{何}}}$ だか ${\overset{\textnormal{ねむ}}{\text{眠}}}$ い。 \hfill\break
\emph{Nandaka nemui. }\hfill\break
I'm a little tired. }

\begin{center}
\textbf{\emph{N da ka }んだか } 
\end{center}

\par{ Showing 80-100\% indecisiveness\slash uncertainty is possible with \emph{n da ka }んだか, especially in the pattern “A… \emph{n da ka }+ B… \emph{n da ka }(A…んだか+B…んだか).” The “A” and “B” can be a noun, adjective, adjectival noun, or verb. If it\textquotesingle s a noun or adjectival noun, you will need to place \emph{na }な before \emph{n da ka }んだか. For adjectives and verbs, this pattern becomes indistinguishable from the previous one, which is why it may or may not show absolute indecisiveness. }

\par{41. ${\overset{\textnormal{やす}}{\text{安}}}$ いんだか、 ${\overset{\textnormal{たか}}{\text{高}}}$ いんだか、もうわからないんですよ。 \hfill\break
\emph{Yasui n da ka, takai n da ka, mō wakaranai n desu yo. \hfill\break
}I have no idea if whether it\textquotesingle s cheap or expensive. }

\par{42. ${\overset{\textnormal{しゅじんこう}}{\text{主人公}}}$ は ${\overset{\textnormal{いったいだれ}}{\text{一体誰}}}$ なんだかわからない。 \hfill\break
\emph{Shujinkō wa ittai dare na n da ka wakaranai. }\hfill\break
I have no earthly idea who the protagonist is. }

\par{43. ${\overset{\textnormal{すく}}{\text{少}}}$ ないんだか、 ${\overset{\textnormal{おお}}{\text{多}}}$ いんだか ${\overset{\textnormal{わ}}{\text{分}}}$ からなくなるよな。 \hfill\break
\emph{Sukunai n da ka, ōi n da ka wakaranaku naru yo na. \hfill\break
}You end up not having a clue if there\textquotesingle s little or a lot of it, you know. \emph{ }}

\par{\textbf{Particle Note }: The particle \emph{na }な at the end of this sentence is used similarly to ending a sentence with “you know” in English. }

\begin{center}
\textbf{Or } 
\end{center}

\par{ As you can see, using more than one \emph{n da ka }んだか is used to show a desperate\slash completely indecisive “whether…or…” This, though, is used in the sense of making heads or tails of a situation, not as in presenting indecisiveness on a benign decision. }

\par{ In the same light, all the other patterns shown in this lesson can be used the same way with various degrees of uncertainty implied. }

\begin{ltabulary}{|P|P|P|}
\hline 

Pattern & Uncertainty & Note \\ \cline{1-3}

 \emph{…ka…ka }~か~か & 0~20\% & Simply lists options. \\ \cline{1-3}

 \emph{…no ka…no ka }~のか~のか & 50\% & A rather confident yet uncertain “or.” \\ \cline{1-3}

 \emph{…da ka…da ka }\emph{ }~だか~だか & 80\% & Very uncertain and suspicious “or.” \\ \cline{1-3}

 \emph{…n da ka…n da ka }~んだか~んだか & 100\% & Completely indecisive “or.” \\ \cline{1-3}

\end{ltabulary}

\par{ As for “… \emph{ka }… \emph{ka }(~か~か),” the final \emph{ka }か isn\textquotesingle t there if you\textquotesingle re just listing options of an action\slash situation unless you are actually questioning something. It\textquotesingle s when you list situations\slash actions that both \emph{ka }か are needed. }

\par{44. ${\overset{\textnormal{なま}}{\text{生}}}$ の ${\overset{\textnormal{すいどうすい}}{\text{水道水}}}$ か ${\overset{\textnormal{ふっとう}}{\text{沸騰}}}$ した ${\overset{\textnormal{すいどうすい}}{\text{水道水}}}$ かお ${\overset{\textnormal{ちゃ}}{\text{茶}}}$ か、どれが ${\overset{\textnormal{いちばんけんこう}}{\text{一番健康}}}$ にいいんですか。 \hfill\break
\emph{Nama no suidōsui ka futtō shita suidōsui ka ocha ka, dore ga ichiban kenkō ni ii n desu ka? }\hfill\break
Which is the healthiest, plain tap water, boiled tap water, or tea? }

\par{45. ${\overset{\textnormal{みず}}{\text{水}}}$ かミルクを ${\overset{\textnormal{の}}{\text{飲}}}$ んでください。 \hfill\break
\emph{Mizu ka miruku wo nonde kudasai. }\hfill\break
Drink water or milk. }

\par{46. すしか ${\overset{\textnormal{さ}}{\text{刺}}}$ し ${\overset{\textnormal{み}}{\text{身}}}$ を ${\overset{\textnormal{ちゅうもん}}{\text{注文}}}$ してください。 \hfill\break
\emph{Sushi ka sashimi wo chūmon shite kudasai. \hfill\break
}Order either sushi or sashimi. }

\par{47. ${\overset{\textnormal{こうえん}}{\text{公園}}}$ に ${\overset{\textnormal{い}}{\text{行}}}$ くか、デパートに ${\overset{\textnormal{い}}{\text{行}}}$ くかですね。 \hfill\break
\emph{Kōen ni iku ka, depāto ni iku ka desu ne. }\hfill\break
Either go to the park or go to the department store, right? }

\par{48. ${\overset{\textnormal{とうきょう}}{\text{東京}}}$ では ${\overset{\textnormal{おとこ}}{\text{男}}}$ だか ${\overset{\textnormal{おんな}}{\text{女}}}$ だか ${\overset{\textnormal{わ}}{\text{分}}}$ からない ${\overset{\textnormal{ひと}}{\text{人}}}$ が ${\overset{\textnormal{おお}}{\text{多}}}$ いの? \hfill\break
\emph{T }\emph{ō }\emph{ky }\emph{ō }\emph{de wa otoko da ka on\textquotesingle na da ka wakaranai hito ga ōi no? }\hfill\break
Are there are lot of people in Tokyo you can\textquotesingle t tell whether they\textquotesingle re a man or a woman? }

\par{49. ${\overset{\textnormal{さむ}}{\text{寒}}}$ いのか ${\overset{\textnormal{あつ}}{\text{暑}}}$ いのか ${\overset{\textnormal{わ}}{\text{分}}}$ からないんじゃないの? \hfill\break
\emph{Samui no ka atsui no ka wakaranai n ja nai no? }\hfill\break
You don\textquotesingle t know whether it\textquotesingle s cold or hot (outside)? }

\par{50. もう ${\overset{\textnormal{だれ}}{\text{誰}}}$ が ${\overset{\textnormal{し}}{\text{死}}}$ ぬんだか、みんな ${\overset{\textnormal{し}}{\text{死}}}$ ぬんだかわからない。 \hfill\break
\emph{Mō dare ga shinu n da ka, min\textquotesingle na shinu n da ka wakaranai. \hfill\break
}I have no idea anymore who\textquotesingle s going to die or if everyone\textquotesingle s going to die. }
    