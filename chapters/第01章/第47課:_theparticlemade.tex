    
\chapter{The Particle まで}

\begin{center}
\begin{Large}
第47課: The Particle まで 
\end{Large}
\end{center}
 
\par{ The particle まで often goes hand in hand with から, with it being the "to" in "from X to Y." Don't let classification confuse you as this particle is either a case or adverbial particle. }

\par{\textbf{Pronunciation Note }: Do not pronounce this like the English word "maid". Don't even think about it. It's just so bad that it could cause a Japanese's ear to bleed. Seriously, don't. }
      
\section{The Case Particle まで}
 
\par{ The particle まで is most commonly seen in the pattern  から\dothyp{}\dothyp{}\dothyp{}まで, which means "from\dothyp{}\dothyp{}\dothyp{}to\dothyp{}\dothyp{}\dothyp{}". This can be used to refer to temporal, location, spatial, or quantity boundaries. This pattern can be used in situations like "from 1 to 3 o' clock," "from the car to the street," or "from 1 meter to 4 meters." }

\par{ まで is the equivalent of "to\slash until\slash till." This particle is seen a lot after nouns and temporal adverbial nouns like 今. However, it may also be seen after verbs but never in the past tense. }

\begin{ltabulary}{|P|P|P|}
\hline 

With Nouns & N + まで & 明日まで; 京都まで; そこまで; 1リットルまで \\ \cline{1-3}

With Verbs &  \textbf{Non }-past V (連体形) + まで & 終わるまで; 消えるまで; 歌い始めるまで \\ \cline{1-3}

\end{ltabulary}

\par{\textbf{Tense Note }: Do not use it with the past tense. This doesn't make sense because "until" is describing an event that has yet to actually happen. It's illogical to use the past tense which refers to something already done\slash taken place. }

\begin{center}
\textbf{Examples }
\end{center}

\par{1. 9時から5時まで仕事をします。 \hfill\break
I work from nine to five. }

\par{2. 僕は9 ${\overset{\textnormal{さい}}{\text{歳}}}$ までローマで ${\overset{\textnormal{そだ}}{\text{育}}}$ ちました。 \hfill\break
I was raised in Rome until I was nine. }

\par{3. ${\overset{\textnormal{さいきん}}{\text{最近}}}$ までアメリカに住んでいました。 \hfill\break
I was living in America until recently. }

\par{4. 仕事が始まるまでの休み時間は ${\overset{\textnormal{すば}}{\text{素晴}}}$ らしい。 \hfill\break
The free time until work starts is wonderful! }

\par{5. ${\overset{\textnormal{さいご}}{\text{最後}}}$ の ${\overset{\textnormal{ひとつぶ}}{\text{一粒}}}$ まで食べてください。 \hfill\break
Eat to the last grain. }

\par{6. いつまでここにいるのか。(Harsh) \hfill\break
Until when are you going to be here? }

\par{7. 6月までオースティンに住む ${\overset{\textnormal{よてい}}{\text{予定}}}$ です。 \hfill\break
I plan to live in Austin until June. }

\par{8. ちょっと大学まで行ってくる! \hfill\break
I'm going to go to the university for a little bit and come back! }

\par{9. カチッと音が ${\overset{\textnormal{な}}{\text{鳴}}}$ るまで ${\overset{\textnormal{かぎ}}{\text{鍵}}}$ を ${\overset{\textnormal{お}}{\text{押}}}$ し ${\overset{\textnormal{こ}}{\text{込}}}$ む。 \hfill\break
To push a key in until there is a click sound. }

\par{10. きょうまでよく ${\overset{\textnormal{た}}{\text{耐}}}$ えたな、 ${\overset{\textnormal{おれ}}{\text{俺}}}$ 。 (Casual, male speech) \hfill\break
I've endured pretty well thus far\dothyp{}\dothyp{}\dothyp{} }

\par{11. 「どちらへ」「ちょっとそこまで」 \hfill\break
"Where are you going?" "No where particular" }

\par{\textbf{Culture Note }: In Japanese you may be asked by people as a gesture where you are going, and you may respond in this manner. You are not obliged like you would be to Americans to tell exactly where you're going. }

\begin{center}
\textbf{までに } 
\end{center}

\par{ Be careful in adding the particle に to まで. までに = "by (the time)", which is quite different. This can be seen after stuff as simple as 一時. So, 一時までに = by 1 o' clock. It could also be after the non-past form of a verb, such as in 死ぬ(とき)までにやりとげる = to accomplish by (the time) one dies. Note that you can also easily add とき in case this makes things easier. This pattern shows the time for which something is to be realized. }

\par{12 . ${\overset{\textnormal{れいじ}}{\text{零時}}}$ までに ${\overset{\textnormal{さんさつ}}{\text{三冊}}}$ 読む。 \hfill\break
To read three books by midnight. }

\par{13. 日が出るまでに寝る。 \hfill\break
To sleep until the sun rises. }

\par{14. 明日までに ${\overset{\textnormal{ゆうびんきょく}}{\text{郵便局}}}$ に行かないといけない。 \hfill\break
I have to go to the post office by tomorrow. }

\par{\textbf{Grammar Note }: ~ないといけない is a "must" construction. We will formally study it in Lesson 102  . }
 
\par{15. ご ${\overset{\textnormal{さんこう}}{\text{参考}}}$ まで(に) \hfill\break
For your information\slash reference }

\par{\textbf{Particle Note }: This phrase is so common that it might as well be viewed as a set phrase. It is a little more polite with に. The role of まで(に) loosely falls under the usages described thus far. The idea is that the speaker is trying to indirectly suggest a reference, and if the listener gets to the point by which it would be useful to use, that person can do so. }

\par{16.ご挨拶(まで)に伺いました。 \hfill\break
I have come by to say hello and introduce myself. }

\par{\textbf{Culture Note }: This phrase is often said when simply coming to greet someone. Just like in America, there are times when we feel obliged to visit someone for the first time and exchange salutations. In Japanese, this means you need to show due respect. The verb 伺う in this sentence is a humble form of 来る. }
 
\par{\textbf{Culture Section: Trains }}
 
\par{ The train is the most important mode of transportation in Japan. The 新幹線 (bullet train) continues to connect more of Japan together. 緑の ${\overset{\textnormal{まどぐち}}{\text{窓口}}}$ is the "Green Window" where you buy reserved seat tickets ( ${\overset{\textnormal{していけん}}{\text{指定券}}}$ ) and long distance tickets. Most stations also have a machine to do this. However, you're more likely to find about deals if you go to an actual window. You can buy tickets in ticket vending machines ( ${\overset{\textnormal{けんばいき}}{\text{券売機}}}$ ). A ${\overset{\textnormal{にゅうじょうけん}}{\text{入場券}}}$ lets you see people off on the platform, ${\overset{\textnormal{ていきけん}}{\text{定期券}}}$ are season tickets, ${\overset{\textnormal{かいすうけん}}{\text{回数券}}}$ are discounted tickets. }

\par{17. 大阪までの ${\overset{\textnormal{きっぷ}}{\text{切符}}}$ を4 ${\overset{\textnormal{まい}}{\text{枚}}}$ ください。 \hfill\break
Four tickets to Osaka please. }
 
\par{18. オースティンからダラスまで ${\overset{\textnormal{よじかんはん}}{\text{4時間半}}}$ かかる 。 \hfill\break
It will take 4 and a half hours from Austin to Dallas. }
 
\par{19. 私たちは ${\overset{\textnormal{いっしょ}}{\text{一緒}}}$ に駅まで行きました。 \hfill\break
We went together up to the train station. }
 
\par{20. このバスは ${\overset{\textnormal{きょうと}}{\text{京都}}}$ まで行きます。 \hfill\break
This bus goes (up) to Kyoto. }

\par{21. 新宿までいくらですか。 \hfill\break
How much is it to Shinjuku? }
 
\par{\textbf{Geography Note }: 新宿 is a major commercial and administrative center. }

\par{ご ${\overset{\textnormal{さんこう}}{\text{参考}}}$ まで \hfill\break
For your information }

\par{ご参考までに (A little more polite than if it were without に) \hfill\break
For your reference }
      
\section{The Adverbial Particle まで}
 
\par{ The adverbial particle まで exemplifies an extremity. This is by nature showing degree, but extremity has a great sense of intensity to it. The emotional aspect separates it from the case particle. Aside from that, it also has peculiar usages. Even so, when just after a noun or verb, it's not that obvious that it's any different from what you've just seen above. }

\par{22. オレのガールフレンドまでもオレを ${\overset{\textnormal{うたが}}{\text{疑}}}$ ってるんだよ。(Masculine; rough) \hfill\break
Even my own girlfriend doubts me. }

\par{23. どこまでオレを ${\overset{\textnormal{にく}}{\text{憎}}}$ むのか。(Masculine; rough) \hfill\break
To what extent do you detest me? }

\par{24. 今まで(に)見たこともない ${\overset{\textnormal{けしき}}{\text{景色}}}$ だよ。 \hfill\break
I've never seen such scenery before. }

\par{25. 私は今までずっとスウェーデンに住んでいます。 \hfill\break
I've been living in Sweden all this while. }

\par{26. そこまで言う?ちょっとひどくない? \hfill\break
Wow, isn't that a little harsh saying that much? }

\par{27. あの人ま で ${\overset{\textnormal{しゅうきょう}}{\text{宗教}}}$ にはまりだした 。 \hfill\break
Even that person became obsessed with religion.  }

\par{\textbf{Spelling Note }: The verb はまりだすcan be spelled in 漢字 as 嵌り出す.   }
    