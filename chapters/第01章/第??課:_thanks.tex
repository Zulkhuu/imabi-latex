    
\chapter*{Expressions of Gratitude}

\begin{center}
\begin{Large}
第??課: Expressions of Gratitude 
\end{Large}
\end{center}
 
\par{ Showing gratitude to others is one of many things we do on a daily basis. In English, we primarily express our gratitude via the expression "thank you" and other phrases like "I appreciate it." Many Westerns recognize the Japanese word \emph{arigatō }ありがとう meaning "thank you," but there is a lot more to showing thanks in Japanese than just this one word. }

\par{ In this lesson, you will be introduced to all sorts of ways to show thanks to others. Dialectical and speech style variation will become rather complex, but the focus for you should be to remember the core phrases introduced. For grammar that hasn't been introduced up to this point, you are not required to know how to constructively use them outside the phrases that are discussed. }

\par{\textbf{Grammar Note }: You will notice the prefix \emph{o\slash go }- お・ご in front of many phrases discussed in this lesson. This prefix is an honorific marker which helps make what it attaches to be more respectful. Much later on, we will learn how to use this constructively. }
      
\section{Gratitude: Kansha 感謝}
 
\par{ The Western world is very familiar with the expression \emph{arigatō }ありがとう. This word is very important to Japanese speakers and is indeed used in everyday life. However, there is still quite a bit to know on how to properly give thanks to someone. It derives from the adjective \emph{arigatai }ありがたい, meaning “to be grateful.” }

\par{ The intonation pattern of \emph{arigat }\emph{ō }ありがとう differs depending on where you are in Japan. In Standard Japanese, \emph{arigat }\emph{ō gozaimasu }ありがとうございます is pronounced with the following intonation (morae in bold being those with a high pitch as opposed to a low pitch): あ \textbf{り }がとうご \textbf{ざいま }す. }

\par{ For those of you that find yourself outside the greater Kanto Region ( \emph{Kant }\emph{ō Chih }\emph{ō }関東地方), you\textquotesingle ll notice native speakers pronouncing \emph{arigat }\emph{ō }ありがとう  differently. The most important variations to note are as follows. }

\begin{itemize}

\item あ \textbf{り }がとう (Standard Japanese – \emph{Hy }\emph{ōjungo }標準語) 
\item あり \textbf{が }とう (Nagoya Dialect – \emph{Nagoya-ben }名古屋弁) 
\item ありが \textbf{と }う (Kansai Dialects – \emph{Kansai-ben }関西弁) 
\item ありが \textbf{とう }(Kagoshima Dialect – \emph{Kagoshima-ben }鹿児島弁) 
\end{itemize}

\par{\textbf{Spelling Note }: This phrase can be spelled in \emph{Kanji }漢字 as 有り難う御座います. This is quite commonly used, especially in e-mails and business chats. }

\par{ Of course, whenever we tell people "thank you," we usually add adverbs like "a lot" and "very much." We also usually explain what we're thankful for. All sorts of contexts are provided for intricate thank-yous in the examples below. }

\par{1.「ご ${\overset{\textnormal{しんぱい}}{\text{心配}}}$ は ${\overset{\textnormal{い}}{\text{要}}}$ りません。 ${\overset{\textnormal{わた(く)し}}{\text{私}}}$ のほうでやります。」「どうもありがとうございます。」 \hfill\break
 \emph{“Go-shimpai wa irimasen. Wata(ku)shi no h }\emph{ō de yarimasu.” “D }\emph{ōmo arigat }\emph{ō gozaimasu.” \hfill\break
 }“There is no need to worry. I\textquotesingle ll do it on my end.” “Thank you very much.” }

\par{\textbf{Phrase Note }: To add the sense of “very much,” use the adverb \emph{d }\emph{ōmo }どうも as seen in Ex. 1. }

\par{2. あ、 ${\overset{\textnormal{ほんとう}}{\text{本当}}}$ に ${\overset{\textnormal{たす}}{\text{助}}}$ かります。ありがとうございます。 \hfill\break
 \emph{A, hont }\emph{ō ni tasukarimasu. Arigat }\emph{ō gozaimasu. \hfill\break
 }Wow, that really helps a lot. Thank you. }

\par{3. ${\overset{\textnormal{たいおう}}{\text{対応}}}$ してくれてありがとうございます。 \hfill\break
 \emph{Tai }\emph{ō shite kurete arigat }\emph{ō gozaimasu \hfill\break
 }Thank you for handling it. }

\par{\textbf{Grammar Note }: The pattern \emph{-te kurete }~てくれて, which indicates a favor being done by someone else for you or someone in one\textquotesingle s in-group, is frequently paired with thank-you phrases. }

\par{4. 「もう ${\overset{\textnormal{す}}{\text{済}}}$ ませてあるよ。」「お、ありがと。」 \hfill\break
 \emph{“M }\emph{ō sumasete aru yo.” “O, arigato.” \hfill\break
 }“It\textquotesingle s already taken care of.” “Oh, thanks.” }

\par{\textbf{Pronunciation Note }: Shortening the phrase to \emph{arigato }ありがと is very common in speech. The pitch of the phrase remains the same with a high pitch on the \slash ri\slash  mora. }

\par{5. ${\overset{\textnormal{なかま}}{\text{仲間}}}$ の ${\overset{\textnormal{おうえん}}{\text{応援}}}$ が ${\overset{\textnormal{ほんとう}}{\text{本当}}}$ にありがたいです。 \hfill\break
\emph{Nakama no ōen ga hontō ni arigatai desu. }\hfill\break
My pals\textquotesingle  support is really appreciated. }

\par{6. ${\overset{\textnormal{われわれ}}{\text{我々}}}$ への ${\overset{\textnormal{しえん}}{\text{支援}}}$ は ${\overset{\textnormal{ほんとう}}{\text{本当}}}$ にありがたい。 \hfill\break
\emph{Wareware e no shien wa hont }\emph{ō ni arigatai. \hfill\break
}The support to us is really appreciated. }
 
\par{\textbf{Word Note }: \emph{Ōen }応援 is support as in “cheering on” whereas \emph{shien }支援 is support as in “assistance.” }

\par{7. ${\overset{\textnormal{そうちょう}}{\text{早朝}}}$ はありがたいことに ${\overset{\textnormal{ゆき}}{\text{雪}}}$ が ${\overset{\textnormal{つ}}{\text{積}}}$ もっていなかった。 \hfill\break
\emph{S }\emph{ōch }\emph{ō wa arigatai koto ni yuki ga tsumotte inakatta. \hfill\break
}Tha nkfully, snow hadn\textquotesingle t piled up in the early morning. }
 
\par{\textbf{Grammar Note }: You can express “thankfully” with \emph{arigatai koto ni }ありがたいことに. }

\par{8. 「 ${\overset{\textnormal{かさ}}{\text{傘}}}$ はあそこに ${\overset{\textnormal{お}}{\text{置}}}$ いてある。」「あ、どうも。」 \hfill\break
 \emph{“Kasa wa asoko in oite aru.” “A, d }\emph{ōmo.” \hfill\break
 }The umbrella(s) have been placed over there.” “Ah, thanks.” }

\par{\textbf{Phrase Note }: A quick way to tell someone “thanks” is \emph{d }\emph{ōmo }どうも. However, this shouldn\textquotesingle t be used when cutting one\textquotesingle s thanks short isn't appropriate--speaking to superiors. }

\par{9. ホント、いつもありがとうです! \hfill\break
 \emph{Honto, itsumo arigat }\emph{ō desu! \hfill\break
 }Thank you so much as always! }

\par{\textbf{Grammar Note }: When people feel like \emph{arigat }\emph{ō gozaimasu }ありがとうございます is too polite but they still wish to be polite to some degree, they often opt for \emph{arigat }\emph{ō desu }ありがとうです. However, this is ungrammatical to most speakers. Nonetheless, it is still used a lot. }

\par{10. おお、あんがとう! \hfill\break
 \emph{Ō, angat }\emph{ō! \hfill\break
 }Oh, thanks! }

\par{\textbf{Sentence Note }: \emph{Angat }\emph{ō }あんがとう is a very common, casual contraction. }

\par{ Grammatically speaking, \emph{arigat }\emph{ō gozaimasu }ありがとうございます is in the non-past tense. Literally, it means “to be grateful.” This gratefulness is typically in response to what has just taken place or is currently taking place. If, however, the act of kindness is markedly in the past, then \emph{arigat }\emph{ō gozaimashita }ありがとうございました becomes viable. \hfill\break
 \hfill\break
11. ${\overset{\textnormal{いろいろ}}{\text{色々}}}$ とお ${\overset{\textnormal{せわ}}{\text{世話}}}$ になりまして ${\overset{\textnormal{まこと}}{\text{誠}}}$ にありがとう\{ございました・ございます\}。 \hfill\break
 \emph{Iroiro to o-sewa ni narimashite makoto ni arigat }\emph{ō [gozaimashita\slash gozaimasu]. \hfill\break
 }I\slash we are truly grateful for all the favor you\textquotesingle ve given me\slash us. }

\par{\textbf{Grammar Note }: By using \emph{gozaimasu }ございます instead of \emph{gozaimashita }ございました, one\textquotesingle s gratitude can be emphasized as being still ongoing despite the event of kindness done by the listener was still in the past. }

\par{12. わざわざお ${\overset{\textnormal{こ}}{\text{越}}}$ しいただいてありがとう\{ございました・ございます\}。 \hfill\break
 \emph{Wazawaza o-koshi itadaite arigat }\emph{ō [gozaimashita\slash gozaimasu]. \hfill\break
 }Thank you so much for going through all the trouble to come. }

\par{\textbf{Grammar Note }: The adverb \emph{wazawaza }わざわざ is used to stress the trouble someone went to do something for the speaker. Additionally, the word \emph{o-koshi }お越し comes from a respectful verb for “to come.” }

\par{13. ${\overset{\textnormal{せんじつ}}{\text{先日}}}$ はわざわざお ${\overset{\textnormal{えつ}}{\text{越}}}$ しいただいてありがとうございました。 \hfill\break
 \emph{Senjitsu wa wazawaza o-koshi itadaite arigat }\emph{ō gozaimashita. \hfill\break
 }Thank you so much for going through all the trouble to come the other day. }

\par{14. ${\overset{\textnormal{せんじつ}}{\text{先日}}}$ はどうも(ありがとうございました)。 \hfill\break
 \emph{Senjitsu wa d }\emph{ōmo (arigat }\emph{ō gozaimashita). \hfill\break
 }Thank you for the other day.  }

\par{\textbf{Sentence Note }: Using the unabbreviated version is most appropriate in formal settings such as conversations in business. }

\par{15. この ${\overset{\textnormal{たび}}{\text{度}}}$ は ${\overset{\textnormal{まこと}}{\text{誠}}}$ にありがとうございました。 \hfill\break
\emph{Kono tabi wa makoto ni arigat }\emph{ō gozaimashita. \hfill\break
}Thank you so much for this occasion. }
 
\par{\textbf{Grammar Note }: The event in Ex. 15 would have already been completed at the time of utterance. }

\begin{center}
\textbf{Conjugation Recap } \hfill\break

\end{center}

\par{ As a recap of the forms we've seen so far, they\textquotesingle re listed again below. }

\begin{ltabulary}{|P|P|}
\hline 

Plain Non-Past & Polite Non-Past \\ \cline{1-2}

 \emph{Arigat }\emph{ō\slash arigato }ありがと(う) \hfill\break
\emph{Arigatai }ありがたい &  \emph{Arigato gozaimasu }ありがとうございます \hfill\break
\emph{Arigatai desu }ありがたいです \\ \cline{1-2}

Plain Past & Polite Past \\ \cline{1-2}

 \emph{Arigatakatta }ありがたかった &  \emph{Arigat }\emph{ō gozaimashita }ありがとうございました \hfill\break
\emph{Arigatakatta desu }ありがたかったです \\ \cline{1-2}

\end{ltabulary}

\par{\textbf{Variation Notes }: }

\par{1. \emph{Arigatakatta }ありがたかった(です) would be interpreted as “I appreciated it.” Similar, \emph{arigatai (desu) }ありがたい(です) is interpreted as “That\textquotesingle s appreciated.“ \hfill\break
2. Another respectful form is \emph{arigatō zonjimasu }ありがとう存じます, which is occasionally used by women who aim to use the politest phrases possible. Note that the \emph{t }\emph{ō }とう in \emph{arigat }\emph{ō }ありがとうis actually a contraction of \emph{-taku }たく. This is its adverbial form. }

\begin{center}
 \textbf{Thanking for Food }
\end{center}

\par{ When giving thanks upon receiving food, Japanese people say \emph{itadakimasu }頂きます. This literally means “I\textquotesingle m receiving (food).” The intonation of this phrase is い \textbf{ただきま }す. }

\par{16. では、 ${\overset{\textnormal{いただ}}{\text{頂}}}$ きます。 \hfill\break
 \emph{De wa, itadakimasu. \hfill\break
 }Well then, bon appetit! }

\par{ After finishing a meal, it is customary to give thanks again by saying \emph{go-chis }\emph{ō-sama deshita }ご馳走様でした. The word \emph{chis }\emph{ō }馳走 means “feast” and literally means “having to ride by horse to gather ingredients.” Although this is no longer modern reality, this expression gives recognition of the effort and quality put into the food that was given to you. Whenever you are familiar with the person, \emph{deshita }でした can be dropped, or you can simply say \emph{go-chis }\emph{ō }ご馳走. However, the shorter you make the expression, the stronger friendship you should have with the individual. }

\par{\textbf{Intonation Note }: The intonation of this phrase is ご \textbf{ちそうさまで }した. }

\par{17. ${\overset{\textnormal{きょう}}{\text{今日}}}$ は ${\overset{\textnormal{しょくじ}}{\text{食事}}}$ に ${\overset{\textnormal{さそ}}{\text{誘}}}$ ってくださって、ありがとうございました。ご ${\overset{\textnormal{ちそうさま}}{\text{馳走様}}}$ でした。 \hfill\break
 \emph{Ky }\emph{ō wa shokuji ni sasotte kudasatte, arigat }\emph{ō gozaimashita. Go-chis }\emph{ō-sama deshita. \hfill\break
 }Thank you very much for inviting me to dinner today. It was a wonderful meal. }

\par{\textbf{Sentence Notes }: \hfill\break
1. Although translated as “dinner,” \emph{shokuji }食事 simply means “meal” and can be used in the same sense as “dinner” would in English. \hfill\break
2. - \emph{te kudasatte }てくださって is the respectful version of \emph{-te kurete }てくれて. Its meaning of marking the kind action of someone outside one\textquotesingle s in-group remains the same. \hfill\break
3. The response to this phrase is \emph{o-somatsu-sama deshita }お粗末さまでした. Similarly, the speaker may reduce this phrase to either \emph{o-somatsu-sama }お粗末様 or just \emph{o-somatsu }お粗末 depending on how casual the tone is. This phrase may be used in replying to the use of any services other than just food and drink. For instance, it can be used at bath houses ( \emph{sentō }銭湯). }

\begin{center}
\textbf{\emph{Kansha shimasu }感謝します } \hfill\break

\end{center}

\par{ In addition to the phrases above surrounding \emph{arigatō }ありがとう, there is also the verb \emph{kansha suru }感謝する (to be grateful\slash thank you) that can be used. The noun \emph{kansha }感謝 means gratitude. In polite speech, this phrase is rendered as \emph{kansha shimasu }感謝します. }

\par{\textbf{Intonation Note }: The intonation of this phrase is \textbf{か }んしゃします. }

\par{18. ご ${\overset{\textnormal{きょうりょく}}{\text{協力}}}$ に ${\overset{\textnormal{ふか}}{\text{深}}}$ く ${\overset{\textnormal{かんしゃ}}{\text{感謝}}}$ します。 \hfill\break
 \emph{Go-ky }\emph{ōryoku ni fukaku kansha shimasu. \hfill\break
 }I am deeply grateful for your cooperation. }

\par{19a. ${\overset{\textnormal{こころ}}{\text{心}}}$ から ${\overset{\textnormal{かんしゃ}}{\text{感謝}}}$ しています。 \hfill\break
19b. ${\overset{\textnormal{こころ}}{\text{心}}}$ より ${\overset{\textnormal{かんしゃ}}{\text{感謝}}}$ しております。 \hfill\break
 \emph{Kokoro kara kansha shite imasu. \hfill\break
Kokoro yori kansha shite orimasu. }\hfill\break
I am profoundly grateful. }

\par{\textbf{Grammar Note }: \emph{Kokoro kara\slash yori }心\{から・より\} literally means “from the heart.” Using the particle \emph{yori }より is more respectful. Additionally, \emph{shite orimasu }しております is the humble form of \emph{shite imasu }しています. The use of these forms instead of just \emph{shimasu }します is done to emphasize how one\textquotesingle s state of gratitude has been an ongoing and continuing emotion. }

\par{20. ${\overset{\textnormal{ながねん}}{\text{長年}}}$ のご ${\overset{\textnormal{しじ}}{\text{支持}}}$ に ${\overset{\textnormal{かんしゃ}}{\text{感謝}}}$ しております。 \hfill\break
 \emph{Naganen no go-shiji ni kansha shite orimasu. \hfill\break
 }I am grateful for your long-time support. }

\begin{center}
\textbf{\emph{Shai }謝意 }\hfill\break

\end{center}

\par{ A very formal means of expressing gratitude that is frequently used in business settings, primarily in speeches and\slash or the written language is \emph{shai wo hyō suru }謝意を表する (to express gratitude). }

\par{\textbf{Intonation Note }: The intonation of this phrase is \textbf{しゃ }いをひょ \textbf{うしま }す. }

\par{21. ご ${\overset{\textnormal{こうい}}{\text{厚意}}}$ に ${\overset{\textnormal{こころ}}{\text{心}}}$ より ${\overset{\textnormal{しゃい}}{\text{謝意}}}$ を ${\overset{\textnormal{あらわ}}{\text{表}}}$ します。 \hfill\break
 \emph{Go-kōi ni kokoro yori shai wo hyō shimasu. }\hfill\break
I am profoundly grateful of your kindness. }

\begin{center}
\textbf{\emph{Sankyū }サンキュー }
\end{center}

\par{ Lastly, it's impossible to ignore how the English phrase “thank you” has made its way into Japanese as \emph{sanky }\emph{ū }サンキュー. A lot of speakers use this in conversation among friends. Online, it may even be seen colloquially spelled as 3Q or 三Q. At McDonald\textquotesingle s in Japan, \emph{sanky }\emph{ū }サンキュー has also incorporated a meaning of “understood,” mixing gratitude to the customer orders along with confirming what needs to be served. }

\par{22. スタッフA氏:オレンジジュースのSサイズおひとつですね。 \hfill\break
スタッフB氏:Sサイズのオレンジジュース、サンキュー! \hfill\break
 \emph{Sutaffu Ei-shi: Orenjij }\emph{ūsu no esu-saizu o-hitotsu desu ne. \hfill\break
Sutaffu Bii-shi: Esu-saizu no orenjij }\emph{ūsu, sanky }\emph{ū! }\hfill\break
Staff A: One small-sized orange juice, correct? \hfill\break
Staff B: Small orange juice, coming up! }

\begin{center}
\textbf{Dialectical Variation }
\end{center}

\par{ Lastly, there are some important variations of "thank you” that are widely known about and s till prevalent in their unique ways. \hfill\break
}

\par{ The most popular dialectical version of "thank you" in Japanese is \emph{ōkini }大きに, which is emb lematic of Kansai Dialects ( \emph{Kansai-ben }関西弁). Younger speakers tend to not gravitate towards this phrase, but it is still prevalent among older generations and among store clerks. }

\par{23. ( ${\overset{\textnormal{まいど}}{\text{毎度}}}$ ) ${\overset{\textnormal{だい}}{\text{大}}}$ きに。 \hfill\break
 \emph{(Maido) }\emph{ōkini. \hfill\break
 }Thank you (as always). }

\par{ In most of Western Japan \emph{, arigatō-san (desu) }ありがとうさん(です) is very prevalent. Variation exists as to whether the “o” is long or short. }

\par{24. ${\overset{\textnormal{みな}}{\text{皆}}}$ いつもありがと(う)さん(です)。 \hfill\break
 \emph{Mina itsumo arigatō\slash arigato-san (desu). \hfill\break
 }Thank you, everyone as always. }

\par{ In Yamagata Prefecture ( \emph{Yamagata-ken }山形県), the phrase \emph{mokke }もっけ is used. In Standard Japanese, this word can be found in the expression \emph{mokke no saiwai }勿怪の幸い, which means “windfall\slash piece of good luck.” In this dialect, the word is used to refer to a sense of gratitude that is seldom had. }

\par{25. あいや、もっけだちゃ! ${\overset{\textnormal{ほんとう}}{\text{本当}}}$ でありがどのお。(山形弁) \hfill\break
 \emph{Aiya, mokke da cha! Hontō de arigado nō.  (Yamagata-ben) \hfill\break
 }Oh wow, thank you so much! Really, thank you, thank you! (Yamagata Dialect) \hfill\break
\hfill\break
 In many parts of Northern Japan, \emph{arigatō }can be heard pronounced as \emph{arigado }ありがど. This is because non-voiced consonants tend to be voiced in the dialects spoken there. This is still the case for even younger speakers, but this depends on the exact locality. \hfill\break
 \hfill\break
 In prefectures such as Shimane ( \emph{Shimane-ken }島根県), Ehime ( \emph{Ehime-ken }愛媛県), Kumamoto ( \emph{Kumamoto-ken }熊本県), and Miyazaki ( \emph{Miyazaki-ken }宮崎県), the adverb \emph{dandan }だんだん has been used as an intensifier in conjunction with \emph{arigatō }ありがとう, so much so that it can stand for “thank you” by itself. Although this has died out of use, it is still widely known throughout Japan and is still used by older generations in those prefectures. }

\par{26. だんだん(ありがとう)。 \hfill\break
 \emph{Dandan (arigatō). \hfill\break
 }Thank you. }

\par{ In the Hokuriku Region ( \emph{Hokuriku Chihō }北陸地方), the phrase \emph{ki no doku }気の毒 can be heard used for “thank you.” In Standard Japanese, this is seen in phrases like \emph{o-ki no doku ni }お気の毒に, which is used to express sympathy for someone\textquotesingle s misfortune. We will learn more about how it is used in Standard Japanese in the next lesson. }

\par{27. ${\overset{\textnormal{き}}{\text{気}}}$ の ${\overset{\textnormal{どく}}{\text{毒}}}$ な。 \hfill\break
 \emph{Ki no doku na. \hfill\break
 }Thanks (even despite the trouble I put you through). }

\par{ In Okinawa, one must understand that the local indigenous dialects aren\textquotesingle t really dialects of Japanese. They are sister languages of Japanese. The Standard Japanese spoken in each individual locality will be influenced to some degree by these languages, however. In the main island of Okinawa, you will hear Ex. 28 used. }

\par{28. ${\overset{\textnormal{にふぇー}}{\text{御拝}}}$ でーびる。 \hfill\break
 \emph{Nifēdēbiru. \hfill\break
 }Thank you very much. }

\begin{center}
\textbf{You\textquotesingle re Welcome }
\end{center}

\par{ The standard direct translation of “you\textquotesingle re welcome” in Japanese is \emph{dō itashimashite }どういたしまして. This implies that the speaker hasn\textquotesingle t really done anything extraordinary, which is quite opposite of the nuance found in the English “you\textquotesingle re welcome.” Traditionally, this has been a rather humble expression, but in many circumstances people often interpret it as downplaying the situation at hand, which can make it seem that the speaker is of higher status than the listener. Because of this, speakers typically avoid using it, opting for expressions that emphasize how the speaker was only trying to help. }

\par{I \textbf{ntonation Note }: The intonation of this phrase is \textbf{ど }うい \textbf{たしま }して. }

\par{29. いえいえ(、お ${\overset{\textnormal{やく}}{\text{役}}}$ に ${\overset{\textnormal{た}}{\text{立}}}$ てれば ${\overset{\textnormal{なに}}{\text{何}}}$ よりです)。 \hfill\break
 \emph{Ieie, (o-yaku ni tatereba nani yori desu). \hfill\break
 }No, no, so long as I\textquotesingle ve been of any help. }

\par{30. どういたしまして。ご ${\overset{\textnormal{りよう}}{\text{利用}}}$ ありがとうございます。 \hfill\break
 \emph{D }\emph{ō itashimashite. Go-riy }\emph{ō arigat }\emph{ō gozaimasu. \hfill\break
 }You\textquotesingle re welcome. Thank you for using us. }

\par{\textbf{Sentence Note }: In customer service, \emph{d }\emph{ō itashimashite }どういたしまして is still used as, traditionally, it is meant to be humble. }

\par{31a. とんでもございません。お ${\overset{\textnormal{やく}}{\text{役}}}$ に ${\overset{\textnormal{た}}{\text{立}}}$ てれば ${\overset{\textnormal{うれ}}{\text{嬉}}}$ しいです。 \hfill\break
31b. とんでもないことでございます。お ${\overset{\textnormal{やく}}{\text{役}}}$ に ${\overset{\textnormal{た}}{\text{立}}}$ てれば ${\overset{\textnormal{うれ}}{\text{嬉}}}$ しいです。 \hfill\break
 \emph{Ton demo gozaimasen. O-yaku ni tatereba ureshii desu. \hfill\break
Ton demo nai koto de gozaimasu. O-yaku ni tatereba ureshii desu. \hfill\break
 }Don\textquotesingle t mention it. I\textquotesingle m glad if I can be of any help. }

\par{\textbf{Grammar Note }: Many speakers feel that \emph{ton demo nai koto de gozaimasu }とんでもないことでございます is more grammatical than \emph{ton demo gozaimasen }とんでもございません despite the fact that both are grammatical. }

\par{\textbf{Intonation Note }: と \textbf{んでもな }いです. }

\par{32. お ${\overset{\textnormal{やく}}{\text{役}}}$ に ${\overset{\textnormal{た}}{\text{立}}}$ てて ${\overset{\textnormal{さいわ}}{\text{幸}}}$ いです。 \hfill\break
 \emph{O-yaku ni tatete saiwai desu. \hfill\break
 }I\textquotesingle m happy to be of help. }

\par{33. お ${\overset{\textnormal{てつだ}}{\text{手伝}}}$ いできてよかったです。 \hfill\break
 \emph{O-tetsudai dekite yokatta desu. \hfill\break
 }I\textquotesingle m happy to have been able to help. }

\par{34. いや、とんでもないです。 \hfill\break
 \emph{Iya, ton demo nai desu. \hfill\break
 }Oh no, don\textquotesingle t mention it. }
    