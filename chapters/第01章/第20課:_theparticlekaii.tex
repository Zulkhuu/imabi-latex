    
\chapter{The Particle Ka か II}

\begin{center}
\begin{Large}
第20課: The Particle Ka か II: With the Negative 
\end{Large}
\end{center}
 
\par{ Asking questions in the negative is not straightforward in either English or Japanese. In fact, things can get rather complicated right off the bat with nuance. What exactly do you mean? Consider the following examples in English. }

\begin{enumerate}

\item Are you not going? \hfill\break

\item Is this not new? \hfill\break

\item Why not go? 
\end{enumerate}

\par{ The first example expresses surprise and doubt about the situation. The second example purposely uses the negative to make a more affirmative statement, and the third example is an invitation to do something. All these things translate quite well into Japanese. After this lesson, your knowledge of \emph{ka }か should become even stronger. }

\par{\textbf{Intonation Note }: Although this will not be stated again as we go through this lesson, the final syllable of a question in Japanese should have a high intonation. }
      
\section{Vocabulary List}
 
\par{\textbf{Nouns }}

\par{・年下 \emph{Toshishita }– Younger }

\par{・夕食 \emph{Y }\emph{ūshoku }– Dinner }

\par{・車 \emph{Kuruma }– Car }

\par{・事実 \emph{Jijitsu }- Fact }

\par{・一人暮らし \emph{Hitorigurashi }– Living alone }

\par{・トイレ \emph{Toire }– Bathroom\slash toilet }

\par{・パーティ \emph{P }\emph{āti }– Party }

\par{・コンサート \emph{Kons }\emph{āto }– Concert }

\par{・彼氏 \emph{Kareshi }– Boyfriend }

\par{・醤油 \emph{Sh }\emph{ōyu }– Soy sauce }

\par{・動物 \emph{D }\emph{ōbutsu }– Animal }

\par{・パイナップル \emph{Painappuru }– Pineapple }

\par{・海 \emph{Umi }– Sea }

\par{・タバコ \emph{Tabako }- Tobacco }

\par{・糖質 \emph{T }\emph{ōshitsu }– Carbohydrates }

\par{・折り紙 \emph{Origami }– Origami }

\par{・キャンディー \emph{Kyandii }– Candy }

\par{・宿題 \emph{Shukudai }– Homework }

\par{・ピアノ \emph{Piano }– Piano }

\par{・ピザ \emph{Piza }– Pizza }

\par{・カメ \emph{Kame }– Turtle }

\par{・おやつ \emph{Oyatsu }– Snacks }

\par{・洋服 \emph{Y }\emph{ōfuku }– Clothes }

\par{・タヌキ \emph{Tanuki }– Tanuki (Japanese raccoon dog) }

\par{\textbf{Pronouns }}

\par{・彼女 \emph{Kanojo }– She }

\par{\textbf{Proper Nouns }}

\par{・京子さん \emph{Ky }\emph{ōko-san }– Mr\slash M(r)s. Kyoko }

\par{・木下さん \emph{Kinoshita-san }– Mr\slash M(r)s. Kinoshita }
\textbf{Adjectives }
\par{・忙しい \emph{Isogashii }– Busy }

\par{・新しい \emph{Atarashii }– New }

\par{・真新しい \emph{Ma\textquotesingle atarashii }– Brand new }

\par{・悲しい \emph{Kanashii }– Sad }

\par{・寂しい \emph{Sabishii }– Lonely\slash desolate }

\par{・いい \emph{Ii }– Good }

\par{\textbf{Adjectival Nouns }}

\par{・好きだ \emph{Suki da }– To like }

\par{\textbf{Interrogatives (Question Words) }}

\par{・何 \emph{Nani\slash nan }– What }
 
\par{\textbf{Idioms }}

\par{・いい加減にする \emph{Ii kagen ni suru }– To get…over with }

\par{\textbf{Interjections }}

\par{・はい \emph{Hai }– Yes }

\par{・い(い)え \emph{I(i)e }– No }

\par{・あ \emph{A }– Ah }

\par{\textbf{Adverbs }}

\par{・毎日 \emph{Mainichi }– Every day }

\par{・そう \emph{S }\emph{ō }– So }

\par{・ちょっと \emph{Chotto }– A little }

\par{・一緒に \emph{Issho ni }– Together }

\par{・昨日 \emph{Kin }\emph{ō }- Yesterday }

\par{・すぐに \emph{Sugu ni }- Immediately }

\par{・明日 \emph{Ashita\slash asu }– Tomorrow }

\par{・さっき \emph{Sakki }– Just now }

\par{\textbf{\emph{(ru) Ichidan }Verbs }}

\par{・やめる \emph{Yameru }– To quit (trans.) }

\par{・漬ける \emph{Tsukeru }– To pickle (trans.) }

\par{・食べる \emph{Taberu }– To eat (trans.) }

\par{・見る \emph{Miru }– To see\slash look (trans.) }

\par{\textbf{\emph{(u) Godan }Verbs }}

\par{・使う \emph{Tsukau }– To use (trans.) }

\par{・行く \emph{Iku }– To go (intr.) }

\par{・祈る \emph{Inoru }– To pray (trans.) }

\par{・遊ぶ \emph{Asobu }– To play (trans.) }

\par{・泳ぐ \emph{Oyogu }– To swim (intr.) }

\par{・減らす \emph{Herasu }– To decrease (trans.) }

\par{・飲む \emph{Nomu }– To drink\slash swallow\slash take (medicine) (trans.) }

\par{・作る \emph{Tsukuru }– To make (trans.) }

\par{・やる \emph{Yaru }– To do (casual) (trans.) }

\par{・終わる \emph{Owaru }– To end (intr.) }

\par{・選ぶ \emph{Erabu }– To choose (trans.) }

\par{・弾く \emph{Hiku }– To play (piano\slash guitar) (trans.) }

\par{・買う \emph{Kau }– To buy (trans.) }

\par{・知る \emph{Shiru }- To know\slash recognize (trans.) }

\par{\textbf{Demonstratives }}

\par{・この \emph{Kono }– This (adj.) }

\par{・その \emph{Sono }– That (adj.) }

\par{・あれ \emph{Are }– That over there (noun) }
      
\section{-nai (no) desu ka? ない(の)ですか}
 
\par{ When \emph{ka }か follows \emph{-nai desu }ないです, the resultant question translates to “is it not…”  When one is rather certain of the answer being the affirmative, this pattern stays as in. However, when there is any degree of doubt, the particle \emph{no }の usually intervenes. In conversation, the particle \emph{no }の is often contracted to \emph{n }ん in this pattern, resulting in \emph{-nai n desu ka }ないんですか? }

\par{1. ${\overset{\textnormal{かのじょ}}{\text{彼女}}}$ は ${\overset{\textnormal{いそが}}{\text{忙}}}$ しくないのですか。 \hfill\break
 \emph{Kanojo wa isogashikunai no desu ka? }\hfill\break
Is she not busy? }

\par{2. ${\overset{\textnormal{きょうこ}}{\text{京子}}}$ さんは ${\overset{\textnormal{としした}}{\text{年下}}}$ じゃないんですか。 \hfill\break
 \emph{Kyōko-san wa toshishita ja nai n desu ka? }\hfill\break
Is Mrs\slash Ms. Kyoko not younger? }

\par{3. ${\overset{\textnormal{ゆうしょく}}{\text{夕食}}}$ を ${\overset{\textnormal{た}}{\text{食}}}$ べないのですか。 \hfill\break
 \emph{Yūshoku wo tabenai no desu ka? }\hfill\break
Are you not going to have dinner? }

\par{4. ${\overset{\textnormal{ひとりぐ}}{\text{一人暮}}}$ らしは ${\overset{\textnormal{さび}}{\text{寂}}}$ しくないですか。 \hfill\break
 \emph{Hitorigurashi wa sabishikunai desu ka? \hfill\break
}Isn\textquotesingle t living alone lonely? }

\par{5. その ${\overset{\textnormal{じじつ}}{\text{事実}}}$ は ${\overset{\textnormal{かな}}{\text{悲}}}$ しくないですか。 \hfill\break
 \emph{Sono jijitsu wa kanashikunai desu ka. }\hfill\break
Is that fact not sad? }

\par{6. この ${\overset{\textnormal{くるま}}{\text{車}}}$ は ${\overset{\textnormal{あたら}}{\text{新}}}$ しくないですか。 \hfill\break
 \emph{Kono kuruma wa atarashikunai desu ka? }\hfill\break
Isn\textquotesingle t this car new? }

\par{ The use of \emph{no\slash n }の・ん provides emphasis on the question element and adds weight to the overall statement. When you are asking about intention and you don\textquotesingle t use \emph{no\slash n }の・ん, then your question sounds more like a solicitation because you are in effect telling the person that you\textquotesingle re pretty certain that he\slash she is doing so. }

\par{7. パーティに ${\overset{\textnormal{い}}{\text{行}}}$ かないんですか。 \hfill\break
 \emph{Pāti ni ikanai n desu ka? }\hfill\break
Are you not going to the party? }

\par{8. コンサートに ${\overset{\textnormal{い}}{\text{行}}}$ かないですか。 \hfill\break
 \emph{Konsāto ni ikanai desu ka? }\hfill\break
Why not go to a\slash the concert? }

\par{\textbf{Particle Note }: In the example sentences here, the particle \emph{ni }に is used to mark where at one is going. }

\par{ As has been demonstrated thus far, \emph{no\slash n }の・ん doesn\textquotesingle t simply add doubt to a question, it also demonstrates the speaker\textquotesingle s interest to the answer in the question as an effect. After all, you aren\textquotesingle t 100\% as to what the answer will be. This is also true when \emph{no\slash n desu ka }の・んですか is used with the affirmative. }

\par{9. ${\overset{\textnormal{かれし}}{\text{彼氏}}}$ は ${\overset{\textnormal{い}}{\text{行}}}$ くんですか。 \hfill\break
 \emph{Kareshi wa iku n desu ka? }\hfill\break
Will your boyfriend go? }

\par{10. ${\overset{\textnormal{しょうゆ}}{\text{醤油}}}$ を ${\overset{\textnormal{つか}}{\text{使}}}$ うんですか。 \hfill\break
 \emph{Shōyu wo tsukau n desu ka? \hfill\break
 }Will you be using soy sauce? }

\par{11. ${\overset{\textnormal{まいにちいの}}{\text{毎日祈}}}$ るんですか。 \hfill\break
 \emph{Mainichi inoru n desu ka? }\hfill\break
Do you pray every day? }

\begin{center}
\textbf{The Japanese Yes \& No }
\end{center}

\par{12. 「そう ${\overset{\textnormal{おも}}{\text{思}}}$ わないですか」「いえ、そうは ${\overset{\textnormal{おも}}{\text{思}}}$ いません。」 \hfill\break
 \emph{ }\emph{“Sō omowanai desu ka” “Ie, sō wa omoimasen.” \hfill\break
}“Don\textquotesingle t you think so?” “No, I don\textquotesingle  think so.” }

\par{\textbf{Particle Note }: The particle \emph{wa }は is used here to emphasize the negative situation of not thinking as such. }

\par{ When responded to a question in the affirmative\slash negative in Japanese, your answer is about whether it is or is not so first and foremost. In this example, the negative is used for the affirmative, which explains why “no” began the second speaker\textquotesingle s response; however, consider the following. }

\par{13. 「 ${\overset{\textnormal{どうぶつ}}{\text{動物}}}$ が ${\overset{\textnormal{す}}{\text{好}}}$ きじゃないのですか」「はい、 ${\overset{\textnormal{す}}{\text{好}}}$ きじゃありません。」 \hfill\break
 \emph{“Dōbutsu ga suki ja nai no desu ka?” “Hai, suki ja arimasen.” \hfill\break
}“Do you not like any animals?” “No, I don\textquotesingle t like them.” }

\par{ As you can see, the answer in Japanese was actually “yes” because it affirms that the doubt\slash question that the first speaker had about the second speaker not liking animals. }
      
\section{-masen ka ませんか}
 
\par{ Just as \emph{-nai desu ka }ないですか can be used to solicit, \emph{-masen ka }ませんか is perhaps the most common way in polite speech to invite someone to do something. When used with verbs that describes states (verbs that don't express an action), this is used as a polite means of asking the affirmative. }

\par{14. パイナップルを ${\overset{\textnormal{つ}}{\text{漬}}}$ けませんか。 \hfill\break
 \emph{Painappuru wo tsukemasen ka? \hfill\break
}\emph{ }Why not pickle pineapple? }

\par{15. ちょっと ${\overset{\textnormal{あそ}}{\text{遊}}}$ びませんか。 \hfill\break
 \emph{Chotto asobimasen ka? }\hfill\break
Why not have some fun? }

\par{16. ${\overset{\textnormal{うみ}}{\text{海}}}$ で ${\overset{\textnormal{およ}}{\text{泳}}}$ ぎませんか。 \hfill\break
 \emph{Umi de oyogimasen ka? }\hfill\break
Why not swim in the ocean? }

\par{17. タバコをやめませんか。 \hfill\break
 \emph{Tabako wo yamemasen ka? }\hfill\break
How about quitting smoking? }

\par{18. ${\overset{\textnormal{とうしつ}}{\text{糖質}}}$ を ${\overset{\textnormal{へ}}{\text{減}}}$ らしませんか。 \hfill\break
 \emph{Tōshitsu wo herashimasen ka? \hfill\break
}\emph{ }How about decreasing carbohydrates? }

\par{19.知りませんか。 \hfill\break
 \emph{Shirimasen ka? }\hfill\break
Do you happen to know? }
      
\section{-nai (ka)? ない(か)}
 
\par{ In plain speech, you can solicit someone with the negative with \emph{-nai (ka)? }ない(か). The use of \emph{ka }か indicates that you are very familiar with the speaker. As we learned last lesson, this tends to be used by male speakers, and if the listener is not either equal or lower in status than oneself, the invite could be taken badly at the least. }

\par{20. ${\overset{\textnormal{お}}{\text{折}}}$ り ${\overset{\textnormal{がみ}}{\text{紙}}}$ を ${\overset{\textnormal{つく}}{\text{作}}}$ らないか? \hfill\break
 \emph{Origami wo tsukuranai ka? \hfill\break
}\emph{ }Why not make some origami? }

\par{21. ${\overset{\textnormal{いっしょ}}{\text{一緒}}}$ に ${\overset{\textnormal{の}}{\text{飲}}}$ まない? \hfill\break
 \emph{Issho ni nomanai? }\hfill\break
How about we drink together? }

\par{22. いい ${\overset{\textnormal{かげん}}{\text{加減}}}$ にしないか。(Idiom) \hfill\break
 \emph{Ii kagen ni shinai ka? }\hfill\break
Can\textquotesingle t you give me a break? }

\par{23. キャンディー、 ${\overset{\textnormal{た}}{\text{食}}}$ べない? \hfill\break
 \emph{Kyandii, tabenai? \hfill\break
}\emph{ }Why not eat some candy? }

\par{\textbf{Particle Note }: This sentence shows how the particle \emph{wo }を can be dropped in casual conversation. }
      
\section{-nai no ka? ないのか}
 
\par{\emph{ - }\emph{nai no ka }ないのか is used in plain speech to seriously and or harshly ask a question about whether something truly isn\textquotesingle t so. This is used frequently by both male and female speakers, especially when emotions are flaring. Thus, caution is needed when using it. }

\par{24. ${\overset{\textnormal{み}}{\text{見}}}$ ないのか。 \hfill\break
 \emph{Minai no ka? }\hfill\break
You\textquotesingle re not going to watch? }

\par{25. すぐにやめないのか。 \hfill\break
 \emph{Sugu ni yamenai no ka? \hfill\break
}\emph{ }You\textquotesingle re not going to stop immediately? }

\par{26. ${\overset{\textnormal{きのう}}{\text{昨日}}}$ 、 ${\overset{\textnormal{しゅくだい}}{\text{宿題}}}$ やらなかったのか。 \hfill\break
\emph{Kinō, shukudai yaranakatta no ka. \hfill\break
}\emph{ }You didn\textquotesingle t do your homework yesterday? }

\begin{center}
\textbf{\emph{-nai no? }ないの }
\end{center}

\par{ When the particle \emph{ka }か is dropped and you're left with \emph{-nai no }ないの, the question becomes a lot softer, but you are still seriously asking the question. The same goes for when this is used with the affirmative. }

\par{27. ピアノ ${\overset{\textnormal{ひ}}{\text{弾}}}$ かないの? \hfill\break
 \emph{Piano hikanai no? \hfill\break
}\emph{ }You\textquotesingle re not going to\slash won\textquotesingle t play the piano? }

\par{28. ${\overset{\textnormal{お}}{\text{終}}}$ わらないの? \hfill\break
 \emph{Owaranai no? }\hfill\break
Is not going to end\slash are you ever going to finish? }

\par{29. ${\overset{\textnormal{なに}}{\text{何}}}$ を ${\overset{\textnormal{えら}}{\text{選}}}$ ぶの? \hfill\break
 \emph{Nani wo erabu no? \hfill\break
 }What\textquotesingle ll you choose? }

\par{30. ${\overset{\textnormal{あした}}{\text{明日}}}$ ${\overset{\textnormal{い}}{\text{行}}}$ くの? \hfill\break
 \emph{Ashita iku no? }\hfill\break
Are you going tomorrow? }

\par{31. ピザを ${\overset{\textnormal{た}}{\text{食}}}$ べたの? \hfill\break
 \emph{Piza wo tabeta no? \hfill\break
}\emph{ }Did you eat pizza? }
      
\section{Verb + (n) ja nai desu ka\slash ja arimasen ka (ん)じゃないですか・じゃありませんか}
 
\par{ When \emph{ja nai desu ka\slash ja arimasen ka }じゃないですか・じゃありませんか follows a verb\slash adjective\slash copula, it is used to mean “isn\textquotesingle t it (the case that…)?” The lack of ん, as was the case before, indicates greater confidence in one\textquotesingle s statement being right in the first place. This may follow any of the conjugations we\textquotesingle ve studied thus far, but when it is after the non-past form of the copula, the copula must become \emph{na }な.This is because \emph{no\slash n }の・ん actually functions as a nominalizer in all the examples here in this lesson. }

\par{32. あれはカメ \textbf{なんじゃないですか }? \hfill\break
 \emph{Are wa kame na n ja nai desu ka? }\hfill\break
Isn\textquotesingle t that over there a turtle? }

\par{33. おやつ ${\overset{\textnormal{た}}{\text{食}}}$ べ \textbf{ないんじゃない }? \hfill\break
 \emph{Oyatsu tabenai n ja nai? }\hfill\break
Aren\textquotesingle t you not going to eat the snacks? }

\par{34. ${\overset{\textnormal{あした}}{\text{明日}}}$ ${\overset{\textnormal{か}}{\text{買}}}$ うじゃない? \hfill\break
 \emph{Ashita kau ja nai? } \hfill\break
Aren't you buying it tomorrow? }

\par{35. この ${\overset{\textnormal{ようふく}}{\text{洋服}}}$ 、 ${\overset{\textnormal{まあたら}}{\text{真新}}}$ しい \textbf{んじゃありませんか }。 \hfill\break
 \emph{Kono yōfuku, ma\textquotesingle atarashii ja arimasen ka? }\hfill\break
Aren\textquotesingle t these clothes brand new? }

\par{36. いい \textbf{んじゃないですか }。 \hfill\break
 \emph{Ii n ja nai desu ka? }\hfill\break
Isn\textquotesingle t that fine? }

\par{37. いいんじゃない! \hfill\break
 \emph{Ii n ja nai! }\hfill\break
That\textquotesingle s great! }

\par{38. さっき ${\overset{\textnormal{か}}{\text{買}}}$ ったんじゃない(の)? \hfill\break
 \emph{Sakki katta n ja nai (no)? \hfill\break
}\emph{ }Didn\textquotesingle t you buy (that) just now? }

\par{\textbf{Particle Note }: As is demonstrated in the previous example, the particle \emph{no }の may follow \emph{n ja nai }んじゃない to add a final hook to the seriousness of one\textquotesingle s question. }

\par{ After \emph{verbs and adjectives }, じゃ is hardly ever uncontracted as では in the spoken language, but the entire phrase \emph{ja nai desu ka }じゃないですか can actually be simplified to \emph{jan (ka) }じゃん(か). This is a general contraction that may be applied to what we\textquotesingle ve already covered. }

\par{39. あ、 ${\overset{\textnormal{きのした}}{\text{木下}}}$ さんではないですか。 \hfill\break
 \emph{A, Kinoshita-san de wa nai desu ka! }\hfill\break
My, isn\textquotesingle t it Mr\slash M(r)s. Kinoshita! }

\par{40.  いいじゃん! \hfill\break
 \emph{Ii jan! }\hfill\break
That\textquotesingle s awesome! }

\par{41. ${\overset{\textnormal{い}}{\text{行}}}$ くんじゃん(か)。 \hfill\break
 \emph{Iku n jan (ka)? }\hfill\break
Aren\textquotesingle t you going, right? }

\par{42. あれはタヌキじゃん! \hfill\break
 \emph{Are wa tanuki jan! }\hfill\break
Well is that not a tanuki! }
    