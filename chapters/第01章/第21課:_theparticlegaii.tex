    
\chapter{The Particle Ga が II}

\begin{center}
\begin{Large}
第21課: The Particle Ga が II: The Object Marker Ga が 
\end{Large}
\end{center}
 
\par{ In Lesson 15, we learned how the particle \emph{wo }を marks the direct object of a sentence. Direct objects are typically acted upon by an active agent (doer). Usually, there is someone or something that is purposefully and willingly exerted his\slash her\slash its intent on another entity. }

\par{i. Sam \emph{chopped } \textbf{the lettuce }. \hfill\break
ii. Sarah \emph{threw }\textbf{the ball }. \hfill\break
iii. Dusty \emph{petted } \textbf{the cat }. }

\par{ However, not all verbs involve an activity with an active agent. Some verbs simply express a stative state. In these situations, the agent isn\textquotesingle t necessarily exerting his\slash her control over the object. The relation between agent and object, then, can be viewed as a mere statement of reality. }

\par{iv. I \emph{understand } \textbf{the situation }. \hfill\break
v. I \emph{have } \textbf{three dogs }. \hfill\break
vi. I \emph{need } \textbf{money }. \hfill\break
 \hfill\break
 In English, these verbs (in italics) are all treated as transitive verbs because they have objects (in bold). However, their meanings are stative in nature. A stative verb is one that expresses a state\slash condition rather than an activity. In English, stative verbs can either be intransitive or transitive depending on the existence of an object or the lack thereof. }

\par{vii. There are horses here. (intransitive) \hfill\break
viii. The flag stands still. (intransitive) \hfill\break
ix. The fire burns brightly. (intransitive) \hfill\break
x. Everyone wants money. (transitive) \hfill\break
xi. I like dogs. (transitive) \hfill\break
xii. I hate cats. (transitive) }

\par{ In English, objects are typically linked to verbs of activity, but this is not the case for stative-transitive verbs. In fact, these verbs share much in common with adjectives. After all, adjectives are primarily used in expressing the state\slash condition of something. }

\par{xiii. I\textquotesingle m good at math. \hfill\break
xiv. I\textquotesingle m bad at physics. \hfill\break
xv. Spiders are scary. \hfill\break
xvi. I\textquotesingle m scared of spiders. }

\par{ Whether it be a stative-transitive verb or an adjective with an object, they both constitute what are called stative-transitive predicates. In Japanese, the objects of these so-called “stative-transitive predicates” are marked by \emph{ga }が rather than \emph{wo }を. The types that exist include those concerned with human attributes such as perception, necessity, possession, desire, etc. These attributes are outside the realm of the subject\textquotesingle s control. Even when someone “wants something,” the want is treated as an emotion the speaker can\textquotesingle t control. This lack of control prompts the use of \emph{ga }が instead of \emph{wo }を. }

\par{ Stative-transitive predicates in Japanese involve intransitive verbs, adjectives, or adjectival nouns. The stative-transitive verbs of English are not transitive verbs in Japanese. Instead, they either correspond to intransitive verbs or adjectives\slash adjectival nouns. In Japanese, \emph{jid }\emph{ōshi }自動詞 refers not only to verbs that have no objects, but also to verbs whose objects are marked by \emph{ga }が. The term \emph{tad }\emph{ōshi }他動詞 is reserved only to verbs whose objects are marked by \emph{wo }を. Lastly, when an English stative-transitive verb corresponds to an adjective\slash adjectival noun, it\textquotesingle s because of differences in morphology between the two languages. }

\par{ Now that we\textquotesingle ve learned what stative-transitive predicates are in English, it\textquotesingle s time to see what they look like in Japanese. These predicates will be divided into two broad categories with further semantic divisions: }

\begin{center}
\textbf{Stative-Transitive Predicates of Objective Fact }
\end{center}

\par{・Possession \hfill\break
・Necessity \hfill\break
・Non-intentional Perception \hfill\break
・Non-intentional Understanding \hfill\break
・Ability }
\textbf{Stative-Transitive Predicates of Subjective Emotions }・Internal Feeling \hfill\break
・Like\slash Dislike \hfill\break
・Want\slash Desire \hfill\break
・Competence 
\par{\textbf{Curriculum Note }: Some of these categories exhibit interchangeability between \emph{ga }が and \emph{wo }を. In this lesson, we will focus first on the instances \emph{ga }が can and does mark the object of a sentence. For more information about the interchangeability between \emph{ga }が and \emph{wo }を, you will be directed to later lessons. }
      
\section{Stative-Transitive Predicates of Objective Fact}
 
\par{ The subject\slash agent of a stative-transitive predicate is limited to nouns\slash pronouns which refer to people or things that are personified. Often, the subject and topic are the same, in which case the subject becomes a zero-pronoun, and because it isn\textquotesingle t spoken, there is no issue with two \emph{ga }が being used in the same sentence. However, the exhaustive-listing function of \emph{ga }が can still manifest in stative-transitive predicates. When this happens, the first \emph{ga }が marks the subject and the second \emph{ga }が marks the object (thing involved). }

\par{\textbf{Curriculum Note }: For some stative-transitive predicates, especially those of this type, the subject (experiencer) may alternatively be marked by \emph{ni (wa) }に(は). The use of \emph{ni (wa) }に(は) adds various kinds of emotions\slash semantic nuancing not intrinsic to the meanings of these sort of predicates. Because of this, we will discuss how \emph{ni(wa) }に(は) can alternatively mark the subject of stative-transitive predicates in a later lesson. }

\par{・Possession: ある \& いる }

\par{ On top of meaning “to be,” the verbs \emph{aru }ある and \emph{iru }いる can also mean “to have.” Aru is used to show possession of inanimate objects that may or may not be alive. \emph{Iru }おる is primarily used to show possession of human relations. Although we will be avoiding instances of \emph{ni wa }には marking the subject in this lesson, it is impossible to ignore for these two verbs. Whenever the subject is stated (not omitted), the use of \emph{ga }が brings about the exhaustive-listing meaning, and the use of \emph{wa }は bring out its contrastive meaning. When \emph{ni wa }には is used, it is very like saying “X for one.” }

\par{ These two verbs primarily show existence, and this is reflected in their secondary meaning of “to have,” so much so that the use of \emph{ni wa }には can still be literally interpreted as showing the place where possession of something exists.” The connection between the possessor and possessed entity becomes heavily emphasized as an effect, which is what brings out the translation “X for one.” }

\par{1. ${\overset{\textnormal{わたし}}{\text{私}}}$ (に)は ${\overset{\textnormal{うでどけい}}{\text{腕時計}}}$ があります。 \hfill\break
 \emph{Watashi (ni) wa udedokei ga arimasu. \hfill\break
 }I (for one) have an arm watch. }

\par{2. ${\overset{\textnormal{わたし}}{\text{私}}}$ たち(に)は ${\overset{\textnormal{じゅうぶん}}{\text{十分}}}$ なお ${\overset{\textnormal{かね}}{\text{金}}}$ があります。 \hfill\break
 \emph{Watashitachi (ni) wa j }\emph{ūbun na okane ga arimasu. \hfill\break
 }We (for one) have enough money. }

\par{3. ( ${\overset{\textnormal{わたし}}{\text{私}}}$ 【には・は】) ${\overset{\textnormal{あに}}{\text{兄}}}$ がいます。 \hfill\break
 \emph{(Watashi [ni wa\slash wa]) ani ga imasu. \hfill\break
 }[I, for one,\slash As for me,] have an older brother. }

\par{4. ${\overset{\textnormal{わたし}}{\text{私}}}$ (に)は ${\overset{\textnormal{ともだち}}{\text{友達}}}$ が【います・いません】。 \hfill\break
 \emph{Watashi (ni) wa tomodachi ga [imasu\slash imasen]. \hfill\break
 }I (for one) [have\slash have no] friends. }

\begin{center}
\textbf{Having Pets }
\end{center}

\par{ Having pets\slash livestock is usually expressed with \emph{katte iru }飼っている. }

\par{5. ${\overset{\textnormal{わたし}}{\text{私}}}$ は【 ${\overset{\textnormal{いぬ}}{\text{犬}}}$ ・ ${\overset{\textnormal{ねこ}}{\text{猫}}}$ ・ウサギ・ ${\overset{\textnormal{かめ}}{\text{亀}}}$ ・アライグマ・ ${\overset{\textnormal{きつね}}{\text{狐}}}$ ・ ${\overset{\textnormal{しか}}{\text{鹿}}}$ 】を ${\overset{\textnormal{か}}{\text{飼}}}$ っています。 \hfill\break
 \emph{Watashi wa [inu\slash neko\slash usagi\slash kame\slash araiguma\slash kitsune\slash shika] wo katte imasu. \hfill\break
 }I have a [dog\slash cat\slash rabbit\slash turtle\slash raccoon\slash fox\slash deer]. }

\par{6. ${\overset{\textnormal{わたくし}}{\text{私}}}$ には ${\overset{\textnormal{ねこ}}{\text{猫}}}$ がいます。 \hfill\break
 \emph{Watashi ni wa neko ga imasu. \hfill\break
 }I, for one, have a cat. }

\par{\textbf{Sentence Note }: Students are often tempted to use \emph{iru }いる to express having pets. Although this isn\textquotesingle t technically wrong, \emph{ni wa }には would need to mark the subject, and the sense of “for one” becomes heavily emphasized. }

\par{7. 【 ${\overset{\textnormal{わ}}{\text{我}}}$ が ${\overset{\textnormal{や}}{\text{家}}}$ ・ ${\overset{\textnormal{いえ}}{\text{家}}}$ 】には ${\overset{\textnormal{ねこ}}{\text{猫}}}$ がいます。 \hfill\break
 \emph{[Wagaya\slash ie] ni wa neko ga imasu. \hfill\break
 }I have a cat at my\slash our house. }

\par{\textbf{Sentence Note }: A more practical way to say one has a cat or any other kind of pet is to say that you have it at one\textquotesingle s home. This will mean that you\textquotesingle re using \emph{iru }いる in the literal sense of the pet existing at a certain location but also implying ownership because of where that location happens to be, your home. }

\par{・Necessity: \emph{Iru }要る }

\par{ Pronounced the same as \emph{iru }いる but is instead a \emph{Godan }五段 verb, \emph{iru }要る means “to need.”  It is used to express a need and\slash or want that is felt to be a certainty. It is used most with nouns regarding time, resources, money, etc. }

\par{8. ${\overset{\textnormal{にゅうじょうりょう}}{\text{入場料}}}$ は ${\overset{\textnormal{い}}{\text{要}}}$ らないですが、 ${\overset{\textnormal{の}}{\text{乗}}}$ り ${\overset{\textnormal{もの}}{\text{物}}}$ はお ${\overset{\textnormal{かね}}{\text{金}}}$ が ${\overset{\textnormal{い}}{\text{要}}}$ ります。 \hfill\break
 \emph{Ny }\emph{ūj }\emph{ōry }\emph{ō wa iranai desu ga, norimono wa okane ga irimasu. \hfill\break
 }Admission fee isn\textquotesingle t needed, but you need money for the rides. }

\par{9. ${\overset{\textnormal{さいしょ}}{\text{最初}}}$ は ${\overset{\textnormal{すこ}}{\text{少}}}$ しコツが ${\overset{\textnormal{い}}{\text{要}}}$ ります。 \hfill\break
 \emph{Saisho wa sukoshi kotsu ga irimasu. \hfill\break
 }Some skill is needed from the start. }

\par{10.このお ${\overset{\textnormal{かし}}{\text{菓子}}}$ 、 ${\overset{\textnormal{い}}{\text{要}}}$ る? \hfill\break
 \emph{Kono okashi, iru? \hfill\break
 }You really want this candy? }

\par{11. 【 ${\overset{\textnormal{ばん}}{\text{晩}}}$ ご ${\overset{\textnormal{はん}}{\text{飯}}}$ ・ ${\overset{\textnormal{よる}}{\text{夜}}}$ ご ${\overset{\textnormal{はん}}{\text{飯}}}$ 】、 ${\overset{\textnormal{い}}{\text{要}}}$ る? ${\overset{\textnormal{い}}{\text{要}}}$ らない? \hfill\break
 \emph{[Bangohan\slash yorugohan], iru? Iranai? \hfill\break
 }Do you need dinner or no? }

\begin{center}
\textbf{\emph{Iru }要るVS \emph{Hitsuy }\emph{ō da }必要だ }
\end{center}

\par{\emph{ Iru }要る is similar to the expression \emph{hitsuy }\emph{ō da }必要だ, which means “to be necessary.” \emph{Hitsuy }\emph{ō da }必要だ is far more broad in usage as anything could be presented as being a necessity\slash must, whereas \emph{iru }要る is typically limited to nouns that can be conceptualized as some sort of resource. To “need” has a somewhat subjective tone to it, as is also the case with \emph{iru }いる, but \emph{hitsuy }\emph{ō da }必要だ can be used in very objective contexts. }

\par{12. ${\overset{\textnormal{ちゅうい}}{\text{注意}}}$ が ${\overset{\textnormal{ひつよう}}{\text{必要}}}$ です。 \hfill\break
 \emph{Ch }\emph{ūi ga hitsuy }\emph{ō desu. \hfill\break
 }Caution is necessary. }

\par{13. お ${\overset{\textnormal{かね}}{\text{金}}}$ が ${\overset{\textnormal{ひつよう}}{\text{必要}}}$ だ。 \hfill\break
 \emph{Okane ga hitsuy }\emph{ō da. \hfill\break
 }Money is needed\slash necessary. }

\par{・Non-intentional Perception: \emph{Mieru }見える \& \emph{Kikoeru }聞こえる }

\par{ The verbs \emph{mieru }見える and \emph{kikoeru }聞こえる express that something is visible and audible respectively. Neither verb implies volition. Rather, they merely express the natural phenomena of sight and hearing. They are typically translated as “can see” and “can hear,” respectively, but it may be best to perceive them as meaning “visible” and “audible” so that you don\textquotesingle t accidentally attribute volition to them. }

\par{14. ${\overset{\textnormal{やま}}{\text{山}}}$ の ${\overset{\textnormal{けしき}}{\text{景色}}}$ が ${\overset{\textnormal{み}}{\text{見}}}$ えます。 \hfill\break
 \emph{Yama no keshiki ga miemasu. \hfill\break
 }The mountain scenery is visible. \hfill\break
I\slash one can see the mountain scenery. }

\par{15. ${\overset{\textnormal{なに}}{\text{何}}}$ が ${\overset{\textnormal{み}}{\text{見}}}$ えますか。 \hfill\break
 \emph{Nani ga miemasu ka? \hfill\break
 }What\textquotesingle s visible? \hfill\break
What can you see? \hfill\break
 \hfill\break
16. ${\overset{\textnormal{き}}{\text{聞}}}$ こえましたか? \hfill\break
 \emph{Kikoemashita ka? \hfill\break
 }Did you hear it? }

\par{17. ${\overset{\textnormal{おお}}{\text{大}}}$ きな ${\overset{\textnormal{ばくはつ}}{\text{爆発}}}$ が ${\overset{\textnormal{き}}{\text{聞}}}$ こえました。 \hfill\break
 \emph{Ōkina bakuhatsu ga kikoemashita. \hfill\break
 }I heard a large explosion. }

\par{・Non-intentional Understanding: 分かる }

\par{ There are two different meanings of \emph{wakaru }分かる. It can be used to mean “to understand” or “to become known.” For the first meaning, it's clear that \emph{wakaru }分かる falls under the same syntactic situation as other stative-transitive predicates. There is an object to understanding, but the understanding is out of the volition of the person who knows, thus the object is marked with \emph{ga }が. However, for the meaning “to become known,” \emph{ga }が functions more as a subject marker than object marker. }

\par{18. ${\overset{\textnormal{にほんご}}{\text{日本語}}}$ が ${\overset{\textnormal{わ}}{\text{分}}}$ かりますか。 \hfill\break
 \emph{Nihongo ga wakarimasu ka? \hfill\break
 }Do you understand Japanese? }

\par{19. ${\overset{\textnormal{わたし}}{\text{私}}}$ は ${\overset{\textnormal{かんこくご}}{\text{韓国語}}}$ が ${\overset{\textnormal{わ}}{\text{分}}}$ かりません。 \hfill\break
 \emph{Watashi wa kankokugo ga wakarimasen. \hfill\break
 }I don\textquotesingle t understand Korean. }

\par{20. ${\overset{\textnormal{かんとうじん}}{\text{関東人}}}$ は ${\overset{\textnormal{かんさいべん}}{\text{関西弁}}}$ が ${\overset{\textnormal{わ}}{\text{分}}}$ かりますか。 \hfill\break
 \emph{Kant }\emph{ōjin wa Kansaiben ga wakarimasu ka? \hfill\break
 }Do people from Kanto understand the Kansai dialect? }

\par{\textbf{Word Note }: \emph{Kant }\emph{ō }関東 is the region of Japan where Tokyo resides. }

\par{21. その ${\overset{\textnormal{げんいん}}{\text{原因}}}$ が ${\overset{\textnormal{わ}}{\text{分}}}$ かりました。 \hfill\break
 \emph{Sono gen\textquotesingle in ga wakarimashita. \hfill\break
 }The cause has become known. \hfill\break
 \hfill\break
22. ${\overset{\textnormal{ぼく}}{\text{僕}}}$ の ${\overset{\textnormal{きも}}{\text{気持}}}$ ちをわかってくれ。 \hfill\break
 \emph{Boku no kimochi wo wakatte kure. \hfill\break
 }Understand my emotions! }

\par{\textbf{Particle Note }: The particle \emph{wo }を is occasionally used instead of \emph{ga }が with \emph{wakaru }分かる whenever it is used with contexts involving emotion. This usage is quite different as it does imply volition in the knowledge had about said emotion. This usage, though, is ungrammatical to many speakers, especially those of older generations. Properly, this meaning is carried about by the verb \emph{rikai suru }理解する, which is a transitive verb. }

\par{\textbf{Grammar Note }: The ending \emph{-te kure }てくれ is a very strong means of commanding someone to do something. }

\par{・Ability: \emph{Dekiru }出来る }

\par{ One of the meanings of the verb \emph{dekiru }出来る is “to be able to.” It is treated as the potential form of the verb \emph{suru }する (to do). The potential form of verbs is not something we have discussed yet, and so we will only look at \emph{dekiru }出来る for now. As for \emph{dekiru }出来る, the object is always marked with \emph{ga }が. For other potential verbs, including those that also incorporate \emph{dekiru }出来る, there is interchangeability between \emph{ga }が and \emph{wo }を to mark the object, which is delved into in Lesson 167. }

\par{23. ${\overset{\textnormal{にほんご}}{\text{日本語}}}$ が ${\overset{\textnormal{でき}}{\text{出来}}}$ ますか。 \hfill\break
 \emph{Nihongo ga dekimasu ka? \hfill\break
 }Can you speak Japanese? }

\par{24. ${\overset{\textnormal{かれ}}{\text{彼}}}$ はテニスが ${\overset{\textnormal{でき}}{\text{出来}}}$ ます。 \hfill\break
 \emph{Kare wa tenisu ga dekimasu. \hfill\break
 }He can play tennis? }

\par{25. ${\overset{\textnormal{わたし}}{\text{私}}}$ は ${\overset{\textnormal{すこ}}{\text{少}}}$ し ${\overset{\textnormal{しゅわ}}{\text{手話}}}$ が ${\overset{\textnormal{でき}}{\text{出来}}}$ ます。 \hfill\break
 \emph{Watashi wa sukoshi shuwa ga dekimasu. \hfill\break
 }I can speak a little sign language. }

\par{26. どんなスポーツが ${\overset{\textnormal{でき}}{\text{出来}}}$ ますか。 \hfill\break
 \emph{Don\textquotesingle na sup }\emph{ōtsu ga dekimasu ka? \hfill\break
 }What kind of sports can you play? }
      
\section{Stative-Transitive Predicates of Subjective Emotions}
 
\par{・Internal Feeling }

\par{ Broadly speaking, all stative-transitive predicates that relate to subjective emotion deal with one\textquotesingle s internal feelings. }

\par{27. ${\overset{\textnormal{ようかい}}{\text{妖怪}}}$ が ${\overset{\textnormal{こわ}}{\text{怖}}}$ い! \hfill\break
 \emph{Y }\emph{ōkai ga kowai! \hfill\break
 }Ghosts are scary! \hfill\break
I\textquotesingle m scared of ghosts. \emph{}}

\par{28. 私が怖い? \hfill\break
 \emph{Watashi ga kowai? \hfill\break
 }I\textquotesingle m scary? \hfill\break
Are you scared of me? }

\par{29. 「 ${\overset{\textnormal{おれ}}{\text{俺}}}$ 、クモが ${\overset{\textnormal{こわ}}{\text{怖}}}$ くないよ」「 ${\overset{\textnormal{ほんとう}}{\text{本当}}}$ ? ${\overset{\textnormal{ぼく}}{\text{僕}}}$ は ${\overset{\textnormal{こわ}}{\text{怖}}}$ いけど。」 \hfill\break
 \emph{“Ore, kumo ga kowakunai yo.” “Hont }\emph{ō? Boku wa kowai kedo.” \hfill\break
 }“Me, I\textquotesingle m not afraid of spiders.” “Really? Well, I am.” }

\par{30. ${\overset{\textnormal{わたし}}{\text{私}}}$ は ${\overset{\textnormal{こわ}}{\text{怖}}}$ い ${\overset{\textnormal{ははおや}}{\text{母親}}}$ です。 \hfill\break
 \emph{Watashi wa kowai hahaoya desu. \hfill\break
 }I\textquotesingle m a ‘scary\textquotesingle  mother. }

\par{\textbf{Grammar Note }: In isolation, \emph{X wa kowai }Xは怖い is grammatically ambiguous. It could be that there is a zero-pronoun, which would indicate that “X” is the subject and not the object of fear. “X” could also be the object of fear like in Ex. 30. Although \emph{kowai }怖い is modifying a noun, making \emph{watashi }私 function as the subject, \emph{watashi }私 still behaves as the object of fear of the mother\textquotesingle s child(ren). In short, in isolation, \emph{watashi wa kowai }私は怖い can mean both “I\textquotesingle m scary” or “I\textquotesingle m afraid.” Remember, the comment is what dictates the usage of \emph{wa }は, and this can\textquotesingle t be more true for this situation. }

\par{31. ${\overset{\textnormal{ぼく}}{\text{僕}}}$ は ${\overset{\textnormal{かき}}{\text{柿}}}$ が待ち ${\overset{\textnormal{どお}}{\text{遠}}}$ しい。 \hfill\break
 \emph{Boku wa kaki ga machid }\emph{ōshii. \hfill\break
 }I look forward to persimmons. }

\par{・Like\slash Dislike }

\par{ The adjectival expressions \emph{suki da }好きだ and \emph{kirai da }嫌いだ respectively show personal like and dislike. They may be used to either express first person like\slash dislike or ask about second person like\slash dislike. However, there is a general principle in Japanese that one can never definitively state the mindset in third person. In such a situation, a qualifier must be added to make clear that one isn\textquotesingle t asserting absolute knowledge pertaining someone else\textquotesingle s feelings (Ex. 34). }

\par{32. どんな ${\overset{\textnormal{りょうり}}{\text{料理}}}$ が ${\overset{\textnormal{す}}{\text{好}}}$ きですか。 \hfill\break
 \emph{Don\textquotesingle na ry }\emph{ōri ga suki desu ka? \hfill\break
 }What kind of food do you like? }

\par{33. ${\overset{\textnormal{わたし}}{\text{私}}}$ は ${\overset{\textnormal{かれ}}{\text{彼}}}$ が ${\overset{\textnormal{す}}{\text{好}}}$ きです。 \hfill\break
 \emph{Watashi wa kare ga suki desu. \hfill\break
 }I like him. }

\par{34. ${\overset{\textnormal{ねこ}}{\text{猫}}}$ はみんな ${\overset{\textnormal{ぼく}}{\text{僕}}}$ が ${\overset{\textnormal{す}}{\text{好}}}$ きみたいですねえ。 \hfill\break
 \emph{Neko wa min\textquotesingle na boku ga suki mitai desu n }\emph{ē. \hfill\break
 }All cats seem to like me, don\textquotesingle t they? }

\par{35. ${\overset{\textnormal{ひと}}{\text{人}}}$ が【 ${\overset{\textnormal{いせい}}{\text{異性}}}$ ・ ${\overset{\textnormal{どうせい}}{\text{同性}}}$ 】\{が・を\} ${\overset{\textnormal{す}}{\text{好}}}$ きになる ${\overset{\textnormal{りゆう}}{\text{理由}}}$ は ${\overset{\textnormal{なん}}{\text{何}}}$ ですか。 \hfill\break
 \emph{Hito ga [isei\slash d }\emph{ōsei] [ga\slash wo] suki ni naru riyu wa nan desu ka? \hfill\break
 }What are the reasons for why people like the [opposite sex\slash same sex]? }

\par{\textbf{Particle Note }: The particle \emph{wo }を is occasionally seen instead of \emph{ga }が with \emph{suki da }好きだin casual speech, especially when doing so can prevent two \emph{ga }が\textquotesingle s in the same sentence. However, switching \emph{ga }が for \emph{wo }を like this is ungrammatical regardless of the circumstances to many speakers as Japanese grammar allows for such instances of the doubling of \emph{ga }が as we\textquotesingle ve come to learn. }

\par{36. ${\overset{\textnormal{じょうし}}{\text{上司}}}$ が ${\overset{\textnormal{きら}}{\text{嫌}}}$ いです。 \hfill\break
 \emph{J }\emph{ōshi ga kirai desu. \hfill\break
 }I hate my boss. }

\par{37. 【 ${\overset{\textnormal{むすめ}}{\text{娘}}}$ ・ ${\overset{\textnormal{むすこ}}{\text{息子}}}$ 】の ${\overset{\textnormal{かれし}}{\text{彼氏}}}$ が ${\overset{\textnormal{きら}}{\text{嫌}}}$ いです。 \hfill\break
 \emph{[Musume\slash musuko] no kareshi ga kirai desu. \hfill\break
 }I hate my [daughter\textquotesingle s\slash son\textquotesingle s] boyfriend. }

\par{38. ${\overset{\textnormal{た}}{\text{食}}}$ べ ${\overset{\textnormal{もの}}{\text{物}}}$ は ${\overset{\textnormal{なに}}{\text{何}}}$ が ${\overset{\textnormal{きら}}{\text{嫌}}}$ いですか。 \hfill\break
 \emph{Tabemono wa nani ga kirai desu ka? \hfill\break
 }What foods do you hate? }

\par{39. ${\overset{\textnormal{わたし}}{\text{私}}}$ がすぐ ${\overset{\textnormal{ひと}}{\text{人}}}$ \{が・を\} ${\overset{\textnormal{きら}}{\text{嫌}}}$ いになる ${\overset{\textnormal{りゆう}}{\text{理由}}}$ は ${\overset{\textnormal{なん}}{\text{何}}}$ だろう。 \hfill\break
 \emph{Watashi ga sugu hito [ga\slash wo] kirai ni naru riy }\emph{ū wa nan dar }\emph{ō? \hfill\break
 }I wonder what the reasons are for why I end up hating people immediately? }

\par{\textbf{Particle Note }: Although not as common as with \emph{suki da }好きだ, some speakers will occasionally replace the particle \emph{ga }が with \emph{wo }を when using \emph{kirai da }嫌いだ; however, this is usually done to avoid the doubling of \emph{ga }が. Nonetheless, this is still prescriptively incorrect as well as ungrammatical to most speakers. }

\par{・Want\slash Desire }

\par{ The adjective \emph{hoshii }欲しい is used to show personal want\slash desire for something. It is not used to show third person want. This adjective is also not used to show personal want to do something. These grammar points will be discussed in Lesson 99. }

\par{40. ${\overset{\textnormal{こども}}{\text{子供}}}$ が ${\overset{\textnormal{ほ}}{\text{欲}}}$ しいです。 \hfill\break
 \emph{Kodomo ga hoshii desu. \hfill\break
 }I want a child\slash children. }

\par{41. ${\overset{\textnormal{はな}}{\text{話}}}$ し ${\overset{\textnormal{あいて}}{\text{相手}}}$ が ${\overset{\textnormal{ほ}}{\text{欲}}}$ しいです。 \hfill\break
 \emph{Hanashi aite ga hoshii desu. \hfill\break
 }I want someone to talk to. }

\par{42. ${\overset{\textnormal{めいかく}}{\text{明確}}}$ な ${\overset{\textnormal{へんじ}}{\text{返事}}}$ がほしいです。 \hfill\break
 \emph{Meikaku na henji ga hoshii desu. \hfill\break
 }I want a clear response. }

\par{43. どちらがほしいですか。 \hfill\break
 \emph{Dochira ga hoshii desu ka? \hfill\break
 }Which one do you want? }

\par{・Competence }

\par{ There are several phrases in Japanese for “to be good at” and “to be bad at.” How they mainly differ is to what degree they qualify someone and who the subject can be. }

\begin{center}
\textbf{Good At }
\end{center}

\par{・ \emph{J }\emph{ōzu da }上手だ – This is used to express that someone is good at something. \hfill\break
・ \emph{Umai }うまい – This is used to express that someone is good at something. \hfill\break
・ \emph{Tokui da }得意だ – This is used to express one\textquotesingle s forte or someone else\textquotesingle s forte. }

\begin{center}
\textbf{Bad At }
\end{center}

\par{・ \emph{Nigate da }苦手だ – Neutral way of expressing “bad at.” \hfill\break
・ \emph{Futokui da }不得意だ – Neutral way of expressing that something is not one\textquotesingle s forte. \hfill\break
・ \emph{Mazui }まずい  - Subjective way of expressing “bad at.” \hfill\break
・ \emph{Heta da }下手だ – Often rude way of showing poor skill. \hfill\break
・ \emph{Hetakuso da }下手くそだ – “Shitty.” }

\par{ For “bad at” phrases, they can all essentially be used to refer to oneself as well as others. When commenting about others, a word of caution must be had because they can all be taken the wrong way if the person in question hears your comment. }

\par{44. ${\overset{\textnormal{にほんご}}{\text{日本語}}}$ が(お) ${\overset{\textnormal{じょうず}}{\text{上手}}}$ ですね。 \hfill\break
 \emph{Nihongo ga (o-)j }\emph{ōzu desu ne. \hfill\break
 }Your Japanese is good. }

\par{\textbf{Grammar Note }: Adding \emph{o }- お to \emph{j }\emph{ōzu }上手 makes it politer. }

\par{45. ${\overset{\textnormal{わたし}}{\text{私}}}$ は ${\overset{\textnormal{りょうり}}{\text{料理}}}$ が ${\overset{\textnormal{とくい}}{\text{得意}}}$ です。 \hfill\break
 \emph{Watashi wa ry }\emph{ōri ga tokui desu. \hfill\break
 }I\textquotesingle m good at cooking. }

\par{46. ${\overset{\textnormal{うんてん}}{\text{運転}}}$ がうまいですね。 \hfill\break
 \emph{Unten ga umai desu ne. \hfill\break
 }Wow, your driving is good. }

\par{47. ${\overset{\textnormal{えいご}}{\text{英語}}}$ が ${\overset{\textnormal{にがて}}{\text{苦手}}}$ ですが、 ${\overset{\textnormal{がんば}}{\text{頑張}}}$ ります。 \hfill\break
 \emph{Eigo ga nigate desu ga, gambarimasu. \hfill\break
 }I\textquotesingle m bad at English, but I\textquotesingle ll try. }

\par{48. ${\overset{\textnormal{ぼく}}{\text{僕}}}$ は ${\overset{\textnormal{はやお}}{\text{早起}}}$ きが ${\overset{\textnormal{にがて}}{\text{苦手}}}$ です。 \hfill\break
 \emph{Boku wa hayaoki ga nigate desu. \hfill\break
 }I\textquotesingle m bad at waking up early. }

\par{49. ${\overset{\textnormal{かのじょ}}{\text{彼女}}}$ はメイクが ${\overset{\textnormal{へた}}{\text{下手}}}$ だ。 \hfill\break
 \emph{Kanojo wa meiku ga heta da. \hfill\break
 }She\textquotesingle s bad at makeup. }

\par{50. あいつはどうも ${\overset{\textnormal{にがて}}{\text{苦手}}}$ だ。 \hfill\break
 \emph{Aitsu wa d }\emph{ōmo nigate da. \hfill\break
 }He\textquotesingle s terribly hard to deal with. }

\par{\textbf{Word Note }: \emph{Nigate da }苦手だ may also express that one has difficulties dealing with someone. }

\par{51. お ${\overset{\textnormal{まえ}}{\text{前}}}$ は ${\overset{\textnormal{じ}}{\text{字}}}$ が ${\overset{\textnormal{へた}}{\text{下手}}}$ くそだな。 \hfill\break
 \emph{Omae wa ji ga hetakuso da na. \hfill\break
 }Your handwriting is shitty, you know. }

\par{\textbf{Word Note }: \emph{Omae }お前 is another word for “you,” but it is used in coarse situations and should only be used with people you are incredibly familiar with like a childhood friend. This is used mainly by male speakers. }
    