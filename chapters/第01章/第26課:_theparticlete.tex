    
\chapter{The Particle Te て}

\begin{center}
\begin{Large}
第26課: The Particle Te て 
\end{Large}
\end{center}
 
\par{ The particle \emph{te }て is the most important conjunctive particle in Japanese. Conjunctive particles correspond to words like "and" and "but." As you will see, you cannot use \emph{te }て for the \emph{and }in " \emph{And }, I saw him" or "dogs \emph{and }cats do this," but its use is profoundly important. This particle will be seen constantly in various grammatical patterns. For now, though, we'll learn about how it's used when it's by itself. }
      
\section{The Conjunctive Particle Te て}
 
\par{ Conjugating with the particle \emph{te }て isn't particularly difficult, but conjugating is done differently depending on what part of speech you're using. In the table below, you will see the affirmative and negative \emph{te }て forms for each independent conjugatable part of speech. }

\begin{ltabulary}{|P|P|P|P|}
\hline 

Part of Speech & Example Verb & Affirmative & Negative \\ \cline{1-4}

 \emph{Ichidan }Verbs & ・ \emph{Miru }見る \hfill\break
・ \emph{Taberu }食べる & ・みて \hfill\break
・たべて & ・みなくて \hfill\break
・たべなくて \\ \cline{1-4}

 \emph{Godan }Verbs & ・ \emph{Kau }買う (To buy) \hfill\break
・ \emph{Iu }言う (To say) \hfill\break
・ \emph{Tō }問う (To question) \hfill\break
・ \emph{Kaku }書く (To write) \hfill\break
・ \emph{Kiku }聞く (To listen) \hfill\break
・ \emph{Iku }行く (To go) \hfill\break
・ \emph{Kagu }嗅ぐ (To sniff) \hfill\break
・ \emph{Katsu }勝つ (To win) \hfill\break
・ \emph{Shiru }知る (To know) \hfill\break
・ \emph{Shinu }死ぬ (To die) \hfill\break
・ \emph{Yobu }呼ぶ (To call) \hfill\break
・ \emph{Yomu }読む (To read) & ・かって \hfill\break
・いって \hfill\break
・とうて \hfill\break
・かいて \hfill\break
・きいて \hfill\break
・いって \hfill\break
・かいで \hfill\break
・かって \hfill\break
・しって \hfill\break
・しんで \hfill\break
・よんで \hfill\break
・よんで & ・かわなくて \hfill\break
・いわなくて \hfill\break
・とわなくて \hfill\break
・かかなくて \hfill\break
・きかなくて \hfill\break
・いかなくて \hfill\break
・かがなくて \hfill\break
・かたなくて \hfill\break
・しらなくて \hfill\break
・しななくて \hfill\break
・よばなかくて \hfill\break
・よまなくて \\ \cline{1-4}

Auxiliaries & ・- \emph{masu }ます & ・~まして &  \\ \cline{1-4}

Irregular & ・ \emph{Kuru }来る (To come) \hfill\break
・ \emph{Suru }する (To do) & ・きて \hfill\break
・して & ・こなくて \hfill\break
・しなくて \\ \cline{1-4}

Adjectives & ・ \emph{Atarashii }新しい (New) \hfill\break
・ \emph{Ii\slash yoi }良い (Good) & ・あたらしくて \hfill\break
・よくて & ・あたらしくなくて \hfill\break
・よくなくて \\ \cline{1-4}

Adj. Nouns &  \emph{Kantan da }簡単だ (Easy) & ・かんたんで & ・かんたんではなくて \\ \cline{1-4}

Copula &  \emph{Da }だ (To be) & ・で & ・ではなくて \\ \cline{1-4}

\end{ltabulary}

\par{\textbf{Conjugation Notes }: \hfill\break
}

\par{1. With a select numb er of verbs such as \emph{t }\emph{ō }問う (to question) , the particle \emph{te }て simply follows the verb. Very few verbs do this, and whenever we come across them, this will be brought up. \hfill\break
2. Note how the conjugations for the affirmative forms are, in the case of verbs, made exactly how the past tense is made, utilizing the exact same sound changes. \hfill\break
3. As always, \emph{de wa }では can be contracted to \emph{ja }じゃ. \hfill\break
4. As implied by the chart, the polite affirmative forms of verbs can be put in the \emph{te }て form. \hfill\break
Ex. \emph{Suru }する \textrightarrow  \emph{shimashite }しまして. }
 
\begin{center}
 \textbf{Usages of the Particle \emph{Te }て }\hfill\break

\end{center}

\par{1. The first and most important role of the particle \emph{te }て is to \textbf{connect two clauses }. In doing so, it can also implicitly indicate reason for feelings, states, and\slash or the past. However, the action in the latter part(s) can't contain volition. }

\par{1. ニュースを ${\overset{\textnormal{き}}{\text{聞}}}$ いて、びっくりした。 \hfill\break
 \emph{Nyūsu wo kiite, bikkuri shita. }\hfill\break
I was surprised to hear the news. }

\par{2. この ${\overset{\textnormal{にほんご}}{\text{日本語}}}$ の ${\overset{\textnormal{ぶんしょう}}{\text{文章}}}$ は ${\overset{\textnormal{ふくざつ}}{\text{複雑}}}$ で、よく ${\overset{\textnormal{わ}}{\text{分}}}$ かりませんでした。 \hfill\break
 \emph{Kono Nihongo no bunshō wa fukuzatsu de, yoku wakarimasendeshita. }\hfill\break
The Japanese was complicated, so I didn't understand it well. }

\par{2. The particle \emph{te }てcan \textbf{list actions or qualities }, and \textbf{indicate a method for action }(like putting sugar in tea and drinking it). }

\par{3. そのリンゴは ${\overset{\textnormal{あか}}{\text{赤}}}$ くて、 ${\overset{\textnormal{おお}}{\text{大}}}$ きいです。 \hfill\break
 \emph{Sono ringo wa akakute, ōkii desu. }\hfill\break
The apple is red and big. }

\par{4. ${\overset{\textnormal{にほんご}}{\text{日本語}}}$ は ${\overset{\textnormal{かんたん}}{\text{簡単}}}$ で、 ${\overset{\textnormal{すば}}{\text{素晴}}}$ らしいです。 \hfill\break
 \emph{Nihongo wa kantan de, subarashii desu. }\hfill\break
Japanese is easy and awesome. }

\par{5. ${\overset{\textnormal{かわだ}}{\text{川田}}}$ さんの ${\overset{\textnormal{いえ}}{\text{家}}}$ は ${\overset{\textnormal{あたら}}{\text{新}}}$ しくて、 ${\overset{\textnormal{きれい}}{\text{綺麗}}}$ ですね。 \hfill\break
 \emph{Kawada-san no ie wa atarashikute, kirei desu ne. }\hfill\break
Mrs. Kawada's house is new and pretty, isn't it? }

\par{6. ${\overset{\textnormal{とうきょう}}{\text{東京}}}$ は ${\overset{\textnormal{にぎ}}{\text{賑}}}$ やかで ${\overset{\textnormal{おもしろ}}{\text{面白}}}$ い。 \hfill\break
 \emph{Tōkyō wa nigiyaka de omoshiroi. }\hfill\break
Tokyo is lively and interesting. }

\par{7. このクラスは ${\overset{\textnormal{しゅくだい}}{\text{宿題}}}$ が ${\overset{\textnormal{おお}}{\text{多}}}$ くて、 ${\overset{\textnormal{しけん}}{\text{試験}}}$ も ${\overset{\textnormal{むずか}}{\text{難}}}$ しかったです。 \hfill\break
 \emph{Kono kurasu wa shukudai ga okute, shiken mo muzukashikatta desu. }\hfill\break
The class had a lot of homework, and the exams were difficult. }

\par{8. ${\overset{\textnormal{かれ}}{\text{彼}}}$ は ${\overset{\textnormal{かのじょ}}{\text{彼女}}}$ の ${\overset{\textnormal{いえ}}{\text{家}}}$ に ${\overset{\textnormal{よ}}{\text{寄}}}$ って、 ${\overset{\textnormal{てがみ}}{\text{手紙}}}$ を ${\overset{\textnormal{とど}}{\text{届}}}$ けた。 \hfill\break
 \emph{Kare wa kanojo no ie ni yotte, tegami wo todoketa. }\hfill\break
He stopped by her house and delivered a letter. }

\par{9. あの ${\overset{\textnormal{ひと}}{\text{人}}}$ は ${\overset{\textnormal{やさ}}{\text{優}}}$ しくて ${\overset{\textnormal{とう}}{\text{頭}}}$ がいいです。 \hfill\break
Ano hita wa yasashikute atama ga ii desu. \hfill\break
That person is kind and smart. }

\par{10. ${\overset{\textnormal{た}}{\text{立}}}$ ち ${\overset{\textnormal{ど}}{\text{止}}}$ まって ${\overset{\textnormal{あた}}{\text{辺}}}$ りを ${\overset{\textnormal{みまわ}}{\text{見回}}}$ す。 \hfill\break
 \emph{Tachidomatte atari wo mimawasu. }\hfill\break
To stop and look around. }

\par{11. ${\overset{\textnormal{かぜ}}{\text{風}}}$ が ${\overset{\textnormal{つよ}}{\text{強}}}$ \textbf{くて }${\overset{\textnormal{さむ}}{\text{寒}}}$ \textbf{い }${\overset{\textnormal{ひ}}{\text{日}}}$ \hfill\break
 \emph{Kaze ga tsuyokute samui hi \hfill\break
 }A \emph{cold }day where the wind \emph{is strong } }

\par{12. ${\overset{\textnormal{やまだ}}{\text{山田}}}$ さんはきれい \textbf{で }${\overset{\textnormal{やさ}}{\text{優}}}$ し \textbf{い }です。 \hfill\break
 \emph{Yamada-san wa kirei de yasashii desu. }\hfill\break
Ms. Yamada \emph{is } \emph{pretty and nice }. }

\par{13. ジェシカさんはきれい \textbf{で }、 ${\overset{\textnormal{しんせつ}}{\text{親切}}}$ \textbf{な }人です。 \hfill\break
 \emph{Jeshika-san wa kirei de, shinsetsu na hito desu. }\hfill\break
Jessica \emph{is }a \emph{pretty and kind }person. }

\par{14. ${\overset{\textnormal{うつく}}{\text{美}}}$ し \textbf{くて }${\overset{\textnormal{しず}}{\text{静}}}$ か \textbf{な }${\overset{\textnormal{じょせい}}{\text{女性}}}$ \hfill\break
 \emph{Utsukushikute shizuka na josei }\hfill\break
A \emph{beautiful, quiet }woman }

\par{15. ${\overset{\textnormal{せかい}}{\text{世界}}}$ は ${\overset{\textnormal{すば}}{\text{素晴}}}$ らし \textbf{くて }、 ${\overset{\textnormal{おもしろ}}{\text{面白}}}$ い。 \hfill\break
 \emph{Sekai wa subarashikute, omoshiroi. }\hfill\break
The world is wonderful and interesting. }

\par{16. ${\overset{\textnormal{かる}}{\text{軽}}}$ \textbf{くて }\{かっこいい・スマート\}なケータイがほしいです。 \hfill\break
 \emph{Karukute [kakko-ii\slash sumāto] na kētai ga hoshii desu. }\hfill\break
I want a light and stylish cell phone. }

\par{17. ${\overset{\textnormal{あつ}}{\text{暑}}}$ \textbf{く }なりましたね。 \hfill\break
 \emph{Atsuku narimashita ne. }\hfill\break
It's become hot, hasn't it? }

\begin{center}
 \textbf{Verb Deletion }
\end{center}

\par{ What if you are just repeating the same verb over and over again? You can delete all but the last, and you can even delete the particle that goes with it except in the last clause. }

\par{18a. ランスはフランスへ ${\overset{\textnormal{い}}{\text{行}}}$ って、セスは ${\overset{\textnormal{にほん}}{\text{日本}}}$ へ ${\overset{\textnormal{い}}{\text{行}}}$ って、サムは ${\overset{\textnormal{ちゅうごく}}{\text{中国}}}$ へ ${\overset{\textnormal{い}}{\text{行}}}$ きました。 \hfill\break
18b. ランスはフランスへ、セスは ${\overset{\textnormal{にほん}}{\text{日本}}}$ へ、サムは ${\overset{\textnormal{ちゅうごく}}{\text{中国}}}$ へ ${\overset{\textnormal{い}}{\text{行}}}$ きました。 \hfill\break
18c. ランスはフランス、 セスは ${\overset{\textnormal{にほん}}{\text{日本}}}$ 、サムは ${\overset{\textnormal{ちゅうごく}}{\text{中国}}}$ へ行きました。 \hfill\break
18a. \emph{Ransu wa Furansu e itte, Sesu wa Nihon e itte, Samu wa Chūgoku e ikimashita. }\hfill\break
18b. \emph{Ransu wa Furansu e, Sesu wa Nihon e, Samu wa Chūgoku e ikimashita. }\hfill\break
18c. \emph{Ransu wa Furansu, Sesu wa Nihon, Samu wa Chūgoku e ikimashita. }\hfill\break
Lance went to France, Seth went to Japan, and Sam went to China. }

\begin{center}
\textbf{Negative \emph{Te }て Forms } \hfill\break

\end{center}

\par{ There are two possible negative \emph{te }て forms: - \emph{naide }ないで and \emph{-nakute }なくて. \emph{-naide }ないで means "without" as in "without doing something," whereas \emph{-nakute }なくて simply conjoins two clauses (in line with the usages of \emph{te }て explained above), with the first happening to be in the negative. Note that only verbs are capable of being used with \emph{-naide }ないで. }

\begin{ltabulary}{|P|P|P|P|}
\hline 

Part of Speech & Example Verb & - \emph{nakute }なくて & - \emph{naide }ないで \\ \cline{1-4}

 \emph{Ichidan }Verbs & ・ \emph{Miru }見る \hfill\break
・ \emph{Taberu }食べる & ・みなくて \hfill\break
・たべなくて & ・みないで \hfill\break
・たべないで \\ \cline{1-4}

 \emph{Godan }Verbs & ・ \emph{Kau }買う (To buy) \hfill\break
・ \emph{Iu }言う (To say) \hfill\break
・ \emph{Tō }問う (To question) \hfill\break
・ \emph{Kaku }書く (To write) \hfill\break
・ \emph{Kiku }聞く (To listen) \hfill\break
・ \emph{Iku }行く (To go) \hfill\break
・ \emph{Kagu }嗅ぐ (To sniff) \hfill\break
・ \emph{Katsu }勝つ (To win) \hfill\break
・ \emph{Shiru }知る (To know) \hfill\break
・ \emph{Shinu }死ぬ (To die) \hfill\break
・ \emph{Yobu }呼ぶ (To call) \hfill\break
・ \emph{Yomu }読む (To read) & ・かわなくて \hfill\break
・いわなくて \hfill\break
・とわなくて \hfill\break
・かかなくて \hfill\break
・きかなくて \hfill\break
・いかなくて \hfill\break
・およがなくて \hfill\break
・かたなくて \hfill\break
・かわらなくて \hfill\break
・しななくて \hfill\break
・よばなかくて \hfill\break
・よまなくて & ・かわないで \hfill\break
・いわないで \hfill\break
・とわないで \hfill\break
・かかないで \hfill\break
・きかないで \hfill\break
・いかないで \hfill\break
・かがないで \hfill\break
・かたないで \hfill\break
・しらないで \hfill\break
・しなないで \hfill\break
・よばないで \hfill\break
・よまないで \\ \cline{1-4}

Irregular & ・ \emph{Kuru }来る (To come) \hfill\break
・ \emph{Suru }する (To do) & ・こなくて \hfill\break
・しなくて & ・こないで \hfill\break
・しないで \\ \cline{1-4}

Adjectives & ・ \emph{Atarashii }新しい (New) \hfill\break
・ \emph{Ii\slash yoi }良い (Good) & ・あたらしくなくて \hfill\break
・よくなくて &  \\ \cline{1-4}

Adj. Nouns &  \emph{Kantan da }簡単だ (Easy) & ・かんたんではなくて &  \\ \cline{1-4}

Copula &  \emph{Da }だ (To be) & ・ではなくて &  \\ \cline{1-4}

\end{ltabulary}

\par{19. ${\overset{\textnormal{じしょ}}{\text{辞書}}}$ を\{ ${\overset{\textnormal{つか}}{\text{使}}}$ わないで・ ${\overset{\textnormal{ひ}}{\text{引}}}$ かないで\}、 ${\overset{\textnormal{てがみ}}{\text{手紙}}}$ を ${\overset{\textnormal{か}}{\text{書}}}$ いた。 \hfill\break
 \emph{Jisho wo [tsukawanaide\slash hikanaide], tegami wo kaita. }\hfill\break
I wrote a letter without using a dictionary. }

\par{20. ${\overset{\textnormal{とも}}{\text{友}}}$ だちが ${\overset{\textnormal{こ}}{\text{来}}}$ なくて、 ${\overset{\textnormal{こま}}{\text{困}}}$ りました。 \hfill\break
 \emph{Tomodachi ga konakute, komarimashita. }\hfill\break
My friend didn't come, and I was upset. }

\par{21. ${\overset{\textnormal{ね}}{\text{寝}}}$ ないで ${\overset{\textnormal{ま}}{\text{待}}}$ つ。 \hfill\break
 \emph{Nenaide matsu. }\hfill\break
Wait without sleeping. }

\par{22. ${\overset{\textnormal{きょう}}{\text{今日}}}$ は ${\overset{\textnormal{でんしゃ}}{\text{電車}}}$ に ${\overset{\textnormal{の}}{\text{乗}}}$ らないで、 ${\overset{\textnormal{ある}}{\text{歩}}}$ いてきました。 \hfill\break
 \emph{Kyō wa densha ni noranaide, aruite kimashita. }\hfill\break
I came by walking instead of riding the train today. }

\par{\textbf{Curriculum Note }: There is some considerable overlap between these two forms. Because of this, this section is scheduled to become its own lesson in the not too distant future. }

\begin{center}
\textbf{More Examples }
\end{center}

\par{23. ${\overset{\textnormal{わたし}}{\text{私}}}$ は ${\overset{\textnormal{ひる}}{\text{昼}}}$ ご ${\overset{\textnormal{はん}}{\text{飯}}}$ を ${\overset{\textnormal{た}}{\text{食}}}$ べて、テレビを ${\overset{\textnormal{み}}{\text{見}}}$ て、 ${\overset{\textnormal{おんがく}}{\text{音楽}}}$ を ${\overset{\textnormal{き}}{\text{聞}}}$ いて、 ${\overset{\textnormal{かえ}}{\text{帰}}}$ りました。 \hfill\break
 \emph{Watashi wa hirugohan wo tabete, terebi wo mite, ongaku wo kiite, kaerimashita. }\hfill\break
I ate lunch, watched TV, listened to music, and came home. }

\par{24. ${\overset{\textnormal{わたし}}{\text{私}}}$ は ${\overset{\textnormal{なつふく}}{\text{夏服}}}$ をしまって、 ${\overset{\textnormal{あき}}{\text{秋}}}$ の ${\overset{\textnormal{ふく}}{\text{服}}}$ を ${\overset{\textnormal{だ}}{\text{出}}}$ しました。 \hfill\break
 \emph{Watashi wa natsufuku wo shimatte, aki no fuku wo dashimashita. }\hfill\break
I put up my summer clothes and got out my fall clothes. }

\par{25. ${\overset{\textnormal{さくら}}{\text{桜}}}$ の ${\overset{\textnormal{はな}}{\text{花}}}$ が ${\overset{\textnormal{ち}}{\text{散}}}$ って、 ${\overset{\textnormal{わかば}}{\text{若葉}}}$ が ${\overset{\textnormal{で}}{\text{出}}}$ ました。 \hfill\break
 \emph{Sakura no hana ga chitte, wakaba ga demashita. }\hfill\break
The cherry blossoms have scattered, and the leaves have appeared. }

\par{26. ${\overset{\textnormal{はたら}}{\text{働}}}$ いて、 ${\overset{\textnormal{つか}}{\text{疲}}}$ れた。 \hfill\break
 \emph{Hataraite, tsukareta. }\hfill\break
I worked and got exhausted. }

\begin{center}
\textbf{\emph{Te }て Phrases Deemed as One Word }
\end{center}

\par{ There are a few instances when the particle \emph{te }て phrases result in single vocabulary items. }

\par{27. ${\overset{\textnormal{しんい}}{\text{真意}}}$ を ${\overset{\textnormal{み}}{\text{見}}}$ て ${\overset{\textnormal{と}}{\text{取}}}$ る。 \hfill\break
 \emph{Shin'i wo mite-toru. }\hfill\break
To grasp the real sense. }

\par{28. ${\overset{\textnormal{せんきょ}}{\text{選挙}}}$ に ${\overset{\textnormal{う}}{\text{打}}}$ って ${\overset{\textnormal{で}}{\text{出}}}$ る。 \hfill\break
 \emph{Senkyo ni utte-deru. }\hfill\break
To make one's debut in an election. }

\par{29. ${\overset{\textnormal{いそ}}{\text{急}}}$ いで ${\overset{\textnormal{と}}{\text{取}}}$ って ${\overset{\textnormal{かえ}}{\text{返}}}$ す。 \hfill\break
 \emph{Isoide totte-kaesu. }\hfill\break
To hurry and come back. }
    