    
\chapter{Adjectives II}

\begin{center}
\begin{Large}
第14課: Adjectives II: 形容動詞 Keiyōdōshi 
\end{Large}
\end{center}
 
\par{ The second part of speech that acts like adjectives is called \emph{Keiyōdōshi }形容動詞. This name literally means “adjectival verb.” The name comes from the fact that all adjectives in this class must use the copula to conjugate. Confusingly, however, they\textquotesingle re essentially adjectival nouns. \hfill\break
}

\par{ Take, for example, the word “fat.” In English, this is both a noun and an adjective, but when it is used as an adjective, we need to use forms of “to be fat.” Although this example doesn\textquotesingle t translate over into Japanese, this same principle applies to \emph{Keiyōdōshi }形容動詞. }
      
\section{Vocabulary List}
 
\par{\textbf{Nouns }}
 
\par{・病院 \emph{Byōin }– Hospital }
 
\par{・チェコ語 \emph{Chekogo }– Czech language }
 
\par{・図書館 – \emph{Toshokan }– Library }
 
\par{・子供 \emph{Kodomo }– Child\slash children }
 
\par{・歯 \emph{Ha }– Tooth\slash teeth }
 
\par{・月 \emph{Tsuki - }Moon }
 
\par{・人 \emph{Hito }– Person }
 
\par{・建物 \emph{Tatemono }- Building }
 
\par{・気持ち \emph{Kimochi }– Feeling(s) }
 
\par{・酸素 \emph{Sanso }– Oxygen }
 
\par{・資格 \emph{Shikaku }- Qualifications }
 
\par{・傾斜 \emph{Keisha }– Slant\slash slope }
 
\par{・歌手 \emph{Kashu }– Singer }
 
\par{・ハンドバッグ \emph{Handobaggu }- Handbug }
 
\par{・ビジネス \emph{Bijinesu }– Business }
 
\par{・問題 \emph{Mondai }– Question\slash problem }
 
\par{・登録 \emph{Tōroku }– Registration }
 
\par{・町 \emph{Machi }– Town }
 
\par{・スタッフ \emph{Sutaffu }- Staff }
 
\par{・野菜 \emph{Yasai }- Vegetables }
 
\par{・英語 \emph{Eigo }– English language }
 
\par{・昔 \emph{Mukashi }– Olden days }
 
\par{・簡易ベッド \emph{Kan\textquotesingle i beddo }– Bunk bed }
 
\par{・ルール \emph{Rūru }– Rule(s) }
 
\par{・兵士 \emph{Heishi }- Soldier }
 
\par{・文章 \emph{Bunshō }– Sentence }
 
\par{・初恋 \emph{Hatsukoi }– First love }
 
\par{・ゲーム \emph{Gēmu }– Game }
 
\par{\textbf{Pronouns }}
 
\par{・私 \emph{Wata(ku)shi }– I }
 
\par{・僕 \emph{Boku }– I (male) }
 
\par{・彼 \emph{Kare }– He }
 
\par{・彼ら \emph{Karera }– They }
 
\par{\textbf{Demonstratives }}
 
\par{・この \emph{Kono }– This (adj.) }
 
\par{・それ \emph{Sore }– That (n.) }
 
\par{・あれ \emph{Are }– That (n.) }
 
\par{・あの \emph{Ano }– That (adj.) }
 
\par{\textbf{Adverbs }}
 
\par{・もはや \emph{Mohaya }- No more }
 
\par{・全然 \emph{Zenzen }– At all }
 
\par{・別に \emph{Betsu ni }– Not particularly }
 
\par{・特に \emph{Toku ni }– Especially }
 
\par{・まだ \emph{Mada }– Yet\slash still }
 
\par{・少しも \emph{Sukushi mo }– Not one bit }

\par{\textbf{Adjectival Nouns } }

\par{・正確だ  \emph{Seikaku da } – To be accurate }

\par{・特殊だ \emph{Tokushu da }– To be peculiar }

\par{・微妙だ \emph{Bimyō da }– To be subtle }

\par{・明確だ \emph{Meikaku da }– To be precise }

\par{・勇敢だ \emph{Yūkan da }– To be brave }

\par{・スマートだ \emph{Sumāto da }– To be stylish }

\par{・リアルだ \emph{Riaru da }– To be realistic }

\par{・好きだ \emph{Suki da }– To like }

\par{・嫌いだ \emph{Kirai da }– To hate }

\par{・フォーマルだ \emph{Fōmaru da }– To be formal }
 \textbf{Adjectival Nouns }
\par{・静かだ \emph{Shizuka da }– To be quiet }
 
\par{・安全だ \emph{Anzen da }– To be safe }
 
\par{・健康だ \emph{Kenkō da }– To be healthy }
 
\par{・元気だ \emph{Genki da }– To be lively }
 
\par{・簡単だ \emph{Kantan da }– To be easy }
 
\par{・綺麗だ \emph{Kirei da }– To be pretty }
 
\par{・大切だ \emph{Taisetsu da }– To be indispensable }
 
\par{・真面目だ \emph{Majime da }– To be serious }
 
\par{・ユニークだ \emph{Yuniiku da }– To be unique }
 
\par{・モダンだ \emph{Modan da }– To be modern }
 
\par{・駄目だ \emph{Dame da }– To be no good }
 
\par{・馬鹿だ \emph{Baka da }– To be stupid\slash dumb }
 
\par{・重要だ \emph{Jūyō da }– To be important }
 
\par{・変だ \emph{Hen da }– To be strange }
 
\par{・優秀だ \emph{Yūshū da }– To excellent at }
 
\par{・大事だ \emph{Daiji da }– To be valuable }
 
\par{・有害だ \emph{Yūgai da }– To be harmful }
 
\par{・平気だ \emph{Heiki da }– To be cool\slash calm }
 
\par{・必要だ \emph{Hitsuyō da }– To be necessary }
 
\par{・急だ \emph{Kyū da }- To be urgent\slash steep\slash rapid }
 
\par{・カジュアルだ \emph{Kajuaru da }– To be casual }
 
\par{・有名だ \emph{Yūmei da }– To be famous }
 
\par{・苦手だ \emph{Nigate da }– To be poor at }
 
\par{・高価だ \emph{Kōka da }– To be high price }
 
\par{・容易だ \emph{Yōi da }– To be simple }
 
\par{・稀だ \emph{Mare da }– To be rare }
 
\par{・危険だ \emph{Kiken da }– To be dangerous }
 
\par{・複雑だ \emph{Fukuzatsu da }– To be complicated }
 
\par{・熱心だ \emph{Nesshin da }– To be enthusiastic }
 
\par{・適切だ \emph{Tekisetsu da }– To be appropriate }
 
\par{・失礼だ \emph{Shitsurei da }– To be rude }
 
\par{・多忙だ \emph{Tabō da }– To be very busy }
 
\par{・冷静だ \emph{Reisei da }– To be calm\slash composed }
 
\par{・特別だ \emph{Tokubetsu da }– To be special }
 
\par{・親切だ \emph{Shinsetsu da }– To be kind }
 
\par{・公平だ \emph{Kōhei da }– To be fair }
 
\par{・最高だ \emph{Saikō da }– To be the best }
 
\par{・自由だ \emph{Jiyū da }– To be free }
 
\par{・新鮮だ \emph{Shinsen da }– To be fresh }
 
\par{・得意だ \emph{Tokui da }– To be good at }
 
\par{・自然だ \emph{Shizen da }– To be nature }
 
\par{・不自然だ \emph{Fushizen da }– To be unnatural }
 
\par{・残酷だ \emph{Zankoku da }– To be harsh\slash cruel }
 
\par{・切実だ \emph{Setsujitsu da }– To be earnest }
 
\par{・詳細だ \emph{Shōsai da }– To be detailed }
 
\par{・未熟だ \emph{Mijuku da }– To be inexperienced }
 
\par{・簡易だ \emph{Kan\textquotesingle i da }– To be simplistic }
 
\par{・慎重だ \emph{Shinchō da }– To be prudent }
 
\par{・便利だ \emph{Benri da }– To be convenient }
 
\par{・不便だ \emph{Fuben da }– To be inconvenient }
 
\par{・困難だ \emph{Kon\textquotesingle nan da }– To be difficult }
 
\par{・不思議だ \emph{Fushigi da }– To be mysterious }
 
\par{・安易だ \emph{An\textquotesingle i da }– To be easy(-going) }

\par{・ハンサムだ \emph{Hansamu da }– To be handsome }

\par{・シックだ \emph{Shikku da }– To be chic }
       
\section{Conjugating Keiyōdōshi 形容動詞}
  \hfill\break
 As mentioned above, \emph{Keiyōdōshi }形容動詞 are essentially adjectival nouns. Grammatically, an abstract noun is used with a form of the copula. Remember, the copula simply refers to either \emph{da }だ, \emph{desu }です, or any of their conjugations. Although this means the conjugations will be exactly the same as the copula, we will still go through each one individually as we did for \emph{Keiyōshi }形容詞.  
\begin{center}
\textbf{Non-Past Forms: \emph{Da }だ \& \emph{Desu }です }
\end{center}
 
\par{ To use a \emph{Keiyōdōshi }形容動詞 in the non-past tense, simply attach either \emph{da }だ (for plain speech) or \emph{desu }です (for polite speech). When using a \emph{Keiyōdōshi }形容動詞 before a noun, though, \emph{da }だ and \emph{desu }です must be replaced with \emph{na }な. }
 
\begin{ltabulary}{|P|P|P|P|P|}
\hline 
 
  Meaning 
 &   Adj. Noun 
 &   + \emph{na }な   (Before Nouns) 
 &   + \emph{da }だ 
 &   + \emph{desu }です 
 \\ \cline{1-5} 
 
  Quiet 
 &    \emph{Shizuka }静か 
 &    \emph{Shizuka na }静かな 
 &    \emph{Shizuka da }静かだ 
 &    \emph{Shizuka desu }静かです 
 \\ \cline{1-5} 
 
  Safe 
 &    \emph{Anzen }安全 
 &    \emph{Anzen na }安全な 
 &    \emph{Anzen da }安全だ 
 &    \emph{Anzen desu }安全です 
 \\ \cline{1-5} 
 
  Healthy 
 &    \emph{Kenkō }健康 
 &    \emph{Kenkō na }健康な 
 &    \emph{Kenkō da }健康だ 
 &    \emph{Kenkō desu }健康です 
 \\ \cline{1-5} 
 
  Lively 
 &    \emph{Genki }元気 
 &    \emph{Genki na }元気な 
 &    \emph{Genki da }元気だ 
 &    \emph{Genki desu }元気です 
 \\ \cline{1-5} 
 
  Easy 
 &    \emph{Kantan }簡単 
 &    \emph{Kantan na }簡単な 
 &    \emph{Kantan da }簡単だ 
 &    \emph{Kantan desu }簡単です 
\\ \cline{1-5}

\end{ltabulary}
 
\par{\hfill\break
1. ${\overset{\textnormal{びょういん}}{\text{病院}}}$ は ${\overset{\textnormal{あんぜん}}{\text{安全}}}$ です。 \hfill\break
 \emph{Byōin wa anzen desu. }\hfill\break
The hospital is safe.\slash Hospitals are safe. }
 
\par{2. チェコ ${\overset{\textnormal{ご}}{\text{語}}}$ は ${\overset{\textnormal{かんたん}}{\text{簡単}}}$ だ。 \hfill\break
 \emph{Chekogo wa kantan da. \hfill\break
 }Czech is easy. }
 
\par{3. あの ${\overset{\textnormal{としょかん}}{\text{図書館}}}$ は ${\overset{\textnormal{しず}}{\text{静}}}$ かです。 \hfill\break
 \emph{Ano toshokan wa shizuka desu. \hfill\break
 }That library is quiet. }
 
\par{4. ${\overset{\textnormal{げんき}}{\text{元気}}}$ な ${\overset{\textnormal{こども}}{\text{子供}}}$ ですね。 \hfill\break
 \emph{Genki na kodomo desu ne. \hfill\break
 }What a lively kid. }
 
\par{5. ${\overset{\textnormal{わたし}}{\text{私}}}$ の ${\overset{\textnormal{は}}{\text{歯}}}$ は ${\overset{\textnormal{けんこう}}{\text{健康}}}$ です。 \hfill\break
 \emph{Watashi no ha wa kenkō desu. \hfill\break
 }My teeth are healthy. }
 
\begin{center}
\textbf{Past Forms: \emph{Datta }だった\& \emph{Deshita }でした }
\end{center}
 
\par{ To use a \emph{Keiy }\emph{ōd }\emph{ōshi }形容動詞 in the past tense, you need to conjugate the copula to \emph{datta }だった for plain speech or \emph{deshita }でした for polite speech. To use the past tense to modify a noun, you must use \emph{datta }だった. This is because there is a general rule in Japanese that polite forms shouldn\textquotesingle t modify nouns. For instance, if you wanted to say “something that was easy,” you\textquotesingle d need to say \emph{kantan datta koto }簡単だったこと. }
 
\begin{ltabulary}{|P|P|P|P|}
\hline 
 
  Meaning 
 &   Adj. Noun 
 &   + \emph{datta }だった (Before Nouns) 
 &   + \emph{deshita }でした 
 \\ \cline{1-4} 
 
  Pretty 
 &    \emph{Kirei }綺麗 
 &    \emph{Kirei datta }綺麗だった 
 &    \emph{Kirei deshita }綺麗でした 
 \\ \cline{1-4} 
 
  Indispensable 
 &    \emph{Taisetsu }大切 
 &    \emph{Taisetsu datta }大切だった 
 &    \emph{Taisetsu deshita }大切でした 
 \\ \cline{1-4} 
 
  Serious 
 &    \emph{Majime }真面目 
 &    \emph{Majime datta }真面目だった 
 &    \emph{Majime deshita }真面目でした 
 \\ \cline{1-4} 
 
  Unique 
 &    \emph{Yuniiku }ユニーク 
 &    \emph{Yuniiku datta }ユニークだった 
 &    \emph{Yuniiku deshita }ユニークでした 
 \\ \cline{1-4} 
 
  Modern 
 &    \emph{Modan }モダン 
 &    \emph{Modan datta }モダンだった 
 &    \emph{Modan deshita }モダンでした 
 \\ \cline{1-4} 
 
\end{ltabulary}
 
\par{\textbf{Grammar Note }: Not all adjectives that end in \slash i\slash  are \emph{Keiy }\emph{ō }\emph{shi }形容詞. This is demonstrated by the existence of \emph{kirei da }綺麗だ. This is why you shouldn't just gravitate toward conjugating something without first examining at least one conjugation. Just one conjugation will help distinguish misnomer \emph{Keiy }\emph{ōd }\emph{ōshi }形容動詞 . If the \slash i\slash  part is written in \emph{Kanji }漢字, or can be written in \emph{Kanji }漢字, it will always be a \emph{Keiy }\emph{ōd }\emph{ōshi }形容動詞. Even if the final \slash i\slash  is written in \emph{Kana }かな, if it can be followed by \emph{d }a だ, it's a \emph{Keiy }\emph{ōd }\emph{ōshi }形容動詞.  }

\par{6. ${\overset{\textnormal{つき}}{\text{月}}}$ は ${\overset{\textnormal{きれい}}{\text{綺麗}}}$ でした。 \hfill\break
 \emph{Tsuki wa kirei deshita. \hfill\break
 }The moon was pretty. }
 
\par{7. ${\overset{\textnormal{たいせつ}}{\text{大切}}}$ だった ${\overset{\textnormal{ひと}}{\text{人}}}$ \hfill\break
 \emph{Taisetsu datta hito \hfill\break
 }A person who was indispensable\slash important }
 
\par{8. ${\overset{\textnormal{ぼく}}{\text{僕}}}$ は ${\overset{\textnormal{まじめ}}{\text{真面目}}}$ だった。 \hfill\break
 \emph{Boku wa majime datta. \hfill\break
 }I was serious. }
 
\par{9. ${\overset{\textnormal{たてもの}}{\text{建物}}}$ がユニークでした。 \hfill\break
 \emph{Tatemono ga yuniiku deshita. \hfill\break
 }The building(s) were unique. }
 
\begin{center}
\textbf{Plain Negative: }\textbf{ \emph{[de wa\slash ja] nai }\{では・じゃ\}ない }
\end{center}
 
\par{ To make \emph{Keiy }\emph{ōd }\emph{ōshi }形容動詞 negative in plain speech, conjugate the copula to \emph{de wa nai }ではない. In the spoken language, this is usually contracted to \emph{ja nai }じゃない. }
 
\begin{ltabulary}{|P|P|P|}
\hline 
 
  Meaning 
 &   Adj. Noun 
 &   + \emph{[de wa\slash ja] nai }\{では・じゃ\}ない (Before Nouns) 
 \\ \cline{1-3} 
 
  No good 
 &    \emph{Dame }駄目 
 &    \emph{Dame [de wa\slash ja] nai }駄目\{では・じゃ\}ない 
 \\ \cline{1-3} 
 
  Dumb 
 &    \emph{Baka }馬鹿 
 &    \emph{Baka [de wa\slash ja] nai }馬鹿\{では・じゃ\}ない 
 \\ \cline{1-3} 
 
  Important 
 &    \emph{Jūyō }重要 
 &    \emph{Jūyō [de wa\slash ja] nai }重要\{では・じゃ\}ない 
 \\ \cline{1-3} 
 
  Strange 
 &    \emph{Hen }変 
 &    \emph{Hen [de wa\slash ja] nai }変\{では・じゃ\}ない 
 \\ \cline{1-3} 
 
  Excellent 
 &    \emph{Yūshū }優秀 
 &    \emph{Yūshū [de wa\slash ja] nai }優秀\{では・じゃ\}ない 
\\ \cline{1-3}

\end{ltabulary}
 
\par{\hfill\break
10. この ${\overset{\textnormal{きも}}{\text{気持}}}$ ちは ${\overset{\textnormal{へん}}{\text{変}}}$ じゃない。 \hfill\break
 \emph{Kono kimochi wa hen ja nai. \hfill\break
 }This feeling\slash these feelings are not strange. }
 
\par{11. あれはもはや ${\overset{\textnormal{じゅうよう}}{\text{重要}}}$ ではない。 \hfill\break
 \emph{Are wa mohaya jūyō de wa nai. \hfill\break
 }That is no longer important. }
 
\par{12. この ${\overset{\textnormal{ひと}}{\text{人}}}$ は ${\overset{\textnormal{ゆうしゅう}}{\text{優秀}}}$ じゃない。 \hfill\break
 \emph{Kono hito wa yūshū ja nai. \hfill\break
 }This person isn\textquotesingle t excellent. }
 
\begin{center}
\textbf{Polite Negative 1: }\textbf{\emph{[de wa\slash ja] nai }desu \{では・じゃ\}ないです }
\end{center}
 
\par{ To make the negative form polite, all you need to do is add \emph{desu }です to the forms above. This option is not as polite as the following one, but it suffices for most instances you\textquotesingle ll encounter in daily conversation. }
 
\begin{ltabulary}{|P|P|P|}
\hline 
 
  Meaning 
 &   Adj. Noun 
 &   + \emph{[de wa\slash ja] nai }desu \{では・じゃ\}ないです 
 \\ \cline{1-3} 
 
  Valuable 
 &    \emph{Daiji }大事 
 &    \emph{Daiji [de wa\slash ja] nai desu }大事\{では・じゃ\}ないです 
 \\ \cline{1-3} 
 
  Harmful 
 &    \emph{Yūgai }有害 
 &    \emph{Yūgai [de wa\slash ja] nai desu }有害\{では・じゃ\}ないです 
 \\ \cline{1-3} 
 
  Calm\slash cool 
 &   \emph{Heiki }平気 
 &    \emph{Heiki [de wa\slash ja] nai desu }平気\{では・じゃ\}ないです 
 \\ \cline{1-3} 
 
  Necessary 
 &    \emph{Hitsuyō }必要 
 &    \emph{Hitsuyō [de wa\slash ja] nai desu }必要\{では・じゃ\}ないです 
 \\ \cline{1-3} 
 
  Urgent\slash steep\slash rapid 
 &    \emph{Kyū }急 
 &    \emph{Kyū [de wa\slash ja] nai desu }急\{では・じゃ\}ないです 
 \\ \cline{1-3} 
 
\end{ltabulary}
 
\par{\textbf{Grammar Note }: Do not use these forms when directly modifying a noun! Use their plain speech counterparts! }

\par{13. ${\overset{\textnormal{さんそ}}{\text{酸素}}}$ は ${\overset{\textnormal{ゆうがい}}{\text{有害}}}$ ではないです。 \hfill\break
 \emph{Sanso wa y }\emph{ūgai de wa nai desu. \hfill\break
 }Oxygen is not harmful. }
 
\par{14. ${\overset{\textnormal{わたし}}{\text{私}}}$ は ${\overset{\textnormal{へいき}}{\text{平気}}}$ じゃないです。 \hfill\break
 \emph{Watashi wa heiki ja nai desu. \hfill\break
 }I\textquotesingle m not cool. }
 
\par{15. ${\overset{\textnormal{しかく}}{\text{資格}}}$ は ${\overset{\textnormal{ひつよう}}{\text{必要}}}$ ではないです。 \hfill\break
 \emph{Shikaku wa hitsuyō de wa nai desu. \hfill\break
 }Qualifications are not necessary. }
 
\par{16. ${\overset{\textnormal{けいしゃ}}{\text{傾斜}}}$ は ${\overset{\textnormal{きゅう}}{\text{急}}}$ ではないです。 \hfill\break
 \emph{Keisha wa kyū de wa nai desu. \hfill\break
 }The incline is not steep. }
 
\begin{center}
\textbf{Polite Negative 2: }\textbf{\emph{[de wa\slash ja] arimasen }\{では・じゃ\}ありません }
\end{center}
 
\par{ The forms below are the politest means of making the negative form. Note again that the use of \emph{ja }じゃ is most common in the spoken language. It is not used much in the written language. }
 
\begin{ltabulary}{|P|P|P|}
\hline 
 
  Meaning 
 &   Adj. Noun 
 &   + \emph{[de wa\slash ja]   arimasen }\{では・じゃ\}ありません 
 \\ \cline{1-3} 
 
  Casual 
 &    \emph{Kajuaru }カジュアル 
 &    \emph{Kajuaru [de wa\slash ja] arimasen }カジュアル\{では・じゃ\}ありません 
 \\ \cline{1-3} 
 
  Famous 
 &    \emph{Yūmei }有名 
 &    \emph{Yūmei [de wa\slash ja] arimasen }有名\{では・じゃ\}ありません 
 \\ \cline{1-3} 
 
  Poor at 
 &    \emph{Nigate }苦手 
 &    \emph{Nigate [de wa\slash ja] arimasen }苦手\{では・じゃ\}ありません 
 \\ \cline{1-3} 
 
  High price 
 &    \emph{Kōka }高価 
 &    \emph{Kōka [de wa\slash ja] arimasen }高価\{では・じゃ\}ありません 
 \\ \cline{1-3} 
 
  Simple 
 &    \emph{Yōi }容易 
 &    \emph{Yōi [de wa\slash ja] arimasen }容易\{では・じゃ\}ありません 
\\ \cline{1-3}

\end{ltabulary}
 
\par{\textbf{Grammar Note }: Do not use these forms when modifying a noun! Use their plain speech counterparts!  }

\par{17. あの ${\overset{\textnormal{かしゅ}}{\text{歌手}}}$ は ${\overset{\textnormal{ゆうめい}}{\text{有名}}}$ ではありません。 \hfill\break
 \emph{Ano kashu wa yūmei de wa arimasen. \hfill\break
 }That singer is not famous. }
 
\par{18. このハンドバッグは ${\overset{\textnormal{こうか}}{\text{高価}}}$ ではありません。 \hfill\break
 \emph{Kono handobaggu wa kōka de wa arimasen. \hfill\break
 }This handbag is not high price. }
 
\par{19. ビジネスカジュアルはカジュアルじゃありません。 \hfill\break
 \emph{Bijinesu kajuaru wa kajuaru ja arimasen. \hfill\break
 }Business casual isn\textquotesingle t casual. }
 
\begin{center}
\textbf{Plain Negative-Past: }\textbf{\emph{[de wa\slash ja] nakatta }\{では・じゃ\}なかった }
\end{center}
 
\par{ The use of \emph{nakatta }なかった shows up here as it did with adjectives in Lesson 13. Remember, \emph{de wa }では > \emph{ja }じゃ in regards to politeness. However, the former takes precedent in the written language; the opposite is so in the spoken language. }
 
\begin{ltabulary}{|P|P|P|}
\hline 
 
  Meaning 
 &   Adj. Noun 
 &   + \emph{[de wa\slash ja] nakatta }\{では・じゃ\}なかった   (Before Nouns) 
 \\ \cline{1-3} 
 
  Rare 
 &    \emph{Mare }稀 
 &    \emph{Mare [de wa\slash ja] nakatta }稀\{では・じゃ\}なかった 
 \\ \cline{1-3} 
 
  Dangerous 
 &    \emph{Kiken }危険 
 &    \emph{Kiken [de wa\slash ja] nakatta }危険\{では・じゃ\}なかった 
 \\ \cline{1-3} 
 
  Complicated 
 &    \emph{Fukuzatsu }複雑 
 &    \emph{Fukuzatsu [de wa\slash ja] nakatta }複雑\{では・じゃ\}なかった 
 \\ \cline{1-3} 
 
  Enthusiastic 
 &    \emph{Nesshin }熱心 
 &    \emph{Nesshin [de wa\slash ja] nakatta }熱心\{では・じゃ\}なかった 
 \\ \cline{1-3} 
 
  Appropriate 
 &    \emph{Tekisetsu }適切 
 &    \emph{Tekisetsu [de wa\slash ja] nakatta }適切\{では・じゃ\}なかった 
\\ \cline{1-3}

\end{ltabulary}
 
\par{\textbf{Grammar Note }: These forms can be used to directly modify a noun! For instance, a "problem that wasn't complicated" would be \emph{fukuzatsu de wa\slash ja nakatta mondai }複雑\{では・じゃ\}なかった問題. }

\par{20. ${\overset{\textnormal{とうろく}}{\text{登録}}}$ が ${\overset{\textnormal{ふくざつ}}{\text{複雑}}}$ ではなかった。 \hfill\break
 \emph{Tōroku ga fukuzatsu de wa nakatta. \hfill\break
 }Registration was not difficult. }
 
\par{21. それは ${\overset{\textnormal{てきせつ}}{\text{適切}}}$ ではなかった。 \hfill\break
 \emph{Sore wa tekisetsu de wa nakatta. \hfill\break
 }That was not appropriate. }
 
\par{22. あの ${\overset{\textnormal{まち}}{\text{町}}}$ は ${\overset{\textnormal{きけん}}{\text{危険}}}$ じゃなかった。 \hfill\break
 \emph{Ano machi wa kiken ja nakatta. \hfill\break
 }That town wasn\textquotesingle t dangerous. }
 
\par{23. ${\overset{\textnormal{かれ}}{\text{彼}}}$ らは ${\overset{\textnormal{ねっしん}}{\text{熱心}}}$ じゃなかった。 \hfill\break
 \emph{Karera wa nesshin ja nakatta. \hfill\break
 }They weren\textquotesingle t enthusiastic. }
 
\begin{center}
\textbf{Polite Negative-Past 1: }\textbf{\emph{[de wa\slash ja] nakatta desu }\{では・じゃ\}なかったです }
\end{center}
 
\par{ To make negative-past polite, just add \emph{desu }です. The same dynamics apply to the negative-past as it did for the negative forms in regards to politeness. Meaning, \emph{de wa }では is more polite than \emph{ja }じゃ, but \emph{ja }じゃ is more common in the spoken language and the opposite is true in the written language. Furthermore, these forms will not be as polite as the ones that will be discussed after this section. }
 
\begin{ltabulary}{|P|P|P|}
\hline 
 
  Meaning 
 &   Adj. Noun 
 &   + \emph{[de wa\slash ja] nakatta desu }\{では・じゃ\}なかったです 
 \\ \cline{1-3} 
 
  Rude 
 &    \emph{Shitsurei }失礼 
 &    \emph{Shitsurei [de wa\slash ja] nakatta desu }失礼\{では・じゃ\}なかったです 
 \\ \cline{1-3} 
 
  Very busy 
 &    \emph{Tabō }多忙 
 &    \emph{Tabō [de wa\slash ja] nakatta desu }多忙\{では・じゃ\}なかったです 
 \\ \cline{1-3} 
 
  Calm 
 &    \emph{Reisei }冷静 
 &    \emph{Reisei [de wa\slash ja] nakatta desu }冷静\{では・じゃ\}なかったです 
 \\ \cline{1-3} 
 
  Special 
 &    \emph{Tokubetsu }特別 
 &    \emph{Tokubetsu [de wa\slash ja] nakatta desu }特別\{では・じゃ\}なかったです 
 \\ \cline{1-3} 
 
  Kind 
 &    \emph{Shinsetsu }親切 
 &    \emph{Shinsetsu [de wa\slash ja] nakatta desu }親切\{では・じゃ\}なかったです 
 \\ \cline{1-3} 
 
\end{ltabulary}
 
\par{\textbf{Grammar Note }: Do not use these forms when directly modifying a noun! Use their plain speech counterparts!  }

\par{24. ${\overset{\textnormal{ぜんぜんしつれい}}{\text{全然失礼}}}$ じゃなかったですよ。 \hfill\break
 \emph{Zenzen shitsurei ja nakatta desu yo. \hfill\break
 }(That) wasn\textquotesingle t rude at all. }
 
\par{25. ${\overset{\textnormal{わたし}}{\text{私}}}$ 、 ${\overset{\textnormal{れいせい}}{\text{冷静}}}$ じゃなかったですね。 \hfill\break
 \emph{Watashi, reisei ja nakatta desu ne. \hfill\break
 }I wasn\textquotesingle t calm, huh. }
 
\par{26. ${\overset{\textnormal{ぼく}}{\text{僕}}}$ は ${\overset{\textnormal{べつ}}{\text{別}}}$ に ${\overset{\textnormal{とくべつ}}{\text{特別}}}$ じゃなかったです。 \hfill\break
 \emph{Boku wa betsu ni tokubetsu ja nakatta desu. \hfill\break
 }I wasn\textquotesingle t particularly special. }
 
\par{27. スタッフが ${\overset{\textnormal{ぜんぜんしんせつ}}{\text{全然親切}}}$ ではなかったです。 \hfill\break
 \emph{Sutaffu ga zenzen shinsetsu de wa nakatta desu. \hfill\break
 }The staff wasn\textquotesingle t kind at all. }
\textbf{Polite Negative-Past 2: }\textbf{\emph{[de wa\slash ja] arimasendeshita }\{では・じゃ\}ありませんでした } \hfill\break
 Notice how the negative-past forms that are most polite end in \emph{arimasendeshita }ありませんでした. Don\textquotesingle t forget to insert \emph{de wa }\slash  \emph{ja }では・じゃ!  Lastly, remember that \emph{ja }じゃ is more indicative of the spoken language. Although it is not as polite as its uncontracted counterpart \emph{de wa }では, it is very common in conversation. However, it's best to avoid when writing. \hfill\break
\hfill\break
 
\begin{ltabulary}{|P|P|P|}
\hline 
 
  Meaning 
 &   Adj. Noun 
 &   + \emph{[de wa\slash ja] arimasendeshita }\{では・じゃ\}ありませんでした 
 \\ \cline{1-3} 
 
  Fair 
 &    \emph{Kōhei }公平 
 &    \emph{Kōhei [de wa\slash ja] arimasendeshita }公平\{では・じゃ\}ありませんでした 
 \\ \cline{1-3} 
 
  Best 
 &    \emph{Saikō }最高 
 &    \emph{Saikō [de wa\slash ja] arimasendeshita }最高\{では・じゃ\}ありませんでした 
 \\ \cline{1-3} 
 
  Free 
 &    \emph{Jiyū }自由 
 &   \emph{Jiyū [de wa\slash ja] arimasendeshita }自由\{では・じゃ\}ありませんでした 
 \\ \cline{1-3} 
 
  Fresh 
 &    \emph{Shinsen }新鮮 
 &    \emph{Shinsen [de wa\slash ja] arimasendeshita }新鮮\{では・じゃ\}ありませんでした 
 \\ \cline{1-3} 
 
  Good at 
 &    \emph{Tokui }得意 
 &    \emph{Tokui [de wa\slash ja] arimasendeshita }得意\{では・じゃ\}ありませんでした 
\\ \cline{1-3}

\end{ltabulary}
 
\par{\textbf{Grammar Note }: Do not use these forms when directly modifying a noun! Use their plain speech counterparts!  }

\par{28. ${\overset{\textnormal{やさい}}{\text{野菜}}}$ は ${\overset{\textnormal{しんせん}}{\text{新鮮}}}$ じゃありませんでした。 \hfill\break
 \emph{Yasai wa shinsen ja arimasendeshita. \hfill\break
 }The vegetables weren\textquotesingle t fresh. }
 
\par{29. ${\overset{\textnormal{わたし}}{\text{私}}}$ は ${\overset{\textnormal{えいご}}{\text{英語}}}$ が ${\overset{\textnormal{とくい}}{\text{得意}}}$ ではありませんでした。 \hfill\break
 \emph{Watashi wa eigo ga tokui de wa arimasendeshita. \hfill\break
 }I wasn\textquotesingle t good at English. }
 
\par{30. ${\overset{\textnormal{むかし}}{\text{昔}}}$ は ${\overset{\textnormal{こうへい}}{\text{公平}}}$ ではありませんでした。 \hfill\break
 \emph{Mukashi wa k }\emph{ōhei de wa arimasendeshita. \hfill\break
 }The olden days were not fair. }
 
\begin{center}
\textbf{More Essential \emph{Keiy }\emph{ōd }\emph{ōshi }形容動詞 }
\end{center}
 
\par{ Now that we have learned the basic conjugations for \emph{Keiy }\emph{ōd }\emph{ōshi }形容動詞, we will spend some time studying more essential example words. After that, we\textquotesingle ll review the conjugations taught in this lesson. }
 
\begin{ltabulary}{|P|P|P|P|}
\hline 
 
  Meaning 
 &   Adj. Noun 
 &   Meaning 
 &   Adj. Noun 
 \\ \cline{1-4} 
 
  Natural 
 &    \emph{Shizen }自然 
 &   Harsh\slash cruel 
 &    \emph{Zankoku }残酷 
 \\ \cline{1-4} 
 
  Unnatural 
 &    \emph{Fushizen }不自然 
 &   Earnest 
 &    \emph{Setsujitsu }切実 
 \\ \cline{1-4} 
 
  Detailed 
 &    \emph{Shōsai }詳細 
 &   Inexperienced 
 &    \emph{Mijuku }未熟 
 \\ \cline{1-4} 
 
  Simplistic 
 &    \emph{Kan\textquotesingle i }簡易 
 &   Prudent 
 &    \emph{Shinch }\emph{ō }慎重 
 \\ \cline{1-4} 
 
  Convenient 
 &    \emph{Benri }便利 
 &   Inconvenient 
 &    \emph{Fuben }不便 
 \\ \cline{1-4} 
 
  Difficult 
 &    \emph{Kon\textquotesingle nan }困難 
 &   Mysterious 
 &    \emph{Fushigi }不思議 
 \\ \cline{1-4} 
 
  Easy(-going) 
 &    \emph{An\textquotesingle i }安易 
 &   Accurate 
 &    \emph{Seikaku }正確 
 \\ \cline{1-4} 
 
  Peculiar 
 &    \emph{Tokushu }特殊 
 &   Subtle 
 &    \emph{Bimyō }微妙 
 \\ \cline{1-4} 
 
  Precise 
 &    \emph{Meikaku }明確 
 &   Brave 
 &    \emph{Yūkan }勇敢 
 \\ \cline{1-4} 
 
  Stylish 
 &    \emph{Sumāto }スマート 
 &   Realistic 
 &    \emph{Riaru }リアル 
\\ \cline{1-4}

\end{ltabulary}

\par{\textbf{Meaning Notes }: \hfill\break
1. \emph{Kon\textquotesingle nan }困難 is “difficult” as in dealing with hardship. It is the opposite of \emph{yōi }容易. \hfill\break
2. \emph{An\textquotesingle i }安易 is like \emph{yōi }容易, but it implies that not much effort is needed to get whatever done. Because of this, it often has a negative image attached to it. It is the opposite of \emph{shinchō }慎重. \hfill\break
3. \emph{Kan\textquotesingle i }簡易 means that something isn\textquotesingle t complicated. It is the opposite of \emph{fukuzatsu }複雑. \hfill\break
4. \emph{Kantan }簡単 is similar to \emph{kan\textquotesingle i }簡易 in that both mean something isn\textquotesingle t complicated; however, it is used more widely in the spoken language. Incidentally, \emph{kan\textquotesingle i }簡易 is usually used in a compound expressions such as \emph{kan\textquotesingle i beddo }簡易ベッド (bunk bed). }
 
\par{31. ルールは ${\overset{\textnormal{めいかく}}{\text{明確}}}$ です。 \hfill\break
 \emph{R }\emph{ūru wa meikaku desu. \hfill\break
 }The rules clear. }
 
\par{32. まだ ${\overset{\textnormal{みじゅく}}{\text{未熟}}}$ でした。 \hfill\break
 \emph{Mada mijuku deshita. \hfill\break
 }I was still inexperienced. }
 
\par{33. ${\overset{\textnormal{すこ}}{\text{少}}}$ しも ${\overset{\textnormal{ふしぎ}}{\text{不思議}}}$ ではありません。 \hfill\break
 \emph{Sukoshi mo fushigi de wa arimasen. \hfill\break
 }It is not even a little mysterious. }
 
\par{34. ${\overset{\textnormal{とく}}{\text{特}}}$ に ${\overset{\textnormal{ふべん}}{\text{不便}}}$ じゃなかったです。 \hfill\break
 \emph{Toku ni fuben ja nakatta desu. \hfill\break
 }It wasn\textquotesingle t especially inconvenient. }
 
\par{35. ${\overset{\textnormal{かれ}}{\text{彼}}}$ は ${\overset{\textnormal{ゆうかん}}{\text{勇敢}}}$ な ${\overset{\textnormal{へいし}}{\text{兵士}}}$ だった。 \hfill\break
 \emph{Kare wa y }\emph{ūkan na heishi datta. \hfill\break
 }He was a brave soldier. }
 
\par{36. ${\overset{\textnormal{ほんとう}}{\text{本当}}}$ に ${\overset{\textnormal{しぜん}}{\text{自然}}}$ ですね。 \hfill\break
 \emph{Hont }\emph{ō ni shizen desu ne. \hfill\break
 }It\textquotesingle s really natural, huh. }
 
\par{37. ${\overset{\textnormal{ふしぜん}}{\text{不自然}}}$ な ${\overset{\textnormal{ぶんしょう}}{\text{文章}}}$ です。 \hfill\break
 \emph{Fushizen na bunsh }\emph{ō desu. \hfill\break
 }This is an unnatural sentence. }
 
\par{38. ${\overset{\textnormal{きじゅつ}}{\text{記述}}}$ が ${\overset{\textnormal{せいかく}}{\text{正確}}}$ じゃなかった。 \hfill\break
 \emph{Kijutsu ga seikaku ja nakatta. \hfill\break
 }The description wasn\textquotesingle t accurate. }
 
\par{39. ${\overset{\textnormal{はつこい}}{\text{初恋}}}$ は ${\overset{\textnormal{ざんこく}}{\text{残酷}}}$ でした。 \hfill\break
 \emph{Hatsukoi wa zankoku deshita. \hfill\break
 }My first love was cruel. }
 
\par{40. リアルなゲームが ${\overset{\textnormal{す}}{\text{好}}}$ きです。 \hfill\break
 \emph{Riaru na g }\emph{ēmu ga suki desu. \hfill\break
 }I like realistic games. }
 
\par{\textbf{Part of Speech Note }: “To like” is actually expressed in Japanese as a \emph{Keiy }\emph{ōd }\emph{ōshi }形容動詞 with \emph{suki da }好きだ. The opposite of this, "to hate," is \emph{kirai da }嫌いだ. It too is not a verb in Japanese. }
 
\begin{center}
\textbf{Conjugation Recap }
\end{center}

\begin{ltabulary}{|P|P|P|P|}
\hline 

Meaning \textrightarrow  & Formal & Handsome & Chic \\ \cline{1-4}

Adjective \textrightarrow  \hfill\break
Conjugations ↓ &  \emph{F }\emph{ōmaru }フォーマル &  \emph{Hansamu }ハンサム &  \emph{Shikku }シック \\ \cline{1-4}

Before Nouns & フォーマルな & ハンサムな & シックな \\ \cline{1-4}

Plain Non-Past & フォーマルだ & ハンサムだ & シックだ \\ \cline{1-4}

Polite Non-Past & フォーマルです & ハンサムです & シックです \\ \cline{1-4}

Plain Past & フォーマルだった & ハンサムだった & シックだった \\ \cline{1-4}

Polite Past & フォーマルでした & ハンサムでした & シックでした \\ \cline{1-4}

Plain Neg. & フォーマル\{では・じゃ\}ない & ハンサム\{では・じゃ\}ない & シック\{では・じゃ\}ない \\ \cline{1-4}

Polite Neg. 1 & フォーマル\{では・じゃ\}ないです & ハンサム\{では・じゃ\}ないです & シック\{では・じゃ\}ないです \\ \cline{1-4}

Polite Neg. 2 & フォーマル\{では・じゃ\}ありません & ハンサム\{では・じゃ\}ありません & シック\{では・じゃ\}ありません \\ \cline{1-4}

Plain Neg-Past & フォーマル\{では・じゃ\}なかった & ハンサム\{では・じゃ\}なかった & シック\{では・じゃ\}なかった \\ \cline{1-4}

Polite Neg-Past 1 & フォーマル\{では・じゃ\}ないです & ハンサム\{では・じゃ\}ないです & シック\{では・じゃ\}ないです \\ \cline{1-4}

Polite Neg-Past 2 & フォーマル\{では・じゃ\}ありませんでした & ハンサム\{では・じゃ\}ありませんでした & シック\{では・じゃ\}ありませんでした \\ \cline{1-4}

\end{ltabulary}

\par{\textbf{Chart Note }: For brevity, the chart above is only rendered in \emph{Kana }かな. Use this as an opportunity for review. }
    