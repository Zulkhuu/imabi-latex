    
\chapter{The Particle Ka か}

\begin{center}
\begin{Large}
第19課: The Particle Ka か 
\end{Large}
\end{center}
 
\par{ In order to make a question in English, we utilize changes in word order and stress. Take for example the following. }

\par{i. Sam will go to the park. \hfill\break
ii. Will Sam go to the park? }

\par{ As you can see, the words "Sam" and "will" flip whenever you turn the statement in Ex. i into a question. In addition, stress is placed on the word "park." In Japanese no word order change is necessary to form a question, but the question must be marked somehow at the end of the sentence. There are many ways to do this, and they all differ in the exact tone and purpose of the question. In a way, what English gets across via intonation is more explicitly stated with the aid of verbal emoticons (final particles.) }

\par{ Because Japanese is more complicated in this respect, it's important for us to first start off with the most basic of questions and work our way up, understanding that more grammar is to come. In this endeavor, we will begin by studying the particle \emph{ka }か. }
      
\section{Vocabulary List}
 
\par{\textbf{Nouns }}

\par{・高校生 \emph{Kōkōsei }–  High school student }

\par{・休憩 \emph{Kyūkei }– Break }

\par{・(お)名前 \emph{(O)namae }– Name }

\par{・(お)誕生日 \emph{(O)tanjōbi }- Birthday }

\par{・試験 \emph{Shiken }– Exam(ination) }

\par{・結婚式 \emph{Kekkonshiki }– Wedding }

\par{・趣味 \emph{Shumi }- Hobby }

\par{・トイレ \emph{Toire }– Bathroom\slash toilet }

\par{・住所 \emph{Jūsho }- Address }

\par{・サラダ \emph{Sarada }- Salad }

\par{・人 \emph{Hito }- Person }

\par{・座席 \emph{Zaseki }– Seat(s) }

\par{・都合 \emph{Tsugō }- Convenience }

\par{・足 \emph{Ashi }– Foot\slash feet }

\par{・お土産 \emph{Omiyage }– Souvenir(s) }

\par{・(お)飲み物 \emph{(O)nomimono }Drink }

\par{・社長 \emph{Shachō }– Company president }

\par{・問題 \emph{Mondai }- Problem }

\par{・時間 \emph{Jikan }- Time }

\par{・クモ \emph{Kumo }- Spider }

\par{・雨 \emph{Ame }- Rain }

\par{・鍵 \emph{Kagi }– Key(s) }

\par{\textbf{Pronouns }}

\par{・あんた \emph{Anta }– You (Rough) }

\par{・君 \emph{Kimi }– You  (Informal) }

\par{\textbf{Proper Nouns }}

\par{・山田さん \emph{Yamada-san }– Mr.\slash Mr(s). Yamada }

\par{・山下さん \emph{Yamashita-san }– Mr.\slash Mr(s). Yamashita }

\par{・日本 \emph{Nihon\slash Nippon }– Japan }

\par{・上野公園 \emph{Ueno kōen }– Ueno Park }

\par{\textbf{Demonstratives }}

\par{・これ \emph{Kore }– This (noun) }

\par{・それ \emph{Sore }– That (noun) }

\par{・あれ \emph{Are }– That over there (noun) }

\par{・あの \emph{Ano }– That over there (adj.) }
  \textbf{Adjectives }
\par{・かわいい \emph{Kawaii }– Cute }

\par{・都合がいい \emph{Tsugō ga ii }- Convenient }

\par{・痛い \emph{Itai }- Painful }

\par{・いい \emph{Ii }– Good }

\par{\textbf{Adjectival Nouns }}

\par{・可能だ \emph{Kanō da }– To be possible }

\par{・好きだ \emph{Suki da }– To like }

\par{・嫌いだ \emph{Kirai da }– To hate }

\par{・元気だ \emph{Genki da }– To be well }

\par{・大丈夫だ \emph{Daijōbu da }– To be alright }

\par{・あほだ \emph{Aho da }– To be dumb\slash stupid }

\par{\textbf{Interjections }}

\par{・はい \emph{Hai }– Yes }

\par{・さ(あ) \emph{Sa(a) }– Well (now)\slash come now }

\par{・では \emph{De wa }– Well (now)\slash (well) then }

\par{・ああ \emph{Ā }– Ah }

\par{\textbf{Adverbs }}

\par{・もう \emph{Mō }– Already\slash (not) anymore\slash before long }

\par{\textbf{\emph{ (u) Godan }}\textbf{Verbs }}

\par{・変わる Kawaru – To change (intr.) }

\par{・取る Toru – To take (trans.) }

\par{・行く Iku – To go (intr.) }

\par{・思う Omou – To think (trans.) }

\par{・分かる \emph{Wakaru }– To become clear\slash be known\slash understand (intr.) }

\par{・違う \emph{Chigau }– To differ\slash be wrong (intr.) }

\par{\textbf{Interrogatives (Question Words) }}

\par{・誰 \emph{Dare }– Who }

\par{・何 \emph{Nani\slash nan }– What }

\par{・いつ \emph{Itsu }– When }

\par{・どこ \emph{Doko }– Where }

\par{・どうして \emph{Dōshite }– Why }

\par{・どう \emph{Dō }– How }
      
\section{Questions Hurt}
 
\par{ The majority of questions we make on a daily basis revolve around the words "who," "what," when," "where," "why," and "how." Japanese is similar in this regard, but because a lot more complexity is placed on things like politeness, tone, and purpose of the question, things can get tricky very quickly. Putting all that aside, the basic means of expressing these questions in Japanese are as follows: }

\begin{itemize}

\item \emph{Dare desu ka? }だれですか ー Who is it? \hfill\break

\item \emph{Nan desu ka? }なんですか ー What is it? 
\item \emph{Itsu desu ka? }いつですか ー When is it? 
\item \emph{Doko desu ka? }どこですか ー Where is it? 
\item \emph{D \emph{ō }shite desu ka? }どうしてですか ー Why is it? 
\item \emph{D \emph{ō }desu ka? }どうですか ー How is it? 
\end{itemize}

\par{ In English, these question words can be used more than just to literally create a question. For instance, they may denote a subordinate clause like in "I forgot \textbf{what }I did yesterday." They may also deviate further such as in "when I go to school" or "use this when you need help.'  These unique circumstances call for particular grammar to be used in Japanese, some of which involves more than the basics we're going over now. You must first understand what exactly the Japanese words refer to in order to build upon them. }

\begin{itemize}

\item \emph{Dare }だれー "Who" as in an unknown someone. \hfill\break

\item \emph{Nani }なに ー "What" as in an unknown something. 
\item \emph{Itsu }いつ ー "When" as in an unknown time. 
\item \emph{Doko }どこ ー "Where" as in an unknown location. 
\item \emph{D \emph{ō }shite }どうして ー "Why" as in an unknown reason. 
\item \emph{D \emph{ō } }どう ー "How" as in an unknown situation\slash means. 
\end{itemize}

\par{\textbf{Form Note }: The base form of "what" in Japanese is \emph{nani }何. }

\par{ As you can see, these words are treated far more literally in Japanese. \emph{Itsu }いつ, for instance, cannot be used to mean the "when" in "when I go to school" because there isn't anything unknown about this statement unless you were to change it to "when am I going to school?" }

\begin{center}
\textbf{The Basic Question }
\end{center}

\par{ The formula for a \emph{basic }question in Japanese will be defined as a polite sentence with no deviation in tone from a simple, harmless question. Add anything to the mix and the result will likely be different. Now that we've seen what the basic question words are in Japanese, let's go over the very straightforward means of using \emph{ka }か in a sentence: just add it to the end and you're done. That is quite literally all you have to do. Just so that questions don't take over your mind, however, we will see how exactly this all looks like with the parts of speech we've covered thus far. }

\begin{ltabulary}{|P|P|}
\hline 

Part of Speech & + \emph{ka }か \\ \cline{1-2}

Noun &  \emph{K }\emph{ō }\emph{k }\emph{ō }\emph{sei desu ka? }高校生ですか (Is..a high school student?) \\ \cline{1-2}

Adjective &  \emph{Kawaii desu ka? }かわいいですか (Is\dothyp{}\dothyp{}\dothyp{}cute?) \\ \cline{1-2}

Adjectival Noun &  \emph{Kan }\emph{ō }\emph{desu ka? }可能ですか (Is\dothyp{}\dothyp{}\dothyp{}possible?) \\ \cline{1-2}

Verb &  \emph{Kawarimasu ka? }変わりますか (Will\dothyp{}\dothyp{}\dothyp{}change?) \\ \cline{1-2}

\end{ltabulary}

\par{ As you can see, there isn't anything particularly difficult about all this. With that being said, it's time to see how this all works in very real examples. Although you may not be able to ask all the questions in the world in Japanese just yet, you can still ask away. }

\par{1. ${\overset{\textnormal{きゅうけい}}{\text{休憩}}}$ を ${\overset{\textnormal{と}}{\text{取}}}$ りますか。 \hfill\break
\emph{Ky }\emph{ūkei wo torimasu ka? }\hfill\break
Will you take a break? }

\par{2. ${\overset{\textnormal{やまだ}}{\text{山田}}}$ さんはどこですか。 \hfill\break
\emph{Yamada-san wa doko desu ka? \hfill\break
}Where is Mr.\slash Mr(s). Yamada? }

\par{3.(お) ${\overset{\textnormal{なまえ}}{\text{名前}}}$ は ${\overset{\textnormal{なん}}{\text{何}}}$ ですか。 \hfill\break
\emph{(O)namae wa nan desu ka? }\hfill\break
What is your name? }

\par{4.(お) ${\overset{\textnormal{たんじょうび}}{\text{誕生日}}}$ はいつですか。 \hfill\break
\emph{(O)tanjōbi wa itsu desu ka? }\hfill\break
When is your birthday? }

\par{5. ${\overset{\textnormal{しけん}}{\text{試験}}}$ はいつですか。 \hfill\break
\emph{Shiken wa itsu desu ka? }\hfill\break
When is the exam(ination)? }

\par{6. ${\overset{\textnormal{けっこんしき}}{\text{結婚式}}}$ はいつですか。 \hfill\break
\emph{Kekkonshiki wa itsu desu ka? }\hfill\break
When is the wedding? }

\par{7. 行きます(か)? \hfill\break
\emph{Ikimasu (ka)? \hfill\break
}Will you go?\slash Shall we go? }

\par{8. ${\overset{\textnormal{しゅみ}}{\text{趣味}}}$ は ${\overset{\textnormal{なん}}{\text{何}}}$ ですか。 \hfill\break
\emph{Shumi wa nan desu ka? }\hfill\break
What are your hobbies? }

\par{9. トイレはどこですか。 \hfill\break
\emph{Toire wa doko desu ka? }\hfill\break
Where is the bathroom? }

\par{10. ${\overset{\textnormal{じゅうしょ}}{\text{住所}}}$ はどこですか。 \hfill\break
\emph{Jūsho wa doko desu ka? }\hfill\break
What is your address? }

\par{11. どう ${\overset{\textnormal{おも}}{\text{思}}}$ いますか。 \hfill\break
\emph{Dō omoimasu ka? \hfill\break
}What do you think (of it)? }

\par{12. ${\overset{\textnormal{わ}}{\text{分}}}$ かりますか。 \hfill\break
\emph{Wakarimasu ka? }\hfill\break
Do you follow\slash understand? }

\par{13. ${\overset{\textnormal{わ}}{\text{分}}}$ かりましたか。 \hfill\break
\emph{Wakarimashita ka? }\hfill\break
Have you got it?\slash Do you understand? }

\par{14. ${\overset{\textnormal{ちが}}{\text{違}}}$ いますか。 \hfill\break
\emph{Chigaimasu ka? }\hfill\break
Is it wrong?\slash Am I wrong? }

\par{15. これは ${\overset{\textnormal{なん}}{\text{何}}}$ ですか。 \hfill\break
\emph{Kore wa nan desu ka? }\hfill\break
What is this? }

\par{16. サラダが ${\overset{\textnormal{きら}}{\text{嫌}}}$ いですか? \hfill\break
\emph{Sarada ga kirai desu ka? }\hfill\break
Do you hate salad? }

\par{17. あの ${\overset{\textnormal{ひと}}{\text{人}}}$ は ${\overset{\textnormal{やました}}{\text{山下}}}$ さんですか。 \hfill\break
\emph{Ano hito wa Yamada-san desu ka? }\hfill\break
Is that person over there Mr.\slash Mr(s). Yamashita? }

\par{18. 「お元気ですか」「はい、元気です」 \hfill\break
\emph{“Ogenki desu ka” “Hai, genki desu.” }\hfill\break
“How are you?” “I\textquotesingle m doing well.” \hfill\break
Literally: “Are you doing well?” “Yes, I\textquotesingle m doing well.” }

\par{\textbf{Phrase Note: }This phrase is perhaps one of the most iconic phrases in Japanese. The \emph{o }attached to \emph{genki }元気 indicates politeness, and it will continue to appear a few more times in this lesson. }

\begin{center}
 \textbf{X \emph{wa }Y (question word) \emph{ga }Z }
\end{center}

\par{ When question words aren\textquotesingle t used as the predicate of the sentence, the differences between \emph{wa }は and \emph{ga }が become most apparent. Instead of seeing the question word at  the end of the sentence preceded by \emph{wa }は, you see that the question word is now marked by \emph{ga }が and that the question word is pinpointing information about the topic. Thus, it's no longer a general question. }

\par{iii. What is a pet?  \textrightarrow  Question word at the end \hfill\break
iv. What would be good for a pet?  \textrightarrow   X \emph{wa }Y (question word) \emph{ga }Z }

\par{ iii. and iv. illustrate how this grammatical difference works in English. iii. follows the same line of questioning seen in the previous section whereas iv. is indicative of the sorts of questions that will soon follow. }

\par{ A ll the question words discussed can be used as either nouns or adverbs except \emph{d }\emph{ō }どう (how), which can only be used as an adverb. }

\par{19. ${\overset{\textnormal{ざせき}}{\text{座席}}}$ はどこがいいですか。 \hfill\break
\emph{Zaseki wa doko ga ii desu ka? }\hfill\break
What seat(s) is\slash are good? \hfill\break
Literally: As for seat(s), where at is good? }

\par{20. いつ(が) ${\overset{\textnormal{つごう}}{\text{都合}}}$ がいいですか。 \hfill\break
\emph{Itsu (ga) tsug }\emph{ō ga ii desu ka? }\hfill\break
When will be convenient for you? \hfill\break
Literally: As for you, when is convenient? }

\par{\textbf{Particle Note }: Although \emph{itsu }いつ can be used as a noun, this is not nearly as common, and so \emph{ga }が is always optional after it. In this example sentence, \emph{tsug }\emph{ō ga ii }都合がいい is a set phrase meaning “convenient,” and because it is grammatically treated as a single adjective, two \emph{ga }が become possible in the same clause. }

\par{21. ${\overset{\textnormal{にほん}}{\text{日本}}}$ のどこが ${\overset{\textnormal{す}}{\text{好}}}$ きですか? \hfill\break
\emph{Nihon no doko ga suki desu ka? }\hfill\break
What part of Japan do you like? \hfill\break
Literally: What of Japan do you like? }

\par{\textbf{Particle Note }: The particle \emph{no }の is used here to indicate “of.” We will learn more about it later in IMABI, but it\textquotesingle s useful here to help make more substantive questions. }

\par{22. ${\overset{\textnormal{あし}}{\text{足}}}$ のどこが ${\overset{\textnormal{いた}}{\text{痛}}}$ いですか。 \hfill\break
\emph{Ashi no doko ga itai desu ka? }\hfill\break
What part of your leg hurts?\slash Where on your leg is it that you are hurting? \hfill\break
Literary: What of your leg hurts? }

\par{23. お ${\overset{\textnormal{みやげ}}{\text{土産}}}$ は ${\overset{\textnormal{なに}}{\text{何}}}$ がいいですか。 \hfill\break
\emph{Omiyage wa nani ga ii desu ka? }\hfill\break
What would be good for souvenirs? \hfill\break
Literally: As for souvenirs, what is good? }

\par{24.(お) ${\overset{\textnormal{の}}{\text{飲}}}$ み ${\overset{\textnormal{もの}}{\text{物}}}$ は ${\overset{\textnormal{なに}}{\text{何}}}$ がいいですか。 \hfill\break
\emph{(O)nomimono wa nani ga ii desu ka? }\hfill\break
What would you like to drink? }

\par{ As you can see, the very fundamental pattern "X + \emph{wa }は + Y + \emph{ga }が + Z" affects question words the same way as any other words, but this also means you\textquotesingle ll have to pay some attention to nuance. Consider the difference between the two following sentences. }

\par{25. ${\overset{\textnormal{しゃちょう}}{\text{社長}}}$ は ${\overset{\textnormal{だれ}}{\text{誰}}}$ ですか。 \hfill\break
\emph{Shachō wa dare desu ka? }\hfill\break
Who is the company president? }

\par{ The topic of conversation here is clearly the company president. The question is “who is he\slash she”? This sentence would be used when you are asking someone to identify who the person is out of a group of people. You are simply confirming what can be verifiable in front of you, and the conversation naturally led to this question. }

\par{26. ${\overset{\textnormal{だれ}}{\text{誰}}}$ が ${\overset{\textnormal{しゃちょう}}{\text{社長}}}$ ですか? \hfill\break
\emph{Dare ga shachō desu ka? }\hfill\break
\emph{Who }is the company president? }

\par{ By itself, this sentence will catch most native speakers off-guard as an odd question. This is because more context is needed for this to be used naturally. The basic nature of \emph{ga }が presenting new information comes into play here. The person in question may or may not be around the speaker, but the speaker is still directly asking someone to identify the individual. The direct nature of \emph{dare ga }誰が happens to be far stronger than with the other question words, which is why is not preferred over other phrasing. The directness can be felt with the other question words as well. }

\par{27. ${\overset{\textnormal{もんだい}}{\text{問題}}}$ は ${\overset{\textnormal{なん}}{\text{何}}}$ ですか。 \hfill\break
\emph{Mondai wa nan desu ka? }\hfill\break
What\textquotesingle s wrong? }

\par{28. ${\overset{\textnormal{なに}}{\text{何}}}$ が ${\overset{\textnormal{もんだい}}{\text{問題}}}$ ですか? \hfill\break
\emph{Nani ga mondai desu ka? }\hfill\break
\emph{What }is the problem? }

\par{ Just as in English, the same sternness that this question possesses comes across in the Japanese as well. Although both sentences could be translated as "what is the problem," the former is not as direct and is merely innocently asking the question at hand. }

\begin{center}
 \textbf{Basic Questions in Plain Speech }
\end{center}

\par{ The lack of \emph{desu }です or \emph{-masu }ます in forming questions in plain speech makes using \emph{ka }か a little bit more tricky, largely because it's not typically used at all. Rather, the phrase becomes a question by the use of high intonation. }

\par{29. ${\overset{\textnormal{だいじょうぶ}}{\text{大丈夫}}}$ ? \hfill\break
\emph{Daijōbu? }\hfill\break
Are you okay? }

\par{30. それは ${\overset{\textnormal{なに}}{\text{何}}}$ ? \hfill\break
\emph{Sore wa nani? }\hfill\break
What is that? }

\par{31. あれはクモ? \hfill\break
\emph{Are wa kumo? }\hfill\break
Is that a spider over there? }

\par{32. ${\overset{\textnormal{うえのこうえん}}{\text{上野公園}}}$ はどこ? \hfill\break
\emph{Ueno kōen wa doko? }\hfill\break
Where's Ueno Park? }

\par{ When \emph{ka }か does happen to be used, a few words of caution are needed. First, it does not attach to \emph{da }だ like it does with \emph{desu }です. The only time this is acceptable is when \emph{ka }か is used to make subordinate clauses, which we'll study in the next lesson. Therefore, \emph{da ka }だか is wrong and must be changed to either \emph{ka }か or dropped entirely. This means it will always attach straight to nouns and adjectival nouns without the copula intervening. }

\par{ Secondly, \emph{ka }か is primarily used in this fashion by male speakers among friends and or toward people of lower status. When it is used out of these arenas, you create a question that shows no reservation\slash modesty toward the listener. As such, it is typically favored by men in very casual situations among each other or whenever they are speaking to people inferior to themselves. If this pattern is used toward someone who is not one\textquotesingle s friend nor someone who has a lower status that oneself, the question will create a tone that borders on interrogation, making the speaker sound like a pompous brute, to say the least. }

\par{33. ${\overset{\textnormal{だいじょうぶ}}{\text{大丈夫}}}$ か? \hfill\break
\emph{Daijōbu ka? }\hfill\break
You alright? }

\par{34. ${\overset{\textnormal{きみ}}{\text{君}}}$ はあほか? \hfill\break
\emph{Kimi wa aho ka? }\hfill\break
Are you stupid or something? }

\par{35. あんた、 ${\overset{\textnormal{い}}{\text{行}}}$ くか? \hfill\break
\emph{Anta, iku ka? }\hfill\break
You coming? }

\par{\textbf{Tone Note }: The use of 35 is largely restricted to males in coarse conversation. }

\par{ Say if the question isn't directed at anyone, but instead, you're talking to oneself or reacting to something and make a rhetorical question to that effect, then か loses its potency. As the following examples demonstrate, this applies to polite speech as well. }

\par{36. もう ${\overset{\textnormal{じかん}}{\text{時間}}}$ (です)か。 \hfill\break
\emph{Mō jikan (desu) ka. }\hfill\break
It's already time, huh\dothyp{}\dothyp{}\dothyp{} }

\par{37. さ、 ${\overset{\textnormal{い}}{\text{行}}}$ くか。 \hfill\break
\emph{Sa, iku ka. }\hfill\break
Well, time to go. }

\par{38. では、 ${\overset{\textnormal{い}}{\text{行}}}$ きますか。 \hfill\break
 \emph{De wa, ikimasu ka. }\hfill\break
Alright, time to go. }

\par{39. ${\overset{\textnormal{あめ}}{\text{雨}}}$ 、 ${\overset{\textnormal{ふ}}{\text{降}}}$ ったか。 \emph{\hfill\break
Ame, futta ka. }\hfill\break
It rained, huh. }

\par{40. ああ、そうか。 \hfill\break
\emph{Ā, sō ka. }\hfill\break
Ah, really?\slash I see. }
 
\begin{center}
\textbf{Question Word + \emph{ka }か: X }
\end{center}

\par{ Ignoring grammar concerning subordinate clauses which we haven't covered yet, you can't simply add \emph{ka }か to question words like you can with other nouns in plain speech. Instead, you either need to add the particle \emph{yo }よ for exclamatory effect or use \emph{da }だ instead, as non-intuitive as that might seem. }

\begin{ltabulary}{|P|P|P|}
\hline 

Who &  \emph{Dare }\textrightarrow  \emph{Dare da (yo) }& だれ \textrightarrow  だれだ(よ) \\ \cline{1-3}

What & \emph{Nani }\textrightarrow  \emph{Nan da (yo) }& なに \textrightarrow  なんだ(よ) \\ \cline{1-3}

When &  \emph{Itsu }\textrightarrow  \emph{Itsu }& いつ \textrightarrow  いつ \\ \cline{1-3}

Where &  \emph{Doko }\textrightarrow  \emph{Doko da yo }& どこ \textrightarrow  どこだ(よ) \\ \cline{1-3}

Why &  \emph{Dō }\emph{shite }\textrightarrow  \emph{Dōshite da (yo) }\hfill\break
& どうして \textrightarrow  どうしてだ(よ) \\ \cline{1-3}

\end{ltabulary}

\par{ The use of the particle \emph{yo }よ adds to the weight of frustration the speaker has, and as such, it is not a given that the question is rhetorical or not. As the chart above suggests, because \emph{itsu }いつ is almost always adverbial, it is treated differently in heated questions such as these. }

\par{41. ${\overset{\textnormal{なん}}{\text{何}}}$ だ、これ? \hfill\break
\emph{Nan da, kore ? }\hfill\break
What the heck is this? }

\par{42. えっ、 ${\overset{\textnormal{なん}}{\text{何}}}$ だよ! \hfill\break
\emph{Eh, nan da yo! }\hfill\break
W-what the heck? }

\par{43. ${\overset{\textnormal{かぎ}}{\text{鍵}}}$ はどこだ。 \hfill\break
\emph{Kagi wa doko da? }\hfill\break
Where are the keys?\slash Where is the key? }

\par{44. ${\overset{\textnormal{かぎ}}{\text{鍵}}}$ はどこだよ! \hfill\break
\emph{Kagi wa doko da yo!? }\hfill\break
Where are the dang keys!? \slash Where is the dang key? }

\par{45. ${\overset{\textnormal{きみ}}{\text{君}}}$ 、どうしてだよ! \hfill\break
\emph{Kimi, dōshite da yo!? }\hfill\break
You\dothyp{}\dothyp{}\dothyp{}why!? }
    