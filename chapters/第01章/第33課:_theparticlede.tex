    
\chapter{The Particle で I}

\begin{center}
\begin{Large}
第33課: The Particle で I 
\end{Large}
\end{center}
 
\par{ In this lesson we'll study another crucial particle of Japanese, で. Many confuse it with に. However, after this lesson, you will realize that it is not the same thing. There are times when they overlap, but you'll leave this lesson with the knowledge needed to distinguish them. }
      
\section{The Case Particle で}
 
\par{1. で and に both frequently translate as "at." With で, however, there is no deep connection implied between the action and the place of said action as there is with に. The best example of this contrast between the two particles can be seen with the concept of "to work." }
 
\par{1. ここで ${\overset{\textnormal{はたら}}{\text{働}}}$ きます。    VS    ここに ${\overset{\textnormal{つと}}{\text{勤}}}$ めています。 \hfill\break
I work here.             I work here. }
 
\par{The first one says that you just work there. You may just be an extra hand, someone that just does things, or the job may just be temporary. The second says that you have been working there; you're employed there. It's the kind of work where you are going to probably be there for a long time. For example, ${\overset{\textnormal{かいしゃ}}{\text{会社}}}$ に勤めている. }
 
\par{に relates to the state of something whereas で relates to the occurrence of something. That is why に is used with verbs like  ある, いる, and 勤める and で is used verbs like 働く, ${\overset{\textnormal{}}{\text{会}}}$ う, etc. However, it is very possible that you will have to use both in the same sentence because of other usages they have. }
 
\par{${\overset{\textnormal{}}{\text{2. 学校}}}$ で ${\overset{\textnormal{}}{\text{勉強}}}$ します。 \hfill\break
I study at school. }

\par{3. ${\overset{\textnormal{としょかん}}{\text{図書館}}}$ で ${\overset{\textnormal{}}{\text{勉強}}}$ しました。 \hfill\break
I studied at the library. }

\par{4. ${\overset{\textnormal{えきまえ}}{\text{駅前}}}$ で ${\overset{\textnormal{}}{\text{買}}}$ い ${\overset{\textnormal{もの}}{\text{物}}}$ (を)する。 \hfill\break
To shop in front of the train station. }
 
\par{\textbf{Culture Note }: There are almost always lots of shops in and around train stations in Japan. }

\par{5. ${\overset{\textnormal{じむしょ}}{\text{事務所}}}$ で ${\overset{\textnormal{}}{\text{電話}}}$ をかけます。 \hfill\break
I will make a call at the office. }
 
\par{${\overset{\textnormal{}}{\text{6. 彼女}}}$ は ${\overset{\textnormal{}}{\text{海}}}$ で ${\overset{\textnormal{}}{\text{泳}}}$ いだ。 \hfill\break
She swam in the sea. }

\par{\textbf{Particle Note }: 海に泳ぐ in Modern Japanese is incorrect to the majority of speakers. When you use で, you are specifying the location of swimming. Otherwise, you would say 海 \textbf{を }泳ぐ. }

\par{${\overset{\textnormal{}}{\text{7. 私}}}$ はいまびというサイトで ${\overset{\textnormal{}}{\text{日本語}}}$ を ${\overset{\textnormal{}}{\text{勉強}}}$ しています。 \hfill\break
I'm studying Japanese at IMABI. }

\par{8. ${\overset{\textnormal{こども}}{\text{子供}}}$ たちが ${\overset{\textnormal{にわ}}{\text{庭}}}$ で ${\overset{\textnormal{}}{\text{遊}}}$ んでいる。 \hfill\break
The children are playing in the garden. }
 
\par{${\overset{\textnormal{}}{\text{9. 北海道}}}$ で ${\overset{\textnormal{じしん}}{\text{地震}}}$ がありました。 \hfill\break
There was an earthquake in Hokkaido. }
 
\par{2. で shows the condition or method in which something is done--"of\slash with\slash by." }
 
\par{${\overset{\textnormal{}}{\text{10. 日本語}}}$ で話してください。 \hfill\break
Please speak in Japanese. }
 
\par{11. お ${\overset{\textnormal{}}{\text{金}}}$ の ${\overset{\textnormal{りょうがえ}}{\text{両替}}}$ はどこでできますか。 \hfill\break
Where can I exchange money? }
 
\par{12. ステレオで ${\overset{\textnormal{おんがく}}{\text{音楽}}}$ を ${\overset{\textnormal{}}{\text{聞}}}$ く。 \hfill\break
Listen to music with a stereo. }
 
\par{${\overset{\textnormal{}}{\text{13. 彼}}}$ は ${\overset{\textnormal{はやくち}}{\text{早口}}}$ でしゃべった。 \hfill\break
He chattered rapidly. }

\par{14. ${\overset{\textnormal{けっこんしき}}{\text{結婚式}}}$ に ${\overset{\textnormal{きもの}}{\text{着物}}}$ で ${\overset{\textnormal{}}{\text{行}}}$ きます。 \hfill\break
I will go to the wedding in kimono. }

\par{15. ${\overset{\textnormal{}}{\text{雨}}}$ の日(に)はバスで学校に行きますか。 \hfill\break
Do you go by bus to school on a rainy day? }
 
\par{${\overset{\textnormal{}}{\text{16. 僕}}}$ は ${\overset{\textnormal{かみ}}{\text{紙}}}$ でカモを ${\overset{\textnormal{つく}}{\text{作}}}$ った。 \hfill\break
I made a duck out of paper. }

\par{17. ${\overset{\textnormal{ちかてつ}}{\text{地下鉄}}}$ で ${\overset{\textnormal{}}{\text{行}}}$ きます。 \hfill\break
I('ll) go by subway. }
 
\par{18. このCDを3 ${\overset{\textnormal{}}{\text{千円}}}$ で ${\overset{\textnormal{}}{\text{買}}}$ った。 \hfill\break
I bought this CD for three thousand yen. }
 
\par{19. もはやデパートで ${\overset{\textnormal{}}{\text{働いてい}}}$ ませんよ。 \hfill\break
I no longer work at the department store. }

\par{20. ${\overset{\textnormal{われわれ}}{\text{我々}}}$ は ${\overset{\textnormal{とうひょう}}{\text{投票}}}$ で ${\overset{\textnormal{き}}{\text{決}}}$ めました。 \hfill\break
We decided by voting. }

\par{\textbf{Word Note }: Remember that 我々 is used in very formal situations. }
 
\par{21. 私はナイフで ${\overset{\textnormal{にく}}{\text{肉}}}$ を ${\overset{\textnormal{}}{\text{切}}}$ りました。 \hfill\break
I cut the meat up with a knife. }

\par{22. ${\overset{\textnormal{みせ}}{\text{店}}}$ の ${\overset{\textnormal{しんぶん}}{\text{新聞}}}$ でその記事を読みました 。 \hfill\break
I read that in the newspaper in the store. }

\par{23. これは ${\overset{\textnormal{でんき}}{\text{電気}}}$ で ${\overset{\textnormal{うご}}{\text{動}}}$ きます。 \hfill\break
This works with electricity. }
 
\par{24. お ${\overset{\textnormal{}}{\text{茶}}}$ は ${\overset{\textnormal{い}}{\text{要}}}$ りません。 ${\overset{\textnormal{}}{\text{水}}}$ でけっこうです。 \hfill\break
I don't need\slash want tea. I'm fine with water. }

\par{25. わたしはいつも(お) ${\overset{\textnormal{はし}}{\text{箸}}}$ で ${\overset{\textnormal{}}{\text{食}}}$ べます。 \hfill\break
I always eat with chopsticks. }
 
\par{\textbf{Culture Note }: Do not point chopsticks completely vertically in a bowl of rice as this resembles burning incense sticks in funerals. Don't pass food with chopsticks because this is how bones are handled. Pass food by placing it on a small plate or using the ends. Mismatched chopsticks aren't used. Pointing chopsticks at someone may be considered a threat. }

\begin{center}
 \textbf{Curve Ball: に働く Exists\dothyp{}\dothyp{}\dothyp{} }
\end{center}

\par{Many students are penalized all the time making the annoying mistake of writing ~に働いています on a test. Unbeknownst to these students, the phrase isn't so incorrect as they have been taught. In fact, ~に働く is pretty common. }

\par{Why, then, must teachers distill false information? For one, telling students to google search examples of ~に働いています doesn't result in thousands of instances of "place + に働いています." The word preceding it can be a number of things such as adverbs, time phrases, etc. Furthermore, if a student were to look up 働く in a Japanese dictionary in Japanese, the verb is defined in way indicative of a temporary job position to simply sustain one's livelihood. It doesn't intrinsically imply working in a career sense. }

\par{However, when this is all blurred, the state of the job doesn't necessarily matter. After all, native speakers aren't going to consider the state of their job to determine which particle to use. }

\par{26. 彼女はうちの店に働いていますわ。 \hfill\break
She's working at my store. }
    