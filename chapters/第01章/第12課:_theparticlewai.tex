    
\chapter{The Particle Wa は I}

\begin{center}
\begin{Large}
第12課: The Particle Wa は I: The Topic\slash Contrast Marker Wa は 
\end{Large}
\end{center}
 
\par{  The particle \emph{wa }\emph{ }は, unlike \emph{ga }が, is not a case particle. This means that it grammatically doesn't stand for the subject or even the object of a sentence. Instead, it is a special kind of particle called a bound particle. \emph{Wa }は is bound to the comment that follows. In return, the comment dictates the nature of \emph{wa }は. }

\par{ It is not possible to know exactly what will be said with \emph{wa }は alone when in total isolation without context. The only th ing the listener would know in such a situation is that \emph{wa }は will mark the topic of the discussion to come. In addition, the topic marked by \emph{wa }は is differentiated from other things that could be the topic, which is in and of itself contrast, which will also be looked at in depth in this lesson. }

\par{ In this lesson, we will continue our discussion on \emph{ga }が vs \emph{wa }は by looking closely at the usages of \emph{wa }は . After reading through this lesson, you will have learned enough about both particles to adequately differentiate them in most circumstances. }

\par{\textbf{Curriculum Note }: This lesson also requires that we look at gra mmatical items which haven\textquotesingle t been fully covered. This includes adjectives, adjectival nouns, verbs and their conjugation. As such, your goal should be to focus only on the particles \emph{ga }が and \emph{wa }は.  }
      
\section{Vocabulary List (Under Construction)}
       
\section{The Bound Particle Wa は}
 
\par{1. \textbf{The Topic Marker \emph{Wa }は }}

\begin{center}
\textbf{i. What is a “Topic”? }
\end{center}

\par{ To understand \emph{wa }は, we must first understand what is meant by the word “topic.” The topic ( \emph{shudai }主題) of a sentence can be an animate or inanimate entity (of one or more components), and that entity is what provides a starting point for conversation. A topic must also be something based on previously established information, whether it be from the ongoing conversation, one not too far back in the past, or from common sense. }

\par{ The topic is, thus, “old information.” In order for something to be register ed information, though, you may need to use \emph{ga } が first to establish it. Essentially, information needs to be new before it can be grammatically treated as old information. This distinction between new information and known information is exemplified in Ex. 1. }

\par{1. ${\overset{\textnormal{むかしむかし}}{\text{昔々}}}$ 、あるところに、おじいさんとおばあさんが ${\overset{\textnormal{す}}{\text{住}}}$ んでいました。おじいさんは ${\overset{\textnormal{やま}}{\text{山}}}$ へ ${\overset{\textnormal{しばか}}{\text{柴刈}}}$ りに、おばあさんは ${\overset{\textnormal{かわ}}{\text{川}}}$ へ ${\overset{\textnormal{せんたく}}{\text{洗濯}}}$ に ${\overset{\textnormal{い}}{\text{行}}}$ きました。 \hfill\break
\emph{Mukashi mukashi, aru tokoro ni, ojiisan to obāsan ga sunde imashita. Ojiisan wa yama e shibakari ni, obāsan wa kawa e sentaku ni ikimashita. } \hfill\break
Long, long ago, there lived an old man and woman. One day, the old man went to the mountains to gather firewood, and the old woman went to the river to wash clothes. }

\par{This sentence is the opening to one of the most important fairy-tales of Japan, \emph{Momotarō }桃太郎 . At the beginning, the reader doesn't know anything about the story. This is why the particle \emph{ga } が is used to mark the subject. Once the characters are established, they are then treated as the topic in the following sentence, thus marked by \emph{wa }は . }

\par{2. あれは ${\overset{\textnormal{わたし}}{\text{私}}}$ の ${\overset{\textnormal{ぼうし}}{\text{帽子}}}$ です。 \hfill\break
\emph{Are wa watashi no bōshi desu. \hfill\break
}That's my hat. }

\par{\textbf{Sentence Note }: Although the comment, the hat being the speaker's, is "new information," the recognition of the hat is not. }

\par{ In Japanese, phrases may be topicalized and put at or near the front of the sentence, after which point a comment is made about said topic. The comment could be already kn own or new information, but the topic is something implied to be known to both speaker and listener(s). The topic, as mentioned above, is deemed to be an entity known to others and oneself. Often times, this is based on a common sense assessment of reality. }

\par{3. お ${\overset{\textnormal{なまえ}}{\text{名前}}}$ は ${\overset{\textnormal{なん}}{\text{何}}}$ ですか。 \hfill\break
\emph{O-namae wa nan desu ka? }\hfill\break
What's your name? }
 
\par{\textbf{Sentence Note }: Everyone has a name. Even if this statement weren't completely true, it's practically true. This is all the information one needs to know about the human world to understand how "your name" can be grammatically treated as "old\slash registered" knowledge. You know the person you're talking to has a name; you just don't know what that person's name is, which is why the question forms the comment about the topic. }
 
\par{4. トイレはどこですか。 \hfill\break
\emph{Toire wa doko desu ka? }\hfill\break
Where is the toilet? }
 
\par{\textbf{Sentence Note }: When you ask this to someone, you're assuming that there is a toilet nearby. The existence of toilets can be rather easily ascertained based on one's surroundings. The fact that you're asking this means you've already determined that there is one, and you're also implying that the existence and knowledge of its location is something that others might help you find out. }
 
\par{5. ${\overset{\textnormal{かせい}}{\text{火星}}}$ は ${\overset{\textnormal{あか}}{\text{赤}}}$ いです。 \hfill\break
\emph{Kasei wa akai desu. }\hfill\break
Mars is red. }
 
\par{\textbf{Sentence Note }: Most people know about Mars. It has been a part of human fascination for a long time, and so the acknowledgment of its existence is well established. It being red is also something that is so well known that it can be viewed as a generic statement. }
 
\par{6. ${\overset{\textnormal{にほん}}{\text{日本}}}$ は ${\overset{\textnormal{しま}}{\text{島}}}$ の ${\overset{\textnormal{くに}}{\text{国}}}$ です。 \hfill\break
\emph{Nihon wa shima no kuni desu. }\hfill\break
Japan is an island nation. }
 
\par{\textbf{Sentence Note }: Japan is known by both all Japanese speakers as well as most of the world, and it's also known by most people that it is an island nation. }
 
\par{7. ウサギはかわいいですね。  \hfill\break
\emph{Usagi wa kawaii desu ne. }\hfill\break
Rabbits are cute, aren't they? }
 
\par{\textbf{Sentence Note }: Wherever rabbits exist, there are humans that know about them. }

\begin{center}
\textbf{The Zero-Pronoun }
\end{center}

\par{ Whenever the topic is semantically the same as the subject or even the object of a sentence, the particle \emph{wa }は does not mark both. It only function as the topic marker. All sorts of things can be topicalized, which makes it seem like \emph{wa }は has far more functions than it actually does. Semantically, it is very similar to the English expression “as for X.” It's the "X" in this expression that \emph{wa }は stands for, and nothing more. However, using “as for” heavily in translation will result in unnatural English. Using one\textquotesingle s own intuition on what is proper English will come to play here. Nonetheless, it's a perfect stepping stone for understanding how this particle functions grammatically. }

\par{8a. ${\overset{\textnormal{わたし}}{\text{私}}}$ は ${\overset{\textnormal{まいにち}}{\text{毎日}}}$ ジムに ${\overset{\textnormal{い}}{\text{行}}}$ きます。 \hfill\break
\emph{Watashi wa mainichi jimu ni ikimasu. \hfill\break
}I go to the gym every day. \hfill\break
\hfill\break
 Ex. 8a can alternatively be translated as, “As for me, I go to the gym every day.” The purpose of \emph{wa }は is two-fold. It establishes that “I” is the topic, but it also differentiates it from other possible topics like “he” or “she.” As such, the reason why \emph{watashi }私 would even be used instead of just being dropped—which is usually the case—is because the speaker has become the center of conversation. Although the subject of this sentence is “I,” the \emph{watashi }私 of this sentence corresponds to the "me" in “as for me.” The “I” that corresponds to the subject is not spoken because it would be semantically redundant. In fact, \emph{watashi wa watashi ga }私は私が is ungrammatical. }

\par{ This is where the concept of a zero-pronoun comes into play. A zero-pronoun is a pronoun used to refer to the subject of a Japanese sentence when it is omitted because it is juxtaposed with a topic that happens to be the same thing. It is the grammatical fix to the grammaticalized rule of omitting semantically redundant elements. More broadly, a zero-pronoun is used in place of an entity that is semantically the same as the topic. Thus, this can be applied to other situations as we will see as well. With a zero-pronoun in mind, we can view 8a as follows: }

\par{8b. ${\overset{\textnormal{わたし}}{\text{私}}}$ は ${\overset{\textnormal{まいにち}}{\text{毎日}}}$ ジムに ${\overset{\textnormal{い}}{\text{行}}}$ きます。 \hfill\break
\emph{Watashi-wa ( }\emph{ø-ga) jimu-ni ikimasu. \hfill\break
}As for me, (I) go to the gym every day. \hfill\break
\emph{ø = Watashi }${\overset{\textnormal{わたし}}{\text{私}}}$ }

\par{9a. ケーキはもう ${\overset{\textnormal{た}}{\text{食}}}$ べました。 \hfill\break
\emph{Kēki wa mō tabemashita. \hfill\break
}The cake, I already ate it. }

\par{\textbf{Grammar Note }: The particle \emph{wa }は appears to mark the direct object, but in reality, it simply marks the topic which also happens to be the object, but the object is expressed with an unexpressed zero-pronoun. Thus, Ex. 9a can be viewed alternatively as follows: }

\par{9b. ケーキはもう ${\overset{\textnormal{た}}{\text{食}}}$ べました。 \hfill\break
\emph{Kēki-wa mō ( }\emph{ø-wo) tabemashita. \hfill\break
}The cake, I already ate it. \hfill\break
\emph{ø = Kēki }ケーキ }

\begin{center}
\textbf{The Variety of Topicalized Phrases }
\end{center}

\par{ The particle \emph{wa }は has few restrictions on what it can topicalize. It may topicalize time phrases, location phrases, etc. This is exemplified in the following examples. }

\par{10. ${\overset{\textnormal{にほん}}{\text{日本}}}$ では ${\overset{\textnormal{じしん}}{\text{地震}}}$ がよく ${\overset{\textnormal{お}}{\text{起}}}$ きます。 \hfill\break
\emph{Nihon de wa jishin ga yoku okimasu. \hfill\break
}In Japan, earthquakes often happen. }

\par{11. ${\overset{\textnormal{きょう}}{\text{今日}}}$ は ${\overset{\textnormal{かんこくご}}{\text{韓国語}}}$ を ${\overset{\textnormal{べんきょう}}{\text{勉強}}}$ します。 \hfill\break
\emph{Kyō wa kankokugo wo benkyō shimasu. \hfill\break
}Today, I will study Korean. }

\par{12. ${\overset{\textnormal{わたし}}{\text{私}}}$ はお ${\overset{\textnormal{ちゃ}}{\text{茶}}}$ です。 \hfill\break
\emph{Watashi wa ocha desu. \hfill\break
}I\textquotesingle ll have tea. }

\par{\textbf{Grammar Note }: Whenever learners don\textquotesingle t fully understand the concept of topicalization, they fail to understand that topic ≠ subject. It\textquotesingle s best to never consider them one of the same thing. If this means having to deconstruct sentences and translate them literally first to figure out what the subject is and whether it\textquotesingle s being represented by a zero-pronoun so that you don\textquotesingle t end up misunderstanding sentences like Ex. 12 as meaning “I am tea,” then it would be worth it  }

\par{13. こちらは ${\overset{\textnormal{わたし}}{\text{私}}}$ の ${\overset{\textnormal{おとうと}}{\text{弟}}}$ です。 \hfill\break
\emph{Kochira wa watashi no otōto desu. \hfill\break
}This is my little brother. }

\par{14. ${\overset{\textnormal{かのじょ}}{\text{彼女}}}$ は\{ ${\overset{\textnormal{ちゅうごくじん}}{\text{中国人}}}$ ・ ${\overset{\textnormal{にほんじん}}{\text{日本人}}}$ ・アメリカ ${\overset{\textnormal{じん}}{\text{人}}}$ ・イギリス ${\overset{\textnormal{じん}}{\text{人}}}$ \}です。 \hfill\break
\emph{Kanojo wa [chūgokujin\slash nihonjin\slash amerikajin\slash igirisujin] desu. \hfill\break
}She is [Chinese\slash Japanese\slash American\slash British]. }

\par{15. ${\overset{\textnormal{ちゅうごくけいざい}}{\text{中国経済}}}$ には ${\overset{\textnormal{もんだい}}{\text{問題}}}$ がある。 \hfill\break
\emph{Chūgoku keizai ni wa mondai ga aru. \hfill\break
}There is\slash are problem(s) in the Chinese economy. \hfill\break
\hfill\break
\textbf{Grammar Note }: Due to English phrasing constraints, it may not always be possible to place the topicalized phrase of a Japanese sentence at the front of the English translation. However, the fact that the \emph{wa }は phrase in question is being topicalized and the fact that said \emph{wa }は phrase forms the basis for the upcoming conversation do not change. }

\par{16. ${\overset{\textnormal{わたし}}{\text{私}}}$ は ${\overset{\textnormal{い}}{\text{行}}}$ きません。 \hfill\break
\emph{Watashi wa ikimasen. \hfill\break
}I won\textquotesingle t go. }

\par{17. ${\overset{\textnormal{かれ}}{\text{彼}}}$ は ${\overset{\textnormal{せんせい}}{\text{先生}}}$ ではありません。 \hfill\break
\emph{Kare wa sensei de wa arimasen. \hfill\break
}He is not a teacher. }

\par{\textbf{Grammar Notes }: }

\par{1. Ex. 16 and Ex. 17 are examples of the particle \emph{wa }は bringing out the meaning of “X isn\textquotesingle t but something\slash someone else might be\slash do Z.” This implicit contrast is something that, depending on the context, may become even more profound (See Usage 2). As for Ex. 17, it could be that another person is a teacher, or "he" could be something other than a teacher. If the particle \emph{ga }が were used, the sentences would become examples of exhaustive-listing. Remember, exhaustive-listing is still exhaustive if X simply refers to one entity and one entity only. }

\par{2. The \emph{wa }は in \emph{de wa arimasen }ではありません is not the topic \emph{wa }は. Rather, it is one usage of the contrast marker \emph{wa }は (Usage 2). }

\begin{center}
\textbf{ii. Generic Statements }
\end{center}

\par{ Many conversations are started off by mentioning something everyone already knows. However, implying that the listener(s) knows is subjective in nature. This is because one can never definitively know what someone else does or doesn\textquotesingle t know. This usage of \emph{wa }は is very different from the exhaustive-listing statements that \emph{ga }が can make. Whereas an exhaustive-listing sentence is limited semantically solely to what\textquotesingle s explicitly stated, \emph{wa }は is far more open-ended due to its generic nature. There is always a chance for the speaker to imply “I know that X is Z, but I don\textquotesingle t know about Y.” }

\par{18. リンゴは ${\overset{\textnormal{ちい}}{\text{小}}}$ さい。 \hfill\break
\emph{Ringo wa chiisai. \hfill\break
}(The) apples are small. }

\par{\textbf{Spelling Note }: \emph{Ringo }is only seldom spelled as 林檎. }

\par{19. ${\overset{\textnormal{そら}}{\text{空}}}$ は ${\overset{\textnormal{あお}}{\text{青}}}$ い。 \hfill\break
\emph{Sora wa aoi. \hfill\break
}The sky is blue. }

\par{20. ${\overset{\textnormal{うちゅう}}{\text{宇宙}}}$ は ${\overset{\textnormal{ひろ}}{\text{広}}}$ い。 \hfill\break
\emph{Uchū wa hiroi. \hfill\break
}The universe is wide. }

\par{21. ${\overset{\textnormal{たいよう}}{\text{太陽}}}$ は ${\overset{\textnormal{あか}}{\text{明}}}$ るい。 \hfill\break
\emph{Taiyō wa akarui. }\hfill\break
The sun is bright. }

\par{22. ${\overset{\textnormal{よる}}{\text{夜}}}$ は ${\overset{\textnormal{くら}}{\text{暗}}}$ い。 \hfill\break
\emph{Yoru wa kurai. }\hfill\break
Night is dark. }

\par{23. ${\overset{\textnormal{はな}}{\text{花}}}$ は ${\overset{\textnormal{うつく}}{\text{美}}}$ しい。 \hfill\break
\emph{Hana wa utsukushii. }\hfill\break
Flowers are beautiful. }

\par{24. ${\overset{\textnormal{はる}}{\text{春}}}$ は ${\overset{\textnormal{すば}}{\text{素晴}}}$ らしいですね。 \hfill\break
\emph{Haru wa subarashii desu ne. }\hfill\break
Spring is wonderful, isn't it? }

\par{25. ${\overset{\textnormal{せかい}}{\text{世界}}}$ は ${\overset{\textnormal{ちい}}{\text{小}}}$ さいですね。  \hfill\break
\emph{Sekai wa chiisai desu ne. }\hfill\break
The world is small, isn't it? }

\par{26. ${\overset{\textnormal{すうがく}}{\text{数学}}}$ は ${\overset{\textnormal{むずか}}{\text{難}}}$ しいですね。 \hfill\break
\emph{Sūgaku wa muzukashii desu ne. \hfill\break
}Math is difficult, isn\textquotesingle t it? }

\par{\textbf{Sentence Note }: As a demonstration of the last point from above, this statement should be interpreted as meaning “I\textquotesingle m not sure about other subjects being hard, but math is, isn\textquotesingle t it?” }

\begin{center}
\textbf{iii. Attribute Phrases: X \emph{wa }は Y \emph{ga }が }
\end{center}

\par{ One of the most common ways to describe something is by following a topicalized phrase (X) with \emph{wa }は with a neutral statement (Y) followed by \emph{ga }が. In the examples below, there are generally two kinds of translations. The first reflects the nature of the Japanese grammar whereas the second rephrases it into more natural English. As you will see, the resulting translation indicates how this grammar is essentially identical to making generic statements. }

\par{27. ${\overset{\textnormal{ぞう}}{\text{象}}}$ は ${\overset{\textnormal{はな}}{\text{鼻}}}$ が ${\overset{\textnormal{なが}}{\text{長}}}$ い。 \hfill\break
\emph{Zō wa hana ga nagai. \hfill\break
}As for elephants, their noses are long. \hfill\break
Elephants have long noses. }

\par{28. ${\overset{\textnormal{にほん}}{\text{日本}}}$ は ${\overset{\textnormal{じんじゃ}}{\text{神社}}}$ が ${\overset{\textnormal{おお}}{\text{多}}}$ い。 \hfill\break
\emph{Nihon wa jinja ga ōi. \hfill\break
}As for Japan, there are many Shinto shrines. \hfill\break
Japan has many Shinto shrines. }

\par{29. ${\overset{\textnormal{あき}}{\text{秋}}}$ はサンマが最高だ。 \hfill\break
\emph{Aki wa samma ga saikō da. \hfill\break
}As for autumn, Pacific saury is the best. \hfill\break
In autumn, Pacific saury is the best. }

\par{30. ${\overset{\textnormal{ふゆ}}{\text{冬}}}$ には ${\overset{\textnormal{きおん}}{\text{気温}}}$ が ${\overset{\textnormal{さ}}{\text{下}}}$ がります。 \hfill\break
\emph{Fuyu ni wa kion ga sagarimasu. \hfill\break
}In winter, the temperature goes down. }

\par{31. その ${\overset{\textnormal{しごと}}{\text{仕事}}}$ は、 ${\overset{\textnormal{わたし}}{\text{私}}}$ がします。 \hfill\break
\emph{Sono shigoto wa, watashi ga shimasu. \hfill\break
}As for that job, I\textquotesingle ll do it. \hfill\break
I\textquotesingle ll do that job. }

\par{32. キリンは ${\overset{\textnormal{くび}}{\text{首}}}$ が ${\overset{\textnormal{なが}}{\text{長}}}$ い。 \hfill\break
\emph{Kirin wa kubi ga nagai. \hfill\break
}As for giraffes, their necks are long. \hfill\break
Giraffes have long necks. }

\par{\textbf{Spelling Note }: Only rarely is \emph{kirin }spelled as 麒麟. }

\par{33. ( ${\overset{\textnormal{わたし}}{\text{私}}}$ は) ${\overset{\textnormal{あたま}}{\text{頭}}}$ が ${\overset{\textnormal{いた}}{\text{痛}}}$ いです。 \hfill\break
\emph{(Watashi wa) atama ga itai desu. \hfill\break
}(As for me), my head hurts. \hfill\break
I have a headache. }

\par{34. ( ${\overset{\textnormal{わたし}}{\text{私}}}$ は)お ${\overset{\textnormal{なか}}{\text{腹}}}$ が ${\overset{\textnormal{す}}{\text{空}}}$ きました。 \hfill\break
\emph{(Watashi wa) onaka ga sukimashita. \hfill\break
}I\textquotesingle m hungry. \hfill\break
Literally: (As for me), my stomach is empty. }

\par{35. ( ${\overset{\textnormal{わたし}}{\text{私}}}$ は) ${\overset{\textnormal{のど}}{\text{喉}}}$ が ${\overset{\textnormal{かわ}}{\text{渇}}}$ きました。 \hfill\break
\emph{(Watashi wa) nodo ga kawakimashita. \hfill\break
}I\textquotesingle m thirsty. \hfill\break
Literally: (As for me), my throat is parched. }

\par{\textbf{Grammar Note }: If distinguishing oneself from other people is necessary in expressing hunger or thirst, Ex. 34 and Ex. 35 are both examples of the pattern X \emph{wa }はY \emph{ga }が. }

\begin{center}
\textbf{iv. Questions }
\end{center}

\par{ As opposed to the questions made with \emph{ga }が, those made with \emph{wa }は have the interrogatives as part of the predicate. This is because the questions formed with \emph{wa }は imply that the question (topic) at hand is already known to the listener(s), and this knowledge is then topicalized to bring forth the question (comment) you\textquotesingle d like the discussion to be about. This pattern will be how most of the questions you ask are formed. }

\par{\textbf{Word Note }: As seen in Ex. 3, when \emph{nani }何 (what) is used as the predicate and followed by the copula, it undergoes a sound change and becomes \emph{nan }なん. }

\par{36. サムはいつ ${\overset{\textnormal{く}}{\text{来}}}$ る? \hfill\break
\emph{Samu wa itsu kuru? \hfill\break
}When is Sam coming? }

\par{37. ${\overset{\textnormal{きょう}}{\text{今日}}}$ は ${\overset{\textnormal{なんようび}}{\text{何曜日}}}$ ですか。 \hfill\break
\emph{Kyō wa nan\textquotesingle yōbi desu ka? \hfill\break
}What day is it today? }

\par{38.(あなたは) ${\overset{\textnormal{だれ}}{\text{誰}}}$ ですか。 \hfill\break
\emph{(Anata wa) dare desu ka? \hfill\break
}Who are you? }

\par{39. ${\overset{\textnormal{びょういん}}{\text{病院}}}$ はどこですか。 \hfill\break
\emph{Byōin wa doko desu ka? \hfill\break
}Where is the hospital? }

\par{40. ${\overset{\textnormal{しゅみ}}{\text{趣味}}}$ は何ですか。 \hfill\break
\emph{Shumi wa nan desu ka? \hfill\break
}What are your hobbies? }

\par{2. \textbf{The Contrast Marker }}

\par{On top of being a topic marker, \emph{wa } は is also the particle of contrast ( \emph{taihi }対比 ), which can be seen in its usage of marking the topic. There is a line of thought that the contrast meaning of \emph{wa } は is actually the primary meaning of \emph{wa } は . Within a given sentence, several \emph{wa } は may appear. Each one will have a different level of contrast implied. When a \emph{wa } は phrase's degree of contrast is really weak, it can be viewed as the topic. }

\par{41. ${\overset{\textnormal{わたし}}{\text{私}}}$ は ${\overset{\textnormal{きのう}}{\text{昨日}}}$ は ${\overset{\textnormal{ちゅうしょく}}{\text{昼食}}}$ は ${\overset{\textnormal{と}}{\text{取}}}$ らなかったんです。 \hfill\break
\emph{Watashi wa kinō wa chūshoku wa toranakatta n desu. \hfill\break
}Yesterday, I didn't have \emph{lunch }. }

\par{Although the presence of \emph{watashi wa } 私は could imply a contrast with other people, the sentence is bringing oneself to the forefront of conversation. With this being this case, it is viewed as the topic. Both the words for "yesterday" and "lunch" are marked with \emph{wa } は because they contrast with other scenarios. For instance, the speaker may have eaten lunch today, and he may have eaten breakfast and\slash or dinner that day. }

\par{42. ${\overset{\textnormal{きょう}}{\text{今日}}}$ は ${\overset{\textnormal{い}}{\text{行}}}$ きます。( \textrightarrow   ${\overset{\textnormal{あす}}{\text{明日}}}$ は行きません) \hfill\break
\emph{Kyō wa ikimasu }. (\textrightarrow  \emph{Asu wa ikimasen }) \hfill\break
I'm going \emph{today }. (\textrightarrow  I'm not going \emph{tomorrow }) }

\par{43. ${\overset{\textnormal{だんな}}{\text{旦那}}}$ さんは ${\overset{\textnormal{しゃんはい}}{\text{上海}}}$ へ ${\overset{\textnormal{い}}{\text{行}}}$ きます。( \textrightarrow   ${\overset{\textnormal{おく}}{\text{奥}}}$ さんは ${\overset{\textnormal{ぺきん}}{\text{北京}}}$ へ ${\overset{\textnormal{い}}{\text{行}}}$ きます) \hfill\break
\emph{Dan'na-san wa Shanhai e ikimasu }. (\textrightarrow  \emph{Oku-san wa Pekin e ikimasu }) \hfill\break
\emph{His\slash her husband }is going to Shanghai. (\textrightarrow  \emph{His\slash her wife }is going to Beijing) }

\par{44. ${\overset{\textnormal{おおさか}}{\text{大阪}}}$ へは ${\overset{\textnormal{い}}{\text{行}}}$ きます。( \textrightarrow   ${\overset{\textnormal{きょうと}}{\text{京都}}}$ へは ${\overset{\textnormal{い}}{\text{行}}}$ きません) \hfill\break
\emph{Ōsaka e wa ikimasu }(\textrightarrow  \emph{Ky }\emph{ō }\emph{to e wa ikimasen }) \hfill\break
I'm going to Osaka. (\textrightarrow  I'm not going to \emph{Kyoto }) }

\par{45. ${\overset{\textnormal{ほんとう}}{\text{本当}}}$ は ${\overset{\textnormal{うれ}}{\text{嬉}}}$ しいです。 \hfill\break
\emph{Hontō wa ureshii desu. }\hfill\break
I'm actually happy. }

\par{\textbf{Grammar Note }: \emph{Hontō }本当 is used here as a noun meaning "reality\slash actuality." The speaker may not appear happy, but internally he\slash she is happy. }

\par{46. 「 ${\overset{\textnormal{にほんりょうり}}{\text{日本料理}}}$ は好きですか」「タイ ${\overset{\textnormal{りょうり}}{\text{料理}}}$ は ${\overset{\textnormal{す}}{\text{好}}}$ きです」  \hfill\break
\emph{"Nihon ryōri wa o-suki desu ka?" "Tai ryōri wa suki desu." }\hfill\break
"Do you like Japanese cuisine?" I like Thai food(, but as far as other cuisine\dothyp{}\dothyp{}\dothyp{}) }

\par{\textbf{Grammar Note }: The reply provides an indirect means of saying that one doesn't like Japanese cuisine. Although this is inferred by the reply, it's politer to reply as such than simply saying no. }

\par{47. ${\overset{\textnormal{いぬ}}{\text{犬}}}$ は ${\overset{\textnormal{す}}{\text{好}}}$ きですが、 ${\overset{\textnormal{ねこ}}{\text{猫}}}$ はどうも・・・ \hfill\break
\emph{Inu wa suki desu ga, neko wa domo\dothyp{}\dothyp{}\dothyp{} }\hfill\break
I like dogs, but cats \dothyp{}\dothyp{}\dothyp{} }

\par{\textbf{Grammar Note }: The \emph{ga }が seen after \emph{desu }です is the conjunctive particle \emph{ga }が, which is separate from its use as a subject marker. For now, simply know that it is the "but" in this example and the ones that follow. }

\par{48. コーヒーは ${\overset{\textnormal{の}}{\text{飲}}}$ まないが、ビールは ${\overset{\textnormal{の}}{\text{飲}}}$ むよ。 \hfill\break
\emph{Kōhii wa nomanai ga, biiru wa nomu yo. }\hfill\break
I don't drink coffee, but I drink beer. }

\par{49. ${\overset{\textnormal{えんぴつ}}{\text{鉛筆}}}$ はありませんが、ペンはありますよ。 \hfill\break
\emph{Empitsu wa arimasen ga, pen wa arimasu yo. }\hfill\break
There aren't pencils, but there are pens. \hfill\break
I don't have pencils, but I have pens. }

\par{50. あれはオオカミではない、 ${\overset{\textnormal{きつね}}{\text{狐}}}$ だよ。 \hfill\break
\emph{Are wa ōkami de wa nai, kitsune da yo. }\hfill\break
That isn't \emph{a wolf }; it's a fox. }

\par{\textbf{Grammar Note }: This example demonstrates how the \emph{wa }は in \emph{de wa nai }ではない is the contrasting \emph{wa }は. The grammar behind this actually goes beyond its use in the negative forms of the copula. However, due to the complexity of this grammar point, it will be discussed in a later lesson. }

\par{3. Another usage of the particle \emph{wa }は is to express a bare minimum ( \emph{saiteigen } 最低限)--"at least." This is primarily used with number expressions, which will be studied later on. However, this usage is not limited to such expressions, as is demonstrated by Ex. 54. }

\par{51. ${\overset{\textnormal{すく}}{\text{少}}}$ なくとも ${\overset{\textnormal{に}}{\text{2}}}$ ${\overset{\textnormal{じかん}}{\text{時間}}}$ はかかります。  \hfill\break
\emph{Sukunakutomo nijikan wa kakarimasu. }\hfill\break
It will take at least two hours. }

\par{52. ${\overset{\textnormal{じゅう}}{\text{10}}}$ ${\overset{\textnormal{にん}}{\text{人}}}$ は ${\overset{\textnormal{き}}{\text{来}}}$ ます。 \hfill\break
\emph{Jūnin wa kimasu. }\hfill\break
At least ten people will come. }

\par{53. ${\overset{\textnormal{じゅう}}{\text{10}}}$ ${\overset{\textnormal{まんえん}}{\text{万円}}}$ は ${\overset{\textnormal{ひつよう}}{\text{必要}}}$ です。 \hfill\break
\emph{Jūman'en wa hitsuy }\emph{ō }\emph{desu. }\hfill\break
It will need at least 100,00 yen. }

\par{54. ${\overset{\textnormal{ぎゅうにゅう}}{\text{牛乳}}}$ ぐらいは ${\overset{\textnormal{か}}{\text{買}}}$ ってください。 \hfill\break
\emph{Gyūnyū gurai wa katte kudasai. }\hfill\break
At least buy milk, please. }

\par{\textbf{Grammar Note }: The particle \emph{kurai\slash gurai }くらい・ぐらい is frequently used with this function of the particle \emph{wa }は to express "at least." It can actually be inserted similarly to the other example sentences in this section. Its addition creates a greater emphatic tone. }

\par{55. ${\overset{\textnormal{もうちょう}}{\text{盲腸}}}$ の ${\overset{\textnormal{しゅじゅつ}}{\text{手術}}}$ でも ${\overset{\textnormal{せん}}{\text{1000}}}$ ドルはかかります。 \hfill\break
\emph{Mōchō no shujutsu demo sen-doru wa kakarimasu. }\hfill\break
Even appendix surgery will cost at least a thousand dollars. }

\par{\textbf{Grammar Note }: The particle \emph{demo }でも means "even" and will be discussed in Lesson 67. }
    