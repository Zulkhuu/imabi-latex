    
\chapter{The Particle Wo を}

\begin{center}
\begin{Large}
第15課: The Particle Wo を: The Direct Object\slash Transition Marker 
\end{Large}
\end{center}
 
\par{ As we saw in Lesson 11 with the particle \emph{ga }が, \textbf{case particles }are what \emph{state grammatical function of a noun phrase in relation to the predicate }. Interpreting case particles correctly, however, requires a considerable amount of knowledge as to what nouns and predicates are used with a given particle to then accurately understand them. In this lesson, we will learn about the case particle \emph{wo }を and its various usages. }

\par{ The case that the particle \emph{wo }を marks is called the accusative case . The \textbf{accusative case } \emph{marks the direct object of a transitive verb }. This definition present us with more terminology to understand. }

\par{\textbf{Pronunciation Note }: を is usually pronounced as “o” but is still pronounced as “wo” depending on the dialect and\slash or context. For instance, it is frequently pronounced as “wo” in clearly enunciated speech and music. }

\begin{center}
\textbf{What is a Direct Object? }
\end{center}

\par{ First, a \textbf{direct object } \emph{is a noun phrase denoting a person or thing that is the recipient of the action of a transitive verb }. Putting aside what a transitive verb is, below are examples of direct objects in English. }

\par{i. I kissed \textbf{a boy }. \hfill\break
ii. I ate \textbf{a hamburger }. \hfill\break
iii. I drank \textbf{milk }. \hfill\break
iv. She bought \textbf{a television }. \hfill\break
v. He sold \textbf{his stereo }. }

\begin{center}
 \textbf{Intransitive \& Transitive Verbs }
\end{center}

\par{ The next question is what a transitive verb is, but in figuring this out, it\textquotesingle s best to understand both what transitive and intransitive verbs are. }

\par{\textbf{Intransitive Verb }: \emph{A verb that doesn\textquotesingle t need an object to complete its meaning. }}

\par{\textbf{Transitive Verb }: \emph{A verb that requires one or more objects to complete its meaning. }}

\par{ Here are some examples of both intransitive and transitive verbs in English. }

\par{vi. I saw a monkey. (transitive) \hfill\break
vii. He stood still. (intransitive) \hfill\break
viii. He found a raccoon. (transitive) \hfill\break
ix. The alligator swam away. (intransitive) \hfill\break
x. She cooked the meal. (transitive) }

\begin{center}
\textbf{\emph{Jid }\emph{ōshi }自動詞 vs \emph{Tad }\emph{ōshi }他動詞 }
\end{center}

\par{ In Japanese, intransitive verbs are called \emph{jid }\emph{ōshi }自動詞. Such verbs only require that a subject be used in concert with the predicate, and of course, the subject is marked by \emph{ga }が. Transitive verbs are called \emph{tad }\emph{ōshi }他動詞. Such verbs require that both a subject and direct object be used in concert with the predicate. Although the subject is still marked by \emph{ga }が, the direct object is marked by \emph{wo }を. The fact that \emph{tad }\emph{ōshi }他動詞 need one more element (argument) to the sentence to be grammatical, the presence of a direct object or lack thereof, is frequently used to distinguish \emph{jid }\emph{ōshi }自動詞 and \emph{tad }\emph{ōshi }他動詞. }

\par{ In fact, it is often the case that the subject of an intransitive predicate can become the direct object of a transitive predicate. This is also the case in English, which can be seen in the following examples. }

\par{1. ${\overset{\textnormal{でんき}}{\text{電気}}}$ がついた。(Intransitive) \hfill\break
 \emph{Denki ga tsuita. \hfill\break
 }The lights turned on. }

\par{2. ジェイムズが ${\overset{\textnormal{でんき}}{\text{電気}}}$ をつけた。(Transitive) \hfill\break
 \emph{Jeimuzu ga denki wo tsuketa. \hfill\break
 }James turned on the light. }

\par{\textbf{Sentence Note }: The subject of Ex. 1 is the direct object of Ex. 2. }

\par{3. ビールが ${\overset{\textnormal{ひ}}{\text{冷}}}$ えた。(Intransitive) \hfill\break
 \emph{Biiru ga hieta. \hfill\break
 }The beer chilled. }

\par{4. ${\overset{\textnormal{ゆうた}}{\text{雄太}}}$ がビールを ${\overset{\textnormal{ひ}}{\text{冷}}}$ やした。(Transitive) \hfill\break
 \emph{Y }\emph{ūta ga biiru wo hiyashita. \hfill\break
 }Yuta chilled the beer. }

\par{\textbf{Sentence Note }: The subject of Ex. 3 is the direct object of Ex. 4. }

\begin{center}
\textbf{Intransitive-Transitive Verb Pairs }
\end{center}

\par{ In English, the verbs “to turn on” and “to chill” can be both used as either intransitive or transitive verbs without any change to their conjugations; however, the same cannot be said for Japanese. "To turn on" is \emph{tsuku }つく and \emph{tsukeru }つける for the intransitive and transitive sense respectively, and "to chill" is \emph{hieru }冷える and \emph{hiyasu }冷やす in the intransitive and transitive sense respectively. }

\par{ In Japanese, many verbs have intransitive-transitive verb pairs, which is another means of figuring out when and when not to use the particle \emph{wo }を. Some of the most common examples of these so-called intransitive-transitive verb pairs in Japanese are listed below. Note that for some of them, they also create verb pairs in English. }

\begin{ltabulary}{|P|P|P|}
\hline 

Definition & Intransitive & Transitive \\ \cline{1-3}

To break &  \emph{Kowareru }壊れる &  \emph{Kowasu }壊す \\ \cline{1-3}

To change &  \emph{Kawaru }変わる &  \emph{Kaeru }変える \\ \cline{1-3}

To start &  \emph{Hajimaru }始まる &  \emph{Hajimeru }始める \\ \cline{1-3}

To stop &  \emph{Tomaru }止まる &  \emph{Tomeru }止める \\ \cline{1-3}

To open &  \emph{Aku }開く &  \emph{Akeru }開ける \\ \cline{1-3}

\end{ltabulary}

\par{\hfill\break
5. ${\overset{\textnormal{くるま}}{\text{車}}}$ が ${\overset{\textnormal{と}}{\text{止}}}$ まった。 \hfill\break
 \emph{Kuruma ga tomatta. \hfill\break
 }The car stopped. }

\par{6. ${\overset{\textnormal{けいさつかん}}{\text{警察官}}}$ が ${\overset{\textnormal{ぼく}}{\text{僕}}}$ の ${\overset{\textnormal{くるま}}{\text{車}}}$ を ${\overset{\textnormal{と}}{\text{止}}}$ めた。 \hfill\break
 \emph{Keisatsukan ga boku no kuruma wo tometa. \hfill\break
 }A police officer stopped me (my car). }

\par{7. ${\overset{\textnormal{まど}}{\text{窓}}}$ が ${\overset{\textnormal{ひら}}{\text{開}}}$ いた。 \hfill\break
 \emph{Mado ga aita. \hfill\break
 }The window opened. }

\par{8. ${\overset{\textnormal{ゆう}}{\text{悠}}}$ は ${\overset{\textnormal{まど}}{\text{窓}}}$ を ${\overset{\textnormal{あ}}{\text{開}}}$ けた。 \hfill\break
 \emph{Y }\emph{ū wa mado wo aketa. \hfill\break
 }Yu opened a\slash the window. }

\begin{center}
\textbf{The Usages of The Particle \emph{Wo }を }
\end{center}

\par{ So far, we have seen how the particle \emph{wo }を primarily marks the direct object, but that isn\textquotesingle t all it can do. \emph{Wo }を can be explained as having two broad purposes with various applications. Depending on the usage, you may see it with transitive verbs, intransitive verbs, or both. }
 
\begin{itemize}

\item 1. Direct Object Marker \hfill\break

\item 2. Transition Marker \hfill\break
i. Through\slash Along \hfill\break
ii. Toward\slash Around \hfill\break
iii. Object of Departure: “From” \hfill\break
iv. Flux in Degree \hfill\break
v. Time 
\end{itemize}
 
\par{  For the remainder of this lesson, we would closely at these usages. Note that at times, there will be some grammar points used that have not been fully introduced. In those instances, you are only expected to focus on learning how to use the particle \emph{wo }を. }
      
\section{Vocabulary List (Under Construction)}
       
\section{The Direct Object Marker Wo を}
 
\par{ The basic usage of \emph{wo }を is to mark the direct object of a transitive verb. Not all transitive verbs in English are transitive verbs in Japanese, and so you can\textquotesingle t always expect \emph{wo }を to be the correct particle to choose, but for verbs that are unequivocally transitive }

\par{9. ${\overset{\textnormal{つくえ}}{\text{机}}}$ を ${\overset{\textnormal{う}}{\text{売}}}$ る。 \hfill\break
 \emph{Tsukue wo uru. \hfill\break
 }To sell a desk. }

\par{10. ${\overset{\textnormal{ゆか}}{\text{床}}}$ を ${\overset{\textnormal{は}}{\text{掃}}}$ く。 \hfill\break
 \emph{Yuka wo haku. \hfill\break
 }To sweep the floor. }

\par{11. ${\overset{\textnormal{ざっし}}{\text{雑誌}}}$ を ${\overset{\textnormal{よ}}{\text{読}}}$ む。 \hfill\break
 \emph{Zasshi wo yomu. \hfill\break
 }To read a magazine. }

\par{12. お ${\overset{\textnormal{この}}{\text{好}}}$ み ${\overset{\textnormal{や}}{\text{焼}}}$ きを ${\overset{\textnormal{た}}{\text{食}}}$ べる。 \hfill\break
 \emph{Okonomiyaki wo taberu. \hfill\break
 }To eat okonomiyaki. }

\par{\textbf{Word Note }: \emph{Okonomiyaki }お好み焼き is known as the Japanese pancake. In its predominant form, the batter is made of \emph{komugiko }小麦粉 (flour), grated \emph{nagaimo }長芋 (Chinese yam), water\slash  \emph{dashi }出汁 (soup stalk), eggs, and \emph{sengiri-kyabetsu }千切りキャベツ (shredded cabbage). Other ingredients such as \emph{aonegi }青ネギ (green onion), \emph{niku }肉 (meat), \emph{tako }タコ (octopus), \emph{ika }イカ (squid), \emph{ebi }エビ (shrimp), \emph{yasai }野菜 (vegetables), \emph{kon\textquotesingle nyaku }コンニャク (konjac), \emph{mochi }餅 (sticky rice cake), and \emph{chiizu }チーズ (cheese) are typically added. }

\par{13. テレビを ${\overset{\textnormal{み}}{\text{見}}}$ る。 \hfill\break
 \emph{Terebi wo miru. \hfill\break
 }To watch TV. }

\par{14. オレンジジュースを ${\overset{\textnormal{の}}{\text{飲}}}$ む。 \hfill\break
 \emph{Orenjij }\emph{ūsu wo nomu. \hfill\break
 }To drink orange juice. }

\par{15. ${\overset{\textnormal{しょうり}}{\text{勝利}}}$ を ${\overset{\textnormal{めざ}}{\text{目指}}}$ す。 \hfill\break
 \emph{Sh }\emph{ōri wo mezasu. \hfill\break
 }To aim for victory. }

\par{16. ${\overset{\textnormal{かんじ}}{\text{漢字}}}$ を ${\overset{\textnormal{べんきょう}}{\text{勉強}}}$ する。 \hfill\break
 \emph{Kanji wo benky }\emph{ō suru. \hfill\break
 }To study Kanji. }

\par{17. ${\overset{\textnormal{いし}}{\text{石}}}$ を ${\overset{\textnormal{な}}{\text{投}}}$ げる。 \hfill\break
 \emph{Ishi wo nageru. \hfill\break
 }To throw (a) rock(s). }

\par{18. お ${\overset{\textnormal{ゆ}}{\text{湯}}}$ を ${\overset{\textnormal{わ}}{\text{沸}}}$ かす。 \hfill\break
 \emph{Oyu wo wakasu. \hfill\break
 }To boil water. }

\par{ \textbf{Word Note }: \emph{Oyu }お湯 (hot water) must be used instead of \emph{mizu }水 (water) with the verb \emph{wakasu }沸かす (to boil). }
      
\section{The Transition Marker Wo を}
 
\par{ Transitive verbs in Japanese usually involve actions in which the agent\textquotesingle s volition is a fundamental aspect. In other words, the agent is in control of what happens (to a direct object). It is this quality that the particle \emph{wo }を brings attention to. Because intransitive verbs can also represent actions in which the agent is fully in control of the situation, the particle \emph{wo }を can be used with them so long as the agent is acting upon something. In the following usages, the concept of ‘something\textquotesingle  is broadened to indicate transition. As the \textbf{transition marker }, whether it is used with intransitive or transitive verbs, the particle \emph{wo }を \emph{indicates transition in time, space, or degree }. }

\par{・ \textbf{Motion through all or part of a dimension }: Transitioning through a dimension of time is one of the most important applications of the transition marker \emph{wo }を. To visualize how this works, think of a circle and arrow going through it. The action at hand happens anywhere throughout the space \emph{wo }を marks. Transition-wise, it may equate to various English phrases such as “along,” “through,” and “across.” }

\par{\textbf{Transitivity Note }: The verbs this usage is used with are intransitive verbs that all involve motion. }

\par{19. ${\overset{\textnormal{ふじさん}}{\text{富士山}}}$ を ${\overset{\textnormal{のぼ}}{\text{登}}}$ りました。 \hfill\break
 \emph{Fujisan wo noborimashita. \hfill\break
 }I climbed Mt. Fuji. }

\par{20. ${\overset{\textnormal{こうえん}}{\text{公園}}}$ を ${\overset{\textnormal{はし}}{\text{走}}}$ りました。 \hfill\break
 \emph{K }\emph{ōen wo hashirimashita. \hfill\break
 }I ran through the park. }

\par{21. ${\overset{\textnormal{にほんばし}}{\text{日本橋}}}$ を ${\overset{\textnormal{わた}}{\text{渡}}}$ りました。 \hfill\break
 \emph{Nihombashi wo watarimashita. \hfill\break
 }I crossed the Nihon Bridge. }

\par{22. ミシシッピ ${\overset{\textnormal{がわ}}{\text{川}}}$ を ${\overset{\textnormal{およ}}{\text{泳}}}$ ぎました。 \hfill\break
 \emph{Mishishippi-gawa wo oyogimashita. \hfill\break
 }I swam across the Mississippi River. }

\par{23. ${\overset{\textnormal{そら}}{\text{空}}}$ を ${\overset{\textnormal{と}}{\text{飛}}}$ びました。 \hfill\break
 \emph{Sora wo tobimashita. \hfill\break
 }I flew across the sky. }

\par{24. ${\overset{\textnormal{かれ}}{\text{彼}}}$ は ${\overset{\textnormal{くだ}}{\text{下}}}$ り ${\overset{\textnormal{ざか}}{\text{坂}}}$ を ${\overset{\textnormal{はし}}{\text{走}}}$ った。 \hfill\break
 \emph{Kare wa kudarizaka wo hashitta. \hfill\break
 }He ran downhill. }

\par{25. アユは ${\overset{\textnormal{かわ}}{\text{川}}}$ を ${\overset{\textnormal{くだ}}{\text{下}}}$ った。 \hfill\break
 \emph{Ayu wa kawa wo kudatta. \hfill\break
 }The sweetfish descended the river. }

\begin{center}
\emph{Ni iku }に行く VS \emph{Wo iku }を行く 
\end{center}

\par{ As the examples above have demonstrated, the transition marker \emph{wo }を is used to indicate what dimension movement is taking place. However, the particle \emph{wo }を says nothing about destination or what may happen internally within a certain dimension. Those situations are handled by other particles. The verb \emph{iku }行く means “to go,” and is frequently described as taking the particle \emph{ni }に, which indicates destination. }

\par{26. スーパーに ${\overset{\textnormal{い}}{\text{行}}}$ きました。 \hfill\break
 \emph{S }\emph{ūp }\emph{ā ni ikimashita. \hfill\break
 }I went to the supermarket. }

\par{ However, it too can be used with the particle \emph{wo }を. In the case of \emph{iku }行く, the sentence becomes figurative as it goes beyond the typical application of “to go (somewhere).” }

\par{27. ${\overset{\textnormal{ぼく}}{\text{僕}}}$ は ${\overset{\textnormal{ぼく}}{\text{僕}}}$ の ${\overset{\textnormal{みち}}{\text{道}}}$ を ${\overset{\textnormal{い}}{\text{行}}}$ く。 \hfill\break
 \emph{Boku wa boku no michi wo iku. \hfill\break
 }I walk along my path. }

\par{\textbf{Reading Note }: For this usage, 行く may alternatively be read as “ \emph{yuku }.” }

\par{28. ラクサウルを ${\overset{\textnormal{とお}}{\text{通}}}$ ってネパールとインドの ${\overset{\textnormal{こっきょう}}{\text{国境}}}$ を ${\overset{\textnormal{こ}}{\text{越}}}$ えました。 \hfill\break
 \emph{Rakusauru wo t }\emph{ōtte Nep }\emph{ā }\emph{ru to Indo no kokky }\emph{ō wo koemashita. \hfill\break
 }Passing through Raxaul, I crossed the border between Nepal and India. }

\par{\textbf{Grammar Note }: The verb \emph{t }\emph{ōru }通る is used together with the conjunctive particle \emph{te }て to make a dependent clause. This means \emph{t }\emph{ōtte }通って is used to mean “passing through.” We\textquotesingle ll learn more about this conjugation in Lesson 26. As for this sentence, it enables \emph{wo }を to be used twice to mark transition. }

\par{・ \textbf{Direction of an action }: Another use of the transition marker \emph{wo }を is to indicate the direction of an action that, although being an outward action, isn\textquotesingle t necessarily going through something. The action could be done towards or around some entity, with entity being broadened to include direction. }

\par{\textbf{Transitivity Note }: This usage may be used with both intransitive and transitive verbs. }

\par{29. ${\overset{\textnormal{かど}}{\text{角}}}$ を ${\overset{\textnormal{ま}}{\text{曲}}}$ がった。 \hfill\break
 \emph{Kado wo magatta. \hfill\break
 }I turned the corner. }

\par{30. ${\overset{\textnormal{まわ}}{\text{周}}}$ りを ${\overset{\textnormal{まわ}}{\text{回}}}$ った。 \hfill\break
 \emph{Mawari wo mawatta. \hfill\break
 }I circled around. }

\begin{center}
The Noun \emph{H }\emph{ō }方 
\end{center}

\par{ The noun \emph{h }\emph{ō }方, frequently spelled simply as ほう, is used to help \emph{wo }を create the meaning of “toward.” The insertion of \emph{h }\emph{ō }ほう is imperative whenever the noun it precedes is not a literal direction-word (north, south, east, and west). However, it is still frequently inserted regardless. }

\begin{ltabulary}{|P|P|P|P|}
\hline 
 
  North 
 &    \emph{Kita }北 
 &   South 
 &    \emph{Minami }南 
 \\ \cline{1-4} 
 
  East 
 &    \emph{Higashi }東 
 &   West 
 &   \emph{Nishi }西 
 \\ \cline{1-4} 
 
  Up 
 &    \emph{Ue }上 
 &   Down 
 &    \emph{Shita }下 
 \\ \cline{1-4} 
 
  Left 
 &    \emph{Hidari }左 
 &   Right 
 &   \emph{Migi }右 
 \\ \cline{1-4} 
 
  Forward 
 &    \emph{Mae }前 
 &   Back(ward) 
 &    \emph{Ushiro }後ろ 
\\ \cline{1-4}

\end{ltabulary}

\par{\hfill\break
31. ${\overset{\textnormal{した}}{\text{下}}}$ (のほう)を ${\overset{\textnormal{み}}{\text{見}}}$ ました。 \hfill\break
 \emph{Shita (no h }\emph{ō) wo mimashita. \hfill\break
 }I looked down(ward). }

\par{32. ${\overset{\textnormal{ぎふ}}{\text{岐阜}}}$ のほうを ${\overset{\textnormal{なが}}{\text{眺}}}$ めました。 \hfill\break
 \emph{Gifu no h }\emph{ō wo nagamemashita. \hfill\break
 }I gazed toward Gifu. }

\par{33. ${\overset{\textnormal{しょうへい}}{\text{翔平}}}$ は ${\overset{\textnormal{せんせい}}{\text{先生}}}$ のほうを ${\overset{\textnormal{む}}{\text{向}}}$ いた。 \hfill\break
 \emph{Sh }\emph{ōhei wa sensei no h }\emph{ō wo muita. \hfill\break
 }Shohei faced the teacher. }

\begin{center}
\textbf{Two Transition Marker \emph{Wo }を in the Same Clause }
\end{center}

\par{ The transition marker \emph{wo }を, as we\textquotesingle re discovering, has more than one application. Although these individual applications are all interrelated to each other, they are different enough to the point that more than one can manifest in a sentence. Because sentences can be composed of several clauses (sections) as was seen in Ex. 28, this isn\textquotesingle t hard to fathom. }

\par{ We have also seen how the same particle can easily be used more than once in a sentence even in the same clause, as is frequently the case with the particles \emph{ga }が and \emph{wa }は. In those discussions, the word sentence was used in place of clause, but clause is simply one stage below a sentence in terms of grammar. Clauses come in two kinds: independent and dependent. An independent clause is something that can stand alone as a proper sentence, whereas a dependent clause cannot stand alone as sentence. A clause, nonetheless, will always have the same hallmarks of a sentence in regards to composition. }

\par{ In regards to \emph{wo }を, it is possible to have two usages of the transition marker function manifest in a single clause. In Ex. 34, the first \emph{wo }を marks the direction of the action of “to walk.” The subject walked “in” the rain. The second \emph{wo }を marks the dimension of transit, which is “through” the park. There is a principle of Japanese grammar, however, that aims to avoid such doubling of case particles. Because of this, the first such \emph{wo }を is usually left omitted despite grammatically still being there. }

\par{34. ${\overset{\textnormal{あめ}}{\text{雨}}}$ の ${\overset{\textnormal{なか}}{\text{中}}}$ (を)、 ${\overset{\textnormal{こうえん}}{\text{公園}}}$ を ${\overset{\textnormal{ある}}{\text{歩}}}$ いた。 \hfill\break
 \emph{Ame no naka (wo), k }\emph{ōen wo aruita. \hfill\break
 }I walked through the park in the rain. }

\par{・ \textbf{Origin of Departure }: The transition marker \emph{wo }を may also mark what the agent is departing from. In this sense, it is interchangeable with another case particle \emph{kara }から. This is only true, however, for when the point of departure is a physical, concrete location. }

\par{\textbf{Transitivity Note }: This usage may be used with both intransitive and transitive verbs. }

\par{35. ${\overset{\textnormal{でんしゃ}}{\text{電車}}}$ \{を・から\} ${\overset{\textnormal{お}}{\text{降}}}$ りました。 \hfill\break
 \emph{Densha [wo\slash kara] orimashita. \hfill\break
 }I got off the train. \hfill\break
 \hfill\break
\textbf{Nuance Note }: The use of \emph{wo }を simply implies getting off a train. The use of \emph{kara }から indicates that the speaker is purposely heading elsewhere upon getting off the train. }

\par{36. ${\overset{\textnormal{いえ}}{\text{家}}}$ \{を・から\} ${\overset{\textnormal{で}}{\text{出}}}$ ました。 \hfill\break
 \emph{Ie [wo\slash kara] demashita. \hfill\break
 }I left the home\slash I went out of my home. }

\par{\textbf{Nuance Note }: The use of \emph{wo }を indicates the former interpretation, which implies leaving the one to live by oneself, whereas the use of \emph{kara }から indicates the latter interpretation, which simply implies going from the home for a bit. }

\par{37. 【 ${\overset{\textnormal{つま}}{\text{妻}}}$ ・ ${\overset{\textnormal{しゅじん}}{\text{主人}}}$ 】は ${\overset{\textnormal{しち}}{\text{7}}}$ ${\overset{\textnormal{じ}}{\text{時}}}$ に ${\overset{\textnormal{うち}}{\text{家}}}$ を ${\overset{\textnormal{で}}{\text{出}}}$ ました。 \hfill\break
 \emph{[Tsuma\slash shujin] wa shichiji ni uchi wo demashita. \hfill\break
 }My wife\slash husband left home at seven o\textquotesingle  clock. }

\par{\textbf{Sentence Note }: The use of \emph{uchi }instead of \emph{ie }, both words being spelled as 家, helps make it clear that one is not uprooting oneself from one\textquotesingle s home. If the particle \emph{kara }から were used instead of \emph{wo }を, it would imply there\textquotesingle s a certain destination in mind with the home being the starting point. To simply express one leaving home, use \emph{uchi wo deru }家を出る. }

\par{\textbf{Grammar Note }: The particle \emph{ni }に indicates “at” when used after time phrases. }

\par{38. ${\overset{\textnormal{そと}}{\text{外}}}$ に ${\overset{\textnormal{で}}{\text{出}}}$ ました。 \hfill\break
 \emph{Soto ni demashita. }\hfill\break
I went outside. \hfill\break
 \textbf{\hfill\break
Grammar Note }: Like the issue with the verb \emph{iku }行くas to which particle should be used, if you are intending to use the verb \emph{deru }出る to indicate where one has exited “to,” then you must use the particle \emph{ni }に instead of \emph{wo }を as it is \emph{ni }に that indicates place of destination. }

\par{39. ${\overset{\textnormal{きょねんだいがく}}{\text{去年大学}}}$ \{を 〇・から X\} ${\overset{\textnormal{そつぎょう}}{\text{卒業}}}$ しました。 \hfill\break
 \emph{Kyonen daigaku wo sotsugy }\emph{ō shimashita. \hfill\break
 }I graduated (from) college last year. }

\par{40. ${\overset{\textnormal{かいしゃ}}{\text{会社}}}$ \{を 〇・から X\} ${\overset{\textnormal{や}}{\text{辞}}}$ めました。 \hfill\break
 \emph{Kaisha wo yamemashita \hfill\break
 }I quit the company. }

\par{41. ${\overset{\textnormal{ふね}}{\text{船}}}$ が ${\overset{\textnormal{みなと}}{\text{港}}}$ \{を・から\} ${\overset{\textnormal{しゅっぱつ}}{\text{出発}}}$ した。 \hfill\break
 \emph{Fune ga minato [wo\slash kara] shuppatsu shita. \hfill\break
 }The boat departed from the harbor. }

\par{\textbf{Grammar Note }: The use of \emph{wo }を makes it sound that the harbor is just the origin of the act of departure. The use of \emph{kara }から makes it sound that harbor is specifically the starting point of a venture. This distinction between the two particles is profound enough to have both particles show up in the same sentence for their own respective purposes. }

\par{42. シアトルタコマ ${\overset{\textnormal{こくさいくうこう}}{\text{国際空港}}}$ からアメリカを ${\overset{\textnormal{しゅっぱつ}}{\text{出発}}}$ した。 \hfill\break
 \emph{Shiatoru-Takoma Kokusai K }\emph{ūk }\emph{ō kara Amerika wo shuppatsu shita. \hfill\break
 }I departed America from Seattle-Tacoma International Airport. }

\par{・ \textbf{Flux in Degree }: There are a handful of verbs in Japanese that regard fluctuation in degree. These verbs indicate how a certain value goes beyond or below a certain standard. }

\par{\textbf{Transitivity Note }: This usage is used with intransitive verbs. }

\par{43. ${\overset{\textnormal{きょう}}{\text{今日}}}$ は ${\overset{\textnormal{さんじゅうご}}{\text{35}}}$ ${\overset{\textnormal{ど}}{\text{度}}}$ を ${\overset{\textnormal{こ}}{\text{超}}}$ える ${\overset{\textnormal{もうしょび}}{\text{猛暑日}}}$ でした。 \hfill\break
 \emph{Ky }\emph{ō wa sanj }\emph{ūgodo wo koeru m }\emph{ōshobi deshita. \hfill\break
 }Today was an extremely hot day exceeding 35 degrees Celsius. }

\par{44. その ${\overset{\textnormal{きんがく}}{\text{金額}}}$ が ${\overset{\textnormal{いち}}{\text{1}}}$ ${\overset{\textnormal{おく}}{\text{億}}}$ ドルを ${\overset{\textnormal{うわまわ}}{\text{上回}}}$ りました。 \hfill\break
 \emph{Sono kingaku ga ichioku-doru wo uwamarimashita. \hfill\break
 }The amount exceeded 100 million dollars. }

\par{45. ${\overset{\textnormal{きじゅん}}{\text{基準}}}$ を ${\overset{\textnormal{したまわ}}{\text{下回}}}$ る。 \hfill\break
\emph{Kijun wo shitamawaru. \hfill\break
}To fall below a standard. }

\par{・ \textbf{Transition of Time }: The use of the transition marker of \emph{wo }を to show transition in a temporal sense is not as productive as the grammatical situations above. As is the case with expressing transit through space, \emph{wo }を can mark transiting through a certain time period. }

\par{\textbf{Transitivity Note }: This usage may be used with both intransitive and transitive verbs. }

\par{46. カナダで ${\overset{\textnormal{なつやす}}{\text{夏休}}}$ みを ${\overset{\textnormal{す}}{\text{過}}}$ ごました。 \hfill\break
 \emph{Kanada de natsuyasumi wo sugoshimashita. \hfill\break
 }I spent my summer break in Canada. }

\par{\textbf{Sentence Note }: \emph{Sugosu }過ごす is a transitive verb. It is indicative of how this use of \emph{wo }を, when used with transitive verbs, is used with nouns that are not literally time phrases but inherently imply a period of time. }

\par{\textbf{Grammar Note }: The particle \emph{de }で is the particle used to indicate where an action is done “at.” }

\par{47. ${\overset{\textnormal{しゅんこう}}{\text{竣工}}}$ から ${\overset{\textnormal{じゅう}}{\text{10}}}$ ${\overset{\textnormal{ねん}}{\text{年}}}$ を ${\overset{\textnormal{へ}}{\text{経}}}$ た ${\overset{\textnormal{たてもの}}{\text{建物}}}$ の ${\overset{\textnormal{ちょうさ}}{\text{調査}}}$ を ${\overset{\textnormal{おこな}}{\text{行}}}$ います。 \hfill\break
 \emph{Shunk }\emph{ō kara j }\emph{ūnen wo heta tatemono no ch }\emph{ōsa wo okonaimasu. \hfill\break
 }We perform investigations of buildings which have passed ten years since completion. }

\par{\textbf{Sentence Note }: \emph{Heru }経る is an intransitive verb. It is indicative of how this use of wo, when used with intransitive verbs, is used primarily with explicit time phrases. }

\par{\textbf{Grammar Note }: The particle \emph{kara }から, when used with nouns that relate to time, indicate a starting point in time. This shows how the concepts of “from” and “since” are the same in Japanese. }

\par{48. その ${\overset{\textnormal{さつじんじけん}}{\text{殺人事件}}}$ の ${\overset{\textnormal{ようぎしゃ}}{\text{容疑者}}}$ は ${\overset{\textnormal{じゅう}}{\text{10}}}$ ${\overset{\textnormal{ねん}}{\text{年}}}$ もの ${\overset{\textnormal{あいだ}}{\text{間}}}$ (を)、 ${\overset{\textnormal{いき}}{\text{息}}}$ を ${\overset{\textnormal{ひそ}}{\text{潜}}}$ めて、 ${\overset{\textnormal{かく}}{\text{隠}}}$ れていました。 \hfill\break
 \emph{Sono satsujin jiken no y }\emph{ōgisha wa j }\emph{ūnen mo no aida (wo), iki wo hisomete, kakurete imashita. \hfill\break
 }The suspect of that murder case had been hiding, breath bated, for ten years. }

\par{\textbf{Grammar Note }: In this example, the temporal \emph{wo }を is optional. For one, it is not normally paired with the intransitive verb \emph{kakureru }隠れる (to hide) as the length of hiding is not a detail that must be explicitly stated. The reason why it would appear in a sentence like Ex. 48 is to emphasize the fact that the suspect was hiding for ten years. The sense of the hiding having been something that was ongoing is the heart of what the temporal \emph{wo }を means. }

\par{\textbf{Grammar Note }: The phrase \emph{mo no aida }もの間 equates to “for” in this sentence. }

\par{49. ${\overset{\textnormal{わたし}}{\text{私}}}$ たちは ${\overset{\textnormal{きび}}{\text{厳}}}$ しい ${\overset{\textnormal{げんじつ}}{\text{現実}}}$ の ${\overset{\textnormal{なか}}{\text{中}}}$ を ${\overset{\textnormal{い}}{\text{生}}}$ きている。 \hfill\break
 \emph{Watashitachi wa kibishii genjitsu no naka wo ikite iru. \hfill\break
 }We are living through a harsh reality. }

\par{\textbf{Grammar Note }: Although \emph{genjitsu no naka }現実の中 can be interpreted as being a spatial phrase; “living through” a situation implies that time is also passing. This shows just how intertwined spatial and temporal phrases often are in Japanese. }

\par{50. ${\overset{\textnormal{いったんしゃかい}}{\text{一旦社会}}}$ を ${\overset{\textnormal{はな}}{\text{離}}}$ れた ${\overset{\textnormal{じょせい}}{\text{女性}}}$ がブランクを ${\overset{\textnormal{へ}}{\text{経}}}$ て ${\overset{\textnormal{ふたた}}{\text{再}}}$ び ${\overset{\textnormal{しごと}}{\text{仕事}}}$ に ${\overset{\textnormal{つ}}{\text{就}}}$ くことはなかなか ${\overset{\textnormal{こんなん}}{\text{困難}}}$ だ。 \hfill\break
 \emph{Ittan shakai wo hanareta josei ga buranku wo hete futatabi shigoto ni tsuku koto wa nakanaka kon\textquotesingle nan da. } \hfill\break
A woman who has separated herself from the public taking a job again upon going through a gap is a fairly difficult thing. }

\par{\textbf{Grammar Notes }: \hfill\break
1. The word \emph{koto }こと is used to nominalize the entire phrase before it. \emph{Koto }こと literally means “situation.” \hfill\break
2. The \emph{wo }を before \emph{hanareta }離れた is the \emph{wo }を of origin of departure. \hfill\break
3. \emph{Hete }経て is the verb \emph{heru }経る with the conjunctive particle \emph{te }て. The phrase \emph{buranku wo hete }ブランクを経て means “going through a gap…” This \emph{wo }を is the temporal \emph{wo }を. \hfill\break
4. The use of the particle \emph{ni }に before the verb \emph{tsuku }就く indicates that one becomes seated into an occupation. }
    