    
\chapter{Regular Verbs II}

\begin{center}
\begin{Large}
第17課: Regular Verbs II: 五段 Godan Verbs 
\end{Large}
\end{center}
 
\par{ The next class of verbs that we will study is the \emph{Godan }五段 verb class. These verbs behave just like \emph{ru }る verbs \emph{Ichidan }一段 verbs. These verbs conjugate and function the same way with the only difference  being that their roots end in consonants. Because their stems end in consonants, which are then followed by \emph{u }う in their basic form, they are frequently referred to as \emph{u }verbs. This difference does have one drawback, which is that you will have to learn contraction rules with certain conjugations. }
      
\section{Vocabulary List}
 
\par{\textbf{Nouns }}
 
\par{・小説 \emph{Sh }\emph{ōsetsu }– Novel }
 
\par{・世界 \emph{Sekai }– World }
 
\par{・食べ物 \emph{Tabemono }– Food }
 
\par{・道 \emph{Michi }– Road }
 
\par{・警察 \emph{Keisatsu }– Police }
 
\par{・バター \emph{Bat }\emph{ā }– Butter }
 
\par{・お金 \emph{Okane }– Money }
 
\par{・宿題 \emph{Shukudai }– Homework }
 
\par{・鳥 \emph{Tori }– Bird }
 
\par{・卵 \emph{Tamago }– Egg(s) }
 
\par{・肉団子 \emph{Nikudango }– Meatball(s) }
 
\par{・アニソン \emph{Anison }– Anime song }
 
\par{・悪口 \emph{Warukuchi }– Slander }
 
\par{・人 \emph{Hito }– Person }
 
\par{・喉 \emph{Nodo }– Throat }
 
\par{・香り \emph{Kaori }- Scent }
 
\par{・タイプミス \emph{Taipumisu }– Typo }
 
\par{・祖母 \emph{Sobo }– Grandmother }
 
\par{・竹 \emph{Take }– Bamboo }
 
\par{・こと \emph{Koto }– Thing\slash matter\slash event }
 
\par{・自然薯 \emph{Jinenjo }– Japanaese yam }
 
\par{・心 \emph{Kokoro }– Heart (emotional) }
 
\par{・電気 \emph{Denki }– Electricity }
 
\par{・薬 \emph{Kusuri }– Medicine }
 
\par{・風邪 \emph{Kaze }– Cold }
 
\par{・風邪薬 \emph{Kazegusuri }– Cold medicine }
 
\par{・銃 \emph{J }\emph{ū }– Gun }
 
\par{・服 \emph{Fuku }– Clothes }
 
\par{・夫 \emph{Otto }– Husband }
 
\par{・英会話 \emph{Eikaiwa }– English conversation }
 
\par{・クリスマス \emph{Kurisumasu }– Christmas }
 
\par{・歯 \emph{Ha }– Tooth }
 
\par{・ワニ \emph{Wani }– Crocodile\slash alligator }
 
\par{・とどめ \emph{Todome }– Finishing blow }
 
\par{・ハリー・ポッター \emph{Harii Pott }\emph{ā }– Harry Potter }
 
\par{・爪 \emph{Tsume }– Nail(s) }
 
\par{・商品 \emph{Sh }\emph{ōhin }– Product\slash item }
 
\par{・タバコ \emph{Tabako }– Tobacco }
 
\par{・本 \emph{Hon }– Book }
 
\par{・電話 \emph{Denwa }– Phone }
 
\par{・着物 \emph{Kimono }– Kimono }
 
\par{・車 \emph{Kuruma }– Car }
 
\par{・スタイル \emph{Sutairu }– Style }
 
\par{・魚 \emph{Sakana\slash Uo }– Fish }
 
\par{・街 \emph{Machi }– Town }
 
\par{・助け \emph{Tasuke }– Help }
 
\par{\textbf{Pronouns }}
 
\par{・私 \emph{Wata(ku)shi }– I }
 
\par{・僕 \emph{Boku }– I (male) }
 
\par{・彼 \emph{Kare }– He }
 
\par{・彼女 \emph{Kanojo }– She }
 
\par{\textbf{Proper Nouns }}
 
\par{・小泉さん \emph{Koizumi-san }– Mr.\slash Mr(s). Koizumi }
 
\par{\textbf{Demonstratives }}
 
\par{・この \emph{Kono }– This (adj.) }
 
\par{・その \emph{Sono }– That (adj.) }
 
\par{\textbf{Adjectives }}
 
\par{・美味しい \emph{Oishii }– Delicious }
 
\par{・黄色い \emph{Kiiroi }– Yellow }
 
\par{\textbf{Adjectival Nouns }}
 
\par{・好きだ \emph{Suki da }– To like }
 
\par{・同じだ \emph{Onaji da }– To be the same }
 
\par{\textbf{Adverbs }}
 
\par{・いつか \emph{Itsuka }– One day\slash someday }
 
\par{・全員 \emph{Zen\textquotesingle in }– Everyone\slash everybody }
 
\par{・本気で \emph{Honki de }– All out\slash seriously\slash in earnest }
 
\par{・やっと \emph{Yatto – }Finally }
 
\par{・決して \emph{Kesshite }– Never }
 
\par{・まったく \emph{Mattaku }– Completely\slash at all (neg.) }
 
\par{・もう \emph{M }\emph{ō }– Already\slash (not) anymore\slash before long }
 
\par{・まだ \emph{Mada }– Still\slash yet }
 
\par{・多分 \emph{Tabun }– Probably }
 
\par{・あまり \emph{Amari }– Not really }
 
\par{・たくさん \emph{Takusan }– A lot }
 
\par{・ずっと \emph{Zutto }– All along\slash by far }
 
\par{\textbf{( \emph{ru }) \emph{Ichidan }Verbs }}
 
\par{・変える \emph{Kaeru }– To change (trans.) }
 
\par{・着る \emph{Kiru }– To wear (trans.) }
\hfill\break
  \textbf{( \emph{u }) \emph{Godan }Verbs }
\par{・書く \emph{Kaku }– To write (trans.) }

\par{・泳ぐ \emph{Oyogu }– To swim  (intr.) }

\par{・話す \emph{Hanasu }– To talk\slash speak (trans.) }

\par{・勝つ \emph{Katsu }– To win (intr.) }

\par{・死ぬ \emph{Shinu }– To die (intr.) }

\par{・選ぶ \emph{Erabu }– To choose\slash select (trans.) }

\par{・読む \emph{Yomu }– To read (trans.) }

\par{・変わる \emph{Kawaru }– To change (intr.) }

\par{・買う \emph{Kau }– To buy (trans.) }

\par{・聞く \emph{Kiku }– To listen\slash ask (trans.) }

\par{・繋ぐ \emph{Tsunagu }– To connect (trans.) }

\par{・指す \emph{Sasu }– To point (out)\slash identify (trans.) }

\par{・待つ \emph{Matsu }– To wait (for) (trans.) }

\par{・呼ぶ \emph{Yobu }– To call\slash invoke\slash summon (trans.) }

\par{・挑む \emph{Idomu }– To challenge (intr.) }

\par{・凍る \emph{Kōru }- To freeze\slash be frozen over (intr.) }

\par{・会う \emph{Au }– To meet\slash encounter (intr.) }

\par{・巻く \emph{Maku }– To wind\slash envelope (trans.\slash intr.) }

\par{・受け継ぐ \emph{Uketsugu }– To inherit (trans.) }

\par{・貸す \emph{Kasu }– To lend (trans.) }

\par{・終わる \emph{Owaru }– To end (intr.) }

\par{・叫ぶ \emph{Sakebu }– To scream\slash shout (intr.) }

\par{・包む \emph{Tsutsumu }– To wrap\slash conceal (trans.) }

\par{・売る \emph{Uru }– To sell (trans.) }

\par{・歌う \emph{Utau }– To sing (trans.) }

\par{・行く \emph{Iku }– To go (intr.) }

\par{・言う \emph{Iu }– To say (trans.) }

\par{・乾く \emph{Kawaku }– To dry up (intr.) }

\par{・嗅ぐ \emph{Kagu }– To sniff\slash smell (trans.) }

\par{・直す \emph{Naosu }– To fix (trans.) }

\par{・治す \emph{Naosu }– To heal\slash cure (trans.) }

\par{・打つ \emph{Utsu }– To hit\slash beat\slash type\slash etc. (trans.) }

\par{・撃つ \emph{Utsu }– To shoot (trans.) }

\par{・運ぶ \emph{Hakobu }– To carry\slash transport (trans.) }

\par{・育む \emph{Hagukumu }– To raise\slash rear (trans.) }

\par{・掘る \emph{Horu }– To dig (trans.) }

\par{・思う \emph{Omou }– To think (trans.) }

\par{・つく \emph{Tsuku }– To be lit\slash turn on (intr.) }

\par{・揺るぐ \emph{Yurugu }– To waver (intr.) }

\par{・遊ぶ \emph{Asobu }– To play (trans.) }

\par{・畳む \emph{Tatamu }– To fold (trans.) }

\par{・実る \emph{Minoru }– To bear fruit (intr.) }

\par{・習う \emph{Narau }– To take lessons in (trans.) }

\par{・働く \emph{Hataraku }– To work (intr.) }

\par{・引き継ぐ \emph{Hikitsugu }– To transfer\slash hand over (trans.) }

\par{・騙す \emph{Damasu }– To deceive (trans.) }

\par{・乱す \emph{Midasu }– To throw out of order (trans.) }

\par{・立つ \emph{Tatsu }– To stand (intr.) }

\par{・学ぶ \emph{Manabu }– To study (trans.) }

\par{・悩む \emph{Nayamu }– To be troubled (intr.) }

\par{・帰る \emph{Kaeru }– To return home (intr.) }

\par{・祝う \emph{Iwau }– To celebrate (trans.) }

\par{・動く \emph{Ugoku }– To move (intr.) }

\par{・騒ぐ \emph{Sawagu }– To clamor (intr.) }

\par{・刺す \emph{Sasu }– To stab (trans.) }

\par{・持つ \emph{Motsu }– To hold (trans.) }

\par{・喜ぶ \emph{Yorokobu }– To be delighted (intr.) }

\par{・噛む \emph{Kamu }– To bite (trans.) }

\par{・祈る \emph{Inoru }– To pray (trans.) }

\par{・縫う \emph{N }\emph{ū }– To sew (trans.) }

\par{・届く \emph{Todoku }– To reach\slash arrive (intr.) }

\par{・移す \emph{Utsusu }– To swap\slash infect (trans.) }

\par{・保つ \emph{Tamotsu }– To preserve\slash retain (trans.) }

\par{・並ぶ \emph{Narabu }– To form a line (intr.) }

\par{・吸う \emph{S }\emph{ū }– To inhale\slash smoke\slash suck (trans.) }

\par{・飲む \emph{Nomu }– To drink\slash swallow\slash take (medicine) (trans.) }

\par{・切る \emph{Kiru }– To cut (trans.) }

\par{・走る \emph{Hashiru }– To run (intr.) }

\par{・喋る \emph{Shaberu }– To chat(ter)\slash talk (intr.) }

\par{・触る \emph{Sawaru }– To touch\slash feel (trans.\slash intr.) }

\par{・入る \emph{Hairu }– To enter (intr.) }

\par{・滑る \emph{Suberu }– To slip\slash slide (intr.) }

\par{・握る \emph{Nigiru }– To grasp (trans.) }

\par{・限る \emph{Kagiru }– To limit\slash restrict (trans.) }

\par{・釣る \emph{Tsuru }– To fish\slash lure }

\par{・蹴る \emph{Keru }– To kick }

\par{・嘲る \emph{Azakeru }– To ridicule }

\par{・捻る \emph{Hineru }– To twist }
      
\section{Conjugating Godan Verbs}
 
\par{ \emph{Godan }五段 verbs conjugate as do adjectives, adjectival nouns with the help of \emph{da }だ or \emph{desu }です, and the basic tenses are still the non-past and past tenses. As we've learned in the lesson prior, the non-past tense can be used to stand for either the present tense, future tense, or the gerund depending on context. The past tense correlates to both the English past tense and perfect tense. Using this knowledge that we've already attained, we will take the next step of our journey by learning how to conjugate the remaining 49\% of verbs. }

\begin{center}
\textbf{Plain Non-Past Form: No Conjugation }
\end{center}

\par{ As was the case for \emph{Ichidan }一段 verbs, no conjugation is required for the non-past tense in plain speech. Of course, the non-past needs \emph{-masu }ます when you use polite speech, but we'll get to that shortly. Because a \emph{Godan }五段 verb can have any consonant end its stem, we'll need to first see all the possibilities. }

\begin{ltabulary}{|P|P|P|}
\hline 

K & To write &  \emph{Ka \textbf{k }u }書く \\ \cline{1-3}

G & To swim &  \emph{Oyo \textbf{g }u }泳ぐ \\ \cline{1-3}

S & To talk\slash speak &  \emph{Hana \textbf{s }u }話す \\ \cline{1-3}

T & To win &  \emph{Ka \textbf{ts }u }勝つ \\ \cline{1-3}

N & To die &  \emph{Shi \textbf{n }u }死ぬ \\ \cline{1-3}

B & To choose &  \emph{Era \textbf{b }u }選ぶ \\ \cline{1-3}

M & To read &  \emph{Yo \textbf{m }u }読む \\ \cline{1-3}

R & To change &  \emph{Kawa \textbf{r }u }変わる \\ \cline{1-3}

W* & To buy &  \emph{Kau }買う \\ \cline{1-3}

\end{ltabulary}

\par{\textbf{Verb Notes }: }

\par{1. The consonant "w" becomes important in certain conjugations, so even though it may be silent here, it does play a role. \hfill\break
2. \emph{Shinu }死ぬ is the only verb in Standard Japanese that ends in \emph{nu }ぬ. }

\par{1. ${\overset{\textnormal{しょうせつ}}{\text{小説}}}$ を ${\overset{\textnormal{か}}{\text{書}}}$ く。 \hfill\break
\emph{Shōsetsu wo kaku. \hfill\break
}To write a novel. }

\par{2. いつか ${\overset{\textnormal{し}}{\text{死}}}$ ぬ。 \hfill\break
\emph{Itsuka shinu. }\hfill\break
To one-day die. }

\par{3. ${\overset{\textnormal{せかい}}{\text{世界}}}$ が ${\overset{\textnormal{か}}{\text{変}}}$ わる。 \hfill\break
\emph{Sekai ga kawaru. \hfill\break
}The world will change. }

\par{4. ${\overset{\textnormal{おい}}{\text{美味}}}$ しい ${\overset{\textnormal{た}}{\text{食}}}$ べ ${\overset{\textnormal{もの}}{\text{物}}}$ を ${\overset{\textnormal{か}}{\text{買}}}$ う。 \hfill\break
\emph{Oishii tabemono wo kau. \hfill\break
}To buy delicious food. }

\begin{center}
\textbf{Polite Non-Past Form: }\textbf{- \emph{masu }}\textbf{ます }
\end{center}
 
\par{ To make a \emph{Godan }五段 verb polite in the non-past tense, change the vowel after the stem to \slash i\slash  い, and then add \emph{-masu }\emph{ます. }Note that the consonant t will turn to ch when followed by the vowel i. }

\begin{ltabulary}{|P|P|P|}
\hline 

Meaning & Verb & Stem + \emph{i }+ - \emph{masu }\emph{ます }\\ \cline{1-3}

To ask\slash listen &  \emph{Kiku }聞く &  \emph{Kikimasu }聞きます \\ \cline{1-3}

To connect &  \emph{Tsunagu }繋ぐ &  \emph{Tsunagimasu }繋ぎます \\ \cline{1-3}

To point (out)\slash identify &  \emph{Sasu }指す &  \emph{Sashimasu }指します \\ \cline{1-3}

To wait (for) &  \emph{Matsu }待つ &  \emph{Machimasu }待ちます \\ \cline{1-3}

To die &  \emph{Shinu }死ぬ & \emph{Shinimasu }死にます \\ \cline{1-3}

To call\slash invoke\slash summon &  \emph{Yobu }呼ぶ & \emph{Yobimasu }呼びます \\ \cline{1-3}

To challenge &  \emph{Idomu }挑む & \emph{Idomimasu }挑みます \\ \cline{1-3}

To freeze\slash be frozen over &  \emph{Kōru }凍る \hfill\break
&  \emph{Kōrimasu }凍ります \\ \cline{1-3}

To meet\slash encounter &  \emph{Au }会う &  \emph{Aimasu }会います \\ \cline{1-3}

\end{ltabulary}

\par{\textbf{Usage Note }: This form cannot modify nouns. To modify nouns with verbs in the non-past tense, you must use the plain form without \emph{-masu }ます. }

\par{5. ${\overset{\textnormal{みち}}{\text{道}}}$ が ${\overset{\textnormal{こお}}{\text{凍}}}$ ります。 \hfill\break
\emph{M }\emph{ichi ga k }\emph{ōrimasu. \hfill\break
}The road(s) (will) freeze over. }

\par{6. ${\overset{\textnormal{こいずみ}}{\text{小泉}}}$ さんを ${\overset{\textnormal{ま}}{\text{待}}}$ ちます。 \hfill\break
\emph{Koizumi-san wo machimasu. \hfill\break
}I\textquotesingle ll wait for Mr. Koizumi. }

\par{7. ${\overset{\textnormal{ぜんいんし}}{\text{全員死}}}$ にます。 \hfill\break
\emph{Zen\textquotesingle in shinimasu. \hfill\break
}Everyone dies. }

\par{8. ${\overset{\textnormal{けいさつ}}{\text{警察}}}$ を ${\overset{\textnormal{よ}}{\text{呼}}}$ びますよ。 \hfill\break
\emph{Keisatsu wo yobimasu yo. \hfill\break
}I\textquotesingle ll call the police. }

\par{9. ${\overset{\textnormal{ほんき}}{\text{本気}}}$ で ${\overset{\textnormal{いど}}{\text{挑}}}$ みます! \hfill\break
\emph{Honki de idomimasu! \hfill\break
}I\textquotesingle m going to challenge all out! }
  
\begin{center}
\textbf{Plain Past Form: \emph{-ta }}\textbf{た }
\end{center}
 
\par{ To make a \emph{Godan }一段 verb past tense in plain speech, \emph{-ta }\emph{た is used just like with Ichidan  一段 verbs }. However, a series of sound changes affect them in different ways depending on what consonant the verb stem ends with. }

\begin{itemize}

\item If the verb stem ends in k, the k drops, \slash i\slash  is inserted, and - \emph{ta }た is attached. 
\item If the verb stem ends in g, the g drops, \slash i\slash  is inserted, but - \emph{ta }た \textrightarrow  -da だ when attached. \hfill\break

\item If the verb stem ends in s, -ta attaches normally like with -masu ます. 
\item If the verb stem ends in n, b, or m, these consonants \textrightarrow  n, and - \emph{ta }た \textrightarrow  - \emph{da }だ when attached. 
\item If the verb stem ends in t, r, or w, these consonants \textrightarrow  small tsu っ, and then \emph{-ta }た attaches. 
\end{itemize}

\par{ Examples affected by the sound changes described in these rules are shown in bold below along with verbs that are irregular. }

\begin{ltabulary}{|P|P|P|}
\hline 

Meaning & Verb & Past Tense \\ \cline{1-3}

To wind\slash envelope &  \emph{Maku }巻く &  \textbf{\emph{Maita }}巻いた \\ \cline{1-3}

To inherit &  \emph{Uketsugu }受け継ぐ &  \textbf{\emph{Uketsuida }}受け継いだ \\ \cline{1-3}

To lend &  \emph{Kasu }貸す &  \emph{\textbf{Kashita }}貸した \\ \cline{1-3}

To end &  \emph{Owaru }終わる &  \emph{\textbf{Owatta }}終わった \\ \cline{1-3}

To die &  \emph{Shinu }死ぬ &  \emph{\textbf{Shinda } }死んだ \\ \cline{1-3}

To scream\slash shout &  \emph{Sakebu }叫ぶ &  \textbf{\emph{Sakenda }}叫んだ \\ \cline{1-3}

To wrap up\slash conceal &  \emph{Tsutsumu }包む &  \emph{\textbf{Tsutsunda }}包んだ \\ \cline{1-3}

To sell &  \emph{Uru }売る &  \textbf{\emph{Utta }}売った \\ \cline{1-3}

To sing &  \emph{Utau }歌う &  \textbf{\emph{Utatta }}歌った \\ \cline{1-3}

To go* &  \emph{Iku }行く &  \emph{\textbf{Itta }}行った \\ \cline{1-3}

To say* &  \emph{Iu }言う & \textbf{\emph{Itta\slash Yutta }}言った \\ \cline{1-3}

\end{ltabulary}

\par{\textbf{Usage Note }: This form can also be used to modify nouns without any change in form. Most conjugations are able to do so as long as they are not in their polite forms. }

\par{\textbf{Grammar Notes }: \hfill\break
1. The verb "to go" becomes \emph{itta }行った in the past tense unlike all other verbs whose stems end in k. \hfill\break
2. The verb "to say" is actually pronounced as "yū." This discrepancy manifests in the past tense as well, making both "itta" and "yutta" viable and equally correct pronunciations. }

\par{10. バターを ${\overset{\textnormal{う}}{\text{売}}}$ った。 \hfill\break
\emph{Bat }\emph{ā wo utta. \hfill\break
}I sold butter. }

\par{11. お ${\overset{\textnormal{かね}}{\text{金}}}$ を ${\overset{\textnormal{か}}{\text{貸}}}$ した。 \hfill\break
\emph{Okane wo kashita. \hfill\break
}I lent money. }
 
\par{12. やっと ${\overset{\textnormal{しゅくだい}}{\text{宿題}}}$ が ${\overset{\textnormal{お}}{\text{終}}}$ わった! \hfill\break
\emph{Yatto shukudai ga owatta. }\hfill\break
My homework is finally finished! }
 
\par{13. その ${\overset{\textnormal{きいろ}}{\text{黄色}}}$ い ${\overset{\textnormal{とり}}{\text{鳥}}}$ が ${\overset{\textnormal{し}}{\text{死}}}$ んだ。 \hfill\break
\emph{Sono kiiroi tori ga shinda. }\hfill\break
That yellow bird died. }

\par{14. ${\overset{\textnormal{たまご}}{\text{卵}}}$ を ${\overset{\textnormal{つつ}}{\text{包}}}$ んだ ${\overset{\textnormal{にくだんご}}{\text{肉団子}}}$ \hfill\break
\emph{Tamago wo tsutsunda nikudango \hfill\break
}Meatballs with egg packed in }
 
\par{15. アニソンを ${\overset{\textnormal{うた}}{\text{歌}}}$ った。 \hfill\break
\emph{Anison wo utatta. }\hfill\break
I sang an anime song. }

\par{16. ${\overset{\textnormal{わるくち}}{\text{悪口}}}$ を言った ${\overset{\textnormal{ひと}}{\text{人}}}$ \hfill\break
\emph{Warukuchi wo itta\slash yutta hito \hfill\break
}Person who said slander }
  
\begin{center}
\textbf{Polite Past Form: \emph{-mashita }}\textbf{\emph{まし }}\textbf{た } 
\end{center}
 
\par{  To make a \emph{Godan }五段 verb past tense in polite speech, change the vowel after the stem to \slash i\slash  い, and then add \emph{-mashita }\emph{ました }. }

\begin{ltabulary}{|P|P|P|}
\hline 

Meaning & Verb & Stem + \emph{i }+ - \emph{mashita }\emph{ました }\\ \cline{1-3}

To dry up &  \emph{Kawaku }乾く &  \emph{Kawakimashita }乾きました \\ \cline{1-3}

To sniff\slash smell &  \emph{Kagu }嗅ぐ &  \emph{Kagimashita }嗅ぎました \\ \cline{1-3}

To fix &  \emph{Naosu }直す &  \emph{Naoshimashita }直しました \\ \cline{1-3}

To hit\slash beat\slash type\slash etc. &  \emph{Utsu }打つ & \emph{Uchimashita }打ちました \\ \cline{1-3}

To die &  \emph{Shinu }死ぬ &  \emph{Shinimashita }死にました \\ \cline{1-3}

To carry\slash transport &  \emph{Hakobu }運ぶ &  \emph{Hakobimashita }運びました \\ \cline{1-3}

To raise\slash rear & \emph{Hagukumu }育む &  \emph{Hagukumimashita }育みました \\ \cline{1-3}

To dig &  \emph{Horu }掘る &  \emph{Horimashita }掘りました \\ \cline{1-3}

To think &  \emph{Omou }思う &  \emph{Omoimashita }思いました \\ \cline{1-3}

\end{ltabulary}

\par{\textbf{Usage Note }: This form cannot modify nouns. To modify nouns with verbs in the past tense, you must use the plain form \emph{-ta }た. }

\par{17. ${\overset{\textnormal{のど}}{\text{喉}}}$ が ${\overset{\textnormal{かわ}}{\text{渇}}}$ きました。 \hfill\break
\emph{Nodo ga kawakimashita. \hfill\break
}I\textquotesingle m thirsty\slash My throat is dry. }

\par{18. ${\overset{\textnormal{す}}{\text{好}}}$ きな ${\overset{\textnormal{かお}}{\text{香}}}$ りを ${\overset{\textnormal{か}}{\text{嗅}}}$ ぎました。 \hfill\break
\emph{Suki na kaori wo kagimashita. \hfill\break
}I smelled a scent that I like. }

\par{19. タイプミスを ${\overset{\textnormal{なお}}{\text{直}}}$ しました。 \hfill\break
\emph{Taipumisu wo naoshimashita. \hfill\break
}I fixed the typo(s). }

\par{20. ${\overset{\textnormal{そぼ}}{\text{祖母}}}$ は ${\overset{\textnormal{きょねんし}}{\text{去年死}}}$ にました。 \hfill\break
\emph{Sobo wa kyonen shinimashita. \hfill\break
}My grandmother died last year. }

\par{21. ${\overset{\textnormal{たけ}}{\text{竹}}}$ を ${\overset{\textnormal{はこ}}{\text{運}}}$ びました。 \hfill\break
\emph{Take wo hakobimashita. \hfill\break
}I carried\slash transported bamboo. }

\par{22. ${\overset{\textnormal{おな}}{\text{同}}}$ じことを ${\overset{\textnormal{おも}}{\text{思}}}$ いました。 \hfill\break
\emph{Onaji koto wo omoimashita. \hfill\break
}I thought the same thing. }

\par{23. ${\overset{\textnormal{じねんじょ}}{\text{自然薯}}}$ を ${\overset{\textnormal{ほ}}{\text{掘}}}$ りました。 \hfill\break
\emph{Jinenjo wo horimashita. \hfill\break
}I dug up Japanese yams. }
  
\begin{center}
\textbf{Plain Negative Form: \emph{-nai } }\textbf{ない } 
\end{center}
 
\par{ To make a \emph{Godan }五段 verb negative in plain speech, change the vowel after the stem to \slash a\slash  あ, and then add - \emph{nai }\emph{ない }. Note how in this conjugation, silent w becomes spoken. }

\begin{ltabulary}{|P|P|P|}
\hline 

Meaning & Verb & Stem + \emph{a }+ - \emph{nai }ない \\ \cline{1-3}

To be lit\slash turn on & \emph{Tsuku }つく &  \emph{Tsukanai つ }かない \\ \cline{1-3}

To waver &  \emph{Yurugu }揺るぐ &  \emph{Yuruganai }揺るがない \\ \cline{1-3}

To heal\slash cure &  \emph{Naosu }治す &  \emph{Naosanai }治さない \\ \cline{1-3}

To shoot & \emph{Utsu }撃つ &  \emph{Utanai }撃たない \\ \cline{1-3}

To die &  \emph{Shinu }死ぬ &  \emph{Shinanai }死なない \\ \cline{1-3}

To play &  \emph{Asobu }遊ぶ &  \emph{Asobanai }遊ばない \\ \cline{1-3}

To fold & \emph{Tatamu }畳む &  \emph{Tatamanai }畳まない \\ \cline{1-3}

To bear fruit &  \emph{Minoru }実る &  \emph{Minoranai }実らない \\ \cline{1-3}

To take lessons in & \emph{Narau }習う &  \emph{Narawanai }習わない \\ \cline{1-3}

\end{ltabulary}

\par{\textbf{Usage Note }: This form can also be used to modify nouns without any change in form. Most conjugations are able to do so as long as they are not in their polite forms. }

\par{24. ${\overset{\textnormal{こころ}}{\text{心}}}$ は ${\overset{\textnormal{ゆ}}{\text{揺}}}$ るがない。 \hfill\break
\emph{Kokoro wa yuruganai. \hfill\break
}My heart won\textquotesingle t waver. }

\par{25. ${\overset{\textnormal{でんき}}{\text{電気}}}$ がつかない。 \hfill\break
\emph{Denki ga tsukanai. \hfill\break
}The electricity won\textquotesingle t\slash doesn\textquotesingle t turn on. }

\par{26. ${\overset{\textnormal{かぜぐすり}}{\text{風邪薬}}}$ は ${\overset{\textnormal{かぜ}}{\text{風邪}}}$ を ${\overset{\textnormal{なお}}{\text{治}}}$ さない。 \hfill\break
\emph{Kazegusuri wa kaze wo naosanai. \hfill\break
}Cold medicine won\textquotesingle t cure a cold. }

\par{27. ${\overset{\textnormal{じゅう}}{\text{銃}}}$ を ${\overset{\textnormal{う}}{\text{撃}}}$ たない。 \hfill\break
\emph{J }\emph{ū wo utanai. \hfill\break
}I won\textquotesingle t shoot a gun. }

\par{28. ${\overset{\textnormal{けっ}}{\text{決}}}$ して ${\overset{\textnormal{し}}{\text{死}}}$ なない。 \hfill\break
\emph{Kesshite shinanai. \hfill\break
}I\textquotesingle ll never die. }

\par{29. ${\overset{\textnormal{ふく}}{\text{服}}}$ を ${\overset{\textnormal{たた}}{\text{畳}}}$ まない ${\overset{\textnormal{おっと}}{\text{夫}}}$ \hfill\break
\emph{Fuku wo tatamanai otto \hfill\break
}Husband who won\textquotesingle t fold clothes }

\par{30. ${\overset{\textnormal{えいかいわ}}{\text{英会話}}}$ を ${\overset{\textnormal{なら}}{\text{習}}}$ わない。 \hfill\break
Eikaiwa wo narawanai. \hfill\break
To not take lessons in English conversation. }
  
\begin{center}
\textbf{Polite Negative Form: \emph{-masen } }\textbf{ません }
\end{center}
 
\par{ To make a \emph{Godan }五段 verb negative in polite speech, change the vowel after the stem to \slash i\slash  い, and then add \emph{-masen }\emph{ません }. }

\begin{ltabulary}{|P|P|P|}
\hline 

Meaning & Verb & Stem + \emph{i }+ \emph{-masen }ません \\ \cline{1-3}

To work &  \emph{Hataraku }働く &  \emph{Hatarakimasen }働きません \\ \cline{1-3}

To transfer\slash hand over &  \emph{Hikitsugu }引き継ぐ &  \emph{Hikitsugimasen }引き継ぎません \\ \cline{1-3}

To deceive &  \emph{Damasu }騙す &  \emph{Damashimasen }騙しません \\ \cline{1-3}

To throw out of order &  \emph{Midasu }乱す & \emph{Midashimasen }乱しません \\ \cline{1-3}

To stand &  \emph{Tatsu }立つ &  \emph{Tachimasen }立ちません \\ \cline{1-3}

To die &  \emph{Shinu }死ぬ &  \emph{Shinimasen }死にません \\ \cline{1-3}

To study &  \emph{Manabu }学ぶ &  \emph{Manabimasen }学びません \\ \cline{1-3}

To be troubled &  \emph{Nayamu }悩む &  \emph{Nayamimasen }悩みません \\ \cline{1-3}

To return home &  \emph{Kaeru }帰る &  \emph{Kaerimasen }帰りません \\ \cline{1-3}

To celebrate &  \emph{Iwau }祝う &  \emph{Iwaimasen }祝いません \\ \cline{1-3}

\end{ltabulary}

\par{\textbf{Usage Note }: This form cannot modify nouns. To modify nouns with verbs in the negative, you must use the plain form \emph{-nai }ない. }

\par{31. ${\overset{\textnormal{おっと}}{\text{夫}}}$ がまったく ${\overset{\textnormal{はたら}}{\text{働}}}$ きません。 \hfill\break
My husband won\textquotesingle t work at all. }

\par{32. ${\overset{\textnormal{ぼく}}{\text{僕}}}$ は ${\overset{\textnormal{し}}{\text{死}}}$ にません! (Male speech) \hfill\break
\emph{Boku wa shinimasen! \hfill\break
}I won\textquotesingle t die! }

\par{33. もう ${\overset{\textnormal{なや}}{\text{悩}}}$ みません。 \hfill\break
\emph{M }\emph{ō nayamimasen. }\hfill\break
I won\textquotesingle t be troubled anymore. }

\par{34. ${\overset{\textnormal{わたし}}{\text{私}}}$ はまだ ${\overset{\textnormal{かえ}}{\text{帰}}}$ りません。 \hfill\break
\emph{Watashi wa mada kaerimasen. \hfill\break
}I\textquotesingle m not going home yet. }

\par{35. クリスマスは ${\overset{\textnormal{いわ}}{\text{祝}}}$ いません。 \hfill\break
\emph{Kurisumasu wa iwaimasen. }\hfill\break
I won\textquotesingle t celebrate \emph{Christmas }. }

\par{36. ${\overset{\textnormal{は}}{\text{歯}}}$ が ${\overset{\textnormal{た}}{\text{立}}}$ ちません。 \hfill\break
\emph{Ha ga tachimasen. \hfill\break
}I can\textquotesingle t make a dent in it. }

\par{ \textbf{Idiom Note }: This idiom comes from a phrase literally meaning “unable to bite through something.” }
  
\begin{center}
\textbf{Plain Negative Past Form: \emph{-nakatta } }\textbf{なかった } 
\end{center}
 
\par{ To make a \emph{Godan }五段 verb negative past in plain speech, change the vowel after the stem to \slash a\slash  あ, and then add \emph{-nakatta }\emph{なかった }. }

\begin{ltabulary}{|P|P|P|}
\hline 

Meaning & Verb & Stem + \emph{a }+ \emph{-nakatta }なかった \\ \cline{1-3}

To work\slash move &  \emph{Ugoku }動く &  \emph{Ugokanakatta }動かなかった \\ \cline{1-3}

To clamor &  \emph{Sawagu }騒ぐ &  \emph{Sawaganakatta }騒がなかった \\ \cline{1-3}

To stab &  \emph{Sasu }刺す &  \emph{Sasanakatta }刺さなかった \\ \cline{1-3}

To hold &  \emph{Motsu }持つ &  \emph{Motanakatta }持たなかった \\ \cline{1-3}

To die &  \emph{Shinu }死ぬ &  \emph{Shinanakatta }死ななかった \\ \cline{1-3}

To be delighted &  \emph{Yorokobu }喜ぶ &  \emph{Yorokobanakatta }喜ばなかった \\ \cline{1-3}

To bite &  \emph{Kamu }噛む &  \emph{Kamanakatta }噛まなかった \\ \cline{1-3}

To pray &  \emph{Inoru }祈る &  \emph{Inoranakatta }祈らなかった \\ \cline{1-3}

To sew &  \emph{Nū }縫う &  \emph{Nuwanakatta }縫わなかった \\ \cline{1-3}

\end{ltabulary}

\par{\textbf{Usage Note }: This form can also be used to modify nouns without any change in form. Most conjugations are able to do so as long as they are not in their polite forms. }
 
\par{37. ワニが動かなかった。 \hfill\break
\emph{Wani ga ugokanakatta. }\hfill\break
The crocodile didn\textquotesingle t move. }

\par{38. とどめを ${\overset{\textnormal{さ}}{\text{刺}}}$ さなかった。 \hfill\break
\emph{Todome wo sasanakatta. \hfill\break
}I didn\textquotesingle t put an end to it. }

\par{39. ハリー・ポッターは ${\overset{\textnormal{し}}{\text{死}}}$ ななかった。 \hfill\break
\emph{Harii Potta wa shinanakatta. }\hfill\break
Harry Potter didn\textquotesingle t die. }

\par{40. ${\overset{\textnormal{つめ}}{\text{爪}}}$ を ${\overset{\textnormal{か}}{\text{噛}}}$ まなかった。 \hfill\break
\emph{Tsume wo kamanakatta. \hfill\break
}I didn\textquotesingle t bite my nails. }
  
\begin{center}
\textbf{Polite Negative Past Form: \emph{-masendeshita } }\textbf{ませんでした } 
\end{center}
 
\par{ To make a \emph{Godan }五段 verb negative past in polite speech, change the vowel after the stem to \slash i\slash  い, and then add \emph{-masendeshita }\emph{ませんでした }. }

\begin{ltabulary}{|P|P|P|}
\hline 

Meaning & Verb & Stem + \emph{i }+ \emph{-masendeshita }ませんでした \\ \cline{1-3}

To reach\slash arrive &  \emph{Todoku }届く &  \emph{Todokimasendeshita }届きませんでした \\ \cline{1-3}

To swim & \emph{Oyogu }泳ぐ & \emph{Oyogimasendeshita }泳ぎませんでした \\ \cline{1-3}

To swap\slash infect & \emph{Utsusu }移す &  \emph{Utsushimasendeshita }移しませんでした \\ \cline{1-3}

To preserve\slash retain &  \emph{Tamotsu }保つ &  \emph{Tamochimasendeshita }保ちませんでした \\ \cline{1-3}

To die &  \emph{Shinu }死ぬ &  \emph{Shinimasendeshita }死にませんでした \\ \cline{1-3}

To form a line &  \emph{Narabu }並ぶ &  \emph{Narabimasendeshita }並びませんでした \\ \cline{1-3}

To drink\slash swallow\slash take (medicine) & \emph{Nomu }飲む & \emph{Nomimasendeshita }飲みませんでした \\ \cline{1-3}

To inhale\slash smoke &  \emph{Sū }吸う &  \emph{Suimasendeshita }吸いませんでした \\ \cline{1-3}

\end{ltabulary}

\par{\textbf{Usage Note }: This form cannot modify nouns. To modify nouns with verbs in the negative past, you must use the plain form \emph{-nakatta }なかった. }

\par{41. ${\overset{\textnormal{くすり}}{\text{薬}}}$ を ${\overset{\textnormal{の}}{\text{飲}}}$ みませんでした。 \hfill\break
\emph{Kusuri wo nomimasendeshita. \hfill\break
}I didn\textquotesingle t take (the) medicine. }

\par{42. ${\overset{\textnormal{しょうひん}}{\text{商品}}}$ が ${\overset{\textnormal{とど}}{\text{届}}}$ きませんでした。 \hfill\break
\emph{Sh }\emph{ōhin ga todokimasendeshita. \hfill\break
}The item didn\textquotesingle t arrive. }
 
\par{43. カニは ${\overset{\textnormal{し}}{\text{死}}}$ にませんでした。 \hfill\break
\emph{Kani wa shinimasendeshita. }\hfill\break
The crab didn\textquotesingle t die. }
 
\par{44. タバコを ${\overset{\textnormal{す}}{\text{吸}}}$ いませんでした。 \hfill\break
\emph{Tabako wo suimasendeshita. \hfill\break
}I didn\textquotesingle t smoke. }
  
\begin{center}
\textbf{Alternative Polite Negative \& Neg-Past Forms: \emph{-nai desu \& -nakatta desu } }\textbf{ないです・なかったです } 
\end{center}
 
\par{ To make a \emph{Godan }五段 verb negative or negative past in polite yet casual speech, change the vowel after the stem to \emph{i }い, and then add \emph{-nai desu }\emph{ないです }or \emph{nakatta desu }なかったです respectively. }

\begin{ltabulary}{|P|P|P|}
\hline 

Meaning & Verb & Stem + \emph{a }あ + - \emph{nai desu }\slash  \emph{nakatta desu }ないです・なかったです \\ \cline{1-3}

To talk &  \emph{Hanasu }話す & \emph{Hanasanai desu }話さないです \hfill\break
 \emph{Hanasanakatta desu }話さなかったです \\ \cline{1-3}

To die & \emph{Shinu }死ぬ &  \emph{Shinanai desu }死なないです \hfill\break
 \emph{Shinanakatta desu }死ななかったです \\ \cline{1-3}

To read &  \emph{Yomu }読む & \emph{Yomanai desu }読まないです \hfill\break
 \emph{Yomanakatta desu }読まなかったです \\ \cline{1-3}

\end{ltabulary}
 
\par{\textbf{Usage Note }: These forms cannot modify nouns. To modify nouns with verbs in the negative or negative past, you must use their plain forms - \emph{nai }ない and \emph{-nakatta }なかった respectively.  }

\par{45. 彼女は本を ${\overset{\textnormal{よ}}{\text{読}}}$ まなかったです。 \hfill\break
\emph{Kanojo wa hon wo yomanakatta desu. }\hfill\break
She didn't read a\slash the book. }

\par{46. ${\overset{\textnormal{たぶんし}}{\text{多分死}}}$ なないです。 \hfill\break
\emph{Tabun shinanai desu. \hfill\break
}I probably won\textquotesingle t die. }
 
\par{47. あまり ${\overset{\textnormal{はな}}{\text{話}}}$ さなかったです。 \hfill\break
\emph{Amari hanasanakatta desu. \hfill\break
}I really didn\textquotesingle t talk\slash speak. }

\begin{center}
 \textbf{Conjugation Recap } 
\end{center}

\par{ As review, here are the conjugations you learned in this lesson. Each kind of \emph{Godan }五段 verb is represented with one verb. For brevity, all conjugations will be rendered in \emph{Hiragana }ひらがな. }

\begin{ltabulary}{|P|P|P|P|P|P|}
\hline 

う・く・ぐ・す・つ &  \emph{Kau }買う \hfill\break
(To buy) &  \emph{Kaku }書く \hfill\break
(To write) &  \emph{Oyogu }泳ぐ \hfill\break
(To swim) &  \emph{Hanasu }話す \hfill\break
(To talk) &  \emph{Matsu }待つ \hfill\break
(To wait) \\ \cline{1-6}

Plain Non-Past & かう & かく & およぐ & はなす & まつ \\ \cline{1-6}

Polite Non-Past & かいます & かきます & およぎます & はなします & まちます \\ \cline{1-6}

Plain Past & かった & かいた & およいだ & はなした & まった \\ \cline{1-6}

Polite Past & かいました & かきました & およぎました & はなしました & まちました \\ \cline{1-6}

Plain Neg. & かわない & かかない & およがない & はなさない & またない \\ \cline{1-6}

Polite Neg. 1 & かわないです & かかないです & およがないです & はなさないです & またないです \\ \cline{1-6}

Polite Neg. 2 & かいません & かきません & およぎません & はなしません & まちません \\ \cline{1-6}

Plain Neg-Past & かわなかった & かかなかった & およがなかった & はなさなかった & またなかった \\ \cline{1-6}

Polite Neg-Past 1 & かわなかったです & かかなかったです & およがなかったです & はなさなかったです & またなかったです \\ \cline{1-6}

Polite Neg-Past 2 & かいませんでした & かきませんでした & およぎませんでした & はなしませんでした & まちませんでした \\ \cline{1-6}

ぶ・む・ぬ・る &  \emph{Yobu }呼ぶ \hfill\break
(To call) &  \emph{Yomu }読む \hfill\break
(To read) &  \emph{Shinu }死ぬ \hfill\break
(To die) &  \emph{Horu }掘る \hfill\break
(To dig) &  \\ \cline{1-6}

Plain Non-Past & よぶ & よむ & しぬ & ほる &  \\ \cline{1-6}

Polite Non-Past & よびます & よみます & しにます & ほります &  \\ \cline{1-6}

Plain Past & よんだ & よんだ & しんだ & ほった &  \\ \cline{1-6}

Polite Past & よびました & よみました & しにました & ほりました &  \\ \cline{1-6}

Plain Neg. & よばない & よまない & しなない & ほらない &  \\ \cline{1-6}

Polite Neg. 1 & よばないです & よまないです & しなないです & ほらないです &  \\ \cline{1-6}

Polite Neg. 2 & よびません & よみません & しにません & ほりません &  \\ \cline{1-6}

Plain Neg-Past & よばなかった & よまなかった & しななかった & ほらなかった &  \\ \cline{1-6}

Polite Neg-Past 1 & よばなかったです & よまなかったです & しななかったです & ほらなかったです &  \\ \cline{1-6}

Polite Neg-Past 2 & よびませんでした & よみませんでした & しにませんでした & ほりませんでした &  \\ \cline{1-6}

\end{ltabulary}

\begin{center}
\textbf{Conjugation Recap } 
\end{center}

\par{ As review, here are the conjugations you learned in this lesson with the verbs \emph{miru }見る (to see) and \emph{taberu }食べる (to eat). }

\begin{ltabulary}{|P|P|P|P|}
\hline 

Verb Form & Conjugation &  \emph{Miru }見る (To see) &  \emph{Taberu }食べる (To eat) \\ \cline{1-4}

Plain Non-Past & N\slash A &  \emph{Miru }見る &  \emph{Taberu }食べる \\ \cline{1-4}

Polite Non-Past & - \emph{masu }ます &  \emph{Mimasu }見ます &  \emph{Tabemasu }食べます \\ \cline{1-4}

Plain Past & - \emph{ta }た &  \emph{Mita }見た &  \emph{Tabeta }食べた \\ \cline{1-4}

Polite Past & - \emph{mashita }ました &  \emph{Mimashita }見ました &  \emph{Tabemashita }食べました \\ \cline{1-4}

Plain Negative & - \emph{nai }ない &  \emph{Minai }見ない &  \emph{Tabenai }食べない \\ \cline{1-4}

Polite Neg. 1 & - \emph{nai desu } \hfill\break
ないです &  \emph{Minai desu \hfill\break
}見ないです &  \emph{Tabenai desu } \hfill\break
食べないです \\ \cline{1-4}

Polite Neg. 2 & - \emph{masen }ません &  \emph{Mimasen }見ません &  \emph{Tabemasen }食べません \\ \cline{1-4}

Plain Neg-Past & - \emph{nakatta }なかった &  \emph{Minakatta }見なかった &  \emph{Tabenakatta }食べなかった \\ \cline{1-4}

Polite Neg-Past 1 & - \emph{nakatta desu } \hfill\break
なかったです &  \emph{Minakatta desu } \hfill\break
見なかったです &  \emph{Tabenakatta desu } \hfill\break
食べなかったです \\ \cline{1-4}

Polite Neg-Past 2 & - \emph{masendeshita \hfill\break
}ませんでした &  \emph{Mimasendeshita \hfill\break
}見ませんでした &  \emph{Tabemasendeshita \hfill\break
}食べませんでした \\ \cline{1-4}

\end{ltabulary}
 
\begin{center}
\textbf{Conjugation Recap } 
\end{center}

\par{ As review, here are the conjugations you learned in this lesson with the verbs \emph{miru }見る (to see) and \emph{taberu }食べる (to eat). }

\begin{ltabulary}{|P|P|P|P|}
\hline 

Verb Form & Conjugation &  \emph{Miru }見る (To see) &  \emph{Taberu }食べる (To eat) \\ \cline{1-4}

Plain Non-Past & N\slash A &  \emph{Miru }見る &  \emph{Taberu }食べる \\ \cline{1-4}

Polite Non-Past & - \emph{masu }ます &  \emph{Mimasu }見ます &  \emph{Tabemasu }食べます \\ \cline{1-4}

Plain Past & - \emph{ta }た &  \emph{Mita }見た &  \emph{Tabeta }食べた \\ \cline{1-4}

Polite Past & - \emph{mashita }ました &  \emph{Mimashita }見ました &  \emph{Tabemashita }食べました \\ \cline{1-4}

Plain Negative & - \emph{nai }ない &  \emph{Minai }見ない &  \emph{Tabenai }食べない \\ \cline{1-4}

Polite Neg. 1 & - \emph{nai desu } \hfill\break
ないです &  \emph{Minai desu \hfill\break
}見ないです &  \emph{Tabenai desu } \hfill\break
食べないです \\ \cline{1-4}

Polite Neg. 2 & - \emph{masen }ません &  \emph{Mimasen }見ません &  \emph{Tabemasen }食べません \\ \cline{1-4}

Plain Neg-Past & - \emph{nakatta }なかった &  \emph{Minakatta }見なかった &  \emph{Tabenakatta }食べなかった \\ \cline{1-4}

Polite Neg-Past 1 & - \emph{nakatta desu } \hfill\break
なかったです &  \emph{Minakatta desu } \hfill\break
見なかったです &  \emph{Tabenakatta desu } \hfill\break
食べなかったです \\ \cline{1-4}

Polite Neg-Past 2 & - \emph{masendeshita \hfill\break
}ませんでした &  \emph{Mimasendeshita \hfill\break
}見ませんでした &  \emph{Tabemasendeshita \hfill\break
}食べませんでした \\ \cline{1-4}

\end{ltabulary}
       
\section{Godan Verbs that Unfortunately Look Like Ichidan Verbs}
 
\par{ After having learned all about both kinds of verbs, the sad realization that sometimes they look exactly like each other hits you like a bowling ball the face. For verbs that end in "eru" or "iru," you won't be able to initially tell what kind of verb they are until you see at least one conjugation. Usually, verbs that are homophonous almost always have different pitches as well. }

\par{ There is at least one bit of good news from this debacle, which is that the fact there are two classes of verbs helps distinguish verbs that sound alike in their base forms. }

\begin{ltabulary}{|P|P|P|P|}
\hline 

\emph{ }Meaning &  \emph{Ichidan }Verb & Meaning &  \emph{Godan }Verb \\ \cline{1-4}

To change &  \emph{Kae.ru }変える & To go home &  \emph{Kaer.u }帰る \\ \cline{1-4}

To wear &  \emph{Ki.ru }着る & To cut &  \emph{Kir.u }切る \\ \cline{1-4}

\end{ltabulary}

\par{ If a verb ends in \emph{-ru }る but the vowel preceding it is an \slash a\slash , \slash u\slash , or \slash o\slash , it will always be a \emph{Godan }五段 verb. It's also the case that a lot of verbs won't sound like an \emph{Ichidan }一段 verb. All the verbs below are \emph{Godan }五段 verbs. }

\begin{ltabulary}{|P|P|P|P|P|P|}
\hline 

To run & \emph{Hashiru }走る & To chatter & \emph{Shaberu }喋る & To decrease &  \emph{Heru }減る \\ \cline{1-6}

To be proud &  \emph{Hokoru }誇る & To rub & \emph{Kosuru }擦る & To touch &  \emph{Sawaru }触る \\ \cline{1-6}

To enter &  \emph{Hairu }入る & To slip\slash slide &  \emph{Suberu }滑る & To grasp & \emph{Nigiru }握る \\ \cline{1-6}

To change (intr.) &  \emph{Kawaru }変わる & To limit\slash restrict &  \emph{Kagiru }限る & To fish\slash lure &  \emph{Tsuru }釣る \\ \cline{1-6}

To kick &  \emph{Keru }蹴る & To ridicule & \emph{Azakeru }嘲る & To twist & \emph{Hineru }捻る \\ \cline{1-6}

\end{ltabulary}

\par{48. ${\overset{\textnormal{でんわ}}{\text{電話}}}$ を ${\overset{\textnormal{き}}{\text{切}}}$ りました。 \hfill\break
\emph{Denwa wo kirimashita. \hfill\break
}I hung up the phone. }

\par{49. ${\overset{\textnormal{かのじょ}}{\text{彼女}}}$ は ${\overset{\textnormal{きもの}}{\text{着物}}}$ を ${\overset{\textnormal{き}}{\text{着}}}$ ました。 \hfill\break
\emph{Kanojo wa kimono wo kimashita. \hfill\break
}She wore a kimono. }
 
\par{50. スタイルを ${\overset{\textnormal{か}}{\text{変}}}$ えませんでした。 \hfill\break
\emph{Sutairu wo kaemasendeshita. \hfill\break
}I didn\textquotesingle t change the style. }

\par{51. ${\overset{\textnormal{かれ}}{\text{彼}}}$ は ${\overset{\textnormal{かえ}}{\text{帰}}}$ りませんでした。 \hfill\break
\emph{Kare wa kaerimasendeshita. \hfill\break
}He didn\textquotesingle t return\slash go home. }

\par{52. ${\overset{\textnormal{くるま}}{\text{車}}}$ を ${\overset{\textnormal{こす}}{\text{擦}}}$ る。 \hfill\break
\emph{Kuruma wo kosuru. \hfill\break
}To scratch a car. }
 
\par{53. この ${\overset{\textnormal{まち}}{\text{街}}}$ はずっと ${\overset{\textnormal{か}}{\text{変}}}$ わらない。 \hfill\break
\emph{Kono machi wa zutto kawaranai. }\hfill\break
This town won\textquotesingle t ever change. }
 
\par{54. たくさん ${\overset{\textnormal{さかな}}{\text{魚}}}$ を ${\overset{\textnormal{つ}}{\text{釣}}}$ った。 \hfill\break
\emph{Takusan sakana wo tsutta. \hfill\break
}I caught a lot of fish. }

\par{55. ${\overset{\textnormal{たす}}{\text{助}}}$ けは ${\overset{\textnormal{い}}{\text{要}}}$ らない。 \hfill\break
\emph{Tasuke wa iranai. }\hfill\break
I don't need help. }
    