    
\chapter*{No Phrases}

\begin{center}
\begin{Large}
第??課: No Phrases: いいえ, いえ, いや, 否, \& ううん 
\end{Large}
\end{center}
 
\par{ The words that will be taught in this lesson are as follows. There will be a lot of overlap in how they\textquotesingle re used, but it is important nonetheless to see how each of them are used. }

\par{・いいえ \hfill\break
・いえ \hfill\break
・いや \hfill\break
・否 \hfill\break
・ううん }

\par{ Japanese culture is known for its favoring of indirect expressions. That is not to say that a direct and affirmative “no” can\textquotesingle t be expressed in Japanese. To the contrary, the most basic words for “no” are all capable of expressing opposition, disapproval, or rejecting. It is because of Japanese\textquotesingle s sensitivity to social circumstances that causes all “no” expressions to be multifaceted and seemingly ambiguous. }
      
\section{いいえ}
 
\par{ The antonym of はい is いいえ. By default, it is the direct translation of “no.” It is the correct word for any non-verbal “no” you may encounter. For example, a “yes” and “no” window will display はい (Y) and いいえ (N). いいえ is also the most formal and polite of all the variants of it. Although this point will be important to remember as we learn more about these phrases, understand that this is not an absolute and that there will be times when using いいえ may cause an unwanted reaction. }

\par{Usage 1: To express negation when presented a question that seeks affirmative or negative response (yes-no question). This is the most basic understanding of いいえ. }

\par{1. 「 ${\overset{\textnormal{ゆき}}{\text{雪}}}$ はまだ ${\overset{\textnormal{ふ}}{\text{降}}}$ っていますか?」「いいえ、 ${\overset{\textnormal{に}}{\text{2}}}$ ${\overset{\textnormal{じかんまえ}}{\text{時間前}}}$ に ${\overset{\textnormal{や}}{\text{止}}}$ みました。」 \hfill\break
“Is it still snowing?” “No, it stopped snowing two hours ago.” }

\par{2. \hfill\break
 ${\overset{\textnormal{じょうし}}{\text{上司}}}$ : ${\overset{\textnormal{おれ}}{\text{俺}}}$ 、こうやれって ${\overset{\textnormal{い}}{\text{言}}}$ った? \hfill\break
 ${\overset{\textnormal{ぶか}}{\text{部下}}}$ :いいえ、(そう ${\overset{\textnormal{い}}{\text{言}}}$ ってません)。 \hfill\break
Boss: “Did I say for you to do it this way?” \hfill\break
Subordinate: “No(, you did not).” }

\par{\textbf{Usage Note }: At face value, it seems that the responder is replying politely. However, such a response, especially to a confrontational yes-no question would be conversely inflammatory and undoubtedly stir animosity. This is because a direct denial, even if the response is the truth, is not demonstrating the responsibility of fixing the perceived problem that the questioner is seeking. To stop confrontation, adding fuel to the fire is counterproductive. Ex. 3 would be an appropriate use of いいえ whilst still being polite in this sort of circumstance. }

\par{3. いいえ、 ${\overset{\textnormal{かちょう}}{\text{課長}}}$ はそんな ${\overset{\textnormal{ふう}}{\text{風}}}$ には ${\overset{\textnormal{おっしゃ}}{\text{仰}}}$ っていません。私が ${\overset{\textnormal{まちが}}{\text{間違}}}$ えたのです。 ${\overset{\textnormal{たいへんもう}}{\text{大変申}}}$ し ${\overset{\textnormal{わけ}}{\text{訳}}}$ ございません。 \hfill\break
No, that is not how you said (to do it), Chief. I was mistaken. I am terribly sorry. \hfill\break
 \hfill\break
\textbf{Sentence Note }: Practically speaking, this response would not necessarily come to mind. A more expedient answer to the question that would still be polite would be Ex. 4. }

\par{4. すみません、 ${\overset{\textnormal{ちが}}{\text{違}}}$ っていたでそしょうか。 \hfill\break
I\textquotesingle m sorry, was I incorrect? }

\par{\textbf{Sentence Note }: This response would bring about a response that would inevitably result in admitting personal responsibility, which is what Ex. 3 explicitly states from the beginning. }

\par{Usage 2: Declining someone\textquotesingle s proposal\slash request— “no thank you.” }

\par{5.「コーヒーお ${\overset{\textnormal{か}}{\text{代}}}$ わりはいかがですか。」「いいえ、 ${\overset{\textnormal{けっこう}}{\text{結構}}}$ です。」 \hfill\break
“Would you like a coffee refill?” “No thank you, I\textquotesingle m fine.” }

\par{\textbf{Usage Note }: Many speakers, especially young people, tend to use 大丈夫です for “no thanks,” as it is less direct than saying いいえ、結構です. In formal, business situations, いいえ、結構です is preferred as a certain degree of directness is expected. Additionally, colloquialisms are typically frowned upon in such situations. The same can be said for the English-speaking world. }

\par{Usage 3: Negating someone\textquotesingle s idea\slash statement. }

\par{6. 「 ${\overset{\textnormal{きょう}}{\text{今日}}}$ のことは ${\overset{\textnormal{ぐうぜん}}{\text{偶然}}}$ なんだ」「いいえ、そんなはずないと ${\overset{\textnormal{おも}}{\text{思}}}$ うわ」 \hfill\break
“Today\textquotesingle s incident was just a coincidence.” “No, that can\textquotesingle t be true.” }

\par{Usage 4: Rejecting the premise of a question. }

\par{7. 「 ${\overset{\textnormal{かあ}}{\text{母}}}$ さん、 ${\overset{\textnormal{ぼく}}{\text{僕}}}$ の ${\overset{\textnormal{かばんみ}}{\text{鞄見}}}$ なかったね」「いいえ、 ${\overset{\textnormal{し}}{\text{知}}}$ らないよ」 \hfill\break
“Mom, you wouldn\textquotesingle t happen to have seen my bag, huh?” “No, I don\textquotesingle t know anything about it.” }

\par{8.「 ${\overset{\textnormal{きみ}}{\text{君}}}$ 、きのうは ${\overset{\textnormal{しゅっしゃ}}{\text{出社}}}$ しなかったか」「いいえ、いつものとおり ${\overset{\textnormal{しゅっしゃ}}{\text{出社}}}$ しましたよ」 \hfill\break
“Hey, did you not come to work yesterday? “No, I came just as usual.” }

\par{Usage 5: Indicating one has no intentions of furthering the conversation or intends to end the conversation upon being pressed for explanation. }

\par{9. 「 ${\overset{\textnormal{なに}}{\text{何}}}$ がいまにわかるのだ」「いいえ、なんでもありません」 \hfill\break
"What exactly am I going to find out soon enough?” “Oh no, it\textquotesingle s nothing.” }

\par{Usage 6: Rejecting thanks and praise out of courtesy. }

\par{10. 「 ${\overset{\textnormal{ほんとう}}{\text{本当}}}$ に ${\overset{\textnormal{たす}}{\text{助}}}$ かります、ありがとうございます。」「いいえ、とんでもないですよ。」 \hfill\break
“I really appreciate it. Thank you.” “Oh no, it\textquotesingle s nothing.” }

\par{11. 「ありがとうございます」「いいえ、どういたしまして」 \hfill\break
“Thank you.” “Oh no, you\textquotesingle re welcome.” }
      
\section{いえ \& いや}
 
\par{ Both いえ and いや can be seen as variants of いいえ, but they have morphed to be somewhat different in their unique ways. Intrinsically, they carry all six usages seen above, but when it\textquotesingle s appropriate to use them will determine how the other person interprets your response. }

\par{12. 「 ${\overset{\textnormal{やましたくん}}{\text{山下君}}}$ も、 ${\overset{\textnormal{えんりょ}}{\text{遠慮}}}$ しないで ${\overset{\textnormal{の}}{\text{飲}}}$ みなさい」「いえ、 ${\overset{\textnormal{けっこう}}{\text{結構}}}$ です」 \hfill\break
“Yamashita-kun, don\textquotesingle t hold back and drink.” “Oh no, I\textquotesingle m fine.” }

\par{13. いえいえ、お ${\overset{\textnormal{やく}}{\text{役}}}$ に ${\overset{\textnormal{た}}{\text{立}}}$ てれば ${\overset{\textnormal{なに}}{\text{何}}}$ よりです。 \hfill\break
No, no, what counts more than anything is that I am of use to you. }

\begin{center}
\textbf{Negating Presentation of Information }
\end{center}

\par{ A situation where いえ and いや can be used interchangeably in plain or polite speech but cannot be replaced by いいえ is when one is negating and\slash or criticizing how information is being presented or gently yet affirmatively correcting someone. }

\par{14. 「 ${\overset{\textnormal{たいよう}}{\text{太陽}}}$ は ${\overset{\textnormal{ちきゅう}}{\text{地球}}}$ の ${\overset{\textnormal{まわ}}{\text{周}}}$ りを ${\overset{\textnormal{まわ}}{\text{回}}}$ ってるって ${\overset{\textnormal{し}}{\text{知}}}$ ってますか?」「\{いえ・いや\}、 ${\overset{\textnormal{ちきゅう}}{\text{地球}}}$ が ${\overset{\textnormal{たいよう}}{\text{太陽}}}$ の ${\overset{\textnormal{まわ}}{\text{周}}}$ りを ${\overset{\textnormal{まわ}}{\text{回}}}$ ってるんですよ」 \hfill\break
“Did you know that the Sun revolves around the Earth?” “Um, actually, the Earth revolves around the Sun.” }

\par{\textbf{Sentence Note }: In this situation, the responder to the first person\textquotesingle s statement is rejecting the information trying to be shared. In such circumstances, both いえ and いや are appropriate. }

\par{15. 「どうしたの? ${\overset{\textnormal{かおいろ}}{\text{顔色}}}$ が ${\overset{\textnormal{わる}}{\text{悪}}}$ いよ。」「いや、なんでもない。」 \hfill\break
“What\textquotesingle s wrong? You\textquotesingle re all pale.” “Oh no, it\textquotesingle s nothing.” }

\par{\textbf{Sentence Note }: In this example, it isn\textquotesingle t that the responder is negating that he\slash she is pale. The responder is negating the question itself which implies that something is up. }

\par{16. }

\par{${\overset{\textnormal{エイ}}{\text{A}}}$ ${\overset{\textnormal{し}}{\text{氏}}}$ :あれ、どこ ${\overset{\textnormal{い}}{\text{行}}}$ くの? \hfill\break
 ${\overset{\textnormal{ビー}}{\text{B}}}$ ${\overset{\textnormal{し}}{\text{氏}}}$ :いえ、ちょっと ${\overset{\textnormal{きゅう}}{\text{急}}}$ な ${\overset{\textnormal{ようじ}}{\text{用事}}}$ が ${\overset{\textnormal{はい}}{\text{入}}}$ ってさ・・・。 \hfill\break
Person A: Huh? Where are you going? \hfill\break
Person B: Oh no, you see, there\textquotesingle s a small urgent matter that\textquotesingle s come up… }

\par{17. }

\par{${\overset{\textnormal{エイ}}{\text{A}}}$ ${\overset{\textnormal{し}}{\text{氏}}}$ :なんでそんな ${\overset{\textnormal{はなし}}{\text{話}}}$ をするの? \hfill\break
 ${\overset{\textnormal{ビー}}{\text{B}}}$ ${\overset{\textnormal{し}}{\text{氏}}}$ :いえね、 ${\overset{\textnormal{じつ}}{\text{実}}}$ を ${\overset{\textnormal{い}}{\text{言}}}$ うと・・・。 \hfill\break
Person A: Why are you talking about something like that? \hfill\break
Person B: Oh no, you see, the thing is…. }

\par{\textbf{Particle Note }: Unlike いいえ, いえ and いや can both be followed by the final particle ね. }

\par{18. いやね、やっぱりシェアハウスなんてするものじゃないわ。 \hfill\break
Oh no, share houses and what not are definitely something not to do. }

\par{\textbf{Variation Note }: いや can alternatively sometimes be heard as いーや. }

\par{19. 「では、 ${\overset{\textnormal{しゅっせき}}{\text{出席}}}$ を ${\overset{\textnormal{と}}{\text{取}}}$ ります。 ${\overset{\textnormal{おざわ}}{\text{小沢}}}$ さん」「いや、あのー、 ${\overset{\textnormal{しらいし}}{\text{白石}}}$ ${\overset{\textnormal{せんせい}}{\text{先生}}}$ のクラスは ${\overset{\textnormal{となり}}{\text{隣}}}$ ですよ」 \hfill\break
“Alright, I\textquotesingle m going to take attendance. Ozawa-san…” “Hey, um… Shiraishi Sensei, your class is the one next over.” }

\par{\textbf{Usage Note }: Whenever one is simply negating how information is being presented but not negating the truth of the statement, then only いや becomes appropriate. }

\par{20. 「 ${\overset{\textnormal{よる}}{\text{夜}}}$ ご ${\overset{\textnormal{はん}}{\text{飯}}}$ は ${\overset{\textnormal{なにた}}{\text{何食}}}$ べようか」「いや、 ${\overset{\textnormal{いま}}{\text{今}}}$ は ${\overset{\textnormal{かんけい}}{\text{関係}}}$ ないでしょ、その ${\overset{\textnormal{はなし}}{\text{話}}}$ は」 \hfill\break
“What\textquotesingle ll we eat for dinner?” “Uh, that has nothing to do with right now.” }

\par{21. 「 ${\overset{\textnormal{はや}}{\text{早}}}$ すぎる、といいますと?」「いやあ、これは ${\overset{\textnormal{くち}}{\text{口}}}$ が ${\overset{\textnormal{すべ}}{\text{滑}}}$ りました。いや、あなた ${\overset{\textnormal{がた}}{\text{方}}}$ に ${\overset{\textnormal{みょう}}{\text{妙}}}$ に ${\overset{\textnormal{かんぐ}}{\text{勘繰}}}$ られると ${\overset{\textnormal{こま}}{\text{困}}}$ りますから、 ${\overset{\textnormal{しゃくめい}}{\text{釈明}}}$ しますがね」 \hfill\break
“When you say ‘too soon\textquotesingle …?” “Ahh, I made a slip of the tongue. Well, since it\textquotesingle d be troublesome if this were oddly suspected by you all, I\textquotesingle ll explain things.” \hfill\break
From 繁昌するメス by 松本清張. }

\par{\textbf{Variation Note }: Elongated as いやー, いや can express a feeling of having gone too forward toward oneself. }

\begin{center}
\textbf{Politeness Difference }
\end{center}

\par{ Generally speaking, いいえ and いえ are both seen as typically being more appropriate in polite speech than いや. }

\par{22. 「 ${\overset{\textnormal{やまのてせん}}{\text{山手線}}}$ に ${\overset{\textnormal{の}}{\text{乗}}}$ るんですか」「\{いえ 〇・いいえ 〇・いや X\}、 ${\overset{\textnormal{ちが}}{\text{違}}}$ います」 \hfill\break
“Do you ride the Yamanote Line” “No, that\textquotesingle s not the one.” }

\par{23. 「エクセル ${\overset{\textnormal{つか}}{\text{使}}}$ うの?」「\{いや 〇・いえ X・いいえ X\}、 ${\overset{\textnormal{ちが}}{\text{違}}}$ う。エクセルは不要だね。」 \hfill\break
“Do you use Excel? “No, Excel isn\textquotesingle t needed.” }

\par{24.「 ${\overset{\textnormal{はんにん}}{\text{犯人}}}$ は ${\overset{\textnormal{あ}}{\text{挙}}}$ がったのか」「いや、まだだ」 \hfill\break
“Do you got the criminal arrested?” “No, not yet.” }

\par{ However, it is not always the case that いや can\textquotesingle t be used in polite speech. For instance, it can be used just like いいえ and いえ when rejecting thanks\slash praise, and using it instead can bring about a more familial sense. }

\par{25. 「 ${\overset{\textnormal{かんけん}}{\text{漢検}}}$ ${\overset{\textnormal{いっ}}{\text{1}}}$ ${\overset{\textnormal{きゅうごうかく}}{\text{級合格}}}$ しんたんだって?おめでとう!」「いや、まあ、おかげさまで、 ${\overset{\textnormal{ぶじ}}{\text{無事}}}$ に。」 \hfill\break
“So you passed Kanken Level 1? Congrats!” “Oh no, well, thankfully it went all fine.” }

\begin{center}
\textbf{Usages Unique to いや }
\end{center}

\par{ At times, いや isn\textquotesingle t just a variant of いえ but a contraction of 嫌だ, and it can be seen elongated as いーや, but unlike the いや that\textquotesingle s equivalent to “no,” the intonation drops sharply after the mora for this usage. }

\par{26. 「お ${\overset{\textnormal{かねか}}{\text{金貸}}}$ してくれない?」「いや(だ)!」 \hfill\break
“Could you lend me some money?” “No way!” }

\par{ Only いや can be used when talking to oneself, and this is for any capacity of self-directed commentary, even when one is talking to someone. }

\par{27. ${\overset{\textnormal{ゆうびんきょく}}{\text{郵便局}}}$ はですね、えーと、そこを ${\overset{\textnormal{うせつ}}{\text{右折}}}$ して ${\overset{\textnormal{ふた}}{\text{2}}}$ つ ${\overset{\textnormal{め}}{\text{目}}}$ の、いや、 ${\overset{\textnormal{みっ}}{\text{3}}}$ つ ${\overset{\textnormal{め}}{\text{目}}}$ の ${\overset{\textnormal{しんごう}}{\text{信号}}}$ を ${\overset{\textnormal{うせつ}}{\text{右折}}}$ して、しばらく ${\overset{\textnormal{い}}{\text{行}}}$ くと、右手のほうにあります。 \hfill\break
The post office, um… if you make a right there and then on the second, no, the third light take a right, and then keep going for a while, it\textquotesingle ll be on the right. }
      
\section{否}
 
\par{ A variant of these two unique senses of いや is 否. This negation word is most often seen in the grammar point か否か meaning “whether or not,” but it also finds itself synonymous with いやだ or いや, although it is somewhat old-fashioned in these regards. It also happens to be used in the literary sense of “nay.” }

\par{28. ${\overset{\textnormal{げんじょう}}{\text{現状}}}$ では ${\overset{\textnormal{ざんねん}}{\text{残念}}}$ ながら ${\overset{\textnormal{いな}}{\text{否}}}$ と ${\overset{\textnormal{こた}}{\text{答}}}$ えるほかない。 \hfill\break
Unfortunately under the circumstances, there is no other alternative but to answer with “no.” }

\par{29. ${\overset{\textnormal{いな}}{\text{否}}}$ 、 ${\overset{\textnormal{だん}}{\text{断}}}$ じて ${\overset{\textnormal{いな}}{\text{否}}}$ ! \hfill\break
No, absolutely not! }

\par{30. ${\overset{\textnormal{かいしゃ}}{\text{会社}}}$ のため、 ${\overset{\textnormal{いな}}{\text{否}}}$ 、 ${\overset{\textnormal{しゃかい}}{\text{社会}}}$ に ${\overset{\textnormal{こうけん}}{\text{貢献}}}$ するために ${\overset{\textnormal{わ}}{\text{我}}}$ が ${\overset{\textnormal{み}}{\text{身}}}$ を ${\overset{\textnormal{ぎせい}}{\text{犠牲}}}$ にしている。 \hfill\break
I\textquotesingle m sacrificing myself for the company, no, to contribute to society. }
      
\section{ううん}
 
\par{ In casual conversation, ううん can be used to mean “no.” It holds the six basic meanings of いいえ that were discussed earlier but solely within the confines of casual, plain speech. In addition to those meanings, it can also be used as an interjection similar to “uh” or as an interjection indicating strain \slash struggling. }

\par{\textbf{Intonation Note }: To mean “no,” pitch rises as the end. }

\par{31. ううん、もう ${\overset{\textnormal{だめ}}{\text{駄目}}}$ だ。 \hfill\break
No, it\textquotesingle s no use. }

\par{32. ううん、そんなことないよ? \hfill\break
No, that\textquotesingle s not true\dothyp{}\dothyp{}\dothyp{}? }

\par{33. 「 ${\overset{\textnormal{けいこ}}{\text{恵子}}}$ はもう ${\overset{\textnormal{き}}{\text{来}}}$ たの?」「ううん、まだ ${\overset{\textnormal{き}}{\text{来}}}$ てないよ」 \hfill\break
“Has Keiko already here?” “No, she hasn\textquotesingle t come yet.” }

\par{34. ううん、 ${\overset{\textnormal{なん}}{\text{何}}}$ だっけ。 \hfill\break
Er, what was it? }

\par{35. ううん、いいよ。 \hfill\break
No, it\textquotesingle s fine. }
    