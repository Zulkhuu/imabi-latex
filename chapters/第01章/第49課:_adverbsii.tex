    
\chapter{Adverbs II}

\begin{center}
\begin{Large}
第49課: Adverbs II: From Adjectives 
\end{Large}
\end{center}
 
\par{ This lesson is about how you can create adverbs out of any adjective in Japanese. }
      
\section{Adjective \textrightarrow  Adverb}
 
\par{ To make an adjective an adverb, drop い and add く. For ${\overset{\textnormal{けいようどうし}}{\text{形容動詞}}}$ , you add に. These adverbs are normally translated with "-ly". However, for whenever English is weird and doesn't let us use -ly, we have to go with another translation. }

\begin{ltabulary}{|P|P|P|P|P|P|P|P|}
\hline 

Adjective &  & Adverb &  & Adjective &  & Adverb &  \\ \cline{1-8}

強い & Strong & 強く & Strongly & 静か(な) & Quiet & 静かに & Quietly \\ \cline{1-8}

弱い & Weak & 弱く & Weakly & かんたん(な) & Easy & かんたんに & Easily \\ \cline{1-8}

遅い & Late & 遅く & Late & まじめ(な) & Serious & まじめに & Seriously \\ \cline{1-8}

小さい & Small & 小さく & Small & きれい(な) & Pretty; nice & きれいに & Nicely \\ \cline{1-8}

\end{ltabulary}

\begin{center}
 \textbf{Example Sentences }
\end{center}

\par{1. ${\overset{\textnormal{かんたん}}{\text{簡単}}}$ に ${\overset{\textnormal{せつめい}}{\text{説明}}}$ する。 \hfill\break
To easily explain. }

\par{2. 楽しく休日を ${\overset{\textnormal{す}}{\text{過}}}$ ごす。 \hfill\break
To spend the holidays merrily. }

\par{3. 何かを小さく切る。 \hfill\break
To cut something into small pieces. }

\par{4. わたしは毎日自分の部屋で一人静かに勉強します。 \hfill\break
I study in my own room quietly every day. }

\par{5. インディゴで ${\overset{\textnormal{せんめんだい}}{\text{洗面台}}}$ が青く ${\overset{\textnormal{そ}}{\text{染}}}$ まった。 \hfill\break
The washbasin was dyed blue with indigo. }

\par{6. あの(向こう)の電気が赤く ${\overset{\textnormal{かがや}}{\text{輝}}}$ いた。 \hfill\break
That light over there shined red. }

\par{\textbf{Phrase Note }: あの is literally translated in English as "that over there" as it refers to something not directly near the speaker or listener. However, when the object of reference is truly literally "over there" as in on the other side of the speaker and listener, あの向こうの is more appropriate. }

\par{7. ${\overset{\textnormal{あね}}{\text{姉}}}$ は ${\overset{\textnormal{やさ}}{\text{優}}}$ しくなりました。 \hfill\break
My older sister became nice. }

\par{\textbf{Grammar Notes }: \hfill\break
1. When we want to say that something "becomes X" but X is actually an adjectival attribute, we turn the adjective into an adverb and then add なる. So, "to become red" is 赤くなる. 赤くになる is wrong. However, you say きれいになる for "to become pretty". Remember that there are two classes of adjectives and that they always conjugate differently. \hfill\break
2. Continuing on 2 , ${\overset{\textnormal{しんごう}}{\text{信号}}}$ が青くなった actually means "the light turned green". 青, not ${\overset{\textnormal{みどり}}{\text{緑}}}$ , is the color used for streetlights for "green". }

\par{8. もう少し静かにしてください。 \hfill\break
Please be more quiet }

\par{\textbf{Grammar Note }: The opposite of "(adjectival) adverb + なる" is "(Adjectival) adverb + する", which means "to make\dothyp{}\dothyp{}\dothyp{}" as in implementing a change. }

\par{9. 冬には ${\overset{\textnormal{たいよう}}{\text{太陽}}}$ は早く ${\overset{\textnormal{しず}}{\text{沈}}}$ む。 \hfill\break
The sun sets early in the winter. }

\par{10. ${\overset{\textnormal{ふか}}{\text{深}}}$ く ${\overset{\textnormal{いき}}{\text{息}}}$ を ${\overset{\textnormal{す}}{\text{吸}}}$ う。 \hfill\break
To take a deep breath. }

\begin{center}
\textbf{${\overset{\textnormal{たし}}{\text{確}}}$ \textbf{か VS }\textbf{確かに }}
\end{center}

\par{確か means "certain", but as an adverb, it's often paired with だろう・でしょう to mean "if I'm not mistaken". The other adverb form 確かに means "certainly". So, they're slightly different. }

\par{11. ${\overset{\textnormal{せかい}}{\text{世界}}}$ の人口は、確か70億(人)ぐらいだ(った)と思います。 \hfill\break
The population of the world, if I'm not mistaken is around 7 billion people. }

\par{12. 確かにその木が ${\overset{\textnormal{たお}}{\text{倒}}}$ れるでしょう。 \hfill\break
The tree will certainly fall. }

\par{13a. 確か(か)? (Very casual and a little blunt) \hfill\break
13b. 確かですか。 \hfill\break
Is that for certain? }

\par{14. それは確かな証拠ではないです。 \hfill\break
That is not definitive evidence. }

\par{\textbf{Nuance Note }: 多分 is less certain than 確か and きっと is more certain than 確かに. }

\begin{center}
 \textbf{Exceptional Phrases }
\end{center}

\par{ Not all adverb phrases will be made similarly. For example, ${\overset{\textnormal{ひつ}}{\text{必}}}$ ${\overset{\textnormal{よう}}{\text{要}}}$ に is not used. "Necessarily" is instead ${\overset{\textnormal{かなら}}{\text{必}}}$ ず. Another example is けっこう. Although it too is a 形容動詞, it's adverbial form is just けっこう. Lastly, we have 少しく meaning  "just a little". The word comes from when 少し was an adjective. It is occasionally used in the written language. }

\par{15. 今日はけっこう寒いです。 \hfill\break
Today is quite cold. }

\par{16. ${\overset{\textnormal{すこ}}{\text{少}}}$ しく思うところを ${\overset{\textnormal{の}}{\text{述}}}$ べる。(書き言葉) \hfill\break
To state a little bit of what you think. }

\begin{center}
 \textbf{Ends in に but not from 形容詞 }
\end{center}

\par{ Even though an adverb may end in に, this doesn't mean it necessarily comes from an adjective. Though this is usually the case, there are still very commonly used exceptions to this. }

\par{17. ${\overset{\textnormal{げんじょう}}{\text{現状}}}$ は ${\overset{\textnormal{ただ}}{\text{直}}}$ ちに問題はありません。 \hfill\break
There are no problems present right now. }

\par{18. レンタカーで直ちに ${\overset{\textnormal{くうこう}}{\text{空港}}}$ を出発しました。 \hfill\break
(I\slash we) immediately left the airport in a rental car. }
    