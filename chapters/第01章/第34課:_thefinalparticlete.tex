    
\chapter{The Final Particle て}

\begin{center}
\begin{Large}
第34課: The Final Particle て 
\end{Large}
\end{center}
 
\par{ Particle classification is very important to keep in mind as you learn more about particles. The particle て discussed in this lesson is a final particle and mustn't be confused with the conjunctive one. Although the conjunctive て may be at the end of a sentence in the instance that the remaining part of a sentence is not said, overall context should always help you differentiate between the two. }
      
\section{The Final Particle て}
 
\par{ The main purpose of the final particle て is to \textbf{make a light command }. This is the contraction of ~てください, the polite command. Likewise, the negative of this is ~ないで(ください), making the absence of ください a contracted form of the pattern. Thus, its absence for both the affirmative and negative makes it go from polite to plain speech. A more polite form of both can be made by changing ください to くださいませんか. }

\par{1. ちょっと ${\overset{\textnormal{}}{\text{待}}}$ って。(Casual) \hfill\break
Hold on. }

\par{2. ${\overset{\textnormal{はら}}{\text{払}}}$ い ${\overset{\textnormal{もど}}{\text{戻}}}$ ししてください。 \hfill\break
Please refund this. }

\par{3. ${\overset{\textnormal{たす}}{\text{助}}}$ けて! \hfill\break
Help! }

\par{${\overset{\textnormal{}}{\text{4.   教科書を\{閉じて・しまって\}}}}$ テストを受けてください。 \hfill\break
Please do your test by closing your textbooks. }

\par{5. やめて! \hfill\break
Quit it! }

\par{6. これを ${\overset{\textnormal{}}{\text{見}}}$ て。 \hfill\break
Look at this. }

\par{7. タクシーを ${\overset{\textnormal{よ}}{\text{呼}}}$ んでください。 \hfill\break
Please call a taxi (for me). }

\par{8. はやく ${\overset{\textnormal{いそ}}{\text{急}}}$ いで! \hfill\break
Hurry quickly! }

\par{9. もっとゆっくりと ${\overset{\textnormal{}}{\text{話}}}$ してください。 \hfill\break
Please speak more slowly. }

\par{10. ${\overset{\textnormal{はや}}{\text{速}}}$ くしてよ! \hfill\break
(Do it) faster! }

\par{11. まっすぐ ${\overset{\textnormal{}}{\text{行}}}$ ってください。 \hfill\break
Please go straight. }

\par{12. ${\overset{\textnormal{ざぶとん}}{\text{座布団}}}$ を ${\overset{\textnormal{し}}{\text{敷}}}$ いてください。 \hfill\break
Please sit on a cushion. }

\par{\textbf{Culture Note }: 座布団 are floor cushions used instead of chairs in traditional Japanese rooms. }

\par{\textbf{Contraction Note }: ~てて is the contraction of ~ていて. ~ている means "-ing". ~ていて creates a command similar to "be\dothyp{}\dothyp{}\dothyp{}-ing!". }

\par{ In women's speech, ~(っ)てよ asserts opinion. However, this phrase has essentially disappeared in the younger generations and is most likely to be used by older women or seen in literature dating back a few decades ago. It is replaced by things such as ~てるよ. }

\par{13. あたくし、ちっとも ${\overset{\textnormal{よ}}{\text{酔}}}$ って なんかいなくてよ。 \hfill\break
I'm not even the least bit drunk. \hfill\break
From 永すぎた春 by 三島由紀夫. }

\par{\textbf{Particle Note }: なんか is essentially a filler word here. }
14. かまわなくってよ。 \hfill\break
I don't care. 
\par{ ~て, with a high intonation, can make a question. These two usages, though, are \textbf{hackneyed yet refined }. This, too, would be replaced with something like ~てる?. }
15. あなた、 ${\overset{\textnormal{}}{\text{私}}}$ のいうことが ${\overset{\textnormal{わ}}{\text{分}}}$ かって? \hfill\break
Do you understand what I'm saying? 
\par{ It is to note that ~てよ is used by everyone to make a command. The sound of your voice is what matters. }

\par{ ~て may also follow things like だ・じゃ to \textbf{tell someone to do }something or \textbf{give some sort of instruction\slash warning }. This, though, is uncommon. }

\par{16. ${\overset{\textnormal{たいへん}}{\text{大変}}}$ なことじゃて。(Old person; dialectical) \hfill\break
That's a horrible thing! }
    