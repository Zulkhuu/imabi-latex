    
\chapter{The Particle から}

\begin{center}
\begin{Large}
第46課: The Particle から 
\end{Large}
\end{center}
 
\par{ In this lesson we will learn how to say "from" with the particle から. }
      
\section{The Case Particle から}
 
\par{ から means "from". Just as in English it may mark a point of transit or starting, place of departure, starting time, order, in XからYまで to mean "from\dothyp{}\dothyp{}\dothyp{}to\dothyp{}\dothyp{}\dothyp{}," cause as in "due to," and is also the "from" used with "change," "made," "receive," "hear from," etc. This mainly goes after nominal phrases, but it is also possible for it to come after temporal adverbial expressions. }

\par{1. 左から    (左 is a noun) \hfill\break
From the left }

\par{2. ${\overset{\textnormal{せいごま}}{\text{生後間}}}$ もなくから  (間もなく is adverbial) \hfill\break
Not long after birth }

\par{\textbf{Particle Notes }: }

\par{1. As for "receive," から can show receiving from anything whereas に is limited to people. Thus, one could say that に shows a closer relationship as it is limited to human interaction. }
 
\par{2. When used to show the place of transit or departure, it may be replaced by を. }

\begin{center}
 \textbf{Examples }
\end{center}

\par{3. ${\overset{\textnormal{たいよう}}{\text{太陽}}}$ は ${\overset{\textnormal{ひがし}}{\text{東}}}$ から ${\overset{\textnormal{のぼ}}{\text{昇}}}$ ります。 \hfill\break
The sun rises in the east. }

\par{4. ${\overset{\textnormal{すきま}}{\text{隙間}}}$ から ${\overset{\textnormal{ほうしゃせいぶっしつ}}{\text{放射性物質}}}$ が ${\overset{\textnormal{も}}{\text{漏}}}$ れる ${\overset{\textnormal{おそ}}{\text{恐}}}$ れがあります。 \hfill\break
There is a fear that radioactive material is leaking from an opening. }

\par{5. ${\overset{\textnormal{しんげん}}{\text{震源}}}$ は ${\overset{\textnormal{せんだい}}{\text{仙台}}}$ から ${\overset{\textnormal{}}{\text{数}}}$ キロ ${\overset{\textnormal{はな}}{\text{離}}}$ れています。 \hfill\break
The epicenter is a couple kilometers from Sendai. }

\par{6. ヨーロッパからブラジルへ船で行きました。 \hfill\break
I went to Brazil from Europe by boat. }

\par{${\overset{\textnormal{}}{\text{7. 日本語}}}$ の ${\overset{\textnormal{じゅぎょう}}{\text{授業}}}$ は ${\overset{\textnormal{}}{\text{朝}}}$ 9時から ${\overset{\textnormal{}}{\text{始}}}$ まります。 \hfill\break
Japanese class begins at 9 o'clock in the morning. }

\par{8. ${\overset{\textnormal{かいぎ}}{\text{会議}}}$ は3 ${\overset{\textnormal{}}{\text{時}}}$ から ${\overset{\textnormal{}}{\text{始}}}$ まります。 \hfill\break
The meeting will begin at 3 o' clock. }
 
\par{${\overset{\textnormal{}}{\text{9. 来年}}}$ から ${\overset{\textnormal{かいし}}{\text{開始}}}$ します。 \hfill\break
We will start next year. }
 
\par{${\overset{\textnormal{}}{\text{10. 子犬}}}$ から ${\overset{\textnormal{そだ}}{\text{育}}}$ てたよ。 \hfill\break
I raised it from (since it was) a puppy. }
 
\par{${\overset{\textnormal{}}{\text{11. 彼女}}}$ はウェイトレスからそのレストランの ${\overset{\textnormal{しはいにん}}{\text{支配人}}}$ になりました。 \hfill\break
She rose from a waitress to become the manager of that restaurant. }
 
\par{12. ハーバート・フーヴァーは ${\overset{\textnormal{しょうむちょうかん}}{\text{商務長官}}}$ から ${\overset{\textnormal{だいとうりょう}}{\text{大統領}}}$ になりました。 \hfill\break
Herbert Hoover rose from the Secretary of Commerce to the presidency. }
 
\par{13. その ${\overset{\textnormal{きじ}}{\text{記事}}}$ を ${\overset{\textnormal{すみ}}{\text{隅}}}$ から ${\overset{\textnormal{}}{\text{隅}}}$ まで ${\overset{\textnormal{}}{\text{読}}}$ んだ。 \hfill\break
I read that article from beginning to the end. }

\par{14. ${\overset{\textnormal{にさい}}{\text{二歳}}}$ から ${\overset{\textnormal{ろくさい}}{\text{六歳}}}$ までの ${\overset{\textnormal{こども}}{\text{子供}}}$ がいた。 \hfill\break
There were children from two to six years old. }
 
\par{15. この ${\overset{\textnormal{}}{\text{川}}}$ からあの ${\overset{\textnormal{}}{\text{山}}}$ まで ${\overset{\textnormal{}}{\text{走}}}$ れ! \hfill\break
Run from this river to the mountain over there! }
 
\par{${\overset{\textnormal{}}{\text{16. 彼}}}$ は ${\overset{\textnormal{せきにんかん}}{\text{責任感}}}$ から ${\overset{\textnormal{じしょく}}{\text{辞職}}}$ した。 \hfill\break
He resigned from office due to a sense of responsibility. }

\par{17. ${\overset{\textnormal{さいばんかん}}{\text{裁判官}}}$ は ${\overset{\textnormal{めつ}}{\text{目付}}}$ きから ${\overset{\textnormal{かがいしゃ}}{\text{加害者}}}$ の ${\overset{\textnormal{はんだん}}{\text{判断}}}$ をした。 \hfill\break
The judge judged the assailant from his looks. }

\par{18. ${\overset{\textnormal{おとな}}{\text{大人}}}$ から ${\overset{\textnormal{しんさつ}}{\text{診察}}}$ します。 \hfill\break
I will examine people starting from the adults. }

\par{19. これらのコンピューターは ${\overset{\textnormal{かくしゅ}}{\text{各種}}}$ 5万 ${\overset{\textnormal{}}{\text{円}}}$ からです。 \hfill\break
These computers start at 50,000 yen. }
 
\par{20. パンは ${\overset{\textnormal{こむぎこ}}{\text{小麦粉}}}$ から ${\overset{\textnormal{つく}}{\text{作}}}$ る。 \hfill\break
You make bread out of\slash from flour. }

\par{21. 彼は ${\overset{\textnormal{ふちゅうい}}{\text{不注意}}}$ で ${\overset{\textnormal{かいだん}}{\text{階段}}}$ から ${\overset{\textnormal{お}}{\text{落}}}$ ちた。 \hfill\break
He fell down the staircase due to carelessness. }

\par{22. ${\overset{\textnormal{だいざ}}{\text{台座}}}$ を ${\overset{\textnormal{ぎん}}{\text{銀}}}$ から ${\overset{\textnormal{}}{\text{作}}}$ る。 \hfill\break
To make a pedestal from silver. }
 
\par{${\overset{\textnormal{}}{\text{23. 日本}}}$ は ${\overset{\textnormal{}}{\text{北海道}}}$ 、 ${\overset{\textnormal{}}{\text{本州}}}$ 、 ${\overset{\textnormal{}}{\text{九州}}}$ 、 ${\overset{\textnormal{}}{\text{四国}}}$ という4つの ${\overset{\textnormal{}}{\text{大}}}$ きい ${\overset{\textnormal{}}{\text{島}}}$ からなっています。 \hfill\break
Japan consists of four large islands Hokkaido, Honshu, Kyushu, and Shikoku. }
 
\par{24. その ${\overset{\textnormal{}}{\text{話}}}$ は ${\overset{\textnormal{}}{\text{先生}}}$ から ${\overset{\textnormal{}}{\text{聞}}}$ きました。 \hfill\break
I heard that story from my teacher. }
 
\par{\textbf{Particle Note }: から is interchangeable with に in the sense of receiving. }

\par{25. ${\overset{\textnormal{きのう}}{\text{昨日}}}$ からずっとここにいる。 \hfill\break
I have been here since yesterday. }

\par{26. ${\overset{\textnormal{はんげき}}{\text{反撃}}}$ が ${\overset{\textnormal{みなみ}}{\text{南}}}$ から来た。 \hfill\break
A counterattack came from the south. }

\par{27. ${\overset{\textnormal{ふちゅうい}}{\text{不注意}}}$ からくる失敗 \hfill\break
 Failure (coming) from carelessness }
    