    
\chapter{The Particle と}

\begin{center}
\begin{Large}
第40課: The Particle と 
\end{Large}
\end{center}
 
\par{ There is a lot to know about the particle と. However, this lesson will only focus on the case particle と. If you don't quite remember what a case particle is, don't worry. These particles simply show basic grammatical relationships in a sentence. }
      
\section{The Case Particle と}
 
\begin{center}
 \textbf{Noun+と+Noun(と) }
\end{center}

\par{ と's basic meaning is "and" when placed \textbf{in between nominal phrases }. \textbf{ }If you want to use something else, it has to become a noun first. \textbf{ }と \textbf{can't }be used at the beginning of a sentence . In that case, you should use something like そして. When listing more than two things, consecutive と may be omitted, and this is usually the most natural thing to do. と is typically not used after the last item, but it can be in older language. }

\par{1a. 私は日本人と医者です X \hfill\break
1b. 私は日本人 \textbf{で }医者です。 〇 \hfill\break
I am Japanese and a doctor. }

\par{\textbf{Grammar Note }: Again, this meaning is for when と is between two or more \emph{\textbf{noun }}s! 1a is wrong because this rule is violated in a more complicated way. In English "I am Japanese and a doctor" is technically short for "I am Japanese and I am a doctor." Here it is clear that you are actually connecting two predicate phrases, not two simple noun phrases. If you wanted to say that "X and Y are doctors," you would use と. }

\par{2a. すばしっこいと茶色の ${\overset{\textnormal{きつね}}{\text{狐}}}$ はのろまな犬を ${\overset{\textnormal{と}}{\text{飛}}}$ び ${\overset{\textnormal{こ}}{\text{越}}}$ える。 X \hfill\break
2b. すばしっこい茶色の狐はのろまな犬を飛び越える。〇 \hfill\break
The quick brown fox jumps over the lazy dog. }

\par{\textbf{Example Note }: 2b is grammatically correct, but because it is a translation of a sentence in English that has each letter of the alphabet in it, the Japanese sounds somewhat like a direct translation. For instance, this sentence has an unnecessarily lengthy subject. In Japanese, such lengthy subjects tend to sound unnatural. }

\begin{center}
\textbf{Examples } 
\end{center}
 
\par{3. リンゴとブドウがテーブルの ${\overset{\textnormal{}}{\text{上}}}$ にあります。 \hfill\break
There are apples and grapes on the table. }
 
\par{4. これとそれは同じです。 \hfill\break
This and that are the same. }
 
\par{${\overset{\textnormal{}}{\text{5. ノート}}}$ と ${\overset{\textnormal{きょうかしょ}}{\text{教科書}}}$ と ${\overset{\textnormal{じしょ}}{\text{辞書}}}$ を持ってきてください。 \hfill\break
Please come with your notes, textbook, and dictionary. }
 
\par{${\overset{\textnormal{}}{\text{6. 犬}}}$ と ${\overset{\textnormal{}}{\text{猫}}}$ がいる。 \hfill\break
I have a dog and a cat; there is a dog and a cat. \hfill\break
${\overset{\textnormal{}}{\text{7. 犬}}}$ と ${\overset{\textnormal{}}{\text{猫}}}$ を ${\overset{\textnormal{か}}{\text{飼}}}$ っている。 \hfill\break
I have a dog and a cat. }
 
\par{${\overset{\textnormal{}}{\text{8. 日本語}}}$ のクラスにはアメリカ ${\overset{\textnormal{}}{\text{人}}}$ と、イギリス ${\overset{\textnormal{}}{\text{人}}}$ と、カナダ ${\overset{\textnormal{}}{\text{人}}}$ がいます。 \hfill\break
There are Americans, English, and Canadians in (my) \emph{Japanese }class. }
 
\par{${\overset{\textnormal{}}{\text{9. 朝}}}$ と ${\overset{\textnormal{}}{\text{夜}}}$ はちょっと ${\overset{\textnormal{}}{\text{寒}}}$ いです。 \hfill\break
Morning and night are a little cold. }

\par{\textbf{Particle Note }: This pattern used to be XとY(と) for "X and Y." If you ever see something like XとYとが, you're not seeing a typo. It's just not frequently used. }

\begin{center}
 \textbf{With }
\end{center}
 
\par{ と may show the partner in which a person of interest is doing an \emph{action } \textbf{with }. Both the "and" and "with" definitions can be in the same sentence as they're not the same and have different requirements. The word after と when it means "with" is not part of the same phrase as と. }
 
\par{${\overset{\textnormal{}}{\text{10. \{}}}$ 両親・父と母\}と ${\overset{\textnormal{}}{\text{公園}}}$ に ${\overset{\textnormal{}}{\text{行}}}$ きます。 \hfill\break
I'm going to the park with my father and mother. }
 
\par{${\overset{\textnormal{}}{\text{11a. 犬}}}$ と ${\overset{\textnormal{さんぽ}}{\text{散歩}}}$ に ${\overset{\textnormal{}}{\text{出}}}$ かけた。 \hfill\break
11b.  犬の散歩に出かけた。 \hfill\break
I went out on a walk with my dog. }

\par{\textbf{Nuance Note }: Both 11a and 11b are grammatical, but 11a sounds as if you are elevating the dog to the same level of importance as yourself whereas 11b means that you are simply taking the dog out for its walk. }
 
\par{ と ${\overset{\textnormal{いっしょ}}{\text{一緒}}}$ に and と ${\overset{\textnormal{とも}}{\text{共}}}$ に both may mean "together" or "jointly". と一緒に is only used with two compatible nouns that are same in status and kind. You can't use it in "I talked together with my teacher". ご一緒する is honorific but means "to go with". とともに is more formal and shows that things\slash people are doing something or in the same state together. }
 ${\overset{\textnormal{}}{\text{12. 彼女}}}$ と ${\overset{\textnormal{}}{\text{彼}}}$ は ${\overset{\textnormal{}}{\text{結婚}}}$ する。 (Don't need 一緒に) \hfill\break
He and she will marry (each other). 
\par{${\overset{\textnormal{}}{\text{13. 宏}}}$ と ${\overset{\textnormal{}}{\text{花子は同時}}}$ に ${\overset{\textnormal{けっこん}}{\text{結婚}}}$ する。 \hfill\break
Hiroshi and Hanako will marry simultaneously. }
 
\par{${\overset{\textnormal{}}{\text{14. 彼}}}$ らは ${\overset{\textnormal{}}{\text{一緒}}}$ に ${\overset{\textnormal{てき}}{\text{敵}}}$ と ${\overset{\textnormal{たたか}}{\text{戦}}}$ っていた。 \hfill\break
They fought the enemy together. }
 
\par{${\overset{\textnormal{}}{\text{15. 彼}}}$ らは敵と一緒に ${\overset{\textnormal{}}{\text{戦}}}$ っていた。 \hfill\break
They fought mutually with the enemy. }
 
\par{16. あたしは ${\overset{\textnormal{}}{\text{日}}}$ の ${\overset{\textnormal{で}}{\text{出}}}$ とともに ${\overset{\textnormal{お}}{\text{起}}}$ きます。(Feminine) \hfill\break
I rise with the sun. }
 
\par{${\overset{\textnormal{}}{\text{17.  夫婦ともに元気です。}}}$ \hfill\break
The husband and wife are both fine. }
 
\par{18a. 誰かと悲しみを共にすることはおかしなことではない。(Lyrical\slash poetic) \hfill\break
18b. 誰かと悲しみを共有することはおかしなことではない。(Normal) \hfill\break
It is not strange to share sadness with someone. }
 
\par{19. (あなたと) ${\overset{\textnormal{}}{\text{一緒}}}$ に ${\overset{\textnormal{}}{\text{行}}}$ きます。 \hfill\break
I am going together (with you). }
 
\par{20. あの ${\overset{\textnormal{}}{\text{二人}}}$ はいつも ${\overset{\textnormal{}}{\text{一緒}}}$ です。 \hfill\break
Those two are always together. }

\par{21. 私は父と電話で話しました。 \hfill\break
I talked with my father on the telephone. }

\par{22. ${\overset{\textnormal{かれし}}{\text{彼氏}}}$ と ${\overset{\textnormal{}}{\text{一緒}}}$ に ${\overset{\textnormal{しゅくだい}}{\text{宿題}}}$ をした。 \hfill\break
I did homework together with my boyfriend. }

\par{23a. ${\overset{\textnormal{}}{\text{同}}}$ じ ${\overset{\textnormal{はね}}{\text{羽}}}$ の ${\overset{\textnormal{}}{\text{鳥}}}$ は ${\overset{\textnormal{}}{\text{一緒}}}$ に ${\overset{\textnormal{あつ}}{\text{集}}}$ まる。 \hfill\break
23b. ${\overset{\textnormal{るい}}{\text{類}}}$ は ${\overset{\textnormal{とも}}{\text{友}}}$ を ${\overset{\textnormal{}}{\text{呼}}}$ ぶ。 \hfill\break
Birds of a feather flock together. }
 
\par{${\overset{\textnormal{}}{\text{24. 一緒}}}$ に ${\overset{\textnormal{}}{\text{来}}}$ ませんか。 \hfill\break
Won't you come with me? }
 
\par{${\overset{\textnormal{}}{\text{25. 彼}}}$ は ${\overset{\textnormal{きょうふ}}{\text{恐怖}}}$ と ${\overset{\textnormal{}}{\text{戦}}}$ って、ついに ${\overset{\textnormal{}}{\text{勝}}}$ った。 \hfill\break
He fought against his fears, and he finally won. }
 
\par{${\overset{\textnormal{}}{\text{26. 彼女}}}$ と ${\overset{\textnormal{}}{\text{時々けんか}}}$ する。 \hfill\break
I sometimes argue with her. }
 
\par{${\overset{\textnormal{}}{\text{27. 車}}}$ とぶつかった。 \hfill\break
I collided with a car. }
 
\par{${\overset{\textnormal{}}{\text{28. 子供}}}$ と ${\overset{\textnormal{}}{\text{遊}}}$ ぶ。 \hfill\break
To play with a child. }
 
\par{${\overset{\textnormal{}}{\text{29. 彼}}}$ と ${\overset{\textnormal{しょくじ}}{\text{食事}}}$ をともにした。 \hfill\break
I had dinner with him. }

\par{30. ${\overset{\textnormal{ともだち}}{\text{友達}}}$ と ${\overset{\textnormal{}}{\text{会}}}$ う。 \hfill\break
To meet with a friend. }
 
\par{\textbf{Grammar Note }: と ${\overset{\textnormal{}}{\text{会}}}$ う shows that both sides move (to see one another) while に ${\overset{\textnormal{}}{\text{会}}}$ う shows that only one party moves, thus leading to meeting the person. Due to this discrepancy, for "to happen to meet someone\slash encounter someone," you can see X\{と・に\}ばったり(と)[会う・出会う]. }

\begin{center}
\textbf{Comparing }
\end{center}
 
\par{ と may mark a subject being compared. As for ${\overset{\textnormal{に}}{\text{似}}}$ る, ~に似る and ~と似る are possible but slightly different. と in this case marks one side of a mutual relation(ship) whereas に only shows the standard of comparison. They both, though, make the second person the basis of comparison when the pattern is XはY\{に・と\}似ている. Consider the following.  }

\par{31a その父と子は似ている。〇 \hfill\break
 ${\overset{\textnormal{}}{\text{31b.その子}}}$ はお ${\overset{\textnormal{}}{\text{父さん}}}$ と ${\overset{\textnormal{}}{\text{似}}}$ ている。 ${\overset{\textnormal{}}{\text{〇}}}$ \hfill\break
 ${\overset{\textnormal{}}{\text{31c. その子}}}$ はお ${\overset{\textnormal{}}{\text{父さん}}}$ に ${\overset{\textnormal{}}{\text{似}}}$ ている。 ${\overset{\textnormal{}}{\text{〇}}}$ \hfill\break
 ${\overset{\textnormal{}}{\text{31d. その}}}$ ${\overset{\textnormal{}}{\text{父}}}$ は息子 と ${\overset{\textnormal{}}{\text{似}}}$ ている。 〇 \hfill\break
${\overset{\textnormal{}}{\text{31e. その父}}}$ は息子 に ${\overset{\textnormal{}}{\text{似}}}$ ている。  X \hfill\break
The child resembles his father. }
 
\par{32a.この ${\overset{\textnormal{じけん}}{\text{事件}}}$ は ${\overset{\textnormal{いぜん}}{\text{以前}}}$ とは ${\overset{\textnormal{こと}}{\text{異}}}$ なる。 \hfill\break
 ${\overset{\textnormal{}}{\text{32b.以前}}}$ の ${\overset{\textnormal{}}{\text{事件}}}$ とは ${\overset{\textnormal{}}{\text{異}}}$ なる。 \hfill\break
 ${\overset{\textnormal{}}{\text{32c. 以前}}}$ の ${\overset{\textnormal{}}{\text{事件}}}$ とは ${\overset{\textnormal{ちが}}{\text{違}}}$ う。 \hfill\break
The case is different from before. }
 
\par{33. Xと ${\overset{\textnormal{くら}}{\text{比}}}$ べる \hfill\break
To compare with X. }
 
\par{${\overset{\textnormal{}}{\text{34. 彼}}}$ と ${\overset{\textnormal{}}{\text{同}}}$ じ ${\overset{\textnormal{}}{\text{考}}}$ えです。 \hfill\break
That is the same idea as his. }
 
\par{${\overset{\textnormal{}}{\text{35. 人間}}}$ と ${\overset{\textnormal{どうぶつ}}{\text{動物}}}$ との ${\overset{\textnormal{}}{\text{違}}}$ い \hfill\break
The difference between humans and animals. }

\par{\textbf{Particle Note }: The second instance of と here is not optional. The different と are different. The first is "and" and the second is "with" used here to show comparison, or in this case the opposite of comparison. }

\par{36. スペイン語と日本語はまったく似ていません。 \hfill\break
The Spanish and Japanese don't resemble\slash match each other at all. }

\par{\textbf{Particle Note }: We've seen XはYと似ている and XはYに似ている, but XとYは似ている exists too. If the referents weren't mentioned, then you'd need something like お互いに for "each other", but you don't use the word every time you use 似ている. }

\begin{center}
 \textbf{Content of a Result }
\end{center}
 
\par{ と and に are sometimes similar, but there is always a difference. と may show the result of something. It's used with verbs like ${\overset{\textnormal{}}{\text{決}}}$ める (to decide) and なる (to become). As for なる, になる and となる are possible. However, になる shows an \textbf{end point }of some change. Therefore, there is a duration to it. As for となる, it shows the content\slash substance of a result. }

\par{ と shows a \textbf{discrete }change. This means that it shows the result of something not from a continuous process. と often sounds more stiff and formal. Another verb where there is this contrast is ${\overset{\textnormal{あらた}}{\text{改}}}$ める (to reform; revise). This means that there are some phrases where と doesn't make a lot of sense. }
 
\par{37. いよいよ ${\overset{\textnormal{うんどうかい}}{\text{運動会}}}$ の ${\overset{\textnormal{}}{\text{日}}}$ となりました。 \hfill\break
Field day has come at last. }

\par{38. ${\overset{\textnormal{きまつ}}{\text{期末}}}$ ${\overset{\textnormal{しけん}}{\text{試験}}}$ は ${\overset{\textnormal{にしゅうかんご}}{\text{二週間後}}}$ と ${\overset{\textnormal{}}{\text{決}}}$ まった。 \hfill\break
It's been decided that the final test will be in 2 weeks. }

\par{39. ${\overset{\textnormal{こさめ}}{\text{小雨}}}$ となった。 \hfill\break
It became a drizzle. }

\par{40a. 元気となる。△\slash X \hfill\break
40b. 元気になる。〇 \hfill\break
To become better. }
 
\par{41. みぞれが雪となりました。 \hfill\break
The sleet turned to snow. }

\par{\textbf{漢字 Note }: みぞれ is rarely written in 漢字 as 霙. }
    