    
\chapter{Kosoado こそあど II Here \& There}

\begin{center}
\begin{Large}
第24課: Kosoado こそあど II: Here \& There: Koko ここ, Soko そこ, \& Asoko あそこ 
\end{Large}
\end{center}
 
\par{ In this lesson, we will learn about the \emph{kosoado }こそあど words that describe location\slash situation. The dynamics involved as for which to use are exactly the same as they were for “this” and “that.” }
      
\section{Here: Koko ここ}
 
\par{ Word of warning, this word does NOT mean cocoa as in cocoa puffs. As tempting as that might sound, it is most certainly the word for “here.” \emph{Koko }ここ refers to a location\slash situation that is in close proximity\slash association with the speaker and listener(s). }

\par{1. ここは ${\overset{\textnormal{きょうしつ}}{\text{教室}}}$ です。 \hfill\break
 \emph{Koko wa kyōshitsu desu. }\hfill\break
Here is a\slash the classroom. }

\par{2. ここは ${\overset{\textnormal{かまた}}{\text{蒲田}}}$ です。 \hfill\break
 \emph{Koko wa Kamata desu. }\hfill\break
This is Kamata. }

\par{\textbf{Sentence Note }: Sometimes in English, “this” is used instead of “here” for the same purpose. However, in Japanese, \emph{koko }ここ remains the word of choice. }

\par{\textbf{Location Note }: \emph{Kamata }蒲田 is a neighborhood in Ōta Ward ( \emph{Ōta-ku }大田区) of Tokyo, Japan. }

\par{3. ここのラーメンは ${\overset{\textnormal{あじ}}{\text{味}}}$ がうまいです。 \hfill\break
 \emph{Koko no rāmen wa aji ga umai desu. }\hfill\break
The ramen here has a delicious taste. }

\par{\textbf{Grammar Note }: There are no unique changes to \emph{kosoado }こそあど for location to make them adjectival. All you do is add the particle \emph{no }の after. }

\par{\textbf{Phrase Note }: \emph{Umai }うまい is another way of saying “delicious.” However, the word is not as polite or refined as \emph{oishii }美味しい.  This is because the latter originated from refined feminine speech which eventually became widely used by sex and ages. Although this is the case, \emph{umai }うまい can still be used in casual yet polite speech as is shown in this example. }

\par{4. ${\overset{\textnormal{ごたんだえき}}{\text{五反田駅}}}$ はここら ${\overset{\textnormal{へん}}{\text{辺}}}$ でしたよね。 \hfill\break
 \emph{Gotanda-eki wa kokorahen deshita yo ne? }\hfill\break
Gotanda Station was around here, wasn\textquotesingle t it? }

\par{\textbf{Particle Note }: The particles \emph{yo }よ and \emph{ne }ね are used together at the end of the sentence to express direct seeking of confirmation from the listener. }

\par{\textbf{Tense Note }: The use of the past tense here is not literal. Instead, it is used in part to seek confirmation, just as is the case in the English translation. }

\par{\textbf{Suffix Note }: The suffix \emph{-rahen }ら辺  may be added to any of the \emph{kosoado }こそあど phrases mentioned in this lesson to add the nuance “about.” }
      
\section{There: Soko そこ}
 
\par{ The word for “there” in Japanese is \emph{soko }そこ. It is "there" as in a location in close proximity to the listener. When neither speaker nor listener is talking about a place in proximity, then the place indicated by \emph{soko }そこ is one that just one party is fully aware of. Conversely, \emph{soko }そこ is a situation that both listener and speaker are aware of, but the degree to which they are involved will likely not be equal. }

\par{5. そこは ${\overset{\textnormal{かいだん}}{\text{階段}}}$ です。 \hfill\break
\emph{Soko wa kaidan desu. }\hfill\break
There is a\slash the staircase there. }

\par{6. そこはどこですか。 \hfill\break
\emph{Soko wa doko desu ka? }\hfill\break
Where is that? }

\par{\textbf{Sentence Note }: In this example, it is English that is odd. Instead of referring to "there" with “there,” the word “that” is used. However, this is a problem with English and not Japanese, as this example demonstrates. }

\par{7. そこが ${\overset{\textnormal{むずか}}{\text{難}}}$ しいところですね。 \hfill\break
\emph{Soko ga muzukashii tokoro desu ne. }\hfill\break
Yeah, \emph{that }\textquotesingle s the difficult part. }

\par{\textbf{Sentence Note }: In this example, both the speaker and listener may be heavily involved in whatever is going on. However, it is the listener who must have mentioned how something about it was terribly difficult, and it is the speaker who is simply responding. The tone indicated by \emph{ne }ね in this sentence indicates that the speaker must be less emotionally taxed than the listener. Thus, some distance is to be had in the mind of the speaker. }

\par{8. そこら ${\overset{\textnormal{へん}}{\text{辺}}}$ に ${\overset{\textnormal{お}}{\text{置}}}$ いてください。 \hfill\break
\emph{Sokorahen ni oite kudasai. }\hfill\break
Please place it around there. }

\par{9. そこまで ${\overset{\textnormal{い}}{\text{言}}}$ う ${\overset{\textnormal{ひつよう}}{\text{必要}}}$ はない。 \hfill\break
\emph{Soko made iu hitsuyo wa nai. } \hfill\break
There\textquotesingle s no need to go (talk) that far. }

\par{\textbf{Sentence Note }: \emph{Soko made }そこまで means “to that extent\slash go that far.” This is a perfect example of how “there” doesn\textquotesingle t necessarily have to literally mean “there” but can also mean “that (part\slash extent\slash situation).” }

\par{10. そこのお ${\overset{\textnormal{ねえ}}{\text{姉}}}$ さん、あの、 ${\overset{\textnormal{さいふ}}{\text{財布}}}$ を ${\overset{\textnormal{お}}{\text{落}}}$ としましたよ。 \hfill\break
\emph{Soko no onē-san, ano, saifu wo otoshimashita yo. }\hfill\break
Miss, um, you dropped your wallet. }

\par{\textbf{Sentence Note }: In English, no word indicating the physical proximity of the lady is needed, but in Japanese, it aids in grabbing the lady\textquotesingle s attention. This sentence also demonstrates how the word \emph{ano }あの may be used as an interjection meaning “um.” }

\par{There: \emph{Soko }そこ }

\par{The word for “there” in Japanese is \emph{soko }そこ. It is "there" as in a location in close proximity to the listener. When neither speaker nor listener is talking about a place in proximity, then the place indicated by \emph{soko }そこ is one that just one party is fully aware of. Conversely, \emph{soko }そこ is a situation that both listener and speaker are aware of, but the degree to which they are involved will likely not be equal. }

\par{5. そこは ${\overset{\textnormal{かいだん}}{\text{階段}}}$ です。 \hfill\break
\emph{Soko wa kaidan desu. }\hfill\break
There is a\slash the staircase there. }

\par{6. そこはどこですか。 \hfill\break
\emph{Soko wa doko desu ka? }\hfill\break
Where is that? }

\par{Sentence Note: In this example, it is English that is odd. Instead of referring to "there" with “there,” the word “that” is used. However, this is a problem with English and not Japanese, as this example demonstrates. }

\par{7. そこが ${\overset{\textnormal{むずか}}{\text{難}}}$ しいところですね。 \hfill\break
\emph{Soko ga muzukashii tokoro desu ne. }\hfill\break
Yeah, \emph{that }\textquotesingle s the difficult part. }

\par{Sentence Note: In this example, both the speaker and listener may be heavily involved in whatever is going on. However, it is the listener who must have mentioned how something about it was terribly difficult, and it is the speaker who is simply responding. The tone indicated by \emph{ne }ね in this sentence indicates that the speaker must be less emotionally taxed than the listener. Thus, some distance is to be had in the mind of the speaker. }

\par{8. そこら ${\overset{\textnormal{へん}}{\text{辺}}}$ に ${\overset{\textnormal{お}}{\text{置}}}$ いてください。 \hfill\break
\emph{Sokorahen ni oite kudasai. }\hfill\break
Please place it around there. }

\par{9. そこまで ${\overset{\textnormal{い}}{\text{言}}}$ う ${\overset{\textnormal{ひつよう}}{\text{必要}}}$ はない。 \hfill\break
\emph{Soko made iu hitsuyo wa nai. } \hfill\break
There\textquotesingle s no need to go (talk) that far. }

\par{Sentence Note: Soko madeそこまで means “to that extent\slash go that far.” This is a perfect example of how “there” doesn\textquotesingle t necessarily have to literally mean “there” but can also mean “that (part\slash extent\slash situation).” }

\par{10. そこのお ${\overset{\textnormal{ねえ}}{\text{姉}}}$ さん、あの、 ${\overset{\textnormal{さいふ}}{\text{財布}}}$ を ${\overset{\textnormal{お}}{\text{落}}}$ としましたよ。 \hfill\break
\emph{Soko no onē-san, ano, saifu wo otoshimashita yo. }\hfill\break
Miss, um, you dropped your wallet. }

\par{Sentence Note: In English, no word indicating the physical proximity of the lady is needed, but in Japanese, it aids in grabbing the lady\textquotesingle s attention. This sentence also demonstrates how the word \emph{ano }あの may be used as an interjection meaning “um.” }

\par{(Over) There: \emph{Asoko }あそこ }

\par{In a physical sense, \emph{asoko }あそこ refers to a place away from both the speaker and the listener. When said place, however, is out of eyesight and is being referred to in context, then the place must be known by all parties in the conversation. Of course, this is assumed in natural discourse. Similarly to \emph{soko }そこ, \emph{asoko }あそこ may also refer to a situation that is known by both the speaker and listener, but as for \emph{asoko }あそこ, the situation is of a severe degree. }

\par{11. あそこは ${\overset{\textnormal{じむしつ}}{\text{事務室}}}$ です。 \hfill\break
\emph{Asoko wa jimushitsu desu. }\hfill\break
Over there is the office. }

\par{12. ${\overset{\textnormal{きよこ}}{\text{清子}}}$ さんの ${\overset{\textnormal{かばん}}{\text{鞄}}}$ はあそこにあります。 \hfill\break
\emph{Kiyoko-san no kaban wa asoko ni arimasu. }\hfill\break
Ms. Kiyoko\textquotesingle s bag is over there. }

\par{13. あそこの ${\overset{\textnormal{む}}{\text{向}}}$ こうは ${\overset{\textnormal{ふくおかし}}{\text{福岡市}}}$ ですね。 \hfill\break
\emph{Asoko no mukō wa Fukuoka-shi desu ne. }\hfill\break
Beyond there\slash on the opposite side of there is Fukuoka City, right? }

\par{14. あそこのお ${\overset{\textnormal{まわ}}{\text{巡}}}$ りさんに ${\overset{\textnormal{き}}{\text{聞}}}$ いてください。 \hfill\break
\emph{Asoko no omawari-san ni kiite kudasai. }\hfill\break
Please ask that police officer over there. }

\par{Phrase Note: Although the word for police officer is \emph{kei(satsu)kan }警(察)官, policemen are generally referred to as \emph{omawari-san }お巡りさん. }

\par{15. ${\overset{\textnormal{かのじょ}}{\text{彼女}}}$ もあそこら ${\overset{\textnormal{へん}}{\text{辺}}}$ に ${\overset{\textnormal{す}}{\text{住}}}$ んでいます。 \hfill\break
\emph{Kanojo mo asokorahen ni sunde imasu. }\hfill\break
She too lives around there. }

\par{16. 「 ${\overset{\textnormal{ぎんこう}}{\text{銀行}}}$ はどこですか。」「あそこです。」 \hfill\break
\emph{“Ginkō wa doko desu ka?” “Asoko desu.” \hfill\break
}Where is the bank? }

\par{17. あそこに ${\overset{\textnormal{どうぶつえん}}{\text{動物園}}}$ があります。 \hfill\break
\emph{Asoko ni dōbutsuen ga arimasu. }\hfill\break
There is a zoo over there. }

\par{18. ${\overset{\textnormal{わたし}}{\text{私}}}$ もあそこに ${\overset{\textnormal{かぞく}}{\text{家族}}}$ がいます。 \hfill\break
\emph{Watashi mo asoko ni kazoku ga imasu. }\hfill\break
I too have family there. }

\par{19. そこに ${\overset{\textnormal{しょうがっこう}}{\text{小学校}}}$ があります。 \hfill\break
\emph{Soko ni shōgakkō ga arimasu. }\hfill\break
There is an elementary school there. }

\par{20. あそこが ${\overset{\textnormal{いた}}{\text{痛}}}$ いです。 \hfill\break
\emph{Asoko ga itai desu. }\hfill\break
My private area hurts. }
 
\par{Phrase Note: \emph{Asoko }あそこ may also be used to euphemistically refer to one\textquotesingle s private parts. }

\par{There: \emph{Soko }そこ }

\par{The word for “there” in Japanese is \emph{soko }そこ. It is "there" as in a location in close proximity to the listener. When neither speaker nor listener is talking about a place in proximity, then the place indicated by \emph{soko }そこ is one that just one party is fully aware of. Conversely, \emph{soko }そこ is a situation that both listener and speaker are aware of, but the degree to which they are involved will likely not be equal. }

\par{5. そこは ${\overset{\textnormal{かいだん}}{\text{階段}}}$ です。 \hfill\break
\emph{Soko wa kaidan desu. }\hfill\break
There is a\slash the staircase there. }

\par{6. そこはどこですか。 \hfill\break
\emph{Soko wa doko desu ka? }\hfill\break
Where is that? }

\par{Sentence Note: In this example, it is English that is odd. Instead of referring to "there" with “there,” the word “that” is used. However, this is a problem with English and not Japanese, as this example demonstrates. }

\par{7. そこが ${\overset{\textnormal{むずか}}{\text{難}}}$ しいところですね。 \hfill\break
\emph{Soko ga muzukashii tokoro desu ne. }\hfill\break
Yeah, \emph{that }\textquotesingle s the difficult part. }

\par{Sentence Note: In this example, both the speaker and listener may be heavily involved in whatever is going on. However, it is the listener who must have mentioned how something about it was terribly difficult, and it is the speaker who is simply responding. The tone indicated by \emph{ne }ね in this sentence indicates that the speaker must be less emotionally taxed than the listener. Thus, some distance is to be had in the mind of the speaker. }

\par{8. そこら ${\overset{\textnormal{へん}}{\text{辺}}}$ に ${\overset{\textnormal{お}}{\text{置}}}$ いてください。 \hfill\break
\emph{Sokorahen ni oite kudasai. }\hfill\break
Please place it around there. }

\par{9. そこまで ${\overset{\textnormal{い}}{\text{言}}}$ う ${\overset{\textnormal{ひつよう}}{\text{必要}}}$ はない。 \hfill\break
\emph{Soko made iu hitsuyo wa nai. } \hfill\break
There\textquotesingle s no need to go (talk) that far. }

\par{Sentence Note: Soko madeそこまで means “to that extent\slash go that far.” This is a perfect example of how “there” doesn\textquotesingle t necessarily have to literally mean “there” but can also mean “that (part\slash extent\slash situation).” }

\par{10. そこのお ${\overset{\textnormal{ねえ}}{\text{姉}}}$ さん、あの、 ${\overset{\textnormal{さいふ}}{\text{財布}}}$ を ${\overset{\textnormal{お}}{\text{落}}}$ としましたよ。 \hfill\break
\emph{Soko no onē-san, ano, saifu wo otoshimashita yo. }\hfill\break
Miss, um, you dropped your wallet. }

\par{Sentence Note: In English, no word indicating the physical proximity of the lady is needed, but in Japanese, it aids in grabbing the lady\textquotesingle s attention. This sentence also demonstrates how the word \emph{ano }あの may be used as an interjection meaning “um.” }

\par{(Over) There: \emph{Asoko }あそこ }

\par{In a physical sense, \emph{asoko }あそこ refers to a place away from both the speaker and the listener. When said place, however, is out of eyesight and is being referred to in context, then the place must be known by all parties in the conversation. Of course, this is assumed in natural discourse. Similarly to \emph{soko }そこ, \emph{asoko }あそこ may also refer to a situation that is known by both the speaker and listener, but as for \emph{asoko }あそこ, the situation is of a severe degree. }

\par{11. あそこは ${\overset{\textnormal{じむしつ}}{\text{事務室}}}$ です。 \hfill\break
\emph{Asoko wa jimushitsu desu. }\hfill\break
Over there is the office. }

\par{12. ${\overset{\textnormal{きよこ}}{\text{清子}}}$ さんの ${\overset{\textnormal{かばん}}{\text{鞄}}}$ はあそこにあります。 \hfill\break
\emph{Kiyoko-san no kaban wa asoko ni arimasu. }\hfill\break
Ms. Kiyoko\textquotesingle s bag is over there. }

\par{13. あそこの ${\overset{\textnormal{む}}{\text{向}}}$ こうは ${\overset{\textnormal{ふくおかし}}{\text{福岡市}}}$ ですね。 \hfill\break
\emph{Asoko no mukō wa Fukuoka-shi desu ne. }\hfill\break
Beyond there\slash on the opposite side of there is Fukuoka City, right? }

\par{14. あそこのお ${\overset{\textnormal{まわ}}{\text{巡}}}$ りさんに ${\overset{\textnormal{き}}{\text{聞}}}$ いてください。 \hfill\break
\emph{Asoko no omawari-san ni kiite kudasai. }\hfill\break
Please ask that police officer over there. }

\par{Phrase Note: Although the word for police officer is \emph{kei(satsu)kan }警(察)官, policemen are generally referred to as \emph{omawari-san }お巡りさん. }

\par{15. ${\overset{\textnormal{かのじょ}}{\text{彼女}}}$ もあそこら ${\overset{\textnormal{へん}}{\text{辺}}}$ に ${\overset{\textnormal{す}}{\text{住}}}$ んでいます。 \hfill\break
\emph{Kanojo mo asokorahen ni sunde imasu. }\hfill\break
She too lives around there. }

\par{16. 「 ${\overset{\textnormal{ぎんこう}}{\text{銀行}}}$ はどこですか。」「あそこです。」 \hfill\break
\emph{“Ginkō wa doko desu ka?” “Asoko desu.” \hfill\break
}Where is the bank? }

\par{17. あそこに ${\overset{\textnormal{どうぶつえん}}{\text{動物園}}}$ があります。 \hfill\break
\emph{Asoko ni dōbutsuen ga arimasu. }\hfill\break
There is a zoo over there. }

\par{18. ${\overset{\textnormal{わたし}}{\text{私}}}$ もあそこに ${\overset{\textnormal{かぞく}}{\text{家族}}}$ がいます。 \hfill\break
\emph{Watashi mo asoko ni kazoku ga imasu. }\hfill\break
I too have family there. }

\par{19. そこに ${\overset{\textnormal{しょうがっこう}}{\text{小学校}}}$ があります。 \hfill\break
\emph{Soko ni shōgakkō ga arimasu. }\hfill\break
There is an elementary school there. }

\par{20. あそこが ${\overset{\textnormal{いた}}{\text{痛}}}$ いです。 \hfill\break
\emph{Asoko ga itai desu. }\hfill\break
My private area hurts. }
 
\par{Phrase Note: \emph{Asoko }あそこ may also be used to euphemistically refer to one\textquotesingle s private parts. }

\par{There: \emph{Soko }そこ }

\par{The word for “there” in Japanese is \emph{soko }そこ. It is "there" as in a location in close proximity to the listener. When neither speaker nor listener is talking about a place in proximity, then the place indicated by \emph{soko }そこ is one that just one party is fully aware of. Conversely, \emph{soko }そこ is a situation that both listener and speaker are aware of, but the degree to which they are involved will likely not be equal. }

\par{5. そこは ${\overset{\textnormal{かいだん}}{\text{階段}}}$ です。 \hfill\break
\emph{Soko wa kaidan desu. }\hfill\break
There is a\slash the staircase there. }

\par{6. そこはどこですか。 \hfill\break
\emph{Soko wa doko desu ka? }\hfill\break
Where is that? }

\par{Sentence Note: In this example, it is English that is odd. Instead of referring to "there" with “there,” the word “that” is used. However, this is a problem with English and not Japanese, as this example demonstrates. }

\par{7. そこが ${\overset{\textnormal{むずか}}{\text{難}}}$ しいところですね。 \hfill\break
\emph{Soko ga muzukashii tokoro desu ne. }\hfill\break
Yeah, \emph{that }\textquotesingle s the difficult part. }

\par{Sentence Note: In this example, both the speaker and listener may be heavily involved in whatever is going on. However, it is the listener who must have mentioned how something about it was terribly difficult, and it is the speaker who is simply responding. The tone indicated by \emph{ne }ね in this sentence indicates that the speaker must be less emotionally taxed than the listener. Thus, some distance is to be had in the mind of the speaker. }

\par{8. そこら ${\overset{\textnormal{へん}}{\text{辺}}}$ に ${\overset{\textnormal{お}}{\text{置}}}$ いてください。 \hfill\break
\emph{Sokorahen ni oite kudasai. }\hfill\break
Please place it around there. }

\par{9. そこまで ${\overset{\textnormal{い}}{\text{言}}}$ う ${\overset{\textnormal{ひつよう}}{\text{必要}}}$ はない。 \hfill\break
\emph{Soko made iu hitsuyo wa nai. } \hfill\break
There\textquotesingle s no need to go (talk) that far. }

\par{Sentence Note: Soko madeそこまで means “to that extent\slash go that far.” This is a perfect example of how “there” doesn\textquotesingle t necessarily have to literally mean “there” but can also mean “that (part\slash extent\slash situation).” }

\par{10. そこのお ${\overset{\textnormal{ねえ}}{\text{姉}}}$ さん、あの、 ${\overset{\textnormal{さいふ}}{\text{財布}}}$ を ${\overset{\textnormal{お}}{\text{落}}}$ としましたよ。 \hfill\break
\emph{Soko no onē-san, ano, saifu wo otoshimashita yo. }\hfill\break
Miss, um, you dropped your wallet. }

\par{Sentence Note: In English, no word indicating the physical proximity of the lady is needed, but in Japanese, it aids in grabbing the lady\textquotesingle s attention. This sentence also demonstrates how the word \emph{ano }あの may be used as an interjection meaning “um.” }

\par{(Over) There: \emph{Asoko }あそこ }

\par{In a physical sense, \emph{asoko }あそこ refers to a place away from both the speaker and the listener. When said place, however, is out of eyesight and is being referred to in context, then the place must be known by all parties in the conversation. Of course, this is assumed in natural discourse. Similarly to \emph{soko }そこ, \emph{asoko }あそこ may also refer to a situation that is known by both the speaker and listener, but as for \emph{asoko }あそこ, the situation is of a severe degree. }

\par{11. あそこは ${\overset{\textnormal{じむしつ}}{\text{事務室}}}$ です。 \hfill\break
\emph{Asoko wa jimushitsu desu. }\hfill\break
Over there is the office. }

\par{12. ${\overset{\textnormal{きよこ}}{\text{清子}}}$ さんの ${\overset{\textnormal{かばん}}{\text{鞄}}}$ はあそこにあります。 \hfill\break
\emph{Kiyoko-san no kaban wa asoko ni arimasu. }\hfill\break
Ms. Kiyoko\textquotesingle s bag is over there. }

\par{13. あそこの ${\overset{\textnormal{む}}{\text{向}}}$ こうは ${\overset{\textnormal{ふくおかし}}{\text{福岡市}}}$ ですね。 \hfill\break
\emph{Asoko no mukō wa Fukuoka-shi desu ne. }\hfill\break
Beyond there\slash on the opposite side of there is Fukuoka City, right? }

\par{14. あそこのお ${\overset{\textnormal{まわ}}{\text{巡}}}$ りさんに ${\overset{\textnormal{き}}{\text{聞}}}$ いてください。 \hfill\break
\emph{Asoko no omawari-san ni kiite kudasai. }\hfill\break
Please ask that police officer over there. }

\par{Phrase Note: Although the word for police officer is \emph{kei(satsu)kan }警(察)官, policemen are generally referred to as \emph{omawari-san }お巡りさん. }

\par{15. ${\overset{\textnormal{かのじょ}}{\text{彼女}}}$ もあそこら ${\overset{\textnormal{へん}}{\text{辺}}}$ に ${\overset{\textnormal{す}}{\text{住}}}$ んでいます。 \hfill\break
\emph{Kanojo mo asokorahen ni sunde imasu. }\hfill\break
She too lives around there. }

\par{16. 「 ${\overset{\textnormal{ぎんこう}}{\text{銀行}}}$ はどこですか。」「あそこです。」 \hfill\break
\emph{“Ginkō wa doko desu ka?” “Asoko desu.” \hfill\break
}Where is the bank? }

\par{17. あそこに ${\overset{\textnormal{どうぶつえん}}{\text{動物園}}}$ があります。 \hfill\break
\emph{Asoko ni dōbutsuen ga arimasu. }\hfill\break
There is a zoo over there. }

\par{18. ${\overset{\textnormal{わたし}}{\text{私}}}$ もあそこに ${\overset{\textnormal{かぞく}}{\text{家族}}}$ がいます。 \hfill\break
\emph{Watashi mo asoko ni kazoku ga imasu. }\hfill\break
I too have family there. }

\par{19. そこに ${\overset{\textnormal{しょうがっこう}}{\text{小学校}}}$ があります。 \hfill\break
\emph{Soko ni shōgakkō ga arimasu. }\hfill\break
There is an elementary school there. }

\par{20. あそこが ${\overset{\textnormal{いた}}{\text{痛}}}$ いです。 \hfill\break
\emph{Asoko ga itai desu. }\hfill\break
My private area hurts. }
 
\par{Phrase Note: \emph{Asoko }あそこ may also be used to euphemistically refer to one\textquotesingle s private parts. }
      
\section{(Over) There: Asoko あそこ}
 
\par{ In a physical sense, \emph{asoko }あそこ refers to a place away from both the speaker and the listener. When said place, however, is out of eyesight and is being referred to in context, then the place must be known by all parties in the conversation. Of course, this is assumed in natural discourse. Similarly to \emph{soko }そこ, \emph{asoko }あそこ may also refer to a situation that is known by both the speaker and listener, but as for \emph{asoko }あそこ, the situation is of a severe degree. }

\par{11. あそこは ${\overset{\textnormal{じむしつ}}{\text{事務室}}}$ です。 \hfill\break
\emph{Asoko wa jimushitsu desu. }\hfill\break
Over there is the office. }

\par{12. ${\overset{\textnormal{きよこ}}{\text{清子}}}$ さんの ${\overset{\textnormal{かばん}}{\text{鞄}}}$ はあそこにあります。 \hfill\break
\emph{Kiyoko-san no kaban wa asoko ni arimasu. }\hfill\break
Ms. Kiyoko\textquotesingle s bag is over there. }

\par{13. あそこの ${\overset{\textnormal{む}}{\text{向}}}$ こうは ${\overset{\textnormal{ふくおかし}}{\text{福岡市}}}$ ですね。 \hfill\break
\emph{Asoko no mukō wa Fukuoka-shi desu ne. }\hfill\break
Beyond there\slash on the opposite side of there is Fukuoka City, right? }

\par{14. あそこのお ${\overset{\textnormal{まわ}}{\text{巡}}}$ りさんに ${\overset{\textnormal{き}}{\text{聞}}}$ いてください。 \hfill\break
\emph{Asoko no omawari-san ni kiite kudasai. }\hfill\break
Please ask that police officer over there. }

\par{\textbf{Phrase Note }: Although the word for police officer is \emph{kei(satsu)kan }警(察)官, policemen are generally referred to as \emph{omawari-san }お巡りさん. }

\par{15. ${\overset{\textnormal{かのじょ}}{\text{彼女}}}$ もあそこら ${\overset{\textnormal{へん}}{\text{辺}}}$ に ${\overset{\textnormal{す}}{\text{住}}}$ んでいます。 \hfill\break
\emph{Kanojo mo asokorahen ni sunde imasu. }\hfill\break
She too lives around there. }

\par{16. 「 ${\overset{\textnormal{ぎんこう}}{\text{銀行}}}$ はどこですか。」「あそこです。」 \hfill\break
\emph{“Ginkō wa doko desu ka?” “Asoko desu.” \hfill\break
}Where is the bank? }

\par{17. あそこに ${\overset{\textnormal{どうぶつえん}}{\text{動物園}}}$ があります。 \hfill\break
\emph{Asoko ni dōbutsuen ga arimasu. }\hfill\break
There is a zoo over there. }

\par{18. ${\overset{\textnormal{わたし}}{\text{私}}}$ もあそこに ${\overset{\textnormal{かぞく}}{\text{家族}}}$ がいます。 \hfill\break
\emph{Watashi mo asoko ni kazoku ga imasu. }\hfill\break
I too have family there. }

\par{19. そこに ${\overset{\textnormal{しょうがっこう}}{\text{小学校}}}$ があります。 \hfill\break
\emph{Soko ni shōgakkō ga arimasu. }\hfill\break
There is an elementary school there. }

\par{20. あそこが ${\overset{\textnormal{いた}}{\text{痛}}}$ いです。 \hfill\break
\emph{Asoko ga itai desu. }\hfill\break
My private area hurts. }

\par{\textbf{Phrase Note }: \emph{Asoko }あそこ may also be used to euphemistically refer to one\textquotesingle s private parts. }

\par{(Over) There: \emph{Asoko }あそこ }

\par{In a physical sense, \emph{asoko }あそこ refers to a place away from both the speaker and the listener. When said place, however, is out of eyesight and is being referred to in context, then the place must be known by all parties in the conversation. Of course, this is assumed in natural discourse. Similarly to \emph{soko }そこ, \emph{asoko }あそこ may also refer to a situation that is known by both the speaker and listener, but as for \emph{asoko }あそこ, the situation is of a severe degree. }

\par{11. あそこは ${\overset{\textnormal{じむしつ}}{\text{事務室}}}$ です。 \hfill\break
\emph{Asoko wa jimushitsu desu. }\hfill\break
Over there is the office. }

\par{12. ${\overset{\textnormal{きよこ}}{\text{清子}}}$ さんの ${\overset{\textnormal{かばん}}{\text{鞄}}}$ はあそこにあります。 \hfill\break
\emph{Kiyoko-san no kaban wa asoko ni arimasu. }\hfill\break
Ms. Kiyoko\textquotesingle s bag is over there. }

\par{13. あそこの ${\overset{\textnormal{む}}{\text{向}}}$ こうは ${\overset{\textnormal{ふくおかし}}{\text{福岡市}}}$ ですね。 \hfill\break
\emph{Asoko no mukō wa Fukuoka-shi desu ne. }\hfill\break
Beyond there\slash on the opposite side of there is Fukuoka City, right? }

\par{14. あそこのお ${\overset{\textnormal{まわ}}{\text{巡}}}$ りさんに ${\overset{\textnormal{き}}{\text{聞}}}$ いてください。 \hfill\break
\emph{Asoko no omawari-san ni kiite kudasai. }\hfill\break
Please ask that police officer over there. }

\par{Phrase Note: Although the word for police officer is \emph{kei(satsu)kan }警(察)官, policemen are generally referred to as \emph{omawari-san }お巡りさん. }

\par{15. ${\overset{\textnormal{かのじょ}}{\text{彼女}}}$ もあそこら ${\overset{\textnormal{へん}}{\text{辺}}}$ に ${\overset{\textnormal{す}}{\text{住}}}$ んでいます。 \hfill\break
\emph{Kanojo mo asokorahen ni sunde imasu. }\hfill\break
She too lives around there. }

\par{16. 「 ${\overset{\textnormal{ぎんこう}}{\text{銀行}}}$ はどこですか。」「あそこです。」 \hfill\break
\emph{“Ginkō wa doko desu ka?” “Asoko desu.” \hfill\break
}Where is the bank? }

\par{17. あそこに ${\overset{\textnormal{どうぶつえん}}{\text{動物園}}}$ があります。 \hfill\break
\emph{Asoko ni dōbutsuen ga arimasu. }\hfill\break
There is a zoo over there. }

\par{18. ${\overset{\textnormal{わたし}}{\text{私}}}$ もあそこに ${\overset{\textnormal{かぞく}}{\text{家族}}}$ がいます。 \hfill\break
\emph{Watashi mo asoko ni kazoku ga imasu. }\hfill\break
I too have family there. }

\par{19. そこに ${\overset{\textnormal{しょうがっこう}}{\text{小学校}}}$ があります。 \hfill\break
\emph{Soko ni shōgakkō ga arimasu. }\hfill\break
There is an elementary school there. }

\par{20. あそこが ${\overset{\textnormal{いた}}{\text{痛}}}$ いです。 \hfill\break
\emph{Asoko ga itai desu. }\hfill\break
My private area hurts. }
 
\par{Phrase Note: \emph{Asoko }あそこ may also be used to euphemistically refer to one\textquotesingle s private parts. }
    