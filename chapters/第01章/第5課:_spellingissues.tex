    
\chapter{Kana III}

\begin{center}
\begin{Large}
第5課: Kana III: Long Vowels, Double Consonants, \& Yotsugana 
\end{Large}
\end{center}
 
\par{ Every language has an \textbf{orthography }for its script(s). In any orthography, \emph{there are lots of rules that govern the use of its writing system(s) }. With Japanese being written with a mixed script, there are plenty of rules that have to be accounted for in its orthography. For the most part, Japanese orthography in regards to \emph{Kana }is rather straightforward. You've learned how the basic sounds are written in both \emph{Hiragana }and \emph{Katakana }. What has not been taught yet is how long vowels and double consonants are covered. We've also yet to learn about the true differences are between the variant ways of writing \slash ji\slash , \slash zu\slash , \slash ja\slash , \slash ju\slash , and \slash jo\slash . The \emph{Kana }used to write these sounds are called \emph{Yotsugana }. }

\par{ As we learned in Lesson 1, Japanese distinguishes between short and long vowels, and as we learned in Lesson 2, Japanese also distinguishes between single and double consonants. The change in mora count that occurs when lengthening a vowel or consonant isn't negligible. First, we will learn about how \emph{Hiragana }and \emph{Katakana }typically spell out long vowels. Then, we'll learn about how both systems write double consonants. Then, we'll learn about how to differentiate between \emph{Yotsugana }. Although we've already learned about the characters themselves, nothing has been said on when to use which. }

\begin{center}
\textbf{Contents of This Lesson }
\end{center}

\begin{itemize}

\item Long Vowels in \emph{Hiragana } 
\item Long Vowels in \emph{Katakana } 
\item Double Consonants in \emph{K }\emph{ana } 
\item \emph{Yotsugana  }
\end{itemize}
      
\section{Long Vowels in Hiragana}
 
\par{ In \emph{Hiragana }, long vowels are typically written by doubling the vowel. As you can see below, only long \slash e\slash  or \slash o\slash  sounds are extra complicated. The reason why these two long vowels are two possible spellings is because of all the words that have been borrowed from Chinese. Sometimes, spelling doesn't always match pronunciation. As readers of English, you should know this oh too well. }

\begin{ltabulary}{|P|P|P|P|P|}
\hline 

Long \slash a\slash  & Long \slash i\slash  & Long \slash u\slash  & Long \slash e\slash  & Long \slash o\slash  \\ \cline{1-5}

ああ & いい & うう & ええ \hfill\break
えい & おお \hfill\break
おう \\ \cline{1-5}

\end{ltabulary}

\par{ The next thing to do is see actual words with each of these long vowels. The information we learned about long \slash e\slash  and long \slash o\slash  sounds in Lesson 1 will be extremely relevant in this lesson. }

\begin{center}
\textbf{Long \slash a\slash , \slash i\slash , \& \slash u\slash  } 
\end{center}

\par{ To create long vowels for \slash a\slash , \slash i\slash , and \slash u\slash , all you do is double the vowel symbol. In the word charts below, the first column shows their spellings in Hiragana. Because word type is a major factor later on in this lesson, the word type for all words shown in this section are also provided. There are three main sources of vocabulary in Japanese: native (words that are indigenous to Japanese), Sino-Japanese, and loan-words. Sino-Japanese words are words that were either borrowed or created with roots from Chinese. These words are alternatively referred to as Kango (the Japanese terminology for Sino-Japanese) in the charts below. Loan-words are borrowings from modern world languages that have managed to find their way into Japanese. In the third column. }

\par{\textbf{Transcription Note }: \hfill\break
1. Because pitch contours will be marked on the \emph{Hiragana }spellings, long vowels will be romanized with macrons in the charts below except for long \slash i\slash , which will be written as "ii." \hfill\break
2. High pitch and pitch drops will be denoted the same way as previous lessons, just with their \emph{Hiragana }spellings. }

\par{\textbf{Curriculum Note }: False long vowels, vowels that happened to be juxtaposed next to each other but are in fact belong to separate word elements, are not represented as examples of long vowels in the charts below. }

\begin{ltabulary}{|P|P|P|}
\hline 

 \emph{Long \slash a\slash  }& Word Type & Meaning \\ \cline{1-3}

 \emph{Ā }\textbf{あ }あ \hfill\break
& Native & Ah \\ \cline{1-3}

 \emph{Okāsan }お \textbf{か }あさん \hfill\break
& Native & (Someone's) mother \\ \cline{1-3}

 \emph{Obasan }お \textbf{ばさん }\hfill\break
& Native & Aunt; middle-aged woman \\ \cline{1-3}

 \emph{Obāsan }お \textbf{ば }あさん \hfill\break
& Native & Grandmother\slash old woman \\ \cline{1-3}

\end{ltabulary}

\par{\textbf{Usage Note }: Long \slash a\slash  is not a common long vowel. In Hiragana, long \slash a\slash  is limited to native words. }

\begin{ltabulary}{|P|P|P|}
\hline 

 \emph{Long \slash i\slash  }& Word Type & Meaning \\ \cline{1-3}

 \emph{Ojisan }お \textbf{じさん }\hfill\break
& Native & Uncle\slash middle-aged man \\ \cline{1-3}

 \emph{Ojiisan }お \textbf{じ }いさん \hfill\break
& Native & Grandfather\slash old man \\ \cline{1-3}

\end{ltabulary}

\par{\textbf{Usage Note }: Long \slash i\slash  is also not a common long vowel. In Hiragana, long \slash i\slash  is limited to native words. }

\begin{ltabulary}{|P|P|P|}
\hline 

 \emph{Long \slash u\slash  }& Word Type & Meaning \\ \cline{1-3}

 \emph{Sūgaku }す \textbf{うがく }\hfill\break
&  \emph{Kango }& Math \\ \cline{1-3}

 \emph{Fūfu }\textbf{ふ }うふ \hfill\break
&  \emph{Kango }& Married couple \\ \cline{1-3}

 \emph{Gyūniku }ぎゅ \textbf{うにく }\hfill\break
&  \emph{Kango }& Beef \\ \cline{1-3}

\end{ltabulary}
\textbf{\hfill\break
Usage Note }: Though common, long \slash u\slash  is limited to Sino-Japanese words in \emph{Hiragana }. \hfill\break

\begin{center}
\textbf{Long \slash e\slash : ええ vs えい }
\end{center}

\par{ Whereas long \slash e\slash  in native words is always spelled with ええ, it is spelled as えい in Sino-Japanese, in which case it may alternatively be literally pronounced as [ei]. This literal pronunciation is preferred in many regions of Japan as well as in conversation pronunciation, especially in singing. Note that all other instances of えい outside Sino-Japanese vocabulary must be pronounced as [ei]. }

\begin{ltabulary}{|P|P|P|}
\hline 

[ē] & Word Type & Meaning \\ \cline{1-3}

 \emph{On }\emph{ēsan }お \textbf{ね }えさん \hfill\break
& Native & Older sister\slash young lady\slash miss \\ \cline{1-3}

 \emph{Hē }\textbf{へ }え \hfill\break
& Native & Really? \\ \cline{1-3}

[ē] or [ei] & Word Type & Meaning \\ \cline{1-3}

 \emph{Ēga }\slash  \emph{Eiga }え \textbf{いが }\hfill\break
&  \emph{Kango }& Movie \\ \cline{1-3}

 \emph{Mēshi }\slash  \emph{Meishi }め \textbf{いし }&  \emph{Kango }& Business card \\ \cline{1-3}

[ei] & Word Type & Meaning \\ \cline{1-3}

 \emph{Mei }め \textbf{い }& Native & Niece \\ \cline{1-3}

 \emph{Hei }へ \textbf{い }& Native & Wall\slash fence \\ \cline{1-3}

 \emph{Ei }えい \hfill\break
& Native & Stingray \\ \cline{1-3}

\end{ltabulary}

\begin{center}
\textbf{Long \slash o\slash : おお vs \emph{おう }}
\end{center}

\par{ Long \slash o\slash  is usually spelled in native words as おお. Historically, the second "o" would have originally been ほ or を, depending on the word. In Sino-Japanese words, long \slash o\slash  is written as おう. When おう is used in native words, it either stands for a long \slash o\slash  or "o.u." Typically, おう in native words is always a long \slash o\slash  except when it is at the end of a verb. The ending of a verb is treated as a separate element, thus breaking apart what otherwise would be a long vowel. }

\begin{ltabulary}{|P|P|P|}
\hline 

[ō] \hfill\break
& Word Type & Meaning \\ \cline{1-3}

 \emph{Kōri }こ \textbf{おり }\hfill\break
& Native & Ice \\ \cline{1-3}

 \emph{Tōi }と \textbf{おい }\hfill\break
& Native & Far away \\ \cline{1-3}

 \emph{Ō kii }お \textbf{おき }い \hfill\break
& Native & Big \\ \cline{1-3}

 \emph{Ō i }\textbf{お }おい \hfill\break
& Native & Many \\ \cline{1-3}

 \emph{Mō }\textbf{も }う \hfill\break
& Native & Already \\ \cline{1-3}

 \emph{Otōsan }お \textbf{と }うさん \hfill\break
& Native & (Someone's) father \\ \cline{1-3}

 \emph{Kanj }\emph{ō }か \textbf{んじょう }&  \emph{Kango }& Emotion \\ \cline{1-3}

 \emph{Gakkō }が \textbf{っこう }\hfill\break
&  \emph{Kango }& School \\ \cline{1-3}

 \emph{Nōgyō } \textbf{の }うぎょう & Kango & Agriculture \\ \cline{1-3}

[ou] & Word Type & Meaning \\ \cline{1-3}

 \emph{Ou }お \textbf{う }& Native & To chase \\ \cline{1-3}

 \emph{Ōu }お \textbf{おう }\hfill\break
& Native & To cover \\ \cline{1-3}

\end{ltabulary}
      
\section{Long Vowels in Katakana}
 
\par{ For \emph{Katakana }, long vowels are typically represented with a mark that looks similar to a hyphen: ー. It's normally either called a \emph{c }\emph{hō\textquotesingle ompu }ちょ \textbf{うお }んぷ or \emph{bōbiki }ぼ \textbf{うび }\textbf{き }. As \emph{Katakana }is used primarily to write foreign words, you are primarily going to use and see this with foreign words. }

\begin{ltabulary}{|P|P|P|P|}
\hline 

Word & Meaning & Word & Meaning \\ \cline{1-4}

 \emph{Tēburu }テ \textbf{ーブル }& Table &  \emph{Aisukuriimu }ア \textbf{イスクリ }ーム & Ice cream \\ \cline{1-4}

 \emph{Intāchenji }イ \textbf{ンターチェ }ン ジ & Interchange &  \emph{Mēru } \textbf{メ }ール & Email \\ \cline{1-4}

 \emph{Fināre }フィ \textbf{ナ }ーレ & Finale &  \emph{Kōchi } \textbf{コ }ーチ & Coach \\ \cline{1-4}

 \emph{Sōda } \textbf{ソ }ーダ & Soda &  \emph{Kompyūtā }コ \textbf{ンピュ }ーター & Computer \\ \cline{1-4}

 \emph{Aisutii }ア \textbf{イスティ }ー & Ice tea &  \emph{Sēru } \textbf{セ }ール & Sale \\ \cline{1-4}

 \emph{Orenjijūsu }オ \textbf{レンジジュ }ース & Orange juice &  \emph{Chiizu } \textbf{チ }ーズ & Cheese \\ \cline{1-4}

 \emph{Daunrōdo }ダ \textbf{ウンロ }ード & Download &  \emph{Kōhii }コ \textbf{ーヒ }ー & Coffee \hfill\break
\\ \cline{1-4}

 \emph{Intaby }\emph{ū } \textbf{イ }ンタビュー & Interview &  \emph{S }\emph{ū tsuk ē }\emph{su }ス \textbf{ーツケ }ース & Suitcase \\ \cline{1-4}

\end{ltabulary}

\par{\textbf{Curriculum Note }: A lot can be said about how to transcribe and pronounce loan-words. For now, know that long vowels are typically written with ー in \emph{Katakana }. }
      
\section{Double Consonants in Kana}
 
\par{ In both \emph{Hiragana }and \emph{Katakana }, double consonants are created by preceding a symbol with a shrunken \emph{tsu }. In \emph{Hiragana }, this is っ. In \emph{Katakana }, this is ッ. As we have learned previously, unvoiced consonants are typically the only consonants doubled. However, \slash n\slash  and \slash m\slash  can technically be long, but the symbol for N will be what precedes the main symbol (ん in \emph{Hiragana }and ン in \emph{Katakana }). }

\begin{ltabulary}{|P|P|P|P|}
\hline 

Word & Meaning & Word & Meaning \\ \cline{1-4}

 \emph{Chotto }\textbf{ちょ }っと \hfill\break
& A little &  \emph{Matto }\textbf{マ }ット \hfill\break
& Mat \\ \cline{1-4}

 \emph{Hokkē }\textbf{ホ }ッケー \hfill\break
& Hockey &  \emph{Shippai }し \textbf{っぱい }\hfill\break
& Failure \\ \cline{1-4}

 \emph{Jetto }\textbf{ジェ }ット \hfill\break
& Jet \hfill\break
&  \emph{Intānetto }イ \textbf{ンターネ }ット \hfill\break
& Internet \hfill\break
\\ \cline{1-4}

 \emph{Sakkā }\textbf{サ }ッカー \hfill\break
& Soccer &  \emph{Robotto }\textbf{ロ }ボット \hfill\break
& Robot \\ \cline{1-4}

\end{ltabulary}

\par{ With \emph{Katakana }, voiced consonants are only voiced in certain loan-words or in exaggerated pronunciations. Even in such expressions, these doubled voiced consonants are still usually pronounced as if they were unvoiced so long as there is an unvoiced equivalent. For instance, "bed" is \emph{\textbf{be }ddo }but is normally pronounced as \emph{\textbf{be }tto }. Nonetheless, it remains spelled as ベッド. Consonants for which this all applies include: g, z, d, h, f, b, r, w and y. }

\begin{ltabulary}{|P|P|P|P|}
\hline 

Word & Meaning & Word & Meaning \\ \cline{1-4}

 \emph{Baggu }\textbf{バ }ッグ \hfill\break
& Bag &  \emph{Beddo }\textbf{ベ }ッド & Bed \\ \cline{1-4}

 \emph{Suggoi }す \textbf{っご }い \hfill\break
& Cool! &  \emph{Reddo Sokkkusu }レ \textbf{ッドソ }ックス & The Red Socks \\ \cline{1-4}

 \emph{Aipaddo }\textbf{ア }イパッド \hfill\break
& iPad & \emph{Bagudaddo }バ \textbf{グダ }ッド & Baghdad \\ \cline{1-4}

 \emph{Hottodoggu }ホ \textbf{ットド }ッグ \hfill\break
& Hot dog &  \emph{Bahha }\textbf{バ }ッハ & Bach \\ \cline{1-4}

\end{ltabulary}

\begin{center}
\textbf{Glottal Stops }
\end{center}

\par{ In Lesson 1, we learned about what glottal stops were. A glottal stop is made by forcibly stopping air in one's Adam's apple. When an expression ends in a glottal stop, a small \emph{tsu }is used to indicate this pronunciation. An example of this is \emph{itah }いたっ (ouch!). }
      
\section{Yotsugana}
 
\par{ \emph{Yotsugana }refer to \emph{Kana }that spell what were traditionally four distinct consonants: \slash z\slash , \slash dz\slash , \slash j\slash , and \slash dj\slash . Pronunciation-wise, \slash z\slash  is usually pronounced as \slash dz\slash  and can only be pronounced as \slash z\slash  inside words. As for \slash j\slash  and \slash dj\slash , the two sounds are overwhelmingly both pronounced as [dj]. Previously, we learned when these consonants are used, but we haven't gone over the rules for how to write them correctly in \emph{Kana }. }

\par{ Below are the symbols in question in both \emph{Hiragana }and \emph{Katakana }. In the chart, symbols are listed as "common", "uncommon" or "rare." }

\begin{ltabulary}{|P|P|P|P|P|}
\hline 

Sound &  \emph{Hiragana }& Rarity &  \emph{Katakana }& Rarity \\ \cline{1-5}

JI & じ & Common & ジ & Common \\ \cline{1-5}

ZU & ず & Common & ズ & Common \\ \cline{1-5}

DZU & づ & Uncommon & ヅ & Rare \\ \cline{1-5}

DJI & ぢ & Uncommon & ヂ & Rare \\ \cline{1-5}

JA & じゃ & Common & ジャ & Common \\ \cline{1-5}

JU & じゅ & Common & ジュ & Common \\ \cline{1-5}

JO & じょ & Common & ジョ & Common \\ \cline{1-5}

DJA & ぢゃ & Uncommon & ヂャ & Rare \\ \cline{1-5}

DJU & ぢゅ & Rare & ヂュ & Rare \\ \cline{1-5}

DJO & ぢょ & Rare & ヂョ & Rare \\ \cline{1-5}

\end{ltabulary}

\par{ Because \emph{Katakana }is used largely for loan-word transcriptions, which is why symbols traditionally associated with the consonants \slash dj\slash  and \slash dz\slash  are all rare. Typically, the symbols traditionally associated with the consonants \slash z\slash  and \slash j\slash  are used regardless of how the consonant is pronounced. The only times when づ・ヅ and ぢ・ヂ are used is when they are immediately preceded by つ・ツ and ち・チ respectively, or when they are the voiced forms of つ・ツ and ち・チ respectively in compound expressions. }

\begin{ltabulary}{|P|P|P|P|}
\hline 

Nosebleed &  \emph{Hana(d)ji }は \textbf{なぢ }\hfill\break
& Instruction &  \emph{Shiji }\textbf{し }じ \hfill\break
\\ \cline{1-4}

Shrinkage &  \emph{Chi(d)jimi }ち \textbf{ぢみ }\hfill\break
& Bell &  \emph{Suzu }す \textbf{ず }\hfill\break
\\ \cline{1-4}

Continuation &  \emph{Tsu(d)zuki }つ \textbf{づき }\hfill\break
& Monopoly &  \emph{Hitorijime }ひ \textbf{とりじめ }\hfill\break
\\ \cline{1-4}

Class &  \emph{Jugyō } \textbf{じゅ }ぎょう & Jaguar &  \emph{Jaga } \textbf{ジャ }ガー \\ \cline{1-4}

Monotone &  \emph{Ippon(d)jōshi }い \textbf{っぽんぢょ }うし & Information &  \emph{Jōhō }じょ \textbf{うほう }\\ \cline{1-4}

Suggestion\slash hint &  \emph{Ire(d)jie }い \textbf{れぢえ }& Crescent Moon &  \emph{Mika(d)zuki }み \textbf{かづき }\\ \cline{1-4}

Within reach & \emph{Te(d)jika }て \textbf{ぢか }& To spell &  \emph{Tsu(d)zuru }つ \textbf{づる }\\ \cline{1-4}

Proximity &  \emph{Ma(d)jika }ま \textbf{ぢか }& Hairpiece &  \emph{Zura }ヅ \textbf{ラ }\\ \cline{1-4}

\end{ltabulary}

\par{\textbf{Word Note }: ヅラ is an abbreviation of \emph{katsura }か \textbf{つら }(hairpiece), and it is usually spelled in Katakana largely to emphasize its existence as an abbreviation. }

\par{\textbf{Curriculum Note }: To learn more, see Lesson 355. }
    