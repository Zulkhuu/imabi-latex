    
\chapter{Kosoado こそあど I This \& That}

\begin{center}
\begin{Large}
第23課: Kosoado こそあど I: This \& That: Kore\slash Kono これ・この, Sore\slash Sono それ・その, \& Are\slash Ano あれ・あの 
\end{Large}
\end{center}
 
\par{ In English, the words “this” and “that” are perhaps the most important words to refer to things, regardless of whether the things they refer to are physically present. No distinction is made between their uses as a noun (pronoun more specifically) or as an adjective. }
 
\par{i. This is a beautiful house. \hfill\break
ii. That is a very tall tree. \hfill\break
iii. This song is amazing. \hfill\break
iv. That blade is sharp. }
 
\par{ In Japanese, “this” and “that” are not this simple. Instead, they both have a pronoun and an adjectival form. As you will see, though, this will not be the only thing to consider as you learn about what word to use and when. }
\textbf{Curriculum Note }: Words like “this” and “that” fall under a category of words called demonstratives ( \emph{Shijishi }指示詞). Demonstratives are pronoun\slash adjectival phrases that indicate which entities are being referred to and how. In Japanese, they are typically called \emph{kosoado }こそあど because each syllable represents the syllables that can start these kinds of words.       
\section{This: Kore これ \& Kono この}
 
\par{ In Japanese there are two forms of the word “this”: \emph{kore }これ and \emph{kono }この. The first is its pronoun form and the second is its adjectival form. }

\par{1. これは ${\overset{\textnormal{たまご}}{\text{卵}}}$ です。 \hfill\break
 \emph{Kore wa tamago desu. }\hfill\break
This is the egg\slash these are eggs. }

\par{2. この ${\overset{\textnormal{つくえ}}{\text{机}}}$ は ${\overset{\textnormal{ふる}}{\text{古}}}$ いです。 \hfill\break
 \emph{Kono tsukue wa furui desu. }\hfill\break
This desk is old\slash these desks are old. }

\par{\textbf{Grammar Note }: As seen by these two examples, there is no distinct difference in Japanese between “this” and “these.” Although a distinction is possible, we will leave that for a later discussion. }

\par{ There are two fundamentally different uses of the word “this.” You could be speaking about something physically present, or you could be speaking about a “this” in context. }

\par{ When you are using "this" to refer to something that is physically close to you, you use \emph{kore }これ or \emph{kono }この depending on whether “this” is the subject or “this” is a part of the subject\textquotesingle s description respectively. }

\par{3. これは ${\overset{\textnormal{なん}}{\text{何}}}$ ですか。 \hfill\break
 \emph{Kore wa nan desu ka? }\hfill\break
What is this? }

\par{4. これは ${\overset{\textnormal{まんねんひつ}}{\text{万年筆}}}$ です。 \hfill\break
 \emph{Kore wa man\textquotesingle nenhitsu desu. }\hfill\break
This is a fountain pen. }

\par{5. これは ${\overset{\textnormal{えいわじてん}}{\text{英和辞典}}}$ です。 \hfill\break
 \emph{Kore wa eiwa jiten desu. }\hfill\break
This is an English-Japanese dictionary. }

\par{6. かつて、この ${\overset{\textnormal{あた}}{\text{辺}}}$ りは ${\overset{\textnormal{しず}}{\text{静}}}$ かなところでした。 \hfill\break
 \emph{Katsute, kono atari wa shizukana tokoro deshita. }\hfill\break
This neighborhood was once a quiet place. }

\par{\textbf{Spelling Note }: \emph{Katsute }means “once” and can alternatively be spelled as 嘗て. }

\par{7. このヘビを ${\overset{\textnormal{ころ}}{\text{殺}}}$ しました。 \hfill\break
 \emph{Kono hebi wo koroshimashita. }\hfill\break
I killed this snake. }

\par{\textbf{Spelling Note }: The noun \emph{hebi }meaning “snake” can alternatively be spelled as 蛇. }

\par{ When a “this” in context is being spoken about, it is the speaker who holds the information and is referring to his own information with “this.” What we refer to as “this” tends to be important information. This dialogue about the word “this” itself is important to the author, it is one written by the author, and it is giving the reader information important to learning Japanese. As you can see, English and Japanese do not differ in this use of “this,” and no different words are needed to use this “this.” }

\par{8. これは ${\overset{\textnormal{じゅうよう}}{\text{重要}}}$ な手がかりです。 \hfill\break
 \emph{Kore wa jūyō na tegakari desu. }\hfill\break
This is an important clue. }

\par{9. これは ${\overset{\textnormal{たいへん}}{\text{大変}}}$ です! \hfill\break
 \emph{Kore wa taihen desu! }\hfill\break
This is serious! }

\par{10. この ${\overset{\textnormal{はなし}}{\text{話}}}$ は ${\overset{\textnormal{ひみつ}}{\text{秘密}}}$ ですよ。 \hfill\break
 \emph{Kono hanashi wa himitsu desu yo. }\hfill\break
This conversation is a secret. }

\par{\textbf{Particle Note }: The particle \emph{yo }よ at the end of this sentence adds emphasis to the importance of the predicate, which is “is a secret.” }
      
\section{That: Sore それ \& Sono その}
 
\par{ The words \emph{sore }それ and \emph{sono }その mean “that,” and the former is the pronoun form and the latter is the adjectival form. }

\par{11. それはオランウータンですよ。 \hfill\break
 \emph{Sore wa oran\textquotesingle ūtan desu yo. }\hfill\break
That\textquotesingle s an orangutan. }

\par{12. そのつもりはない。 \hfill\break
 \emph{Sono tsumori wa nai. }\hfill\break
I don\textquotesingle t have that intention. }

\par{13. そのアリはかわいいですね。 \hfill\break
 \emph{Sono ari wa kawaii desu ne. }\hfill\break
This ant is cute, isn\textquotesingle t it? \hfill\break
Those ants are cute, aren\textquotesingle t they? }

\par{\textbf{Particle Note }: The particle \emph{ne }ね at the end of this sentence seeks agreement from the listener. }

\par{\textbf{Spelling Note }: The noun \emph{ari }means “ant” and can alternatively be spelled as 蟻. }

\par{ As you can see, there is no important distinction between “that” and “those” in Japanese. Just as was the case with “this,” both \emph{sore }それ and \emph{sono }その can be used in reference to “that” which is in physical proximity and “that” in context.  There is, however, a catch. }

\par{ The "that" must be close to the listener when speaking about something in physical proximity. }

\par{14. それは ${\overset{\textnormal{なん}}{\text{何}}}$ ですか。 \hfill\break
 \emph{Sore wa nan desu ka? }\hfill\break
What is that? }

\par{\textbf{Sentence Note }: The "that" which the speaker is asking about is close to the listener but not to the speaker. }

\par{15. それはワニです。 \hfill\break
 \emph{Sore wa wani desu. }\hfill\break
That is a crocodilian. }

\par{16. その ${\overset{\textnormal{きょうかしょ}}{\text{教科書}}}$ は ${\overset{\textnormal{やす}}{\text{安}}}$ かったですか。 \hfill\break
 \emph{Sono kyōkasho wa yasukatta desu ka? }\hfill\break
Was that textbook cheap? }

\par{ Furthermore, the "that" must either be familiar to either the speaker or the listener but not both parties. It also works when you know something about the “that” but not everything about it. }

\par{17. その ${\overset{\textnormal{がくせい}}{\text{学生}}}$ さんは ${\overset{\textnormal{だれ}}{\text{誰}}}$ ですか。 \hfill\break
 \emph{Sono gakusei-san wa dare desu ka? }\hfill\break
Who is that student? }

\par{\textbf{Sentence Note }: This sentence would be used in context of the speaker mentioning the student and then asking the listener to tell him\slash her who that student actually is. }

\par{18. あ、その ${\overset{\textnormal{はなし}}{\text{話}}}$ を ${\overset{\textnormal{き}}{\text{聞}}}$ きました。 \hfill\break
 \emph{A, sono hanashi wo kikimashita. }\hfill\break
Oh yeah, I heard about that. }

\par{\textbf{Sentence Note }: Even though both the speaker and listener know something about the conversation being referenced, only the listener would know the full story. }

\par{19. えー。それ、 ${\overset{\textnormal{ほんとう}}{\text{本当}}}$ ですか? \hfill\break
 \emph{Ē. Sore, hontō desu ka? }\hfill\break
Eh? Is that true? }

\par{20. その ${\overset{\textnormal{ひ}}{\text{日}}}$ は ${\overset{\textnormal{くも}}{\text{曇}}}$ りでした。 \hfill\break
 \emph{Sono hi wa kumori deshita. }\hfill\break
That day was cloudy. }

\par{\textbf{Sentence Note }: In this example, the speaker is informing the listener that the day in question was cloudy. }

\par{21. それは ${\overset{\textnormal{い}}{\text{要}}}$ りません。 \hfill\break
 \emph{Sore wa irimasen. }\hfill\break
I don't need that\slash that's not necessary. }

\par{22. それの何がいけないですか。 \hfill\break
\emph{Sore no nani ga ikenai desu ka? }\hfill\break
What is wrong with that? }

\par{\textbf{Sentence Note }: The use of \emph{no }の after \emph{sore }それ is not wrong. In this sentence, \emph{sore no nani }それの何 literally means "what of that." As such, whenever you are using "this" or "that" as the subject but are using it in a possessive manner, you cannot drop the \slash re\slash . }
      
\section{That (Over There): Are あれ \& Ano あの}
 
\par{ Japanese has two more phrases for “that,” but they\textquotesingle re not the same as those above. \emph{Are }あれ and \emph{ano }あの both refer to “that” which is neither close to the speaker not to the listener. In context when the “that” is not present, the information regarding it is known fully well by all parties involved. If it\textquotesingle s just oneself, it\textquotesingle s “that” which one has an image of. Again, as was the case above, \emph{are }あれ is the pronoun form and \emph{ano }あの is the adjectival form. }

\par{23. あれはカフェテリアです。 \hfill\break
 \emph{Are wa cafeteria desu. }\hfill\break
That over there is a cafeteria. }

\par{24. あの ${\overset{\textnormal{たてもの}}{\text{建物}}}$ は ${\overset{\textnormal{ぎじどう}}{\text{議事堂}}}$ です。 \hfill\break
\emph{Ano tatemono wa gijidō desu. }\hfill\break
That building is The National Diet. }

\par{\textbf{Culture Note }: The National Diet is the parliament house of Japanese government.  The word \emph{gijidō }議事堂 can in principle refer to other country\textquotesingle s legislative branch building, but context would be needed to clarify. }

\par{25. 私もあの ${\overset{\textnormal{ねこ}}{\text{猫}}}$ が ${\overset{\textnormal{す}}{\text{好}}}$ きです。 \hfill\break
 \emph{Watashi mo ano neko ga suki desu. }\hfill\break
I too like that cat. }

\par{\textbf{Sentence Note }: The cat in question may be physically in sight but away from the speaker or listener, or it may be a cat that both the speaker and listener personally know but is not in sight. }

\par{26. 「あれは ${\overset{\textnormal{なん}}{\text{何}}}$ ですか。」「 ${\overset{\textnormal{こいのぼり}}{\text{鯉幟}}}$ です。」 \hfill\break
 \emph{“Are wa nan desu ka?” “Koinobori desu.” }\hfill\break
“What is that?” “It's a koinobori”. }

\par{\textbf{Culture Note }: A \emph{koinobori }鯉幟 is a giant paper carp flown atop poles next to houses that are celebrating Children's Day on May 5th with male children. }

\par{\textbf{Sentence Note }: The \emph{koinobori }would be positioned away from both the speaker and the listener. }

\par{27. あのレストラン、 ${\overset{\textnormal{おい}}{\text{美味}}}$ しかったなあ。 \hfill\break
 \emph{Ano resutoran, oishikatta nā. } \hfill\break
Aah, that restaurant was delicious. }

\par{\textbf{Particle Note }: The particle \emph{nā }なあ is used to give a heightened sense of appreciation and yearning for the restaurant\textquotesingle s food. }

\par{28. ${\overset{\textnormal{たし}}{\text{確}}}$ かにあれを ${\overset{\textnormal{た}}{\text{食}}}$ べたね。 \hfill\break
 \emph{Tashika ni are wo tabeta ne. }\hfill\break
I definitely ate that, right? }

\par{\textbf{Sentence Note }: Even when you forget a certain detail, you still use these forms of “that” because you are simply conjuring up knowledge that had already been established between you and the listener.  }
      
\section{Korya こりゃ, Sorya そりゃ, \& Arya ありゃ}
 
\par{ In more casual speech, the particle \emph{wa }は contracts with the pronoun forms of the words for “this” and “that (over there).” }

\begin{ltabulary}{|P|P|P|}
\hline 

Standard Speech & Casual Speech & Dialect\slash When Angry\slash Other \\ \cline{1-3}

 \emph{Kore wa }これは &  \emph{Korya }こりゃ &  \emph{Kora }こら \\ \cline{1-3}

 \emph{Sore wa }それは &  \emph{Sorya }そりゃ &  \emph{Sora }そら \\ \cline{1-3}

 \emph{Are wa }あれは &  \emph{Arya }ありゃ &  \emph{Ara }あら \\ \cline{1-3}

\end{ltabulary}

\par{ These casual forms can be used in rather coarse yet informal situations. The latter column, however, is trickier. \emph{Kora }こら is either used to mean “hey!” when angry, or it\textquotesingle s the same as \emph{kore wa }これは in other dialects. \emph{Sora }そら is either used to mean “look!” in which case it\textquotesingle s interchangeable with \emph{hora }ほら, or it\textquotesingle s the same as \emph{sore wa }それは. \emph{Ara }あら is used to mean “oh?” in female speech. The gender neutral way of saying “oh?” happens to be \emph{are }あれ. }

\par{29. ${\overset{\textnormal{なん}}{\text{何}}}$ だ、\{こりゃ・こら\}! \hfill\break
 \emph{Nan da, [korya\slash kora]. } \hfill\break
What the heck (is this\slash going on)! }

\par{\textbf{Sentence Note }: This example shows that the casual forms are just as capable of being used when one is angry as is the case for \emph{kora }こら and \emph{sora }そら. This sentence also shows that given the enhanced angry tone of the statement, you find \emph{korya\slash kora }こりゃ・こら inverted to the end of the sentence. }

\par{30. そら、おもろい。 \hfill\break
 \emph{Sora, omoroi. }\hfill\break
That\textquotesingle s interesting. }

\par{\textbf{Dialect Note }: In Kansai dialects, which are well known for being spoken in West Japan in places like Ōsaka 大阪, the adjective \emph{omoshiroi }面白い (interesting) it's contracted to \emph{omoroi }おもろい. }

\par{31. ありゃ、 ${\overset{\textnormal{たいへん}}{\text{大変}}}$ だったねぇ。 \hfill\break
 \emph{Arya, taihen datta ne. }\hfill\break
That was difficult, huh. }

\par{\textbf{Particle Note }: \emph{Nē }ねぇ is the same as \emph{ne }ね but with a trailing pronunciation that gives a tone of relief\slash consolation. }

\par{32. あら、その ${\overset{\textnormal{おと}}{\text{音}}}$ 、 ${\overset{\textnormal{き}}{\text{聞}}}$ こえた? (Feminine) \hfill\break
 \emph{Ara, sono oto, kikoeta? }\hfill\break
Oh? Did you hear that sound? }
    