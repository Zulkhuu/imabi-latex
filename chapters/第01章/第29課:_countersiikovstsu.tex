    
\chapter{Counters II}

\begin{center}
\begin{Large}
第29課: Counters II: 個 vs つ 
\end{Large}
\end{center}
 
\par{ In our second lesson on counters, we will tackle how to count things in general with - \emph{tsu }つ and - \emph{ko } ${\overset{\textnormal{こ}}{\text{個}}}$ . }
      
\section{The Counter -tsu つ}
 
\par{ Before Chinese loan words inundated Japanese, the language already had a counter system. Even then, there was one counter most frequently used. That counter continues to be used to count things in general. This counter is - \emph{tsu }つ. Naturally, this counter is used with native numbers. However, it is because of this that it is greatly limited. After the number 9, 10 can be expressed with a native expression, but the system for counting things in general is largely limited up to 10 as an effect. }

\begin{ltabulary}{|P|P|P|P|P|P|}
\hline 
 
  0 
 &    \emph{Nashi }なし 
 &   1 
 &    \emph{Hitotsu }ひとつ 
 &   2 
 &    \emph{Futatsu }ふたつ 
 \\ \cline{1-6} 
 
  3 
 &    \emph{Mi(t)tsu }み(っ)つ 
 &   4 
 &    \emph{Yo(t)tsu }よ(っ)つ 
 &   5 
 &    \emph{Itsutsu }いつつ 
 \\ \cline{1-6} 
 
  6 
 &    \emph{Mu(t)tsu }む(っ)つ 
 &   7 
 &    \emph{Nanatsu }ななつ 
 &   8 
 &    \emph{Ya(t)tsu }や(っ)つ 
 \\ \cline{1-6} 
 
  9 
 &    \emph{Kokonotsu }ここのつ 
 &   10 
 &    \emph{Tō }とお 
 &   ? 
 &    \emph{Ikutsu }いくつ 
 \\ \cline{1-6} 
 
\end{ltabulary}

\par{\textbf{Notes }: \emph{Nashi }なし literally means “nothing” and is where - \emph{nai }ない derives. The small っ, although shown as being optional, is almost always pronounced. Other native numbers do exist, but they survive only in set phrases. As such, these numbers will be addressed later in IMABI. }

\par{ The \emph{Kanji }漢字 spellings for these phrases are as follows. }

\begin{ltabulary}{|P|P|P|P|P|P|P|P|P|P|}
\hline 
 
  0 
 &   無し 
 &   1 
 &   一つ 
 &   2 
 &   二つ 
 &   3 
 &   三つ 
 &   4 
 &   四つ 
 \\ \cline{1-10} 
 
  5 
 &   五つ 
 &   6 
 &   六つ 
 &   7 
 &   七つ 
 &   8 
 &   八つ 
 &   9 
 &   九つ 
 \\ \cline{1-10} 
 
  10 
 &   十 
 &   ? 
 &   幾つ 
 &  &     &     &     &     &  \\ \cline{1-10} 
 
\end{ltabulary}

\par{\textbf{Orthography Note }: Typically, 1-9 are written with Arabic numerals. }

\par{ Overusing - \emph{tsu }つ will cause your speech to sound uneducated. This is partly because it is a native word and not a Sino-Japanese word. Sino-Japanese words tend to sound more sophisticated, especially when a native counterpart exists. When you use - \emph{tsu }つ for just anything, there is also the risk that you\textquotesingle ll ignore counters that have always been in Japanese. By extension, blanketly using - \emph{tsu }つ to count anything may make it seem as if you\textquotesingle re ignoring the whole system. }

\par{ Howbeit, it is still used heavily in natural conversation. As such, we will now discuss the instances - \emph{tsu }つ is used so that your usage of it may be as natural as possible. }

\par{\textbf{Variation Note }: Because the counter - \emph{tsu }つ often replaces the ‘proper\textquotesingle  counter, variation may include counters not yet introduced. In such an event, you are not required to memorize said counters for now. However, it is still important to know that you have options and that those options may be more appropriate than - \emph{tsu }つ. }

\par{1. Used to count three-dimensional items: This usage is the most common and the most problematic. Most physical items have counters to count them, but there are also some things solely counted with - \emph{tsu }つ. }

\par{1. ${\overset{\textnormal{ふた}}{\text{2}}}$ つ ${\overset{\textnormal{くだ}}{\text{下}}}$ さい。 \hfill\break
 \emph{Futatsu kudasai. }\hfill\break
Please give me two (of them). }

\par{2. ブースは ${\overset{\textnormal{よっ}}{\text{4}}}$ つあります。 \hfill\break
 \emph{B }\emph{ūsu wa yottsu arimasu. \hfill\break
 }There are four booths. }

\par{3. オフィスにテーブルが ${\overset{\textnormal{いつ}}{\text{5}}}$ つあります。 \hfill\break
 \emph{Ofisu ni t }\emph{ēburu ga itsutsu arimasu. \hfill\break
 }There are five tables in the office. }

\par{4. ${\overset{\textnormal{にほん}}{\text{日本}}}$ に ${\overset{\textnormal{おんせん}}{\text{温泉}}}$ は\{いくつ・ ${\overset{\textnormal{なん}}{\text{何}}}$ か ${\overset{\textnormal{しょ}}{\text{所}}}$ \}ありますか。 \hfill\break
 \emph{Nihon ni onsen wa [ikutsu\slash nankasho] arimasu ka? \hfill\break
 }How many hot springs are there in Japan? }

\par{5. ベッドは\{ ${\overset{\textnormal{いつ}}{\text{五}}}$ つ △・ ${\overset{\textnormal{ご}}{\text{5}}}$ ${\overset{\textnormal{しょう}}{\text{床}}}$ 〇・ ${\overset{\textnormal{ご}}{\text{5}}}$ ${\overset{\textnormal{だい}}{\text{台}}}$ 〇\}あります。 \hfill\break
 \emph{Beddo wa [itsutsu\slash gosh }\emph{ō\slash godai] arimasu. \hfill\break
 }There are five beds. }

\par{6. ${\overset{\textnormal{あめ}}{\text{飴}}}$ を\{ ${\overset{\textnormal{ひと}}{\text{1}}}$ つ・ ${\overset{\textnormal{いっ}}{\text{1}}}$ ${\overset{\textnormal{こ}}{\text{個}}}$ \} ${\overset{\textnormal{た}}{\text{食}}}$ べます。 \hfill\break
 \emph{Ame wo [hitotsu\slash ikko] tabemasu. \hfill\break
 }I\textquotesingle ll eat one piece of candy. }

\par{2. Used to count things with an indeterminate form: }

\par{7. タオルに ${\overset{\textnormal{きいろ}}{\text{黄色}}}$ い ${\overset{\textnormal{し}}{\text{染}}}$ みが ${\overset{\textnormal{ふた}}{\text{2}}}$ つあります。 \hfill\break
 \emph{Taoru ni kiiroi shimi ga futatsu arimasu. \hfill\break
 }There are two yellow stains on the towel. }

\par{8. ${\overset{\textnormal{せんめんだい}}{\text{洗面台}}}$ の ${\overset{\textnormal{した}}{\text{下}}}$ に ${\overset{\textnormal{おお}}{\text{大}}}$ きな ${\overset{\textnormal{みずた}}{\text{水溜}}}$ まりがひとつあります。 \hfill\break
 \emph{Semmendai no shita ni }\emph{ōkina mizutamari ga hitotsu arimasu. \hfill\break
 }There is a large puddle beneath the washstand. }

\par{9. ${\overset{\textnormal{はい}}{\text{肺}}}$ に ${\overset{\textnormal{かげ}}{\text{影}}}$ が ${\overset{\textnormal{みっ}}{\text{3}}}$ つありました。 \hfill\break
 \emph{Hai ni kage ga mittsu arimashita. \hfill\break
 }There were three shadows on the\slash my lungs. }

\par{3. Used to count age (1-9) or to ask someone\textquotesingle s age: }

\par{10. おいくつですか? \hfill\break
 \emph{O-ikutsu desu ka? \hfill\break
 }How old are you? }

\par{11. うちの ${\overset{\textnormal{こ}}{\text{子}}}$ は ${\overset{\textnormal{みっ}}{\text{3}}}$ つです。 \hfill\break
 \emph{Uchi no ko wa mittsu desu. \hfill\break
 }My child is three. }

\par{\textbf{Grammar Note }: \emph{Tō } ${\overset{\textnormal{とお}}{\text{十}}}$ can be used to mean “10 years old” to complete the series from 1-10. }

\par{4. Used to count abstract things: }

\par{12. ${\overset{\textnormal{りゆう}}{\text{理由}}}$ は ${\overset{\textnormal{むっ}}{\text{6}}}$ つあります。 \hfill\break
 \emph{Riy }\emph{ū wa muttsu arimasu. \hfill\break
 }There are six reasons. }

\par{13. ミッションは ${\overset{\textnormal{なな}}{\text{7}}}$ つあります。 \hfill\break
 \emph{Misshon wa nanatsu arimasu. \hfill\break
 }There are seven missions. }

\par{14. ${\overset{\textnormal{じょうけん}}{\text{条件}}}$ は ${\overset{\textnormal{ひと}}{\text{1}}}$ つあります。 \hfill\break
 \emph{J }\emph{ōken wa hitotsu arimasu. \hfill\break
 }There is one condition. }

\par{15. ${\overset{\textnormal{か}}{\text{交}}}$ わした ${\overset{\textnormal{じゅう}}{\text{10}}}$ の ${\overset{\textnormal{やくそく}}{\text{約束}}}$ はひとつも ${\overset{\textnormal{まも}}{\text{守}}}$ らなかった。 \hfill\break
 \emph{Kawashita j }\emph{ū no yakusoku wa hitotsu mo mamoranakatta. \hfill\break
 }You didn\textquotesingle t even protect one of the ten promises we made. }

\par{\textbf{Grammar Note }: Ex. 15 shows an example of a number being used without a counter. This is because the series for counting “promises” switches from - \emph{tsu }つ to no counter at all after 9. The counter - \emph{ko } ${\overset{\textnormal{こ}}{\text{個}}}$ , which is to be discussed shortly, could be used. The choice is up to the speaker in situations like this. }

\par{5. Used when ordering things: }

\par{16. ラーメン\{ ${\overset{\textnormal{ひと}}{\text{1}}}$ つ・ ${\overset{\textnormal{いっ}}{\text{1}}}$ ${\overset{\textnormal{ちょう}}{\text{丁}}}$ \}! \hfill\break
 \emph{R }\emph{āmen [hitotsu\slash itch }\emph{ō]! }\hfill\break
One ramen! }

\par{17. ハムサンドイッチとコーヒーを ${\overset{\textnormal{ひと}}{\text{1}}}$ つ ${\overset{\textnormal{くだ}}{\text{下}}}$ さい。 \hfill\break
 \emph{Hamu-sandoitchi to k }\emph{ōhii wo hitotsu kudasai. \hfill\break
 }One ham sandwich and coffee please. }

\par{18. アイスコーヒーを\{ ${\overset{\textnormal{ひと}}{\text{1}}}$ つ・ ${\overset{\textnormal{いっ}}{\text{1}}}$ ${\overset{\textnormal{はい}}{\text{杯}}}$ \} ${\overset{\textnormal{くだ}}{\text{下}}}$ さい。 \hfill\break
 \emph{Aisuk }\emph{ōhii wo [hitotsu\slash ippai] kudasai. \hfill\break
 }One iced coffee please. }

\par{6. Used to replace the proper counter when deemed most convenient: }

\par{19. ${\overset{\textnormal{きんようび}}{\text{金曜日}}}$ に ${\overset{\textnormal{じゅぎょう}}{\text{授業}}}$ は\{いくつ・ ${\overset{\textnormal{なん}}{\text{何}}}$ クラス\}ありますか。 \hfill\break
 \emph{Kin\textquotesingle y }\emph{ōbi ni jugy }\emph{ō wa [ikutsu\slash nankurasu] arimasu ka? \hfill\break
 }How many classes do you have on Friday? }

\par{20. ${\overset{\textnormal{いえ}}{\text{家}}}$ の\{ ${\overset{\textnormal{ふた}}{\text{2}}}$ つ ${\overset{\textnormal{まえ}}{\text{前}}}$ の ${\overset{\textnormal{えき}}{\text{駅}}}$ ・ ${\overset{\textnormal{ふた}}{\text{2}}}$ ${\overset{\textnormal{えきまえ}}{\text{駅前}}}$ \}で ${\overset{\textnormal{ともだち}}{\text{友達}}}$ と ${\overset{\textnormal{ま}}{\text{待}}}$ ち ${\overset{\textnormal{あ}}{\text{合}}}$ わせました。 \hfill\break
 \emph{Ie no [futatsu-mae no eki\slash futaeki-mae] de tomodachi to machiawasemashita. \hfill\break
 }I met with a friend at the station two stations before my place. }

\par{\textbf{Usage Note }: It is this usage that causes students problems. This is because the decision to replace the proper counter with - \emph{tsu }つ is one that is often very difficult even to native speakers. As such, it is best to use the counters you know and listen to when natives use - \emph{tsu }つ. }

\par{7. Used in set phrases: }

\par{ 21. ${\overset{\textnormal{しちょうしゃ}}{\text{視聴者}}}$ (たち)の ${\overset{\textnormal{こころ}}{\text{心}}}$ がひとつになったでしょう。 \hfill\break
 \emph{Shich }\emph{ōsha(-tachi) no kokoro ga hitotsu ni natta deshō. \hfill\break
 }The hearts of the viewers were surely one. }
      
\section{The Counter -ko 個}
 
\par{ The counter - \emph{ko } ${\overset{\textnormal{こ}}{\text{個}}}$ can be used to count round items or items that form a cluster. Incidentally, this counter is also used by some speakers to count anything. This comes from how it is used to count things when a specific counter can\textquotesingle t be thought of by the speaker. In a way, you can view it as meaning “article” as in “an article of belongings.” }

\begin{ltabulary}{|P|P|P|P|P|P|P|P|}
\hline 

1 & いっこ & 2 & にこ & 3 & さんこ & 4 & よんこ \\ \cline{1-8}

5 & ごこ & 6 & ろっこ & 7 & ななこ & 8 &  \textbf{はちこ }\hfill\break
はっこ \\ \cline{1-8}

9 & きゅうこ & 10 &  \textbf{じゅっこ }\hfill\break
じっこ & 100 & ひゃっこ & ? & なんこ \\ \cline{1-8}

\end{ltabulary}

\par{ Though generic, it accounts for physical items in general with which categorization is irrelevant. Even when it is used with something that is not necessary a physical item, the thing in question will be treated as if it were a concrete item. This is a fundamental difference between it and - \emph{tsu }つ. Thus, although - \emph{ko } ${\overset{\textnormal{こ}}{\text{個}}}$ may be more frequently used because of prestige it gets as a Sino-Japanese word, the number of things it can be used with (provided the speaker isn\textquotesingle t one who uses it with anything and everything) is less than with - \emph{tsu }つ. Nevertheless, the two do still overlap as some of the examples below illustrate. }

\par{\textbf{Variation Note }: In the examples below, variants are mentioned regardless of whether they\textquotesingle ve been taught by this point or not. Because - \emph{ko } ${\overset{\textnormal{こ}}{\text{個}}}$ does at times replace the ‘proper\textquotesingle  counter, it\textquotesingle s best to know your options. }

\par{22. ${\overset{\textnormal{けさ}}{\text{今朝}}}$ 、 ${\overset{\textnormal{たまご}}{\text{卵}}}$ を\{ ${\overset{\textnormal{さん}}{\text{3}}}$ ${\overset{\textnormal{こ}}{\text{個}}}$ ・ ${\overset{\textnormal{みっ}}{\text{3}}}$ つ\} ${\overset{\textnormal{た}}{\text{食}}}$ べました。 \hfill\break
 \emph{Kesa, tamago wo [sanko\slash mittsu] tabemashita. \hfill\break
 }I ate three eggs this morning. }

\par{23. リンゴを\{ ${\overset{\textnormal{いっ}}{\text{1}}}$ ${\overset{\textnormal{こ}}{\text{個}}}$ ・ ${\overset{\textnormal{ひと}}{\text{1}}}$ ${\overset{\textnormal{たま}}{\text{玉}}}$ \} ${\overset{\textnormal{か}}{\text{買}}}$ いました。 \hfill\break
 \emph{Ringo wo [ikko\slash hitotama] kaimashita. \hfill\break
 }I bought one apple. }

\par{\textbf{Spelling Note }: \emph{Ringo }is only seldom spelled as 林檎. }

\par{24. ${\overset{\textnormal{しろ}}{\text{白}}}$ い ${\overset{\textnormal{たま}}{\text{玉}}}$ が ${\overset{\textnormal{に}}{\text{2}}}$ ${\overset{\textnormal{こ}}{\text{個}}}$ あります。 \hfill\break
 \emph{Shiroi tama ga niko arimasu. \hfill\break
 }There are two white balls\slash beads. }

\par{25. ${\overset{\textnormal{にもつ}}{\text{荷物}}}$ は ${\overset{\textnormal{なんこ}}{\text{何個}}}$ ですか。 \hfill\break
 \emph{Nimotsu wa nanko desu ka? \hfill\break
 }How many parcels\slash bags do you have? }

\par{26. ${\overset{\textnormal{あたら}}{\text{新}}}$ しい ${\overset{\textnormal{せいひん}}{\text{製品}}}$ が ${\overset{\textnormal{いち}}{\text{1}}}$ ${\overset{\textnormal{まんこ}}{\text{万個}}}$ あります。 \hfill\break
 \emph{Atarashii seihin ga ichimanko arimasu. \hfill\break
 }There are ten thousand new products. }

\par{27. ${\overset{\textnormal{うし}}{\text{牛}}}$ には ${\overset{\textnormal{い}}{\text{胃}}}$ が\{ ${\overset{\textnormal{よん}}{\text{4}}}$ ${\overset{\textnormal{こ}}{\text{個}}}$ ・ ${\overset{\textnormal{よっ}}{\text{4}}}$ つ\}もあります。 \hfill\break
 \emph{Ushi ni wa i ga [yonko\slash yottsu] mo arimasu. \hfill\break
 }A cow has \emph{four }stomachs. }

\par{28. ${\overset{\textnormal{にんげん}}{\text{人間}}}$ には ${\overset{\textnormal{さいぼう}}{\text{細胞}}}$ が\{ ${\overset{\textnormal{なんこ}}{\text{何個}}}$ ・いくつ\}ありますか。 \hfill\break
 \emph{Ningen ni wa saib }\emph{ō ga [nanko\slash ikutsu] arimasu ka? \hfill\break
 }How many cells does a human have? }

\par{29. ${\overset{\textnormal{つ}}{\text{積}}}$ み ${\overset{\textnormal{き}}{\text{木}}}$ は ${\overset{\textnormal{なんこ}}{\text{何個}}}$ ありますか。 \hfill\break
 \emph{Tsumiki wa nanko arimasu ka? \hfill\break
 }How many building blocks are there? }

\par{30. ${\overset{\textnormal{とけい}}{\text{時計}}}$ は ${\overset{\textnormal{なんこ}}{\text{何個}}}$ ありますか。 \hfill\break
 \emph{Tokei wa nanko arimasu ka? \hfill\break
 }How many watches are there\slash do you have? }

\par{\textbf{Counter Note }: For arm watches ( \emph{udedokei }${\overset{\textnormal{うでどけい}}{\text{腕時計}}}$ ), the counters - \emph{ko } ${\overset{\textnormal{こ}}{\text{個}}}$ and - \emph{hon }${\overset{\textnormal{ほん}}{\text{本}}}$ can be used. For alarm clocks ( \emph{mezamashidokei } ${\overset{\textnormal{めざ}}{\text{目覚}}}$ まし ${\overset{\textnormal{どけい}}{\text{時計}}}$ ), the counter - \emph{ko } ${\overset{\textnormal{こ}}{\text{個}}}$ is used. For wall clocks ( \emph{hashiradokei } ${\overset{\textnormal{はしらどけい}}{\text{柱時計}}}$ ) or those that hang on the wall ( \emph{kakedokei } ${\overset{\textnormal{か}}{\text{掛}}}$ け ${\overset{\textnormal{どけい}}{\text{時計}}}$ ), the counters - \emph{ko } ${\overset{\textnormal{こ}}{\text{個}}}$ and - \emph{dai }${\overset{\textnormal{だい}}{\text{台}}}$ can be used. }

\par{31. ${\overset{\textnormal{しかく}}{\text{四角}}}$ が ${\overset{\textnormal{じゅっ}}{\text{10}}}$ ${\overset{\textnormal{こ}}{\text{個}}}$ あります。 \hfill\break
 \emph{Shikaku ga jukko arimasu. }\hfill\break
There are ten squares. }

\par{32. ピアスの ${\overset{\textnormal{あな}}{\text{穴}}}$ は ${\overset{\textnormal{なんこ}}{\text{何個}}}$ ありますか。 \hfill\break
 \emph{Piasu no ana wa nanko arimasu ka? \hfill\break
 }How many pierces do you have? }

\par{33. ${\overset{\textnormal{ゆめ}}{\text{夢}}}$ は ${\overset{\textnormal{ひゃっ}}{\text{100}}}$ ${\overset{\textnormal{こ}}{\text{個}}}$ あります。 \hfill\break
 \emph{Yume wa hyakko arimasu. \hfill\break
 }I have a hundred dreams. }

\par{34. ${\overset{\textnormal{せいすう}}{\text{整数}}}$ は ${\overset{\textnormal{ぜんぶ}}{\text{全部}}}$ で ${\overset{\textnormal{なんこ}}{\text{何個}}}$ ありますか。 \hfill\break
 \emph{Seis }\emph{ū wa zembu de nanko arimasu ka? \hfill\break
 }How many integers are there in total? }

\par{35. ${\overset{\textnormal{にほん}}{\text{日本}}}$ に、 ${\overset{\textnormal{しちょうそん}}{\text{市町村}}}$ は ${\overset{\textnormal{なんこ}}{\text{何個}}}$ ありますか。 \hfill\break
\emph{Nihon ni, shich }\emph{ōson wa nanko arimasu ka? \hfill\break
}How many municipalities are there in Japan? }

\par{\textbf{Counter Note }: In bureaucratic documentation, municipalities will be counted with each kind functioning as a counter. Meaning, “three cities” would be ${\overset{\textnormal{さん}}{\text{3}}}$ ${\overset{\textnormal{し}}{\text{市}}}$ and five villages would be }

\par{${\overset{\textnormal{ご}}{\text{5}}}$ ${\overset{\textnormal{そん}}{\text{村}}}$ etc. Notice that the Sino-Japanese readings are used in this case. }

\par{36. \{ ${\overset{\textnormal{いっ}}{\text{1}}}$ ${\overset{\textnormal{こした}}{\text{個下}}}$ △・ ${\overset{\textnormal{いっ}}{\text{1}}}$ ${\overset{\textnormal{さいした}}{\text{歳下}}}$ 〇・ひとつ ${\overset{\textnormal{した}}{\text{下}}}$ 〇\}の ${\overset{\textnormal{かれし}}{\text{彼氏}}}$ がいます。 \hfill\break
 \emph{[Ikko-shita\slash issai-shita\slash hitotsu-shita] no kareshi ga imasu. \hfill\break
 }I have a boyfriend who is one year younger. }

\par{37. ジョウロが\{ ${\overset{\textnormal{ご}}{\text{5}}}$ ${\overset{\textnormal{こ}}{\text{個}}}$ ・ ${\overset{\textnormal{ご}}{\text{5}}}$ ${\overset{\textnormal{ほん}}{\text{本}}}$ \}あります。 \hfill\break
 \emph{J }\emph{ōro ga [goko\slash gohon] arimasu. \hfill\break
 }There are five watering cans? }

\par{\textbf{Spelling Note }: ジョウロ can seldom be spelled as 如雨露. }

\par{38. ${\overset{\textnormal{ふうせん}}{\text{風船}}}$ が\{ ${\overset{\textnormal{いっ}}{\text{1}}}$ ${\overset{\textnormal{こ}}{\text{個}}}$ ・ ${\overset{\textnormal{いち}}{\text{1}}}$ ${\overset{\textnormal{まい}}{\text{枚}}}$ ・ ${\overset{\textnormal{いっ}}{\text{1}}}$ ${\overset{\textnormal{ほん}}{\text{本}}}$ \}あります。 \hfill\break
 \emph{F }\emph{ūsen ga [ikko\slash ichimai\slash ippon] arimasu. \hfill\break
 }There is one balloon. }

\par{\textbf{Counter Note }: When not inflated, balloons are counted with - \emph{mai }${\overset{\textnormal{まい}}{\text{枚}}}$ . When balloons are shaped in long, cylindrical shapes, they\textquotesingle re counted with - \emph{hon } ${\overset{\textnormal{ほん}}{\text{本}}}$ . When counting typical inflated balloons, you use - \emph{ko } ${\overset{\textnormal{こ}}{\text{個}}}$ . }

\par{39. クモの ${\overset{\textnormal{す}}{\text{巣}}}$ が\{5個・5つ\}あります。 \hfill\break
 \emph{Kumo no su ga [goko\slash itsutsu] arimasu. \hfill\break
 }There are five spider webs. }

\par{\textbf{Spelling Note }: \emph{Kumo }is occasionally spelled as 蜘蛛. }

\par{40. ジャガイモが ${\overset{\textnormal{よん}}{\text{4}}}$ ${\overset{\textnormal{こ}}{\text{個}}}$ あります。 \hfill\break
 \emph{Jagaimo ga yonko arimasu. \hfill\break
 }There are four potatoes. }

\par{\textbf{Spelling Note }: \emph{Jagaimo }is often spelled as じゃが芋. }
    