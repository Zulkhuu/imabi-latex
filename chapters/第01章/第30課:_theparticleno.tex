    
\chapter{The Particle の I}

\begin{center}
\begin{Large}
第30課: The Particle の I  
\end{Large}
\end{center}
 
\par{ The particle の is an essential particle. Although much can be said about it, this lesson will solely be about its most important usage, which is its role as an attribute marker. We attribute details to everything, and in Japanese, this particle is what makes these attribute phrases possible. }
      
\section{The Case Particle の}
 
\par{ の's primary role is to indicate that Noun 1 is an attribute of Noun 2, but Noun 2 will always be the main noun of the phrase. Treating の as the Japanese equivalent of "of" may be tempting, but know that both "of" and の have usages the other does not. }

\begin{center}
 \textbf{Examples }
\end{center}

\par{ Notice how in all the examples below, の helps \emph{qualify }the noun that follows. }

\par{${\overset{\textnormal{たかだい}}{\text{1. 高台}}}$ の ${\overset{\textnormal{たてもの}}{\text{建物}}}$ \hfill\break
A building on high ground }
 
\par{${\overset{\textnormal{こおり}}{\text{2. 氷}}}$ の上を ${\overset{\textnormal{すべ}}{\text{滑}}}$ る。 \hfill\break
To slide on top of ice. }
 
\par{3. ニューヨークの冬はとても寒いですね。 \hfill\break
New York winters are really cold, isn't it? }
 
\par{4. ハワイの ${\overset{\textnormal{かいがん}}{\text{海岸}}}$ はきれいです。 \hfill\break
The Hawaiian coast is pretty. }

\par{${\overset{\textnormal{きせつ}}{\text{5. 季節}}}$ の ${\overset{\textnormal{うつ}}{\text{移}}}$ り ${\overset{\textnormal{か}}{\text{変}}}$ わりは ${\overset{\textnormal{おもしろ}}{\text{面白}}}$ い。 \hfill\break
The changing of the seasons is interesting. }

\par{6. 指(の)先を ${\overset{\textnormal{やけど}}{\text{火傷}}}$ する。 \hfill\break
To burn the tip of one's fingers. }

\par{${\overset{\textnormal{ふぶき}}{\text{7. 吹雪}}}$ の中を ${\overset{\textnormal{そうさく}}{\text{捜索}}}$ する。 \hfill\break
To search through the blizzard. }

\par{${\overset{\textnormal{おおがた}}{\text{8. 大型}}}$ (の)クラゲが ${\overset{\textnormal{しゅつげん}}{\text{出現}}}$ した。 \hfill\break
A big jelly fish appeared. }

\par{9. ${\overset{\textnormal{わたしあて}}{\text{私宛}}}$ のメッセージはありますか。 \hfill\break
Are there any messages for me? }

\par{\textbf{Definition Note }: 宛(て) is "for" as in directed to. }
 
\par{10. 私はクラスの ${\overset{\textnormal{いいん}}{\text{委員}}}$ を ${\overset{\textnormal{つと}}{\text{務}}}$ めました。 \hfill\break
I served on the class committee. }
 
\par{11. 世界(の) ${\overset{\textnormal{へいわ}}{\text{平和}}}$ をもたらす。 \hfill\break
To achieve world peace. }
 
\par{\textbf{Particle Note }: At times の is dropped. This is common in long chains of Sino-Japanese words. Particles are dropped when the grammatical relation the particle marks is obvious in context. In this context, の's role is obvious. So, it may be dropped. Japanese has a lot of compound nouns, so those would be examples of の not being needed. When it doubt, use の. It's not a guarantee what you'll say is 100\% natural, but it will guarantee that you're being grammatically correct. }

\par{12. その赤ちゃんの顔は日焼けしている。 \hfill\break
The baby's face is sunburned. }

\par{13. (私は)彼の時計の ${\overset{\textnormal{じこく}}{\text{時刻}}}$ を ${\overset{\textnormal{なお}}{\text{直}}}$ しました。 \hfill\break
I fixed the time on his watch. }

\par{\textbf{Word Note }: 時刻 is used rather than 時間, which only refers to a period of time. Also note that 時計 is read as とけい. }

\par{${\overset{\textnormal{ひしょ}}{\text{14. 秘書}}}$ の ${\overset{\textnormal{じょうし}}{\text{上司}}}$ VS 上司の秘書 \hfill\break
The secretary's boss VS  The boss's secretary. }
 
\par{\textbf{Phrase Note }: The difference between these phrases is simply the relationship you're trying to show. Switch the nouns and you'll switch the roles. The particle の doesn't change at all. }
 
\par{ の can also follow other case particles--never が nor に. It can also be after some adverbs. When this happens, it is best to treat the adverb as a nominal phrase. }
 
\par{15. 神への道 \hfill\break
Road to God\slash the gods }
 
\par{16. 母からの手紙 \hfill\break
A letter from my mother }
 
\par{17.しばらくの間 \hfill\break
For a while }
 
\par{\textbf{Practice (1) }: Translate the following. You may use a dictionary. }
 
\par{1. To fix part of a computer. \hfill\break
2. She is the secretary of the company president. \hfill\break
3. A person with pretty eyes. \hfill\break
4. Takamura the bank employee. \hfill\break
5. This is a Japanese textbook. \hfill\break
6. Hokkaido in the winter. \hfill\break
7. I have three children. \hfill\break
8. Her house is beautiful. }
 
\par{ We can describe の in the following situations. Remember that these are just different situations of の qualifying a noun to describe what its attributes are. The grammar is the same. }
 
\begin{enumerate}
 
\item Shows the \textbf{nature of something }. 
\begin{itemize}
 
\item It may mark possession. Ex. 私の ${\overset{\textnormal{とけい}}{\text{時計}}}$ = \textbf{my }watch.  
\item It may mark location or time. Ex. 夏のテキサス = Texas       in summer.  
\item It may mark a characteristic. Ex. 木の ${\overset{\textnormal{はし}}{\text{橋}}}$ =       wooden bridge.  
\item It may mark quantity. Ex. 3つの大学 = three colleges.  
\item It may indicate position. Ex. ${\overset{\textnormal{しゃちょう}}{\text{社長}}}$ の ${\overset{\textnormal{まさき}}{\text{正木}}}$ =       Masaki the company president.  
\end{itemize}
 Shows the \textbf{standard for a relative relationship }. 
\begin{itemize}
 
\item It may show a part related to a whole. Ex. ペンの ${\overset{\textnormal{さき}}{\text{先}}}$ =       tip of pen.  
\item It may show a relative placement. This placement may be       physical, temporal, etc. Ex. の ${\overset{\textnormal{あと}}{\text{後}}}$ means       "after". It is also how you show the location of something.  
\end{itemize}
 Shows the \textbf{characteristics of affairs }. 
\begin{itemize}
 
\item It can show solid information about a matter. Ex. ${\overset{\textnormal{じこ}}{\text{事故}}}$ の ${\overset{\textnormal{ほうこく}}{\text{報告}}}$ =       accident report.  
\item It may show purpose. Ex. ${\overset{\textnormal{しゅっちょう}}{\text{出張}}}$ の ${\overset{\textnormal{じゅんび}}{\text{準備}}}$ =       business trip preparations.  
\item It may show the object of action. Ex. 日本語の勉強をする = to       study Japanese. 
\end{itemize}

\end{enumerate}

\begin{center}
\textbf{Conflicts with Other Means of Making Attribute Expressions }
\end{center}
 
\par{ Adjectives also make attributes. There are also words like 特別 for which な or の are used to create attributes. One form may be more common than the other, but there may also be semantic differences. }

\par{ばかな人 VS ばかの人 \hfill\break
 \hfill\break
 The first is "stupid person", but the second sounds like "the person of an idiot", which isn't something you'd necessarily say. }

\par{ Verbs can be used as attributes as well. So, we get some variation in phrases such as in どしゃ降りの雨 and どしゃ降る雨 being acceptable for "pouring rain," though the first is more proper. In similar expressions, both may be fine or be narrowed in usage. }
 
\par{18. 平等の権利を与える。 \hfill\break
To give equal rights. }
 
\par{19. 平等な\{扱い・世界・社会・問題・ルール\} \hfill\break
Fair \{treatment\slash world\slash society\slash problem\slash rule\} }

\par{${\overset{\textnormal{なぐ}}{\text{20. 殴}}}$ る ${\overset{\textnormal{け}}{\text{蹴}}}$ るの ${\overset{\textnormal{ぼうりょく}}{\text{暴力}}}$ を ${\overset{\textnormal{う}}{\text{受}}}$ ける。 \hfill\break
To receive violence of punches and kicks. }

\par{\textbf{Part of Speech Note }: There are rare expressions with の after verbs to make an attribute phrase. Treat these instances as separate words. You don't have to bother yourself with this example, but it is grammatically intriguing. }

\begin{center}
\textbf{A }\textbf{+の+れんたいけい+の }
\end{center}

\par{In this pattern it moreover limits something in noting a condition about what is expressed by the noun. It's like に ${\overset{\textnormal{}}{\text{}}}$ して, which means "in regard to". }

\par{コーヒーの ${\overset{\textnormal{}}{\text{}}}$ めたの \hfill\break
Cold in regards to coffee }

\par{コーヒーに ${\overset{\textnormal{}}{\text{}}}$ しては ${\overset{\textnormal{}}{\text{}}}$ たい。 \hfill\break
It's cold in relation to coffee. }

\par{オレンジの ${\overset{\textnormal{}}{\text{}}}$ りなの \hfill\break
A comparatively small orange }

\par{Note: The word 小振り is normally only used in reference to things like fruits, fish, etc. However, its usage may be expanded some in speaking. Nevertheless, something like 小振りなテレビ is very weird. You should use ${\overset{\textnormal{}}{\text{}}}$ のテレビ instead. }
      
\section{Key}
 
\par{Practice }

\par{1. コンピュータ(ー)の ${\overset{\textnormal{}}{\text{一部}}}$ を ${\overset{\textnormal{}}{\text{直}}}$ す。 \hfill\break
2. ${\overset{\textnormal{}}{\text{彼女}}}$ は ${\overset{\textnormal{}}{\text{社長}}}$ の ${\overset{\textnormal{}}{\text{秘書}}}$ です。 \hfill\break
3. きれいな ${\overset{\textnormal{}}{\text{眼}}}$ の ${\overset{\textnormal{}}{\text{人}}}$ 。 \hfill\break
4. ${\overset{\textnormal{ぎんこういん}}{\text{銀行員}}}$ の ${\overset{\textnormal{}}{\text{高村}}}$ 。 \hfill\break
5. これは ${\overset{\textnormal{}}{\text{日本語}}}$ の ${\overset{\textnormal{}}{\text{教科書}}}$ だ。 \hfill\break
6. ${\overset{\textnormal{}}{\text{冬}}}$ の ${\overset{\textnormal{}}{\text{北海道}}}$ 。 \hfill\break
7. ${\overset{\textnormal{}}{\text{子供}}}$ が ${\overset{\textnormal{}}{\text{三人}}}$ いる。 \hfill\break
8. 彼女の ${\overset{\textnormal{いえ}}{\text{家}}}$ は ${\overset{\textnormal{}}{\text{美}}}$ しい。 }

\par{Note: Sometimes, trick question are needed to see if you can learn the content at hand and know whether or not it is the most natural grammar point to use in a given situation. }
    