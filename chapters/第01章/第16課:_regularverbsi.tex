    
\chapter{Regular Verbs I}

\begin{center}
\begin{Large}
第16課: Regular Verbs I: 一段 Ichidan Verbs 
\end{Large}
\end{center}
 
\par{In Japanese, there is a handful of verb classes that differ in conjugation. In this lesson, we will specifically only learn about verbs called " \emph{ru }る verbs," more specifically called \emph{Ichidan }一段 verbs. This class constitutes half of all verbs in Japanese. }

\par{Before we learn about these verbs, let's go over some basic grammatical terminology. }

\begin{ltabulary}{|P|P|}
\hline 

Verb & An action or state of being. \\ \cline{1-2}

Auxiliary Verb & An ending that shows some grammatical function. \\ \cline{1-2}

Transitive Verb & A (willful) act that takes a direct object. \\ \cline{1-2}

Intransitive Verb & An action, state of being, or happening that does not take a direct object. \\ \cline{1-2}

\end{ltabulary}

\par{Auxiliary verbs, often called "helper verbs," are not standalone words like in English. They must always be attached to verbs. Verbs and auxiliary verbs conjugate, and their conjugations are always very systematic with hardly any exceptions. }
      
\section{Vocabulary List}
 
\par{\textbf{Nouns }}

\par{・一段 \emph{Ichidan }– \emph{ru }verbs }

\par{・看板 \emph{Kamban }- Billboard }

\par{・服 \emph{Fuku }– Clothes }

\par{・時間 \emph{Jikan }– Time }

\par{・野菜 \emph{Yasai }– Vegetables }

\par{・ドア \emph{Doa }– Door }

\par{・教科書 \emph{Kyōkasho }– Textbook }

\par{・絵 \emph{E }– Picture }

\par{・切手 \emph{Kitte }– Stamp }

\par{・クマ \emph{Kuma }– Bear }

\par{・木 \emph{Ki }– Tree }

\par{・言葉 \emph{Kotoba }– Word\slash language }

\par{・人口 \emph{Jinkō }- Population }

\par{・調味料 \emph{Chōmiryō }– Spice(s) }

\par{・引き戸 \emph{Hikido }– Sliding door }

\par{・シャワー \emph{Shawā }– Shower }

\par{・足 \emph{Ashi }– Foot }

\par{・心 \emph{Kokoro }– Heart (emotional entity) }

\par{・(お)肉 \emph{(O)niku }- Meat }

\par{・神 \emph{Kami }– God\slash god\slash deity\slash kami }

\par{・証拠 \emph{Shōko }– Proof\slash evidence }

\par{・ごみ \emph{Gomi }– Trash\slash rubbish }

\par{・話題 \emph{Wadai }– Topic }

\par{・ピザ \emph{Piza }– Pizza }

\par{・内容 \emph{Naiyō }– Content }

\par{・魚 \emph{Sakana }– Fish }

\par{\textbf{Pronouns }}

\par{・僕 \emph{Boku }– I (male) }

\par{・彼 \emph{Kare }– He }

\par{\textbf{Adjectives }}

\par{・汚い \emph{Kitanai }– Dirty }

\par{\textbf{Demonstratives }}

\par{・あの \emph{Ano }– That (adj.) }

\par{\textbf{Adverbs }}

\par{・毎朝 \emph{Maiasa }– Every morning }

\par{・ちょっと \emph{Chotto }– A little }

\par{・だいぶ \emph{Daibu }– Fairly\slash considerably }

\par{・今夜 \emph{Kon\textquotesingle ya }– Tonight }

\par{・何も \emph{Nani mo }– Nothing }

\par{・特に \emph{Toku ni }– Particularly }

\par{・なかなか \emph{Nakanaka }– Considerably\slash not really (neg.) }

\par{・今日 \emph{Kyō }– Today }
 \textbf{Verbs }
\par{・食べる \emph{Taberu }– To eat (trans.) }

\par{・見る \emph{Miru }– To see\slash look at (trans.) }

\par{・着る \emph{Kiru }– To wear (trans.) }

\par{・起きる \emph{Okiru }– To occur\slash get up (trans.) }

\par{・過ぎる \emph{Sugiru }– To pass (intr.) }

\par{・煮る \emph{Niru }– To simmer (intr.) }

\par{・閉じる \emph{Tojiru }– To  close (trans.) }

\par{・落ちる \emph{Ochiru }– To drop\slash fall (intr.) }

\par{・借りる \emph{Kariru }– To borrow (trans.) }

\par{・考える \emph{Kangaeru }– To think\slash ponder\slash intend (trans.) }

\par{・覚える \emph{Oboeru }– To remember (trans.) }

\par{・倒れる \emph{Taoreru }– To fall down\slash collapse (trans.) }

\par{・見つける \emph{Mitsukeru }– To find (trans.) }

\par{・集める \emph{Atsumeru }– To collect (trans.) }

\par{・慣れる \emph{Nareru }– To get used to (trans.) }

\par{・得る \emph{Eru }– To get (trans.) }

\par{・増える \emph{Fueru }– To increase (intr.) }

\par{・並べる \emph{Naraberu }– To line up\slash put in order (intr.) }

\par{・混ぜる \emph{Mazeru }– To mix (trans.) }

\par{・植える \emph{Ueru }– To plant (trans.) }

\par{・壊れる \emph{Kowareru }– To break (intr.) }

\par{・浴びる \emph{Abiru }– To bathe in (trans.) }

\par{・負ける \emph{Makeru }– To lose (intr.) }

\par{・濡れる \emph{Nureru }– To get wet (intr.) }

\par{・答える \emph{Kotaeru }– To answer (intr.) }

\par{・信じる \emph{Shinjiru }– To believe (trans.) }

\par{・感じる \emph{Kanjiru }– To feel\slash sense (trans.) }

\par{・出る \emph{Deru }– To go out (intr.) }

\par{・寝る \emph{Neru }– To sleep (intr.) }

\par{・消える \emph{Kieru }– To vanish\slash disappear (intr.) }

\par{・枯れる \emph{Kareru }– To wither (intr.) }

\par{・捨てる \emph{Suteru }– To throw away (trans.) }

\par{・忘れる \emph{Wasureru }– To forget (trans.) }

\par{・加える \emph{Kuwaeru }– To add to (trans.) }

\par{・降りる \emph{Oriru }– To get off\slash go down (intr.) }

\par{・変える \emph{Kaeru }– To change (trans.) }

\par{・認める \emph{Mitomeru }– To admit\slash recognize (trans.) }

\par{・焦げる \emph{Kogeru }– To get burned\slash charred (intr.) }

\par{・調べる \emph{Shiraberu }– To check\slash investigate (trans.) }

\par{・漏れる \emph{Moreru }– To leak (intr.) }

\par{・晴れる \emph{Hareru }– To clear up (intr.) }

\par{・もてる \emph{Moteru }– To be popular (intr.) }
      
\section{Conjugating Verbs}
 
\par{ Just as with adjectives and adjectival nouns, the basic tenses of Japanese are non-past (present or future) and past. Although tense may not be the best word to describe what goes on in Japanese, we'll stick to it to have things feel more familiar. }

\begin{center}
\textbf{Plain Non-Past Form: No Conjugation } 
\end{center}

\par{ Unlike the previous parts of speech we've covered, no conjugation is ever required to use the non-past tense in plain speech. This includes times when you use a verb to modify a noun. That means you won't have to learn any rules like \emph{da }だ becoming \emph{na }な. }

\par{ It is when you use polite speech that goes beyond the basic form that you begin to conjugate. Nevertheless, as you will see, the non-past form can also stand for the gerund, which is the "to\dothyp{}\dothyp{}\dothyp{}" form of verbs in English. This means quite a few slightly different things can be expressed just with the basic form of a verb. }

\par{1. ${\overset{\textnormal{かんばん}}{\text{看板}}}$ を ${\overset{\textnormal{み}}{\text{見}}}$ る。 \hfill\break
\emph{Kamban wo miru. }\hfill\break
To look at a billboard. }

\par{2. ${\overset{\textnormal{きたな}}{\text{汚}}}$ い ${\overset{\textnormal{ふく}}{\text{服}}}$ を ${\overset{\textnormal{き}}{\text{着}}}$ る。 \hfill\break
\emph{Kitanai fuku wo kiru. }\hfill\break
To wear dirty clothes. }

\par{3. ${\overset{\textnormal{まいあさお}}{\text{毎朝起}}}$ きる。 \hfill\break
\emph{Maiasa okiru. }\hfill\break
To happen every morning\slash To get up every morning. }

\par{4. ${\overset{\textnormal{じかん}}{\text{時間}}}$ が ${\overset{\textnormal{す}}{\text{過}}}$ ぎる。 \hfill\break
\emph{Jikan ga sugiru. }\hfill\break
Time passes. }

\par{5. ${\overset{\textnormal{やさい}}{\text{野菜}}}$ を ${\overset{\textnormal{に}}{\text{煮}}}$ る。 \hfill\break
\emph{Yasai wo niru. }\hfill\break
To simmer vegetables. }

\begin{center}
 \textbf{Polite Non-Past Form: - \emph{masu }ます }
\end{center}

\par{ To make an \emph{Ichidan }一段 verb polite in the non-past tense, drop \emph{ru }る and add \emph{-masu ます. } }

\begin{ltabulary}{|P|P|P|}
\hline 

Meaning & Verb &  \emph{Drop ru る, add -masu ます }\\ \cline{1-3}

To close &  \emph{Tojiru }閉じる &  \emph{Tojimasu }閉じます \\ \cline{1-3}

To fall\slash drop &  \emph{Ochiru }落ちる &  \emph{Ochimasu }落ちます \\ \cline{1-3}

To borrow &  \emph{Kariru }借りる &  \emph{Karimasu }借ります \\ \cline{1-3}

To think &  \emph{Kangaeru }考える &  \emph{Kangaemasu }考えます \\ \cline{1-3}

To remember &  \emph{Oboeru }覚える &  \emph{Oboemasu }覚えます \\ \cline{1-3}

\end{ltabulary}

\par{6. ドアを ${\overset{\textnormal{と}}{\text{閉}}}$ じます。 \hfill\break
\emph{Doa wo tojimasu. \hfill\break
}I\textquotesingle ll close the door. }

\par{7. ちょっと ${\overset{\textnormal{かんが}}{\text{考}}}$ えます。 \hfill\break
\emph{Chotto kangaemasu. }\hfill\break
I\textquotesingle ll think about it. }

\par{8. ${\overset{\textnormal{きょうかしょ}}{\text{教科書}}}$ を ${\overset{\textnormal{か}}{\text{借}}}$ ります。 \hfill\break
\emph{Kyōkasho wo karimasu. \hfill\break
}I\textquotesingle ll borrow the textbook. }

\par{\textbf{Usage Note }: This form almost always cannot modify nouns. Meaning, you can't place it before a noun for the purpose of modifying said noun. To modify nouns with verbs in the non-past tense, you must use the plain form without \emph{-masu }ます. }

\par{Ex. よく ${\overset{\textnormal{か}}{\text{借}}}$ りる ${\overset{\textnormal{きょうかしょ}}{\text{教科書}}}$ 〇 \hfill\break
\emph{Yoku kariru kyōkasho }\hfill\break
よく ${\overset{\textnormal{か}}{\text{借}}}$ ります ${\overset{\textnormal{きょうかしょ}}{\text{教科書}}}$ X \hfill\break
\emph{Yoku karimasu kyōkasho }\hfill\break
Textbook(s) that I often borrow }

\begin{center}
\textbf{Plain Past Form: \emph{-ta }た }
\end{center}

\par{ To make an \emph{Ichidan }一段 verb past tense in plain speech, drop \emph{ru }る and add \emph{-ta た }. }

\begin{ltabulary}{|P|P|P|}
\hline 

Meaning & Verb &  \emph{Drop ru る, add -ta た }\\ \cline{1-3}

To see\slash look &  \emph{Miru }見る &  \emph{Mita }見た \\ \cline{1-3}

To fall down\slash collapse &  \emph{Taoreru }倒れる &  \emph{Taoreta }倒れた \\ \cline{1-3}

To find &  \emph{Mitsukeru }見つける & \emph{Mitsuketa }見つけた \\ \cline{1-3}

To gather\slash collect &  \emph{Atsumeru }集める &  \emph{Atsumeta }集めた \\ \cline{1-3}

To get used to &  \emph{Nareru }慣れる &  \emph{Nareta }慣れた \\ \cline{1-3}

\end{ltabulary}

\par{\textbf{Usage Note }: This form can also be used to modify nouns without any change in form. Most conjugations are able to do so as long as they are not in their polite forms. }

\par{9. あの ${\overset{\textnormal{え}}{\text{絵}}}$ を ${\overset{\textnormal{み}}{\text{見}}}$ た。 \hfill\break
\emph{Ano e wo mita. }\hfill\break
I saw that picture. }

\par{10. ${\overset{\textnormal{きって}}{\text{切手}}}$ を ${\overset{\textnormal{あつ}}{\text{集}}}$ めた。 \hfill\break
\emph{Kitte wo atsumeta. \hfill\break
}I collected stamps. }
 
\par{11. クマが ${\overset{\textnormal{たお}}{\text{倒}}}$ れた。 \hfill\break
\emph{Kuma ga taoreta. \hfill\break
}The bear collapsed. }

\par{12. だいぶ ${\overset{\textnormal{な}}{\text{慣}}}$ れた。 \hfill\break
\emph{Daibu nareta. }\hfill\break
I've gotten fairly  used to it. }

\par{\textbf{Grammar Note }: Perfect past tense involving "to have\dothyp{}\dothyp{}\dothyp{}" in English is usually translated with the past tense. }

\begin{center}
\textbf{Polite Past Form: \emph{-mashita まし }た } 
\end{center}

\par{ To make an \emph{Ichidan }一段 verb past tense in polite speech, drop \emph{ru }る and add \emph{-mashita ました }. }

\begin{ltabulary}{|P|P|P|}
\hline 

Meaning & Verb &  \emph{Drop ru る, add -mashita ました }\\ \cline{1-3}

To get &  \emph{Eru }得る &  \emph{Emashita }得ました \\ \cline{1-3}

To increase &  \emph{Fueru }増える &  \emph{Fuemashita }増えました \\ \cline{1-3}

To line up\slash arrange in order &  \emph{Naraberu }並べる & \emph{Narabemashita }並べました \\ \cline{1-3}

To mix &  \emph{Mazeru }混ぜる &  \emph{Mazemashita }混ぜました \\ \cline{1-3}

To plant &  \emph{Ueru }植える &  \emph{Uemashita }植えました \\ \cline{1-3}

\end{ltabulary}

\par{13. ${\overset{\textnormal{き}}{\text{木}}}$ を ${\overset{\textnormal{う}}{\text{植}}}$ えました。 \hfill\break
\emph{Ki wo uemashita. \hfill\break
}I planted a tree. }

\par{14. ${\overset{\textnormal{ことば}}{\text{言葉}}}$ を ${\overset{\textnormal{なら}}{\text{並}}}$ べました。 \hfill\break
\emph{Kotoba wo narabemashita. \hfill\break
}I lined up the words. }
 
\par{15. ${\overset{\textnormal{じんこう}}{\text{人口}}}$ が ${\overset{\textnormal{ふ}}{\text{増}}}$ えました。 \hfill\break
\emph{Jinkō ga fuemashita. }\hfill\break
The population increased\slash grew. }

\par{16. ${\overset{\textnormal{ちょうみりょう}}{\text{調味料}}}$ を ${\overset{\textnormal{ま}}{\text{混}}}$ ぜました。 \hfill\break
\emph{Chōmiryō wo mazemashita. \hfill\break
}I mixed in spices. }

\par{\textbf{Usage Note }: Usually, this form cannot modify nouns. Meaning, you can't place it before a noun for the purpose of modifying said noun. To modify nouns with verbs in the past tense, you must use the plain form \emph{-ta }た. }

\par{Ex. ${\overset{\textnormal{ま}}{\text{混}}}$ ぜた ${\overset{\textnormal{ちょうみりょう}}{\text{調味料}}}$ 〇 \hfill\break
\emph{Mazeta chōmiryō }\hfill\break
${\overset{\textnormal{ま}}{\text{混}}}$ ぜました ${\overset{\textnormal{ちょうみりょう}}{\text{調味料}}}$ X \hfill\break
\emph{Mazemashita chōmiryō }\hfill\break
Spice(s) that I mixed }

\begin{center}
\textbf{Plain Negative Form: \emph{-nai }ない } 
\end{center}

\par{ To make an \emph{Ichidan }一段 verb negative in plain speech, drop \emph{ru }る and add \emph{-nai ない }. }

\begin{ltabulary}{|P|P|P|}
\hline 

Meaning & Verb &  \emph{Drop ru る, add -nai ない }\\ \cline{1-3}

To break &  \emph{Kowareru }壊れる &  \emph{Kowarenai }壊れない \\ \cline{1-3}

To bathe in &  \emph{Abiru }浴びる &  \emph{Abinai }浴びない \\ \cline{1-3}

To lose &  \emph{Makeru }負ける & \emph{Makenai }負けない \\ \cline{1-3}

To get wet &  \emph{Nureru }濡れる &  \emph{Nurenai }濡れない \\ \cline{1-3}

To answer &  \emph{Kotaeru }答える &  \emph{Kotaenai }答えない \\ \cline{1-3}

\end{ltabulary}

\par{\textbf{Usage Note }: This form can also be used to modify nouns without any change in form. Most conjugations are able to do so as long as they are not in their polite forms. }

\par{17. ${\overset{\textnormal{ひ}}{\text{引}}}$ き ${\overset{\textnormal{ど}}{\text{戸}}}$ が ${\overset{\textnormal{こわ}}{\text{壊}}}$ れない。 \hfill\break
\emph{Hikido ga kowarenai. \hfill\break
}The sliding door won\textquotesingle t break. }

\par{18. シャワーを ${\overset{\textnormal{あ}}{\text{浴}}}$ びない。 \hfill\break
\emph{Shawā wo abinai. }\hfill\break
To not\slash will not take a shower. }

\par{19. ${\overset{\textnormal{ぼく}}{\text{僕}}}$ は ${\overset{\textnormal{ま}}{\text{負}}}$ けない。(Male speech) \hfill\break
\emph{Boku wa makenai. \hfill\break
}I won\textquotesingle t lose. }

\par{20. ${\overset{\textnormal{あし}}{\text{足}}}$ が ${\overset{\textnormal{ぬ}}{\text{濡}}}$ れない。 \hfill\break
\emph{Ashi ga nurenai. \hfill\break
}One\textquotesingle s feet won\textquotesingle t get wet. }

\par{21. ${\overset{\textnormal{ま}}{\text{負}}}$ けない ${\overset{\textnormal{こころ}}{\text{心}}}$ \hfill\break
\emph{Makenai kokoro \hfill\break
}A heart that won\textquotesingle t lose }

\begin{center}
\textbf{Polite Negative Form: \emph{-masen }ません }
\end{center}

\par{ To make an \emph{Ichidan }一段 verb negative in polite speech, drop \emph{ru }る and add \emph{-masen ません }. }

\begin{ltabulary}{|P|P|P|}
\hline 

Meaning & Verb &  \emph{Drop ru る, add -masen ません }\\ \cline{1-3}

To believe &  \emph{Shinjiru }信じる &  \emph{Shinjimasen }信じません \\ \cline{1-3}

To eat &  \emph{Taberu }食べる &  \emph{Tabemasen }食べません \\ \cline{1-3}

To feel\slash sense &  \emph{Kanjiru }感じる & \emph{Kanjimasen }感じません \\ \cline{1-3}

To go out &  \emph{Deru }出る &  \emph{Demasen }出ません \\ \cline{1-3}

To sleep &  \emph{Neru }寝る &  \emph{Nemasen }寝ません \\ \cline{1-3}

\end{ltabulary}

\par{\textbf{Usage Note }: This form almost always cannot modify nouns. Meaning, you can't place it before a noun for the purpose of modifying said noun. To modify nouns with verbs in the negative, you must use the plain form \emph{-nai }ない. }

\par{22. ${\overset{\textnormal{こんや}}{\text{今夜}}}$ ${\overset{\textnormal{ね}}{\text{寝}}}$ ません。 \hfill\break
\emph{Kon\textquotesingle ya nemasen. \hfill\break
}I won\textquotesingle t sleep tonight. }

\par{23. ${\overset{\textnormal{なに}}{\text{何}}}$ も ${\overset{\textnormal{かん}}{\text{感}}}$ じません。 \hfill\break
\emph{Nani mo kanjimasen. \hfill\break
}I don\textquotesingle t feel anything. }
 
\par{24. (お) ${\overset{\textnormal{にく}}{\text{肉}}}$ は ${\overset{\textnormal{た}}{\text{食}}}$ べません。 \hfill\break
\emph{(O-)niku wa tabemasen. }\hfill\break
I don\textquotesingle t\slash won\textquotesingle t eat meat. }

\par{25. ${\overset{\textnormal{かみ}}{\text{神}}}$ を ${\overset{\textnormal{しん}}{\text{信}}}$ じません。 \hfill\break
\emph{Kami wo shinjimasen. \hfill\break
}I don\textquotesingle t believe in God. }

\par{\textbf{Usage Note }: This form almost always cannot modify nouns. Meaning, you can't place it before a noun for the purpose of modifying said noun. To modify nouns with verbs in the negative, you must use the plain form \emph{-nai }ない. }

\par{Ex. ${\overset{\textnormal{た}}{\text{食}}}$ べない (お) ${\overset{\textnormal{にく}}{\text{肉}}}$ 〇 \hfill\break
\emph{T }\emph{abenai (o-)niku \hfill\break
}${\overset{\textnormal{た}}{\text{食}}}$ べません (お) ${\overset{\textnormal{にく}}{\text{肉}}}$ X \hfill\break
\emph{Tabemasen (o-)niku \hfill\break
}Meat that I don't\slash won't eat }

\begin{center}
\textbf{Plain Negative Past Form: \emph{-nakatta }なかった } 
\end{center}

\par{ To make an \emph{Ichidan }一段 verb negative past in plain speech, drop \emph{ru }る and add \emph{-nakatta なかった }. }

\begin{ltabulary}{|P|P|P|}
\hline 

Meaning & Verb &  \emph{Drop ru る, add - }\emph{nakatta なかった }\\ \cline{1-3}

To vanish\slash go out &  \emph{Kieru }消える &  \emph{Kienakatta }消えなかった \\ \cline{1-3}

To wither &  \emph{Kareru }枯れる &  \emph{Karenakatta }枯れなかった \\ \cline{1-3}

To throw away &  \emph{Suteru }捨てる & \emph{ }\emph{Sutenakatta }捨てなかった \\ \cline{1-3}

To forget &  \emph{Wasureru }忘れる &  \emph{Wasurenakatta }忘 \emph{れなかった }\\ \cline{1-3}

To add to &  \emph{Kuwaeru }加える &  \emph{Kuwaenakatta }加えなかった \\ \cline{1-3}

\end{ltabulary}

\par{\textbf{Usage Note }: This form can also be used to modify nouns without any change in form. Most conjugations are able to do so as long as they are not in their polite forms. }

\par{26. ${\overset{\textnormal{き}}{\text{消}}}$ えなかった ${\overset{\textnormal{しょうこ}}{\text{証拠}}}$ \hfill\break
\emph{Kienakatta shōko \hfill\break
}Evidence that didn\textquotesingle t disappear }

\par{27. ${\overset{\textnormal{かれ}}{\text{彼}}}$ はごみを ${\overset{\textnormal{す}}{\text{捨}}}$ てなかった。 \hfill\break
\emph{Kare wa gomi wo sutenakatta. \hfill\break
}He didn\textquotesingle t throw away the trash. }

\par{28. ${\overset{\textnormal{わす}}{\text{忘}}}$ れなかったよ。 \hfill\break
\emph{Wasurenakatta yo. \hfill\break
}I didn\textquotesingle t forget. }

\par{\textbf{Particle Note }: The particle \emph{yo }よ adds exclamation to the sentence. }

\begin{center}
\textbf{Polite Negative Past Form: \emph{-masendeshita }ませんでした } 
\end{center}

\par{ To make an \emph{Ichidan }一段 verb negative past in polite speech, drop \emph{ru }る and add \emph{-masendeshita ませんでした }. }

\begin{ltabulary}{|P|P|P|}
\hline 

Meaning & Verb &  \emph{Drop ru る, add -masendeshita }\emph{ませんでした }\\ \cline{1-3}

To get off\slash go down &  \emph{Oriru }降りる &  \emph{Orimasendeshita }降りませんでした \\ \cline{1-3}

To change &  \emph{Kaeru }変える &  \emph{Kaemasendeshita }変えませんでした \\ \cline{1-3}

To admit\slash recognize &  \emph{Mitomeru }認める & \emph{Mitomemasendeshita }認めませんでした \\ \cline{1-3}

To be burned\slash charred &  \emph{Kogeru }焦げる &  \emph{Kogemasendeshita }焦げませんでした \\ \cline{1-3}

To check\slash investigate &  \emph{Shiraberu }調べる &  \emph{Shirabemasendeshita }調べませんでした \\ \cline{1-3}

\end{ltabulary}

\par{29. ${\overset{\textnormal{わだい}}{\text{話題}}}$ を ${\overset{\textnormal{か}}{\text{変}}}$ えませんでした。 \hfill\break
\emph{Wadai wo kaemasendeshita. } \hfill\break
I didn't change the topic. }

\par{30. ピザを食べませんでした。 \hfill\break
\emph{Piza wo tabemasendeshita. }\hfill\break
I didn't eat pizza. }

\par{31. なかなか ${\overset{\textnormal{こ}}{\text{焦}}}$ げませんでした。 \hfill\break
\emph{Nakanaka kogemasendeshita. \hfill\break
}It didn\textquotesingle t really get charred. }

\par{32. ${\overset{\textnormal{とく}}{\text{特}}}$ に ${\overset{\textnormal{しら}}{\text{調}}}$ べませんでした。 \hfill\break
\emph{Toku ni shirabemasendeshita. \hfill\break
}I didn\textquotesingle t particularly check\slash investigate it. }

\par{\textbf{Usage Note }: This form cannot modify nouns. Meaning, you can't place it before a noun for the purpose of modifying said noun. To modify nouns with verbs in the negative past, you must use the plain form \emph{-nakatta }なかった. }

\par{Ex. ${\overset{\textnormal{た}}{\text{食}}}$ べなかったピザ \hfill\break
\emph{Tabenakatta piza }\hfill\break
${\overset{\textnormal{た}}{\text{食}}}$ べませんでしたピザ \hfill\break
\emph{Tabemasendeshita piza }\hfill\break
The pizza I didn\textquotesingle t eat }

\begin{center}
\textbf{Alternative Polite Negative \& Neg-Past Forms: \emph{-nai desu \& -nakatta desu }ないです・なかったです } 
\end{center}

\par{ To make an \emph{Ichidan }一段 verb negative or negative past in polite yet casual speech, drop \emph{ru }る and add \emph{-nai desu ないです }or \emph{nakatta desu }なかったです respectively. }

\begin{ltabulary}{|P|P|P|}
\hline 

Meaning & Verb &  \emph{Change to Plain Negative\slash Negative-Past, add desu です }\\ \cline{1-3}

To leak &  \emph{Moreru }漏れる &  \emph{Morenai desu }漏れないです \hfill\break
 \emph{Morenakatta desu }漏れなかったです \\ \cline{1-3}

To clear up &  \emph{Hareru }晴れる &  \emph{Harenai desu }晴れないです \hfill\break
 \emph{Harenakatta desu }晴れなかったです \\ \cline{1-3}

To be popular &  \emph{Moteru }もてる &  \emph{Motenai desu }もてないです \hfill\break
 \emph{Motenakatta desu }もてなかったです \\ \cline{1-3}

\end{ltabulary}

\par{33. ${\overset{\textnormal{ないよう}}{\text{内容}}}$ は ${\overset{\textnormal{も}}{\text{漏}}}$ れないです。 \hfill\break
\emph{Naiyō wa morenai desu. }\hfill\break
(The) content won\textquotesingle t leak. }

\par{34. ${\overset{\textnormal{きょう}}{\text{今日}}}$ は ${\overset{\textnormal{は}}{\text{晴}}}$ れなかったです。 \hfill\break
\emph{Kyō wa harenakatta desu. }\hfill\break
It didn\textquotesingle t clear up today. (Weather) }

\par{35. ${\overset{\textnormal{さかな}}{\text{魚}}}$ は ${\overset{\textnormal{た}}{\text{食}}}$ べないです。 \hfill\break
\emph{Sakana wa tabenai desu. \hfill\break
}I don\textquotesingle t eat fish. }

\par{\textbf{Usage Note }: These forms cannot modify nouns. To modify nouns with verbs in the negative or negative past, you must use their plain forms - \emph{nai }ない and \emph{-nakatta }なかった respectively. }

\par{Ex. もてない男 〇 \hfill\break
\emph{Motenai otoko }\hfill\break
もてないです男 X \hfill\break
\emph{Motenai desu otoko }\hfill\break
A man who isn\textquotesingle t popular }

\begin{center}
\textbf{Conjugation Recap } 
\end{center}

\par{ As review, here are the conjugations you learned in this lesson with the verbs \emph{miru }見る (to see) and \emph{taberu }食べる (to eat). }

\begin{ltabulary}{|P|P|P|P|}
\hline 

Verb Form & Conjugation &  \emph{Miru }見る (To see) &  \emph{Taberu }食べる (To eat) \\ \cline{1-4}

Plain Non-Past & N\slash A &  \emph{Miru }見る &  \emph{Taberu }食べる \\ \cline{1-4}

Polite Non-Past & - \emph{masu }ます &  \emph{Mimasu }見ます &  \emph{Tabemasu }食べます \\ \cline{1-4}

Plain Past & - \emph{ta }た &  \emph{Mita }見た &  \emph{Tabeta }食べた \\ \cline{1-4}

Polite Past & - \emph{mashita }ました &  \emph{Mimashita }見ました &  \emph{Tabemashita }食べました \\ \cline{1-4}

Plain Negative & - \emph{nai }ない &  \emph{Minai }見ない &  \emph{Tabenai }食べない \\ \cline{1-4}

Polite Neg. 1 & - \emph{nai desu } \hfill\break
ないです &  \emph{Minai desu \hfill\break
}見ないです &  \emph{Tabenai desu } \hfill\break
食べないです \\ \cline{1-4}

Polite Neg. 2 & - \emph{masen }ません &  \emph{Mimasen }見ません &  \emph{Tabemasen }食べません \\ \cline{1-4}

Plain Neg-Past & - \emph{nakatta }なかった &  \emph{Minakatta }見なかった &  \emph{Tabenakatta }食べなかった \\ \cline{1-4}

Polite Neg-Past 1 & - \emph{nakatta desu } \hfill\break
なかったです &  \emph{Minakatta desu } \hfill\break
見なかったです &  \emph{Tabenakatta desu } \hfill\break
食べなかったです \\ \cline{1-4}

Polite Neg-Past 2 & - \emph{masendeshita \hfill\break
}ませんでした &  \emph{Mimasendeshita \hfill\break
}見ませんでした &  \emph{Tabemasendeshita \hfill\break
}食べませんでした \\ \cline{1-4}

\end{ltabulary}
     