    
\chapter{Pronunciation I}

\begin{center}
\begin{Large}
第1課: Pronunciation I: Vowels     
\end{Large}
\end{center}
 
\par{ Japanese is spoken by over 125 million speakers, and despite largely only being spoken in Japan, it is a very influential language on the international stage. Japanese may seem daunting, but by beginning your journey with this curriculum, you will be on the road to mastering it. To start, we will learn about the fundamentals of Japanese pronunciation. }

\par{ In this first lesson on Japanese, our focus will be on how to pronounce vowel sounds. Vowel sounds are sounds like "ah" and "eh," and they contrast with sounds like "k" and "m," which are called consonants. It's rather impossible to showcase many words without using both vowels and consonants, but once we've covered the vowels, we'll move onto how to properly pronounce the consonants. }

\begin{center}
\textbf{Use of Terminology: Japanese ≠ English }\hfill\break

\end{center}

\par{ As obvious as it may be, Japanese is not English. You cannot assume that its sounds are the same as those in English. Creating an accurate perception of how Japanese sounds requires some terminology. However, e xplanations are always given when new terminology is used. }

\par{\textbf{Lesson Note }: Japanese words will be transcribed in this lesson with the English alphabet. This practice is called \emph{Rōmaji }. There is nothing special about English letters when used in Japanese other than what their intended pronunciations are. }
      
\section{The Syllabic Structure of Japanese: The Mora}
  \hfill\break
 In both English and Japanese, words are made by combining consonants and vowels. \textbf{Consonants (C) }\emph{are sounds that obstruct air in vocalization }. For instance, \slash b\slash , \slash t\slash , \slash m\slash , etc. are all examples of consonants. \textbf{Vowels (V) } \emph{are sounds made by vibrating the vocal folds without obstructing air from the lungs }. All varieties of English have many vowels, but in Japanese only five exist: \slash a\slash , \slash i\slash , \slash u\slash , \slash e\slash  and \slash o\slash . \hfill\break
\hfill\break
 Phonetically, American English has at least fourteen vowels in its standard form. To demonstrate this fact, read aloud this paragraph and you will notice that there are more than five distinct vowel-sounds in each of them.   In English, consonants and vowels can be combined in all sorts of ways to create syllables. A \textbf{syllable } \emph{is a unit of sound composed of a vowel that is with or without surrounding consonants }. In English, there are many syllabic structures. For instance, the word "blond" has the structure CCVCC. T wo consonants are put before (\slash b\slash  and \slash l\slash ) and after (\slash n\slash  and \slash d\slash ) a vowel (o). \hfill\break
 \hfill\break
 In Japanese, consonants don't cluster together within syllables. In fact, there are only three possible syllabic combinations CV, V, and C. Even when consonants happen to be next to each other, they're always in their own syllables. The word "syllable," though, is not quite the right word to describe these combinations. The technical word to describe syllables in Japanese is what's called a "mora." A \textbf{mora } \emph{is a unit of sound that is equivalent to a single beat }. Each "beat" is conceptualized as being equal in length, and each beat is assigned a high or low pitch. \hfill\break
\textbf{Transcription Note }: "\slash \slash " indicate sounds that are deemed as unique sounds ( \textbf{phonemes }) within the confines of Japanese. Variations of a single sound ( \textbf{allophones }) within the confines of Japanese are marked with "[]." Sometimes, the variant of one sound may end up sounding the same as a separate sound, which is why these two separate kinds of brackets are needed.       
\section{Pitch Accent}
 
\par{ In Standard Japanese--the standard form of the language any Japanese speaker will understand and the form that you are beginning to learn-- there is what is called a \textbf{pitch accent system }. \emph{In this system, every mora of a phrase is assigned a pitch. This assigned pitch may either be \textbf{high }or \textbf{low }}. The assignment of pitch \textbf{doesn't fundamentally change the meaning of words }. The pitch accent system is simply an acoustic observation that helps describe how phrases sound. Incidentally, because there is an audible difference between a mora that is low in pitch and a mora that is high in pitch, words are occasionally distinguished via pitch. However, context can easily tell what meaning is meant when two or more words are otherwise pronounced the same, putting pitch differences aside. }

\par{ In Standard Japanese, there are four pitch contours that a phrase can have. Regardless of how short or long a phrase is, the pitch contour will always be one of the following four patterns. Think of a phrase as being a pitch roller coaster. Every mora isn't an individual ride with its own loops and turns. Rather, a mora is only one loop of a ride--nothing more. You must put the loops (morae) together to see the course of the ride. }

\par{\textbf{Chart Notes }: \hfill\break
1. "L" and "H" both stand for a single mora. That means H-L is two morae, whereas H-L-L is three morae. As a reminder of this, numbers will be placed after these contour notations to tell you how many morae words involved have. \hfill\break
2. The "L" and "H" in parentheses indicate what the pitch of something attached to words would be per pattern. }

\begin{ltabulary}{|P|P|P|}
\hline 

1 & Pitch is \textbf{high for the first mora, }drops on the second mora, and stays low for any remaining morae that follow. \hfill\break
Ex. H(-L) ①, H-L(-L) ②, H-L-L(-L) ③, H-L-L-L(-L) ④ &  \emph{\textbf{há }shì }(chopsticks) \\ \cline{1-3}

2 & Pitch starts low on the first mora, \textbf{peaks at high pitch on the middle mora(e) }, drops back to low pitch on the third morae, and stays low for any following morae after the word. \hfill\break
Ex. L-H-L ③, L-H-H-L ④ &  \emph{ha \textbf{n }}\emph{\textbf{á }}\emph{su }(to speak) \\ \cline{1-3}

3 & Pitch starts low on the first mora, peaks at high pitch on the last mora, and then \textbf{drops to low pitch on any morae that follow the word }. \hfill\break
Ex. L-H-(L) ②, L-H-H(-L) ③ &  \emph{hà \textbf{shí }}(bridge) \hfill\break
\\ \cline{1-3}

4 & Pitch starts low on the first mora, becomes high pitch on the second mora, and then the pitch stays high even once the word is over unto anything that follows. \hfill\break
Ex. L(-H) ①, L-H(-H) ②, L-H-H(-H) ③, L-H-H-H(-H) ④ &  \emph{ha \textbf{shi }}\textbf{ }(edge) \\ \cline{1-3}

\end{ltabulary}
\hfill\break

\par{\textbf{Transcription Note }: In this lesson, morae with a high pitch will be in bold. If the pitch were to fall directly after the word in question, a ↓ arrow will follow to indicate this. What this all means will be explained at the end of the lesson. }
      
\section{The Five Vowels}
 
\par{ Below you will see how the five vowels of Japanese are roughly pronounced. The vowels are pronounced clearly and sharply  \emph{like }the American English approximates provided . However, it cannot be stressed enough that these are approximates. }

\begin{ltabulary}{|P|P|P|}
\hline 

 \textbf{A }& Like the "a" sound in the word "buy." &  \emph{Ta }↓ (field) \\ \cline{1-3}

 \textbf{I }& Like the "i" in "police." &  \emph{Ki }↓ (tree) \\ \cline{1-3}

 \textbf{U }& Like the "oo" in "mood." Compress your lips without protruding them. &  \emph{U \textbf{ta }}↓ (song) \\ \cline{1-3}

 \textbf{E }& Like the "e" in "set." &  \emph{I \textbf{ke }}↓ (pond) \\ \cline{1-3}

O & Like the "o" in "oh." &  \emph{O \textbf{ka }}(hill) \\ \cline{1-3}

\end{ltabulary}

\par{ Although the chart says that \slash u\slash  is like the "oo" in the word "mood," this isn't quite accurate. There is no form of English that has the \slash u\slash  found in Standard Japanese. However, by compressing your lips rather than protruding them forward, the resulting \slash u\slash  will be something like the one in Japanese. }

\par{ Although the other vowels are almost identical to the ones found in American English, the Japanese \slash a\slash  actually only shows up in diphthongs in American English. A diphthong is when \textbf{two vowels blend together to form a complex vowel sound }. You start off pronouncing one vowel sound, but at the end it sounds like something else. For example, the vowel sound in the word "height" is an example of a diphthong, and the onset of this word is exactly how the Japanese \slash a\slash  is pronounced. }

\begin{center}
\textbf{Juxtaposed Vowels }\hfill\break

\end{center}

\par{ In English, complex vowel sounds called \textbf{diphthongs }are created by beginning a vowel sound with one quality but ending the sound with a different vowel sound. For instance, in the word "kite," the vowel sound written with the letter "i" starts out as sounding like the Japanese \slash a\slash  but ends sounding like the Japanese \slash i\slash . The opposite of a diphthong is a \textbf{monophthong }, which \emph{is a vowel whose quality doesn't change during its pronunciation }. This is what all vowels in Japanese are thought to intrinsically be. }

\par{ In Japanese, diphthongs are said not to exist because of how the moraic structure of the language dictates how sounds are organized. Instead of viewing something like \emph{\textbf{ha }i }(yes) as one syllable, you view it as two morae: \slash ha\slash  + \slash i\slash . However, there are plenty of instances in which Japanese speakers pronounce consecutive vowels similarly to how they would be in English. }

\par{ Acoustically, juxtaposed vowels sound like they blend together. However, native speakers still conceptualize them as being separate. This is because pitch can fall or rise without the need of consonants f rom mora to mora, and if two vowels next to each other count as two morae, then there is room for pitch changes. }

\begin{center}
\textbf{Examples } 
\end{center}

\par{ Even in words just composed of vowels, pitch contours cannot be ignored, as is demonstrated below. }

\begin{ltabulary}{|P|P|P|P|P|P|}
\hline 

Love\slash indigo \hfill\break
&  \emph{\textbf{A }i }& To meet & \emph{\textbf{A }u }&  \emph{U \textbf{e }}↓ \hfill\break
& Starvation \\ \cline{1-6}

Fish \hfill\break
&  \emph{U \textbf{o }}& Blue &  \emph{\textbf{A }o }&  \emph{U \textbf{e }}& Above \\ \cline{1-6}

Nephew &  \emph{O \textbf{i }}& Hey! &  \emph{\textbf{O }i }&  \emph{I }\textbf{\emph{u } }& To say \\ \cline{1-6}

\end{ltabulary}

\par{\textbf{Exception Note }: The word \emph{iu }is pronounced as \slash  \emph{yu }\textbf{u }\slash . Exceptions like this, though, are few and far between. }
\emph{}      
\section{Short Vowels vs Long Vowels}
 
\par{ In Japanese, \textbf{\emph{short vowels are distinguished from long vowels }}. A \textbf{short vowel }\emph{is a vowel utterance equal to one mora in length }. If \emph{a vowel is elongated to take up two morae }, it becomes a \textbf{long vowel }. Pitch can consequently rise or drop inside long vowels because they're treated as two morae. }

\par{ Consequently, vowel length contrasts thousands and thousands of words. Mistakes at the beginning are inevitable, but recognizing distinctions like this now will spare you a lot of potential heartache. }

\begin{ltabulary}{|P|P|P|P|}
\hline 

Short \slash a\slash  &  \emph{O \textbf{basan }}(aunt) & Long \slash a\slash  &  \emph{O \textbf{ba }asan }(grandma) \\ \cline{1-4}

 Short \slash i\slash  &  \emph{I }\textbf{\emph{e↓ } }(house) &  Long \slash i\slash  &  \emph{I \textbf{ie }}(no) \\ \cline{1-4}

Short \slash u\slash  & \emph{Yu \textbf{ki }}\textbf{↓ }(snow) \emph{\textbf{\hfill\break
}}& Long \slash u\slash  &  \emph{\textbf{Yu }uki }(courage) \\ \cline{1-4}

 Short \slash e\slash  &  \textbf{\emph{E }}↓ (painting) &  Long \slash e\slash  &  \emph{\textbf{E }e }(yes) \\ \cline{1-4}

Short \slash o\slash  &  \emph{\textbf{To } }(door) & Long \slash o\slash  &  \emph{To \textbf{o }}\emph{ }(ten things) \\ \cline{1-4}

\end{ltabulary}

\par{\textbf{Pronunciation Note }: Do not pronounce "oo" as a long \slash u\slash  sound. This is incorrect! \hfill\break
}

\begin{center}
 \textbf{False Long Vowels }
\end{center}

\par{ What makes a long vowel truly a long vowel and not just the same vowel next to each other is there being nothing that obstructs the pronunciation of the vowel as it spans two morae. In both English and Japanese, the pronunciations of vowels begin with glottal stops. Whenever you say the phrase "uh-oh", you should feel an audible release of air after completely stopping airflow from the glottis (Adam's apple) at the start of "uh" and "oh." }

\par{ In Japanese, a long vowel will always be inside a single element of a word. If one element of a word only has a short vowel but is followed by another word element beginning with the same vowel, that second element's vowel is still going to begin with a glottal stop like any other initial vowel sound. This acoustically interrupts what would otherwise be a long vowel. Unsurprisingly, in the context of Japanese, this glottal stop insertion is influential enough to change the pitch contour of a phrase and make phrases even more distinct--and not just from an etymological standpoint. \hfill\break
\hfill\break
\textbf{Transcription Note }: In the words below, to show where elements of a word begin and end, periods will be inserted to indicate these boundaries. }

\begin{center}
\textbf{Examples }
\end{center}

\begin{ltabulary}{|P|P|P|P|}
\hline 

Scene &  \emph{\textbf{Shi }in }& Consonant\slash Cause of death & \emph{Shi. \textbf{in }}\\ \cline{1-4}

\end{ltabulary}

\par{\textbf{Trivia Note }: Vowels are called \emph{bo \textbf{in }}in Japanese. }

\begin{center}
 \textbf{The Pronunciation of " \emph{Ei }": [ei] or [ }ē \textbf{] }
\end{center}

\par{ In Japanese, the vowel combination "ei" is usually pronounced as a long \slash e\slash  (ē). All such words come from Chinese roots. Because this sound change is technically optional, you don't have to worry so much about whether to pronounce an "ei" as [ei] or [ē]. After all, we haven't even learned about what exactly words made from Chinese roots look like. For this lesson, alternative pronunciations of a word are listed for you. }

\begin{center}
\textbf{Examples } 
\end{center}

\par{\textbf{Transcription Notes }: \hfill\break
1. Long \slash e\slash  are written as "ee" so that pitch contours can be designated. However, do not be confused by this spelling and pronounce "ee" as a long "i" sound as would be the case in English words such as "cheese." This is incorrect, and so try to stay focused on what is going on in Japanese. \hfill\break
2. To show where elements of a word begin and end, periods will be inserted to indicate these boundaries. "Ei" can only be pronounced as [ē] if it's within the same element of a word, so these boundaries are very important. \hfill\break
3. If a word is not derived from Chinese roots, the pronounce [ē] becomes impossible even if the vowel combination is found in a single word element. }

\begin{ltabulary}{|P|P|P|P|P|P|}
\hline 

Plan &  \emph{Ke \textbf{i.kaku }\hfill\break
Ke \textbf{e.kaku }}& Student &  \emph{\textbf{Se }i.to \hfill\break
 \textbf{Se }e.to }& Stingray & \emph{ \textbf{E }i }\\ \cline{1-6}

English &  \emph{E \textbf{i.go }\hfill\break
E \textbf{e.go }}& Native &  \emph{\textbf{Ne }itibu }& Cellphone &  \emph{Ke \textbf{i.tai }\hfill\break
Ke \textbf{e.tai }}\\ \cline{1-6}

Management &  \emph{Ke \textbf{iei }\hfill\break
Ke \textbf{e.ee }}& Clock &  \emph{To. \textbf{kei }}& Correct answer &  \emph{Se \textbf{i }. \textbf{kai \hfill\break
 }Se \textbf{e.kai }}\\ \cline{1-6}

Proclamation &  \emph{Se \textbf{i.mei }\hfill\break
Se \textbf{e.mee }}& Process &  \emph{\textbf{Ke }i.ro \hfill\break
 \textbf{Ke }e.ro }& Hair color &  \emph{Ke. \textbf{iro }}\\ \cline{1-6}

\end{ltabulary}
\hfill\break

\begin{center}
\textbf{The Pronunciation of "Ou": [ou] or [ō]? } 
\end{center}

\par{ In Japanese, the vowel combination "ou" is usually pronounced as a long \slash o\slash  ( ō ). Most such words come from Chinese roots, but this is not always the case. This sound change, unlike the one above, is not optional for the words it affects. }

\par{ Because knowing which words are and aren't affected is a luxury that comes about from knowing a lot word origins. For this lesson and the next, any word in which a word that would be spelled as "ou" but is instead pronounced as a long \slash o\slash  will be spelled as "oo." This means if you do see " \emph{ou }," you should pronounce it literally as such. Try not to read "oo" as a long "u" sound as this is incorrect. }

\begin{center}
\textbf{Examples  }
\end{center}

\par{\textbf{Transcription Note }: To show where elements of a word begin and end, periods will be inserted to indicate these boundaries. }

\begin{ltabulary}{|P|P|P|P|P|P|}
\hline 

Already &  \emph{\textbf{Mo }o }& King &  \emph{\textbf{O }o }& Method &  \emph{Ho \textbf{o.hoo }}\\ \cline{1-6}

To think &  \emph{O \textbf{mo. }u }& Large &  \emph{O \textbf{oki }i }& Calf &  \emph{Ko. \textbf{ushi }}\\ \cline{1-6}

Action &  \emph{Ko \textbf{odoo }}& Robbery &  \emph{Go \textbf{otoo }}& Sauce &  \emph{\textbf{So }osu }\\ \cline{1-6}

\end{ltabulary}
    