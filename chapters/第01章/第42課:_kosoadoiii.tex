    
\chapter{Kosoado こそあど III}

\begin{center}
\begin{Large}
第42課: Kosoado こそあど III: Kochira こちら, Sochira そちら, \& Achira あちら 
\end{Large}
\end{center}
 
\par{ In this lesson, we will discuss three more \emph{kosoado }こそあど phrases that are politer counterparts to both the \emph{kosoado }こそあど phrases for “this” and “that” and those for “here” and “there.” The reason for why they would be intertwined with each other is because phrases indicating direction have always been used in Japanese to also refer to physical entities, and by extension, people as you will soon see. }
      
\section{Kochira こちら}
 
\par{ The first main usage of \emph{kochira }こちら is as a politer version of \emph{kore }これ. }

\par{1. こちらは ${\overset{\textnormal{しんせいひん}}{\text{新製品}}}$ でございます。 \hfill\break
 \emph{Kochira wa shinseihin de gozaimasu. }\hfill\break
This is a new product. }

\par{\textbf{Grammar Note }: In respectful language, more than just one word here and there will be different. Verb forms also change. Instead of using \emph{desu }です, \emph{de gozaimasu }でございます may be used instead to both be more respectful yet humble at the same time. }

\par{\textbf{Prefix Note }: \emph{Shin- }新 is seen in a lot of phrases in place of \emph{atarashii }新しい to mean “new” to form compound phrases. Although these should be learned individually, some phrases like \emph{shinseihin }新製品will also be common in the spoken language as they are in the written language. }

\par{2. こちら、 ${\overset{\textnormal{おやこどん}}{\text{親子丼}}}$ です。 \hfill\break
 \emph{Kochira, oyakodon desu. \hfill\break
 }This is your oyakodon (you ordered). }

\par{\textbf{Culture Note }: \emph{Oyakodon }親子丼 is a bowl of rice topped with chicken and eggs. \emph{Oyako }親子 means parent and child, and the use of chicken and eggs refers to the age-old question “which came first, the chicken or the egg?” This is one variety of rice bowl dishes called \emph{domburi }丼. }

\par{ The second main usage of \emph{kochira }こちら is as a politer version of \emph{koko }ここ. }

\par{3. どうぞこちらへ。 \hfill\break
 \emph{Dōzo kochira e. }\hfill\break
Please, come this way. }

\par{4. こちらを ${\overset{\textnormal{む}}{\text{向}}}$ いてください。 \hfill\break
 \emph{Kochira wo muite kudasai. \hfill\break
 }Please face this way. }

\par{5. こちらは ${\overset{\textnormal{きのう}}{\text{昨日}}}$ 、 ${\overset{\textnormal{ひど}}{\text{酷}}}$ い ${\overset{\textnormal{あめ}}{\text{雨}}}$ でした。 \hfill\break
 \emph{Kochira wa kinō, hidoi ame deshita. \hfill\break
 }It rained here heavily yesterday. }

\begin{center}
\textbf{As a Pronoun }
\end{center}

\par{ Just as alluded to in the introduction, \emph{kochira }こちら may also be used as a pronoun, and when it is, it can either be a first person or a third person pronoun. As a first person pronoun, it is used to indirectly refer to oneself. }

\par{6. こちらこそどうぞ ${\overset{\textnormal{よろ}}{\text{宜}}}$ しくお ${\overset{\textnormal{ねが}}{\text{願}}}$ いします。 \hfill\break
 \emph{Kochira koso dōzo yoroshiku onegai shimasu. }\hfill\break
It\textquotesingle s very nice to meet you too. }

\par{7. こちらはいつでも ${\overset{\textnormal{けっこう}}{\text{結構}}}$ です。 \hfill\break
 \emph{Kochira wa itsu demo kekkō desu. \hfill\break
 }I\textquotesingle m fine whenever. }

\par{ In the third person, \emph{Kochira }こちら may also be used to refer to “this person.” In this case, the person is either equal or above one\textquotesingle s own status. The phrase is synonymous with \emph{kono kata }この方. When the person is of especially high status, the suffix \emph{-sama }様 should be added, creating \emph{kochira-sama }こちら様. }

\par{8. こちらはジョーンズ ${\overset{\textnormal{ふじん}}{\text{夫人}}}$ です。 \hfill\break
 \emph{Kochira wa Jōnzu-fujin desu. }\hfill\break
This is Mrs. Jones. }

\par{ 9. こちら ${\overset{\textnormal{さま}}{\text{様}}}$ にお ${\overset{\textnormal{みず}}{\text{水}}}$ を ${\overset{\textnormal{さ}}{\text{差}}}$ し ${\overset{\textnormal{あ}}{\text{上}}}$ げてください。 \hfill\break
 \emph{Kochira-sama ni omizu wo sashiagete kudasai. }\hfill\break
Please give this individual water. }
      
\section{Sochira そちら}
 
\par{ The first main usage of \emph{sochira }そちら is as a politer version of \emph{sore }それ. }

\par{10. そちらはお ${\overset{\textnormal{か}}{\text{買}}}$ い ${\overset{\textnormal{どく}}{\text{得}}}$ ですよ。 \hfill\break
 \emph{Sochira wa okaidoku desu yo. }\hfill\break
That is a good bargain. }

\par{ The second main usage of \emph{sochira }そちら is used as a politer version of \emph{soko }そこ. }

\par{11. そちらはもう ${\overset{\textnormal{さむ}}{\text{寒}}}$ くなりましたか。 \hfill\break
 \emph{Sochira wa m }\emph{ō samuku narimashita ka? }\hfill\break
Has it already gotten cold there? }

\begin{center}
\textbf{As a Pronoun } 
\end{center}

\par{ As a pronoun, \emph{sochira }そちら frequently refers to the person whom you are interacting with. Essentially, this is a respectful “you.” }

\par{12. そちらのご ${\overset{\textnormal{いけん}}{\text{意見}}}$ を ${\overset{\textnormal{き}}{\text{聞}}}$ かせてください。 \hfill\break
 \emph{Sochira no go-iken wo kikasete kudasai. }\hfill\break
Please let me hear your opinion. }

\par{\textbf{Grammar Note }: \emph{Kikaseru }聞かせる means “to let hear\slash ask.” Additionally, the prefix \emph{go- }ご attached to \emph{iken }意見 adds respect to the noun. }

\par{13. そちらこそお ${\overset{\textnormal{つか}}{\text{疲}}}$ れ ${\overset{\textnormal{さま}}{\text{様}}}$ です。 \hfill\break
 \emph{Sochira koso otsukare-sama desu. \hfill\break
 }Thanks to you as well for your work. }

\par{\textbf{Sentence Note }: This phrase is used to very politely respond to also being told “ \emph{otsukare-sama }お疲れ様.” This phrase is used at the end of the day to thank colleagues for their efforts in the day\textquotesingle s work. It is also used by the public in service industries to people who have gotten off work. }

\par{ As an extension of this, it may also refer to someone in proximity\slash relation with whom you are interacting with. In this case, you are not who is completely in the know about the individual. If the person happens to be in eyesight, then the individual is simply close to the listener. }

\par{ Just as is the case for \emph{kochira }こちら, if the person in question is of especially high social status, then the suffix \emph{-sama }様 should be added, creating \emph{sochira-sama }そちら様. }

\par{14. そちらを ${\overset{\textnormal{わたし}}{\text{私}}}$ に ${\overset{\textnormal{しょうかい}}{\text{紹介}}}$ してもらえませんか。 \hfill\break
 \emph{Sochira wo watakushi ni sh }\emph{ōkai shite moraemasen ka? \hfill\break
 }Could I have you introduce the\slash that person to me? }
      
\section{Achira あちら}
 
\par{ The first main usage of \emph{achira }あちら is as a politer version of \emph{are }あれ. }

\par{15. あちらをご ${\overset{\textnormal{らん}}{\text{覧}}}$ ください。 \hfill\break
 \emph{Achira wo goran kudasai. }\hfill\break
Look at that over there. }

\par{\textbf{Grammar Note }: The respectful form of “please see” is \emph{goran kudasai }ご覧ください. }

\par{ The second main usage of \emph{achira }あちら is as a politer version of \emph{asoko }あそこ. }

\par{16. お ${\overset{\textnormal{てあら}}{\text{手洗}}}$ いはあちらです。 \hfill\break
 \emph{Otearai wa achira desu. }\hfill\break
The bathroom is over there. }

\par{17. ${\overset{\textnormal{かねこ}}{\text{金子}}}$ さんはあちら ${\overset{\textnormal{じこ}}{\text{仕込}}}$ みの ${\overset{\textnormal{りゅうちょう}}{\text{流暢}}}$ な ${\overset{\textnormal{えいご}}{\text{英語}}}$ が ${\overset{\textnormal{はな}}{\text{話}}}$ せますよ。 \hfill\break
 \emph{Kaneko-san wa achira-jikomi no ry }\emph{ūch }\emph{ō na eigo ga hanasemasu yo. \hfill\break
 }Mr. Kaneko can speak fluent English acquired abroad. }

\par{\textbf{Grammar Note }: \emph{Hanasemasu }話せます utilizes the potential form of the verb \emph{hanasu }話す, which we\textquotesingle ll get to later. Additionally, the suffix - \emph{jikomi }仕込み is used to mean “acquired at.” }

\begin{center}
\textbf{As a Pronoun } 
\end{center}

\par{\emph{ Achira }あちら may be used to refer to a third person that is not in near proximity to the speaker or the listener. When the individual is not in eyesight, all parties in the conversation are assumed to know who the person in question is. However, simple distance from parties involved determines its usage if said person is in eyesight. }

\par{18. あちらがお ${\overset{\textnormal{かあさま}}{\text{母様}}}$ ですか。 \hfill\break
 \emph{Achira ga ok }\emph{ā-sama desu ka? }\hfill\break
Is that person over there your mother? }

\par{19. あちら ${\overset{\textnormal{さま}}{\text{様}}}$ は ${\overset{\textnormal{いえがら}}{\text{家柄}}}$ もよろしいのです。 \hfill\break
 \emph{Achira-sama wa iegara mo yoroshii no desu. }\hfill\break
That person\textquotesingle s pedigree is also very good. }
      
\section{Kotchi こっち, Sotchi そっち, \& Atchi あっち}
 
\par{ Quite simply, these are contractions of the forms \emph{kochira }こちら, \emph{sochira }そちら, and \emph{achira }あちら respectively. They possess the same meanings but without heightened sense of politeness. To the contrary, they are casual since they are contractions. }

\par{20. こっち、こっち! \hfill\break
 \emph{Kotchi, kotchi! }\hfill\break
This way, this way! }

\par{21. それはこっちの ${\overset{\textnormal{せりふ}}{\text{台詞}}}$ だ。 \hfill\break
 \emph{Sore wa kotchi no serifu da. }\hfill\break
That\textquotesingle s my line. }

\par{22. そっちの ${\overset{\textnormal{ひと}}{\text{人}}}$ たちのことを ${\overset{\textnormal{はな}}{\text{話}}}$ してたの? \hfill\break
 \emph{Sotchi no hitotachi no koto wo hanashiteta no? }\hfill\break
Were you talking about those people? }

\par{23. あっちの ${\overset{\textnormal{みず}}{\text{水}}}$ は ${\overset{\textnormal{にが}}{\text{苦}}}$ いぞ。こっちの ${\overset{\textnormal{みず}}{\text{水}}}$ は ${\overset{\textnormal{あま}}{\text{甘}}}$ いぞ。 \hfill\break
 \emph{Atchi no mizu wa nigai zo. Kotchi no mizu wa amai zo. \hfill\break
 }The water over there is bitter! The water here is sweet! }

\par{\textbf{Culture Note }: These lines are a part of an old children\textquotesingle s song for catching fireflies. }

\par{\textbf{Grammar Note }: The particle \emph{zo }ぞ is used here to add force to the tone of the sentence. }

\par{24. ${\overset{\textnormal{わたし}}{\text{私}}}$ は ${\overset{\textnormal{ことし}}{\text{今年}}}$ 、 ${\overset{\textnormal{せかい}}{\text{世界}}}$ をあちこち ${\overset{\textnormal{りょこう}}{\text{旅行}}}$ して ${\overset{\textnormal{まわ}}{\text{回}}}$ りました。 \hfill\break
 \emph{Watashi wa kotoshi, sekai wo achikochi ryokō shite mawarimashita. }\hfill\break
I traveled around the world this year. }

\par{\textbf{Phrase Note }: Achikochi あちこち literally means here and there. Although it may be pronounced as \emph{atchikotchi }あっちこっち, that is not its normal pronunciation. However, \emph{achikochi }あちこちis not a contraction of it. In fact, it is simply composed of \emph{achira }あちら and \emph{kochira }こちら without the \slash ra\slash . }
    