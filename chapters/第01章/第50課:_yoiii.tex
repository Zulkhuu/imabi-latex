    
\chapter{Adjectives}

\begin{center}
\begin{Large}
第50課: Adjectives: Yoi\slash Ii 良い 
\end{Large}
\end{center}
 
\par{ The word for “good” is \emph{yoi }良い. However, it is not as easy as simply using it as the opposite of \emph{warui }悪い. The first problem we encounter is that it\textquotesingle s usually replaced with its contracted form: \emph{ii }いい. However, it is \emph{yoi }良い that is used for conjugation. There is also the problem of nuance, which is the hardest problem for learners. }
      
\section{Good Ole Yoi\slash Ii 良い}
 
\par{ When you conjugate yoi\slash ii よい・いい, you must use \emph{yoi }よい for everything aside from the non-past tense. }

\begin{ltabulary}{|P|P|P|}
\hline 

Form & Plain Speech & Polite Speech \\ \cline{1-3}

Non-past &  \emph{Yoi\slash ii }\emph{ }よい・いい &  \emph{Yoi desu\slash ii desu }\emph{ }よいです・いいです \\ \cline{1-3}

Past &  \emph{Yokatta }よかった &  \emph{Yokatta desu }よかったです \\ \cline{1-3}

Negative &  \emph{Yokunai }\emph{ }よくない &  \emph{Yokunai desu }\emph{ }よくないです \\ \cline{1-3}

Negative Past &  \emph{Yokunakatta }よくなかった &  \emph{Yokunakatta desu }\emph{ }よくなかったです \\ \cline{1-3}

 \emph{Te }て Form &  \emph{Yokute }よくて &  \emph{Yokute }よくて \\ \cline{1-3}

Adverbial Form &  \emph{Yoku }よく &  \emph{Yoku }よく \\ \cline{1-3}

\end{ltabulary}

\par{ In the example sentences below, \emph{yoi\slash ii }よい・いい are used in the various forms above with nuances that all revolve around “good (for)\slash fine\slash excellent\slash pleasant\slash agreeable\slash ready\slash sufficient\slash beneficial\slash okay.” }

\par{1. ${\overset{\textnormal{せいせき}}{\text{成績}}}$ が ${\overset{\textnormal{よ}}{\text{良}}}$ くなった。 \hfill\break
 \emph{Seiseki ga yoku natta. }\hfill\break
My grades become good\slash got better. }

\par{2. ${\overset{\textnormal{こんしゅう}}{\text{今週}}}$ は(お) ${\overset{\textnormal{てんき}}{\text{天気}}}$ が ${\overset{\textnormal{よ}}{\text{良}}}$ くないですね。 \hfill\break
 \emph{Konshū wa (o)tenki ga yokunai desu ne. }\hfill\break
The weather this week isn\textquotesingle t good, huh. }

\par{3. ${\overset{\textnormal{うん}}{\text{運}}}$ が ${\overset{\textnormal{よ}}{\text{良}}}$ かったですね。 \hfill\break
 \emph{Un ga yokatta desu ne. }\hfill\break
My\slash our\slash your\slash his\slash her luck was good, huh. }

\par{4. ${\overset{\textnormal{かれ}}{\text{彼}}}$ はかっこよくて ${\overset{\textnormal{やさ}}{\text{優}}}$ しいですね。 \hfill\break
 \emph{Kare wa kakko yokute yasashii desu ne. }\hfill\break
He is cool and nice, isn\textquotesingle t he? }

\par{\textbf{Phrase Note }: \emph{Kakkō yoi }格好良い means “attractive\slash good-looking\slash stylish,” and in the spoken language, it is typically contracted to \emph{kakko ii }かっこいい. It is frequently alternatively spelled as カッコいい. }

\par{5. ${\overset{\textnormal{いんしょう}}{\text{印象}}}$ が ${\overset{\textnormal{よ}}{\text{良}}}$ くなかったです。 \hfill\break
 \emph{Inshō ga yokunakatta desu. }\hfill\break
Its impression wasn\textquotesingle t good. }

\par{6. ${\overset{\textnormal{じゅうぎょういん}}{\text{従業員}}}$ の ${\overset{\textnormal{たいおう}}{\text{対応}}}$ が ${\overset{\textnormal{よ}}{\text{良}}}$ くなかった。 \hfill\break
 \emph{Jūgyōin no taiō ga yokunakatta. }\hfill\break
The employees\textquotesingle  handling wasn\textquotesingle t good. }

\par{7. ${\overset{\textnormal{よ}}{\text{良}}}$ かった! \hfill\break
 \emph{Yokatta! }\hfill\break
Thank goodness! }

\par{8. ${\overset{\textnormal{よ}}{\text{良}}}$ いお ${\overset{\textnormal{とし}}{\text{年}}}$ を! \hfill\break
 \emph{Yoi o-toshi wo! }\hfill\break
Have a good New Year! \hfill\break
 \hfill\break
\textbf{Phrase Note }: This phrase is more or less the same as saying “Happy New Year” in the West, and as such, it is not used once the New Year has begun. Even in the spoken language, \slash yoi\slash  is still the predominant pronunciation in this set phrase. However, \slash ii\slash  would not be wrong. }

\par{9. それはよかったですね。 \hfill\break
 \emph{Sore wa yokatta desu ne. \hfill\break
 }I\textquotesingle m glad to hear that. }

\par{\textbf{Sentence Note }: One could more literally be expressed as \emph{sore wo kiite yokatta desu ne }それを聞いてよかったですね. However, this would emphasize being glad that you heard whatever “that” is. }

\par{10. タバコは ${\overset{\textnormal{からだ}}{\text{体}}}$ に ${\overset{\textnormal{よ}}{\text{良}}}$ くないです。 \hfill\break
 \emph{Tabako wa karada ni yokunai desu. }\hfill\break
Tobacco is not good for the body. }

\par{\textbf{Spelling Note }: \emph{Tabako }may be alternatively spelled as たばこ or 煙草. }

\par{11. いい ${\overset{\textnormal{けしき}}{\text{景色}}}$ ですね。 \hfill\break
 \emph{Ii keshiki desu ne. }\hfill\break
What nice scenery, no? }

\par{\textbf{Sentence Note }: This may be more literally expressed by adding the adverb \emph{nanto }なんと at the beginning of the sentence, but because this is very emphatic, the adjective \emph{subarashii }素晴らしい (wonderful) would be more appropriate. }

\par{12. ${\overset{\textnormal{かれ}}{\text{彼}}}$ も ${\overset{\textnormal{うで}}{\text{腕}}}$ がいいですよ。 \hfill\break
 \emph{Kare mo ude ga ii desu yo. } \hfill\break
He too has good skill. }

\par{\textbf{Phrase Note }: \emph{Ude ga yoi\slash ii }腕が良い is a set phrase meaning “able\slash skilled,” and both \slash yoi\slash  and \slash ii\slash  are correct pronunciations; however, the latter is most common in the spoken language. }

\par{13. ${\overset{\textnormal{とうにょうびょう}}{\text{糖尿病}}}$ に ${\overset{\textnormal{よ}}{\text{良}}}$ い ${\overset{\textnormal{しょくひん}}{\text{食品}}}$ を ${\overset{\textnormal{おし}}{\text{教}}}$ えてください。 \hfill\break
 \emph{Tōnyōbyō ni yoi shokuhin wo oshiete kudasai. }\hfill\break
Could you please tell me foods that are good for diabetes? }

\par{\textbf{Spelling Note }: As you may have noticed, writing \emph{yoi\slash ii }よい・いいin \emph{Kanji }usually indicates that the pronunciation is \slash yoi\slash . Although this not a guarantee, it shows that the sentence is stilted to the written language. }

\par{14. ${\overset{\textnormal{かくご}}{\text{覚悟}}}$ はいいか? \hfill\break
 \emph{Kakugo wa ii ka? }\hfill\break
You prepared? }

\par{\textbf{Tone Note }: This sentence is both casual and indicative of a superior-inferior relationship. Meaning, the speaker is in no way below the listener in status. }

\par{15. ${\overset{\textnormal{あいしょう}}{\text{相性}}}$ が良いカップルの ${\overset{\textnormal{とくちょう}}{\text{特徴}}}$ は ${\overset{\textnormal{なに}}{\text{何}}}$ ですか。 \hfill\break
 \emph{Aishō ga yoi\slash ii kappuru no tokuchō wa nan desu ka? }\hfill\break
What are the characteristics of a couple that suits each other. }

\par{\textbf{Phrase Note }: \emph{Aishō }相性 means affinity, and so \emph{aishō ga yoi\slash ii }相性が良い literally means “affinity is good.” The opposite of this is \emph{aishō ga warui }相性が悪い. }

\par{16. ${\overset{\textnormal{えいが}}{\text{映画}}}$ を ${\overset{\textnormal{み}}{\text{観}}}$ て ${\overset{\textnormal{よ}}{\text{良}}}$ かったです。 \hfill\break
 \emph{Eiga wo mite yokatta desu. }\hfill\break
I\textquotesingle m glad I watched a\slash the movie. }

\par{\textbf{Grammar Note }: \emph{Te yokatta }てよかった is used to mean “I\textquotesingle m glad that (I)…” It may be used in reference to being glad that an action was done or that a particular situation came to be. }

\par{\textbf{Spelling Note }: Spelling \emph{miru }as 観る indicates that you watched the movie somewhere, most likely a theatre. }

\par{\textbf{Rejecting an Offer }}

\par{ Just as in English with the word “fine,” ii may be used in rejecting offers. }

\par{17. ええ、いいですよ。 \hfill\break
 \emph{Ē, ii desu yo. }\hfill\break
Sure, that\textquotesingle s fine. }

\par{18. (いや、)いいです。 \hfill\break
 \emph{(Iya,) ii desu. }\hfill\break
(No,) I\textquotesingle m fine. }

\par{ However, just as in English, many people just don\textquotesingle t get it and do whatever you intended to say “no” to anyway. }

\par{\textbf{Spelling Notes }: \hfill\break
 \hfill\break
1. When 良い\textquotesingle s nuance focuses on the nature\slash behavior\slash actions\slash status of someone\slash something is satisfactory, it may be alternatively spelled as 善い. However, this is rather rare in today\textquotesingle s writing. }

\par{2. When 良い's nuance focuses on the auspicious nature of something, then it may be alternatively yet rarely spelled as 好い, 佳い, or 吉い. Unless you read works of the famous \emph{Natsume Sōseki }夏目漱石, you\textquotesingle ll likely never see them. }

\begin{center}
\textbf{No Need\slash Insulting\slash Irony }
\end{center}

\par{ As an extension of rejecting with ii, it may also be used in an insulting\slash ironic manner in several set phrases. }

\par{19. もういいです。 \hfill\break
 \emph{Mō ii desu. }\hfill\break
That\textquotesingle s enough. }

\par{\textbf{Sentence Note }: Even in English, this phrase may be quite offensive depending on the situation. }

\par{20. いい ${\overset{\textnormal{かげん}}{\text{加減}}}$ にしてください! \hfill\break
 \emph{Ii kagen ni shite kudasai! }\hfill\break
Please cut it out! }

\par{21. いい ${\overset{\textnormal{とし}}{\text{歳}}}$ (を)して ${\overset{\textnormal{じっかぐ}}{\text{実家暮}}}$ らしは ${\overset{\textnormal{は}}{\text{恥}}}$ ずかしい。 \hfill\break
 \emph{Ii toshi (wo) shite jikkagurashi wa hazukashii. }\hfill\break
Living at one's parents\textquotesingle  house despite being old enough to know better is embarrassing. }

\par{\textbf{Phrase Note }: \emph{Ii toshi (wo) shite }いい歳(を)して is a set phrase used to insult someone for something that is unbecoming of his age. \emph{Jikkagurashi }実家暮らし is a set phrase meaning living with one\textquotesingle s parents, particularly at their home. }

\par{22. あんた、いい ${\overset{\textnormal{めいわく}}{\text{迷惑}}}$ だよ。 \hfill\break
 \emph{Anta, ii meiwaku da yo. }\hfill\break
You\textquotesingle re a real nuisance. }

\par{\textbf{Phrase Note }: \emph{Anta }あんた is a coarse contraction of \emph{anata }あなた (you). Only in non-Standard Japanese dialects is it used in a less casual and coarse manner. Male speakers of Arabic should especially take caution in not overusing it as it may be tempting to use it due to it coincidentally sounding like the word for “you.” }

\par{23. いいざまだ。 \hfill\break
 \emph{Ii zama da. }\hfill\break
It serves you\slash him\slash her right! }

\par{24. いい ${\overset{\textnormal{きみ}}{\text{気味}}}$ だ。 \hfill\break
 \emph{Ii kimi da. } \hfill\break
It serves you\slash him\slash her right! }

\par{\textbf{Phrase Note }: This is synonymous to Ex. 23. Kimi literally means “feeling\slash sensation” and \emph{zama }ざま literally means “sorry state.” It may also be seen as a suffix attached to verb stems to mean “manner of.” For example, \emph{ikizama }生き様 means “way of life.” }

\par{25. いいご ${\overset{\textnormal{みぶん}}{\text{身分}}}$ だね。 \hfill\break
 \emph{Ii go-mibun da ne. }\hfill\break
How can you afford it? }

\par{\textbf{Phrase Note }: \emph{Go-mibun }ご身分 is literally a respectful phrase referring to someone\textquotesingle s status. Here, it is being used sarcastically to lead to a question about how the other person could possible afford the item of discussion. }

\par{\textbf{Spelling Note }: Although not really common at all, these negative nuances of \emph{ii }いい may be spelled in \emph{Kanji }alternatively as 好い. }

\begin{center}
\textbf{Set Phrases }
\end{center}

\par{ There are plenty of set phrases in which \emph{yoi\slash ii }よい・いい are attached to nouns to create a compound expression. In this case, the main difference is typically whether the sentence is made for the written or spoken language. For the written language, \emph{yoi }よい will be your choice, and for the spoken language, \emph{ii }いい will be your choice. }

\par{26. ${\overset{\textnormal{あさぶろ}}{\text{朝風呂}}}$ も ${\overset{\textnormal{きも}}{\text{気持}}}$ ち\{よい・いい\}です。 \hfill\break
 \emph{Asaburo mo kimochi-yoi\slash ii. }\hfill\break
The morning bath also feels good. }

\par{27. ${\overset{\textnormal{とり}}{\text{鳥}}}$ たちの ${\overset{\textnormal{ここち}}{\text{心地}}}$ \{よい・いい\}さえずりに ${\overset{\textnormal{みみ}}{\text{耳}}}$ を ${\overset{\textnormal{かたむ}}{\text{傾}}}$ ける。 \hfill\break
 \emph{Toritachi no kokochi-yoi\slash ii saezuri ni mimi wo katamukeru. }\hfill\break
To listen carefully to the pleasant songs of the birds. }

\par{\textbf{Phrase Note }: \emph{Mimi wo katamukeru }耳を傾ける literally means “to tilt one\textquotesingle s ears.” \emph{Kokochi }心地 means “sensation,” and it is seen following the stem of verbs as \emph{gokochi }to show the “sensation of doing.” In which case, these phrases are very frequently followed by \emph{yoi\slash ii }よい・いい. }

\par{\textbf{Grammar Note }: \emph{-tachi }たち is a suffix that indicates a group of something. }

\par{\textbf{Spelling Note }: Although rare and difficult, \emph{saezuri }may be alternatively written in \emph{Kanji }as 囀り. }

\par{28. ${\overset{\textnormal{いごこち}}{\text{居心地}}}$ 良い ${\overset{\textnormal{ばしょ}}{\text{場所}}}$ で ${\overset{\textnormal{とくべつ}}{\text{特別}}}$ な ${\overset{\textnormal{じかん}}{\text{時間}}}$ を ${\overset{\textnormal{す}}{\text{過}}}$ ごす。 \hfill\break
 \emph{Igokochi-yoi\slash ii basho de tokubetsu na jikan wo sugosu. }\hfill\break
To spend special time at a cozy place. }

\par{29. ${\overset{\textnormal{こうそう}}{\text{高層}}}$ マンションの ${\overset{\textnormal{さいじょうかい}}{\text{最上階}}}$ は ${\overset{\textnormal{す}}{\text{住}}}$ み ${\overset{\textnormal{ごこち}}{\text{心地}}}$ 良いのですか。 \hfill\break
 \emph{Kōsō manshon no saijōkai wa sumigokochi-yoi no desu ka? }\hfill\break
Is it comfortable living on the top floor of a high-rise apartment complex? \hfill\break
 \hfill\break
\textbf{Reading Note }: Due to the presence of \emph{no desu ka }のですか, it becomes more unlikely that 良い is read as \slash ii\slash . }

\begin{center}
\textbf{\emph{Yoku }よく }
\end{center}

\par{ The adverbial form \emph{yoku }よく may be used to mean “nicely\slash well” or “frequently\slash often,” but differentiating between these usages will require contextual clues. }

\par{30. ${\overset{\textnormal{は}}{\text{歯}}}$ をよく ${\overset{\textnormal{みが}}{\text{磨}}}$ いてください。 \hfill\break
 \emph{Ha wo yoku migaite kudasai. }\hfill\break
Brush your teeth well. }

\par{31. ${\overset{\textnormal{かれ}}{\text{彼}}}$ は ${\overset{\textnormal{ほんとう}}{\text{本当}}}$ によく ${\overset{\textnormal{は}}{\text{歯}}}$ を ${\overset{\textnormal{みが}}{\text{磨}}}$ いているかどうかわかりません。 \hfill\break
 \emph{Kare wa hontō ni yoku ha wo migaite iru ka dō ka wakarimasen. }\hfill\break
I don\textquotesingle t really know whether or not he brushes his teeth often. }

\par{32. よくやりましたね。 \hfill\break
 \emph{Yoku yarimashita ne. }\hfill\break
Wow, you did well. }

\par{33. あの ${\overset{\textnormal{こ}}{\text{子}}}$ はお ${\overset{\textnormal{かあ}}{\text{母}}}$ さんとよく ${\overset{\textnormal{に}}{\text{似}}}$ ていますね。 \hfill\break
 \emph{Ano ko wa okā-san to yoku nite imasu ne. }\hfill\break
That child closely resembles his\slash her mother. }

\par{34. それ、 ${\overset{\textnormal{さいきん}}{\text{最近}}}$ よく ${\overset{\textnormal{き}}{\text{聞}}}$ きますね。 \hfill\break
 \emph{Sore, saikin yoku kikimasu ne. }\hfill\break
You here that a lot recently, don\textquotesingle t you. }

\par{35. よく ${\overset{\textnormal{でんわ}}{\text{電話}}}$ する ${\overset{\textnormal{あいて}}{\text{相手}}}$ を ${\overset{\textnormal{とうろく}}{\text{登録}}}$ する。 \hfill\break
 \emph{Yoku denwa suru aite wo tōroku suru. } \hfill\break
To register those one often calls. }

\par{36. ${\overset{\textnormal{わたし}}{\text{私}}}$ はよくピザを ${\overset{\textnormal{ちゅうもん}}{\text{注文}}}$ します。 \hfill\break
 \emph{Watashi wa yoku piza wo chūmon shimasu. }\hfill\break
I often order pizza. }

\par{37. イチゴをよく ${\overset{\textnormal{た}}{\text{食}}}$ べますか。 \hfill\break
 \emph{Ichigo wo yoku tabemasu ka. }\hfill\break
Do you often eat strawberries? }

\par{\textbf{Spelling Note }: \emph{Ichigo }may alternatively be spelled in \emph{Kanji }as 苺. }

\par{38. よく ${\overset{\textnormal{き}}{\text{聞}}}$ いてください。 \hfill\break
 \emph{Yoku kiite kudasai. }\hfill\break
Please listen closely. }

\par{39. ${\overset{\textnormal{かれ}}{\text{彼}}}$ は、 ${\overset{\textnormal{てぎわ}}{\text{手際}}}$ よく ${\overset{\textnormal{しょっき}}{\text{食器}}}$ を ${\overset{\textnormal{かさ}}{\text{重}}}$ ねていました。 \hfill\break
 \emph{Kare wa, tegiwa yoku shokki wo kasanete imashita. }\hfill\break
He was skillfully stacking tableware. }

\par{40. そのカメは ${\overset{\textnormal{うん}}{\text{運}}}$ よく ${\overset{\textnormal{ながい}}{\text{長生}}}$ きしました。 \hfill\break
 \emph{Sono kame wa un yoku nagaiki shimashita. }\hfill\break
The turtle luckily lived a long life. }
    