    
\chapter{Daily Expressions I}

\begin{center}
\begin{Large}
第25課: Daily Expressions I 
\end{Large}
\end{center}
 
\par{There are many expressions that native speakers of a language use on a constant basis. In the English-speaking world, it would be hard, for instance, to go a day without telling someone “hello.” This lesson will introduce you to many such phrases in Japanese. }

\par{Many of the phrases of this lesson are used in greetings ( \emph{aisatsu } 挨拶 ). These phrases are especially important to Japanese culture. As we\textquotesingle ve seen, Japanese places a lot of stress in how one addresses others. This will be very true for the phrases introduced in this lesson. Dialectical differences will also be of importance. Because speech styles and dialects bring about a lot of grammar that you haven\textquotesingle t encountered, the purpose for this lesson will be simply to learn the set phrases explicitly introduced. }
      
\section{Greetings of the Day: Hibi no Aisatsu 日々の挨拶}
 
\par{ To begin, we'll learn about the greeting phrases for morning, afternoon, and evening. There will be variation depending on dialect and speech style, but try not to stress over the variation too much. }

\par{\textbf{Grammar Note }: You will notice the prefix \emph{o\slash go- }お・ご in front of many phrases discussed in this lesson. This prefix is an honorific marker which helps make what it attaches to be more respectful. Much later on, we will learn how to use this constructively. }

\begin{center}
\textbf{Good Morning }
\end{center}

\par{ “Good morning” is expressed with the adjective for early, \emph{hayai }早い, in its traditional honorific form \emph{o-hayō gozaimasu }お早うございます. In casual situations, this is shortened to \emph{o-hayō }お早う. Note that the pronunciation of \emph{o-hayō }お早う is not the same as the state of Ohio. In Japanese, the latter would be rendered as \emph{Ohaio(-shū) }オハイオ(州). Do not mishear \slash io\slash  as \slash yo\slash  as they are not the same. }

\par{\textbf{Intonation Note }: The intonation of this phrase is お \textbf{はよう }ご \textbf{ざいま }す. }

\par{1. ${\overset{\textnormal{せんせい}}{\text{先生}}}$ 、お ${\overset{\textnormal{はよ}}{\text{早}}}$ うございます。 \hfill\break
 \emph{Sensei, o-hayō gozaimasu. }\hfill\break
Good morning, teacher. }

\par{2. よー、 ${\overset{\textnormal{けんじくん}}{\text{健二君}}}$ !お ${\overset{\textnormal{はよ}}{\text{早}}}$ う! ${\overset{\textnormal{げんき}}{\text{元気}}}$ ? \hfill\break
 \emph{Y }ō \emph{, Kenji-kun! O-hayō! Genki? \hfill\break
 }Hey, Kenji-kun! Morning! How are ya? }

\par{ For those who live in the Kinki Region ( \emph{Kinki Chihō }近畿地方) where Kansai Dialects ( \emph{Kansai-ben }関西弁) are spoken, you will frequently hear people use \emph{o-hayō-san }お早うさん instead. }

\par{3. ${\overset{\textnormal{げんき}}{\text{元気}}}$ に ${\overset{\textnormal{いち}}{\text{一}}}$ 、 ${\overset{\textnormal{に}}{\text{二}}}$ おはようさん!お ${\overset{\textnormal{てて}}{\text{手手}}}$ を ${\overset{\textnormal{ふ}}{\text{振}}}$ っておはようさん! \hfill\break
\emph{Genki ni ichi, ni o-hayō-san! O-tete wo futte o-hay }\emph{ō }\emph{-san! }\hfill\break
Now a lively good morning in one, two! Wave your hands good morning! }

\par{\textbf{Sentence Note }: This example comes from a well-known children\textquotesingle s song, \emph{bōkaru shoppu }ボーカル・ショップ. }

\par{ There are also two casual variants of "good morning" that would be wise to remember as well. The first is \emph{ohayōn }おはよーん, which adds a little cutesy flare to the greeting. Another variant is \emph{osoyō }おそよう, which is a portmanteau of \emph{osoi }遅い and \emph{ohayō }おはよう. Unsurprisingly, this is used in a sarcastic manner towards friends that have woken up far later than they should have. }

\begin{center}
\textbf{Hello\slash Good Afternoon }
\end{center}

\par{ “Hello” is associated with the phrase \emph{kon\textquotesingle nichi wa }こんにちは (alternatively written as 今日は). In face-to-face encounters, it is used primarily in the afternoon. This is why it is closer to the English expression “good afternoon.” However, when the time of day is not relevant or ascertainable, especially in situations online, it is used just like “hello.” The reason why the particle \emph{wa }は is used is because at one time, Japanese people used to greet each other by first making a comment about the day\textquotesingle s weather. Though this still happens, this phrase can still stand alone regardless whether a complete sentence is made of it. }

\par{\textbf{Intonation Note }: The intonation of this phrase is こ \textbf{んにちは }↓. }

\par{4. すみません、こんにちは! \hfill\break
 \emph{Sumimsen, kon\textquotesingle nichi wa! \hfill\break
 }Excuse me. Hello! }

\par{5. こんにちは、ご ${\overset{\textnormal{へんじ}}{\text{返事}}}$ ありがとうございます。 \hfill\break
 \emph{Kon\textquotesingle nichi wa, go-henji arigatō gozaimasu. \hfill\break
 }Hello, thank you for replying. }

\par{6. こんにちは、 ${\overset{\textnormal{さいしん}}{\text{最新}}}$ のニュースをお ${\overset{\textnormal{つた}}{\text{伝}}}$ えします。 \hfill\break
 \emph{Kon\textquotesingle nichi wa, saishin no nyūsu wo o-tsutae shimasu. \hfill\break
 }Good afternoon, here\textquotesingle s the latest news. }

\begin{center}
\textbf{Okinawan Hello }
\end{center}

\par{ For those of you who may be stationed in Okinawa, you may also be familiar with the expressions \emph{haisai }はいさい (men) and \emph{haitai }はいたい (for women). However, it is important to treat these words as belonging to a separate language, Okinawan, that have managed to stick around in the daily lives of Japanese speakers in Okinawa (almost all of whom share cultural affinity as Okinawans). Who one should and shouldn't use these expressions also depends on where you are in Okinawa. Therefore, if you are in Okinawa, finding out who you can say these expressions to may open many more conversations to become familiarized with the local culture. }

\begin{center}
\textbf{Casual Hello }
\end{center}

\par{ In many circles, “hello” is often expressed with \emph{chiwassu }ちわっす. This is only used with people who you are close to where there isn\textquotesingle t emphasis on social order. }

\par{ Similarly to how English speakers jokingly use “howdy,” Japanese speakers sometimes use \emph{koncha(ssu) }こんちゃ(っす). However, this gets most currency on Internet forums. The added \emph{ssu }っすcomes from \emph{desu }です. }

\par{ Another well-known casual way to say “hello” is \emph{ossu }おっす or \emph{ussu }うっす. These phrases are used a lot by guys as well as among team sport players, to name just a few instances where this is frequently used. Both variants come from abbreviating \emph{ohayō }おはよう. In response, one responds with \emph{ossu(ssu) }おっす(っす)or \emph{ussu(ssu) }うっす(っす) . The added \emph{ssu }っすcomes from \emph{desu }です. }

\par{ Similarly, Japanese people have also had a custom of making a comment about the evening whenever they would meet past daylight hours. This has brought about the phrase \emph{komban wa }今晩は, which translates as “good evening.” The time of day when you switch from \emph{kon\textquotesingle nichi wa }こんにちは and \emph{komban wa }今晩は is mostly determined by whether the sun is still out. If it is still bright outside, you use the former. If it is already darkening outside, you use the latter. }

\par{\textbf{Intonation Note }: The intonation of this phrase is こ \textbf{んばんは }↓. }

\par{7. ${\overset{\textnormal{こんばん}}{\text{今晩}}}$ は、ニュース ${\overset{\textnormal{セブン}}{\text{7}}}$ です。 \hfill\break
 \emph{Komban wa, nyūsu sebun desu. \hfill\break
 }Good evening, this is News 7. }

\par{8. お ${\overset{\textnormal{つき}}{\text{月}}}$ さん、 ${\overset{\textnormal{こんばん}}{\text{今晩}}}$ は。 \hfill\break
 \emph{O-tsuki-san, komban wa. \hfill\break
 }Good evening, moon. }

\par{9. ${\overset{\textnormal{こんばん}}{\text{今晩}}}$ は、 ${\overset{\textnormal{ねこ}}{\text{猫}}}$ ちゃん。 ${\overset{\textnormal{かわい}}{\text{可愛}}}$ いね。 \hfill\break
 \emph{Komban wa, neko-chan. Kawaii ne. \hfill\break
 }Good evening, kitty. Aren\textquotesingle t you cute? }

\par{\textbf{Dialect Notes }: }

\par{1. In traditional Kyoto Dialect ( \emph{Kyōto-ben }京都弁), “good evening” is expressed with the phrase \emph{oshimaiyasu }おしまいやす, which is still used by those who wish to preserve the beauty of the local dialect. \hfill\break
2.In various parts of Western Japan, you\textquotesingle ll hear \emph{banjimashite }晩じまして used instead. \hfill\break
3. In various parts of Northeastern Japan ( \emph{Tōhoku Chihō }東北地方) as well as Hokkaido ( \emph{Hokkaidō }北海道), you may also hear \emph{o-bankata }お晩方 and\slash or \emph{o-ban desu }お晩です。 }

\par{\textbf{Casual Speech Note }: Casually, “good evening” can be expressed with \emph{kombancha }こんばんちゃ, but this usually only gets currency online, where it may alternatively be further shortened and then spelled as \emph{bancha }番茶, which means "coarse tea" if taken literally. }

\begin{center}
\textbf{Good Night }\hfill\break

\end{center}

\par{ To say “good night,” you use the phrase \emph{o-yasumi-nasai }お休みなさい. Literally, this means “please go rest,” but it is still used whenever you are parting with someone at night. This phrase is also appropriate in figurative uses like the English phrase “may you rest” when at a funeral. Meaning, this phrase is very multi-faceted. In casual speech where it would simply be used to mean “good night,” it can be shortened to \emph{o-yasumi }お休み. }

\par{ Also, whenever you wish to be exceptionally formal in saying “good night,” there is also the longer variant \emph{o-yasumi-nasaimase }\emph{ }お休みなさいませ. Additionally, instead of \emph{nasai }なさい, \emph{kudasai }ください may be used instead, creating \emph{o-yasumi-kudasai(mase) }お休み下さい(ませ). Lastly, it's important to note that it doesn't have to be nighttime to tell someone "good night" as the verb used in these expressions literally means “to rest.” }

\par{\textbf{Intonation Notes }: }

\par{1. お \textbf{やすみなさ }い. \hfill\break
2. お \textbf{やすみくださ }い. }

\par{10. ${\overset{\textnormal{まつい}}{\text{松井}}}$ さん、お ${\overset{\textnormal{やす}}{\text{休}}}$ みなさい。 \hfill\break
 \emph{Matsui-san, o-yasumi-nasai. \hfill\break
 }Good night, Matsui-san. }

\par{11. ${\overset{\textnormal{よる}}{\text{夜}}}$ はぐっすりとお ${\overset{\textnormal{やす}}{\text{休}}}$ みください。 \hfill\break
 \emph{Yoru wa gussuri to o-yasumi-kudasai. }\emph{\hfill\break
 }Please sleep tight at night. }

\par{\textbf{Culture Note }: When you have already greeted someone once in the day, it is customary to simply give a small bow. This is called an \emph{eshaku }会釈. }
      
\section{Farewell: Wakare no Kotoba 別れの言葉}
 \hfill\break
\textbf{\emph{Sayōnara }さようなら }\hfill\break
 There are many expressions for “see your later” and “farewell” in Japanese. The most well-known expression to the Western world is \emph{sayōnara }さようなら, which is a shortening of \emph{sayō-naraba }左様ならば, which literally means “if that\textquotesingle s so.” Its non-abbreviated form only lives on in purposely old-fashioned samurai-mimicking speech, but \emph{sayōnara }さようなら nonetheless remains an important expression today. \hfill\break
 
\par{\textbf{Intonation Note }: The intonation of this phrase is さ \textbf{ような }ら. }
 
\par{\emph{ Sayōnara }さようなら is a very formal expression. It is used by students at school to their instructors at the end of each day from elementary school to high school. Outside school, it is usually perceived as a literal “farewell,” thus making its use rather rare. It may sometimes be seen shortened as \emph{sayonara }さよなら  and used as such in expressions like \emph{sayonara pātii }さよならパーティー (farewell party). It may also be seen in some dialects as \emph{sainara }さいなら, in which case it can be more broadly used to mean “bye.” }
 
\par{12. }
 
\par{${\overset{\textnormal{にっちょく}}{\text{日直}}}$ 「 ${\overset{\textnormal{かえ}}{\text{帰}}}$ りの ${\overset{\textnormal{あいさつ}}{\text{挨拶}}}$ をしましょう」 \hfill\break
 ${\overset{\textnormal{ぜんいん}}{\text{全員}}}$ 「 ${\overset{\textnormal{せんせい}}{\text{先生}}}$ 、さようなら! ${\overset{\textnormal{みな}}{\text{皆}}}$ さん、さようなら! ${\overset{\textnormal{くるま}}{\text{車}}}$ に ${\overset{\textnormal{き}}{\text{気}}}$ を ${\overset{\textnormal{つ}}{\text{付}}}$ けて ${\overset{\textnormal{かえ}}{\text{帰}}}$ ります」 \hfill\break
 \emph{Nitchoku “Kaeri no aisatsu wo shimashō” \hfill\break
Zen\textquotesingle in “Sensei, sayōnara! Mina-san, sayōnara! Kuruma ni ki wo tsukete kaerimasu” }\hfill\break
Kid on Duty: Let\textquotesingle s give our going-home salutations. \hfill\break
Everyone: Goodbye, teacher! Goodbye, everyone! I\textquotesingle ll watch out for cars as I go home. }
 
\par{13. }
 
\par{${\overset{\textnormal{にっちょく}}{\text{日直}}}$ 「 ${\overset{\textnormal{れい}}{\text{礼}}}$ 、さようなら!」 \hfill\break
 ${\overset{\textnormal{ぜんいん}}{\text{全員}}}$ 「さようなら!」 \hfill\break
 \emph{Nitchoku “Rei, sayōnara!” \hfill\break
Zen\textquotesingle in “Sayōnara!” }\hfill\break
Kid on Duty: Bow and goodbye! \hfill\break
Everyone: Goodbye! }
 
\par{14. さよならパーティーをしました。 \hfill\break
 \emph{Sayonara pātii wo shimashita. }\hfill\break
We had a farewell party. }
 
\begin{center}
\textbf{A Simple Goodbye } \hfill\break

\end{center}

\par{ In the realm of casual conversation, friends say “goodbye” to each other with all sorts of phrases based off certain key words like \emph{mata }また (again), \emph{ato de }後で (later), \emph{ashita }明日 (tomorrow) and \emph{raishū }来週 (next week). }

\begin{ltabulary}{|P|P|}
\hline 

See you later, k? &  \emph{(Ja,) mata [ne\slash na]! }(じゃ、)また\{ね・な\}! \\ \cline{1-2}

Later! &  \emph{Ja(a) [ne\slash na]! }じゃ(あ)\{ね・な\}! \\ \cline{1-2}

Bye-bye! &  \emph{Baibai! }バイバイ! \\ \cline{1-2}

See you tomorrow! &  \emph{Mata ashita (ne) }また明日(ね) \\ \cline{1-2}

See you next week! &  \emph{Mata raishū (ne) }また来週(ね) \\ \cline{1-2}

Well… + ↑ &  \emph{(Sore) [de wa\slash ja(a)] }(それ)\{では・じゃ(あ)\} \\ \cline{1-2}

\end{ltabulary}
 
\par{\textbf{Variation Note }: The particle \emph{ne }ね is often switched out for \emph{na }な by male speakers. }
 
\par{15. (それ)じゃあ、また ${\overset{\textnormal{らいしゅう}}{\text{来週}}}$ ! \hfill\break
 \emph{(Sore) j }\emph{ā, mata raish }\emph{ū! \hfill\break
 }Well, see you next week! }
 
\par{16. ${\overset{\textnormal{かん}}{\text{寛}}}$ ちゃん、バイバイ! \hfill\break
 \emph{Kan-chan, baibai! \hfill\break
 }Bye-bye, Kan-chan! }
 
\par{17. もちろん ${\overset{\textnormal{い}}{\text{行}}}$ きますよ。それじゃ、また! \hfill\break
 \emph{Mochiron ikimasu yo. Sore ja, mata! \hfill\break
 }Of course I\textquotesingle m going. Well, see you! }
 
\par{18. やあ、 ${\overset{\textnormal{きょう}}{\text{今日}}}$ は ${\overset{\textnormal{ほんとう}}{\text{本当}}}$ に ${\overset{\textnormal{たの}}{\text{楽}}}$ しかった!みんなありがとう、また ${\overset{\textnormal{あした}}{\text{明日}}}$ ね! \hfill\break
 \emph{Yā, kyō wa hontō ni tanoshikatta! Min\textquotesingle na arigatō, mata ashita ne! \hfill\break
 }Wow, today was really fun! Thanks, everyone; see you all tomorrow! }
 
\par{ Of course, there are plenty other variants that you may encounter wherever you might be in Japan. These expressions, though, don\textquotesingle t cover what adults would use in the realm of polite\slash formal interactions. }
 
\begin{center}
\textbf{Leaving the Office }
\end{center}
 
\par{ When leaving before one\textquotesingle s coworkers and boss, you will need to say \emph{o-saki ni shitsurei shimasu }お先に失礼します. This can be translated as “Forgive me for leaving first.” When leaving coworkers at the same time, \emph{o-tsukare-sama desu }お疲れ様です is used. It may also be shortened to \emph{o-tsukare-sama }お疲れ様 , \emph{o-tsukare }お疲れ, or even be seen as \emph{o-tsukare-san }お疲れさん depending on how casual and cordial you are with your coworkers. }
 
\par{ As \emph{o-tsukare-sama desu }お疲れ様です, it's also used by those who greet those who've just come from a hard day at work. It's also frequently used at the start of business e-mails, recognizing the addressee\textquotesingle s involvement in a matter. Alternatively, \emph{o-tsukare-sama deshita }お疲れ様でした adds a sense of gratitude for that person\textquotesingle s work, but it mustn\textquotesingle t be used if you think that coworker isn\textquotesingle t actually leaving just yet. }
 
\par{19. }
 
\par{${\overset{\textnormal{やましたさま}}{\text{山下様}}}$ \hfill\break
お ${\overset{\textnormal{つか}}{\text{疲}}}$ れ ${\overset{\textnormal{さま}}{\text{様}}}$ です。 ${\overset{\textnormal{まるまるかぶしきがいしゃ}}{\text{〇〇株式会社}}}$ の ${\overset{\textnormal{かねだりょうた}}{\text{金田亮太}}}$ です。 \hfill\break
 \emph{Yamashita-sama \hfill\break
O-tsukare-sama desu. Maru-maru-kabushikigaisha no Kaneda Ryōta desu. \hfill\break
 }Mr. Yamashita \hfill\break
First, let me thank you for your work. I am Ryota Kaneda from \#\# Incorporated. }
 
\par{20. これで ${\overset{\textnormal{お}}{\text{終}}}$ わりにしましょう。 ${\overset{\textnormal{みな}}{\text{皆}}}$ さん、お ${\overset{\textnormal{つか}}{\text{疲}}}$ れ ${\overset{\textnormal{さま}}{\text{様}}}$ でした。 \hfill\break
 \emph{Kore de owari ni shimasho. Mina-san, o-tsukare-sama deshita. \hfill\break
 }Let\textquotesingle s end here. Thank you for your hard work, everyone. }
 
\par{ In addition to this, there is also the phrase \emph{go-kurō-sama desu }ご苦労様です, alternatively \emph{as go-kurō-san desu }ご苦労さんです,  used by superiors to their underlings. Depending on how they view their relationship with you, this may be shortened to \emph{go-kurō-sama }ご苦労様 or even just \emph{go-kurō }ご苦労. }
 
\par{21. よく ${\overset{\textnormal{がんば}}{\text{頑張}}}$ った、ご ${\overset{\textnormal{くろう}}{\text{苦労}}}$ さん! (Boss Talk) \hfill\break
 \emph{Yoku gambatta, go-kurō-san! \hfill\break
 }You worked hard. Thanks for your work! }
 
\par{ Typically, when parting with someone you should show respect to, you use \emph{shitsurei shimasu }失礼します when leaving them. For instance, say you\textquotesingle re a student that went to your teacher\textquotesingle s office hours, you\textquotesingle d part with him\slash her by saying this. When leaving somewhere in a hurry but feeling still inclined to give respect to whomever may have been of service to you, you may also hear people say \emph{dōmo }どうも. }
 
\par{22. ${\overset{\textnormal{しゃいん}}{\text{社員}}}$ 「 ${\overset{\textnormal{ほか}}{\text{他}}}$ に ${\overset{\textnormal{しごと}}{\text{仕事}}}$ はありませんか。」 \hfill\break
 ${\overset{\textnormal{じょうし}}{\text{上司}}}$ 「いや、 ${\overset{\textnormal{きょう}}{\text{今日}}}$ は ${\overset{\textnormal{だいじょうぶ}}{\text{大丈夫}}}$ です。」 \hfill\break
 ${\overset{\textnormal{しゃいん}}{\text{社員}}}$ 「わかりました。では、お ${\overset{\textnormal{さき}}{\text{先}}}$ に ${\overset{\textnormal{しつれい}}{\text{失礼}}}$ します。」 \hfill\break
 ${\overset{\textnormal{じょうし}}{\text{上司}}}$ 「はい、お ${\overset{\textnormal{つか}}{\text{疲}}}$ れ。」 \hfill\break
 \emph{Shain “Hoka ni shigoto wa arimasen ka?” \hfill\break
Jōshi “Iya, kyō wa daijōbu desu.” \hfill\break
Shain “Wakarimashita. De wa, o-saki ni shitsurei shimasu.” \hfill\break
Jōshi “Hai, o-tsukare.” }\hfill\break
Employee: Is there anything else to do? \hfill\break
Boss: No, we\textquotesingle re good for today. \hfill\break
Employee: Understood. In which case, do pardon me for leaving first.” \hfill\break
Boss: That\textquotesingle s fine. Good work. }
 
\begin{center}
\textbf{\emph{Saraba }さらば }
\end{center}
 
\par{ One last expression we\textquotesingle ll go over that means “farewell” is \emph{saraba }さらば. This is a contraction of an archaic expression meaning “if that\textquotesingle s so.” Nowadays, this phrase is old-fashioned or adds some sense of affection to the situation, which can be interpreted in various ways depending on what\slash who one is departing with. It is mainly used by men. }
 
\par{23. さらば、 ${\overset{\textnormal{みらい}}{\text{未来}}}$ 。 \hfill\break
 \emph{Saraba, mirai. \hfill\break
 }Farewell, future. }
 
\par{24. (お)さらばだ。 \hfill\break
 \emph{O-saraba da. \hfill\break
 }Farewell. }
 
\par{\textbf{Sentence Note }: \emph{O- }お may be added with no change to meaning. It just makes the phrase have better cadence. }
 
\par{25. さらばじゃ。 \hfill\break
 \emph{Saraba ja. \hfill\break
 }Farewell. }
 
\par{\textbf{Sentence Note }: Ex. 21 would be indicative of an elderly man. }
 
\par{26. さらば、 ${\overset{\textnormal{はこだて}}{\text{函館}}}$ よ。 \hfill\break
 \emph{Saraba, Hakodate yo. \hfill\break
 }Farewell, Hakodate. }
 
\begin{center}
\textbf{Leaving \& Returning }
\end{center}
 
\par{ Whenever you leave somewhere, it\textquotesingle s important to say \emph{itte kimasu }行ってきます or some variant of it. }

\begin{ltabulary}{|P|P|P|}
\hline 

Plain Speech & Polite Speech & Humble Speech \\ \cline{1-3}

 \emph{Itte kuru }行ってくる &  \emph{Itte kimasu }行ってきます &  \emph{Itte mairimasu }行ってまいります \\ \cline{1-3}

\end{ltabulary}
 
\par{ These phrases literally mean that one is going but coming right back. If seen in the past tense, it implies that you went to go do something but have since returned. }
 
\par{27. ${\overset{\textnormal{いま}}{\text{今}}}$ から ${\overset{\textnormal{えいかいわ}}{\text{英会話}}}$ に ${\overset{\textnormal{い}}{\text{行}}}$ ってきます。 \hfill\break
 \emph{Ima kara eikaiwa ni itte kimasu. \hfill\break
 }I\textquotesingle m heading to English conversation now (and will be back). }
 
\par{28. ちょっと ${\overset{\textnormal{い}}{\text{行}}}$ ってくるね。 \hfill\break
 \emph{Chotto itte kuru ne. \hfill\break
 }I\textquotesingle m going to be out for a bit, okay? }
 
\par{29. それでは ${\overset{\textnormal{ゆめ}}{\text{夢}}}$ の ${\overset{\textnormal{せかい}}{\text{世界}}}$ へ ${\overset{\textnormal{い}}{\text{行}}}$ って ${\overset{\textnormal{まい}}{\text{参}}}$ ります。 \hfill\break
 \emph{Sore de wa, yume no sekai e itte mairimasu. \hfill\break
 }Well now, I will be heading to a\slash the world of dreams! }
 
\par{30. ロンドンに ${\overset{\textnormal{い}}{\text{行}}}$ って ${\overset{\textnormal{まい}}{\text{参}}}$ ります。 \hfill\break
 \emph{Rondon ni itte mairimasu. \hfill\break
 }I\textquotesingle m going to London (and will be back). }
 
\par{31. ${\overset{\textnormal{しゃいんりょこう}}{\text{社員旅行}}}$ で ${\overset{\textnormal{おきなわ}}{\text{沖縄}}}$ に ${\overset{\textnormal{い}}{\text{行}}}$ ってきました! \hfill\break
 \emph{Shain ryokō de Okinawa ni itte kimashita! \hfill\break
 }I went on a company trip to Okinawa. }
 
\par{32. ${\overset{\textnormal{ぜんいん}}{\text{全員}}}$ で ${\overset{\textnormal{いんがいけんしゅう}}{\text{院外研修}}}$ に ${\overset{\textnormal{い}}{\text{行}}}$ ってまいりました。 \hfill\break
 \emph{Zen\textquotesingle in de ingai kenshu ni itte mairimashita. \hfill\break
 }We all went together to an outside training. }

\begin{center}
 \textbf{Going Out to Do\dothyp{}\dothyp{}\dothyp{} }
\end{center}
 
\par{ The above grammar can be extended by replacing the verb \emph{iku }行く (to go) with any activity verb. }
 
\par{33. じゃ、 ${\overset{\textnormal{くすり}}{\text{薬}}}$ を ${\overset{\textnormal{か}}{\text{買}}}$ ってきます。 \hfill\break
 \emph{Ja, kusuri wo katte kimasu. \hfill\break
 }Well then, I\textquotesingle ll go buy medicine (and be right back) }
 
\par{34. ${\overset{\textnormal{もど}}{\text{戻}}}$ ってくるから、 ${\overset{\textnormal{あんしん}}{\text{安心}}}$ してね。 \hfill\break
 \emph{Modotte kuru kara, anshin shite ne. \hfill\break
 }I\textquotesingle ll be right back, so relax. }
 
\par{35. ミュウツーをゲットしてきました。 \hfill\break
 \emph{Myūtsū wo getto shite kimashita. \hfill\break
 }I\textquotesingle ve come back having caught Mewtwo. }
 
\par{ In response to someone leaving for somewhere, those present customarily say \emph{itte rasshai }行ってらっしゃい, literally meaning “go and come back.”  In more formal situations, \emph{o-ki wo tsukete (itte rasshaimase) }お気をつけて(行ってらっしゃいませ) is used instead. This literally means “please be careful and come back). }
 
\par{36.  はい、 ${\overset{\textnormal{い}}{\text{行}}}$ ってらっしゃい。 \hfill\break
 \emph{Hai, itte rasshai. \hfill\break
 }Well then, be back safely. }
 
\par{37. それでは、お ${\overset{\textnormal{き}}{\text{気}}}$ をつけて( ${\overset{\textnormal{い}}{\text{行}}}$ ってらっしゃいませ)。 \hfill\break
 \emph{Sore de wa, o-ki wo tsukete (itte rasshaimase). \hfill\break
 }Well then, please be careful and get back safely. }
 
\par{ When returning to the office or any other place, you\textquotesingle ll use phrases like the following depending on how formal you need to be. }
 
\par{38. ただいま ${\overset{\textnormal{もど}}{\text{戻}}}$ りました。 \hfill\break
 \emph{Tadaima modorimashita. \hfill\break
 }I\textquotesingle ve arrived just now. }
 
\par{\textbf{Sentence Note }: Ex. 36 would be especially appropriate when returning to the office. }
 
\par{39. ${\overset{\textnormal{もど}}{\text{戻}}}$ ってきたよ。 \hfill\break
 \emph{Modotte kita yo. \hfill\break
 }I\textquotesingle m back. }
 
\par{40. もう ${\overset{\textnormal{かいしゃ}}{\text{会社}}}$ に ${\overset{\textnormal{もど}}{\text{戻}}}$ ってますよ。 \hfill\break
 \emph{Mō kaisha ni modottemasu yo. \hfill\break
 }I\textquotesingle m already back at work (company). }

\begin{center}
\textbf{Returning Home }
\end{center}
 
\par{ When returning home to your family, it is customary to say \emph{tadaima }ただいま, which literally means “now,” emphasizing that you\textquotesingle re at home at last. Those present say \emph{o-kaeri(-nasai) }お帰りなさい. The addition of \emph{-nasai }なさい depends on the dynamics in the home. Wives often use this to their husbands, but if children are taught to speak formally to their parents as a sign of class, they too will not abbreviate the phrase. Those not in the immediate family but happen to be present will also choose to say the full \emph{o-kaeri-nasai }お帰りなさい. }
 
\par{41. }
 ${\overset{\textnormal{だんな}}{\text{旦那}}}$ :ただいま! \hfill\break
 ${\overset{\textnormal{つま}}{\text{妻}}}$ :お ${\overset{\textnormal{かえ}}{\text{帰}}}$ り(なさい)。 \hfill\break
 \emph{Dan\textquotesingle na: Tadaima! \hfill\break
Tsuma: O-kaeri(-nasai). }\hfill\break
Husband: I\textquotesingle m home! \hfill\break
Wife: Welcome back. \hfill\break
 \hfill\break
      
\section{Welcome: Kangei no Kotoba 歓迎の言葉}
 
\par{ The basic word for “welcome” in Japanese is \emph{yōkoso }ようこそ. It can either go at the start or the end of a sentence. When at the beginning, the grammar of the sentence must be inverted. As you'll notice in the examples below, \emph{yōkoso }ようこそ can be used with both the particles \emph{e }へ and \emph{ni }に when it is placed at the end of the sentence. }

\par{ The use of \emph{e }へ implies a sense of adventure and\slash or having gone a long way to get to said point. This sense of being welcomed to a new place is heightened with the sentence is inverted to let \emph{yōkoso }ようこそ be at the front, and in this situation, the place in question can only be marked by \emph{e }へ. }

\par{42. ようこそ、 ${\overset{\textnormal{じごく}}{\text{地獄}}}$ へ。 \hfill\break
 \emph{Yōkoso, jigoku e. }\hfill\break
Welcome to hell. }

\par{43. ようこそ、 ${\overset{\textnormal{にっぽん}}{\text{日本}}}$ へ! \hfill\break
 \emph{Yōkoso, Nippon e. \hfill\break
 }Welcome to Japan. }

\par{44. スイス\{へ・に\}ようこそ。 \hfill\break
 \emph{Suisu [ni\slash e] yōkoso. \hfill\break
 }Welcome to Switzerland. }

\par{45. この ${\overset{\textnormal{せかい}}{\text{世界}}}$ へようこそ。 \hfill\break
 \emph{Kono sekai e yōkoso. \hfill\break
 }Welcome to this world. }

\par{46. ようこそ ${\overset{\textnormal{ウインドウズ}}{\text{Windows}}}$  ${\overset{\textnormal{テン}}{\text{10}}}$ へ。 \hfill\break
 \emph{Yōkoso Uindouzu ten e. \hfill\break
 }Welcome to Windows 10. }

\par{ There are also more phrases for “welcome” that you\textquotesingle ll hear, some that either do or don't incorporate \emph{yōkoso }ようこそ somehow, but their purpose will always be apparent. }

\par{47. ${\overset{\textnormal{ほんじつ}}{\text{本日}}}$ はようこそ\{お ${\overset{\textnormal{い}}{\text{出}}}$ で・お ${\overset{\textnormal{こ}}{\text{越}}}$ し\}くださいました。 \hfill\break
 \emph{Honjitsu wa yōkoso [o-ide\slash o-koshi] kudasaimashita. \hfill\break
 }Thank you for coming today. }

\par{48. \{ようこそ・よく\}( ${\overset{\textnormal{にほん}}{\text{日本}}}$ に)いらっしゃいました。 \hfill\break
 \emph{[Yōkoso\slash yoku] (Nihon ni) irasshaimashita. \hfill\break
 }Thank you for coming (to Japan)! }

\par{49. いっらしゃいませ! \hfill\break
 \emph{Irasshaimase! \hfill\break
 }Welcome! }

\par{\textbf{Sentence Note }: You\textquotesingle ll hear this when entering many restaurants. }

\par{50. よう、いらっしゃい! \hfill\break
 \emph{Yō, irasshai! \hfill\break
 }Welcome! }

\par{ \textbf{Sentence Note }: Ex. 50 is more familial than Ex. 49. }
    