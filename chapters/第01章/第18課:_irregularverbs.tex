    
\chapter{The Irregular Verbs}

\begin{center}
\begin{Large}
第18課: The Irregular Verbs: Suru する \& Kuru 来る 
\end{Large}
\end{center}
 
\par{ Out of all the verbs in Modern Japanese, there are only two that can be called truly irregular. These verbs are \emph{suru }する and \emph{kuru }来る. Yet, even though they're called irregular, their conjugations are only minutely different from any other verbs. Meaning-wise, their basic usages are simple. As such, this lesson will not be as difficult as you might have thought. }
      
\section{Vocabulary List}
 \textbf{Nouns }
\par{・宿題 \emph{Shukudai }– Homework }
 
\par{・上司 \emph{Jōshi }–  Boss\slash superior authority }
 
\par{・こと \emph{Koto }– Thing\slash matter\slash event }
 
\par{・無理 \emph{Muri }– Something unreasonable }
 
\par{・仕事 \emph{Shigoto }– Job\slash work }
 
\par{・弟 \emph{Otōto }– Little brother }
 
\par{・掃除 \emph{Sōji }– Cleaning\slash sweeping }
 
\par{・皿洗い \emph{Sara\textquotesingle arai }– Dish-washing }
 
\par{・研究 \emph{Kenkyū }– Research\slash study }
 
\par{・不正行為 \emph{Fusei kōi }– Malpractice\slash unfair practice }
 
\par{・息 \emph{Iki }– Breath\slash breathing }
 
\par{・あくび \emph{Akubi }– Yawn\slash yawning }
 
\par{・恋 \emph{Koi }– (Romantic) love }
 
\par{・旅 \emph{Tabi }– Travel\slash journey }
 
\par{・運転 \emph{Unten }– Driving }
 
\par{・英文 \emph{Eibun }– English text }
 
\par{・Tシャツ \emph{Tii-shatsu }– T-shirt }
 
\par{・カレンダー \emph{Karendā }– Calendar }
 
\par{・勉強 \emph{Benkyō }– Study\slash studying }
 
\par{・フィンランド語 \emph{Finrandogo }– Finnish }
 
\par{・広東語 \emph{Kantongo }– Cantonese }
 
\par{・レンタル \emph{Rentaru }– Rental }
 
\par{・ワイン \emph{Wain }– Wine }
 
\par{・関税 \emph{Kanzei }– Customs duties }
 
\par{・料金 \emph{Ryōkin }– Fee\slash charge\slash fare }
 
\par{・手袋 \emph{Tebukuro }– Gloves }
 
\par{・サラリーマン \emph{Sarariiman }– Businessman }
 
\par{・ネクタイ \emph{Nekutai }– Necktie }
 
\par{・靴 \emph{Kutsu }– Shoe(s) }
 
\par{・眼鏡 \emph{Megane }– Glasses }
 
\par{・帽子 \emph{Bōshi }– Hat }
 
\par{・手錠 \emph{Tejō }– Handcuffs }
 
\par{・匂い \emph{Nioi }– Smell\slash scent }
 
\par{・味 \emph{Aji }– Flavor\slash taste }
 
\par{・感じ \emph{Kanji }– Feeling\slash sense }
 
\par{・量 \emph{Ryō }– Quantity }
 
\par{・音 \emph{Oto }– Sound }
 
\par{・気持ち \emph{Kimochi }– Feeling }
 
\par{・家庭教師 \emph{Katei kyōshi }– Private tutor }
 
\par{・料理人 \emph{Ryōrinin }– Cook }
 
\par{・銀行員 \emph{Ginkōin }– Bank clerk }
 
\par{・形 \emph{Katachi }– Form\slash shape }
 
\par{・顔 \emph{Kao }– Face }
 
\par{・目付き \emph{Metsuki }– Expression of the eyes }
 
\par{・目 \emph{Me }– Eye(s) }
 
\par{・格好 \emph{Kakkō }– Appearance\slash posture }
 
\par{・振り \emph{Furi }– Pretense\slash behavior\slash swing }
 
\par{・建物 \emph{Tatemono }– Building }
 
\par{・スタンプ \emph{Sutampu }– Stamp (on SNS) }
 
\par{・パチンコ \emph{Pachinko }– Japanese pinball }
 
\par{・生け花 \emph{Ikebana }– Flower arrangement }
 
\par{・スポーツ \emph{Supōtsu }– Sport(s) }
 
\par{・野球 \emph{Yakyū }– Baseball }
 
\par{・サッカー \emph{Sakkā }– Soccer }
 
\par{・アメフト \emph{Amefuto }– American football }
 
\par{・相撲 \emph{Sumō }– Sumo wrestling }
 
\par{・水泳 \emph{Suiei }– Swimming }
 
\par{・格闘技 \emph{Kakutōgi }– Martial arts }
 
\par{・スキー \emph{Sukii }– Skiing }
 
\par{・ゴルフ \emph{Gorufu }– Golf }
 
\par{・テニス \emph{Tenisu }– Tennis }
 
\par{・卓球 \emph{Takkyū }– Ping pong }
 
\par{・子 \emph{Ko }– Child }
 
\par{・CD \emph{Shiidii }– CD }
 
\par{・春 \emph{Haru }– Spring }
 
\par{・電車 \emph{Densha }– (Electric) train }
 
\par{・連絡 \emph{Renraku }– Communication }
 
\par{・返事 \emph{Henji }– Response\slash reply }
 
\par{・嵐 \emph{Arashi }– Storm }
 
\par{・雪 \emph{Yuki }– Snow }
 
\par{・手紙 \emph{Tegami }– Letter }
 
\par{・日 \emph{Hi }– Day\slash sun }
  \textbf{Pronouns }
\par{・私 \emph{Wata(ku)shi }– I }

\par{・僕 \emph{Boku }– I (male) }

\par{・彼 \emph{Kare }– He }

\par{\textbf{Proper Nouns }}

\par{・LINE \emph{Rain }– LINE }

\par{・サンタさん \emph{Santa-san }– Santa }

\par{\textbf{Adjective }}

\par{・いい \emph{Ii }– Good }

\par{・強い \emph{Tsuyoi }– Strong }

\par{・甘い \emph{Amai }– Sweet }

\par{・少ない \emph{Sukunai }– Few\slash lacking\slash insufficient }

\par{・悲しい \emph{Kanashii }– Sad }

\par{・丸い \emph{Marui }– Round }

\par{・四角い \emph{Shikakui }- Square }

\par{・可愛い \emph{Kawaii }– Cute }

\par{・鋭い \emph{Surudoi }– Sharp }

\par{・冷たい \emph{Tsumetai }– Cold }

\par{\textbf{Adjectival Nouns }}

\par{・変\{な\} \emph{Hen [na] }– Weird }

\par{・ラフ\{な\} \emph{Rafu [na] }– Rough }

\par{\textbf{Demonstratives }}

\par{・この \emph{Kono }– This (adj.) }

\par{・その \emph{Sono }– That (adj.) }

\par{・あの \emph{Ano }– That (over there) (adj.) }

\par{\textbf{Number Phrases }}

\par{・100円 \emph{Hyakuen }– 100 yen }

\par{・2千円 \emph{Nisen\textquotesingle en }– 2,000 yen }

\par{・5000円 \emph{Gosen\textquotesingle en }– 5,000 yen }

\par{・2万円 \emph{Niman\textquotesingle en }– 20,000 yen }

\par{・500ユーロ \emph{Gohyaku-yūro }– 500 euro }

\par{\textbf{Adverbs }}

\par{・今年 \emph{Kotoshi }– This year }

\par{・今夜 \emph{Kon\textquotesingle ya }– Tonight }

\par{・今朝 \emph{Kesa }– This morning }

\par{・今回 \emph{Konkai }– This time }

\par{・あまり \emph{Amari }– Much }

\par{・おおよそ \emph{Ōyoso }– About\slash approximately }

\par{・普通 \emph{Futsū }– Usually }

\par{・昨日 \emph{Kinō }– Yesterday }

\par{・やっと \emph{Yatto }–  Finally }

\par{・ついに \emph{Tsui ni }– At last }

\par{\textbf{\emph{Ichidan (ru) }Verbs }}

\par{・着る \emph{Kiru }– To wear }

\par{・付ける \emph{Tsukeru }– To put on\slash append\slash turn on }

\par{・締める \emph{Shimeru }– To fasten\slash wear }

\par{・かける \emph{Kakeru }– To put on (glasses), etc. }

\par{・はめる \emph{Hameru }- To put on (gloves, etc.)\slash insert  }

\par{\textbf{\emph{Godan (u) }Verbs }}

\par{・かかる \emph{Kakaru }– To cost, etc. }

\par{・穿く \emph{Haku }– To wear (on legs) }

\par{・履く \emph{Haku }– To wear (on feet) }

\par{・被る \emph{Kaburu }–  To put on (one\textquotesingle s head), etc. }

\par{・脱ぐ \emph{Nugu }– To take off }

\par{・取る \emph{Toru }– To take }

\par{・遊ぶ \emph{Asobu }– To have a good time }

\par{・やる \emph{Yaru }– To do\slash play, etc. }

\par{\textbf{\emph{Suru }-Verbs }}

\par{・運転する \emph{Unten suru }– To drive }

\par{・販売する \emph{Hambai suru }– To sell\slash market }

\par{・注文する \emph{Chūmon suru }– To order }

\par{・確保する \emph{Kakuho suru }– To guarantee }

\par{・翻訳する \emph{Hon\textquotesingle yaku suru }– To translate }

\par{・サインインする \emph{Sain\textquotesingle in suru }– To sign-in }

\par{・キャンセルする \emph{Kyanseru suru }– To cancel }

\par{・勉強する \emph{Benkyō suru }– To study }

\par{・電話する \emph{Denwa suru }– To call on the phone }

\par{・投票する \emph{Tōhyō suru }– To vote }

\par{・混乱する \emph{Konran suru }– To be confused }

\par{・構築する \emph{Kōchiku suru }– To construct }
      
\section{Suru する: To Do}
 
\par{ The basic meaning of \emph{suru }する is “to do.” As the "to do" verb of Japanese, it is extensively used. Consequently, it also has many other usages that may or may not relate to the English understanding of “to do” as in actions. }
 
\par{ Before we look at its most important usages, we will first learn how to conjugate it into its basic forms. As you can see from the chart below, it isn\textquotesingle t that different from other verbs. }
 
\begin{center}
\textbf{Plain Conjugations }
\end{center}

\begin{ltabulary}{|P|P|P|P|}
\hline 

Non-Past &  \emph{Suru }する & Past &  \emph{Shita }した \\ \cline{1-4}

Negative &  \emph{Shinai }しない & Negative Past &  \emph{Shinakatta }しなかった \\ \cline{1-4}

\end{ltabulary}

\par{ As you can see, all you do is change \slash su\slash  to \slash shi\slash  and then add any of the endings you\textquotesingle ve learned thus far. To see how these forms look in practice, here are a few examples using \emph{suru }する in the sense of “to do” in ways that don\textquotesingle t require additional considerations. }
 
\par{1. ${\overset{\textnormal{ぼく}}{\text{僕}}}$ は ${\overset{\textnormal{しゅくだい}}{\text{宿題}}}$ をしなかった。 \hfill\break
 \emph{Boku wa shukudai wo shinakatta. \hfill\break
 }I didn\textquotesingle t do my homework. }
 
\par{2. ${\overset{\textnormal{じょうし}}{\text{上司}}}$ は ${\overset{\textnormal{ふせいこうい}}{\text{不正行為}}}$ をした。 \hfill\break
 \emph{Jōshi wa fusei kōi wo shita. \hfill\break
 }The boss did something illegal\slash improper. }
 
\par{3. ${\overset{\textnormal{へん}}{\text{変}}}$ なことをする。 \hfill\break
 \emph{Hen na koto wo suru. \hfill\break
 }To do something weird. }
 
\par{4. ${\overset{\textnormal{むり}}{\text{無理}}}$ はしない。 \hfill\break
 \emph{Muri wa shinai. \hfill\break
 }I won\textquotesingle t do anything unreasonable. }
 
\begin{center}
\textbf{Polite Conjugations }
\end{center}

\begin{ltabulary}{|P|P|P|P|}
\hline 

Non-Past &  \emph{Shimasu }します & Past &  \emph{Shimashita }しました \\ \cline{1-4}

Negative &  \emph{Shinai desu }しないです \hfill\break
\emph{Shimasen }しません & Negative Past &  \emph{Shinakatta desu }しなかったです \hfill\break
\emph{Shimasendeshita }しませんでした \\ \cline{1-4}

\end{ltabulary}

\par{ Just as with other verbs, using \emph{-nai desu }ないです and \emph{-nakatta desu }なかったです instead of \emph{–masen }ません and \emph{–masendeshita }ませんでした is done in colloquial polite speech. In other words, they're not as polite as their latter counterparts, but they\textquotesingle re appropriate in lax situations. }
 
\par{5. いい ${\overset{\textnormal{しごと}}{\text{仕事}}}$ をしますね。 \hfill\break
 \emph{Ii shigoto wo shimasu ne. \hfill\break
 }You do a good job, don\textquotesingle t you. }
 
\par{6. ${\overset{\textnormal{おとうと}}{\text{弟}}}$ は ${\overset{\textnormal{そうじ}}{\text{掃除}}}$ をしました。 \hfill\break
 \emph{Otōto wa sōji wo shimashita. \hfill\break
 }My little brother did cleaning. }
 
\par{7. ${\overset{\textnormal{いや}}{\text{嫌}}}$ なことはしないです。 \hfill\break
 \emph{Iya na koto wa shinai desu. \hfill\break
 }I won\textquotesingle t do anything unpleasant. }
 
\par{8. ${\overset{\textnormal{さらあら}}{\text{皿洗}}}$ いはしません。 \hfill\break
 \emph{Sara\textquotesingle arai wa shimasen. \hfill\break
 }I won\textquotesingle t do the dishes. }
 
\par{9. ${\overset{\textnormal{かれ}}{\text{彼}}}$ も ${\overset{\textnormal{しゅくだい}}{\text{宿題}}}$ をしなかったです。 \hfill\break
 \emph{Kare mo shukudai wo shinakatta desu. \hfill\break
 }He too didn\textquotesingle t do his homework. }
 
\par{10. ${\overset{\textnormal{わたし}}{\text{私}}}$ は ${\overset{\textnormal{ことし}}{\text{今年}}}$ 、あまり ${\overset{\textnormal{けんきゅう}}{\text{研究}}}$ をしませんでした。 \hfill\break
 \emph{Watashi wa kotoshi, amari kenkyū shimasendeshita. \hfill\break
 }I didn\textquotesingle t do much studies\slash research this year. }
  \textbf{Making Nouns Verbs } Now that we have gone through the basic conjugations for both plain speech and polite speech, it is now time for us to go over the most important usages of \emph{suru }する. 
\par{ Firstly, \emph{suru }する is “to do” in a much larger sense both grammatically and semantically speaking. In English, many words can be used as a noun or verb depending on context, but in Japanese, words usually can\textquotesingle t change their part of speech without some sort of change in appearance. For example, the noun for “yawn” is \emph{akubi }あくび, but “to yawn” is \emph{akubi wo suru }あくびをする. Even if  “to do" is not used in any way in English, \emph{suru }する is used to help a wide variety of nouns behave as verbs in this manner. }
 
\par{11. ${\overset{\textnormal{いき}}{\text{息}}}$ をする。 \hfill\break
 \emph{Iki wo suru. \hfill\break
 }To breathe. }
 
\par{12. ${\overset{\textnormal{こい}}{\text{恋}}}$ をする。 \hfill\break
 \emph{Koi wo suru. \hfill\break
 }To be in love }
 
\par{13. ${\overset{\textnormal{たび}}{\text{旅}}}$ をする。 \hfill\break
 \emph{Tabi wo suru. \hfill\break
 }To go on a journey. }
 
\par{ Perhaps most importantly, \emph{suru }する helps make countless words, many of Chinese origin, usable as verbs. For instance, \emph{unten }運転 means “driving,” but to say “to drive,” you say \emph{unten (wo) suru }運転(を)する. Whether you use \emph{wo suru }をする or \emph{suru }する is a discussion we'll leave for another time because it can get complicated, but before we move on to other meanings, here are more examples of “ \emph{suru }-verbs.” }

\begin{ltabulary}{|P|P|P|P|}
\hline 

 \emph{Hambai suru }販売する & To sell\slash market &  \emph{Chūmon suru }注文する & To order \\ \cline{1-4}

 \emph{Kakuho suru }確保する & To guarantee &  \emph{Hon\textquotesingle yaku suru }翻訳する & To translate \\ \cline{1-4}

 \emph{Sain\textquotesingle in suru }サインインする & To sign-in &  \emph{Kyanseru suru }キャンセルする & To cancel \\ \cline{1-4}

 \emph{Benkyō suru }勉強する & To study &  \emph{Denwa suru }電話する & To call on the phone \\ \cline{1-4}

\emph{Tōhyō suru }投票する & To vote &  \emph{Konran suru }混乱する & To be confused \\ \cline{1-4}

\end{ltabulary}

\par{14. ${\overset{\textnormal{えいぶん}}{\text{英文}}}$ を ${\overset{\textnormal{ほんやく}}{\text{翻訳}}}$ しました。 \hfill\break
\emph{Eibun wo hon\textquotesingle yaku shimashita. \hfill\break
}I translated the English sentences. }
 
\par{15. Tシャツを ${\overset{\textnormal{はんばい}}{\text{販売}}}$ します。 \hfill\break
 \emph{Tii-shatsu wo hambai shimasu. \hfill\break
 }(I\slash we) will sell T-shirts. }
 
\par{16. カレンダーを ${\overset{\textnormal{ちゅうもん}}{\text{注文}}}$ しませんか。 \hfill\break
 \emph{Karendā wo chūmon shimasen ka? \hfill\break
 }How about ordering a calendar? }
 
\par{17. ${\overset{\textnormal{こんかい}}{\text{今回}}}$ は ${\overset{\textnormal{とうひょう}}{\text{投票}}}$ しませんでした。 \hfill\break
 \emph{Konkai wa tōhyō shimasendeshita. \hfill\break
 }I did not vote this time. }
 
\par{18. フィンランド ${\overset{\textnormal{ご}}{\text{語}}}$ の ${\overset{\textnormal{べんきょう}}{\text{勉強}}}$ をしました。 \hfill\break
 \emph{Finrandogo no benkyo wo shimashita. \hfill\break
 }I did my Finnish studies. \hfill\break
 \hfill\break
19. ${\overset{\textnormal{かんとんご}}{\text{広東語}}}$ を ${\overset{\textnormal{べんきょう}}{\text{勉強}}}$ しませんか。 \hfill\break
 \emph{Kantongo wo benkyō shimasen ka? \hfill\break
 }Why don\textquotesingle t you study Cantonese? }
 
\par{20. ${\overset{\textnormal{こんやでんわ}}{\text{今夜電話}}}$ しますね。 \hfill\break
 \emph{Kon\textquotesingle ya denwa shimasu ne. \hfill\break
 }I\textquotesingle ll call you tonight, ok? }
 
\par{\textbf{Grammar Note }: In Ex. 18, the word \emph{benkyō }勉強 is used as a noun attached to the previous word with no to create the compound phrase “Finnish studies.” This is done to simply mention studying, and the studying incidentally happens to be for Finnish. However, simply saying “I study…” involves using \emph{benkyō }勉強 as a verb in \emph{benkyō suru }勉強する. You will see that manner \emph{suru }-verbs can be rephrased to “ \emph{X no Y wo suru },” with Y standing for what makes up the \emph{suru }-verb. }
 
\begin{center}
\textbf{Other Meanings of \emph{suru }する } \hfill\break

\end{center}

\par{ Now that we\textquotesingle ve seen to some extent how \emph{suru }する can be used to mean “to do” and help nouns become verbs, it\textquotesingle s time to look at some of its other usages. It's important to note that not all usages will use \emph{wo }を. }
 
\par{\textbf{To Cost }: When used to mean “to cost,” \emph{suru }する can indicate the sticker price of something. }
 
\par{21. レンタルもおおよそ ${\overset{\textnormal{ごせん}}{\text{5000}}}$ ${\overset{\textnormal{えん}}{\text{円}}}$ します。 \hfill\break
 \emph{Rentaru mo ōyoso gosen\textquotesingle en shimasu. \hfill\break
 }Rental also costs about 5000 yen. }

\par{22. LINE スタンプは ${\overset{\textnormal{ひゃく}}{\text{100}}}$ ${\overset{\textnormal{えん}}{\text{円}}}$ します。 \hfill\break
 \emph{Rain sutampu wa hyakuen shimasu. \hfill\break
 }LINE stamps cost 100 yen. }

\par{23. このワインは ${\overset{\textnormal{に}}{\text{2}}}$ ${\overset{\textnormal{せんえん}}{\text{千円}}}$ もしません。 \hfill\break
 \emph{Kono wain wa nisen\textquotesingle en mo shimasen. \hfill\break
 }This wine doesn\textquotesingle t even cost 2000 yen. }
 
\par{ The verb \emph{kakaru }かかる is used to mean “to cost” as well, and it is used in contexts regarding costs beyond price tags on items. You could also just use the copula, \emph{da }だ\slash  \emph{desu }です, to discuss charges and fees. }
 
\par{24. ${\overset{\textnormal{かんぜい}}{\text{関税}}}$ は2 ${\overset{\textnormal{まんえん}}{\text{万円}}}$ かかりました。 \hfill\break
 \emph{Kanzei wa niman\textquotesingle en kakarimashita \hfill\break
 }The customs duties cost 20,000 yen. \hfill\break
 \hfill\break
25. その ${\overset{\textnormal{りょうきん}}{\text{料金}}}$ は ${\overset{\textnormal{ごひゃく}}{\text{500}}}$ ユーロです。 \hfill\break
 \emph{Sono ryōkin wa gohyaku-yūro desu. \hfill\break
 }That fee is 500 euros. \hfill\break
 \hfill\break
\textbf{To Wear (Accessories) }: To express “to wear,” you have a lot of options. You use \emph{suru }する for accessories. The other “to wear” verbs, though, are important to know. While we\textquotesingle re at it, we will learn how to say “to take off,” which is also somewhat complicated. }
 
\begin{ltabulary}{|P|P|P|}
\hline 
 
  Part of the Body 
 &   To Wear 
 &   To Take Off 
 \\ \cline{1-3} 
 
  The upper body\slash torso 
 &   \emph{Kiru }着る ( \emph{ichidan }) 
 &  \multirow{3}*{ \emph{ } \emph{Nugu }脱ぐ ( \emph{godan }) 
 }\\ \cline{1-2} 
 
  Legs 
 &   \emph{Haku }穿く ( \emph{godan }) 
  & \\ \cline{1-2} 
 
  Feet 
 &   \emph{Haku }履く( \emph{godan }) \emph{}
  & \\ \cline{1-3} 
 
  Head 
 &   \emph{Kaburu }被る ( \emph{godan }) 
 &   \emph{Nugu }脱ぐ\slash  \emph{Toru }取る ( \emph{godan }) 
 \\ \cline{1-3} 
 
  Hands\slash fingers 
 &   \emph{Hameru }はめる ( \emph{ichidan }) 
 &  \multirow{4}*{   \emph{Toru }取る\slash  \emph{Hazusu }外す ( \emph{godan }) 
   }\\ \cline{1-2} 
 
  General accessories 
 &   \emph{Suru }する\slash  \emph{Tsukeru }付ける ( \emph{ichidan }) 
  & \\ \cline{1-2} 
 
  Neckties, etc. 
 &   \emph{Shimeru }締める ( \emph{ichidan })\slash  \emph{Suru }する 
  & \\ \cline{1-2} 
 
  Glasses, etc. 
 &   \emph{Kakeru }かける ( \emph{ichidan }) 
  & \\ \cline{1-3} 
 
\end{ltabulary}
 
\par{ Consequently, some of these verbs overlap with each other and create variation among speakers depending on what item you\textquotesingle re wearing.  }

\par{26. ${\overset{\textnormal{てぶくろ}}{\text{手袋}}}$ を\{する・はめる・ ${\overset{\textnormal{つ}}{\text{付}}}$ ける\}。 \hfill\break
 \emph{Tebukuro wo [suru\slash hameru\slash tsukeru]. \hfill\break
 }To wear gloves. }
 
\par{27. サラリーマンは ${\overset{\textnormal{ふつう}}{\text{普通}}}$ 、ネクタイを\{します・ ${\overset{\textnormal{し}}{\text{締}}}$ めます\}。 \hfill\break
 \emph{Sarariiman wa futsū, nekutai wo [shimasu\slash shimemasu]. \hfill\break
 }Businessmen usually wear neckties. }
 
\par{28. ${\overset{\textnormal{くつ}}{\text{靴}}}$ を ${\overset{\textnormal{は}}{\text{履}}}$ く。 \hfill\break
 \emph{Kutsu wo haku. \hfill\break
 }To wear shoes. }
 
\par{29. ズボンを ${\overset{\textnormal{ぬ}}{\text{脱}}}$ ぐ。 \hfill\break
 \emph{Zubon wo nugu. \hfill\break
 }To take off one\textquotesingle s pants. }
 
\par{30. ${\overset{\textnormal{めがね}}{\text{眼鏡}}}$ はかけません。 \hfill\break
 \emph{Megane wa kakemasen. \hfill\break
 }I don\textquotesingle t wear glasses. }
 
\par{31. ${\overset{\textnormal{ぼうし}}{\text{帽子}}}$ を\{ ${\overset{\textnormal{と}}{\text{取}}}$ る・ ${\overset{\textnormal{ぬ}}{\text{脱}}}$ ぐ\}。 \hfill\break
 \emph{Bōshi wo [toru\slash nugu]. \hfill\break
 }To take off one\textquotesingle s hat. }
 
\par{32. ${\overset{\textnormal{てじょう}}{\text{手錠}}}$ を ${\overset{\textnormal{はず}}{\text{外}}}$ す。 \hfill\break
 \emph{Tejō wo hazusu. \hfill\break
 }To take off handcuffs. }
 
\par{\textbf{To be Sensed }: The sensing of natural phenomena can be expressed with \emph{suru }する. In this usage, the particle \emph{wo }を is not used. }
 
\par{33. ${\overset{\textnormal{つよ}}{\text{強}}}$ い ${\overset{\textnormal{にお}}{\text{匂}}}$ いがしました。 \hfill\break
 \emph{Tsuyoi nioi ga shimashita. \hfill\break
 }There was a strong scent. }
 
\par{34. ${\overset{\textnormal{あま}}{\text{甘}}}$ い ${\overset{\textnormal{あじ}}{\text{味}}}$ がしますね。 \hfill\break
 \emph{Amai aji ga shimasu ne. \hfill\break
 }It has a sweet taste, doesn\textquotesingle t it? }
 
\par{35. ${\overset{\textnormal{りょう}}{\text{量}}}$ が ${\overset{\textnormal{すく}}{\text{少}}}$ ない ${\overset{\textnormal{かん}}{\text{感}}}$ じがしました。 \hfill\break
 \emph{Ryō ga sukunai kanji ga shimashita. \hfill\break
 }I felt that the amount was lacking. }
 
\par{36. ${\overset{\textnormal{へん}}{\text{変}}}$ な ${\overset{\textnormal{おと}}{\text{音}}}$ がした。 \hfill\break
 \emph{Hen na oto ga shita. \hfill\break
 }There was a strange sound. }
 
\par{37. ${\overset{\textnormal{かな}}{\text{悲}}}$ しい ${\overset{\textnormal{きも}}{\text{気持}}}$ ちがする。 \hfill\break
 \emph{Kanashii kimochi ga suru. \hfill\break
 }To have sad feelings. }
 
\par{\textbf{To Be… (Occupation\slash Role) }: Whenever you want to say what you do as in what your job\slash role\slash occupation is, you typically use \emph{suru }する. However, to use this correctly, you need to use the ending ~ている. This is used to show that what you ‘do\textquotesingle  is an ongoing state. Although  we haven\textquotesingle t covered this grammar point yet, all you need to know now is to use “ \emph{…wo shite imasu }~をしています” in this situation for generic, polite conversations. }
 
\par{38. ${\overset{\textnormal{かていきょうし}}{\text{家庭教師}}}$ をしています。 \hfill\break
 \emph{Katei kyōshi wo shite imasu. \hfill\break
 }I am a private tutor. \hfill\break
 \hfill\break
39. ${\overset{\textnormal{りょうりにん}}{\text{料理人}}}$ をしています。 \hfill\break
 \emph{Ryōrinin wo shite imasu. \hfill\break
 }I am a cook. }
 
\par{40. ${\overset{\textnormal{ぎんこういん}}{\text{銀行員}}}$ をしています。 \hfill\break
 \emph{Ginkōin wo shite imasu. \hfill\break
 }I am a bank clerk. }
 
\par{\textbf{To Be\slash Have }: Another instance in which \emph{suru }する may function as “to be” is in the sense of taking a certain state or condition. Key phrases to remember for this include the following. }

\begin{ltabulary}{|P|P|P|P|}
\hline 

~形をする \hfill\break
\emph{\dothyp{}\dothyp{}\dothyp{}katachi wo suru }& To take the form of… & ~顔をする \hfill\break
 \emph{\dothyp{}\dothyp{}\dothyp{}kao wo suru }& To have…face \\ \cline{1-4}

~目付きをする \hfill\break
 \emph{\dothyp{}\dothyp{}\dothyp{}metsuki wo suru }& To have a…expression & ~目をする \hfill\break
 \emph{\dothyp{}\dothyp{}\dothyp{}me wo suru }& To have…eyes \\ \cline{1-4}

~格好をする \hfill\break
 \emph{\dothyp{}\dothyp{}\dothyp{}kakk }\emph{ō wo suru }& To have a…figure\slash appearance & ~振りをする \hfill\break
\emph{\dothyp{}\dothyp{}\dothyp{}furi wo suru }& To pretend to be… \\ \cline{1-4}

\end{ltabulary}

\par{ Similarly to the grammar point above, to use these phrases in truly functioning sentences, you\textquotesingle ll need to use - \emph{te imasu }~ています to indicate the above phrases as an “ongoing state.” For general purposes, when before nouns, \emph{suru }する  should be changed to \emph{shita }した. This is not the literal past tense, but we will revisit why this is grammatically so later in IMABI. }
 
\par{41. ${\overset{\textnormal{かれ}}{\text{彼}}}$ はラフな ${\overset{\textnormal{かっこう}}{\text{格好}}}$ をしていますね。 \hfill\break
\emph{Kare wa rafu na kakk }\emph{ō }\emph{wo shite imasu ne. }\hfill\break
He has a rough appearance, doesn\textquotesingle t he? }
 
\par{42. \{ ${\overset{\textnormal{まる}}{\text{丸}}}$ い・ ${\overset{\textnormal{しかく}}{\text{四角}}}$ い\} ${\overset{\textnormal{かたち}}{\text{形}}}$ をした ${\overset{\textnormal{たてもの}}{\text{建物}}}$ を ${\overset{\textnormal{こうちく}}{\text{構築}}}$ する。 \hfill\break
 \emph{[Marui\slash shikakui] katachi wo shita tatemono wo kōchiku suru. \hfill\break
 }To construct a [round\slash square] shaped building. }
 
\par{43. あの ${\overset{\textnormal{こ}}{\text{子}}}$ は ${\overset{\textnormal{かわい}}{\text{可愛}}}$ い ${\overset{\textnormal{かお}}{\text{顔}}}$ をしていますね。 \hfill\break
\emph{Ano ko wa kawaii kao wo shite imasu ne. }\hfill\break
That kid has a cute face, doesn\textquotesingle t he\slash she? }
 
\par{44. ${\overset{\textnormal{するど}}{\text{鋭}}}$ い ${\overset{\textnormal{めつ}}{\text{目付}}}$ きをする。 \hfill\break
\emph{Surudoi metsuki wo suru. }\hfill\break
To have a sharp expression. }
 
\par{45. ${\overset{\textnormal{つめ}}{\text{冷}}}$ たい ${\overset{\textnormal{め}}{\text{目}}}$ をする。 \hfill\break
\emph{Tsumetai me wo suru. }\hfill\break
To have cold eyes. }
 
\par{\textbf{To Play }: Although the word \emph{asobu }遊ぶ is often translated as "to play,” it is best translated as “to have a fun time.” As such, the act of playing some specific game or sport is expressed with \emph{suru }する. In fact, \emph{suru }する also encompasses doing general activities of any sort. In casual contexts, you may also use the verb \emph{yaru }やる for this meaning. }
 
\par{46. パチンコをしませんか。 \hfill\break
 \emph{Pachinko wo shimasen ka? \hfill\break
 }Why not play pachinko (Japanese pinball)? }
 
\par{47. ${\overset{\textnormal{い}}{\text{生}}}$ け ${\overset{\textnormal{ばな}}{\text{花}}}$ をします。 \hfill\break
 \emph{Ikebana wo shimasu. \hfill\break
 }I will practice flower arrangement. }
 
\par{ Since sports ( \emph{supōtsu }スポーツ) are used frequently with this meaning, it\textquotesingle s best to learn the Japanese words for some of the most common sports out there. }

\begin{ltabulary}{|P|P|P|P|}
\hline 

Baseball &  \emph{Yakyū }野球 & American football &  \emph{Amefuto }アメフト \\ \cline{1-4}

Soccer &  \emph{Sakkā }サッカー & Basketball &  \emph{Basuke(ttobōru) }バスケ(ットボール) \\ \cline{1-4}

Sumo &  \emph{S }\emph{umō }相撲 & Swimming &  \emph{Suiei }水泳 \\ \cline{1-4}

Martial arts &  \emph{Kakutōgi }格闘技 & Gymnastics &  \emph{Taisō }体操 \\ \cline{1-4}

Skiing &  \emph{Sukii }スキー & Golf &  \emph{Gorufu }ゴルフ \\ \cline{1-4}

Tennis &  \emph{Tenisu }テニス & Ping pong &  \emph{Takkyū }卓球 \\ \cline{1-4}

\end{ltabulary}

\par{48. アメフトを\{します・やります\}。 \hfill\break
\emph{Amefuto wo [shimasu\slash yarimasu]. \hfill\break
}I\textquotesingle ll play American football. }
 
\par{49. スキーはしません。 \hfill\break
 \emph{Sukii wa shimasen. \hfill\break
 }I don\textquotesingle t \emph{ }ski. }
 
\par{50. ${\overset{\textnormal{わたし}}{\text{私}}}$ は ${\overset{\textnormal{きのう}}{\text{昨日}}}$ 、サッカーをしました。 \hfill\break
 \emph{Watashi wa kinō, sakkā wo shimashita. \hfill\break
}I played soccer yesterday.  }
      
\section{Kuru 来る: To Come}
 
\par{ You may be wondering if \emph{kuru }来る will be as intricate as \emph{suru }する, but rest assured, it\textquotesingle s extremely straightforward. It means “to come” and is used in much the same way as in English. The only difference is that in its basic understanding as a direction verb, it refers to entities coming toward the speaker. Movement away from the speaker, regardless of the situation, is expressed with \emph{iku }行く (to go). }
 
\par{ Before we see example sentences, we need to know how to conjugate it and see just how irregular it really is. Don\textquotesingle t worry, though. It isn\textquotesingle t all that different from the other verbs. }
 
\begin{ltabulary}{|P|P|P|P|P|}
\hline 
 
  Plain Non-Past 
 &  \multicolumn{2}{|c|}{  \emph{Kuru }くる 
 }&   Polite   Non-Past 
 &    \emph{Kimasu }きます 
 \\ \cline{1-5} 
 
 \multicolumn{2}{|c|}{ Plain Past 
 }&    \emph{Kita }きた 
 &   Polite Past 
 &    \emph{Kimashita }きました 
 \\ \cline{1-5} 
 
 \multicolumn{2}{|c|}{ Plain Negative 
 }&    \emph{Konai }こない 
 &   Polite   Negative 
 &    \emph{Konai desu }こないです \hfill\break
 \emph{Kimasen }きません 
 \\ \cline{1-5} 
 
 \multicolumn{2}{|c|}{ Plain Neg. Past 
 }&    \emph{Konakatta }こなかった 
 &   Polite Neg. Past 
 &    \emph{Konakatta desu }こなかったです \hfill\break
 \emph{Kimasendeshita }きませんでした 
 \\ \cline{1-5} 
 
\end{ltabulary}
 
\par{ As you can see, the reason why it's irregular is because of what happens to the vowel after the initial \slash k\slash . Other than that, all the endings are the same as you\textquotesingle re used to. }
 
\par{51. ${\overset{\textnormal{わたし}}{\text{私}}}$ が ${\overset{\textnormal{ちゅうもん}}{\text{注文}}}$ したCDが ${\overset{\textnormal{き}}{\text{来}}}$ ました。 \hfill\break
 \emph{Watashi ga chūmon shita shiidii ga kimashita. \hfill\break
 }The CD(s) I ordered have arrived\slash come. }
 
\par{52. ${\overset{\textnormal{けさ}}{\text{今朝}}}$ サンタさんが ${\overset{\textnormal{こ}}{\text{来}}}$ なかった。 \hfill\break
 \emph{Kesa Santa-san ga konakatta. \hfill\break
 }Santa didn\textquotesingle t come this morning. }
 
\par{53. ついに ${\overset{\textnormal{はる}}{\text{春}}}$ が ${\overset{\textnormal{き}}{\text{来}}}$ た! \hfill\break
 \emph{Tsui ni haru ga kita! \hfill\break
 }Spring has finally arrived\slash come! }
 
\par{54. ${\overset{\textnormal{でんしゃ}}{\text{電車}}}$ 、 ${\overset{\textnormal{こ}}{\text{来}}}$ ないよ。 \hfill\break
 \emph{Densha, konai yo. \hfill\break
 }The train won\textquotesingle t come\slash hasn\textquotesingle t come! }
 
\par{55. ${\overset{\textnormal{く}}{\text{来}}}$ る ${\overset{\textnormal{ひ}}{\text{日}}}$ も ${\overset{\textnormal{く}}{\text{来}}}$ る ${\overset{\textnormal{ひ}}{\text{日}}}$ も ${\overset{\textnormal{ゆき}}{\text{雪}}}$ だった。 \hfill\break
 \emph{Kuru hi mo kuru hi mo yuki datta. \hfill\break
 }It was snow day after day. }
 
\par{56. ${\overset{\textnormal{れんらく}}{\text{連絡}}}$ が ${\overset{\textnormal{き}}{\text{来}}}$ ませんでした。 \hfill\break
 \emph{Renraku ga kimasendeshita. \hfill\break
 }No contact came. }
 
\par{57. ${\overset{\textnormal{てがみ}}{\text{手紙}}}$ が ${\overset{\textnormal{こ}}{\text{来}}}$ なかったです。 \hfill\break
 \emph{Tegami ga konakatta desu. \hfill\break
 }A letter didn\textquotesingle t come. }
 
\par{58. ${\overset{\textnormal{あらし}}{\text{嵐}}}$ が ${\overset{\textnormal{き}}{\text{来}}}$ たぞ! \hfill\break
 \emph{Arashi ga kita zo! \hfill\break
 }The storm\textquotesingle s here! }
 
\par{\textbf{Particle Note }: Using \emph{zo }ぞ at the end of the sentence is used in this casual context to draw attention to the situation at hand. }
 
\par{59. ${\overset{\textnormal{へんじ}}{\text{返事}}}$ も ${\overset{\textnormal{こ}}{\text{来}}}$ なかった。 \hfill\break
 \emph{Henji mo konakatta. \hfill\break
 }Not even a reply has come. }
 
\par{60. やっとこの ${\overset{\textnormal{ひ}}{\text{日}}}$ が ${\overset{\textnormal{き}}{\text{来}}}$ ました! \hfill\break
 \emph{Yatto kono hi ga kimashita! \hfill\break
 }This day has arrived\slash come at last! }

\par{${\overset{\textnormal{りょう}}{\text{寮}}}$ に ${\overset{\textnormal{かえ}}{\text{帰}}}$ りませんか。 }

\par{Why don't we go back to the dorms? }
    