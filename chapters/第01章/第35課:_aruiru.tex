    
\chapter{ある \& いる}

\begin{center}
\begin{Large}
第35課: ある \& いる 
\end{Large}
\end{center}
 
\par{ ある and いる contrast each other in many ways. There are some important exceptions to conjugation that you need to keep in mind. }

\begin{ltabulary}{|P|P|P|P|P|}
\hline 

Verb & Class & Plain Negative & Polite Negative & Plain Negative Past \\ \cline{1-5}

ある & 五段 & ない & ありません & なかった \\ \cline{1-5}

いる & 一段 & いない & いません & いなかった \\ \cline{1-5}

\end{ltabulary}

\par{\textbf{Grammar Note }: The negative of ある is \textbf{ない }, not あらない. This is not the only anomaly in regards to conjugation for these two verbs, but for now, just keep this in mind. }
      
\section{ある}
 
\par{ Both ある and いる show that \textbf{there is something }. They may or may not be interchangeable depending on their use. Even when you can choose between the two, one will \textbf{always }be more prevalent\slash natural than the other. ある normally shows \textbf{existence of things besides people and animals }. ${\overset{\textnormal{そんざい}}{\text{存在}}}$ する is more appropriate, especially with non-physical items, for the explicit meaning of "to exist." }
 
\par{\textbf{Particle Note }: に, not で, should be used with existential verbs. で shows location of action, not state. Existing in a place is a state of being. Furthermore, をある and をいる are always wrong. }
 
\begin{center}
\textbf{Examples } 
\end{center}

\par{1. そこに ${\overset{\textnormal{さいふ}}{\text{財布}}}$ がありました。 \hfill\break
There was a wallet there. }
 
\par{${\overset{\textnormal{}}{\text{2. 高}}}$ い ${\overset{\textnormal{}}{\text{木}}}$ があります。 \hfill\break
There is a tall tree. }
 
\par{${\overset{\textnormal{}}{\text{3. 高}}}$ い ${\overset{\textnormal{}}{\text{木}}}$ が ${\overset{\textnormal{}}{\text{公園}}}$ にたくさんあります。 \hfill\break
There are a lot of tall trees in the park. }

\par{4. ${\overset{\textnormal{こうしゅうでんわ}}{\text{公衆電話}}}$ はありますか。 \hfill\break
Is there a public telephone? }

\par{${\overset{\textnormal{とかい}}{\text{5. 都会}}}$ に ${\overset{\textnormal{びょういん}}{\text{病院}}}$ がある。 \hfill\break
There is a hospital in the city. }

\par{6. ${\overset{\textnormal{ほんしゃ}}{\text{本社}}}$ は ${\overset{\textnormal{}}{\text{東京}}}$ にあります。 \hfill\break
The main office is in Tokyo. }

\par{7. ${\overset{\textnormal{しんせいひん}}{\text{新製品}}}$ は ${\overset{\textnormal{しさくだんかい}}{\text{試作段階}}}$ にない。 \hfill\break
The new product is not in the trial stage. }

\par{8. ${\overset{\textnormal{りょうこくかん}}{\text{両国間}}}$ には ${\overset{\textnormal{}}{\text{国交}}}$ がありません。      (Not Spoken Language) \hfill\break
There isn't any diplomatic relations between both of the nations. }

\par{${\overset{\textnormal{}}{\text{9. 日本}}}$ では ${\overset{\textnormal{こうさてん}}{\text{交差点}}}$ に ${\overset{\textnormal{こうばん}}{\text{交番}}}$ があります。 \hfill\break
There are police stations at intersections in Japan. }
 
\par{${\overset{\textnormal{}}{\text{10. 名古屋}}}$ には ${\overset{\textnormal{なごやじょう}}{\text{名古屋城}}}$ がある。 \hfill\break
In Nagoya, there is the Nagoya Castle (but not anywhere else). }
 
\par{\textbf{Culture Note }: The Nagoya Castle was one of the most important stops on the Minoji 美濃路 Highway, which was a major roadway in the Edo Period. The castle still exists is and is still very significant. }
 
\par{11. あの ${\overset{\textnormal{}}{\text{車}}}$ には ${\overset{\textnormal{がんじょう}}{\text{頑丈}}}$ な ${\overset{\textnormal{と}}{\text{取}}}$ っ ${\overset{\textnormal{て}}{\text{手}}}$ があるんじゃないか。 \hfill\break
Isn't there a firm handle in that car? }
 
\begin{center}
\textbf{Possession and Occurrence } 
\end{center}

\par{ ある may also show the \textbf{possession of inanimate non-living and non-physical things }. This usage may also show \emph{physical }attributes of people and things. It can even mean "to occur". Depending on the usage of "to occur", there are alternative ways to say it. ${\overset{\textnormal{はっせい}}{\text{発生}}}$ する is "to occur" as in a break out of some kind. ${\overset{\textnormal{お}}{\text{起}}}$ こる means "to occur" as in some event. For this usage で can be used because "to occur" is not existential. }

\par{12. ${\overset{\textnormal{きん}}{\text{金}}}$ も ${\overset{\textnormal{じかん}}{\text{時間}}}$ もたっぷりある ${\overset{\textnormal{}}{\text{王}}}$ \hfill\break
A king ample in both gold and time }

\par{13. ${\overset{\textnormal{す}}{\text{好}}}$ き ${\overset{\textnormal{きら}}{\text{嫌}}}$ いがある。 \hfill\break
To have one's likes and dislikes. }

\par{${\overset{\textnormal{}}{\text{14. 彼女}}}$ の ${\overset{\textnormal{ぶんしょう}}{\text{文章}}}$ にはユーモアがある。 \hfill\break
There is humor in her composition. }

\par{${\overset{\textnormal{}}{\text{15. 奈良}}}$ で ${\overset{\textnormal{じしん}}{\text{地震}}}$ がありました。 \hfill\break
There was an earthquake in Nara. }

\par{16a. 事故があった。 \hfill\break
16b. 事故に遭った。 \hfill\break
16a. There was an accident. \hfill\break
16b. I was in an accident. }

\par{17. ${\overset{\textnormal{はっぴょう}}{\text{発表}}}$ があった。 \hfill\break
There was an announcement. }

\par{18. ${\overset{\textnormal{なや}}{\text{悩}}}$ みがある。 \hfill\break
To have worries. }

\par{19. その ${\overset{\textnormal{どうぶつ}}{\text{動物}}}$ は ${\overset{\textnormal{たいじゅう}}{\text{体重}}}$ が200キロもある。 \hfill\break
The animal has a weight of at least 200 kilograms. }

\par{20. ${\overset{\textnormal{たが}}{\text{互}}}$ いに ${\overset{\textnormal{めんしき}}{\text{面識}}}$ がある。 \hfill\break
To have an acquaintance with each other. }

\par{21. ${\overset{\textnormal{きょうよう}}{\text{教養}}}$ がある。 \hfill\break
To have an education. }
 
\par{22. もう ${\overset{\textnormal{に}}{\text{逃}}}$ げ ${\overset{\textnormal{ば}}{\text{場}}}$ はありません。 \hfill\break
There is nowhere to hide. }
 
\par{${\overset{\textnormal{}}{\text{23. 高}}}$ さは ${\overset{\textnormal{}}{\text{六十}}}$ メートルある。 \hfill\break
The height is 60 meters. }

\par{24. ${\overset{\textnormal{ざいさん}}{\text{財産}}}$ はありますか。 \hfill\break
Do you have assets? }

\par{${\overset{\textnormal{ふろづ}}{\text{25. 風呂付}}}$ きの ${\overset{\textnormal{へや}}{\text{部屋}}}$ はありますか。 \hfill\break
Do you have a room with a bath? }
 
\par{${\overset{\textnormal{}}{\text{26a. 私}}}$ には、 ${\overset{\textnormal{}}{\text{強}}}$ い ${\overset{\textnormal{みかた}}{\text{味方}}}$ がある。? \hfill\break
 ${\overset{\textnormal{}}{\text{26b. 私}}}$ には、 ${\overset{\textnormal{}}{\text{強}}}$ い ${\overset{\textnormal{}}{\text{味方}}}$ がいる。 \hfill\break
As for me, I have a strong ally. }
 
\par{\textbf{Usage Note }: This ある would most certainly \textbf{almost always be replaced with いる today }. }

\par{27. ${\overset{\textnormal{ぎゅうにゅう}}{\text{牛乳}}}$ はありますか。 \hfill\break
Do you have milk? }

\par{28. ${\overset{\textnormal{じこくひょう}}{\text{時刻表}}}$ はありますか。 \hfill\break
Do you have a timetable? }
 
\par{\textbf{Other Usages }}
 
\par{\textbf{Old Usages }: ある may be used in the introduction of characters, but \textbf{いる is predominant }in this case. More so literary, ある may show that someone is in a certain position\slash role. Again, いる is more common here. It may also be used in this manner with things, in which case いる would not be applicable. }

\par{29a. ${\overset{\textnormal{おおむかし}}{\text{大昔}}}$ 、ある ${\overset{\textnormal{ところ}}{\text{所}}}$ におじいさんとおばあさんがありました。? \hfill\break
${\overset{\textnormal{}}{\text{29b. 大昔}}}$ 、ある ${\overset{\textnormal{}}{\text{所}}}$ におじいさんとおばあさんがいました。 \hfill\break
${\overset{\textnormal{}}{\text{30c. 大昔}}}$ 、ある ${\overset{\textnormal{}}{\text{所}}}$ におじいさんとおばあさんがおりました。 \hfill\break
Long time ago, there was an old man and an old woman. }
\textbf{Usage Note }: This usage is no longer used in the spoken language. This nostalgic line is now normally stated with either いました or おりました (a dialectal and or humble version of いる) instead of ありました。 
\par{ The last old usage that we will look at is とある. This is not used in the spoken language. However, you can find it in the written language a lot in contexts such as in Ex. 30. }

\par{30. ${\overset{\textnormal{せいしょ}}{\text{聖書}}}$ には「 ${\overset{\textnormal{はじ}}{\text{初}}}$ めに ${\overset{\textnormal{かみ}}{\text{神}}}$ は ${\overset{\textnormal{てん}}{\text{天}}}$ と ${\overset{\textnormal{ち}}{\text{地}}}$ とを ${\overset{\textnormal{そうぞう}}{\text{創造}}}$ された」とある。 \hfill\break
It is written in the Bible that, "In the beginning, God created the heavens and the earth." }
      
\section{いる}
 
\par{\textbf{ いる }shows that an \textbf{animate and alive }thing exists in a certain location. It may also show the possession of something animate and alive. }

\par{31. ${\overset{\textnormal{さとう}}{\text{佐東}}}$ さんは ${\overset{\textnormal{いえ}}{\text{家}}}$ にいませんでした。 \hfill\break
Mr. Sato was not home. }
 
\par{${\overset{\textnormal{}}{\text{32. 犬}}}$ がいません。 \hfill\break
I don't have a dog (since it left somewhere). }
 
\par{\textbf{Nuance Note }: Using が here makes it sound like either there isn't a dog, and in the case of ownership, something happened to the dog, like it ran away, was stolen, or died. If you were to say 犬はいません, you would be able to say I don't have a dog, but it sounds like you may own something else like a cat. You could say 犬を ${\overset{\textnormal{か}}{\text{飼}}}$ っていません to avoid this extra nuance. 飼う means "to raise an animal." You could just also naturally respond with いません or 飼っていません. }

\par{33. ${\overset{\textnormal{わ}}{\text{我}}}$ が ${\overset{\textnormal{や}}{\text{家}}}$ には ${\overset{\textnormal{}}{\text{鳥}}}$ が ${\overset{\textnormal{さんば}}{\text{三羽}}}$ いる。 \hfill\break
We have three birds in our house. }

\par{34a. ${\overset{\textnormal{よい}}{\text{宵}}}$ の ${\overset{\textnormal{みょうじょう}}{\text{明星}}}$ がいる。? \hfill\break
${\overset{\textnormal{}}{\text{34b. 宵}}}$ の ${\overset{\textnormal{}}{\text{明星}}}$ が ${\overset{\textnormal{}}{\text{出}}}$ ています。〇 \hfill\break
${\overset{\textnormal{}}{\text{のがえます。〇}}}$ \hfill\break
There is the evening Venus. }
 
\par{\textbf{Sentence Note }: Ex. 34a is an example of いる in literary language that wouldn't be used in the spoken language. Originally, this usage was meant to show something that normally moves but is in fact not moving. }

\begin{center}
\textbf{ある VS いる Troubles }
\end{center}

\par{ In reality, natives don't all agree with each other on case by case instances of ある VS いる. What do you use with words like 家族, バス, エレベーター, 人形, or a dead cat? None of these example nouns are necessarily straightforward. }

\par{35. 家族が\{いる・ある\}。 \hfill\break
I have a family. }

\par{\textbf{Sentence Note }: More speakers would choose いる. However, ある is not strange at all. }

\par{36. 神が\{ある・いる\}。 \hfill\break
There is God\slash god\slash kami. }

\par{\textbf{Sentence Note }: Ignoring the various interpretations of the word 神, describing such a spiritual force's existence with いる is most common today, but ある does get used. It may sound less religious, stiff, or old-fashioned, but this Japanese does exist. }

\par{37. 私(に)は ${\overset{\textnormal{きょうだい}}{\text{兄弟}}}$ が3 ${\overset{\textnormal{にん}}{\text{人}}}$ \{ (△・X) ある・〇 いる\}。 \hfill\break
I have three siblings. }

\par{\textbf{Sentence Note }: Using ある for people causes heart attacks for Japanese learners, but the reality is that there are speakers who find this instance of ある perfectly fine, though even these individuals will admit that it is old-fashioned. }

\par{38. 私の2人の兄弟は神奈川に\{〇 います・ X あります\}。 \hfill\break
My two siblings are in Kanagawa. \hfill\break
\hfill\break
39. ${\overset{\textnormal{にんぎょう}}{\text{人形}}}$ がいっぱい \{いる・ある\}ね。 \hfill\break
There are a lot of dolls, aren't there. }

\par{\textbf{Sentence Note }: いる would most especially be used when in contact or in the vicinity of the dolls, but it's important to know that speakers use both. The decision hinges on how human-like you wish to view dolls. }

\par{40. あそこにロボットが\{います・あります\}。 \hfill\break
There is a robot over there. }

\par{\textbf{Sentence Note }: If you use いる, the robot is in service and is truly human-like. If you use ある, it's in the same position as a regular inanimate object. }

\par{41. あそこにバスが\{〇・△ いる・〇 ある・〇 止まっている\}。 \hfill\break
There is a bus over there. }

\par{\textbf{Sentence Note }: いる has the nuance of that it is in service, but some speakers still think that this is wrong. Other speakers think that this is even OK for elevators and what not in service, but others still disagree and believe you should use 止まっている when its stopped there indefinitely. If you wanted to show it's just stopped there, the particle に should be changed to で.  If you use ある, you would simply state that there is a bus\slash elevator, and it would most certainly mean it's not in service. \hfill\break
\hfill\break
42. ${\overset{\textnormal{と}}{\text{捕}}}$ らわれの ${\overset{\textnormal{み}}{\text{身}}}$ \{〇 にある・X でいる\}。 \hfill\break
To be in captivity. }

\par{\textbf{Sentence Note }: This is a set phrase, but the grammar is also a little tricky because にある is actually the original copula. It shows up here as set phrases tend to hold onto old grammar. However, when we try replacing it with grammar we've learned today, we see that the result is ungrammatical. }

\begin{center}
 \textbf{The Dead: いる or ある? }
\end{center}

\par{ What if something is dead? Certainly, when you use words such as 死体 or 遺体 which mean corpse, you use ある. When discussing the existence of dead people, いる is overwhelmingly used. However, consider this counterexample. }

\par{43. ピーナッツアレルギーで死んだ人が\{いました・ありました\}。 \hfill\break
There were people who died from peanut allergies. }

\par{ Many speakers would not like ありました. Those that do, though, would say it is rarer, but it is more emphatic and focusing on the severity of the matter than いました. What about dogs and cats? Generally, people would hate using ある for dead pets. If you really hate pet animals, you could use ある. This would imply you don't value them as much. \hfill\break
\hfill\break
44. ${\overset{\textnormal{どく}}{\text{毒}}}$ ${\overset{\textnormal{い}}{\text{入}}}$ りのものを食べて、 ${\overset{\textnormal{し}}{\text{死}}}$ んだ ${\overset{\textnormal{ねこ}}{\text{猫}}}$ が \{〇 いた・ △ あった\}。 \hfill\break
There was a dead cat that ate a poison-laced item. }

\par{ What about a dead fly? The phrase 死んだハエ is possible. Most would still say いる is OK and ある is not so OK, but saying 死んだハエがいる is not practical. Thus, some say that a sentence like below would be more practical. }

\par{45. コップの中に死んだハエが入っている。 \hfill\break
There is a dead fly in my cup. }

\par{ Another word to consider is けが人 (injured). It turns out that even if the noun is a person noun, if the concept is abstract, it can still take ある. Or, if there is any wavering of whether something exists or not, ある more easily appears. }

\par{46. 乗客の中にけが人は\{ありません 〇\slash X・いません〇\}でした。 \hfill\break
There were no people injured among the passengers. }
    