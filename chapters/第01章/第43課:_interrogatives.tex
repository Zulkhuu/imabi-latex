    
\chapter{Interrogatives}

\begin{center}
\begin{Large}
第43課: Interrogatives: Who, What, When, Where, \& Why? 
\end{Large}
\end{center}
 
\par{ In English, we create questions with what are colloquially called "wh-words." Although there are quite a few of these words in English, it's customary to view "who," "what," "when," "where," and "why" as the basic five. }

\par{ In grammar, these words are called "interrogatives." These words in English, though, have two fundamental kinds of usage. The following examples demonstrate this with the words "when" and "where." }

\par{i. When did he go home? \hfill\break
ii. I like it when you dance. \hfill\break
iii. Where do you live. \hfill\break
iv. The place where I bought my dog is really nice. }

\par{ In Japanese, interrogatives are called \emph{gimonshi }疑問詞. This word literally means "part of speech for questioning." As this name implies, the meanings of these words revolve around questioning. Although each individual word may have a variety of usages, unlike English, they will always be used in some sense in questions\slash describing uncertainty. }

\par{ In English, there is a grammatical rule that interrogatives go at the start of a sentence and\slash or clause. In Japanese, there is no such rule. To put this in perspective, consider the following examples. }

\par{v. Who went with you to the park? \hfill\break
vi. Last night when I came home from work, [who was it to that you were talking on the phone with]? }

\par{ In v. and vi., "who" must be at the start of their respective independent clauses. However, this as you will see in this lesson, is not the case in Japanese. As we learn about the rules of Japanese interrogatives, try to superimpose your understanding of English interrogatives on to them. }

\par{ In this lesson, we will learn about the basic question words of Japanese. To simplify things, we will only look at polite speech. This is because there is a lot of variation regarding speech styles. Due to the emphasis placed on politeness in Japanese, questioning is something that you have to go about with caution. Yet, basic personal questions, unlike in Western culture, are deemed natural and key to starting conversations as icebreakers. }

\par{ As a Japanese learner, being able to ask questions is especially important. Otherwise, you'll lose the ability to figure out how and when questions are formed. As the Japanese learner, however, there is a general understanding that you will be asking all sorts of questions, similar to children as they are still trying to grasp what things mean. Use this expectation to your advantage as it will certainly speed up the language learning process. }
      
\section{Vocabulary List}
 
\par{\textbf{Nouns }}

\par{・疑問詞 \emph{Gimonshi }– Interrogatives }

\par{・社長 \emph{Shach }\emph{ō }– Company president }

\par{・妻 \emph{Tsuma }– Wife }

\par{・科学者 \emph{Kagakusha }- Scientist }

\par{・担当者 \emph{Tant }\emph{ōsha }– Manager\slash person in charge }

\par{・お土産 \emph{Omiyage }- Souvenir }

\par{・履き物 \emph{Hakimono }– Footwear }

\par{・日 \emph{Hi }– Day\slash sun }

\par{・仕事 \emph{Shigoto }– Job\slash work }

\par{・名前 \emph{Namae }– Name }

\par{・趣味 \emph{Shumi }– Hobby }

\par{・エアコン \emph{Eakon }– Air-conditioner }

\par{・住宅 \emph{J }\emph{ūtaku }– Residence\slash housing }

\par{・時期 \emph{Jiki }– Time\slash period }

\par{・請求 \emph{Seiky }\emph{ū }– Billing }

\par{・雛人形 \emph{Hina-ningy }\emph{ō }– Hina doll }

\par{・サプリ(メント) \emph{Sapuri(mento) }– Supplement }

\par{・タイミング \emph{Taimingu }- Timing }

\par{・苺 \emph{Ichigo }- Strawberry }

\par{・携帯電話 \emph{Keitai denwa }– Cellphone }

\par{・地震 \emph{Jishin }– Earthquake }

\par{・速報 \emph{Sokuh }\emph{ō }– News flash\slash bulletin }

\par{・復興 \emph{Fukk }\emph{ō }– Reconstruction\slash restoration }

\par{・電池 \emph{Denchi }– Battery }

\par{・数学 \emph{S }\emph{ūgaku }- Math }

\par{・住所 \emph{J }\emph{ūsho }– Address }

\par{・トイレ \emph{Toire }– Toilet\slash bathroom }

\par{・お手洗い \emph{Otearai }– Restroom\slash bathroom\slash lavatory }

\par{・街 \emph{Machi }– Town }

\par{・コンビニ \emph{Kombini }– Convenience store }

\par{・お握り \emph{Onigiri }– Rice ball }

\par{・飛行機 \emph{Hik }\emph{ōki }– (Air)plane }

\par{・座席 \emph{Zaseki }- Seat }

\par{・郵便局 \emph{Y }\emph{ūbinkyoku }– Post office }

\par{・牛 \emph{Ushi }– Cow }

\par{・乳 \emph{Chichi }– Milk (from mammalian breasts)\slash breasts }

\par{・火山 \emph{Kazan }– Volcano }

\par{・名詞 \emph{Meishi }– Noun }

\par{・人々 \emph{Hitobito }– People }

\par{・神 \emph{Kami }– god\slash deity }

\par{\textbf{Proper Nouns }}

\par{・マレーシア \emph{Mar }\emph{ēshia }– Malaysia }

\par{・スミスさん \emph{Sumisu-san }Mr. Smith }

\par{・日本 \emph{Nihon\slash Nippon }– Japan }

\par{\textbf{\emph{(ru) Ichidan }Verbs }}

\par{・似る \emph{Niru }– To resemble  (intr.) }

\par{・出る \emph{Deru }– To exit\slash come out\slash leave\slash appear (intr.) }

\par{\textbf{\emph{suru }Verbs }}

\par{・発明する \emph{Hatsumei suru }– To invent (trans.) }

\par{・噴火する  \emph{Funka suru } – To erupt (intr.) }
  \textbf{Adjectives }
\par{・いい \emph{Ii }– Good\slash nice }

\par{・安い \emph{Yasui }– Cheap }

\par{・美味しい \emph{Oishii }– Delicious }

\par{\textbf{Adjectival Nouns }}

\par{・好き\{な\} \emph{Suki [na] }– To like }

\par{・一緒\{の\} \emph{Issho [no] }– Together }

\par{・有利\{な\} \emph{Y }\emph{ūri [na] }– Advantageous\slash lucrative }

\par{・ベスト\{の・な\} \emph{Besuto [no\slash na] }- Best }

\par{・緊急\{の・な Demonstratives }

\par{・この \emph{Kono }– This (adj.) }

\par{\textbf{Question Words }}

\par{・誰 \emph{Dare }– Who? }

\par{・何 \emph{Nani\slash nan }– What? }

\par{・いつ \emph{Itsu }– When? }

\par{・どこ \emph{Doko }– Where? }

\par{・何故 \emph{Naze }– Why? }

\par{\textbf{Number Phrases }}

\par{・2週間 \emph{Nish }\emph{ūkan }– Two weeks }

\par{\textbf{Prefixes }}

\par{・お~ \emph{O- }– Honorific prefix }

\par{\textbf{Suffixes }}

\par{・~頃 \emph{-goro }– Around }

\par{・~後 \emph{-go }– After… }

\par{\textbf{Interjections }}

\par{・え  \emph{E } – Eh?\slash What }

\par{\textbf{Adverbs }}

\par{・一番 \emph{Ichiban }– Best\slash most }

\par{・今日 \emph{Ky }\emph{ō }– Today }

\par{・今 \emph{Ima }– Now }

\par{・毎日  \emph{Mainichi }– Every day ? \} \emph{Kinky }\emph{ū [no\slash na] }– Urgent }

\par{\textbf{\emph{(u) Godan }Verbs }}

\par{・住む \emph{Sumu }– To live  (intr.) }

\par{・思う \emph{Omou }– To think  (trans.) }

\par{・成る \emph{Naru }– To become\slash come into fruition (intr.) }

\par{・生る \emph{Naru }– To ripen  (intr.) }

\par{・鳴る \emph{Naru }– To sound\slash ring (intr.) }

\par{・変わる \emph{Kawaru }– To change (intr.) }

\par{・買う \emph{Kau }– To buy (trans.) }

\par{・飾る \emph{Kazaru }– To decorate (trans.) }

\par{・仕上がる \emph{Shiagaru }– To be finished (intr.) }

\par{・飲む \emph{Nomu }– To drink\slash swallow\slash take (medicine) (trans.) }

\par{・空く \emph{Aku }– To be open\slash empty\slash hungry (intr.) }

\par{・壊す \emph{Kowasu }– To break (trans.) }

\par{・行く \emph{Iku }– To go (intr.) }

\par{・祈る \emph{Inoru }– To pray (trans.) }

\par{・太る \emph{Futoru }– To gain weight (intr.) }
      
\section{Interrogatives in Polite Speech}
 
\par{ The first interrogatives that we will be going over are "who," "what," "when," "where," and "why." In Japanese, these words can appear in more places in the sentence than in English. This difference happens to affect various factors in how they're used. In the following chart, take note of the differences between interrogatives used at the start of a sentence and interrogatives used at the end of a sentence. }

\begin{ltabulary}{|P|P|P|}
\hline 

Interrogative & Start of a Sentence & End of a Sentence \\ \cline{1-3}

Who &  \emph{Dare ga\dothyp{}\dothyp{}\dothyp{}desu ka? }誰が…ですか? &  \emph{\dothyp{}\dothyp{}\dothyp{}wa dare desu ka? }…は誰ですか? \\ \cline{1-3}

What &  \emph{Nani ga\dothyp{}\dothyp{}\dothyp{}desu ka? }何が\dothyp{}\dothyp{}\dothyp{}ですか? &  \emph{\dothyp{}\dothyp{}\dothyp{}wa nan desu ka? }\dothyp{}\dothyp{}\dothyp{}は何ですか? \\ \cline{1-3}

When &  \emph{Itsu (ga)\dothyp{}\dothyp{}\dothyp{}desu ka? }いつ(が)\dothyp{}\dothyp{}\dothyp{}ですか? &  \emph{\dothyp{}\dothyp{}\dothyp{}wa itsu desu ka? }\dothyp{}\dothyp{}\dothyp{}はいつですか? \\ \cline{1-3}

Where &  \emph{Doko ga\dothyp{}\dothyp{}\dothyp{}desu ka? }どこが\dothyp{}\dothyp{}\dothyp{}ですか? &  \emph{\dothyp{}\dothyp{}\dothyp{}wa doko desu ka? }\dothyp{}\dothyp{}\dothyp{}はどこですか? \\ \cline{1-3}

Why &  \emph{Naze\dothyp{}\dothyp{}\dothyp{}no\slash n desu ka? }何故\dothyp{}\dothyp{}\dothyp{}\{の・ん\}ですか? &  \emph{\dothyp{}\dothyp{}\dothyp{}(no) wa naze desu ka? }\dothyp{}\dothyp{}\dothyp{}(の)は何故ですか? \\ \cline{1-3}

\end{ltabulary}

\par{ The placement of interrogatives is determined by how much emphasis you are placing on the individual words in the sentence. The closer to the front of a sentence something is, the more emphasis is placed on it. This principle helps determine the nuances in the examples below designed to give you a thorough range of grammatical complexity that comes about simply from the patterns above. }

\par{1a. ${\overset{\textnormal{だれ}}{\text{誰}}}$ が ${\overset{\textnormal{しゃちょう}}{\text{社長}}}$ ですか。 \hfill\break
\emph{Dare ga shach }\emph{ō desu ka? \hfill\break
}1b. ${\overset{\textnormal{しゃちょう}}{\text{社長}}}$ は ${\overset{\textnormal{だれ}}{\text{誰}}}$ ですか。 \hfill\break
\emph{Shach }\emph{ō wa dare desu ka? \hfill\break
}Who is the company president? }
 
\par{\textbf{Sentence Note }: Ex. 1a. would only be used in the sense of “ \emph{who }is the company president?” It would only be used when asking about who the company president is in a conversation where said person would not be present. This is unlike 1b. which could be used to ask who the company president is out of a group of people visible\slash near the speaker and listener(s). }
 
\par{2. ${\overset{\textnormal{だれ}}{\text{誰}}}$ が ${\overset{\textnormal{す}}{\text{好}}}$ きですか? \hfill\break
 \emph{Dare ga suki desu ka? \hfill\break
 }Who do you like? }
 
\par{3. ${\overset{\textnormal{つま}}{\text{妻}}}$ は ${\overset{\textnormal{だれ}}{\text{誰}}}$ に ${\overset{\textnormal{に}}{\text{似}}}$ てると ${\overset{\textnormal{おも}}{\text{思}}}$ いますか? \hfill\break
\emph{Tsuma wa dare ni niteru to omoimasu ka? \hfill\break
 }Who do you think my wife looks like? }
 
\par{4. ${\overset{\textnormal{だれ}}{\text{誰}}}$ と ${\overset{\textnormal{いっしょ}}{\text{一緒}}}$ に ${\overset{\textnormal{す}}{\text{住}}}$ んでいますか。 \hfill\break
 \emph{Dare to issho ni sunde imasu ka? \hfill\break
 }Who do you live together with? }
 
\par{5. ${\overset{\textnormal{いちばんす}}{\text{一番好}}}$ きな ${\overset{\textnormal{かがくしゃ}}{\text{科学者}}}$ は ${\overset{\textnormal{だれ}}{\text{誰}}}$ ですか。 \hfill\break
 \emph{Ichiban suki na kagakusha wa dare desu ka? \hfill\break
 }Who is your favorite scientist? }
 
\par{6. ${\overset{\textnormal{たんとうしゃ}}{\text{担当者}}}$ は ${\overset{\textnormal{だれ}}{\text{誰}}}$ になりますか。 \hfill\break
 \emph{Tant }\emph{ōsha wa wa dare ni narimasu ka? \hfill\break
 }Who will be(come) the manager? }
 
\par{7. ${\overset{\textnormal{なに}}{\text{何}}}$ が ${\overset{\textnormal{か}}{\text{変}}}$ わったのですか。 \hfill\break
 \emph{Nani ga kawatta no desu ka? \hfill\break
 }What has changed? }
 
\par{8. お ${\overset{\textnormal{みやげ}}{\text{土産}}}$ は ${\overset{\textnormal{なに}}{\text{何}}}$ がいいですか。 \hfill\break
 \emph{Omiyage wa nani ga ii desu ka? \hfill\break
 }What would be good for souvenirs? }
 
\par{9. ${\overset{\textnormal{は}}{\text{履}}}$ き ${\overset{\textnormal{もの}}{\text{物}}}$ は ${\overset{\textnormal{なに}}{\text{何}}}$ がいいですか。 \hfill\break
 \emph{Hakimono wa nani ga ii desu ka? \hfill\break
 }What would be good for footwear? }
 
\par{10. ${\overset{\textnormal{きょう}}{\text{今日}}}$ は ${\overset{\textnormal{なん}}{\text{何}}}$ の ${\overset{\textnormal{ひ}}{\text{日}}}$ ですか。 \hfill\break
 \emph{Ky }\emph{ō wa nan no hi desu ka? \hfill\break
 }What day is it today? }
 
\par{11. え、 ${\overset{\textnormal{なん}}{\text{何}}}$ ですか。 \hfill\break
 \emph{E, nan desu ka? \hfill\break
 }Uh, what? }
 
\par{12. お ${\overset{\textnormal{しごと}}{\text{仕事}}}$ は ${\overset{\textnormal{なん}}{\text{何}}}$ ですか。 \hfill\break
 \emph{O-shigoto wa nan desu ka? \hfill\break
 }What is your job? }

\par{\textbf{Phrasing Note }: Ex. 12 could be rephrased as \emph{nan no shigoto wo shite imasu ka? }何の仕事をしていますか. In this phrasing, greater emphasis is placed on "what." As such, it could be translated as "What line of work are you in?" }
 
\par{13. お ${\overset{\textnormal{なまえ}}{\text{名前}}}$ は ${\overset{\textnormal{なん}}{\text{何}}}$ ですか。 \hfill\break
 \emph{O-namae wa nan desu ka? \hfill\break
 }What is your name? }
 
\par{14. ${\overset{\textnormal{しゅみ}}{\text{趣味}}}$ は ${\overset{\textnormal{なん}}{\text{何}}}$ ですか。 \hfill\break
 \emph{Shumi wa nan desu ka? \hfill\break
 }What are your hobbies? }
 
\par{15. エアコンはいつが ${\overset{\textnormal{やす}}{\text{安}}}$ いんですか。 \hfill\break
 \emph{Eakon wa itsu ga yasui n desu ka? }\hfill\break
When is it that air conditioning is cheap? }

\par{\textbf{Grammar Note }: いつ, along with なぜ,  are typically used as adverbs. The other interrogatives are always used as nouns. As for いつ, it can also be used as a noun, and when it is used as a noun, it's very similar to "what time (period)?" }
 
\par{16. ${\overset{\textnormal{じゅうたく}}{\text{住宅}}}$ を ${\overset{\textnormal{ゆうり}}{\text{有利}}}$ に ${\overset{\textnormal{か}}{\text{買}}}$ う ${\overset{\textnormal{じき}}{\text{時期}}}$ はいつがいいんですか。 \hfill\break
 \emph{J }\emph{ūtaku wo y }\emph{ūri ni kau jiki wa itsu ga ii n desu ka? \hfill\break
 }What time is good to lucratively buy a home? }
 
\par{17. ${\overset{\textnormal{ひなにんぎょう}}{\text{雛人形}}}$ を ${\overset{\textnormal{かざ}}{\text{飾}}}$ る ${\overset{\textnormal{じき}}{\text{時期}}}$ はいつごろが ${\overset{\textnormal{い}}{\text{良}}}$ いですか。 \hfill\break
 \emph{Hina-ningy }\emph{ō wo kazaru jiki wa itsu-goro ga ii desu ka? \hfill\break
 }About what time would be alright to display hina dolls? }

\par{\textbf{Phrase Note }: - \emph{goro }頃 is used after time phrases like "when" to mean "around\dothyp{}\dothyp{}\dothyp{}" }
 
\par{18. このサプリ(メント)は、 ${\overset{\textnormal{の}}{\text{飲}}}$ むタイミングがいつがベストなんですか? \hfill\break
 \emph{Kono sapuri(mento) wa, nomu taimingu ga itsu ga besuto na n desu ka? \hfill\break
 }As for this supplement, what timing would be best to drink it? }
 
\par{19. ${\overset{\textnormal{せいきゅう}}{\text{請求}}}$ はいつになりますか。 \hfill\break
 \emph{Seiky }\emph{ū wa itsu ni narimasu ka? \hfill\break
 }When will billing be? }
 
\par{20. いつ ${\overset{\textnormal{しあ}}{\text{仕上}}}$ がりますか。 \hfill\break
 \emph{Itsu shiagarumasu ka? \hfill\break
 }When will you be finished? }

\par{\textbf{Grammar Note }: Using \emph{ni }に after \emph{itsu }いつ in this sentence would be grammatically incorrect. }
 
\par{21. イチゴはいつ ${\overset{\textnormal{な}}{\text{生}}}$ るんですか? \hfill\break
 \emph{Ichigo wa itsu naru n desu ka? \hfill\break
 }When do strawberries ripen? }
 
\par{\textbf{Spelling Note }: \emph{Ichigo }may occasionally be spelled as 苺. }

\par{\textbf{Grammar Note }: \emph{Itsu ni naru }いつになる translates as "when will\dothyp{}\dothyp{}\dothyp{}be?" The "when" is essentially the same as "what time?" If you were to not use \emph{ni }に, \emph{naru }なる would have to interpreted as 生る (to ripen), 鳴る (to sound\slash ring), or 成る (to come into fruition). }
 
\par{22. ケータイの ${\overset{\textnormal{きんきゅうじしんそくほう}}{\text{緊急地震速報}}}$ はいつ ${\overset{\textnormal{な}}{\text{鳴}}}$ るんですか。 \hfill\break
 \emph{K }\emph{ētai no kinky }\emph{ū jishin sokuh }\emph{ō wa itsu naru n desu ka? \hfill\break
 }When does the mobile emergency earthquake alert go off? }
 
\par{\textbf{Word Note }: The formal word for cellular phone in Japanese is \emph{keitai denwa }携帯電話. Because \slash ei\slash  can be pronounced as \slash ē\slash , this explains the colloquial spelling ケータイ. }
 
\par{23. ${\overset{\textnormal{ふっこう}}{\text{復興}}}$ はいつ ${\overset{\textnormal{な}}{\text{成}}}$ るのですか。 \hfill\break
 \emph{Fukk }\emph{ō wa itsu naru no desu ka? \hfill\break
 }When will restoration come into fruition? }
 
\par{24. 「いつが ${\overset{\textnormal{あ}}{\text{空}}}$ いてます?」「 ${\overset{\textnormal{に}}{\text{2}}}$ ${\overset{\textnormal{しゅうかんご}}{\text{週間後}}}$ が ${\overset{\textnormal{あ}}{\text{空}}}$ いてます!」 \hfill\break
 \emph{“Itsu ga aitemasu?” “Nish }\emph{ūkan-go ga aitemasu!” \hfill\break
 }“What time is available?” “Two weeks from now is available!” }

\par{\textbf{Grammar Note }: With the use of \emph{ga }が after the respective time phrases, it is apparent that openings in schedules are being referred to. }
 
\par{25. ${\overset{\textnormal{いま}}{\text{今}}}$ (は)、いつですか。 \hfill\break
\emph{Ima (wa), itsu desu ka? \hfill\break
}What point in time is it now? }

\par{\textbf{Sentence Note }: One can imagine this sentence would be used by someone from the future confused as to what point in time he has traveled to. The use of the particle \emph{wa }は enhances the emphasis placed on the "now" in the sentence. }
 
\par{26. いつマレーシアに ${\overset{\textnormal{き}}{\text{来}}}$ ましたか。 \hfill\break
 \emph{Itsu Mar }\emph{ēshia ni kimashita ka? \hfill\break
 }When did you come to Malaysia? }
 
\par{27. ${\overset{\textnormal{でんち}}{\text{電池}}}$ は、いつ、 ${\overset{\textnormal{だれ}}{\text{誰}}}$ が ${\overset{\textnormal{はつめい}}{\text{発明}}}$ しましたか。 \hfill\break
 \emph{Denchi wa, itsu, dare ga hatsumei shimashita ka? \hfill\break
 }As for the battery, when and who invented it? }
 
\par{28. スミスさんはどこで ${\overset{\textnormal{なに}}{\text{何}}}$ を ${\overset{\textnormal{こわ}}{\text{壊}}}$ しましたか。 \hfill\break
 \emph{Sumisu-san wa doko de nani wo kowashimashita ka? \hfill\break
 }What did Mr. Smith break and where? }
 
\par{29. ${\overset{\textnormal{にほん}}{\text{日本}}}$ の\{ ${\overset{\textnormal{なに}}{\text{何}}}$ ・どこ\}が ${\overset{\textnormal{す}}{\text{好}}}$ きですか。 \hfill\break
 \emph{Nihon no [nani\slash doko] ga suki desu ka? \hfill\break
 }What about\slash where in Japan do you like? }

\par{\textbf{Sentence Note }: The use of \emph{doko }どこ can actually stand for either nuance whereas the use of \emph{nani }何 would only result in the first nuance. }
 
\par{30. ${\overset{\textnormal{すうがく}}{\text{数学}}}$ のどこが ${\overset{\textnormal{す}}{\text{好}}}$ きですか。 \hfill\break
 \emph{S }\emph{ūgaku no doko ga suki desu ka? \hfill\break
 }What part about math do you like? }
 
\par{31. ${\overset{\textnormal{じゅうしょ}}{\text{住所}}}$ はどこですか。 \hfill\break
 \emph{J }\emph{ūsho wa doko desu ka? \hfill\break
 }What is your address? }

\par{\textbf{Grammar Note }: Contrary to English, the word for "where" needs to be used in asking what someone's address is. }
 
\par{32. \{トイレ・お ${\overset{\textnormal{てあら}}{\text{手洗}}}$ い\}はどこですか。 \hfill\break
 \emph{[Toire\slash otearai] wa doko desu ka? \hfill\break
 }Where is the bathroom? }

\par{\textbf{Phrase Note }: \emph{Otearai }お手洗い is a more refined means of saying "bathroom" in the same way "restroom" is more refined than saying "bathroom." }
 
\par{33. なぜ ${\overset{\textnormal{まち}}{\text{街}}}$ へ ${\overset{\textnormal{い}}{\text{行}}}$ ったのですか。 \hfill\break
 \emph{Naze machi e itta no desu ka? \hfill\break
 }Why did you go to town? }
 
\par{34. コンビニのおにぎり、どこが ${\overset{\textnormal{おい}}{\text{美味}}}$ しいと ${\overset{\textnormal{おも}}{\text{思}}}$ いますか。 \hfill\break
 \emph{Kombini no onigiri, doko ga oishii to omoimasu ka? \hfill\break
 }As for convenience store rice balls, where are they really good at? }
 
\par{35. ${\overset{\textnormal{ひこうき}}{\text{飛行機}}}$ の ${\overset{\textnormal{ざせき}}{\text{座席}}}$ はどこがいいですか。 \hfill\break
 \emph{Hik }\emph{ōki no zaseki wa doko ga ii desu ka? \hfill\break
 }What seats on a(n air)plane are good? }
 
\par{36. ${\overset{\textnormal{ゆうびんきょく}}{\text{郵便局}}}$ はどこですか。 \hfill\break
 \emph{Y }\emph{ūbinkyoku wa doko desu ka? \hfill\break
 }Where is the post office? }

\begin{center}
\textbf{Why? } 
\end{center}

\par{ To use the word for "why" other than when you're just saying \emph{naze (desu ka?) }なぜ(ですか), you need to use the particle \emph{no }の with it in some fashion. When \emph{naze }なぜ is at the front of a sentence, the sentence should end in \emph{n\slash no desu ka }ん・のですか.  Contracting \emph{no }の to \emph{n }ん is done largely in the spoken language. When following verbs or adjectives, nothing else need to be done to the sentence, but if following nouns or adjectival nouns, then \emph{na }な will need to placed in between the noun\slash adjectival noun and \emph{n\slash no desu ka }ん・のですか. }
 
\par{37. ${\overset{\textnormal{うし}}{\text{牛}}}$ は、なぜ ${\overset{\textnormal{まいにちちち}}{\text{毎日乳}}}$ が ${\overset{\textnormal{で}}{\text{出}}}$ るんですか。 \hfill\break
 \emph{Ushi wa, naze mainichi chichi ga deru n desu ka? \hfill\break
 }Why do cows produce milk every day? }
 
\par{38. ${\overset{\textnormal{かざん}}{\text{火山}}}$ はなぜ ${\overset{\textnormal{ふんか}}{\text{噴火}}}$ するんですか。 \hfill\break
 \emph{Kazan wa naze funka suru n desu ka? \hfill\break
 }Why do volcanoes erupt? }

\par{39. なぜ ${\overset{\textnormal{めいし}}{\text{名詞}}}$ なんですか。 \hfill\break
\emph{Naze meishi na n desu ka? }\hfill\break
Why is it a noun? }

\par{ Under the same principles, if \emph{naze }なぜ is at the end of the sentence, a noun phrase has to precede it. If the phrase is a full sentence, then the particle \emph{no }の needs to be used to make it into a noun before \emph{naze }なぜ can create the question. In this situation, \emph{no }の shouldn't ever be contracted. }
 
\par{40. ${\overset{\textnormal{ひとびと}}{\text{人々}}}$ が ${\overset{\textnormal{かみ}}{\text{神}}}$ に ${\overset{\textnormal{いの}}{\text{祈}}}$ るのはなぜですか。 \hfill\break
 \emph{Hitobito ga kami ni inoru no wa naze desu ka? \hfill\break
 }Why do people pray to God\slash the gods. }
 
\par{41. ${\overset{\textnormal{ふと}}{\text{太}}}$ ったのはなぜだと ${\overset{\textnormal{おも}}{\text{思}}}$ いますか。 \hfill\break
 \emph{Futotta no wa naze da to omoimasu ka? \hfill\break
 }Why do you think it is you got fat? }
    