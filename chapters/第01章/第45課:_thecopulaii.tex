    
\chapter{The Copula II}

\begin{center}
\begin{Large}
第45課: The Copula II 
\end{Large}
\end{center}
 
\par{ This is one of the shortest lessons in IMABI. So, feel relaxed as you go through more information about the copula in Japanese. }
      
\section{A Closer Look at である, だ, \& です}
 
\begin{center}
\textbf{である }
\end{center}

\par{ である is quite \textbf{ceremonious }and normally used only in writing, but when it is used in the spoken language, it becomes an objective-sounding means of presenting something as fact. A comical example of this is the title of a famous book written by the renowned author 夏目 ${\overset{\textnormal{そうせき}}{\text{漱石}}}$ . }

\par{1. ${\overset{\textnormal{わがはい}}{\text{我輩}}}$ は ${\overset{\textnormal{ねこ}}{\text{猫}}}$ である。 \hfill\break
I am a cat. }
 
\par{\textbf{Word Note }: わがはい is a very pompous, old-fashioned pronoun, and it is well-known from this title in and outside of Japan. Aside from this title, this pronoun is hardly ever used. It is certainly not used in everyday conversation, and even if it were, it would be for comedic relief. }
 
\par{2. 「あの君島という人は、そっちの会社の人 \textbf{な }わけですね」 \hfill\break
So, that Kimishima guy is someone from that company, no? \hfill\break
From 顔に降りかかる雨 by 桐野夏生. }
 
\par{\textbf{Grammar Note }: The reason why this example is cited is because of a peculiar avoidance of である due to the lack of a need to be formal. Typically, having a whole phrase be a participial for another noun simply involves な. At times, である can be used instead but usually in writing. In a way, it can be viewed as the non-contracted form of this grammar. }
 
\par{ However, it is important to note that when you wish to use something like んだ・のだ, which is for explicit emphasis, to "nominalize" an entire phrase, な is obligatory after a noun if it is the final word. If you have それは魚 and want to add んだ, you say それは魚なんだ. These two grammatical situations demonstrate how である and な perform the same function but are not 100\% interchangeable. }
 
\begin{center}
\textbf{だ \& }\textbf{です }
\end{center}
 
\par{As far as usage is concerned, both だ and です make a declarative sentence. Also, both can act as a final particle to strengthen an appeal to the listener. }
 
\par{3. 日本語は簡単です。 \hfill\break
Japanese is easy. }
 
\par{4. 明日は休みだ。 \hfill\break
Tomorrow is a break. }
 
\par{5. ぼくはアイスティー。 \hfill\break
Note: This sentence can be interpreted differently depending on context. It could show that you want to drink iced tea, brought iced tea, etc. }
 
\par{6. げんき。 \hfill\break
I'm fine. }
 
\par{\textbf{Grammar Note }: The omission of the copula is allowed, but the statement loses some of its assertiveness, which could be helpful to not sound rude or blunt in the case of だ. }
 
\par{7. まあ、 ${\overset{\textnormal{しゅくだい}}{\text{宿題}}}$ だ。 \hfill\break
Well, it's homework (time). }
 
\par{8. さあ、 ${\overset{\textnormal{しゅっぱつ}}{\text{出発}}}$ です。 \hfill\break
Well, it's (time to) depart. }
 
\par{9. そこでだ、 ${\overset{\textnormal{きみ}}{\text{君}}}$ はだね \hfill\break
So, you }
 
\par{\textbf{Grammar Note }: The copula may be used as a filler word like in Ex. 9. \hfill\break
\hfill\break
10. あっかんべーだ。 \hfill\break
Sticking out your tongue. }

\par{ \textbf{Culture Note }: あっかんべー is where you stick out your tongue while pulling down your eyelid. }
 
\par{11. 明日はですね。 \hfill\break
So, tomorrow. }
 
\par{です can make some conjugations polite whereas だ \textbf{doesn't }! だ is plain and the basic form of adjectives and verbs are already plain. }
 
\par{12a. 簡単ではないだ X \hfill\break
12b. 簡単ではないです。 〇 \hfill\break
It's not easy. }
      
\section{Variants of the Copula}
 
\par{ The copula can look quite different depending on where you are and what speech style you are using. Aside from である, だ, and です, the following variants are important to keep in mind, but the only extra two that you will be immediately responsible for are や and じゃ. The honorific variants will be touched on at a later time. }

\begin{ltabulary}{|P|P|P|P|}
\hline 

Kansai Region & や & Southern Japan & や、じゃ、ちゃ \\ \cline{1-4}

Western Japan & や、じゃ & Colloquial & っす \\ \cline{1-4}

Older generations & じゃ & Classical & にあり・なり \\ \cline{1-4}

Plain Speech & だ & Polite Speech & です \\ \cline{1-4}

Respectful & でいらっしゃいます & Humble & でございます \\ \cline{1-4}

\end{ltabulary}
 
\par{13. うそ \textbf{や }。(Kansai) \hfill\break
That's a lie. }

\par{${\overset{\textnormal{}}{\text{14. 私}}}$ は ${\overset{\textnormal{}}{\text{}}}$ \textbf{でございます }。(Humble) \hfill\break
I am Ryoji Tanaka. }

\par{${\overset{\textnormal{}}{\text{}}}$ \textbf{でいらっしゃいます }。(Honorific) \hfill\break
(He) is President Kurogi. }

\par{${\overset{\textnormal{}}{\text{}}}$ \textbf{だ }。(Plain)  \hfill\break
It's real. }

\par{${\overset{\textnormal{}}{\text{}}}$ \textbf{っす }。 (Colloquial) \hfill\break
It's easy. }
    