    
\chapter{The Particle に II}

\begin{center}
\begin{Large}
第41課: The Particle に II  
\end{Large}
\end{center}
 
\par{ There is a lot to learn about the particle に. So, instead of having to go through it all in one lesson, this lesson will go over the rest of the basic information that you should know about it. }
      
\section{More on the Case Particle に}
 
\begin{center}
 \textbf{A }\textbf{をB }\textbf{にする }
\end{center}

\par{ "AをBにする" is "to make\slash have A be a B way". You can alternatively see the particle と instead, but the difference between the two is that に shows the endpoint of a duration of some sort whereas と shows the content\slash substance of a result. This may seem arbitrary, but it will become more important later on. }

\par{ "AをB\{に・と\}する" may also show what someone deems as or decides on. Using と sounds more formal and rigid due to its aforementioned quality. This is most prominently seen in こと\{に・と\}する. }

\par{ This pattern also applies to the adverbial forms of adjectives. For 形容詞, you would alternatively use ~くする, not ~くにする. }

\par{1. ${\overset{\textnormal{さつがいじけん}}{\text{殺害事件}}}$ を ${\overset{\textnormal{しょうせつ}}{\text{小説}}}$ にする。 \hfill\break
To make the murder case into a novel. }

\par{2. ${\overset{\textnormal{おんがく}}{\text{音楽}}}$ のボリュームを高くする。 \hfill\break
To make the music louder. }

\par{3. 彼は彼女を ${\overset{\textnormal{しあわ}}{\text{幸}}}$ せにした。 \hfill\break
He made her happy. }

\par{4. ${\overset{\textnormal{いもうと}}{\text{妹}}}$ が ${\overset{\textnormal{へや}}{\text{部屋}}}$ をきれいにしました。 \hfill\break
My younger sister made the\slash her room clean. }

\par{5. ${\overset{\textnormal{こども}}{\text{子供}}}$ を ${\overset{\textnormal{まんがか}}{\text{漫画家}}}$ にする。 \hfill\break
To have a kid become a manga artist. }

\par{6. ${\overset{\textnormal{におくえん}}{\text{2億円}}}$ を ${\overset{\textnormal{みのしろきん}}{\text{身代金}}}$ とする。 \hfill\break
To use 200 million yen as ransom money. }

\par{7. ${\overset{\textnormal{しゃしん}}{\text{写真}}}$ を ${\overset{\textnormal{しゅみ}}{\text{趣味}}}$ にする。 \hfill\break
To do pictures as a hobby. }

\par{9. すしを昼ご飯にする。 \hfill\break
To have sushi for lunch. }

\par{10. ポンドを円に ${\overset{\textnormal{りょうがえ}}{\text{両替}}}$ してください。 \hfill\break
Please change these pounds into yen. }

\par{11. ${\overset{\textnormal{いわ}}{\text{岩}}}$ を ${\overset{\textnormal{まくら}}{\text{枕}}}$ にして ${\overset{\textnormal{ねむ}}{\text{眠}}}$ る。 \hfill\break
To sleep with a rock as a pillow. }

\par{\textbf{Grammar Note }: There are instances in Japanese where certain adjective れんようけい do become functional nouns such as 近く (vicinity) and 遠く (afar). This, though, combined with this phrase, which is possible, would not cause any structural problems given the definitions. }

\par{\textbf{Curriculum Note }: There will be more things discussed about とする over time. }

\begin{itemize}
 
\item Shows the \textbf{standard of action, condition, or state }.      Ex. "to lack \textbf{in }creativity".  
\end{itemize}
12. ${\overset{\textnormal{じんこう}}{\text{人口}}}$ は ${\overset{\textnormal{かこ}}{\text{過去}}}$ ${\overset{\textnormal{よ}}{\text{4}}}$ ${\overset{\textnormal{ねんかん}}{\text{年間}}}$ に3 ${\overset{\textnormal{ばい}}{\text{倍}}}$ に ${\overset{\textnormal{ぞうか}}{\text{増加}}}$ しました。 \hfill\break
The population has increased three times in the last four years. 
\par{\textbf{Word Note }: 人口 is either ${\overset{\textnormal{おお}}{\text{多}}}$ い or ${\overset{\textnormal{すく}}{\text{少}}}$ ない, not 大きい or 小さい. Although the last sentence was different, you definitely need to know this. }

\par{13. ${\overset{\textnormal{よふ}}{\text{夜更}}}$ かしは体によくありません。 \hfill\break
Staying up late is not good for your health. }

\par{14. 僕は ${\overset{\textnormal{いちにち}}{\text{一日}}}$ に ${\overset{\textnormal{さんかい}}{\text{3回}}}$ ${\overset{\textnormal{さんぽ}}{\text{散歩}}}$ に行く。    (Guy speech) \hfill\break
I go on a walk three times a day. }

\par{15. 私の家は駅に近いです。 \hfill\break
My house is close to the train station. \hfill\break
\hfill\break
16. ${\overset{\textnormal{しょうぶ}}{\text{勝負}}}$ に勝ったよ! \hfill\break
I won the match! }

\par{\textbf{Word Note }: 勝つ should only be used to mean "to win" when talking about games of some sort. If you want to say, "to win the lottery", you need to say something like "くじがあたる". }

\begin{itemize}
 
\item Shows \textbf{what brings about some sort of measure, feelings,      situation, or work }. It is "by" as in "to      be\dothyp{}\dothyp{}\dothyp{}by\dothyp{}\dothyp{}\dothyp{}" and the "from" in receiving.  
\end{itemize}

\par{17. ${\overset{\textnormal{ひろし}}{\text{宏}}}$ にしばしば ${\overset{\textnormal{はら}}{\text{腹}}}$ を立てる。 \hfill\break
I frequently get angry at Hiroshi. }

\par{18. ${\overset{\textnormal{きみ}}{\text{君}}}$ に ${\overset{\textnormal{どうじょう}}{\text{同情}}}$ する。 \hfill\break
I sympathize with you. }

\par{19. 僕は ${\overset{\textnormal{そぼ}}{\text{祖母}}}$ にお ${\overset{\textnormal{こづか}}{\text{小遣}}}$ いをもらった。 \hfill\break
I received some spending money from my grandmother. }

\begin{center}
 \textbf{More Usages }
\end{center}

\begin{itemize}
 
\item Shows direct time. Although true, it is not used after 今日, 去年, 翌年, 来月, etc.  
\end{itemize}
 
\par{20. 学校は ${\overset{\textnormal{まいあさ}}{\text{毎朝}}}$ 8時15分に ${\overset{\textnormal{はじ}}{\text{始}}}$ まります。 \hfill\break
School begins every morning at 8:15. }

\par{21. ${\overset{\textnormal{かいぎ}}{\text{会議}}}$ は2時に終わります。 \hfill\break
The meeting will end at 2. }
 
\par{22. 私は五月の終わりに ${\overset{\textnormal{おおさか}}{\text{大阪}}}$ に ${\overset{\textnormal{たいざい}}{\text{滞在}}}$ しました。 \hfill\break
I stayed in Osaka at the end of May. }
 
\begin{itemize}
 
\item Creates pairs with a deep connection. It's often in set      phrases. This is also actually used in ordering items as well.  
\end{itemize}
23. ${\overset{\textnormal{しんろう}}{\text{新郎}}}$ に ${\overset{\textnormal{しんぷ}}{\text{新婦}}}$ 。 \hfill\break
A bridegroom and a bride. \hfill\break
24. ${\overset{\textnormal{おに}}{\text{鬼}}}$ に ${\overset{\textnormal{かなぼう}}{\text{金棒}}}$ 。 \hfill\break
A demon and a metal club.  25. ${\overset{\textnormal{たぜい}}{\text{多勢}}}$ に ${\overset{\textnormal{ぶぜい}}{\text{無勢}}}$ 。 \hfill\break
To be outnumbered.  
\par{\textbf{Definition Note }: 多勢 = "a lot of people"; 無勢 = "few people". }
 
\begin{itemize}
 
\item Verb stem + に + same verb      creates a special emphasis. The base form of this pattern may be in the non-past form, but in actual practice, the final verb is basically always in the past tense. The grammar pattern itself is usually used in the written language. 
\end{itemize}
 
\par{26. 待ちに待った。 \hfill\break
To wait and wait. }
 
\par{27a. 赤ちゃんは泣きに泣いた。(Slightly literary) \hfill\break
27b. 赤ちゃんは大泣きしていた。 (More natural) \hfill\break
The baby cried and cried. }
    