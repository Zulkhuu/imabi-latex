    
\chapter{~てある}

\begin{center}
\begin{Large}
第37課: ~てある  
\end{Large}
\end{center}
 
\par{ ~てある is quite different from ~ている, but it is still often confused with it. Just as was the case with いる, ある functions as a supplementary verb when paired with the particle て. As a recap, remember that in Japanese, a supplementary verb is \emph{a verb that loses some or all of its literal meaning(s) to serve a specific grammatical purpose }. }
      
\section{~てある: Creating Stative Verbs}
 
\par{ ~てある is exclusively used with verbs but is restricted to only certain kinds of verbs. }

\begin{ltabulary}{|P|P|P|P|P|P|}
\hline 

Class & Example Verb & + ~てある & Class & Example Verb & + ~てある \\ \cline{1-6}

 \emph{Ichidan }Verbs & 伝える & 伝えてある &  \emph{Godan }Verbs & 洗う & 洗ってある \\ \cline{1-6}

 \emph{Kuru }& 来る & 来てある X &  \emph{Suru }& する & してある \\ \cline{1-6}

\end{ltabulary}

\par{\textbf{Grammar Note }: For grammatical reasons to be discussed, 来てある is ungrammatical. }

\begin{center}
\textbf{What is ~てある? }
\end{center}

\par{~てある shows a \emph{current }state caused by someone's action. This state results from purposeful action done by someone, not something. In other words, the someone has to be a person. Personification, however, can be used to make non-humans treated as people in this situation. It's used with transitive verbs, but を is not used for this usage. The basic pattern is (だれかに)XがYてある. }

\par{ 花が ${\overset{\textnormal{い}}{\text{生}}}$ けてある means that "the flowers have been arranged (by someone)" and the flowers are still arranged as such. 生ける is transitive. Do not confuse with 生きる. }

\par{ The following example exhibits a static nuance. The resultant state is "to have informed." }

\par{1. 彼女には前もって ${\overset{\textnormal{つた}}{\text{伝}}}$ えてある。 \hfill\break
To have informed her beforehand. }

\par{Similarly, ~ことにしてある has it that something is deemed as such by someone but really isn't. ~にする means "to make as..". For example, ばかにする means "to make an idiot of". Similarly, "Verb + ことにする" means "to decide that". Together, you can make sentences like the one below. }

\par{2. ${\overset{\textnormal{げんき}}{\text{元気}}}$ でいることにしてある。 \hfill\break
I decided to (make myself out to) be well (even though I'm really not). }

\par{Let's put some context to this statement. Your friend is in a stressful situation at home. In order to not worry her family, she has decided to be spirited on the phone to keep them at ease. This wouldn't be referring to actually seeing them in person. }

\par{\textbf{Examples }}

\par{3. その ${\overset{\textnormal{とけい}}{\text{時計}}}$ は ${\overset{\textnormal{ごふんすす}}{\text{五分進}}}$ めてあります。 \hfill\break
(I) have set the clock five minutes forward. \hfill\break
State: The state is that the clock has been set five minutes forward. }

\par{\textbf{Sentence Note }:  Ex. 3 shows that this pattern occasionally implies that the speaker is who did the action. }

\par{4. ガラスが ${\overset{\textnormal{わ}}{\text{割}}}$ ってある。 \hfill\break
The glass has been broken (because of the actions of someone and still is broken). \hfill\break
State: The glass is still broken due to the fact that someone purposely broke it in the first place. }

\par{5. アンケートは ${\overset{\textnormal{あつ}}{\text{集}}}$ めてある。 \hfill\break
The questionnaires have been gathered. \hfill\break
State: The questionnaires are now gathered due to the actions of a person or people. }

\par{6. ${\overset{\textnormal{まど}}{\text{窓}}}$ が ${\overset{\textnormal{}}{\text{開}}}$ けてあります。 \hfill\break
The window is open (by someone). \hfill\break
State: The window has been left open by someone. }

\par{7. ストーブがつけてあります。 \hfill\break
The heater was turned on (by someone) and has been kept that way. \hfill\break
State: The heater has been turned on by someone. }

\par{8. 木が ${\overset{\textnormal{たお}}{\text{倒}}}$ してある。 \hfill\break
The tree has been toppled down (by someone). \hfill\break
State: A person knocked the tree down and the tree is still on the ground. }

\par{9. 彼の ${\overset{\textnormal{ねつ}}{\text{熱}}}$ ${\overset{\textnormal{}}{\text{は計}}}$ ってある。 \hfill\break
His temperature was checked (by someone). \hfill\break
State: The person's temperature has been checked. }

\par{10. ${\overset{\textnormal{ばん}}{\text{晩}}}$ ご ${\overset{\textnormal{はん}}{\text{飯}}}$ が作ってある。 \hfill\break
Dinner has been made. \hfill\break
State: The dinner is made. }

\par{11. シャツが ${\overset{\textnormal{あら}}{\text{洗}}}$ ってある。 \hfill\break
The shirt has been washed. \hfill\break
State: The shirt is washed. }

\par{12. 机に本が置いてある。 \hfill\break
The book was placed on the desk (by someone). }

\par{13. ${\overset{\textnormal{にもつ}}{\text{荷物}}}$ が ${\overset{\textnormal{らんざつ}}{\text{乱雑}}}$ に ${\overset{\textnormal{つ}}{\text{積}}}$ んである。 \hfill\break
The luggage was piled up in a clutter (by someone). }

\par{\textbf{Other Conjugations }}

\par{ ~てある has other conjugations. For instance, you can still see ~てあった and ~てない. Because particle usage and other parts of grammar differ between the two patterns, it shouldn't be too difficult to distinguish this ~てない from the contraction of ~ていない, ~てない. }

\par{14. 一階は古本屋に ${\overset{\textnormal{か}}{\text{貸}}}$ してあった。 \hfill\break
The first floor [had been\slash was] rented to an old book store. }

\par{15. 新聞に書いてあった。 \hfill\break
It [had been\slash was] written in the newspaper. }

\par{16. あまり ${\overset{\textnormal{かぐ}}{\text{家具}}}$ が置いてありません。 \hfill\break
There isn't much furniture placed. }
    