    
\chapter{Pronunciation II}

\begin{center}
\begin{Large}
第2課: Pronunciation II: Consonants 
\end{Large}
\end{center}
 
\par{ In this lesson, we will learn how to pronounce the consonants of Japanese. Several of them don't exist in English, so it will take some time to get them down. Just as was the case in Lesson 1, English letters will be used to transcribe Japan ese words ( \emph{Rōmaji }). }
      
\section{Unvoiced Consonants}
 
\begin{center}
\textbf{What are Unvoiced Consonants? } 
\end{center}

\par{ A consonant is a speech sound that obstructs airflow from the lungs. To understand what an "unvoiced" consonant is, it's necessary to know what a "voiced" consonant is. A "voiced" sound is a sound that causes the vocal folds in one's throat to vibrate. You can verify this by putting your hand over your Adam's apple, where your vocal folds are located, as you speak. When you pronounce a voiced consonant, you'll feel your Adam's apple vibrate. Voiced consonants include sounds like \slash b\slash  and \slash d\slash . They're not the only things that are voiced, though, as vowels are also examples of "voiced" sounds. }

\par{ Conversely, an "unvoiced" sound is a sound that does not cause the vocal folds to vibrate. Examples of unvoiced consonants include \slash p\slash  and \slash t\slash . When you pronounce something like \slash ka\slash , you feel the contrast between non-voiced and voiced sounds, with \slash k\slash  being unvoiced and \slash a\slash , a vowel, being voiced. }

\par{ In English, unvoiced consonants are typically pronounced with aspiration. Aspiration is the strong burst of air that accompanies the release of certain kinds of consonants. For example, pronounce the word "king" with one hand directly in front of your mouth. If your pronunciation is native(-like), you should notice a puff of air hitting your hand. This is what aspiration is. In Japanese, unvoiced consonants tend to be slightly more aspirated than they are in languages like Spanish but not nearly as so as in English or Korean. }

\begin{center}
\textbf{The Unvoiced Consonants of Japanese }  
\end{center}

\par{ The basic unvoiced consonants of Japanese are \slash k\slash , \slash s\slash , \slash t\slash , \slash h\slash , and \slash p\slash . Overall, all these consonants are less aspirated than their English counterparts. Other differences exist, of course, which is why each consonant will be introduced individually. }

\begin{center}
\textbf{Pronouncing \slash k\slash  }
\end{center}

\par{ Just as is the case in English, \slash k\slash  is made by placing the back of the tongue against the soft palate in the back of the mouth. }

\begin{ltabulary}{|P|P|P|P|P|}
\hline 

 \emph{Ku \textbf{ki }}↓ &  \emph{\textbf{Ka }ki }& Oyster &  \emph{Ke \textbf{ika }}& Progress \\ \cline{1-5}

\end{ltabulary}

\begin{center}
\textbf{Pronouncing \slash t\slash  }
\end{center}

\par{ \slash t\slash  is made by placing the blade of the tongue behind the upper teeth. When the vowel \slash u\slash  follows \slash t\slash , it becomes [ts]. This is an example of an allophone. An allophone is a variation of the same consonant in the confounds of a particular language. This [ts] is the same as the \slash ts\slash  consonant cluster found in words like "its" in English. Unlike English, you must never drop the "t" in [ts] in Japanese. This means that "tsunami" is not pronounced as \slash sunami\slash . It's pronounced as \slash tsu. \textbf{na.mi }\slash . \hfill\break
}

\par{ Additionally, \slash t\slash  becomes [ch] when followed by the vowel \slash i\slash . However, the [ch] in Japanese is not like the "ch" in "chair." The Japanese [ch] is produced by first stopping air flow and then placing the blade of the tongue right behind the gum line while the middle of the tongue touches the hard palate of the mouth. }

\begin{ltabulary}{|P|P|P|P|P|P|}
\hline 

 \emph{Ta \textbf{ke }}& Bamboo &  \emph{To \textbf{chi }}& Land &  \emph{Tsu \textbf{ki }}↓ & Moon \\ \cline{1-6}

\end{ltabulary}

\begin{center}
\textbf{Pronouncing \slash s\slash  } \hfill\break

\end{center}

\par{ The consonant \slash s\slash  is pronounced just like it is in English, but it becomes [sh] when followed by \slash i\slash . When pronouncing [sh], the middle of the tongue is bowed and raised towards the hard palate of the mouth. Note that [sh] is made not as farther back in the mouth as is the case in English. }

\begin{ltabulary}{|P|P|P|P|P|P|}
\hline 

 \emph{I \textbf{su }}& Chair &  \emph{Ke \textbf{isatsu }}& Police &  \emph{\textbf{Shi }itsu }& (Bed) sheet(s) \\ \cline{1-6}

\end{ltabulary}

\begin{center}
\textbf{Pronouncing \slash h\slash  \& \slash p\slash  } 
\end{center}

\par{ \slash p\slash  is known as a plosive sound. It is made by releasing air upon opening one's lips. In Japanese, it isn't all that common because most words with \slash p\slash  come from other languages. Both \slash p\slash  and \slash h\slash  are pronounced the same as in English, but \slash h\slash   has two allophones. When followed by \slash i\slash , \slash h\slash  sounds most like the "h" in "hue." When followed by \slash u\slash , it becomes [f]. The Japanese [f], though, is created by bringing the lips together and blowing air through them without using the teeth. }

\begin{ltabulary}{|P|P|P|P|P|P|P|P|}
\hline 

 \emph{Ha \textbf{ko }}& Box &  \emph{\textbf{Hi }fu }& Skin &  \emph{Fu \textbf{tsuu }}& Ordinary &  \emph{\textbf{Pe }nki }& Paint \\ \cline{1-8}

\end{ltabulary}

\begin{center}
\textbf{More Example Words }
\end{center}

\begin{ltabulary}{|P|P|P|P|P|P|P|P|}
\hline 

 \emph{\textbf{A }sa }& Morning &  \textbf{\emph{Te }}↓ & Hand(s) &  \emph{Ko \textbf{ko }}& Here &  \emph{Sa \textbf{ke }}& Alcohol \\ \cline{1-8}

 \emph{A \textbf{shi }}↓ & Foot\slash leg &  \emph{Shi \textbf{ai }}& Match\slash game &  \emph{Pa \textbf{supo }oto }& Passport &  \emph{\textbf{Sa }ke }& Salmon \\ \cline{1-8}

 \emph{\textbf{Ka }tsu }& To win &  \emph{\textbf{Tsu }aa }& Tour &  \emph{\textbf{A }ka }& Red &  \emph{\textbf{Pa }i }& Pie\slash pi \\ \cline{1-8}

 \emph{Ka }↑ & Mosquito &  \emph{\textbf{Su }utsu }& Suit &  \emph{\textbf{So }to }& Outside &  \emph{Tsu \textbf{chi }↓ }& Dirt\slash earth \\ \cline{1-8}

 \emph{\textbf{A }se }& Sweat &  \emph{Sa \textbf{to }o }& Sugar &  \emph{\textbf{Ta }ko }& Octopus\slash kite &  \emph{\textbf{Chi }ka }& Underground \\ \cline{1-8}

 \emph{\textbf{Ka }ko }& Past &  \emph{\textbf{He }ishi }& Soldier &  \emph{Ho \textbf{shi }}& Star &  \emph{Ku \textbf{tsu }}↓ & Shoes \\ \cline{1-8}

 \emph{\textbf{Hi }↓ }& Fire\slash day &  \emph{Ko \textbf{tatsu }}& Kotatsu &  \emph{Tsu \textbf{kau }}& To use &  \emph{\textbf{Ka }sa }& Umbrella \\ \cline{1-8}

\end{ltabulary}
\hfill\break

\begin{center}
\textbf{Vowel Devoicing }
\end{center}

\par{ When the vowels \slash i\slash  and \slash u\slash  are in between and\slash or after unvoiced consonants--\slash k\slash , \slash t\slash , \slash s\slash , and \slash p\slash  along with their respective allophones--they become devoiced (silent). Devoicing is a very distinctive feature of Standard Japanese pronunciation. As an example, the phrase for "good morning" sounds like "o- \textbf{ha-yo-o go-za-i-ma }-s". However, it is important to note that many speakers, especially those that don't come from East Japan, do not devoice vowels. }

\par{\textbf{Practice }: Pronounce the words below with the underlined vowels devoiced. }

\par{\emph{K u \textbf{sha }mi }(Sneeze)  \emph{\textbf{Ta }f u }(Tough)  \emph{H i \textbf{to }}(↓) (Person) }
      
\section{Voiced Consonants}
 
\par{ The unvoiced consonants and their allophones mentioned above all have a voiced consonant counterpart. For every voiced consonant, its pronunciation is the same as its unvoiced counterpart minus voicing. }

\begin{ltabulary}{|P|P|}
\hline 

Unvoiced Counterpart & Voiced Counterpart \\ \cline{1-2}

\slash k\slash  & \slash g\slash  \\ \cline{1-2}

\slash s\slash  & \slash z\slash  \\ \cline{1-2}

\slash sh\slash  & \slash j\slash  \\ \cline{1-2}

\slash t\slash  & \slash d\slash  \\ \cline{1-2}

\slash ts\slash  & \slash dz\slash  \\ \cline{1-2}

\slash ch\slash  & \slash dj\slash  \\ \cline{1-2}

\slash h\slash  (and allophones) & \slash b\slash  \\ \cline{1-2}

\slash p\slash  & \slash b\slash  \\ \cline{1-2}

\end{ltabulary}

\par{ There are a few peculiarities that need to be discussed. However, before going into too much detail, \slash j\slash  and \slash dj\slash  will be mentioned later in this lesson. }

\par{1. \slash z\slash  typically becomes [dz] at the start of words. \slash dz\slash  tends to become [z] inside words, but this isn't always so. \slash z\slash  sounds like the "z" in "zoo," whereas \slash dz\slash  sounds like the "ds" in "kids." However, it is important to note that many speakers cannot tell the difference between the two sounds. }

\par{2. \slash h\slash , its allophones, and \slash p\slash  correspond with \slash b\slash . \slash b\slash  is made by bringing the lips together and then releasing them. This means its articulation is the same as \slash p\slash  but not as \slash h\slash . }

\par{3. \slash g\slash  can be pronounced as \slash ng\slash  inside words. This pronunciation is particularly common in the north and east of Japan. }

\par{ Try pronouncing the following example words. }

\begin{ltabulary}{|P|P|P|P|P|P|}
\hline 

 \emph{\textbf{So }kudo }& Speed &  \emph{\textbf{Ba }ka }& Idiot &  \emph{Zu \textbf{tsuu }}& Headache \\ \cline{1-6}

 \emph{Tsu \textbf{(d)zuki }}& Continuance &  \emph{Ka \textbf{ze }}& Wind &  \emph{\textbf{De }eto }& A date \\ \cline{1-6}

 \emph{\textbf{Ka }zu }& Number &  \emph{\textbf{Ka }ge }& Shadow &  \emph{\textbf{Gi }taa }& Guitar \\ \cline{1-6}

 \emph{\textbf{Ba }taa }& Butter &  \emph{To \textbf{ge }}↓ & Thorn &  \emph{Do \textbf{ku }}↓ & Poison \\ \cline{1-6}

\end{ltabulary}
\hfill\break

\begin{center}
\textbf{More Voiced Consonants } 
\end{center}

\par{ There are also voiced consonants that do not have unvoiced counterparts. These sounds are listed in the chart below. }

\begin{ltabulary}{|P|P|}
\hline 

[n] & Made with the blade of the tongue on the back of the upper teeth with \slash a\slash , \slash e\slash , and \slash o\slash , behind the ridge of the mouth with \slash i\slash  (like in news), and behind the teeth with \slash u\slash  (like in noon). \\ \cline{1-2}

[m] & Pronounced by bringing the two lips together just as in English. \\ \cline{1-2}

[r] & Its pronunciation varies drastically. It is typically pronounced as a flap, which is only seen in American English as the "t" in many words such as "water." At the beginning of a word, it sounds almost like \slash d\slash . Sometimes it's pronounced as a trill or like \slash l\slash . 
\\ \cline{1-2}

[y] & Pronounced the same in English by bringing the tongue up to the hard palate. This means it is a palatal consonant. \\ \cline{1-2}

[w] & Its pronunciation is very similar to the Japanese \slash u\slash . Rather than protruding your lips, you compress them. It is only used with the vowels \slash a\slash  and \slash o\slash , but its use with \slash o\slash  won't even become important until later on in your studies. \\ \cline{1-2}

\end{ltabulary}

\par{ The differences in pronunciation detailed above make Japanese sound significantly different from English. Many sounds tend to be closer to the teeth, which is the case for [n] and [r], and movement of the tongue and parts of the mouth are more limited in range. To practice pronouncing these consonants, try saying the following words out loud. }

\begin{ltabulary}{|P|P|P|P|P|P|}
\hline 

 \emph{\textbf{Fu }ne }& Boat &  \emph{\textbf{Ne }ko }& Cat &  \emph{Mu \textbf{ne }}↓ & Chest \\ \cline{1-6}

 \emph{Yo \textbf{wa }i }& Weak &  \emph{Ya \textbf{kusoku }}& Promise &  \emph{Ka \textbf{wa }}↓ & River \\ \cline{1-6}

Mu \textbf{ra }↓ & Village &  \emph{\textbf{U }mi }& Sea &  \emph{Ta \textbf{na }}& Shelf \\ \cline{1-6}

 \emph{\textbf{Ka }mi }& God &  \emph{Ka \textbf{mi }}& Hair\slash paper &  \emph{Yu \textbf{bi }}↓ & Finger \\ \cline{1-6}

 \emph{\textbf{Wa }na }& Trap &  \emph{Ka \textbf{rada }}& Body &  \emph{\textbf{Re }i }& Example\slash zero \\ \cline{1-6}

\end{ltabulary}
      
\section{Palatal Consonants}
 
\par{ Palatal consonants are made by the body of the tongue touching against the hard palate of the mouth. In Japanese, these consonants are usually limited to the vowels \slash a\slash , \slash u\slash , and \slash o\slash , and they're all created with the help of the consonant \slash y\slash . First, we'll look at those palatal consonants shown below in the chart. }

\begin{ltabulary}{|P|P|P|P|}
\hline 

Consonant & C + \slash a\slash  & C + \slash u\slash  & C + \slash o\slash  \\ \cline{1-4}

\slash y\slash  & \slash ya\slash  & \slash yu\slash  & \slash yo\slash  \\ \cline{1-4}

\slash ky\slash  & \slash kya\slash  & \slash kyu\slash  & \slash kyo\slash  \\ \cline{1-4}

\slash gy\slash  & \slash gya\slash  & \slash gyu\slash  & \slash gyo\slash  \\ \cline{1-4}

\slash ny\slash  & \slash nya\slash  & \slash nyu\slash  & \slash nyo\slash  \\ \cline{1-4}

\slash hy\slash  & \slash hya\slash  & \slash hyu\slash  & \slash hyo\slash  \\ \cline{1-4}

\slash py\slash  & \slash pya\slash  & \slash pyu\slash  & \slash pyo\slash  \\ \cline{1-4}

\slash by\slash  & \slash bya\slash  & \slash byu\slash  & \slash byo\slash  \\ \cline{1-4}

\slash my\slash  & \slash mya\slash  & \slash myu\slash  & \slash myo\slash  \\ \cline{1-4}

\slash ry\slash  & \slash rya\slash  & \slash ryu\slash  & \slash ryo\slash  \\ \cline{1-4}

\end{ltabulary}

\par{\textbf{Usage Note }: In loan-words, these consonants may be used with other vowels. \hfill\break
}

\par{ Most of these combination are very common in Japanese. They are most frequently found in words that come from Chinese. Below are some examples. }

\begin{ltabulary}{|P|P|P|P|P|P|}
\hline 

 \emph{Hyo \textbf{o }}& Vote &  \emph{Kya \textbf{ku }}& Customer &  \emph{\textbf{Kyu }u }& Nine \\ \cline{1-6}

 \emph{\textbf{Myo }o }& Weird &  \emph{Hya \textbf{ku }}↓ & 100 &  \emph{\textbf{Kyo }o }& Today \\ \cline{1-6}

 \emph{Mya \textbf{ku }}& Pulse &  \emph{\textbf{Ryu }u }& Dragon &  \emph{Byo \textbf{oki }}& Illness \\ \cline{1-6}

 \emph{Gyak \textbf{u }}& Reverse &  \emph{\textbf{Ryo }o }& Quantity &  \emph{\textbf{Rya }ku }& Abbreviation \\ \cline{1-6}

\end{ltabulary}

\begin{center}
\textbf{Other Palatal Consonants } 
\end{center}

\par{ The remaining palatal sounds that have yet to be looked at are \slash sh\slash , \slash ch\slash , \slash j\slash , and \slash dj\slash . These palatal consonants aren't just allophones of \slash s\slash , \slash t\slash , \slash z\slash , or \slash d\slash  respectively. In fact, they are full-fledged consonants that can be used with any vowel. }

\begin{ltabulary}{|P|P|P|P|P|P|}
\hline 

 \emph{\textbf{Shu }u }& Week &  \emph{Ka \textbf{isha }}& Company &  \emph{O \textbf{cha }}& Tea \\ \cline{1-6}

 \emph{\textbf{Cho }osa }& Investigation &  \textbf{Che }ro & Cello &  \emph{\textbf{She }fu }& Chef \\ \cline{1-6}

 \emph{Sho \textbf{oko }}& Proof &  \emph{Shi \textbf{ma }}↓ & Island &  \emph{\textbf{Ji }ko }& Accident \\ \cline{1-6}

 \emph{\textbf{Ka }ji }& Housework\slash fire &  \emph{Jo \textbf{okyoo }}& Situation &  \emph{\textbf{Ju }kyoo }& Confucianism \\ \cline{1-6}

\end{ltabulary}

\par{\textbf{Pronunciation Note }: The Japanese \slash j\slash  is like the j-sound in "seizure." \slash dj\slash , on the other hand, sounds more like the j-sound in "judge."  \slash j\slash  and \slash dj\slash  are merged completely as [dj] for most speakers, with [j] becoming a less common pronunciation. [j] is usually not used at the start of words or before the moraic nasal (see below). }
      
\section{Long Consonants}
 
\par{ Consonants may be lengthened in Japanese just like vowels. When you make a long consonant, the sound is perceived as sounding harder. The length of time you use to pronounce it increases from one mora to somewhere in between one and two morae. However, speakers conceptualize long consonants as being two morae. }

\par{ The consonants that are typically doubled in Japanese are non-voiced consonants. These consonants include \slash p\slash , \slash k\slash , \slash t\slash , \slash s\slash , \slash sh\slash , \slash ch\slash , and \slash ts\slash . As far as transcribing them is concerned, they will be written as \slash pp\slash , \slash kk, \slash tt\slash , \slash ss\slash , \slash ssh\slash , \slash tch\slash , and \slash tts\slash  respectively. }

\begin{ltabulary}{|P|P|P|P|P|P|P|P|}
\hline 
  
 \emph{Shi \textbf{ppai }}& Failure &  \emph{\textbf{Ma }tchi }& A match &  \emph{Yo \textbf{kka }}& Four days & \emph{Za \textbf{sshi }}& Magazine \\ \cline{1-8}

 \emph{Ha \textbf{ppa }}& Leaf\slash leaves &  \emph{\textbf{Ko }kka }& Nation &  \emph{Shu \textbf{ppatsu }}& Departure &  \emph{Ha \textbf{ssoo }}& Conception \\ \cline{1-8}

 \emph{Ka \textbf{sso }oro }& Runway &  \emph{\textbf{Sa }tchi }& Inferring &  \emph{S \textbf{akka }}& Author &  \emph{\textbf{Sa }kkaa }& Soccer \\ \cline{1-8}

\end{ltabulary}

\par{\textbf{Usage Note }: Voiced consonants are only voiced in a handful of loanwords from other languages, but even then they're usually pronounced as their long unvoiced counterparts. }
      
\section{The Moraic Nasal}
 
\par{ There is a special consonant in Japanese called the "moraic nasal." It counts as a mora on its own. Although usually transcribed as an "n" in some fashion, its pronunciation varies depending on the environment. In its basic understanding, it is what's called a uvular "n" that is best transcribed as \slash N\slash . The uvula is back in the mouth, but when you pronounce it, the mouth constricts as if you were producing a regular \slash n\slash , which makes it sound more like the \slash n\slash  you're used to hearing but not quite. }

\par{ This sound has a lot of allophones because it assimilates (becomes more similar) with the sound that follows. Because things can get quite complicated, we'll go over each situation separately with plenty of examples along the way. In Standard Japanese, this sound can't start words, but it is still quite complicated. }

\begin{itemize}

\item Pronounced as [m]. 
\end{itemize}

\par{ When \slash N\slash  is before a \slash p\slash , \slash b\slash , or \slash m\slash , it becomes [m]. This means that \slash m\slash  can in fact be a doubled with the aid of \slash N\slash . }

\begin{ltabulary}{|P|P|P|P|P|P|}
\hline 

 \emph{Sa \textbf{mpo }}& A walk &  \emph{Shi \textbf{mpai }}& Worry &  \emph{Ka \textbf{mpeki }}& Perfect \\ \cline{1-6}

 \emph{\textbf{Bi }mboo }& Destitute &  \emph{\textbf{Ka }mbu }& Executive &  \emph{\textbf{Se }mbei }& Rice cracker \\ \cline{1-6}

 \emph{Sa \textbf{mmyaku }}& Mountain range &  \emph{Ta \textbf{mmatsu }}& Device &  \emph{\textbf{Chi }mmi }& Delicacy \\ \cline{1-6}

\end{ltabulary}

\begin{itemize}

\item Pronounced as [n]. 
\end{itemize}
 When \slash N\slash  is before \slash t\slash , \slash d\slash , \slash n\slash , \slash r\slash , it becomes [n]. When before \slash t\slash , \slash d\slash , and \slash n\slash , the blade of the tongue is used, and when before \slash r\slash , the tip of the tongue is used, but in any case, the natural pronunciation of these consonants will result in you pronounce [n] correctly. Incidentally, by using this sound change, \slash n\slash  can become long. \hfill\break
\hfill\break

\begin{ltabulary}{|P|P|P|P|P|P|}
\hline 

 \emph{\textbf{So }ntoku }& Loss and gain &  \emph{Se \textbf{ntaku }}& Choice\slash laundry &  \emph{\textbf{Ka }ntoo }& The Kanto Region \\ \cline{1-6}

 \emph{Ko \textbf{ndate }}& \textbf{ }Menu &  \emph{Shi \textbf{ndai }}& Bed &  \emph{Ka \textbf{ndoo }}& Impression \\ \cline{1-6}

 \emph{Mi }\textbf{\emph{nna }↓ }& Everyone &  \emph{\textbf{Te }nnai }& Inside store &  \emph{Te \textbf{nno }o }& Emperor \\ \cline{1-6}

 \emph{Shi \textbf{nrai }}& Faith &  \emph{\textbf{Ka }nri }& Management &  \emph{\textbf{Shi }nri }& Mentality \\ \cline{1-6}

\end{ltabulary}

\begin{itemize}

\item Pronounced as [ng]. 
\end{itemize}
 When \slash N\slash  is before \slash k\slash  and \slash g\slash , it becomes [ng]. Because \slash n\slash  is pronounced the same way in English under these circumstances, this [ng] will be spelled as "n" for simplicity.  
\begin{ltabulary}{|P|P|P|P|P|P|}
\hline 

 \emph{\textbf{Shi }nka }& Evolution &  \emph{Ka \textbf{nkaku }}& Feeling &  \emph{Sa \textbf{nka }}& Participation \\ \cline{1-6}

 \emph{\textbf{Ki }ngyo }& Gold fish &  \emph{Ka \textbf{ngo }}& Sino-Japanese word & \emph{Ka \textbf{ng }ae }& Idea \\ \cline{1-6}

\end{ltabulary}

\begin{itemize}

\item Pronounced as [ny]. \hfill\break

\end{itemize}
 When before \slash ch\slash  or \slash dj, it is pronounced in the same place of the mouth as these consonants. This technically makes it an [ny] with the "y" only indicating that it's palatal. Remember that \slash ch\slash  and \slash dj\slash  are pronounced by temporarily stopping the flow of air at the gum line in the mouth. This is where [ny] is produced. If a vowel were to follow \slash ny\slash --separate from the [ny] of \slash N\slash  being discussed now--you'd hear the "y." Since the "y" here is only meant to indicate where in the mouth this variant is pronounced, it will simply be denoted as "n."  
\begin{ltabulary}{|P|P|P|P|P|P|}
\hline 

 \emph{Shi \textbf{nchuu }}& Brass &  \emph{Ta \textbf{nchi }ki }& Detector &  \emph{Sa \textbf{nchoo }}& Summit \\ \cline{1-6}

 \emph{Ka \textbf{nja }}& Patient &  \emph{Ta \textbf{njo }obi }& Birthday &  \emph{Shi \textbf{njuku }}& Shinjuku \\ \cline{1-6}

\end{ltabulary}

\par{\textbf{Transcription Note }: Typically, \slash dj\slash  is spelled as "j" since \slash j\slash  is largely pronounced as [dj]. }

\begin{itemize}

\item Pronounced as [ũ]. 
\end{itemize}

\par{ When before vowels, \slash y\slash , \slash w\slash , \slash s\slash , \slash sh\slash , \slash z\slash , \slash h\slash , and \slash f\slash , \slash N\slash  sounds like a nasal vowel from the back of the mouth. At any rate, the vowel before \slash N\slash  is always nasalized, but when \slash N\slash  is followed by a vowel, all you may hear is a really nasal vowel and then the following vowel. Typically, this \slash N\slash  is usually just a very nasal ũ. Although this is usually spelled as "n" for simplicity, it'll be spelled as "ũ" below. }

\begin{ltabulary}{|P|P|P|P|P|P|}
\hline 

 \emph{\textbf{Ta }ũ'i }& Unit &  \emph{Ko \textbf{ũwaku }}& Perplexity &  \emph{\textbf{De }ũsha }& Train \\ \cline{1-6}

 \emph{Ka \textbf{ũzei }}& Tariff &  \emph{Ki \textbf{ũyuu }}& Finance &  \emph{\textbf{Ka }ũsai }& The Kansai Region \\ \cline{1-6}

\end{ltabulary}

\par{\textbf{Pronunciation Notes }: }

\par{1. When before \slash z\slash , some speakers pronounced \slash N\slash  as [n]. \hfill\break
2. \slash  \emph{\textbf{De }ũ }\emph{sha }\slash  may also be pronounced as [ \emph{d }\emph{e \textbf{ũ }}\emph{sha }]. }

\begin{itemize}

\item Pronounced as [N]\slash [m]\slash [ũ]. \hfill\break

\end{itemize}

\par{  At the end of words, \slash N\slash 's default pronunciation is [N]. However, there are plenty of speakers that pronounce it like a nasal vowel as seen above in this position. In singing, it will even be pronounced as [m]. This is actually the case for any instance of \slash N\slash  in singing. For the purpose of this section, [N] will be written below as "N." }

\begin{ltabulary}{|P|P|P|P|P|P|}
\hline 

 \emph{Ni \textbf{ho }N }& Japan & \emph{ \textbf{Ka }N }& Can &  \emph{\textbf{Ho }N }& Book \\ \cline{1-6}

\end{ltabulary}

\par{ Many of these sound changes should come naturally since the English \slash n\slash  undergoes similar changes under the same environments. However, you should not replace \slash N\slash  with the English \slash n\slash  except in the situations described above. Of course, regardless of the situation, remember to pronounce it as its own mora. }

\begin{center}
Next Lesson \textrightarrow  第3課: \emph{Hiragana }ひらがな   
\end{center}
    