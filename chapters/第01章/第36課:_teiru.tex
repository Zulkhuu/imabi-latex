    
\chapter{~ている}

\begin{center}
\begin{Large}
第36課: ~ている 
\end{Large}
\end{center}
 
\par{ When used with the particle て, いる functions as a supplementary verb. In Japanese a supplementary verb is a \emph{verb that loses some or all of its literal connotations to serve (a) specific grammatical purpose(s) }. Although it retains some resemblance to its basic meaning of indicating state, ~ている should be treated separately from いる. }
      
\section{~ている}
 
\par{ Correctly interpreting ~ている depends on the verb being used with it. Therefore, pay attention to the kinds of verbs used for each meaning introduced in this lesson. "Kind" here does not refer to how the verb conjugates, but rather what it means semantically and its relation to verbs of similar meaning. }

\begin{center}
 \textbf{-ing }
\end{center}

\par{ The first usage of ~ている equates to "-ing." You're doing something, thus it is a continuation in the present time. This is also linked to ongoing action, which is typically expressed with verbs of process--食べる、飲む、走る. }

\par{1. ${\overset{\textnormal{たろう}}{\text{太朗}}}$ は朝ご飯を食べています。 \hfill\break
Taro is eating breakfast. }

\par{2. ${\overset{\textnormal{きょうし}}{\text{教師}}}$ をしています。 \hfill\break
I am a teacher. }

\par{\textbf{Phrase Note }: Remember that using ~をする in this manner shows profession and is more appropriate in this situation than です. }

\begin{center}
 \textbf{Habit }
\end{center}

\par{ Then there are instances when the action is not literally being done now, but it's a habit of some sort. }

\par{3. A学校に行っています。 \hfill\break
I go to School A. }

\par{4. ${\overset{\textnormal{てんさい}}{\text{天才}}}$ はいつも勉強に ${\overset{\textnormal{う}}{\text{打}}}$ ち ${\overset{\textnormal{こ}}{\text{込}}}$ んでいる。 \hfill\break
Geniuses are always diving into studies. }

\begin{center}
 \textbf{State of Being }
\end{center}

\par{ When used with verbs like 着る (to wear), it shows a state of being dressed. This is in contrast to putting clothing on, which has to be expressed differently to avoid ambiguity as 着ている ${\overset{\textnormal{さいちゅう}}{\text{最中}}}$ . Other verbs are just like this. }

\par{5. ネクタイが ${\overset{\textnormal{まが}}{\text{曲}}}$ っている。 \hfill\break
His necktie is tangled. }

\par{6. ${\overset{\textnormal{ふるぎ}}{\text{古着}}}$ を着ています。 \hfill\break
I'm wearing old clothes. }

\begin{center}
 \textbf{State: Motion Verbs }
\end{center}

\par{ For verbs of motion like 行く and 帰る, it shows state of having done that movement. Interpret it as a completed action and the result being the state in effect. }

\par{7. 彼女は東京に来ています。 \hfill\break
She has come to Tokyo. }

\par{8. 彼はもう帰っている。 \hfill\break
He's already gone home. }

\begin{center}
\textbf{More Examples }
\end{center}

\par{9. 赤い顔をしている。 \hfill\break
To have a red face. }

\par{10. 彼女は長い ${\overset{\textnormal{かみ}}{\text{髪}}}$ をしている。 \hfill\break
She has long hair. }

\par{11. 私は東京 ${\overset{\textnormal{えき}}{\text{駅}}}$ の近くに住んでいます。 \hfill\break
I live near Tokyo Station. }

\par{${\overset{\textnormal{}}{\text{12. 彼}}}$ は ${\overset{\textnormal{かい}}{\text{会}}}$ ${\overset{\textnormal{ちょう}}{\text{長}}}$ をしていた。 \hfill\break
He had been the chairman. }

\par{\textbf{Nuance Note }: 会長 is a chairman of an organization; ${\overset{\textnormal{ぎちょう}}{\text{議長}}}$ is a chairman of an assembly. }

\par{13. お ${\overset{\textnormal{かあ}}{\text{母}}}$ さんによく ${\overset{\textnormal{に}}{\text{似}}}$ ています。 \hfill\break
You resemble your mother well. }

\par{\textbf{Word Note }: お母さん is used instead of 母 because the speaker is referencing the listener's mother. }

\par{14. その ${\overset{\textnormal{はし}}{\text{橋}}}$ は石でできている。 \hfill\break
The bridge is made of stone. }

\par{15. この机は ${\overset{\textnormal{こわ}}{\text{壊}}}$ れています。 \hfill\break
This desk is broken. }

\par{16. ${\overset{\textnormal{さとう}}{\text{砂糖}}}$ はもう ${\overset{\textnormal{はい}}{\text{入}}}$ っています。 \hfill\break
Sugar has already been put in. }

\par{17. 今日も ${\overset{\textnormal{おだ}}{\text{穏}}}$ やかなお天気が続いていますね。 \hfill\break
The moderate weather is continuing today, isn't it? }

\par{18. ${\overset{\textnormal{こや}}{\text{小屋}}}$ は山へ ${\overset{\textnormal{めん}}{\text{面}}}$ している。 \hfill\break
The lodge faces the mountain. }

\par{19. 彼は通りをのそのそと歩き続けていた。 \hfill\break
He continued to flop along the street. }

\par{20. その教科書は ${\overset{\textnormal{しょがくしゃ}}{\text{初学者}}}$ に ${\overset{\textnormal{てき}}{\text{適}}}$ しています。 \hfill\break
The textbook is suitable for beginners. }

\par{21. お ${\overset{\textnormal{ふろ}}{\text{風呂}}}$ はもう ${\overset{\textnormal{わ}}{\text{沸}}}$ いていますか。 \hfill\break
Is the bath hot yet? }

\par{21. その ${\overset{\textnormal{とけい}}{\text{時計}}}$ は ${\overset{\textnormal{ご}}{\text{5}}}$ ${\overset{\textnormal{ふん}}{\text{分}}}$ ${\overset{\textnormal{}}{\text{ほど}}}$ ${\overset{\textnormal{すす}}{\text{進}}}$ んでいます。 \hfill\break
The clock is five minutes fast. }

\par{22. 私は車を持っています。 \hfill\break
I own a vehicle. }

\par{23a. 木になっていたリンゴを集めた。〇 \hfill\break
23b. 木にあっていたリンゴを集めた。X \hfill\break
23c. 木にあったリンゴを集めた。?? \hfill\break
I gathered the apples that were on the tree. }

\par{\textbf{Word Note }: This なる is 生る, which means "to bear fruit". So, this sentence more literally reads "I gathered the apples that ripened on the tree". If you were to say the third line, it sounds like the apple is somehow out of place inside a tree. It definitely isn't talking about picking apples from an apple tree. }

\par{24. 町は ${\overset{\textnormal{たに}}{\text{谷}}}$ に ${\overset{\textnormal{いち}}{\text{位置}}}$ している。 \hfill\break
The town lies in the valley. }

\par{25. 山がそびえている。 \hfill\break
The mountain towers above (everything). }

\par{\textbf{漢字 Note }: そびえる in 漢字 is 聳える, but you don't need to know this spelling for now. }

\par{26. ${\overset{\textnormal{つか}}{\text{疲}}}$ れています。 \hfill\break
I'm tired. }

\par{27. 明治大学で ${\overset{\textnormal{ほうりつ}}{\text{法律}}}$ を勉強しています。 \hfill\break
I am studying law at Meiji University. }

\par{28. 今晩 ${\overset{\textnormal{あ}}{\text{空}}}$ いている ${\overset{\textnormal{へや}}{\text{部屋}}}$ はありますか。 \hfill\break
Do you have any vacant rooms this evening? }

\par{29. この ${\overset{\textnormal{きんがく}}{\text{金額}}}$ は ${\overset{\textnormal{そうごうほけん}}{\text{総合保険}}}$ を ${\overset{\textnormal{ふく}}{\text{含}}}$ んでいますか。 \hfill\break
Does this price include fully comprehensive insurance? }

\par{30. 彼は ${\overset{\textnormal{でんりゅう}}{\text{電流}}}$ を ${\overset{\textnormal{なが}}{\text{流}}}$ している。 \hfill\break
He's passing an electric current. }

\par{31. 通りは込んでいる。 \hfill\break
The road is crowded. }

\par{\textbf{Attribute Note }: When 込んでいる is an attribute, it's often just 込んだ. Also, you wouldn't use 込む to describe Tokyo or Japan. You could say 東京はどこへ行っても込んでいる, which means "Tokyo is crowded wherever you go". }

\par{\textbf{Spelling Note }: This usage of the verb 込む can also be spelled as 混む. }

\par{32. ${\overset{\textnormal{そんざい}}{\text{存在}}}$ している。 ? }

\par{\textbf{Phrase Note }: The above phrase is only used when telling for how long something has existed. }

\par{33. ${\overset{\textnormal{かず}}{\text{数}}}$ では ${\overset{\textnormal{あっとう}}{\text{圧倒}}}$ している ${\overset{\textnormal{じょうたい}}{\text{状態}}}$ だ。 \hfill\break
As far as numbers are concerned, the situation is overwhelming. }

\begin{center}
\textbf{~ている Negation }
\end{center}

\par{The negative is ~ていない, but ~ず(に)いる and ~ないでいる mean "without\dothyp{}\dothyp{}\dothyp{}-ing". ~ずにいる is used in more formal, poetic-like speech. }

\par{34. 覚えていません。 \hfill\break
I don't remember. }

\par{35. ${\overset{\textnormal{ぼく}}{\text{僕}}}$ は何もしていません。(男性語) \hfill\break
I'm not doing anything. }

\par{36. ${\overset{\textnormal{けっ}}{\text{決}}}$ して病気にならないでいることは ${\overset{\textnormal{ふかのう}}{\text{不可能}}}$ だ。 \hfill\break
It is impossible to never get sick. }

\par{37. 彼はいつも ${\overset{\textnormal{お}}{\text{落}}}$ ち ${\overset{\textnormal{つ}}{\text{着}}}$ かないでいる。 \hfill\break
He is always ill at ease. }

\par{38. 彼女は ${\overset{\textnormal{かよ}}{\text{通}}}$ えないでいる。 \hfill\break
She has not been able to go to school. }

\par{\textbf{Grammar Note }: 通える is the potential form of the verb 通う. }

\begin{center}
\textbf{Contractions }
\end{center}

\par{ ~ている is usually contracted to ~てる in casual conversation. Even in polite speech, it is commonplace to hear ~てます instead of ~ています. However, in truly polite situations such as being in an interview, it is NG (エヌジー = No good) to use such contractions as humanly possible. }

\par{\textbf{Dialect Note }: In other regions of Japan, you will hear ~とる or even ~ちょる instead of ~てる. These are both contractions of ~ておる, which in Standard Japanese is the plain humble form of ~ている. However, in the dialects these variants are used, they are treated as the standard non-honorific form for conversation purposes. }

\par{39. 今の、聞いてましたか。(ちょっとくだけた話し言葉) \hfill\break
Were you listening to what I was saying just now? }

\par{40. 父は私が何を勉強してるか知らない。 \hfill\break
My dad doesn't know what I am studying. }

\par{41. 動いてる! \hfill\break
It's moving. }

\par{\textbf{Word Note }: 動く is "to move " as in to physically move about, not "to move to a different house". That meaning of the English verb "to move" is carried out by the verb 引っ ${\overset{\textnormal{こ}}{\text{越}}}$ す. }

\par{\textbf{States \& Appearances }\textbf{状態・ }\textbf{${\overset{\textnormal{}}{\text{様子}}}$ }}

\par{ In the chart below, several verbs are shown in different forms. For each row, the same verb is used, but for each column, the grammar pattern being used is different. }

\begin{ltabulary}{|P|P|P|}
\hline 

~た + Noun & V+ている & V+た \\ \cline{1-3}

${\overset{\textnormal{わ}}{\text{割}}}$ れた ${\overset{\textnormal{たまご}}{\text{卵}}}$  \hfill\break
A broken egg. &  ${\overset{\textnormal{}}{\text{卵が}}}$ 割れている。 \hfill\break
The egg is broken. & 卵が割れた。 \hfill\break
The egg broke. \\ \cline{1-3}

やせた ${\overset{\textnormal{すがた}}{\text{姿}}}$ の \hfill\break
A slim figure & 姿がやせている。 \hfill\break
To have a slim figure & 姿がやせた。 \hfill\break
Figure got skinny. \\ \cline{1-3}

 ${\overset{\textnormal{ふと}}{\text{太}}}$ った彼 \hfill\break
He, who is fat & 彼は太っている。 \hfill\break
He is fat. & 彼は太った。 \hfill\break
He got fat. \\ \cline{1-3}

 ${\overset{\textnormal{あな}}{\text{穴}}}$ が ${\overset{\textnormal{あ}}{\text{開}}}$ いたポケット \hfill\break
A pocket with a hole & ポケットに穴が開いている \hfill\break
There's a hole in my pocket & ポケットに穴が開いた。 \hfill\break
A hole opened up in the pocket. \\ \cline{1-3}

 ${\overset{\textnormal{ゆが}}{\text{歪}}}$ んだ ${\overset{\textnormal{みかた}}{\text{見方}}}$  \hfill\break
A distorted viewpoint & 見方が歪んでいる。 \hfill\break
Your viewpoint is distorted. &  \\ \cline{1-3}

 ${\overset{\textnormal{くさ}}{\text{腐}}}$ った ${\overset{\textnormal{はし}}{\text{橋}}}$ の \hfill\break
A rotten bridge \hfill\break
腐っている橋 \hfill\break
A rotting\slash rotten bridge & 橋が腐っている。 \hfill\break
The bridge is rotting\slash rotten. & 橋が腐った。 \hfill\break
The bridge rotted. \\ \cline{1-3}

 ${\overset{\textnormal{へこ}}{\text{凹}}}$ んだドア \hfill\break
A dented door & ドアが凹んでいる。 \hfill\break
The door is dented. & ドアが凹んだ。 \hfill\break
The door got dented. \\ \cline{1-3}

 ${\overset{\textnormal{こお}}{\text{凍}}}$ った川 \hfill\break
A frozen river & 川が凍っている。 \hfill\break
The river is frozen\slash freezing. & 川が凍った。 \hfill\break
The river froze. \\ \cline{1-3}

 ${\overset{\textnormal{かわ}}{\text{乾}}}$ いた ${\overset{\textnormal{すな}}{\text{砂}}}$ の \hfill\break
Dry sand & 砂が乾いている。 \hfill\break
The sand is dry\slash drying. & 砂が乾いた。 \hfill\break
The sand dried. \\ \cline{1-3}

ひびが入った ${\overset{\textnormal{かべ}}{\text{壁}}}$ の \hfill\break
Cracked wall & 壁にひびが入っている。 \hfill\break
There are cracks in the wall. & 壁にひびが入った。 \hfill\break
Cracks have gotten in the wall. \\ \cline{1-3}

曲がった ${\overset{\textnormal{ほそみち}}{\text{細道}}}$ の \hfill\break
A twisted narrow path & 細道が曲がっている。 \hfill\break
The narrow path is twisted. &  \\ \cline{1-3}

 ${\overset{\textnormal{か}}{\text{欠}}}$ けた ${\overset{\textnormal{ちゃわん}}{\text{茶碗}}}$  \hfill\break
A chipped teacup & 茶碗が欠けている。 \hfill\break
The teacup is chipped. & 茶碗が欠けた。 \hfill\break
The teacup got chipped. \\ \cline{1-3}

\end{ltabulary}

\par{\textbf{Grammar Notes }: }

\par{1. ~た is often preferred over ~ている when used as an attribute. You can substitute it and still create a grammatical phrase, but 90\% of the time, ~た is used. }

\par{2. Context determines whether ~ている is the ${\overset{\textnormal{しんこうけい}}{\text{進行形}}}$ (progressive) or the ${\overset{\textnormal{かんりょうけい}}{\text{完了形}}}$ (perfect tense). }

\par{\textbf{漢字 Note }: やせる has 漢字 spellings 痩せる・瘦せる・瘠せる, and they are listed in usefulness. But, you're not responsible for any of them. They all show up in books, though. ひび has the spelling 罅, which although cool looking, is not very useful to remember. }
    