    
\chapter{The Particle Ga が I}

\begin{center}
\begin{Large}
第11課: The Particle Ga が I: The Subject Marker Ga が  
\end{Large}
\end{center}
 
\par{ As mentioned in Lesson 8, particles indicate the function of what they attach to has in a sentence. Just as there are many functions a word can have in a sentence, there are also many particles. Each particle is complex with its own grammatical rules. }

\par{ Particles are akin to the prepositions of English. In English, prepositions are words that indicate what function the word that follows has in the sentence. }

\par{i. The pen \textbf{in } \emph{the drawer }is yours. \hfill\break
ii. The bird \textbf{on }\emph{the fence }is an endangered species. \hfill\break
iii. The statue \textbf{at } \emph{the park }is brand-new. \hfill\break
iv. He went \textbf{to }\emph{Japan  }\textbf{with } \emph{his other half }. \hfill\break
v. I fought \textbf{for } \emph{freedom }. }

\par{ Particles, however, are post-positions. This means they go after what they modify instead of before. Furthermore, there are functions that some particles have that may not have an English equivalent. Each word in bold below is a particle. }

\par{vi. ${\overset{\textnormal{かれ}}{\text{彼}}}$ が ${\overset{\textnormal{しお}}{\text{塩}}}$ と ${\overset{\textnormal{こしょう}}{\text{胡椒}}}$ だけで ${\overset{\textnormal{やさい}}{\text{野菜}}}$ や ${\overset{\textnormal{にく}}{\text{肉}}}$ などを ${\overset{\textnormal{あじつ}}{\text{味付}}}$ けした。 \hfill\break
 \emph{Kare- \textbf{ga }shio- \textbf{to }koshō- \textbf{dake }- \textbf{de }yasai- \textbf{ya }niku- \textbf{nado }- \textbf{wo }ajitsukeshita. }\hfill\break
Gloss: He-subject marker salt-and pepper-only-with vegetables-such as meat-et cetera-object marker seasoned. \hfill\break
Translation: He seasoned (the) vegetables, meat, etc. with only salt and pepper. }

\par{\emph{ Ga }が and \emph{wa }は—written as \slash ha\slash  but always pronounced as \slash wa\slash  — are very different particles, but they are nonetheless very difficult to distinguish in the most basic of sentences. }

\par{\emph{ Ga }が is a case particle. A \textbf{case particle }is \emph{used to mark grammatical case }. The purpose of \textbf{grammatical case }is \emph{to explicitly state the grammatical function of the noun phrase it attaches to in relation to the predicate }. A  \textbf{predicate } \emph{can be a copular verb, adjective, adjectival noun, or a verb }. }

\par{\textbf{Definition Notes }: }
1. An \textbf{adjective }in Japanese \emph{is a word that describes a state which has its own conjugations }. \hfill\break
2. An \textbf{adjectival noun }in Japanese \emph{is a word that describes a state like an adjective, but it requires the copula to be part of the predicate like a noun }. \hfill\break
3. A \textbf{verb }in Japanese \emph{is a word that describes an action, state, or occurrence. Its conjugations are distinct from those of adjectives, but the principles of conjugation are the same }. 
\par{\emph{ Ga }が marks the \textbf{subject }— \emph{person\slash thing that performs an action (with verbs) or is what exhibits a certain state (with adjectives\slash adjectival nouns) }. By doing so, it is implied that the listener(s) are receiving new information, potentially even the speaker. Contrary to generic statements, it is the objective voice needed in making neutral statements as well as answering questions with the information the asker seeks. }

\par{\emph{ Wa }は, unlike \emph{ga }が, is not a case particle. It is a special kind of particle called a \textbf{bound particle }: \emph{its purpose is lived out by the comment that follows }, which means it is not restricted by what comes before it. \emph{Wa }は is bound to the comment that follows. In return, the comment dictates the function of \emph{wa }は. The only thing the listener can know is that \emph{wa }は marks the topic of the discussion to come. Its motto is “I don\textquotesingle t know about other things, but as for X…” \hfill\break
 \hfill\break
 The complexity of \emph{ga }が and \emph{wa }は doesn\textquotesingle t end here, though. Due to the complexity of the matter at hand, this discussion will be split into two lessons. The first lesson will focus on the fundamentals of \emph{ga }が, and the second lesson on the fundamentals of \emph{wa }は. }

\par{\textbf{Curriculum Note }: This lesson requires that we look at grammatical items which haven\textquotesingle t been fully covered. This includes adjectives, adjectival nouns, verbs and their conjugations, as well as other particles. As such, your goal should be to focus only on the particles \emph{ga }が and \emph{wa }は. Anything aside from the particle \emph{ga }が and what has been taught up to this point can be safely put to the side for now. }
      
\section{Vocabulary List (Under Construction)}
       
\section{The Case Particle Ga が}
 
\par{ The purpose of marking the subject ( \emph{shukaku }主格) of a sentence in Japanese is to indicate information that is newly registered to the speaker, and that information is thus being distilled to the listener(s) as \textbf{new information }. This distinction helps \emph{ga }が serve as an objective means of making \textbf{neutral statements }and \textbf{providing answers to questions }, as well as \textbf{asking direct questions }such as “what is…?” or “who is…?” }

\par{1. \textbf{New information }}

\par{ Whereas the purpose of \emph{wa }は is to topicalize something and bring attention to the comment that follows, the particle \emph{ga }が is used mostly to present new information in the form of neutral statements. This is especially true with statements regarding the existence of something, the five senses, and simple intransitive sentences. Intransitive sentences involve an intransitive verb. These verbs only concern a subject and a predicate, which makes the particle \emph{ga }が the perfect particle as the basic particle for such a grammatical relation. }

\begin{center}
\textbf{i. Existential Sentences }
\end{center}

\par{ Existential sentences are those that state something exists. Typically, these sentences include information such as location. In English, the subject of an existential sentence is “there” and the item that exists ends up being treated as an object. }

\par{vii. \textbf{There }is a \emph{dog }in the yard. \hfill\break
viii. \textbf{There }are \emph{oranges }on the table. \hfill\break
ix. There isn\textquotesingle t a dragon here. \hfill\break
x. There aren\textquotesingle t any pens in the room. }

\par{ In Japanese existential sentences, the thing that exists is treated as the subject. Furthermore, the “to be” verb for showing existence is carried out by two verbs. \emph{Aru } ある is used to express existence of (non-living) inanimate objects whereas \emph{iru } いる i s used to express living animate objects. }

\par{1. ${\overset{\textnormal{あめ}}{\text{飴}}}$ がある。 \hfill\break
 \emph{Ame ga aru. \hfill\break
 }There is candy. }

\par{2. ${\overset{\textnormal{えんぴつ}}{\text{鉛筆}}}$ がある。 \hfill\break
 \emph{Empitsu ga aru. \hfill\break
 }There is\slash are pencil(s). }

\par{3. ${\overset{\textnormal{とり}}{\text{鳥}}}$ がいる。 \hfill\break
 \emph{Tori ga iru. \hfill\break
 }There is\slash are (a) bird(s). }

\par{4. ${\overset{\textnormal{うし}}{\text{牛}}}$ がいる。 \hfill\break
 \emph{Ushi ga iru. \hfill\break
 }There is\slash are (a) cow(s). }

\par{5. ${\overset{\textnormal{さかな}}{\text{魚}}}$ が\{ある・いる\}。 \hfill\break
 \emph{Sakana ga [aru\slash iru]. \hfill\break
}There is\slash are (a) fish. }

\par{\textbf{Sentence Note }: When the verb aru ある is used, “fish” is being treated as a food item that is no longer living. When the verb \emph{iru }いる is used, the fish is still alive and well. }

\par{ The subject\textquotesingle s \textbf{location }is marked with the particle \emph{ni }に. In English, this role may be expressed with “in,” “on,” or no preposition at all. In Japanese, the subject doesn\textquotesingle t have to be the first thing stated. In fact, because anything topicalized with \emph{wa }は always takes precedence, it\textquotesingle s not even true that the subject is usually stated first. In this same token, location phrases \emph{usually }take precedence in existential sentences. }

\par{6. あそこに ${\overset{\textnormal{がっこう}}{\text{学校}}}$ がある。 \hfill\break
 \emph{Asoko ni gakkō ga aru. \hfill\break
 }There is a school over there. }

\par{7. ${\overset{\textnormal{へや}}{\text{部屋}}}$ に ${\overset{\textnormal{ねこ}}{\text{猫}}}$ がいる。 \hfill\break
 \emph{Heya ni neko ga iru. \hfill\break
 }There is\slash are (a) cat(s) in the room. }

\par{8. ${\overset{\textnormal{つくえ}}{\text{机}}}$ の ${\overset{\textnormal{うえ}}{\text{上}}}$ に ${\overset{\textnormal{ほん}}{\text{本}}}$ がある。 \hfill\break
 \emph{Tsukue no ue ni hon ga aru. \hfill\break
 }There is\slash are a book(s) on top of the table. }

\par{9. テーブルの ${\overset{\textnormal{した}}{\text{下}}}$ にネズミがいる。 \hfill\break
 \emph{Tēburu no shita ni nezumi ga iru. \hfill\break
 }There is\slash are (a) mouse\slash mice underneath the table. }

\par{10. ${\overset{\textnormal{はし}}{\text{橋}}}$ の ${\overset{\textnormal{となり}}{\text{隣}}}$ に ${\overset{\textnormal{たき}}{\text{滝}}}$ がある。 \hfill\break
 \emph{Hashi no tonari ni taki ga aru. \hfill\break
 }There is a waterfall next to the bridge. }

\begin{center}
\textbf{ii. Neutral Statements }
\end{center}

\par{ Neutral statements are those that describe temporary states and\slash or actions. They form the objective truth of the recent past, the now, or the near future. The most cited example of this usage of the particle \emph{ga }が, however, happens to be Ex. 11. Monkey business is taken seriously in grammar. }

\par{11. ${\overset{\textnormal{さる}}{\text{猿}}}$ が ${\overset{\textnormal{き}}{\text{木}}}$ から ${\overset{\textnormal{お}}{\text{落}}}$ ちた。 \hfill\break
 \emph{Saru ga ki kara ochita. \hfill\break
 }A monkey fell from tree. \hfill\break
Alternatively: It is the monkey that fell from the tree (See Usage 2). }

\par{\textbf{Particle Note }: The particle \emph{kara }から is the "from" of the sentence. }

\par{12. ${\overset{\textnormal{ひんしつ}}{\text{品質}}}$ がいい。 \hfill\break
 \emph{Hinshitsu ga ii. \hfill\break
 }The quality is good. }

\par{13. ${\overset{\textnormal{にっしょく}}{\text{日食}}}$ が ${\overset{\textnormal{お}}{\text{起}}}$ きます。 \hfill\break
\emph{Nisshoku ga okimasu. \hfill\break
 }There will be a solar eclipse. }

\par{14. ( ${\overset{\textnormal{かれ}}{\text{彼}}}$ は) ${\overset{\textnormal{れいぎ}}{\text{礼儀}}}$ が ${\overset{\textnormal{わる}}{\text{悪}}}$ い。 \hfill\break
 \emph{(Kare wa) reigi ga warui. \hfill\break
 }His manners are bad. \hfill\break
Literally: As for him, (his) manners are bad. }

\par{15. (あなたは) ${\overset{\textnormal{あたま}}{\text{頭}}}$ がいい。 \hfill\break
 \emph{(Anata wa) atama ga ii. \hfill\break
 }You\textquotesingle re smart. \hfill\break
Literally: As for you, your mind is good. }
\textbf{iii. Five senses \hfill\break
\hfill\break
} Another facet of expressing new information\slash neutral statements is creating statements regarding the five senses: sight, sound, smell, taste, and touch. \hfill\break
 \hfill\break
16. ${\overset{\textnormal{さむけ}}{\text{寒気}}}$ がする。 \hfill\break
 \emph{Samuke ga suru. \hfill\break
 }I\textquotesingle m chilly. 
\par{17. ${\overset{\textnormal{くさ}}{\text{臭}}}$ い ${\overset{\textnormal{にお}}{\text{匂}}}$ いがする。 \hfill\break
 \emph{Kusai nioi ga suru. \hfill\break
 }There\textquotesingle s an awful smell. }

\par{18. ${\overset{\textnormal{へん}}{\text{変}}}$ な ${\overset{\textnormal{おと}}{\text{音}}}$ がする。 \hfill\break
 \emph{Hen na oto ga suru. \hfill\break
 }There\textquotesingle s a strange noise. }

\par{19. ${\overset{\textnormal{やま}}{\text{山}}}$ が ${\overset{\textnormal{み}}{\text{見}}}$ える。 \hfill\break
 \emph{Yama ga mieru. \hfill\break
 }The mountain\slash mountains are visible. }

\par{20. ${\overset{\textnormal{はごた}}{\text{歯応}}}$ えがいい。 \hfill\break
 \emph{Hagotae ga ii. \hfill\break
 }The feel (of the food) is good. }

\par{21. ${\overset{\textnormal{しおから}}{\text{塩辛}}}$ い ${\overset{\textnormal{あじ}}{\text{味}}}$ がする。 \hfill\break
\emph{Shiokarai aji ga suru. }\hfill\break
It tastes salty. }
\textbf{i }\textbf{v. Intransitive sentences }
\par{ One of the most practical applications of expressing new information is speaking about what happens, is happening, or has happened. Intransitive verbs are verbs that, put simply, discuss what happens. }

\par{22. ${\overset{\textnormal{ゆき}}{\text{雪}}}$ が ${\overset{\textnormal{つ}}{\text{積}}}$ もる。 \hfill\break
 \emph{Yuki ga tsumoru. \hfill\break
 }Snow accumulates. }

\par{\textbf{Grammar Note }: The speaker is seeing the event occur before his eyes. }

\par{23. ${\overset{\textnormal{つよ}}{\text{強}}}$ い ${\overset{\textnormal{かぜ}}{\text{風}}}$ が ${\overset{\textnormal{ふ}}{\text{吹}}}$ きました。 \hfill\break
 \emph{Tsuyoi kaze ga fukimashita. \hfill\break
 }Strong wind blew. }

\par{24. ${\overset{\textnormal{あめ}}{\text{雨}}}$ が ${\overset{\textnormal{ふ}}{\text{降}}}$ ります。 \hfill\break
 \emph{Ame ga furimasu. \hfill\break
 }It\textquotesingle s going to rain. \hfill\break
Literally: Rain will fall. }

\par{25. ドアが ${\overset{\textnormal{し}}{\text{閉}}}$ まります! \hfill\break
 \emph{Doa ga shimarimasu! \hfill\break
 }The door is (about to) close! }

\par{26. ${\overset{\textnormal{たいふう}}{\text{台風}}}$ が ${\overset{\textnormal{じょうりく}}{\text{上陸}}}$ しました。 \hfill\break
\emph{Taifū ga jōriku shimashita. \hfill\break
}The\slash a typhoon landed. }

\par{2. \textbf{Exhaustive-listing: It is X that… }}

\par{ There are times when \emph{ga }が isn\textquotesingle t meant as a mere statement of new information. Instead, it can also explicitly state that it is “X” that is the subject of the predicate. The “X” can be one entity or several entities, which is where the name “exhaustive-listing” comes into play. When the predicate describes a static state, one that is not necessarily a temporary reality, this interpretation is typically meant. A static state can be expressed with a copular sentence, adjectives, adjectival nouns, or verbs which describe states. In fact, this interpretation reigns supreme over the existential sentences studied above. With \emph{ga }が, the things mentioned to exist in a certain place are what\textquotesingle s there. }

\par{27. ${\overset{\textnormal{かれ}}{\text{彼}}}$ が ${\overset{\textnormal{がくせい}}{\text{学生}}}$ です。 \hfill\break
 \emph{Kare ga gakusei desu. \hfill\break
 }He is the student. }

\par{28. この ${\overset{\textnormal{きょうかしょ}}{\text{教科書}}}$ が ${\overset{\textnormal{べんり}}{\text{便利}}}$ です。 \hfill\break
 \emph{Kono kyōkasho ga benri desu. \hfill\break
 }This is the textbook that is useful. }

\par{29. ${\overset{\textnormal{なみ}}{\text{波}}}$ が ${\overset{\textnormal{たか}}{\text{高}}}$ い! \hfill\break
 \emph{Nami ga takai! \hfill\break
 }These waves are high! }

\par{30. このサンマのほうが ${\overset{\textnormal{はごた}}{\text{歯応}}}$ えが ${\overset{\textnormal{よわ}}{\text{弱}}}$ い。 \hfill\break
 \emph{Kono samma no hō ga hagotae ga yowai. \hfill\break
}The consistency of \emph{this }Pacific saury is weak. }

\par{\textbf{Grammar Note }: The use of \emph{no hō }のほう (side of a comparison) intensifies the exhaustive nature of \emph{ga }が. Whenever there are two \emph{ga }が phrases next to each other like this, the first \emph{ga }が phrase is always treated as the subject of the main clause. The secondary \emph{ga }が phrase is embedded in the predicate. }

\begin{center}
\textbf{ii. Asking Questions }
\end{center}

\par{ Exhaustive-listing is a feature of \emph{ga }が that is not normally brought out without cause. Meaning, just as is the case for the English equivalents seen in translation, such phrasing is usually brought about some sort of question being asked, for which a direct and substantive answer is required. Unsurprisingly, \emph{ga }が is involved in the making and answering of those questions. To ask the direct questions, you add \emph{ga }が to an interrogative (question word). The basic question words in Japanese are as follows: }

\par{・ \emph{Dare }誰 \hfill\break
・ \emph{Nani }何 \hfill\break
・ \emph{Itsu }\slash  \emph{Nanji }いつ・何時 \hfill\break
・ \emph{Doko }どこ \hfill\break
・ \emph{Naze }何故 \hfill\break
 \hfill\break
\textbf{Meaning Note }: \emph{Nanji }何時 literally means “what time?” }

\par{31. どこが ${\overset{\textnormal{びょういん}}{\text{病院}}}$ ですか。 \hfill\break
 \emph{Doko ga byōin desu ka? \hfill\break
Where }is the hospital? }

\par{\textbf{Sentence Note }: This sentence is not a simple question about where the hospital is. Imagine a person looking at a line of buildings and wondering which is the hospital. That is a situation where this sentence would be appropriate. Although not as smooth of a translation, Ex. 31 can also be interpreted as “Where is it that the hospital is?” }

\par{32. ${\overset{\textnormal{なぜ}}{\text{何故}}}$ ここに ${\overset{\textnormal{ゆうれい}}{\text{幽霊}}}$ が ${\overset{\textnormal{そんざい}}{\text{存在}}}$ するんですか。 \hfill\break
 \emph{Naze koko ni yūrei ga sonzai suru n desu ka? \hfill\break
 }Why is it that ghosts exist here? }

\par{\textbf{Grammar Note }: In polite speech, "why" questions must end in \emph{n desu ka? }んですか. }

\par{33. ${\overset{\textnormal{なに}}{\text{何}}}$ がおかしい!? \hfill\break
 \emph{Nani ga okashii!? \hfill\break
 }What (is it that) is so funny!? }

\par{34a. ${\overset{\textnormal{だれ}}{\text{誰}}}$ が ${\overset{\textnormal{しゃちょう}}{\text{社長}}}$ ? \emph{ \hfill\break
Dare ga shachō? }\hfill\break
34b. ${\overset{\textnormal{しゃちょう}}{\text{社長}}}$ は ${\overset{\textnormal{だれ}}{\text{誰}}}$ ? \hfill\break
 \emph{Shachō wa dare? \hfill\break
Who }\textquotesingle s the company president? (34a) \hfill\break
Who is the company president? (34b) }

\par{\textbf{Grammar Note }: Ex. 34a would be appropriate to say when you are somewhere where there is a group of people, one of which you would like identified as the company president by who you\textquotesingle re asking the question to. Ex. 34b, on the other hand, would be used in a situation where the company president is already at the forefront of conversation and the speaker, you, is simply asking the listener about who that person is. This conversation doesn\textquotesingle t have to be held where the company president happens to be at. }

\par{35. ${\overset{\textnormal{あした}}{\text{明日}}}$ は\{いつ・ ${\overset{\textnormal{なんじ}}{\text{何時}}}$ \}が ${\overset{\textnormal{つごう}}{\text{都合}}}$ がいいですか。 \hfill\break
 \emph{Ashita wa [itsu\slash nanji] ga tsugō ga ii desu ka? \hfill\break
 }As for tomorrow, when is convenient (for you)? }

\begin{center}
\textbf{iii. Answers to Questions }
\end{center}

\par{ Questions brought about with \emph{ga }が are typically answered back with the information sought. \emph{Ga }が provides an exhaustive answer to the question at hand. }

\par{36. 「 ${\overset{\textnormal{だれ}}{\text{誰}}}$ が ${\overset{\textnormal{い}}{\text{行}}}$ く?」「 ${\overset{\textnormal{ぼく}}{\text{僕}}}$ が ${\overset{\textnormal{い}}{\text{行}}}$ きます。」 \hfill\break
 \emph{“Dare ga iku?” “Boku ga ikimasu.” \hfill\break
 }“Who\textquotesingle s the one going?” “I\textquotesingle m the one going.” }

\par{37. 「何がいい?」「ラーメンがいいでしょう。」 \hfill\break
 \emph{“Nani ga ii?” “Rāmen ga ii deshō.” \hfill\break
 }“What would be good.” “Ramen would be good.” }

\begin{center}
\textbf{iv. Spontaneous Reply }
\end{center}

\par{ Whenever someone spontaneously utters something, it is often in reference to some immediate concern. }

\par{38. この ${\overset{\textnormal{くすり}}{\text{薬}}}$ が ${\overset{\textnormal{き}}{\text{効}}}$ くよ。 \hfill\break
 \emph{Kono kusuri ga kiku yo. }\hfill\break
This medicine will work. }

\par{\textbf{Sentence Note }: Suppose you find out a friend has a cold and you have some cold medicine on you. The moment you hear about your friend's condition, you take out the medicine and say this'll help him. This is one way Ex. 38 could be used. }

\par{39. お ${\overset{\textnormal{きゃく}}{\text{客}}}$ さんが ${\overset{\textnormal{き}}{\text{来}}}$ た。 \hfill\break
 \emph{O-kyaku-san ga kita. }\hfill\break
Customer(s) are here. }

\par{\textbf{Sentence Note }: You're the owner of a restaurant. It's nearing lunch hour and at last you hear the first guest(s) entering. Just as you hear this, you utter Ex. 39. }

\begin{center}
\textbf{v. Sense of Discovery }
\end{center}

\par{ Another application of the exhaustive-listing interpretation of \emph{ga }が  is expressing surprise in discovery what something truly is. This application translates as “X is what Y is…” This usage is essentially the same as the one for expressing a spontaneous reply. }

\par{40. あ、これが ${\overset{\textnormal{ゆき}}{\text{雪}}}$ だ! \hfill\break
 \emph{A, kore ga yuki da! \hfill\break
 }Ah, this is what snow is! }

\par{41. あ、あの ${\overset{\textnormal{ひと}}{\text{人}}}$ が ${\overset{\textnormal{うわさ}}{\text{噂}}}$ の ${\overset{\textnormal{やまだししょう}}{\text{山田師匠}}}$ だ! \hfill\break
 \emph{A, ano hito ga uwasa no Yamada-shish }\emph{ō da! \hfill\break
 }Ah, \emph{that person }is the rumored Master Yamada! }

\par{4. \textbf{Object Marker with Stative-Transitive Predicates }}

\par{ Having already learned quite a lot about how \emph{ga }が functions as a subject-marker, we will study its function as an object-marker in Lesson 21. }

\par{${\overset{\textnormal{もんだい}}{\text{問題}}}$ がある。 \hfill\break
There is a problem. }

\par{お ${\overset{\textnormal{かね}}{\text{金}}}$ がある。 \hfill\break
There is money; to have money. }

\par{${\overset{\textnormal{けいかん}}{\text{警官}}}$ がいる。 \hfill\break
There is a police officer. }
    