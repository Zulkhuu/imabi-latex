    
\chapter{The Particle Mo も}

\begin{center}
\begin{Large}
第22課: The Particle Mo も 
\end{Large}
\end{center}
 
\par{ Although this won't be the last time that you learn about the particle \emph{mo }も, this lesson will introduce you to its most important usages. }
      
\section{The Adverbial Particle Mo も}
 
\par{ The adverbial particle \emph{mo }も follows nouns to mean "also\slash too." Similarly to what happens when these two English words are used heavily in conversation, the particle \emph{mo }も often helps soften the tone of a sentence. }

\par{ This particle must never be used immediately after the particles \emph{ga }が or \emph{wa }は. Whenever it is after a noun that functions as a subject and\slash or topic, these particles are thought of as simply not being spoken. Usually, the particle \emph{mo }もis not used after the particle wo either; however, you will see the combination “ \emph{wo mo }をも” in contexts such as older literature. }

\par{${\overset{\textnormal{てんき}}{\text{1. (お)天気}}}$ もいいですね。 \hfill\break
 \emph{O-tenki mo ii desu ne. \hfill\break
 }The weather's good, too. }

\par{2. ${\overset{\textnormal{ぼく}}{\text{僕}}}$ もお ${\overset{\textnormal{なか}}{\text{腹}}}$ が ${\overset{\textnormal{す}}{\text{空}}}$ いた。 \hfill\break
 \emph{Boku mo onaka ga suita. \hfill\break
 }I too am hungry. }

\par{4. ${\overset{\textnormal{とうきょう}}{\text{東京}}}$ も ${\overset{\textnormal{きょうと}}{\text{京都}}}$ も ${\overset{\textnormal{あめ}}{\text{雨}}}$ です。 \hfill\break
 \emph{Tōkyō mo Kyoto mo ame desu. }\emph{\hfill\break
 }There's rain in Tokyo and in Kyoto. }

\par{5. ${\overset{\textnormal{せいこう}}{\text{成功}}}$ も ${\overset{\textnormal{せいこう}}{\text{成功}}}$ 、 ${\overset{\textnormal{だいせいこう}}{\text{大成功}}}$ だ。 \hfill\break
 \emph{Seikō mo seikō, daiseikō da. }\emph{\hfill\break
 }What a success, it was a great success. }

\par{ Similarly to the contrastive \emph{wa }は, the particle \emph{mo }も may also emphasize sheer lack in negative sentences, or sheer intensity in positive sentences. }

\par{6. あのコンピューターは ${\overset{\textnormal{ご}}{\text{5}}}$ ${\overset{\textnormal{まんえん}}{\text{万円}}}$ もかからない。 \hfill\break
 \emph{Ano kompyūtā wa goman\textquotesingle en mo kakaranai. }\emph{\hfill\break
 }That computer doesn't even cost 50,000 yen. }

\par{3. ${\overset{\textnormal{いっ}}{\text{一}}}$ センチも ${\overset{\textnormal{うご}}{\text{動}}}$ かない。 \hfill\break
 \emph{Issenchi mo ugokanai. }\emph{\hfill\break
 }To not move even a centimeter. }

\par{7. ${\overset{\textnormal{ごじかん}}{\text{五時間}}}$ も ${\overset{\textnormal{ま}}{\text{待}}}$ った。 \hfill\break
 \emph{Gojikan mo matta. }\emph{\hfill\break
 }I waited at least\slash about five hours. }
 
\par{\textbf{Particle Note }: In Ex. 7, \emph{mo }も here implies perhaps a much longer wait, that is, you've already waited "at least five hours" and the wait has become unreasonable. This same logic can also explain Ex. 5. Here, it's implied that the cost of such a computer doesn't even exceed 50,000 yen. Since this equates roughly to \$500, we can imagine that any much higher than that would be an exuberant price without the quality of the PC being far higher than standard expectations. In both cases, \emph{mo }も is seen after some counter phrase. }

\par{${\overset{\textnormal{こうぼう}}{\text{弘法}}}$ にも ${\overset{\textnormal{ふで}}{\text{筆}}}$ の ${\overset{\textnormal{あやま}}{\text{誤}}}$ り。 \hfill\break
Anyone can make a mistake. (Proverb) }

\par{\textbf{Culture Note }: 弘法 was an outstanding master at writing, but even he made mistakes. }

\par{${\overset{\textnormal{お}}{\text{老}}}$ いも ${\overset{\textnormal{し}}{\text{若}}}$ きも ${\overset{\textnormal{かんどう}}{\text{感動}}}$ した。 \hfill\break
Everyone was moved. \hfill\break
Literally: Even the young and old were moved. }
    