    
\chapter*{Yes Phrases}

\begin{center}
\begin{Large}
第??課: Yes Phrases: Hai はい, Hā はあ, Ē  ええ, \& Un うん 
\end{Large}
\end{center}
 
\par{ Way back in Lesson 7, we learned briefly that the words for “yes” and “no” were \emph{hai }はい and \emph{iie }いいえ respectively. However, it couldn\textquotesingle t be further from the truth that this is all you need to know to use these two words correctly, or even to express “yes” and “no” correctly in every circumstance. These words, aside from being literal replies to yes-no questions, can also be used as \emph{aizuchi }相槌 in back-channeling, where one interjects to indicate that one is paying attention. }

\par{ Incidentally, a lot can be said about both the words for “yes” and about the words for “no.” Meaning, there are several ways to go about saying both, and each word brings along with it usages that you may not necessarily think of as an English speaker. In this lesson, we will focus solely on the words that translate as “yes.” }

\par{・ \emph{Hai }はい \hfill\break
・ \emph{Hā }はあ \hfill\break
・ \emph{Ē }ええ \hfill\break
・ \emph{Un }うん }

\par{\textbf{Curriculum Note }: Because these words, at times, require context to be understood, there will be some grammar used that has not been formally introduced. At those points, simply focus on understanding the yes-phrase at hand. }
      
\section{The Usages of Hai はい・Hā はあ}
 
\par{1. The most fundamental usage of \emph{hai }はい is meaning "yes" in  \textbf{answering “yes-no” questions }. }

\par{1. 「はい」を ${\overset{\textnormal{お}}{\text{押}}}$ してください。 \hfill\break
 \emph{“Hai” wo oshite kudasai. \hfill\break
 }Please press “Yes.” }

\par{2. }

\par{${\overset{\textnormal{エイ}}{\text{A}}}$ ${\overset{\textnormal{し}}{\text{氏}}}$ : ${\overset{\textnormal{あした}}{\text{明日}}}$ 、 ${\overset{\textnormal{か}}{\text{買}}}$ い ${\overset{\textnormal{もの}}{\text{物}}}$ に ${\overset{\textnormal{い}}{\text{行}}}$ きますか。 \hfill\break
 ${\overset{\textnormal{ビー}}{\text{B}}}$ ${\overset{\textnormal{し}}{\text{氏}}}$ :はい(、コストコに ${\overset{\textnormal{か}}{\text{買}}}$ い ${\overset{\textnormal{もの}}{\text{物}}}$ (し)に ${\overset{\textnormal{い}}{\text{行}}}$ きます)。 \hfill\break
 \emph{Ei-shi: Ashita, kaimono ni ikimasu ka? \hfill\break
Bii-shi: Hai(, Kosutoko ni kaimono (shi) ni ikimasu). \hfill\break
 }Person A: Are you going to go shopping tomorrow? \hfill\break
Person B: Yes(, I\textquotesingle m going to go shop at Costco). }

\par{ Although simply saying \emph{hai }はい could suffice in answering a yes-no question, explicitly stating an answer is often needed because this isn\textquotesingle t the only function \emph{hai }はい has. Tone, for one, is very important. If one\textquotesingle s tone doesn\textquotesingle t convey affirmation, then \emph{hai }はい won\textquotesingle t likely be interpreted as a simple “yes.” Also, if you are speaking to a superior\slash client where giving a succinct answer is expected, just answering with \emph{hai }はい would be inappropriate. The same could be said in the English-speaking business world if one were to continuously say “yes” without providing anything substantive. }

\par{2. \emph{Hai }はい also shows \textbf{confirmation }, which is in and of itself an offshoot of above. If the question is in the affirmative, \emph{hai }はい confirms the affirmative. However, if it the question is in the negative, \emph{hai }はい confirms the negative. This is unlike English where the latter situation would be answered with “no.” }

\par{3. }

\par{${\overset{\textnormal{エイ}}{\text{A}}}$ ${\overset{\textnormal{し}}{\text{氏}}}$ : ${\overset{\textnormal{ことし}}{\text{今年}}}$ セスは ${\overset{\textnormal{にじゅうよん}}{\text{24}}}$ ${\overset{\textnormal{さい}}{\text{歳}}}$ になりますね。 \hfill\break
 ${\overset{\textnormal{ビー}}{\text{B}}}$ ${\overset{\textnormal{し}}{\text{氏}}}$ :はい(、そうですね)。 \hfill\break
 \emph{Ei-shi: Kotoshi Sesu wa nijūyonsai ni narimasu ne. \hfill\break
Bii-shi: Hai(, sō desu ne). }\hfill\break
Person A: Seth turns 24 this year, right? \hfill\break
Person B: Yes(, that\textquotesingle s right). }

\par{\textbf{Phrase Note }: \emph{Sō desu ne }そうですね and its variants can often be treated as beings synonymous to “yes” and do not need to be used with a yes-word to be used as such. This is no different than the English expression “that\textquotesingle s right.” }

\par{4. }

\par{${\overset{\textnormal{エイ}}{\text{A}}}$ ${\overset{\textnormal{し}}{\text{氏}}}$ : ${\overset{\textnormal{あす}}{\text{明日}}}$ 、 ${\overset{\textnormal{きょうと}}{\text{京都}}}$ へ ${\overset{\textnormal{い}}{\text{行}}}$ かないんですか。 \hfill\break
 ${\overset{\textnormal{ビー}}{\text{B}}}$ ${\overset{\textnormal{し}}{\text{氏}}}$ :はい、 ${\overset{\textnormal{い}}{\text{行}}}$ きませんね。 ${\overset{\textnormal{たぶんことし}}{\text{多分今年}}}$ は ${\overset{\textnormal{い}}{\text{行}}}$ かないと ${\overset{\textnormal{おも}}{\text{思}}}$ います。 \hfill\break
 \emph{Ei-shi: Asu, Kyōto e ikanai n desu ka? \hfill\break
Bii-shi: Hai, ikimasen ne. Tabun kotoshi wa ikanai to omoimasu. }\hfill\break
Person A: Are you not going to Kyoto tomorrow? \hfill\break
Person B: No, I\textquotesingle m not. I probably won\textquotesingle t go this year. }

\par{5. }

\par{${\overset{\textnormal{さちこ}}{\text{紗智子}}}$ : ${\overset{\textnormal{やす}}{\text{休}}}$ むんですか。 \hfill\break
 ${\overset{\textnormal{やすひこ}}{\text{泰彦}}}$ :はい、ちょっと ${\overset{\textnormal{きゅう}}{\text{急}}}$ な ${\overset{\textnormal{ようじ}}{\text{用事}}}$ がありまして。 \hfill\break
 \emph{Sachiko: Yasumu n desu ka? \hfill\break
Yasuhiko: Hai, chotto kyū na yoji ga arimashite. } \hfill\break
Sachiko: Are you taking the day off? \hfill\break
Yasuhiko: Yes, I have an urgent thing to attend to. }

\par{6. }

\par{${\overset{\textnormal{どうりょう}}{\text{同僚}}}$ ${\overset{\textnormal{エイ}}{\text{A}}}$ : ${\overset{\textnormal{い}}{\text{行}}}$ かないんですか。 \hfill\break
 ${\overset{\textnormal{どうりょう}}{\text{同僚}}}$ ${\overset{\textnormal{ビー}}{\text{B}}}$ :はい、 ${\overset{\textnormal{い}}{\text{行}}}$ きません。 \hfill\break
 \emph{Dōryō Ei: Ikanai n desu ka?” \hfill\break
Dōryō Bii: Hai, ikimasen. }\hfill\break
Colleague A: Are you not going? \hfill\break
Colleague B: No, I\textquotesingle m not going. }

\par{3. \emph{Hai }はい may also show agreement to a request, in which case it is accompanied with an affirmative tone. This can also be viewed as an offshoot of answering to a yes-no question, only with “no” not being an option. }

\par{7. }

\par{${\overset{\textnormal{かちょう}}{\text{課長}}}$ :このファイルを ${\overset{\textnormal{つじ}}{\text{辻}}}$ さんに ${\overset{\textnormal{てんそう}}{\text{転送}}}$ してください。 \hfill\break
 ${\overset{\textnormal{しゃいん}}{\text{社員}}}$ :はい(、 ${\overset{\textnormal{しょうち}}{\text{承知}}}$ しました)。 \hfill\break
 \emph{Kachō: Kono fairu wo Tsuji-san ni tensō shite kudasai. \hfill\break
Shain: Hai(, shōchi shimashita). }\hfill\break
Section Manager: Please forward these files to Tsuji-san. \hfill\break
Employee: Yes(, understood). }

\par{8. はい、わかりました。 \hfill\break
 \emph{Hai, wakarimashita. \hfill\break
 }Understood. }

\par{\textbf{Phrase Note }: The phrases for "understood" are listed below in order of how polite\slash humble they are. }

\begin{ltabulary}{|P|P|}
\hline 

Most Humble &  \emph{Kashikomarimashita }畏まりました \\ \cline{1-2}

Very Humble &  \emph{Shōchi itashimashita }承知いたしました \\ \cline{1-2}

Humble &  \emph{Shōchi shimashita }承知しました \\ \cline{1-2}

Polite (To those equal or below oneself) &  \emph{Ryōkai shimashita }了解しました \\ \cline{1-2}

Polite (General-Use) &  \emph{Wakarimashita }わかりました \\ \cline{1-2}

Polite (To equals) &  \emph{Ryōkai desu }了解です \\ \cline{1-2}

Casual &  \emph{Wakatta }わかった・ \emph{Ryōkai }了解 \\ \cline{1-2}

\end{ltabulary}

\par{4. \emph{Hai }はい is the go-to phrase when responding to someone calling (for) you, both in person and on the phone. }

\par{9. }

\par{${\overset{\textnormal{ふじた}}{\text{藤田}}}$ : ${\overset{\textnormal{かねこ}}{\text{金子}}}$ さん。 \hfill\break
 ${\overset{\textnormal{かねこ}}{\text{金子}}}$ :はい。 \hfill\break
 ${\overset{\textnormal{ふじた}}{\text{藤田}}}$ :こちらに ${\overset{\textnormal{き}}{\text{来}}}$ てもらっていいですか。 \hfill\break
 ${\overset{\textnormal{かねこ}}{\text{金子}}}$ :はい、 ${\overset{\textnormal{なに}}{\text{何}}}$ ですか。 \hfill\break
 \emph{Fujita: Kaneko-san. \hfill\break
Kaneko: Hai. \hfill\break
Fujita: Kochira ni kite moratte ii desu ka? }\hfill\break
Fujita: Kaneko-san. \hfill\break
Kaneko: Yes. \hfill\break
Fujita: Could you please come here? \hfill\break
 \hfill\break
10. }

\par{${\overset{\textnormal{ははおや}}{\text{母親}}}$ :もしもし。 ( ${\overset{\textnormal{でんわ}}{\text{電話}}}$ ) \hfill\break
 ${\overset{\textnormal{むすめ}}{\text{娘}}}$ :はい。あ、お ${\overset{\textnormal{かあ}}{\text{母}}}$ さん! \hfill\break
 \emph{Hahaoya: Moshimoshi. (Denwa) \hfill\break
Musume: Hai. A, o-kā-san! }\hfill\break
Mother: Hello. (Phone) \hfill\break
Daughter: Hello. Oh, mom! }

\par{5. Another usage of \emph{hai }はい is to indicate the start of a joint activity. }

\par{11. では、 ${\overset{\textnormal{ちょうりかいし}}{\text{調理開始}}}$ ですね、はい。 \hfill\break
 \emph{De wa, chōri kaishi desu ne, hai. \hfill\break
 }Well then, cooking begins. Yes. }

\par{12. これをあのベンチに ${\overset{\textnormal{お}}{\text{置}}}$ いときましょう。ちょっと ${\overset{\textnormal{おも}}{\text{重}}}$ いので ${\overset{\textnormal{き}}{\text{気}}}$ をつけましょう。じゃ、いち、にいの、さん、はい! \hfill\break
 \emph{Kore wo ano benchi ni oitokimashō. Chotto omoi node ki wo tsukemashō. Ja, ichi, nii no, san, hai! \hfill\break
 }Let\textquotesingle s place this on the bench over there. It\textquotesingle s a little heavy, so let\textquotesingle s be careful. Alright, one, two, three, here we go! }

\par{6. One usage of \emph{hai }はい that is far removed from the English concept of “yes” is indicating to the speaker that one is listening. The purpose here is not to interrupt who is talking, which is how it may seem to an English speaker. Once the other person is done speaking, however, simply responding with \emph{hai }はい would be inappropriate as its interpretation would default to this usage. }

\par{13. }

\par{${\overset{\textnormal{エイ}}{\text{A}}}$ ${\overset{\textnormal{し}}{\text{氏}}}$ : ${\overset{\textnormal{つぎ}}{\text{次}}}$ のグラフを ${\overset{\textnormal{み}}{\text{見}}}$ てもらうと。 \hfill\break
 ${\overset{\textnormal{ビー}}{\text{B}}}$ ${\overset{\textnormal{し}}{\text{氏}}}$ :はい。 \hfill\break
 ${\overset{\textnormal{エイ}}{\text{A}}}$ ${\overset{\textnormal{し}}{\text{氏}}}$ : ${\overset{\textnormal{わ}}{\text{分}}}$ かると ${\overset{\textnormal{おも}}{\text{思}}}$ いますが。 \hfill\break
 ${\overset{\textnormal{ビー}}{\text{B}}}$ ${\overset{\textnormal{し}}{\text{氏}}}$ :はい。 \hfill\break
 \emph{Ei-shi: Tsugi no gurafu wo mite morau to. \hfill\break
Bii-shi: Hai. \hfill\break
Ei-shi: Wakaru to omoimasu ga. \hfill\break
Bii-shi: Hai. }\hfill\break
Person A: If I have you look at the next graph. \hfill\break
Person B: Mm. \hfill\break
Person A: I believe you\textquotesingle ll understand, but… \hfill\break
Person B: Mm. }

\par{ It may not always be the case that one is the intended listener when using \emph{hai }はい in this fashion. This is especially the case in TV programs where one person may be directing comments to the audience while another person is to the side nodding off and making interjections as the person speaks. This is done largely to encourage engagement from the audience. }

\par{ The applications of this usage thus far have been harmless, but it can also be used to tell the speaker you\textquotesingle ve heard enough. }

\par{14. はいはい、もう ${\overset{\textnormal{わ}}{\text{分}}}$ かってますよ。 \hfill\break
 \emph{Hai hai, mō wakattemasu yo. \hfill\break
 }Yes, yes, I know already. }

\par{ With a change in intonation to that of a question, \emph{hai }はい can be used when you would otherwise indicate that you\textquotesingle re listening to imply that you\textquotesingle re shocked by what is said—simultaneously asking that the speaker repeat oneself. }

\par{15.はい? \hfill\break
 \emph{Hai? }\hfill\break
Come again? }

\par{7. Responding to \emph{sō desu ka? }そうですか is an important function of \emph{hai }はい. }

\par{16. \hfill\break
 \hfill\break
 ${\overset{\textnormal{りょうこ}}{\text{涼子}}}$ :きのう、しらこばと ${\overset{\textnormal{すいじょうこうえん}}{\text{水上公園}}}$ に ${\overset{\textnormal{い}}{\text{行}}}$ きましたよ。 \hfill\break
 ${\overset{\textnormal{さら}}{\text{沙良}}}$ :あ、そうですか? \hfill\break
 ${\overset{\textnormal{りょうこ}}{\text{涼子}}}$ :はい、 ${\overset{\textnormal{すご}}{\text{凄}}}$ く ${\overset{\textnormal{たの}}{\text{楽}}}$ しかったですよ。 ${\overset{\textnormal{ぜひおこな}}{\text{是非行}}}$ ってみてください。 \hfill\break
 \emph{Ryōko: Kinō, Shirakobato Suijō Kōen ni ikimashita yo. \hfill\break
Sara: A, sō desu ka? \hfill\break
Ryōko: Hai, sugoku tanoshikatta desu yo. Zehi itte mite kudasai. }\hfill\break
Ryoko: I went to Shirakobato Water Park yesterday. \hfill\break
Sara: Oh, really? \hfill\break
Ryoko: Yeah, it was really fun. Definitely try going there. }

\par{8. \emph{Hai }はい is often used as an anticipatory answer at the end of a sentence, especially by store clerks. }

\par{17. }

\par{${\overset{\textnormal{きゃく}}{\text{客}}}$ : ${\overset{\textnormal{いま}}{\text{今}}}$ の ${\overset{\textnormal{しゅん}}{\text{旬}}}$ の ${\overset{\textnormal{しょくざい}}{\text{食材}}}$ はなんでしょうか? \hfill\break
 ${\overset{\textnormal{てんちょう}}{\text{店長}}}$ :そうですねえ。 ${\overset{\textnormal{いま}}{\text{今}}}$ の ${\overset{\textnormal{じき}}{\text{時期}}}$ 、 ${\overset{\textnormal{おい}}{\text{美味}}}$ しい ${\overset{\textnormal{さかな}}{\text{魚}}}$ はいっぱいあるんですけど、 ${\overset{\textnormal{とく}}{\text{特}}}$ に ${\overset{\textnormal{おい}}{\text{美味}}}$ しいのはやっぱり ${\overset{\textnormal{たい}}{\text{鯛}}}$ ですね、はい。 \hfill\break
\emph{Kyaku: Ima no shun no shokuzai wa nan deshō ka? \hfill\break
}\emph{Tenchō: So desu nē. Ima no jiki, oishii sakana wa ippai aru n desu kedo, toku ni oishii no wa yappari tai desu ne, hai. \hfill\break
}Customer: What ingredients are in season now? \hfill\break
Shop Manager: Well, there are a lot of delicious fish this season, but the one that\textquotesingle s especially delicious would have to be sea bream. Yes. }

\par{9. \emph{Hai }はい is used when presenting something to someone. This can also be said when arriving somewhere, with the destination being what one is bringing attention to. }

\par{18. はい、 ${\overset{\textnormal{た}}{\text{炊}}}$ き ${\overset{\textnormal{こ}}{\text{込}}}$ みご ${\overset{\textnormal{はん}}{\text{飯}}}$ です。どうぞ。 \hfill\break
 \emph{Hai, takikomi-gohan desu. Dōzo. \hfill\break
 }Here\textquotesingle s mixed rice. Feel free. }

\par{19. はい、こちらはスターバックスコーヒー ${\overset{\textnormal{いち}}{\text{1}}}$ ${\overset{\textnormal{ごうてん}}{\text{号店}}}$ でございます。 \hfill\break
 \emph{Hai, kochira wa Sutābakkusu Kōhii Ichigōten de gozaimasu. \hfill\break
 }Now right here is Starbucks Coffee\textquotesingle s first store. }

\par{20. はい、 ${\overset{\textnormal{とうちゃく}}{\text{到着}}}$ しました。 \hfill\break
 \emph{Hai, tōchaku shimashita. \hfill\break
 }Alright, we\textquotesingle ve arrived. }

\par{10. \emph{Hai }はい is frequently used right as one is giving instructions. }

\par{21. はい、スタート! \hfill\break
 \emph{Hai, sutāto! \hfill\break
 }Start! }

\par{22. はい、 ${\overset{\textnormal{れいぞうこ}}{\text{冷蔵庫}}}$ に ${\overset{\textnormal{い}}{\text{入}}}$ れて。 \hfill\break
 \emph{Hai, reizōko ni irete. \hfill\break
 }Yeah, put it in the refrigerator. }

\par{11. \emph{Hai }はい is used to indicate to others they stop what they\textquotesingle re doing. }

\par{23. はい、そこまで(だ)! \hfill\break
 \emph{Hai, soko made (da)! \hfill\break
 }Alright, stop there! }

\par{24. }

\par{${\overset{\textnormal{せいと}}{\text{生徒}}}$ ${\overset{\textnormal{エイ}}{\text{A}}}$ : ${\overset{\textnormal{わたし}}{\text{私}}}$ は ${\overset{\textnormal{しょうらいうちゅうひこうし}}{\text{将来宇宙飛行士}}}$ になりたいです! \hfill\break
 ${\overset{\textnormal{せんせい}}{\text{先生}}}$ :はい。じゃ、 ${\overset{\textnormal{けんたくん}}{\text{健太君}}}$ は? \hfill\break
 \emph{Seito Ei: Watashi wa shōrai uchū-hikōshi ni naritai desu! \hfill\break
Sensei: Hai. Ja, Kenta-kun wa? }\hfill\break
Student A: I want to become an astronaut in the future! \hfill\break
Teacher: Alright. Now how about you, Kenta-kun? }

\par{12. \emph{Hai }はい can also be used to get people\textquotesingle s attention. }

\par{25. ${\overset{\textnormal{せんせい}}{\text{先生}}}$ :はい、はい。みな ${\overset{\textnormal{お}}{\text{落}}}$ ち ${\overset{\textnormal{つ}}{\text{着}}}$ いてください。 \hfill\break
 \emph{Sensei: Hai, hai. Mina ochitsuite kudasai. \hfill\break
 }Okay, okay, everyone settle down, please. }

\par{13. \emph{Hai }はい is frequently used in response to being asked for commentary. }

\par{26. }

\par{${\overset{\textnormal{さいとう}}{\text{斉藤}}}$ : ${\overset{\textnormal{かとう}}{\text{加藤}}}$ さん、どう ${\overset{\textnormal{おも}}{\text{思}}}$ いますか。 \hfill\break
 ${\overset{\textnormal{かとう}}{\text{加藤}}}$ :はい、あの、 ${\overset{\textnormal{わたし}}{\text{私}}}$ も ${\overset{\textnormal{はんたい}}{\text{反対}}}$ ですね。 \hfill\break
 \emph{Saitō: Katō-san, dō omoimasu ka? \hfill\break
Katō: Hai, ano, watashi mo hantai desu ne. }\hfill\break
Saito: What do you think, Kato-san? \hfill\break
Kato: Yes, um, I\textquotesingle m also against it. }

\par{14. \emph{Hai }はい is also frequently used after someone makes a comment to get the individual to say more. This can be viewed as an offshoot of Usage 6. }

\par{27. }

\par{${\overset{\textnormal{ひがいしゃ}}{\text{被害者}}}$ : ${\overset{\textnormal{わたし}}{\text{私}}}$ は ${\overset{\textnormal{あお}}{\text{青}}}$ に ${\overset{\textnormal{か}}{\text{変}}}$ わったのを ${\overset{\textnormal{かくにん}}{\text{確認}}}$ してから ${\overset{\textnormal{ちょくしん}}{\text{直進}}}$ に ${\overset{\textnormal{はっしゃ}}{\text{発車}}}$ させました。 \hfill\break
 ${\overset{\textnormal{けんさつかん}}{\text{検察官}}}$ :はい。 \hfill\break
 ${\overset{\textnormal{ひがいしゃ}}{\text{被害者}}}$ :はい、その ${\overset{\textnormal{ご}}{\text{後}}}$ 、 ${\overset{\textnormal{ひだりがわ}}{\text{左側}}}$ から ${\overset{\textnormal{くるま}}{\text{車}}}$ が ${\overset{\textnormal{み}}{\text{見}}}$ えてブレーキを ${\overset{\textnormal{ふ}}{\text{踏}}}$ みましたが、 ${\overset{\textnormal{ま}}{\text{間}}}$ に ${\overset{\textnormal{あ}}{\text{合}}}$ わず ${\overset{\textnormal{じこ}}{\text{事故}}}$ になりました。 \hfill\break
 ${\overset{\textnormal{けんさつかん}}{\text{検察官}}}$ :はい。 \hfill\break
 ${\overset{\textnormal{ひがいしゃ}}{\text{被害者}}}$ :そのとき、 ${\overset{\textnormal{たいこうしゃ}}{\text{対向車}}}$ は ${\overset{\textnormal{いちだい}}{\text{一台}}}$ でした。 \hfill\break
 \emph{Higaisha: Watashi wa ao ni kawatta no wo kakunin shite kara chokushin ni hassha sasemashita. \hfill\break
Kensatsukan: Hai. \hfill\break
Higaisha: Hai, sono ato, hidarigawa kara kuruma ga miete burēki wo fumimashita ga, ma ni awazu jiko ni narimashita. \hfill\break
Kensatsukan: Hai. \hfill\break
Higaisha: Sono toki, taikōsha wa ichidai deshita. }\hfill\break
Injured Party: I went straight in my car after verifying that the light had turned green. \hfill\break
Prosecutor: Continue. \hfill\break
Injured Party: Ok, afterward, I could see a car from the left. I braked but didn\textquotesingle t make it in time, which led to the accident. \hfill\break
Prosecutor: Continue. \hfill\break
Injured Party: At that time, there was one oncoming car. }

\par{15. At times, \emph{hai }はい is simply used to create a rhythm, especially in folk songs. It may also be used as an interjection when pounding \emph{mochi }餅 (sticky rice cake). }

\par{28. はい、はい、はい、はい。 \hfill\break
 \emph{Hai, hai, hai, hai. \hfill\break
 }One, two, one, two. }

\par{\textbf{Variation Note }: There also exists \emph{hā }はあ, which is treated as a simple alteration of \emph{hai }はい that is used by male speakers, but its usage remains exactly the same. }

\par{29. はあ、 ${\overset{\textnormal{しょうち}}{\text{承知}}}$ しました。 \hfill\break
 \emph{Hā, shōchi shimashita. \hfill\break
 }Yes, understood. }

\par{30. はあ! \hfill\break
 \emph{Hā! }\hfill\break
Yes, sir! }

\par{31. はあ、その ${\overset{\textnormal{じしん}}{\text{地震}}}$ は ${\overset{\textnormal{たいへん}}{\text{大変}}}$ でしたね。 \hfill\break
 \emph{Hā, sono jishin wa taihen deshita ne. }\hfill\break
Yeah, that earthquake was terrible, huh. }

\par{32. はあ、なんでしょうか。 \hfill\break
 \emph{Hā, nan deshō ka? \hfill\break
 }Yes, what is it? }

\par{33. はあ、それはそうですが。 \hfill\break
 \emph{Hā, sore wa sō desu ga. }\hfill\break
Yes, that\textquotesingle s true, but… }

\par{34. はあ、 ${\overset{\textnormal{うそ}}{\text{嘘}}}$ でしょう? \hfill\break
 \emph{Hā, uso deshō? } \hfill\break
What, you\textquotesingle ve got to be kidding!? }

\par{35. はあ、しまった! \hfill\break
 \emph{Hā, shimatta! \hfill\break
 }Ah, damn it! \hfill\break
 \hfill\break
\textbf{Usage Note }: One usage that \emph{hā }はあ \emph{ }has that it doesn\textquotesingle t share with \emph{hai }はい is being used as an interjection when one is really in a rut over failing at something, etc. }
      
\section{The Usages of Ē ええ}
 
\par{\emph{ Hai }はい is the most multi-faceted word used in this lesson. It is also the most formal. With that being said, \emph{ē }ええ is not as complicated. Regardless of how it\textquotesingle s used, it is an affirmative response of one\textquotesingle s thought and\slash or emotions, which is why it can at times be perceived as rude if used out of place. Below are various scenarios most suited for using \emph{ē }ええ: }

\par{・To show confidence. \hfill\break
・To make it known that you already know about what\textquotesingle s been talked about. \hfill\break
・Indicative of being older, composed, and being able to affirmatively look back. \hfill\break
・To show elitism. \hfill\break
・To give an at-home feeling to those especially close. \hfill\break
・Responding to audience but without appeal unless mixed together with \emph{hai }はい. \hfill\break
・Seemingly able to talk on and on, indicative of female conversation. }

\par{1. Yes-No Questions }

\par{36. }

\par{${\overset{\textnormal{さえき}}{\text{佐伯}}}$ : ${\overset{\textnormal{しゃちょう}}{\text{社長}}}$ は ${\overset{\textnormal{ながさき}}{\text{長崎}}}$ へ ${\overset{\textnormal{しゅっちょう}}{\text{出張}}}$ するのですか。 \hfill\break
${\overset{\textnormal{さとう}}{\text{佐藤}}}$ :ええ。 \hfill\break
\emph{Saeki: Shachō wa Nagasaki e shutchō suru no desu ka? \hfill\break
Satō: }\emph{Ē. } \hfill\break
Saeki: Is the company director going to Nagasaki on business. \hfill\break
Sato: Yes. }

\par{2. Acknowledgement of Listening }

\par{37. }

\par{${\overset{\textnormal{しょうへい}}{\text{昌平}}}$ :ちょっとお ${\overset{\textnormal{ねが}}{\text{願}}}$ いがあるんですが。 \hfill\break
${\overset{\textnormal{さとし}}{\text{聡}}}$ :ええ(、 ${\overset{\textnormal{なん}}{\text{何}}}$ でしょう)。 \hfill\break
${\overset{\textnormal{しょうへい}}{\text{昌平}}}$ : ${\overset{\textnormal{あ}}{\text{空}}}$ いてるときは ${\overset{\textnormal{ぎんこう}}{\text{銀行}}}$ に ${\overset{\textnormal{つ}}{\text{連}}}$ れていってくれませんか。 \hfill\break
${\overset{\textnormal{さとし}}{\text{聡}}}$ :ええ、 ${\overset{\textnormal{もんだい}}{\text{問題}}}$ ありませんよ。 \hfill\break
\emph{Sh }\emph{ōhei: Chotto o-negai ga aru n desu ga. \hfill\break
Satoshi: Ē(, nan deshō)? \hfill\break
Shōhei: Aiteru toki wa ginkō ni tsurete itte kuremasen ka? \hfill\break
Satoshi: Ē, mondai arimasen yo. }\hfill\break
Shohei: I have a small request. \hfill\break
Satoshi: Yes, what is it? \hfill\break
Shohei: When you\textquotesingle re free, could you take me to the bank? \hfill\break
Satoshi: Yeah, no problem. }

\par{3. Confirmation }

\par{38. ええ、そうですよ。 \hfill\break
\emph{Ē, sō desu yo. \hfill\break
}Yes, that\textquotesingle s right. }

\par{39. ええ、 ${\overset{\textnormal{せいかい}}{\text{正解}}}$ です。 \hfill\break
\emph{Ē, seikai desu. \hfill\break
}Yes, that\textquotesingle s the correct answer. }

\par{40. ええ、よくあることです。 \hfill\break
\emph{Ē, yoku aru koto desu. \hfill\break
}Yes, it happens often. }

\par{41. ええ、 ${\overset{\textnormal{い}}{\text{言}}}$ うとおりだと ${\overset{\textnormal{おも}}{\text{思}}}$ います。 \hfill\break
\emph{Ē, iu tōri da to omoimasu. }}

\par{4. Responding to Request\slash Suggestion: Agreement\slash Sympathy }

\par{42. ええ、ぜひ。 \hfill\break
\emph{Ē, zehi. \hfill\break
}Yes, by all means. }

\par{43. ええ、いい ${\overset{\textnormal{かんが}}{\text{考}}}$ えだと ${\overset{\textnormal{おも}}{\text{思}}}$ いますな。 \hfill\break
\emph{Ē, ii kangae da to omoimasu na. \hfill\break
}Yes, I think that\textquotesingle s a good idea. }

\par{5. Surprise (Low-High Intonation) }

\par{44. ええ、 ${\overset{\textnormal{ほんとう}}{\text{本当}}}$ ですか。 \hfill\break
\emph{Ē, hontō desu ka. \hfill\break
}What, really? }

\par{45. ええ、 ${\overset{\textnormal{うそ}}{\text{嘘}}}$ ! \hfill\break
\emph{Ē, uso! \hfill\break
What, you\textquotesingle re kidding!? }}

\par{46. ええ!? \hfill\break
\emph{Ē!? \hfill\break
}What!? }

\par{\textbf{Usage Note }: This usage does not follow the restrictions outlined at the start of this section. }
      
\section{The Usages of Un うん}
 
\par{ \emph{Un }うん is a very casual means of saying "yes" that should only be used with those who one is very friendly and close with. When behaving as a means of saying "yes," it has the following purposes. }

\par{
\begin{enumerate}

\item Yes-No Questions \hfill\break

\item Acknowledgement of Listening \hfill\break

\item Confirmation \hfill\break

\item Responding to Request\slash Suggestion: Agreement\slash Sympathy \hfill\break

\item Response to a Command 
\end{enumerate}
}

\par{\textbf{Examples }}

\par{47. 「ペンギンは ${\overset{\textnormal{とり}}{\text{鳥}}}$ なの?」「うん(、 ${\overset{\textnormal{とり}}{\text{鳥}}}$ だよ)」 \hfill\break
\emph{“Pengin wa tori na no?” “Un(, tori da yo)” }\hfill\break
“Are penguins birds?” “Yes(, they\textquotesingle re birds).” }

\par{48. }

\par{花子:明日は… \hfill\break
涼子:うん。 \hfill\break
花子:あたしの誕生日なのよ。 \hfill\break
\emph{Hanako: Ashita… \hfill\break
Ryōko: Un. \hfill\break
Hanako: Atashi no tanjōbi na no yo. }\hfill\break
Hanako: Tomorrow… \hfill\break
Ryoko: Yeah. \hfill\break
Hanako: Is my birthday. }

\par{49. いや、そのままでいいんじゃない、うん。 \hfill\break
\emph{Iya, sono mama de ii n ja nai, un. \hfill\break
}“No, it\textquotesingle s fine as is, right? Yeah.” }

\par{50. うんうん、 ${\overset{\textnormal{きみ}}{\text{君}}}$ の ${\overset{\textnormal{い}}{\text{言}}}$ うとおりだね。 \hfill\break
\emph{Un un, kimi no iu tōri da ne. \hfill\break
}Yep, yep, it\textquotesingle s exactly what you said. }

\par{51. うん、その ${\overset{\textnormal{きも}}{\text{気持}}}$ ちわかるよ。 \hfill\break
\emph{Un, sono kimochi wakaru yo. \hfill\break
}Yeah, I get what you\textquotesingle re feeling. }

\par{52. 「ごめんなさい。」「うん、いいんだよ。」 \hfill\break
\emph{“Gomen-nasai.” “Un, ii n da yo.” \hfill\break
}“Sorry.” “No, it\textquotesingle s ok.” }

\par{53. }

\par{${\overset{\textnormal{じゅうぎょういん}}{\text{従業員}}}$ : ${\overset{\textnormal{こいけ}}{\text{小池}}}$ さんこの ${\overset{\textnormal{しゅ}}{\text{種}}}$ の ${\overset{\textnormal{しごと}}{\text{仕事}}}$ に ${\overset{\textnormal{む}}{\text{向}}}$ いてないんですか。 \hfill\break
${\overset{\textnormal{しはいにん}}{\text{支配人}}}$ :うん、 ${\overset{\textnormal{む}}{\text{向}}}$ いてないな。 \hfill\break
\emph{J }\emph{ūgyōin: Koike-san wa kono shu no shigoto ni muitenai n desu ka? \hfill\break
Shihainin: Un, muitenai na. }\hfill\break
Employee: Is Koike-san not suited for this \hfill\break
}

\par{54. うん、そうだよな。はい、そうですね。 \hfill\break
\emph{Un, sō da yo na. Hai, sō desu ne. \hfill\break
}Yeah, that\textquotesingle s right. Yes, you\textquotesingle re definitely right. }

\par{\textbf{Sentence Note }: The first half of this example would be an example of answering one\textquotesingle s own question. Only the latter part would be directed at the listener, as indicated by the difference in speech style. }

\par{55. }

\par{${\overset{\textnormal{ちちおや}}{\text{父親}}}$ :ボール ${\overset{\textnormal{な}}{\text{投}}}$ げて! \hfill\break
${\overset{\textnormal{むすこ}}{\text{息子}}}$ :うん! \hfill\break
\emph{Chichioya: Bōru nagete! \hfill\break
Musuko: Un! }\hfill\break
Father: Throw the ball! \hfill\break
Son: Ok! }

\par{ Unrelated to the meanings of "yes" outlined so far, \emph{un }うん may also be used as onomatopoeia for the following purposes. }

\par{・The sound of bees, horseflies, etc. in great number flying about. \hfill\break
・The faint sound of machinery. \hfill\break
・The sound of someone in anguish. }

\par{56. パソコンがうんうんいってるけど ${\overset{\textnormal{だいじょうぶ}}{\text{大丈夫}}}$ かな? \hfill\break
\emph{Pasokon ga un\textquotesingle un itteru kedo daijōbu kana? }\hfill\break
The computer is sure making noise. I wonder if it\textquotesingle s alright? }

\par{57. ${\overset{\textnormal{いた}}{\text{痛}}}$ みが ${\overset{\textnormal{ひど}}{\text{酷}}}$ くてベッドでうんうんと ${\overset{\textnormal{うな}}{\text{唸}}}$ っていた。 \hfill\break
\emph{Itami ga hidokute beddo de un\textquotesingle un to unatte ita. \hfill\break
}The pain was awful; I was groaning in bed. }
    