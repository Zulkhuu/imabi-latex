    
\chapter{The Particle へ}

\begin{center}
\begin{Large}
第32課: The Particle へ 
\end{Large}
\end{center}
 
\par{ The particle へ, pronounced as "e," is interchangeable with に for one usage. This particle is much easier. }
      
\section{The Case Particle へ}
 
\par{  へ indicates movement to somewhere away from where the subject currently is. For the most part, it is interchangeable with に. However, \textbf{\emph{i }t must not be used to replace に for any other usage other than movement! } \hfill\break
}

\par{ へ is often associated with extravagant distances, or at least movement away from where the subject currently is . However, this is not always the case. As a grammatical rule, へ must never be replaced by に when preceded by の since \textbf{に can't be used with の }. \hfill\break
}

\par{\textbf{Particle Note }: へと is a much stronger variant that emphasizes direction. You will see this mainly in music and literature. }
 
\par{1. 私は ${\overset{\textnormal{おおさか}}{\text{大阪}}}$ へ行きます。 \hfill\break
I'm going to Osaka. }
 
\par{2. 私は明日徳島へ行きます。 \hfill\break
I will go to Tokushima tomorrow. }

\par{3. ${\overset{\textnormal{こうべ}}{\text{神戸}}}$ へ ${\overset{\textnormal{ふね}}{\text{船}}}$ を見に行きました。 \hfill\break
I went to Kobe to see the boats. }
 
\par{4. 彼女はドアのところへ走って行った。 \hfill\break
She ran to the door. }
 
\par{5. よく中国へ行きますか。 \hfill\break
Do you often go to China? }

\par{6. ${\overset{\textnormal{しょくどう}}{\text{食堂}}}$ へ行きませんか。 \hfill\break
Why don't we go to the restaurant? }
 
\par{7. 左へ ${\overset{\textnormal{まが}}{\text{曲}}}$ ってください。 \hfill\break
Please turn left. }
 
\par{8. 川へ ${\overset{\textnormal{と}}{\text{飛}}}$ び ${\overset{\textnormal{こ}}{\text{込}}}$ む。 \hfill\break
To jump in the river. }
 
\par{9. 机の上\{に 〇・へ △・X\} ${\overset{\textnormal{きょうかしょ}}{\text{教科書}}}$ を ${\overset{\textnormal{お}}{\text{置}}}$ く。 \hfill\break
To place a textbook on top of the desk. }
 
\par{11. 学校へ行く ${\overset{\textnormal{とちゅう}}{\text{途中}}}$ だ。 \hfill\break
I'm on my way to school. }

\par{12. ${\overset{\textnormal{へいわ}}{\text{平和}}}$ への ${\overset{\textnormal{あゆ}}{\text{歩}}}$ み \hfill\break
Steps toward peace }
 
\par{13. 私は ${\overset{\textnormal{すしや}}{\text{寿司屋}}}$ へ行きました。 \hfill\break
I went to a sushi restaurant. }
 
\par{14a. よくこの ${\overset{\textnormal{いざかや}}{\text{居酒屋}}}$ へ来るんですか。X \hfill\break
14b. よくこの居酒屋に来るんですか。〇 \hfill\break
Do you often come to this pub? }
 
\par{\textbf{Word Note }: An 居酒屋 is a Japanese style pub where you can order a wide variety of foods and drinks for a low cost. They often have times or special deals for 飲み ${\overset{\textnormal{ほうだい}}{\text{放題}}}$ (all you can drink) and 食べ放題 (all you can eat). }
 
\par{\textbf{Particle Note }: へ shows direction in going away to somewhere. Here, then, doesn't count in the last sentence because the addressee is already there. }
 
\par{15. 彼は ${\overset{\textnormal{ま}}{\text{間}}}$ もなくここへ来ます。 \hfill\break
He will be here shortly. }
 
\par{16. ここへ座って話を ${\overset{\textnormal{き}}{\text{聴}}}$ いてください。 \hfill\break
Come and sit down here and listen. }
 
\par{17. 去年イギリスへ行きました。オランダ(へ)も行きました。 \hfill\break
Last year, I went to England. I also went to Holland. }
 
\par{\textbf{Particle Note }: Notice how へ isn't needed when followed by も. }
 
\par{18. 私はニューヨークを ${\overset{\textnormal{た}}{\text{発}}}$ ってシドニーへ ${\overset{\textnormal{む}}{\text{向}}}$ かいました。 \hfill\break
I left New York for Sydney. }
 
\par{19. 東京へ ${\overset{\textnormal{しゅっちょう}}{\text{出張}}}$ します。 \hfill\break
I will take a business trip to Tokyo. }
 
\par{\textbf{Culture Note }: Presenting business cards, ${\overset{\textnormal{めいし}}{\text{名刺}}}$ , is extremely important to business etiquette. You should present your card with both hands and take it out of a business card box, and you are to receive the other person's card, read it, and say ${\overset{\textnormal{ちょうだい}}{\text{頂戴}}}$ します. When exchanging cards with someone of higher status, you should make sure yours is below the other. You should place business cards in the back of your leather case, and if you are at a table, wait until the meeting is over before putting it in. Don't write on, damage, or fold business cards, at least not in front of the person. }
 
\par{20. 日本へ帰りたい。 \hfill\break
I want to return to Japan. }
 
\par{\textbf{Culture Note }: The above phrase is very weird for a non-Japanese person to say. It's even wrong if someone Japanese is not actually born in Japan. The reason is that 帰る = "to go home."In the case that you want to say that you want to go Japan again, you should say something like 日本へまた行きたいです. Another way to say to return to one's country is ${\overset{\textnormal{きこく}}{\text{帰国}}}$ する.  This phrase is used more when referring to other people going back home. }
 
\par{Practice: Translate the following. }
 
\par{1. I'm going home. }
 
\par{2. This is a present to you. }
 
\par{3. A letter to my mother }
 
\par{4. I'm going to Tokyo. }
 
\par{5. To go to the left. }
      
\section{Key}
 
\par{Practice }

\par{1. 家に帰る。or 家へ帰る。 }

\par{2. これは君へのプレゼントだ。 }

\par{3. 母への手紙; お母さんへの手紙 }

\par{4. 東京に行きます。or 東京へ行きます。 }

\par{5. 左\{に・へ\}行く。 }
    