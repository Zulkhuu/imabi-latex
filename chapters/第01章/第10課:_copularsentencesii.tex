    
\chapter{Copular Sentences II}

\begin{center}
\begin{Large}
第10課: Copular Sentences II: Polite Speech 
\end{Large}
\end{center}
 
\par{ As mentioned in Lesson 9, \textbf{polite speech } \emph{is used in everyday interactions with people who are neither family nor close friends }. Polite speech, at times, can also be spoken in a casual manner, but its purpose is to keep some form of distance\slash formality between the speaker and the listener(s). }
 In polite speech, the part of the sentence that changes the most is the predicate, which is at the end of the sentence. Remember, the predicate simply means the part of sentence that gives some information about the subject. As will once more be the case in this lesson, the predicate will be the copula. This is always the case when the copula is used in an \textbf{independent clause }— \emph{something that can stand alone as a sentence }. Another characteristic of polite speech is that the verbal component—the predicate—is longer. There is a rule of thumb that the longer something is, the politer it is.       
\section{Vocabulary List}
 
\par{\textbf{Nouns }}
 
\par{・ \emph{Nihonjin }日本人 – Japanese person }
 
\par{・ \emph{Ch }\emph{ūgokujin }中国人 – Chinese person }
 
\par{・ \emph{Taiwanjin }台湾人 – Taiwanese person }
 
\par{・ \emph{O-isha-san }お医者さん – Doctor }
 
\par{・Shōgakusei 小学生 – Elementary student }
 
\par{・ \emph{Ch }\emph{ūgakusei }中学生 – Junior high student }
 
\par{・ \emph{K }\emph{ōk }\emph{ōsei }高校生 – High school student }
 
\par{・ \emph{Daigakusei }大学生 – College\slash university student }
 
\par{・ \emph{Giin }議員 – Legislator }
 
\par{・ \emph{Eiy }\emph{ū }英雄 – Hero }
 
\par{・ \emph{Nisemono }偽物 – A fake }
 
\par{・ \emph{Machigai }間違い - Mistake }
 
\par{・ \emph{Jikan }時間 – Time }
 
\par{・ \emph{Hebi }蛇 – Snake }
 
\par{・ \emph{Sotsugy }\emph{ōshiki }卒業式 – Graduation ceremony }
 
\par{・ \emph{Kaig }\emph{ō }会合 – Assembly }
 
\par{・ \emph{Ny }\emph{ūsu }ニュース – News }
 
\par{・ \emph{Tero }テロ – Terrorism }
 
\par{・ \emph{Ōkami }オオカミ – Wolf }
 
\par{・ \emph{Uso }嘘 – Lie }
 
\par{・ \emph{Mogi shiken }模擬試験 – Mock exam }
 
\par{・ \emph{Yasumi }休み – Rest\slash absence\slash holiday }

\par{\textbf{Adverbs }}

\par{・ \emph{S }\emph{ō }そう – So }

\par{・ \emph{M }\emph{ō sugu }もうすぐ – Soon }

\par{\textbf{Interjections }}

\par{・ \emph{Etto }えっと – Uh\slash um }

\par{・ \emph{Hai }はい – Yes }

\par{・ \emph{Iie }いいえ – No }
 ・ \emph{Goendama } 五円玉 – five-yen coin 
\par{・ \emph{J }\emph{ōdan }冗談 – Joke }

\par{・ \emph{Kiseki }奇跡 – Miracle }

\par{・ \emph{Shokud }\emph{ō }食堂 – Diner }

\par{・ \emph{Sensei }先生 – Teacher }

\par{・ \emph{J }\emph{ūmin }住民 – Resident }

\par{・ \emph{O-hiru }お昼 – Lunch }

\par{・ \emph{K }\emph{ōhii }コーヒー – Coffee }

\par{・ \emph{Onigiri }お握り – Onigiri }

\par{・ \emph{Saru }猿 – Monkey }

\par{・ \emph{Hitori }1人 – One person }

\par{・ \emph{Futari }2人 – Two people }

\par{・ \emph{Otoko }男 – Man }

\par{・ \emph{Otoko-no-ko }男の子 – Boy }

\par{・ \emph{Keisatsukan }警察官 – Police officer }

\par{・ \emph{Sh }\emph{ūry }\emph{ōbi }終了日 – End date }

\par{・ \emph{Nichiy }\emph{ōbi }日曜日 – Sunday }

\par{・ \emph{Mokuy }\emph{ōbi }木曜日 – Thursday }

\par{・ \emph{Kin }\emph{ō }昨日 – Yesterday }

\par{・ \emph{Ashita\slash asu } 明日 – Tomorrow }

\par{\textbf{Pronouns }}

\par{・ \emph{Watashi }私 – I }

\par{・ \emph{Kare }彼 – He }

\par{・ \emph{Kanojo }彼女 – She }

\par{・ \emph{Kore }これ – This }

\par{・ \emph{Sore }それ – That }

\par{・ \emph{Are }あれ – That (over there) }

\par{・ \emph{Ano }あの – That (over there) (adj.) }

\par{・ \emph{Soko }そこ – There }

\par{・ \emph{Tanaka-san }田中さん – Mr.\slash Ms. Tanaka }

\par{・ \emph{Oda-san }小田さん – Mr.\slash Ms. Oda }

\par{・ \emph{Rii-san }リーさん  - Mr.\slash Ms. Lee }

\par{・ \emph{Kenta-kun }健太君 – Kenta-kun }
      
\section{Copular Conjugations in Polite Speech}
 
\begin{center}
\textbf{Polite Non-Past Form: \emph{Desu }です }
\end{center}

\par{ In polite speech, the non-past form of the copula is \emph{desu }です. Just like \emph{da }だ, \emph{desu }です can stand for “will be,” “is,” and “are.” Below are examples of the basic noun-predicate sentence in polite speech: “X \emph{wa }は Y \emph{desu }です.” }

\par{\textbf{Pronunciation Note }: In Standard Japanese, the “u” in \emph{desu }です is typically devoiced. It is still perceived as two morae but phonetically rendered as \slash de.s\slash . However, devoicing does not mean dropping the vowel altogether. The mouth is still articulated to form the sound. It\textquotesingle s simply not vocalized at that point. It is important to note that this phenomenon doesn\textquotesingle t occur much outside Eastern Japan. This means that you will hear speakers that fully articulate both “ \emph{de }” and “ \emph{su }.” Lastly, whenever something directly follows \emph{desu }です, the \slash u\slash  becomes fully pronounced. }

\begin{center}
\textbf{Present Tense }
\end{center}

\par{1. ${\overset{\textnormal{かのじょ}}{\text{彼女}}}$ は ${\overset{\textnormal{たいわんじん}}{\text{台湾人}}}$ です。 \hfill\break
 \emph{Kanojo wa Taiwanjin desu. \hfill\break
 }She is Taiwanese. }

\par{2. ${\overset{\textnormal{わたし}}{\text{私}}}$ はアメリカ ${\overset{\textnormal{じん}}{\text{人}}}$ です。 \hfill\break
 \emph{Watashi wa Amerikajin desu. \hfill\break
 }I\textquotesingle m an American. }

\par{3. ${\overset{\textnormal{じかん}}{\text{時間}}}$ です。 \hfill\break
 \emph{Jikan desu. \hfill\break
 }It\textquotesingle s time. }

\par{4. えっと、 ${\overset{\textnormal{にほんじん}}{\text{日本人}}}$ です。 \hfill\break
 \emph{Etto, Nihonjin desu. \hfill\break
 }Um, I\textquotesingle m Japanese. }

\par{\textbf{Grammar Note }: Remember that the subject is often omitted. This isn\textquotesingle t just for “it.” In fact, “I” is frequently not stated in a sentence, so long as it is contextually obvious. }

\par{5. あれは ${\overset{\textnormal{へび}}{\text{蛇}}}$ ですよ。 \hfill\break
 \emph{Are wa hebi desu yo. \hfill\break
 }That is a snake. }

\par{\textbf{Grammar Note }: The particle \emph{yo }よ is added to the end of a sentence to emphasize something you\textquotesingle re trying to bring to someone\textquotesingle s attention. It is implied that the listener doesn\textquotesingle t already know what you\textquotesingle re saying. }

\par{6. はい、そうです。 \hfill\break
 \emph{Hai, sō desu. \hfill\break
 }Yes, that\textquotesingle s right. }

\par{\textbf{Grammar Note }: \emph{Sō }そう is an adverb, not a noun, which literally translates as "so." In English grammar, "so" as in "that is so," is in place of an adjective, but in all other instances of English grammar, it is apparent that it inherently behaves as an adverb (Ex. "This is \emph{so }cool"). In Japanese, words don't change part of speech unless they're able to conjugate. In Japanese, adverbs are incapable of conjugating, just like nouns, which allows most adverbs and nouns to be followed by the copula verb in the same fashion. }

\begin{center}
\textbf{Future Tense }
\end{center}

\par{7. ${\overset{\textnormal{そつぎょうしき}}{\text{卒業式}}}$ は ${\overset{\textnormal{あした}}{\text{明日}}}$ です。 \hfill\break
 \emph{Sotsugyōshiki wa ashita desu. \hfill\break
 }Graduation is\slash will be tomorrow }

\par{8. ${\overset{\textnormal{かいごう}}{\text{会合}}}$ は ${\overset{\textnormal{あした}}{\text{明日}}}$ です。 \hfill\break
 \emph{Kaigō wa ashita desu. \hfill\break
 }The assembly is\slash will be tomorrow. }

\par{9. ${\overset{\textnormal{しゅうりょうび}}{\text{終了日}}}$ は ${\overset{\textnormal{もくようび}}{\text{木曜日}}}$ です。 \hfill\break
 \emph{Shūryōbi wa mokuyōbi desu. \hfill\break
 }The end date is\slash will be Thursday. }

\par{10. もうすぐです。 \hfill\break
 \emph{Mō sugu desu. \hfill\break
 }It\textquotesingle ll be soon. }

\par{\textbf{Grammar Note }: \emph{Mō sugu }もうすぐ is also an adverb, but the grammar is still the same. }

\par{\textbf{Grammar Note }: Unlike with \emph{da }だ, the polite form \emph{desu }です is not usually omitted at the end of a sentence. This is because its purpose is to provide politeness. }

\begin{center}
\textbf{Polite Past Tense: \emph{Deshita }でした }
\end{center}

\par{ The past tense form of \emph{desu }です is \emph{deshita }でした. As you can see, -TA appears once more. }

\begin{center}
Conjugation Recap 
\end{center}

\begin{ltabulary}{|P|P|}
\hline 

Non-Past Tense & Past Tense \\ \cline{1-2}

 \emph{Desu }です &  \emph{Deshita }でした \\ \cline{1-2}

\end{ltabulary}

\par{\hfill\break
11. ${\overset{\textnormal{にほんじん}}{\text{日本人}}}$ は ${\overset{\textnormal{ふた}}{\text{2}}}$ ${\overset{\textnormal{り}}{\text{人}}}$ でした。 \hfill\break
 \emph{Nihonjin wa futari deshita. \hfill\break
 }There were two Japanese people. }

\par{12. ${\overset{\textnormal{ちゅうごくじん}}{\text{中国人}}}$ は ${\overset{\textnormal{ひと}}{\text{1}}}$ ${\overset{\textnormal{り}}{\text{人}}}$ でした。 \hfill\break
 \emph{Chūgokujin wa hitori deshita. \hfill\break
 }There was one Chinese person. }

\par{13. あの ${\overset{\textnormal{おとこ}}{\text{男}}}$ は ${\overset{\textnormal{けいさつかん}}{\text{警察官}}}$ でした。 \hfill\break
 \emph{Ano otoko wa keisatsukan deshita. \hfill\break
 }That man was a policeman. }

\par{\textbf{Grammar Note }: \emph{Ano }あの is the attributive form of are あれ. }

\par{14. ニュースでした。 \hfill\break
 \emph{Nyūsu deshita. \hfill\break
 }This has been the news. }

\par{\textbf{Grammar Note }: In Japanese, the past tense form also extends to perfect tenses (completion). }

\par{15. あれはテロでした。 \hfill\break
 \emph{Are wa tero deshita. \hfill\break
 }That was terrorism. }

\par{16. あの ${\overset{\textnormal{こ}}{\text{子}}}$ は ${\overset{\textnormal{おとこ}}{\text{男}}}$ の ${\overset{\textnormal{こ}}{\text{子}}}$ でした。 \hfill\break
 \emph{Ano ko wa otoko-no-ko deshita. \hfill\break
 }That child was a boy. }

\par{\textbf{Variation Note }: Some speakers use \emph{datta desu }だったです instead of \emph{deshita }でした, but this is deemed incorrect by most native speakers. As such, it is best to always use \emph{deshita }でした but understand what people mean when they use \emph{datta desu }だったです instead. }

\begin{center}
\textbf{Polite Negative 1: \emph{De wa nai desu }ではないです }
\end{center}

\par{ To make the copula negative in polite speech, you have two options at your disposal. The path you take determines how formal you are. The first method takes the least amount of effort, which is adding \emph{desu }です to \emph{[de wa\slash ja] nai }【では・じゃ】ない. This method is typically avoided in more formal, serious situations, but it is very common in conversation. With the contraction \emph{ja }じゃ also being most common in speech, you will hear \emph{ja nai desu }じゃないです a lot. }

\begin{center}
Conjugation Recap 
\end{center}

\begin{ltabulary}{|P|P|P|}
\hline 

 Non-Past Tense & Past Tense & Negative 1 \\ \cline{1-3}

 \emph{Desu }です &  \emph{Deshita }でした &  \emph{De wa nai desu }では ないです \hfill\break
\emph{Ja nai desu }じゃないです  \\ \cline{1-3}

\end{ltabulary}

\par{\hfill\break
17. ${\overset{\textnormal{わたし}}{\text{私}}}$ は ${\overset{\textnormal{ちゅうがくせい}}{\text{中学生}}}$ ではないです。 \hfill\break
 \emph{Watashi wa chūgakusei de wa nai desu. \hfill\break
 }I am not a junior high student. }

\par{18. ${\overset{\textnormal{かのじょ}}{\text{彼女}}}$ はお ${\overset{\textnormal{いしゃ}}{\text{医者}}}$ さんではないです。 \hfill\break
 \emph{Kanojo wa o-isha-san de wa nai desu. \hfill\break
 }She is not a doctor. }

\par{19. ${\overset{\textnormal{かれ}}{\text{彼}}}$ は ${\overset{\textnormal{こうこうせい}}{\text{高校生}}}$ ではないです。 \hfill\break
 \emph{Kare wa kōkōsei de wa nai desu. \hfill\break
 }He is not a high school student. }

\par{20. ${\overset{\textnormal{けんたくん}}{\text{健太君}}}$ は ${\overset{\textnormal{しょうがくせい}}{\text{小学生}}}$ じゃないです。 \hfill\break
 \emph{Kenta-kun wa shōgakusei ja nai desu. \hfill\break
 }Kenta-kun isn\textquotesingle t a student. }

\par{\textbf{Grammar Note }: - \emph{kun }君 is often added affectionately to male names. \hfill\break
 \hfill\break
21. あれはオオカミじゃないです。 \hfill\break
 \emph{Are wa ōkami ja nai desu. \hfill\break
 }That isn\textquotesingle t a wolf. }

\par{22. それは ${\overset{\textnormal{うそ}}{\text{嘘}}}$ じゃないです。 \hfill\break
 \emph{Sore wa uso ja nai desu. \hfill\break
 }That\textquotesingle s not a lie. }

\begin{center}
\textbf{Polite Negative 2: \emph{De wa arimasen }ではありません }
\end{center}

\par{ The second method to make the polite negative form is by using \emph{de wa arimasen }ではありません. This form is considerably politer, and as such, its contracted form \emph{ja arimasen }じゃありません is on par with \emph{de wa arimasen }ではありません even in conversation as a result. This is because people typically wish to capitalize on how polite they are when the situation calls for it, and avoiding contractions is one way to accomplish this. }

\par{ You may be wondering; how do you get \emph{arimasen }ありません out of \emph{nai }ない? The answer is that \emph{nai }ない is the negative form of an actual verb, \emph{aru }ある. Although we haven't learned about verbs just yet, \emph{aru }ある is the basic existential verb of Japanese. Meaning, it demonstrates that something "is" and is actually embedded etymologically into all copular phrases of the language. }

\par{ Just like the copula, verbs also have their own plain and polite conjugations. Just like the copula, there are two means of making the polite negative. The less polite form of \emph{aru }ある is \emph{nai desu }ないです. Its politer form is \emph{arimasen }ありません. Although there isn't any need to break down \emph{arimasen }ありません, it's important to note that the \slash n\slash  at the end is what brings about the negative meaning. Since \emph{na }i ない also has \slash n\slash  in it, this should be easy to remember. }

\begin{ltabulary}{|P|P|P|}
\hline 

Non-Past & Past Tense & Negative 1 \\ \cline{1-3}

 \emph{Desu }です &  \emph{Deshita }でした &  \emph{De wa nai desu }ではないです \hfill\break
\emph{Ja nai desu }じゃないです \\ \cline{1-3}

 &  & Negative 2 \\ \cline{1-3}

 &  &  \emph{De wa arimasen }ではありません \hfill\break
\emph{Ja arimasen }じゃありません \\ \cline{1-3}

\end{ltabulary}
\hfill\break
23. ${\overset{\textnormal{かれ}}{\text{彼}}}$ は ${\overset{\textnormal{ぎいん}}{\text{議員}}}$ ではありません。  \emph{\emph{Kare wa giin de wa arimasen. } }He is not a legislator. \hfill\break
\hfill\break
24. リーさんは ${\overset{\textnormal{だいがくせい}}{\text{大学生}}}$ ではありません。 \emph{Rii-san wa daigakusei de wa arimasen. } Mr. Lee is not a college student. \hfill\break
\hfill\break
25. あれは ${\overset{\textnormal{にせもの}}{\text{偽物}}}$ ではありません。 \emph{Are wa nisemono de wa arimasen. } That is not a fake. \hfill\break
\hfill\break
26. ${\overset{\textnormal{かれ}}{\text{彼}}}$ は ${\overset{\textnormal{えいゆう}}{\text{英雄}}}$ じゃありません。 \emph{Kare wa eiyū ja arimasen. } He isn\textquotesingle t a hero. \hfill\break
\hfill\break
27. それは ${\overset{\textnormal{まちが}}{\text{間違}}}$ いじゃありません。 \emph{Sore wa machigai ja arimasen. } That isn\textquotesingle t a mistake. \hfill\break
\hfill\break
28. いいえ、そうじゃありません。 \emph{Iie, sō ja arimasen. } No, that isn\textquotesingle t so. \textbf{Polite Negative-Past 1: \emph{De wa nakatta desu }ではなかったです }
\par{ Just like above, there are two methods to making the polite negative-past form. The first simply involves adding \emph{desu }です to \emph{[de wa\slash ja] nakatta }【では・じゃ】なかった. This form, though not as polite as the one that will follow, is still frequently used in conversation. Given that it is used a lot in conversation, you will hear it as \emph{ja nakatta desu }じゃなかったです the most. }

\begin{ltabulary}{|P|P|P|P|}
\hline 

Non-Past & Past & Negative 1 & Negative-Past 1 \\ \cline{1-4}

 \emph{Desu }です &  \emph{Deshita }でした &  \emph{De wa nai desu }ではないです \hfill\break
\emph{Ja nai desu }じゃないです &  \emph{De wa nakatta desu } \hfill\break
ではなかった です \emph{Ja nakatta desu } \hfill\break
じゃなかったです 
\\ \cline{1-4}

 &  & Negative 2 &  \\ \cline{1-4}

 &  &  \emph{De wa arimasen }ではありません \hfill\break
\emph{Ja arimasen }じゃありません &  \\ \cline{1-4}

\end{ltabulary}
\hfill\break
29. ${\overset{\textnormal{きのう}}{\text{昨日}}}$ は ${\overset{\textnormal{やす}}{\text{休}}}$ みではなかったです。 \emph{Kinō wa yasumi de wa nakatta desu. } Yesterday was not a holiday. \hfill\break
\hfill\break
30. ${\overset{\textnormal{きのう}}{\text{昨日}}}$ は ${\overset{\textnormal{にちようび}}{\text{日曜日}}}$ ではなかったです。 \emph{Kinō wa Nichiyōbi de wa nakatta desu. } Yesterday was not Sunday. \hfill\break
\hfill\break
31. それは ${\overset{\textnormal{ごえんだま}}{\text{五円玉}}}$ ではなかったです。 \emph{Sore wa goendama de wa nakatta desu. } That wasn\textquotesingle t a five-yen coin. \hfill\break
\hfill\break
32. これは ${\overset{\textnormal{もぎしけん}}{\text{模擬試験}}}$ じゃなかったです。 \emph{Kore wa mogi shiken ja nakatta desu. } This wasn\textquotesingle t a mock exam. \hfill\break
\hfill\break
33. ${\overset{\textnormal{じょうだん}}{\text{冗談}}}$ じゃなかったですよ。 \emph{Jōdan ja nakatta desu yo. } It wasn\textquotesingle t a joke. \hfill\break
\hfill\break
34. ${\overset{\textnormal{きせき}}{\text{奇跡}}}$ じゃなかったですよ。 \emph{Kiseki ja nakatta desu yo. } It wasn\textquotesingle t a miracle. 
\begin{center}
\textbf{Polite Negative-Past 2: \emph{De wa arimasendeshita }ではありませんでした }
\end{center}

\par{ To make the negative-past form politer, you need to conjugate \emph{arimasen }ありません into the past tense. To this, you add \emph{deshita }でした to the end, giving \emph{arimasendeshita }ありませんでした. Altogether, you get \emph{[de wa\slash ja] arimasendeshita }【では・じゃ】ありませんでした. Due to the nature of this form being inherently polite, both \emph{ja arimasendeshita }じゃありませんでした and \emph{de wa arimasendeshita }ではありませんでした are used frequently in the spoken language. However, in the written language, \emph{de wa arimasendeshita }ではありませんでした is overwhelmingly preferred. }

\begin{center}
Conjugation Recap 
\end{center}

\begin{ltabulary}{|P|P|P|P|}
\hline 

Non-Past & Past & Negative 1 & Negative-Past 1 \\ \cline{1-4}

 \emph{Desu }です &  \emph{Deshita }でした &  \emph{De wa nai desu }ではないです \hfill\break
\emph{Ja nai desu }じゃないです &  \emph{De wa nakatta desu } \hfill\break
ではなかったです \emph{Ja nakatta desu } \hfill\break
じゃなかったです 
\\ \cline{1-4}

 &  & Negative 2 & Negative-Past 2 \\ \cline{1-4}

 &  &  \emph{De wa arimasen }ではありません \hfill\break
\emph{Ja arimasen }じゃありません &  \emph{De wa arimasendeshita } \hfill\break
ではありませんでした \hfill\break
\emph{Ja arimasendeshita } \hfill\break
じゃありませんでした \\ \cline{1-4}

\end{ltabulary}

\par{ 35. (そこは) ${\overset{\textnormal{しょくどう}}{\text{食堂}}}$ ではありませんでした。 \hfill\break
\emph{(Soko wa) shokudō de wa arimasendeshita. \hfill\break
}(That\slash it) was not a diner. }
36. ${\overset{\textnormal{たなか}}{\text{田中}}}$ さんは ${\overset{\textnormal{せんせい}}{\text{先生}}}$ ではありませんでした。 \emph{Tanaka-san wa sensei de wa arimasendeshita. } Mr. Tanaka was not a teacher. \hfill\break
\hfill\break
37. ${\overset{\textnormal{おだ}}{\text{小田}}}$ さんは ${\overset{\textnormal{じゅうみん}}{\text{住民}}}$ ではありませんでした。 \emph{Oda-san wa jūmin de wa arimasendeshita. } Mr. Oda was not a resident. \hfill\break
\hfill\break
38. お ${\overset{\textnormal{ひる}}{\text{昼}}}$ はお ${\overset{\textnormal{にぎ}}{\text{握}}}$ りじゃありませんでした。 \emph{O-hiru wa onigiri ja arimasendeshita. } Lunch wasn\textquotesingle t onigiri. \hfill\break
\hfill\break
39. それはコーヒーじゃありませんでした。 \emph{Sore wa kōhii ja arimasendeshita. } That wasn\textquotesingle t coffee. \hfill\break
\hfill\break
40. あれは ${\overset{\textnormal{さる}}{\text{猿}}}$ じゃありませんでした。 \emph{Are wa saru ja arimasendeshita. } That wasn\textquotesingle t a monkey.  7. リーさんは ${\overset{\textnormal{がくせい}}{\text{学生}}}$ \textbf{ではありません }。(Polite negative) Mr. Lee isn't a student.   8. いいえ、そう \textbf{じゃありません }。(Polite negative) No, it isn't.   7. リーさんは ${\overset{\textnormal{がくせい}}{\text{学生}}}$ \textbf{ではありません }。(Polite negative) Mr. Lee isn't a student.   8. いいえ、そう \textbf{じゃありません }。(Polite negative) No, it isn't.   7. リーさんは ${\overset{\textnormal{がくせい}}{\text{学生}}}$ \textbf{ではありません }。(Polite negative) Mr. Lee isn't a student.   8. いいえ、そう \textbf{じゃありません }。(Polite negative) No, it isn't.   7. リーさんは ${\overset{\textnormal{がくせい}}{\text{学生}}}$ \textbf{ではありません }。(Polite negative) Mr. Lee isn't a student.   8. いいえ、そう \textbf{じゃありません }。(Polite negative) No, it isn't.      