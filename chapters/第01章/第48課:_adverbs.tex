    
\chapter{Adverbs I}

\begin{center}
\begin{Large}
第48課: Adverbs I 
\end{Large}
\end{center}
 
\par{ Adverbs ( ${\overset{\textnormal{ふくし}}{\text{副詞}}}$ ) modify verbs or adjectives, and although adverbs may show up in various places in a sentence, they are typically used at the beginning of a sentence. This isn't a rule, and you still see adverbs just about anywhere. It is simply a trend that you will become familiarized with. }

\par{ Adverbs come from various sources. However, not every kind of adverb will be discussed in this lesson, and we will continue learning more about adverbs in consecutive lessons. }
      
\section{Adverbs}
 
\par{ Most adverbs come from nouns. This sometimes makes it hard to tell when an adverb phrase is actually being used as a noun or not, but we'll learn more about how to discern between them. You've actually already gotten used to such words. \textbf{Temporal words }and \textbf{counters }are great examples of things that can be \emph{nouns or adverbs }. }

\par{ Only a handful of adverbs are true adverbs--もう (already). A "true adverb" can never be used as nouns, and not surprisingly, they don't appear to come from nouns at all. }

\par{ Below are some of the most common adverbs in Japanese. Before we get to example sentences, we'll need to take into account the notes that follow the chart. }

\begin{ltabulary}{|P|P|P|P|P|P|}
\hline 

Now & 今 & Today & 本日、今日 & Yesterday & 昨日 \\ \cline{1-6}

Tomorrow & 明日 & Already & もう & Still; yet & まだ \\ \cline{1-6}

At that time & その時 & A little & 少し、ちょっと & Sometimes & 時々 \\ \cline{1-6}

Immediately & すぐに & Fairly; quite & かなり & Almost & ほとんど \hfill\break
\\ \cline{1-6}

Again & また \hfill\break
& Completely & まったく \hfill\break
& Suddenly & とつぜん \\ \cline{1-6}

\end{ltabulary}

\par{\textbf{Usage Notes }: }
 
\par{1. The general reading for 今日 is きょう. こんにち is formal and usually expected if used in formal writing. In general speech, it is usually limited to こんにちは. こんじつ is a rather outdated reading that you should not expect to see. \hfill\break
2. 本日 is read as ほんじつ, not ほんにち. It is very formal and usually treated as a 書き言葉. \hfill\break
3. 明日 is formally read as みょうにち. あす is slightly more formal than あした, but both are common readings in the spoken language. \hfill\break
4. 昨日 is formally read as さくじつ. However, this is usually restricted to very honorific speech or writing. Its normal reading is きのう. }

\begin{center}
 \textbf{Examples }
\end{center}
 
\par{${\overset{\textnormal{}}{\text{1. 彼女}}}$ はまだ ${\overset{\textnormal{}}{\text{時計}}}$ の ${\overset{\textnormal{みかた}}{\text{見方}}}$ を ${\overset{\textnormal{し}}{\text{知}}}$ らない。 \hfill\break
She still doesn't know how to tell time. }

\par{2. とりわけ ${\overset{\textnormal{}}{\text{今日}}}$ は ${\overset{\textnormal{}}{\text{寒}}}$ い。 \hfill\break
Today is particularly cold. }

\par{3. ${\overset{\textnormal{きゅうきゅうしゃ}}{\text{救急車}}}$ が ${\overset{\textnormal{にだいき}}{\text{二台来}}}$ ました。(Polite) \hfill\break
Two ambulances came. }

\par{4. ついに ${\overset{\textnormal{ふゆ}}{\text{冬}}}$ も ${\overset{\textnormal{お}}{\text{終}}}$ わった。 \hfill\break
Winter has finally ended. }

\par{5. ${\overset{\textnormal{やくひん}}{\text{薬品}}}$ は ${\overset{\textnormal{げんざい}}{\text{現在}}}$ かなり ${\overset{\textnormal{}}{\text{安い}}}$ です。(Formal) \hfill\break
Pharmaceuticals are currently fairly cheap. }

\par{6a. きのうと比べてちょっと ${\overset{\textnormal{あたた}}{\text{暖}}}$ かいね。(Casual) \hfill\break
6b. きのうと比べて\{すこし・ちょっと\} ${\overset{\textnormal{}}{\text{暖}}}$ かいですね。(Polite) \hfill\break
It's a little warm compared to yesterday, isn't it? }

\par{\textbf{Vocab Note }: あたたかい is written as ${\overset{\textnormal{}}{\text{温}}}$ かい in reference to heat of touch or emotion and ${\overset{\textnormal{}}{\text{暖}}}$ かい as in reference to climate, body temperature, and even color. あたたかい may also be あたたかな or even あったか\{い・な\} in casual speech. Remember that な is used before nouns. As for "hot", ${\overset{\textnormal{あつ}}{\text{熱}}}$ い refers to things being hot and ${\overset{\textnormal{あつ}}{\text{暑}}}$ い refers to the weather being hot. }

\par{7. たくさんありますか。 \hfill\break
Is there a lot?\slash Do you have a lot? }

\par{\textbf{Grammar Note }: たくさん may either be used as a noun or an adverb. Never say たくさんな when using this as a noun quantifier. }

\begin{center}
 \textbf{もう VS もっと }
\end{center}

\par{  Both have meanings of "more." The first, though, can be used in the sense of "once more\slash further" and the second is "more" as in the degree of something. Aside from that, もう also means "already" and "shortly." }

\par{8a. もう ${\overset{\textnormal{いっかい}}{\text{一回}}}$ 〇 \hfill\break
8b. もっと一回  X \hfill\break
Once more }
 
\par{9. もっと ${\overset{\textnormal{}}{\text{時間}}}$ が ${\overset{\textnormal{ひつよう}}{\text{必要}}}$ だ。 \hfill\break
I need more time. }

\par{10. ${\overset{\textnormal{すし}}{\text{寿司}}}$ をもう ${\overset{\textnormal{}}{\text{食}}}$ べた。 \hfill\break
I already ate the sushi. }

\par{11. もっとお金が ${\overset{\textnormal{ひつよう}}{\text{必要}}}$ です。 \hfill\break
More money is necessary. }
 
\par{12. もう ${\overset{\textnormal{}}{\text{一}}}$ つ ${\overset{\textnormal{}}{\text{上}}}$ のクラスに ${\overset{\textnormal{うつ}}{\text{移}}}$ った。 \hfill\break
I've already switched to a class above (the previous one). }
 
\par{13. もうすこしがんばって。 \hfill\break
Just keep on a little bit more. }

\par{14. もう ${\overset{\textnormal{}}{\text{終}}}$ わった。 \hfill\break
It has already ended. }
      
\section{Adverbs from the Particle て}
 
\par{ Some て phrases are adverbial. These phrases have no "conjunctive" role as the literal grammatical function of て has been lost in them. It is best to treat them as separate words in your vocabulary. }

\par{${\overset{\textnormal{}}{\text{15. 歩}}}$ いて10 ${\overset{\textnormal{ぷん}}{\text{分}}}$ かかります。 \hfill\break
It will take 10 minutes by foot. }

\par{16a. それは ${\overset{\textnormal{}}{\text{初}}}$ めて ${\overset{\textnormal{}}{\text{聞}}}$ く ${\overset{\textnormal{}}{\text{話}}}$ です。 \hfill\break
16b. それは初めて聞きました。 \hfill\break
16c. それは初耳です。(Idiom) \hfill\break
That's a new story to me\slash That's first in my ears. }
    