    
\chapter{Adjectives I}

\begin{center}
\begin{Large}
第13課: Adjectives I: 形容詞 Keiyōshi 
\end{Large}
\end{center}
 
\par{  \emph{Keiyōshi } ( ${\overset{\textnormal{けいようし}}{\text{形容詞}}}$ ) means "adjective." As you will learn, using adjectives is somewhat similar to English. Because we don't need words like "is\slash are" when we use them, they more or less end up where they do in an English sentence. }
      
\section{Vocabulary List}
 
\par{\textbf{Nouns  }}

\par{・本 \emph{Hon }– Book }

\par{・映画 \emph{Eiga }– Movie }

\par{・海 \emph{Umi }– Sea }

\par{・光 \emph{Hikari }– Light }

\par{・建物 \emph{Tatemono }– Building }

\par{・ウサギ(兎) \emph{Usagi }- Rabbit\slash bunny }

\par{・クマ(熊) \emph{Kuma }- Bear }

\par{・歯茎 \emph{Haguki }- Gums }

\par{・携帯 \emph{Keitai }- Cellphone }

\par{・手 \emph{Te }- Hand(s) }

\par{・水 \emph{Mizu }- Water }

\par{・冬 \emph{Fuyu }- Winter }

\par{・公園 \emph{Kōen }- Park }

\par{・景色 \emph{Keshiki }- Scenery }

\par{・ラーメン \emph{Rāmen }- Ramen }

\par{・仕事 \emph{Shigoto }- Job\slash work }

\par{・猫 \emph{Neko }- Cat }

\par{・川 \emph{Kawa }- River }

\par{・冗談 \emph{J }\emph{ōdan }- Joke }

\par{・影響 \emph{Eikyō }- Influence\slash effect }

\par{・信号 \emph{Shingō }- Traffic light }

\par{・リンゴ \emph{Ringo }- Apple }

\par{・パソコン \emph{Pasokon }- Computer }

\par{・給料 \emph{Ky }\emph{ūr }\emph{yō }- Salary\slash wages }

\par{・城 \emph{Shiro }- Castle }

\par{・葉っぱ \emph{Happa }- Leaves }

\par{・タイヤ \emph{Taiya }- Tire }

\par{・味 \emph{Aji }- Flavor }

\par{・扱い \emph{Atsukai }- Handling }

\par{・品質 \emph{Hinshitsu }- Quality }

\par{・空気 \emph{K }\emph{ūki } - Air }

\par{・スコア \emph{Sukoa }- Score }

\par{・足 \emph{Ashi }- Foot\slash feet }

\par{・虫 \emph{Mushi }- Bug }

\par{・道 \emph{Michi }- Street\slash road }

\par{・答え \emph{Kotae }- Answer }

\par{・試験 \emph{Shiken }- Exam }

\par{・問題 \emph{Mondai }- Problem\slash question }

\par{\textbf{Proper Nouns }}

\par{・山田先生 \emph{Yamada-sensei }– Professor\slash Dr.\slash Mr. Yamada (teacher) }

\par{・長谷川先生 \emph{Hasegawa-sensei }- Professor\slash Dr.\slash Mr. Hasegawa (teacher) }

\par{・南米 Nambei - South America }

\par{\textbf{Pronouns }}

\par{・私 \emph{Watashi }– I }

\par{・僕 Boku - I (male) }

\par{・彼 \emph{Kare }– He }

\par{\textbf{Demonstratives }}

\par{・この \emph{Kono }– This (adj.) }

\par{・その \emph{Sono }- That (adj.) }

\par{・あれ \emph{Are }- That (over there) (n.) }

\par{・あの \emph{Ano }– That (over there) (adj.) }

\par{・ここ \emph{Koko }- Here }

\par{\textbf{Copula Verb }}

\par{ \textbf{・ }です \emph{Desu }– Is\slash are \hfill\break
}
 
\par{ \textbf{Adjectives }\hfill\break
}
 
\par{・新しい \emph{Atarashii }– New }
 
\par{・早い \emph{Hayai }– Early }
 
\par{・遅い \emph{Osoi }– Late\slash slow }
 
\par{・速い \emph{Hayai }– Fast }
 
\par{・可愛い \emph{Kawaii }- Cute }
 
\par{・強い \emph{Tsuyoi }– Strong }
 
\par{・弱い \emph{Yowai }– Weak }
 
\par{・古い \emph{Furui }– Old }
 
\par{・暑い \emph{Atsui }– Hot (weather) }
 
\par{・熱い \emph{Atsui }– Hot (in general) }
 
\par{・寒い \emph{Samui }– Cold (weather) }
 
\par{・冷たい \emph{Tsumetai }– Cold (in general) }
 
\par{・嬉しい \emph{Ureshii }– Happy }
 
\par{・悲しい \emph{Kanashii }– Sad }
 
\par{・遠い \emph{Tōi }– Far }
 
\par{・近い \emph{Chikai }– Nearby\slash close }
 
\par{・美しい \emph{Utsukushii }– Beautiful }
 
\par{・醜い \emph{Minikui }– Ugly }
 
\par{・忙しい \emph{Isogashii }– Busy }
 
\par{・楽しい \emph{Tanoshii }– Fun }
 
\par{・美味しい \emph{Oishii }– Delicious }
 
\par{・不味い \emph{Mazui }– Nasty }
 
\par{・優しい \emph{Yasashii }– Kind }
 
\par{・厳しい \emph{Kibishii }– Strict\slash harsh }
 
\par{・良い \emph{Yoi }- Good }
 
\par{・悪い \emph{Warui }- Bad }
 
\par{・素晴らしい \emph{Subarashii }- Wonderful }
 
\par{・凄い \emph{Sugoi }- Amazing }
 
\par{・汚い \emph{Kitanai }- Dirty }
 
\par{・明るい \emph{Akarui }- Bright }
 
\par{・広い \emph{Hiroi }- Wide }
 
\par{・狭い \emph{Semai }- Narrow }
 
\par{・暗い \emph{Kurai }- Dark }
 
\par{・怖い \emph{Kowai }- Scary }
 
\par{・低い \emph{Hikui }- Low }
 
\par{・長い \emph{Nagai }- Long }
 
\par{・短い \emph{Mijikai }- Short (length) }
 
\par{・正しい \emph{Tadashii }- Correct }
 
\par{・寂しい \emph{Sabishii }- Lonely }
 
\par{・痛い \emph{Itai }- Painful }
 
\par{・易しい \emph{Yasashii }- Easy }
 
\par{・難しい \emph{Muzukashii }- Difficult }
 
\par{・背が低い \emph{Se ga hikui }- Short (height) }
 
\par{・軽い \emph{Karui }- Light }
 
\par{・重い \emph{Omoi }- Heavy }
 
\par{・苦い \emph{Nigai }- Bitter }
 
\par{・辛い \emph{Karai }- Spicy }
・塩辛い \emph{Shiokarai }- Salty       
\section{What is an Adjective?}
 
\par{ An \textbf{adjective }is a word that describes some attribute of a noun. In English, an adjective can either come directly before a noun or modify it from afar as the predicate (part of the sentence which makes a statement about the subject) with the help of a “to be” verb. }

\par{i. I sold the \textbf{\emph{old }}book. \hfill\break
ii. I ate a \textbf{\emph{green }}apple. \hfill\break
iii. The moose was \textbf{\emph{large }}. \hfill\break
iv. Textbook are \textbf{\emph{expensive }}. \hfill\break
v. The \textbf{\emph{young }}man bought a \textbf{\emph{new }}car. }

\par{ In Japanese, \emph{Keiy }\emph{ōshi }形容詞 can modify a noun directly before it as well as modify it from afar as the predicate of the sentence. However, unlike English, there is no need for a “to be (= copula).” The adjective may either be directly before a noun or at the end of the sentence. A good example word to showcase this is the adjective for “new”: \emph{atarashii }新しい. This single word can stand for “to be new,” “is new,” “are new,” etc. To see how it works in actual sentences, consider the following. }

\par{1. この ${\overset{\textnormal{ほん}}{\text{本}}}$ は ${\overset{\textnormal{あたら}}{\text{新}}}$ しい。 \hfill\break
 \emph{Kono hon wa atarashii. \hfill\break
 }This book is new. }

\par{2. ${\overset{\textnormal{あたら}}{\text{新}}}$ しい ${\overset{\textnormal{ほん}}{\text{本}}}$ です。 \hfill\break
 \emph{Atarashii hon desu. \hfill\break
 }(This is) a new book. }

\par{  In Ex. 1, \emph{atarashii }新しい functions as the predicate. In Ex. 2, it only modifies the noun \emph{hon }本 (book). Understanding word order, though, is only one aspect of knowing how to use adjectives. You will now need to learn how to conjugate adjectives. Note that the conjugations you learn in this lesson only apply to the part of speech being introduced to you. Although the same principles will apply, conjugatable parts of speech in Japanese will always look different from one another, but words in any given part of speech will share the same endings. }
      
\section{Conjugating Keiyōshi 形容詞}
 
\begin{center}
\textbf{Plain Non-Past Form: No Conjugation }
\end{center}

\par{ All \emph{Keiy }\emph{ōshi }形容詞 end in the vowel \slash i\slash . This is no coincidence. This is because the \slash i\slash  is an ending that attaches to the stem. The stem of a word is its root, what makes up the meaning of the word. The - \emph{i }い is a grammatical item that adds a finishing touch. The resultant form is the plain non-past form of an adjective. }

\par{ Although conjugation is technically involved, speakers treat the “stem” and the ending - \emph{i }い as parts of a single word. Knowing why the - \emph{i }い is there simply helps you know when something is a \emph{Keiy }\emph{ōshi }形容詞.  Before we learn about other conjugations, let's first learn a handful of adjectives to add context to the information discussed so far. }
 
\begin{ltabulary}{|P|P|P|P|}
\hline 
 
Meaning & Adjective & Meaning & Adjective \\ \cline{1-4}

  Strong 
 &    \emph{Tsuyoi }強い 
 &   Weak 
 &    \emph{Yowai }弱い 
 \\ \cline{1-4} 
 
  Old 
 &    \emph{Furui }古い 
 &   New 
 &    \emph{Atarashii }新しい 
 \\ \cline{1-4} 
 
  Hot (weather) 
 &    \emph{Atsui }暑い 
 &   Hot 
 &    \emph{Atsui }熱い 
 \\ \cline{1-4} 
 
  Cold (weather) 
 &    \emph{Samui }寒い 
 &   Cold 
 &    \emph{Tsumetai }冷たい 
 \\ \cline{1-4} 
 
\end{ltabulary}
 
\par{\hfill\break
1. クマは ${\overset{\textnormal{つよ}}{\text{強}}}$ い。 \hfill\break
 \emph{Kuma wa tsuyoi. \hfill\break
 }Bears are strong. }
 
\par{2. ${\overset{\textnormal{はぐき}}{\text{歯茎}}}$ が ${\overset{\textnormal{よわ}}{\text{弱}}}$ い。 \hfill\break
 \emph{Haguki ga yowai. \hfill\break
 }My gums are weak. }
 
\par{3. ${\overset{\textnormal{なんべい}}{\text{南米}}}$ は ${\overset{\textnormal{あつ}}{\text{暑}}}$ い! \hfill\break
 \emph{Nambei wa atsui }\emph{! \hfill\break
 }South America is hot! }
 
\par{4. ${\overset{\textnormal{て}}{\text{手}}}$ が ${\overset{\textnormal{あつ}}{\text{熱}}}$ い。 \hfill\break
 \emph{Te ga atsui. \hfill\break
 }My\slash your hands are hot. }
 
\par{5. あれは ${\overset{\textnormal{ふる}}{\text{古}}}$ い ${\overset{\textnormal{でんわ}}{\text{電話}}}$ だ。 \hfill\break
 \emph{Are wa furui denwa da. \hfill\break
 }That is an old phone. }
 
\par{6. ${\overset{\textnormal{みず}}{\text{水}}}$ が ${\overset{\textnormal{つめ}}{\text{冷}}}$ たい。 \hfill\break
 \emph{Mizu ga tsumetai. \hfill\break
 }The water is cold. \hfill\break
 \hfill\break
7. ${\overset{\textnormal{ふゆ}}{\text{冬}}}$ は ${\overset{\textnormal{さむ}}{\text{寒}}}$ い! \hfill\break
 \emph{Fuyu wa samui! \hfill\break
 }Winter is cold! }
 
\begin{center}
\textbf{Polite Non-Past Form: + \emph{desu }です }
\end{center}
 
\par{ To conjugate an adjective into its polite non-past form, attach \emph{desu }です to the end. }
 
\begin{ltabulary}{|P|P|P|}
\hline 
 
  Meaning 
 &   Adjective 
 &   Add \emph{desu }です 
 \\ \cline{1-3} 
 
  Close\slash nearby 
 &    \emph{Chikai }近い 
 &    \emph{Chikai desu }近いです 
 \\ \cline{1-3} 
 
  Far 
 &   \emph{ }\emph{Tōi }遠い 
 &    \emph{Tōi desu }遠いです 
 \\ \cline{1-3} 
 
  Beautiful 
 &    \emph{Utsukushii }美しい 
 &    \emph{Utsukushii desu }美しいです 
 \\ \cline{1-3} 
 
  Ugly 
 &   \emph{ }\emph{Minikui }醜い 
 &    \emph{Minikui   desu }醜いです 
 \\ \cline{1-3} 
 
\end{ltabulary}
 
\par{\hfill\break
8. ${\overset{\textnormal{こうえん}}{\text{公園}}}$ が ${\overset{\textnormal{ちか}}{\text{近}}}$ いです。 \hfill\break
 \emph{K }\emph{ōen ga chikai desu. \hfill\break
 }A park is nearby. }
 
\par{9. ${\overset{\textnormal{けしき}}{\text{景色}}}$ が ${\overset{\textnormal{うつく}}{\text{美}}}$ しいです。 \hfill\break
 \emph{Keshiki ga utsukushii desu. \hfill\break
 }The scenery is beautiful. }
 
\begin{center}
\textbf{Past Forms: - \emph{katta }かった \& - \emph{katta desu }かったです }
\end{center}
 
\par{ To make an adjective past tense, drop - \emph{i }い and add - \emph{katta }かった. To make it polite, just add \emph{desu }です at the end of - \emph{katta }かった. }
 
\begin{ltabulary}{|P|P|P|P|}
\hline 
 
  Meaning 
 &   Adjective 
 &   - \emph{katta }かった (plain) 
 &   - \emph{katta   desu }かったです (polite) 
 \\ \cline{1-4} 
 
  Delicious 
 &    \emph{Oishii }美味しい 
 &   Oishikatta 美味しかった 
 &   Oishikatta   desu 美味しかったです 
 \\ \cline{1-4} 
 
  Nasty 
 &   \emph{ }\emph{Mazui }不味い 
 &   Mazukatta 不味かった 
 &   Mazukatta desu 不味かったです 
 \\ \cline{1-4} 
 
  Kind 
 &    \emph{Yasashii }優しい 
 &   Yasashikatta 優しかった 
 &   Yasashikatta   desu 優しかったです 
 \\ \cline{1-4} 
 
  Strict 
 &   \emph{ }\emph{Kibishii }厳しい 
 &   Kibishikatta 厳しかった 
 &   Kibishikatta desu 厳しかったです 
 \\ \cline{1-4} 
 
  Busy 
 &    \emph{Isogashii }忙しい 
 &   Isogashikatta 忙しかった 
 &   Isogashikatta desu 忙しかったです 
 \\ \cline{1-4} 
 
  Fun 
 &   \emph{ }\emph{Tanoshii }楽しい 
 &   Tanoshikatta 楽しかった 
 &   Tanoshikatta desu楽しかったです 
\\ \cline{1-4}

\end{ltabulary}

\par{\hfill\break
10. ラーメンは ${\overset{\textnormal{おい}}{\text{美味}}}$ しかったです。 \hfill\break
 \emph{Ramen wa oishikatta desu. \hfill\break
 }The ramen was delicious. }
 
\par{11. ${\overset{\textnormal{しごと}}{\text{仕事}}}$ が ${\overset{\textnormal{いそが}}{\text{忙}}}$ しかった。 \hfill\break
 \emph{Shigoto ga isogashikatta. \hfill\break
 }Work was busy. }
 
\par{12. あの ${\overset{\textnormal{えいが}}{\text{映画}}}$ は ${\overset{\textnormal{たの}}{\text{楽}}}$ しかったです。 \hfill\break
 \emph{Ano eiga wa tanoshikatta desu. \hfill\break
 }That movie was fun. }
 
\par{13. ${\overset{\textnormal{かれ}}{\text{彼}}}$ が ${\overset{\textnormal{やさ}}{\text{優}}}$ しかった。 \hfill\break
 \emph{Kare ga yasashikatta. \hfill\break
 }He was kind. }
 
\par{14. ${\overset{\textnormal{はせがわせんせい}}{\text{長谷川先生}}}$ は ${\overset{\textnormal{きび}}{\text{厳}}}$ しかったです。 \hfill\break
 \emph{Hasegawa-sensei wa kibishikatta desu. \hfill\break
 }Hasegawa-sensei was strict. }
 
\begin{center}
\textbf{Plain Negative Form: - \emph{kunai }くない }
\end{center}
 
\par{ To make the negative (to not be…) form of an adjective, drop - \emph{i }い and add - \emph{kunai }くない. The - \emph{nai }ない part also functions as an adjective. This fact will come in handy here shortly. }
 
\begin{ltabulary}{|P|P|P|}
\hline 
 
  Meaning 
 &   Adjective 
 &   - \emph{kunai }くない 
 \\ \cline{1-3} 
 
  Early 
 &    \emph{Hayai }早い 
 &    \emph{Hayakunai }早くない 
 \\ \cline{1-3} 
 
  Late\slash Slow 
 &   \emph{ }\emph{Osoi }遅い 
 &    \emph{Osokunai }遅くない 
 \\ \cline{1-3} 
 
  Fast 
 &    \emph{Hayai }速い 
 &    \emph{Hayakunai }速くない 
 \\ \cline{1-3} 
 
  Cute 
 &   \emph{ }\emph{Kawaii }可愛い 
 &    \emph{Kawaikunai }可愛くない 
 \\ \cline{1-3} 
 
  Happy 
 &    \emph{Ureshii }嬉しい 
 &    \emph{Ureshikunai }嬉しくない 
 \\ \cline{1-3} 
 
  Sad 
 &    \emph{Kanashii }悲しい 
 &    \emph{Kanashikunai }悲しくない 
\\ \cline{1-3}

\end{ltabulary}
 
\par{\hfill\break
15. あの ${\overset{\textnormal{ねこ}}{\text{猫}}}$ は ${\overset{\textnormal{かわい}}{\text{可愛}}}$ くない。 \hfill\break
 \emph{Ano neko wa kawaikunai. \hfill\break
 }That cat is not cute. }
 
\par{16. ${\overset{\textnormal{ぼく}}{\text{僕}}}$ は ${\overset{\textnormal{かな}}{\text{悲}}}$ しくない。 \hfill\break
 \emph{Boku wa kanashikunai. \hfill\break
 }I am not sad. }
 
\par{17. ${\overset{\textnormal{おそ}}{\text{遅}}}$ くない? \hfill\break
 \emph{Osokunai? \hfill\break
 }Aren\textquotesingle t you\slash isn\textquotesingle t that slow? \hfill\break
 \hfill\break
\textbf{Intonation Note }: With a raised intonation at the end, you can use the negative ending to make a rhetorical question in the affirmative. }
 
\begin{center}
\textbf{Polite Negative Forms: - \emph{kunai desu }くないです \& - \emph{ku arimasen }くありません }
\end{center}
 
\par{ There are two methods to make an adjective negative in polite speech. The first method, which is not as polite but perfect for general conversation is attaching \emph{desu }です to - \emph{kunai }くない, giving - \emph{kunai desu }くないです. The second method involves dropping - \emph{i }い and adding - \emph{ku arimasen }くありません. In Japanese, there is a trend of longer phrases being politer. This is certainly the case here. }
 
\begin{ltabulary}{|P|P|P|P|}
\hline 
 
  Meaning 
 &   Adjective 
 &   - \emph{kunai desu }くないです 
 &   - \emph{ku   arimasen }くありません 
 \\ \cline{1-4} 
 
  Interesting\slash funny 
 &    \emph{Omoshiroi }面白い 
 &    \emph{Omoshirokunai desu }面白くないです 
 &    \emph{Omoshiroku arimasen }面白くありません 
 \\ \cline{1-4} 
 
  Small 
 &    \emph{Chiisai }小さい 
 &   Chiisakunai desu 小さくないです 
 &   Chiisaku arimasen 小さくありません 
 \\ \cline{1-4} 
 
  Large 
 &    \emph{Ōkii }大きい 
 &    \emph{Ōkikunai desu }大きくないです 
 &    \emph{Ōkiku arimasen }大きくありません 
 \\ \cline{1-4} 
 
  Blue\slash green 
 &    \emph{Aoi }青い 
 &    \emph{Aokunai desu }青くないです 
 &    \emph{Aoku arimasen }青くありません 
 \\ \cline{1-4} 
 
  Red 
 &    \emph{Akai }赤い 
 &    \emph{Akakunai desu }赤くないです 
 &    \emph{Akaku arimasen }赤くありません 
 \\ \cline{1-4} 
 
  Yellow 
 &   \emph{Kiiroi }黄色い 
 &    \emph{Kiirokunai desu }黄色くないです 
 &    \emph{Kiiroku arimasen }黄色くありません 
 \\ \cline{1-4} 
 
\end{ltabulary}
 
\par{\hfill\break
18. その ${\overset{\textnormal{じょうだん}}{\text{冗談}}}$ は ${\overset{\textnormal{おもしろ}}{\text{面白}}}$ くないです。 \hfill\break
 \emph{Sono j }\emph{ōdan wa omoshirokunai desu. \hfill\break
 }That joke is not funny. }
 
\par{19. その ${\overset{\textnormal{えいきょう}}{\text{影響}}}$ は ${\overset{\textnormal{ちい}}{\text{小}}}$ さくありません。 \hfill\break
 \emph{Sono eiky }\emph{ō wa chiisaku arimasen. \hfill\break
 }The influence of that is not small. }
 
\par{20. 川は青くないです。 \hfill\break
 \emph{Kawa wa aokunai desu. \hfill\break
 }The river is not blue. }
 
\par{21. 信号は青くないです。 \hfill\break
 \emph{Shing }\emph{ō wa aokunai desu. \hfill\break
 }The light is not green. }
 
\par{22. あのリンゴは赤くありません。 \hfill\break
 \emph{Ano ringo wa akaku arimasen. \hfill\break
 }That apple is not red. }
 
\begin{center}
\textbf{Plain Negative Past Form: - \emph{kunakatta }くなかった }
\end{center}
 
\par{ To create the negative past in plain speech, drop - \emph{i }い and add - \emph{kunakatta }くなかった.  As you can see, what\textquotesingle s going on is the - \emph{i }い in - \emph{nai }ない is dropped and then - \emph{katta }かった is added. }
 
\begin{ltabulary}{|P|P|P|}
\hline 
 
  Meaning 
 &   Adjective 
 &   - \emph{kunakatta }くなかった 
 \\ \cline{1-3} 
 
  Black 
 &    \emph{Kuroi }黒い 
 &    \emph{Kurokunakatta }黒くなかった 
 \\ \cline{1-3} 
 
  White 
 &    \emph{Shiroi }白い 
 &   \emph{Shirokunakatta }白くなかった 
 \\ \cline{1-3} 
 
  Brown 
 &    \emph{Chairoi }茶色い 
 &    \emph{Chairokunakatta }茶色くなかった 
 \\ \cline{1-3} 
 
  Strange 
 &    \emph{Okashii }可笑しい 
 &    \emph{Okashikunakatta }可笑しくなかった 
 \\ \cline{1-3} 
 
  Tall\slash expensive 
 &    \emph{Takai }高い 
 &    \emph{Takakunakatta }高くなかった 
 \\ \cline{1-3} 
 
  Cheap 
 &    \emph{Yasui }安い 
 &    \emph{Yasukunakatta }安くなかった 
 \\ \cline{1-3} 
 
\end{ltabulary}
 
\par{\hfill\break
23. このパソコンは ${\overset{\textnormal{やす}}{\text{安}}}$ くなかった。 \hfill\break
 \emph{Kono pasokon wa yasukunakatta. \hfill\break
 }This computer wasn\textquotesingle t cheap. }
 
\par{24. ${\overset{\textnormal{きゅうりょう}}{\text{給料}}}$ は ${\overset{\textnormal{たか}}{\text{高}}}$ くなかった。 \hfill\break
 \emph{Ky }\emph{ūry }\emph{ō wa takakunakatta. \hfill\break
 }The salary wasn\textquotesingle t\slash wages weren\textquotesingle t high. }
 
\par{25. その ${\overset{\textnormal{しろ}}{\text{城}}}$ は ${\overset{\textnormal{しろ}}{\text{白}}}$ くなかった。 \hfill\break
 \emph{Sono shiro wa shirokunakatta. \hfill\break
 }That castle wasn\textquotesingle t white. }
 
\par{26. ${\overset{\textnormal{は}}{\text{葉}}}$ っぱは ${\overset{\textnormal{ちゃいろ}}{\text{茶色}}}$ くなかった。 \hfill\break
 \emph{Happa wa chairokunakatta. \hfill\break
 }The leaves weren\textquotesingle t brown. }
 
\par{27. タイヤは ${\overset{\textnormal{くろ}}{\text{黒}}}$ くなかった。 \hfill\break
 \emph{Taiya wa kurokunakatta. \hfill\break
 }The tire(s) weren\textquotesingle t black. }
 
\par{28. ${\overset{\textnormal{あじ}}{\text{味}}}$ は ${\overset{\textnormal{おか}}{\text{可笑}}}$ しくなかった。 \hfill\break
 \emph{Aji wa okashikunakatta. \hfill\break
 }The flavor wasn\textquotesingle t strange. }

\begin{center}
\textbf{Polite Negative-Past Forms: - \emph{kunakatta desu }くなかったです \& - \emph{ku arimasendeshita }くありませんでした }\hfill\break

\end{center}

\par{  Just as there were two methods to making an adjective negative in polite speech, there are also two methods to conjugating an adjective into negative past (was not) in polite speech. The first method involves simply attaching \emph{desu }です to - \emph{kunakatta }くなかった, giving - \emph{kunakatta desu }くなかったです. The other is by using - \emph{ku arimasendeshita }くありませんでした, which utilizes \emph{deshita }でした, the past form of \emph{desu }です. }
 
\begin{ltabulary}{|P|P|P|P|}
\hline 
 
  Meaning 
 &   Adjective 
 &   - \emph{kunakatta desu \hfill\break
 }くなかったです 
 &   - \emph{ku   arimasendeshita } \hfill\break
くありませんでした 
 \\ \cline{1-4} 
 
  Good 
 &    \emph{Yoi }良い 
 &   良くなかったです 
 &   良くありませんでした 
 \\ \cline{1-4} 
 
  Bad 
 &    \emph{Warui }悪い 
 &   悪くなかったです 
 &   悪くありませんでした 
 \\ \cline{1-4} 
 
  Wonderful 
 &    \emph{Subarashii }素晴らしい 
 &   素晴らしくなかったです 
 &   素晴らしくありませんでした 
 \\ \cline{1-4} 
 
  Amazing 
 &    \emph{Sugoi }凄い 
 &   凄くなかったです 
 &   凄くありませんでした 
 \\ \cline{1-4} 
 
  Dirty 
 &    \emph{Kitanai }汚い 
 &   汚くなかったです 
 &   汚くありませんでした 
 \\ \cline{1-4} 
 
  Bright 
 &    \emph{Akarui }明るい 
 &   明るくなかったです 
 &   明るくありませんでした 
\\ \cline{1-4}

\end{ltabulary}
 
\par{\hfill\break
29. その ${\overset{\textnormal{あつか}}{\text{扱}}}$ いは ${\overset{\textnormal{わる}}{\text{悪}}}$ くなかったです。 \hfill\break
 \emph{Sono atsukai wa warukunakatta desu. \hfill\break
 }That handling wasn\textquotesingle t bad. }
 
\par{30. ${\overset{\textnormal{ひんしつ}}{\text{品質}}}$ が ${\overset{\textnormal{よ}}{\text{良}}}$ くありませんでした。 \hfill\break
 \emph{Hinshitsu ga yoku arimasendeshita. \hfill\break
 }The quality was not good. }
 
\par{31. ${\overset{\textnormal{くうき}}{\text{空気}}}$ は ${\overset{\textnormal{きたな}}{\text{汚}}}$ くなかったです。 \hfill\break
 \emph{K }\emph{ūki wa kitanakunakatta desu. \hfill\break
 }The air wasn\textquotesingle t dirty. }
 
\par{32. スコアは ${\overset{\textnormal{すば}}{\text{素晴}}}$ らしくありませんでした。 \hfill\break
 \emph{Sukoa wa subarashiku arimasendeshita. \hfill\break
 }The score was not wonderful. }
\textbf{More Essential Adjectives }\hfill\break
 Now that you have learned the basic conjugations, we will conclude this lesson by learning more adjectives and then finish off with even more adjectives conjugated into all the forms that have been introduced.   
\begin{ltabulary}{|P|P|P|P|}
\hline 
 
  Meaning 
 &   Adjective 
 &   Meaning 
 &   Adjective 
 \\ \cline{1-4} 
 
  Wide 
 &    \emph{Hiroi }広い 
 &   Difficult 
 &    \emph{Muzukashii }難しい 
 \\ \cline{1-4} 
 
  Narrow 
 &    \emph{Semai }狭い 
 &   Easy 
 &    \emph{Yasashii }易しい 
 \\ \cline{1-4} 
 
  Dark 
 &    \emph{Kurai }暗い 
 &   Painful 
 &    \emph{Itai }痛い 
 \\ \cline{1-4} 
 
  Scary 
 &    \emph{Kowai }怖い 
 &   Lonely 
 &    \emph{Sabishii }寂しい 
 \\ \cline{1-4} 
 
  Low 
 &    \emph{Hikui }低い 
 &   Correct 
 &    \emph{Tadashii }正しい 
 \\ \cline{1-4} 
 
  Long 
 &    \emph{Nagai }長い 
 &   Short 
 &    \emph{Mijikai }短い  
\\ \cline{1-4}

\end{ltabulary}
 
\par{\hfill\break
33. ${\overset{\textnormal{あし}}{\text{足}}}$ が ${\overset{\textnormal{いた}}{\text{痛}}}$ い! \hfill\break
 \emph{Ashi ga itai! \hfill\break
 }My feet hurt! }
 
\par{34. ${\overset{\textnormal{むし}}{\text{虫}}}$ 、 ${\overset{\textnormal{こわ}}{\text{怖}}}$ いです! \hfill\break
 \emph{Mushi, kowai desu! \hfill\break
 }The bugs are scary!\slash I\textquotesingle m afraid of bugs. }
 
\par{35. ${\overset{\textnormal{みち}}{\text{道}}}$ が ${\overset{\textnormal{せま}}{\text{狭}}}$ かったです。 \hfill\break
 \emph{Michi ga semakatta desu. \hfill\break
 }The street(s) were narrow. }
 
\par{36. その ${\overset{\textnormal{こた}}{\text{答}}}$ えは ${\overset{\textnormal{ただ}}{\text{正}}}$ しくないです。 \hfill\break
 \emph{Sono kotae wa tadashikunai desu. \hfill\break
 }That answer isn\textquotesingle t correct. }
 
\par{37. ${\overset{\textnormal{わたし}}{\text{私}}}$ は ${\overset{\textnormal{せ}}{\text{背}}}$ が ${\overset{\textnormal{ひく}}{\text{低}}}$ くありません。 \hfill\break
 \emph{Watashi wa se ga hikuku arimasen. \hfill\break
 }I am not short. }
 
\par{\textbf{Word Note }: “Short” as in "height” is \emph{se ga hikui }背が低い. }
 
\par{38. ${\overset{\textnormal{しけん}}{\text{試験}}}$ は ${\overset{\textnormal{むずか}}{\text{難}}}$ しくありませんでした。 \hfill\break
 \emph{Shiken wa muzukashiku arimasendeshita. \hfill\break
 }The exam was not difficult. }
 
\par{39. ${\overset{\textnormal{しけんもんだい}}{\text{試験問題}}}$ は ${\overset{\textnormal{やさ}}{\text{易}}}$ しかったです。 \hfill\break
 \emph{Shiken mondai wa yasashikatta desu. \hfill\break
 }The exam questions were easy. }
 
\par{40. ${\overset{\textnormal{ぼく}}{\text{僕}}}$ 、 ${\overset{\textnormal{さび}}{\text{寂}}}$ しい… \hfill\break
 \emph{Boku, sabishii… \hfill\break
 }I\textquotesingle m lonely… }
 
\par{41. 山田先生は ${\overset{\textnormal{きび}}{\text{厳}}}$ しいです。 \hfill\break
 \emph{Yamada-sensei wa kibishii desu. \hfill\break
}Yamada-sensei is strict. }
 
\par{42. ${\overset{\textnormal{わたし}}{\text{私}}}$ は ${\overset{\textnormal{いそが}}{\text{忙}}}$ しくありませんでした。 \hfill\break
 \emph{Watashi wa isogashiku arimasendeshita. \hfill\break
}I was not busy. }
 
\par{43. あの ${\overset{\textnormal{たてもの}}{\text{建物}}}$ は ${\overset{\textnormal{ちか}}{\text{近}}}$ いです。 \hfill\break
 \emph{Ano tatemono wa chikai desu. \hfill\break
 }That building is close\slash nearby. }
 
\par{44.  ここは ${\overset{\textnormal{あつ}}{\text{暑}}}$ くないです \hfill\break
 \emph{Koko wa atsukunai desu. \hfill\break
 }It isn't hot here. }

\par{45. ウサギはかわいいです。 \hfill\break
 \emph{Usagi wa kawaii desu. }\hfill\break
Rabbits are cute. }

\begin{center}
 \textbf{Conjugation Recap }
\end{center}

\begin{ltabulary}{|P|P|P|P|P|P|}
\hline 

Meaning \textrightarrow  & Light & Heavy & Bitter & Spicy & Salty \\ \cline{1-6}

Adjective \textrightarrow  \hfill\break
Conjugations ↓ &  \emph{Karui }軽い &  \emph{Omoi }重い &  \emph{Nigai }苦い &  \emph{Karai }辛い &  \emph{Shiokarai }塩辛い \\ \cline{1-6}

Plain Non-Past & かるい & おもい & にがい & からい & しおからい \\ \cline{1-6}

Polite Non-Past & かるいです & おもいです & にがいです & からいです & しおからいです \\ \cline{1-6}

Plain Past & かるかった & おもかった & にがかった & からかった & しおからかった \\ \cline{1-6}

Polite Past & かるかったです & おもかったです & にがかったです & からかったです & しおからかったです \\ \cline{1-6}

Plain Negative & かるくない & おもくない & にがくない & からくない & しおからくない \\ \cline{1-6}

Polite Negative 1 & かるくないです & おもくないです & にがくないです & からくないです & しおからくないです \\ \cline{1-6}

Polite Negative 2 & かるくありません & おもくありません & にがくありません & からくありません & しおからくありません \\ \cline{1-6}

Plain Negative-Past & かるくなかった & おもくなかった & にがくなかった & からくなかった & しおからくなかった \\ \cline{1-6}

Polite Negative-Past 1 & かるくなかったです & おもくなかったです & にがくなかったです & からくなかったです & しおからくなかったです \\ \cline{1-6}

Polite Negative-Past 2 & かるくありませんでした & おもくありませんでした & にがくありませんでした & からくありませんでした & しおからくありませんでした \\ \cline{1-6}

\end{ltabulary}

\par{\textbf{Chart Note }: For brevity, conjugations are shown only in \emph{Hiragana }ひらがな. }
    