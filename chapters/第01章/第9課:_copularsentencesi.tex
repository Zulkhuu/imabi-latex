    
\chapter{Copular Sentences I}

\begin{center}
\begin{Large}
第9課: Copular Sentences I: Plain Speech 
\end{Large}
\end{center}
 
\par{ The Japanese language possesses a speech-level hierarchy that determines how one should address any given person based on various factors: relationship, age, role, respect, etc. The social dynamics that set a discourse also shape how said discourse is worded. }
 
\par{ Typically, Japanese learners are first introduced to polite speech. This is because polite speech is what is used in most daily interactions; it helps establish courtesy between oneself and those around. As the Japanese learner, the most practical means of using Japanese early on will be centered around speaking to native speakers who are neither family nor close friends. Because the most important use of plain speech involves casual conversation, it is kept to the side until the learner can handle situations where polite speech is imperative. }

\par{ However, the greatest flaw made by introducing polite speech first lies in the fact that it is not the base form of speech. Within each speech style you will find unique vocabulary, grammar, and endings. Grammatically, plain speech is what makes up the base form of the language. Contrary to its name, plain speech is not limited to conversation among peers or family. In fact, it is grammaticalized in all sorts of grammar points. }

\par{ Plain speech is also inherently direct, which is why it is heavily used in academic writing. Most importantly, it is what\textquotesingle s used in one\textquotesingle s inner monologue. Plain speech also makes up the heart of most music and literature. The very essence of being able to think in Japanese requires oneself to truly understand the language from the ground up. That cannot be possible if the base is left ignored. }
 Politeness is an auxiliary element to conversation. Its purpose is not to provide information other than social implications. Strip it away and you get the actual message a sentence is trying to get across. Naturally, plain speech becomes the basis for conjugation, to which politeness is then added.       
\section{Vocabulary List}
 
\par{\textbf{Nouns }}
 
\par{・ \emph{Tera }寺 – Buddhist temple }

\par{・ \emph{Jijitsu }事実 – Fact }

\par{・ \emph{K }\emph{ōmori }コウモリ – Bat }

\par{・ \emph{Tori }鳥 – Bird }

\par{・ \emph{Gakusei }学生 – Student }

\par{・ \emph{Kankokujin }韓国人 – Korean }

\par{・ \emph{Shod }\emph{ō }書道 – Calligraphy }

\par{・ \emph{Geijutsu }芸術 – Art }

\par{・ \emph{Baka }馬鹿 – Idiot }

\par{・ \emph{Neko }猫 – Cat }

\par{・ \emph{P }\emph{ātii }パーティー – Party }

\par{・ \emph{Kaishi }開始 – Start\slash beginning }

\par{・ \emph{D }\emph{ōbutsuen }動物園 – Zoo }

\par{・ \emph{Hikiwake }引き分け – A tie }

\par{・ \emph{Kawauso }カワウソ – Otter }

\par{・ \emph{Furansugo }フランス語 – French }

\par{・ \emph{Gakk }\emph{ō }学校 – School }

\par{・ \emph{Seikai }正解 – Correct answer }

\par{・ \emph{Wakusei }惑星 – Planet }

\par{・ \emph{Mei }\emph{ōsei }冥王星 – Pluto }

\par{・ \emph{Ocha }お茶 – Tea }

\par{・ \emph{Daihy }\emph{ō }代表 – Representative }

\par{・ \emph{Gen\textquotesingle in }原因 – Cause }

\par{・ \emph{Tabako }煙草・タバコ・たばこ – Tobacco }

\par{・ \emph{Kodomo }子供 – Child(ren) }

\par{・ \emph{Pen }ペン – Pen }
 
\par{・ \emph{Kujira }鯨 – Whale }

\par{・ \emph{Sakana }魚 – Fish }

\par{・ \emph{Mogi shiken }模擬試験 – Mock exam }

\par{・ \emph{Mizu }水 – Water }

\par{・ \emph{Jikan }時間 – Time }

\par{・ \emph{Ganjitsu }元日 – New Year\textquotesingle s Day }

\par{・ \emph{Getsuy }\emph{ōbi }月曜日 – Monday }

\par{・ \emph{Suiy }\emph{ōbi }水曜日 – Wednesday }

\par{・ \emph{Kaishibi }開始日 – Start date }

\par{・ \emph{Kin }\emph{ō }昨日 – Yesterday }

\par{・ \emph{Ashita\slash asu }明日 – Tomorrow }

\par{・ \emph{Yoru }夜 – Night }

\par{\textbf{Pronouns }}

\par{・ \emph{Watashi }私 – I }
 
\par{・ \emph{Boku }僕 – I (male) }
 
\par{・ \emph{Kare }彼 – He }
 
\par{・ \emph{Kanojo }彼女 – She }
 
\par{・ \emph{Kore }これ – This }
 
\par{・ \emph{Sore }それ – That }
 
\par{・ \emph{Sono }その – That (adj.) }
 
\par{・ \emph{Are }あれ – That (over there) }
 
\par{・ \emph{Koko }ここ – Here }
 
\par{・ \emph{Kimu }キム – Kim }
 
\par{・ \emph{Pikach }\emph{ū }ピカチュウ – Pikachu }
 
\par{\textbf{Interjections }}
・ \emph{A }あ – Ah       
\section{The Copular Sentence}
 
\par{ The first thing you must learn about Japanese sentence structure is its most basic form: the copular sentence. This is otherwise known as a “noun-predicate” sentence. In other words, “X is Y.” As trivial as it may sound, many far more complex sentences can be broken down to this very structure. First, let\textquotesingle s cover some basic terminology to better understand this topic. }

\begin{itemize}

\item \textbf{\textbf{Subject }: \emph{The person\slash thing that performs the action or exhibits the description found in the predicate. }}
\item \textbf{Predicate }: \emph{The part of a sentence that gives some information about the subject. }
\item \textbf{Copula }: \emph{A word used to link the subject and predicate of a sentence. }\hfill\break

\item \textbf{Noun }: \emph{In its most basic definition, a word that refers to a person, place, thing, event, substance, or quality. }\hfill\break

\item \textbf{Auxiliary }: \emph{An ending that helps construct verbal conjugations. }
\end{itemize}
 
\par{ The predicate of a sentence may take on different forms depending on what the statement is. In the context of this discussion, the copula is the predicate because we are learning how to simply say “X”—the subject—is “Y.” “Y” In this lesson, “Y” will be another noun, which is why “copular sentences” can alternatively be called “noun-predicate sentences.” }
 
\par{ In English, the copula verb is “to be,” and it manifests itself in various forms such as “is,” “are,” “was,” “were,” etc. Their use in the English language is profoundly important as they form the basis of a great portion of the statements we make. }
 
\par{i. The dog is a German shepherd. \hfill\break
ii. My husband is a banker. \hfill\break
iii. Apples are fruits. \hfill\break
iv. It was a fossil. \hfill\break
v. A bat is not a bird. }
 
\par{ Similarly, Japanese has its own copular verb, which in turn has its own various forms. Before discussing what this all looks like in Japanese, we must first understand what sort of basic conjugations exist in general. Using the English examples i.-v. as a basis, we see that \textbf{tense }and \textbf{affirmation\slash negation }are major components to a sentence. In English, there are three tenses: \textbf{past, present, and future }. As their names suggest, the \textbf{past tense } \emph{refers to an event\slash state which occurred in the past }, the \textbf{present tense } \emph{refers to a current event\slash state }, and the \textbf{future tense } \emph{refers to an event\slash state that hasn\textquotesingle t yet realized }. \textbf{Affirmation }is \emph{positively stating that something is so }. \textbf{Negation }is \emph{rejecting a premise }. }
 
\par{ Japanese only has two tenses: non-past and past tense. Unlike English, tense is not so straightforward, but the speaker\textquotesingle s intent is to always make the “time” factor of any statement obvious in context. The non-past tense encompasses both the concepts of the English present tense and future tense. The past tense corresponds to the past tense, but the form that expresses past tense covers a wider semantic scope than the English -ed. }
      
\section{The Copula Da だ}
 
\par{ Putting all this aside, it is now time to familiarize yourself with the base form of the copula in plain speech. This verb is \emph{\textbf{da }}\textbf{だ }.  As is the case for any base form of verb, it alone may stand for the non-past tense. As such, \emph{da }だ can translate as “is,” “are,” or “will be.” Japanese lacks grammatical number, so there is no difference between “is” or “are.” }

\par{ Because the verb of a Japanese sentence must always be at the end, we can\textquotesingle t simply insert \emph{da }だ between “X” and “Y.” “X” remains at the start of the sentence, and the sentence ends in “Y \emph{da }だ.”  To complete the sentence, we will insert the particle \emph{wa }は in between X and Y. In Lessons 11-12, we\textquotesingle ll learn about the sort of nuances that are expressed with this particle as well as what else can be between X and Y. For now, though, our goal will be to master the basic pattern “X \emph{wa }は Y \emph{da }だ.” }

\begin{center}
\textbf{Non-past: Present }
\end{center}

\par{1. あれは ${\overset{\textnormal{てら}}{\text{寺}}}$ だ。 \hfill\break
\emph{Are wa tera da. \hfill\break
}That (over there) is a Buddhist temple. }

\par{2. それは ${\overset{\textnormal{うそ}}{\text{嘘}}}$ だ。 \hfill\break
 \emph{Sore wa uso da. \hfill\break
 }That\textquotesingle s a lie. }

\par{3. これは ${\overset{\textnormal{じじつ}}{\text{事実}}}$ だ。 \hfill\break
 \emph{Kore wa jijitsu da. \hfill\break
 }This is the truth. }

\par{4. ${\overset{\textnormal{わたし}}{\text{私}}}$ は ${\overset{\textnormal{がくせい}}{\text{学生}}}$ だ。 \hfill\break
 \emph{Watashi wa gakusei da. \hfill\break
 }I\textquotesingle m a student. }

\par{5. キムは ${\overset{\textnormal{かんこくじん}}{\text{韓国人}}}$ だ。 \hfill\break
 \emph{Kimu wa kankokujin da. \hfill\break
 }Kim is Korean. }

\par{6. ${\overset{\textnormal{しょどう}}{\text{書道}}}$ は ${\overset{\textnormal{げいじゅつ}}{\text{芸術}}}$ だ。 \hfill\break
 \emph{Shod }\emph{ō wa geijutsu da. \hfill\break
 }Calligraphy is art. }

\par{7. ${\overset{\textnormal{かれ}}{\text{彼}}}$ は ${\overset{\textnormal{ばか}}{\text{馬鹿}}}$ だ。 \hfill\break
 \emph{Kare wa baka da. \hfill\break
 }He\textquotesingle s an idiot. }

\begin{center}
\textbf{Non-Past: Future }
\end{center}

\par{8. ${\overset{\textnormal{がんじつ}}{\text{元日}}}$ は ${\overset{\textnormal{げつようび}}{\text{月曜日}}}$ だ。 \hfill\break
 \emph{Ganjitsu wa getsuy }\emph{ōbi da. \hfill\break
 }New Year\textquotesingle s Day is\slash will be on Monday. }

\par{9. ${\overset{\textnormal{かいしび}}{\text{開始日}}}$ は ${\overset{\textnormal{あした}}{\text{明日}}}$ だ。 \hfill\break
 \emph{Kaishibi wa ashita da. \hfill\break
 }The start date is\slash will be tomorrow. \hfill\break
 \hfill\break
10. パーティーは ${\overset{\textnormal{よる}}{\text{夜}}}$ だ。 \hfill\break
 \emph{P }\emph{ātii wa yoru da. \hfill\break
 }The party will be at night. }

\begin{center}
\textbf{Omitting “X” }
\end{center}

\par{ In Japanese, the subject is often dropped in the sentence. This tends to be the case, especially when the subject is “it. }

\par{11. ${\overset{\textnormal{あした}}{\text{明日}}}$ だ。 \hfill\break
 \emph{Ashita da. \hfill\break
 }It\textquotesingle s tomorrow. \hfill\break
It\textquotesingle ll be tomorrow. }

\par{12. あ、 ${\overset{\textnormal{ねこ}}{\text{猫}}}$ だ! \hfill\break
 \emph{A, neko da. \hfill\break
 }Ah, (it\textquotesingle s) a cat! }

\par{13. ${\overset{\textnormal{じかん}}{\text{時間}}}$ だ。 \hfill\break
 \emph{Jikan da. \hfill\break
 }(It\textquotesingle s) time. }

\begin{center}
\textbf{Omitting \emph{Da }だ }
\end{center}

\par{ The copula \emph{da }だ is also occasionally dropped altogether with a heightened intonation at the end to express various emotions such as anger or surprise. Dropping the copula may also be done in this fashion in English. }

\par{14. あ、ピカチュウ(だ)! \hfill\break
 \emph{A, Pikachū (da)! \hfill\break
 }Ah, (it\textquotesingle s) Pikachu! }

\par{15. ${\overset{\textnormal{かいし}}{\text{開始}}}$ (だ)! \hfill\break
 \emph{Kaishi (da)! \hfill\break
 }Start! \hfill\break
Literally: This is the start! \hfill\break
 \hfill\break
16. ${\overset{\textnormal{どうぶつえん}}{\text{動物園}}}$ (だ)! \hfill\break
 \emph{Dōbutsuen (da)! \hfill\break
 }(It\textquotesingle s) a zoo! }

\par{17. ${\overset{\textnormal{ひ}}{\text{引}}}$ き ${\overset{\textnormal{わ}}{\text{分}}}$ げ(だ)! \hfill\break
 \emph{Hikiwake (da)! \hfill\break
 }(It\textquotesingle s a) draw! }

\par{18. あ、カワウソ(だ)! \hfill\break
 \emph{A, kawauso (da)! \hfill\break
 }Ah, (it\textquotesingle s) an otter! }

\begin{center}
\textbf{Past Tense: \emph{Datta }だった }
\end{center}

\par{ To express past tense with the copula \emph{da }だ, you must conjugate to \emph{datta }だった. As you learn more, you will see that -TA stands for -ed in anything that conjugates. Remember, Japanese makes no distinctions with grammatical number. This means that “was” and “were” are both expressed with \emph{datta }だった. }

\begin{center}
Conjugation Recap 
\end{center}

\begin{ltabulary}{|P|P|}
\hline 

 Non-Past Tense & Past Tense \\ \cline{1-2}

 \emph{Da }だ &  \emph{Datta }だった \\ \cline{1-2}

\end{ltabulary}

\par{\hfill\break
19. あれはフランス ${\overset{\textnormal{ご}}{\text{語}}}$ だった。 \hfill\break
 \emph{Are wa Furansugo datta. \hfill\break
 }That was French. }

\par{20. ここは ${\overset{\textnormal{がっこう}}{\text{学校}}}$ だった。 \hfill\break
 \emph{Koko wa gakk }\emph{ō datta. \hfill\break
 }This here was a school. }

\par{21. ${\overset{\textnormal{せいかい}}{\text{正解}}}$ は ${\overset{\textnormal{エイ}}{\text{A}}}$ だった。 \hfill\break
 \emph{Seikai wa ei datta. \hfill\break
 }The correct answer was A. }

\par{22. ${\overset{\textnormal{きのう}}{\text{昨日}}}$ は ${\overset{\textnormal{すいようび}}{\text{水曜日}}}$ だった。 \hfill\break
 \emph{Kin }\emph{ō wa suiy }\emph{ōbi datta. \hfill\break
 }Yesterday was Wednesday. }

\par{23. ${\overset{\textnormal{かれ}}{\text{彼}}}$ は ${\overset{\textnormal{こども}}{\text{子供}}}$ だった。 \hfill\break
 \emph{Kare wa kodomo datta. \hfill\break
 }He was a child. }

\par{\textbf{Grammar Note }: The past tense form need not always be interpreted literally. Ex. 23 implies that a male individual happened to be a child and is said as a remember to oneself and\slash or to others. }

\begin{center}
\textbf{Negation: \emph{De wa nai }ではない }
\end{center}

\par{ Conjugating \emph{da }だ into its plain non-past negative form is not as easy as the past tense form. First, you must change \emph{da }だ to \emph{de }で. Then, you add \emph{wa nai }はない. In reality, it\textquotesingle s the \emph{nai }ない that brings about the negation, which you\textquotesingle ll continue seeing in negative conjugations. Lastly, in conversation, “ \emph{de wa }では” typically contracts to “ \emph{ja }じゃ.” }

\begin{center}
Conjugation Recap 
\end{center}

\begin{ltabulary}{|P|P|P|}
\hline 

Non-Past Tense & Past Tense & Non-Past Negative \\ \cline{1-3}

 \emph{Da }だ &  \emph{Datta }だった &  \emph{De wa nai }ではない \hfill\break
 \emph{Ja nai }じゃない  \\ \cline{1-3}

\end{ltabulary}
\hfill\break
24. これはペンではない。 \emph{Kore wa pen de wa nai. } This is not a pen. \hfill\break
\hfill\break
25. コウモリは ${\overset{\textnormal{とり}}{\text{鳥}}}$ ではない。 \emph{K }\emph{ōmori wa tori de wa nai. } Bats are not birds. \hfill\break
\hfill\break
26. ${\overset{\textnormal{くじら}}{\text{鯨}}}$ は ${\overset{\textnormal{さかな}}{\text{魚}}}$ ではない。 \emph{Kujira wa sakana de wa nai. } Whales are not fish. \hfill\break
\hfill\break
27. ${\overset{\textnormal{めいおうせい}}{\text{冥王星}}}$ は ${\overset{\textnormal{わくせい}}{\text{惑星}}}$ じゃない。 \emph{Mei }\emph{ōsei wa wakusei ja nai. } Pluto isn\textquotesingle t a planet. \hfill\break
\hfill\break
28. あれは ${\overset{\textnormal{いぬ}}{\text{犬}}}$ じゃない。 \emph{Are wa inu ja nai. } That isn\textquotesingle t a dog. \hfill\break
\hfill\break
29. これはお ${\overset{\textnormal{ちゃ}}{\text{茶}}}$ じゃない。 \emph{Kore wa ocha ja nai. } This isn\textquotesingle t tea. 
\par{\textbf{Grammar Note }: Saying \emph{de nai }でない isn\textquotesingle t wrong, but it is typically only seen in literature. }

\begin{center}
\textbf{Negative-Past: \emph{De wa nakatta }ではなかった }
\end{center}

\par{ The last conjugation we will study in this lesson is the plain negative-past form of the copula. To begin, we start with the negative form from above. We then add -TA to it. When you add -TA to the negative auxiliary - \emph{nai }ない, you get \emph{nakatta }なかった. Altogether, this gives you \emph{de wa nakatta }ではなかった. Just like above, “ \emph{de wa }では” often contracts to “ \emph{ja }じゃ” in conversation, which results in \emph{ja nakatta }じゃなかった. }

\begin{center}
Conjugation Recap 
\end{center}

\begin{ltabulary}{|P|P|P|P|}
\hline 

Non-past & Past & Negative & Negative-Past \\ \cline{1-4}

 \emph{Da }だ &  \emph{Datta }だった & \emph{De wa nai }ではない \hfill\break
\emph{Ja nai }じゃない &  \emph{De wa nakatta } ではなかった \hfill\break
\emph{Ja nakatta } じゃなかった  \\ \cline{1-4}

\end{ltabulary}
\hfill\break
30. その ${\overset{\textnormal{だいひょう}}{\text{代表}}}$ は ${\overset{\textnormal{かのじょ}}{\text{彼女}}}$ ではなかった。 \emph{Sono daihy }\emph{ō wa kanojo de wa nakatta. } The representative was not her.  \textbf{Grammar Note }: The demonstrative pronouns briefly mentioned in Lesson 8 actually have adjectival forms. The adjectival form for \emph{sore }それ is \emph{sono }その, as demonstrated in Ex. 30. This form often translates to "the" whenever the concepts of "the" and "that" overlap. \hfill\break
\hfill\break
31. ${\overset{\textnormal{げんいん}}{\text{原因}}}$ は ${\overset{\textnormal{たばこ}}{\text{煙草}}}$ ではなかった。 \emph{Gen\textquotesingle in wa tabako de wa nakatta. } The cause was not tobacco. \hfill\break
\hfill\break
32. あれは ${\overset{\textnormal{もぎしけん}}{\text{模擬試験}}}$ ではなかった。 \emph{Are wa mogi shiken de wa nakatta. } That was not a mock exam. \hfill\break
\hfill\break
33. ${\overset{\textnormal{かれ}}{\text{彼}}}$ は ${\overset{\textnormal{ぼく}}{\text{僕}}}$ の ${\overset{\textnormal{ともだち}}{\text{友達}}}$ じゃなかった。 \emph{Kare wa boku no tomodachi ja nakatta } He wasn\textquotesingle t my friend. \hfill\break
\hfill\break
34. あれは ${\overset{\textnormal{みず}}{\text{水}}}$ じゃなかった。 \emph{Are wa mizu ja nakatta. } That wasn\textquotesingle t water. \hfill\break
\hfill\break
35. それは ${\overset{\textnormal{うそ}}{\text{嘘}}}$ じゃなかった。 \emph{Sore wa uso ja nakatta. } That wasn\textquotesingle t a lie.     