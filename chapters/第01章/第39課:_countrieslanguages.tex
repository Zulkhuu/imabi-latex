    
\chapter{Countries, Nationality, \& Languages}

\begin{center}
\begin{Large}
第39課: Countries, Nationality, \& Languages 
\end{Large}
\end{center}
 
\par{ In this lesson, you will be introduced to the Japanese names of the countries, nationalities, and languages found in the world. We will also learn about the diversity found in Japanese dialects. There is no need to learn the name of every country, nationality, or language mentioned. Use this opportunity to improve upon your transliteration skill of foreign place names so that you may become better able to describe yourself in Japanese. }
      
\section{Countries of the World}
 
\par{ Currently, there are 197 recognized independent states in the world. Each country has a Japanese name, many of which come from transliterations of their true names. As such, you will be able to recognize many country names without having to necessarily study them. To refer to nationality ( \emph{kokuseki }国籍), all you need to do is add the suffix - \emph{jin }人 to the name of the country in question. To refer to the language of said country, simply add the suffix - \emph{go }語. These suffixes are also used regardless if a group of people has its own country or not. Because there are far more languages than there are countries, we will first go over the names of countries and nationalities. }

\par{ Spelling Note: Most country names are written in \emph{Katakana }カタカナ. Only a handful, those typically found in Asia, have spellings in \emph{Kanji }漢字. Although all country names have \emph{Kanji }漢字 spellings, only a handful are used. In this lesson, we will only concern ourselves with the standard spelling of each country name. After all, you have plenty of countries to go over. }

\begin{center}
\textbf{Countries of Eurasia }
\end{center}

\par{ Eurasia in Japanese is \emph{Yūrashia(-tairiku) }ユーラシア(大陸). In America, Europe ( \emph{Yōroppa(-tairiku) }ヨーロッパ(大陸)) and Asia ( \emph{Ajia-(tairiku) }アジア(大陸)) are typically treated as separate continents, but in Japan they are viewed as one. In Japanese, the six continents are either referred to as \emph{rokudaishū }六大州 or \emph{rokutairiku }六大陸. The difference between \emph{taishū }大州 and \emph{tairiku }大陸 is that the former refers to all landmasses on a continental plate whereas the latter refers solely to the continents. The - \emph{tairiku }大陸 seen after the name of the continents is only used when referring to the continents in the sense of geography. In everyday speech, the continents simply go by their names. }

\par{\textbf{Chart Notes }: }

\par{1. \emph{Kokumei }国名 = Country name; \emph{kokuseki }国籍 = Nationality. \hfill\break
2. \emph{Tsūshō }通称 means “popular name” and refers to the fact that the names listed are not the full-length names every nation has. These names are the ones that are used in actual conversation. \hfill\break
3. Romanization will not be provided in the country name charts of this lesson for brevity and conciseness. Use this opportunity to practice \emph{Katakana }カタカナ. }

\begin{ltabulary}{|P|P|P|P|P|P|}
\hline 
 
   ${\overset{\textnormal{こくめい}}{\text{国名}}}$ 
 &    ${\overset{\textnormal{つうしょう}}{\text{通称}}}$ 
 &    ${\overset{\textnormal{こくせき}}{\text{国籍}}}$ 
 &    ${\overset{\textnormal{こくめい}}{\text{国名}}}$ 
 &    ${\overset{\textnormal{つうしょう}}{\text{通称}}}$ 
 &    ${\overset{\textnormal{こくせき}}{\text{国籍}}}$ 
 \\ \cline{1-6} 
 
  China 
 &    ${\overset{\textnormal{ちゅうごく}}{\text{中国}}}$ 
 &    ${\overset{\textnormal{ちゅうごくじん}}{\text{中国人}}}$ 
 &   India 
 &   インド 
 &   インド人 
 \\ \cline{1-6} 
 
  Indonesia 
 &   インドネシア 
 &   インドネシア人 
 &   Pakistan 
 &   パキスタン 
 &   パキスタン人 
 \\ \cline{1-6} 
 
  Bangladesh 
 &   バングラデシュ 
 &   バングラデシュ人 
 &   Russia 
 &   ロシア 
 &   ロシア人 
 \\ \cline{1-6} 
 
  Japan 
 &    ${\overset{\textnormal{にほん}}{\text{日本}}}$ 
 &    ${\overset{\textnormal{にほんじん}}{\text{日本人}}}$ 
 &   Philippines 
 &   フィリピン 
 &   フィリピン人 
 \\ \cline{1-6} 
 
  Vietnam 
 &   ベトナム 
 &   ベトナム人 
 &   Germany 
 &   ドイツ 
 &   ドイツ人 
 \\ \cline{1-6} 
 
  Iran 
 &   イラン 
 &   イラン人 
 &   Turkey 
 &   トルコ 
 &   トルコ人 
 \\ \cline{1-6} 
 
  Thailand 
 &   タイ 
 &   タイ人 
 &   United Kingdom 
 &   イギリス 
 &   イギリス人 
 \\ \cline{1-6} 
 
  France 
 &   フランス 
 &   フランス人 
 &   Italy 
 &   イタリア 
 &   イタリア人 
 \\ \cline{1-6} 
 
  Myanmar 
 &   ミャンマー 
 &   ミャンマー人 
 &   South Korea 
 &    ${\overset{\textnormal{かんこく}}{\text{韓国}}}$ 
 &    ${\overset{\textnormal{かんこくじん}}{\text{韓国人}}}$ 
 \\ \cline{1-6} 
 
  Spain 
 &   スペイン 
 &   スペイン人 
 &   Ukraine 
 &   ウクライナ 
 &   ウクライナ人 
 \\ \cline{1-6} 
 
  Poland 
 &   ポーランド 
 &   ポーランド人 
 &   Iraq 
 &   イラク 
 &   イラク人 
 \\ \cline{1-6} 
 
  Saudi Arabia 
 &   サウジアラビア 
 &   サウジアラビア人 
 &   Uzbekistan 
 &   ウズベキスタン 
 &   ウズベキスタン人 
 \\ \cline{1-6} 
 
  Malaysia 
 &   マレーシア 
 &   マレーシア人 
 &   Nepal 
 &   ネパール 
 &   ネパール人 
 \\ \cline{1-6} 
 
  Afghanistan 
 &   アフガニスタン 
 &   アフガニスタン 
 &   Yemen 
 &   イエメン 
 &   イエメン人 
 \\ \cline{1-6} 
 
  North Korea 
 &    ${\overset{\textnormal{きたちょうせん}}{\text{北朝鮮}}}$ 
 &    ${\overset{\textnormal{きたちょうせんじん}}{\text{北朝鮮人}}}$ 
 &   Taiwan 
 &    ${\overset{\textnormal{たいわん}}{\text{台湾}}}$ 
 &    ${\overset{\textnormal{たいわんじん}}{\text{台湾人}}}$ 
 \\ \cline{1-6} 
 
  Syria 
 &   シリア 
 &   シリア人 
 &   Sri Lanka 
 &   スリランカ 
 &   スリランカ人 
 \\ \cline{1-6} 
 
  Romania 
 &   ルーマニア 
 &   ルーマニア人 
 &   Kazakhstan 
 &   カザフスタン 
 &   カザフスタン人 
 \\ \cline{1-6} 
 
  Netherlands 
 &   オランダ 
 &   オランダ人 
 &   Cambodia 
 &   カンボジア 
 &   カンボジア人 
 \\ \cline{1-6} 
 
  Belgium 
 &   ベルギー 
 &   ベルギー人 
 &   Greece 
 &   ギリシャ 
 &   ギリシャ人 
 \\ \cline{1-6} 
 
  Czech Republic 
 &   チェコ 
 &   チェコ人 
 &   Portugal 
 &   ポルトガル 
 &   ポルトガル人 
 \\ \cline{1-6} 
 
  Hungary 
 &   ハンガリー 
 &   ハンガリー人 
 &   Sweden 
 &   スウェーデン 
 &   スウェーデン人 
 \\ \cline{1-6} 
 
  Azerbaijan 
 &   アゼルバイジャン 
 &   アゼルバイジャン人 
 &   Belarus 
 &   ベラルーシ 
 &   ベラルーシ人 
 \\ \cline{1-6} 
 
  UAE 
 &   アラブ ${\overset{\textnormal{しゅちょうこくれんぽう}}{\text{首長国連邦}}}$ 
 &   アラブ ${\overset{\textnormal{しゅちょうこくれんぽうじん}}{\text{首長国連邦人}}}$ 
 &   Austria 
 &   オーストリア \hfill\break
オーストリー 
 &   オーストリア人 
 \\ \cline{1-6} 
 
  Tajikistan 
 &   タジキスタン 
 &   タジキスタン人 
 &   Israel 
 &   イスラエル 
 &   イスラエル人 
 \\ \cline{1-6} 
 
  Switzerland 
 &   スイス 
 &   スイス人 
 &   Hong Kong 
 &    ${\overset{\textnormal{ほんこん}}{\text{香港}}}$ 
 &    ${\overset{\textnormal{ほんこんじん}}{\text{香港人}}}$ 
 \\ \cline{1-6} 
 
  Bulgaria 
 &   ブルガリア 
 &   ブルガリア人 
 &   Serbia 
 &   セルビア 
 &   セルビア人 
 \\ \cline{1-6} 
 
  Jordan 
 &   ヨルダン 
 &   ヨルダン人 
 &   Laos 
 &   ラオス 
 &   ラオス人 
 \\ \cline{1-6} 
 
  Kyrgyzstan 
 &   キルギス 
 &   キルギス人 
 &   Denmark 
 &   デンマーク 
 &   デンマーク人 
 \\ \cline{1-6} 
 
  Singapore 
 &   シンガポール 
 &   シンガポール人 
 &   Finland 
 &   フィンランド 
 &   フィンランド人 
 \\ \cline{1-6} 
 
  Slovakia 
 &   スロバキア 
 &   スロバキア人 
 &   Norway 
 &   ノルウェー 
 &   ノルウェー人 
 \\ \cline{1-6} 
 
  Turkmenistan 
 &   トルクメニスタン 
 &   トルクメニスタン人 
 &   Palestine 
 &   パレスチナ 
 &   パレスチナ人 
 \\ \cline{1-6} 
 
  Ireland 
 &   アイルランド 
 &   アイルランド人 
 &   Lebanon 
 &   レバノン 
 &   レバノン人 
 \\ \cline{1-6} 
 
  Croatia 
 &   クロアチア 
 &   クロアチア人 
 &   Oman 
 &   オマーン 
 &   オマーン人 
 \\ \cline{1-6} 
 
  Kuwait 
 &   クウェート 
 &   クウェート人 
 &   Bosnia and Herzegovina 
 &   ボスニアヘルツェゴビナ 
 &   ボスニアヘルツェゴビナ人 \hfill\break
ボスニア人 \hfill\break
ヘルツェゴビナ人 
 \\ \cline{1-6} 
 
  Georgia 
 &   ジョージア \hfill\break
グルシア 
 &   グルシア人 
 &   Moldova 
 &   モルドバ 
 &   モルドバ人 
 \\ \cline{1-6} 
 
  Mongolia 
 &   モンゴル 
 &   モンゴル人 
 &   Armenia 
 &   アルメニア 
 &   アルメニア人 
 \\ \cline{1-6} 
 
  Lithuania 
 &   リトアニア 
 &   リトアニア人 
 &   Albania 
 &   アルバニア 
 &   アルバニア人 
 \\ \cline{1-6} 
 
  Qatar 
 &   カタール 
 &   カタール人 
 &   Macedonia 
 &   マケドニア 
 &   マケドニア人 
 \\ \cline{1-6} 
 
  Slovenia 
 &   スロベニア 
 &   スロベニア人 
 &   Latvia 
 &   ラトビア 
 &   ラトビア人 
 \\ \cline{1-6} 
 
  Kosovo 
 &   コソボ 
 &   コソボ人 
 &   Bahrain 
 &   バハレーン 
 &   バハレーン人 
 \\ \cline{1-6} 
 
  Estonia 
 &   エストニア 
 &   エストニア人 
 &   East Timor 
 &    ${\overset{\textnormal{ひがし}}{\text{東}}}$ ティモール 
 &    ${\overset{\textnormal{ひがし}}{\text{東}}}$ ティモール人 
 \\ \cline{1-6} 
 
  Cyprus 
 &   キプロス 
 &   キプロス人 
 &   Bhutan 
 &   ブータン 
 &   ブータン人 
 \\ \cline{1-6} 
 
  Macau 
 &   マカオ 
 &   マカオ人 
 &   Montenegro 
 &   モンテネグロ 
 &   モンテネグロ人 
 \\ \cline{1-6} 
 
  Luxembourg 
 &   ルクセンブルク 
 &   ルクセンブルク人 
 &   Malta 
 &   マルタ 
 &   マルタ人 
 \\ \cline{1-6} 
 
  Brunei 
 &   ブルネイ 
 &   ブルネイ人 
 &   Maldives 
 &   モルディブ 
 &   モルディブ人 
 \\ \cline{1-6} 
 
  Iceland 
 &   アイスランド 
 &   アイスランド人 
 &   Andorra 
 &   アンドラ 
 &   アンドラ人 
 \\ \cline{1-6} 
 
  Monaco 
 &   モナコ 
 &   モナコ人 
 &   Liechtenstein 
 &   リヒテンシュタイン 
 &   リヒテンシュタイン人 
 \\ \cline{1-6} 
 
  San Marino 
 &   サンマリノ 
 &   サンマリノ人 
 &   Vatican City 
 &   バチカン ${\overset{\textnormal{しこく}}{\text{市国}}}$ 
 &   バチカン ${\overset{\textnormal{しこくみん}}{\text{市国民}}}$ 
 \\ \cline{1-6} 
 
\end{ltabulary}

\par{\textbf{Word Notes }: }

\par{1. In formal situations, 日本 is read as \emph{Nippon }. }

\par{2. \emph{Kokumin }国民 means “citizen” and is a more appropriate word to refer to the citizens of Vatican City, who are all associated with the religious establishment of the nation itself. \hfill\break
3. Some countries of Eurasia are abbreviated to single-character \emph{Kanji }漢字 spellings in writing, especially in news reports or non-fiction literature. Europe itself can also be abbreviated as \emph{Ō }欧. This is because another name for Europe is \emph{Ōshū }欧州, which is usually used in the written language. Asia may also be abbreviated as \emph{A }亜. This comes from the \emph{Kanji }漢字 spelling 亜細亜. }

\par{\textbf{Chart Notes }: }
1. Only abbreviations that are used to a significant degree are listed below. \hfill\break
2. \emph{Ryakushō }略称 = abbreviation(s); \emph{Kanji hyōki }漢字表記 = Kanji Spelling. \hfill\break
\hfill\break

\begin{ltabulary}{|P|P|P|P|P|P|}
\hline 
 
   ${\overset{\textnormal{こくめい}}{\text{国名}}}$ 
 &    ${\overset{\textnormal{かんじひょうき}}{\text{漢字表記}}}$ 
 &    ${\overset{\textnormal{りゃくしょう}}{\text{略称}}}$ 
 &    ${\overset{\textnormal{こくめい}}{\text{国名}}}$ 
 &    ${\overset{\textnormal{かんじひょうき}}{\text{漢字表記}}}$ 
 &    ${\overset{\textnormal{りゃくしょう}}{\text{略称}}}$ 
 \\ \cline{1-6} 
 
  Italy 
 &   伊太利亜 
 &    \emph{I }伊 
 &   Netherlands 
 &   阿蘭陀 
 &    \emph{Ran }蘭 
 \\ \cline{1-6} 
 
  France 
 &   仏蘭西 
 &    \emph{Futsu }仏 
 &   England 
 &   英吉利 
 &    \emph{Ei }英 
 \\ \cline{1-6} 
 
  Switzerland 
 &   瑞西 
 &    \emph{Sui }瑞 
 &   Russia 
 &   露西亜 
 &    \emph{Ro }露 
 \\ \cline{1-6} 
 
  Myanmar 
 &   緬甸 
 &    \emph{Men }緬 
 &   Turkey 
 &   土耳古 
 &    \emph{To }土 
 \\ \cline{1-6} 
 
  Portugal 
 &   葡萄牙 
 &    \emph{Po }葡 
 &   Germany 
 &   独逸 
 &    \emph{Doku }独 
 \\ \cline{1-6} 
 
  Spain 
 &   西班牙 
 &    \emph{Sei }西 
 &   Poland 
 &   波蘭 
 &    \emph{Po }波 
 \\ \cline{1-6} 
 
  Taiwan 
 &   台湾 
 &    \emph{Tai }台 
 &   Sweden 
 &   瑞典 
 &    \emph{Ten }典 
 \\ \cline{1-6} 
 
  Norway 
 &   諾威 
 &    \emph{Daku }諾 
 &   Thailand 
 &   泰 
 &    \emph{Tai }泰 
 \\ \cline{1-6} 
 
  Laos 
 &   羅宇 
 &    \emph{Ryō }寮・ \emph{Rō }老 
 &   Malaysia 
 &   馬来西亜 
 &    \emph{Ma }馬 
 \\ \cline{1-6} 
 
  Belgium 
 &   白耳義 
 &    \emph{Haku }白 
 &   Greece 
 &   希臘 
 &    \emph{Gi }希 
 \\ \cline{1-6} 
 
  Denmark 
 &   丁抹 
 &    \emph{Tei }丁 
 &   Finland 
 &   芬蘭 
 &    \emph{Fun }芬 
 \\ \cline{1-6} 
 
  India 
 &   印度 
 &    \emph{In }印 
 &   Singapore 
 &   新嘉坡・新加坡 
 &    \emph{Sei }星 
 \\ \cline{1-6} 
 
  Philippines 
 &   比律賓 
 &    \emph{Hi }比 
 &   Vietnam 
 &   越南 
 &    \emph{Etsu }越 
 \\ \cline{1-6} 
 
  South Korea 
 &   韓国 
 &    \emph{Kan }韓 
 &   China 
 &   中国 
 &    \emph{Chū }中・ \emph{Ka }華 
 \\ \cline{1-6} 
 
  Mongolia 
 &   蒙古 
 &    \emph{Mō }蒙 
 &   Japan 
 &   日本 
 &    \emph{Nichi }日・ \emph{Wa }和 
 \\ \cline{1-6} 
 
  North Korea 
 &   北朝鮮 
 &    \emph{Chō }朝 
 &     &     &     \\ \cline{1-6} 
 
\end{ltabulary}

\par{\textbf{Usage Notes }: }

\par{1. 中 is the typical abbreviation for China, but 華 can also be seen every so often, especially in various terminology. \hfill\break
2. The use of 和 to stand for Japan is ancient. Japan was once called \emph{Wa }in antiquity, and this name still lives on in many situations. The use of 和 is typically limited to compounds such as \emph{wayaku }和訳 (translating into Japanese). \hfill\break
3. 蒙古 is the old name for Mongolia and is read as \emph{Mōko }. \hfill\break
4. 老 for Laos comes from the Chinese name of the country; however, in other parts of Asia, the country is referred to as 寮國, and it is because of this that 寮 is also used in Japan. \hfill\break
4. Abbreviations tend to be added together, especially in news headlines. For instance, China-Korea will be abbreviated as \emph{Chūkan }中韓. Japan-Korea will be abbreviated as \emph{Nikkan }日韓. Another example is \emph{Nichiryō }日寮 (Japan-Laos). For other nation combinations, a \emph{nakaguro }中黒 (・) is usually inserted between the abbreviations. For example, 伊・仏stands for “Italy-France.” \hfill\break
5. Although some of these are not frequently used, those in this list are at least widely known. You will still see many of these used. }

\begin{center}
\textbf{Countries of the Americas }
\end{center}

\par{ North America, Central America, and South America are \emph{Kita Amerika }北アメリカ, \emph{Chūō Amerika }中央アメリカ, and \emph{Minami Amerika }南アメリカ respectively. They also go by the shortenings \emph{Hokubei }北米, \emph{Chūbei }中米, and \emph{Nambei }南米 respectively. In fact, these shortenings are used the most in conversation. The continents North America and South America are called \emph{Kita-Amerika-tairiku }北アメリカ大陸 and \emph{Minami-Amerika-tairiku }南アメリカ大陸 respectively. Below are the nations that can be found on these two continents: }

\begin{ltabulary}{|P|P|P|P|P|P|}
\hline 
 
   ${\overset{\textnormal{こくめい}}{\text{国名}}}$ 
 &    ${\overset{\textnormal{つうしょう}}{\text{通称}}}$ 
 &    ${\overset{\textnormal{こくせき}}{\text{国籍}}}$ 
 &    ${\overset{\textnormal{こくめい}}{\text{国名}}}$ 
 &    ${\overset{\textnormal{つうしょう}}{\text{通称}}}$ 
 &    ${\overset{\textnormal{こくせき}}{\text{国籍}}}$ 
 \\ \cline{1-6} 
 
  United States 
 &   アメリカ \hfill\break
 ${\overset{\textnormal{べいこく}}{\text{米国}}}$ 
 &   アメリカ人 \hfill\break
 ${\overset{\textnormal{べいこくじん}}{\text{米国人}}}$ 
 &   Brazil 
 &   ブラジル 
 &   ブラジル人 
 \\ \cline{1-6} 
 
  Mexico 
 &   メキシコ 
 &   メキシコ人 
 &   Colombia 
 &   コロンビア 
 &   コロンビア人 
 \\ \cline{1-6} 
 
  Argentina 
 &   アルゼンチン 
 &   アルゼンチン人 
 &   Canada 
 &   カナダ 
 &   カナダ人 
 \\ \cline{1-6} 
 
  Peru 
 &   ペルー 
 &   ペルー人 
 &   Venezuela 
 &   ベネズエラ 
 &   ベネズエラ人 
 \\ \cline{1-6} 
 
  Chile 
 &   チリ 
 &   チリ人 
 &   Ecuador 
 &   エクアドル 
 &   エクアドル人 
 \\ \cline{1-6} 
 
  Guatemala 
 &   グアテマラ 
 &   グアテマラ人 
 &   Cuba 
 &   キューバ 
 &   キューバ人 
 \\ \cline{1-6} 
 
  Haiti 
 &   ハイチ 
 &   ハイチ人 
 &   Bolivia 
 &   ボリビア 
 &   ボリビア人 
 \\ \cline{1-6} 
 
  Dominican Republic 
 &   ドミニカ ${\overset{\textnormal{きょうわこく}}{\text{共和国}}}$ 
 &   ドミニカ人 
 &   Honduras 
 &   ホンジュラス 
 &   ホンジュラス人 
 \\ \cline{1-6} 
 
  Paraguay 
 &   パラグアイ 
 &   パラグアイ人 
 &   Nicaragua 
 &   ニカラグア 
 &   ニカラグア人 
 \\ \cline{1-6} 
 
  El Salvador 
 &   エルサルバドル 
 &   エルサルバドル人 
 &   Costa Rica 
 &   コスタリカ 
 &   コスタリカ人 
 \\ \cline{1-6} 
 
  Panama 
 &   パナマ 
 &   パナマ人 
 &   Uruguay 
 &   ウルグアイ 
 &   ウルグアイ人 
 \\ \cline{1-6} 
 
  Jamaica 
 &   ジャマイカ 
 &   ジャマイカ人 
 &   Trinidad and Tobago 
 &   トリニダード・トバゴ 
 &   トリニダード・トバゴ人 
 \\ \cline{1-6} 
 
  Guyana 
 &   ガイアナ 
 &   ガイアナ人 
 &   Suriname 
 &   スリナム 
 &   スリナム人 
 \\ \cline{1-6} 
 
  Bahamas 
 &   バハマ 
 &   バハマ人 
 &   Belize 
 &   ベリーズ 
 &   ベリーズ人 
 \\ \cline{1-6} 
 
  Barbados 
 &   バルバドス 
 &   バルバドス人 
 &   French Guiana 
 &   フランス ${\overset{\textnormal{りょう}}{\text{領}}}$ ギアナ 
 &   ギアナ人 
 \\ \cline{1-6} 
 
  Saint Lucia 
 &   セントルシア 
 &   セントルシア人 
 &   Curaçao 
 &   キュラソー ${\overset{\textnormal{とう}}{\text{島}}}$ 
 &   キュラソー ${\overset{\textnormal{とうじん}}{\text{島人}}}$ 
 \\ \cline{1-6} 
 
  Aruba 
 &   アルバ 
 &   アルバ人 
 &   Saint Vincent and the   Grenadines 
 &   セントビンセント・グレナディーン 
 &   セントビンセント・グレナディーン 
 \\ \cline{1-6} 
 
  Grenada 
 &   グレナダ 
 &   グレナダ人 
 &   Antigua and Barbuda 
 &   アンティグア・バーブーダ 
 &   アンティグア・バーブーダ人 
 \\ \cline{1-6} 
 
  Dominica 
 &   ドミニカ 
 &   ドミニカ 
 &   Saint Kitts and Nevis 
 &   セントクリストファー・ネイビス 
 &   セントクリストファー・ネイビス 
 \\ \cline{1-6} 
 
\end{ltabulary}

\par{\textbf{Word Notes: }}

\par{1. \emph{Beikoku }米国 and \emph{Beikokujin }米国人 are typically only used in the written language. The full name of the United States in Japanese is \emph{Amerika Gasshūkoku }アメリカ合衆国. \hfill\break
2. \emph{Furansu-ryō }フランス領 means “French controlled.” \hfill\break
3. For nations that aren\textquotesingle t well known, the suffix - \emph{jin }人 may frequently be seen replaced with \emph{no hito }の人. \hfill\break
4. Because both the nationalities for the Dominican Republic and Dominica are rendered as \emph{Dominikajin }ドミニカ人, to distinguish the two, you can refer to people of the Dominican Republic as \emph{Dominika Kyōwakoku no hito }ドミニカ共和国の人 and people from Dominica as \emph{Dominika-koku no hito }ドミニカ国の人. \emph{Dominika-koku }ドミニカ国 is the full name for Dominica. \hfill\break
5. Just as was the case with Eurasian countries, there are several countries in the Americas whose names can be abbreviated to single-character spellings based on their \emph{Kanji }漢字 spellings. Below are the most common abbreviations: }

\begin{ltabulary}{|P|P|P|P|P|P|}
\hline 
 
   ${\overset{\textnormal{こくめい}}{\text{国名}}}$ 
 &    ${\overset{\textnormal{かんじひょうき}}{\text{漢字表記}}}$ 
 &    ${\overset{\textnormal{りゃくしょう}}{\text{略称}}}$ 
 &    ${\overset{\textnormal{こくめい}}{\text{国名}}}$ 
 &    ${\overset{\textnormal{かんじひょうき}}{\text{漢字表記}}}$ 
 &    ${\overset{\textnormal{りゃくしょう}}{\text{略称}}}$ 
 \\ \cline{1-6} 
 
  United States 
 &   亜米利加 
 &    \emph{Bei }米 
 &   Canada 
 &   加奈陀 
 &    \emph{Ka }加 
 \\ \cline{1-6} 
 
  Mexico 
 &   墨西哥 
 &    \emph{Boku }墨 
 &   Brazil 
 &   伯剌西爾 
 &    \emph{Haku }伯 
 \\ \cline{1-6} 
 
  Argentina 
 &   亜爾然丁 
 &    \emph{Ji }爾 
 &   Cuba 
 &   玖馬 
 &    \emph{Kyū }玖 
 \\ \cline{1-6} 
 
  Chile 
 &   智利 
 &    \emph{Chi }智 
 &   Peru 
 &   秘露 
 &    \emph{Hi }秘 
 \\ \cline{1-6} 
 
\end{ltabulary}

\par{\textbf{Usage Note }: Of these, the only ones important to remember are those for the United States and Canada. }

\begin{center}
\textbf{Countries of Africa }
\end{center}

\par{ Africa in Japanese is \emph{Afurika(-tairiku) }アフリカ(大陸) and is composed of the following nations: }

\begin{ltabulary}{|P|P|P|P|P|P|}
\hline 
 
   ${\overset{\textnormal{こくめい}}{\text{国名}}}$ 
 &    ${\overset{\textnormal{つうしょう}}{\text{通称}}}$ 
 &    ${\overset{\textnormal{こくせき}}{\text{国籍}}}$ 
 &    ${\overset{\textnormal{こくめい}}{\text{国名}}}$ 
 &    ${\overset{\textnormal{つうしょう}}{\text{通称}}}$ 
 &    ${\overset{\textnormal{こくせき}}{\text{国籍}}}$ 
 \\ \cline{1-6} 
 
  Nigeria 
 &   ナイジェリア 
 &   ナイジェリア人 
 &   Ethiopia 
 &   エチオピア 
 &   エチオピア人 
 \\ \cline{1-6} 
 
  Egypt 
 &   エジプト 
 &   エジプト人 
 &   Democratic Republic of the   Congo 
 &   コンゴ ${\overset{\textnormal{みんしゅきょうわこく}}{\text{民主共和国}}}$ 
 &   コンゴ人 
 \\ \cline{1-6} 
 
  Tanzania 
 &   タンザニア 
 &   タンザニア人 
 &   Sudan 
 &   スーダン 
 &   スーダン人 
 \\ \cline{1-6} 
 
  Morocco 
 &   モロッコ 
 &   モロッコ人 
 &   Kenya 
 &   ケニア 
 &   ケニア人 
 \\ \cline{1-6} 
 
  Algeria 
 &   アルジェリア 
 &   アルジェリア人 
 &   Uganda 
 &   ウガンダ 
 &   ウガンダ人 
 \\ \cline{1-6} 
 
  Angola 
 &   アンゴラ 
 &   アンゴラ人 
 &   Ghana 
 &   ガーナ 
 &   ガーナ人 
 \\ \cline{1-6} 
 
  Mozambique 
 &   モザンビーク 
 &   モザンビーク人 
 &   Madagascar 
 &   マダガスカル 
 &   マダガスカル人 
 \\ \cline{1-6} 
 
  Cameroon 
 &   カメルーン 
 &   カメルーン人 
 &   Ivory Coast 
 &   コートジボワール 
 &   コートジボワール人 
 \\ \cline{1-6} 
 
  Zambia 
 &   ザンビア 
 &   ザンビア人 
 &   Niger 
 &   ニジェール 
 &   ニジェール人 
 \\ \cline{1-6} 
 
  Mali 
 &   マリ 
 &   マリ人 
 &   Burkina Faso 
 &   ブルキナファソ 
 &   ブルキナファソ人 
 \\ \cline{1-6} 
 
  Zimbabwe 
 &   ジンバブエ 
 &   ジンバブエ人 
 &   Malawi 
 &   マラウイ 
 &   マラウイ人 
 \\ \cline{1-6} 
 
  Senegal 
 &   セネガル 
 &   セネガル人 
 &   Somalia 
 &   ソマリア 
 &   ソマリア人 
 \\ \cline{1-6} 
 
  Chad 
 &   チャド 
 &   チャド人 
 &   Tunisia 
 &   チュニジア 
 &   チュニジア人 
 \\ \cline{1-6} 
 
  Guinea 
 &   ギニア 
 &   ギニア人 
 &   Benin 
 &   ベニン 
 &   ベニン人 
 \\ \cline{1-6} 
 
  South Sudan 
 &    ${\overset{\textnormal{みなみ}}{\text{南}}}$ スーダン 
 &    ${\overset{\textnormal{みなみ}}{\text{南}}}$ スーダン人 
 &   Rwanda 
 &   ルワンダ 
 &   ルワンダ人 
 \\ \cline{1-6} 
 
  Burundi 
 &   ブルンジ 
 &   ブルンジ人 
 &   Togo 
 &   トーゴ 
 &   トーゴ人 
 \\ \cline{1-6} 
 
  Sierra Leone 
 &   シエラレオネ 
 &   シエラレオネ人 
 &   Libya 
 &   リビア 
 &   リビア人 
 \\ \cline{1-6} 
 
  Eritrea 
 &   エリトリア 
 &   エリトリア人 
 &   Central African Republic 
 &    ${\overset{\textnormal{ちゅうおう}}{\text{中央}}}$ アフリカ 
 &   中央アフリカ人 
 \\ \cline{1-6} 
 
  Liberia 
 &   リベリア 
 &   リベリア人 
 &   Republic of the Congo 
 &   コンゴ ${\overset{\textnormal{きょうわこく}}{\text{共和国}}}$ 
 &   コンゴ人 
 \\ \cline{1-6} 
 
  Mauritania 
 &   モーリタニア 
 &   モーリタニア人 
 &   Lesotho 
 &   レソト 
 &   レソト人 
 \\ \cline{1-6} 
 
  Namibia 
 &   ナミビア 
 &   ナミビア人 
 &   Botswana 
 &   ボツワナ 
 &   ボツワナ人 
 \\ \cline{1-6} 
 
  Guinea-Bissau 
 &   ギニアビサウ 
 &   ギニアビサウ人 
 &   Gambia 
 &   ガンビア 
 &   ガンビア人 
 \\ \cline{1-6} 
 
  Gabon 
 &   ガボン 
 &   ガボン人 
 &   Mauritius 
 &   モーリシャス 
 &   モーリシャス人 
 \\ \cline{1-6} 
 
  Swaziland 
 &   スワジランド 
 &   スワジランド人 
 &   Djibouti 
 &   ジブチ 
 &   ジブチ人 
 \\ \cline{1-6} 
 
  Comoros 
 &   コモロ 
 &   コモロ人 
 &   Equatorial Guinea 
 &    ${\overset{\textnormal{せきどう}}{\text{赤道}}}$ ギニア 
 &    ${\overset{\textnormal{せきどう}}{\text{赤道}}}$ ギニア人 
 \\ \cline{1-6} 
 
  Cape Verde 
 &   カーボベルデ 
 &   カーボベルデ人 
 &   São Tomé and Príncipe 
 &   サントメ・プリンシペ 
 &   サントメ・プリンシペ人 
 \\ \cline{1-6} 
 
  Seychelles 
 &   セーシェル 
 &   セーシェル人 
 &     &     &     \\ \cline{1-6} 
 
\end{ltabulary}

\par{\textbf{Word Note }: \hfill\break
1. The nationalities for both the Democratic Republic of the Congo and the Republic of the Congo are rendered as \emph{Kongojin }コンゴ人. To distinguish, you can refer to people of the former nation as \emph{Kongo Minshu Kyōwakoku no hito }コンゴ民主共和国の人 and people of the latter nation as \emph{Kongo Kyōwakoku no hito }コンゴ共和国の人. \hfill\break
2. Unlike other continents, very few country names have commonly known single-character spellings. The ones that do seldom get used are below. Although the continent and not a country, Africa is also listed as there are so few examples. }

\begin{ltabulary}{|P|P|P|P|P|P|}
\hline 
 
   ${\overset{\textnormal{ちめい}}{\text{地名}}}$ 
 &    ${\overset{\textnormal{かんじひょうき}}{\text{漢字表記}}}$ 
 &    ${\overset{\textnormal{りゃくしょう}}{\text{略称}}}$ 
 &    ${\overset{\textnormal{ちめい}}{\text{地名}}}$ 
 &    ${\overset{\textnormal{かんじひょうき}}{\text{漢字表記}}}$ 
 &    ${\overset{\textnormal{りゃくしょう}}{\text{略称}}}$ 
 \\ \cline{1-6} 
 
  Egypt 
 &   埃及 
 &    \emph{Ai }埃 
 &   Africa 
 &   阿弗利加 
 &    \emph{A }阿 
 \\ \cline{1-6} 
 
\end{ltabulary}

\par{\textbf{Chart Note }: \emph{Chimei }地名 = Place name. }

\par{\textbf{Usage Note }: Neither of these examples are commonly used. }

\begin{center}
\textbf{Countries of the Pacific }
\end{center}

\par{ Oceania in Japanese is \emph{Oseania }オセアニア. The Pacific Ocean is called \emph{Taiheiyō }太平洋. Within Oceania is Australia, which is called \emph{Ōsutoraria(-tairiku) }オーストラリア(大陸). Below are the nations that can be found in this region of the world: }
 
\begin{ltabulary}{|P|P|P|P|P|P|}
\hline 
 
   ${\overset{\textnormal{こくめい}}{\text{国名}}}$ 
 &    ${\overset{\textnormal{つうしょう}}{\text{通称}}}$ 
 &   国籍 
 &    ${\overset{\textnormal{こくめい}}{\text{国名}}}$ 
 &    ${\overset{\textnormal{つうしょう}}{\text{通称}}}$ 
 &    ${\overset{\textnormal{こくせき}}{\text{国籍}}}$ 
 \\ \cline{1-6} 
 
  Australia 
 &   オーストラリア 
 &   オーストラリア人 
 &   Papua New Guinea 
 &   パプアニューギニア 
 &   パプアニューギニア人 
 \\ \cline{1-6} 
 
  New Zealand 
 &   ニュージーランド 
 &   ニュージーランド人 
 &   Fiji 
 &   フィジー 
 &   フィジー人 
 \\ \cline{1-6} 
 
  Solomon Islands 
 &   ソロモン ${\overset{\textnormal{しょとう}}{\text{諸島}}}$ 
 &   ソロモン ${\overset{\textnormal{しょとうじん}}{\text{諸島人}}}$ 
 &   Vanuatu 
 &   バヌアツ 
 &   バヌアツ人 
 \\ \cline{1-6} 
 
  Samoa 
 &   サモア 
 &   サモア人 
 &   Kiribati 
 &   キリバス 
 &   キリバス人 
 \\ \cline{1-6} 
 
  Tonga 
 &   トンガ 
 &   トンガ人 
 &   Federated States of Micronesia 
 &   ミクロネシア ${\overset{\textnormal{れんぽう}}{\text{連邦}}}$ 
 &   ミクロネシア人 
 \\ \cline{1-6} 
 
  Marshall Islands 
 &   マーシャル ${\overset{\textnormal{しょとう}}{\text{諸島}}}$ 
 &   マーシャル ${\overset{\textnormal{しょとうじん}}{\text{諸島人}}}$ 
 &   Palau 
 &   パラオ 
 &   パラオ人 
 \\ \cline{1-6} 
 
  Tuvalu 
 &   ツバル 
 &   ツバル人 
 &   Nauru 
 &   ナウル 
 &   ナウル人 
\\ \cline{1-6}

\end{ltabulary}

\par{\textbf{Word Notes }: \hfill\break
1. The two examples of single-character spellings for the Pacific are as follows: }

\begin{ltabulary}{|P|P|P|P|P|P|}
\hline 
 
   ${\overset{\textnormal{ちめい}}{\text{地名}}}$ 
 &    ${\overset{\textnormal{かんじひょうき}}{\text{漢字表記}}}$ 
 &    ${\overset{\textnormal{りゃくしょう}}{\text{略称}}}$ 
 &    ${\overset{\textnormal{ちめい}}{\text{地名}}}$ 
 &    ${\overset{\textnormal{かんじ}}{\text{漢字}}}$ 
 &    ${\overset{\textnormal{りゃくしょう}}{\text{略称}}}$ 
 \\ \cline{1-6} 
 
  Australia 
 &   濠太剌利 
 &    \emph{Gō }豪・ \emph{Gō }濠 
 &   New Zealand 
 &   新西蘭 
 &    \emph{Shin }新 
\\ \cline{1-6}

\end{ltabulary}

\par{\textbf{Usage Note }: Of these, \emph{Gō }豪 for Australia is the most commonly used. }
      
\section{Languages}
 
\par{ Below are some of the world\textquotesingle s most spoken and\slash or important languages. With few exceptions, they all end in the suffix - \emph{go }語. }

\begin{ltabulary}{|P|P|P|P|P|P|}
\hline 
 
  言語 
 &   通称 
 &   言語 
 &   通称 
 &   言語 
 &   通称 
 \\ \cline{1-6} 
 
  Mandarin Chinese 
 &    ${\overset{\textnormal{ちゅうごくご}}{\text{中国語}}}$ \hfill\break
 ${\overset{\textnormal{ふつうわ}}{\text{普通話}}}$ 
 &   English 
 &    ${\overset{\textnormal{えいご}}{\text{英語}}}$ 
 &   Hindi 
 &   ヒンディー語 
 \\ \cline{1-6} 
 
  Spanish 
 &   スペイン語 
 &   Arabic 
 &   アラビア語 
 &   Bengali 
 &   ベンガル語 
 \\ \cline{1-6} 
 
  Portuguese 
 &   ポルトガル語 
 &   Russian 
 &   ロシア語 
 &   Japanese 
 &    ${\overset{\textnormal{にほんご}}{\text{日本語}}}$ 
 \\ \cline{1-6} 
 
  German 
 &   ドイツ語 
 &   French 
 &   フランス語 \hfill\break
 ${\overset{\textnormal{ふつご}}{\text{仏語}}}$ 
 &   Punjabi 
 &   パンジャーブ語 
 \\ \cline{1-6} 
 
  Javanese 
 &   ジャワ語 
 &   Korean 
 &    ${\overset{\textnormal{かんこくご}}{\text{韓国語}}}$ \hfill\break
 ${\overset{\textnormal{ちょうせんご}}{\text{朝鮮語}}}$ \hfill\break
コリア語 \hfill\break
ハングル △ 
 &   Vietnamese 
 &   ベトナム語 
 \\ \cline{1-6} 
 
  Telugu 
 &   テルグ語 
 &   Marathi 
 &   マラーティー語 
 &   Tamil 
 &   タミル語 
 \\ \cline{1-6} 
 
  Farsi 
 &   ペルシア語 
 &   Urdu 
 &   ウルドゥー語 
 &   Italian 
 &   イタリア語 
 \\ \cline{1-6} 
 
  Turkish 
 &   トルコ語 
 &   Gujarati 
 &   グジャラート語 
 &   Polish 
 &   ポーランド語 
 \\ \cline{1-6} 
 
  Ukrainian 
 &   ウクライナ語 
 &   Malayalam 
 &   マラヤ―ラム語 
 &   Kannada 
 &   カンナダ語 
 \\ \cline{1-6} 
 
  Azerbaijani 
 &   アゼルバイジャン語 
 &   Odia 
 &   オリヤー語 
 &   Burmese 
 &   ビルマ語 
 \\ \cline{1-6} 
 
  Thai 
 &   タイ語 
 &   Sundanese 
 &   スンダ語 
 &   Kurdish 
 &   クルド語 
 \\ \cline{1-6} 
 
  Pashto 
 &   パシュトー語 
 &   Hausa 
 &   ハウサ語 
 &   Romanian 
 &   ルーマニア語 
 \\ \cline{1-6} 
 
  Indonesian 
 &   インドネシア語 
 &   Uzbek 
 &   ウズベク語 
 &   Sindhi 
 &   シンド語 
 \\ \cline{1-6} 
 
  Cebuano 
 &   セブアノ語 
 &   Yoruba 
 &   ヨルバ語 
 &   Somali 
 &   ソマリ語 
 \\ \cline{1-6} 
 
  Lao 
 &   ラーオ語 
 &   Oromo 
 &   オモロ語 
 &   Malay 
 &   マレー語 
 \\ \cline{1-6} 
 
  Igbo 
 &   イボ語 
 &   Dutch 
 &   オランダ語 
 &   Amharic 
 &   アムハラ語 
 \\ \cline{1-6} 
 
  Malagasy 
 &   マダガスカル語 
 &   Tagalog 
 &   タガログ語 
 &   Nepali 
 &   ネパール語 
 \\ \cline{1-6} 
 
  Hebrew 
 &   ヘブライ語 
 &   Norwegian 
 &   ノルウェー語 
 &   Icelandic 
 &   アイスランド語 
 \\ \cline{1-6} 
 
  Swedish 
 &   スウェーデン語 
 &   Czech 
 &   チェコ語 
 &   Zulu 
 &   ずールー語 
 \\ \cline{1-6} 
 
\end{ltabulary}

\par{\textbf{Chart Notes }: }

\par{1. For languages not listed here that happen to be the national language of a certain nation, simply add - \emph{go }語 to the name of that nation and it is most likely the Japanese name for that language. \hfill\break
2. Many of the languages in this list are spoken in India. India has the second largest population in the world and is very linguistically diverse. \hfill\break
3. In many aspects of life such as education, Japanese is referred to as \emph{Kokugo }国語, meaning “national language.” However, this word can be used in context of any nation to refer to its national language(s). \hfill\break
4. \emph{Futsugo }仏語 is an outdated word for “French” which is seldom used. }

\par{\textbf{Usage Notes }: \hfill\break
1. Chinese is perhaps the most controversial language on this list. Although it is still the most spoken language in the world, there are many Chinese languages. Mandarin Chinese is either referred to as \emph{Chūgokugo }中国語 or \emph{Futsūwa (Pūtonfa) }普通話. The latter name comes from the actual name for the language in Mandarin Chinese. The other Chinese languages that were neglected in the chart above that are worth noting include the following: }

\par{・ \emph{Kantongo }広東語 - Cantonese \hfill\break
・ \emph{Gogo }呉語 – Wu Chinese \hfill\break
・ \emph{Hakkago }客家語 -  Hakka }

\par{2. Korean is typically referred to as \emph{Kankokugo }韓国語. However, linguistically speaking, it is referred to as \emph{Chōsengo }朝鮮語. Some speakers prefer \emph{Koriago }コリア語 to be politically neutral. Some speakers erroneously call it \emph{Hanguru }ハングル, which is the name of the Korean alphabet. }

\begin{center}
\textbf{Languages \& Dialects of Japan }
\end{center}

\par{ Other languages that are important to know about are those spoken in Japan. These include the following: }

\begin{ltabulary}{|P|P|P|P|P|P|}
\hline 
 
  Ainu 
 &   アイヌ語 
 &   Amami 
 &    ${\overset{\textnormal{あまみご}}{\text{奄美語}}}$ 
 &   Kunigami 
 &    ${\overset{\textnormal{くにがみご}}{\text{国頭語}}}$ 
 \\ \cline{1-6} 
 
  Okinawan 
 &    ${\overset{\textnormal{おきなわご}}{\text{沖縄語}}}$ \hfill\break
ウチナー ${\overset{\textnormal{ぐち}}{\text{口}}}$ 
 &   Miyako 
 &    ${\overset{\textnormal{みやこご}}{\text{宮古語}}}$ 
 &   Yaeyama 
 &    ${\overset{\textnormal{やえやまご}}{\text{八重山語}}}$ 
 \\ \cline{1-6} 
 
  Yonaguni 
 &    ${\overset{\textnormal{よなぐにご}}{\text{与那国語}}}$ 
 &     &     &     &     \\ \cline{1-6} 
 
\end{ltabulary}
 
\par{\textbf{Word Note }: \emph{Uchināguchi }ウチナー口 is Okinawan for the name of the language itself. }

\par{ Of these languages, all of them are related to Japanese except Ainu. For those related to Japanese, the Japanese government has tried over the past century to claim that they are dialects ( \emph{hōgen }方言). However, they are all mutually unintelligible to Japanese and vice versa. Generally speaking, dialects end in either \emph{hōgen }方言 or - \emph{ben }弁. Dialects in Japan can be very divergent from one another. It is merely that their accents ( \emph{namari }訛り) are different. Vocabulary and grammar will also be different. For the most part, any area of Japan will have its own dialect which may or may not be like those around it. }

\par{【 \emph{Kagoshima-ben }鹿児島弁】 \hfill\break
ゆ来たね!\textrightarrow  よく来たね! \hfill\break
 \emph{Yu kita ne! }\emph{Yoku kita ne! \hfill\break
 }How good it is you came! }

\par{【 \emph{Ōsaka-ben }大阪弁】 \hfill\break
どないしたん? \textrightarrow  どうしたの? \hfill\break
 \emph{Donai shita n? }\emph{\textrightarrow    Dō shita no? \hfill\break
 }What\textquotesingle s wrong? \emph{}}

\par{【 \emph{Kyōto-ben }京都弁】 \hfill\break
お ${\overset{\textnormal{とふ}}{\text{豆腐}}}$ さん ${\overset{\textnormal{ようい}}{\text{用意}}}$ しといて。 \textrightarrow  ${\overset{\textnormal{とうふ}}{\text{豆腐}}}$ を ${\overset{\textnormal{ようい}}{\text{用意}}}$ しておいて。 \hfill\break
 \emph{O-tofu-san yōi shitoite. }\emph{Tōfu wo yōi shite oite. \hfill\break
 }Prepare the tofu. }

\par{【 \emph{Nagasaki-ben }長崎弁】 \hfill\break
 ${\overset{\textnormal{きのう}}{\text{昨日}}}$ ${\overset{\textnormal{なん}}{\text{何}}}$ ばしよったと? \textrightarrow  ${\overset{\textnormal{きのうなに}}{\text{昨日何}}}$ をしていたの? \hfill\break
 \emph{Kinō nam ba shiyotta to? }\emph{Kinō nani wo shite ita no? \hfill\break
 }What were you doing yesterday? }

\par{【 \emph{Kōbe-ben }神戸弁】 \hfill\break
これ、知っとお? \textrightarrow  これ、知ってる? \hfill\break
 \emph{Kore, shittō? }\emph{Kore, shitteru? \hfill\break
 }Did you know this? }

\par{\textbf{Curriculum Note }: Dialect coverage will continue to be limited to the most essential knowledge that is known by speakers of Japanese at large. This is because the lingua franca of Japan is overwhelmingly Standard Japanese. }
    