    
\chapter{Counters I}

\begin{center}
\begin{Large}
第28課: Counters I: 円, 冊, 課, 人, 名, 歩, 枚, ページ, 頭, 匹, 足, 台, 階, 歳, \& 杯 
\end{Large}
\end{center}
 
\par{ When we learned about Sino-Japanese numbers, we learned that most number phrases in Japanese are largely inherited from Chinese. However, there is more to counting than ${\overset{\textnormal{いち}}{\text{一}}}$ , ${\overset{\textnormal{に}}{\text{二}}}$ , ${\overset{\textnormal{さん}}{\text{三}}}$ , etc. When counting \emph{things }rather than just counting in general, you must use what are called counters, \emph{josūshi } ${\overset{\textnormal{じょすうし}}{\text{助数詞}}}$ , along with the numbers. The name \emph{josūshi }助数詞 can literally be interpreted as “helper numbers,” and that is exactly what they are: they help numbers count things. }
 
\par{ There are a lot of counters that exist. This is because there are endless things in the world that we count. Incidentally, the infinite things one would want to count are neatly categorized and allotted to a finite number of counters. Counters, in a way, help indicate how nouns are conceptualized semantically speaking. Before we start conquering some of the most commonly used counters, first consider the following English phrases. }
 
\par{i.       When you go to the supermarket, could you buy a \textbf{gallon }of milk? \hfill\break
ii.       How many \textbf{loaves }of bread are left in the cupboard? \hfill\break
iii.       Why did you only give me three \textbf{slices }of ham? \hfill\break
iv.       How many \textbf{times }did you go? \hfill\break
v.       I have four \textbf{volumes }of the same book. }
 
\par{ Counters in Japanese work just like the words in bold. The purpose of counters considering their English equivalents may seem easy enough, but there is a much greater challenge that has yet to be addressed. The challenge is how to read counter phrases. }
 
\par{ When we learned about \emph{on\textquotesingle yomi }音読み and \emph{kun\textquotesingle yomi }訓読み, we learned that the distinction wasn\textquotesingle t arbitrary or trivial. This dichotomy comes from words from Chinese and native words being spelled with the same writing system. Consequently, the Japanese counting system was also heavily influenced by Chinese. Most counters come from Chinese, but a combination of Sino-Japanese and native counters is used with both Sino-Japanese and native numbers. }
 
\par{ For now, we\textquotesingle ll be avoiding counters that use native numbers not taught yet. In this lesson, we\textquotesingle ll only cover fifteen of the most frequently used counters to lessen the load of what needs to be learned for now. You will learn what they count, how they interact with numbers, and become familiar enough with them so that when we study counters again, learning an additional set won\textquotesingle t be so bad. }

\begin{center}
\textbf{Counters Covered in This Lesson }
\end{center}
1.       - \emph{en }円 2.       - \emph{satsu }冊 3.       - \emph{ka }課 4.       - \emph{nin\slash -ri }人 5.       - \emph{mei }名 6.       - \emph{ho }歩 7.       - \emph{mai }枚 8.       - \emph{p }\emph{ēji }ページ 9.       - \emph{t }\emph{ō }頭 10.   - \emph{hiki }匹 11.   - \emph{soku }足 12.   - \emph{dai }台 13.   - \emph{kai }階 14.   - \emph{sai }歳 15.   - \emph{hai }杯  
\par{\textbf{Reading Note }: Whenever a counter expression has more than one reading, readings will be listed in order of most to least commonly used. Additionally, the most commonly used variant will be in bold for easier identification. }
 
\par{\textbf{Chart Note: }In all counter lessons in IMABI, readings will be given in charts in \emph{hiragana }ひらがな only. This is so that readings can be concisely displayed with extraneous information. }
      
\section{Essential Counters}
 \textbf{- \emph{en }円 } Yen is the currency of Japan. - \emph{en } ${\overset{\textnormal{}}{\text{円}}}$ is both the name of the currency and the counter to count said currency. Despite being a Sino-Japanese counter used primarily with Sino-Japanese numbers, you\textquotesingle ll still need to use the native numbers for 4 and 7 for most counters. Therefore, we will look at each counter on an individual basis. \hfill\break
\hfill\break

\begin{ltabulary}{|P|P|P|P|P|P|P|P|}
\hline 

1 & いちえん & 2 & にえん & 3 & さんえん & 4 & よえん \\ \cline{1-8}

5 & ごえん & 6 & ろくえん & 7 &  \textbf{ななえん }\hfill\break
しちえん & 8 & はちえん \\ \cline{1-8}

9 & きゅうえん & 10 & じゅうえん & 14 & じゅうよえん & 100 & ひゃくえん \\ \cline{1-8}

1000 & (いっ)せんえん & 10000 & いちまんえん & ? & なんえん &  &  \\ \cline{1-8}

\end{ltabulary}

\par{\textbf{Reading Note }: With 4, the native number for 4 is よ instead of よん. This variation is rare but never optional whenever it is used. }

\par{1. ${\overset{\textnormal{わたし}}{\text{私}}}$ も ${\overset{\textnormal{ご}}{\text{5}}}$ ${\overset{\textnormal{えんだま}}{\text{円玉}}}$ を ${\overset{\textnormal{か}}{\text{搔}}}$ き ${\overset{\textnormal{あつ}}{\text{集}}}$ めました。 \hfill\break
 \emph{Watashi mo goen-dama wo kakiatsumemashita. \hfill\break
 }I also scraped up five-yen coins. }

\par{2. ${\overset{\textnormal{きょう}}{\text{今日}}}$ 、 ${\overset{\textnormal{いち}}{\text{1}}}$ ${\overset{\textnormal{まんえんさつ}}{\text{万円札}}}$ を ${\overset{\textnormal{ひろ}}{\text{拾}}}$ いました。 \hfill\break
 \emph{Ky }\emph{ō, ichiman\textquotesingle en-satsu wo hiroimashita. \hfill\break
 }Today, I picked up a 10,000-yen bill. }

\par{3. ${\overset{\textnormal{にひゃくごじゅう}}{\text{250}}}$ ${\overset{\textnormal{えん}}{\text{円}}}$ のお ${\overset{\textnormal{つ}}{\text{釣}}}$ りでございます。 \hfill\break
 \emph{Nihyakugoj }\emph{ūen no otsuri de gozaimasu. \hfill\break
 }Here is 250 yen in change. }

\par{\textbf{Grammar Note }: \emph{De gozaimasu }でございます is used as a respectful form of \emph{desu }です in this example sentence and would be expected in situations like this where the speaker ought to be respectful to the listener, which in this situation would be the customer. }

\begin{center}
- \emph{satsu }冊 
\end{center}

\par{ The counter - \emph{satsu }${\overset{\textnormal{}}{\text{冊}}}$ is used to count books, magazines, etc. 冊 itself is a pictograph of volumes of books next to each other, which should aid in remembering what it means. }

\begin{ltabulary}{|P|P|P|P|P|P|P|P|}
\hline 

1 & いっさつ & 2 & にさつ & 3 & さんさつ & 4 & よんさつ \\ \cline{1-8}

5 & ごさつ & \textbf{ }6 & \textbf{ }ろくさつ & \textbf{ }7 & \textbf{ }ななさつ & 8 & はっさつ \\ \cline{1-8}

\textbf{ }9 & きゅうさつ & \textbf{ }10 & \textbf{じゅっさつ \hfill\break
 }じっさつ & 11 & じゅういっさつ & 20 & \textbf{にじゅっさつ }\hfill\break
にじっさつ \\ \cline{1-8}

30 &  \textbf{さんじゅっさつ }\hfill\break
さんじっさつ & 50 &  \textbf{ごじゅっさつ }\hfill\break
ごじっさつ & 100 & ひゃくさつ & ? & なんさつ \\ \cline{1-8}

\end{ltabulary}

\par{\textbf{Reading Notes }: }

\par{1. The contraction じっ for 10 is the original contraction. じゅっ, however, is what\textquotesingle s largely used in the spoken language. The prescriptive, older form is often limited to formal situations such as news reports. \hfill\break
2. A minority of speakers pronounce 100冊 as “ \emph{hyassatsu }ひゃっさつ.” }

\par{4. ${\overset{\textnormal{わたし}}{\text{私}}}$ は ${\overset{\textnormal{ほん}}{\text{本}}}$ が ${\overset{\textnormal{ごじゅういっ}}{\text{51}}}$ ${\overset{\textnormal{さつ}}{\text{冊}}}$ あります。 \hfill\break
 \emph{Watashi wa hon ga goj }\emph{ūissatsu arimasu. \hfill\break
 }I have fifty one books. }

\par{5. ${\overset{\textnormal{ほん}}{\text{本}}}$ を ${\overset{\textnormal{6}}{\text{6}}}$ ${\overset{\textnormal{さつか}}{\text{冊借}}}$ りました。 \hfill\break
 \emph{Hon wo rokusatsu karimashita. \hfill\break
 }I borrowed six books. }

\par{6. もう ${\overset{\textnormal{いっさつよ}}{\text{一冊読}}}$ みました。 \hfill\break
 \emph{M }\emph{ō issatsu yomimashita. \hfill\break
 }I read one more book. }

\begin{center}
\textbf{- \emph{ka }課 }
\end{center}

\par{ When the counter - \emph{ka } ${\overset{\textnormal{}}{\text{課}}}$ is paired with the prefix \emph{dai }- ${\overset{\textnormal{}}{\text{第}}}$ , it helps create the expression “Lesson \#.” Before we get into the tricky part about this counter, let\textquotesingle s see how to pronounce it with the following numbers. }

\begin{ltabulary}{|P|P|P|P|P|P|P|P|}
\hline 

1 & いっか & 2 & にか & 3 & さんか & 4 & よんか \\ \cline{1-8}

5 & ごか & 6 & ろっか & 7 &  \textbf{ななか }\hfill\break
しちか & 8 &  \textbf{はちか }\hfill\break
はっか \\ \cline{1-8}

9 & きゅうか & 10 &  \textbf{じゅっか }\hfill\break
じっか & 11 & じゅういっか & 20 &  \textbf{にじゅっか }\hfill\break
にじっか \\ \cline{1-8}

30 &  \textbf{さんじゅっか }\hfill\break
さんじっか & 50 &  \textbf{ごじゅっか }\hfill\break
ごじっか & 100 & ひゃっか & ? & なんか \\ \cline{1-8}

\end{ltabulary}

\par{7 . ${\overset{\textnormal{だい}}{\text{第}}}$ ${\overset{\textnormal{はち}}{\text{8}}}$ ${\overset{\textnormal{か}}{\text{課}}}$ を ${\overset{\textnormal{がくしゅう}}{\text{学習}}}$ します。 \hfill\break
\emph{Dai-hachika wo gakush }\emph{ū shimasu. \hfill\break
}We will study Lesson 8. }

\par{Ex. 7 shows exactly how this counter can be used in a practical classroom setting to refer to which lesson is being studied. Unfortunately, the many phrases that English speakers would use such as “how many lessons are in this textbook?” are not easily expressed in Japanese. }

\par{ Japanese speakers conceptualize books as being broken up into sections just like English speakers do. A written work may be broken up into chapters ( \emph{sh }\emph{ō } ${\overset{\textnormal{しょう}}{\text{章}}}$ )\slash units ( \emph{tangen } ${\overset{\textnormal{たんげん}}{\text{単元}}}$ ), sections ( \emph{sekushon }セクション), or lessons ( \emph{ressun }レッスン・ \emph{ka }${\overset{\textnormal{か}}{\text{課}}}$ ). These various counters, though, are different words for the same thing. None indicate volume of content. In the Japanese mind, it makes more sense to indicate from what to which section is being indicated rather than the number of sections itself. This contrasts with how an English-speaking student might think, who may feel gratification in having reached Lesson 100 in a language learning course. }

\par{ With all this in mind,“how many lessons does the textbook have?” is most naturally expressed with something like Ex. 8. }

\par{8. この ${\overset{\textnormal{きょうかしょ}}{\text{教科書}}}$ は ${\overset{\textnormal{だいなんか}}{\text{第何課}}}$ までありますか。 \hfill\break
 \emph{Kono ky }\emph{ōkasho wa dai-nanka made arimasu ka? \hfill\break
 }How many lessons does the textbook have? \slash Up to what lesson is there in a textbook? }

\par{Ex. 8 literally translates as “Up to what lesson is there in this textbook?” This allows the Japanese language student to ask how many lessons there are while also following the Japanese mindset to figure out about up to what point one might study with the textbook. A context in which the student already knows how the curriculum is organized would make the most sense so that there is a concrete reference point as to where in the series the textbook gets the student. }

\par{\textbf{Grammar Note }: The particle \emph{made }まで means “until\slash up to” and will be discussed in greater depth later on in IMABI. }

\par{ This is not to say it isn\textquotesingle t possible to ask this question in an English-like manner. However, the counter - \emph{ka } ${\overset{\textnormal{か}}{\text{課}}}$ would not typically be used. Instead, \emph{sekushon }セクション\slash  \emph{tangen } ${\overset{\textnormal{たんげん}}{\text{単元}}}$ would be most suitable. }

\par{9. この ${\overset{\textnormal{きょうかしょ}}{\text{教科書}}}$ には ${\overset{\textnormal{なん}}{\text{何}}}$ セクションありますか。 \hfill\break
 \emph{Kono ky }\emph{ōkasho ni wa nansekushon arimasu ka? \hfill\break
 }How many sections are in this textbook? }

\par{10. その ${\overset{\textnormal{きょうかしょ}}{\text{教科書}}}$ にはいくつの ${\overset{\textnormal{たんげん}}{\text{単元}}}$ がありますか。 \hfill\break
 \emph{Sono ky }\emph{ōkasho ni wa ikutsu no tangen ga arimasu ka? \hfill\break
 }How many units are in that textbook? }

\par{\emph{ Tangen }単元 is a well-known word among educators and concerned college students, but it\textquotesingle s not quite the word you would encounter. As such, for it to be used as naturally as possible in this sentence, \emph{ikutsu no tangen }いくつの ${\overset{\textnormal{たんげん}}{\text{単元}}}$ is used instead of \emph{nantangen } ${\overset{\textnormal{なんたんげん}}{\text{何単元}}}$ . Both are possible phrases and are synonymous with each other. In Ex. 10, it is possible to replace \emph{tangen }単元 with \emph{ka }課 or \emph{ressun }レッスン, but the sentence would sound rather unnatural. \emph{Ressun }レッスン denotes a very English-based context. For instance, レッスン1 is possible and 1 is typically pronounced as “ \emph{wan }ワン” in this circumstance. }

\par{ Another reason for why there are restrictions on how \emph{ka } ${\overset{\textnormal{か}}{\text{課}}}$ can be used to mean “lesson” is because it\textquotesingle s normally used to mean “department.” In fact, 何課 is typically read as “ \emph{nanika }なにか” meaning “what department?” }

\par{11. ${\overset{\textnormal{えいぎょうにか}}{\text{営業二課}}}$ は変更(が)ありません。 \hfill\break
 \emph{Eigy }\emph{ō Nika wa henk }\emph{ō (ga) arimasen. \hfill\break
 }There will be no changes to Sales Department No. 2. }

\begin{center}
\textbf{- \emph{nin }\slash - \emph{ri }人 }
\end{center}

\par{ Counting people in Japanese is not particularly easy. This is because the Sino-Japanese counter - \emph{nin }にん coexists with the native counter – \emph{ri }り, both of which are written as 人. For numbers 1, 2, 4, 7, and any number ending in 4 or 7, native numbers can be seen used. However, the native counter – \emph{ri }り is used only with 1 and 2. This means 4 and 7 behave just how we\textquotesingle ve seen them used thus far. }

\begin{ltabulary}{|P|P|P|P|P|P|P|P|}
\hline 

1 & ひとり & 2 & ふたり & 3 & さんにん & 4 & よにん \\ \cline{1-8}

5 & ごにん & 6 & ろくにん & 7 &  \textbf{しちにん }\hfill\break
ななにん & 8 & はちにん \\ \cline{1-8}

9 &  \textbf{きゅうにん }\hfill\break
くにん & 10 & じゅうにん & 11 & じゅういちにん & 14 & じゅうよにん \\ \cline{1-8}

19 &  \textbf{じゅうくにん }\hfill\break
じゅうきゅうにん & 20 & にじゅうにん & 100 & ひゃくにん & ? & なんにん \\ \cline{1-8}

\end{ltabulary}

\par{\textbf{Reading Notes }: As you can see, there are several peculiarities in this chart. \emph{Hitori }ひとり (one person) and \emph{futari }ふたり (two people) are both inherited from native vocabulary. It is perhaps because of their high frequency of use that has spared them from being replaced. The reading of 4 is also よ just as with - \emph{en } ${\overset{\textnormal{えん}}{\text{円}}}$ . For 7, \emph{shichinin }しちにん is the predominant reading. As for the reading “ \emph{ku }く” for 9, it becomes the predominant reading for 19, 29, etc. The reason why \emph{kunin }くにん is avoided for “9 people” is because the connection of \emph{ku } ${\overset{\textnormal{く}}{\text{九}}}$ being homophonous to \emph{ku }${\overset{\textnormal{く}}{\text{苦}}}$ (suffering) becomes more apparent. However, both \emph{ky }\emph{ūnin }きゅうにん and \emph{kunin }くにん are correct and used. }

\par{12. ${\overset{\textnormal{なんにん}}{\text{何人}}}$ いますか。 \hfill\break
 \emph{Nan\textquotesingle nin imasu ka? \hfill\break
 }How many people are there? }

\par{13. ${\overset{\textnormal{こども}}{\text{子供}}}$ が ${\overset{\textnormal{じゅう}}{\text{10}}}$ ${\overset{\textnormal{にん}}{\text{人}}}$ います。 \hfill\break
 \emph{Kodomo ga j }\emph{ūnin imasu. \hfill\break
 }There are ten children. }

\par{14. ${\overset{\textnormal{けいさつかん}}{\text{警察官}}}$ が ${\overset{\textnormal{はち}}{\text{8}}}$ ${\overset{\textnormal{にんき}}{\text{人来}}}$ た。 \hfill\break
 \emph{Keisatsukan ga hachinin kita. \hfill\break
 }Eight police officers came. }
\textbf{- \emph{mei }名 }
\par{ In formal situations, people are counted with - \emph{mei }${\overset{\textnormal{めい}}{\text{名}}}$ instead of - \emph{nin } ${\overset{\textnormal{にん}}{\text{人}}}$ . - \emph{mei } ${\overset{\textnormal{めい}}{\text{名}}}$ replaces - \emph{nin }${\overset{\textnormal{にん}}{\text{人}}}$ especially when counting members, participants, staff, etc. However, regardless of the situation, - \emph{nin }${\overset{\textnormal{にん}}{\text{人}}}$ is still used when referring to population, occupancy, family, etc. You\textquotesingle ll see that in the news or newspapers that - \emph{nin } ${\overset{\textnormal{にん}}{\text{人}}}$ is used across the board to give the most objective tone possible. }

\begin{ltabulary}{|P|P|P|P|P|P|P|P|}
\hline 

1 & いちめい & 2 & にめい & 3 & さんめい & 4 & よんめい \\ \cline{1-8}

5 & ごめい & 6 & ろくめい & 7 &  \textbf{ななめい }\hfill\break
しちめい & 8 & はちめい \\ \cline{1-8}

9 & きゅうめい & 10 & じゅうめい & 100 & ひゃくめい & ? & なんめい \\ \cline{1-8}

\end{ltabulary}

\par{\hfill\break
15. 「 ${\overset{\textnormal{なんめいさま}}{\text{何名様}}}$ ですか?」「\{ ${\overset{\textnormal{よん}}{\text{4}}}$ ${\overset{\textnormal{めい}}{\text{名}}}$ ・ ${\overset{\textnormal{よ}}{\text{4}}}$ ${\overset{\textnormal{にん}}{\text{人}}}$ \}です。」 \hfill\break
 \emph{“Nammei-sama desu ka?” “[Yommei\slash yonin] desu.” }\hfill\break
“How many people?” “Four.” }

\par{16. スタッフが ${\overset{\textnormal{さん}}{\text{3}}}$ ${\overset{\textnormal{めい}}{\text{名}}}$ います。 \hfill\break
 \emph{Sutaffu ga sammei imasu. \hfill\break
 }There are three staff members. }

\par{17. ${\overset{\textnormal{じゅうぎょういん}}{\text{従業員}}}$ が ${\overset{\textnormal{ひゃく}}{\text{100}}}$ ${\overset{\textnormal{めい}}{\text{名}}}$ を ${\overset{\textnormal{こ}}{\text{超}}}$ えています。 \hfill\break
 \emph{J }\emph{ūgy }\emph{ōin ga hyakumei wo koete imasu. \hfill\break
 }There are over 100 employees. }

\par{\textbf{Phrase Note }: The phrase \emph{wo koete iru }を ${\overset{\textnormal{こ}}{\text{超}}}$ えている means “to exceed\slash to be over.” }

\begin{center}
\textbf{- \emph{ho }歩 }
\end{center}

\par{ The counter - \emph{ho }歩 counts steps. }

\begin{ltabulary}{|P|P|P|P|P|P|P|P|}
\hline 

1 & いっぽ & 2 & にほ & 3 & さんぽ & 4 & よんほ \\ \cline{1-8}

5 & ごほ & 6 & ろっぽ & 7 & ななほ & 8 & はっぽ \\ \cline{1-8}

9 & きゅうほ & 10 &  \textbf{じゅっぽ }\hfill\break
じっぽ & 100 & ひゃっぽ & ? &  \textbf{なんぽ }\hfill\break
なんほ \\ \cline{1-8}

\end{ltabulary}

\par{\hfill\break
18. ${\overset{\textnormal{まいにち}}{\text{毎日}}}$ ${\overset{\textnormal{ご}}{\text{5}}}$ ${\overset{\textnormal{せんぽ}}{\text{千歩}}}$ くらい ${\overset{\textnormal{ある}}{\text{歩}}}$ きます。 \hfill\break
 \emph{Mainichi gosempo kurai arukimasu. \hfill\break
 }I walk about five thousand steps every day. }

\par{\textbf{Grammar Note }: The particle \emph{kurai }くらい means “about” and will be discussed in greater detail later in IMABI. }

\par{19. \{ ${\overset{\textnormal{いっぽいっぽ}}{\text{一歩一歩}}}$ ・ ${\overset{\textnormal{いっぽ}}{\text{一歩}}}$ ずつ\} ${\overset{\textnormal{すす}}{\text{進}}}$ む。 \hfill\break
 \emph{[Ippo ippo\slash ippo zutsu] susumu. }\hfill\break
To proceed one step at a time. }

\begin{center}
\textbf{- \emph{mai }枚 }
\end{center}

\par{ The counter - \emph{mai } ${\overset{\textnormal{}}{\text{枚}}}$ counts thin and flat objects. As such, it is frequently used to count “paper” and the likes. Interestingly enough, it can also count fields (in a general sense). This is because, just like paper, fields are typically flat. Ironically, this counter is not used to count pages. The counter for that will be discussed next. }

\begin{ltabulary}{|P|P|P|P|P|P|P|P|}
\hline 

1 & いちまい & 2 & にまい & 3 & さんまい & 4 & よんまい \\ \cline{1-8}

5 & ごまい & 6 & ろくまい & 7 &  \textbf{ななまい }\hfill\break
しちまい & 8 & はちまい \\ \cline{1-8}

9 & きゅうまい & 10 & じゅうまい & 100 & ひゃくまい & ? & なんまい \\ \cline{1-8}

\end{ltabulary}

\par{\hfill\break
20. ${\overset{\textnormal{かみいちまい}}{\text{紙一枚}}}$ \hfill\break
 \emph{Kami ichimai }\hfill\break
One piece of paper }

\par{21. シャツが ${\overset{\textnormal{さん}}{\text{3}}}$ ${\overset{\textnormal{まい}}{\text{枚}}}$ あります。 \hfill\break
 \emph{Shatsu ga sammai arimasu. \hfill\break
 }There are three shirts. }

\par{22. ${\overset{\textnormal{こうえん}}{\text{公園}}}$ には ${\overset{\textnormal{ひょうしき}}{\text{標識}}}$ が ${\overset{\textnormal{ご}}{\text{5}}}$ ${\overset{\textnormal{まい}}{\text{枚}}}$ あります。 \hfill\break
 \emph{K }\emph{ōen ni wa hyōshiki ga gomai arimasu. \hfill\break
 }There are five signs in the park. }

\begin{center}
\textbf{- \emph{p }\emph{ēji }ページ }
\end{center}

\par{\emph{ P }\emph{ēji }ページ means “page.” It has been incorporated into Japanese as both a noun and a counter to count pages. }

\begin{ltabulary}{|P|P|P|P|P|P|P|P|}
\hline 

1 &  \textbf{いっぺーじ }\hfill\break
いちぺーじ & 2 & にぺーじ & 3 & さんぺーじ & 4 & よんぺーじ \\ \cline{1-8}

5 & ごぺーじ & 6 &  \textbf{ろくぺーじ }\hfill\break
ろっぺーじ & 7 & ななぺーじ & 8 &  \textbf{はっぺーじ }\hfill\break
はちぺーじ \\ \cline{1-8}

9 & きゅうぺーじ & 10 &  \textbf{じゅっぺーじ }\hfill\break
じっぺーじ & 100 & ひゃくぺーじ & ? & なんぺーじ \\ \cline{1-8}

\end{ltabulary}

\par{\textbf{Grammar Note }: Some speakers view the contracted forms as meaning “x quantity of pages” and use the non-abbreviated forms to mean “page \#.” However, this distinction is not set in stone and the choice is made by personal preference. }

\par{23. この ${\overset{\textnormal{ほん}}{\text{本}}}$ は ${\overset{\textnormal{にひゃくごじゅう}}{\text{250}}}$ ページあります。 \hfill\break
 \emph{Kono hon wa nihyakugojupp }\emph{ēji arimasu. \hfill\break
 }This book has two hundred fifty pages. }

\par{24. ${\overset{\textnormal{なん}}{\text{何}}}$ ページ ${\overset{\textnormal{よ}}{\text{読}}}$ みましたか。 \hfill\break
 \emph{Namp }\emph{ēji yomimashita ka? \hfill\break
 }How many pages did you read? }

\par{25. ${\overset{\textnormal{じゅうご}}{\text{15}}}$ ページを ${\overset{\textnormal{ひら}}{\text{開}}}$ いてください。 \hfill\break
 \emph{J }\emph{ūgopēji wo hiraite kudasai. \hfill\break
 }Please open Page 15. }

\par{\textbf{Grammar Note }: These examples show both instances of this counter being used as a \emph{noun }or \emph{adverb }. In Japanese, counter phrases are most often used as adverbs , which explains why the word order seems so different from English. However, counters can be used as nouns. For this lesson, only a handful of instances of counters used as nouns will be presented. This is to allow you to study how counters are most frequently used and leave the more complex issue of part of speech to a later lesson. }

\par{\textbf{Spelling Note }: This counter can also be spelled as 頁. }

\begin{center}
\textbf{- \emph{tō }頭 }
\end{center}

\par{ The counter - \emph{tō } ${\overset{\textnormal{とう}}{\text{頭}}}$ counts naturally large animals. It is also the counter for animals in general that are present at a zoo. In the world of zoology, even insects are counted with this (including butterflies). As you will discover, depending on the animal and its physical characteristics, one or more counters may be applicable. }

\begin{ltabulary}{|P|P|P|P|P|P|P|P|}
\hline 

1 & いっとう & 2 & にとう & 3 & さんとう & 4 & よんとう \\ \cline{1-8}

5 & ごとう & 6 &  \textbf{ろくとう }\hfill\break
ろっとう & 7 & ななとう & 8 &  \textbf{はっとう }\hfill\break
はちとう \\ \cline{1-8}

9 & きゅうとう & 10 &  \textbf{じゅっとう }\hfill\break
じっとう & 100 &  \textbf{ひゃくとう }\hfill\break
ひゃっとう & ? & なんとう \\ \cline{1-8}

\end{ltabulary}

\par{26. イルカが ${\overset{\textnormal{さん}}{\text{3}}}$ ${\overset{\textnormal{とう}}{\text{頭}}}$ います。 \hfill\break
\emph{Iruka ga sant }\emph{ō imasu. \hfill\break
}There are three dolphins. }

\par{27. ヒョウが ${\overset{\textnormal{よん}}{\text{4}}}$ ${\overset{\textnormal{とう}}{\text{頭}}}$ います。 \hfill\break
 \emph{Hy }\emph{ō ga yont }\emph{ō imasu. \hfill\break
 }There are four leopards. }

\par{28. パンダが ${\overset{\textnormal{はっ}}{\text{8}}}$ ${\overset{\textnormal{とう}}{\text{頭}}}$ います。 \hfill\break
 \emph{Panda ga hatt }\emph{ō imasu. \hfill\break
 }There are eight pandas. }

\begin{center}
\textbf{- \emph{hiki }匹 }
\end{center}

\par{ The counter - \emph{hiki } ${\overset{\textnormal{ひき}}{\text{匹}}}$ counts small animals. Birds are typically counted with other counters. Because that counter is especially tricky, we\textquotesingle ll learn about it later. As for this counter, it is the most common counter for animals in general in layman\textquotesingle s speech. }

\begin{ltabulary}{|P|P|P|P|P|P|P|P|}
\hline 

1 & いっぴき & 2 & にひき & 3 & さんびき & 4 & よんひき \\ \cline{1-8}

5 & ごひき & 6 & ろっぴき & 7 & ななひき & 8 & はっぴき \\ \cline{1-8}

9 & きゅうひき & 10 &  \textbf{じゅっぴき }\hfill\break
じっぴき & 100 & ひゃっぴき & ? & なんびき \\ \cline{1-8}

\end{ltabulary}

\par{\hfill\break
29. この ${\overset{\textnormal{いえ}}{\text{家}}}$ には ${\overset{\textnormal{いぬ}}{\text{犬}}}$ が10匹、 ${\overset{\textnormal{ねこ}}{\text{猫}}}$ が5匹います。 \hfill\break
 \emph{Kono ie ni wa inu ga juppiki, neko ga gohiki imasu. \hfill\break
 }There are ten dogs and five cats in this house. }

\par{30. ${\overset{\textnormal{わたし}}{\text{私}}}$ は ${\overset{\textnormal{こねこ}}{\text{子猫}}}$ (が) ${\overset{\textnormal{さん}}{\text{3}}}$ ${\overset{\textnormal{びき}}{\text{匹}}}$ います。 \hfill\break
 \emph{Watashi wa koneko (ga) sambiki imasu. \hfill\break
 }I have three kittens. }

\par{31. ${\overset{\textnormal{わたし}}{\text{私}}}$ は ${\overset{\textnormal{こいぬ}}{\text{子犬}}}$ (が) ${\overset{\textnormal{いっ}}{\text{1}}}$ ${\overset{\textnormal{びき}}{\text{匹}}}$ います。 \hfill\break
 \emph{Watashi wa koinu (ga) ippiki imasu. \hfill\break
 }I have one puppy. }

\begin{center}
\textbf{- \emph{soku }足 }
\end{center}

\par{ The counter - \emph{soku }${\overset{\textnormal{そく}}{\text{足}}}$ counts pairs of footwear such as shoes and socks. }

\begin{ltabulary}{|P|P|P|P|P|P|P|P|}
\hline 

1 & いっそく & 2 & にそく & 3 &  \textbf{さんぞく }\hfill\break
さんそく & 4 & よんそく \\ \cline{1-8}

5 & ごそく & 6 & ろくそく & 7 & ななそく & 8 & はっそく \\ \cline{1-8}

9 & きゅうそく & 10 &  \textbf{じゅっそく }\hfill\break
じっそく & 100 & ひゃくそく & ? &  \textbf{なんぞく }\hfill\break
なんそく \\ \cline{1-8}

\end{ltabulary}

\par{\textbf{Reading Notes }: The reading - \emph{zoku }is also frequently seen with 1000 ( \emph{senzoku }\slash  \emph{sensoku }千足) and 10000 ( \emph{ichimanzoku }\slash  \emph{ichimansoku }一万足). Additionally, any number ending in 3 may utilize this reading. }

\par{32. ${\overset{\textnormal{くつ}}{\text{靴}}}$ を ${\overset{\textnormal{いっ}}{\text{1}}}$ ${\overset{\textnormal{そく}}{\text{足}}}$ ${\overset{\textnormal{か}}{\text{買}}}$ いました。 \hfill\break
 \emph{Kutsu wo issoku kaimashita. \hfill\break
 }I bought one pair of shoes. }

\par{33. サンダルが ${\overset{\textnormal{に}}{\text{2}}}$ ${\overset{\textnormal{そく}}{\text{足}}}$ あります。 \hfill\break
 \emph{Sandaru ga nisoku arimasu. \hfill\break
 }There are\slash I have two pairs of sandals. }

\par{34. ${\overset{\textnormal{げた}}{\text{下駄}}}$ は ${\overset{\textnormal{いま}}{\text{今}}}$ も ${\overset{\textnormal{ご}}{\text{5}}}$ ${\overset{\textnormal{そく}}{\text{足}}}$ あります。 \hfill\break
 \emph{Geta wa ima mo gosoku arimasu. \hfill\break
 }I still have five pairs of geta. \hfill\break
 \hfill\break
\textbf{Word Note }: \emph{Geta }are wooden clogs. This is the traditional Japanese sandal. }

\par{\textbf{Word Note }: To refer to one part of a pair, there are two options: \emph{katah }\emph{ō }${\overset{\textnormal{かたほう}}{\text{片方}}}$ or \emph{hansoku } ${\overset{\textnormal{はんそく}}{\text{半足}}}$ , with the former option being the most common and usable for any kind of pair. To say something like “two and a half pairs, you would use \emph{nisoku-han(bun) } ${\overset{\textnormal{に}}{\text{2}}}$ ${\overset{\textnormal{そくはん}}{\text{足半}}}$ ( ${\overset{\textnormal{ぶん}}{\text{分}}}$ ). }

\begin{center}
\textbf{- \emph{dai }台 }
\end{center}

\par{ The counter - \emph{dai }${\overset{\textnormal{だい}}{\text{台}}}$ is used to count mechanical objects both large and small and non-electric and electric ones alike. This means it can counts vehicles of any kind, bicycles, pianos, devices of any kind, etc. }

\begin{ltabulary}{|P|P|P|P|P|P|P|P|}
\hline 

1 & いちだい & 2 & にだい & 3 & さんだい & 4 & よんだい \\ \cline{1-8}

5 & ごだい & 6 & ろくだい & 7 &  \textbf{ななだい }\hfill\break
しちだい & 8 & はちだい \\ \cline{1-8}

9 & きゅうだい & 10 & じゅうだい & 100 & ひゃくだい & ? & なんだい \\ \cline{1-8}

\end{ltabulary}

\par{\hfill\break
35. ${\overset{\textnormal{にほんこくない}}{\text{日本国内}}}$ には ${\overset{\textnormal{くるま}}{\text{車}}}$ は ${\overset{\textnormal{なんだい}}{\text{何台}}}$ ありますか。 \hfill\break
 \emph{Nihon kokunai ni wa kuruma wa nandai arimasu ka? \hfill\break
 }How many cars in Japan? }

\par{36. ${\overset{\textnormal{わたし}}{\text{私}}}$ はこの ${\overset{\textnormal{に}}{\text{2}}}$ ${\overset{\textnormal{だい}}{\text{台}}}$ の ${\overset{\textnormal{ピーシー}}{\text{PC}}}$ を ${\overset{\textnormal{つな}}{\text{繋}}}$ ぎます。 \hfill\break
 \emph{Watashi wa kono nidai no piishii wo tsunagimasu. \hfill\break
 }I\textquotesingle m going to connect these two PCs. }

\par{37. ${\overset{\textnormal{たんまつ}}{\text{端末}}}$ を ${\overset{\textnormal{じゅう}}{\text{10}}}$ ${\overset{\textnormal{だいはっそう}}{\text{台発送}}}$ しました。 \hfill\break
 \emph{Tammatsu wo j }\emph{ūdai hass }\emph{ō shimashita. \hfill\break
 }I shipped ten devices. }

\begin{center}
\textbf{- \emph{kai }階 }
\end{center}

\par{ The counter - \emph{kai } ${\overset{\textnormal{かい}}{\text{階}}}$ counts stairs. In English, some difference exists across dialects as to what “first floor, second floor, etc.” refer to. However, in American English, the ground floor is referred to as the “first floor” as is the case in Japanese. Basement floors follow the same naming scheme as in American English as well. B1 becomes “ \emph{chika ikkai } ${\overset{\textnormal{ちか}}{\text{地下}}}$ ${\overset{\textnormal{いっ}}{\text{1}}}$ ${\overset{\textnormal{かい}}{\text{階}}}$ = underground floor 1.” }

\begin{ltabulary}{|P|P|P|P|P|P|P|P|}
\hline 

1 & いっかい & 2 & にかい & 3 &  \textbf{さんがい }\hfill\break
さんかい & 4 & よんかい \\ \cline{1-8}

5 & ごかい & 6 & ろっかい & 7 & ななかい & 8 &  \textbf{はちかい }\hfill\break
はっかい \\ \cline{1-8}

9 & きゅうかい & 10 &  \textbf{じゅっかい }\hfill\break
じっかい & 100 & ひゃっかい & ? &  \textbf{なんがい }\hfill\break
なんかい \\ \cline{1-8}

\end{ltabulary}

\par{\textbf{Reading Note }: For any number that ends in 3, 階 may be read as “ \emph{gai }がい.” For numbers with 8, the reading “ \emph{hachi }はち” for 8 is preferred by announcers of all sorts to avoid any form of confusion. In fact, in announcements on planes, trains, buses, elevators, etc., counter phrases are frequently used without any sound changes. \hfill\break
 \hfill\break
38. ${\overset{\textnormal{はなや}}{\text{花屋}}}$ は ${\overset{\textnormal{さん}}{\text{3}}}$ ${\overset{\textnormal{かい}}{\text{階}}}$ にあります。 \hfill\break
 \emph{Hanaya wa sangai ni arimasu. \hfill\break
 }The florist is on the third floor. }

\par{39. ${\overset{\textnormal{わたし}}{\text{私}}}$ はマンションの ${\overset{\textnormal{よん}}{\text{4}}}$ ${\overset{\textnormal{かい}}{\text{階}}}$ に ${\overset{\textnormal{す}}{\text{住}}}$ んでいます。 \hfill\break
 \emph{Watashi wa manshon no yonkai ni sunde imasu. \hfill\break
 }I live on Floor 4 of an apartment complex. }

\par{40. この ${\overset{\textnormal{たてもの}}{\text{建物}}}$ は ${\overset{\textnormal{なんがいだ}}{\text{何階建}}}$ てですか。 \hfill\break
 \emph{Kono tatemono wa nangaidate desu ka? \hfill\break
 }How many floors does this building have? }

\par{\textbf{Word Note }: To refer to how many floors a building has, - \emph{kaidate } ${\overset{\textnormal{かいだ}}{\text{階建}}}$ て must be used. }

\begin{center}
\textbf{- \emph{sai }歳・才 }
\end{center}

\par{ The counter - \emph{sai }歳 ・才counts how many “years old” one is. The first character is used the most, but in abbreviated writing, the latter character is used. }

\begin{ltabulary}{|P|P|P|P|P|P|P|P|}
\hline 

1 & いっさい & 2 & にさい & 3 & さんさい & 4 & よんさい \\ \cline{1-8}

5 & ごさい & 6 & ろくさい & 7 & ななさい & 8 & はっさい \\ \cline{1-8}

9 & きゅうさい & 10 &  \textbf{じゅっさい }\hfill\break
じっさい & 15 & じゅうごさい & 18 & じゅうはっさい \\ \cline{1-8}

20 &  \textbf{はたち \hfill\break
にじゅっさい }\hfill\break
にじっさい & 21 & にじゅういっさい & 100 & ひゃくさい & ? & なんさい \\ \cline{1-8}

\end{ltabulary}

\par{\textbf{Reading Notes }: To ask “how old are you?” you may also be asked “ \emph{o-ikutsu desu ka? }おいくつですか” using native phrasing instead of “ \emph{nansai desu ka? } ${\overset{\textnormal{なんさい}}{\text{何歳}}}$ ですか.” The former is politer. }

\par{41. ${\overset{\textnormal{じゅうはっ}}{\text{18}}}$ ${\overset{\textnormal{さい}}{\text{歳}}}$ の ${\overset{\textnormal{みせいねん}}{\text{未成年}}}$ が ${\overset{\textnormal{さん}}{\text{3}}}$ ${\overset{\textnormal{にん}}{\text{人}}}$ います。 \hfill\break
 \emph{J }\emph{ūhassai no miseinen ga san\textquotesingle nin imasu. \hfill\break
 }There are three minors aged 18. }

\par{\textbf{Culture Note }: In Japan, individuals under the age of 20 are considered minors. \hfill\break
 \hfill\break
42. ${\overset{\textnormal{はちじゅっ}}{\text{80}}}$ ${\overset{\textnormal{さい}}{\text{歳}}}$ のおばさんも ${\overset{\textnormal{さんか}}{\text{参加}}}$ しました。 \hfill\break
 \emph{Hachijussai no obasan mo sanka shimashita. \hfill\break
 }An eighty-year old lady also participated. }

\begin{center}
\textbf{\emph{-hai }杯 }
\end{center}

\par{ The counter - \emph{hai }${\overset{\textnormal{はい}}{\text{杯}}}$ counts a cup\slash bowl\slash glass full of something or squid\slash octopus\slash crab (when taken out of the water and then sold). }

\begin{ltabulary}{|P|P|P|P|P|P|P|P|}
\hline 

 1 & いっぱい & 2 & にはい & 3 &  \textbf{さんばい }\hfill\break
さんぱい & 4 & よんはい \\ \cline{1-8}

5 & ごはい & 6 & ろっぱい & 7 &  \textbf{ななはい }\hfill\break
しちはい & 8 & はっぱい \\ \cline{1-8}

9 & きゅうはい & 10 &  \textbf{じゅっぱい }\hfill\break
じっぱい & 100 & ひゃっぱい & ? &  \textbf{なんばい }\hfill\break
なんぱい \\ \cline{1-8}

\end{ltabulary}

\par{\textbf{Reading Notes }: Traditionally, counters that start with “h” have a sound change with 3 or \emph{nan }- ${\overset{\textnormal{なん}}{\text{何}}}$ of “h” to “b,” not “p.” This, though, has changed for most counters and is currently changing for - \emph{hai }${\overset{\textnormal{はい}}{\text{杯}}}$ . }

\par{43. コーヒーを ${\overset{\textnormal{いっぱいの}}{\text{一杯飲}}}$ みました。 \hfill\break
 \emph{K }\emph{ōhii wo ippai nomimashita. \hfill\break
 }I drank a cup of coffee. }

\par{44. イカ ${\overset{\textnormal{いっ}}{\text{1}}}$ ${\overset{\textnormal{ぱい}}{\text{杯}}}$ の ${\overset{\textnormal{おも}}{\text{重}}}$ さを ${\overset{\textnormal{はか}}{\text{量}}}$ りました。 \hfill\break
 \emph{Ika ippai no omosa wo hakarimashita. \hfill\break
 }I measured the weight of one squid. }

\par{\textbf{Spelling Note }: \emph{Ika }is seldom spelled as 烏賊. }

\par{45. ${\overset{\textnormal{じもと}}{\text{地元}}}$ のカニを ${\overset{\textnormal{に}}{\text{2}}}$ ${\overset{\textnormal{はいつか}}{\text{杯使}}}$ いました。 \hfill\break
 \emph{Jimoto no kani wo nihai tsukaimashita. \hfill\break
 }I used two local crabs. }

\par{\textbf{Spelling Note }: \emph{Kani }is occasionally spelled as 蟹. }

\par{46. ${\overset{\textnormal{おおがた}}{\text{大型}}}$ のタコを ${\overset{\textnormal{さん}}{\text{3}}}$ ${\overset{\textnormal{ばい}}{\text{杯}}}$ も ${\overset{\textnormal{つ}}{\text{釣}}}$ りました! \hfill\break
 \emph{Ōgata no tako wo sambai mo tsurimashita! \hfill\break
 }I caught three large octopuses! }

\par{\textbf{Spelling Note }: \emph{Tako }is occasionally spelled as 蛸. }

\par{47. ${\overset{\textnormal{いっぱい}}{\text{一杯}}}$ のご ${\overset{\textnormal{はん}}{\text{飯}}}$ \hfill\break
 \emph{Ippai no gohan \hfill\break
 }A bowl of food\slash rice }

\par{\textbf{Word Note }: \emph{Gohan }ご ${\overset{\textnormal{はん}}{\text{飯}}}$ is often used to mean “meal” even though it literally means “cooked rice.” }

\par{\emph{ Ippai }${\overset{\textnormal{いっぱい}}{\text{一杯}}}$ is used in a lot of expressions. It can show that a container is full of something. It can also be used with time phrases to indicate the full extent of time one has. Showing extent also translates into showing the extent something can be done. Think of this whenever you come across it in a set expression. }
 
\par{48. ${\overset{\textnormal{せいいっぱい}}{\text{精一杯}}}$ \hfill\break
 \emph{Sei ippai \hfill\break
 }As hard as possible  }
      
\section{Sound Change Rules for Sino-Japanese Counters}
 
\par{ After seeing fifteen counters, you may have noticed patterns to the sound changes you\textquotesingle ve seen. In the chart below, all these rules are detailed in a systematic fashion. Some rules involve counters not taught in this lesson, but you won\textquotesingle t need to memorize them until they are introduced. }
 
\begin{ltabulary}{|P|P|P|P|P|}
\hline 
 
  No. 
 &   Contraction 
 &   Counters 
 &   Effect 
 &   Examples 
 \\ \cline{1-5} 
 
  1 
 &   いち\textrightarrow いっ 
 &   \#k, s, sh, t, ch, h, f. 
 &   H\slash F   \textrightarrow P 
 &   いっかい・いっさい・いっぱい 
 \\ \cline{1-5} 
 
  3 
 &   n\slash a 
 &   n\slash a 
 &   H\slash F   \textrightarrow B\slash P* 
 &   さんぼん・さんばい・さんぷん 
 \\ \cline{1-5} 
 
  4 
 &   n\slash a 
 &   n\slash a 
 &   H \textrightarrow P 
 &   よんぷん・よんぱい 
 \\ \cline{1-5} 
 
  4 
 &   よ\textrightarrow よん 
 &   Almost all. 
 &   n\slash a 
 &   よんかい・よんさい 
 \\ \cline{1-5} 
 
  6 
 &   ろく   \textrightarrow  ろっ 
 &   \#k, h, and f. 
 &   H\slash F   \textrightarrow  P. 
 &   ろっかい・ ろっぽ 
 \\ \cline{1-5} 
 
  8 
 &   はち\textrightarrow はっ 
 &   \#k, s, sh, t, ch, h, f. 
 &   H\slash F   \textrightarrow P 
 &   はっさい・はっぽ 
 \\ \cline{1-5} 
 
  10 
 &   じゅう\textrightarrow    じゅっ 
 &   \#k, s, sh, t, ch, h, f. 
 &   H\slash F   \textrightarrow P 
 &   じゅっかい・じゅっさい 
 \\ \cline{1-5} 
 
  100 
 &   ひゃく   \textrightarrow  ひゃっ 
 &   \#k, h, and f. 
 &   H\slash F   \textrightarrow  P. 
 &   ひゃっかい・ ひゃっぴき 
 \\ \cline{1-5} 
 
  1000 
 &   n\slash a 
 &   n\slash a 
 &   H\slash F   \textrightarrow B\slash P* 
 &   せんぼん・せんぱい・ せんぷん 
 \\ \cline{1-5} 
 
  1万 
 &   n\slash a 
 &   n\slash a 
 &   H\slash F   \textrightarrow B\slash P* 
 &   いちまんぼん・ いちまんぷん 
 \\ \cline{1-5} 
 
  何 
 &   n\slash a 
 &   n\slash a 
 &   H\slash F   \textrightarrow B\slash P* 
 &   なんぼん・なんぱい・なんぷん 
 \\ \cline{1-5} 
 
\end{ltabulary}
 
\par{\textbf{Chart Notes }: \hfill\break
1. Almost all counters use よん instead of よ. However, some of the most important counters use よ, so be careful whenever those counters are discussed. \hfill\break
2. H\textrightarrow P has slowly been replacing H\textrightarrow B, but the choice between the two is currently all over the place. \hfill\break
3. The contraction for 10 is formally じっ, but most speakers now use じゅっ. }
As we study more counters, you\textquotesingle ll see that these rules can be extended to non-Sino-Japanese counters, but the pronunciation of counter phrases is undergoing major flux in Modern Japanese. Ultimately, you\textquotesingle ll have to study how people say counter phrases and follow accordingly.     