    
\chapter{The Particle に I}

\begin{center}
\begin{Large}
第31課: The Particle に I 
\end{Large}
\end{center}
 
\par{ に is the particle of \textbf{"establishment." }It encompasses place, time, direction, and destination . These things may not seem inherently related to each other, but as you learn more about these usages and how they are expressed with に, you'll see how they are quite interrelated in the realm of Japanese grammar. }
      
\section{The Case Particle に}
 
\par{ に shows deep establishment of an action or state. For the most part, it is equivalent to "at." The next sections describe the most important usages of に. }
 
\begin{itemize}
 
\item に may show where something exists or occurs and even mark the person of possession. With this comes a strong sense of relation between the verb and what に attaches to.  
\end{itemize}
 
\par{1. 彼には子どもがいる。 \hfill\break
As for him, he has children. }
 
\par{2. この ${\overset{\textnormal{にわ}}{\text{庭}}}$ に花はない。 \hfill\break
There aren't any flowers in this garden. }
3. ${\overset{\textnormal{しょうがっこう}}{\text{小学校}}}$ はどこ\{にあります・です\}か \hfill\break
Where is the elementary school? \hfill\break
 
\par{\textbf{Phrase Note }: どこにありますか and どこですか are "where is it at?" and "where is it" respectively. At times, they really aren't different. However, there are situations where one is more appropriate than the other. The same goes for the English equivalents. }
 
\par{4. ${\overset{\textnormal{とし}}{\text{都市}}}$ にたくさんの ${\overset{\textnormal{たてもの}}{\text{建物}}}$ を ${\overset{\textnormal{た}}{\text{建}}}$ てる。 \hfill\break
To build a lot of buildings in the city. }
 
\par{5. いすに座る。 \hfill\break
To sit in a chair. }
 
\par{6. ${\overset{\textnormal{げんかん}}{\text{玄関}}}$ に犬が ${\overset{\textnormal{いっぴき}}{\text{一匹}}}$ いる。 \hfill\break
There is a dog at the door. }
 
\par{7. 彼は ${\overset{\textnormal{だいどころ}}{\text{台所}}}$ の ${\overset{\textnormal{とぐち}}{\text{戸口}}}$ にいます。 \hfill\break
He is in the kitchen doorway. }
 
\par{8. ${\overset{\textnormal{こおり}}{\text{氷}}}$ は水に ${\overset{\textnormal{う}}{\text{浮}}}$ く。 \hfill\break
Ice floats on water. }
 
\par{9. ${\overset{\textnormal{たいよう}}{\text{太陽}}}$ は東に ${\overset{\textnormal{のぼ}}{\text{昇}}}$ ります。 \hfill\break
The sun (always) rises in the east. }
 
\par{10. 太陽は西に ${\overset{\textnormal{しず}}{\text{沈}}}$ む。 \hfill\break
The sun sets in the west. }
 
\par{11. 彼女は男の子に ${\overset{\textnormal{にんき}}{\text{人気}}}$ がある。 \hfill\break
She is popular with boys. }
 
\par{12. 私は ${\overset{\textnormal{こんなん}}{\text{困難}}}$ に ${\overset{\textnormal{おちい}}{\text{陥}}}$ った。 \hfill\break
I fell into difficulties. }
 
\par{\textbf{Word Note }: Notice that 陥る is used instead of ${\overset{\textnormal{お}}{\text{落}}}$ ちる. The words are related to each other. You would be right in thinking the おち in these words is the same thing. }
 
\par{13. 私の学校には韓国人の先生がいます。 \hfill\break
There is a teacher who's Korean in my school. }
 
\par{14. 東京に ${\overset{\textnormal{す}}{\text{住}}}$ んでいますか。 \hfill\break
Do you live in Tokyo? }
 
\par{\textbf{Definition Note }: 住む is "to live (somewhere)" and denotes the moment of moving in. ~ている shows the state of living there. 住む isn't used for animals. ${\overset{\textnormal{せいそく}}{\text{生息}}}$ する is. }
 
\par{15. この ${\overset{\textnormal{ぞう}}{\text{象}}}$ はインドに生息する。 \hfill\break
This elephant lives\slash inhabits India. }
 
\begin{itemize}
 
\item Shows \textbf{destination or      the direction of an action or goal }. As an extension, it also      marks \textbf{indirect objects }. Both usages are used with      intransitive verbs.  
\end{itemize}
 
\par{16a. 彼に会う。〇 \hfill\break
16b. 彼を会う。X \hfill\break
To meet him. }
 
\par{17. 雪が ${\overset{\textnormal{ふ}}{\text{降}}}$ って、 ${\overset{\textnormal{なだれ}}{\text{雪崩}}}$ に ${\overset{\textnormal{あ}}{\text{遭}}}$ った。 \hfill\break
Snow fell and then we got caught in an avalanche. }
 
\par{18. 私はいとこに手紙を書きました。 \hfill\break
I wrote a letter to my cousin. }
 
\par{19. ${\overset{\textnormal{われわれ}}{\text{我々}}}$ はニューヨークに ${\overset{\textnormal{つ}}{\text{着}}}$ きました。 \hfill\break
We arrived in New York City. }
 
\par{20. ${\overset{\textnormal{へや}}{\text{部屋}}}$ に ${\overset{\textnormal{もど}}{\text{戻}}}$ る。 \hfill\break
To return to a room. }
 
\par{21. ${\overset{\textnormal{うち}}{\text{家}}}$ に帰る。 \hfill\break
To return home. }
 
\par{22. ${\overset{\textnormal{きみ}}{\text{君}}}$ に ${\overset{\textnormal{でんわ}}{\text{電話}}}$ だよ。(Masculine; casual) \hfill\break
There's a phone call for you. }
 
\par{23. 友だちがカリフォルニアに来ます。 \hfill\break
My friend is coming to California. }
 
\par{24. スポーツに ${\overset{\textnormal{ねっちゅう}}{\text{熱中}}}$ する。 \hfill\break
To be bent on sports. }
 
\par{25. 日本に ${\overset{\textnormal{りょこう}}{\text{旅行}}}$ する ${\overset{\textnormal{きかい}}{\text{機会}}}$ に任天堂の ${\overset{\textnormal{ほんしゃ}}{\text{本社}}}$ に行く。 \hfill\break
Taking the opportunity of traveling to Japan and going to the headquarters of Nintendo. }
 
\par{\textbf{Word Note }: The word "旅行" does mean "journey or travel" but more in the sense of a physical "travel" or "trip." The word ${\overset{\textnormal{たび}}{\text{旅}}}$ means the same thing as 旅行 but it also implies more than the physical aspects of "travel"; it also can include the emotional, psychological and other abstract feelings that are sometimes involved in a "journey." }
 
\par{26. バスが来なくて、授業に ${\overset{\textnormal{おく}}{\text{遅}}}$ れました。 \hfill\break
The bus didn't come, and so I was late to class. }
 
\par{27. 我々は一つの ${\overset{\textnormal{けつろん}}{\text{結論}}}$ に ${\overset{\textnormal{たっ}}{\text{達}}}$ しました。 \hfill\break
We arrived at a conclusion. }
 
\par{28. ${\overset{\textnormal{こんき}}{\text{今期}}}$ の ${\overset{\textnormal{りえき}}{\text{利益}}}$ は百億円に達しました。 \hfill\break
This term's profits reached ten billion yen. }
 
\par{29. 彼女は ${\overset{\textnormal{えき}}{\text{駅}}}$ に ${\overset{\textnormal{とうちゃく}}{\text{到着}}}$ しました。 \hfill\break
She arrived at the station. }
 
\par{30. 今日、 ${\overset{\textnormal{じゅぎょう}}{\text{授業}}}$ に行きます。 \hfill\break
I will go to class today. }
 
\begin{itemize}
 
\item Shows the \textbf{effect, condition,      state, or goal of an action }, including going somewhere. It may also      show pretext as in "you make your hands \textbf{as }a      pillow."  
\end{itemize}
 
\par{31. 彼女は ${\overset{\textnormal{びょうき}}{\text{病気}}}$ になった。 \hfill\break
She became sick. }
 
\par{\textbf{Word Note }: 病気 refers to an "illness" worthy of going to the hospital. When you don't feel well, say something like ${\overset{\textnormal{きぶん}}{\text{気分}}}$ が悪い or (体の) ${\overset{\textnormal{ちょうし}}{\text{調子}}}$ が悪い. }
 
\par{32. ${\overset{\textnormal{しごと}}{\text{仕事}}}$ に出かける。 \hfill\break
Start to go to work. }
 
\par{33. 彼らは ${\overset{\textnormal{わかもの}}{\text{若者}}}$ を ${\overset{\textnormal{こうほしゃ}}{\text{候補者}}}$ に ${\overset{\textnormal{た}}{\text{立}}}$ てた。 \hfill\break
They put a young person as the candidate. }
 
\par{34. 手を ${\overset{\textnormal{さゆう}}{\text{左右}}}$ に ${\overset{\textnormal{ふ}}{\text{振}}}$ った。 \hfill\break
I waved my hands left and right. }
 
\par{35. ${\overset{\textnormal{けいさつ}}{\text{警察}}}$ は彼らの ${\overset{\textnormal{きゅうじょ}}{\text{救助}}}$ に行った。 \hfill\break
The police went to their rescue. }
 
\par{36. 私は ${\overset{\textnormal{おきなわりょこう}}{\text{沖縄旅行}}}$ のお ${\overset{\textnormal{みやげ}}{\text{土産}}}$ にシーサーを買いました。 \hfill\break
I bought a shisa as a souvenir of the trip to Okinawa. }
 
\par{\textbf{Culture Note }: A shisa is a guardian dog\slash lion hybrid often seen in statues and souvenirs in Okinawa. シーサー comes from the Okinawan word for lion, which is シシ in Japanese. }
    