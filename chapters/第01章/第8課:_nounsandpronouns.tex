    
\chapter{Nouns \& Pronouns}

\begin{center}
\begin{Large}
第8課: Nouns \& Pronouns 
\end{Large}
\end{center}
 
\par{ \textbf{Nouns }are the easiest words to learn in a foreign language. Memorizing them all, however, is no easy task. By learning nouns, though, you will create a framework which, when paired with grammar, will allow you to express the things you want to talk about. }

\par{ In addition to nouns, this lesson will also serve to introduce you to \textbf{pronouns }, which are words that indirectly refer to people, direction, and things that require context to be properly understood. }
      
\section{Nouns}
 
\par{ In a basic understanding, a \textbf{noun }( \emph{meishi }名詞) \emph{represents a person, place, state, quality, event, or thing }. In Japanese, nouns have no number or gender. This means that there is no fundamental distinction between singular and plural forms or masculine and feminine forms. In addition, there are no articles like "a," "an," or "the" that accompany nouns like is the case in English.  }

\par{ Some of the most common nouns in Japanese include the following. Many of these words are also written with the most basic \emph{Kanji }漢字 that are taught early on in Japanese education. }

\begin{ltabulary}{|P|P|P|P|P|P|}
\hline 

Karaoke &  \emph{Karaoke }カラオケ \hfill\break
&  \emph{Ramen }&  \emph{Rāmen }ラーメン \hfill\break
& Karate &  \emph{Karate }空手 \\ \cline{1-6}

Alcohol &  \emph{(O-)sake }(お)酒 & Sushi &  \emph{Sushi }寿司 & Mountain &  \emph{Yama }山 \\ \cline{1-6}

Anime &  \emph{Anime }アニメ & Manga &  \emph{Manga }マンガ & Dog &  \emph{Inu }犬 \hfill\break
\\ \cline{1-6}

Cat &  \emph{Neko }猫 & Tea &  \emph{Ocha }お茶 & Water &  \emph{Mizu }水 \\ \cline{1-6}

Sea &  \emph{Umi }海 & Fire &  \emph{Hi }火 & Bamboo &  \emph{Take }竹 \\ \cline{1-6}

Hill &  \emph{Oka }丘 & Tree &  \emph{Ki }木 & Grass &  \emph{Kusa }草 \\ \cline{1-6}

Person &  \emph{Hito }人 & Car &  \emph{Kuruma }車 & Yen &  \emph{En }円 \\ \cline{1-6}

Flower &  \emph{Hana }花 & Sound &  \emph{Oto }音 & Sky &  \emph{Sora }空 \\ \cline{1-6}

Mouth &  \emph{Kuchi }口 & Hand &  \emph{Te }手 & Leg\slash foot &  \emph{Ashi }脚・足 \\ \cline{1-6}

Ear &  \emph{Mimi }耳 & Man &  \emph{Otoko }男 & Woman &  \emph{On'na }女 \\ \cline{1-6}

Sun &  \emph{Hi\slash taiy }\emph{ō }日・太陽 & Stone &  \emph{Ishi }石 & River &  \emph{Kawa }川 \\ \cline{1-6}

Village &  \emph{Mura }村 & Town &  \emph{Machi }町 & Bug &  \emph{Mushi }虫 \\ \cline{1-6}

Countryside &  \emph{Inaka }田舎 & Ground &  \emph{Tsuchi }土 & Book &  \emph{Hon }本 \\ \cline{1-6}

Name &  \emph{Namae }名前 & Strength &  \emph{Chikara }力 & Eye(s) &  \emph{Me }目 \\ \cline{1-6}

King &  \emph{Ō }王 & Queen &  \emph{Jo' }\emph{ō }女王 & Rain &  \emph{Ame }雨 \\ \cline{1-6}

Gold &  \emph{Kin }金 & Silver &  \emph{Gin }銀 & Money &  \emph{Okane }お金 \\ \cline{1-6}

School &  \emph{Gakk }\emph{ō }学校 & Thread &  \emph{Ito }糸 & Year &  \emph{Toshi }年 \\ \cline{1-6}

Cloud &  \emph{Kumo }雲 & Song &  \emph{Uta }歌 & Fish &  \emph{Sakana }魚 \\ \cline{1-6}

Face &  \emph{Kao }顔 & Cow &  \emph{Ushi }牛 & Shape &  \emph{Katachi }形 \\ \cline{1-6}

\end{ltabulary}

\par{\textbf{Grammar Note }: Making nouns plural, although not common, is still possible. One method involves the suffix - \emph{tachi }たち, which is typically used to refer to a group of p eople or (living) things. }

\par{1. ${\overset{\textnormal{じょせい}}{\text{女性}}}$ たち \hfill\break
\emph{Joseitachi } \hfill\break
(Group of ) women }

\par{2. ${\overset{\textnormal{だんせい}}{\text{男性}}}$ たち \hfill\break
\emph{Danseitachi }\hfill\break
(Group of) men }

\par{3. ${\overset{\textnormal{いぬ}}{\text{犬}}}$ たち \hfill\break
\emph{Inutachi \hfill\break
}(A group of) dogs }

\begin{center}
\textbf{Proper Nouns } 
\end{center}

\par{ In English, a proper noun is a noun that indicates an individual person, place, organization, etc. and is spelled with initial capital letters. In Japanese, words are not capitalized, but the concept of proper noun ( \emph{koyū meishi }固有名詞) still exists. Below is a chart with some very important examples. }

\begin{ltabulary}{|P|P|P|P|}
\hline 
 
  Tokyo 
 &    \emph{Tōkyō }東京 
 &   Kyoto 
 &   \emph{Kyōto }京都 
 \\ \cline{1-4} 
 
  Osaka 
 &    \emph{Ōsaka }大阪 
 &   Yokohama 
 &    \emph{Yokohama }横浜 
 \\ \cline{1-4} 
 
  Japan 
 &    \emph{Nihon\slash Nippon }日本 
 &   America 
 &    \emph{Amerika }アメリカ 
 \\ \cline{1-4} 
 
  Russia 
 &    \emph{Roshia }ロシア 
 &   China 
 &   Chūgoku 中国 
 \\ \cline{1-4} 
 
  Korea 
 &    \emph{Kankoku }韓国 
 &   Hokkaido 
 &    \emph{Hokkaidō }北海道 
 \\ \cline{1-4} 
 
  Honshu 
 &    \emph{Honshū }本州 
 &   Shikoku 
 &    \emph{Shikoku }四国 
 \\ \cline{1-4} 
 
  Kyushu 
 &    \emph{Kyūshū }九州 
 &   Okinawa 
 &    \emph{Okinawa }沖縄 
 \\ \cline{1-4} 
 
  Asia 
 &    \emph{Ajia }アジア 
 &   Europe 
 &    \emph{Yōroppa }ヨーロッパ 
 \\ \cline{1-4} 
 
  Africa 
 &    \emph{Afurika }アフリカ 
 &   Australia 
 &    \emph{Ōsutoraria }オーストリア 
 \\ \cline{1-4} 
 
  Antarctica 
 &    \emph{Nankyokutairiku }南極大陸 
 &   India 
 &    \emph{Indo }インド 
 \\ \cline{1-4} 
 
  Kanto Region 
 &    \emph{Kanto Chihō }関東地方 
 &   Kinki Region 
 &    \emph{Kinki Chihō }近畿地方 
 \\ \cline{1-4} 
 
  Shinzo Abe 
 &    \emph{Abe Shinzō }安倍晋三 
 &   Barack Obama 
 &    \emph{Baraku Obama }バラク・オバマ 
 \\ \cline{1-4} 
 
  Tokyo   Skytree 
 &    \emph{Tokyo Sukaitsurii }東京スカイツリー 
 &   Ueno Park 
 &    \emph{Ueno Kōen }上野公園 
 \\ \cline{1-4} 

\end{ltabulary}

\par{\textbf{Word Notes }: \hfill\break
1. There are four main islands of Japan. The northernmost island is Hokkaido. South of it is the largest island, Honshu. Further south are the islands of Shikoku and Kyushu, with Kyushu being the southernmost island. Further south is a chain of islands referred to as Okinawa. \hfill\break
2. The Kanto Region encompasses the capital of Japan, Tokyo, as well as the surrounding area. \hfill\break
3. The Kinki Region encompasses both Osaka and Kyoto and their surrounding areas. \hfill\break
4. Shinzo Abe is the current Prime Minister of Japan. \hfill\break
5. Barack Obama is the 44th president of the United States. \hfill\break
6. Tokyo Skytree is the second tallest structure found in the world and is located in Tokyo. \hfill\break
7. Ueno Park is a very spacious park found in Tokyo. }

\begin{center}
 \textbf{Loan-words }
\end{center}

\par{ A \textbf{loan-word } \emph{is a word borrowed from another language }. In Japanese, there are many loanwords from all sorts of languages. Loan-words are called \emph{Gairaigo }外来語, and this term typically refers to words that have been borrowed in the last two, three centuries from the world's modern languages. Words borrowed from Chinese during the language's development--Kango 漢語--are usually treated separately. Words that have come from Chinese lan guages in recent centuries like \emph{ch }\emph{ā }\emph{han }炒飯 (fried rice), though, are treated as \emph{Gairaigo }外来語. }

\par{ Although loan-words come from dozens of languages, the overwhelmingly majority of them come from English. As convenient as that may be, you must still treat these loan words as Japanese words. This means you can't simply pronounce it as if it were English. You will likely not be understood. It is always important that you pronounce words in Japanese like any other word in Japanese regardless of whether or not it comes from English. }

\begin{ltabulary}{|P|P|P|P|P|P|}
\hline 

Meter &  \emph{Mētoru }メートル \hfill\break
& Game &  \emph{G }\emph{ēmu }ゲーム \hfill\break
& Bus & \emph{Basu }バス \\ \cline{1-6}

Pen &  \emph{Pen }ペン \hfill\break
& Sofa &  \emph{Sofā }ソファー \hfill\break
& Pie\slash pi &  \emph{Pai }パイ \hfill\break
\\ \cline{1-6}

Point &  \emph{Pointo }ポイント \hfill\break
& Cola &  \emph{Kōra }コーラ \hfill\break
& Coffee &  \emph{Kōhii }コーヒー \hfill\break
\\ \cline{1-6}

Tobacco &  \emph{Tabako }タバコ & Tomato &  \emph{Tomato }トマト & Banana & Banana バナナ \\ \cline{1-6}

\end{ltabulary}
\hfill\break

\begin{center}
\textbf{Pronouns: Grammatical Person }
\end{center}

\par{ A \textbf{pronoun }( \emph{daimeishi }代名詞) \emph{indirectly refers to an entity that involves a person, direction, or thing }. The meaning of said entity is determined by context. For instance, proper names are pronouns because they stand in place of the actual person\slash thing they reference. Proper names can also be shared with others or other things, and so we need context to truly understand what is meant by say the name "Seth." This can refer to the creator of this curriculum, or it can refer to any other person whose name is "Seth." Because of this, the word "Seth" is a pronoun. }

\par{ Similarly, words like "here" and "there" or even words like "this" and "that" are also pronouns. This is because no one can ascertain what they refer to without context. }

\par{ Generally, when we think of pronouns, we think about pronouns that are used to establish grammatical person. For instance, in English we make the following distinctions in grammatical person. }

\begin{ltabulary}{|P|P|P|}
\hline 

Person & Singular & Plural \\ \cline{1-3}

1st & I & We \\ \cline{1-3}

2nd & You & You (all) \\ \cline{1-3}

3rd & He\slash she\slash it & They \\ \cline{1-3}

\end{ltabulary}

\par{ In English, gender and number both play roles in determining what grammatical person is used in a sentence. In Japanese, however, there isn't a single pronoun that corresponds to each of the pronouns for grammatical person. Meaning, there is more than one word for "I," "we," etc. This is because all pronouns in Japanese started out as typical nouns, or they were far vaguer pronouns that didn't necessarily match up with the concept of showing grammatical person. }

\par{ In Japanese, pronouns differ by their politeness and by who actually uses them. Many pronouns are reserved for whether the speaker is male or female, or whether the person is young or old. Dialects also differ majorly in what pronouns are used. }

\par{ For the purposes of understanding basic Standard Japanese, the pronouns listed below are the most essential. As you will see, the notes provided for them show just how different they are from their English counterparts. }

\begin{ltabulary}{|P|P|P|}
\hline 

Person & Singular & Plural \\ \cline{1-3}

1st &  \emph{Wata(ku)shi }私 \hfill\break
 \emph{Boku }僕 &  \emph{Wata(ku)shitachi }私たち \hfill\break
 \emph{Bokutachi }僕たち \\ \cline{1-3}

2nd &  \emph{Anata }あなた &  \emph{Anatatachi }あなたたち \\ \cline{1-3}

3rd &  \emph{Kare }彼 (He) \hfill\break
 \emph{Kanojo }彼女 (She) &  \emph{Karera }彼ら (They) \hfill\break
 \emph{Kanojotachi }彼女たち (They) \\ \cline{1-3}

\end{ltabulary}

\par{\textbf{Usage Notes }: }
1. \emph{Watakushi }わたくし is the respectful form of \emph{watashi }わたし. Typically, \emph{watashi }わたし will suffice in most situations. However, "I" is often dropped altogether. So long as it is known that the sentence is about oneself, there is no need to have to use a pronoun for "I." \hfill\break
2. \emph{Boku }僕 is another pronoun for "I" which is used heavily by male speakers, both young and old, in various situations. \hfill\break
3. There are many pronouns for "you" in Japanese, but the most neutral in terms of politeness and purpose is \emph{anata }あなた. However, it is typically dropped from most sentences altogether. \emph{Anata }あなた is inherently direct, which is a quality not typically found in statements directed at others. Second person statements are rarely stated in absolute terms. Usually, referring to others in third person is preferred. \hfill\break
4. \emph{Anata }あなた may also be used as a term of endearment to refer to one's male partner\slash spouse, both when one is happy and mad at one's significant other. \hfill\break
5. In casual settings, the pronouns \emph{kare }彼 and \emph{kanojo }彼女 may respectively mean "boyfriend" and "girlfriend." \hfill\break
6. The plural form of \emph{kare }彼, \emph{karera }彼ら, uses another suffix for making plurals, - \emph{ra }ら. We will revisit the concept of pluralization in Lesson 92. \hfill\break
8. \emph{Karera }彼ら may refer to a group of people with both men and women, but \emph{kanojotachi }彼女たち only refers to groups of women. 7. T he pronoun "it" is omitted because it is combined in Japanese with the concept of "that," which is also a pronoun. It will be introduced later in this lesson.   \hfill\break
 In English, pronouns change form depending on their grammatical purpose in a sentence. In grammar, this is referred to as grammatical case. Grammatical case reflects the function that a given phrase has in a sentence. In English, pronouns are what change the most depending on their case. For instance, "my" is the possessive form of "I." \hfill\break
 \hfill\break
 In Japanese, grammatical case is marked by the use of what are called "case particles." Case particles take the place of form change to nouns\slash pronouns to indicate what function said word has in a sentence. For the purpose of pronouns, there are two grammatical cases you should know about: the nominative and the possessive. 
\begin{itemize}

\item Nominative Case: Noun\slash pronoun used as the subject of the sentence. \hfill\break

\item Possessive Case: Noun\slash pronoun form that shows ownership. 
\end{itemize}
i. I am an American man. (I = nominative) \hfill\break
ii. Your dog is adorable. (Your = possessive) \hfill\break
iii. We are scientists. (We = nominative) \hfill\break
iv. Our goal is world peace. (Our = possessive) \hfill\break
\hfill\break
 The subject of a sentence is the person\slash thing that performs an action or exhibits some description which is what the sentence is about. In Japanese, case particles attached to noun\slash pronouns express these grammatical concepts. To mark something as being in the nominative case, you add the particle \emph{ga }が to said noun\slash pronoun. To mark something as being in the possessive case, you add the particle \emph{no }の to said noun\slash pronoun. \hfill\break
\hfill\break

\begin{ltabulary}{|P|P|P|P|}
\hline 

 & Nominative &  & Possessive \\ \cline{1-4}

I\dothyp{}\dothyp{}\dothyp{} &  \emph{Wata(ku)shi ga }私が \hfill\break
 \emph{Boku ga }僕が & My &  \emph{Wata(ku)shi no }私の \hfill\break
 \emph{Boku no }僕の \\ \cline{1-4}

We\dothyp{}\dothyp{}\dothyp{} &  \emph{Wata(ku)shitachi ga }私たちが \hfill\break
 \emph{Bokutachi ga }僕たちが & Our &  \emph{Wata(ku)shitachi no }私たちの \hfill\break
 \emph{Bokutachi no }僕たちの \\ \cline{1-4}

You\dothyp{}\dothyp{}\dothyp{} &  \emph{Anata ga }あなたが & Your &  \emph{Anata no }あなたの \\ \cline{1-4}

You (all)\dothyp{}\dothyp{}\dothyp{} &  \emph{Anatatachi ga }あなたたちが & Your &  \emph{Anatatachi no }あなたたちの \\ \cline{1-4}

He\dothyp{}\dothyp{}\dothyp{} \hfill\break
She\dothyp{}\dothyp{}\dothyp{} &  \emph{Kare ga }彼が (He) \hfill\break
 \emph{Kanojo ga }彼女が (She) & His \hfill\break
Her &  \emph{Kare no }彼の \hfill\break
 \emph{Kanojo no }彼女の \\ \cline{1-4}

They\dothyp{}\dothyp{}\dothyp{} &  \emph{Karera ga }彼らが \hfill\break
 \emph{Kanojotachi ga }彼女たちが & Their &  \emph{Karera no }彼らの \hfill\break
 \emph{Kanojotachi no }彼女たちの \\ \cline{1-4}

\end{ltabulary}

\par{\textbf{Grammar Note }: To make a third person reference into the possessive case, just add no to whatever name you're using. This means that "Seth's" would be expressed as \emph{Sesu no }セスの. }

\par{ When we start learning how to make sentences in Lesson 9, it will be important to remember that different things will happen in a Japanese sentence. However, we won't revisit the particle \emph{ga }が until Lesson 11. }
\textbf{Pronouns: Place \& Things (Demonstratives) }\hfill\break
  In addition to the pronouns shown above for grammatical person, we still need to learn about the basic pronouns used to indicate place or thing. As was the case with the pronouns above, none of these words are used exactly like their English counterparts. This means that later on, we will have to revisit them to learn more about how they're truly used. For now, it's important to simply have them in your vocabulary.  
\begin{ltabulary}{|P|P|P|}
\hline 

Close to Speaker & Close to Listener\slash  \hfill\break
Known only to Speaker & Far from Speaker and Listener\slash  \hfill\break
Known to both Speaker and Listener \\ \cline{1-3}

\textbf{Here }& \textbf{There }& \textbf{Over There }\\ \cline{1-3}

 \emph{Koko }ここ &  \emph{Soko }そこ &  \emph{Asoko }あそこ \\ \cline{1-3}

 \textbf{This }& \textbf{That }& \textbf{That over there }\\ \cline{1-3}

 \emph{Kore }これ &  \emph{Sore }それ &  \emph{Are }あれ \\ \cline{1-3}

\end{ltabulary}
\hfill\break
\textbf{Chart Notes }:  \hfill\break
1. \emph{Sore }それ is the closest Japanese equivalent to "it." \hfill\break
2. For now, we will forego covering what these words look like in the grammatical cases mentioned above. This is because things become slightly more complicated with these sorts of pronouns than with pronouns for grammatical person.   When speaking about entities \textbf{physically visible }, \emph{there is a three-way distinction made based on the proximity of the entity from the speaker and listener }. An entity may be close to the speaker, close to the listener but not the speaker, or far from both the speaker and the listener(s). When the entity discussed is \textbf{not physically visible }, \emph{there is a two-way distinction made based on who knows about the entity in question }. The criterion then becomes whether only the speaker knows about the entity or if both the speaker and listener(s) know about it.     