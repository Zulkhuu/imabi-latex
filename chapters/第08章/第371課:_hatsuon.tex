    
\chapter{Phonology V}

\begin{center}
\begin{Large}
第371課: Phonology V: 撥音添加 \& 撥音化 
\end{Large}
\end{center}
 
\par{ There are two interesting phenomena that deal with ん. ん is known as the 撥音, which interestingly sounds the same as the word for pronunciation, 撥音. Anyway, the topics for this lesson are actually very easy. The first phenomenon, 撥音添加, involves the insertion of ん in phrases mainly before nasal sounds but also other voiced stops. 撥音化 refers to sounds becoming ん. There are obligatory instances of 撥音化 as well as colloquial\slash dialectical instances of it. Both sound changes involve many phrases with various social stigma added to them. So, this will probably be the hardest thing to grasp about these words.  }
      
\section{撥音添加}
 
\par{ Long ago, there would not have been any uvular nasal ん in Japanese. As is reflected in older spelling, it no doubt came about from the contraction of む, which was the Japanese equivalent for final nasal consonants in Chinese loans. So, one could imagine that this "mu" in words from Chinese was actually being pronounced as a syllabic m. After all, the script had no way of writing out final consonants or syllabic consonants. It was only until modern spelling reform that む and ん were officially distinguished from each other as only referring to \slash mu\slash  and \slash N\slash  respectively. }

\par{ Anyway, most consonants can be made long in Japanese, so the nasal sounds are no exceptions. When this occurs, you get things like みな (Everyone) \textrightarrow  みんな. This is usually the first example the comes into mind. However, as you should know, the form みんな has restrictions. For instance, it is forbidden from being used in polite forms such as in みなさん. Take note that さん actually comes from さまand is less polite than it. So, keep this in mind for the next section on 撥音化. }

\par{ So, we know that there is a general tendency for words with 撥音添加 tend to be less polite. However, the word for kite as in the bird is almost always とんび instead of とび for the majority of speakers. This is interesting because the word comes from the 連用形 of the verb 飛ぶ and all other similar examples are not treated so kindly as とんび. }

\par{ There is also the tendency of 撥音添加 being for nasal sounds, but what about other voiced sounds? For instance, in casual\slash dialectical speech, など can appear as なんぞ. In older language and in some dialects, you can't even still find なんど. It may be tempting to call this 撥音添加, and many people do. However, in Middle Japanese, all voiced sounds had prenasalization. So, it would be best to analyze this words as the prenasalization becoming a fully nasal syllable. など, then, would be the deletion of ん. The same goes for とんび (鳶). How prenasalization came to be a feature of voiced consonants is uncertain for all words, but in these cases, it is clearly not from an intervening の. So, we can't call it 撥音化. Thus, this is just a finer case of 撥音添加. }

\begin{center}
  \textbf{List of 撥音添加 Examples }
\end{center}

\par{ The following table will introduce you to the many examples you can find in Standard Japanese and also describe how they are used. }

\begin{ltabulary}{|P|P|P|P|}
\hline 

同じ \textrightarrow  おんなじ & Colloquial & 度 \textrightarrow  たんび* & Casual\slash dialectical \\ \cline{1-4}

まま \textrightarrow  まんま & Colloquial & とび \textrightarrow  とんび* & Standard \\ \cline{1-4}

尖る (to become sharp) \textrightarrow  とんがる & Colloquial & くだり \textrightarrow  くんだり** & Standard but old \\ \cline{1-4}

クマバチ \textrightarrow  クマンバチ*** & Standard & 黙り \textrightarrow  だんまり & Standard \\ \cline{1-4}

小締まり \textrightarrow  こぢんまり (snugly)**** & Standard & 金(かな) \textrightarrow  鉋(かんな)**** & Standard \\ \cline{1-4}

こぶ \textrightarrow  昆布(こんぶ)***** & Standard & すで \textrightarrow  すんで****** & Set phrases \\ \cline{1-4}

すごい \textrightarrow  すんごい* & Emphatic & 見事 \textrightarrow  みんごと & Emphatic \\ \cline{1-4}

あまり \textrightarrow  あんまり & Casual\slash emphatic & 真丸 \textrightarrow  真ん丸******* & Standard \\ \cline{1-4}

真中 \textrightarrow  真ん中******* & Standard & 真前 \textrightarrow  真ん前******* & Standard \\ \cline{1-4}

\end{ltabulary}

\par{*: For instances other than double m and n, deletion of nasalization takes place rather than the addition of ん. However, we can still call this 撥音添加 because that's how the forms with ん came about. It's just it appears that voiced stops (other than m and n) came in at the start with pre-nasalization, and it's just that the modern forms without ん are the new forms. \hfill\break
**: くんだり and くだり cannot be treated synonymous anymore and have become separate words. くんだり is a rather emphatic suffix that attaches to place names to show where one is going far off too, and it is now almost never seen in the spoken language. \hfill\break
***: It is not certain whether the ん comes from の or not because it is clearly a compound. However, because not all compounding in Japanese requires the particle の, it is fair to say for this discussion that this word could very well either be an example of 撥音添加 or 撥音化 of の \textrightarrow  ん.  \hfill\break
****: The first forms of these words no longer exist anymore. 鉋 = plane (for cutting wood). \hfill\break
*****: The etymology of 昆布 is obscure as there are many historical terms for it in both Japan and China. ん deletion is not uncommon, and the mainstream of thought is that こぶ is a dialectical pronunciation brought about by this. If this is the case, then this is not a good example of 撥音添加 because the 音読み of 昆 is コン. Thus, the ん would not derive from the prenasalization of ぶ from an earlier pronunciation. \hfill\break
******: すんでの所・すんでのこと = very nearly. \hfill\break
*******: ん can be viewed as obligatory in these phrases. まなか exists but is rarely used. ままる exist as a surname, but it would not be used to mean "perfect circle" as a regular noun without 撥音添加. }
      
\section{撥音化}
 
\begin{center}
\textbf{撥音便 }
\end{center}

\par{ The first major instance of 撥音化 to occur in Japanese is called 撥音便. 撥音便 refers to the 連用形 of 五段・四段 verbs ending in ぬ・ぶ・む going from に・び・み respectively to ん when used with the particle て and the auxiliary ~た. This began in the 平安時代 and was here to stay by the 室町時代. It is during the 鎌倉時代 when ん is fully treated as a separate phoneme in Japanese. }

\begin{ltabulary}{|P|P|}
\hline 

ナ行撥音便 &  死ぬ \textrightarrow  死に + た・て = 死んだ・死んで \\ \cline{1-2}

バ行撥音便 &  呼ぶ \textrightarrow  呼び + た・て = 呼んだ・呼んで \\ \cline{1-2}

マ行撥音便 &  読む \textrightarrow  読み + た・て = 読んだ・読ん で \\ \cline{1-2}

\end{ltabulary}

\begin{center}
 \textbf{History and Review of the Pronunciation of ん }
\end{center}

\par{ Because of the nature of ん, the Japanese syllabic structure is called moraic rather than syllabic. However, there are some dialects in which ん is not moraic and forms CVC syllables. This aids in accent systems sounding more distinct because if ん is syllabic in some dialects and not in others, the dialects which treat it syllabically will allow pitch to rise and fall on it. Those which don't would not be able to do this. Regions of Japan which don't treat ん as a mora include northern 東北 and southern 九州. Treating ん as a mora, though, has existed as early as the late 平安時代. So, it's been around for a long time as such. }

\par{ Another thing that you must not forget is the assimilation rules that go along with the modern ん. }

\begin{ltabulary}{|P|P|P|}
\hline 

Before bilabials & ん \textrightarrow  [m] & 3枚 = [sa m̩ mai] \\ \cline{1-3}

Before n or z & ん \textrightarrow  [n] & 女 = [o n̩ na] \\ \cline{1-3}

Before velars & ん \textrightarrow  [ ŋ] & 珊瑚 (coral) = [sa ŋ̩̩ ŋ o] \\ \cline{1-3}

Before vowels, fricatives, and approximants & ん \textrightarrow  [ ĩ~ɯ̃] & 単位 = [ta ĩ i] \\ \cline{1-3}

\end{ltabulary}

\par{\textbf{Word Note }: おんな is actually an example of 撥音化 because it comes from をむな. }

\begin{center}
\textbf{The Contraction of の }
\end{center}

\par{ の has been contracted far before the moraic ん ever arrived. In fact, many believe that voicing in compounds is resultant of an intervening の contracting to prenasalization before the voiced consonant. An example of this would be 水. 水 is believed to derive from みのつ (身の津). Then, it contracted to みづ. づ would have been pronounced as [ ⁿdz] . }

\par{ の \textrightarrow  ん examples from more modern Japanese are usually treated as being examples of 口語. In other words, you would most likely not use such contractions in the written language. For instance, んだ would more likely be のだ or のである. If we consider very common examples of this such as こんな (このよな), ここんとこ (ここのところ), 僕んち (僕の家), this statement holds true. こんな, そんな, あんな, どんな may be more likely as they are slightly older than the examples after them, but people still treat them as contraction and do avoid using them in writing. }

\begin{center}
 \textbf{Word Initial 撥音化 }
\end{center}

\par{ In many dialects, especially in 東北, ん can be seen word initially due to 撥音化. For instance, you may hear うまい pronounced as んめえ. A common example throughout this same region is んだ, which can be viewed as being cognate to うん、そうだ. Note that word initial ん in foreign transliteration does not count as 撥音化. }

\begin{center}
 \textbf{ラ行 \textrightarrow  ん }
\end{center}

\par{ The most recent form of 撥音化 in Japanese is r-sounds going to ん. ら is frequently seen as ん in colloquial Standard Japanese in the negative. The most common example is 分からない. However, any ら in conjugation can be found seen as ん. The boundaries of ラ行撥音化 are not set in stone. Many feel that any instance of it is bad Japanese and consider such phrases to be 悪い or 汚い. This most likely stems from the fact that rates of usage of ラ行撥音化 are highest in other dialects, some of which that are not thought highly. There are limits to this sound change. For instance, phrases such as 降んなければ・降んなけりゃ (降らなければ) are still quite uncommon. }

\par{ り \textrightarrow  ん is actually seen a lot in onomatopoeia where it is surprisingly accepted without any negative stigma. For example, you have surely heard きちんと a lot, but this actually is a variant of きっちりと. The two are slightly different at times now, but they still come from the same source. }

\par{ If you've ever heard people say すんな, you know of る becoming ん. This has become so pervasive that some speakers insert ん before な even when the verb doesn't end in る. So, you get things like 叫ぶんな. However, this is yet another sound change that is spreading slowly. So, you can still find people in 東京 who have never heard of this sound change extending into 撥音添加. る \textrightarrow  ん in all situations when something follows is rather dialectical, but it does exist. Usage of phrases such as 寝んから (寝るから) and すんと (すると) vary greatly even around the capital according to origin, gender, and age. }

\par{ れ \textrightarrow  ん is generally deemed to be even more dialectical or rough. For instance, けんど is commonly found in Japanese dialects instead of けれど or けど. In 東京弁, you can find the れ in passives and the potential contracted to ん in very colloquial\slash rough speech. For instance, 来らんない. Though, the potential form competes with ら抜き. So, 来らんない is not as common as 来れない. }
    