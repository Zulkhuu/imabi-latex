    
\chapter{幼児語}

\begin{center}
\begin{Large}
第374課: 幼児語 
\end{Large}
\end{center}
 
\par{ These words are of great interest. What are some of the first expressions that a Japanese speaker learns? How do these words stay into their personal lexicon? Do they not get said again after childhood until one becomes a parent? In English there are several words like "birdy" or "potty" that get used frequently by children, but they often remain as important euphemisms for the rest of one's lives. This lesson will try to investigate into such phrases that exist in Japanese. }
      
\section{幼児語}
 
\par{ Like in English, there are several special phrases used a lot in the conversations of little kids but normally not at all in actual adult conversations. Some are simply used to try to get a small child to speak. Such words are often used to help young children practice making sounds by using words that are relatively easy to say. }

\par{ One source of child words are onomatopoeia used as a nouns. Some become verb phrases to replace the more difficult actually used phrase. }

\begin{ltabulary}{|P|P|P|P|P|P|}
\hline 

幼児語 & 意味 & 幼児語 & 意味 & 幼児語 & 意味 \\ \cline{1-6}

ニャンニャン・まーお & 猫 & ワンワン & 犬 & モーモー & 牛 \\ \cline{1-6}

ブ(ー)ブ(ー) & 自動車 & ポッポ & 鳩 & コッコー & 鶏 \\ \cline{1-6}

チッチ & 排尿 & ワンワン・モーモーする & 四つん這いになる & ブンブン & 蜂 \\ \cline{1-6}

チュンチュン・小鳥 & ねずみ & カーカー & 烏 & しーしー & 小便 \\ \cline{1-6}

\end{ltabulary}

\par{ The repetition of sounds or whole words is also common. Many words contain only certain kinds of sounds. This is assimilation of speech to make things easier to say. }

\begin{ltabulary}{|P|P|P|P|P|P|}
\hline 

幼児語 & 意味 & 幼児語 & 意味 & 幼児語 & 意味 \\ \cline{1-6}

まんま・んま & ご飯 & ぶぶ & 飲み水 & ジージー & 祖父 \\ \cline{1-6}

うまうま & うまい & しゅっしゅ & 汽車 & たっち & 立つ \\ \cline{1-6}

ぺっぺ & 汚い & ぱちぱち(する) & 拍手(する) & かえっこ & 交換する \\ \cline{1-6}

ぴーぽぴーぽ & 消防車 & ぺろぺろ & 飴ん棒 & ぴー & 笛 \\ \cline{1-6}

おじょじょ & 怖い & じょんじょ・ぞぞ & 草履 & ぜんぜ & 銭 \\ \cline{1-6}

くっく & 靴 & てて & 手 & のんの & 伸ばす \\ \cline{1-6}

ぷう & おなら & すいすい & 魚 & ぽりぽり & おっぱい \\ \cline{1-6}

キューキュー & 救急車 & オンモ & 表面 & バーバー & 祖母 \\ \cline{1-6}

おめめ・めんめ & 眼 & ねんね & 就寝・安心毛布 & ト(ッ)ト & 魚・お父さん・鳥 \\ \cline{1-6}

じじ & 字 & かんかん & 髪の毛 & きれきれ & 拭く \\ \cline{1-6}

きしゃぽっぽ & 汽車 & くさいくさい & 大便 & まっか(か) & 赤い \\ \cline{1-6}

よしよし & 撫でる & ずるっこ & 苛める & ぺっ・げ(げ) & 吐き捨てる \\ \cline{1-6}

たんたん & お風呂 & ちゃんこ & 座る & きれいきれい & 洗う \\ \cline{1-6}

あちち & 熱い & べべ & 洋服 & はあは & 歯 \\ \cline{1-6}

\end{ltabulary}

\par{ Using ちゃん with onomatopoeia can also be used to refer to animal. For example, ワンちゃん (dog). This kind of abbreviation is common in making nicknames for your friends which can last life times. Some like うんこ for dog crap have become mainstream for all ages. Other words like this include 抱っこ (carrying in one's arms) and おんぶ (carrying on one's back). }

\par{高い高い高い: This expression is used when lifting up your kid up and down in the air. }

\par{ Of course this is not exhaustive. Each household is different. Some people just try to avoid using these phrases with their kids and try to make them say the correct word. To others, they use these phrases to their kids a lot in trying to get them to speak. There is a lot of regional variety in these expressions as well. Many kids often try mimicking what things sound like and use those sounds to call those things. }
    