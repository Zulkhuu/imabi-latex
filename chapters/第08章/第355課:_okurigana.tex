    
\chapter{送り仮名}

\begin{center}
\begin{Large}
第355課: 送り仮名 
\end{Large}
\end{center}
 
\par{ 送り仮名 in its simplest understanding is かな used for conjugation purposes. However, the rules for 送り仮名are not quite set in stone. Though variant spellings in this regard are disappearing in replace of one spelling, literature is still pervaded with variant spellings. }

\par{ Natives still show relative inconsistency in the matter. This is because guidelines by the government have not been crafted to the point that they should touch on individual spelling practices. }

\par{\textbf{Resource Note }: Much of this information is adopted from the Ministry of Education guidelines for 送り仮名 usage, which can be found at http:\slash \slash www.bunka.go.jp\slash kokugo\_ nihongo\slash joho\slash kijun\slash naikaku\slash okurikana\slash index.html  . }
      
\section{About 送り仮名}
 
\par{ In writing words in 漢字 and making clear what reading should be used, it has become orthodox to affixかな. For instance, in order to read 送 as おくる rather than ソウ or even some conjugation likeおくらない, る is affixed to it. However, things that do have 漢字spellings but are replaced with かな, 交ぜ書き, is not called 送り仮名. }

\par{ Rules for properly using 送り仮名 have been passed by the government, and it wouldn\textquotesingle t be surprising if these guidelines were modified again. However, even the preface of the guidelines published in 1973 admits that these rules do not pervasively apply to every aspect of Japanese. For instance, standards in science, the arts, and special fields are immune to having to follow these rules, which is why there have been plenty of mentions of how to address these situations thus far. }

\par{ However, it is without a doubt that these guidelines do play a substantial role in orthography used in broadcast, official documents, newspapers, etc. }

\par{ In studying the use of 送り仮名, you must first distinguish words in two groups: those that conjugate and those that don\textquotesingle t. Compound words are also important to consider. Though these different faces of 送り仮名 are rather complex, there are basic principles to keep in mind that work for the majority of cases. }
      
\section{Rule 1}
 
\par{ \emph{\textbf{If a word conjugates, it will have 送り仮名 affixed }}. }

\begin{ltabulary}{|P|P|P|P|P|P|}
\hline 

憤る & いきどおる & To resent & 承る & うけたまわる & To undertake; to hear\slash know (humble) \\ \cline{1-6}

書く & かく & To write & 生きる & いきる & To live \\ \cline{1-6}

考える & かんがえる & To think & 陥れる & おちいれる & To trick into; assault (castle); drop into \\ \cline{1-6}

助ける & たすける & To help & 催す & もよおす & To hold (event); feel (sensation) \\ \cline{1-6}

荒い & あらい & Rough; wild & 潔い & いさぎよい & Gallant; unsullied \\ \cline{1-6}

賢い & かしこい & Wise & 濃い & こい & Thick \\ \cline{1-6}

薄い & うすい & Thin & 主な & おもな & Main \\ \cline{1-6}

\end{ltabulary}

\begin{center}
\textbf{Irregularities }
\end{center}

\par{1. Adjectives that end in しい・じい such as 美しい and 凄まじい(terrible) were historically 美し and 凄まじ. So, this is why し・じ haven\textquotesingle t been dropped in the spellings. }

\par{2: Native suffixes that create adjectives such as か, らか, andやか are also left as 送り仮名. }

\begin{ltabulary}{|P|P|P|P|P|P|P|P|P|}
\hline 

細かな & こまかな & Fine; detailed & 静かな & しずかな & Quiet & 暖かな & あたたかな & Warm \\ \cline{1-9}

平らかな & たいらかな & Level; peaceful & 柔らかな & やわらかな & Tender; meek & 和やかな & なごやかな & Harmonious \\ \cline{1-9}

健やかな & すこやか & Vigorous & 鮮やかな & あざやかな & Vivid; adroit & 穏やかな & おだやかな & Calm; gentle \\ \cline{1-9}

\end{ltabulary}

\par{3: The following words are irregular. }

\begin{ltabulary}{|P|P|P|P|P|P|}
\hline 

明らむ & あからむ & To break dawn & 味わう & あじわう & To taste; savor \\ \cline{1-6}

哀れむ & あわれむ & To pity & 慈しむ & いつくしむ & To be affectionate to \\ \cline{1-6}

教わる & おそわる & To be taught & 脅かす & おどかす・おびやかす & To menace \\ \cline{1-6}

関わる & かかわる & To be concerned with & 食らう & くらう & To eat\slash drink (Vulgar) \\ \cline{1-6}

異なる & ことなる & To differ & 逆らう & さからう & To defy \\ \cline{1-6}

捕まる & つかまる & To be caught & 群がる & むらがる & To swarm; gather \\ \cline{1-6}

和らぐ & やわらぐ & To be mitigated & 揺する & ゆする & To shake; jolt \\ \cline{1-6}

明るい & あかるい & Bright & 危ない & あぶない & Dangerous \\ \cline{1-6}

危うい & あやうい & Dangerous & 大きい & おおきい & Big \\ \cline{1-6}

少ない & すくない & Few & 小さい & ちいさい & Small \\ \cline{1-6}

冷たい & つめたい & Cold & 平たい & ひらたい & Flat \\ \cline{1-6}

新たな & あらたな & New & 同じ & おなじ & Same \\ \cline{1-6}

盛んな & さかんな & Popular; enthusiastic & 平らな & たいらな & Flat; smooth \\ \cline{1-6}

\end{ltabulary}
      
\section{Rule 2}
 
\par{ \textbf{\emph{Derivations are included in 送り仮名. }}}

\par{1. Conjugations\slash derivations of verbs. }

\begin{ltabulary}{|P|P|P|P|P|P|}
\hline 

Derivation & Base Word & Derivation & Base Word & Derivation & Base Word \\ \cline{1-6}

動かす & 動く & 照らす & 照る & 語らう & 語る \\ \cline{1-6}

浮かぶ & 浮く & 生れる & 生む & 押える & 押す \\ \cline{1-6}

捕える & 捕る & 勇ましい & 勇む & 輝かしい & 輝く \\ \cline{1-6}

喜ばしい & 喜ぶ & 晴れやかだ & 晴れる & 及ぼす & 及ぶ \\ \cline{1-6}

積もる & 積む & 聞こえる & 聞く & 頼もしい & 頼む \\ \cline{1-6}

起こる & 起きる & 落とす & 落ちる & 暮らす & 暮れる \\ \cline{1-6}

冷やす & 冷える & 当たる & 当てる & 終わる & 終える \\ \cline{1-6}

変わる & 変える & 集まる & 集める & 定まる & 定める \\ \cline{1-6}

連なる & 連ねる & 交わる & 交える & 混ざる・混じる & 混ぜる \\ \cline{1-6}

恐ろしい & 恐れる & 恨めしい & 恨む & 痛ましい & 痛む \\ \cline{1-6}

\end{ltabulary}

\par{2. Words including an adjectival root. }

\begin{ltabulary}{|P|P|P|P|P|P|}
\hline 

Derivative & Base Word & Derivative & Base Word & Derivative & Base Word \\ \cline{1-6}

重んずる & 重い & 若やぐ & 若い & 怪しむ & 怪しい \\ \cline{1-6}

悲しむ & 悲しい & 苦しがる & 苦しい & 確かめる & 確かだ \\ \cline{1-6}

重たい & 重い & 憎らしい & 憎い & 古めかしい & 古い \\ \cline{1-6}

細かい & 細かだ & 柔らかい & 柔らかだ & 清らかだ & 清い \\ \cline{1-6}

高らかだ & 高い & 寂しげだ & 寂しい & 可愛げ & 可愛い \\ \cline{1-6}

\end{ltabulary}

\par{3. Things with nouns in them. }

\begin{ltabulary}{|P|P|P|P|P|P|P|P|}
\hline 

Verb & Base Noun & Verb & Base Noun & Verb & Base Noun & Verb & Base Noun \\ \cline{1-8}

汗ばむ & 汗 & 先んずる & 先 & 春めく & 春 & 後ろめたい & 後ろ \\ \cline{1-8}

\end{ltabulary}

\par{\textbf{許容 }: When there is no worry of being misread, 送り仮名 may be abbreviated as in the following words. }

\begin{ltabulary}{|P|P|P|P|}
\hline 

浮かぶ \textrightarrow  浮ぶ & 生まれる \textrightarrow  生れる & 押さえる \textrightarrow  押える & 捕らえる \textrightarrow  捕える \\ \cline{1-4}

晴れやかだ \textrightarrow  晴やかだ & 聞こえる \textrightarrow  聞える & 積もる \textrightarrow  積る & 起こる \textrightarrow  起る \\ \cline{1-4}

落とす \textrightarrow  落す & 暮らす \textrightarrow  暮す & 当たる \textrightarrow  当る & 終わる \textrightarrow  終る \\ \cline{1-4}

変わる \textrightarrow  変る &  &  &  \\ \cline{1-4}

\end{ltabulary}

\par{Note: The following words are deemed to follow Rule 1 instead: 明るい・荒い・悔しい・恋しい. }
      
\section{Rule 3}
 
\par{ \textbf{\emph{Excluding words dealt with via Rule 4, nouns shouldn\textquotesingle t have 送り仮名. }}}

\begin{ltabulary}{|P|P|P|P|P|P|P|P|P|P|P|}
\hline 

月 & 鳥 & 花 & 山 & 男 & 女 & 彼 & 何 & 草 & 上 & 下 \\ \cline{1-11}

\end{ltabulary}

\begin{center}
\textbf{Irregularities }
\end{center}

\par{1. }

\begin{ltabulary}{|P|P|P|P|P|P|P|P|P|P|P|P|}
\hline 

辺り & 哀れ & 勢い & 幾ら & 後ろ & 傍ら & 幸い & 幸せ & 全て & 互い & 便り & 半ば \\ \cline{1-12}

半ば & 情け & 斜め & 独り & 誉れ & 自ら & 幸い &  &  &  &  &  \\ \cline{1-12}

\end{ltabulary}

\par{2. With the counter つ: 一つ, 二つ, 三つ, 四つ, 五つ, 六つ, 七つ, 八つ, 九つ, 幾つ }
      
\section{Rule 4}
 
\par{ \textbf{\emph{Nouns that come from a conjugatable part of speech or those made with the suffixes ~さ, ~み, or ~げ abide by the 送り仮名 spellings of the base word. }}}

\begin{ltabulary}{|P|P|P|P|P|P|P|P|P|P|P|P|P|P|}
\hline 

動き & 仰せ & 恐れ & 薫り & 香り & 曇り & 調べ & 届け & 願い & 晴れ & 当たり & 代わり & 向かい & 狩り \\ \cline{1-14}

泳ぎ & 答え & 祭り & 群れ & 憩い & 愁い & 極み & 初め & 近く & 遠く & 暑さ & 大きさ & 正しさ & 確かさ \\ \cline{1-14}

明るみ & 哀しみ & 憎しみ & 重み & 惜しげ & 可愛げ &  &  &  &  &  &  &  &  \\ \cline{1-14}

\end{ltabulary}

\begin{center}
 \textbf{Irregularities }
\end{center}

\par{ The following words do not have 送り仮名. }

\begin{ltabulary}{|P|P|P|P|P|P|P|P|P|P|P|P|P|P|P|P|P|P|P|P|P|P|P|P|P|P|}
\hline 

謡 & 虞 & 趣 & 氷 & 印 & 頂 & 帯 & 畳 & 卸 & 煙 & 志 & 恋 & 次 & 隣 & 富 & 恥 & 話 & 光 & 舞 & 折 & 組 & 肥 & 並 & 巻 & 割 & 掛 \\ \cline{1-26}

\end{ltabulary}

\par{Note: 送り仮名is only lost when their verbal sense is lost. Ex. 話し VS 話, 氷り VS 氷. }

\par{\textbf{許容 }: In the case when there is no worry of a word being misread, 送り仮名 may be dropped in the following fashion for the words below. }

\begin{ltabulary}{|P|P|P|P|P|P|P|}
\hline 

曇り \textrightarrow  曇 & 届け \textrightarrow  届 & 願い \textrightarrow  願 & 晴れ \textrightarrow  晴 & 当たり \textrightarrow  当り & 代わり \textrightarrow  代り & 狩り \textrightarrow  狩 \\ \cline{1-7}

向かい \textrightarrow  向い & 祭り \textrightarrow  祭 & 群れ \textrightarrow  群 & 憩い \textrightarrow  憩 & 答え \textrightarrow  答 & 問い \textrightarrow  問 &  \\ \cline{1-7}

\end{ltabulary}
      
\section{Rule 5}
 
\par{ \textbf{\emph{The final mora in adverbs, attributes, and conjugations is usually 送り仮名. }}}

\begin{ltabulary}{|P|P|P|P|P|P|P|P|}
\hline 

Adverbs & 必ず & 更に & 少し & 既に & 全く & 再び & 最も \\ \cline{1-8}

Attributes & 来る & 去る &  &  &  &  &  \\ \cline{1-8}

Conjunctions & 及び & 且つ & 但し &  &  &  &  \\ \cline{1-8}

\end{ltabulary}

\begin{center}
\textbf{Irregularities }
\end{center}

\par{・With more 送り仮名: }

\begin{ltabulary}{|P|P|P|P|P|}
\hline 

明くる & 大いに & 直ちに & 並びに & 若しくは \\ \cline{1-5}

\end{ltabulary}

\par{・With no 送り仮名: 又 }

\par{Adverbs\slash Conjunctions from Verbs\slash Adjectives or + particles: }

\begin{ltabulary}{|P|P|P|P|P|}
\hline 

併せて 〔併せる〕 & 至って 〔至る〕 & 恐らく 〔恐れる〕 & 絶えず 〔絶える〕 & 例えば 〔例える〕 \\ \cline{1-5}

努めて 〔努める〕 & 辛うじて 〔辛い〕 & 少なくとも 〔少ない〕 & 互いに 〔互い & 必ずしも 〔必ず〕 \\ \cline{1-5}

\end{ltabulary}
      
\section{Rule 6}
 
\par{ \textbf{\emph{In regards to compound words excluding those dealt with via Rule 7, 送り仮名 is determined by the individual components\textquotesingle  音訓. }}}

\par{1. Examples of words that conjugate: }

\begin{ltabulary}{|P|P|P|P|P|P|P|P|P|}
\hline 

書き抜く & 流れ込む & 申し込む & 打ち合せる & 長引く & 若返る & 裏切る & 旅立つ & 聞苦しい \\ \cline{1-9}

薄暗い & 草深い & 心細い & 待遠しい & 軽々しい & 女々しい & 気軽だ & 望み薄だ &  \\ \cline{1-9}

\end{ltabulary}

\par{2. Examples of words that do not conjugate: }

\begin{ltabulary}{|P|P|P|P|P|P|P|P|P|P|}
\hline 

石橋 & 竹馬 & 山津波 & 後ろ姿 & 斜め左 & 花便り & 独り言 & 卸商 & 水煙 & 目印 \\ \cline{1-10}

物知り & 落書き & 雨上がり & 墓参り & 日当たり & 夜明かし & 先駆け & 巣立ち & 手渡し & 入り江 \\ \cline{1-10}

合わせ鏡 & 封切り & 教え子 & 生き物 & 落ち葉 & 預かり金 & 寒空 & 深情け & 愚か者 & 行き帰り \\ \cline{1-10}

乗り降り & 抜け駆け & 田植え & 飛び火 & 伸び縮み & 作り笑い & 暮らし向き & 売り上げ & 取り扱い & 乗り換え \\ \cline{1-10}

引き換え & 歩み寄り & 申し込み & 移り変わり & 長生き & 早起き & 苦し紛れ & 大写し & 次々 & 常々 \\ \cline{1-10}

近々 & 深々 & 休み休み & 行く行く &  &  &  &  &  &  \\ \cline{1-10}

\end{ltabulary}

\par{\textbf{許容 }: When there is no worry of being misread, 送り仮名 can be dropped in the following fashion in the example words. }

\begin{ltabulary}{|P|P|P|}
\hline 

書き抜く \textrightarrow  書抜く & 申し込む \textrightarrow  申込む & 打ち合わせる \textrightarrow  打ち合せる・打合せる \\ \cline{1-3}

雨上がり \textrightarrow  雨上り & 申し込み \textrightarrow  申込み・申込 & 向かい合わせる \textrightarrow  向い合せる \\ \cline{1-3}

日当たり \textrightarrow  日当り & 引き換え \textrightarrow  引換え・引換 & 立ち居振る舞い \textrightarrow  立ち居振舞い・立ち居振舞・立居振舞 \hfill\break
\\ \cline{1-3}

封切り \textrightarrow  封切 & 有り難み \textrightarrow  有難み & 呼び出し電話 \textrightarrow  呼出し電話・呼出電話 \\ \cline{1-3}

夜明かし \textrightarrow  夜明し & 暮らし向き \textrightarrow  暮し向き & 移り変わり \textrightarrow  移り変り \\ \cline{1-3}

入り江 \textrightarrow  入江 & 飛び火 \textrightarrow  飛火 & 合わせ鏡 \textrightarrow  合せ鏡 \\ \cline{1-3}

抜け駆け \textrightarrow  抜駆け & 待ち遠しい \textrightarrow  待遠しい & 売り上げ \textrightarrow  売上げ・売上 \\ \cline{1-3}

田植え \textrightarrow  田植 & 預かり金 \textrightarrow  預り金 & 取り扱い \textrightarrow  取扱い・取扱 \\ \cline{1-3}

落書き \textrightarrow  落書 & 聞き苦しい \textrightarrow  聞苦しい & 乗り換え \textrightarrow  乗換え・乗換 \\ \cline{1-3}

待ち遠しさ \textrightarrow  待遠しさ &  &  \\ \cline{1-3}

\end{ltabulary}

\par{Note: In cases like こけら落とし, さび止め, 洗いざらし, 打ちひも whether either the front or end part of the word is written in かな instead of in漢字, you should not abbreviate 送り仮名 out. }
      
\section{Rule 7}
 
\par{ \emph{\textbf{Compounds may or may not have 送り仮名 according to convention. }}}

\par{1.The first in the guidelines with this rule are particular words of domain in which  conventional spelling is recognized. }

\par{ア: Names of positions and titles: }

\begin{ltabulary}{|P|P|P|P|}
\hline 

関取 & 頭取 & 取締役 & 事務取扱 \\ \cline{1-4}

\end{ltabulary}

\par{イ: Handicraft words that end in 「織」、「染」、「塗」、「彫」、「焼」等. }

\begin{ltabulary}{|P|P|P|P|P|}
\hline 

《博多》織 & 《型絵》染 & 《春慶》塗 & 《鎌倉》彫 & 《備前》焼 \\ \cline{1-5}

\end{ltabulary}

\par{ウ; Others: }

\begin{ltabulary}{|P|P|P|P|P|P|P|P|P|}
\hline 

書留 & 気付 & 切手 & 消印 & 小包 & 振替 & 切符 & 踏切 & 手当 \hfill\break
\\ \cline{1-9}

仲買 & 両替 & 割引 & 組合 & 売値 & 買値 & 倉敷料 & 作付面積 & 請負 \\ \cline{1-9}

借入《金》 & 小売《商》 & 取扱《所》 & 取扱《注意》 & 繰越《金》 & 乗換《駅》 & 取次《店》 & 取引《所》 & 乗組《員》 \\ \cline{1-9}

引受《人》 & 引換《券》 & 《代金》引換 & 引受《時刻》 & 振出《人》 & 待合室 & 見積《書》 & 売上《高》 & 貸付《金》 \\ \cline{1-9}

申込《書》 &  &  &  &  &  &  &  &  \\ \cline{1-9}

\end{ltabulary}

\par{2. Spellings that are generally conventional. }

\begin{ltabulary}{|P|P|P|P|P|P|P|P|P|P|P|P|P|}
\hline 

奥書 & 木立 & 子守 & 献立 & 座敷 & 試合 & 字引 & 場合 & 羽織 & 葉巻 & 番組 & 番付 & 日付 \\ \cline{1-13}

水引 & 物置 & 物語 & 役割 & 屋敷 & 夕立 & 割合 & 合図 & 合間 & 植木 & 置物 & 織物 & 貸家 \\ \cline{1-13}

敷石 & 立場 & 建物 & 並木 & 巻紙 & 浮世絵 & 絵巻物 & 仕立屋 &  &  &  &  &  \\ \cline{1-13}

\end{ltabulary}

\par{Notes: }

\par{1. Even when the items in 《》 are different, these guidelines still apply. }

\par{2. This list is not exhaustive. Therefore, as far as convention may be recognized, similar words are to be dealt with likewise. When it is hard to determine whether Rule 7 should be applied or not, use Rule 6. }
      
\section{Specific Exceptions}
 
\par{ Words specifically mentioned in the Cabinet guidelines from words within the bounds of the 常用漢字表, which includes not only lists of general characters but general readings with exceptions. }

\par{浮 つく The reasoning for this is that the verb comes from the use of a suffix, つく. }

\par{お巡 り さん  This spelling is motivated by convention. }

\par{差 し 支 える 立 ち 退 く  }

\par{Note: Alternatively, 送り仮名 within compound words such as this are often dropped. Thus, 差支える and 立退く. For compounds nouns, all 送り仮名 can be dropped. Ex. 話し合い・話合い・話合. }

\par{ The following nouns were deemed to not ever have 送り仮名. }

\begin{ltabulary}{|P|P|P|P|P|P|P|P|P|}
\hline 

息吹 & 桟敷 & 時雨 & 築山 & 名残 & 雪崩 & 吹雪 & 迷子 & 行方 \\ \cline{1-9}

\end{ltabulary}
\hfill\break
 \hfill\break
    