    
\chapter{Phonology II}

\begin{center}
\begin{Large}
第368課: Phonology II: Pitch (高低アクセント) 
\end{Large}
\end{center}
 
\par{ An accent is the combination of phonetic properties attributed to words. English exhibits a stress system called a "stress accent (強弱アクセント)" in which syllables differ in how much stress is put on them. Japanese, on the other hand, has a "pitch accent (高低アクセント)" system in which syllables only differ in pitch. }

\par{ To demonstrate how pitch accent is utilized in Japanese, consider the following example well-known to demonstrate how it is sometimes what makes sentences understood correctly. }

\par{1. ニ \textbf{ワニ }ワ・ \textbf{ニ }ワ・ニ \textbf{ワトリガイル }。(庭には2羽鶏がいる。) \hfill\break
L H  H L   H  L L H H H H H H }

\par{ As important as pitch might be to help bring clarity to a sentence, its use is not uniform throughout Japan. You should expect each major dialect to exhibit its own unique twist to the same formula. Some Japanese dialects don't even utilize pitch to distinguish words. The point of this lesson, however, is to become familiar with the Standard Japanese pitch accent system. }

\par{\textbf{Technicality Note }: It is humanly impossible to transition immediately from low to high pitch or vice versa. Changes are gradual with natural curvatures. }
      
\section{Standard Japanese Pitch Accent System}
 
\par{ Unlike its neighboring language Chinese, Japanese is not considered to be tonal because pitch accent isn't fundamental to interpreting it. Pitch may be used to distinguish words every now and then, but this is not consistent throughout dialects, and pitch is not viewed as an intrinsic part of the language's lexicon. A foreigner could incidentally mess up the pitch of every syllable in a sentence and still likely be understood provided other aspects of pronunciation are fine. This is unlike Chinese in which messing up the tone of the world will inevitably change the word or make whatever you're attempting to say hard to understood. }

\par{As far as the rules of Standard Japanese's pitch accent system, there are four basic patterns to consider. These rules are designed with the forethought that affixes will attach to words. For instance, particles, auxiliary verbs, etc. influence the pitch accent of a phrase. }

\begin{ltabulary}{|P|P|P|}
\hline 

頭高型 & The pitch starts \textbf{high }, drops suddenly, then steadily goes down. & \_ \\ \cline{1-3}

中高型 & The pitch rises, reaches a maximum, then drops suddenly. & /\ \\ \cline{1-3}

尾高型 & The pitch rises then drops when it reaches an attached element. & / ̄(\) \\ \cline{1-3}

平板型 & The pitch \textbf{rises from start to end }. & / ̄( ̄) \\ \cline{1-3}

\end{ltabulary}
 
\begin{center}
 \textbf{Examples }
\end{center}
 
\begin{ltabulary}{|P|P|P|P|}
\hline 
 
  Rule 1 
 &   Rule 2 
 &   Rule 3 
 &   Rule 4 
 \\ \cline{1-4} 
 
  háɕì   (chopsticks) 
 &   hàɕí   (bridge) 
 &     &   haɕi   (edge) 
 \\ \cline{1-4} 
 
  ímà (now) 
 &   ìmá (living room) 
 &     &     \\ \cline{1-4} 
 
  kákì (oyster) 
 &   kàkí (fence) 
 &     &   kaki (persimmon) 
 \\ \cline{1-4} 
 
  níhòN' (two long items) 
 &   nìhóN' (Japan) 
 &     &     \\ \cline{1-4} 
 
  ámè (rain) 
 &     &     &   ame (candy) 
 \\ \cline{1-4} 
 
  ásà (morning) 
 &   àsá (hemp) 
 &     &     \\ \cline{1-4} 
 
  káɽàsu   (crow) 
 &   tàmágò (egg) 
 &   otoko (man) 
 &   otona (adult) 
 \\ \cline{1-4} 

\end{ltabulary}

\par{\textbf{Curriculum Note }: IPA is used in this chart. }

\begin{center}
\textbf{東京弁のアクセント型一覧 } 
\end{center}

\begin{ltabulary}{|P|P|P|P|P|P|P|}
\hline 

\multicolumn{3}{|c|}{Kind \slash  Syllable Count 
}& 2 Morae 
& 3 Morae 
& 4 Morae 
& 5 Morae 
\\ \cline{1-7}

平板式 
& \multicolumn{2}{|c|}{平板型0 
}& LH(H) \hfill\break
咲く, 鼻, 腰 
& LHH(H) \hfill\break
兎, 桜, 赤い 
& LHHH(H) \hfill\break
洗濯, 鶏 
& LHHHH(H) \hfill\break
アルコール 
\\ \cline{1-7}

\multirow{5}*{ 起 
  \hfill\break
伏 
  \hfill\break
式 
}& \multicolumn{2}{|c|}{尾高型 
}& LH(L) \hfill\break
犬, 川, 花, 山 
& LHH(L) \hfill\break
男, 女, 二人 
& LHHH(L) \hfill\break
妹,弟, 極楽 
& LHHHH(L) \hfill\break
お正月 
\\ \cline{1-0} \cline{1-7}

\multirow{3}*{中 
 高 
\hfill\break
型 
}& ② 
&  & LHL(L) \hfill\break
心, 卵, 絵本 
& LHLL(L) \hfill\break
歩いた 
& LHLLL(L) \hfill\break
不完全 
\\ \cline{1-0} \cline{1-0} \cline{1-7}

③ 
& \multicolumn{2}{|c|}{ }& LHHL(L) \hfill\break
湖, 花嫁 
& LHHLL(L) \hfill\break
ありました 
\\ \cline{1-0} \cline{1-0} \cline{1-7}

④ 
& \multicolumn{3}{|c|}{ }& LHHHL(L) \hfill\break
面白い, 暖まる 
\\ \cline{1-0} \cline{1-7}

\multicolumn{2}{|c|}{頭高型① 
}& HL(L) \hfill\break
雨, 猿, 秋, 窓 
& HLL(L) \hfill\break
命, クラス 
& HLLL(L) \hfill\break
大臣, 大きな 
& HLLLL(L) \hfill\break
アクセント 
\\ \cline{1-7}

\end{ltabulary}

\par{\textbf{Terminology Notes }: }

\par{1. 起伏 = Highs and lows \hfill\break
2. 平板 = Flat }
 Morae count doesn't change the basic patterns identified at the beginning. However, it is important to understand how these patterns are applied to words of different lengths. Take for instance the difference between 花 and 鼻. Without anything affixed to them, they are completely homophonous. However, when in phrases, they are distinguishable as being LH(L) and LH(H).   Pitch contours in phrases is not necessarily just taking the individual pitch contours of words and putting them together. For instance, the pitch of 咲く is LH(H) and the pitch of 桃 is LH(H). When you say 桃が咲く, the pitch is LHHHH(H) instead of LHHLH(H).   One aspect of 東京弁 is that although particles such as が, は, に, と, や, へ, も, を, から, and しか don't affect the pitch of nouns, の causes a similar assimilation of pitch. For instance, say you have 花 and 桃. The first is 尾高型 and the second is 平板型. However, when used with the particle の, their pitch contours become the same. The particle だけ optionally does the same as の.  3. ハ \textbf{ナノヤマ }↓  モ \textbf{モノヤマ }↓ \hfill\break
花の山        桃の山 \hfill\break
\textbf{Notation Note }: The down arrow shows a drop in pitch on the following affixed item.   What does this mean for 鼻の山 or 腿の山? Well, that is why puns exist to manipulate rules that cause such things to be possible.   What about particles more than one morae? Except for から, しか, and だけ, the pitch of such particles, which include さえ, すら, だに, でも, のみ, まで, かな, とも, など, なり, のに, ので, and combinations, is determined by the pitch of the noun. If the noun is 平板型, the first morae of the particle is accented. If the noun is accented, then there is no pitch change in the particle.  4. \textbf{う }みでも  う \textbf{りものさ }え   Despite there being so much apparently fluidity to pitch in Japanese, there is still innate pitch rules built into the language. Otherwise, how would continuously coined words and compounds be naturally assigned pitch? Of course, this "innate pitch" changes from generation to generation, and it's not completely constant across all Japanese speakers. So, those that say pitch allocation is unpredictable are not right. Pitch is indeed fluid, but it's not quite like a gas.  \textbf{Systematic Rules for the Patterns }   There are rules, after all, to what kinds of words to expect in these patterns. Everything is subject to speaker variation, but the following guidelines will definitely help you sound a lot more like a native speaker.  
\begin{ltabulary}{|P|P|P|}
\hline 

Rule 1 & All こそあど expressions are 平板型. & こ \textbf{こ }, そ \textbf{れ }, あ \textbf{の }, こ \textbf{う }, あ \textbf{あ }\\ \cline{1-3}

Rule 2 & All nouns from 平板型 verbs are also 平板型. & あ \textbf{そび }, た \textbf{たみ }, は \textbf{じめ }, た \textbf{たかい }\\ \cline{1-3}

Rule 3 & 3 morae names from verbs \textrightarrow  平板型 & イ \textbf{サム }, シ \textbf{ゲル }, \\ \cline{1-3}

Rule 4 & When the first part is 3(+) morae or 2(+) 漢字, \hfill\break
compounds with the following at the end are \hfill\break
平板型: 色, 組, 型, 玉, 寺(てら), 村, 山, 科, 家, 課, \hfill\break
語, 座, 派, 教, 産, 制, 線, 党, 病 & 血液型, 耳鼻科, 人事課, キリスト教, \hfill\break
心臓病, 政治家 \\ \cline{1-3}

Rule 5 & Female names that end in 江, 枝, 恵, 代, 世 & 良枝, 正世 \\ \cline{1-3}

Rule 6 & Male names 3 morae (+) ending in 夫, 男, 雄, 助, 介, \hfill\break
輔, 吉, 作 & 正夫, 和男, 寿輔(じゅすけ) \\ \cline{1-3}

Rule 7 & 4 morae onomatopoeic expressions with だ, に, or な \hfill\break
affixed & ざ \textbf{らざらだ }, つ \textbf{るつるに }, べ \textbf{たべたな }\\ \cline{1-3}

Rule 8 & Nouns from compound verb expressions \textrightarrow  平板型 & 試合, 立ち入り, 買い戻し, 取り扱い \\ \cline{1-3}

Rule 9 & Interrogative are 頭高型. &  \textbf{だ }れ, \textbf{な }ぜ, \textbf{い }つ \\ \cline{1-3}

Rule 10 & 2 morae adjectives from the stems of adjectives \hfill\break
\textrightarrow  頭高型 &  \textbf{あ }お, \textbf{し }ろ, \textbf{く }ろ, \textbf{あ }か, \textbf{ふ }る \\ \cline{1-3}

Rule 11 & 2 morae names \textrightarrow  頭高型 & 真理, 綾, 哲(てつ) \\ \cline{1-3}

Rule 12 & 3 morae names from adjectives \textrightarrow  頭高型 & 敦(あつし), 清志, 毅(つよし) \\ \cline{1-3}

Rule 13 & 3 morae female names ending in 子 \textrightarrow  頭高型 & 喜世子, 華子, 千代子 \\ \cline{1-3}

Rule 14 & 3 morae male names ending in 樹, 吾, 二, 次, 治, 太, \hfill\break
一, 市, 平, 兵衛, 郎 \textrightarrow  頭高型 & 正樹, 信吾, 真二, 太郎, 四郎, \\ \cline{1-3}

Rule 15 & 4 morae male names ending in 助, 輔, 介 \textrightarrow  頭高型 \hfill\break
when the second morae is special, having a long vowel, \hfill\break
ん, etc. & 庄助, 勘助, 泰輔 \\ \cline{1-3}

Rule 16 & Onomatopoeia independent or with と \textrightarrow  頭高型 &  \textbf{は }らはら(と), \textbf{で }れでれ(と), \hfill\break
 \textbf{と }んとん(と), \textbf{つ }るつる(と) \\ \cline{1-3}

Rule 17 & Compound words with the first half at least 3 morae or \hfill\break
more than 2 漢字 \textrightarrow  中高型 ending with 子, 歌, 川, 豆, \hfill\break
虫, 器, 区, 市, 府, 部, 員, 駅, 園, 会, 海, 学, 群, 県, 省, 料, \hfill\break
力, 湾 & 受 \textbf{話 }器, 千 \textbf{代田 }区, 子 \textbf{守 }唄, \hfill\break
世 \textbf{田谷 }区, 動物園(ど \textbf{うぶつ }えん) \\ \cline{1-3}

Rule 18 & 4 morae (+) male names ending in 彦, 介, 輔, 助 \textrightarrow  \hfill\break
中高型 & 朝彦, 靖彦, 福助, 孝介 \\ \cline{1-3}

Rule 19 & 3 morae female names ending in 子 with the first morae \hfill\break
devoiced \textrightarrow  中高型 & 菊子, 比沙子, 富貴子, 芙紗子 \\ \cline{1-3}

Rule 20 & When verbs with 起伏式 pitch are sent to nouns, the \hfill\break
pitch goes to 尾高型. However, if 4 morae (+), the word \hfill\break
could also be 平板型 or 中高型. &  \textbf{降 }る \textrightarrow  降 \textbf{り↓ }帰る( \textbf{か }える) \hfill\break
\textrightarrow  帰り(か \textbf{えり↓ }) \hfill\break
集まる(あ \textbf{つま }る) \textrightarrow  あ \textbf{つまり↓ }\hfill\break
\textrightarrow  あ \textbf{つまり }\hfill\break
\textrightarrow  あ \textbf{つま }り \\ \cline{1-3}

\end{ltabulary}
 \textbf{Notes }: Lists are not exhaustive. The best way to look up the pitch of a word is still by looking it up in a pitch accent dictionary. NHK編『日本語発音アクセント辞典』 is a great choice.   Another thing to consider is 呼称. 呼称 don't fit within the bounds of these 20 rules because their pitch is determined by the pitch of the noun phrase itself, although independently they may have their own pitch accent assigned. For instance, the very important endings ~さん, ~ちゃん, and ~さま behave like most particles. The pitch does not change. If the name is 頭高型, it still is with them. With 君, there is typically no change, but if the name is 平板型, the resulting phrase may become 尾高型. Lastly, titles tend to be accented, but they don't have to be when the surname is accented.  \textbf{Generational Change }  Of course, when options exist, you are simply asking for speaker variation. Even in the 首都圏, there is wide variation, which will only continue to be the case for as long people migrate in and out of the area. As that won't probably ever start, new trends will no doubt have to be written down as the "new rules" perhaps thirty years from now. As a matter of fact, there are many common words in the past century that have had a generational shift in pitch.  
\begin{ltabulary}{|P|P|P|}
\hline 

言葉 & 老年層 & 若年層 \\ \cline{1-3}

赤蜻蛉 &  \textbf{あ }かとんぼ & あ \textbf{かと }んぼ \\ \cline{1-3}

鬼が島 &  \textbf{お }にがしま & お \textbf{にが }しま \\ \cline{1-3}

朝日 & あ \textbf{さ }ひ &  \textbf{あ }さひ \\ \cline{1-3}

若葉 & わ \textbf{か }ば &  \textbf{わ }かば \\ \cline{1-3}

熊 & ク \textbf{マ↓ }&  \textbf{ク }マ \\ \cline{1-3}

神 & か \textbf{み↓ }&  \textbf{か }み \\ \cline{1-3}

寿司 & す \textbf{し }↓ &  \textbf{す }し \\ \cline{1-3}

姉 & あ \textbf{ね }&  \textbf{あ }ね \\ \cline{1-3}

梅雨 & つ \textbf{ゆ }&  \textbf{つ }ゆ \\ \cline{1-3}

鍬 & く \textbf{わ }&  \textbf{く }わ \\ \cline{1-3}

僕 &  \textbf{ぼ }く & ぼ \textbf{く }\\ \cline{1-3}

姪 &  \textbf{め }い & め \textbf{い }\\ \cline{1-3}

\end{ltabulary}
\textbf{ Conjugation }   As we have seen before, what you add to something significantly affects pitch. Are you using a 単純語? A 複合語? Are you using conjugations? What particle are you using? All of this plays a large role in Standard Japanese pitch accent. So far, we have yet to see how conjugations come into play. Again, as was the case with the previous 20 rules, morae count is crucial.   As for the variations of the copula, if the noun\slash 形容動詞 is unaccented, then the resulting phrase is 尾高型. If the noun\slash 形容動詞 is accented, there is no pitch change.  5. 赤です ( \textbf{あ }かです)   城です (し \textbf{ろで }す)  雲だ ( \textbf{く }もだ)  空港ではありません (く \textbf{うこうではありませ }ん)   Adjectives and verb pitch allocation is quite challenging to say the least. As the number of morae and what endings you are using are all important, information here will be based off of http:\slash \slash accent.u-biq.org  , which illustrates not only the pitch allocation of nouns, names, number phrases, etc., but it also demonstrates the pitch accent contours of verbs and adjectives of various morae counts in numerous forms.  
\begin{ltabulary}{|P|P|P|P|P|}
\hline 

拍数 & Type 1 & Type 2 & Type 3 & Type 4 \\ \cline{1-5}

2拍動詞 & LH & LH & HL & HL \\ \cline{1-5}

3拍動詞 & LHH & LHH & LHL & LHL \\ \cline{1-5}

4拍動詞 & LHHH & LHHH & LHHL & LHHL \\ \cline{1-5}

\end{ltabulary}
  Though types may have very similar pitch contours, the class of verb does cause different morae counts with certain conjugations. This accounts for seemingly minor differences. Also keep in mind that there are verbs\slash compound verbs longer than 4 morae. So, these charts can't account for everything. Given that Japanese is agglutinative (endings attach in chains onto inflectional items), there are numerous possibilities. This first table just accounts for the 辞書形.   Notice how pitch differs with the following forms. See any patterns depending on the ending? These patterns can be used with each other when dealing with long verbal phrases. Regardless of these charts, though, if you are really wanting to have better pitch accent, the best things to do just boil down to having a decent pitch accent dictionary and ear to mimic native speakers. In time, you'll see how this information comes out quite naturally. http:\slash \slash accent.u-biq.org  .is a great online source to learn more! 
\begin{ltabulary}{|P|P|P|}
\hline 

頭高型 \hfill\break
& The pitch starts \textbf{high }, drops suddenly, then steadily goes down. & \_ \\ \cline{1-3}

中高型 \hfill\break
& The pitch rises, reaches a maximum, then drops suddenly. \hfill\break
& /\ \\ \cline{1-3}

尾高型 \hfill\break
& The pitch rises then drops when it reaches an attached element. & / ̄(\) \\ \cline{1-3}

平板型 \hfill\break
& The pitch \textbf{rises from start to end }. & / ̄( ̄) \\ \cline{1-3}

\end{ltabulary}

\par{ \textbf{Examples }}

\begin{ltabulary}{|P|P|P|P|}
\hline 

Rule 1 \hfill\break
& Rule 2 \hfill\break
& Rule 3 \hfill\break
& Rule 4 \\ \cline{1-4}

há ɕ ì (chopsticks) & hà ɕ í (bridge) &  \hfill\break
& ha ɕ i (edge) \\ \cline{1-4}

ímà (now) \hfill\break
& ìmá (living room) \hfill\break
&  &  \\ \cline{1-4}

kákì (oyster) \hfill\break
& kàkí (fence) \hfill\break
&  \hfill\break
& kaki (persimmon) \\ \cline{1-4}

níhòN' (two long items) & nìhó N' (Japan) &  &  \\ \cline{1-4}

ámè (rain) &  &  & ame (candy) \\ \cline{1-4}

ásà (morning) & àsá (hemp) &  &  \\ \cline{1-4}

ká ɽ àsu (crow) & tàmágò (egg) \hfill\break
& otoko (man) \hfill\break
& otona (adult) \\ \cline{1-4}

\end{ltabulary}

\begin{ltabulary}{|P|P|P|}
\hline 

頭高型 \hfill\break
& The pitch starts \textbf{high }, drops suddenly, then steadily goes down. & \_ \\ \cline{1-3}

中高型 \hfill\break
& The pitch rises, reaches a maximum, then drops suddenly. \hfill\break
& /\ \\ \cline{1-3}

尾高型 \hfill\break
& The pitch rises then drops when it reaches an attached element. & / ̄(\) \\ \cline{1-3}

平板型 \hfill\break
& The pitch \textbf{rises from start to end }. & / ̄( ̄) \\ \cline{1-3}

\end{ltabulary}

\par{ \textbf{Examples }}

\begin{ltabulary}{|P|P|P|P|}
\hline 

Rule 1 \hfill\break
& Rule 2 \hfill\break
& Rule 3 \hfill\break
& Rule 4 \\ \cline{1-4}

há ɕ ì (chopsticks) & hà ɕ í (bridge) &  \hfill\break
& ha ɕ i (edge) \\ \cline{1-4}

ímà (now) \hfill\break
& ìmá (living room) \hfill\break
&  &  \\ \cline{1-4}

kákì (oyster) \hfill\break
& kàkí (fence) \hfill\break
&  \hfill\break
& kaki (persimmon) \\ \cline{1-4}

níhòN' (two long items) & nìhó N' (Japan) &  &  \\ \cline{1-4}

ámè (rain) &  &  & ame (candy) \\ \cline{1-4}

ásà (morning) & àsá (hemp) &  &  \\ \cline{1-4}

ká ɽ àsu (crow) & tàmágò (egg) \hfill\break
& otoko (man) \hfill\break
& otona (adult) \\ \cline{1-4}

\end{ltabulary}

\begin{ltabulary}{|P|P|P|}
\hline 

頭高型 \hfill\break
& The pitch starts \textbf{high }, drops suddenly, then steadily goes down. & \_ \\ \cline{1-3}

中高型 \hfill\break
& The pitch rises, reaches a maximum, then drops suddenly. \hfill\break
& /\ \\ \cline{1-3}

尾高型 \hfill\break
& The pitch rises then drops when it reaches an attached element. & / ̄(\) \\ \cline{1-3}

平板型 \hfill\break
& The pitch \textbf{rises from start to end }. & / ̄( ̄) \\ \cline{1-3}

\end{ltabulary}

\par{ \textbf{Examples }}

\begin{ltabulary}{|P|P|P|P|}
\hline 

Rule 1 \hfill\break
& Rule 2 \hfill\break
& Rule 3 \hfill\break
& Rule 4 \\ \cline{1-4}

há ɕ ì (chopsticks) & hà ɕ í (bridge) &  \hfill\break
& ha ɕ i (edge) \\ \cline{1-4}

ímà (now) \hfill\break
& ìmá (living room) \hfill\break
&  &  \\ \cline{1-4}

kákì (oyster) \hfill\break
& kàkí (fence) \hfill\break
&  \hfill\break
& kaki (persimmon) \\ \cline{1-4}

níhòN' (two long items) & nìhó N' (Japan) &  &  \\ \cline{1-4}

ámè (rain) &  &  & ame (candy) \\ \cline{1-4}

ásà (morning) & àsá (hemp) &  &  \\ \cline{1-4}

ká ɽ àsu (crow) & tàmágò (egg) \hfill\break
& otoko (man) \hfill\break
& otona (adult) \\ \cline{1-4}

\end{ltabulary}

\begin{ltabulary}{|P|P|P|}
\hline 

頭高型 \hfill\break
& The pitch starts \textbf{high }, drops suddenly, then steadily goes down. & \_ \\ \cline{1-3}

中高型 \hfill\break
& The pitch rises, reaches a maximum, then drops suddenly. \hfill\break
& /\ \\ \cline{1-3}

尾高型 \hfill\break
& The pitch rises then drops when it reaches an attached element. & / ̄(\) \\ \cline{1-3}

平板型 \hfill\break
& The pitch \textbf{rises from start to end }. & / ̄( ̄) \\ \cline{1-3}

\end{ltabulary}

\par{ \textbf{Examples }}

\begin{ltabulary}{|P|P|P|P|}
\hline 

Rule 1 \hfill\break
& Rule 2 \hfill\break
& Rule 3 \hfill\break
& Rule 4 \\ \cline{1-4}

há ɕ ì (chopsticks) & hà ɕ í (bridge) &  \hfill\break
& ha ɕ i (edge) \\ \cline{1-4}

ímà (now) \hfill\break
& ìmá (living room) \hfill\break
&  &  \\ \cline{1-4}

kákì (oyster) \hfill\break
& kàkí (fence) \hfill\break
&  \hfill\break
& kaki (persimmon) \\ \cline{1-4}

níhòN' (two long items) & nìhó N' (Japan) &  &  \\ \cline{1-4}

ámè (rain) &  &  & ame (candy) \\ \cline{1-4}

ásà (morning) & àsá (hemp) &  &  \\ \cline{1-4}

ká ɽ àsu (crow) & tàmágò (egg) \hfill\break
& otoko (man) \hfill\break
& otona (adult) \\ \cline{1-4}

\end{ltabulary}
    