    
\chapter{Dialectical 自他動詞}

\begin{center}
\begin{Large}
第353課: Dialectical 自他動詞 
\end{Large}
\end{center}
 
\par{ There are several intransitive verb forms unique to specific areas of Japan. There are many verbs with several transitive forms. However, as we have seen in previous lessons, they're all used extensively. They may just not mean the same things. The intransitive verb forms we'll see in this lesson are almost exclusively dialectal. The interesting thing about these words is that they fill in lexical gaps hard to explain away through 標準語. Some of them have weird origins, which makes things even more interesting. We'll even learn about a conjugation unique to only one dialect! Let's begin. }
      
\section{Love Variation!}
 
\par{ Because the process of deriving intransitive and transitive verbs is not completely straightforward, there is variation in how you should express transitivity on a case by case basis. Below are some common regional variation that you may encounter. The arrow indicates correct, Standard Japanese. }

\par{1. ガムが ${\overset{\textnormal{くつ}}{\text{靴}}}$ にくっつかっている。 (東北弁)  \textrightarrow  ガムが靴にくっついている。 \hfill\break
Gum is stuck to my shoes. }

\par{\hfill\break
\textbf{Form Note }: Speakers who use くっつかる use it as a more emphatic version of くっつく. This is one of those words which if you use it, you may not even notice it's dialectical until you realize no one else in the country uses it. }

\par{2. ${\overset{\textnormal{かい}}{\text{解}}}$ が ${\overset{\textnormal{もと}}{\text{求}}}$ まる。  \textrightarrow  解が求められる・解が ${\overset{\textnormal{ととの}}{\text{整}}}$ う・解にたどり着く。 \hfill\break
For a solution to be solved. }

\par{\textbf{Form Note }: 求まる has existed but has fallen out of use except by some people in math. When people hear it for the first time, they are often taken back and think it is grammatically incorrect. }

\par{3. 単語が覚わった。 ( ${\overset{\textnormal{なごや}}{\text{名古屋}}}$ ~ ${\overset{\textnormal{ぎふ}}{\text{岐阜}}}$ )  \textrightarrow   覚えきった。 \hfill\break
I got the words down. (memory) }

\par{\textbf{Form Note }: This form should be impossible as this would mean that it ultimately derived from 覚える + ある. 覚える shares 思える, and both can be classified as verbs of spontaneity. Other verbs like this include 見える and 聞こえる. They describe things that happen naturally on their own. But, 覚わる, a form which is supposed to be impossible accidentally came about and is used a lot in some areas of Japan. Another word that came about from a similar mistake is 報われる. The verb for "to reward" was 報ゆ. Rather than using the passive ending normally, which results in the modern 報いられる, the verb was analyzed as being 報う instead. Today, 報う, 報いる, 報われる, and 報いられる exist. The spontaneity verbs ended in ゆ in the past as well, so this may have something to deal with 覚わる coming about. }

\par{4. ${\overset{\textnormal{きんにく}}{\text{筋肉}}}$ が ${\overset{\textnormal{きた}}{\text{鍛}}}$ わる。 (名古屋 Area?) \textrightarrow   筋肉が鍛えられる。 \hfill\break
For one's muscles to be built. }

\par{5. あごが鍛わった気がする。 \hfill\break
I feel like my jaw is hardened up. }

\par{6. ロープが ${\overset{\textnormal{こお}}{\text{凍}}}$ って ${\overset{\textnormal{むす}}{\text{結}}}$ ばらない。 ${\overset{\textnormal{}}{\text{\textrightarrow }}}$ ( ${\overset{\textnormal{ひだべん}}{\text{飛騨弁}}}$ )    ロープが凍って結ぶことができない。 \hfill\break
The rope froze and I can't get a knot in it.  }

\par{ \textbf{The Auxiliary Verb ~さる: Anti-Causative\slash Spontaneous Occurrence } }

\par{ In 北海道弁 and other northern dialects, you may see さる attached to the 未然形 of verbs to create intransitive verb pairs. Examples of this include くっつかさる, 溶かさる = 溶ける, and 終わらさる = 終わる. However, it is not the case that speakers here don't use the regular intransitive forms. So, what exactly is さる used for? First, let's go over its conjugation. }

\begin{ltabulary}{|P|P|P|}
\hline 

一段 Verbs & 未然形 + らさる & 食べらさる \\ \cline{1-3}

五段 Verbs & 未然形+ さる & 読まさる \\ \cline{1-3}

\end{ltabulary}

\par{ This is actual used to create spontaneity phrases just like 見える. Because these words are intransitive too, this is yet another means of 自動詞化. However, this is completely unique to 北海道弁. This conjugation doesn't exist in any other dialect area. The meaning is along the lines of not intending to do something, but conditions proceed in a way that you find yourself naturally in the situation. }

\par{7. この小説は面白くて、どんどん読まさります。 \hfill\break
This book is so interesting that I just end up reading more and more of it. }

\par{ This conjugation is perfect for not implying one's incapability or incompetence when used in the negative ~(ら)さんない. For instance, say your pen doesn't work. If you were to say in Standard Japanese このペンは書けない. It's unclear why you can't use the pen. Is it because the ink is out or because somehow you're too stupid to know how to use the pen? In 北海道弁, you can place all the blame on the natural order of things by saying the following. }

\par{8. このペン書かさんない。 \hfill\break
This pen won't write. }
    