    
\chapter{Idioms IV}

\begin{center}
\begin{Large}
第351課: Idioms IV: 四字熟語 
\end{Large}
\end{center}
 
\par{  四字熟語 literally means "four character compound word". Technically, the term refers to all words made up of four characters. In a narrow-sense, the word refers to four character set expressions. That is what we will be discussing. For academic punctuality, the term "四字成語" is preferred because it means "four character set phrase". However, most people refer to them as "四字熟語". Due to their origin, most can also be called "四字漢語". }
\textbf{ What is an Idiom? }
\par{Japanese people will easily tell you that the following is a set phrase. }

\par{弁慶の泣き所 }

\par{ ${\overset{\textnormal{べんけい}}{\text{弁慶}}}$ means "strong man" and 泣き所 means "place where one cries". Together, the phrase refers to the shin. If you are kicked in the shin, even if you are a strong man, you are going to cry. }
 
\par{Idioms are rated by "idiomaticity"--how idiomatic something is. Every Japanese person knows ${\overset{\textnormal{しめんそか}}{\text{四面楚歌}}}$ (to be surrounded by enemies on all four sides) and ${\overset{\textnormal{はらんばんじょう}}{\text{波瀾万丈}}}$ (stormy and full of drama). Phrases like 大学教育 (college education) are non-idiomatic. The idiomatic ones, though, are important to entrance exams in Japan, and they're often used in showing off to other people. }
 
\par{Again, the term normally refers to such words that deviate from a literal direct translation. Some compounds create a problem. For example, many idioms in Japanese have been introduced through direct translation. }

\begin{ltabulary}{|P|P|P|P|}
\hline 

A 四字熟語 \hfill\break
& 教室崩壊 & きょうしつほうかい & Destruction of the classroom \\ \cline{1-4}

A Generic Idiom \hfill\break
& 氷山の一角 & ひょうざんのいっかく & Tip of the iceberg \\ \cline{1-4}

Idiom of the Body & 鼻の差で & はなのさで & By a nose \\ \cline{1-4}

\end{ltabulary}

\par{To the English speaker, these expressions are no doubt idiomatic. As for the typical 四字熟語, they are not 'deviant' from the original definition. Most 四字熟語 come from Buddhist texts\slash Chinese literature. \hfill\break
\hfill\break
It is to note that non-idiomatic (four character) expressions can over time obtain an idiomatic usage. 一時停止, which means "suspension", can now describe a guy stopping traffic because he won't go. }

\begin{center}
\textbf{The 6 Sources }
\end{center}

\begin{enumerate}

\item From modern society-- 官官接待"bureaucrats entertaining bureaucrats using public funds". 
\item Made in Japan, many are reading with 訓読み--手練手管(てれんてくだ) "art of coaxing". \hfill\break

\item From Chinese literature--臥薪嘗胆(がしんしょうたん) "enduring unspeakable hardships for the sake of vengeance". 
\item Buddhist terminology--四苦八苦 "the four and eight kinds of suffering" \hfill\break

\item The insertion of 之 which has been used in neo-classical and Chinese-style Japanese literature in place of の. These phrases are arguably not 四字熟語 
\item From 訓読, the process of reading 漢文. So, they're Classical Japanese sentences fit to Chinese syntax and pronunciation. These phrases are often considered anachronistic. As such, they tend to not be in dictionaries. 
\end{enumerate}

\par{At times, 四字熟語 originating from Chinese have deviated in meaning. 七転八倒(しちてんばっとう) means "writhing in agony" in Japanese and "very confused" in Chinese. There is also obvious significant phonetic deviation. }
      
\section{Analysis of 64 四字熟語}
  
\par{ You aren't required to memorize them all, but they are great practice. There isn't a particular ordering of the idioms. }

\par{一刀両断(いっとうりょうだん) \hfill\break
One-sword-both (sides)-cut into \hfill\break
1. To cut in two with a single blow \hfill\break
2. To take drastic measures }

\par{色即是空(しきそくぜくう) \hfill\break
Rupa-in other words-there is-vanity \hfill\break
1. All is vanity \hfill\break
Note: 色 (rupa) means all things have shape }

\par{手前味噌(てまえみそ) \hfill\break
My-miso \hfill\break
1. Self-flattery \hfill\break
Note: 手前 is used with its humble first person pronoun definition. }

\par{付和雷同 \hfill\break
ふわらいどう \hfill\break
Attach-peace-thunder-same \hfill\break
1. To follow people blindly. }

\par{暖衣飽食(だんいほうしょく) \hfill\break
Warm-clothes-tired of-eating \hfill\break
1. Well-fed and well-dressed \hfill\break
2. Being blessed materially \hfill\break
Note: May also be written as 煖衣飽食. }

\par{一期一会(いちごいちえ) \hfill\break
One-time-one-meet \hfill\break
1. Once in a lifetime (encounter) \hfill\break
Note: Of Japanese in origin }

\par{美人薄命(びじんはくめい) \hfill\break
Beauty-person-thin-life \hfill\break
1. A beautiful woman is destined to die young \hfill\break
2. Beauty and fortune seldom go together \hfill\break
3. The beautiful die young }

\par{月下氷人(げっかひょうじん) \hfill\break
Moon-under-ice-person \hfill\break
1. Matchmaker; go-between }

\par{唯我独尊(ゆいがどくそん) \hfill\break
Only-self-alone-honor \hfill\break
1. Self-righteousness \hfill\break
2. Holy am I alone }

\par{二股膏薬(ふたまたごうやく) \hfill\break
Two-sides-salve \hfill\break
1. Double-dealer \hfill\break
2. Moving back and forth between two sides in a conflict \hfill\break
Also read as ふたまこうやく, and it is Japanese in origin. }

\par{一石二鳥(いっせきにちょう) \hfill\break
One-stone-two-birds \hfill\break
1. Killing two birds with one stone \hfill\break
Note: Originally an English expression. }

\par{呉越同舟(ごえつどうしゅう) \hfill\break
Go-Etsu-same-boat \hfill\break
1. Bitter enemies in the same boat by fate \hfill\break
Note: From a Chinese story where Go and Etsu were old rivals that toppled the vessel when they had to be together, which in turn made them have to save each other. }

\par{順風満帆(じゅんぷうまんぱん) \hfill\break
Gentle-wind-full-sails \hfill\break
1. Smooth sailing; everything going smoothly }

\par{悪因悪果(あくいんあっか) \hfill\break
Evil-cause-evil-result \hfill\break
1. You reap what you sow \hfill\break
Note: Originally from Buddhist scriptures. }

\par{弱肉強食(じゃくにくきょうしょく) \hfill\break
Weak-meat-strong-eat \hfill\break
1. Survival of the fittest \hfill\break
2. Laws of the Jungle }

\par{寂滅為楽(じゃくめついらく) \hfill\break
Loneliness-destroy-to do-happiness \hfill\break
1. Freedom from one's desires \hfill\break
2. Nirvana is true bliss \hfill\break
Note: Originally from Buddhist scriptures }

\par{色恋沙汰(いろこいざた) \hfill\break
Sensual-love-affair \hfill\break
1. Love affair \hfill\break
2. Romantic entanglement }

\par{酔生夢死(すいせいむし) \hfill\break
Drunken-life-dream-death \hfill\break
1. Idling one's life away \hfill\break
2. Dreaming away one's life accomplishes nothing significant }

\par{十人十色(じゅうにんといろ) \hfill\break
Ten-people-ten-colors \hfill\break
1. To each his own \hfill\break
2. So many people, so many minds \hfill\break
3. Everyone has his own interests and ideas }

\par{朝三暮四(ちょうさんぼし) \hfill\break
Morning-three-evening-four \hfill\break
1. Being preoccupied with immediate superficial differences without realizing that there is no difference in substance \hfill\break
Note: From a story where a monkey feeder would feed a monkey three horse chestnuts in the morning and four in the evening and switched it around because he got mad. }

\par{異体同心 \hfill\break
いたいどうしん \hfill\break
Different-body-different-mind \hfill\break
1. Being of one mind; acting in one accord\slash perfect harmony }

\par{会者定離 \hfill\break
えしゃじょうり \hfill\break
Meeting-person-always-separates \hfill\break
1. Those who meet must part \hfill\break
Note: Originated from Buddhist scriptures }

\par{生者必滅 \hfill\break
しょうじゃひつめつ \hfill\break
Living-people-certainly-die \hfill\break
1. All living things must die \hfill\break
Note: Originated from Buddhist scriptures }

\par{我田引水 \hfill\break
がでんいんすい \hfill\break
Own-field-draw-water \hfill\break
1. Self-seeking \hfill\break
2. Feathering one's own nest \hfill\break
3. Drawing water for one's own field \hfill\break
Originated from the rich agrarian society of old China and Japan. }

\par{五里夢中 \hfill\break
ごりむちゅう \hfill\break
Five-ri (2.44 miles)-fog-inside \hfill\break
1. Totally at a loss; in a fog }

\par{古色蒼然 \hfill\break
こしょくそうぜん \hfill\break
Old-color-blue-is \hfill\break
1. Antique-looking \hfill\break
2. With the patina of age }

\par{羊頭狗肉 \hfill\break
ようとうくにく \hfill\break
Sheep-head-dog-meat \hfill\break
1. Crying wine and selling vinegar; using a better name to sell inferior goods \hfill\break
2. Extravagant advertisement }
 黒風白雨(こくふうはくう) \hfill\break
Black-wind-white-rain \hfill\break
1. Sudden rain shower in a dust storm 
\par{高山流水(こうざんりゅうすい) \hfill\break
Tall-mountain-running-water \hfill\break
1. Beauty of nature \hfill\break
2. Beautifully played music \hfill\break
3. Tall mountains and rushing water }

\par{花鳥風月 \hfill\break
かちょうふうげつ \hfill\break
Flower-bird-wind-moon \hfill\break
1. The beauties of nature }

\par{三々五々(さんさんごご) \hfill\break
Three-three-five-five \hfill\break
1. In small groups }

\par{一望千里(いちぼうせんり) \hfill\break
One-hope-thousand-ri \hfill\break
1. Sweeping expanse }

\par{五風十雨(ごふうじゅうう) \hfill\break
Five-wind-ten-rain \hfill\break
1. Halcyon weather\slash times }

\par{九死一生(きゅうしいっしょう) \hfill\break
Nine-deaths-one-life \hfill\break
1. Narrow escape from death }

\par{五十知命(ごじゅうちめい) \hfill\break
Fifty-wisdom-life \hfill\break
1. At 50 one comes to know the will of Heaven \hfill\break
Note: This is a Confucius saying. }

\par{朝雲暮雨(ちょううんぼう) \hfill\break
Morning-clouds-evening-rain \hfill\break
Sexual liaison \hfill\break
}

\par{三千世界(さんぜんせかい) \hfill\break
Three-thousand-worlds \hfill\break
1. The whole universe }

\par{一文半銭(いちもんはんせん) \hfill\break
One-mon-half-sen \hfill\break
1. Meager sum of money }

\par{一文不通(いちもんふつう) \hfill\break
One-sentence-block \hfill\break
Totally illiterate }

\par{一陽来復(いちようらいふく) \hfill\break
One-sun-return \hfill\break
The return of spring }

\par{一利一害(いちりいちがい) \hfill\break
One-profit-one gain \hfill\break
1. Advantage and its disadvantage }

\par{一挙一動(いっきょいちどう) \hfill\break
One-action-one-move \hfill\break
1. One's every move }

\par{一死報国(いっしほうこく) \hfill\break
One-death-patriotism \hfill\break
1. Dying for one's country }

\par{浅学非才(せんがくひさい) \hfill\break
Shallow-learning-no-ability \hfill\break
1. Lack of learning and ability }

\par{円満具足(えんまんぐそく) \hfill\break
Satisfactory-completeness \hfill\break
1. Complete, tranquil, and in harmony }

\par{再三再四(さいさんさいし) \hfill\break
Re-three-re-four \hfill\break
1. Repeatedly }

\par{人心一新(じんしんいっしん) \hfill\break
Person-heart-one-new \hfill\break
1. Complete change in public sentiment }

\par{一心同体(いっしんどうたい) \hfill\break
One-heart-same-body \hfill\break
1. One in body and soul }

\par{一路平安(いちろへいあん) \hfill\break
One-road-security \hfill\break
Bon voyage }

\par{一路順風(いちろじゅんぷう) \hfill\break
One-road-gentle-wind \hfill\break
1. Everything going well }

\par{心機一転(しんきいってん) \hfill\break
Heart-move-turn \hfill\break
Change in attitude }

\par{意気投合(いきとうごう) \hfill\break
Spirit-congeniality \hfill\break
1. Find a kindred spirit }

\par{才子多病(さいしたびょう) \hfill\break
Genius-child-many-diseases \hfill\break
1. Talented people tend to be delicate \hfill\break
2. Men of genius tend to be of delicate health }

\par{九牛一毛(きゅうぎゅういちもう) \hfill\break
Nine-cows-one-hair \hfill\break
1. A drop in the ocean }

\par{一衣帯水(いちいたいすい) \hfill\break
One-clothes-obi-water \hfill\break
1. Being separated by a narrow stretch of water }

\par{清風明月(せいふうめいげつ) \hfill\break
Pure-wind-bright-moon \hfill\break
1. Refreshing breeze and the bright moon }

\par{純一無雑(じゅんいちむざつ) \hfill\break
Pure-one-no-disarray \hfill\break
1. Pure in heart }

\par{下意上達(かいじょうたつ) \hfill\break
Lower idea-advance \hfill\break
1. Conveyance of lower class opinions to the powers that be }

\par{急転直下(きゅうてんちょっか) \hfill\break
Sudden-change-directly-under \hfill\break
1. Suddenly and precipitately }

\par{悠々自適(ゆうゆうじてき) \hfill\break
Leisure-leisure-self-capable \hfill\break
1. Otium cum dignitate \hfill\break
2. Living a life of leisure and dignity }

\par{海千山千(うみせんやません) \hfill\break
Ocean-thousand-mountain-thousand \hfill\break
1. A sly old dog of much worldly wisdom \hfill\break
Note: Japanese in origin }

\par{乾坤一擲(けんこんいってき) \hfill\break
Universe-one-off away \hfill\break
1. Play for all or nothing }

\par{厚顔無恥(こうがんむち) \hfill\break
Heavy-face-no-shame \hfill\break
1. Brazen and unscrupulous; shameless }

\par{傍目八目 \hfill\break
おかめはちもく \hfill\break
Side-eye-eight-eyes \hfill\break
1. A bystander's vantage point \hfill\break
2. Onlookers of a game see more than the actual players do }
      
\section{Additional 四字熟語}
  
\begin{ltabulary}{|P|P|P|}
\hline 

一分一厘 & いちぶいちりん & Not even a bit \\ \cline{1-3}

一網打尽 & いちもうだじん & Wholesale arrest \\ \cline{1-3}

暗雲低迷 & あんうんていめい & Dark clouds hanging low \\ \cline{1-3}

曖昧模糊 & あいまいもこ & Ambiguous \\ \cline{1-3}

悪戦苦闘 & あくせんくとう & Hard struggle \\ \cline{1-3}

一瞬絶句 & いっしゅんぜっく & Rendered speechless for a moment \\ \cline{1-3}

一生懸命 & いっしょうけんめい & With all one's might \\ \cline{1-3}

一心不乱 & いっしんふらん & With heart and soul \\ \cline{1-3}

一体全体 & いったいぜんたい & What on earth? \\ \cline{1-3}

迂闊千万 & うかつせんばん & Very careless \\ \cline{1-3}

宇宙開闢 & うちゅうかいびゃく & Since the dawn of time \\ \cline{1-3}

容貌端正 & ようぼうたんせい & To have handsome and clean-cut features \\ \cline{1-3}

有史以来 & ゆうしいらい & Since the dawn of history \\ \cline{1-3}

和洋折衷 & わようせっちゅう & A blending of Japanese and Western styles \\ \cline{1-3}

和光同塵 & わこうどうじん & Wise men softening their light while with the mundane world \hfill\break
\\ \cline{1-3}

老少不定 & ろうしょうふじょう & Death comes to old and young alike \\ \cline{1-3}

劣弱意識 & れつじゃくいしき & Inferior complex \\ \cline{1-3}

縷々綿々 & るるめんめん & To go on and on in tedious detail \\ \cline{1-3}

流転生死 & るてんしょうじ & The cycle of transmigration \\ \cline{1-3}

風声鶴唳 & ふうせいかくれい & To be frightened by the slightest noise \\ \cline{1-3}

風林火山 & ふうりんかざん & Swift as wind, quiet as forest, fierce as fire, and immovable as mountains \hfill\break
\\ \cline{1-3}

武運長久 & ぶうんちょうきゅう & Continued luck in the fortunes of war \\ \cline{1-3}

複雑多岐 & ふくざつたき & Labyrinthine \\ \cline{1-3}

表裏一体 & ひょうりいったい & The two sides of the same coin \\ \cline{1-3}

比翼連理 & ひよくれんり & Perfect conjugal harmony between husband and wife \\ \cline{1-3}

百人百様 & ひゃくにんひゃくよう & So many men, so many ways \\ \cline{1-3}

眉目秀麗 & びもくしゅうれい & Having a handsome face \\ \cline{1-3}

飛耳長目 & ひじちょうもく & Being well-versed on something \\ \cline{1-3}

煩悶懊悩 & はんもんおうのう & Anguish \\ \cline{1-3}

万緑一紅 & ばんりょくいっこう & One red flower standing out in a see of green vegetation \\ \cline{1-3}

万事承知 & ばんじしょうち & Being well aware of \\ \cline{1-3}

八方美人 & はっぽうびじん & A woman who looks beautiful from all angles \\ \cline{1-3}

拈華微笑 & ねんげみしょう & Heart-to-heart communication \\ \cline{1-3}

如是我聞 & にょぜがもん & Thus I hear \\ \cline{1-3}

如露如電 & にょろにょでん & Existence is as incorporeal as the morning dew or flash of lightning \hfill\break
\\ \cline{1-3}

南無八幡 & なむはちまん & I beseech your aid against my enemy! \\ \cline{1-3}

内剛外柔 & ないごうがいじゅう & Being gentle on the outside but tough on the inside \\ \cline{1-3}

土崩瓦解 & どほうがかい & Complete collapse \\ \cline{1-3}

道聴塗説 & どうちょうとせつ & Shallow-minded mouthing of secondhand information \\ \cline{1-3}

闘志満々 & とうしまんまん & Brimming with fighting spirit \\ \cline{1-3}

東西古今 & とうざいここん & All times and places \\ \cline{1-3}

桃紅柳緑 & とうこうりゅうりょく & The beautiful scenery of spring \\ \cline{1-3}

同工異曲 \hfill\break
& どうこういきょく & Different in appearance but the same in content \\ \cline{1-3}

天網恢恢 & てんもうかいかい & Heaven's vengeance is slow but sure \\ \cline{1-3}

恬淡虚無 & てんたんきょむ & Rising above the trivia of life and remaining calm and selfless \hfill\break
\\ \cline{1-3}

天真流露 & てんしんりゅうろ & Manifestation of one's natural sincerity \\ \cline{1-3}

天地晦冥 & てんちかいめい & The world is covered in darkness \\ \cline{1-3}

天空海闊 & てんくうかいかつ & The open sky and serene sea \\ \cline{1-3}

天下泰平 & てんかたいへい & Peaceful and tranquil \\ \cline{1-3}

痛快淋漓 & つうかいりんり & To be extremely delightful \\ \cline{1-3}

痴話喧嘩 & ちわげんか & Lover's quarrel \\ \cline{1-3}

跳梁跋扈 & ちょうりょうばっこ & Evildoers being rampant and roaming at will \\ \cline{1-3}

寵愛一身 & ちょうあいっしん & To stand highest in one's master's favor \\ \cline{1-3}

忠言逆耳 & ちゅうげんぎゃくじ & Good advice is harsh to the ear \\ \cline{1-3}

躊躇逡巡 & ちゅうちょしゅんじゅん & Hesitation and vacillation \hfill\break
\\ \cline{1-3}

魑魅魍魎 & ちみもうりょう & All sorts of weird creatures \\ \cline{1-3}

着眼大局 & ちゃくがんたいきょく & To take a broad view of things \\ \cline{1-3}

智慧不惑 & ちえふわく & A wise man never wavers \\ \cline{1-3}

知行合一 & ちこうごういつ & The doctrine of the unity of knowledge and action \\ \cline{1-3}

単刀直入 & たんとうちょくにゅう & Going right to the point \\ \cline{1-3}

男尊女卑 & だんそんじょひ & Male domination of women \\ \cline{1-3}

多岐亡羊 & たきぼうよう \hfill\break
& Truth is as hard to find as a sheep lost in a vast plain \hfill\break
\\ \cline{1-3}

大漁貧乏 & たいりょうびんぼう & Impoverishment of fisherman due to a bumper catch \\ \cline{1-3}

大胆巧妙 & だいたんこうみょう & Bold and clever \\ \cline{1-3}

大山鳴動 & たいざんめいどう & A big fuss over nothing \\ \cline{1-3}

大器晩成 & たいきばんせい & Great talents mature late \\ \cline{1-3}

則天去私 & そくてんきょし & Following heaven and abandoning self \\ \cline{1-3}

造反有理 & ぞうはんゆうり & To rebel is justified \\ \cline{1-3}

先憂後楽 & せんゆうこうらく & Hardship now, pleasure later \\ \cline{1-3}

戦々恐々 & せんせんきょうきょう & To be filled with trepidation \\ \cline{1-3}

絶体絶命 & ぜったいぜつめい & Desperate situation \\ \cline{1-3}

切磋琢磨 & せっさたくま & Cultivating one's mind by studying hard \\ \cline{1-3}

世上万般 & せじょうばんぱん & Everything in this world \\ \cline{1-3}

青天白日 & せいてんはくじつ & To be cleared of all charges \\ \cline{1-3}

正邪善悪 & せいじゃぜんあく & Right and wrong \\ \cline{1-3}

政権亡者 & せいけんもうじゃ & One who is obsessed with political power \\ \cline{1-3}

酔眼朦朧 & すいがんもうろう & With drunken eyes \\ \cline{1-3}

垂涎三尺 & すいぜんさんじゃく & Avid desire \\ \cline{1-3}

衰退一途 & すいたいいっと & Being on the wane \\ \cline{1-3}

頭寒足熱 & ずかんそくねつ & Keeping the head cool and the feet warm \\ \cline{1-3}

人馬一体 & じんばいったい & Unity of rider and horse \\ \cline{1-3}

心頭滅却 & しんとうめっきゃく & Clearing one's mind of all mundane thoughts \\ \cline{1-3}

神色自若 & しんしょくじじゃく & Perfect composure \\ \cline{1-3}

心身一如 & しんしんいちにょ & Body and mind as one \\ \cline{1-3}

辛酸甘苦 & しんさんかんく & Hardships and joy \\ \cline{1-3}

支離滅裂 & しりめつれつ & Incoherent \\ \cline{1-3}

諸行無常 & しょぎょうむじょう & All worldly things are transitory \\ \cline{1-3}

四面楚歌 & しめんそか & To be surrounded by enemies on all sides \\ \cline{1-3}

生者必滅 & しょうじゃひつめつ & All living things must die \\ \cline{1-3}

純真無垢 & じゅんしんむく & Pure and innocent \\ \cline{1-3}

秋風落莫 & しゅうふうらくばく & Forlorn and helpless \\ \cline{1-3}

遮二無二 & しゃにむに & Recklessly \\ \cline{1-3}

詩人墨客 & しじんぼっかく & Poets and artists \\ \cline{1-3}

残念無念 & ざんねんむねん & What a pity! \\ \cline{1-3}

昨非今是 \hfill\break
& さくひこんぜ \hfill\break
& Complete reversal of values \hfill\break
\\ \cline{1-3}

\end{ltabulary}
     