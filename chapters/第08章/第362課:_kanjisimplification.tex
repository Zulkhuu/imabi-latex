    
\chapter{旧字体 \& 新字体}

\begin{center}
\begin{Large}
第362課: 旧字体 \& 新字体 
\end{Large}
\end{center}
 
\par{ In 1945 a lot of the most  commonly used complex characters were simplified. The now  defunct characters are called 旧字体. Ironically, the term has two  simplified characters in it. So, the term should be written as 舊字體 to follow old orthography. 新字体, of course, are the resultant simplified forms. }
      
\section{旧字体 and 新字体}
 
\par{ Simplification took a more passive role in Japan as only a handful of methods were used to simplify a handful of 漢字. In the following decade after reform, typesets were still not entirely equipped to comply with simplification. So, it is very easy to find publications from the 1950s with 旧字体. 旧字体, because of their lasting cultural impression, similar to 変体仮名, were often referred to as 正字(体), meaning "correct characters". Many felt that 新字体 deserved to keep their 略字 status. In modern usage, 旧字体 can still be seen in personal and geographical names. It is still important to learn 旧字体 because they have been used throughout Japanese literary history. }
      
\section{Making 新字体 a Reality}
 
\par{  The major methods used to create 新字体 forms are: }

\begin{itemize}

\item Unifying variants 
\item Phonetic replacement 
\item Part removal 
\item Cursive script 
\item Radical Change 
\end{itemize}
\textbf{Unifying Variants }\hfill\break

\par{At times there were variants of a character. For example, 島 could be seen as 嶋, 嶌 and 島 before reform. The radical in 近 has one dot but originally had two. For characters that are not in the 常用漢字 List, it is still common to see two dots. Other differences between the written and printed form were resolved in favor of the written form in many other characters such as the 八 change to 丷 in Kanji such as 判. }

\par{\textbf{Phonetic Replacement }}

\par{A lot of characters classified under the 六書 as 形成文字(けいせいもじ) had their phonetic replaced with another character with the same phonetic without regards to the possible change in meaning in some cases. This allowed many characters to become much simpler. The process continues on in colloquial 略字 handwriting with カタカナ. }

\begin{ltabulary}{|P|P|P|}
\hline 

旧字体 & 新字体 & ON Reading Used for Simplification \hfill\break
\\ \cline{1-3}

圍 & 囲 & イ \\ \cline{1-3}

廰 & 庁 & チョウ \\ \cline{1-3}

膽 & 胆 & タン \\ \cline{1-3}

\end{ltabulary}

\par{\textbf{Part Removal } }

\par{Characters often got parts taken out during simplification. Duplicate parts were either reduced completely or considerably simplified to cut down stroke count. Some 旧字体 that were simplified this way became the same as an already existing character. The instances that this is so is far from many, but some like 藝 to 芸 caused problems. The latter character was a character with an 音読み of ウン and is in the name of Japan's first public library, 芸亭(うんてい). }

\begin{ltabulary}{|P|P|}
\hline 

旧字体 & 新字体 \\ \cline{1-2}

 蟲 & 虫 \\ \cline{1-2}

壘 & 塁 \\ \cline{1-2}

縣 & 県 \\ \cline{1-2}

\end{ltabulary}
 \textbf{Cursive Script } 
\par{漢字 have massive appearance change and stroke reduction when written in cursive\slash grass script. The resultant script, at times, gave shape to a more solidified and simple 略字 that were made proper in reform. }

\begin{ltabulary}{|P|P|P|P|}
\hline 

旧字体 & 新字体 & 旧字体 & 新字体 \\ \cline{1-4}

圖 & 図 & 覺 & 覚 \\ \cline{1-4}

晝 & 昼 & 樂 & 楽 \\ \cline{1-4}

學 & 学 & 靑 & 青 \\ \cline{1-4}

\end{ltabulary}

\par{ \textbf{Radical Change }\hfill\break
}

\par{Radical change happened primarily from the use of cursive script. With such a simple change like this, many characters received a reduction in stroke number of as much as 2 or 3 strokes. An example of such radical change would be the change of 神 to 神 where the radical on the left changed to fit its cursive form. }

\par{\textbf{新字体 Extension \hfill\break
}}

\par{Simplification was only applied to standardized 漢字, but it is common to see patterns used in 新字体 affect non-常用漢字i. 略字 are simplifications that are considered improper but are not uncommon. }

\begin{ltabulary}{|P|P|P|P|}
\hline 

旧字体 & 新字体 & 旧字体 & 新字体 \\ \cline{1-4}

潑 & 溌 & 曾 & 曽 \\ \cline{1-4}

 欅 &  﨔 & 鬱 & 欝 \\ \cline{1-4}

 鷗  & 鴎 & 頰 & 頬 \\ \cline{1-4}

\end{ltabulary}
      
\section{List of 旧字体}
 
\begin{ltabulary}{|P|P|P|P|}
\hline 

 旧字体 & 新字体 & 旧字体 & 新字体 \\ \cline{1-4}

萬 & 万 & 與 & 与 \\ \cline{1-4}

兩 & 両 & 竝 & 並 \\ \cline{1-4}

乘 & 乗 & 亂 & 乱 \\ \cline{1-4}

豫 & 予 & 爭 & 争 \\ \cline{1-4}

亞 & 亜 & 佛 & 仏 \\ \cline{1-4}

假 & 仮 & 會 & 会 \\ \cline{1-4}

傳 & 伝 &  伴 & 伴 \\ \cline{1-4}

體 & 体 & 餘 & 余 \\ \cline{1-4}

倂 & 併 & 價 & 価 \\ \cline{1-4}

侮 & 侮 & 僞 & 偽 \\ \cline{1-4}

儉 & 倹 & 僧 & 僧 \\ \cline{1-4}

免 & 免 & 兒 & 児 \\ \cline{1-4}

黨 & 党 &  具 & 具 \\ \cline{1-4}

內 & 内 & 圓 & 円 \\ \cline{1-4}

\emph{}册 \hfill\break
& 冊 & 寫 & 写 \\ \cline{1-4}

Slashes look like ン \hfill\break
& 冬 & 處 & 処 \\ \cline{1-4}

兇 & 凶 &  判 & 判 \\ \cline{1-4}

Strokes are slashes in 月 \hfill\break
& 前 & 劍 & 剣 \\ \cline{1-4}

劑 & 剤 & 剩 & 剰 \\ \cline{1-4}

勵 & 励 & 勞 & 労 \\ \cline{1-4}

效 & 効 & 敕 & 勅 \\ \cline{1-4}

 勉 & 勉 &  勝 & 勝 \\ \cline{1-4}

 勤 & 勤 & 勸 & 勧 \\ \cline{1-4}

勳 & 勲 &  包 & 包 \\ \cline{1-4}

區 & 区 & 醫 & 医 \\ \cline{1-4}

 半 & 半 &  卑 & 卑 \\ \cline{1-4}

單 & 単 & 卽 & 即 \\ \cline{1-4}

釐 & 厘 & 嚴 & 厳 \\ \cline{1-4}

參 & 参 &  又 & 又 \\ \cline{1-4}

雙 & 双 & 收 & 収 \\ \cline{1-4}

敍 & 叙 & 臺 & 台 \\ \cline{1-4}

號 & 号 & 吳 & 呉 \\ \cline{1-4}

土 changed from stake with two lines & 周 &  咊、 龢 & 和 \\ \cline{1-4}

八 changed to 丷 & 咲 &  啓 & 啓 \\ \cline{1-4}

 喝 & 喝 & 營 & 営 \\ \cline{1-4}

 嘆 & 嘆 & 囑 & 嘱 \\ \cline{1-4}

 器 、噐 & 器 & 團 & 団 \\ \cline{1-4}

圍 & 囲 & 圖 & 図 \\ \cline{1-4}

國 & 国 & 圈 & 圏 \\ \cline{1-4}

壓 & 圧 &  坪 & 坪 \\ \cline{1-4}

墮 & 堕 &  塀 & 塀 \\ \cline{1-4}

壘 & 塁 & 塚 & 塚 \\ \cline{1-4}

鹽、盬 & 塩 & 增 & 増 \\ \cline{1-4}

 墨 & 墨 & 壞 & 壊 \\ \cline{1-4}

壤 & 壌 & 壯 & 壮 \\ \cline{1-4}

聲 & 声 & 壹 & 壱 \\ \cline{1-4}

賣 & 売 & 變 & 変 \\ \cline{1-4}

奧 & 奥 & 奬 & 奨 \\ \cline{1-4}

姬 & 姫 &  娛 & 娯 \\ \cline{1-4}

孃 & 嬢 & 學 & 学 \\ \cline{1-4}

寶 & 宝 & 實 & 実 \\ \cline{1-4}

 寒 & 寒 & 寬 & 寛 \\ \cline{1-4}

寢 & 寝 & 對 & 対 \\ \cline{1-4}

壽 & 寿 & 專 & 専 \\ \cline{1-4}

將 & 将 &  尊 & 尊 \\ \cline{1-4}

 尙 & 尚 & 盡 & 尽 \\ \cline{1-4}

屆 & 届 & 屬 & 属 \\ \cline{1-4}

層 & 層 & 嶽 & 岳 \\ \cline{1-4}

峽 & 峡 & 峯 & 峰 \\ \cline{1-4}

嶌、嶋 & 島 & Strokes are slashes in 月 \hfill\break
& 崩 \\ \cline{1-4}

巢 & 巣 & 卷 & 巻 \\ \cline{1-4}

帶 & 帯 & 歸 & 帰 \\ \cline{1-4}

幤 & 幣 &  平 & 平 \\ \cline{1-4}

廳 & 庁 & 廣 & 広 \\ \cline{1-4}

廢 & 廃 &  廉 & 廉 \\ \cline{1-4}

廊 & 廊 & 獘 & 弊 \\ \cline{1-4}

辨、瓣、辯 & 弁 & 貳 & 弐 \\ \cline{1-4}

 弱 & 弱 &  强 & 強 \\ \cline{1-4}

彈 & 弾 & 當 & 当 \\ \cline{1-4}

徑 & 径 & 從 & 従 \\ \cline{1-4}

德 & 徳 & 徵 & 徴 \\ \cline{1-4}

應 & 応 & 戀 & 恋 \\ \cline{1-4}

恆 & 恒 & 惠 & 恵 \\ \cline{1-4}

悔 & 悔 & 惱 & 悩 \\ \cline{1-4}

惡 & 悪 &  情 & 情 \\ \cline{1-4}

慘 & 惨 &  慈 & 慈 \\ \cline{1-4}

愼 & 慎 & 慨 & 慨 \\ \cline{1-4}

憎 & 憎 & 懷 & 懐 \\ \cline{1-4}

懲 & 懲 & 戰 & 戦 \\ \cline{1-4}

戲 & 戯 & 戾 & 戻 \\ \cline{1-4}

拂 & 払 & 拔 & 抜 \\ \cline{1-4}

擇 & 択 & 擔 & 担 \\ \cline{1-4}

 拐 & 拐 & 拜 & 拝 \\ \cline{1-4}

據 & 拠 & 擴 & 拡 \\ \cline{1-4}

擧 & 挙 & 挾 & 挟 \\ \cline{1-4}

插 & 挿 & 搜 & 捜 \\ \cline{1-4}

揭 & 掲 & 搖 & 揺 \\ \cline{1-4}

攜、擕 & 携 & 攝 & 摂 \\ \cline{1-4}

擊 & 撃 &  敏 & 敏 \\ \cline{1-4}

敎 & 教 & 數 & 数 \\ \cline{1-4}

 敷 & 敷 & 齊 & 斉 \\ \cline{1-4}

齋 & 斎 & 斷 & 断 \\ \cline{1-4}

 旣 & 既 & 舊 & 旧 \\ \cline{1-4}

 曐 & 星 & 晝 & 昼 \\ \cline{1-4}

晚 & 晩 & 晴 & 晴 \\ \cline{1-4}

曉 & 暁 &  暑 & 暑 \\ \cline{1-4}

曆 & 暦 & 朗 & 朗 \\ \cline{1-4}

 朕 & 朕 & Strokes are slashes in 月 \hfill\break
& 朝 \\ \cline{1-4}

條 & 条 & 來 & 来 \\ \cline{1-4}

樞 & 枢 & 榮 & 栄 \\ \cline{1-4}

櫻 & 桜 & 棧 & 桟 \\ \cline{1-4}

 梅 & 梅 & Strokes are slashes in 月 \hfill\break
& 棚 \\ \cline{1-4}

檢 \hfill\break
& 検 & 樓 & 楼 \\ \cline{1-4}

樂 & 楽 & 槪 & 概 \\ \cline{1-4}

樣 & 様 & 權 & 権 \\ \cline{1-4}

橫 & 横 & 欄 & 欄 \\ \cline{1-4}

缺 & 欠 & 歐 & 欧 \\ \cline{1-4}

士 changed to 木 & 款 & 歡 & 歓 \\ \cline{1-4}

步 & 歩 & 齒 & 歯 \\ \cline{1-4}

歷 & 歴 & 殘 & 残 \\ \cline{1-4}

毆 & 殴 & 殺 & 殺 \\ \cline{1-4}

殼 & 殻 & 每 & 毎 \\ \cline{1-4}

氣 & 気 & 冰 & 氷 \\ \cline{1-4}

沒 & 没 & 澤 & 沢 \\ \cline{1-4}

淨 & 浄 & 淺 & 浅 \\ \cline{1-4}

濱 & 浜 & 海 & 海 \\ \cline{1-4}

淚 & 涙 & 淸 & 清 \\ \cline{1-4}

渴 & 渇 & 濟 & 済 \\ \cline{1-4}

涉 & 渉 & 澁 & 渋 \\ \cline{1-4}

溪 & 渓 & 氵+廴+ 咼 & 渦 \\ \cline{1-4}

溫 & 温 &  港 & 港 \\ \cline{1-4}

灣 & 湾 \hfill\break
& 濕 & 湿 \\ \cline{1-4}

滿 & 満 & 玄 were separate & 滋 \\ \cline{1-4}

瀧 & 滝 & 滯 & 滞 \\ \cline{1-4}

桼 & 漆 &  漢 & 漢 \\ \cline{1-4}

 潔 & 潔 & 潛 & 潜 \\ \cline{1-4}

瀨 & 瀬 & 燈 & 灯 \\ \cline{1-4}

 灰 & 灰 & 爐 & 炉 \\ \cline{1-4}

點 & 点 & 爲 & 為 \\ \cline{1-4}

燒 & 焼 &  煮 & 煮 \\ \cline{1-4}

犧 & 犠 & 狀 & 状 \\ \cline{1-4}

獨 & 独 & 狹 & 狭 \\ \cline{1-4}

獵 & 猟 & 獻 & 献 \\ \cline{1-4}

獸 & 獣 & 甁 & 瓶 \\ \cline{1-4}

畫 & 画 & 畍 \hfill\break
& 界 \\ \cline{1-4}

 畔 & 畔 & 畱、畄 & 留 \\ \cline{1-4}

疊 & 畳 & 疏 & 疎 \\ \cline{1-4}

癡 & 痴 & 瘉 & 癒 \\ \cline{1-4}

發 & 発 & 益 & 益 \\ \cline{1-4}

盜 & 盗 & 譼 & 監 \\ \cline{1-4}

縣 & 県 & 眞 & 真 \\ \cline{1-4}

硏 & 研 & 碎 & 砕 \\ \cline{1-4}

 碑 & 碑 & 禮 & 礼 \\ \cline{1-4}

玄 were separate & 磁 &  社 & 社 \\ \cline{1-4}

 祈 & 祈 &  祉 & 祉 \\ \cline{1-4}

 祖 & 祖 &  祝 & 祝 \\ \cline{1-4}

神 & 神 & 祥 & 祥 \\ \cline{1-4}

禪 & 禅 &  禍 & 禍 \\ \cline{1-4}

福 & 福 & 祕 & 秘 \\ \cline{1-4}

稱 & 称 & 稻 & 稲 \\ \cline{1-4}

 穀 & 穀 & 穗 & 穂 \\ \cline{1-4}

穩 & 穏 &  突 & 突 \\ \cline{1-4}

竊 & 窃 & 窗、囱、牎、牕、窻 & 窓 \\ \cline{1-4}

竆 & 窮 & 龍 & 竜 \\ \cline{1-4}

竸 & 競 &  節 & 節 \\ \cline{1-4}

粹 & 粋 & 肅 & 粛 \\ \cline{1-4}

精 & 精 & 絲 & 糸 \\ \cline{1-4}

 納 & 納 &  終 & 終 \\ \cline{1-4}

經 & 経 & 繪 & 絵 \\ \cline{1-4}

 絕 & 絶 & 繼 & 継 \\ \cline{1-4}

續 & 続 & 緜 \hfill\break
& 綿 \\ \cline{1-4}

總 & 総 & 綠 & 緑 \\ \cline{1-4}

緖 & 緒 &  練 & 練 \\ \cline{1-4}

緣 & 縁 & 繩 & 縄 \\ \cline{1-4}

縱 & 縦 & 繁 & 繁 \\ \cline{1-4}

纖 & 繊 & 罐 & 缶 \\ \cline{1-4}

 署 & 署 & 飜 & 翻 \\ \cline{1-4}

 者 & 者 & First slash has no line above it \hfill\break
& 耕 \\ \cline{1-4}

秏 & 耗 &  聖 & 聖 \\ \cline{1-4}

聽 & 聴 & 膽 & 胆 \\ \cline{1-4}

 脫 & 脱 & 腦 & 脳 \\ \cline{1-4}

腕 & 捥 & 臟 & 臓 \\ \cline{1-4}

臭 & 臭 & 舍 & 舎 \\ \cline{1-4}

舖 & 舗 & 藝 \hfill\break
& 芸 \\ \cline{1-4}

莖 & 茎 & 莊 & 荘 \\ \cline{1-4}

 著 & 著 & 藏 & 蔵 \\ \cline{1-4}

薰 & 薫 & 藥 & 薬 \\ \cline{1-4}

虛 & 虚 & 虜 & 虜 \\ \cline{1-4}

 虞 & 虞 & 蟲 & 虫 \\ \cline{1-4}

蠶 & 蚕 & 螢 & 蛍 \\ \cline{1-4}

蠻 & 蛮 & 眾 & 衆 \\ \cline{1-4}

衞 & 衛 & 裝 & 装 \\ \cline{1-4}

裡 & 裏 &  褐 & 褐 \\ \cline{1-4}

襃 & 褒 & 霸 & 覇 \\ \cline{1-4}

 視 & 視 & 覺 & 覚 \\ \cline{1-4}

覽 & 覧 & 觀 & 観 \\ \cline{1-4}

觸 & 触 & 譯 & 訳 \\ \cline{1-4}

證 & 証 &  評 & 評 \\ \cline{1-4}

譽 & 誉 &  誤 & 誤 \\ \cline{1-4}

 說 & 説 & 読 & 讀 \\ \cline{1-4}

 請 & 請 & 諸 & 諸 \\ \cline{1-4}

 謁 & 謁 & 謠 & 謡 \\ \cline{1-4}

 謹 & 謹 & 讓 & 譲 \\ \cline{1-4}

豐 & 豊 &  賓 & 賓 \\ \cline{1-4}

贊 & 賛 &  贈 & 贈 \\ \cline{1-4}

踐 & 践 & 轉 & 転 \\ \cline{1-4}

輕 & 軽 & 辭 & 辞 \\ \cline{1-4}

邊 & 辺 &  送 & 送 \\ \cline{1-4}

遞 & 逓 &  逸 & 逸 \\ \cline{1-4}

遲 & 遅 & 郞 & 郎 \\ \cline{1-4}

鄕 & 郷 & 都 & 都 \\ \cline{1-4}

醉 & 酔 & 釀 & 醸 \\ \cline{1-4}

釋 & 釈 & 鐵 & 鉄 \\ \cline{1-4}

鑛 & 鉱 & 錢 & 銭 \\ \cline{1-4}

鑄 & 鋳 & 鍊 & 錬 \\ \cline{1-4}

錄 & 録 & 鎭 & 鎮 \\ \cline{1-4}

兏、镸 & 長 & 關 & 関 \\ \cline{1-4}

鬪 & 闘 & 陷 & 陥 \hfill\break
\\ \cline{1-4}

險 & 険 & 隆 & 隆 \\ \cline{1-4}

隨 & 随 & 隱 & 隠 \\ \cline{1-4}

隸 & 隷 & 雜 & 雑 \\ \cline{1-4}

 難 & 難 &  雪 & 雪 \\ \cline{1-4}

靈 & 霊 & 靜 & 静 \\ \cline{1-4}

 響 & 響 &  頻 & 頻 \\ \cline{1-4}

賴 & 頼 & 顏 & 顔 \\ \cline{1-4}

顯 & 顕 & 類 & 類 \\ \cline{1-4}

飮 \hfill\break
& 飲 & 馱 & 駄 \\ \cline{1-4}

驛 & 駅 & 驅 & 駆 \\ \cline{1-4}

騷 & 騒 & 驗 & 験 \\ \cline{1-4}

髓 & 髄 & 髙 & 高 \\ \cline{1-4}

髮 & 髪 & 鷄 & 鶏 \\ \cline{1-4}

麥 & 麦 & 黃 & 黄 \\ \cline{1-4}

黑 & 黒 & 默 & 黙 \\ \cline{1-4}

 鼻 & 鼻 & 齢 & 齡 \\ \cline{1-4}

龜 \hfill\break
& 亀 &  剝 & 剥 \\ \cline{1-4}

齅 & 嗅 &  嘲 & 嘲 \\ \cline{1-4}

塡 & 填 & 彌 & 弥 \\ \cline{1-4}

捗 & 捗 & 曾 & 曽 \\ \cline{1-4}

 溺 & 溺 &  煎 & 煎 \\ \cline{1-4}

瘦 & 痩 &  箸 & 箸 \\ \cline{1-4}

艷 & 艶 &  葛 & 葛 \\ \cline{1-4}

 蔽 & 蔽 &  賭 & 賭 \\ \cline{1-4}

 遡 & 遡 & 頰 & 頬 \\ \cline{1-4}

餠 & 餅 & 麵 & 麺 \\ \cline{1-4}

\end{ltabulary}

\par{\textbf{Variant Notes }: Radical changes that affected all characters such as 餠 to 餅 in respect to the radical on the left, two dots to one for characters such as 近 have been omitted from the chart. When a character is not supported by UNICODE, a description is given instead. }
    