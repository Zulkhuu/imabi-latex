    
\chapter{Phonology IV}

\begin{center}
\begin{Large}
第370課: Phonology IV: 露出形 \& 被覆形 
\end{Large}
\end{center}
 
\par{ Along with 連濁, you have probably also noticed that the vowel in a reading changes when used in a compound. For instance, あめ becomes あま in あまぐも. This lesson will try to shed led on this without having to go so deep into the history of Japanese. }
      
\section{露出形 \& 被覆形}
 
\par{ Last lesson the word 白玉 was brought up. Though it can be read as しろだま, the most common reading in this case happens to be しらたま. The first form of 白 without a vowel change can be called its 露出形, and the form しら- can be called its 被覆形. It turns out that there are three such vowel shifts in many native Japanese words. }

\begin{ltabulary}{|P|P|}
\hline 

i―o & き + かげ \textrightarrow  こかげ (木陰・木蔭)  \emph{Tree shade }\\ \cline{1-2}

e―a & め + ふた \textrightarrow  まぶた (瞼・目蓋)  \emph{Eyelid }\\ \cline{1-2}

o―a & くろ + やみ \textrightarrow  くらやみ (暗闇) \emph{Darkness }\\ \cline{1-2}

\end{ltabulary}

\par{ The problem is that positing a rule as to when these sound changes should occur or not is troubling. For instance, 白 is often not read as しら in many compounds such as 白馬 (white horse) and 白熊 (polar bear). }
      
\section{I―O\slash U}
 
\par{ The i that changes to "o" in these words was not pronounced like the modern い. This i vowel will be spelled as ï. The 露出形 would end in ï, but the 被覆形 could either end in u or o\slash ə (the other o vowel). }

\begin{ltabulary}{|P|}
\hline 

kï き + tati たち \textrightarrow  kodati こだち (木立)  \emph{Grove of trees }\\

kï き + nə の + ma ま \textrightarrow  kənəma  このま (木の間) \emph{In the trees }\hfill\break
\\

tukï つき + yo よ \textrightarrow  tukuyo つくよ (月夜) \emph{Moonlit night }\hfill\break
\\

pï ひ + teru てる \textrightarrow  poteru ほてる (火照る)  \emph{To feel hot }\\

kutï くち + wa わ \textrightarrow  kutuwa くつわ (轡) \emph{Bit }\\

kami かみ + agari あがり \textrightarrow   kamuagari  (神上り) \emph{Ascension of a kami }\\

\end{ltabulary}

\par{\textbf{Reading Note }: つきよ is now the normal modern reading of 月夜. }

\par{ Polysynthesis is the chaining of things that can't stand alone in languages like Ainu. Verbal clause indefiniteness and transitive subject avoidance are two main characteristics. This prevents multiple verb doers from being expressed in a word and words themselves have multiple parts. }

\par{ Why does this have to do with anything happening in Japanese? In the development of the 露出形 from 被覆形, an affix イ had the ability of making something indefinite definite to stand alone. So, one could call the 被覆形 the 抱合形 (polysynthetic form). This affix イ would replace the vowel in the 露出形. So, it's more so a sound replacement and the other way around. }

\par{ This イ also provides the origin for the 連用形 of verbs. Its presence after the root of a verb has a nominalizing effect and rids it of indefiniteness. For verbs that have a 連用形 that ends in エ, as you will be said again later in this lesson, it ultimately derives from ア + イ. Because Japanese has historically shunned upon vowel sequences, it's also posited that イ is after the 連用形 of 一段 verbs. }

\par{ The following is an example found in the 古事記 of イ being used to show an "instructive case". In most other situations including this case, it is an emphasizer first and foremost. }

\par{1. }

\par{${\overset{\textnormal{おさか}}{\text{忍坂}}}$ の ${\overset{\textnormal{おほむろや}}{\text{大室屋}}}$ に ${\overset{\textnormal{ひとさは}}{\text{人多}}}$ に來入り居り人多に入り居りとも \hfill\break
${\overset{\textnormal{みつみつ}}{\text{厳厳}}}$ し ${\overset{\textnormal{くめ}}{\text{久米}}}$ の子が ${\overset{\textnormal{くぶつつ}}{\text{頭槌}}}$ \textbf{い }石 ${\overset{\textnormal{つつ}}{\text{槌}}}$ \textbf{い }持ち 撃ちてし止まむ \hfill\break
厳厳し  久米の子等が頭槌 \textbf{い }石槌 \textbf{い }持ち 今撃たば宜し \hfill\break
A lot of people have come to the big cellar in Osaka. Even a lot of people were to enter, are the vigorous Kumebe not going to hit at their enemies with their war and stone hammers? It would be best for the vigorous Kumebe to strike them with their war and stone hammers now! }

\par{ There are even times when it acts like a subject marker. Does this remind anyone of Korean? }

\par{2. 菟原壮士 \textbf{い }天仰ぎ \hfill\break
The man Urai looked up at the heavens \hfill\break
From the 万葉集. }

\par{ This イ got used a lot, as you will continue to see. And, for 沼 (swamp), which is read as ぬま, it didn't need this 露出・被覆 contrast, but the reading ぬみ still existed from イ being attached to it. }
      
\section{E―A}
 
\par{ There is a plethora of words with this sound change. Consider the following examples. }

\begin{ltabulary}{|P|P|P|P|P|P|}
\hline 

掌 & たなごころ & Palm & 手綱 & たづな & Rein \\ \cline{1-6}

船大工 & ふなだいく & Boat-builder & 爪弾き & つまはじき & Ostracism \\ \cline{1-6}

船酔い & ふなよい & Seasickness & 爪楊枝 & つまようじ & Toothpick \\ \cline{1-6}

胸算用 & むなざんよう & Rough mental estimate & 金物 & かなもの & Metal utensil \\ \cline{1-6}

胸毛 & むなげ & Chest hair & 上の空 & うわのそら & Absent-mindedness \\ \cline{1-6}

声音 & こわね & Tone of voice & 風上 & かざかみ & Windward \\ \cline{1-6}

酒屋 & さかや & Liquor store & 風下 & かざしも & Leeward \\ \cline{1-6}

眉毛 & まゆげ & Eyebrow & 睫(毛) & まつげ & Eyelash \\ \cline{1-6}

天の川 & まあのがわ & Milky Way & 天の浮橋 & あまのうきはし & Heavenly floating bridge \\ \cline{1-6}

雨傘 & あまがさ & Umbrella & 菅原 & すがわら & Sugawara (surname) \\ \cline{1-6}

猪苗代 & いなわしろ & Inawashiro (Place name) & 苗水 & なわみず & Water for rice nursery \\ \cline{1-6}

\end{ltabulary}

\par{\textbf{Etymology Note }: 天 and 雨 ultimately share origin. }

\begin{center}
 \textbf{Explaining the Origin of E in Japanese? }
\end{center}

\par{ It turns out that very few native words actually start with an e-sound. Sure, we have 手, 瀬, 関, 寺, 目, 根, and 江. But, can you think of any others easily? In Old Japanese in the 奈良時代, there were 甲 and 乙 distinctions in other vowels like o, and they would have been treated as separate vowels. However, エ段 instances were always only 乙. Does this suggest the vowel "e" was not originally in Japanese? }

\par{ It's difficult to initially say this, though, taking 酒 as an example, that さか precedes さけ in origin when you can't read 甘酒 as あまざか. However, it's important to realize that あい \textrightarrow  え(-) in many Modern Japanese dialects. We just need to find a reason for why such a change would have happened centuries ago. }

\par{ Words like 手綱 and 目蓋 show this \slash e\slash  \textrightarrow  \slash a\slash  sound change in the word initial position. Inside words, we don't see this other forms. So, again, you don't see something like 白酒 read as しろざか. However, there is 白髪, which is read as しらが instead of しらげ. This, though, appears to be exceptional. }

\par{ This is when the terms 露出形 and 被覆形 become useful. The form that does not have the affixing of "i" inside compounds is called the 被覆形. The 露出形 represents the affixing, which results in the sound change. }

\par{ There is also a grammatical aspect to this sound change. Let's consider 甘酒 again. 甘 is the root of an adjective. You can't add a particle to it, and for it to behave as a noun, you would need to affix さ or み to the root. It turns out that the イ used in creating the 露出形 of these words possessed a similar grammatical function. }

\par{ If you are still doubtful, there are a number of nouns posited to come from this イ attaching to the root of an adjective. 飴 and 岳 are believed to have come from 甘 and 高 respectively. Other evidence for this nominalizing イ is found in 或 \textbf{い }は (possibly). When after the 連体形  of verbs, it showed a focus. As for these nouns, however, this would have been a grammatically necessary add-on to be used. }

\par{ Haven't you ever thought that the かが in 鏡 is related to 影? You would be right. Have you ever noticed that 音 read as ね reminds you of 泣く and 鳴る? These words are already semantically interrelated, and the expression 音を泣く even exists. The な in these words, then, must have become ね through this sound change caused by イ. }

\par{  Of course, there are typically variations to any given method. There were also times when a consonant was inserted for whatever reason between イ and what it would attach to. This, unsurprisingly, is preserved and several words. For instance, although we may have gotten 毛 from カ + イ. カ +m + イ = 髪, another word for hair. }

\par{ The list of words below also happen to show up in compounds in which there is no longer a 被覆形 because of the dropping of the final mora. Some of these have become new, standalone nouns. }

\begin{ltabulary}{|P|P|P|}
\hline 

Consonant Insertion in 被覆形 & Words with Loss of 被覆形 & New Independent Noun(s)? \\ \cline{1-3}

ハ + s + イ \textrightarrow  端 & 端数(はすう) \emph{Fraction }& 葉?, 歯?, 刃?, 端 〇 \\ \cline{1-3}

ア + s + イ \textrightarrow  足 & 足掻(あが)く \emph{To struggle }& No \\ \cline{1-3}

クス + r + イ \textrightarrow  薬 & 薬師(くすし) \emph{Doctor (Archaism) }& No \\ \cline{1-3}

ト + r + イ \textrightarrow  鳥 & 鳥羽(とば) \emph{Toba }& No \\ \cline{1-3}

カタ + t + イ \textrightarrow  形 & 形見(かたみ) \emph{Memento }& 形 〇, 肩 ? \\ \cline{1-3}

\end{ltabulary}

\par{ There are also a few examples where イ attaches to other things that didn't end in ア and replaced the vowel. This explains the relation between 奥 (inner part) and 沖 (offing). In the 万葉集, you can sometimes see this replacement even when something does end in ア. For instance, ワガ+ モコ (my love) ends up being read as わぎもこ. }

\par{ Back then, Japanese didn't like chains of vowels, which would make such sound changes and vowel replacements uncommon. It also means that vowel deletion altogether should be present. For instance, the place name 明石 is read as あかし instead of as あかいし. }

\par{ There are other instances of vowels being interchanged. For example, 暦 means calendar, so why is it read as こよみ? The こ happens to be the same か as in 四日. Why do you use はた in はたち but はつ in はつか? It's because you remove ア and replace it with ウ. Lastly, 肩 may have very well come from 形, but the かつ in 担ぐ may very well be from 肩 with this sound change. }

\begin{center}
 \textbf{露出形 in Compounds? }
\end{center}

\par{ So, what about instances when you can use the 露出形 in a compound? For instance, 風向き can also be read as かぜむき. Even so, かぜむき is actually rarer, and such exceptions can easily come about from centuries since this interwoven phonological phenomenon took root in Japanese. As another example, まなじり and めじり exist for "corner of the eye", but the latter comes about much later. And, the original is still the most common. }

\par{ Now, we are discussing the relation between the 露出形 and 被覆形 for e―a. Of course there are the other changes. But, these have to be looked at separately. So, be patient. }
      
\section{O―A}
 
\par{ This 露出・被覆 relationship is not as common, and it doesn't have anything to do with イ, unlike above for the other two cases. The motivation for the sound change is the same. }

\begin{ltabulary}{|P|P|P|P|P|P|}
\hline 

白髪 & しらが & White\slash grey hair & 暗がり & くらがり & Darkness \\ \cline{1-6}

\end{ltabulary}

\par{ Instances of this are scattered in the language, but the most interesting one is しな・しの. This was a general word for a celestial body like the moon and sun, and always practically referred to one or the other. When to use which was maintained following the outlined guidelines, and the word can be spotted in Okinawan languages and in the mainland up to the 8th century. }

\par{3. しなてる片岡山に、飯に飢て臥せる、その旅人あわれ。 \hfill\break
That poor traveler who is faced down starved of food at Kataoka Mountain, illuminated by the sources of light in the sky. \hfill\break
From a Song by 聖徳太子. }

\par{ Sometimes, a particular form may avoided because the resulting word would be homophonous to an existing word. The one example that immediately comes to mind is 白身. This is the white of an egg or the white flesh of certain species of fish. This word is read as しろみ. It is not read as しらみ presumably because this word means louse (虱・蝨). }

\par{ At times, the use of a different form can change the meaning of the word. This is rather interesting considering that this 露出 and 被覆 relationship is supposed to be a fundamental principle for compounding. }

\begin{ltabulary}{|P|P|P|}
\hline 

白魚 & しらうお = Icefish & しろうお = Ice goby \\ \cline{1-3}

\end{ltabulary}
      
\section{Instances of Loss of 被覆形}
 
\par{ There have been some words where the 被覆形 has essentially disappeared from the language. This is not surprising as not all words share this kind of phenomenon. }

\begin{ltabulary}{|P|P|P|P|}
\hline 

 & 露出形 & 被覆形 & Now \\ \cline{1-4}

身 & み & む & み \\ \cline{1-4}

茎 & くき & くく & くき \\ \cline{1-4}

藻 & め & も & も \\ \cline{1-4}

\end{ltabulary}

\par{ As the last example shows, the 被覆形 becoming the standalone variant is not unheard of. Of course, there are still examples of rare words where older forms may still be present.   }
    