    
\chapter{あて字}

\begin{center}
\begin{Large}
第361課: あて字 
\end{Large}
\end{center}
 
\par{ あて字 is a blanket term for irregular readings in Japanese. However, there is a more exact definition of it that separates it from other special cases that this lesson will go in depth in. }
      
\section{熟字訓}
 
\par{ Consequently, in using 漢字 to write Japanese, there are plenty of instances in which the set readings of characters are still not efficient to spell particular native words. Thus, to compensate this problem, one solution is to attribute a particular reading of a set of characters for a certain instance, ignoring the phonetic values of the 漢字 chosen. Thus, the characters are chosen for semantic value. This is called 熟字訓 or 義訓. }

\par{ This, as you would imagine, causes major problems. As we have seen with 国字, many characters had to be created to write the names of fish and what not. The other option chosen that was not mentioned in that lesson was the plethora of 熟字訓 created for the same purpose. Thus, from the perspective of the avid Japanese reader and prospecting Japanese learner, spelling is an arduous task to say the least. }

\par{ Luckily, very few instances of 熟字訓 are taught in school. Thus, the average number known by even natives is relatively few. So, in the case of literature, they are generally given reading aids. However, due to the sheer number of them, you will and have surely already encountered them. }

\par{ It must also be noted that not all 熟字訓 are for native words. There are instances of it for loanwords. However, the large majority of examples are still native words. }

\begin{center}
\textbf{Native Word Examples }
\end{center}

\begin{ltabulary}{|P|P|P|P|P|P|P|P|P|}
\hline 

大人 & おとな & Adult & 時雨 & しぐれ & Drizzle & 梅雨 & つゆ & Rainy season \\ \cline{1-9}

五月雨 & さみだれ & May rain & 紫陽花 & あじさい & Hydrangea & 紅葉 & もみじ & Colored leaves \\ \cline{1-9}

田舎 & いなか & Countryside & 今日 & きょう & Today & 今朝 & けさ & This morning \\ \cline{1-9}

昨日 & きのう & Yesterday & 明日 & あす・あした & Tomorrow & 去年 & こぞ & Last year \\ \cline{1-9}

狗母魚 & えそ & Lizardfish & 薔薇 & ばら & Rose & 梔子 & くちなし & Gardenia \\ \cline{1-9}

向日葵 & ひまわり & Sunflower & 硫黄 & いおう & Sulfur & 意気地 & いくじ & Self-confidence \\ \cline{1-9}

乙女 & おとめ & Maiden & 母屋 & おもや & Main building & 河岸 & かし & River bank \\ \cline{1-9}

風邪 & かぜ & A cold & 仮名 & かな & Kana & 為替 & かわせ & Exchange \\ \cline{1-9}

雑魚 & ざこ & Small fry & 桟敷 & さじき & Gallery & 素人 & しろうと & Novice \\ \cline{1-9}

玄人 & くろうと & Expert & 仲人 & なこうど & Go-between & 若人 & わこうど & Young person \\ \cline{1-9}

数珠 & じゅず & Rosary & 師走 & しわす & 12th month & 雪崩 & なだれ & Avalanche \\ \cline{1-9}

博士 & はかせ & Professor & 下手 & へた & Bad at & 土産 & みやげ & Souvenir \\ \cline{1-9}

木綿 & もめん & Cotton & 寄席 & よせ & Vaudeville & 水母 & くらげ & Jellyfish \\ \cline{1-9}

\end{ltabulary}

\par{\textbf{Indigenous Cultural Items Not }e: Many 熟字訓 are for indigenous\slash cultural items. Clearly, China wouldn't have had a spelling for them as they wouldn\textquotesingle t be Chinese things. }

\begin{ltabulary}{|P|P|P|P|P|P|P|P|P|}
\hline 

神楽 & かぐら & Kagura & 出汁 & だし & Dashi & 太刀 & たち & Longsword \\ \cline{1-9}

祝詞 & のりと & Ritual prayer & お神酒 & おみき & Sacred sake & 大和 & やまと & Yamato \\ \cline{1-9}

相撲 & すもう & Sumo & 草履 & ぞうり & Zori & 小豆 & あずき & Azuki bean \\ \cline{1-9}

浴衣 & ゆかた & Yukata & 竹刀 & しない & Fencing stick & 手水 & ちょうず & Water for washing hands \\ \cline{1-9}

\end{ltabulary}

\par{\textbf{Classification Note }: Although 意気地 and仮名 happen to have readings the resemble actual readings of the characters, because they are chosen primarily for meaning, they are classified as 熟字訓. }

\par{\textbf{Counter Note }: Remember that many counter expressions are irregular in this regard. Consider the days of the month like 一日, 二日, etc. Also consider 一人 and 二人 which are half 熟字訓 because of the -り reading of 人. There's also the important exception for 20 years old, ${\overset{\textnormal{はたち}}{\text{二十歳}}}$ . }

\par{\textbf{Archaism Note }: こぞ  is now an archaism. Also, 師走 is the 12th month of the lunar calendar. }

\begin{center}
\textbf{Loanword Examples }
\end{center}

\begin{ltabulary}{|P|P|P|P|P|P|}
\hline 

煙草 & タバコ & Tobacco & 燐寸 & マッチ & Match \\ \cline{1-6}

煙管 & キセル & Pipe & 硝子 & ガラス & Glass \\ \cline{1-6}

洋燈 & ランプ & Lamp & 麦酒 & ビール & Beer \\ \cline{1-6}

\end{ltabulary}

\par{\textbf{外来語 Note }: Most loanwords are normally written in カタカナ. However, there are instances where really old loans such as タバコ are frequently written in ひらがな and 漢字. By the end of the Meiji Period, there were already hundreds of loans that were given 漢字spellings, and they were frequently used up until after World War II. Even so, if you read a lot of literature, you will undoubtedly encounter more example of this. As they are tested to some degree in the upper levels of the 漢検, learning these spellings is not completely trivial. Also, if you read anything from the late 1800s or early 1900s, they obviously show up in large number. }

\begin{center}
 \textbf{Part 熟字訓 }
\end{center}

\par{ There are also instances where only part of a word is 熟字訓. Again, there are a lot of examples regarding indigenous items, plants, and what not. However, it\textquotesingle s important to note that irregular changes to 音読み are also classified as 熟字訓. }

\begin{ltabulary}{|P|P|P|P|P|P|P|P|P|}
\hline 

日和 & ひより & Weather & 息子 & むすこ & Son & 一言居士 & いちげんこじ & Ready critic \\ \cline{1-9}

笑顔 & えがお & Smiling face & 息吹 & いぶき & Breath & 海原 & うなばら & Seabed \\ \cline{1-9}

浮気 & うわき & Affair & 果物 & くだもの & Fruit & 心地 & ここち & Sensation \\ \cline{1-9}

清水 & しみず & Pure water & 三味線 & しゃみせん & Shamisen & 白髪 & しらが & Grey hair \\ \cline{1-9}

砂利 & じゃり & Gravel & 読経 & どきょう & Sutra chanting & 友達 & ともだち & Friend \\ \cline{1-9}

子供 & こども & Friend & 名残 & なごり & Vestige & 上手 & じょうず & Good at \\ \cline{1-9}

眼鏡 & めがね & Glasses & 行方 & ゆくえ & Whereabouts & 蚊帳 & かや & Mosquito net \\ \cline{1-9}

達磨 & だるま & Daruma & 八百屋 & やおや & Greengrocer & 伝馬船 & てんません & Sculling boat \\ \cline{1-9}

\end{ltabulary}

\begin{center}
 \textbf{Battle with Loanwords }
\end{center}

\par{ In their heyday, loanwords brought another problem. For the examples we've seen thus far, there weren't Japanese equivalents. But, there have also been borrowings even though Japanese equivalents already existed. So, what better thing to do then attribute the loanword to the existing spellings! This is exactly what happened and the way one knew for sure how to read them was with ルビ. Though the practice is no longer near as common, you'll no doubt find some examples. }

\begin{ltabulary}{|P|P|P|P|P|P|P|P|}
\hline 

北極光 & ほっきょくこう & オーロラ & Aurora & 郷愁 & きょうしゅう & ノスタルジア & Nostalgia \\ \cline{1-8}

珈琲店 & こーひーてん & カフェ & Cafe & 火酒 & かしゅ & ウォッカ・ヴォッカ & Vodka \\ \cline{1-8}

接吻 & せっぷん & キッス & Kiss & 情調 & じょうちょう & ムード & Mood \\ \cline{1-8}

短艇 & たんてい & ボート & Boat & 並木道 & なみきみち & アベニュー & Avenue \\ \cline{1-8}

裁縫機械 & さいほうきかい & ミシン & Sewing machine & 喞筒 & そくとう & ポンプ & Pump \\ \cline{1-8}

\end{ltabulary}

\begin{center}
 \textbf{漢語 \textrightarrow  和語 } 
\end{center}

\par{ Although not considered 熟字訓 by some, there is a tendency in writing to impose the native word on the Sino-Japanese word and indicate the intended reading with ルビ. }

\begin{ltabulary}{|P|P|P|P|}
\hline 

漢語 & 和語 & 漢語 & 和語 \\ \cline{1-4}

暫時 & しばらく & 失敗 & しくじる \\ \cline{1-4}

墳墓 & 墓 & 生活 & 暮し \\ \cline{1-4}

\end{ltabulary}
      
\section{あて字}
 
\par{ あて字 is the use of 漢字 for phonetic purposes only. The meanings of the characters are, thus, ignored. Characters used in this fashion are called 借字. 音読み being borrowed for this are called音借, and 訓読み being borrowed for this are called 訓借. }

\par{ There are also words spelled with a mixture of 音借 and 訓借. There is also a strong tendency to choose characters that help with meaning. These may resemble 熟字訓, but 熟字訓 clearly show no relation between the characters and their phonetic values. }

\par{ Words like this can easily be written in かな. They may also have alternative non-あて字 spellings. Such spellings are common for loanwords, but as you can imagine, there are also plenty of native word examples. There are also 漢語 examples. }

\par{\textbf{Spelling Note }: あて字 can also be spelled as 当(て)字 and 宛字. }

\begin{center}
\textbf{Native Word Examples }
\end{center}

\begin{ltabulary}{|P|P|P|P|P|P|}
\hline 

寿司 & すし & Sushi & 合羽 & かっぱ & Raincoat \\ \cline{1-6}

野良 & のら & Wild & 波止場 & はとば & Pier \\ \cline{1-6}

出鱈目 & でたらめ & Random & 滅茶苦茶 & めちゃくちゃ & Wreck; absurd \\ \cline{1-6}

芽出度い・目出度い & めでたい & Auspicious & 浦山敷 & うらやましく & Enviously \\ \cline{1-6}

沢山 & たくさん & A lot & 滅多 & めった & Seldom \\ \cline{1-6}

珍紛漢紛・珍糞漢糞・陳奮翰奮 \hfill\break
& ちんぷんかんぷん & Nonsense & 兎に角 & とにかく & Anyway \\ \cline{1-6}

\end{ltabulary}

\begin{center}
 \textbf{Loan Word Examples }
\end{center}

\begin{ltabulary}{|P|P|P|P|P|P|P|P|P|}
\hline 

瓦斯 & ガス & Gas & 珈琲 & コーヒー & Coffee & 曹達 & ソーダ & Soda \\ \cline{1-9}

護謨 & ゴム & Rubber & 羅紗 & ラシャ & Woolen cloth & 沙翁 & シェークスピア & Shakespeare \\ \cline{1-9}

倶楽部 & クラブ & Club & 型録 & カタログ & Catalog & 場穴 & バケツ & Basket \\ \cline{1-9}

浪漫 & ロ(ウ)マン & Romance & 咖[喱・哩] & カレー & Curry & 亜細亜 & アジア & Asia \\ \cline{1-9}

欧羅巴 & ヨーロッパ & Europe & 亜米利加 & アメリカ & America & 加奈陀 & カナダ & Canada \\ \cline{1-9}

\end{ltabulary}

\par{\textbf{Spelling Note }: 珈琲 may still be frequently seen on coffee cans, store signs, etc. }

\par{ Although あて字, because certain loans have been attributed to particular single characters, they are felt to be 訓読み. }

\begin{ltabulary}{|P|P|P|P|P|P|P|P|P|P|P|P|P|P|P|}
\hline 

頁 & ページ & Page & 米 & メートル & Meter & 瓦 & グラム & Gram & 片 & ペンス & Pence & 釦 & ボタン & Button \\ \cline{1-15}

\end{ltabulary}
 
\begin{center}
\textbf{Country Names }
\end{center}

\par{ All country names have been given あて字. Some are the same as used in Chinese, and if you watched the 2008 Beijing Olympics, you no doubt saw all of the country names in 漢字. However, there are still many differences as the sound systems of Mandarin and Japanese are quite different. }

\par{ Some Chinese spellings had great influence. For instance, the spelling for England, 英吉利, begins with 英. So, although the pronunciation of 英吉利 is イギリス, 英 is frequently used in words regarding England. 英語 and 英国 are perfect examples. The first (or second if the first is taken) character of the spellings for the countries are used as abbreviations for them. }

\begin{center}
${\overset{\textnormal{アメリカ}}{\text{亜米利加}}}$ \textrightarrow   米 ベイ ${\overset{\textnormal{ロシア}}{\text{露西亜}}}$ \textrightarrow  露 ロ ${\overset{\textnormal{オーストラリア}}{\text{濠太剌利}}}$ \textrightarrow  豪・濠 ゴウ 
\end{center}

\par{ As you would expect, there are often options. There are even あて字 spellings for important international cities such as Paris (巴里). These spellings were more common in the heyday of writing loans in 漢字 at the beginning of the 20 th century. But, some remain important. Because these spellings in particular are featured in the upper levels of the 漢検, for those that truly love 漢字, by all means try to learn them all. For the casual, normal Japanese learner, it may help to learn the spellings of some of the most important countries and cities. Wikipedia has a lovely chart with a comparison between the Japanese and Chinese spellings. http:\slash \slash ja.wikipedia.org\slash wiki\slash \%E5\%9B\%BD\%E5\%90\%8D\%E3\%81\%AE\%E6\%BC\%A2\%E5\%AD\%97\%E8\%A1\%A8\%E8\%A8\%9 8\%E4\%B8\%80\%E8\%A6\%A7  }
    