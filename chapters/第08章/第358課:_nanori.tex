    
\chapter{名乗り}

\begin{center}
\begin{Large}
第358課: 名乗り 
\end{Large}
\end{center}
 
\par{ Reading names is extremely difficult. Personal names, surnames, and place names are all very difficult for learners of Japanese and Japanese natives to know how to read properly. Though a lifetime of experience in the language makes the process easier, there is not a foolproof way of being 100\% certain 100\% of the time. Despite this difficulty, this lesson will attempt to explain various aspects you can find in the readings of names. }

\par{ Names, though, are truly important to people. The famous author known by the name of 森鷗外 upon his death gave the following statement in his will: 余ハ石見人森林太郎トシテ死セント欲ス。墓ハ森林太郎ノ外一字モホルベカラズ (I wish to die as Iwamijin Mori Rintarou. Do not carve any other letters other than Mori Rintarou on my grave. He was a native of Iwami, a part of present day Chiba Prefecture. His given name was 森林太郎, and he wished to die that way. In Japanese culture a lot of thought is put into a name. Think of this as you learn more about names. }
      
\section{人名用漢字}
 
\par{ Though there have been many characters used in name in both Chinese and Japanese for a very long time, in attempts to practically narrow down the number of characters and readings that could be used in names, a list of characters not already in the list of general use characters was created by National Language Committee in 1951. Although it has been updated several times since, it is enforced by the Ministry of Justice. }

\par{ People can only have the 2136 常用漢字, 861 人名用漢字, and かな in their names. Any character outside of this is considered as a 表外字. Additions to the list are being considered in accordance to requests from parents. The increase of name characters is being done in attempts to increase the list of general use characters. In fact, in 2010 121 characters from the 人名用漢字表 were put into the 常用漢字表. This trend will probably continue as the use of 漢字 is re-surging due to typing technology and people's cultural pride in the use of 漢字 becomes ever stronger. }

\begin{ltabulary}{|P|P|P|P|P|P|P|P|P|P|}
\hline 

 丑 & 丞 & 乃 & 之 & 乎 & 也 & 云 & 亘・亙 & 些 & 亦 \\ \cline{1-10}

亥 & 亨 & 亮 & 仔 & 伊 & 伍 & 伽 & 佃 & 佑 & 伶 \\ \cline{1-10}

侃 & 侑 & 俄 & 俠 & 俣 & 俐 & 倭 & 俱 & 倦 & 倖 \\ \cline{1-10}

偲 & 傭 & 儲 & 允 & 兎 & 兜 & 其 & 冴 & 凌 & 凜・凛 \\ \cline{1-10}

凧 & 凪 & 凰 & 凱 & 函 & 劉 & 劫 & 勁 & 勺 & 勿 \\ \cline{1-10}

匁 & 匡 & 廿 & 卜 & 卯 & 卿 & 厨 & 厩 & 叉 & 叡 \\ \cline{1-10}

叢 & 叶 & 只 & 吾 & 吞 & 吻 & 哉 & 哨 & 啄 & 哩 \\ \cline{1-10}

喬 & 喧 & 喰 & 喋 & 嘩 & 嘉 & 嘗 & 噌 & 噂 & 圃 \\ \cline{1-10}

圭 & 坐 & 尭・堯 & 坦 & 埴 & 堰 & 堺 & 堵 & 塙 & 壕 \\ \cline{1-10}

壬 \hfill\break
& 夷 & 奄 & 奎 & 套 & 娃 & 姪 & 姥 & 娩 & 嬉 \\ \cline{1-10}

孟 & 宏 & 宋 & 宕 & 宥 & 寅 & 寓 & 寵 & 尖 & 尤 \\ \cline{1-10}

屑 & 峨 & 峻 & 崚 & 嵯 & 嵩 & 嶺 & 巌・巖 & 已 & 巳 \\ \cline{1-10}

巴 & 巷 & 巽 & 帖 & 幌 & 幡 & 庄 & 庇 & 庚 & 庵 \\ \cline{1-10}

廟 & 廻 & 弘 & 弛 & 彗 & 彦 & 彪 & 彬 & 徠 & 忽 \\ \cline{1-10}

怜 & 恢 & 恰 & 恕 & 悌 & 惟 & 惚 & 悉 & 惇 & 惹 \\ \cline{1-10}

惺 & 惣 & 慧 & 憐 & 戊 & 或 & 戟 & 托 & 按 & 挺 \\ \cline{1-10}

挽 & 掬 & 捲 & 捷 & 捺 & 捧 & 掠 & 揃 & 摑 & 摺 \\ \cline{1-10}

撒 & 撰 & 撞 & 播 & 撫 & 擢 & 孜 & 敦 & 斐 & 斡 \\ \cline{1-10}

斧 & 斯 & 於 & 旭 & 昂 & 昊 & 昏 & 昌 & 昴 & 晏 \\ \cline{1-10}

晃・晄 & 晒 & 晋 & 晟 & 晦 & 晨 & 智 & 暉 & 暢 & 曙 \\ \cline{1-10}

曝 & 曳 & 朋 & 朔 & 杏 & 杖 & 杜 & 李 & 杭 & 杵 \\ \cline{1-10}

杷 & 枇 & 柑 & 柴 & 柘 & 柊 & 柏 & 柾 & 柚 & 桧・檜 \\ \cline{1-10}

栞 & 桔 & 桂 & 栖 & 桐 & 栗 & 梧 & 梓 & 梢 & 梛 \\ \cline{1-10}

梯 & 桶 & 梶 & 椛 & 梁 & 棲 & 椋 & 椀 & 楯 & 楚 \\ \cline{1-10}

楕 & 椿 & 楠 & 楓 & 椰 & 楢 & 楊 & 榎 & 樺 & 榊 \\ \cline{1-10}

榛 & 槙・槇 & 槍 & 槌 & 樫 & 槻 & 樟 & 樋 & 橘 & 樽 \\ \cline{1-10}

橙 & 檎 & 檀 & 櫂 & 櫛 & 櫓 & 欣 & 欽 & 歎 & 此 \\ \cline{1-10}

殆 & 毅 & 毘 & 毬 & 汀 & 汝 & 汐 & 汲 & 沌 & 沓 \\ \cline{1-10}

沫 & 洸 & 洲 & 洵 & 洛 & 浩 & 浬 & 淵 & 淳 & 渚・渚 \\ \cline{1-10}

淀 & 淋 & 渥 & 湘 & 湊 & 湛 & 溢 & 滉 & 溜 & 漱 \\ \cline{1-10}

漕 & 漣 & 澪 & 濡 & 瀕 & 灘 & 灸 & 灼 & 烏 & 焰 \\ \cline{1-10}

焚 & 煌 & 煤 & 煉 & 熙 & 燕 & 燎 & 燦 & 燭 & 燿 \\ \cline{1-10}

爾 & 牒 & 牟 & 牡 & 牽 & 犀 & 狼 & 猪・猪 & 獅 & 玖 \\ \cline{1-10}

珂 & 珈 & 珊 & 珀 & 玲 & 琢・琢 & 琉 & 瑛 & 琥 & 琶 \\ \cline{1-10}

琵 & 琳 & 瑚 & 瑞 & 瑶 & 瑳 & 瓜 & 瓢 & 甥 & 甫 \\ \cline{1-10}

畠 & 畢 & 疋 & 疏 & 皐 & 皓 & 眸 & 瞥 & 矩 & 砦 \\ \cline{1-10}

砥 & 砧 & 硯 & 碓 & 碗 & 碩 & 碧 & 磐 & 磯 & 祇 \\ \cline{1-10}

祢・禰 & 祐・祐 & 祷・禱 & 禄・祿 & 禎・禎 & 禽 & 禾 & 秦 & 秤 & 稀 \\ \cline{1-10}

稔 & 稟 & 稜 & 穣・穰 & 穹 & 穿 & 窄 & 窪 & 窺 & 竣 \\ \cline{1-10}

竪 & 竺 & 竿 & 笈 & 笹 & 笙 & 笠 & 筈 & 筑 & 箕 \\ \cline{1-10}

箔 & 篇 & 篠 & 簞 & 簾 & 籾 & 粥 & 粟 & 糊 & 紘 \\ \cline{1-10}

紗 & 紐 & 絃 & 紬 & 絆 & 絢 & 綺 & 綜 & 綴 & 緋 \\ \cline{1-10}

綾 & 綸 & 縞 & 徽 & 繫 & 繡 & 纂 & 纏 & 羚 & 翔 \\ \cline{1-10}

翠 & 耀 & 而 & 耶 & 耽 & 聡 & 肇 & 肋 & 肴 & 胤 \\ \cline{1-10}

胡 & 脩 & 腔 & 脹 & 膏 & 臥 & 舜 & 舵 & 芥 & 芹 \\ \cline{1-10}

芭 & 芙 & 芦 & 苑 & 茄 & 苔 & 苺 & 茅 & 茉 & 茸 \\ \cline{1-10}

茜 & 莞 & 荻 & 莫 & 莉 & 菅 & 菫 & 菖 & 萄 & 菩 \\ \cline{1-10}

萌・萠 & 萊 & 菱 & 葦 & 葵 & 萱 & 葺 & 萩 & 董 & 葡 \\ \cline{1-10}

蓑 & 蒔 & 蒐 & 蒼 & 蒲 & 蒙 & 蓉 & 蓮 & 蔭 & 蔣 \\ \cline{1-10}

蔦 & 蓬 & 蔓 & 蕎 & 蕨 & 蕉 & 蕃 & 蕪 & 薙 & 蕾 \\ \cline{1-10}

蕗 & 藁 & 薩 & 蘇 & 蘭 & 蝦 & 蝶 & 螺 & 蟬 & 蟹 \\ \cline{1-10}

蠟 & 衿 & 袈 & 袴 & 裡 & 裟 & 裳 & 襖 & 訊 & 訣 \\ \cline{1-10}

註 & 詢 & 詫 & 誼 & 諏 & 諄 & 諒 & 謂 & 諺 & 讃 \\ \cline{1-10}

豹 & 貰 & 賑 & 赳 & 跨 & 蹄 & 蹟 & 輔 & 輯 & 輿 \\ \cline{1-10}

轟 & 辰 & 辻 & 迂 & 迄 & 辿 & 迪 & 迦 & 這 & 逞 \\ \cline{1-10}

逗 & 逢 & 遥・遙 & 遁 & 遼 & 邑 & 祁 & 郁 & 鄭 & 酉 \\ \cline{1-10}

醇 & 醐 & 醍 & 醬 & 釉 & 釘 & 釧 & 銑 & 鋒 & 鋸 \\ \cline{1-10}

錘 & 錐 & 錆 & 錫 & 鍬 & 鎧 & 閃 & 閏 & 閤 & 阿 \\ \cline{1-10}

陀 & 隈 & 隼 & 雀 & 雁 & 雛 & 雫 & 霞 & 靖 & 鞄 \\ \cline{1-10}

鞍 & 鞘 & 鞠 & 鞭 & 頁 & 頌 & 頗 & 顚 & 颯 & 饗 \\ \cline{1-10}

馨 & 馴 & 馳 & 駕 & 駿 & 驍 & 魁 & 魯 & 鮎 & 鯉 \\ \cline{1-10}

鯛 & 鰯 & 鱒 & 鱗 & 鳩 & 鳶 & 鳳 & 鴨 & 鴻 & 鵜 \\ \cline{1-10}

鵬 & 鷗 & 鷲 & 鷺 & 鷹 & 麒 & 麟 & 麿 & 黎 & 黛 \\ \cline{1-10}

鼎 &  &  &  &  &  &  &  &  &  \\ \cline{1-10}

\end{ltabulary}
 
\par{ The following 常用漢字 have 旧字体 variants that are allowed in names. }

\begin{ltabulary}{|P|P|P|P|P|P|P|P|P|P|P|}
\hline 

 亞 (亜) &  惡(悪) &  爲(為) &  逸(逸) &  榮(栄)  &   衞(衛) &  謁(謁) &  圓(円) &  緣(縁) &  薗(園) &   應(応) \\ \cline{1-11}

 櫻(桜) &  奧(奥) &  橫(横) &  溫(温) &  價(価) &  禍(禍) &  悔(悔) &  海(海) &  壞(壊) &  懷(懐) &  樂(楽) \\ \cline{1-11}

 渴(渇) &  卷(巻) &  陷(陥) &  寬(寛) &  漢(漢) &  氣(気 ) \hfill\break
&  祈(祈) &  器(器) &  僞(偽) &  戲(戯) &  虛(虚) \\ \cline{1-11}

 峽(峡) &  狹(狭) &  響(響) &  曉(暁) &  勤(勤) &  謹(謹) &  駈(駆) &  勳(勲) &  薰(薫) &  惠(恵 ) &  揭(掲) \\ \cline{1-11}

 鷄(鶏) &  藝(芸) &  擊(撃) &  縣(県) &  儉(倹) &  劍(剣) &  險(険) &  圈(圏) &  檢(検) &  顯(顕) &  驗(験) \\ \cline{1-11}

 嚴(厳) &  廣(広) &  恆(恒) &  黃(黄) &  國(国) &  黑(黒) &  穀(穀) &  碎(砕) &  雜(雑) & 祉(祉) &  視(視) \\ \cline{1-11}

 兒(児) &  濕(湿) &  實(実) &  社(社) &  者(者) &  煮(煮) &  壽(寿) &  收(収) &  臭(臭) &  從(従) &  澁(渋) \\ \cline{1-11}

 獸(獣) &  縱(縦) &  祝(祝) &  暑(暑) &  署(署) &  緖(緒) &  諸(諸) &  敍(叙) &  將(将) &  祥(祥) &  涉(渉) \\ \cline{1-11}

 燒(焼) &  奬(奨) &  條(条) &  狀(状) &  乘(乗) &  淨(浄) &  剩(剰) &  疊(畳) &  孃(嬢) &  讓(譲) &  釀(醸) \\ \cline{1-11}

 神(神) &  眞(真) &  寢(寝) &  愼(慎) &  盡(尽) &  粹(粋) &  醉(酔) &  穗(穂) &  瀨(瀬) &  齊(斉) &  靜(静) \\ \cline{1-11}

 攝(摂) &  節(節) &  專(専) &  戰(戦) &  纖(繊) &  禪(禅) &  祖(祖) &  壯(壮) &  爭(争) &  莊(荘) &  搜(捜) \\ \cline{1-11}

 巢(巣) &  曾(曽) &  裝(装) &  僧(僧) &  層(層) &  瘦(痩) &  騷(騒) &  增(増) &  憎(憎) &  藏(蔵) &  贈(贈) \\ \cline{1-11}

 臟(臓) &  卽(即) &  帶(帯) &  滯(滞) &  瀧(滝) &  單(単) &  嘆(嘆) &  團(団) &  彈(弾) &  晝(昼) &  鑄(鋳) \\ \cline{1-11}

 著(著) &  廳(庁) &  徵(徴) &  聽(聴) &  懲(懲) &  鎭(鎮) &  轉(転) &  傳(伝) &  都(都) &  嶋(島) &  燈(灯) \\ \cline{1-11}

 盜(盗) &  稻(稲) &  德(徳) &  突(突) &  難(難) &  拜(拝) &  盃(杯) &  賣(売) &  梅(梅) &  髮(髪) &  拔(抜) \\ \cline{1-11}

 繁(繁) &  晚(晩) &  卑(卑) &  祕(秘) &  碑(碑) &  賓(賓) &  敏(敏) &  冨(富) &  侮(侮) &  福(福) &  拂(払) \\ \cline{1-11}

 佛(仏) &  勉(勉) &  步(歩) &  峯(峰) &  墨(墨) &  飜(翻) &  每(毎) &  萬(万) &  默(黙) &  埜(野) &  彌(弥) \\ \cline{1-11}

 藥(薬) &  與(与) &  搖(揺) &  樣(様) &  謠(謡) &  來(来) &  賴(頼) &  覽(覧) &  欄(欄) &  龍(竜) &  虜(虜) \\ \cline{1-11}

 凉(涼) &  綠(緑) &  淚(涙) &  壘(塁) &  類(類) &  禮(礼) &  &  &  &  &  \\ \cline{1-11}

\end{ltabulary}
      
\section{名乗り}
 
\par{ The word 名乗り has several definitions, but before we hone in on the one to be the focus of this lesson, we'll begin by looking at its definitions from the fifth edition of the 広辞苑. }

\par{な-のり【名告・名乗】 \hfill\break
①自分の名・素性などを告げること。また、その名。特に武士が戦場でおこなうもの。 \hfill\break
To tell your name\slash lineage. Or, that name. Particularly what warriors do on the battlefield. \hfill\break
②売物の名を呼びあるくこと。 \hfill\break
To walk around calling out one's things to sell. \hfill\break
③公家および武家の男子が、元服後に通称以外に加えた実名。通称藤吉郎に対して秀吉と名乗る類。 \hfill\break
Real name aside from one's alias after attaining manhood for boys of the Imperial Court and military families. The sort seen with labeling oneself as Hideyoshi versus the alias Fujikichirou. \hfill\break
④ \textbf{漢字の、通常の読みとは別に、特に名前に用いる訓。 }\hfill\break
Kun readings used particularly in names aside from the normal readings of a Kanji. \hfill\break
⑤能や狂言の構成部分の一。登場人物が自己の身分や、そこに来た趣旨などを述べるせりふ。 \hfill\break
A compositional part of Noh and Kyogen. Speech made by characters to tell one's status and intentions on coming. }

\par{ 名乗り in this lesson refers to meaning 4, which refers to special 訓読み in personal names and surnames. These readings are those that have been historically attributed to names and are well established readings that people have chosen for names for a long time. }

\par{ Many names in Japanese, though, are made with just standard readings of characters. In fact, some of the most common surnames are as such. }

\begin{ltabulary}{|P|P|P|P|P|P|P|P|P|P|}
\hline 

鈴木 & すずき & 山田 & やまだ & 山本 & やまもと & 山口 & やまぐち & 川端 & かわばた \\ \cline{1-10}

夏目 & なつめ & 本田 & ほんだ & 澤田 & さわだ & 川崎 & かわさき & 杉本 & すぎもと \\ \cline{1-10}

加納 & かのう & 竹中 & たけなか & 石井 & いしい & 田口 & たぐち & 坂本 & さかもと \\ \cline{1-10}

大坪 & おおつぼ & 田村 & たむら & 片山 & かたやま & 辻本 & つじもと & 石田 & いしだ \\ \cline{1-10}

清水 & しみず & 細野 & ほその & 沼田 & ぬまた & 根本 & ねもと & 野呂 & のろ \\ \cline{1-10}

\end{ltabulary}

\par{\textbf{Spelling Note }: Old characters such as 澤 instead of 沢 is common in names. }

\par{\textbf{Reading Note }: 清水, although an irregular reading, also happens to be a common word spelled this way. So, しみず will be treated as a standard reading. }

\par{ Special readings can be used with other special readings or regular readings. There is no rule that a 名乗り reading must be used with another 名乗り reading. In fact, there is no standardization on the use of 名乗り. 名乗り will be in bold in the following examples. }

\begin{ltabulary}{|P|P|P|P|P|P|}
\hline 

飯田 &  \textbf{いい }だ (Surname) & 新潟 &  \textbf{にい }がた (Place name) & 圭輔 & けい \textbf{すけ }(Personal name) \\ \cline{1-6}

希 &  \textbf{のぞみ }(Personal name) & 秀吉 &  \textbf{ひでよし }(Personal name) & 英雄 &  \textbf{ひで }お (Personal name) \\ \cline{1-6}

\end{ltabulary}

\par{ There is some concern as to how far one can use 名乗り and what should even be considered 名乗り. In Japan parents have complete liberty in how they wish for to read their child's name. However, the overwhelming majority of Japanese believe that parents should only give names to their children that the intended reading can be figured out. Attributing a reading that is not a standard reading or a 名乗り recognized in dictionaries (and most importantly the public) is not popular. }

\begin{center}
 \textbf{Names of Famous Literary Figures }
\end{center}

\begin{ltabulary}{|P|P|P|P|P|P|}
\hline 

ペンネーム & 読み & 本名 & ペンネーム & 読み & 本名 \\ \cline{1-6}

二葉亭四迷 & ふたばてい しめい & 長谷川辰之助 & 森鷗外 & もり おうがい & 森林太郎 \\ \cline{1-6}

与謝野昌子 & よさの しょうこ & 与謝野志よう & 永井荷風 & ながい かふう & 永井壮吉 \\ \cline{1-6}

正宗白鳥 & まさむね はくちょう & 正宗忠夫 & 高浜虚子 & たかはま きょし & 高濱清 \\ \cline{1-6}

武者小路実篤 & むしゃのこうじ さねあつ & 々 & 平塚らいてう & ひらつか らいちょう & 奥村明 \\ \cline{1-6}

高村光太郎 & たかむら こうたろう & 高村光太郎 & 菊池寛 & きくち かん & 菊池寛 \\ \cline{1-6}

室生犀星 & むろう さいせい & 室生照道 & 佐藤湖鳴 & さとう ちょうめい & 佐藤春夫 \\ \cline{1-6}

金子光晴 & かねこ みつはる & 金子安和 & 尾崎士郎 & おざき しろう & 々 \\ \cline{1-6}

川端康成 & かわばた やすなり & 々 & 草野心平 & くさの しんぺい & 々 \\ \cline{1-6}

井上靖 & いのうえ やすし & 々 & 唐十郎 & から じゅうろう & 大鶴義英 \\ \cline{1-6}

樋口一葉 & ひぐち いちよう & 樋口夏子 & 国木田独歩 & くにきだ どっぽ & 国木田哲夫 \\ \cline{1-6}

夏目漱石 & なつめ そうせき & 夏目金之助 & 田山花袋 & たやま かたい & 田山録弥 \\ \cline{1-6}

長谷川時雨 & はせがわ しぐれ & 長谷川ヤス & 北原白秋 & きたはら はくしゅう & 北原隆吉 \\ \cline{1-6}

志賀直哉 & しが なおや & 々 & 斉藤茂吉 & さいとう もきち & 々 \\ \cline{1-6}

宇野浩二 & うの こうじ & 宇野格次郎 & 芥川龍之介 & あくたがわりゅうのすけ & 々 \\ \cline{1-6}

宇野千代 & うの ちよ & 々 & 山本周五郎 & やまもと しゅうごろう & 清水三十六 \\ \cline{1-6}

佐多稲子 & さた いねこ & 佐多イネ & 種田山頭火 & たねだ さんとうか & 種田正一 \\ \cline{1-6}

太宰治 & だざい おさむ & 津島修治 & 島崎藤村 & しまざき とうそん & 島崎春樹 \\ \cline{1-6}

三島由紀夫 & みしま ゆきお & 平岡公威 & 大田翔子 & おおだ しょうこ & 々 \\ \cline{1-6}

吉行淳之介 & よしゆき じゅんのすけ & 々 & 寺山修司 & てらやま しゅうじ & 々 \\ \cline{1-6}

泉鏡花 & いずみ きょうか & 泉鏡太郎 & 石川啄木 & いしかわ たくぼく & 石川一 \\ \cline{1-6}

野上弥生 & のがみ やえこ & 野上ヤヱ & 中里介山 & なかざと かいざん & 中里弥之介 \\ \cline{1-6}

宮本百合子 & みやもと ゆりこ & 宮本ユリ & 萩原朔太郎 & はぎわら さくたろう & 々 \\ \cline{1-6}

山本有三 & やまもと ゆうぞう & 山本勇造 & 横光利一 & よこみつ りいち & 横光利一 \\ \cline{1-6}

梶井基次郎 & かじい もとじろう & 々 & 小林多喜二 & こばやし たきじ & 々 \\ \cline{1-6}

堀辰雄 & ほり たつお & 々 & 坂口安吾 & さかぐち あんご & 坂口炳五 \\ \cline{1-6}

中原中也 & なかはら ちゅうや & 々 & 壺井栄 & つぼい さかえ & 々 \\ \cline{1-6}

火野葦平 & ひの あしへい & 玉井勝則 & 椎名麟三 & しいな りんぞう & 大坪昇 \\ \cline{1-6}

大岡昇平 & おおおか しょうへい & 々 & 島尾敏雄 & しまお としお & 々 \\ \cline{1-6}

柳田國男 & やなぎた くにお & 々 & 三木露風 & みき ろふう & 三木操 \\ \cline{1-6}

有島武郎 & ありしま たけお & 々 & 葛西善蔵 & かさい ぜんぞう & 々 \\ \cline{1-6}

広津和郎 & ひろつ かずろう & 々 & 原民喜 & はら たみき & 々 \\ \cline{1-6}

\end{ltabulary}

\par{\textbf{Reading Notes }: }

\par{1. 志よう = しょう \hfill\break
2. 壮吉 = そうきち \hfill\break
3. 奥村明 = おくむら はる \hfill\break
4. 高村\textquotesingle s real name is read as たかむら みつたろう. \hfill\break
5. 菊池\textquotesingle s real name is read as きくち ひろし. \hfill\break
6. 清水三十六 is read as しみず さとむ. \hfill\break
7. 君威 = きみたけ \hfill\break
8. 石川一 = いしかわ はじめ. \hfill\break
9. 横光利一\textquotesingle s real name is read as よこみつ としかず. \hfill\break
10. 三木操 = みき みさお. }

\par{ Even though you may never read even one novel of all of these literary figures, at least knowing how to correctly read their names will impress natives as you will inevitably encounter their names being invoked for whatever reason. }

\begin{center}
 \textbf{Using Rare 名乗り }
\end{center}

\par{ Using rare 名乗り that still appear in dictionaries may cause problems as well. Consider the character 和, which has the 名乗り reading とし. Even so, it is usually read as わ or かず in names. If you were to name your male child the common name さとし but spell it as 佐和, people may understandably mistakenly read it as the common female name read as さわ, which is normally spelled that way. }

\par{ This is not to say that ambiguous reading of names isn't a problem, which is why it's so difficult to read the names of people you personally don't know correctly for the first time. Gender ambiguity in names has actually been used by people who change the reading of their name when they get a gender change, and people can also choose to leave the reading of their name ambiguous to have slightly more privacy in their identity. }

\begin{center}
\textbf{きらきらネーム } 
\end{center}

\par{ However, there are names called きらきらネーム that are really flashy names with cute (and some would say bizarre readings). For instance, people have named children 光宙 with the reading ぴかちゅう. On this end of the spectrum, how to read the name is not necessarily that difficult. Yet, societal consequences for names such as this is highly debated. These names may also be called DQNネーム. }

\begin{ltabulary}{|P|P|P|P|P|P|}
\hline 

天響(てぃな) & 緑輝(さふぁいあ) & 火星(まあず) & 姫凛(ぷりん) & 七音(どれみ) & 月(あかり) \\ \cline{1-6}

希星(きらら) & 陽(ぴん) & 神生理(かおり) & 星影夢(ぽえむ) & 美々魅(みみみ) & 姫奈(ぴいな) \\ \cline{1-6}

園風(ぞふぃ) & 男(あだむ) & 束生夏(ばなな) & 晴日(はるひ) & 精飛愛(せぴあ) & 宝物(おうじ) \\ \cline{1-6}

\end{ltabulary}
      
\section{止め字}
 
\par{ Choosing what the final letter of a name is--止め字--is an important decision, and the decision is highly based on what the previous sound. A lot of parents decide the final character before thinking about the rest of the name they want to give to their child. }

\begin{center}
 \textbf{止め字 for Girl Names }
\end{center}

\begin{ltabulary}{|P|P|P|}
\hline 

ア & 亜, 明, 愛, 阿, 有, 綾, 安 & 星愛 \\ \cline{1-3}

イ & 衣, 依, 伊, 意 & 真衣, 優衣 \\ \cline{1-3}

エ & 絵, 恵, 江, 慧, 枝, 瑛, 映, \hfill\break
依, 永, 重, 笑, 詠, 栄, 英 & 千恵, 澄江, 沙恵, 沙絵, 真理江, \hfill\break
真梨恵, 春恵 \\ \cline{1-3}

オ & 央, 生, 於, 緒, 桜, 欧 & 伽緒 \\ \cline{1-3}

オリ & 織 & 香織, 沙織, 詩織 \\ \cline{1-3}

カ & 花, 華, 香, 果, 歌, 夏, 加, \hfill\break
馨, 霞, 佳, 鹿, 伽, 茄, 賀, \hfill\break
可, 嘉, 樺 & 涼香, 穂乃果, 美香, 実夏, 智香, 千香, 麗華, \hfill\break
玲花, 怜香, 玲果, 晴香, 陽香, 舞香, 麻衣香, \hfill\break
澄香, 純美花 \\ \cline{1-3}

キ & 樹, 貴, 輝, 希, 紀, 季, 規, \hfill\break
岐, 記, 起, 姫, 木, 祈, 芸, \hfill\break
黄, 来, 稀, 葵, 綺, 嬉, 伎 & 沙樹, 早希, 真貴, 美樹, 玉樹 \\ \cline{1-3}

コ & 子, 鼓, 湖, 胡, 古, 虹, 瑚 & 菜々子, 萌子 \\ \cline{1-3}

サ & 沙, 紗, 砂, 左, 茶,  彩, 咲, 早, 冴, 採, \hfill\break
嵯, 裟, 瑳 & 美沙, 千紗, 理沙, 莉紗, 茉莉沙 \\ \cline{1-3}

ジ & 路 &  \\ \cline{1-3}

ス & 朱, 寿, 須 &  \\ \cline{1-3}

スミ・ズミ & 澄, 純 & 伽純 \\ \cline{1-3}

セ & 瀬, 勢, 世, 星、 静 &  \\ \cline{1-3}

チ & 千, 地, 知, 智, 小, 稚 &  \\ \cline{1-3}

ツ・ヅ & 津, 都, 鶴、 通 &  \\ \cline{1-3}

ツキ・ヅキ & 月 &  \\ \cline{1-3}

ト & 都, 渡, 登, 富, 音 &  \\ \cline{1-3}

ナ & 那, 奈, 菜, 南, 名, 七 & 鈴菜, 鈴奈, 陽菜, 綾奈, 彩那, 絢奈, 佑奈, \hfill\break
夕菜, 優那, 沙奈, 紘奈, 宏奈, 茉奈, 玲菜, 陽菜 \\ \cline{1-3}

ナミ & 浪, 波 &  \\ \cline{1-3}

ネ & 音, 根, 嶺 & 朱音, 鈴音, 桃音 \\ \cline{1-3}

ノ & 乃, 野, 農, 能, 濃 & 志乃, 知野 \\ \cline{1-3}

ハ & 葉, 羽, 波 &  \\ \cline{1-3}

ヒ & 日, 陽, 斐 &  \\ \cline{1-3}

ブ & 舞 &  \\ \cline{1-3}

ホ & 穂, 保, 帆, 歩, 朋 & 奈保, 菜穂, 真帆, 夏帆 \\ \cline{1-3}

マ & 麻, 摩, 磨, 万, 茉, 真, 舞, 雅, 満 &  \\ \cline{1-3}

ミ & 美, 未, 巳, 見, 実, 海, 満, 水, 光, 身、 味, 泉 & 愛美, 留美, 晴見, 裕未, 聡美 \\ \cline{1-3}

メ & 女, 芽 &  \\ \cline{1-3}

モ & 萌, 望 &  \\ \cline{1-3}

ヤ & 耶, 矢, 弥, 夜, 也, 野, 椰 &  \\ \cline{1-3}

ユ & 由, 優, 悠, 友, 愉, 佑, 侑, 有 &  \\ \cline{1-3}

ユキ & 雪, 幸 &  \\ \cline{1-3}

ヨ & 世, 夜, 容, 代, 葉, 与, 予, 依, 陽 & 小夜, 沙世 \\ \cline{1-3}

ラ & 良, 羅, 楽 &  \\ \cline{1-3}

リ & 里, 莉, 理, 梨, 璃, 利, 李, 理, 吏 & 真理, 万理, 茉莉, 真里, 絵里, 恵理, 絵梨, \hfill\break
衣里, 汐里, 芽里, 優里, 友梨 \\ \cline{1-3}

リン & 鈴, 林, 凛 &  \\ \cline{1-3}

ル & 留, 瑠, 流 &  \\ \cline{1-3}

レイ & 礼, 怜, 伶, 嶺, 麗, 玲 &  \\ \cline{1-3}

ワ & 和, 輪, 羽, 環 &  \\ \cline{1-3}

\end{ltabulary}

\begin{center}
 \textbf{止め字 for Boy Names }
\end{center}

\begin{ltabulary}{|P|P|P|}
\hline 

キ & 貴, 輝, 樹 & 祐樹, 優輝, 航輝 \\ \cline{1-3}

ゴ & 吾, 悟 & 信吾, 真悟, 賢吾, 涼吾 \\ \cline{1-3}

シ & 史, 士, 司, 志, 至 & 靖史, 建志, 忠士, 雅司, 慶至 \\ \cline{1-3}

ジ & 自, 二 & 達自, 浩二 \\ \cline{1-3}

スケ & 介, 助, 甫, 輔 & 大輔, 康介 \hfill\break
\\ \cline{1-3}

タ & 太, 汰 & 雄太, 謙太, 隆太, 翔太 \\ \cline{1-3}

ダイ & 大 & 佑大, 航大 \\ \cline{1-3}

タケ & 健 & 剛健 \\ \cline{1-3}

ト & 人, 斗, 登 & 流斗, 勇人, 駿斗, 健人 \\ \cline{1-3}

ドウ & 童 & 義童 \\ \cline{1-3}

ノリ & 則, 典, 紀 & 忠典, 勝紀 \\ \cline{1-3}

ヘイ & 平 & 陽平, 昭平, 哲平, 晃平 \\ \cline{1-3}

マ & 真, 馬, 磨 & 和馬, 優馬, 達馬, 卓磨 \\ \cline{1-3}

ヤ & 也, 矢 & 信也, 徹也, 竜也, 智也 \\ \cline{1-3}

ユキ & 之 & 弘之, 尚之, 敏行, 智之 \\ \cline{1-3}

ラ & 羅 & 森羅 \\ \cline{1-3}

ロウ & 郎, 朗 & 健太郎, 舜太郎, 晃太郎 \\ \cline{1-3}

\end{ltabulary}
\hfill\break
 \hfill\break
    