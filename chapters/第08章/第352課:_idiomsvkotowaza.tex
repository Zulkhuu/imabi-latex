    
\chapter{Idioms V}

\begin{center}
\begin{Large}
第352課: Idioms V: 諺 
\end{Large}
\end{center}
 
\par{ 諺(ことわざ), proverbs, are often short sayings, 言い習わし. A proverb shows some sort of virtue, a common truth that anyone can relate to. A proverb may also show moral, satire, truth, and a whole array of important cultural items in a short and concise way. Free translation is often needed to make them sensible to English speakers. You don't have to be Japanese to learn Japanese proverbs. This is a bigoted argument that you should never listen to. If you're human and are capable of acquiring new information, you can learn Japanese proverbs. }

\par{ The first sentence of each example will be in Japanese script. The second will be the same sentence in かな. The third sentence shows the literal translation and the fourth sentence shows the idiomatic or general translation. If anymore information needs to be made about a given expression, a fifth line will be used. }

\par{\textbf{漢字 to learn this week }: 鳶、鰹、鷲、鷗、鱒、髭、麹、宏、蜃、禽、麿、蟻、雹、豹、賑、藁、虻、蝦、蕊、葱 }
      
\section{諺}
  
\par{案ずるより生(う)むが易し。 \hfill\break
Giving birth is easier than worrying about it. \hfill\break
Fear is often greater than the danger. }

\par{天は自ら助くるものを助く。 \hfill\break
Heaven helps those who helps themselves. }

\par{鰻(うなぎ)の寝床。 \hfill\break
Sleeping grounds of eels. \hfill\break
Long\slash thin building or room. }

\par{全ての道はローマに通ず。 \hfill\break
All roads lead to Rome. }

\par{見ぬが花。 \hfill\break
The thing that does not see is the flower. \hfill\break
1. Reality can't compete with imagination. \hfill\break
2. Prospect is often better than possession. }

\par{寝耳に水 \hfill\break
Water in the ears when one is asleep \hfill\break
Unexpected and shocking }

\par{一年の計は元旦にあり。 \hfill\break
Sum of the year is at New Year\textquotesingle s. \hfill\break
The whole year's plans should be made on New Year's. }

\par{魚心あれば水心 \hfill\break
If a fish has the mind of being one with the water, the water also has the mind of being one with the fish \hfill\break
If you scratch my back, I'll scratch yours. }

\par{一を聞いて十を知る。 \hfill\break
Hear one and know ten. \hfill\break
To understand it all from only one part. }

\par{猫に小判。 \hfill\break
A koban to a cat. \hfill\break
1. Cast pearls before swine. \hfill\break
2. Give valuable to someone who doesn't value it. \hfill\break
3. Caviar to the general. \hfill\break
4. Really big waste of resources. }

\par{\textbf{Cultural Note }: A koban is an oval coin used in currency in Japan made either of gold or silver. }

\par{漁夫の利 \hfill\break
A fisherman's profit \hfill\break
Profiting while others fight }

\par{毒をもって毒を制す。 \hfill\break
To use poison against poison. \hfill\break
To fight fire with fire. }

\par{宝の持ち腐れ \hfill\break
A held jewel rotting \hfill\break
Unused possessions }

\par{尻切れトンボ \hfill\break
A dragonfly with its ass cut off \hfill\break
Unfinished business }

\par{七転び八起き。 \hfill\break
Seven falls and stand up on eight. \hfill\break
1. Keep trying when life knocks you down. \hfill\break
2. Always rise after repeated failures. }

\par{ペンは剣よりも強し。 \hfill\break
The pen is mightier than the sword. }

\par{どんぐりの背比べ。 \hfill\break
Comparing the height of acorns. \hfill\break
To have no outstanding characteristics. }

\par{言わぬが花。 \hfill\break
The thing that does not speak is the flower. \hfill\break
Silence is golden. }

\par{諸刃の剣 \hfill\break
A double-edged sword }

\par{火のないところには煙は立たぬ。 \hfill\break
Smoke doesn't rise from a place that doesn't have fire. \hfill\break
Where there is smoke, there is fire. }

\par{藪を突付いて蛇を出す。 \hfill\break
Poke through a bush and a snake will come out. \hfill\break
1. To stir up trouble for oneself. \hfill\break
2. Let sleeping dogs lie. }

\par{豚に真珠。 \hfill\break
A pearl to a pig. \hfill\break
Cast pearls before swine. }

\par{鶴の声 \hfill\break
A crane's voice \hfill\break
A powerful voice decides an argument }

\par{月夜に提灯(ちょうちん)。 \hfill\break
A paper lantern in a moonlit night \hfill\break
Superfluousness }

\par{井の中の蛙大海を知らず。 \hfill\break
The frog in the well doesn't know of the great ocean. \hfill\break
A person who is ignorant of the world. }

\par{馬の耳に念仏。 \hfill\break
A sutra in a horse's ear. \hfill\break
A nod is as good as a wink to a blind man. }

\par{縁側の下の力持ち。 \hfill\break
A strong person underneath the veranda. \hfill\break
Someone of great assistance in the background. }

\par{ローマは一日にして成らず。 \hfill\break
Rome wasn't built in a day. }

\par{口は禍の元。 \hfill\break
The mouth is the source of disaster. }

\par{良薬口に苦し。 \hfill\break
Good medicines tastes bitter in the mouth. \hfill\break
Good advice is hard to swallow. }

\par{借りてきた猫。 \hfill\break
A borrowed cat. \hfill\break
Being quiet and meek. }

\par{住めば都。 \hfill\break
If residing, the capital. \hfill\break
Wherever you live, you come to love it. }

\par{多々益々弁ず。 \hfill\break
The more, the better. }

\par{河童も川流れ。 \hfill\break
Even an excellent swimmer can get carried down a river. \hfill\break
Even Homer sometimes nods. \hfill\break
Everyone makes mistakes. }

\par{蚤(のみ)の夫婦 \hfill\break
A flee couple \hfill\break
A couple in which the woman is taller than the man }

\par{上には上がある。 \hfill\break
There is a top on the top. \hfill\break
There's always someone better than you. }

\par{石の上にも三年。 \hfill\break
On top of a stone for three years. \hfill\break
One who endures wins through perseverance. }

\par{落花(らっか)枝に帰らず。 \hfill\break
A fallen blossom doesn't go home to its branch. \hfill\break
What's done is done. }

\par{沈む瀬あれば浮かぶ瀬あり。 \hfill\break
If the current sinks, it will rise again. \hfill\break
Life has its twists and turns. }

\par{為せば成る。 \hfill\break
If you do, it will happen. \hfill\break
You can do anything you have your mind set on doing. }

\par{乞食を三日すればやめられぬ。 \hfill\break
If you beg for three days, you won't be able to quit. \hfill\break
Once a good-for-nothing, always a good-for-nothing. }

\par{急がば回れ。 \hfill\break
If hurried, go around. \hfill\break
Slow and steady wins the race. \hfill\break
When in a hurry it is faster to take a roundabout. }

\par{噂をすれば影(がさす) \hfill\break
Shadows if you gossip \hfill\break
Speak of the Devil }

\par{前門の虎、後門の狼 \hfill\break
A tiger at the front gate and a wolf at the back gate. \hfill\break
Out of the frying pan, into the fire. }

\par{弘法筆を択ばず。 \hfill\break
Koubou doesn't choose the brush. \hfill\break
An expert doesn't blame his tools. \hfill\break
A bad workman always blames his tools. }

\par{早起きは三文の得。 \hfill\break
Waking up early gets you three mon. \hfill\break
The early bird catches the worm. }

\par{年寄りの冷や水。 \hfill\break
An old person's cold water. \hfill\break
Old people acting reckless. }

\par{来年のことを言えば鬼が笑う。 \hfill\break
Demons laugh if you talk of next year. \hfill\break
No one knows what tomorrow brings. }

\par{光陰(こういん)矢の如し。 \hfill\break
Time is like an arrow. \hfill\break
Time flies. }

\par{一寸先は闇。 \hfill\break
A sun inward is darkness. \hfill\break
The future is unpredictable. }

\par{身から出た錆 \hfill\break
Rust from the blade \hfill\break
What goes around comes around. }

\par{捕らぬ狸の皮算用をするな。 \hfill\break
Don't count the tanuki skins that you haven't caught yet. \hfill\break
Don't count your chickens before they're hatched. \hfill\break
Note: A tanuki is an indigenous animal in Japan that looks like a raccoon. }

\par{過ぎたるはなお及ばざるが如し。 \hfill\break
Doing too much is the same as doing naught. \hfill\break
Doing too little and too much are equally bad. }

\par{転石苔(てんせきこけ)を生せず。 \hfill\break
A rolling stone gathers no moss. \hfill\break
1. Those who are active make progress. \hfill\break
2. Those who frequently change jobs can't be successful in life. }

\par{塵(ちり)も積もれば山となる。 \hfill\break
When even dust piles up, it becomes a mountain. \hfill\break
1. Little things add up. \hfill\break
2. Little and often fills the purse. \hfill\break
3. Many a little makes a nickle. }

\par{損して得取る。 \hfill\break
Take a loss, then take a gain. \hfill\break
One step back, two steps forward. }

\par{三度目の正直 \hfill\break
The third's honesty \hfill\break
Third time's a charm. }

\par{釈迦に説法 \hfill\break
Teaching Buddhism to Buddha \hfill\break
An idiot trying to teach a know-it-all }

\par{脳ある鷹(たか)は爪(つめ)を隠す。 \hfill\break
A hawk with talent hides its claws. \hfill\break
1. Still water runs deep. \hfill\break
2. A wise man keeps some of his talents hidden. \hfill\break
3. The one who knows most often says the least. \hfill\break
}

\par{頭を隠して尻(しり)を隠さず。 \hfill\break
Hiding one's face but not one's ass. \hfill\break
Don't expose your weak spot when protecting yourself. }

\par{三人寄れば文殊の智慧(ちえ)。 \hfill\break
If three gather, Manjusri wisdom. \hfill\break
Two heads are better than one. \hfill\break
\hfill\break
\textbf{Buddhism Note }: Manjusri is the bodhisattva of Wisdom. }

\par{溺れる者は藁(わら)をも掴む。 \hfill\break
A drowning person will even grasp straw. }

\par{仏の顔も三度。 \hfill\break
Even the face of Buddha three times. \hfill\break
To try a saint's patience. \hfill\break
There are limits to one's patience. }

\par{悪妻は六十年の不作。 \hfill\break
A bad wife is the bad harvest of sixty years. \hfill\break
A bad wife is the ruin of her husband. }

\par{触らぬ神に祟りなし。 \hfill\break
An undisturbed god doesn't wreak havoc. \hfill\break
Let sleeping dogs lie. }

\par{悪事千里を走る。 \hfill\break
Running from a wicked deed 1000 ri. \hfill\break
1. Ill news runs apace. \hfill\break
2. Bad news travels fast. }

\par{枯れ木も山の賑わい \hfill\break
Dead trees are also a mountain's liveliness \hfill\break
Dead trees are better than no trees. }

\par{猫も杓子(しゃくし)も \hfill\break
Even cats and bamboo ladle. \hfill\break
Anybody; anything. \hfill\break
}

\par{旅の恥は掻き捨てて。 \hfill\break
Get rid of the humiliation of travel. \hfill\break
Once over the border you can do anything. \hfill\break
}

\par{青雲の志。 \hfill\break
The will of a blue sky. \hfill\break
Lofty ambitions. }

\par{雲泥の差。 \hfill\break
Separation between clouds and mud. \hfill\break
A vast difference. }

\par{初心忘るべからず。 \hfill\break
We mustn't forget our beginner's spirit. }

\par{ 小人閑居して不善を為す。 \hfill\break
Small people in free time do vice. \hfill\break
An idle brain is the devil's shop. }

\par{二兎を追う者は一兎をも得ず。 \hfill\break
He who chases two rabbits won't catch one. \hfill\break
Do what you can accomplish rather than wasting your energy at trying to do the impossible. }

\par{門前市(もんぜんいち)をなす。 \hfill\break
To produce a market outside a gate. \hfill\break
To have a constant stream of visitors. }

\par{他人の飯を食う。 \hfill\break
To eat another's person's feed. \hfill\break
To experience the hardships of the world everyday. }

\par{情けは人の為ならず。 \hfill\break
Kindness is not for others. \hfill\break
Compassion is not for other people's benefit. }

\par{餅は餅屋 \hfill\break
As for mochi, a mochi store. \hfill\break
Leave things to the experts. }

\par{宵越しの金を持たぬ。 \hfill\break
To not have the money to pass the evening. \hfill\break
To spend one's money as quickly as one earns it. }

\par{青年重ねて来たらず。 \hfill\break
Prime years don't return. \hfill\break
You're only young once. }

\par{腐っても鯛(たい) \hfill\break
Even if it rots, (it's still a) sea bream \hfill\break
If something has value, it doesn't matter what shape it's in. }

\par{石の上にも三年 \hfill\break
Three years on top of a rock \hfill\break
Persistence leads to success. }

\par{李下瓜田( りかかでん) \hfill\break
A melon field below a plum tree. \hfill\break
Leave no room for scandal. }

\par{三つ子の魂百まで \hfill\break
The spirit of a three year old until 100 \hfill\break
As the twig is bent, so grows the tree. }

\par{かわいい子には旅をさせよ。 \hfill\break
Make pretty kids take trips. \hfill\break
Have kids experience challenges of life. }

\par{すずめの涙 \hfill\break
A sparrow's tear \hfill\break
A very small amount }

\par{焼餅焼くとて手を焼くな。 \hfill\break
Don't burn your hand even when you're making yakimochi. \hfill\break
Don't be susceptible to folly because of jealousy. }

\par{李下の冠を正さず。 \hfill\break
Not correcting a crown below a plum tree. \hfill\break
To avoid the appearance of evil. }

\par{腹八分目に医者いらず \hfill\break
You don't need a doctor if your stomach is only 8\slash 10ths full. \hfill\break
It's best to eat in moderation. }

\par{出る杭は打たれる。 \hfill\break
The stake that sticks out will get hammered. \hfill\break
The nail that sticks out gets hammered down. }

\par{知らぬが仏。 \hfill\break
The one who does not know is Buddha. \hfill\break
Ignorance is bliss. }
笑う門には福来(きた)る 。 \hfill\break
Good fortune comes in the laughing gate. \hfill\break
Good fortune and happiness will come to the home of those who smile. 
\par{猿も木から落ちる。 \hfill\break
Even monkeys fall from trees. \hfill\break
1. Everyone makes mistakes. \hfill\break
2. Pride comes before a fall. }

\par{泥棒を捕らえて縄を綯(な)う。 \hfill\break
Catching a thief and tying him up. \hfill\break
Hastening to do something after an incident. }

\par{人は見かけによらぬもの \hfill\break
Don't judge people by their looks. }

\par{花より団子。 \hfill\break
Dumplings than flowers. \hfill\break
1. To prefer substance over style. \hfill\break
2. People are more interested in the practical over the aesthetic. }

\par{井の中の蛙(かわず)大海を知らず。 \hfill\break
A frog in a well doesn't know of the big ocean. \hfill\break
1. To know nothing of the world. \hfill\break
2. Used to encourage someone to get a wider perspective. }

\par{鳴く猫はねずみを捕らぬ。 \hfill\break
Loud cats don't catch mice. \hfill\break
Empty vessels make the most sound. }

\par{濡れぬ先の傘。 \hfill\break
The dry end of an umbrella. \hfill\break
Better safe than sorry. }

\par{弘法(こうぼう)にも筆の誤り。 \hfill\break
Even Koubou made mistakes with the brush. \hfill\break
Anyone makes mistakes. \hfill\break
Even experts have their shortfalls. \hfill\break
\hfill\break
Note: Koubou, posthumous name, was a Buddhist priest famous for his calligraphy. }

\par{寄らば大樹の陰 \hfill\break
Look for a big tree for shade. }

\par{目には目を、歯には歯を。 \hfill\break
An eye for an eye, a tooth for a tooth. }

\par{河童の川流れ。 \hfill\break
A kappa swept away by the river. \hfill\break
Even experts screw up. }

\par{石橋を叩いて渡る。 \hfill\break
Hitting a stone bridge and crossing. \hfill\break
Safety on top of safety. }

\par{立つ鳥跡を濁さず。 \hfill\break
It is common courtesy to clean after yourself. }

\par{蓼(たで)食う虫も好き好き。 \hfill\break
Even knot-weed eating insects have various tastes. \hfill\break
There is no accounting for taste. }

\par{喉元(のどもと)過ぎれば熱さを忘れる。 \hfill\break
If it passes the throat, you forget the heat. \hfill\break
Danger past and God forgotten. }

\par{時は金なり。 \hfill\break
Time is money. }

\par{人のふり見て我がふり直せ。 \hfill\break
Watch other's actions, and fix one's own. \hfill\break
One man's fault is another's lesson. }

\par{生兵法(なまびょうほう)は大怪我の基(もと)。 \hfill\break
Crude tactics is the source of great blunders. \hfill\break
A little learning is a dangerous thing. }

\par{隣の芝生は青い。 \hfill\break
The next door lawn is green. \hfill\break
The grass is always greener on the other side. }

\par{隣の花は赤い。 \hfill\break
The flowers next door are red. \hfill\break
Grass is always greener on the other side. }

\par{郷(ごう)に入っては郷に従え。 \hfill\break
When entering a village, obey it. \hfill\break
When in Rome, do as the Romans do. }

\par{焼け石に水。 \hfill\break
Water on burning rocks. \hfill\break
Something bound to fail due to inadequacies. }

\par{男心と秋の空。 \hfill\break
A man's heart and the autumn sky. \hfill\break
Both a man's heart and the autumn sky are fickle. }

\par{玉に瑕(きず) \hfill\break
A flaw on a gem }

\par{二階から目薬 \hfill\break
Eye drops from the second floor \hfill\break
Something that can't be done no matter what. }

\par{駄目で元々 \hfill\break
Nothing to lose }

\par{仏の顔も三度 \hfill\break
The buddha's face thrice \hfill\break
To try the patience of a saint. }

\par{痘痕(あばた)も笑窪 \hfill\break
Pockmarks and dimples \hfill\break
Love is blind. }

\par{匙(さじ)を投げる。 \hfill\break
To throw the space. \hfill\break
To throw in the towel. }

\par{鴨が葱(ねぎ)をしょってくる。 \hfill\break
A duck comes back with a leek on its back. \hfill\break
A stroke of luck. }

\par{無い袖(そで)は振れぬ。 \hfill\break
Can't wave without a sleeve. \hfill\break
A man can't give what he doesn't have. }

\par{濡れ衣を着せる。 \hfill\break
To make someone wear wet clothes. \hfill\break
To make someone innocent look guilty. }

\par{覆水盆に返らず。 \hfill\break
Spilled water doesn't go back to the tray. \hfill\break
It's no use crying over spilled milk. }

\par{終わりよければ全てよし。 \hfill\break
All's well that ends well. }

\par{泣き面に蜂(はち)。 \hfill\break
A bee to a crying face. \hfill\break
1. One misfortune comes after another. \hfill\break
2. Misfortunes never come alone. \hfill\break
3. When it rains, it pours. }

\par{叩けば埃が出る。 \hfill\break
If you strike it, dust will come out. \hfill\break
Everything has flaws. }

\par{頭隠して尻隠さず。 \hfill\break
To hide your head but not your ass. \hfill\break
To fail to cover up all your bad deeds. }

\par{虻蜂(あぶはち)取らず \hfill\break
Neither catching the horsefly nor the bee. \hfill\break
To accomplish nothing. }

\par{犬猿の仲。 \hfill\break
The relationship between dogs and monkeys. \hfill\break
Natural enemies. }

\par{挨拶(あいさつ)は時の氏神(うじがみ) \hfill\break
Arbitration is time's god. \hfill\break
Arbitration is a gift from the gods. }

\par{千里の道も一歩から \hfill\break
A journey of a thousand miles starts with one step }

\par{惚れてしまえば痘痕(あばた)も笑窪。 \hfill\break
There are scars and dimples even when falling in love. \hfill\break
She who loves an ugly man thinks him handsome. \hfill\break
Love is blind. }

\par{堪忍袋の緒が切れる。 \hfill\break
For the string of a tolerance bag to snap. \hfill\break
To be out of patience. }

\par{麻の中の蓬(よもぎ)。 \hfill\break
Mugwort inside hemp \hfill\break
People become like those around them. }

\par{朱に交われば赤くなる。 \hfill\break
If you mix something with red, it too will become red. \hfill\break
People become like those around them. }

\par{絵に描いた餅(もち)。 \hfill\break
Rice cake drawn in a picture. \hfill\break
A pie in the sky. }

\par{門前の小僧習わぬ経を読む。 \hfill\break
A novice before a gate reading an untaught sutra. \hfill\break
1. To learn something without being taught it. \hfill\break
2. You learn, without realizing it, from what is around you. \hfill\break
3. A young monk outside the gate can read sutra he has never studied. }

\par{前事を忘れざるは後事の師なり。 \hfill\break
Not forgetting the past is the teacher of the future. }

\par{秋茄子(あきなす) は嫁に食わすな。 \hfill\break
Do not feed autumn eggplant to your wife. \hfill\break
Autumn eggplants will reduce fertility and give a woman the chills. }

\par{備えあれば憂いなし。 \hfill\break
If you're prepared, there need to be no worry. }

\par{袖振り合うも他生(たしょう)の縁。 \hfill\break
Sleeves waving together is even karma from a past life. \hfill\break
Even a chance acquaintance is preordained. }

\par{猫の首に鈴をつける。 \hfill\break
To put a bell on a cat's neck. \hfill\break
Don't talk about doing the impossible. }

\par{糠(ぬか)に釘(くぎ)。 \hfill\break
A nail in rice bran. \hfill\break
All is lost that is given to a fool. }

\par{餅は餅屋。 \hfill\break
Rice cake at a rice cake store. \hfill\break
Every man to his trade. }

\par{二足の草鞋(わらじ)。 \hfill\break
Two pairs of sandals. \hfill\break
Having two different jobs. }

\par{猫を追うより皿を引け。 \hfill\break
Take away the plate rather than chase the cat. \hfill\break
Attack the root of a problem. }
    