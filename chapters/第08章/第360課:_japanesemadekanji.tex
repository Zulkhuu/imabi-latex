    
\chapter{国字 (Japanese-made Kanji)}

\begin{center}
\begin{Large}
第360課: 国字 (Japanese-made Kanji) 
\end{Large}
\end{center}
 
\par{ The development of 国字 is nothing less than fascinating. Despite having already borrowed thousands of 漢字 from mainland China and created its own syllabaries, hundreds of 漢字 have also been made by the Japanese. These 国字, also known as 和(製漢)字, have even at times been sent back to be used in China. }

\par{\textbf{Curriculum Note }: There is a resource page with a lot of 国字. This page, however, is coverage concerning 国字 in lesson format. }
      
\section{What are 国字?}
 
\par{ 国字 have come and gone. By the beginning of the Heian Period (794~1185 A.D), around 400 such characters were recorded in the 新撰字鏡(しんせんじきょう), an early 漢和辞典. Of all the 国字 that have been created, only a relatively small percentage are still extremely important characters. However, this is not meant to disregard those that are important. For instance, 働 was made in Japan and is now used in China to mean “work” as well. It has even acquired its own 音読み ドウ. Another important case is 腺, which surprisingly only has the 音読み セン. It was coined in translating medical texts in Dutch to mean “gland”. }

\par{There are have only been 10 国字in the 常用漢字表: 働 (to work), 畑 (field), 込 (to put into, etc.), 枠 (frame), 搾 (to squeeze), 峠 (mountain pass), 塀 (earthen wall), 栃 (horse chestnut), 腺 (gland), 匁 (monme). The latter was taken out in 2010. There are also others as 人名用漢字 such as 笹 (bamboo grass) and 麿 (phonetic "maro" in names). }

\par{ The large majority of 国字, though, are not included in these lists as they are in large point used in writing the names of plants and animals. Examples include 鰯 (sardines), 鴫 (snipe), 樫 (oak tree), 橡 (sawtooth oak), 椚 (sawtooth oak), etc. A lot of this is due to the fact that there are some things that are just not found in China that are in Japan, so the Japanese felt the need to create characters for these things. }
      
\section{Making 国字}
 
\par{ For the most part, 国字 were created with the 六書 principles kept in mind. Here are a few examples. }

\begin{ltabulary}{|P|P|}
\hline 

鱈 & Created with the cod fish that can be caught when it\textquotesingle s snowing in mind. \\ \cline{1-2}

樫 & Created with the hard Japanese oak tree in mind. \\ \cline{1-2}

辻 & Created with crossroads in a path in mind. \\ \cline{1-2}

畑 & Created with slash and burn agriculture in mind. \\ \cline{1-2}

雫 & Created with rain falling down in mind. \\ \cline{1-2}

俥 & Created with a rickshaw in mind. \\ \cline{1-2}

\end{ltabulary}
 
\par{Although the majority of 国字 do not have 音読み, there are rare instances where some were made with漢字 phonetics in mind. }

\begin{ltabulary}{|P|P|}
\hline 

腺 & Created with the radical meat plus 泉 (セン). Thus, 腺\textquotesingle s sound is セン. \\ \cline{1-2}

塀 & Created with the radical ground plus 屏 (ヘイ). \\ \cline{1-2}

\end{ltabulary}
 
\par{Some, though, were created by combining characters. }

\begin{ltabulary}{|P|P|}
\hline 

粂 & 久 (ク)  + 米 (メ). Thus, read as くめ. It is seen in names. \\ \cline{1-2}

麿 & 麻 has the reading マ, and 呂 has the reading ロ. So, 麿 = まろ. It is seen in names. \\ \cline{1-2}

\end{ltabulary}
 
\begin{center}
 \textbf{国訓 } 
\end{center}

\par{ There are even some characters that may also be found in China whose meanings are so different in Japanese that it begs the question whether such characters changed usage once brought to Japan or happened to be made in Japan as well. The two main examples of this are the following characters. }

\begin{ltabulary}{|P|P|P|}
\hline 

Character & Chinese Meaning & Japanese Meaning \\ \cline{1-3}

偲 & Suddenly & To recollect \\ \cline{1-3}

鮎 & Sheatfish & Ayu fish \\ \cline{1-3}

藤 & Vine & Wisteria \\ \cline{1-3}

沖 & To rinse; minor river & Offing\slash open water \\ \cline{1-3}

\end{ltabulary}
      
\section{Important 国字 to Study}
 
\par{ Though there are many easily searchable lists of 国字 online, below are the most important ones that you should study. Some of these are far rarer than others. So, the table will label these characters with a scale of 1-10 for importance with 10 being extremely important and 1 being minimally important. }

\par{ If you are hoping to do well on the most difficult 漢字 tests like the 漢検1級, then try learning them all and more. If you want to just know those that show up frequently in literature, you may want to only learn half of them. }

\begin{ltabulary}{|P|P|P|P|P|P|P|P|}
\hline 

国字 & 重要度 & 意味 & 音訓 & 国字 & 重要度 & 意味 & 音訓 \\ \cline{1-8}

畑 & 10 & Field & はたけ & 働 & 10 & To work & ドウ・はたら(く) \\ \cline{1-8}

込 & 10 & To put into & こ(む) & 枠 & 10 & Frame & わく \\ \cline{1-8}

峠 & 10 & Ridge & とうげ & 搾 & 10 & To squeeze & サク・しぼ(る) \\ \cline{1-8}

腺 & 9 & Gland & セン & 匂 & 9 & Smell & にお(い) \\ \cline{1-8}

辻 & 9 & Crossroads & つじ & 辷 & 8 & To slip & すべ(る) \\ \cline{1-8}

雫 & 8 & Raindrop & しずく & 凧 & 8 & Kite & たこ \\ \cline{1-8}

笹 & 8 & Bamboo grass & ささ & 俥 & 8 & Rickshaw & くるま \\ \cline{1-8}

畠 & 8 & Field & はたけ & 躾 & 8 & Behavior & しつけ \\ \cline{1-8}

喰 & 8 & To eat & くら(う) & 癪 & 8 & Spasm & シャク \\ \cline{1-8}

凪 & 8 & Calm; lull & なぎ & 鰯 & 8 & Sardine & いわし \\ \cline{1-8}

鋲 & 8 & Tack & ビョウ & 榊 & 8 & Sakaki & さかき \\ \cline{1-8}

膵 & 8 & Pancreas & スイ & 襷 & 7 & Sash & たすき \\ \cline{1-8}

枡 & 7 & Measure & ます & 桝 & 7 & Measure & ます \\ \cline{1-8}

噺 & 7 & Chat & はなし & 鯰 & 7 & Catfish & なまず \\ \cline{1-8}

鑓 & 7 & Spear & やり & 栃 & 7 & Horse chestnut & とち \\ \cline{1-8}

褄 & 7 & Skirt; hem & つま & 鱈 & 7 & Cod & たら \\ \cline{1-8}

俣 & 7 & Crotch; fork & また & 鱚 & 7 & Sillago & きす \\ \cline{1-8}

颪 & 7 & Wind from mountains & おろし & 埖 & 7 & Trash & ごみ \\ \cline{1-8}

錺 & 7 & Metal jewelry & かざり & 梺 & 7 & Mountain base & ふもと \\ \cline{1-8}

縅 & 7 & Braid of armor & おどし & 麿 & 7 & Maro & まろ \\ \cline{1-8}

蓙 & 7 & Mat & ござ & 迚 & 6 & A lot & とても \\ \cline{1-8}

鞆 & 6 & Archer arm protector & とも & 簗 & 6 & Girder & やな \\ \cline{1-8}

裃 & 6 & Samurai garb & かみしも & 硴 & 6 & Oyster & かき \\ \cline{1-8}

毟 & 6 & To pull out (hair) & むし(る) & 毮 & 6 & To pull out (hair) & むし(る) \\ \cline{1-8}

裄 & 6 & Sleeve length & ゆき & 俤 & 6 & Vestige & おもかげ \\ \cline{1-8}

栂 & 6 & Hemlock & つが & 叺 & 6 & Straw bag & かます \\ \cline{1-8}

籾 & 6 & Unhulled rice & もみ & 鴫 & 6 & Snipe & しぎ \\ \cline{1-8}

鶇 & 6 & Thrush & つぐみ & 閊 & 6 & To be obstructed & つかえ(る) \\ \cline{1-8}

杢 & 6 & Woodworker & もく & 椙 & 6 & Japanese cedar & すぎ \\ \cline{1-8}

柾 & 6 & Spindle tree & まさき & 梻 & 6 & Grave tree & しきみ \\ \cline{1-8}

聢 & 6 & Certainly & しかと & 杣 & 6 & Timber & そま \\ \cline{1-8}

鳰 & 5 & Grebe & にお & 鵤 & 5 & Grosbeak & いかる \\ \cline{1-8}

〆 & 5 & Letter ending mark & しめ & 怺 & 5 & To endure & こらえ(る) \\ \cline{1-8}

渕 & 5 & Edge & ふち & 鯑 & 5 & Yellow fish; 寿司 eggs & かずのこ \\ \cline{1-8}

鮗 & 5 & Gizzard shad & このしろ & 圷 & 5 & Low ground & あくつ \\ \cline{1-8}

糀 & 5 & Malt & こうじ & 鶎 & 5 & Goldcrest & きくいた \\ \cline{1-8}

椚 & 5 & Horse chestnut & くぬぎ & 橡 & 5 & Horse chestnut & くぬぎ \\ \cline{1-8}

桛 & 5 & Reel; skein & かせ & 鎹 & 5 & Clamp & かすがい \\ \cline{1-8}

嬶 & 5 & Wife & かかあ & 遖 & 5 & Good job! & あっぱれ \\ \cline{1-8}

籏 & 5 & Banner & はた & 鰙 & 5 & Japanese smelt & わかさぎ \\ \cline{1-8}

圦 & 5 & Penstock; sluice & いり & 綛 & 4 & Reel; skein & かすり・かせ \\ \cline{1-8}

鱩 & 4 & Sailfin sandfish & はたはた & 鎺 & 4 & Habaki & はばき \\ \cline{1-8}

籡 & 4 & Temple (clothing) & しんし & 纐 & 4 & Tie-dying & コウ \\ \cline{1-8}

硲 & 4 & Gorge & はざま & 㐂 \hfill\break
& 4 & To be joyful & よろこ(ぶ) \\ \cline{1-8}

魸 & 4 & Catfish & なまず & 椛 & 4 & Colored leaves & もみじ \\ \cline{1-8}

鮴 & 4 & Flathead & ごり & 鞐 & 4 & Fastener & こはぜ \\ \cline{1-8}

艝 & 4 & Sled & そり & 轌 & 4 & Sled & そり \\ \cline{1-8}

鯏 \hfill\break
& 4 & Chub; dace & あさり・うぐい & 饂 \hfill\break
& 4 & Udon & ウン \\ \cline{1-8}

燵 \hfill\break
& 4 & Foot warmer & タツ & 鵥 & 4 & Eurasian jay & カケス \\ \cline{1-8}

蛯 \hfill\break
& 4 & Shrimp & えび & 軈 \hfill\break
& 4 & At last & やがて \\ \cline{1-8}

鯱 \hfill\break
& 4 & Killer whale & しゃち(ほこ) & 簓 \hfill\break
& 4 & Bamboo whisk & っさら \\ \cline{1-8}

鯲 & 3 & Lamprey & どじょう & 鵇 & 3 & Crested ibis & とき \\ \cline{1-8}

鮟 \hfill\break
& 3 & Anglerfish & アン & 鱇 \hfill\break
& 3 & Anglerfish & コウ \\ \cline{1-8}

蚫 & 3 & Abalone & ホウ・あわび & 苆 & 3 & Finish (plaster, etc.) & すさ \\ \cline{1-8}

魞 & 3 & Fish trap & えり & 逧 \hfill\break
& 3 & Ravine & さこ・たに \\ \cline{1-8}

鵆 & 3 & Plover & ちどり & 鯐 & 3 & Young striped mullet & すばしり \\ \cline{1-8}

鶍 & 3 & Crossbill & いすか & 鮖 & 3 & Bullhead & かじか \\ \cline{1-8}

躮 & 3 & Son\slash brat & せがれ & 鮇 & 2 & Charr & いわな \\ \cline{1-8}

鯎 \hfill\break
& 2 & Japanese dace & うぐい & 鰘 \hfill\break
& 2 & Mackerel scad & むろあじ \\ \cline{1-8}

屶 \hfill\break
& 2 & Wide blade knife & なた & 熕 & 2 & Big cannon & おおづつ \\ \cline{1-8}

塰 & 2 & Fisherman & あま & 鱛 \hfill\break
& 1 & Lizardfish & えそ \\ \cline{1-8}

袰 & 1 & Awning & ほろ & 弖 & 1 & The phoneme te & て \\ \cline{1-8}

\end{ltabulary}
    