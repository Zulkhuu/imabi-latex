    
\chapter{Phonology III}

\begin{center}
\begin{Large}
第369課: Phonology III: 連濁 
\end{Large}
\end{center}
 
\par{ Of all the phonological 'rules' to Japanese that gives learners and natives alike headaches the most, 連濁 is at the top of the list. The fact that it is not 100\% known for what it is in Japanese academia may be way there are so many misinterpretations about what it is floating around online. However, what this lesson will try to do is address the phenomenon for what it is and try to outline potential restrictions on it, which is what trips everyone up. }

\par{ It is believed that voicing came about in Japanese by a reduction of the particle の to n in between things and causing voiced to the next consonant after it. }
      
\section{Voicing in Compounding}
 
\par{ When you put two nouns together to create a new noun phrase in Japanese, the first mora of the second element often gets voiced. For instance, if your two nouns are 赤 and 玉, the resulting word is あかだま, not あかたま. There is no fundamental change in meaning to 玉. How it's realized in the resultant phrase is slightly different. }

\par{ It's extremely easy to find many examples of this voicing phenomenon (有声音化). For instance, consider the following. }

\begin{ltabulary}{|P|P|P|P|P|P|}
\hline 

赤紙 & あかがみ & Red paper; draft paper; callup notice & 白酒 & しろざけ & Sweet white sake \\ \cline{1-6}

小皿 & こざら & Small plate & 板金 & いたがね & Sheet metal \\ \cline{1-6}

生え際 & はえぎわ & (Receding) hairline & 堀端 & ほりばた & Side of a moat \\ \cline{1-6}

青白い & あおじろい & Pale; bluish-white & 笑顔 & えがお & Smiling face \\ \cline{1-6}

白髪 & しらが & White\slash grey hair & 和毛 & にこげ & Downy hair \\ \cline{1-6}

\end{ltabulary}

\par{ We see that voicing can occur even when you prefix something to a noun like in 小皿, but it can also show up when the second element is an adjective like in 青白い. 笑顔 shows that the first element may even be truncated (えみ \textrightarrow  え), but this doesn't stop the voicing of 顔. However, it's important to know that these things aren't being pointed out for the heck of it. These are merely observations, and you need to understand that little details like this are important to fully understanding a phonological rule as complex as 連濁 is. }

\par{ For instance, if we look back at 玉 from earlier, we see that it doesn't always become だま in compounds. }

\begin{ltabulary}{|P|P|P|P|P|P|}
\hline 

替(え)玉 & かえだま & Scapegoat & 焼(き)玉 & やきだま & Hot-bulb \\ \cline{1-6}

白玉 & しらたま & White gem & 金玉 & きんたま & Balls (male) \\ \cline{1-6}

勾玉 & まがたま & Comma-shaped bead & 水玉 & みずたま & Polkadot \\ \cline{1-6}

数数玉 & ずずだま & Rosary bead & 赤玉葱 & あかたまねぎ & Red union \\ \cline{1-6}

\end{ltabulary}

\par{  Although not all individual meanings are given for each example word, you should realize that compounding does generate different phrases. You shouldn't just expect 白玉 to mean white ball\slash bead. In speaking of this word, notice how it is しらたま instead of  しろたま, しろだま (though this is an extant form), or しらだま. This, however, is yet another old and complicated phonological sound change. }

\par{ Another thing that has to be rectified with is the last example: 赤玉葱. Just because you see 玉 inside a word doesn't mean you can read it as だま. In reality, the compound is 赤+玉葱. Because there is a voiced sound in 玉葱, the voicing of た is prevented. We'll learn more about this sort of restriction later in this lesson. }

\par{ There used to be a long standing rule that if the last mora of the first element of the compound were voiced that the first mora of the second element couldn't be voiced. Yet, as this small list of examples shows, it doesn't take much time at all to find exceptions (数数玉). }
      
\section{No 連濁 \_ \slash  Initial Word with Voiced Sound}
 
\par{ There has been a long standing rule in Japanese that if the final mora in the first element of a compound that 連濁 should not occur in the second element. However, as we've already seen above, there are plenty of examples to this such as 数数玉. Exceptions below are in bold. }

\begin{ltabulary}{|P|P|P|P|P|P|}
\hline 

萩原 & はぎはら & Reedy field & 風邪引き & かぜひき & Catching a cold \\ \cline{1-6}

禍事 & まがこと・ \textbf{まがごと }& Ill-omen word & 風邪声 &  \textbf{かぜごえ }& Hoarse voice \\ \cline{1-6}

風邪気味 &  \textbf{かぜぎみ }& Touch of a cold & 風邪薬 &  \textbf{かぜぐすり }& Cold medicine \\ \cline{1-6}

水子 &  \textbf{みずご }& Stillborn fetus & 筆箱 &  \textbf{ふでばこ }& Pencil box \\ \cline{1-6}

腕時計 &  \textbf{うでどけい }& Arm watch & 袖口 &  \textbf{そでぐち }& Cuff; armhole \\ \cline{1-6}

帯紐 & おびひも & Obi strap & 革紐 & かわひも & Leather strap \\ \cline{1-6}

\end{ltabulary}

\par{\textbf{Notes }: }

\par{1. It's not surprising that there are some words that pivot between between regular and exceptions. \hfill\break
2. However, sometimes voicing is obligatory because the second element is used as a suffix and the reading is fixed. This is the case for 風邪気味, with ~気味 being the suffix. }
      
\section{上清めば下濁る}
 
\par{ The title of this section is an old Classical Japanese expression that states that if the top is not voiced that the bottom is. This clearly refers to 連濁,  the initial consonant of a latter element of a  compound word becoming voiced. }

\par{ We've seen how this typically works for nouns and sometimes with adjectival phrases, but 連濁 is actually seen in particles that come from nouns such as だけ, ばかり, and ぐらい. }

\par{ Before we really start looking at exceptions, we'll take some time to look at a lot of examples that do have 連濁 so that you might be able to some commonalities. }

\begin{ltabulary}{|P|P|P|P|P|P|}
\hline 

大和魂 & やまとだましい & The Japanese spirit & 砂埃 & すなぼこり & Cloud of dust\slash sand \\ \cline{1-6}

手品 & てじな & Trick (魔法など) & 星空 & ほしぞら & Starry sky \\ \cline{1-6}

夜空 & よぞら & Night sky & 青空 & あおぞら & Blue sky \\ \cline{1-6}

舌鼓 & したづつみ & Smacking one's lips & 後れ毛 & おくれげ & Straggling hair \\ \cline{1-6}

他人事 & たにんごと & Other person's affairs & 花時 & はなどき & Flowering season \\ \cline{1-6}

唸り声 & うなりごえ & Groan; growl & 本棚 & ほんだな & Bookshelf \\ \cline{1-6}

手紙 & てがみ & Letter & 手助け & てだすけ & Help \\ \cline{1-6}

寝床 & ねどこ & Bed & 棒縞 & ぼうじま & Stripes \\ \cline{1-6}

人々 & ひとびと & People & 手袋 & てぶくろ & Glove(s) \\ \cline{1-6}

時々 & ときどき & Sometimes & 面魂 & つらだましい & Defiant look \\ \cline{1-6}

丸顔 & まるがお & Round face & 中底 & なかぞこ & Insole \\ \cline{1-6}

道端 & みちばた & Road side & 居所 & いどころ & Whereabouts \\ \cline{1-6}

炭火 & すみび & Charcoal fire & 秋口 & あきぐち & Start of fall \\ \cline{1-6}

根絶やし & ねだやし & Extermination & 引け時 & ひけどき & Closing time \\ \cline{1-6}

蝙蝠傘 & こうもりがさ & Western umbrella & 八つ手 & やつで & Fatsia \\ \cline{1-6}

逆剃り & さかぞり & Shaving upwards & 出来心 & できごころ & Sudden impulse \\ \cline{1-6}

帰り支度 & かえりじたく & Preparing to go home & 湯豆腐 & ゆどうふ & Boiled tofu \\ \cline{1-6}

小太鼓 & こだいこ & Small drum & 湯飲み茶碗 & ゆのみぢゃわん & Teacup \\ \cline{1-6}

目覚まし時計 & めざましどけい & Alarm clock & 返す返す & かえすがえす & Repeatedly \\ \cline{1-6}

足早に & あしばやに & At a quick pace & 代わる代わる & かわるがわる & Alternately \\ \cline{1-6}

長引く & ながびく & To be prolonged & 目立つ & めだつ & To stand out \\ \cline{1-6}

\end{ltabulary}

\par{\textbf{Notes }: }

\par{1. 連濁 usually involves native phrases, but as you can see, there are plenty of examples where 連濁 occurs when there is a Sino-Japanese element. However, if you look at all such examples, you see that there is still a native element somewhere. \hfill\break
2. There are times when 連濁 ends up producing an adverbial phrase, and these tend to be bizarre. 足早に is somewhat reasonable, but 返す返す and 代わる代わる involving doubling a verbal expression. These are set expressions, but it's interesting how 連濁 shows up in them. \hfill\break
3. As is the case with the last two examples, 連濁 can also happen in a verbal phrase with the first element being non-verbal. }
      
\section{Lyman's Law}
 
\par{ This law, which is attributed to 本居宣長 in Japan and thus usually called 本居宣長の法則 by Japanese people, tries to explain why 連濁 doesn't occur when there is a voiced sound in the second element. The law states that }

\par{\emph{if the latter consonant of the second element is a voiced obstruent, }\emph{連濁 does not occur }. }

\begin{ltabulary}{|P|P|P|P|P|P|}
\hline 

秋風 & あきかぜ & Autumn breeze & 神風 & かみかぜ & Divine wind \\ \cline{1-6}

鳥籠 & とりかご & Bird cage & 紙芝居 & かみしばい & Picture story show \\ \cline{1-6}

狐蕎麦 & きつねそば & Kitsune soba & 一人旅 & ひとりたび & Travelling alone \\ \cline{1-6}

山火事 & やまかじ & Forest fire\slash wildfile & 着物姿 & きものすがた & Dressed in a kimono \\ \cline{1-6}

\end{ltabulary}

\begin{center}
 \textbf{Exceptions and Issues }
\end{center}

\par{ One exception is ふん ${\overset{\textnormal{じば}}{\text{縛}}}$ る (to tie fast). This is because ~ん is thought to make 連濁 easily occur. This is true as we've seen many words so far that seem to point to this fact like 他人事 and 本棚, and these words both have an initial Sino-Japanese element. However, words like 金玉 can't be explained with this law. }

\par{ What's more, the linguist definition of what a voiced sound does not do much justice to figuring out how this voicing phenomenon works. If we ignore that vowels are voiced because they do not hinder or hasten 連濁, then we need to think of what sorts of consonants are voiced. Yes, anything with a ゛ or ゜ is voiced, but so are n, m, r, y, w sounds and ん. }

\par{ As for the initial element, voicing is traditionally supposed to hinder 連濁. But, if ~ん 90\% of the time hastens it, this causes a big problem aside from the many exceptions mentioned earlier. If these sounds are all supposed to hinder 連濁 if present in the second element, then why are there words like the following? }

\begin{ltabulary}{|P|P|P|P|P|P|}
\hline 

風車 & かざぐるま & Windmill & 乱れ髪 & みだれがみ & Unraveled hair \\ \cline{1-6}

山川 & やまがわ & Mountain-side river & 雪玉 & ゆきだま & Snowball \\ \cline{1-6}

風下 & かざしも & Leeward & 縄梯子 & なわばしご & Rope ladder \\ \cline{1-6}

\end{ltabulary}

\par{ 連濁 has occurred in speech whether people have been literate or not. So, it's hard to think that the lack of diacritics on phonetically voiced consonants has brought about words like 風車. However, it is possible for these sounds to be the only ones phonologically treated as voiced sounds in the language. }

\par{ Words like 風下 have no 連濁 because the 'voiced sound' in the final position in the first element prevents it. ${\overset{\textnormal{なわばしご}}{\text{縄梯子}}}$ screws things over, but if you treat 子 as a suffix, then you could say that なわ + はし \textrightarrow  なわばし and that 子 is obligatorily read as ご in the same way ~気味 is read as ぎみ. }
 
\par{The suffix excuse is useful for words like 独り子 (ひとりご) meaning "only child". But, in names ~子 is read as こ: ${\overset{\textnormal{やすこ}}{\text{保子}}}$  ${\overset{\textnormal{きくこ}}{\text{菊子}}}$  ${\overset{\textnormal{じゅんこ}}{\text{淳子}}}$ . However, names should be treated separately altogether, which will be looked at again later in this lesson. }

\begin{center}
 \textbf{~ん \textrightarrow  Obligatory 連濁 }
\end{center}

\par{  There are cases that ~ん makes 連濁 obligatory. This goes along well with what has been said thus far about ~ん making 連濁 easier to occur. The cases that seem to be obligatory are counter expressions. Consider the following. }

\begin{ltabulary}{|P|P|P|P|P|P|}
\hline 

三千 & さんぜん & Three thousands & 三階 & さんがい & Three stories \\ \cline{1-6}

三軒 & さんげん & Three houses & 何階 & なんがい & What floor? \\ \cline{1-6}

\end{ltabulary}

\par{  There are a few problems, though. Reading 三階 as さんがい is the most common way to read this word with differentiating from 三回 (3 times) being the main reason, but さんかい is still a possible reading that some speakers use, and the same goes for 何階 being read as なんかい. This voicing phenomenon has been weakening over time. }

\par{1. 昨年暮れ   No 連濁 \hfill\break
The end of last year }
      
\section{外来語 Hate 連濁}
 
\par{ Loan-words typically have no 連濁. Whether both elements are from other languages or one element is from another language, it just doesn't happen. Now, there could be that one exception hidden in the great lexicon of Japanese--which would not be surprising given all the exceptions that do exist with how 連濁 should work--but we will not worry ourselves about it with this restriction. }

\begin{ltabulary}{|P|P|P|P|}
\hline 

和風ステーキ & Japanese-style steak & 和風トイレ & Japanese-style toilet \\ \cline{1-4}

カラーテレビ & Color TV & レンタカー & Rental car \\ \cline{1-4}

肉カレー & Curry with meat & パトカー & Patrol car \\ \cline{1-4}

真鰈 & Brown sole & ダンプカー & Dump truck \\ \cline{1-4}

\end{ltabulary}

\par{\textbf{Notes }: }

\par{1. Although 真鰈, read as まがれい, is not a loanword, the point is that although a loanword like カレー may have basically the same pronunciation as a native word like 鰈, 連濁 still doesn't occur in the loanword. \hfill\break
2. This typically applies to 漢語 (Sino-Japanese words) as well, but there are plenty of exceptional words with 連濁. A good example is 胸算用 (rough estimation in one's head) read as むなざんよう. \hfill\break
3. If you were to look through every word used in Japanese, you would find exception to this. For instance, 銀ぎせる (silver cigar pipe) exists, though it would almost certainly be 銀きせる today. }

\par{ Take for instance, also, the word 株式会社 (corporation), which is read as かぶしきがいしゃ. 株 may very well be a native word, but the rest is Sino-Japanese, yet 会社 is voiced. The easy excuse is to say that 会社 is treated as a suffix, so 連濁 not surprisingly occurs. }

\par{ However, 式 is also a suffix and there are no examples of it ever becoming voiced as じき. It may be easier to say that another factor that enables 会社 to be voiced here is that it has more affinity as a 'Japanese word'. Thus, 'more Japanese' phonological changes can be expected. In light of the following additional, exceptional examples with the Sino-Japanese word meaning sugar, ${\overset{\textnormal{さとう}}{\text{砂糖}}}$ , this is probably the better analysis: ${\overset{\textnormal{かくざとう}}{\text{角砂糖}}}$ (cube sugar), ${\overset{\textnormal{くろざとう}}{\text{黒砂糖}}}$ (brown sugar), ${\overset{\textnormal{こおりざとう}}{\text{氷砂糖}}}$ (rock candy), and ${\overset{\textnormal{しろざとう}}{\text{白砂糖}}}$ (white sugar). \hfill\break
3. In speaking of Sino-Japanese words, there are times when suffixes sometimes get voiced and sometimes don't, and only convention seems to be a feasible explanation for the readings. }

\begin{ltabulary}{|P|P|P|P|P|P|}
\hline 

案内所 & あんないじょ & Information desk & 停留所 & ていりゅうじょ & Bus\slash tram stop \\ \cline{1-6}

裁判所 & さいばんしょ & Court(house) & 発電所 & はつでんしょ & Power plant \\ \cline{1-6}

\end{ltabulary}
${\overset{\textnormal{}}{\text{}}}$       
\section{Branching of a Compound}
 
\par{ When a word has 2(+) parts, voicing occurs depending on the branching constraint of the word. Branching of a word is often very subjective, which doesn't help bring definity to any phonological rule you can draw to explain restrictions on 連濁, but it does help to a degree. Branching essentially refers to where the main meaning of a compound lies. In other words, branching explains where the semantic weight of an expressions falls on. }

\par{ When a word is left-branched, devoicing may occur at the beginning of the second element. When a word is right-branched, voicing doesn't occur. Of course, there are exceptions to this. }

\begin{ltabulary}{|P|P|P|P|}
\hline 

Phrase & Reading & Definition & Branching \\ \cline{1-4}

目覚まし時計 & めざましどけい & Alarm clock & Left \\ \cline{1-4}

株式会社 & かぶしきがいしゃ & Corporation & Left \\ \cline{1-4}

着物虱 & きものじらみ & Body lice & Left \\ \cline{1-4}

紋白蝶 & もんしろちょう & Cabbage butterfly & Right \\ \cline{1-4}

物差し & ものさし & Ruler & Right \\ \cline{1-4}

早口 & はやくち & Fast-talking & Right \\ \cline{1-4}

砂原 & すなはら & Sandy plain & Right \\ \cline{1-4}

草原 & くさはら & Grassland(s) & Right \\ \cline{1-4}

絵描き & えかき & Artist & Right \\ \cline{1-4}

後腐れ & あとくされ & Future trouble & Right \\ \cline{1-4}

雨降り & あめふり & Rainfall & Right \\ \cline{1-4}

小鳥 & ことり & Small bird & Right \\ \cline{1-4}

尾白鷲 & おじろわし & White-tailed eagle & Right (Exception) \\ \cline{1-4}

紋切り型 & もんきりがた & Hackneyed\slash stereotypical phrase & Right then left \\ \cline{1-4}

\end{ltabulary}

\par{\textbf{Notes }: }

\par{1. Though both 小玉 and 小鳥 have the prefix 小-, because "bead" is already something that is usually of relatively small size, 小- becomes the most important detail in the word and thus makes the word a left-branching word. This also explains why 小部屋 is read as こべや. \hfill\break
2. 偽薬箱 has two possible readings: にせぐすりばこ and にせくすりばこ. However, they don't mean the same thing, and it's because of the branching of the phrase. If the phrase branches leftward (にせぐすりばこ), it means "a box with fake medicine". If the phrase branches rightward (にせくすりばこ), it means "fake medicine-box". \hfill\break
3. How branching works may explain when voicing occurs or doesn't occur with compound verb expressions. If both parts have equal weight, 連濁 doesn't occur and the phrase is treated as a right branching word. If the phrase is left branching and 連濁 consequently occurs. However, there are inconsistencies. }

\par{ For 着替える, you can see both きかえる and きがえる. This is reasonable, but as the sense of かえる being more independent in origin fades away, voicing becomes more common and expected. This is why きがえる is the most common reading. }

\par{ However, there are words like 眠り損なう that is typically read as ねむりそこなう because ~そこなう has been essentially standardized as being read as such. Yet, not all Japanese speakers agree and appear to either have dialectical reasons or reasons based on branching emphasis to say things like the following. }

\par{2. 修一の大いびきは間もなくやんだけれども、信吾は眠り \textbf{ぞこなった }。 \hfill\break
Shuuichi's loud snoring stopped before long, but Shingo missed out on sleep. \hfill\break
From 山の音 by 川端康成. }
修一の大いびきは間もなくやんだけれども、信吾は眠りぞこなった。 \hfill\break
From 山の音 by 川端康成. \hfill\break
      
\section{Dvandva}
 
\par{ Dvandva is a fancy word from Sanskrit that refers to two things being conjoined together to represent an "and" relationship between two words. So, though the compound results in one word, the meaning of the phrase is still on the lines of "X and Y". When this is the case for a Japanese compound, 連濁 does not occur, and it can help the reader decide whether to read something as voiced or not with proper context. }

\begin{ltabulary}{|P|P|P|P|P|P|}
\hline 

山川 & やまかわ & Mountains and rivers & 足腰 & あしこし & Legs and loins \\ \cline{1-6}

枝葉 & えだは & Branches and leaves & 雲霧 & くもきり & Clouds and mist \\ \cline{1-6}

\end{ltabulary}

\par{ An even more interesting word is 目鼻立ち. This expression means "facial features", but the first part, 目鼻, doesn't have 連濁 because of a dvandva relationship. However, the latter part does. So, it's read as めはなだち. }

\par{\textbf{Reading Note }: Do not confuse this 山川 with the one earlier read as やまがわ meaning mountain-side river. }
      
\section{Onomatopoeia}
 
\par{ If you repeat a sound effect, you don't change unvoiced sounds to voiced sounds. You simply repeat as is. Now, there are pairs of onomatopoeia that differ in intensity with voicing indicated higher intensity, but this is clearly not the same thing as 連濁. }

\begin{ltabulary}{|P|P|P|P|}
\hline 

さくさく (さくざく X) & Crispy; crunchy; doing smoothly\slash clearly & ざくざく & Cutting up roughly; lots of coins \\ \cline{1-4}

かさかさ (かさがさ X) & Rustle; dryness & がさがさ & Rummaging; rough feeling \\ \cline{1-4}

\end{ltabulary}
      
\section{Doesn't Matter}
 
\par{ Though branching weight being equal may be a reason to explain why some words may either have 連濁 or not with no change in meaning, it is a fact that there are such words like this. Now, it is usually the case that one variant may be more common than the other, but when you consider speaker variation, this point becomes useless in the long run. }

\begin{ltabulary}{|P|P|P|P|P|P|}
\hline 

親木 & おやき・おやぎ & Stock (tree) & 根方 & ねかた・ねがた & Root; lower part \\ \cline{1-6}

\end{ltabulary}

\par{  There are also times like the following that you may find something usually always voiced with 連濁 not be. In this example, being preceded by ん meant nothing. }

\par{2. 「さっき、谷崎さんが来ました。八時半 \textbf{ころ }です。」と夏子は不器用に言った。 \hfill\break
Natsuko then awkwardly said, "Tanizaki-san came a while ago. It was around 8:30”. \hfill\break
From 山の音 by 川端康成. }
      
\section{連濁 in Names}
 
\par{  Names, personal and place names, are hard to read. Just as is the case with spelling and pronunciation of names in English, there is no certainty of how to read names in Japanese. Sometimes 連濁 occurs and sometimes it doesn't. Sometimes a particular reading is the most common for a given name, but you can always meet someone with the "uncommon" reading. }

\par{ Sometimes voicing helps distinguish place names with the same spelling. Names that are abbreviations of X\{の・ノ・乃・之\}Y are read with Y unvoiced, but you would have to know the history of the place name in question. }

\par{ So, practically speaking, 連濁 in names is random, and you just have to learn how to read the names of the people and places you come across one by one. This is what native speakers have to do, so don't feel bad. }

\begin{ltabulary}{|P|P|P|P|P|P|}
\hline 

高田 & たかだ・たかた & Surname & 大田 & おおた・おおだ & Surname \\ \cline{1-6}

豊橋 & とよはし・とよばし & Place name\slash surname & 旭川 & あさひかわ・あさひがわ & Place name \\ \cline{1-6}

京橋 & きょうばし & Place name & 鶴橋 & つるはし & Place name \\ \cline{1-6}

任三郎 & にんざぶろう & Given name & 池田 & いけだ・いけた・いげだ & Surname \\ \cline{1-6}

\end{ltabulary}
\hfill\break
    