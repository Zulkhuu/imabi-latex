    
\chapter{サ行とハ行の揺れ}

\begin{center}
\begin{Large}
第373課: サ行とハ行の揺れ 
\end{Large}
\end{center}
 
\par{ Have you ever heard しと instead of 人, ひち instead of しち, or some people always pronouncing ひ as if it were し? This convolution of s and h is found all throughout Japan. }
      
\section{S \textrightarrow  H? H \textrightarrow  S?}
 
\par{ S \textrightarrow  H appears throughout Japan as it is a very natural sound that occurs in many world languages. It is possible to make some conclusions on where the speakers are who show this sound change in particular words (as it has not progressed to affecting all words with s), but because it is so easy to come about, there are several places in Japan where it has independently developed. This causes things to be more difficult for our studies as it allows for some reasons to show the sound change not fully taking root in the words it has affected. Thus, there is a true 揺れ in pronunciation of these words in Japanese. }

\begin{center}
 \textbf{Normal but not 'Lazy' }
\end{center}

\par{ The typical Standard Japanese speaker often considers the sound change s to h a sign of lazy speech as h is easier to pronounce. Although ease of articulation is a motivation for sound change, it is not to say that the people who use this sound change are uneducated or lazy in speech. Thousands of speakers of West Japanese dialects who use ~へん are not stupid because of this. If they're stupid, it's because of another reason. Below are phrases from various parts of Japan with s \textrightarrow  h. }

\begin{ltabulary}{|P|P|P|P|}
\hline 

標準語 & 方言 & 標準語 & 方言 \\ \cline{1-4}

捨てる & ふてる (和歌山弁) & そうだな & ほだな (山形弁) \\ \cline{1-4}

\end{ltabulary}

\par{\textbf{Dialect Note }: Phrases are not necessarily limited to any said dialect. }

\begin{center}
 \textbf{し \textrightarrow  ひ }
\end{center}

\par{ Specifically, し and ひ are thought to be mixed up most frequently. They are very close in articulation, and it is not surprising that there are dialects with し \textrightarrow  ひ and dialects with ひ \textrightarrow  し or both. This usually only affects certain phrases, but it is fair to say that the scale was traditionally far larger. }

\begin{ltabulary}{|P|P|P|P|}
\hline 

標準語 & 方言 & 標準語 & 方言 \\ \cline{1-4}

布団をしく & 布団をひく & しちや (質屋) & ひちや (大阪・名古屋・京都・広島・松山) \\ \cline{1-4}

お七夜 & おひちや & しつこい & ひつこい \\ \cline{1-4}

\end{ltabulary}

\par{ There are even place names with s to h as part of the official name. 七福旅館 in 鳥取市 is read as ひちふくりょかん and 七宗町 in 岐阜県 is ひちそうちょう. }

\begin{center}
 \textbf{ひ \textrightarrow  し }
\end{center}

\par{ True 江戸っ子 are known for saying おしさま (お日様)はしがし(東)に昇る or しばち(火鉢)にし(火)を入れる. In reality, though, it would be hard to find anyone who still sounds like this in 東京. It's safe to say that for this part of the country, this is a traditional feature of pronunciation that some people either purposely do or is only used by people in older generations. Nevertheless, this sound change does show up throughout East Japan. }

\begin{ltabulary}{|P|P|P|P|}
\hline 

標準語 & 方言 & 標準語 & 方言 \\ \cline{1-4}

おひさしぶりですね & おしさしぶりですね & ひやっこい & しゃっこい・しゃっけ (東北弁) \\ \cline{1-4}

\end{ltabulary}

\begin{center}
 \textbf{Well-Known 関西弁 Examples of S \textrightarrow  H }
\end{center}

\par{ 関西 dialects are frequently known as being where "s to h" is most prevalent. This is no doubt due to the fact that it has phrases such as はん (\textrightarrow  さん), へん (from the せん in ません), はる (from the さる in なさる), etc. There are even parts of the region where both へん and せん are used. Sometimes it may be hard to tell what's going on when a phrase is contracted further like in それなら \textrightarrow  そんなら \textrightarrow  ほんなら \textrightarrow  ほな. But, don't let this get the best of you. }

\par{1. }

\par{標準語: かまわないよ。 \hfill\break
関西弁: かまへん、かまへん }

\par{2. }

\par{標準語: 50万円、貸してくれない? \hfill\break
関西弁: 50万円、貸してくれへん? }

\par{3. }

\par{標準語: それなら買うよ。 \hfill\break
関西弁: ほんなら買うわ。 }

\par{4. }

\par{標準語: それから、次に何をするの。 \hfill\break
関西弁: へてから、次何すんねん。 }

\begin{center}
 \textbf{No True サ行圏 Or ハ行圏 }
\end{center}

\par{ In the end, it's hard to say where the サ行圏 and ハ行圏 of Japan is because at the individual dialect level, there are areas like Ibaraki Prefecture where words may be said with either s or h. In much of North and East Japan, ひ often becomes し. Yet, returning to Ibaraki, you can see す \textrightarrow  ひ. In the far north in 津軽弁, s widely becomes h. So, you get へんへ for 先生. }

\par{ So, it is not as simple as saying anything southwest of Osaka uses h in the stereotypical phrases like ~はん or ひちや. If this can appear independently elsewhere in places like Ibaraki, then it shows how natural this sound change is. }

\begin{center}
 \textbf{出雲弁's Abnormality }
\end{center}

\par{ 関西弁 doesn't have near as much s \textrightarrow  h instances as 出雲弁, although most likely did in the past. 出雲弁 is a rather different from the dialects spoken around it. It has phrases like 布団をひく (which is said by many speakers around the country) and oppositely 車にしかれる (for 車にひかれる). Intriguingly, this dialect also allows ひ to become ふ. }

\par{5. }

\par{標準語: そんならゆきましょか。 \hfill\break
出雲弁: ほなゆきまひょか。 }

\par{6. }

\par{標準語: 灯が灯る。 \hfill\break
出雲弁: ${\overset{\textnormal{ふ}}{\text{灯}}}$ が ${\overset{\textnormal{とぼ}}{\text{灯}}}$ ー。 }

\par{7. }

\par{標準語: 久しぶりだね。 \hfill\break
出雲弁: ${\overset{\textnormal{ふさ}}{\text{久}}}$ しぶーだね。 }

\par{ なさる in West Japan dialects is usually a form of なはる. Consider the chart below. }

\begin{ltabulary}{|P|P|P|P|P|P|P|}
\hline 

 & 未然形 & 連用衛 & 終止形 & 連体形 & 已然形 \hfill\break
仮定形 & 命令形 \\ \cline{1-7}

共通語 & なさら & なさり & なさる & なさる & なされ(+ば) & なされ \\ \cline{1-7}

関西弁 & なはら & なはり & なはる & なはる & なはれ(+ば) & なはれ \\ \cline{1-7}

出雲弁 & なはら & なはい & なはー & なはー & なはら・なはりゃ & なはい \\ \cline{1-7}

\end{ltabulary}

\par{ The conjugations for 関西弁 and 出雲弁 are the same aside from the fact that the latter's conjugations have other sound changes. }
    