    
\chapter{音訓 (ON \& KUN Readings)}

\begin{center}
\begin{Large}
第354課: 音訓 (ON \& KUN Readings) 
\end{Large}
\end{center}
 
\par{ Due to the unique fixation of 漢字 to Japanese, a complex set of readings with different origins has produced the hardest method of reading 漢字 of all languages that use 漢字. This lesson will delve deeper into the 音訓 puzzle. A firm understanding of how to read 漢字 affects how efficiently you can acquire new words and new characters. }
      
\section{漢字 \& Meaning}
 
\par{ 漢字 are called ideograms, symbols that display meaning. However, their shapes alone don\textquotesingle t produce meaning itself. Words are produced by the human mind, and these words are attributed to these highly crafted arrangements of strokes we call 漢字. After all, only a human can understand 水 as water. This 漢字 to Mandarin Chinese speakers has the sound Shuǐ. This is what these people think when they see water and this symbol. However, in Japanese, there is more than one reading to this character. There is the native word for water mizu. There is also the 音読み from Chinese SUI. Both are attributed to 水. }

\par{ Thus, in Modern Japanese, there are two language units distinguished in reading 漢字. 音, the sounds of characters as borrowed from various stages of Chinese languages, and 訓, native vocabulary attributed to these foreign characters. }
      
\section{音読み}
 
\par{ Although the history of 音読み is really complicated and the Japanese sound system has also changed over time, 音読み were in large part assimilated to the existing sound constraints of Japanese. Over time, this system of two kinds of readings, with characters often having more than one of each, has solidified and become what it is today. }

\par{As you have encountered numerous times no doubt in your Japanese studies, many 漢字 have more than one 音読み. This is because, as mentioned earlier, they come from different stages of Chinese. All words with 音読み are Sino-Japanese unless simply used for phonetic purposes. For example, although the word 世話 is pronounced as SEWA with 音読み, it is actually a native word. }

\par{\textbf{The Four Kinds of }\textbf{音読み }}

\par{ There are four kinds of 音読み generally recognized. Unlike other languages utilizing Chinese characters, older readings did not get totally replaced with new ones in Japanese. Rather, they are used together in a complex amalgam. As far as 音読み are concerned, many readings can be guessed on because of phonetics in the characters. As you will see, though, many shifts have occurred in these readings. This has all lead to many characters having the same readings. }

\par{\textbf{呉音: Sounds of Wu }}

\par{ The first wave of readings are called 呉音. These readings are the oldest and come from the Wu Dynasty. They entered Japanese during the 5 th to 6 th centuries. Many Buddhist and Ritsuryou System (律令) terms have these readings as well as many for basic vocabulary like SAN for 山. }

\par{\textbf{Historical Note }: The 律令 was the fundamental set of codes by which constituted in large part the early Japanese political system. }

\begin{ltabulary}{|P|P|P|P|P|P|}
\hline 

世間 & SEKEN & World; society & 末期 & MATSUGO & Hour of death \\ \cline{1-6}

男女 & NAN'NYO & Men and women & 正体 & SHOUTAI & True identity \\ \cline{1-6}

経文 & KYOUMON & Sutra & 下人 & GENIN & Menial \\ \cline{1-6}

極楽 & GOKURAKU & Paradise & 正直 & SHOUJIKI & Honest \\ \cline{1-6}

成就 & JOUJU & Success & 会釈 & ESHAKU & Nod; salutation \\ \cline{1-6}

工夫 & KUFUU & Scheme & 祇園 & GION & Gion \\ \cline{1-6}

白衣 & BYAKUE & White robe & 金色 & KONJIKI & Gold color \\ \cline{1-6}

荼毘 & DABI & Cremation & 供米 & KUMAI & Rice offering \\ \cline{1-6}

\end{ltabulary}

\par{\textbf{Word Notes }: }

\par{1. NAN'NYO is an old-fashioned reading of 男女. \hfill\break
2. Gion is the entertainment district of 京都. \hfill\break
3. BYAKUE is an old-fashioned reading of 白衣. }

\par{ There were many alterations to the actual pronunciations used at the time in Wu China. For instance, final –t became –chi such as NICHI for 日. –ng, which didn\textquotesingle t exist in Japanese, was dealt with randomly, sometimes being replaced with a back vowel like I or U. Other times, it was dropped altogether. However, in rare instances this final –ng became preserved as a medial g. Ex. SU \textbf{G }O ROKU 双六. }

\par{\textbf{Historical Notes }: }

\par{1. These readings are also sometimes called 対馬音 つしまおん・つしまごえ and 百済音 くだらおん・くだらごえ because it is said that a Paekje monk by the name of Houmyou 法名 read the Vimalakirti Sutra (維摩経) with 呉音 at Tsushima. \hfill\break
2. Also, it is believed by some due to no actual written evidence that these readings may have actually been Korean alterations to some form of Chinese as they were introduced to the Japanese via the Korean peninsula. \hfill\break
3. Paekje, which is called Kudara by the Japanese, is one of the ancient Korean kingdoms. }

\par{\textbf{漢音: The Sounds of Han China }}

\par{ 漢音 are readings borrowed during the 7 th to 8 th centuries in the Sui and Tang Dynasties. They were sent to Japan via emissaries. They were heavily propagated as the correct pronunciation of characters. Despite efforts by elites to eradicate呉音, they could not be taken out of already commonly used daily words with them. Thus, we have many characters to this day with more than one 音読み. }

\par{ Several systematic changes can be seen when comparing 呉音 and 漢音. The most important phenomenon is denasalization that occurred in Chinese. In this sound change, the latter half of a nasal sound would become an oral sound. This would be reflected over into Japanese in the following ways. }

\begin{ltabulary}{|P|P|P|P|P|P|}
\hline 

馬 & MA \textrightarrow  BA & 日 & NICHI \textrightarrow  JITSU & 美 & MI \textrightarrow  BI \\ \cline{1-6}

\end{ltabulary}

\par{ There are exceptional cases where nasal endings were retained during the borrowing process such as NEN for 年. }

\par{ Another important phenomenon is the devoicing of voiced initials. }

\begin{ltabulary}{|P|P|P|P|P|P|}
\hline 

定 & JOU \textrightarrow  TEI & 従 & JUU \textrightarrow  SHOU & 勤 & GON \textrightarrow  KIN \\ \cline{1-6}

\end{ltabulary}

\par{ It is important to remember that sound changes have also occurred in Japanese. So, these are the modern renderings of the readings. For instance, JOU for 定 was actually traditional spelled in かな as ぢやう. This would have reflected the pronunciation at the time. Also, keep in mind that despite efforts to propagate 漢音, not all such readings were able to displace 呉音. }

\par{ For instance, the 漢音 of 和 and 話 is KWA, but this reading never made it to the present. Other similar instances did in this k-insertion. For example, 会\textquotesingle s E \textrightarrow  KAI. }

\par{It would take too long to go through every detail for every 漢字. So, we will now switch focus to how 漢音 have become used. The overwhelming majority of 漢字 have 漢音, and they were further propagated during the Meiji Restoration by being used to coin hundreds of words. Although 呉音 continued to have overwhelming influence in Buddhist texts, there are some instances were certain sects read particular sutras in 漢音. For instance, the 理趣経 of the 真言宗 is read with 漢音. So, the famous starting of sutras 如是我聞 (Thus, I hear) is pronounced as JOSHIGABUN rather than the typical NYOZEGAMON, which uses 呉音. }

\par{Example Words: }

\begin{ltabulary}{|P|P|P|P|P|P|}
\hline 

末期 & MAKKI & Closing years; terminal & 経世 & KEISEI & Conduct of state affairs \\ \cline{1-6}

文人 & BUNJIN & Literary person & 土地 & TOCHI & Land \\ \cline{1-6}

幕府 & BAKUFU & Shogunate\slash bakufu & 内裏 & DAIRI & Imperial palace \\ \cline{1-6}

女児 & JOJI & Female child & 男女 & DANJO & Men and women \\ \cline{1-6}

白衣 & HAKUI & White robe & 老若 & ROUJAKU & Young and old \\ \cline{1-6}

口頭 & KOUTOU & Oral & 左右 & SAYUU & Left and right \\ \cline{1-6}

遠近 & ENKIN & Far and near & 兵隊 & HEITAI & Soldier \\ \cline{1-6}

達成 & TASSEI & Achievement & 質量 & SHITSURYOU & Mass \\ \cline{1-6}

\end{ltabulary}

\par{\textbf{Word Notes }: }

\par{1. DANJO is the normal reading of 男女. \hfill\break
2. Notice how the different readings of 末期, まつご and まっき, have different meanings. \hfill\break
3. HAKUI is the normal reading of 白衣. \hfill\break
4. 老若 may also be seldom read with the 呉音 ROUNYAKU. }

\par{\textbf{唐音: T\textquotesingle ang Sounds }}

\par{ Although they weren\textquotesingle t actually borrowed during the T\textquotesingle ang Dynasty, 唐音 were borrowed later from the Kamakura to the Edo Periods. They are generally rare and seen mainly in Zen Buddhism, trade, food, and furnishing terminology. }

\par{\textbf{Terminology Note }: These sounds are also called 宋音・唐宋音. }

\begin{ltabulary}{|P|P|P|P|P|P|}
\hline 

布団 & FU TON & Futon & 納戸 &  NAN DO & Barn \\ \cline{1-6}

扇子 & SEN SU & Folding fan & 北京 &  PEKIN & Beijing \\ \cline{1-6}

南京 & NAN KIN & Nanqing & 瓶 &  BIN & Bottle \\ \cline{1-6}

行脚 &  ANGYA & Pilgrimage & 普請 & FU SHIN & Construction; community activities \\ \cline{1-6}

箪笥 & TAN SU & Dresser & 杜撰 &  ZUSAN & Sloppy \\ \cline{1-6}

団栗 &  DON GURI & Acorn & 炭団 &  TADON & Charcoal briquette \\ \cline{1-6}

餡 &  AN & Read bean paste & 湯麺 &  TAN MEN & Tang mian \\ \cline{1-6}

繻子 &  SHUSU & Satin & 楪子 &  CHATSU & Type of lacquerware \\ \cline{1-6}

杏子 &  ANZU & Apricot & 椅子 & I SU & Chair \\ \cline{1-6}

石灰 &  SHIKKUI & Plaster & 胡散 &  U SAN & Suspicious \\ \cline{1-6}

胡乱 &  URON & Fishy & 橘飩 &  KIT TON & Mashed sweet potatoes \\ \cline{1-6}

暖簾 &  NO REN & Shop entrance curtain & 饅頭 & MAN JUU & Manjuu \\ \cline{1-6}

湯湯婆 & YU TANPO & Hot water bottle & 看経 & KANKIN & Silent reading of sutra \\ \cline{1-6}

明 &  MIN & Ming Dynasty & 鈴 &  RIN & Bell \\ \cline{1-6}

提灯 &  CHOUCHIN & Paper lantern & 茴香 &  UIKYOU & Fennel \\ \cline{1-6}

\end{ltabulary}

\par{\textbf{Word Note }: SHIKKUI is now normally spelled as 漆喰. 石灰 is typically read with the 漢音 SEKKAI to mean "lime" as in limestone. }

\par{\textbf{慣用音: Traditional Sounds } }

\par{ 慣用音 are mishap readings that have somehow changed since being incorporated into Japanese. Important examples include the following. }

\begin{ltabulary}{|P|P|P|P|}
\hline 

立 & RYUU \textrightarrow  RITSU & 輸 & SHU \textrightarrow  YU  \\ \cline{1-4}

攪拌 (Agitation) & KOUHAN \textrightarrow  KAKU HAN & 執 & SHUU \textrightarrow  SHITSU \\ \cline{1-4}

洗滌 (Washing) & SENDEKI \textrightarrow  SEN JOU & 涸渇 (Drying up) & KOKUKATSU \textrightarrow  KO KATSU \\ \cline{1-4}

\end{ltabulary}

\par{ \textbf{慣用音 Classification Controversy } }

\par{ Currently, any reading that deviates from 呉音, 漢音, and 唐音 are considered 慣用音. This means that several different situations are lumped together. }

\par{\textbf{Change Influenced by }\textbf{訓読み } }

\par{ At times, 音読み were manipulated due to 訓読み. For example, 早速 should be pronounced SOUSOKU but it is instead pronounced as SASSOKU, with the sa- coming from a native prefix seen in words like sanae 早苗 (rice seedlings) and saotome 早乙女 (young woman). There is also 奥意 OKUI (true intention) and 奥地 OKUCHI (hinterland) which should be OUI and OUCHI respectively, but the 音読み OU\slash IKU has changed to OKU for most speakers. Thus, these extraordinary readings have become labeled as 慣用音. }

\par{\textbf{Changes to Final – }\textbf{フ }}

\par{ Many 音読み in traditional orthography ended in ふ. It is agreed by most scholars that f used to be p in Old Japanese, and eventually became ɸ by Middle Japanese. Due to this effect, readings such as てふ for 蝶 became ちょう. }

\par{ Plosive and fricative sounds in the final position were preserved with 促音 in words like 合戦 KAS SEN (match; engagement), 入声 NIS SHOU (initial tone), and 法度 HAT TO (ordinance). On the other hand, these finals were typically replaced with う. Thus, you get many common words like 合成 GOU SEI (synthesis), 甲子園 KOU SHIEN (Koshien), and 入賞 NYUU SHOU (winning a prize). There are also instances where ふ \textrightarrow  つ. For instance, 立 RYUU \textrightarrow  RITSU, 圧 OU \textrightarrow  ATSU, 執SHUU \textrightarrow  SHITSU. These changes from the original readings are still classified in many 漢和辞典 as 慣用音. }

\par{\textbf{音読み Confusion }}

\par{ When a 漢字 has more than one 音読み and they have become specialized for particular meanings and the expected reading is not used, this accidental misuse of a reading is deemed to be a 慣用音. One of the best examples of this is 罷免 (discharge). 罷 is supposed to be read as HAI when the character is used to mean “to tire” and HI when used to mean “to stop”. Yet, instead of being read as HAIMEN, the word is read as HIMEN. Thus, this usage of HI is classified as a 慣用音. }

\par{\textbf{Readings with Unknown Origins }}

\par{ There are also readings with unknown origins. These readings, despite obviously coming from Chinese variants, are lumped into the term 慣用音. }

\par{ For example, 茶 is typically read as CHA, but this reading is not a 呉音, 漢音, or 唐音. In order, those readings are actually DA, TA, and SA. So, CHA is classified as a 慣用音. Another odd example is the reading PON for 椪. It is typically believed to be from Taiwanese, and all of its other readings are unknown. Thus, it is classified as a 慣用音. }

\par{\textbf{Lots of Homophones } }

\par{ Due to the fact that so many waves of 音読み have been incorporated into Japanese with heavy simplification due to assimilation into the Japanese sound system, many words sound alike. Just typing こう results in dozens of options. There are a lot of homophones in Mandarin Chinese, but at least there is more variety in its phonology. Thankfully, the majority of homophones are avoided in the spoken language. Of course, there are exceptions. For example, the reading EKI for 駅, 益, 液, 易, and 役 are commonly used. }
      
\section{訓読み}
 
\par{ 訓読み have not been immune to change. After all, if a language doesn't change, it\textquotesingle s dead. 訓読み are from native words, and the reason they exist is because the Japanese already had their own language when 漢字were introduced. Although there are plenty of 訓読み that have essentially not changed at all over the course of time such as yama 山 and kusa 草, others like ${\overset{\textnormal{あや}}{\text{危}}}$ うい \textrightarrow  ${\overset{\textnormal{あぶ}}{\text{危}}}$ }

\par{ない have. }

\begin{center}
 \textbf{Sino-Japanese 訓読み }
\end{center}

\par{ There are actually some words of Chinese that were borrowed way before 漢字 were, and they are so integral in the language that they are viewed as native words. Important examples include the following. }

\begin{ltabulary}{|P|P|P|P|P|P|}
\hline 

馬 & うま & 銭 & ぜに & 梅 & うめ \\ \cline{1-6}

\end{ltabulary}

\begin{center}
 \textbf{Number of 訓読み a Character }
\end{center}

\par{ The number of 訓読み a character usually has also changed over time. As time progressed, most characters have come to have only one 訓読み. However, it still doesn't take much effort to find blatant anomalies like 生, which has lots of readings. }

\begin{ltabulary}{|P|P|P|P|P|P|P|P|P|}
\hline 

生きる & いきる & To live & 生- & なま- & Raw & 生む & うむ & To give birth \\ \cline{1-9}

生す & むす & To grow (moss) & 生す & なす & To give birth & 生う & おう & To spring up \\ \cline{1-9}

生\{える・やす\} & はえる・はやす & To cultivate & 生る & なる & To bear seed & 生 & き & Undilated \\ \cline{1-9}

生 & うぶ- & Innocent; birth- & 生 & -ふ & Thick growth &  &  &  \\ \cline{1-9}

\end{ltabulary}

\par{  Just to think, there are more irregularities and 音読み to keep in mind as well. }

\begin{center}
 \textbf{Kinds of Words with 訓読み }
\end{center}

\par{ Typically, the majority of 訓読み are used to write independent words, with all of the examples such far being just that. }

\begin{ltabulary}{|P|P|P|P|P|P|P|P|P|P|}
\hline 

雨 & あめ & 雲 & くも & 日 & ひ & 火 & ひ & 喜び & よろこび \\ \cline{1-10}

歌う & うたう & 思い & おもい & 高い & たかい & 花 & はな & 鼻 & はな \\ \cline{1-10}

\end{ltabulary}

\par{ There are some that are etymologically compound words but have since been fixated to the point that they are treated as simple words. }

\begin{ltabulary}{|P|P|P|P|P|P|P|P|}
\hline 

湖 & みずうみ & 志 & こころざし & 快い & こころよい & 瞼 & まぶた \\ \cline{1-8}

炎 & ほのお & 橘 & たちばな & 睫 & まつげ & 雷 & かみなり \\ \cline{1-8}

\end{ltabulary}

\par{There are also instances where several characters have received the same 訓読み. This shouldn't be surprising given how many 漢字 exist. The hardest part about this is that options typically have specific nuances, which causes correct spelling to be more difficult. }

\begin{ltabulary}{|P|P|}
\hline 

 \textbf{はかる }& 計る・測る・量る・諮る・図る・謀る \\ \cline{1-2}

 \textbf{とる }& 取る・捕る・執る・採る・撮る・獲る・摂る・盗る・録る \\ \cline{1-2}

\end{ltabulary}

\par{Today, 訓読み are generally not used to write ancillary words such as particles and other function words. However, there are exceptions and many such words still do have 漢字 spellings. }

\begin{ltabulary}{|P|P|P|P|P|P|}
\hline 

迄 = まで & 也 = なり & 御 = おん・お・み- & 程 = ほど & 秤 = ばかり & 位 = くらい \\ \cline{1-6}

哉 = かな & 幾 = いく- & に就いて = について & 様に = ように & 事 = こと & 物 = もの \\ \cline{1-6}

\end{ltabulary}

\par{\textbf{Word Notes }: }

\par{1.  なり is the Classical copula. \hfill\break
2. ほど, ばかり, and くらい as particles are usually not written in 漢字. \hfill\break
3. 事 and 物 when used for more grammatical purposes are typically not written in 漢字. \hfill\break
4. Other speech modals like について and ように are usually only written in 漢字 in formal writing. }

\begin{center}
 \textbf{Loanwords with 訓読み }
\end{center}

\par{${\overset{\textnormal{くんよ}}{\text{訓読}}}$ みmay also be loanwords and can be as 5 morae long, often caused by compound words as mentioned before. The reason why loanwords may sometimes be classified as 訓読み is because a lot of characters for things like measurements and stuff were coined during the Meiji Restoration. These, thus, would be 国字 (Japanese-made characters) and would otherwise be expected to not have 音読み. }

\par{Meters:「粉(デシメートル)」、「糎(センチメートル)」、「粍(ミリメートル)」 ― 「籵(デカメートル)」、「粨(ヘクトメートル)」、「粁(キロメートル)」 }

\par{Liters:「竕(デシリットル)」、「竰(センチリットル)」、「竓(ミリリットル)」 ― 「竍(デカリットル)」、「竡(ヘクトリットル)」、「竏(キロリットル)」 }

\par{Grams: 「瓰(デシグラム)」、「甅(センチグラム)」、「瓱(ミリグラム)」 ― 「瓧(デカグラム)」、「瓸(ヘクトグラム)」、「瓩(キログラム)」。「瓲(トン)」はおまけ。 }

\begin{center}
 \textbf{送り仮名 }
\end{center}

\par{ Many 訓読み have 送りがな requirements. Despite government efforts, though, 送りがな is still rather random for lots of words. For instance, wakaru is typically spelled as 分かる. However, it can also be spelled as 分る, 判る, 解る, and 解かる. }

\begin{center}
 \textbf{Sound Changes }
\end{center}

\par{ There also readings morphologically limited to certain situations. For instance, vowel shifts when creating compound words often cause many students to mispronounce words because they don\textquotesingle t understand them. }

\par{ For instance, E \textrightarrow  A and I \textrightarrow  O are extremely common. They are also important to researchers that suggest Old Japanese had an 8 vowel system. }

\begin{ltabulary}{|P|P|P|P|P|}
\hline 

雨 + 雲 \textrightarrow  あまぐも & 手 + 綱 \textrightarrow  たづな & 手 + 紙 \textrightarrow  てがみ & 木 + 霊 \textrightarrow  こだま & 目+ ゆ+ 毛 \textrightarrow  まゆげ \\ \cline{1-5}

\end{ltabulary}

\par{ As you can see, there is also voicing (連濁) to keep in mind, and these sound changes may not happen in all compounds as 手紙 suggests. }

\begin{center}
\textbf{名乗り } 
\end{center}

\par{ There are also 訓読み called 名乗り that are used in names. A given name could be read in many ways. There are usually readings that are more prominent. The sheer number of 名乗り has actually gone down over time, but it still causes headaches for natives and learners on how to properly read a person\textquotesingle s name. }

\par{秀吉・秀義・英義・英吉・英喜・秀芳・英良・秀好・秀良・秀泰・栄良・英美・秀房・秀由・秀剛・秀嘉・秀衛・秀佳, etc. = ひでよし }
      
\section{音読み \& 訓読み Combined}
 
\par{ The large majority of words, 音読み and訓読み are combined and used together. Furthermore, the majority of words with 音読み are used in compounds (熟語), and the majority of 訓読み are used in isolation or with 送りがな. However, there are instances of 音読み used in isolation, and there are also instances of 訓読み used in compounds. }

\par{ Aside from this, the main issue is that there are times when they are combined and used together! There are two such instances. The first is when a compound is 音 and 訓 are compounded (重箱読み), and the second is when a compound is 訓 and 音 are compounded (湯桶読み). }

\par{\textbf{重箱読み }}

\par{ 重箱 is a multi-tiered box, and it has been used in the word 重箱読み because it is a perfect example of an 音 and 訓 combination. Other examples include the following. }

\begin{ltabulary}{|P|P|P|P|P|P|}
\hline 

路肩 & ROkata & Road shoulder & 番組 & BANgumi & TV program \\ \cline{1-6}

木目 & MOKUme & Grain of wood & 客間 & KYAKUma & Parlor \\ \cline{1-6}

台所 & DAIdokoro & Kitchen & 茶筒 & CHAzutsu & Tea caddy \\ \cline{1-6}

団子 & DANgo & Dumplings & 反物 & TANmono & Fabric \\ \cline{1-6}

額縁 & GAKUbuchi & Frame & 本屋 & HON'ya & Book store \\ \cline{1-6}

残高 & ZANdaka & Balance (bank) & 新顔 & SHINgao & New face \\ \cline{1-6}

職場 & SHOKUba & Work place & 役場 & YAKUba & Town hall \\ \cline{1-6}

\end{ltabulary}

\par{ \textbf{湯桶読み }}

\par{ A 湯桶 is a pail-like wooden container used to carry and serve hot liquids, and it is used in the word 湯桶読み because it is a perfect example of a 訓 and 音 combination. Other examples include the following. }

\begin{ltabulary}{|P|P|P|P|P|P|}
\hline 

場所 & baSHO & Place & 身分 & miBUN & Social position \\ \cline{1-6}

消印 & keshi'IN & Postmark & 古本 & furuHON & Old book \\ \cline{1-6}

見本 & miHON & Sample & 夕刊 & yuuKAN & Evening edition \\ \cline{1-6}

荷物 & niMOTSU & Luggage & 踏台 & fumiDAI & Stool; stepping stone \\ \cline{1-6}

雨具 & amaGU & Rain gear & 薄化粧 & usuGESHOU & Light makeup \\ \cline{1-6}

高台 & takaDAI & High ground & 手帳 & teCHOU & Notebook \\ \cline{1-6}

手数 & teSUU & Bother & 鶏肉 & toriNIKU & Chicken (meat) \\ \cline{1-6}

闇市場 & yamiSHIJOU & Black market & 湯茶 & yuCHA & Hot water and tea \\ \cline{1-6}

野宿 & noJUKU & Sleeping outdoors & 大損 & ooZON & Major loss \\ \cline{1-6}

太字 & futoJI & Boldface & 細字 & hosoJI & Small type \\ \cline{1-6}

冬景色 & yukiGESHIKI & Snowy landscape & 雪化粧 & yukiGESHOU & Covered in snow \\ \cline{1-6}

\end{ltabulary}
\hfill\break
 \hfill\break
    