    
\chapter{Phonology VI}

\begin{center}
\begin{Large}
第372課: Phonology VI: 促音化 
\end{Large}
\end{center}
 
\par{ 促音添加 (the addition of a 促音) or 促音化 (sokuon-ization) is a feature that has long been associated with 俗語 in Japanese. Of course, slang or local vernaculars have always existed, and this phonological phenomenon that we will discuss in this lesson has a long history with such speech. Of course, the 促音 finds itself in Standard Japanese conjugation today. So, what are the certain environments or factors that have led and continue to govern 促音化? Another interesting thing to consider is the prevalent appearance of 促音 in loanwords. }
      
\section{促音化}
 
\par{ The definition of what 俗語 itself means has been in flux as well. By earlier definitions, essentially all spoken conversation today would be considered 俗語. Yet, that is certainly not a majority opinion in today\textquotesingle s terms. The changes we see today in Japanese such as いちかい \textrightarrow  いっかい most certainly belonged to the speech to slang when it first began. Of course, today, just as in conjugation, such consonant gemination is standard in Modern Japanese. }

\par{ Knowing when a 促音 shows up in 和語 and 漢語 at the morphological and phonological level is not that difficult to figure out. Consider the following examples. }

\begin{ltabulary}{|P|P|P|}
\hline 

Kind of Word & 促音化 & Before\slash No 促音化 \\ \cline{1-3}

和語 & 会った = met & 会う = to meet \\ \cline{1-3}

漢語 & いっぱん (一般) = General & いちもん (一問) = one\slash first question \\ \cline{1-3}

オノマトペ & ぽきっと (tree\slash branch snap noise) & ぽきんと = ditto \\ \cline{1-3}

\end{ltabulary}

\par{ In most instances in 和語 and 漢語, we know that 促音化 has been caused due to particular contractions meant for. }

\par{ ち・つ・く are the three most likely sounds to cause 促音化. This is not to say we do not get others such as き and り affected in words such as 行った (went) and 刈った (cut\slash mowed). \hfill\break
 }

\par{ We can form a rule that states high vowels [i u] are dropped before k, t, s, h\slash f, and the consonant before them assimilates to that consonant. If the second consonant is an h\slash f, it changes to a p as h\slash f do not geminate in Japanese words (excluding loans as we will see later in this lesson). }

\begin{center}
 \textbf{The Origin of 促音化 }
\end{center}

\par{ Historically speaking, 促音化 is not well recorded. The use of a small っ is a modern fix to the orthography, and previously a regular-sized つ may have been used, and the word may is used as the practice was far from universal. In fact, it is most common overall in Japanese literature to see a long consonant not marked as such. So, even if the writer wrote にき for 日記 (diary), we can still be quite certain that it was pronounced as にっき. }

\par{ Then again, it is agreed upon that at one point, the pronunciation would have started out as being literal to the spelling. Thus, we would get につきwith this example. This would make gemination in Japanese far easier. Then again, this could be wrong and speakers may have easily shifted at one point to what it is now as the rule posited above is linguistically plausible and affects classes of sounds. Yes, one particular instance of gemination would have been necessary to trigger this event, but that\textquotesingle s a given. }

\begin{center}
 \textbf{Distribution of 促音化 }
\end{center}

\par{ The distribution of 促音, as has been mentioned earlier, in native or nativized words, has a deep to vernacular speech. For instance, the following words should not have a 促音, but many speakers pronounce them with one anyway. This is because although the pattern is pervasive, there are still areas in the Japanese lexicon that it has not been universally applied. }

\begin{ltabulary}{|P|P|P|P|P|P|}
\hline 

Word & 無促音 & 促音化 & Word & 無促音 & 促音化 \\ \cline{1-6}

水族館 (Aquarium) & すいぞくかん & すいぞっかん & 適格 (Eligible) & てきかく & てっかく \\ \cline{1-6}

旅客機 (Passenger aircraft) & りょきゃくき & りょきゃっき & 各国 (Every nation) & かくこく & かっこく \\ \cline{1-6}

\end{ltabulary}

\par{ Any word with各 can have kaku contracted to kak if the next character starts with a k. However, it\textquotesingle s only OK because so many speakers have gotten used to and have begun using this newer pronunciation. The other words are still relatively uncommonly sokuonized outside of Western Japan, but in this part of the country, most people don\textquotesingle t use the standard 促音-less pronunciation at all. Of course, this could have a lot to do with honing in on this dialectical difference. }

\par{\textbf{Lesson Note }: The following situations involving 促音化 are complicated in regards to social attitude. Some things are strictly slangish\slash conversational and some things are standard pronunciations. Some forms are even moribund. Thus, these notes will need to be noted on a case by case basis. }

\begin{center}
\textbf{Emphatic 促音化 }
\end{center}

\par{ As we see in onomatopoeic words and regular words at times, a 促音 may often be inserted for emphasis, and it\textquotesingle s basically always placed in the first mora. }

\begin{ltabulary}{|P|P|P|P|P|P|}
\hline 

Very & とても \textrightarrow  とっても & Latter colloquial & Nothing but & ばかり \textrightarrow  ばっかり & Latter colloquial \\ \cline{1-6}

Closely & ぴたり \textrightarrow  ぴったり & Both common & Certainly & しかと \textrightarrow  しっかと & Both rare\slash literary \\ \cline{1-6}

To urge & せつく \textrightarrow  せっつく & Latter most common & Exclusively & もはら \textrightarrow  もっぱら & Latter only used \\ \cline{1-6}

As is & まま \textrightarrow  まんま & Latter dialectical & Everyone & みな \textrightarrow  みんな & Latter colloquial \\ \cline{1-6}

Same & おなじ \textrightarrow  おんなじ & Latter colloquial &  &  &  \\ \cline{1-6}

\end{ltabulary}

\par{ The last three are still examples of gemination. Don\textquotesingle t let spelling get the best of you. For others that are not geminated, you see them essentially get geminated with ~ん insertion. }

\begin{ltabulary}{|P|P|P|}
\hline 

Puzzle\slash mystery & なぞ \textrightarrow  なんぞ & Latter is colloquial and uncommon. \\ \cline{1-3}

Kite (bird) & とび \textrightarrow  とんび & Both are just as common. \\ \cline{1-3}

Down to & くだり \textrightarrow  くんだり & Latter is rare but with a more specific meaning. \\ \cline{1-3}

Just & ただ \textrightarrow  たんだ & Latter is no longer used. \\ \cline{1-3}

Whenever & たび \textrightarrow  たんび & Latter is colloquial\slash dialectical. \\ \cline{1-3}

To become sharp & とがる \textrightarrow  とんがる & Latter is colloquial \\ \cline{1-3}

\end{ltabulary}

\par{ It\textquotesingle s also important to note that sometimes the original form ends up no longer being used. This is the case for 専ら, which used to be read as もはら. Remember that Japanese wants h geminated as p rather than h. }

\begin{center}
 \textbf{In Compound Verbs }
\end{center}

\par{  One thing that is still considered slang-ish is applying 促音化 to compounds. For instance, 追い, 取り, and the like often get added, but they usually get changed to 追っ and 取っ respectively. So, for instance, an emphatic form of はじめ is おっぱじめ. So, the same sound changes involving the 促音 we see with ~た and ~て is extended to compounding. It\textquotesingle s also important to note that gemination may result in ん insertion. However, the process is still the same. Below are more examples. }

\begin{ltabulary}{|P|P|P|}
\hline 

To poke & 突きつく \textrightarrow  突っつく & The first form is not used. \\ \cline{1-3}

To greatly bind & 引き曲げる \textrightarrow  ひん曲げる & The base form is not used. \\ \cline{1-3}

To tie fast & 踏み縛る \textrightarrow  踏ん縛る & Both are uncommon. \\ \cline{1-3}

Grapple & 取り組み合い \textrightarrow  取っ組み合い & The first form is not used. \\ \cline{1-3}

To kick hard & 蹴り飛ばす \textrightarrow  蹴っ飛ばす & The latter is colloquial yet common. \\ \cline{1-3}

To fly\slash jump forcefully & ぶち飛ぶ \textrightarrow  ぶっ飛ぶ & The first form is not used. \\ \cline{1-3}

To throw away & 打ち遣る \textrightarrow  うっちゃる & The first is not used, but the latter is dialectical. \\ \cline{1-3}

To strongly tear & 打ち千切る \textrightarrow  ぶっ千切る & The first form is not used. \\ \cline{1-3}

To pitch forward & 突きのめる \textrightarrow  つんのめる & The first is not used, but the latter is slang. \\ \cline{1-3}

\end{ltabulary}

\begin{center}
 \textbf{In Compounds }
\end{center}

\par{ 促音化 is also often emphatically inserted at the boundary between two words in a native compound. This also holds true for when particular suffixes attach to a word. Sometimes it is part of the regular form of the word and sometimes it\textquotesingle s not. As this is a lesser extension of 促音化, this is to be expected. Below are some examples. }

\begin{ltabulary}{|P|P|P|}
\hline 

To leave open & 開け放す \textrightarrow  開けっ放す & The latter is colloquial but most common. \\ \cline{1-3}

All one\textquotesingle s & ありたけ \textrightarrow  ありったけ & Only the latter is used. \\ \cline{1-3}

Loving deeply & 首丈 \textrightarrow  首っ丈 & Only the latter is used. \\ \cline{1-3}

City slicker & 擦れ枯らし \textrightarrow  擦れっ枯らし & The latter is the most common. \\ \cline{1-3}

Salty & 塩辛い \textrightarrow  塩っ辛い & Both are common. \\ \cline{1-3}

Sugary & 甘たるい \textrightarrow  甘ったるい & The latter is colloquial. \\ \cline{1-3}

Ostracized child\slash miso scum & みそかす \textrightarrow  みそっかす & The latter is colloquial. \\ \cline{1-3}

\end{ltabulary}

\par{\textbf{Exceptions' Note }: Of course, there are also just exceptional cases that 促音化 appears. For instance, ぬすっと (thief) is certainly a contraction, but we don\textquotesingle t see all word with びと changed to っと. }

\begin{center}
\textbf{促音化 in Loanwords }
\end{center}

\par{ The remainder of this lesson will investigate the intricate yet rather regular distribution of the very prevalent 促音化 in loanwords. }

\par{ Though the special timing of mora is not found in English, the 促音 shows up in many loans from English. It often comes about from the stress in the English word, but there are exceptions of this and other motivations (such as mimicking a non-Japanese sound). }

\begin{ltabulary}{|P|P|P|P|}
\hline 

ロック (lock\slash rock) & タップ (tap) & ファックス (Fax) & レッドソックス (Red Sox) \\ \cline{1-4}

バッハ (Bach) & ゴッホ (Gogh) & ドップラー (Doppler) & タックスイーター (tax eater) \\ \cline{1-4}

スノ(ッ)ブ (snob) & マックス (max) & キャップ (cap) & プッシュボタン (push button) \\ \cline{1-4}

ピクニック (picnic) & スタッフ (staff) & アドホック (ad hoc) & フィッシング (fishing\slash phishing)  \\ \cline{1-4}

\end{ltabulary}

\begin{ltabulary}{|P|P|P|P|}
\hline 

ハーフ (half (person)) & X ハッフ & タフ (tough) & X タッフ \\ \cline{1-4}

パフ (puff) & X パッフ & サインアウト (sing-out) & X サインアウット \\ \cline{1-4}

バス (bus) & X バッス & ログアウト (log-out) & X ロッグアウット \\ \cline{1-4}

\end{ltabulary}

\par{ What can we make of all these examples? First, let\textquotesingle s consider segmental factors in these words. For instance, it\textquotesingle s very common to have gemination of p and k, but it is very rare to find geminate b\textquotesingle s and g\textquotesingle s in loanwords. We do find exceptions in which we do not get geminate k\textquotesingle s. Consider words such as アクト (act), ダクト (duct), タクト (tact), etc. }

\par{ We also find exceptions such as スノッブ of consonants that tend not to geminate get geminated. Though, as this exception also demonstrates, exceptions tend to have a not so exceptional form, in this case スノブ. Then again, geminate h\textquotesingle s never appear in the rest of the Japanese lexicon but appear in loanwords like バッハ. Here, a fricative sound non-native to Japanese (and English for that matter) is being conformed as best as possible in the same way it would be replaced by k in English. What we do see is that gemination }

\par{ As we\textquotesingle ve seen with バス, loanwords entering Japanese are less likely to have a 促音 if the consonant is an s. Japanese hates having geminate f\textquotesingle s too, but it surprisingly allows ssh, which we\textquotesingle ve seen already in words like プッシュ. Again, that\textquotesingle s not to say that there are exceptions. For instance, レッスン (lesson) and ワッフル (waffle) are not just exceptional but also very commonly used words. It appears that if the word ends in ン or ル that this restriction is lifted. So, how do we explain words like スタッフ? If it were three morae, it would be スタフ. }

\par{ Most three morae words in Japanese have the accent on the second mora. If this were to happen, this would make the accent system completely contrary to its source stress accent. There are exceptional loans into Japanese like トマト (tomato) and タバコ (tobacco) with completely different pitches, but these are also much older loans relative to スタッフ. If you add the 促音, the problem is resolved. But, couldn\textquotesingle t the pitch accent have been arranged in a way that would have been less weird? What if the accent were on the first mora or the stress were flat? It would still be contrary to English, but it would not be out of the norm. After all, there are words like スミス (Smith) and プラス (plus) with accent on the first mora and words such as ブログ (blog) with flat pitch. However, both flat pitch and gemination does not occur. It\textquotesingle s as if gemination, accent shift, or pitch leveling are different options that don\textquotesingle t mix well. }

\par{ Though Japanese phonological restraints may trump any foreign word\textquotesingle s original pronunciation, lax and tense distinction is maintained well in the borrowing process into Japanese. For instance, if it were not, then we would expect English words like cup” and “carp” to be borrowed in as the same word. However, this is not the case. Rather, we see that they are borrowed into Japanese as カップ・コップ (cup) and カープ (carp) respectively. If the length of the English syllable were to then end in a 促音, the morae count would exceed three morae, which would be quite long for such a short word in the original language. This is why we don\textquotesingle t see something like カーップ. }

\par{ This sort of avoidance is also connected to the avoidance of diphthongs even when the loanword should have one. For instance, Japanese say ステンレス instead of ステインレス for stainless. The overall idea in loanwords is that the word is almost certainly going to have a higher mora count than the original syllable count. So, eliminating features that would necessarily show up from being picky at sounding like the original language results in a more ‘practical\textquotesingle  loan. And, as we know, Japanese has no problem reducing things further if the resultant loan is still too long (Ex. バスケットボール \textrightarrow  バスケ). }

\par{ Consider the following words. Do you see a pattern? }

\begin{ltabulary}{|P|P|}
\hline 

キャップ & キャプテン (captain) \\ \cline{1-2}

ファックス & ファクシミリ (facsimile) \\ \cline{1-2}

サックス (sax) & サキソフォン (saxophone) \\ \cline{1-2}

リラックス (relax) & リラクセーション (relaxation) \\ \cline{1-2}

リッスン (listen) & リスナー (listener) \\ \cline{1-2}

\end{ltabulary}

\par{ The words on the left have a 促音 but the words on the right don\textquotesingle t. There is certainly a tendency for the 促音 to appear close to the end of the word, and it seems that the complexity and length of the word is another restriction. Of course, these restrictions are relative, but they help to explain what\textquotesingle s in front of us for now. A more probable reasoning for the lack of a 促音 in the long words on the right-hand column is that because the environment most suitable for gemination is not at the end of the word, it does not appear. }

\par{  Japanese does like to have regularity, though. If two syllables in the loanword are considered tense, this gets carried over. So, rather than getting タクス, you getタックス. How much of this has to deal with avoiding sounding like an existing Japanese word should be investigated more in depth. }
    