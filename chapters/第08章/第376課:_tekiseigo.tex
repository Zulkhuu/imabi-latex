    
\chapter{敵性語禁止}

\begin{center}
\begin{Large}
第376課: 敵性語禁止 
\end{Large}
\end{center}
 
\par{ During World War II, due to the fact that Japan was fighting Western powers, loan words of Western inspiration were targets of replacement with Japanese equivalents. If equivalents did not exist, they were to be coined. How far this practice went is controversial. There was no law that fully or partially banned English words. However, there were decrees sent out to broadcasting companies and organizations to rename things. }

\par{ Some of these words have continued to live on and have helped Japanese further in nuance splitting. Others never made it onward, but this lesson is designed to introduce you to the scope of this language reform. }

\par{ This lesson is not about why this happened in Japan, nor is it about what things should be. Rather, this is purely about the words affected or created and their current usage. Although background information will have to be given and words such as "the Japanese Empire" and "wartime" will be used out of necessity, they are by no means employed to be offensive. }
      
\section{軽佻浮薄}
 
\par{ This four character idiom essentially means "frivolous and thoughtless". This word was used in reference to English words (and words from other Western languages) in a time the Japanese Empire was in war with these nations, particularly starting in 1940. With this word in mind, it fueled the boycott movement of these words for their 敵性 (inimical character). }

\par{ From a point of law, there was no rule put in place by the government labeling words of foreign origin as either "軽佻浮薄" and "敵性", and from a practical level, the nation still possessed relations with certain nations and the language had been borrowing words from Indo-European languages since the 1950s. Thus, a reversal of centuries of borrowing (although generally light up until the Meiji Restoration), would essentially be possible as any changes in the lexicon of the language would be most effectively implemented in primary education to the younger population. }

\par{ The movement was a natural reaction to the war. So, it didn't have a strong effect. So, in fields such as science, essentially no changes terminology were made. As simple English words as well as 和製英語 had been used before and during the war in not just the media but also by general citizens, it would, again, have been impossible anyway for the effect of English to be fully eliminated. As examples of words that were even used in government publications, シャツ, コンビネーション, チョッキ (from Portuguese meaning vest) never stopped being used. }

\par{ In fact, although there were attempts by the Ministry of Education to rid of English education in schooling, due to English's role as the lingua franca of the world, it was ultimately left as an optional course of study. That is not to say that some schools went ahead and banned it as an enemy language, but it is not the case that English education or the use of English words--whether when speaking English or Japanese--was ever implemented. }

\par{ However, there were a few things that were put in place that did influence the use of language in schooling. For instance, "pro-American" phrases were replaced, the use of Western-style dates was made limited*, and phrases felt to be derogatory to Japan were taken out of texts. }

\par{ As for the second point marked with an asterisk, this has made a lasting impact on Japan. Though the Japanese-style of telling the date has never ceased and has been the most used throughout Japanese history, it is fair to say that before war tensions escalated that the Western calendar as well as customs on telling the date was becoming more and more common. Nowadays, the two systems balance out by being used in specific situations. }

\par{ This sort of bifurcation is not natural for when one system is quickly replacing an older and more antiquated system. If we relate this to the use of Sino-Japanese and native numbers, we see that the newer Sino-Japanese numbers have been slowly replacing native numbers. So, if customs in using the Western and Japanese calendar were explicitly established in educational texts before the newer system fully took over, then these new conventions would take hold. }

\par{  It was also not the case that loanwords were banned in military use. For instance, although we will look at baseball terminology later in this lesson, ストライク was still used throughout the war in war related materials when speaking of baseball. The French word スペリー for "searchlight" and ピスト for "cockpit" were very important words at the time. }

\par{ Some words just couldn't be avoided, and some people including those in comedy and government would use this fact to make puns. For instance, what were people to do about words such as コーヒー (which even had its own 漢字 coined: 珈琲), ウィスキー, エンジン, and レコード? The answer is, of course, nothing. }
      
\section{野球用語}
 
\par{ Baseball was loved by many Japanese at the time, but because it was the sport of the enemy and most prized one at that, the terminology for baseball was drastically revised as a means to still play it in a Japanese way. The change in terminology, however, is in fact not as sudden as it may seem. Nor is it evident that this movement was the reason for why the words were changed. This is because terminology began as early as 1890 by the help of 正岡子規, who tried his best to "translate" these phrases. }

\par{ As these coined words coexisted with the English loans, it would not be a stretch for the Japanese coins to even naturally replace the English. Because baseball organizations were shut down from 1942-1946, this ultimately prevented permanent erasing of the English loans.  In fact, some coins such as 野球 early on replaced English loans, which in this case would be ベースボール, which is still not used a lot today. }

\par{ A strike is a ストライク. However, to avoid using this word, よし#本 was used instead, and the word 正球 was around too. In Modern Japanese, "Strike 1", "Strike 2", and "Strike 3" are ストライク・ストライクツー・ストライクスリー、ユーアーアウト! respectively. If you were to put the number first, you just have a counter expression. A strike out is 三振. Instead of the English, you would hear 「よし、3本それまで!」 or 「よし、退け!」. }

\par{ On the score board, 振 would be used for strikes, 球 would be used for balls, and 無為 would be for "outs". }

\par{ A ボール is "ball" in English baseball terminology, too. It was replaced by だめ#つ or 悪球. To say something like two strikes and three balls today, you would say ツーストライクスリーボール. }

\par{ Below are many more words related to baseball with foreign and Japanese counterparts. A lot of them are still used, and it's most important to know which one's are not used. Those not used anymore will have a triangle next to them. Some words marked with triangles may still be used but not in the realm of baseball. }

\begin{ltabulary}{|P|P|P|P|}
\hline 

敵性語 & 言い替え & 敵性語 & 言い替え \\ \cline{1-4}

(ファール)ボール & だめ △・圏外 △・もとえ △ & アウト & 無為 △ \\ \cline{1-4}

セーフ & 安全 △ & バッテリー & 対打機関 △ \\ \cline{1-4}

タイム & 停止 △ & フェアヒット & 正打 △ \\ \cline{1-4}

ファールチップ & 擦打 △ & バントヒット & 軽打 △ \\ \cline{1-4}

スクイズ & 走軽打 △ & ヒットエンドラン & 走打 △ \\ \cline{1-4}

ボーク & 疑投 △ & ホームイン & 生還 \\ \cline{1-4}

フォースアウト & 封殺 & インターフェア & 妨害 △ \\ \cline{1-4}

スチール & 奪塁 △ & リーグ戦 & 総当戦 △ \\ \cline{1-4}

コーチ & 監督 & コーチャー & 助令 \\ \cline{1-4}

マネージャー & 幹事 & アーンドラン & 自責点 \\ \cline{1-4}

チーム & 球団 & ホームチーム & 迎戦組 △ \\ \cline{1-4}

ビジターチーム & 往戦戦 △ & グラブ・ミット & 手袋 △ \\ \cline{1-4}

フェアグラウンド & 正打区域 △ & ファールグラウンド & 圏外区域 △ \\ \cline{1-4}

ウイニングショット & 決め球 & インコース & 内角 \\ \cline{1-4}

アウトコース & 外角 & オーバースロー & 暴投 \\ \cline{1-4}

ファールライン & 境界線 & スリーフィートライン & 三尺線 △ \\ \cline{1-4}

プレイヤーズライン & 競技者線 & コーチャーズボックス & 助令区域 △ \\ \cline{1-4}

フライ & 飛球 & エラー & 落球 \\ \cline{1-4}

デッドボール & 死球 & ピッチャー & 投手 \\ \cline{1-4}

バッティングピッチャー & 打撃投手 & フォアボール & 四球 \\ \cline{1-4}

ストッパー & 抑え投手 & スターティングメンバー & 先発選手 \\ \cline{1-4}

ライト & 右翼手 & レフト & 左翼手 \\ \cline{1-4}

センター & 中堅手 & インフィールド・ダイヤモンド & 内野 \\ \cline{1-4}

アウトフィールド & 外野 & スチール & 盗塁 \\ \cline{1-4}

ノーアウト & 無死 & ~アウト & ~死 \\ \cline{1-4}

ランナー & 走者 & ゲームセット & 試合終了 \\ \cline{1-4}

スコアリングポジション & 得点圏 & プレート & 投手板 \\ \cline{1-4}

フリーバッティング & 打撃練習 & ホームラン & 本塁打 \\ \cline{1-4}

フルベース & 満塁 & ホームベース & 本塁 \\ \cline{1-4}

ファースト(ベース) & 一塁 & セカンド(ベース) & 二塁 \\ \cline{1-4}

サードベース & 三塁 &  &  \\ \cline{1-4}

\end{ltabulary}

\par{\textbf{Word Notes }: }

\par{1. 一塁, 二塁, and 三塁・サード can be short for 一塁手, 二塁手, and 三塁手 respectively. These are, of course, referring to the basemen. \hfill\break
2. 本塁 may also mean "main fort" outside of baseball. \hfill\break
3. スタメン is a newer 和製英語 for スターティングメンバー used frequently today. \hfill\break
4. 内野 and 外野 may sometimes respectively be abbreviations of 内野手 and 外野手. \hfill\break
5. フォースアウト may be abbreviated to フォース today. }

\begin{center}
 \textbf{Umpire Language } 
\end{center}

\begin{ltabulary}{|P|P|P|P|P|P|}
\hline 

三振アウト & それまで! & ボール & 一つ、二つ、三つ、四つ塁へ & フェアヒット & よし \\ \cline{1-6}

ファールボール & だめ & セーフ & よし・安全 & アウト & 退け \\ \cline{1-6}

ボーク & 反則 & インフィールドフライ & 内野飛球 &  &  \\ \cline{1-6}

\end{ltabulary}
      
\section{Other Sports}
 
\par{ The most important changes in the name of sports are below. None of these coined terms are used anymore with exception to 送球 and 打球 which instead mean "throwing a ball" and "batting\slash batted ball". Use the 音読み of these characters to read. The only exceptions to this are 雪滑 and 氷滑. So, use the 訓読み for these two words. }

\begin{ltabulary}{|P|P|P|P|P|P|}
\hline 

敵性語 & 言い替え & 敵性語 & 言い替え & 敵性語 & 言い替え \\ \cline{1-6}

ラグビー & 闘球 & ゴルフ & 打球・芝球 & ハンドボール & 送球 \\ \cline{1-6}

クロール (Crawl) & 速泳 & アメリカンフットボール・米式蹴球 & 鎧球 & スキー & 雪滑 \\ \cline{1-6}

スケート & 氷滑 &  &  &  &  \\ \cline{1-6}

\end{ltabulary}
      
\section{Other Kinds of 敵性語の言い替え}
 
\par{ For areas in which few words were changed, newer Japanese words held on more frequently. }

\begin{ltabulary}{|P|P|P|P|}
\hline 

地名 & 日本アルプス & 中部山岳 & Both still used in different contexts. \\ \cline{1-4}

地名 & パールハーバー & 真珠湾 & Both still used. \\ \cline{1-4}

地名 & シンガポール & 昭南島 & Latter only used in reference to this period. \\ \cline{1-4}

放送 & ニュース & 報道 & Both used. Latter used a lot in compounds. \\ \cline{1-4}

放送 & 臨時ニュース & 臨時報道 & Latter is more formal. \\ \cline{1-4}

放送 & アナウンサー & 放送員 & Both used. Latter is more 書き言葉的. \\ \cline{1-4}

放送 & レコード & 音盤 & Latter more 書き言葉的. \\ \cline{1-4}

放送 & マイクロホン & 送話器 & Latter rare and old-fashioned. \\ \cline{1-4}

交通 & ロータリー & 円交路 & Latter not used. 環状交差路 used formally today. \\ \cline{1-4}

交通 & プラットホーム & 乗車廊 & Latter no longer used. \\ \cline{1-4}

飲食物 & サイダー & 噴出水 & Latter now means "spouting water". \\ \cline{1-4}

飲食物 & コロッケ & 油揚げ肉饅頭 & Latter no longer used. \\ \cline{1-4}

飲食物 & カレーライス & 辛味入汁掛飯 & Latter no longer used. \\ \cline{1-4}

飲食物 & キャラメル & 軍粮精 & Latter no longer used. \\ \cline{1-4}

飲食物 & フライ & 洋天 & Latter no longer used. \\ \cline{1-4}

動物 & カンガルー & 袋鼠 & Latter no longer used. \\ \cline{1-4}

動物 & ライオン & 獅子 & Both used today. \\ \cline{1-4}

植物 & チューリップ & 鬱金香 & Latter rarely seen. \\ \cline{1-4}

植物 & シクラメン & カガリビソウ & Both still used. \\ \cline{1-4}

植物 & コスモス & 秋桜 & Both still used. \\ \cline{1-4}

植物 & カスタービン & 支那油桐 & Latter still seldom used. \\ \cline{1-4}

植物 & ヒヤシンス & 風信子 & Both still used. Latter can be read as ふうしんす. \\ \cline{1-4}

植物 & プラタナス & 鈴懸の木 & Both still used. \\ \cline{1-4}

音楽 & サクソホン & 金属製曲がり尺八 & The latter would just make people laugh. \\ \cline{1-4}

音楽 & トロンボーン & 抜き差し曲がり金真鍮喇叭 & The latter would just make people laugh. \\ \cline{1-4}

音楽 & バイオリン & 提琴 & Literary but not used. \\ \cline{1-4}

音楽 & コントラバス & 妖怪的四弦 & The latter is not used. \\ \cline{1-4}

音楽 & ピアノ & 洋琴 & The latter is more common in literature than 提琴. \\ \cline{1-4}

音楽 & ドレミファソラシド & ハニホヘトイロハ & Latter not used. \\ \cline{1-4}

髪 & パーマ & 電髪 & Latter not used. \\ \cline{1-4}

科学 & 酸素マスク & 与圧面 & The latter is no longer used. \\ \cline{1-4}

\end{ltabulary}
     