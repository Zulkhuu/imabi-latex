    
\chapter{Verlan (倒語)}

\begin{center}
\begin{Large}
第375課: Verlan (倒語) 
\end{Large}
\end{center}
 
\par{ The title of this lesson includes a word that you probably have never seen before. To put it simply, this word specifically describes words that have undergone internal inversion. How the word gets inverted may differ from others. In Japanese this is technically referred to as 倒語. Usually, though, it is referred to as 逆(さ)読み. }

\par{ These sorts of readings have existed in Japanese for some time. The purpose of this inversion is to bring about some sort of emphasis. This sort of change is seen in various parts of any language regardless of one's generation. Though the use of these inversions is quite varied, they are generally treated as in-group lingo or 隠語. In Japanese 隠語 typically have negative stereotypes attached to them, and they will almost certainly only be known and used by a small group of people. }

\par{ These sorts of words become most popular in Japan during the 江戸時代. Examples that were later passed down to the present day as the common word include しだらない (slovenly) \textrightarrow  だらしない. Others that didn't quite catch on include words like キセル (tobacco pipe) \textrightarrow  セルキ. Other examples like タネ (seed\slash subject matter) \textrightarrow  ネタ ((joke) material\slash topping of nigirizushi) resulted into words with (slightly) different meanings. }

\par{ These words got some publicity in the 1980s due to broadcast writers who previously wrote memos to publications (ハガキ職人) utilizing such expressions. A common example at the name was calling 六本木 ギロッポン. }

\par{ It's also important to note that at an individual level, 逆さ読み come about naturally. }
      
\section{Examples}
 
\par{ Below is a list of examples. Over time, notes of usage and example sentences will be added. Some of these words are r-rated words, but they are examples nonetheless of this phenomenon. Thus, they will be noted. }

\begin{ltabulary}{|P|P|P|P|P|P|P|P|}
\hline 

倒語 & 意味 & 倒語 & 意味 & 倒語 & 意味 & 倒語 & 意味 \\ \cline{1-8}

ロイク & 黒人 & ナオン & 女 & チャンカー & お母さん & チャンネー & お姉さん \\ \cline{1-8}

チャンバー & おばあさん & ザギン & 銀座 & オカジュー & 自由が丘 & ワイハ & ハワイ \\ \cline{1-8}

ジャーマネ & マネージャー & シータク & タクシー & クーイ & 行く & ミーノ & 飲む \\ \cline{1-8}

ダータ & ただ & ラーハ & 腹 & リーヘ & 減る & シーメ & 飯 \\ \cline{1-8}

ベルター & 食べる & マイウー & うまい & チョイモー & もうちょい & シーホー & 欲しい \\ \cline{1-8}

ルーネー & 寝る & クリソツ & そっくり & クリビツ & びっくり & テンギョー & 仰天 \\ \cline{1-8}

イタオドロ & 驚いた & ルナドッホ & なるほど & ビーチ & ちび & ポコチン & ちんぽこ \\ \cline{1-8}

コーマン & おまんこ & パイオツ & おっぱい & カイデ & でかい & ビーチク & 乳首 \\ \cline{1-8}

チャイチ & ちっちゃい & シーアー & 脚 & ガイナー & 長居 & ソイホ(-) & 細い \\ \cline{1-8}

トイフ(-) & 太い & エーケー & 毛 & チータ & 立つ・勃つ & オイニー & 匂い \\ \cline{1-8}

サイクー & 臭い & エーヘー & 屁 & ソーク(-) & 糞 & メーナ(-ン) & 舐める \\ \cline{1-8}

パイイツ & いっぱい & パツキン & 金髪 & グラサン & サングラス & マーヒー & 暇 \\ \cline{1-8}

ロクブテ & 手袋 & コバルド & ドル箱 &  &  &  &  \\ \cline{1-8}

\end{ltabulary}

\par{ It's also important to note that 逆読み are often used in brand names. Some examples include the following. }

\begin{ltabulary}{|P|P|P|P|P|P|}
\hline 

HAKUBI C & HAKUBI from 美白 & バソキヤ & From 焼きそば & EZAK & From 風邪 \\ \cline{1-6}

\end{ltabulary}

\begin{center}
 \textbf{More Historic Examples }
\end{center}

\begin{itemize}

\item \textbf{験を担ぐ }: The 験 read as げん actually ultimately comes from an inversion of 縁起. Both this and 縁起を担ぐ exist, and both mean "to be superstitious. 
\item \textbf{デカ }: Police officers who would wear kimono instead of their uniforms in the 明治時代 were called 角袖巡査. This was then abbreviated to 角袖. Then, the middle syllables were dropped and the remaining were flipped to get デカ. This abbreviation was due in part to intermediary inversions such as そでかく and くそでか. It has since become a word meaning "detective". 
\item \textbf{ポシャる }: From シャッポを脱ぐ. This essentially means "to surrender when on throws off one's hat", but this new verb now has a meaning similar to ”to fizzle\slash break down". 
\item \textbf{道路 }: Though it isn't from ロード, the thought is still interesting. People have noticed this since the 明治時代, it is still sometimes make people laugh when this is pointed out. 
\end{itemize}
\textbf{Reading Phrases Backwards }
\par{ Sometimes some phrases read backwards make sense, but sometimes you get just another sentence. These are often made as brain teasers or jokes. }

\begin{itemize}

\item \textbf{雲雲崖にこんち旅なし }: This makes no sense as is, but when you read backwards, you get a somewhat perverted joke \textrightarrow  しなびたちんこに毛がもくもく 
\item \textbf{問屋の米を買いたい買いたい }: This is a little more X-rated. When you read backwards, yet get 痛い痛いおめこのやいと. おめこのやいと means "vagina moxibustion". 
\item \textbf{「予想」は「嘘よ」 }: This is just clever. 
\item \textbf{手袋を反対から言ってごらん! }: Don't do this or else you'll be hit six times. 
\end{itemize}
    