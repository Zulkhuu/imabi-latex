    
\chapter{Start to}

\begin{center}
\begin{Large}
第79課: Start to: ~始める, ~出す, \& ~かける 
\end{Large}
\end{center}
 
\par{ \textbf{Compound verbs }are created in Japanese by adding one verb to another verb\textquotesingle s stem. Verbs that attach to others in this way are called supplementary verbs. However, not all verbs can be used in this way. The first ones that we will study concern “beginning”, which are ~始める, ~出す, and ~かける. }

\begin{ltabulary}{|P|P|P|P|}
\hline 

一だん & 五だん & する & 来る \\ \cline{1-4}

食べる \textrightarrow  食べ & 思う \textrightarrow  思い & する \textrightarrow  し & 来る \textrightarrow  来(き) \\ \cline{1-4}

\end{ltabulary}
      
\section{~始める}
 
\par{ The ${\overset{\textnormal{いちだん}}{\text{一段}}}$ verb ${\overset{\textnormal{}}{\text{始}}}$ める means "to begin\slash start". It also means this in compound verbs. So, you can use it to create new verbs like the following. }

\begin{ltabulary}{|P|P|}
\hline 

書く + 始める = 書き始める & To begin writing \\ \cline{1-2}

する + 始める = し始める & To begin doing \\ \cline{1-2}

壊す + 始める = 壊し始める & To begin destroying \\ \cline{1-2}

覚える + 始める = 覚え始める & To begin remembering \\ \cline{1-2}

\end{ltabulary}

\par{ Remember that this verb is transitive. However, when it attaches to verbs, the transitivity of the verb expression is determined by the verb that it attaches to. So, it attaches to both intransitive and transitive verbs. So, when it\textquotesingle s used as a stand-alone verb, you have the 始まる and 始める contrast, but not in compound verbs. }

\par{1. 今日、レポートを書き始めました。 \hfill\break
I started writing the report today. }

\par{2. 宿題を(し)始めた。 \hfill\break
I started doing my homework. }

\par{3. 僕が日本語を勉強し始めたのは、6年前です。 \hfill\break
It was six years ago since I started studying Japanese. }

\par{4. ${\overset{\textnormal{つか}}{\text{疲}}}$ れを ${\overset{\textnormal{かん}}{\text{感}}}$ じ始めた。 \hfill\break
I began to feel tired. }

\par{5. ${\overset{\textnormal{らいがっき}}{\text{来学期}}}$ から日本語を勉強し始める。 \hfill\break
I'll start studying Japanese from next semester. }

\par{6. 仕事を始める。 \hfill\break
To start work. }

\par{7. 話を始めた。 \hfill\break
I began a conversation. }

\par{8. 雪は冷たいみぞれに変わり始めた。 \hfill\break
The snow started to turn into icy sleet. }

\par{9. 手がかじかみ始めました。 \hfill\break
My hands began to feel numb. }

\par{\textbf{漢字 Notes }: }

\par{1. みぞれ can be written in 漢字 as 霙, but you are not responsible for this. }

\par{2. かじかむ can be written in 漢字 as 悴む, but you are not responsible for this. }
      
\section{~出す}
 
\par{  First and foremost, ${\overset{\textnormal{だ}}{\text{出}}}$ す is a transitive verb which means "to take out", and it may be used in an array of situations. Although not the main point of this section, here are some examples of how this verb is used as a stand-alone verb. }
 
\par{10. ゴミを出す。 \hfill\break
To take out the trash. }
 
\par{11. 強いにおいを出した。 \hfill\break
It gave off a strong smell. }
 
\par{12. レポートを出す。 \hfill\break
To submit a report. }
 
\par{13. 私はあの ${\overset{\textnormal{みせ}}{\text{店}}}$ で飲み物を出しています。 \hfill\break
I serve drinks at that store. }

\par{14. ${\overset{\textnormal{てがみ}}{\text{手紙}}}$ を出す。 \hfill\break
To mail a letter. }
 
\par{ ~出す is like ~始める in that they both describe the start of an action, circumstance, or change, but they are slightly different. }
 
\par{ \textbf{~始める }is used generally in regards to the starting of an event, and it is used with verbs of volition, actions in which people have control of, in this sense. In the case that it shows a circumstance or change beginning, it is used with non-volitional\slash intransitive verbs. Even still, it \emph{generally }states the beginning of something. }
 
\par{ \textbf{~出す }attaches to verbs of volition\slash non-volition as well. Thus, although the main verb is transitive, the transitivity of the compound verb expression is decided by the verb that it attaches to. As for ~出す, it describes the sudden start of an action or circumstance. In this sense, it is far more emotionally emphatic. }
 
\par{To be clear on how this is attached to verbs, consider the following chart. }

\begin{ltabulary}{|P|P|P|P|P|P|}
\hline 

動く (To move) & + 出す \textrightarrow  & 動き出す & 叫ぶ (To shout) & + 出す \textrightarrow  & 叫び出す \\ \cline{1-6}

歩く (To walk) & + 出す \textrightarrow  & 歩き出す & 飛ぶ (To fly) & + 出す \textrightarrow  & 飛び出す \\ \cline{1-6}

\end{ltabulary}
 
\par{15. 由美子ちゃんが泣き出した。 \hfill\break
Yumiko broke into tears. }
 
\par{16. 彼女は ${\overset{\textnormal{わら}}{\text{笑}}}$ い出した。 \hfill\break
She started to laugh (big time). }
 
\par{17. 走り出した。 \hfill\break
He began to dash. }
 
\par{18. 太陽が顔を出した。 \hfill\break
The sun showed its face. }
 
\par{19. 雨が降り出した。 \hfill\break
It began to rain (a lot). }

\par{Note: 見出す = みいだす (to find; pick out) }

\par{20. 面白い映画を見て、吹き出した。 \hfill\break
Watching a funny movie, I burst out laughing. }

\par{21. ${\overset{\textnormal{せいふ}}{\text{政府}}}$ の ${\overset{\textnormal{しせい}}{\text{姿勢}}}$ に ${\overset{\textnormal{ふまん}}{\text{不満}}}$ が ${\overset{\textnormal{ふ}}{\text{噴}}}$ き出した。 \hfill\break
I was flushed with dissatisfaction at the government's stance. }
 
\par{\textbf{Word Note }: ふきだす is a word with two spellings and several meanings. It can show something spouting out with much force, which can be applied in a physical sense or in a more abstract sense like in Ex. 21. In Ex. 22, it has a meaning of “sprout out”. Just as in Ex. 20, it may be like 笑い出す. The verb is also sometimes used in a transitive sense. }
 
\par{22. ${\overset{\textnormal{しんめ}}{\text{新芽}}}$ が吹き出した。 \hfill\break
The bud sprouted (out). }
 
\par{23. 桜島はいつも真っ白な ${\overset{\textnormal{かざんばい}}{\text{火山灰}}}$ を ${\overset{\textnormal{ふ}}{\text{噴}}}$ き出しますね。 \hfill\break
Mt. Sakurajima always spouts out white volcanic ash. }
 
\par{\textbf{Geography Note }: 桜島 is an island next to 鹿児島, which is a major city on the southern tip of 九州, the southernmost major Japanese island. This island has an active volcano, ${\overset{\textnormal{かっかざん}}{\text{活火山}}}$ (also read more correctly asかつかざん), which is formally called ${\overset{\textnormal{さくらじまおんたけ}}{\text{桜島御岳}}}$ . \hfill\break
 \hfill\break
 御 is an honorific prefix, and although it is usually read as お in this context, it is actually read with the older reading おん here. 岳 is an alternative character for certain mountains\slash volcanos as opposed to the more frequent山. However, this active volcano is usually called 桜島, despite being the name of the island. If the ash was to finally stop and the volcano became dormant, it would be called as ${\overset{\textnormal{きゅうかざん}}{\text{休火山}}}$ . }
 
\par{\textbf{漢字 Note }: Do not read 火山 as かさん. The change from さん to ざん is not because of sequential voicing. Instead, ざん is a later 音 reading that entered Japanese from Chinese. You see this reading in similar words like ${\overset{\textnormal{ひょうざん}}{\text{氷山}}}$ (iceberg). }

\par{ However, 灰 does get voiced due to sequential voicing, which is called ${\overset{\textnormal{れんだく}}{\text{連濁}}}$ . はい is not Chinese. The Chinese reading is かい. This reading is used in words like 灰白色 (grayish white). }
 
\par{\textbf{Prefix Note }: 真っ白 is “pure\slash snow-white”. When you attach 真っ to other colors, be careful of sound changes! 真っ青 = まっさお; 真赤 = まっか; 真っ黒 = まっくろ. Note that 真っ白い is possible for many speakers, but it is not 100\% proper Japanese. }
 
\par{24. 葡萄の木が芽を吹き出す。 \hfill\break
25. ぶどうの芽が吹き出す。 \hfill\break
The grape tree\textquotesingle s shoots will sprout. }
 
\par{\textbf{Transitivity Note }: This pair of sentences shows flexibility in transitivity. However, it is important to note that the latter is more frequently used as it is shorter. This does not, though, make the former incorrect. }
 
\par{\textbf{漢字 Notes }: }
 
\par{1. Many things have 漢字. Although ぶどう is typically not written as 葡萄, it\textquotesingle s important for you to get the feel of what does have a 漢字 spelling, even if it\textquotesingle s not provided to you. }
 
\par{2. 噴き出す may not be used in the plant sense of “to sprout”. 噴, pronounced as ふん in compounds, has the meaning of “erupt”. Thus, 火山の噴火 means “volcano eruption”. }
 
\par{ Another peculiar difference is that ~始める but not ~出す may be used with phrases like ~てください (please…). Even with, ~てもいい (all right to…), ~出す is often unnatural. }
 
\par{26. 食べ始めてください。〇 \hfill\break
食べ出してください。X \hfill\break
Please start to eat. }
 
\par{27. 食べ出してもいいですか。△ \hfill\break
食べ始めてもいいですか。〇 \hfill\break
Is it alright if I start eating? }
      
\section{~かける}
 
\par{ ~かける also marks the starting of an action or circumstance. The exact nuances that it could have are all over the place to the point that it's best to take each case individually and find out what it means. This may sound like a very difficult task to do, but there are a few commonalities across the combinations that you can use to understand this ending better. 1. you already know that it can deal with something starting. Thus, it may show something being \emph{on the verge of }whatever. 2. this "on the verge" meaning is very similar to doing something midway\slash halfway. As a nominal phrase in the form of ~かけ in describing things half-done. }

\par{ In the example sentences below, Examples 28~30 should be straight-forward. The last two involve more thinking, and the latter example will require some more information about the verb かける itself to make sense. }
 
\par{28. 何かを言いかける。 \hfill\break
To be on the verge of starting to say something. }
 
\par{29. 彼女は二度と死にかけた。 \hfill\break
She almost died twice. }

\par{30. 夜が明けかけている。 \hfill\break
The day is breaking. }
 
\par{31. 見知らぬ人に話しかけてはいけない。 \hfill\break
Don't talk to strangers! }
 
\par{32. 彼の顔に ${\overset{\textnormal{けむり}}{\text{煙}}}$ をぷっと吹きかける。 \hfill\break
To puff smoke into his face. }

\par{\textbf{Sentence Note }: 話しかける refers to the point in time when one tries to start a conversation. So, you're still on the verge of something. 吹きかける requires that you understand かける may also mean "to cover\slash put over". Even so, the verb is still somewhat figurative because 煙を吹く also exists with the meaning "to blow smoke", but this wouldn't be used in the context of Ex. 32. }

\begin{center}
 \textbf{~かかる }
\end{center}
 
\par{ ~かかる may also be used to mean "to begin," but it implied that one is on the verge of an end result. It may also be used to show that an action is being directed elsewhere. Either way, because this ending is not that productive, you will have to study how to use it on a case-by-case basis. Furthermore, you will also have to learn the transitivity of the resultant verb. }
 
\par{34. 死にかかる魚って見たことがあまりない。 \hfill\break
I have seldom seen fish at the point of death. }

\par{35. ${\overset{\textnormal{しんつう}}{\text{心痛}}}$ が重く伸しかかった。 \hfill\break
Worries weighed heavily on her. }
 
\par{36. 何時から仕事に取りかかりますか。 \hfill\break
When do you start work? }
 
\par{37. 旅行の ${\overset{\textnormal{じゅんび}}{\text{準備}}}$ をしかかる。 \hfill\break
To begin preparations for the trip. }
    