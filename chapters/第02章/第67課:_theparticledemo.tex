    
\chapter{The Particle でも}

\begin{center}
\begin{Large}
第67課: The Particle でも 
\end{Large}
\end{center}
 
\par{ でも、でも、もう。。。 }
      
\section{The Adverbial Particle でも}
 
\par{ でも is like "even" and often lessens the tone of a sentence. In the latter case, it is seen in contexts where the speaker is often trying to make some sort of suggestion or offer to someone. Options are implied given the literal interpretation of "even". This is meant to be less direct so one doesn't sound pushy. }

\par{ でも is simply the て形 of the copula with も. So, ~ても is very similar but with adjectives and verbs to mean "even though". The negative form of this is ~なくても or ~ないでも, although the latter is uncommon and rather literary. }

\begin{ltabulary}{|P|P|P|}
\hline 

Part of Speech & Pattern & Meaning \\ \cline{1-3}

名詞 & でも & ~Even \\ \cline{1-3}

形容詞 & ~ても & ~(Even) though \\ \cline{1-3}

形容動詞 & ~(であっ)ても & ~(Even) though \\ \cline{1-3}

動詞 & ~ても・でも & ~(Even) though \\ \cline{1-3}

\end{ltabulary}

\par{ Although there are situations where translation may change, the general purpose of these modals is to make a 逆接関係, which is Japanese terminology for a contradictory relationship. This should make sense in light of the translations above. }

\par{\textbf{Particle Note }: It is not used with particles such as が, を, and は, but it can be seen after other particles such as に, と, から, and まで. However, in slightly older Japanese, you can see combinations such as ~をでも. It is not advisable to use this as it is, again, not used anymore.  }

\par{\textbf{Origin Note }: でも comes from the て form of だ + も. }

\begin{center}
 \textbf{Examples } 
\end{center}
 
\par{1. 先生 \textbf{でも }分からないな。(Masculine) \hfill\break
 \textbf{Even }the teacher doesn't understand it. \hfill\break
 \hfill\break
2. お ${\overset{\textnormal{ちゃ}}{\text{茶}}}$ \textbf{でも }飲みますか。 \hfill\break
Literally: Do you want to drink \textbf{even }(if it's) tea? \hfill\break
Shall we drink tea or something? \hfill\break
 \hfill\break
3. ここ \textbf{でも }、あそこ \textbf{でも }、子供が遊んでいます。 \hfill\break
There are children playing here and even over there. \hfill\break
 \hfill\break
4. あの人のいうことは、 ${\overset{\textnormal{まんざらうそ}}{\text{満更嘘}}}$ \textbf{でも }ない。 \hfill\break
What that person says isn't all a lie. }

\par{5.  彼は \textbf{またしても }「お ${\overset{\textnormal{ねが}}{\text{願}}}$ いします」と来た。 \hfill\break
He came \textbf{again }saying, "Please." }

\par{6a. 彼が言わ \textbf{ないでも }、分かるでしょう。 (Usually don't use) \hfill\break
6b. 彼が言わ \textbf{なくても }、分かるでしょう。(More natural) \hfill\break
 \textbf{Even if }he \textbf{doesn't }say, you'll probably understand. }

\par{7. \textbf{どうでも }いい。 \hfill\break
I don't care. \hfill\break
Literally: \textbf{Whatever }it is, it's OK. }
 
\par{8. コーヒー \textbf{でも }どう?おごるよ。 \hfill\break
How about a coffee on me? }
 
\par{9. どんなことが \textbf{あっても }、 ${\overset{\textnormal{ぜったい}}{\text{絶対}}}$ に行くんだ。(Assertive) \hfill\break
 \textbf{No matter }what, I'm definitely going to go! }
 
\par{10. 雨が ${\overset{\textnormal{ふ}}{\text{降}}}$ ってい \textbf{ても }、ゲームは楽しかった。 \hfill\break
 \textbf{Even though }it had been raining, I enjoyed the game. }
 
\par{11. いずれにしても飲酒運転は絶対にしません。 \hfill\break
In any case, I absolutely do not drink and drive. }
 
\par{12. 私でも分かります。 \hfill\break
Even I understand. }
 
\par{13. そんな例は \textbf{いくつでも }見つかる。 \hfill\break
You can find \textbf{ever so many }such examples. \hfill\break
 \hfill\break
14. 彼は ${\overset{\textnormal{まず}}{\text{貧}}}$ しく \textbf{ても }${\overset{\textnormal{しあわ}}{\text{幸}}}$ せだ。 \hfill\break
He is happy \textbf{even }though he is poor. }
 
\par{15. メキシコでは、冬になっ \textbf{ても }、雪が ${\overset{\textnormal{ふ}}{\text{降}}}$ らない。 \hfill\break
It doesn't snow in Mexico \textbf{even }in the winter. }
 
\par{16. 居ても立ってもいられなかった。 \hfill\break
I couldn't sit still. }
 
\par{\textbf{Set Phrase Note }: This last phrase literally means that one couldn't even be sitting or standing. 居る originally meant "to sit", and it is used in its original sense in this phrase. }
 
\begin{center}
 \textbf{何と言っても: Undeniably (lit. = no matter what they say) }
\end{center}
 
\par{17. \textbf{何といっても }この曲が一番だよね。 \hfill\break
No matter what you say, this song is the best. }

\par{18. \textbf{何といっても }${\overset{\textnormal{いのち}}{\text{命}}}$ だけは ${\overset{\textnormal{だいじ}}{\text{大事}}}$ だ。 \hfill\break
Life is \textbf{no doubt }but the only important thing. }
 
\par{ でいい is like "it's all right\slash OK to", でかまわない is like "I don't care\slash mind if you", and でさしつかえない is like "it won't interfere if you". All three generally represent the same thing, and their purpose is to show permission. でも can be used instead of で, and the inclusion of も makes the phrases less direct. でも may appear as ても as these are also て form speech modals. }

\par{${\overset{\textnormal{}}{\text{19. 新}}}$ しく \textbf{なくてもいい }ですか。 \hfill\break
Is it \textbf{OK if }it's \textbf{not }new? }

\par{20. ${\overset{\textnormal{ちゅうこしゃ}}{\text{中古車}}}$ \textbf{でもかまいませんか }。 \hfill\break
 \textbf{Do you mind if }it's a used vehicle? }

\par{21. ${\overset{\textnormal{さんか}}{\text{参加}}}$ し \textbf{なくてもいいでしょうか }。 \hfill\break
 \textbf{Is it all right }for me not to participate? }

\par{22. ${\overset{\textnormal{しんぱい}}{\text{心配}}}$ し \textbf{なくていい }から。 \hfill\break
It's \textbf{all right not }to worry. }

\par{23. トイレを ${\overset{\textnormal{つか}}{\text{使}}}$ っ \textbf{ていいですか }。 \hfill\break
\textbf{May I }use the bathroom? }
 
\par{\textbf{Classroom Note }: The above sentence is an essential classroom phrase. }

\par{24. ${\overset{\textnormal{いえ}}{\text{家}}}$ は ${\overset{\textnormal{}}{\text{古}}}$ く \textbf{てもさしつかえませんか }。 \hfill\break
 \textbf{Is it all right }if the house is old? }

\par{25. ${\overset{\textnormal{じゅういっかい}}{\text{十一階}}}$ のシングルルーム \textbf{でよろしいですか }。 \hfill\break
Is a single room on the eleventh floor \textbf{all right }? }

\par{26. \textbf{どれでも }けっこうですよ。 \hfill\break
 \textbf{Anything }will do. }

\par{27. 何をやっ \textbf{ても }一日と続かない。 \hfill\break
\textbf{No matter }what you do, it won't continue for another day. }

\begin{center}
\textbf{~てでも }
\end{center}

\par{ This pattern is used to show that no matter what you do with whatever means necessary, you carry it out. The verb phrase that follows should be one that shows intention. This phrase is typically translated as "no matter if", but let this not detract from the fact that what follows is a statement of what is\slash should be carried out rather than showing despair. }

\par{28. 若い時の ${\overset{\textnormal{くろう}}{\text{苦労}}}$ は買ってでもせよ。(Proverb) \hfill\break
By all means do so no matter if you buy hardships in youth. }
 
\par{\textbf{Proverb Note }: This proverb says that hardships when you\textquotesingle re young become great experiences for the future, so it is best to seek them out. }
 
\par{29. 金で買ってでも、苦労した方がいいのは、若いうちだけだ。 \hfill\break
No matter if you buy with money, it's only best to have hardships when one is young. }
    