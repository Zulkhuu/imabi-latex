    
\chapter{Potential II}

\begin{center}
\begin{Large}
第83課: Potential II: できる  
\end{Large}
\end{center}
 
\par{ Other than using られる with 一段 verbs and changing 五段 into special potential verbs, you can also use できる in the pattern ~ことができる. できる more or less correlates to the potential form of する. As you can see, though, they do not come from the same word. }
      
\section{できる}
 
\par{ する\textquotesingle s potential form is できる・出来る. You don't have to add or do anything else. In Japanese, you can use できる with things regarding ability. So, the equivalent of "can English" would be a valid expression. }

\par{1. ${\overset{\textnormal{どりょく}}{\text{努力}}}$ と運が ${\overset{\textnormal{あい}}{\text{相}}}$ まって ${\overset{\textnormal{ゆうしょう}}{\text{優勝}}}$ できた。 \hfill\break
Together with effort and luck, I was able to be victorious. }
 
\par{2. 加藤さんは ${\overset{\textnormal{ご}}{\text{碁}}}$ が出来ます。 \hfill\break
Mr. Kato can play go. }
 
\par{3a. 日本語ができます。〇 \hfill\break
3b. 日本語を話しができます。X \hfill\break
I can speak Japanese. \hfill\break
Literally: I can do Japanese. }
 
\par{\textbf{Grammar Note }: You don't make a verb a noun and use "Noun + ができる (see ことができる). This example shows how できる can show ability\slash talent. So, ゴルフができない would also be correct. }
 
\par{4. 彼らは ${\overset{\textnormal{さんせい}}{\text{賛成}}}$ も ${\overset{\textnormal{はんたい}}{\text{反対}}}$ もできない。 \hfill\break
They can't even agree or disagree. }

\par{5. もう ${\overset{\textnormal{いっぱく}}{\text{一泊}}}$ できますか。 \hfill\break
Can I stay for one more day? }

\begin{center}
 \textbf{Meanings Other Than Potential }
\end{center}

\par{ The 可能形 is unable to show appearance or what something is made of, but できる sure can. Those are the "other" usages of it mentioned at the beginning of this lesson. }
 
\par{6. 水は ${\overset{\textnormal{さんそ}}{\text{酸素}}}$ と ${\overset{\textnormal{すいそ}}{\text{水素}}}$ からできている。 \hfill\break
Water is made up of oxygen and hydrogen. }

\par{7. ご飯、できましたよ。 \hfill\break
Dinner is made! }

\par{8. 人生って、悩みと苦しみからできてるんだなあ。 (男性語) \hfill\break
Life really is made from worries and suffering. }

\par{9. 人間は体と心からできているのに、みんな体のことばかりに ${\overset{\textnormal{いっしょうけんめい}}{\text{一生懸命}}}$ 気を使って、心のほうに ${\overset{\textnormal{えいよう}}{\text{栄養}}}$ を与えることを忘れている。 \hfill\break
Although humans are made up of the body and the heart, everyone focuses solely on their bodies and forget about providing nurture to their hearts. }

\begin{center}
\textbf{~ことができる }
\end{center}

\par{  Instead of using ~(ら)れる, you may use ことができる. This pattern is arguably more common despite being longer. Without qualification, it will sound as if you can't do the phrase at all when used in the negative. This does not have the requirement of being only used with verbs of volition. This can be used to show the potential of something to happen. }
 
\par{10. 私は日本語を話すことができます。 \hfill\break
I can speak Japanese. }
 
\par{11. コンピューターを使うことができるか。 (Rough) \hfill\break
Can you use a computer? }
 
\par{12. アメフトの ${\overset{\textnormal{しあい}}{\text{試合}}}$ を見ることができました。 \hfill\break
I was able to see the American football game. }
 
\par{13. 100万からの ${\overset{\textnormal{しょめい}}{\text{署名}}}$ を ${\overset{\textnormal{あつ}}{\text{集}}}$ めることができないでしょう。 \hfill\break
You won't be able to collect over a million signatures. }
 
\par{14. その漢字を読むことができませんでした。 \hfill\break
I could not read that Kanji. }
 
\par{\textbf{Particle Note }: が in this pattern, due to its specific nature, would otherwise make it sound like you in particular are the one that "can"; however, this is not necessarily the case. Nowadays, ~ことができる has more or less become an affirmative, refined pattern. }

\begin{center}
 \textbf{~することができる }
\end{center}

\par{ ~することができる shouldn't be used excessively, and it should be avoided in promotion-oriented settings. However, it shows up a lot in magazines and such because it is felt by many to be more polite. Yet, when there is a word limit, できる should do. For a minority of speakers, this phrase is ungrammatical. However, this is an extreme position based mainly on a misconception of case marking in Japanese. In the potential, が does not mark a subject. It marks an object. This may be confusing, but other languages have similar alignment rules. This is simply a quirk in Japanese, and it's not like natives were confused on how to deal with it until being introduced with half-baked knowledge about Western grammar terms. }
 
\par{15. ダウンロードすることができます! \hfill\break
You can download it!  }

\begin{center}
 \textbf{Other Ways to Make する Potential }
\end{center}

\par{ There are four potential routes that you can take to make the potential with する-verbs. }
 
\begin{enumerate}
 
\item Single characters      like 愛 that become      verbs with する \hfill\break
\textrightarrow + することができる; -し得(え・う)る; + せる.  
\item Compounds\slash loan      words-- ${\overset{\textnormal{けいけん}}{\text{経験}}}$ --that      become verbs with する. \hfill\break
\textrightarrow + (することが)できる; し得る.  
\item Verbs with する that become      voiced such as 感ずる・じる. \hfill\break
\textrightarrow Conjugate as 一段 verbs; -じ得る; + ことができる.  
\item Verbs with a っ such as 発する. \hfill\break
\textrightarrow + ことができる; -得る.  
\end{enumerate}
 
\par{\textbf{Verb Note }: Verbs like 愛 す are treated as 五段 verbs. Also note that something like 理解することができる is typically deemed to be unnecessarily wordy. }
 
\par{\textbf{Curriculum Note }: See Potential III for ~得る. }
 
\par{16. 彼は ${\overset{\textnormal{きけん}}{\text{危険}}}$ を ${\overset{\textnormal{さっ}}{\text{察}}}$ することができないやつだな。 \hfill\break
He's a guy that can't sense danger, isn't he? }
 
\par{17. ${\overset{\textnormal{きみ}}{\text{君}}}$ を愛せない。 \hfill\break
I can't love you. }
 
\par{18. 経験(《すること》が)できる。 \hfill\break
To be able to experience. }
 
\par{19. 彼は ${\overset{\textnormal{だれ}}{\text{誰}}}$ でも愛せる人だね。 \hfill\break
He's a person that can love anybody, isn't he? }
 
\par{20. これでも中学生の ${\overset{\textnormal{とき}}{\text{時}}}$ に日本語を ${\overset{\textnormal{りかい}}{\text{理解}}}$ できませんでした。 \hfill\break
Believe it or not, I couldn't understand Japanese in junior high.  }
    