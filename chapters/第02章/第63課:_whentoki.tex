    
\chapter{Absolute Time VI When}

\begin{center}
\begin{Large}
第63課: Absolute Time VI: When: 時 
\end{Large}
\end{center}
   時 is a noun literally meaning “time.” It is most frequently used after dependent clauses to mean “when” as in "when one does" and not as the question "when". Despite being easy to translate, its usage is actually complex due to tense differences. Pay attention to notes on particles and parts of speech as both issues are extremely relevant to this discussion.       
\section{The Complicated Grammar of 時}
 
\par{ 時 is used to mean "when" when used with time phrases. The tense and part of speech used with 時 is very important as you will see throughout this lesson. }

\begin{center}
\textbf{Tense }
\end{center}

\par{ As mentioned in the introduction, 時 is complicated because of tense. Why? Considering that the following combinations are possible, it\textquotesingle s easy for students to get them all confused. }

\begin{ltabulary}{|P|P|P|P|}
\hline 

~するとき、~する & ~するとき、~した & ~したとき、~する & ~したとき、~した \\ \cline{1-4}

\end{ltabulary}

\par{  English likes tense to be consistent in a sentence, so the first and last combinations tend to be easier for students to acquire. The middle ones, though, are different to say the least. }

\par{1. ご飯を食べるとき、いつも手を洗います。 \hfill\break
When(ever)\slash before I eat a meal, I always wash my hands. }

\par{2. ご飯を食べ始めるとき、「いただきます」という。 \hfill\break
You\slash we say “Itadakimasu” when(ever)\slash before we start eating a meal. }

\par{3. 日本に帰ったとき、まず(お)寿司を食べました。 \hfill\break
I first ate sushi right when\slash after I returned home to Japan. }

\par{4. 昨夜、ご飯を食べるとき、手を洗いませんでした。 \hfill\break
Last night, before I ate, I didn't wash my hands. }

\par{5. ご飯を食べ終わったとき、いつも「ご馳走さまでした」と言います。 \hfill\break
You\slash we say “Gochisou-sama deshita” after finishing a meal. }

\par{ It's not that the other combinations can\textquotesingle t be translated into English. From these last two examples, it looks as if there is a before-after relation being made in the dependent clause in relation to the main clause. This is often the case, but as you will see, it is not the only possible situation. }

\par{ Consider the difference between the following two sentences. }

\par{6. ベトナムへ行く時に、ベトナム語を少し習いました。 (日本で) \hfill\break
Before going to Vietnam, I studied a little Vietnamese. }

\par{7. ベトナムへ行った時に、ベトナム語を少し習いました。 (ベトナムで) \hfill\break
When I went to Vietnam, I studied a little Vietnamese. }

\par{ The only thing different was the form of 行く before 時. The distinction does not tell whether the sentence is in the past or not because that\textquotesingle s already evident. What the tense distinction in the subordinate clause shows is whether the action in that clause has realized or not. Thus, 時 functions as the base time, and the action in the clause before it may either happen before or after the action of the main clause. }

\begin{center}
\textbf{前後関係 and 時 }
\end{center}

\par{ There are instances when ignoring this 前後関係 (before-after relationship) can cause the sentence to be ungrammatical. }

\par{8. 朝起きるときに、口をすすぐ。X \hfill\break
When I wake up in the morning, I rinse my mouth. △ }

\par{Note: You can't be rinsing your mouth the moment you're waking up. Although in English you can get by with saying this sentence with "the very moment you're waking up" not necessarily being implied, this is implied in Japanese, thus causing the sentence to be ungrammatical with 時. }

\par{9. 寝たときに、犬を外に行かせる。X \hfill\break
I make the dog go outside after I go to sleep. X }

\par{Note: You can't be taking your dog outside after you've already gone to bed unless you've mastered sleep walking. }

\par{ When 時 is used to show a 前後関係 such as in the previous examples, each verb can be viewed as being independent from each other. For those that have a problem with a future event being used with the past tense, more thought into “after” statements in English needs to be made. }

\par{10. ソウルの空港に着いたときに、電話します。 \hfill\break
I'll call you after I arrive in the Seoul Airport. }

\begin{center}
 \textbf{X \& Y \textrightarrow  Instantenous }
\end{center}

\par{ It so happens, though, that when action X and Y are instantaneous and seem to happen at the same time, both the non-past and past tense seem to be able to be used interchangeably before とき without a change of meaning. However, note the detail in this situation. You shouldn't run with this and apply it haphazardly to everything else mentioned in this lesson to the contrary. }

\par{11. そのボールが\{ ${\overset{\textnormal{は}}{\text{爆}}}$ ぜる・爆ぜた\}とき、がちゃんと大きな音がした。 \hfill\break
There was a big bang right when the ball exploded. }

\par{  As stated earlier, there are situations where 時 does \emph{\textbf{not }}express a 前後関係. This is evident when both actions X and Y essentially happen simultaneously. }

\par{12. 中国 ${\overset{\textnormal{たいざいちゅう}}{\text{滞在中}}}$ 、中国料理を食べるときは、 ${\overset{\textnormal{はし}}{\text{箸}}}$ を使いました。 \hfill\break
When I was in China, I used chopsticks whenever ate Chinese food. }

\par{13. 神戸へ出張するときは、新幹線に乗りました。 \hfill\break
I rode the Shinkansen when I went to Kobe. }

\par{14. 僕んち(に)入るときは、いつも ${\overset{\textnormal{うらぐち}}{\text{裏口}}}$ から入ってたんだけど、やめて。(Very colloquial) \hfill\break
Whenever you enter my house, you've always come through the back, but quit it. }

\begin{center}
 \textbf{Tense Interchangeability }
\end{center}

\par{ Time phrases in addition to 時 allow for variation in tense before 時. For instance, notice how the word いつも allows either the non-past or the past to be used before 時 though the final verb in the sentence is in the past tense. This is possible because no specific event of the past is being cited. If it were a specific event, the past tense would inevitably be used. }

\par{15. 韓国の方とビビンバを\{食べる・食べた\}ときは、いつも ${\overset{\textnormal{ぎんせい}}{\text{銀製}}}$ のお箸で食べました。 \hfill\break
Whenever I ate bibimbap with Koreans, I always ate with silver chopsticks. }

\par{\textbf{Word Note }: 方 = 人. This is a formal word. You could also say 韓国の人 or 韓国人. }

\par{16. 小林さんが ${\overset{\textnormal{こうえんかい}}{\text{講演会}}}$ を開くときは、 ${\overset{\textnormal{たしょう}}{\text{多少}}}$ 遠くてもいつも ${\overset{\textnormal{き}}{\text{聴}}}$ きに行きました。 \hfill\break
Whenever Mr. Kobayashi opened a lecture seminar, no matter if it was far, I always went to go listen    (to his lectures). }
 
\par{17. 雪が\{降る・降っている\}ときは、図書館で ${\overset{\textnormal{けんきゅうしょ}}{\text{研究書}}}$ を読んだ。 \hfill\break
Whenever it \{snowed\slash was snowing\}, I read studies at the library. }
 
\par{18. 電気が\{切れる・切れている\}ときは、 ${\overset{\textnormal{かいがん}}{\text{海岸}}}$ に泳ぎに行った。 \hfill\break
When the electricity \{cut off, was cut off\}, I went to the beach to swim. }

\begin{center}
 \textbf{With Adjectives }
\end{center}

\par{ When the main clause is past tense and the adjective before 時 is also in the past tense, 時 gives a rather reminiscent feeling. Otherwise, it wouldn't quite have that feeling. }

\par{19. この ${\overset{\textnormal{ろうけん}}{\text{老犬}}}$ は若いとき、とても美しかった。 \hfill\break
When this old dog was young, it was really beautiful. }

\par{20. 君が若かったとき、わしは君を ${\overset{\textnormal{だいじ}}{\text{大事}}}$ にしてやったけど、今はわしの ${\overset{\textnormal{そんざい}}{\text{存在}}}$ を ${\overset{\textnormal{みと}}{\text{認}}}$ めてくりゃしないや。(Old person) \hfill\break
When you were young, I would treat you dearly, but now you won't even recognize my existence. }

\par{\textbf{Contraction Note }: くりゃしない = くれはしない = a very emphatic くれない. }

\par{21. 天気が悪い時に、ビデオゲームをした。 \hfill\break
When the weather was bad, I played video games. }

\par{\textbf{Particle Note }: With the addition of に, the last sentence pinpoints a specific time in the past. }

\begin{center}
 \textbf{With いる }
\end{center}

\par{ This is the same with いる. So, for いる, when the actions in the clauses occur in the same time period, either tense can be used when the main sentence is in the past with no change in meaning. The only difference then is that using the past tense gives a stronger sense of looking back in time. }

\par{22. 私は埼玉に\{いる・いた\}時、 ${\overset{\textnormal{にかいだ}}{\text{二階建}}}$ てのアパートでタイ人の友達と住んでいました。 \hfill\break
When I was in Saitama, I lived with a Thai friend in a two story apartment. }

\par{ When it is the case that the predicate and condition described by いる are not in the same time period, the main clause\textquotesingle s action occurs within the range of the action of the dependent clause. So, if the dependent clause action is Y and the main clause action is X, X happens within Y. }

\par{23. 僕が海外してたとき、彼女は僕を ${\overset{\textnormal{うらぎ}}{\text{裏切}}}$ って、友人の知り合いと付き合ってあいつのマンションに引っ越したんだ! \hfill\break
When I was overseas, my girlfriend dumped me, got together with an acquaintance of a friend of mine and moved into his apartment! }

\par{24. メールを読んでたとき、地震があって、とってもびっくりしたよ。 \hfill\break
An earthquake happened when I was reading my e-mails, and I was really shocked. }

\begin{center}
 \textbf{With Nominal Phrases }
\end{center}

\par{ Judgments seen thus far apply for when 時 follows nominal phrases. Notice that の is interchangeable with である in this case. However, when the copula is used, it gives a stronger sense of retrospection or looking into the future. This is due to the introduction of a tense item. Consider other particles that can be used with 時 like から and でも. }

\par{25. 僕は学生[の・だった]時、あまり本を読まなかった。 \hfill\break
I didn't read a lot of books when I was a student. }

\par{26. どんなときでもずっと側にいるよ。 \hfill\break
I'll be by your side no matter what. \hfill\break
 \hfill\break
27. 高校に入学したときから、数えて20年が経ちました。 \hfill\break
20 years have passed since I entered high school. }

\begin{center}
 \textbf{Particle Troubles: ~とき、~ときに、~ときは、~ときには }
\end{center}

\par{ Of course, に・には・nothing after 時 is problematic. You may have gotten a sense of the differences between them by looking at all of the examples so far, but to address this explicitly, consider the following. As mentioned earlier, ~ときに mentions a certain point in time. }

\par{ So, ambiguity as to whether something happens at a certain time or not is caused by no particle being used. On the other hand, the particle は is emphatic and usually with 時 to contrast with other times. With ときには, you are emphasizing a particular part in time in contrast with other points in time. The following examples give a nice mix of this and the material covered thus far. }

\par{28. 日本に行く時には、空港でウィスキーを買いました。 \hfill\break
I bought whiskey right before I went to Japan. }

\par{29. 食べている時に、テレビは見ません。 \hfill\break
I don't watch TV (right) when I'm eating. }

\par{30. 学校に行く時、電話します。 \hfill\break
When\slash before I go to school, I'll call you. (before). }

\par{31. ${\overset{\textnormal{さくら}}{\text{桜}}}$ がきれい[な・だった]時に、日本へ行った。 \hfill\break
I went\slash used to go to Japan when(ever) the cherry blossoms were pretty (in bloom). }

\par{32. 分からない時は、先生に聞いてください。 \hfill\break
When you don't understand, ask a teacher. }

\par{33. お金がないとき、何をしないで ${\overset{\textnormal{がまん}}{\text{我慢}}}$ しますか。 \hfill\break
When you don't have money, what do you endure without? }

\par{34. 前回台北に来たときは、誰にも会わなかったが、今回は多くの友人に会いました。 \hfill\break
Last time I came to Taipei, I didn't meet anyone, but this time I met a lot of friends. }

\par{\textbf{Reading Note }: 台北 is officially read by NHK as たいほく. However, many speakers no longer read it this way and say たいぺい, perhaps out of respect of the modern Chinese pronunciation. Similarly, though, the standard reading for 北京 (Beijing) is ぺきん, which is based off of the older pronunciation of the city's name (Peking). }
    