    
\chapter{The Non-Question "How"}

\begin{center}
\begin{Large}
第86課: The Non-Question "How": ~方(かた) 
\end{Large}
\end{center}
 
\par{ This lesson is all about how to address the word “how” in Japanese.It requires that you know a little bit about こそあど and the particles か and でも as related phrases are created with the “how” words to be taught in this lesson. The reason for why “how” was not taught in the first lesson on interrogatives is that a lot more information is involved. Now, it is time to tackle the challenge of expressing “how” in Japanese! }
      
\section{~方: The Non-Interrogative "How”}
 
\par{ The "how" as in “how to express” describes means\slash way of doing something is expressed with the suffix ~方. }

\par{1. 誰か、 ${\overset{\textnormal{でんせつ}}{\text{伝説}}}$ のポケモンの ${\overset{\textnormal{つか}}{\text{捕}}}$ まえ方を教えてくださいませんか。 \hfill\break
Could someone please teach me how to catch legendary Pokemon? }

\par{2. どんな言い方をされても ${\overset{\textnormal{た}}{\text{絶}}}$ えるしかない? \hfill\break
No matter what expression is used at me, is there nothing else than to endure? }

\par{3. 生き方を変えたい。 \hfill\break
I want to change my way of living. }

\par{4. 本当の自分の知り方を教えていただきました。 \hfill\break
I was taught how to understand my true self. }

\par{5. 不完全な死に方をすると、不完全な ${\overset{\textnormal{れい}}{\text{霊}}}$ になってしまいますよ。 \hfill\break
If you die an imperfect death, you become an imperfect spirit. }

\par{6. ${\overset{\textnormal{しょく}}{\text{食}}}$ ${\overset{\textnormal{えんすい}}{\text{塩水}}}$ と砂糖水の凍り方を比べてみましょう。 \hfill\break
Let's compare the way to freeze saline solutions and sugar water. }

\par{7. 鉄の ${\overset{\textnormal{と}}{\text{溶}}}$ かし方を教えてください。 \hfill\break
Please teach me how to melt steel? }

\par{8. ${\overset{\textnormal{ぬ}}{\text{縫}}}$ い方が難しいの? \hfill\break
Is the sewing technique difficult? }

\par{9. 村のおじさんに簡単にできる人の ${\overset{\textnormal{のろ}}{\text{呪}}}$ い方を教えてもらったんだぞ。 \hfill\break
I had an old village man teach me how to easily curse people. }

\par{10. 山田先生は今日、 ${\overset{\textnormal{ほにゅう}}{\text{哺乳}}}$ ${\overset{\textnormal{どうぶつ}}{\text{動物}}}$ の産み方について講座をした。 \hfill\break
Yamada Sensei lectured today about the way mammals give birth. }

\par{ When you use する verbs, you have to use ~の仕方 after the stem. }

\begin{ltabulary}{|P|P|P|}
\hline 

料理 \textrightarrow  料理の仕方 & 運転 \textrightarrow  運転の仕方 & 日本語の勉強 \textrightarrow  日本語の勉強の仕方 \\ \cline{1-3}

\end{ltabulary}

\par{11. 海外への電話の仕方を教えてくれませんか。 \hfill\break
Could you tell me how to make overseas phone calls? }

\par{12. 神様から好印象を持たれる、正しい初詣の仕方とは何でしょう? \hfill\break
What exactly is the correct way to pay one's visit to the shrine for the first time in the year to win good favor from the gods? }

\par{13. 化粧の仕方を見直しましょう。 \hfill\break
Let's redo your make-up routine.  }
    