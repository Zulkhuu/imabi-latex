    
\chapter{~てあげる, ~てくれる, \& ~てもらう}

\begin{center}
\begin{Large}
第95課: ~てあげる, ~てくれる, \& ~てもらう 
\end{Large}
\end{center}
 
\par{ These endings normally get taught rather quickly in most textbooks despite the fact that there is so much that can be said about them. Many of the mistakes from students are caused by the lack of guidance. In an attempt to cover the essential information that can be obtained at this level without adding new grammar structures, this lesson will help you explore the differences that cause students the most trouble. }
      
\section{The Basics}
 
\par{ The giving verbs \textbf{show favor }when used with the particle て. Though the following summary may seem easy enough, we will go to compare and contrast to see where exactly the limits to these expressions lie. }
 
\par{~ \textbf{てあげる }, more polite than ~てやる, shows that one('s in-group) gives a favor to another person\slash party. It's generally used towards equals or even (non-)living things. In certain situations, it may be patronizing when "giving the favor" involves demoting of status. }
 
\par{~ \textbf{てくれる }may show that someone does something for another person's benefit or even disadvantage. It is to note that its 命令形 is normally ~ てくれ rather than ~てくれろ. }
 
\par{~ \textbf{てもらう }shows that someone received a favor or disfavor from someone. ~ てもらえませんか can be used to politely ask for favor. The honorific forms ~ていただけませんか and ~ていただけないでしょうか are also very important. To see how から is used with this, click link  . For now, we will only see agents marked with に. }

\par{\textbf{漢字 Note }: These endings do have 漢字, but they are usually not used. ~ていただく is more common in 漢字 than the others, and its spelling is ~て頂く. }
 
\par{\textbf{Variant Note }: The respectful variants replace them in the appropriate environment. }
 
\begin{center}
\textbf{Examples } 
\end{center}

\par{1. 知らない人たちに道を教えてあげた。 \hfill\break
I showed the way to strangers. }
 
\par{2a. ピザを ${\overset{\textnormal{たくはい}}{\text{宅配}}}$ してもらえます。(Rather literary) \hfill\break
2b. ピザを ${\overset{\textnormal{はいたつ}}{\text{配達}}}$ してもらえます。(More natural) \hfill\break
You can have pizza delivered to you. }
 
\par{3. 友だちはプレゼントを贈ってくれました。 \hfill\break
My friend sent me a gift. }

\par{\textbf{漢字 Note }: 贈る is used over 送る because of the bestowing\slash giving of something special, in this case a present. }
 
\par{4. 知っていて教えてくれないものは本当に ${\overset{\textnormal{じこちゅうしんてき}}{\text{自己中心的}}}$ だ。 \hfill\break
A person that knows but doesn't tell anyone is really self-centered. }
 
\par{\textbf{Suffix Note }: ~的 is like "-like; -ic(al)" and is used to make nouns, mainly Sino-Japanese, into adjectives. Other examples include 日本的 (Japanese-like), 男性的 (masculine), 経済的 (economical), etc. }

\par{\textbf{Word Note }: 教える is used instead of 言う because the context shows that information is being withheld and not transmitted\slash taught\slash informed to others. }

\par{\textbf{漢字 Note }: もの could be written as 者, which is a meaning that further downgrades "person". }
 
\par{5a. 医者に ${\overset{\textnormal{み}}{\text{診}}}$ てもらう。 \hfill\break
5b. 医者に見られる。 △ \hfill\break
To be seen by the doctor. }

\par{\textbf{Grammar Note }: The second option would literally mean that you were seen\slash spotted by the doctor. }
 
\par{6. この ${\overset{\textnormal{ぶん}}{\text{分}}}$ は ${\overset{\textnormal{つぐな}}{\text{償}}}$ ってもらうから。 \hfill\break
You'll pay for this. }
 
\par{7. あいつをとっちめてやる!(Vulgar) \hfill\break
I'll teach him a lesson! }
 
\par{8. 友達に手紙をフランス語に ${\overset{\textnormal{ほんやく}}{\text{翻訳}}}$ してもらった。 \hfill\break
I had my letter translated into French by my friend. }
 
\par{9. どうか静かにしていただけませんか。 \hfill\break
Would you kindly stop making that noise? \hfill\break
Literally: Could I have you somehow get quiet? }
 
\par{10. もっと上の階の部屋にしていただきたいんですが。 \hfill\break
Could you give me a room on a higher floor (if it is possible)? }
 
\par{11. もうすこし時間をいただけないでしょうか。 \hfill\break
Could you please give a little more time? \hfill\break
Could I please have a little more time? }
 
\par{12. 「シングル・ダブルルームを ${\overset{\textnormal{よやく}}{\text{予約}}}$ したいんですが。」「 ${\overset{\textnormal{なんぱく}}{\text{何泊}}}$ ですか。」「 ${\overset{\textnormal{にはく}}{\text{二泊}}}$ です。」「いくらですか。」「4万円です。」「今部屋を見せてもらえますか。」「すみませんが、今は ${\overset{\textnormal{まんしつ}}{\text{満室}}}$ です。」「そうですか、分かりました。」「お名前をお ${\overset{\textnormal{ねが}}{\text{願}}}$ いします。予約の ${\overset{\textnormal{かくにん}}{\text{確認}}}$ もお願いします。」 \hfill\break
“I'd like to reserve a single\slash double room.” “How many nights?” “Two, how much is it?” “It's 40,000 yen.” “Can I see the room now?” “Sorry, but we are full now.” “Oh, that's alright.” “Your name please. Please also confirm your reservation.” }
 
\par{13. もう一度歌っていただけませんか。 \hfill\break
Could you sing once more? }
 
\par{14. 彼女に新しい ${\overset{\textnormal{ようふく}}{\text{洋服}}}$ を買ってやった。 \hfill\break
I bought a new dress for her. }
 
\par{\textbf{Vocab Note }: 洋服 = Western clothing. ${\overset{\textnormal{いふく}}{\text{衣服}}}$ ="garments". You buy ${\overset{\textnormal{いりょう}}{\text{衣料}}}$ ( ${\overset{\textnormal{ひん}}{\text{品}}}$ ). Japanese style clothes = ${\overset{\textnormal{わふく}}{\text{和服}}}$ . 服 is clothes as in dresses or suits. ${\overset{\textnormal{いるい}}{\text{衣類}}}$ = article of clothing (caps, robes, etc.). ${\overset{\textnormal{ふくそう}}{\text{服装}}}$ = "clothes; dress: appearance". ${\overset{\textnormal{ふくそうきてい}}{\text{服装規定}}}$ = "dress code". "Dress; outfit" can also be 衣装. "Outfit; costume" = ${\overset{\textnormal{ふんそう}}{\text{扮装}}}$ . The original word for clothes is ${\overset{\textnormal{ころも}}{\text{衣}}}$ . Now, it refers to a "priest's robe", which can also be ${\overset{\textnormal{そうい}}{\text{僧衣}}}$ or ${\overset{\textnormal{ほうえ}}{\text{法衣}}}$ . }
 
\par{\textbf{ちょうだい }}
 
\par{ちょうだい can be used as a "humble verb" to mean "to receive" or "to eat and drink". By women and children, either with て or not, it makes a request. In 漢字 it's 頂戴. It is not written as such with て. }
 
\par{15. 僕にもちょうだい。 \hfill\break
Give me some too. }
 
\par{16. ちょっと待ってちょうだい。 \hfill\break
Please wait a bit. }

\par{17. ${\overset{\textnormal{けっこう}}{\text{結構}}}$ な ${\overset{\textnormal{しな}}{\text{品}}}$ を頂戴いたしましてありがとうございます。 \hfill\break
Thank you so much for giving me this nice gift. }
 
\par{18. 「もっと ${\overset{\textnormal{め}}{\text{召}}}$ し上がりませんか」「ありがとう、もう十分頂戴しました」 \hfill\break
“Would you like some more?” “Thank you, but I have had enough”. }
 
\par{\textbf{Word Note }: The person serving used the respectful form of "to eat", 召し上がる. The recipient didn't just say something like 結構です. They appear to be both high rank. }
 
\par{19. キャンディーをちょうだい。 \hfill\break
Candy, please. }
 
\par{\textbf{~てあがる・~ }\textbf{て(い)やがる }}
 
\par{Although this looks similar to ~てあげる, it's not the same thing. Rather, this is a vulgar and rather offensive version of ~ている. }
 
\par{20. してあがる・しやがる・していやがる・しやがる = している  (Not 1st Person) }
 
\par{21. いつからそこにいやがった!? \hfill\break
Since when have you been there!? }
 
\par{\textbf{Word Note }: いやがった is not as in the past tense of 嫌がる (to detest). The verb いる・居る is being used. }
      
\section{Compare and Contrast: ~てくれる VS ~てもらう}
 
\par{ Having seen several sentences of ~てくれる and ~てもらう, it superficially looks like so long as you address particle differences, you can rephrase a sentence with each other. This for many cases is true, but there are many important things to keep in mind. First, consider the following. }
 
\par{22a. 道に迷った旅行者は、父に、駅までの道を教えてもらいました。〇 \hfill\break
22b. 父は、道に迷った旅行者に、駅までの道を教えてくれました。X \hfill\break
The lost traveler was told the way to the station by my father. ) }
 
\par{ In 「XがYにAてくれる」, X is a person other than the speaker, Y is the speaker or someone in the speaker\textquotesingle s in-group, and A is an action being done that is a \textbf{plus or minus  }to one\textquotesingle s in-group. Even without context to set up the dialogue, there is some relationship with X and Y. So, although the sentence above with ~てくれる was marked as wrong, if the traveler happened to be someone with a close relationship, or if a strong sense of intimacy\slash familiarity were to be felt from that person, then it and similar sentences would be OK. }
 
\par{23. 友だちはプレゼントを贈ってくれました。 \hfill\break
My friend sent me a gift. }
 
\par{24. 隣の赤ちゃん、よく泣いてくれるね。 \hfill\break
The baby next door is crying a lot, isn't it? }
 
\par{25. 兄が友達にアイスクリームを買ってくれた。 \hfill\break
My big brother bought my friend ice cream. }
 
\par{\textbf{Sentence Note }: Your brother and friend may be in your in-group, but the brother is not necessarily in the in-group of your friend. }
 
\par{ As for「YがX\{に・から\}Aてもらう」, in the case of an unrealized event, a certain person Y, shows a command\slash request\slash hope of something A that is a plus to Y to another person X. In the past tense, this command\slash request\slash hope is realized. In other words, Y has it that Y will receive a plus action A from X. }
 
\par{ If the Y in 「YがX\{に・から\}Aてもらう」is the speaker, it can be reworded to 「XがYにAてくれる」, but if the action is not a plus to the speaker, the opposite doesn\textquotesingle t work. Even though there are situations where the two patterns are interchangeable, there will always be a difference in feeling in respect to the speaker. }
 
\par{26. 僕と ${\overset{\textnormal{おど}}{\text{踊}}}$ ってくれる? \hfill\break
Will you dance with me? }

\par{27. ${\overset{\textnormal{じょゆう}}{\text{女優}}}$ は ${\overset{\textnormal{はな}}{\text{鼻}}}$ を ${\overset{\textnormal{せいけい}}{\text{整形}}}$ してもらいました。 \hfill\break
The actress had her nose remodeled. }
 
\par{ One of the first things we\textquotesingle ll now look at are the person restrictions with ~てくれる. First, consider the following sentences. }
 
\par{28. ランスさんは、友達にフランス語を教えてくれます。 \hfill\break
Lance will teach my friend French. }
 
\par{29. 今日、赤ちゃんが歩いてくれたのよ。 \hfill\break
The baby walked (for us) today. }
 
\par{30. 恋人への思いを、風が運んでくれました。 \hfill\break
The wind carried my thoughts to my lover. }
 
\par{ The restriction is that Y is the speaker or a person in the speaker\textquotesingle s in-group and X is not the speaker. When what follows X is a request or sentence of confirmation, anything can follow whether X be a person, animal, thing of nature, etc. }
 
\par{ When ~てくれる is used with ~のだ\slash ~よ, you are showing a promise with a third person. At the same time, the speaker is showing an expectation and a promise from someone\slash something. Although you can't make a promise with a something, in this case, the X is being personified, and if the benefit\slash convenience is carried out, it feels as if X has promised to do so. Of course, you can make this the topic of question and negation. When negated, there is disappointment. Of course, there are similar patterns where this nuance is implied. }
 
\par{31. 友達がWiiを貸してくれないんだよ。 \hfill\break
My friend won't lend me the Wii. }
 
\par{32. お母さんが誕生日にニンテンドウ3DS を買って\{くれます・くれるのです・くれないのです\}か。 \hfill\break
~Is your mom going to buy you a Nintendo 3DS for your birthday? }
 
\par{33. ねえ、ねえ、見ててね。このかわいい子犬が、2本足で歩いてくれるわよ。(Feminine) \hfill\break
Hey, hey, look! This cute puppy \textbf{will }walk on two legs (for us)! }
 
\par{ Though more so an understanding of how to use ~か and ~のか rather than ~てくれる, it\textquotesingle s important to know how they are used together in requesting and confirming. You may see ~てくれるか,  ~てくれ,  or ~てくれるのか. }
 
\par{34. 何かヒントになることを教えてくれないか。 \hfill\break
Could you give me some hint? }
 
\par{35. ねえ、母さん、今度の誕生日、DS買ってくれる(の)? \hfill\break
Hey, mom, you'll buy me a DS my next birthday, right? }
 
\par{36. ちょっとそのハンマーを取って\{〇 くれるか・〇 くれ・ X くれるのか\}。 \hfill\break
Could you get\slash get the hammer for a bit (?) }
 
\par{37a. ちょっとそのハンマーを取ってくれた(の)か。X \hfill\break
37b. ランスさんがあなたにフランス語を教えてくれた。X }

\par{ If it is a benefit for you, why would you be asking if it happened? The second sentence is problematic because of a potential in-group requirement violation. It could, however, be worded to something like the following. }
 
\par{38. ランスはお前にフランス語を教えてくれたんじゃない? \hfill\break
Didn't Lance teach you French? }

\par{ Here, the listener is clearly part of the speaker\textquotesingle s in-group due to the language used. Furthermore, there can be an in-group benefit of the listener having been taught French. }
 
\par{ A huge thing to understand is that 「くれる」is a verb of non-volition. Although we haven\textquotesingle t studied the following items, for future reference, they must never be used with it: つもりだ, Volitional form, たい. だめですよ! }
 
\par{ From this it may seem odd that there is a command form of くれる. However, 「A+くれ」 unlike the command form of a verb of volition like 貸せ, 取れ, etc., it shows not a request for the listener to obey but a request in which the listener will make the decision as to whether to comply or not. }
 
\par{ Now, we will switch to restrictions on ~てもらう. First, consider the following sentences. }
 
\par{39. \{私・友達\}はランスさんにフランス語を教えてもらいます。 \hfill\break
I\slash my friend will have Lance teach me\slash him French. }
 
\par{40. 私はランスさんに辞典を買ってもらったんですよ。 \hfill\break
I had Lance buy me a dictionary. }
 
\par{41. 忘れたの?俺はこのWiiをお前に貸してもらったよ。(Masculine yet casual) \hfill\break
Have you forgotten? I had you lend it to me. }
 
\par{42. 忘れたのか。お前はそのWiiを俺に貸してもらったんだよ。(Rough; masculine) \hfill\break
Have you forgotten? You were lent this Wii by me. }
 
\par{43. 地図で ${\overset{\textnormal{しめ}}{\text{示}}}$ してもらえますか。 \hfill\break
Can you show me on the map? }
 
\par{44a. ねえ、今日、赤ちゃんに、歩いてもらったのよ。X\slash ??? \hfill\break
44b. 桜の花が 咲き始めるのを待ってもらっている。 X }

\par{ There isn't any grammatical person restrictions for what can be X and Y in the pattern "YがXにAてもらう”. However, both X and Y must be someone\slash something that displays willful action. If the person or "it" does not have the inherent free will to not do the role provided in context, the sentence becomes invalid. 44a is given ??? along with X because grammatically is iffy when dealing with infants. It is unclear whether you can have a baby make the conscious choice to walk for your benefit. It's also not completely far-fetched. 44b, though, is undoubtedly wrong as the agent is non-human. Personification can certainly change things, but there is no personification in this example. }

\par{ Remember that A is an action with a benefit to Y, and there is usually a cost (not necessarily in the fiscal sense). Even when there isn't a cost, something balances the situation out. }
 
\par{ Between X and Y there is a sense of responsibility. The action that is requested by Y to X is from a sense of returning the favor, duty, and depending on the time and circumstance, can be in a relation where “responsibility” is being shared. There is no sense of trouble following when X is doing it out of thanks for something Y has already done in return before. This explains most situations. Even if you have someone do something that you don't have a close relationship with, there are numerous situations where the other person is deemed to have a duty to give you a favor. For instance, say you are an old woman in an airport. You will probably have someone carry your luggage. }
 
\par{45. 荷物を運んでもらえませんか。 \hfill\break
Could I have you carry my luggage? }
 
\par{46. わたしは従業員に荷物を運んでもらいました。 \hfill\break
I had an employee carry my luggage. }
 
\par{47. ちょっと ${\overset{\textnormal{てつだ}}{\text{手伝}}}$ っていただけませんか。 \hfill\break
Could you help me? }
 
\par{48. 彼を ${\overset{\textnormal{でんわぐち}}{\text{電話口}}}$ にお呼びいただけませんか。 \hfill\break
Would you mind calling him to the phone? }
 
\par{~てもらう also adds to the speech modals that have the ability to show commands. The speaker has X do A. This is carried out by stating that you will receive the action by X. It is the speaker that is the requester, and it is that the listener that is the person being requested to do A. Things balance out, however. Things are done out of a borrowing of responsibility. }
 
\par{49. 自分で出したゴミが、自分で持って帰ってもらいます。 \hfill\break
I will have people go home with their own trash. }
 
\par{50. 自分で出したゴミが、自分で持って帰ってもらうんです。 \hfill\break
You get (him) to go home with (his) own trash. }
 
\par{ The ~んだ in「YがXにA~てもらうんだ」shows a command has the same meaning as ~もらえ. The speaker, in regards to the listener is ordering to demand a third person A. Y = the listener. X = a third person. This is different from the roles of the basic form! }

\begin{ltabulary}{|P|P|}
\hline 

 YがXにAてもらう \textrightarrow  &  話し手が聞き手にAてもらう \\ \cline{1-2}

 YがXにAてもら\{うのだ・え\} &  聞き手が第三者にAてもら\{うのだ・え\} \\ \cline{1-2}

\end{ltabulary}
 
\par{ The confusing thing is that のだ with this pattern does not always show command. That\textquotesingle s because the pattern has several usages itself, and above is just what happens for that. When use to emphasize, it merely emphasizes the hope or will of Y receiving the action of someone for his\slash her benefit. As we've seen, Y could be the listener, speaker, or a third person depending on the situation. As you should have figured out by now, もらう, unlikeくれる, is a verb of volition. }
 
\par{51. ぼく、お母さんに、Wii買ってもらう(んだ)。 \hfill\break
I\textquotesingle m going to get a Wii from my mom. }
 
\par{52. お母さんに、Wiiを買ってもらう(の)か。 \hfill\break
Are you getting a Wii from your mom? }
 
\par{53. お姉さんはお父さんに、ドレスをもらいますか。 \hfill\break
Will your sister get a dress from your father? }
 
\par{ Things that you should not do are 「私があなたにA+もらう」 and「あなたが私にA+もらう」. They are both very rude and should be replaced. Consider the following. }
 
\par{54. トイレを貸していただけますか。 \hfill\break
Could you let me use your bathroom? }
 
\par{55. 車をお貸しします。(Humble speech) \hfill\break
I will lend you my car. }
    