    
\chapter{Criticizing}

\begin{center}
\begin{Large}
第59課: Criticizing: くせに・くせして 
\end{Large}
\end{center}
 
\par{ Everyone has bad habits, tendencies, or mannerisms. On top of that, everyone in the world loves to complain and ridicule the bad qualities of others—sometimes themselves as well. In Japanese, one of the most effective ways of ridiculing others for something is done so by using the word 癖, which literally means “habit” in the negative sense. }

\par{ In this lesson, we will learn how this word is used in the grammar expression くせに, which is seen in cruder contexts asくせして, to mean “even though\slash in spite of” when used to express annoyance, criticism, anger, etc. }
      
\section{Dissatisfaction}
 
\par{ As mentioned in the introduction, 癖\textquotesingle s basic meaning is to describe habits, mainly bad ones although not limited to them. For instance, “biting one\textquotesingle s nails” would be ${\overset{\textnormal{つめ}}{\text{爪}}}$ を ${\overset{\textnormal{か}}{\text{噛}}}$ む ${\overset{\textnormal{くせ}}{\text{癖}}}$ and “waking up early” can be expressed with ${\overset{\textnormal{はやお}}{\text{早起}}}$ きの ${\overset{\textnormal{くせ}}{\text{癖}}}$ . }

\par{1. ${\overset{\textnormal{わる}}{\text{悪}}}$ い ${\overset{\textnormal{くせ}}{\text{癖}}}$ を ${\overset{\textnormal{なお}}{\text{直}}}$ すには、どうしたらいいでしょうか。 \hfill\break
What should I do to fix a bad habit? }

\par{2. この ${\overset{\textnormal{あじ}}{\text{味}}}$ 、 ${\overset{\textnormal{くせ}}{\text{癖}}}$ になりますね! \hfill\break
I\textquotesingle m going to get hooked to this taste! }

\par{ In Ex. 2, you can see how X\{が・は\} ${\overset{\textnormal{くせ}}{\text{癖}}}$ になる is used to mean “to get hooked to X.” Interestingly, 癖 may also refer to kinks and curls in things like hair, rugs, etc. For instance, “to straighten out one\textquotesingle s hair” can be expressed with ${\overset{\textnormal{かみ}}{\text{髪}}}$ の ${\overset{\textnormal{くせ}}{\text{癖}}}$ を ${\overset{\textnormal{なお}}{\text{直}}}$ す and the word ${\overset{\textnormal{くせげ}}{\text{癖毛}}}$ means “frizzy\slash unruly hair.” }

\par{ The pattern 癖に, usually written out as くせに, attaches to nouns, adjectives, adjectival nouns, and verbs as demonstrated in the chart below. Its meaning, as mentioned in the introduction, is “even though\slash in spite of” and is used when describing annoyance, criticism, anger, a complaint, etc. It is semantically synonymous with the particle のに with the only true difference being the extremely negative connotations implied by くせに. }

\begin{ltabulary}{|P|P|P|}
\hline 

Nouns & Noun + の・である + くせに & ${\overset{\textnormal{おとこ}}{\text{男}}}$ \{の・である\}くせに \hfill\break
Even though you\textquotesingle re a man… \\ \cline{1-3}

Adjectives & Adj. + くせに &  ${\overset{\textnormal{ふる}}{\text{古}}}$ いくせに \hfill\break
Even though it\textquotesingle s old… \\ \cline{1-3}

Adjectival Nouns & Adj. Noun + な + くせに & ${\overset{\textnormal{かんたん}}{\text{簡単}}}$ なくせに \hfill\break
Even though it\textquotesingle s easy… \\ \cline{1-3}

Verbs & Verb + くせに &  ${\overset{\textnormal{か}}{\text{買}}}$ うくせに \hfill\break
In spite of buying… \\ \cline{1-3}

\end{ltabulary}

\par{\textbf{Grammar Note }: Noun + であるくせに is not used in the spoken language and is seen in somewhat old-fashioned speech. }

\par{ Below are example sentences of くせに following the various parts of speech as shown in the table above to show an array of negative critical connotations. }

\par{3. ${\overset{\textnormal{かのじょ}}{\text{彼女}}}$ って ${\overset{\textnormal{おとな}}{\text{大人}}}$ のくせに、まだ ${\overset{\textnormal{ははおや}}{\text{母親}}}$ に ${\overset{\textnormal{せんたく}}{\text{洗濯}}}$ してもらってるそうだ。 \hfill\break
I hear that she still has her mom do her laundry even though she\textquotesingle s an adult. }

\par{4. ${\overset{\textnormal{けんたくん}}{\text{健太君}}}$ は ${\overset{\textnormal{こども}}{\text{子供}}}$ のくせに、 ${\overset{\textnormal{ぜんぜんそと}}{\text{全然外}}}$ で ${\overset{\textnormal{あそ}}{\text{遊}}}$ びたがらないの。 \hfill\break
Kenta doesn\textquotesingle t want to play outside all even though he\textquotesingle s a kid. }

\par{5. ${\overset{\textnormal{たなか}}{\text{田中}}}$ さんは、 ${\overset{\textnormal{にほんじん}}{\text{日本人}}}$ のくせに、 ${\overset{\textnormal{にほんご}}{\text{日本語}}}$ を ${\overset{\textnormal{はな}}{\text{話}}}$ すのが ${\overset{\textnormal{す}}{\text{好}}}$ きじゃないみたいですねえ。 \hfill\break
Mr. Tanaka, despite being Japanese, doesn\textquotesingle t appear to like speaking Japanese, huh. }

\par{6. あの ${\overset{\textnormal{ひと}}{\text{人}}}$ 、 ${\overset{\textnormal{しんそつしゃ}}{\text{新卒者}}}$ のくせに、よく ${\overset{\textnormal{なん}}{\text{何}}}$ でも ${\overset{\textnormal{し}}{\text{知}}}$ ってるように ${\overset{\textnormal{しゃべ}}{\text{喋}}}$ ったりしますね。 \hfill\break
Despite being a new graduate, that person sure often talks like he knows everything, huh. }

\par{7. ${\overset{\textnormal{かのじょ}}{\text{彼女}}}$ 、コメディアンのくせに、 ${\overset{\textnormal{ぜんぜんおもしろ}}{\text{全然面白}}}$ くないよ。 \hfill\break
She isn\textquotesingle t funny at all despite being a comedian. }

\par{8. 「 ${\overset{\textnormal{おとこ}}{\text{男}}}$ のくせに」や、「 ${\overset{\textnormal{おんな}}{\text{女}}}$ のくせに」などといった ${\overset{\textnormal{い}}{\text{言}}}$ い ${\overset{\textnormal{かた}}{\text{方}}}$ はセクハラになり ${\overset{\textnormal{う}}{\text{得}}}$ ると ${\overset{\textnormal{い}}{\text{言}}}$ えよう。 \hfill\break
One could say that expressions like “even though you\textquotesingle re a man” or “even though you\textquotesingle re a woman” can be treated as forms of sexual harassment. }

\par{9. ${\overset{\textnormal{やせいどうぶつ}}{\text{野生動物}}}$ のくせに ${\overset{\textnormal{うご}}{\text{動}}}$ きが ${\overset{\textnormal{にぶ}}{\text{鈍}}}$ いな。 \hfill\break
It\textquotesingle s slow moving despite being a wild animal, huh. }

\par{10. ワニのくせに ${\overset{\textnormal{かいすい}}{\text{海水}}}$ もへっちゃらなんだぁ。 \hfill\break
Huh, sea water isn\textquotesingle t even a big deal even though it\textquotesingle s an alligator. }

\par{\textbf{Spelling Note }: ワニ can seldom be seen spelled as 鰐. }

\par{11. ${\overset{\textnormal{ちゅうごくじん}}{\text{中国人}}}$ のくせに ${\overset{\textnormal{ばくか}}{\text{爆買}}}$ いしないなんておかしい! \hfill\break
It\textquotesingle s so strange that he\slash she won\textquotesingle t spend like crazy despite being Chinese! }

\par{12. ${\overset{\textnormal{いはら}}{\text{猪原}}}$ は ${\overset{\textnormal{あたま}}{\text{頭}}}$ が ${\overset{\textnormal{わる}}{\text{悪}}}$ いくせに、 ${\overset{\textnormal{にんき}}{\text{人気}}}$ があって ${\overset{\textnormal{うらや}}{\text{羨}}}$ ましい! \hfill\break
I\textquotesingle m so jealous that Ihara is popular even though he\textquotesingle s dumb! }

\par{13. ${\overset{\textnormal{ほんとう}}{\text{本当}}}$ は ${\overset{\textnormal{げんき}}{\text{元気}}}$ なくせに ${\overset{\textnormal{なま}}{\text{怠}}}$ けやがって! \hfill\break
How dare you slack off when you\textquotesingle re actually fine! }

\par{14. ${\overset{\textnormal{じぶん}}{\text{自分}}}$ が ${\overset{\textnormal{ばか}}{\text{馬鹿}}}$ なくせに ${\overset{\textnormal{ひと}}{\text{人}}}$ を ${\overset{\textnormal{ばか}}{\text{馬鹿}}}$ にする ${\overset{\textnormal{ひと}}{\text{人}}}$ が ${\overset{\textnormal{だいきら}}{\text{大嫌}}}$ い。 \hfill\break
I hate people who make others out to be idiots even though they themselves are idiots. }

\par{15. ${\overset{\textnormal{かしこ}}{\text{賢}}}$ いくせに、おかしなことを ${\overset{\textnormal{かんが}}{\text{考}}}$ えるんだね。 \hfill\break
You sure think a lot of crazy stuff for someone who\textquotesingle s wise. }

\par{16. ${\overset{\textnormal{がくれき}}{\text{学歴}}}$ がないくせにプライドだけ ${\overset{\textnormal{たか}}{\text{高}}}$ い ${\overset{\textnormal{おとこ}}{\text{男}}}$ 、めっちゃ ${\overset{\textnormal{きら}}{\text{嫌}}}$ いなの! \hfill\break
I really hate men who only have pride with no education! }

\par{17. あんた、 ${\overset{\textnormal{せんそう}}{\text{戦争}}}$ の ${\overset{\textnormal{けいけん}}{\text{経験}}}$ もないくせに ${\overset{\textnormal{せんそう}}{\text{戦争}}}$ に ${\overset{\textnormal{あこが}}{\text{憧}}}$ れを ${\overset{\textnormal{も}}{\text{持}}}$ つんじゃねーぞ。 \hfill\break
You don\textquotesingle t yearn for war when you\textquotesingle ve never experienced war! }

\par{18. あたしの ${\overset{\textnormal{かれし}}{\text{彼氏}}}$ ね、 ${\overset{\textnormal{し}}{\text{知}}}$ ってるくせに ${\overset{\textnormal{おし}}{\text{教}}}$ えてくれないわ。 \hfill\break
My boyfriend, you know, knows but he won\textquotesingle t tell\slash teach me. }

\par{19. やったくせに ${\overset{\textnormal{し}}{\text{知}}}$ らんぷりするんじゃないぞ! \hfill\break
Don\textquotesingle t act like you don\textquotesingle t know when you\textquotesingle re the one who did it! }

\par{20. それだけの ${\overset{\textnormal{けが}}{\text{怪我}}}$ したくせに? \hfill\break
Even though that\textquotesingle s all you\textquotesingle re hurt? }

\par{21. ${\overset{\textnormal{さん}}{\text{3}}}$ ${\overset{\textnormal{かい}}{\text{回}}}$ も ${\overset{\textnormal{は}}{\text{恥}}}$ ずかしい ${\overset{\textnormal{しっぱい}}{\text{失敗}}}$ をしちゃったくせに、 ${\overset{\textnormal{いまさらなに}}{\text{今更何}}}$ が ${\overset{\textnormal{は}}{\text{恥}}}$ ずかしいの? \hfill\break
What are you so embarrassed about now? You\textquotesingle ve already failed three times… }

\par{22. ${\overset{\textnormal{なん}}{\text{何}}}$ だ、 ${\overset{\textnormal{じぶん}}{\text{自分}}}$ がミスしたくせに・・・ \hfill\break
What? Even though you yourself have messed up… }

\begin{center}
\textbf{くせして }\hfill\break

\end{center}

\par{ In far critical, often derogatory contexts, くせに can be seen as くせして. This implies slightly older grammar in which に and して share some degree of interchangeability with the latter being more emphatic, thus this form of the expression. }

\par{23. ${\overset{\textnormal{おれ}}{\text{俺}}}$ って、 ${\overset{\textnormal{にほんじん}}{\text{日本人}}}$ のくせして ${\overset{\textnormal{ぶっきょう}}{\text{仏教}}}$ についてはとんと ${\overset{\textnormal{もんがいかん}}{\text{門外漢}}}$ だよ。 \hfill\break
Even though I'm Japanese, I\textquotesingle m a complete outsider about Buddhism. }

\par{24. ${\overset{\textnormal{きこくしじょ}}{\text{帰国子女}}}$ の ${\overset{\textnormal{くせ}}{\text{癖}}}$ してそんなのも ${\overset{\textnormal{し}}{\text{知}}}$ らないってわけか。 \hfill\break
So you don\textquotesingle t even know stuff like that despite being a kikoku shijo, huh? }

\par{\textbf{Word Note }: 帰国子女 refers to children of Japanese expatriates who then return to Japan. }

\par{25. ${\overset{\textnormal{にほんじん}}{\text{日本人}}}$ のくせして、 ${\overset{\textnormal{にほんご}}{\text{日本語}}}$ の ${\overset{\textnormal{ぶんぽう}}{\text{文法}}}$ がからっきしダメだ。 \hfill\break
Even though I\textquotesingle m Japanese, my Japanese grammar is absolutely hopeless. }

\begin{ltabulary}{|P|}
\hline 

Adj. + くせに 
\\

\end{ltabulary}
    