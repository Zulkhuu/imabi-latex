    
\chapter{Adjectives}

\begin{center}
\begin{Large}
第52課: Adjectives: 多い Ōi  \& Sukunai 少ない 
\end{Large}
\end{center}
 
\par{ The adjectives \emph{ōi }多い and \emph{sukunai }少ない respectively mean “many” and “few” respectively. As straight forward as that may seem, using these words poses its own problems. }

\par{i. ${\overset{\textnormal{おお}}{\text{多}}}$ い ${\overset{\textnormal{ともだち}}{\text{友達}}}$ がいます。X \hfill\break
 \emph{Ōi tomodachi ga imasu. \hfill\break
 }I have many friends. }

\par{ii. ${\overset{\textnormal{じもと}}{\text{地元}}}$ には ${\overset{\textnormal{すく}}{\text{少}}}$ ない ${\overset{\textnormal{てんぽ}}{\text{店舗}}}$ があります。X \hfill\break
 \emph{Jimoto ni wa sukunai tempo ga arimasu. \hfill\break
 }There are few stores at my home town. }

\par{ As these examples show, they cannot be used like all other adjectives in directly modifying a noun like \emph{kawaii }可愛い (cute) can. }

\par{iii. あのカメは ${\overset{\textnormal{かわい}}{\text{可愛}}}$ いですね。 \hfill\break
 \emph{Ano kame wa kawaii desu ne. \hfill\break
 }That turtle is cute, isn\textquotesingle t it? }

\par{iv. ${\overset{\textnormal{きょうしつ}}{\text{教室}}}$ には ${\overset{\textnormal{かわい}}{\text{可愛}}}$ いカメがいます。 \hfill\break
 \emph{Ky }\emph{ōshitsu ni wa kawaii kame ga imasu. \hfill\break
 }There is a cute turtle in the classroom. }

\par{ In this lesson, we will look intently at what exactly defines these adjectives and how they are unique. We will now investigate further into how these two adjectives are used so that you don\textquotesingle t fall victim to the many mistakes that few students don\textquotesingle t end up making. }
      
\section{Quantity}
 
\par{ Interestingly, both \emph{ōi }多い and \emph{sukunai }少ない may be used at the end of the sentence without problem, but they differ regarding their ability to precede nouns. }

\par{1. ${\overset{\textnormal{こうえん}}{\text{公園}}}$ にはベンチが ${\overset{\textnormal{おお}}{\text{多}}}$ いです。 \hfill\break
 \emph{K }\emph{ōen ni wa benchi ga }\emph{ōi desu. }\hfill\break
There are many benches in the park. }

\par{2. ${\overset{\textnormal{こうえん}}{\text{公園}}}$ には、 ${\overset{\textnormal{おお}}{\text{多}}}$ いベンチがあります。X \hfill\break
 \emph{K }\emph{ōen ni wa, }\emph{ōi benchi ga arimasu. }\hfill\break
There are many benches in the park. }

\par{3. この ${\overset{\textnormal{しゅうへん}}{\text{周辺}}}$ には、ベンチが ${\overset{\textnormal{おお}}{\text{多}}}$ い ${\overset{\textnormal{こうえん}}{\text{公園}}}$ があります。 \hfill\break
 \emph{Kono sh }\emph{ūhen ni wa, benchi ga oi k }\emph{ōen ga arimasu. \hfill\break
 }There is a park with many benches nearby. }

\par{4. ${\overset{\textnormal{わたし}}{\text{私}}}$ はお ${\overset{\textnormal{かね}}{\text{金}}}$ が ${\overset{\textnormal{すく}}{\text{少}}}$ ないです。 \hfill\break
 \emph{Watashi wa okane ga sukunai desu. \hfill\break
 }I have little money. }

\par{5. ${\overset{\textnormal{すく}}{\text{少}}}$ ないお ${\overset{\textnormal{かね}}{\text{金}}}$ で ${\overset{\textnormal{く}}{\text{暮}}}$ らす ${\overset{\textnormal{ひと}}{\text{人}}}$ がたくさんいます。 \hfill\break
 \emph{Sukunai okane de kurasu hito ga takusan imasu. \hfill\break
 }There is a lot of people who live with little money. }

\par{6. ${\overset{\textnormal{てんぽ}}{\text{店舗}}}$ が ${\overset{\textnormal{すく}}{\text{少}}}$ ないところに ${\overset{\textnormal{す}}{\text{住}}}$ んでいます。 \hfill\break
 \emph{Tempo ga sukunai tokoro ni sunde imasu. \hfill\break
 }I live in a place where there are few stores. }

\par{ As you can see, it is not always the case that \emph{sukunai }少ない can\textquotesingle t directly precede and modify a noun. It also happens to be the case that both can be before a noun if part of a dependent clause modifying a noun. }

\begin{center}
\textbf{Grammatical Restraint } 
\end{center}

\par{ Because these adjectives indicate quantity\slash size of something, there must be an element in the sentence indicating what the quantity\slash size is. When using these adjectives as the predicate of the sentence, you must use the pattern "X \emph{wa }Y \emph{ga }Z", with these adjectives being Z. The subject can only be dropped if the information has already been supplemented in context. }

\par{7. シアトルでは ${\overset{\textnormal{あめ}}{\text{雨}}}$ が ${\overset{\textnormal{おお}}{\text{多}}}$ いです。 \hfill\break
 \emph{Shiatoru de wa ame ga }\emph{ōi desu. }\hfill\break
There is a lot of rain in Seattle. }

\par{\textbf{Word Note }: Both \emph{ōi }多い and \emph{sukunai }少ない can be used to indicate frequency. }

\par{8. テキサス ${\overset{\textnormal{しゅう}}{\text{州}}}$ では ${\overset{\textnormal{あめ}}{\text{雨}}}$ が ${\overset{\textnormal{すく}}{\text{少}}}$ ないです。 \hfill\break
 \emph{Tekisasu-sh }\emph{ū de wa ame ga sukunai desu. }\hfill\break
There is little rain in Texas. }

\par{9. ${\overset{\textnormal{おきなわ}}{\text{沖縄}}}$ の ${\overset{\textnormal{りょうり}}{\text{料理}}}$ は ${\overset{\textnormal{りょう}}{\text{量}}}$ が ${\overset{\textnormal{おお}}{\text{多}}}$ いですね。 \hfill\break
 \emph{Okinawa no ry }\emph{ōri wa ry }\emph{ō ga }\emph{ōi desu ne. }\hfill\break
The portions are large in Okinawan cuisine. }

\par{10. なぜ ${\overset{\textnormal{こうきゅうりょうり}}{\text{高級料理}}}$ は ${\overset{\textnormal{りょう}}{\text{量}}}$ が ${\overset{\textnormal{すく}}{\text{少}}}$ ないのですか。 \hfill\break
 \emph{Naze k }\emph{ōky }\emph{ū ry }\emph{ōri wa ry }\emph{ō ga sukunai no desu ka? }\hfill\break
Why is it that the portions are small in high class cuisine? }

\par{\textbf{Word Note }: Unlike in English, the word \emph{ry }\emph{ō }量 (quantity) and any word with it are used as if they are countable entities. This is because aside from meaning that there is “few” or “many” of something, they may also indicate the size of quantity. Incidentally, \emph{chiisai }小さい and \emph{ōkii }大きい may occasionally be used in relation to physical entities, but \emph{sukunai }少ない and \emph{ōi }多い are used almost always. }

\par{11. ${\overset{\textnormal{たてもの}}{\text{建物}}}$ が ${\overset{\textnormal{おお}}{\text{多}}}$ いです。(?) \hfill\break
 \emph{Tatemono ga }\emph{ōi desu. }\hfill\break
There are many buildings. }

\par{12. ${\overset{\textnormal{かんじ}}{\text{漢字}}}$ が ${\overset{\textnormal{すく}}{\text{少}}}$ ないです。(?) \hfill\break
 \emph{Kanji ga sukunai desu. }\hfill\break
There are few Kanji. }

\par{\textbf{Context Note }: If the context as for what the quantity statement these adjectives describe is not clear, obvious, and or stated, their use becomes ungrammatical. This, however, is no different than English. }

\par{13. 「 ${\overset{\textnormal{とうきょう}}{\text{東京}}}$ はどうでしたか。」「 ${\overset{\textnormal{たてもの}}{\text{建物}}}$ が ${\overset{\textnormal{おお}}{\text{多}}}$ かったです。」 \hfill\break
 \emph{“T }\emph{ōky }\emph{ō wa d }\emph{ō deshita ka?” “Tatemono ga }\emph{ōkatta desu.” } \hfill\break
“How was Tokyo?” “There were many buildings.” }

\par{14. 「 ${\overset{\textnormal{まち}}{\text{町}}}$ の ${\overset{\textnormal{いちば}}{\text{市場}}}$ はどうですか」「 ${\overset{\textnormal{おお}}{\text{大}}}$ きい ${\overset{\textnormal{さかな}}{\text{魚}}}$ が ${\overset{\textnormal{すく}}{\text{少}}}$ ないです」 \hfill\break
 \emph{“Machi no ichiba wa d }\emph{ō desu ka?” “ }\emph{Ōkii sakana ga sukunai desu.” }\hfill\break
“How is the town\textquotesingle s market?” “There are few large fish.” }

\par{ Now let\textquotesingle s return to how these adjectives are used to directly modify nouns. }

\par{15. ムンバイは ${\overset{\textnormal{じんこう}}{\text{人口}}}$ が ${\overset{\textnormal{おお}}{\text{多}}}$ い ${\overset{\textnormal{まち}}{\text{街}}}$ です。 \hfill\break
 \emph{Mumbai wa jink }\emph{ō ga }\emph{ōi machi desu. }\hfill\break
Mumbai is a town with a large population. }

\par{16. ムンバイは ${\overset{\textnormal{おお}}{\text{多}}}$ い ${\overset{\textnormal{まち}}{\text{街}}}$ です。X \hfill\break
 \emph{Mumbai wa }\emph{ōi machi desu. }\hfill\break
Mumbai is a large town. \hfill\break
 \hfill\break
17. ${\overset{\textnormal{かくすう}}{\text{画数}}}$ が ${\overset{\textnormal{すく}}{\text{少}}}$ ない ${\overset{\textnormal{かんじ}}{\text{漢字}}}$ が ${\overset{\textnormal{おお}}{\text{多}}}$ い。 \hfill\break
 \emph{Kakus }\emph{ū ga sukunai kanji ga }\emph{ōi. }\hfill\break
There are many Kanji with few strokes. }

\par{18. ${\overset{\textnormal{かくすう}}{\text{画数}}}$ が ${\overset{\textnormal{おお}}{\text{多}}}$ い ${\overset{\textnormal{かんじ}}{\text{漢字}}}$ は、 ${\overset{\textnormal{じつ}}{\text{実}}}$ は、 ${\overset{\textnormal{おお}}{\text{多}}}$ くありません。 \hfill\break
 \emph{Kakus }\emph{ū ga }\emph{ōi kanji wa, jitsu wa, }\emph{ōku arimasen. }\hfill\break
As for Kanji with many strokes, in actuality, there aren\textquotesingle t that many. \hfill\break
 \hfill\break
19. ${\overset{\textnormal{すく}}{\text{少}}}$ ない ${\overset{\textnormal{かんじ}}{\text{漢字}}}$ は ${\overset{\textnormal{かくすう}}{\text{画数}}}$ が ${\overset{\textnormal{おお}}{\text{多}}}$ くありません。X \hfill\break
 \emph{Sukunai kanji wa kakusu ga oku arimasen. }\hfill\break
Few Kanji don\textquotesingle t have many strokes. }

\par{ The reason for why \emph{sukunai }少ない and \emph{ōi }多い are incapable of directly modifying nouns is that they don\textquotesingle t express the attributes of something. They solely describe quantity. When they are in dependent clauses which then modify a noun, they still only indicate the quantity of whatever is inside the dependent clause with them, and it is only case that that assertion is being used to qualify another noun. }

\par{ To express “few” and “many” while directly modifying a noun, \emph{sukunai }少ない and \emph{ōi }多い must be replaced with \emph{sukoshi no }少しの and \emph{ōku no }多くの respectively. However, just because you convert to the latter forms, doesn\textquotesingle t mean that there are never any quirks. }

\par{20. ${\overset{\textnormal{おおさか}}{\text{大阪}}}$ には、 ${\overset{\textnormal{おお}}{\text{多}}}$ くお ${\overset{\textnormal{てら}}{\text{寺}}}$ があるかどうか ${\overset{\textnormal{し}}{\text{知}}}$ りません。 \hfill\break
 \emph{Ōsaka ni wa, }\emph{ōku no otera ga aru ka d }\emph{ō ka shirimasen. }\hfill\break
I don\textquotesingle t know whether there are many temples in Osaka. }

\par{21. \{ ${\overset{\textnormal{おお}}{\text{多}}}$ くの・たくさんの・ ${\overset{\textnormal{おおぜい}}{\text{大勢}}}$ の\} ${\overset{\textnormal{ひと}}{\text{人}}}$ が ${\overset{\textnormal{あつ}}{\text{集}}}$ まりました。 \hfill\break
 \emph{[ }\emph{Ōku no\slash takusan no\slash  }\emph{ōzei no] hito ga atsumarimashita. }\hfill\break
Many people gathered. }

\par{\textbf{Word Note }: \emph{Takusan no }沢山の and \emph{ōzei no }大勢の are \emph{no }-adjectival nouns that, in this situation, are slightly more natural and common than \emph{ōku no }多くの. This is because both \emph{sukunai }少ない and \emph{ōi }多い, and by default, \emph{sukoshi no }少しの and \emph{ōku no }多くの, indicate whether an amount exceeds or is below a standard amount, but if there is no standard ascertainable in context, using them becomes somewhat unnatural in contrast to words like \emph{takusan }沢山 (many) and \emph{wazuka }僅か (few), which are more emphatic. }

\par{22. ${\overset{\textnormal{すこ}}{\text{少}}}$ しの ${\overset{\textnormal{きんがく}}{\text{金額}}}$ から ${\overset{\textnormal{おお}}{\text{大}}}$ きな ${\overset{\textnormal{かせ}}{\text{稼}}}$ ぎを ${\overset{\textnormal{つく}}{\text{作}}}$ る。 \hfill\break
 \emph{Sukoshi no kingaku kara }\emph{ōkina kasegi wo tsukuru. \hfill\break
 }To make large earnings from a small amount of money. }

\par{23. ${\overset{\textnormal{かれ}}{\text{彼}}}$ はわずかな ${\overset{\textnormal{きんがく}}{\text{金額}}}$ を ${\overset{\textnormal{よくば}}{\text{欲張}}}$ って ${\overset{\textnormal{1}}{\text{1}}}$ ${\overset{\textnormal{まんえん}}{\text{万円}}}$ の ${\overset{\textnormal{そんしつ}}{\text{損失}}}$ を ${\overset{\textnormal{こうむ}}{\text{被}}}$ りました。 \hfill\break
 \emph{Kare wa wazuka na kingaku wo yokubatte ichiman\textquotesingle en no sonshitsu wo k }\emph{ōmurimashita. \hfill\break
 }He coveted a small amount of money and lost 10,000 yen. }

\begin{center}
\textbf{Exceptions with \emph{Sukunai }少ない }
\end{center}

\par{ When \emph{sukunai }少ない is in fact used in directly modifying a noun, its meaning is the same as \emph{wazuka na }わずかな, which is to show the sheer scarcity\slash lack of quantity. This is made possible because whenever \emph{sukunai }少ない is used in this way, the parameters are deemed as societal general knowledge, and so there is no need to explicitly ‘restate\textquotesingle  said parameters. }

\par{24. ${\overset{\textnormal{すく}}{\text{少}}}$ ない ${\overset{\textnormal{きゅうりょう}}{\text{給料}}}$ でお ${\overset{\textnormal{かね}}{\text{金}}}$ を貯める。 \hfill\break
 \emph{Sukunai ky }\emph{ūry }\emph{ō de okane wo tameru. }\hfill\break
To save money with little income. }

\par{25. ${\overset{\textnormal{さまざま}}{\text{様々}}}$ な、 ${\overset{\textnormal{すく}}{\text{少}}}$ ない ${\overset{\textnormal{ざいりょう}}{\text{材料}}}$ でできています。 \hfill\break
 \emph{Samazama na, sukunai zairy }\emph{ō de dekite imasu. }\hfill\break
It\textquotesingle s made with various, scarce materials. }

\par{26. ${\overset{\textnormal{わたし}}{\text{私}}}$ たちは ${\overset{\textnormal{こんかい}}{\text{今回}}}$ 、もっと ${\overset{\textnormal{すく}}{\text{少}}}$ ない ${\overset{\textnormal{じかん}}{\text{時間}}}$ で ${\overset{\textnormal{たっせい}}{\text{達成}}}$ しました。 \hfill\break
 \emph{Watashitachi wa konkai, motto sukunai jikan de tassei shimashita. }\hfill\break
We accomplished it in far less time this time. }
    