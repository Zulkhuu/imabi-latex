    
\chapter{Potential I}

\begin{center}
\begin{Large}
第82課: Potential I: The Potential Form 
\end{Large}
\end{center}
 
\par{ Expressing potential in Japanese isn't easy. There are many ways not exactly the same to do this, and there are restrictions on these phrases that don't exist for the English "can". In Japanese, the potential is intertwined with the concept of volition. This, not surprisingly, affects the grammar. }
      
\section{Normal Potential Form of Verbs}
 
\par{ The direct object of a potential sentence is treated as the subject of an intransitive verb. The object is still the object. が and を, though, may mark the object. There are rules behind it, but for now, we will use them interchangeably. }

\begin{ltabulary}{|P|P|P|P|P|P|}
\hline 

 & Ex. & 可能形 & Past & Negative & Negative Past \\ \cline{1-6}

一段 & 食べる & 食べられる & 食べられた & 食べられない & 食べられなかった \\ \cline{1-6}

五段 & 行く & 行かれる \textrightarrow  行ける & 行けた & 行けない & 行けなかった \\ \cline{1-6}

来る & 来る & 来(こ)られる & 来られた & 来られない & 来られなかった \\ \cline{1-6}

する & する & できる & できた & できない & できなかった \\ \cline{1-6}

形容詞 & 正しい & 正しく(して)いられる & 正しく(して)いられた & 正しく(して)いられない & 正しく(して)いられなかった \\ \cline{1-6}

形容動詞 & 幸せだ & 幸せでいられる & 幸せでいられた & 幸せでいられない & 幸せでいられなかった \\ \cline{1-6}

\end{ltabulary}
\hfill\break

\begin{center}
\textbf{Examples }
\end{center}

\par{${\overset{\textnormal{}}{\text{1. 漢字など}}}$ ${\overset{\textnormal{}}{\text{書}}}$ けますか。 \hfill\break
Can you write Kanji and what not? }

\par{2. ${\overset{\textnormal{しかた}}{\text{仕方}}}$ がありませんが、 ${\overset{\textnormal{}}{\text{明日}}}$ は ${\overset{\textnormal{}}{\text{公園}}}$ に ${\overset{\textnormal{}}{\text{行}}}$ けません。 \hfill\break
It can't be helped, but I can't go to the park tomorrow. }
 
\par{${\overset{\textnormal{}}{\text{3. 日本語}}}$ の ${\overset{\textnormal{}}{\text{本}}}$ が ${\overset{\textnormal{}}{\text{読}}}$ めますか。 \hfill\break
Can you read a Japanese book? }
 
\par{4. このキノコは ${\overset{\textnormal{}}{\text{食}}}$ べられますか。 \hfill\break
Is this mushroom edible? }
 
\par{${\overset{\textnormal{}}{\text{5. 私}}}$ が ${\overset{\textnormal{かいしゃ}}{\text{会社}}}$ で ${\overset{\textnormal{しゅっせ}}{\text{出世}}}$ できたのは、 ${\overset{\textnormal{うん}}{\text{運}}}$ がよかったまでのことです。 \hfill\break
It's just luck that I was able to succeed at the company. }
 
\par{6. その ${\overset{\textnormal{}}{\text{話}}}$ はできすぎで(、) ${\overset{\textnormal{しん}}{\text{信}}}$ じられません。 \hfill\break
The story is too good to be true. }

\par{7. そこまで歩けますか。 \hfill\break
Can I walk there? }

\par{8. いくつまで ${\overset{\textnormal{かぞ}}{\text{数}}}$ えられるのか。(Somewhat rude; a sense of doubt is portrayed) \hfill\break
How far can you count to? }

\par{9. ${\overset{\textnormal{じてんしゃ}}{\text{自転車}}}$ に ${\overset{\textnormal{}}{\text{乗}}}$ れますか。 \hfill\break
Can you ride a bicycle? }
 
\par{${\overset{\textnormal{}}{\text{10. 鳥}}}$ は ${\overset{\textnormal{}}{\text{空}}}$ を ${\overset{\textnormal{と}}{\text{飛}}}$ べる。 \hfill\break
Birds can fly. \hfill\break
Literally: Birds can fly through the sky. }

\par{${\overset{\textnormal{}}{\text{11. 僕}}}$ の ${\overset{\textnormal{}}{\text{学校}}}$ では ${\overset{\textnormal{}}{\text{日本語}}}$ が ${\overset{\textnormal{なら}}{\text{習}}}$ えません。 \hfill\break
You can't take Japanese at my school. }

\par{${\overset{\textnormal{}}{\text{12. 明日}}}$ は ${\overset{\textnormal{}}{\text{仕事}}}$ があります。ですから、 ${\overset{\textnormal{}}{\text{行}}}$ けません。 \hfill\break
I have work tomorrow. So, I can't go. }

\par{13. ${\overset{\textnormal{れんらく}}{\text{連絡}}}$ が ${\overset{\textnormal{と}}{\text{取}}}$ れなくなったから、帰ってこないか ${\overset{\textnormal{しんぱい}}{\text{心配}}}$ した。 \hfill\break
Because we lost contact with him, we worried whether he would come back home. }

\par{14. お ${\overset{\textnormal{}}{\text{金}}}$ がなくて、バスに ${\overset{\textnormal{}}{\text{乗}}}$ れませんでした。 \hfill\break
I didn't have money, so I couldn't ride the bus. }

\par{15. ピアノが ${\overset{\textnormal{ひ}}{\text{弾}}}$ けますか。 \hfill\break
Can you play the piano? }

\par{16. この ${\overset{\textnormal{}}{\text{次}}}$ の ${\overset{\textnormal{}}{\text{金曜日}}}$ に ${\overset{\textnormal{}}{\text{来}}}$ られますか。 \hfill\break
Can you come next Friday? }

\par{17. うちの ${\overset{\textnormal{}}{\text{赤}}}$ ん ${\overset{\textnormal{ぼう}}{\text{坊}}}$ はもうヨチヨチ歩けますよ。(Feminine) \hfill\break
Our baby can already waddle around. }

\par{18. ファックスを ${\overset{\textnormal{おく}}{\text{送}}}$ れますか。 \hfill\break
Can I send a fax? }

\par{19. それは言えてるね。 \hfill\break
Idiomatic: That's exactly it. }

\par{\textbf{Word Note }: See that いえる, aside from literally mean "can say", can also be used idiomatically. }

\par{20. ${\overset{\textnormal{しょうぎ}}{\text{将棋}}}$ では ${\overset{\textnormal{か}}{\text{勝}}}$ てる。  (Contrasting) \hfill\break
I can win at shogi. }

\par{\textbf{Word Note }: ${\overset{\textnormal{}}{\text{将棋}}}$ is Japanese chess. }

\par{21. お酒ですか。ええ、飲めますよ。 \hfill\break
Liquor? Yes, you can drink it. }

\par{\textbf{Meaning Note }: In the above sentence, the potential is used in showing permission. Or, depending on context, it may refer to the liquor in question being safe to drink. }

\par{22. この水、飲めますか。 \hfill\break
Can I drink this water? }

\par{\textbf{Meaning Note }: This isn't asking about the ability to drink water. Rather, it's about whether it's OK. The water could be dirty. People can still drink dirty water, but should they is the question. }

\par{23. どこに ${\overset{\textnormal{}}{\text{車}}}$ を ${\overset{\textnormal{と}}{\text{停}}}$ められますか。 \hfill\break
Where can I park my car? }

\par{24. ${\overset{\textnormal{おく}}{\text{\{遅}}}$ れ・ ${\overset{\textnormal{うしな}}{\text{失}}}$ った ${\overset{\textnormal{じかん}}{\text{時間\}}}}$ は ${\overset{\textnormal{}}{\text{金}}}$ では ${\overset{\textnormal{つぐな}}{\text{償}}}$ えない。 \hfill\break
Money cannot pay for lost time. }
無くす・なくす・亡くす・失くす 失う・喪う・うしなう \hfill\break
Here we have a classic battle between script and nuance. Let's begin. \hfill\break
無くなる is uncommon, but = なくなる. It can be used to show that something has become no longer in existence, or something is used up, something is lost. Thus, it is intransitive. And, you have several broad definitions to think through rather than one English keyword. \hfill\break
亡くなる  Comes from the same source as above but refers to the passing away of an individual in a respectful\slash euphemistic fashion. \hfill\break
亡くす is light being died on by someone in your loved ones, and losing that individual. 幼時に父を亡くす。This comes from 亡く + す(る) as expected of the same source as 無くす, which can be broken down likewise. So, なくなる can be broken down to なく+なる. It's just for the definitions thus far, they can be viewed as a single word, just like how words in any language are etymologically compounded but no longer thought of as being so. However, when なくなる is used for ないようになる, it is viewed as being separate and is not written in Kanji. \hfill\break
無くす = 失(く)す = (喪す) 1. To lose something that you've had up till now. 2. To get rid of a bid situation. 3. When used in the sense of 亡くす, it can be spelled as 喪す. However, due to the government's attempts to lower spelling options, it is no longer prevalent. \hfill\break
Now, we have 失う・喪う. 1. To lose something that had been in your possession or on you(r person). 2. To lose the chance at getting something. 3. To be gone from having been taken\slash stolen from (you). 4. To end up not knowing the path to take. 5. To lose (a loved one). This is where the spelling 喪う comes into play. \hfill\break
So, we see that there is heavy interchangeability between the two. \hfill\break
鍵が\{無くなる・なくなる\}。 = For the key to be lost\slash disappear\slash be gone. 鍵を\{なくす・無くす・失(く)す\} = To lose a key. 鍵を失う = To lose a key. (Sounds more like it has been in your possession or on your person) To get rid of \dothyp{}\dothyp{}\dothyp{}.X A 鍵 isn't a 事柄. 
\begin{center}
\textbf{連用形+しない }
\end{center}

\par{ 連用形 + は+ しない is like a strong "won't" in the sense of not being able to do something. }

\par{25. ${\overset{\textnormal{だれ}}{\text{誰}}}$ の ${\overset{\textnormal{かぎ}}{\text{鍵}}}$ も ${\overset{\textnormal{あ}}{\text{合}}}$ いはしない。 \hfill\break
Nobody's key will work. }

\begin{center}
\textbf{Verb + に + Negative Potential Verb }
\end{center}

\par{ This is an emphatic pattern used to show that even when you want to do something, you can't. }

\par{26. 雪が ${\overset{\textnormal{ふ}}{\text{降}}}$ り ${\overset{\textnormal{つ}}{\text{積}}}$ もって、出かけるに出かけられない。 \hfill\break
The snow piled up, and we were unable to go out (though we wanted to). }

\par{27. 嵐が強くて、行くに行けない。 \hfill\break
The storm is so strong that I can't even go. }

\par{ \textbf{More Verbs in the Potential }}

\begin{ltabulary}{|P|P|P|P|}
\hline 

Can go home \hfill\break
& 帰られる & Can swim \hfill\break
& 泳げる \\ \cline{1-4}

Can die \hfill\break
& 死ねる & Can buy \hfill\break
& 買える \\ \cline{1-4}

Can drink \hfill\break
& 飲める & Can wait \hfill\break
& 待てる \\ \cline{1-4}

Can take & 取れる & Can sing & 歌える \\ \cline{1-4}

\end{ltabulary}

\begin{center}
\textbf{Verbs that Cannot Have Potentials }
\end{center}

\par{ Non-volitional verbs cannot have potential forms. This includes verbs of natural phenomenon like 降る, ${\overset{\textnormal{ひか}}{\text{光}}}$ る, ${\overset{\textnormal{なが}}{\text{流}}}$ れる, and ${\overset{\textnormal{こお}}{\text{凍}}}$ る, those concerning human emotion and physiology ( ${\overset{\textnormal{いた}}{\text{痛}}}$ む, ${\overset{\textnormal{しび}}{\text{痺}}}$ れる (to be paralyzed), ${\overset{\textnormal{うらや}}{\text{羨}}}$ む (to be jealous), any verbs that end in ある (as they have no volition), and any pattern that has no control involved like phrases with つく and いく such as ${\overset{\textnormal{そうぞう}}{\text{想像}}}$ がつく (one can imagine) and ${\overset{\textnormal{なっとく}}{\text{納得}}}$ がいく (to accept as valid). }
 
\par{Notice that these are all intransitive verbs. However, it\textquotesingle s not to say that all intransitive verbs don't have potential forms. Think of motion verbs like 走る, 行く, 帰る, ${\overset{\textnormal{もど}}{\text{戻}}}$ る, 来る, etc. All of these have potentials because volition is involved in their meanings. }
 
\par{\textbf{Set Phrase Note }: あられる, the potential of ある, does happen to exist in the phrase あられもない, which means "impossible". As this is the case, it doesn't contradict what has been said above because there is no volition in impossibility. }
 
\begin{center}
\textbf{とても }
\end{center}
 
\par{とても in a negative sentence means "can't possibly". }
 
\par{28. とってもじゃないけど、そんなもん(なんか)買えねーよ。(Vulgar) \hfill\break
I can't possibly buy something like that! }
 
\par{29. とても泳ぎきれない。 \hfill\break
I can't possibly completely swim (that distance). }
 
\begin{center}
\textbf{なかなか }
\end{center}
 
\par{なかなか in a negative sentence means "not easily\slash by no means". It's used a lot with potential expressions. }
 
\par{30. 漢字がなかなか ${\overset{\textnormal{おぼ}}{\text{覚}}}$ えられなくて ${\overset{\textnormal{こま}}{\text{困}}}$ っている学生は、たくさんいますね。 \hfill\break
There are a lot of students that are troubled at not being able to quite memorize Kanji, aren't there? }
 
\par{31. ${\overset{\textnormal{さくや}}{\text{昨夜}}}$ 、なかなか ${\overset{\textnormal{ねむ}}{\text{眠}}}$ れなかったから、今日はとっても眠くてたまらない。 \hfill\break
Since I couldn't easily sleep last night, today I really want to sleep and can't stand it. }
 
\par{32. 宿題がなかなかできなくて、困っています。 \hfill\break
I'm troubled that I can't seem to do my homework. }
 
\par{33. タバコはよくないと分かっていても、なかなかやめられなくて、困っている人が多いです。 \hfill\break
There are a lot of people that can't quite quit smoking even though they know that tobacco is bad. }
 
\par{34. ${\overset{\textnormal{じさ}}{\text{時差}}}$ ボケで、なかなか ${\overset{\textnormal{ね}}{\text{寝}}}$ られなくて、 ${\overset{\textnormal{こま}}{\text{困}}}$ りました。 \hfill\break
I was troubled because I couldn't quite sleep due to jet lag. }
35.  漢字を3000覚えるという ${\overset{\textnormal{もくひょう}}{\text{目標}}}$ があるが、なかなか勉強する時間がない。 I have a goal of learning 3,000 Kanji, but I don't quite have the time.       