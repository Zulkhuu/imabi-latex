    
\chapter{Before}

\begin{center}
\begin{Large}
第61課: Before  
\end{Large}
\end{center}
 
\par{ Expressing "before" in Japanese is not as difficult as other similar temporal phrases, but it still come with its own set of challenges. }
      
\section{Before: 前}
 
\par{ When used as a temporal word, 前 means "before". It can be used with verbs in the non-past tense or with other nouns. Because 前 is also noun, you need to use the particle の when it is after another noun phrase. }

\par{1. それはどのくらい ${\overset{\textnormal{}}{\text{前}}}$ のことでしたか。 \hfill\break
How long ago was that? }

\par{2. 私は ${\overset{\textnormal{りょこう}}{\text{旅行}}}$ に行く前に、トラベラーズ・チェックを買いました。 \hfill\break
I bought traveler's checks before I went on my trip. }

\par{\textbf{重言 Note }: Some speakers use 旅行に行く, but this is traditionally incorrect. It is more correctly 旅行(を)する. This is because 旅行 literally means "going on a trip". }

\par{4. ${\overset{\textnormal{かいがいりょこう}}{\text{海外旅行}}}$ の前に、パスポートを ${\overset{\textnormal{と}}{\text{取}}}$ りました。 \hfill\break
I got a passport before going abroad. }

\par{5. 食べる前に、手を ${\overset{\textnormal{あら}}{\text{洗}}}$ います。 \hfill\break
I wash my hands before I eat. }

\par{6. ${\overset{\textnormal{ぼく}}{\text{僕}}}$ が帰る前に、ホストファミリーのお ${\overset{\textnormal{かあ}}{\text{母}}}$ さんは ${\overset{\textnormal{りょうり}}{\text{料理}}}$ をしました。 \hfill\break
Before I got home, my host family mother had cooked. }

\par{7. これは11年前の新聞だ。 \hfill\break
This is a newspaper from 11 years ago. }

\par{8. 日本に行く前に、ちょっと日本語を勉強しました。 \hfill\break
I studied some Japanese before going to Japan. }

\par{9. 彼は3年前に死んだ。 \hfill\break
He died three years ago. }

\par{10. 彼は3年前に死んでいた。   (今になってわかったこと) \hfill\break
He's been dead for three years. }

\par{11. 彼は3年前に死んでいる。 \hfill\break
It's been three years since he died. }

\par{\textbf{Word Note }: ${\overset{\textnormal{な}}{\text{亡}}}$ くなる is more neutral than 死ぬ for when someone passes away. }

\begin{center}
 \textbf{前: ぜん }
\end{center}

\par{ 前 also has the reading ぜん. When used as a suffix after Sino-Japanese words, it is a very formal and suitable for 書き言葉. Its usage is almost unheard of the spoken language, but it is part of the official reading of many phrases in the jargon of many fields. }

\par{12. ${\overset{\textnormal{いち}}{\text{1}}}$ ${\overset{\textnormal{にち}}{\text{日}}}$ 3回 ${\overset{\textnormal{しょくぜん}}{\text{食前}}}$ に ${\overset{\textnormal{ふくよう}}{\text{服用}}}$ してください。 \hfill\break
Please take (medicine) three times a day before eating. }

\par{\textbf{Word Note }: 服用する is more formal than 飲む and is used in settings in which medical terminology is most appropriate. }

\par{\textbf{Reading Note }: 一日 is not read as ついたち when 一日 is used to mean "a day". }

\par{13. ${\overset{\textnormal{}}{\text{使用前}}}$ ${\overset{\textnormal{じしゅ}}{\text{自主}}}$ ${\overset{\textnormal{けんさ}}{\text{検査}}}$ \hfill\break
 Independent\slash autonomous inspection before use }

\par{14. 期日前投票 \hfill\break
Early voting }

\par{\textbf{Reading Note }: Many people would think きじつぜんとうひょう meant 期日全投票. So, in the spoken language, this word should be read as きじつまえとうひょう. }

\par{15. 出生前診断 \hfill\break
Prenatal diagnosis }

\par{\textbf{Reading Note }: 出生 is usually read as しゅっしょう. However, some organizations use the reading しゅっせい. For instance, in medical terminology 出生届 (birth registration) is read as しゅっせいとどけ despite the fact that most people say しゅっしょうとどけ. In the spoken language, 前 would be read as まえ as expected. }

\par{ As a prefix, it is used to mean "ex-". However, it comes with the word 元‐, which also means "ex-" but implies that other "ex-\dothyp{}\dothyp{}\dothyp{}" have existed before. This, though, is not sufficient in distinguishing them all the time. Someone who has stepped back from his position would be 前〇〇. Before this person, anyone with the title would be 元〇〇. What if you had a boat and there were no person to carry the legacy after the vessel sunk? You would use 元 ${\overset{\textnormal{せんちょう}}{\text{船長}}}$ instead of 前船長. }

\par{16. ${\overset{\textnormal{げんしゃちょう}}{\text{現社長}}}$ が前社長を ${\overset{\textnormal{さつがい}}{\text{殺害}}}$ したという ${\overset{\textnormal{じたい}}{\text{事態}}}$ が ${\overset{\textnormal{あか}}{\text{明}}}$ るみに出た。 \hfill\break
The state of affairs of the current company president having murdered the ex-company president \hfill\break
has come to light. }

\par{17. どうしても元カノを気にしてしまうよ。 \hfill\break
My ex-girlfriend always gets to me no matter what. }

\par{\textbf{Contraction Note }: カノ is a contraction of 彼女 when used to mean "girlfriend". }

\par{\textbf{Final Note }: Aside from this, 前 read as ぜん exists in other time phrases such as 前 ${\overset{\textnormal{じつ}}{\text{日}}}$ (the other day). }

\begin{center}
\textbf{直前 } 
\end{center}

\par{ Right before can be expressed with the word 直前. The word also has the literal direction meaning of "right in front of". When used in a temporal sense, it is frequently appended to nouns without the need of の. However, the use of の is most certainly appropriate in the spoken language. }

\par{18. ${\overset{\textnormal{ききょう}}{\text{帰京}}}$ する ${\overset{\textnormal{ひこうき}}{\text{飛行機}}}$ の直前に ${\overset{\textnormal{とうちゃく}}{\text{到着}}}$ する ${\overset{\textnormal{じこく}}{\text{時刻}}}$ のバスにしか ${\overset{\textnormal{ま}}{\text{間}}}$ に ${\overset{\textnormal{あ}}{\text{合}}}$ わなかったこともあります。(Written) \hfill\break
I've also only managed to make it to a bus with an arrival time (to the airport) right before my flight back (to Tokyo). }

\par{\textbf{Word Note }: 帰京する = 東京に帰る. }

\par{19. 車の直前に人が ${\overset{\textnormal{とつぜん}}{\text{突然}}}$ ${\overset{\textnormal{と}}{\text{飛}}}$ び出してきた。 \hfill\break
A person suddenly dashed out right in front of the\slash my car. }

\par{20. ${\overset{\textnormal{ばくはつ}}{\text{爆発}}}$ ${\overset{\textnormal{ちょくぜん}}{\text{直前}}}$ にガソリンが ${\overset{\textnormal{きか}}{\text{気化}}}$ したような ${\overset{\textnormal{にお}}{\text{臭}}}$ いがしていた。 \hfill\break
There was a smell of gasoline vaporizing right before the explosion. }

\par{\textbf{漢字 Note }: 臭い gives a nuance of stench whereas 匂い is any given smell. The word is also frequently spelled in ひらがな. }
    