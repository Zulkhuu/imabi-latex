    
\chapter{Adjectives III}

\begin{center}
\begin{Large}
第51課: Adjectives III: No-Adjectival Nouns ノ形容詞 
\end{Large}
\end{center}
 
\par{ In our coverage of “adjectival nouns ( ${\overset{\textnormal{けいようどうし}}{\text{形容動詞}}}$ ),” we looked at a class of words that behave as adjectives but are etymologically nouns that use the copula to behave as such. The topic of \emph{adjectival nouns }, though, does not stop with adjectives that use the copula to modify nouns. There are also many words that are adjectival that look like nouns but use \emph{no }の to modify nouns and \emph{ni }に to modify verbs as adverbs. In essence, they're hybrids of nouns and adjectives. }

\par{ This lesson will focus on those words as well as their close connection to \emph{na }-adjectival nouns. Although distinguishing between these two groups of words is often impossible to do, you'll learn more about the real dynamics of Japanese adjectives. }
      
\section{Vocabulary List}
 
\par{\textbf{Nouns }}

\par{・修行 \emph{Shugyō }– (Ascetic) training\slash discipline }

\par{・人生 \emph{Jinsei }– (Human) life }

\par{・意味 \emph{Imi }– Meaning }

\par{・人 \emph{Hito }– Person }

\par{・ブッダ(仏陀) \emph{Budda }– (A) Buddha }

\par{・生活 \emph{Seikatsu }– Life\slash livelihood }

\par{・噂 \emph{Uwasa }– Rumor }

\par{・車内アナウンス \emph{Shanai anaunsu }– Train announcement }

\par{・喋り方 \emph{Shaberikata }– Way of speaking }

\par{・音程 \emph{Ontei }– Pitch\slash interval }

\par{・夕暮れ前 \emph{Yūgure-mae }– Before dusk }

\par{・空 \emph{Sora }– Sky }

\par{・使い方 \emph{Tsukaikata }– Way to use }

\par{・注意 \emph{Chūi }– Attention }

\par{・注射 \emph{Chūsha }– Shot\slash injection }

\par{・土鍋 \emph{Donabe }– Earthenware pot }

\par{・底 \emph{Soko }– Bottom }

\par{・左足 \emph{Hidariashi }– Left foot }

\par{・右足 \emph{Migiashi }– Right foot }

\par{・ブレーキ \emph{Burēki }- Break }

\par{・アクセル \emph{Akuseru }- Accelerator }

\par{・景気 \emph{Keiki }- Business }

\par{・時期 \emph{Jiki }– Time\slash period }

\par{・場所 \emph{Basho }– Place }

\par{・株式 \emph{Kabushiki }– Stock (company) }

\par{・価格 \emph{Kakaku }– Price\slash value\slash cost }

\par{・大学 \emph{Daigaku }– University\slash college }

\par{・大人 \emph{Otona }– Adult }

\par{・最初 \emph{Saisho }– Beginning }

\par{・態度 \emph{Taido }– Attitude }

\par{・感じ \emph{Kanji }- Feeling }

\par{・鶏肉 \emph{Toriniku }- Poultry }

\par{・俎板 \emph{Manaita }- Cutting board }

\par{・ボール \emph{Bōru }– Bowl\slash ball }

\par{・調理器具 \emph{Chōri kigu }– Cooking ware }

\par{・新興国 \emph{Shinkōkoku }– Emerging nation }

\par{・経済 \emph{Keizai }– Economy }

\par{・成長 \emph{Seichō }– Growth\slash development }

\par{・食文化 \emph{Shokubunka }– Food culture\slash cuisine }

\par{・舌 \emph{Shita }– Tongue\slash palate }

\par{・必要 \emph{Hitsuyō }- Necessity }

\par{・料理 \emph{Ryōri }– Cuisine\slash cookery }

\par{・思い込み \emph{Omoikomi }– Assumption }

\par{・専業主婦 \emph{Sengyō shufu }- Housewife }

\par{・措置 \emph{Sochi }– Measure }

\par{・女性 \emph{Josei }– Female }

\par{・就労 \emph{Shūrō }– Being employed }

\par{・意欲 \emph{Iyoku }– Will\slash desire\slash urge }

\par{・結果 \emph{Kekka }– Result\slash effect }

\par{・症状 \emph{Shōjō }– Symptom(s) }

\par{・叔父 \emph{Oji }– Uncle (younger than one\textquotesingle s parent) }

\par{・家 \emph{Ie }– House\slash home }

\par{・雨 \emph{Ame }– Rain }

\par{・髪 \emph{Kami }– Hair }

\par{・色 \emph{Iro }- Color }

\par{・草原 \emph{Sōgen }- Grassland }

\par{・裏側 \emph{Uragawa }– Back(side) }

\par{・黒 \emph{Kuro }– Black }

\par{・服 \emph{Fuku }- Clothes }

\par{・体 \emph{Karada }– Body }

\par{・健康 \emph{Kenkō }– Health }

\par{・状態 \emph{Jōtai }– Condition }

\par{・心臓 \emph{Shinzō }– Heart }

\par{・森 \emph{Mori }– Forest }

\par{・奴 \emph{Yatsu }– Guy }

\par{・進化 \emph{Shinka }– Evolution }

\par{・発展 \emph{Hatten }– Development\slash advancement }

\par{・人間 \emph{Ningen }– Human }

\par{・関係 \emph{Kankei }–  Relation }

\par{・意見 \emph{Iken }- Opinion }

\par{・種類 \emph{Shurui }– Species }

\par{・哺乳類 \emph{Honyūrui }– Mammal }

\par{・鳥 \emph{Tori }– Bird }

\par{・担当者 \emph{Tantōsha }– Manager }

\par{・テクニック \emph{Tekunikku }– Technique }

\par{・立場 \emph{Tachiba }– Position\slash footing }

\par{・貿易 \emph{Bōeki }– Trade }

\par{・こと \emph{Koto }– Thing\slash incident\slash situation }

\par{・太陽 \emph{Taiyō }– Sun }

\par{・恩恵 \emph{Onkei }– Grace\slash favor }

\par{・判断 \emph{Handan }– Judgment }

\par{・お酒 \emph{Osake }– Liquor }

\par{・運動 \emph{Undō }– Exercise }

\par{・現象 \emph{Genshō }– Phenomenon }

\par{・友達 \emph{Tomodachi }– Friend(s) }

\par{・別れ \emph{Wakare }– Parting\slash Separation }

\par{・相性 \emph{Aishō }– Affinity }

\par{・女神 \emph{Megami }– Goddess }

\par{・彼氏 \emph{Kareshi }– Boyfriend }

\par{\textbf{Pronouns }}

\par{・私達 \emph{Watashitachi }- We }

\par{\textbf{Proper Nouns }}

\par{・関東人 \emph{Kantōjin }- Kanto(ite) }

\par{・日本人 \emph{Nihonjin }– Japanese person }

\par{\textbf{Question Words }}

\par{・誰にでも \emph{Dare ni demo }– Anyone }

\par{\textbf{Demonstratives }}

\par{・この \emph{Kono }– This (adj.) }

\par{・こういう \emph{Kō iu }– Like this }

\par{・あの \emph{Ano }– That (over there) (adj.) }

\par{・あそこ \emph{Asoko }– That over there }

\par{\textbf{Adjectives }}

\par{・厳しい \emph{Kibishii }– Harsh\slash strict }

\par{・浅い \emph{Asai }– Shallow }

\par{・薄い \emph{Usui }– Thin\slash pale }

\par{・良い \emph{Yoi }– Good }

\par{・麗しい \emph{Uruwashii }– Beautiful\slash lovely }

\par{\textbf{Number Phrases }}

\par{・二人 \emph{Futari }– Two people }
  \textbf{Adjectival Nouns }
\par{・本当\{の\} \emph{Hontō [no] }– Actual\slash true\slash real }

\par{・大変\{な\} \emph{Taihen [na] }– Serious\slash hard\slash immense }

\par{・独特\{な・の\} \emph{Dokutoku [na\slash no] }– Unique\slash peculiar }

\par{・特別\{な・の\} \emph{Tokubetsu [na\slash no] }– Especial }

\par{・普通\{の\} \emph{Futsū [no] }– Ordinary\slash general }

\par{・四角\{の・な\} \emph{Shikaku [no\slash na] }- Square }

\par{・連日\{の \emph{\} }\emph{Renjitsu [no] }– Prolonged\slash every day }

\par{・一般\{の\} \emph{Ippan [no]- }General }

\par{・一流\{の\} \emph{Ichiryū [no] }– First-class\slash top grade }

\par{・大人\{の・な\} \emph{Otona [no\slash na] }– Adult(-like) }

\par{・最初\{の\} \emph{Saisho [no] }- First }

\par{・苦手\{な\} \emph{Nigate [na] }– Bad at }

\par{・生\{の・な\} \emph{Nama [no\slash na] }– Raw\slash unprocessed\slash live\slash crude\slash unprotected }

\par{・他\{の\} \emph{Ta\slash hoka [no] }- Other }

\par{・多く\{の\} \emph{Ōku [no] }- A lot }

\par{・高度\{な・の\} \emph{Kōdo [na\slash no] }– Sophisticated }

\par{・それぞれ\{の\} \emph{Sorezore [no] }– Each }

\par{・必要\{な\} \emph{Hitsuyō [na] }– Necessary }

\par{・世界中\{の\} \emph{Sekai-jū [no] }– Worldwide }

\par{・NG\{な\} \emph{Enujii [na] }– Not good }

\par{・優遇\{の\} \emph{Yūgū [no] }– Preferential }

\par{・皮肉\{な\} \emph{Hiniku [na] }– Ironic }

\par{・個別\{の\} \emph{Kobetsu [no] }– Individual }

\par{・緑色\{の\} \emph{Midori\textquotesingle iro [no] }– Green }

\par{・黒\{の\} \emph{Kuro [no] }– Black }

\par{・健康\{な\} \emph{Kenkō [na] }– Healthy }

\par{・鮮やか\{な\} \emph{Azayaka [na] }– Vivid\slash vibrant }

\par{・大量\{の\} \emph{Tairyō [no] }– Large amount of }

\par{・地球上\{の\} \emph{Chikyūjō [no] }– On Earth }

\par{・色々\{な・の\} \emph{Iroiro [na\slash no] }– Various }

\par{・急速\{な\} \emph{Kyūsoku [na] }– Rapid }

\par{・別\{の・な\} \emph{Betsu [no\slash na] }– Different }

\par{・対等\{な・の\} \emph{Taitō [na\slash no] }– Equal }

\par{・当たり前\{な・の\} \emph{Atarimae [na\slash no] }– Obvious }

\par{・当然\{な・の\} \emph{Tōzen [na\slash no] }– Obvious }

\par{・平気\{な\} \emph{Heiki [na] }– Calm\slash fine }

\par{・大丈夫\{な\} \emph{Daijōbu [na] }– Alright }

\par{・適度\{な・の\} \emph{Tekido [na\slash no] }– Moderate (amount) }

\par{・不思議\{な\} \emph{Fushigi [na] }– Mysterious }

\par{・永遠\{の\} \emph{Eien [no] }– Eternal }

\par{・仲良し\{の\} \emph{Nakayoshi [no] }– Close\slash intimate }

\par{・最悪\{の・な\} \emph{Saiaku [no\slash na] }– Worst }

\par{\textbf{Adverbs }}

\par{・わざと \emph{Waza to }– On purpose }

\par{・若干 \emph{Jakkan }– Somewhat }

\par{・最初から \emph{Saisho kara }– From the beginning }

\par{・ごく \emph{Goku }– Quite }

\par{・ちょっと \emph{Chotto }– A little }

\par{・そのまま \emph{Sono mama }– As is }

\par{・延々と \emph{En\textquotesingle en to }– Endlessly }

\par{・必ず \emph{Kanarazu }– Always }

\par{・極めて \emph{Kiwamete }– Extremely }

\par{・今日 \emph{Kyō }– Today }

\par{・今週 \emph{Konshū }– This week }

\par{\textbf{\emph{(ru) Ichidan }Verbs }}

\par{・積み重ねる \emph{Tsumikasaneru }– To pile up (trans.) }

\par{・上げる \emph{Ageru }– To raise (trans.) }

\par{・別ける \emph{Wakeru }– To divide\slash split\slash share\slash distinguish (trans.) }

\par{・触れる \emph{Fureru }– To touch (intr.) }

\par{・やめる \emph{Yameru }– To stop (trans.) }

\par{・食べる \emph{Taberu }– To eat (trans.) }

\par{・認める \emph{Mitomeru }– To recognize (trans.) }

\par{・着る \emph{Kiru }– To wear (trans.) }

\par{・遂げる \emph{Togeru }– To achieve\slash accomplish (trans.) }

\par{・いる \emph{Iru }– To be (animate and live) (trans.) }

\par{・出る \emph{Deru }– To leave\slash appear\slash emerge\slash come up (intr.) }

\par{・起きる \emph{Okiru }– To get up\slash occur  (intr.) }

\par{・別れる \emph{Wakareru }– To separate\slash break up (intr.) }

\par{\textbf{\emph{(u) Godan }Verbs }}

\par{・悟る \emph{Satoru }– To be enlightened\slash discern (trans.) }

\par{・なる \emph{Naru }– To be(come) (intr.) }

\par{・払う \emph{Harau }– To pay (attention\slash money)\slash brush off (trans.) }

\par{・立ち入る \emph{Tachi\textquotesingle iru }– To enter\slash trespass (int.) }

\par{・違う \emph{Chigau }– To be different\slash wrong (intr.) }

\par{・取る \emph{Toru }– To take (trans.) }

\par{・残る \emph{Nokoru }– To be left (intr.) }

\par{・焼く \emph{Yaku }– To burn\slash bake\slash toast\slash heat up\slash tan\slash burn (disc) (trans.) }

\par{・習う \emph{Narau }– To take lessons in (trans.) }

\par{・持つ \emph{Motsu }– To hold\slash possess (trans.) }

\par{・招く \emph{Maneku }– To invite\slash beckon (trans.) }

\par{・削ぐ \emph{Sogu }– To chip off\slash discourage (trans.) }

\par{・保つ \emph{Tamotsu }– To preserve (trans.) }

\par{・広がる \emph{Hirogaru }– To extend\slash stretch (intr.) }

\par{・行う Okonau – To perform\slash conduct (trans.) }

\par{・棲む \emph{Sumu }– To inhabit (intr.) }

\par{・思う \emph{Omou }– To think (trans.) }

\par{・言う \emph{Iu }– To say (trans.) }

\par{\textbf{\emph{suru }Verbs }}

\par{・留学(を)する \emph{Ryūgaku (wo) suru }– To study abroad (intr.) }

\par{・アナウンスする \emph{Anaunsu suru }– To announce (trans.) }

\par{・運転する \emph{Unten suru }– To drive (trans.) }

\par{・下落する \emph{Geraku suru }– To depreciate (intr.) }

\par{・入学する \emph{Nyūgaku suru }– To enroll (intr.) }

\par{・調理する \emph{Chōri suru }– To prepare (food) (trans.) }

\par{・指定する \emph{Shitei suru }– To designate\slash specify (trans.) }

\par{・料理する \emph{Ryōri suru }– To cook }

\par{・対する \emph{Tai suru }– To face\slash be directed toward\slash contrast with (intr.) }

\par{・管理する \emph{Kanri suru }– To manage }

\par{・対応する \emph{Taiō suru }– To handle }

\par{・構築する \emph{Kōchiku suru }– To construct }

\par{\textbf{Set Phrases }}

\par{・鼻にかかった \emph{Hana ni kakatta }– Nasal }

\par{・特別の機関 \emph{Tokubetsu no kikan }– Attached organization }

\par{・百薬の長 \emph{Hyakuyaku no chō }– Chief of medicines }

\par{・不思議の国 \emph{Fushigi no kuni }– Wonderland }

\par{・ようこそ \emph{Yōkoso }– Welcome }
      
\section{No vs. Na}
 
\par{ At first glance, many \emph{no }-adjectives seem to share the same functions that \emph{na }-adjectival nouns have. These so-called \emph{no }-adjectives may be used to modify nouns, form the predicate of a sentence, or even be used adverbially by switching out \emph{no }の with \emph{ni }に. }

\par{1. ${\overset{\textnormal{きび}}{\text{厳}}}$ しい ${\overset{\textnormal{しゅぎょう}}{\text{修行}}}$ を ${\overset{\textnormal{つ}}{\text{積}}}$ み ${\overset{\textnormal{かさ}}{\text{重}}}$ ねて ${\overset{\textnormal{じんせい}}{\text{人生}}}$ の ${\overset{\textnormal{ほんとう}}{\text{本当}}}$ の ${\overset{\textnormal{いみ}}{\text{意味}}}$ を ${\overset{\textnormal{さと}}{\text{悟}}}$ った ${\overset{\textnormal{ひと}}{\text{人}}}$ を「ブッダ」と ${\overset{\textnormal{い}}{\text{言}}}$ います。 \hfill\break
\emph{Kibishii shugyō wo tsumikasanete jinsei no hontō no imi wo satotta hito wo "Budda" to iimasu. \hfill\break
}We call people who have built up rigid discipline and have become enlightened about the true meaning of life a "Buddha." }

\par{2. ${\overset{\textnormal{りゅうがくせいかつ}}{\text{留学生活}}}$ は ${\overset{\textnormal{ほんとう}}{\text{本当}}}$ に ${\overset{\textnormal{たいへん}}{\text{大変}}}$ でした。 \hfill\break
\emph{Ryūgaku seikatsu wa hontō ni taihen deshita. \hfill\break
}Life while studying abroad was really tough. }

\par{3. あの ${\overset{\textnormal{うわさ}}{\text{噂}}}$ は ${\overset{\textnormal{ほんとう}}{\text{本当}}}$ だった。 \hfill\break
\emph{Ano uwasa wa hontō datta. }\hfill\break
That rumor was true. }

\par{4. ${\overset{\textnormal{しゃない}}{\text{車内}}}$ アナウンスをする人たちは、わざと ${\overset{\textnormal{はな}}{\text{鼻}}}$ にかかった ${\overset{\textnormal{どくとく}}{\text{独特}}}$ の ${\overset{\textnormal{しゃべ}}{\text{喋}}}$ り ${\overset{\textnormal{かた}}{\text{方}}}$ で ${\overset{\textnormal{おんてい}}{\text{音程}}}$ を ${\overset{\textnormal{あ}}{\text{上}}}$ げてアナウンスしているんですよ。 \hfill\break
\emph{Shanai anaunsu wo suru hitotachi wa, waza to hana ni kakatta dokutoku no shaberikata de ontei wo agete anaunsu shite iru n desu yo. }\hfill\break
Train announcers purposely do the announcements with a raised pitch by speaking in a peculiarly nasal fashion. }

\par{5. ${\overset{\textnormal{ゆうぐ}}{\text{夕暮}}}$ れ ${\overset{\textnormal{まえ}}{\text{前}}}$ の ${\overset{\textnormal{そら}}{\text{空}}}$ は ${\overset{\textnormal{どくとく}}{\text{独特}}}$ でした。 \hfill\break
\emph{Yūgure-mae no sora wa dokutoku deshita. \hfill\break
}The sky before dusk was peculiar. }

\par{6. ${\overset{\textnormal{つか}}{\text{使}}}$ い ${\overset{\textnormal{かた}}{\text{方}}}$ も ${\overset{\textnormal{じゃっかんどくとく}}{\text{若干独特}}}$ になっています。 \hfill\break
\emph{Tsukaikata mo jakkan dokutoku ni natte imasu. \hfill\break
}How to use it is also somewhat peculiar. }

\par{7. ${\overset{\textnormal{とくべつ}}{\text{特別}}}$ に ${\overset{\textnormal{ちゅうい}}{\text{注意}}}$ を ${\overset{\textnormal{はら}}{\text{払}}}$ ってください。 \hfill\break
\emph{Tokubetsu ni }\emph{chūi wo haratte kudasai. \hfill\break
}Pay especial attention. }

\par{8. ${\overset{\textnormal{ふつう}}{\text{普通}}}$ の ${\overset{\textnormal{どなべ}}{\text{土鍋}}}$ と ${\overset{\textnormal{ちが}}{\text{違}}}$ って ${\overset{\textnormal{そこ}}{\text{底}}}$ が ${\overset{\textnormal{しかく}}{\text{四角}}}$ で、 ${\overset{\textnormal{あさ}}{\text{浅}}}$ いのです。 \hfill\break
\emph{Futsū no }\emph{donabe to chigatte soko ga shikaku de, asai no desu. }\hfill\break
Different from a regular earthenware pot, the bottom of (this one) is square and shallow. }

\par{9. ${\overset{\textnormal{れんじつちゅうしゃ}}{\text{連日注射}}}$ はごく ${\overset{\textnormal{ふつう}}{\text{普通}}}$ ですよ。 \hfill\break
\emph{Renjitsu chūsha wa goku futsū desu yo. \hfill\break
}Prolonged shots are quite normal. }

\par{\textbf{Grammar Note }: At times, \emph{no }の is omitted from \emph{no }-adjective phrases like in \emph{ \textbf{renjitsu }chuusha }\textbf{連日 }注射. This is very common when the resultant phrase is four characters or longer and when said resultant phrase semantically refers to a particular concept. }

\par{10. ${\overset{\textnormal{さいしょ}}{\text{最初}}}$ から ${\overset{\textnormal{ひだりあし}}{\text{左足}}}$ でブレーキ、 ${\overset{\textnormal{みぎあし}}{\text{右足}}}$ でアクセルと ${\overset{\textnormal{わ}}{\text{別}}}$ けて ${\overset{\textnormal{ふつう}}{\text{普通}}}$ に ${\overset{\textnormal{うんてん}}{\text{運転}}}$ しています。 \hfill\break
\emph{Saisho kara hidariashi de burēki, migashi de akuseru to wakete futsū ni unten shite imasu. \hfill\break
}I\textquotesingle ve been driving like normal from the beginning by putting my left foot on the break and my right foot on the accelerator. }

\par{11. あそこは ${\overset{\textnormal{いっぱん}}{\text{一般}}}$ の ${\overset{\textnormal{ひと}}{\text{人}}}$ が ${\overset{\textnormal{た}}{\text{立}}}$ ち ${\overset{\textnormal{い}}{\text{入}}}$ れない ${\overset{\textnormal{ばしょ}}{\text{場所}}}$ です。 \hfill\break
\emph{Asoko wa ippan no hito ga tachi\textquotesingle irenai basho desu. }\hfill\break
That place over there can\textquotesingle t be entered by people at large. }

\par{\textbf{Grammar Note }: \emph{Tachi\textquotesingle ireru }立ち入れる is the potential (can) form of the verb \emph{tachi\textquotesingle iru }立ち入る, which is used to mean “enter” usually in the sense of “to trespass.” }

\par{12. ${\overset{\textnormal{いっぱん}}{\text{一般}}}$ に ${\overset{\textnormal{けいき}}{\text{景気}}}$ が ${\overset{\textnormal{わる}}{\text{悪}}}$ い ${\overset{\textnormal{じき}}{\text{時期}}}$ には ${\overset{\textnormal{かぶしき}}{\text{株式}}}$ の ${\overset{\textnormal{かかく}}{\text{価格}}}$ が ${\overset{\textnormal{げらく}}{\text{下落}}}$ します。 \hfill\break
\emph{Ippan ni }\emph{keiki ga warui jiki ni wa kabushiki no kakaku ga geraku shimasu. }\hfill\break
Generally when business is bad, the stock price\slash stock prices goes down. }

\par{13. ${\overset{\textnormal{いちりゅう}}{\text{一流}}}$ の ${\overset{\textnormal{だいがく}}{\text{大学}}}$ に ${\overset{\textnormal{にゅうがく}}{\text{入学}}}$ しました。 \hfill\break
\emph{Ichiryū no }\emph{daigaku ni nyūgaku shita. \hfill\break
}I enrolled into a first class university. }

\begin{center}
 \textbf{\emph{No }の or \emph{Na }な }
\end{center}

\par{ Many \emph{no }-adjectival nouns can also be used as \emph{na }-adjectival nouns instead. The nuance will tend to be different, but the interchangeability is still there. }

\par{14. ${\overset{\textnormal{ほんとう}}{\text{本当}}}$ に ${\overset{\textnormal{おとな}}{\text{大人}}}$ な ${\overset{\textnormal{ひと}}{\text{人}}}$ が ${\overset{\textnormal{にがて}}{\text{苦手}}}$ です。 \hfill\break
\emph{Hontō ni otona na }\emph{hito ga nigate desu. } \hfill\break
I\textquotesingle m bad with really adult-like people. }

\par{15. ${\overset{\textnormal{だれ}}{\text{誰}}}$ にでも ${\overset{\textnormal{おとな}}{\text{大人}}}$ の ${\overset{\textnormal{たいど}}{\text{態度}}}$ を ${\overset{\textnormal{と}}{\text{取}}}$ っていますか。 \hfill\break
\emph{Dare ni demo otona no taido wo totte imasu ka? }\hfill\break
Do you take an adult attitude with anyone? }

\par{16. ちょっと ${\overset{\textnormal{なま}}{\text{生}}}$ な ${\overset{\textnormal{かん}}{\text{感}}}$ じが ${\overset{\textnormal{のこ}}{\text{残}}}$ っている。 \hfill\break
\emph{Chotto nama na kanji ga nokotte iru. }\hfill\break
There\textquotesingle s still a somewhat uncooked\slash unprocessed\slash unpolished feel to it. }

\par{\textbf{Meaning Note }: The meaning of \emph{nama }生 can be quite varied, which is why context is needed to know how it\textquotesingle s meant. }

\par{17. ${\overset{\textnormal{なま}}{\text{生}}}$ の ${\overset{\textnormal{とりにく}}{\text{鶏肉}}}$ に ${\overset{\textnormal{ふ}}{\text{触}}}$ れた ${\overset{\textnormal{まないた}}{\text{俎板}}}$ や ${\overset{\textnormal{ぼ}}{\text{ボ}}}$ ー ${\overset{\textnormal{る}}{\text{ル}}}$ などの ${\overset{\textnormal{ちょうりきぐ}}{\text{調理器具}}}$ はそのまま ${\overset{\textnormal{ほか}}{\text{他}}}$ の ${\overset{\textnormal{ちょうり}}{\text{調理}}}$ に ${\overset{\textnormal{しよう}}{\text{使用}}}$ することはやめましょう。 \hfill\break
 \emph{Nama no }\emph{toriniku ni fureta manaita ya bōru nado no chōri kigu wa sono mama hoka no chōri ni shiyō suru koto wa yamemashō. }\hfill\break
Let's stop using cookware, such as cutting boards and bowls, that have come in contact with raw poultry when preparing other dishes. }

\par{\textbf{Grammar Note }: The ending \emph{-mashō }ましょう is used in this context to create a general polite suggestion for everyone to follow. It is the "let's" in the translation. }

\par{18. ${\overset{\textnormal{せかいじゅう}}{\text{世界中}}}$ の ${\overset{\textnormal{しんこうこく}}{\text{新興国}}}$ はそれぞれ ${\overset{\textnormal{こうど}}{\text{高度}}}$ の ${\overset{\textnormal{けいざいせいちょう}}{\text{経済成長}}}$ を ${\overset{\textnormal{めざ}}{\text{目指}}}$ しています。 \hfill\break
\emph{Sekaijū no }\emph{shinkōkoku wa sorezore kōdo no keizai seichō wo mezashite imasu. }\hfill\break
Developing countries worldwide are each aiming for rapid economic growth. }

\par{19. ${\overset{\textnormal{わたし}}{\text{私}}}$ たち ${\overset{\textnormal{かんとうじん}}{\text{関東人}}}$ は、 ${\overset{\textnormal{こうど}}{\text{高度}}}$ な ${\overset{\textnormal{しょくぶんか}}{\text{食文化}}}$ と ${\overset{\textnormal{した}}{\text{舌}}}$ を ${\overset{\textnormal{も}}{\text{持}}}$ っているんですよ。 \hfill\break
\emph{Watashitachi kantōjin wa, kōdo na shokubunka to shita wo motte iru n desu yo. }\hfill\break
We Kanto-ites possess a sophisticated cuisine and palate. }

\par{20. ${\overset{\textnormal{とくべつ}}{\text{特別}}}$ の ${\overset{\textnormal{きかん}}{\text{機関}}}$ を ${\overset{\textnormal{してい}}{\text{指定}}}$ する ${\overset{\textnormal{ひつよう}}{\text{必要}}}$ は ${\overset{\textnormal{みと}}{\text{認}}}$ めない。 \hfill\break
\emph{Tokubetsu no }\emph{kikan wo shitei suru hitsuyō wa mitomenai. \hfill\break
}(We) do not recognize the need to designate an Attached Organization. }

\par{\textbf{Meaning Note }: An “Attached Organization” is an organization in the Japanese government established for some specific purpose. }

\par{ 21. ${\overset{\textnormal{おお}}{\text{多}}}$ くのテクニックや ${\overset{\textnormal{とくべつ}}{\text{特別}}}$ な ${\overset{\textnormal{りょうり}}{\text{料理}}}$ を ${\overset{\textnormal{なら}}{\text{習}}}$ いました。 \hfill\break
\emph{Ōku no }\emph{tekunikku ya tokubetsu na ryōri wo naraimashita. \hfill\break
}I was taught on a lot of techniques and special cuisines. }

\begin{center}
 \textbf{\emph{No }not Allowed }
\end{center}

\par{ Conversely, not all \emph{na- }adjectival nouns can be used as \emph{no }-adjectival nouns instead. }

\par{22. ${\overset{\textnormal{かって}}{\text{勝手}}}$ \{な ○ ・ の X\} ${\overset{\textnormal{おも}}{\text{思}}}$ い ${\overset{\textnormal{こ}}{\text{込}}}$ みはNG! \hfill\break
\emph{Katte }\emph{[na }\emph{○  \slash no X] omoikomi wa NG! \hfill\break
}Arbitrary assumptions are no good! }

\par{23. ${\overset{\textnormal{せんぎょうしゅふ}}{\text{専業主婦}}}$ に ${\overset{\textnormal{たい}}{\text{対}}}$ する ${\overset{\textnormal{ゆうぐうそち}}{\text{優遇措置}}}$ が、 ${\overset{\textnormal{じょせい}}{\text{女性}}}$ の ${\overset{\textnormal{しゅうろういよく}}{\text{就労意欲}}}$ を ${\overset{\textnormal{そ}}{\text{削}}}$ ぐ ${\overset{\textnormal{ひにく}}{\text{皮肉}}}$ \{な ○ ・ の X\} ${\overset{\textnormal{けっか}}{\text{結果}}}$ を ${\overset{\textnormal{まね}}{\text{招}}}$ いている。 \hfill\break
 \emph{Sengyō shufu ni tai suru yūgū sochi ga, josei no shūrō iyoku wo sogu hiniku [na }\emph{○  \slash no X] kekka wo maneite iru. \hfill\break
}Preferential treatment toward housewives has brought about the ironic effect of weakening female desire to work. }

\begin{center}
 \textbf{\emph{Na }\textrightarrow  NG }
\end{center}

\par{ There are some \emph{no }-adjectives that can never have \emph{no }の be replaced by \emph{na }な. There are instances where switching them is grammatically incorrect. }

\par{24. それぞれ\{の ○・な X\} ${\overset{\textnormal{こべつ}}{\text{個別}}}$ \{の ○・な X\} ${\overset{\textnormal{しょうじょう}}{\text{症状}}}$ が ${\overset{\textnormal{で}}{\text{出}}}$ ることがあります。 \hfill\break
\emph{Sorezore [no }\emph{○\slash na X] kobetsu [no ○\slash na X] sh }\emph{ōj }\emph{ō ga deru koto ga arimasu. }\hfill\break
Individual symptoms of each occasionally occur. }

\par{\textbf{Grammar Note }: \emph{Koto ga aru }ことがある, when after the non-past tense of a verb, is used to show "occasional behavior." }

\par{25. ${\overset{\textnormal{おじ}}{\text{叔父}}}$ の ${\overset{\textnormal{いえ}}{\text{家}}}$ でさっき ${\overset{\textnormal{たいりょう}}{\text{大量}}}$ \{の ○・な X\}の ${\overset{\textnormal{あめ}}{\text{雨}}}$ が ${\overset{\textnormal{ふ}}{\text{降}}}$ っていました。 \hfill\break
\emph{Oji no ie de sakki tairy }\emph{ō [no ○\slash na X]  ame ga futte imashita. \hfill\break
}There was a massive amount of rain just now at my uncle\textquotesingle s house. }

\begin{center}
\textbf{Noun or Adjective? }
\end{center}

\par{ Sometimes, telling whether a \emph{no }の should be treated as a noun or an adjective is not easy. The best way to figure this out is by thinking about the phrase's English equivalent. For instance, in the phrase "hair color," both "hair" and "color" are recognized as nouns, but in this phrase, "hair" is an attribute to color. As such, treating it as an adjective-like noun phrase would be appropriate. }

\par{26. ${\overset{\textnormal{くろ}}{\text{黒}}}$ の ${\overset{\textnormal{ふく}}{\text{服}}}$ を ${\overset{\textnormal{き}}{\text{着}}}$ る。(Adjective?) \hfill\break
\emph{Kuro no }\emph{fuku wo kiru. }\hfill\break
To wear black clothes. }

\par{27. ${\overset{\textnormal{にほんじん}}{\text{日本人}}}$ の ${\overset{\textnormal{かみ}}{\text{髪}}}$ の ${\overset{\textnormal{いろ}}{\text{色}}}$ は ${\overset{\textnormal{くろ}}{\text{黒}}}$ ですね。 (Noun?) \hfill\break
\emph{Nihonjin no kami no iro wa kuro desu ne. \hfill\break
}Japanese hair color is black, isn\textquotesingle t it. \emph{}}

\par{28. ${\overset{\textnormal{あざ}}{\text{鮮}}}$ やかな ${\overset{\textnormal{みどりいろ}}{\text{緑色}}}$ の ${\overset{\textnormal{そうげん}}{\text{草原}}}$ が ${\overset{\textnormal{えんえん}}{\text{延々}}}$ と ${\overset{\textnormal{ひろ}}{\text{広}}}$ がっていた。 (Adjective?) \hfill\break
\emph{Azayaka na midori\textquotesingle iro no s }\emph{ōgen ga en\textquotesingle en to hirogatte ita. \hfill\break
}A vibrant, green grassland endlessly stretched out. }

\par{29. ${\overset{\textnormal{うらがわ}}{\text{裏側}}}$ は ${\overset{\textnormal{みどりいろ}}{\text{緑色}}}$ が ${\overset{\textnormal{うす}}{\text{薄}}}$ くなっている。 (Noun?) \hfill\break
\emph{Uragawa wa midori\textquotesingle iro ga usuku natte iru. \hfill\break
}On the back, the green is light. }

\par{30. ${\overset{\textnormal{からだ}}{\text{体}}}$ を ${\overset{\textnormal{けんこう}}{\text{健康}}}$ な ${\overset{\textnormal{じょうたい}}{\text{状態}}}$ に ${\overset{\textnormal{たも}}{\text{保}}}$ つ。 (Adjective?) \hfill\break
\emph{Karada wo kenk }\emph{ō na j }\emph{ōtai ni tamotsu. \hfill\break
}To keep one\textquotesingle s body healthy. }

\par{31. ${\overset{\textnormal{しんぞう}}{\text{心臓}}}$ の ${\overset{\textnormal{けんこう}}{\text{健康}}}$ の ${\overset{\textnormal{じょうたい}}{\text{状態}}}$ を ${\overset{\textnormal{かんり}}{\text{管理}}}$ する。 (Noun?) \hfill\break
\emph{Shinz }\emph{ō no kenk }\emph{ō no j }\emph{ōtai wo kanri suru. }\hfill\break
To manage the condition of one\textquotesingle s heart. }

\begin{center}
\textbf{Interchangeability of \emph{No }の and \emph{Na }な } 
\end{center}

\par{ Even when it may be standard to use either \emph{na }な or n \emph{o }の, variation will still exist between the two. The motivations for why \emph{na }な is used in place of \emph{no }の when either is possible, or when the latter is deemed standard form, are as follows: }

\par{1. The use of \emph{na }な strengthens the sense that one is qualifying the phrase that follows. \hfill\break
2. The use of \emph{na }な becomes more casual, especially when it is not the “standard” choice. \hfill\break
3. Generally speaking, \emph{na }な is typically softer and subjective in nature whereas \emph{no }の can sound colder, more objective, and stiff. \hfill\break
4. \emph{No }の may feel simply as a mere connector of phrases whereas \emph{na }な can embed some of the speaker\textquotesingle s feelings on top of functioning as a modifier. }

\par{ The difference between the two largely falls on the feeling of the individual. Whenever either can be used, a semantic space is opened up to meet the emotional needs of the context. If you are ever corrected on the use of \emph{na }な vs. \emph{no }の, the cadence and tone of what you are saying will likely be faulty rather than the choice itself being fundamentally wrong. }

\par{ In the following sentences, the adjectival nouns used can either use \emph{na }な or \emph{no }の, but the choice between the two is made based on the guidelines above. }

\par{32. ${\overset{\textnormal{ちきゅうじょう}}{\text{地球上}}}$ に ${\overset{\textnormal{いろいろ}}{\text{色々}}}$ の ${\overset{\textnormal{しゅるい}}{\text{種類}}}$ の ${\overset{\textnormal{ほにゅうるい}}{\text{哺乳類}}}$ が ${\overset{\textnormal{きゅうそく}}{\text{急速}}}$ に ${\overset{\textnormal{しんかはってん}}{\text{進化発展}}}$ を ${\overset{\textnormal{と}}{\text{遂}}}$ げました。 \hfill\break
\emph{Chiky }\emph{ūj }\emph{ō ni iroiro no shurui no hony }\emph{ūrui ga ky }\emph{ūsoku ni shinka hatten wo togemashita. \hfill\break
}Various species of mammals on Earth have undergone rapid evolutionary development. }

\par{33. この ${\overset{\textnormal{もり}}{\text{森}}}$ には ${\overset{\textnormal{いろいろ}}{\text{色々}}}$ な ${\overset{\textnormal{とり}}{\text{鳥}}}$ が ${\overset{\textnormal{す}}{\text{棲}}}$ んでいます。 \hfill\break
\emph{Kono mori ni wa iroiro na tori ga sunde imasu. }\hfill\break
There are various kinds of birds that live in this forest. }

\par{34. ${\overset{\textnormal{べつ}}{\text{別}}}$ な ${\overset{\textnormal{いけん}}{\text{意見}}}$ を ${\overset{\textnormal{い}}{\text{言}}}$ う ${\overset{\textnormal{やつ}}{\text{奴}}}$ がいても、 ${\overset{\textnormal{へいき}}{\text{平気}}}$ です。 \hfill\break
\emph{Betsu na iken wo iu yatsu ga ite mo, heiki desu. \hfill\break
}I\textquotesingle m fine even if there\textquotesingle s a guy with a different opinion. }

\par{35. ${\overset{\textnormal{べつ}}{\text{別}}}$ の ${\overset{\textnormal{たんとうしゃ}}{\text{担当者}}}$ が ${\overset{\textnormal{たいおう}}{\text{対応}}}$ しても ${\overset{\textnormal{だいじょうぶ}}{\text{大丈夫}}}$ です。 \hfill\break
\emph{Betsu no tant }\emph{ōsha ga tai }\emph{ō shite mo daij }\emph{ōbu desu. }\hfill\break
It\textquotesingle s okay if a different manager handles it. }

\par{36. ${\overset{\textnormal{たいとう}}{\text{対等}}}$ な ${\overset{\textnormal{にんげんかんけい}}{\text{人間関係}}}$ を ${\overset{\textnormal{こうちく}}{\text{構築}}}$ する。 \hfill\break
\emph{Tait }\emph{ō na ningen kankei wo k }\emph{ōchiku suru. }\hfill\break
To build equal human relations. }

\par{37. ${\overset{\textnormal{たいとう}}{\text{対等}}}$ の ${\overset{\textnormal{たちば}}{\text{立場}}}$ で ${\overset{\textnormal{ぼうえき}}{\text{貿易}}}$ を ${\overset{\textnormal{おこな}}{\text{行}}}$ う。 \hfill\break
\emph{Tait }\emph{ō no tachiba de b }\emph{ōeki wo okonau. }\hfill\break
To conduct trade on equal footing. }

\par{38. こういう ${\overset{\textnormal{あ}}{\text{当}}}$ たり ${\overset{\textnormal{まえ}}{\text{前}}}$ なことを ${\overset{\textnormal{い}}{\text{言}}}$ う ${\overset{\textnormal{ひと}}{\text{人}}}$ は ${\overset{\textnormal{かなら}}{\text{必}}}$ ずいるよね。 \hfill\break
\emph{K }\emph{ō }\emph{iu atarimae na koto wo iu hito wa kanarazu iru yo ne. }\hfill\break
There\textquotesingle s always a person who says obvious things like this, huh. }

\par{39. ${\overset{\textnormal{わたし}}{\text{私}}}$ たち ${\overset{\textnormal{にんげん}}{\text{人間}}}$ は ${\overset{\textnormal{たいよう}}{\text{太陽}}}$ の ${\overset{\textnormal{おんけい}}{\text{恩恵}}}$ を ${\overset{\textnormal{あ}}{\text{当}}}$ たり ${\overset{\textnormal{まえ}}{\text{前}}}$ のことだと ${\overset{\textnormal{おも}}{\text{思}}}$ っています。 \hfill\break
\emph{Watashitachi ningen wa taiyō no onkei wo atarimae no koto da to omotte imasu. }\hfill\break
We humans think of the benefits of the Sun as something of the ordinary. }

\par{40. ${\overset{\textnormal{きわ}}{\text{極}}}$ めて ${\overset{\textnormal{とうぜん}}{\text{当然}}}$ な ${\overset{\textnormal{はんだん}}{\text{判断}}}$ だと ${\overset{\textnormal{おも}}{\text{思}}}$ います。 \hfill\break
\emph{Kiwamete tōzen na handan da to omoimasu. \hfill\break
}I think it is an extremely obvious judgment. }

\par{41. ${\overset{\textnormal{とうぜん}}{\text{当然}}}$ の ${\overset{\textnormal{けっか}}{\text{結果}}}$ が ${\overset{\textnormal{で}}{\text{出}}}$ ました。 \hfill\break
\emph{T }\emph{ōzen no kekka ga demashita. }\hfill\break
The obvious results were made. }

\par{42. ${\overset{\textnormal{てきど}}{\text{適度}}}$ のお ${\overset{\textnormal{さけ}}{\text{酒}}}$ は、「 ${\overset{\textnormal{ひゃくやく}}{\text{百薬}}}$ の ${\overset{\textnormal{ちょう}}{\text{長}}}$ 」なのです。 \hfill\break
\emph{Tekido no osake wa,"hyakuyaku no ch }\emph{ō” na no desu. }\hfill\break
That\textquotesingle s because a moderate amount of liquor is the chief of all medicines. }

\par{43. ${\overset{\textnormal{てきど}}{\text{適度}}}$ な ${\overset{\textnormal{うんどう}}{\text{運動}}}$ は ${\overset{\textnormal{からだ}}{\text{体}}}$ によいです。 \hfill\break
\emph{Tekido na und }\emph{ō wa karada ni yoi desu. \hfill\break
}Moderate exercise is good for the body. }

\begin{center}
 \textbf{Set Phrases }
\end{center}

\par{ Sometimes, variation is typically restricted to set phrases, with one being restricted and the other one being the overwhelming used option. }

\par{44. 不思議の国へようこそ。 \hfill\break
\emph{Fushigi no kuni e y }\emph{ōkoso. }\hfill\break
Welcome to Wonderland. }

\par{45. 不思議\{な 〇・の X\}現象が起きた。 \hfill\break
\emph{Fushigi [na }\emph{〇\slash no X] gensh }\emph{ō ga okita. } \hfill\break
A mysterious phenomenon has occurred. }

\begin{center}
\textbf{\emph{Na }vs. \emph{No }: Dependent Clause }
\end{center}

\par{ While there is great variation as to whether one should use \emph{no }の or \emph{na }な, the decision as to which should be used shifts almost entirely to \emph{na }な whenever it functions as the end of a dependent clause modifying a noun.  For example, in Ex. 46 \emph{kareshi }彼氏 is being modified by [ \emph{aish }\emph{ō ga saiaku da }相性が最悪だ], which makes \emph{saiaku }最悪 seem more like a noun that happens to be next to the copula in the form of \emph{na }な. }

\par{46. 今週、相性が最悪な彼氏と別れました。 \hfill\break
\emph{Konshu, aish }\emph{ō ga saiaku na kareshi to wakaremashita. } \hfill\break
This week, I broke up with my boyfriend who I had the worst compatibility with. }

\begin{center}
 \textbf{Adjective + \emph{No }の }
\end{center}

\par{ There is a small handful of adjectives with which \emph{no }の attaches itself in peculiar ways. One such example is \emph{nakayoshi }仲良し, the original form of \emph{naka ga ii }仲がいい. This can in fact be used as a \emph{no }-adjective as seen below. }

\par{47. ${\overset{\textnormal{きょう}}{\text{今日}}}$ も ${\overset{\textnormal{ふた}}{\text{2}}}$ ${\overset{\textnormal{り}}{\text{人}}}$ は ${\overset{\textnormal{なかよ}}{\text{仲良}}}$ しだった。 \hfill\break
\emph{Ky }\emph{ō mo futari wa nakayoshi datta. } \hfill\break
The two got along well today as well. }

\par{48. ${\overset{\textnormal{なかよ}}{\text{仲良}}}$ しの ${\overset{\textnormal{ともだち}}{\text{友達}}}$ がいない。 \hfill\break
\emph{Nakayoshi no tomodachi ga inai. }\hfill\break
I don\textquotesingle t have any close friends. }

\par{ Lastly, though they are few in number, there are examples of \emph{no }の attaching to the stem of adjectives in literary language. They are always replaceable with something else, but they\textquotesingle re interesting to know about. }

\par{49a. ${\overset{\textnormal{なが}}{\text{永}}}$ の ${\overset{\textnormal{わか}}{\text{別}}}$ れとなった。(Old-fashioned\slash literary) \hfill\break
\emph{Naga no wakare to natta. }\hfill\break
49b. ${\overset{\textnormal{えいえん}}{\text{永遠}}}$ の ${\overset{\textnormal{わか}}{\text{別}}}$ れとなった。 \hfill\break
\emph{Eien no wakare to natta. }\hfill\break
It became an eternal separation. }

\par{ 50. ${\overset{\textnormal{うるわ}}{\text{麗}}}$ しの ${\overset{\textnormal{めがみ}}{\text{女神}}}$ \hfill\break
\emph{Uruwashi no megami }\hfill\break
An outstandingly beautiful goddess }
    