    
\chapter{Interrogatives II}

\begin{center}
\begin{Large}
第87課: Interrogatives II: The Question "How" 
\end{Large}
\end{center}
 
\par{ Following our coverage of the non-question how with ~かた, we'll now cover the many phrases for how to make a question "how" in Japanese. }
      
\section{どう \& Related Words}
 
\par{ どう and どのように are used to mean “how” as in regards to action. However, the two words are not the same when it comes to politeness. どう is typically more casual, and when it is used in polite speech, it still has a somewhat casual feel to it. どのように, on the other hand, is more formal and is expected in more polite contexts as well as in written documents. }

\par{1. この ${\overset{\textnormal{じょうきょう}}{\text{状況}}}$ をどのように ${\overset{\textnormal{くつがえ}}{\text{覆}}}$ すことができるのでしょうか。 \hfill\break
How can be we overturn this condition? }

\par{2. どのように新しい世界を ${\overset{\textnormal{かいたく}}{\text{開拓}}}$ されてきたのか、お聞きしました。 (Humble speech) \hfill\break
I asked how a new world has been pioneered. }

\par{3. その印象はどのように感じますか。 \hfill\break
How do you feel about the impression? }

\par{ どう may also simply describe an uncertain state. A politer form would be いかが. For instance, ~はどうですか meaning “how is…?” is more politely as ~はいかがですか. To further make it more polite, you can replace ~ですか with ~でしょうか as です is often too abrupt for truly polite\slash honorific utterances. Of course, when the situation does not call for such measures, ~でしょうか would be taken the opposite way. }

\par{4. 「仕事はどうですか」「私は ${\overset{\textnormal{じえいぎょう}}{\text{自営業}}}$ です」 \hfill\break
How is your work?” I am self-employed” }

\par{5. お天気はどうですか。 \hfill\break
How is the weather? }

\par{6. 日本をどう思いますか。 \hfill\break
What do you think about Japan? }

\par{7. ${\overset{\textnormal{かんとんご}}{\text{広東語}}}$ のクラスはどうですか。 \hfill\break
How is your Cantonese class? }

\par{8. 皆様いかがお過ごしですか。(Honorific) \hfill\break
How is everyone fairing? }

\par{ \textbf{どう }\textbf{か }}

\par{ど うか, the combination of どう and the particle  か, either means どうぞ (which is ultimately also a combination of どう with the particle ぞ) in the sense of pleading or なんとか. Do not confuse this with かどうか. In かどうか, the か is used to mark the end of a question clause inside a larger sentence. }

\par{9. そのことはどうかしなければなりません。 \hfill\break
We have to do something about it. }

\par{10. この頃彼女はどうかしてるね。 \hfill\break
Something\textquotesingle s been wrong with her lately. }

\par{11. どうか許してください。 \hfill\break
Please forgive me. }

\par{12. 頭がどうかなってしまいそうだ。 \hfill\break
My head feels like it\textquotesingle s going crazy. }

\par{\textbf{どんな風に・どういう風に }}

\par{ どんな ${\overset{\textnormal{ふう}}{\text{風}}}$ に and どういう ${\overset{\textnormal{ふう}}{\text{風}}}$ に are similar words meaning “how”. Both are used to describe how things are done in regards to manner\slash style. The first is more casual while the latter is more polite\slash formal. 風 is read as ふう when used to mean method; manner; way; style; appearance etc. }

\par{13. いまびはどういう風に作られているんですか。 \hfill\break
How has IMABI been being made? }

\par{14. どんな風に生きれば幸せになれるんだろう。 \hfill\break
I wonder how one could live happily. }

\par{15. どんな風に天(国)に導かれるのでしょう? \hfill\break
How is one led to heaven? }

\par{16. 日本語は外国人にどんな風に聞こえているのでしょうか。 \hfill\break
How does Japanese sound to foreigners? }

\par{ \textbf{どうやっ }\textbf{て }}

\par{ どうやって is another word for “how” often used in asking how something is done. This makes it especially useful in asking for directions and how things work. Even so, どう and どのように may also be used in the same way. }

\par{17. ここからどうやって駅に行けばいいのか全く分からないんですよ。 \hfill\break
I have no idea how to get to the train station from here. }

\par{18. どうやって ${\overset{\textnormal{おうぼ}}{\text{応募}}}$ したらいいですか? \hfill\break
How should I apply? }

\par{19. どうやって英語を勉強すればいいの? (Casual) \hfill\break
How should I study English? }

\par{20. ${\overset{\textnormal{しんらい}}{\text{信頼}}}$ はどのように作りますか。 \hfill\break
How does one build trust? }
      
\section{How Much}
 
\par{ “How” as in “how much” in degree is expressed with either どれくらい, どれぐらい, どのくらい, or どのぐらい. There is almost no difference in meaning or nuance. However, the variants are listed from most to least common. This is not to insinuate any one of them is not used a lot because they all are. }

\par{21. 日本にどれくらい滞在するつもりですか。 \hfill\break
How long do you plan to reside in Japan? }

\par{22. どれくらいの ${\overset{\textnormal{ひんど}}{\text{頻度}}}$ で運動したら ${\overset{\textnormal{こうか}}{\text{効果}}}$ を期待できるのでしょうか。 \hfill\break
How often should one exercise and expect to see effects? }

\par{23. 宇宙はどのくらい広いの? \hfill\break
How wide is the universe? }

\par{24. 理想の彼氏の ${\overset{\textnormal{しんちょう}}{\text{身長}}}$ ってどれくらい?  (Casual) \hfill\break
How tall is your ideal boyfriend? }

\par{25. どれくらいの頻度でトイレに行くの? (Casual) \hfill\break
How often do you go to the bathroom? }

\par{26. 愛ちゃんの耳がどのくらい赤くなっているのかが気がかりだ。 \hfill\break
How red Ai-chan's ears have gotten is a big worry. }

\begin{center}
\textbf{いくら }
\end{center}

\par{ “How much” as in money or quantity is expressed with いくら.  It is also seen in the phrases いくらか, いくらも, and いくらでも. }

\begin{itemize}

\item \textbf{いくらか }: Equivalent to 幾分か and 少し. It may also be used in the same sense as 多少 when used as an adverb. Whereas いくらか can be used for degree and quantity, いくぶんか must only be used for quantity. 
\item \textbf{いくらも }: Shows a very large degree. With the negative, it is equivalent to ほとんど. 
\item \textbf{いくらでも }: This may be equivalent to どれほどでも and can be used to exaggerate the amount of something. Whereas いくらでも works for quantity and price, どれほどでも only works for degree. With ない, it\textquotesingle s just like いくらも showing that the degree\slash amount is not as large as one thought. With いい, it means “any amount is fine”. 
\end{itemize}
\textbf{Spelling Note }: いくら may be spelled as 幾ら. 
\par{27. 出来れば、アメリカ人に、いくら払うとかどんな場所でチップを期待されているのかについて、聞いたほうがい いですよ。 \hfill\break
If you can, it would be best to ask an American on how much and in what places it is expected of you to tip. }

\par{28. ${\overset{\textnormal{たいじゅう}}{\text{体重}}}$ はいくらですか。 \hfill\break
What is your weight? }

\par{29. ${\overset{\textnormal{こおり}}{\text{氷}}}$ の ${\overset{\textnormal{おんど}}{\text{温度}}}$ はいくらですか。 \hfill\break
How hot is the ice? }

\par{31. お金がいくらでも使える。 \hfill\break
I can use as much money as I want. }

\par{32a. 立派な品をどれほどでも買おう! \hfill\break
32b. 立派な品をいくらでも買おう! \hfill\break
I'll buy splendid items no matter how much (there are). }

\par{\textbf{Sentence Note }: 32a can only refer to quantity of what is being bought. 32b would usually be interpreted as referring to price, but it is ambiguous enough to refer to quantity as well. }

\par{33. インドの人口はいくらでしょうか。 \hfill\break
What is the population of India? }

\par{34. 値段はいくらでもいいから、売ってくれ。(Vulgar) \hfill\break
The price is no problem, so sell it (to me)! }

\par{35. ${\overset{\textnormal{ていじ}}{\text{提示}}}$ する ${\overset{\textnormal{がく}}{\text{額}}}$ はいくらですか? \hfill\break
How much is your offering price? }

\par{36. ${\overset{\textnormal{ざいこ}}{\text{在庫}}}$ はいくらも ${\overset{\textnormal{のこ}}{\text{残}}}$ っていない。 \hfill\break
Almost nothing remains in the stockpile. }

\par{37. 「 ${\overset{\textnormal{きょう}}{\text{今日}}}$ ${\overset{\textnormal{いち}}{\text{一}}}$ ドルは ${\overset{\textnormal{なんえん}}{\text{何円}}}$ ですか」「 ${\overset{\textnormal{かわせ}}{\text{為替}}}$ レートはいくらか分かりません」 \hfill\break
"How many yen is one dollar today?" "I don't know how much the exchange rate is". }

\par{38. 「日本からアメリカまでいくらぐらいかかりますか」「 ${\overset{\textnormal{りょこうひ}}{\text{旅行費}}}$ はいくらか分かりません」 \hfill\break
“About how much is it from Japan to America? ” “I don't know how much the travel costs are”. }

\begin{center}
 \textbf{何ぼ }
\end{center}

\par{ 何ぼ can be used for not only price but also quantity in general. This is used in 関西弁 and other parts of Japan. }

\par{39. これ、ちょっと何ぼ? \hfill\break
How much is this? }

\par{40. このリンゴ何ぼ? \hfill\break
How much is this apple\slash are these apples? }

\begin{center}
\textbf{いくつ }
\end{center}

\par{ “How many” as in age or counting things is いくつ. However, this is a generic word. So, for counter phrases, you must use the pattern 何+counter. 幾‐ is sometimes seen instead of 何‐ in native phrases in literary contexts. }

\par{ いくつか means “a few”. It may be used as a noun or adverb. Regardless of its syntactic role, it is always in reference to 数. いくつも is on either end of the spectrum, being able to express large and hardly any amount depending on whether the sentence is positive or negative. }

\par{41. おいくつですか。 \hfill\break
How old are you? }

\par{42. ${\overset{\textnormal{にじ}}{\text{虹}}}$ には ${\overset{\textnormal{いろ}}{\text{色}}}$ がいくつありますか。 \hfill\break
How many colors does a rainbow have? }

\par{43. あのクラスには何人いますか。 \hfill\break
How many people do you have in this class? }

\par{44. ${\overset{\textnormal{きょうつうてん}}{\text{共通点}}}$ がいくつかある。 \hfill\break
To have some points in common. }

\par{45. 選択肢はいくつも残っていない。 \hfill\break
There's absolutely no option to be had. }

\par{46. 指サイズはいくつですか? \hfill\break
What's your finger size? \hfill\break
Literally: How much is your finger size? \hfill\break
\hfill\break
\textbf{Word Note }: いくら, although it means "how much", is not used here because "finger size" is not deemed as something of quantity. If you akin finger size to age, the logic behind this may be clearer. }
    