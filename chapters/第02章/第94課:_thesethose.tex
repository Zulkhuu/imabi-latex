    
\chapter{Plural Kosoado こそあど}

\begin{center}
\begin{Large}
第94課: Plural Kosoado こそあど: Korera これら, Sorera それら, \& Arera あれら 
\end{Large}
\end{center}
 
\par{ In Japanese, pluralization is not grammatically imposed in the way it is in English. It is not that Japanese people do not conceptualize the difference between singular and plural entities. Rather, this distinction rarely needs to be verbalized because the language provides context clues to make such a distinction unnecessary. Of course, this means individual sentences may be ambiguous in this regard if put in isolation, but this would not be reflective of actual conversation. }

\par{ Nevertheless, there are in fact plural phrases in Japanese. Each means of creating such phrases brings about nuance distinctions unique to each construct. In English, we make the following distinctions in demonstrative words for singular and plural forms. }

\begin{ltabulary}{|P|P|}
\hline 

Singular Form & Plural Form \\ \cline{1-2}

This & These \\ \cline{1-2}

That & Those \\ \cline{1-2}

It & They \\ \cline{1-2}

\end{ltabulary}

\par{ In this lesson, we\textquotesingle ll learn how Japanese usually doesn\textquotesingle t explicitly make these distinctions, and when it does how it differs from English. }
      
\section{Vocabulary List}
 \textbf{Nouns } 
\par{・リンゴ  \emph{Ringo }– Apple }
 
\par{・値段  \emph{Nedan }– Price\slash cost }
 
\par{・家族  \emph{Kazoku }– Family }
 
\par{・写真  \emph{Shashin }– Picture }
 
\par{・オレンジ  \emph{Orenji }– Orange }
 
\par{・アプリケーション  \emph{Apurik }\emph{ēshon }– Application }
 
\par{・画面  \emph{Gamen }– Screen }
 
\par{・場合  \emph{Ba\textquotesingle ai }– Case\slash situation\slash event }
 
\par{・国々  \emph{Kuniguni }– Countries }
 
\par{・大麻  \emph{Taima }– Hemp\slash cannabis }
 
\par{・テロ  \emph{Tero }– Terrorism }
 
\par{・対策  \emph{Taisaku }– Measure\slash provision }
 
\par{・強化  \emph{Ky }\emph{ōka }– Enforcement }
 
\par{・一環  \emph{Ikkan }– Link }
 
\par{・対象  \emph{Taish }\emph{ō }– Target\slash object }
 
\par{・根拠  \emph{Konkyo }– Evidence }
 
\par{・情報  \emph{J }\emph{ōhō }– Information\slash news }
 
\par{・ミサイル  \emph{Misairu }– Missile }
 
\par{・発射  \emph{Hassha }– Firing\slash discharge }
 
\par{・領土  \emph{Ry }\emph{ōdo }– Territory }
 
\par{・領海  \emph{Ry }\emph{ōkai }– Territorial waters }
 
\par{・可能性  \emph{Kan }\emph{ōsei }– Possibility }
 
\par{・避難  \emph{Hinan }– Evacuation }
 
\par{・呼びかけ  \emph{Yobikake }– Call\slash appeal }
 
\par{・名前  \emph{Namae }– Name }
 
\par{・過去  \emph{Kako }– Past }
 
\par{・履歴  \emph{Rireki }– Background\slash personal history }
 
\par{・決まり  \emph{Kimari }– Rule\slash agreement\slash custom }
 
\par{・疑問  \emph{Gimon }– Question\slash doubt }
 
\par{・改正  \emph{Kaisei }- Revision }
 
\par{・提案  \emph{Teian }- Proposal }
 
\par{・町  \emph{Machi }– Town }
 
\par{・海  \emph{Umi }– Sea\slash ocean }
 
\par{・空  \emph{Sora }– Sky }
 
\par{・満月  \emph{Mangetsu }– Full moon }
 
\par{・季節  \emph{Kisetsu }– Season(s) }
 
\par{・電車  \emph{Densha }– (Electric) train }
 
\par{・公園  \emph{K }\emph{ōen }– Park }
 
\par{・遊園地  \emph{Y }\emph{ūenchi }– Amusement park }
 
\par{・動物  \emph{D }\emph{ōbutsu }– Animal }
 
\par{・スーパー(マーケット)  \emph{S }\emph{ūp }\emph{ā(m }\emph{āketto) }– Supermarket }
 
\par{・おもちゃ  \emph{Omocha }– Toy }
 
\par{・子  \emph{Ko }– Child }
 
\par{・将来  \emph{Sh }\emph{ōrai }– Future }
 
\par{・こと  \emph{Koto }– Thing\slash event\slash circumstance\slash affair }
 
\par{・不安  \emph{Fuan }– Anxiety }
 
\par{・頭  \emph{Atama }– Head\slash mind }
 
\par{・恋愛  \emph{Ren\textquotesingle ai }– Love\slash affections }
 
\par{・余裕  \emph{Yoy }\emph{ū }– Leeway\slash margin }
 
\par{・机  \emph{Tsukue }– Desk }
 
\par{・鉛筆  \emph{Empitsu }– Pencil }
 
\par{・中  \emph{Naka }– In(side)\slash middle }
 
\par{・内  \emph{Uchi }– Inside\slash within\slash among }
 
\par{・サッカー  \emph{Sakk }\emph{ā }– Soccer }
 
\par{・選手権  \emph{Senshuken }– Championship }
 
\par{・人  \emph{Hito }– Person }
 
\par{・トランク  \emph{Toranku }– Trunk\slash suitcase }
 
\par{・鞄  \emph{Kaban }– Bag\slash briefcase\slash basket }
 
\par{・山  \emph{Yama }– Mountain }
 
\par{・子猫・仔猫  \emph{Koneko }– Kitten }
 
\par{・本  \emph{Hon }– Book }
 
\par{\textbf{Proper Nouns }}
 
\par{・米国  \emph{Beikoku }– America (country only) }
 
\par{・アメリカ \emph{Amerika }– America }
 
\par{・フランス  \emph{Furansu }- France }
 
\par{・ドイツ  \emph{Doitsu }- Germany }
 
\par{・トランプ大統領  \emph{Torampu-dait }\emph{ōry }\emph{ō }– President Trump }
 
\par{・日本  \emph{Nihon\slash Nippon }– Japan }
 
\par{\textbf{Adjectives }}
 
\par{・青い  \emph{Aoi }– Blue\slash green\slash pale }
 
\par{・高い  \emph{Takai }– Tall\slash expensive }
 
\par{・美味しい  \emph{Oishii }- Delicious }
 
\par{・多い  \emph{Ōi }– Many\slash a lot }
 
\par{・無い  \emph{Nai }– Nonexistent }
 \textbf{Demonstratives } 
\par{・これ  \emph{Kore }– This (noun) }
 
\par{・この  \emph{Kono }– This (adj.) }
 
\par{・これら  \emph{Korera }– These }
 
\par{・それ  \emph{Sore }– That (noun) }
 
\par{・その  \emph{Sono }– That (adj.) }
 
\par{・それら  \emph{Sorera }– Those }
 
\par{・あれ  \emph{Are }– Those (over there) }
 
\par{・あの  \emph{Ano }– Those (over there) }
 
\par{・どれも  \emph{Dore mo }– Either\slash none }
 
\par{・それだけ  \emph{Sore dake }– Just that }
 
\par{・あちら  \emph{Achira }– That way\slash that one\slash that person }
 
\par{\textbf{Adverbs }}
 
\par{・(お)いくつ  \emph{(O-)ikutsu }– How many? }
 
\par{・いくら  \emph{Ikura }– How many? }
 
\par{・全て  \emph{Subete }– All }
 
\par{・初めに  \emph{Hajime ni }– Firstly\slash to begin with }
 
\par{・一杯  \emph{Ippai }Full of }
 
\par{・やがて  \emph{Yagate }– At last }
 
\par{・更に  \emph{Sara ni }– Furthermore\slash moreover }
 
\par{\textbf{Number Phrases }}
 
\par{・二、三個  \emph{Ni, sanko }– Two, three things }
 
\par{・ 7 か国  \emph{Nanakakoku }– Seven countries }
 
\par{・一つ  \emph{Hitotsu }– One thing }
 
\par{・一位  \emph{Ichi\textquotesingle i }– First place }
 
\par{・ 5 人  \emph{Gonin }– Five people }
 
\par{・三匹  \emph{Sambiki }– Three creatures }
 
\par{・五つ  \emph{Itsutsu }– Five things }
 
\par{\textbf{Interrogatives }}
 
\par{・何故  \emph{Naze }– Why? }
 
\par{・誰  \emph{Dare }– Who? }
 
\par{\textbf{Prefixes }}
 
\par{お~  \emph{O- }– Honorific prefix }
 
\par{ご~  \emph{Go- }– Honorific prefix }
 
\par{\textbf{Suffixes }}
 
\par{・~産  \emph{-san }– Product of }
 
\par{・~屋  \emph{-ya }– Shop\slash someone who sells\slash works as… }
 
\par{\textbf{\emph{Ichidan } \emph{(ru) }Verbs }}
 
\par{・見る  \emph{Miru }– To see\slash look }
 
\par{・閉じる  \emph{Tojiru }– To close }
 
\par{・伝える \emph{Tsutaeru }– To convey\slash report\slash transmit\slash communicate }
 
\par{・流れる  \emph{Nagareru }– To stream\slash flow\slash be carried }
 
\par{・考える  \emph{Kangaeru }– To think about\slash ponder }
 
\par{・詰める  \emph{Tsumeru }– To stuff into\slash cram\slash pack }
 
\par{\textbf{\emph{Godan (u) }Verbs }}
 
\par{・開く  \emph{Hiraku }– To open }
 
\par{・成る  \emph{Naru }– To become }
 
\par{・示す  \emph{Shimesu }– To demonstrate\slash indicate }
 
\par{・手放す  \emph{Tebanasu }– To let go }
 
\par{・ある  \emph{Aru }– To be\slash have (inanimate objects) }
 
\par{・奪う  \emph{Ubau }– To steal }
 
\par{・産む  \emph{Umu }– To give birth }
 
\par{・飼う  \emph{Kau }– To raise (an animal) }
 
\par{・持っていく  \emph{Motte iku }– To take with }
 
\par{\textbf{\emph{Suru }Verbs }}
 
\par{・強調する  \emph{Ky }\emph{ōch }\emph{ō suru }– To emphasize }
 
\par{・落下する  \emph{Rakka suru }– To fall\slash drop }
 
\par{・判断する  \emph{Handan suru }– To judge\slash conclude }
 
\par{・申請する  \emph{Shinsei suru }– To apply\slash request }
 
\par{\textbf{\emph{Kuru }Verb }}
 
\par{・来る \emph{Kuru }– To come }
 
\par{\textbf{Set Phrases }}
 
\par{・ご入り用  \emph{Go-iriy }\emph{ō }– Need }
 
\par{・~下さい  \emph{Kudasai }– Please }

\par{ ・~について  \emph{Ni tsui }\emph{te }– About }

\par{\textbf{Adjectival Nouns }\emph{\hfill\break
}}

\par{・他\{の\} \emph{Ta\slash hoka [no] }– Other }

\par{・違法\{な\} \emph{Ih }\emph{ō [na] }– Illegal\slash invalid }

\par{・具体的\{な\}  \emph{Gutaiteki [na] }– Concrete\slash tangible }

\par{・緊急\{の・な\}  \emph{Kinky }\emph{ū [no\slash na] }– Urgent\slash emergency }

\par{・屋内\{の\} \emph{Okunai [no] }– Indoor }

\par{\textbf{Adnominal Adjectives }}

\par{・大きな  \emph{Ōkina }– Large\slash great \textbf{\hfill\break
}}
\textbf{}
\par{\textbf{Pronouns }}

\par{・私  \emph{Wata(ku)shi }– I }

\par{・彼女  \emph{Kanojo }– She }
      
\section{Plural Kosoado こそあど}
 
\par{ As English speakers, the forms above are so ingrained into our minds that it is often difficult coping with the seemingly drastically simplified Japanese template. In Japanese, the singular and plural usages of the equivalents of these words are not distinguishable. In other words, they look the same. }

\par{1. 「このリンゴを ${\overset{\textnormal{くだ}}{\text{下}}}$ さい。」「おいくつご ${\overset{\textnormal{い}}{\text{入}}}$ り ${\overset{\textnormal{よう}}{\text{用}}}$ でしょうか。」 \hfill\break
\emph{“Kono ringo wo kudasai.” “O-ikutsu go-iriyō deshō ka?” \hfill\break
}“Please give me these apples.” “How many is it that you need?” }

\par{\textbf{Orthography Note }: \emph{Ringo }may also be seldom spelled as 林檎. }

\par{ In this exchange, the customer (Speaker A) is physically next to what\textquotesingle s presumably an apple stand at a market and is directing attention somehow to a certain kind of apple. If Speaker A were pointing specifically at a certain apple in the bunch, Speaker B (the store clerk) may have potentially asked something else, but just as English speakers use phrases like “I want this one,” Speaker A would have somehow indicated clearly through his\slash her actions that the one apple was all he\slash she wanted. With that being the case, if what all Speaker B perceives is that Speaker A wants Aomori Prefecture grown apples, then the next logical question is “how many?” In other words, \emph{kono }この simply replaces the description that would otherwise go with “apple” to merely indicate its position to Speaker A. }

\par{ In other sentences, we can clearly see how \emph{kono }この can and very frequently stand for “these.” }

\par{2. このリンゴは ${\overset{\textnormal{あお}}{\text{青}}}$ くて ${\overset{\textnormal{ねだん}}{\text{値段}}}$ が ${\overset{\textnormal{たか}}{\text{高}}}$ い。 \hfill\break
 \emph{Kono ringo wa aokute nedan ga takai. }\hfill\break
These apples are green and pricey. }

\par{\textbf{Sentence Note }: Of course, this sentence could be interpreted in the singular form. Context is everything. If the speaker is holding an apple that he bought from the store and is merely describing it to someone, there\textquotesingle s no reason why the listener would think he\textquotesingle s talking about more than one apple that happen to be present. However, it is logical to think that the statement can be true for apples of that kind. This would make sense especially if the sentence were preceded by something like “This apple from America is delicious.” Saying that would make it sound like he\textquotesingle s now making a general statement about such apples. Now, say the speaker is with someone at a supermarket and happen to go to the produce section and see apples that are both really green and really pricey. In which case, this statement would need to be translated in the plural sense. }

\par{3. このリンゴはどれもおいしそう。 \hfill\break
 \emph{Kono ringo wa dore mo oishisō. }\hfill\break
Each of these apples looks delicious. }

\par{4. これ、いくらですか。 \hfill\break
 \emph{Kore, ikura desu ka? \hfill\break
 }How much [is this\slash are these]? }

\par{ In general conversation, \emph{kore\slash kono }これ・この, \emph{sore\slash sono }それ・その, and \emph{are\slash ano }あれ・あの are used in both the singular and plural sense. These words, as you know, are used in various set phrases with almost the same capacities as their English equivalents and then some. The reason why this is possible is because these words refer collectively to things. If only one thing is part of the collective, then we interpret that in the singular sense. If the collective has more than one thing in it, we just refer to the collective as a whole and interpret that in the plural sense. }

\par{5. これは ${\overset{\textnormal{わたし}}{\text{私}}}$ の ${\overset{\textnormal{かぞくしゃしん}}{\text{家族写真}}}$ です。 \hfill\break
 \emph{Kore wa watashi no kazoku shashin desu. \hfill\break
 }[This is a\slash these are] picture(s) of my family. }

\par{6. それだけじゃない。 \hfill\break
 \emph{Sore dake ja nai. \hfill\break
 }That\textquotesingle s not all. }

\begin{center}
 \textbf{Using ー \emph{ra }ら }\hfill\break

\end{center}

\par{ It just so happens that \emph{korera }これら, \emph{sorera }それら, and \emph{arera }あれら exist for the plural sense. However, they refer to things at the individual level. This means they can be translated as “(each of) these,” “(each of) those,” “(each of) those (over there).” These words are largely used in the written language unless ignoring the individuality of what “these” and “those” refer to would be illogical. It\textquotesingle s important to note that \emph{arera }あれら is actually not used at all aside from Japanese directly translated from English. }

\par{ To unpack all this, we will now look at several example sentences utilizing these words so that you can see exactly when they\textquotesingle re used. }

\par{7.(2、3個のオレンジを見て)「これらのオレンジは米国産です。」 \hfill\break
 \emph{(Ni, sanko no orenji wo mite) “Korera no orenji wa beikoku-san desu.” \hfill\break
 }(Looking at 2, 3 oranges) “These oranges are America imported.” }

\par{\textbf{Sentence Note }: Using \emph{korera }これら instead of \emph{kono }この allows the speaker to emphasize how each of the oranges at the individual level are America imported. }

\par{8. ほかのアプリケーション画面を開いている場合、それらをすべて閉じてください。 \hfill\break
 \emph{Hoka no apurikēshon gamen wo hiraite iru ba\textquotesingle ai, sorera wo subete tojite kudasai. \hfill\break
 }In the case you have other application screens open, please close all of them. }

\par{9. アメリカ、フランス、ドイツ、これらの国々では大麻は違法ですか。 \hfill\break
 \emph{Amerika, Furansu, Doitsu, korera no kuniguni de wa taima wa ihō desu ka? \hfill\break
 }In America, France, Germany, and all these countries, is marijuana illegal? }

\par{10. トランプ大統領は、テロ対策強化の一環であることを強調していますが、なぜこれらの7か国が対象となったのか、具体的な根拠は示していません。 \hfill\break
 \emph{Torampu-daitōryō wa, tero taisaku kyōka no ikkan de aru koto wo kyōchō shite imasu ga, naze korera no nanakakoku ga taishō to natta no ka, gutaiteki na konkyo wa shimeshite imasen. }\hfill\break
President Trump emphasizes that it is linked to strengthening measures against terrorism, but he hasn\textquotesingle t provided concrete evidence as to why these seven countries became targeted. }

\par{11. これらの緊急情報は、はじめに「ミサイル発射情報」が伝えられ、ミサイルが日本の領土・領海に落下する可能性があると判断された場合、「屋内避難の呼びかけ」という情報が流れることになっています。 \hfill\break
 \emph{Korera no kinkyū jōhō wa, hajime ni “misairu hassha jōhō” ga tsutaerare, misairu ga Nihon no ryōdo\slash ryōkai ni rakka suru kanōsei ga aru to handan sareta ba\textquotesingle ai, “okunai hinan no yobikake” to iu jōhō ga nagareru koto ni natte imasu. }\hfill\break
As for these (means of) emergency information, at first “missile launch information” will be transmitted, and in the event that it\textquotesingle s concluded there exists the possibility a missile could come down on Japanese territory\slash waters, information calling for “indoor evacuation” will play. }

\par{12. 名前も過去も履歴も、それらすべてを手放した。 \hfill\break
 \emph{Namae mo kako mo rireki mo, sorera subete wo tebanashita. \hfill\break
 }My name, my past, my background, I let go of all those things. }

\par{\textbf{Phrase Note }: \emph{Korera }これら and \emph{sorera }それら are frequently followed by \emph{subete }すべて. In this situation, you can\textquotesingle t drop the \slash ra\slash . }

\par{13. ${\overset{\textnormal{き}}{\text{決}}}$ まりに ${\overset{\textnormal{ぎもん}}{\text{疑問}}}$ があれば、それも ${\overset{\textnormal{かいせいていあん}}{\text{改正提案}}}$ を ${\overset{\textnormal{しんせい}}{\text{申請}}}$ できる。 \hfill\break
 \emph{Kimari ni gimon ga areba, sore mo kaisei teian wo shinsei dekiru. \hfill\break
 }If you have any questions about the rules, you can submit them in a reform proposal. }

\par{\textbf{Sentence Note }: \emph{Sore }それ is used because there is no context that implies individuality to potential problems with the rule(s) in question. }

\par{14. ${\overset{\textnormal{まち}}{\text{町}}}$ も、 ${\overset{\textnormal{うみ}}{\text{海}}}$ も、 ${\overset{\textnormal{そら}}{\text{空}}}$ も、 ${\overset{\textnormal{やま}}{\text{山}}}$ も、 ${\overset{\textnormal{まんげつ}}{\text{満月}}}$ も、 ${\overset{\textnormal{きせつ}}{\text{季節}}}$ も、 ${\overset{\textnormal{でんしゃ}}{\text{電車}}}$ も、 ${\overset{\textnormal{こうえん}}{\text{公園}}}$ も、 ${\overset{\textnormal{ゆうえんち}}{\text{遊園地}}}$ も、 ${\overset{\textnormal{どうぶつ}}{\text{動物}}}$ も、スーパーマーケットも、おもちゃ ${\overset{\textnormal{や}}{\text{屋}}}$ も、 ${\overset{\textnormal{わたし}}{\text{私}}}$ はこの ${\overset{\textnormal{こ}}{\text{子}}}$ からそれらをすべて ${\overset{\textnormal{うば}}{\text{奪}}}$ ってきたんだ。 \hfill\break
 \emph{Machi mo, umi mo, sora mo, yama mo, mangetsu mo, kisetsu mo, densha mo, kōen mo, yūnchi mo, sūpāmāketto mo, omochaya mo, watashi wa kono ko kara sorera wo subete ubatte kita n da. \hfill\break
 }Towns, the sea, the sky, mountains, the full moon, the seasons, trains, parks, amusement parks, animals, supermarkets, toy stores, I\textquotesingle ve stolen all those things from this child. }

\par{15. 将来のこととか不安が多いよ。これらのことで頭が一杯で、恋愛について考える余裕は無い。 \hfill\break
 \emph{Shōrai no koto toka fuan ga ōi yo. Korera no koto de atama ga ippai de, ren\textquotesingle ai ni tsuite kangaeru yoyū wa nai. \hfill\break
 }I have a lot of anxiety about the future and all. My mind has been full of these things, and I don\textquotesingle t have time to think about love. }

\par{16. この机にたくさんの鉛筆があるけど、これらの中から一つだけ持っていっていいよ。 \hfill\break
 \emph{Kono tsukue ni takusan no empitsu ga aru kedo, korera no naka kara hitotsu dake motte itte ii yo. \hfill\break
 }There are many pencils in this desk, but it is okay for you to take only one out of these with you. }

\par{\textbf{Sentence Note }: This first \emph{kono }この can\textquotesingle t be interpreted in the plural sense in context. Although \emph{korera no naka }これらの中 could be rewritten by using \emph{kono }この instead, the individuality of the pencils the listener could choose from would not be emphasized. }

\par{17. これらの5人のうち、サッカー世界選手権で一位となった人は誰でしょう。 \hfill\break
 \emph{Korera no gonin no uchi, sakkā sekai senshuken de ichii to natta hito wa dare deshō. \hfill\break
 }Of these five individuals, who will become number one in the soccer championship? }

\par{18. 彼女は猫を三匹飼い出して、これらがやがて子猫を産んだ。 \hfill\break
 \emph{Kanojo wa neko wo sambiki kaidashite, korera ga yagate koneko wo unda. \hfill\break
 }She began to raise three cats, and at last these cats have given birth to kittens. }

\par{19. トランクが五つあって、それらは更に大きなカバンに詰められて一つになった。 \hfill\break
 \emph{Toranku ga itsutsu atte, sorera wa sara ni ōkina kaban ni tsumerarete hitotsu ni natta. \hfill\break
 }There were five trunks, and these trunks become one by being crammed into a much larger briefcase. }

\par{20. \{あちら・あれら △\}の ${\overset{\textnormal{やま}}{\text{山}}}$ は ${\overset{\textnormal{たか}}{\text{高}}}$ いですね。 \hfill\break
 \emph{[Achira\slash arera] no yama wa takai desu ne. \hfill\break
 }The mountain(s) over there are tall. }

\par{\textbf{Sentence Note }: As stated, \emph{arera }あれら doesn't really exist outside the world of translations from Western languages, particularly English. In this sentence, \emph{achira }あちら would be far more natural. Ex. 21 and Ex. 22 are further examples of how \emph{arera }あれら might come about in translated material, but something else will always be what would actually be used. }

\par{21. \{あれ・あれら\}を ${\overset{\textnormal{み}}{\text{見}}}$ てください。 \hfill\break
 \emph{[Are\slash arera] wo mite kudasai. \hfill\break
 }Look at those. }

\par{ 22. \{あの・あれらの\} ${\overset{\textnormal{ほん}}{\text{本}}}$ は ${\overset{\textnormal{だれ}}{\text{誰}}}$ のですか。 \hfill\break
 \emph{[Ano\slash arera no] hon wa dare no desu ka? \hfill\break
 }Whose books are those? }
    