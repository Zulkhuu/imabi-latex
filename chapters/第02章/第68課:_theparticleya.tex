    
\chapter{The Particle や}

\begin{center}
\begin{Large}
第68課: The Particle や 
\end{Large}
\end{center}
 
\par{ Here is yet another particle you can use to list things. It also does other things, though. Listing, as you would imagine, is a case particle usage. After all, that's what と is in this case. And, this usage is specifically seen in between nominal phrases. Notice how the word "nominal" used instead of just "noun." This means that you can link things with や that are grammatically made nouns. }

\par{ After all, や is a particle. Do you really expect one to have just ONE usage? Its other usages are not as important, and some will be deferred to other lessons. Although the remaining usages discussed in this lesson may seem out of place, since they are relatively easy, you might as well know about them. }
      
\section{The Case Particle や}
 
\par{ や \textbf{lists things incompletely }. This means that there is a sense that there are more related things implied. It may list nouns, phrases, and even clauses like "and", but it more so lists in categories . }

\par{1. ピクニックには、お ${\overset{\textnormal{かし}}{\text{菓子}}}$ や飲み ${\overset{\textnormal{もの}}{\text{物}}}$ を ${\overset{\textnormal{も}}{\text{持}}}$ って行った。 \hfill\break
We brought candies, drinks, etc. to the picnic. }

\par{2. 家は ${\overset{\textnormal{でんしゃ}}{\text{電車}}}$ やバスの ${\overset{\textnormal{ていりゅうじょう}}{\text{停留場}}}$ に近くてとても ${\overset{\textnormal{べんり}}{\text{便利}}}$ なところです。 \hfill\break
My home is close to the stations, bus, train, etc., and it's (in) a very convenient place. }

\par{3. 牛や馬が草むらで草を ${\overset{\textnormal{は}}{\text{食}}}$ む。 \hfill\break
Cows and horses graze in grassy places. }

\par{4. 東京や ${\overset{\textnormal{かんこく}}{\text{韓国}}}$ へ行く時や、 ${\overset{\textnormal{ひこうき}}{\text{飛行機}}}$ で行きます。 \hfill\break
Whenever I go to Tokyo, Korea, etc., I go by plane. }

\par{5. 何やかや言っては両親にお金をせびる。 \hfill\break
To say this and that to get money from one's parents. }

\par{\textbf{Word Note }: The か in the above expression is not the particle か. Rather, it can be written in 漢字 as 彼, the same character for "he" and actually has the same origin as かれ. Both are demonstrative words, and in this case, it is being used as an indefinite pronoun. }
      
\section{The Adverbial Particle や}
 
\par{ This classification of や happens to not be as important as the conjunctive usage, which is why more must be said of it. Attached to an adverb and accompanied with a statement of exclamation, it strengthens the meaning of the adverb that it is attached to. This is rather limited, so it is best that you only use this with set phrases and other instances that you see it used in. For instance, the first example has またもや, which happens to be a common set example of this. }
 
\par{6. またもや ${\overset{\textnormal{しっぱい}}{\text{失敗}}}$ しました。 \hfill\break
I lost yet again. }
 
\par{7. 今や ${\overset{\textnormal{おそ}}{\text{遅}}}$ しと ${\overset{\textnormal{しょうり}}{\text{勝利}}}$ を待ち ${\overset{\textnormal{わ}}{\text{侘}}}$ びていた。 \hfill\break
He was just impatiently tired of waiting for victory. }
 
\par{8. 彼の言葉に、祥子はまたもや目頭を熱くした。 \hfill\break
His words made Sachiko yet again move to tears. \hfill\break
From 冷たい誘惑 by 乃南アサ. }
    