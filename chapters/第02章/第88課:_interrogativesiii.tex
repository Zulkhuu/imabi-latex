    
\chapter{Interrogatives III}

\begin{center}
\begin{Large}
第88課: Interrogatives III: With Particles 
\end{Large}
\end{center}
 
\par{ We've seen the basic interrogatives, the many ways to say how, and other phrases involving this thing. Now it is time to see them used with particles. }
      
\section{With か \& も}
 
\par{ When a 疑問詞 is followed by the particles か or も, you get expressions for "some\dothyp{}\dothyp{}\dothyp{}" and "all\dothyp{}\dothyp{}\dothyp{}" respectively. The expressions with も have positive and negative definitions. }

\begin{ltabulary}{|P|P|P|}
\hline 

 & + か & + も \hfill\break
\\ \cline{1-3}

だれ・どなた & Someone & Either\slash Everyone\slash Neither\slash Nobody \\ \cline{1-3}

なに & Something & Everything\slash Nothing \\ \cline{1-3}

なぜ & Somehow & X \\ \cline{1-3}

どうして & X & By any means \\ \cline{1-3}

いつ & Sometime & Always \\ \cline{1-3}

どこ・どちら & Somewhere & Everywhere\slash Nowhere \\ \cline{1-3}

どれ & Every (one\slash thing) & Anything\slash none \\ \cline{1-3}

どう & Somehow or another & Much \\ \cline{1-3}

\end{ltabulary}

\par{か makes the "some". When も follows these words, it gives a meaning of "all". In a negative expression, it does the complete opposite. ${\overset{\textnormal{}}{\text{何}}}$ も is almost always used with the negative, but it is positive in 何もかも and 何も ${\overset{\textnormal{すべ}}{\text{全}}}$ て. }

\par{\textbf{Part of Speech Note }: These words are either used as nouns or adverbs. Remember that these words are no longer interrogatives. }

\begin{center}
\textbf{Examples } 
\end{center}

\par{${\overset{\textnormal{}}{\text{1. 仕事}}}$ も ${\overset{\textnormal{}}{\text{何}}}$ もかも ${\overset{\textnormal{}}{\text{忘}}}$ れて、 ${\overset{\textnormal{きゅうか}}{\text{休暇}}}$ を ${\overset{\textnormal{たの}}{\text{楽}}}$ しんだ。 \hfill\break
I forgot about work and everything, and I enjoyed a vacation. }

\par{\textbf{Grammar Note }: To say "nobody's", use " ${\overset{\textnormal{}}{\text{誰}}}$ のXも". ${\overset{\textnormal{}}{\text{誰}}}$ の ${\overset{\textnormal{}}{\text{車}}}$ も = "nobody's car". "Someone's car" = ${\overset{\textnormal{}}{\text{誰}}}$ かの ${\overset{\textnormal{}}{\text{車}}}$ . ${\overset{\textnormal{}}{\text{誰}}}$ か is an indefinite pronoun. For always, いつもの is right. }

\par{${\overset{\textnormal{}}{\text{2. 彼}}}$ らの ${\overset{\textnormal{}}{\text{住}}}$ んでいる ${\overset{\textnormal{ところ}}{\text{所}}}$ はどこも ${\overset{\textnormal{かいどうすじ}}{\text{街道筋}}}$ に ${\overset{\textnormal{}}{\text{近}}}$ い。 \hfill\break
Wherever they are living, it is close to the highway. }

\par{${\overset{\textnormal{}}{\text{3. 誰}}}$ かがおいしいクッキーを ${\overset{\textnormal{}}{\text{全部食}}}$ べた。 \hfill\break
Somebody ate all of the delicious cookies! }

\par{4. いつも ${\overset{\textnormal{}}{\text{何}}}$ を ${\overset{\textnormal{}}{\text{考}}}$ えているのですか。 \hfill\break
What are you always thinking about? }

\par{${\overset{\textnormal{}}{\text{5. 彼}}}$ は ${\overset{\textnormal{}}{\text{本当}}}$ にいつも ${\overset{\textnormal{しんせつ}}{\text{親切}}}$ です。 \hfill\break
He is really always kind. }

\par{6. あいつはいつも ${\overset{\textnormal{ぐち}}{\text{愚痴}}}$ を ${\overset{\textnormal{こぼ}}{\text{零}}}$ してる。 \hfill\break
That guy is always complaining. \hfill\break
\hfill\break
\textbf{Meaning Note }: ${\overset{\textnormal{}}{\text{愚痴}}}$ を ${\overset{\textnormal{}}{\text{零}}}$ す is a set phrase meaning "to complain". Also note that the character 零 is hardly ever used. }

\par{7. ${\overset{\textnormal{はんにん}}{\text{犯人}}}$ をどこかで ${\overset{\textnormal{}}{\text{見}}}$ たか。 \hfill\break
Did you see the criminal somewhere? }

\par{8. いつか ${\overset{\textnormal{りゅうがく}}{\text{留学}}}$ する。 \hfill\break
To some day study abroad. }

\par{\textbf{Word Note }: The following is \textbf{wrong }: ${\overset{\textnormal{}}{\text{日本}}}$ の ${\overset{\textnormal{}}{\text{大学}}}$ で ${\overset{\textnormal{}}{\text{留学}}}$ する. ${\overset{\textnormal{}}{\text{留学}}}$ する is used with に. So,  日本に ${\overset{\textnormal{}}{\text{留学}}}$ する is correct. ${\overset{\textnormal{}}{\text{日本}}}$ の ${\overset{\textnormal{}}{\text{大学}}}$ で ${\overset{\textnormal{}}{\text{勉強}}}$ する is correct, though. }

\par{${\overset{\textnormal{}}{\text{9. 誰}}}$ もが ${\overset{\textnormal{}}{\text{彼}}}$ の ${\overset{\textnormal{しょうさい}}{\text{商才}}}$ を ${\overset{\textnormal{みと}}{\text{認}}}$ めている。 \hfill\break
Everyone recognizes his business ability. }

\par{${\overset{\textnormal{}}{\text{10. 私}}}$ の ${\overset{\textnormal{}}{\text{英語}}}$ の ${\overset{\textnormal{こうざ}}{\text{講座}}}$ はどれも ${\overset{\textnormal{ぶんぽう}}{\text{文法}}}$ を ${\overset{\textnormal{あつか}}{\text{扱}}}$ った。 \hfill\break
Every one of my English courses dealt with grammar. }

\par{11. あなたのコンピューターはどこも ${\overset{\textnormal{}}{\text{悪}}}$ くありません。 \hfill\break
There's nothing wrong anywhere with your computer. }

\par{${\overset{\textnormal{}}{\text{12. 食}}}$ べ ${\overset{\textnormal{}}{\text{物}}}$ が ${\overset{\textnormal{}}{\text{何}}}$ もない。 \hfill\break
I have nothing to eat. }

\par{${\overset{\textnormal{}}{\text{13. 彼女}}}$ は ${\overset{\textnormal{}}{\text{何}}}$ も ${\overset{\textnormal{こた}}{\text{答}}}$ えませんでした。 \hfill\break
She answered nothing. }

\par{14. アクセルに ${\overset{\textnormal{}}{\text{何}}}$ かおかしいところがあります。 \hfill\break
There's something wrong with the accelerator. }

\par{15. 今日の彼はどうかしている。 \hfill\break
He's not himself today. }

\par{16. ${\overset{\textnormal{いろいろ}}{\text{色々}}}$ どうもありがとうございます。 \hfill\break
Thank you very much for everything. }

\par{17. ${\overset{\textnormal{きのう}}{\text{昨日}}}$ はどうも。 \hfill\break
Thanks for yesterday. }

\par{18. 今日は魚がどうも ${\overset{\textnormal{く}}{\text{食}}}$ わない。 \hfill\break
The fish won't bite for some reason today. }

\par{${\overset{\textnormal{}}{\text{19. 何}}}$ かの ${\overset{\textnormal{ひょうし}}{\text{拍子}}}$ でケイタイを落としたんです。 \hfill\break
By some chance, I dropped my cellphone. }

\par{20. \textbf{${\overset{\textnormal{}}{\text{何}}}$ \textbf{か }}${\overset{\textnormal{しゅみ}}{\text{趣味}}}$ はありますか。 \hfill\break
Do you have \textbf{any }hobbies? }

\par{${\overset{\textnormal{}}{\text{21. 彼}}}$ らはワープロにいつも ${\overset{\textnormal{}}{\text{手}}}$ を ${\overset{\textnormal{や}}{\text{焼}}}$ いている。 \hfill\break
They're always having trouble with their processor. \hfill\break
\hfill\break
\textbf{Phrase Note }: ${\overset{\textnormal{}}{\text{手}}}$ を ${\overset{\textnormal{}}{\text{焼}}}$ く is a set phrase similar to "not know what to do\slash be at a loss with; to be put out".  }
      
\section{With でも・ても}
 
\par{ When an interrogative is followed by でも, it creates an "any" indefinite pronoun. These phrases are only used as adverbs. So, you can't follow them with particles like が. }

\begin{ltabulary}{|P|P|P|}
\hline 

なん & + でも & Anything \\ \cline{1-3}

だれ・どなた & + でも & Anybody\slash anyone \\ \cline{1-3}

いつ & + でも & Anytime \\ \cline{1-3}

どこ・どちら & + でも & Anywhere \\ \cline{1-3}

どれ & + でも & Anything (out of many) \\ \cline{1-3}

\end{ltabulary}

\par{22. \textbf{いつでも }いい。 \hfill\break
\textbf{Anytime }is OK. }

\par{23. この ${\overset{\textnormal{たな}}{\text{棚}}}$ の ${\overset{\textnormal{しょうひん}}{\text{商品}}}$ は \textbf{どれでも }千円です。 \hfill\break
\textbf{Whichever }item on this shelf is 1,000 yen. }

\par{24. \textbf{どこでも }好きなところへ行く。 \hfill\break
I will go \textbf{anywhere }I like. }

\par{25. ${\overset{\textnormal{ろんぴょう}}{\text{論評}}}$ の ${\overset{\textnormal{けっか}}{\text{結果}}}$ は \textbf{いつでも }悪いことだ。 \hfill\break
The effect of criticism is \textbf{always }a bad thing. }

\par{26. 彼は ${\overset{\textnormal{なん}}{\text{何}}}$ \textbf{でも }${\overset{\textnormal{こうか}}{\text{高価}}}$ な ${\overset{\textnormal{もの}}{\text{物}}}$ を ${\overset{\textnormal{この}}{\text{好}}}$ む。 \hfill\break
He likes \textbf{anything }expensive. \hfill\break
\hfill\break
\textbf{Word Note }: 好む is similar to "to prefer". }

\begin{center}
\textbf{Interrogative + }\textbf{~ても }
\end{center}

\par{When a question word is used with ても, it creates a "no matter\dothyp{}\dothyp{}\dothyp{}" statement. }

\par{27. ${\overset{\textnormal{あそさん}}{\text{阿蘇山}}}$ は \textbf{いつ }見 \textbf{ても }きれいですね。 \hfill\break
Mount Aso is beautiful \textbf{no matter when }you see it, isn't it? \hfill\break
\hfill\break
28. 僕は \textbf{何 }を食べ \textbf{ても }、太らない。 \hfill\break
 \textbf{No matter what }I eat, I don't get fat. }

\par{29. \textbf{どこ }に行っ \textbf{ても }僕は君の ${\overset{\textnormal{そば}}{\text{側}}}$ にいるよ。 \hfill\break
 \textbf{No matter where }you go, I will be by your side. }

\par{30. \textbf{誰の }${\overset{\textnormal{かぎ}}{\text{鍵}}}$ がドアを開け \textbf{ても }、僕がいるから君はこの部屋に ${\overset{\textnormal{あんぜん}}{\text{安全}}}$ だよ。 \hfill\break
\textbf{No matter whose }key opens the door, you are safe in this room because I'm here. }

\par{31. \textbf{何度 }見 \textbf{ても }面白いです。 \hfill\break
 \textbf{No matter how many times }I see it, it's funny. }
    