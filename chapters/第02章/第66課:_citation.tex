    
\chapter{The Particle と II}

\begin{center}
\begin{Large}
第66課: The Particle と II: Citation 
\end{Large}
\end{center}
 
\par{ We've looked at several usages of the case particle と before, but now it's time to learn about how it is used in citation. }
      
\section{Citation}
 
\par{ と is also the citation particle, basically spoken quotation marks. This usage often translates to "that". However, grammar within subordinate clauses--those that are embedded within a sentence--is not exactly the same as in English as you will see. }

\par{\textbf{Orthography Note }: Actual quotation marks, 「」, are only used in direct quotes . }

\begin{ltabulary}{|P|P|P|P|P|P|}
\hline 

と言う & To say that & と思う & To think that & と考える & To consider that \\ \cline{1-6}

\end{ltabulary}

\begin{center}
\textbf{Basic Grammatical Restraints in Citation } 
\end{center}

\par{ In direct quotation, the quoted verb may be in any form because you're telling what someone has said. So, sentences like below are completely correct. }

\par{1. 彼女は「雨が ${\overset{\textnormal{ふ}}{\text{降}}}$ りません」と言いました。 \hfill\break
She said, "It won't rain." }

\par{ It is common to just see と in abbreviated speech. This means that the citation verb is just implied. It is often the case that the verb is 言う, but full context will make it clear which citation verb is intended. If such deletion is within a more complex sentence, a comma is \emph{often }after と, and the citation verb is probably the final verb of a following clause. Having said that, though, below is a sentence where neither is the case. }

\par{2. もうそんな寒さか \textbf{と }${\overset{\textnormal{しまむら}}{\text{島村}}}$ は ${\overset{\textnormal{そと}}{\text{外}}}$ を ${\overset{\textnormal{なが}}{\text{眺}}}$ めると、 ${\overset{\textnormal{てつどう}}{\text{鉄道}}}$ の ${\overset{\textnormal{かんしゃ}}{\text{官舎}}}$ らしいバラックが ${\overset{\textnormal{やますそ}}{\text{山裾}}}$ に ${\overset{\textnormal{さむざむ}}{\text{寒々}}}$ と ${\overset{\textnormal{ち}}{\text{散}}}$ らばっているだけで、 \hfill\break
雪の色はそこまで行かぬうちに ${\overset{\textnormal{やみ}}{\text{闇}}}$ に ${\overset{\textnormal{の}}{\text{呑}}}$ まれていた。 \hfill\break
When Shimamura gazed outside, thinking it had already gotten cold, railroad residence-like barracks were desolately dispersed at the foot of the mountains, and before the snow hues could reach that far, the barracks were swallowed by darkness. }

\par{From 雪国 by ${\overset{\textnormal{かわばたやすなり}}{\text{川端康成}}}$ . }

\par{\textbf{Particle Note }: The quotation is of a thought. The thought is もうそんな寒さか. The verb that is implied, then, is 思う. The sentence would, then, be もうそんな寒さか \textbf{と }思い. 思い is just like 思って here. }

\par{\textbf{漢字 Note }: 呑む means "to swallow," but it is usually spelled as 飲む nowadays. }

\par{\textbf{Grammar Note }: 行かぬうちに = 行かないうちに. }

\begin{center}
 \textbf{Ungrammatical? }
\end{center}

\par{ For verbs of thought both the citation and cited verb mustn't be polite . Your thoughts aren't in polite speech. Thus, the final verb should be what takes ~ます. Furthermore, Japanese hates for the same purpose being expressed twice in the same clause or phrase. }

\par{3a. 雨が降りませんと思います。X \hfill\break
3b. 雨が降らないと思います。〇 \hfill\break
I think that it won't rain. }

\par{ For \textbf{other people's }thoughts, the verb of thought must be used with ~ている. You can also use the progressive for first person too. Now, for other citation verbs like "to say", polite speech is OK only in a direct quote. }

\par{4a. 彼はどう思うの? X \hfill\break
4b. 彼はどう思ってるの? 〇  (ちょっとくだけた) \hfill\break
What is he thinking? }

\begin{center}
\textbf{Tense and Negation with Citation }
\end{center}

\par{ What about the past and negative? Either verb can be changed but not both. Again, Japanese avoids function words being doubled. If both are negative, you make a positive. If both are past, your sentence becomes difficult to interpret and sounds unnatural. }

\par{ The choice you make in deciding which is negative or past does slightly change nuance. Some combinations are more common than others, but none are ungrammatical. However, speaker variation does exist. For instance, for the negative, 5b is preferred over 5d by many speakers. }

\par{${\overset{\textnormal{}}{\text{5a. 雨}}}$ が ${\overset{\textnormal{}}{\text{降}}}$ ったと ${\overset{\textnormal{}}{\text{思}}}$ う。 = I think that it rained. \hfill\break
${\overset{\textnormal{}}{\text{5b. 雨}}}$ が ${\overset{\textnormal{}}{\text{降}}}$ らないと ${\overset{\textnormal{}}{\text{思}}}$ う。= I think that it won't rain. \hfill\break
${\overset{\textnormal{}}{\text{5c. 雨}}}$ が ${\overset{\textnormal{}}{\text{降}}}$ ると ${\overset{\textnormal{}}{\text{思}}}$ った。 = I thought it would rain. \hfill\break
${\overset{\textnormal{}}{\text{5d. 雨}}}$ が ${\overset{\textnormal{}}{\text{降}}}$ ると ${\overset{\textnormal{}}{\text{思}}}$ わない。 = I don't think that it will rain. }

\par{The second and last are similar, but the last is more emphatic. This is because you're rejecting that "雨が降る" is what you think. You'd generally say "雨が降らないと思う". }

\par{Consider the following. Do not do this. This is a clear example of why doubling past tense items in Japanese is a big problem. This does not negate the existence of past tense in a relative clause or the main clause. In context, this may work to show that you thought something had ended. }

\par{6.  「映画を見て、どう思いましたか?」 \hfill\break
「スリルがあって面白かったと思いました」 X・△ }

\par{ This is something a kid might say in Japanese. It's best not to mimic this, but it is important to know that some speakers might use it, but it would certainly not be refined Japanese. The speaker meant to say "I thought (the movie) was interesting". Remember, と quotes the thought that one has. When you're watching the movie, you think it's おもしろい. }

\par{ おもしろかったと思う is possible, but it would express that "you think that it \emph{was }cool". For example, a show could be discontinued and you are saying what you think about the show in retrospect. This is similar for other expressions with the cited verb\slash adjective in the past tense and the main citing verb in the non-past. }

\par{ If the ~た were used to show completion (完了), then ~たと~た would be correct. To insert a ~た due to English tense agreement in a Japanese sentence with no need for this, it becomes strange Japanese, to say the least. }

\par{7. 梅雨はもう終わったと思ったのですが、どうやらまだ続いているようですね。 \hfill\break
I thought that the rainy season had already ended, but it appears to still be going, doesn't it? }

\begin{center}
\textbf{Examples }
\end{center}

\par{${\overset{\textnormal{}}{\text{8. 彼}}}$ の ${\overset{\textnormal{そふ}}{\text{祖父}}}$ は ${\overset{\textnormal{いだい}}{\text{偉大}}}$ な ${\overset{\textnormal{がくしゃ}}{\text{学者}}}$ だったという。 \hfill\break
They say that his grandfather was a great scholar. }

\par{${\overset{\textnormal{}}{\text{9. 山田}}}$ さんは「 ${\overset{\textnormal{}}{\text{英語}}}$ は ${\overset{\textnormal{}}{\text{難}}}$ しいですよ」と ${\overset{\textnormal{}}{\text{言}}}$ いました。 \hfill\break
Ms. Yamada said, "English is hard!" }

\par{${\overset{\textnormal{}}{\text{10. 妹}}}$ はサンタ・クロースは ${\overset{\textnormal{}}{\text{来}}}$ ないと ${\overset{\textnormal{}}{\text{思}}}$ っています。 \hfill\break
My little sister thinks that Santa Claus won't come. }

\par{11. 〇〇さんのことは ${\overset{\textnormal{ゆうじん}}{\text{友人}}}$ だと ${\overset{\textnormal{}}{\text{考}}}$ えている。 \hfill\break
To regard 〇〇 as a friend. }

\par{${\overset{\textnormal{}}{\text{12. 花子}}}$ が ${\overset{\textnormal{}}{\text{先生}}}$ に ${\overset{\textnormal{}}{\text{英語}}}$ でいいえと ${\overset{\textnormal{こた}}{\text{答}}}$ えました。 \hfill\break
It is Hanako that told the teacher no in English. }

\par{\textbf{Culture Note }: The above is not a direct quotation because it's highly unlikely that Hanako actually said いいえ. A quote is likely to be a summary. Japanese speakers are always conscious of their relations to others. So, Hanako probably said じつは、そうではありません。 }

\begin{center}
\textbf{Citation と + する }
\end{center}

\par{The last thing we'll look at is とする. This is specifically about uses that are really just an extension of と's citation usage. }

\begin{ltabulary}{|P|P|}
\hline 

1. & To presume that\dothyp{}\dothyp{}\dothyp{} \\ \cline{1-2}

2. & A euphemism of という, と考える, or と主張する \\ \cline{1-2}

\end{ltabulary}

\par{\hfill\break
13. その ${\overset{\textnormal{せいとう}}{\text{政党}}}$ が ${\overset{\textnormal{かはんすう}}{\text{過半数}}}$ を ${\overset{\textnormal{}}{\text{取}}}$ ったとする。 \hfill\break
To presume that the party takes the majority. }

\par{14. ${\overset{\textnormal{きしょうちょう}}{\text{気象庁}}}$ では ${\overset{\textnormal{つなみ}}{\text{津波}}}$ の ${\overset{\textnormal{しんぱい}}{\text{心配}}}$ はないとしています。 \hfill\break
The Meteorological Center is saying that there is no worry of a tsunami. }
    