    
\chapter{Absolute Time I}

\begin{center}
\begin{Large}
第56課: Absolute Time I: Basic Phrases 
\end{Large}
\end{center}
 
\par{ Absolute time expressions indicate a relatively exact time frame and are often sensitive to tense. They are generally adverbial nouns. This means that they usually can either function as nouns or adverbs depending on context. So, pay attention to how this affects particle usage. }
      
\section{Absolute Time}
 
\par{ The following table shows many single word absolute time phrases in Japanese. Patterns to be discussed in this lesson help fill in the gaps. Take note of patterns. There are word notes after the chart. So, be sure to not overlook them. }

\par{\textbf{Chart Note }: Variants are listed from most to least polite. ▽ stands for rare. }

\begin{ltabulary}{|P|P|P|P|P|P|}
\hline 

時間帯 & 2\dothyp{}\dothyp{}\dothyp{}ago & Last\dothyp{}\dothyp{}\dothyp{} & This\dothyp{}\dothyp{}\dothyp{} & Next\dothyp{}\dothyp{}\dothyp{} & 2\dothyp{}\dothyp{}\dothyp{}from now \\ \cline{1-6}

Day & 一昨日 \hfill\break
いっさくじつ \hfill\break
おととい & 昨日 \hfill\break
さくじつ \hfill\break
きのう & 本日、今日 \hfill\break
ほんじつ、 \hfill\break
こんじつ ▽ \hfill\break
こんにち \hfill\break
きょう & 明日 \hfill\break
みょうにち \hfill\break
あす \hfill\break
あした & 明後日 \hfill\break
みょうごにち \hfill\break
あさって \\ \cline{1-6}

Morning &  & 昨朝 \hfill\break
さくちょう & 今朝 \hfill\break
こんちょう ▽ \hfill\break
けさ & 明朝 \hfill\break
みょうちょう 
&  \\ \cline{1-6}

Evening & 一昨晩 \hfill\break
いっさくばん & 昨晩・夕べ \hfill\break
さくばん・ゆうべ & 今晩 \hfill\break
こんばん & 明晩 \hfill\break
みょうばん &  \\ \cline{1-6}

Night & 一昨夜 \hfill\break
いっさくや & 昨夜 \hfill\break
さくや & 今夜 \hfill\break
こんや & 明夜 \hfill\break
みょうや &  \\ \cline{1-6}

Week & 先々週 \hfill\break
せんせんしゅう & 昨週、先週 \hfill\break
さくしゅう ▽、せんしゅう & 今週 \hfill\break
こんしゅう & 来週 \hfill\break
らいしゅう & 再来週 \hfill\break
さらいしゅう \\ \cline{1-6}

Month & 先々月 \hfill\break
せんせんげつ & 先月 \hfill\break
せんげつ & 今月 \hfill\break
こんげつ & 来月 \hfill\break
らいげつ & 再来月 \hfill\break
さらいげつ \\ \cline{1-6}

Year & 一昨年 \hfill\break
いっさくねん \hfill\break
おととし & 昨年、去年 \hfill\break
さくねん、きょねん & 今年 \hfill\break
ことし & 来年 \hfill\break
らいねん & 明後年、再来年 \hfill\break
みょうごねん、さらいねん 
\\ \cline{1-6}

\end{ltabulary}

\par{\textbf{Word Notes }: }

\par{1. ${\overset{\textnormal{せんじつ}}{\text{先日}}}$ means "the other day". ${\overset{\textnormal{ほんげつ}}{\text{本月}}}$ (this month) and ${\overset{\textnormal{ほんねん}}{\text{本年}}}$ (this year) exist, but they are very formal and typically in written addresses. However, note that although ${\overset{\textnormal{ほんちょう}}{\text{本朝}}}$ exists, it means "our nation" instead of "this morning". }

\par{2. Changing ${\overset{\textnormal{いっさく}}{\text{一昨}}}$ to ${\overset{\textnormal{いっさくさく}}{\text{一昨々}}}$ creates "3\dothyp{}\dothyp{}\dothyp{}ago". However, the Chinese readings are not used in the spoken language and should be reserved for the written language. For the native words おととい and おととし, you can さき to get さきおととい (three days ago) and さきおととし (three years ago) respectively. It is these native readings that should be used in actual practice. }

\par{3. "3 days from now" can be ${\overset{\textnormal{(しあさって・みょうごにち)}}{\text{明々後日}}}$ or ${\overset{\textnormal{あした}}{\text{明日}}}$ の ${\overset{\textnormal{よくよくじつ}}{\text{翌々日}}}$ . }

\par{4. 今年 may be read as こんねん in certain contexts such as in the phrase 今年度 (this fiscal year). }

\par{5. It's funny to note that 明々後日 exists too, and its native counterpart is やのあさって. These words are not really used that often because 4日後 is much easier. }

\par{6. 昨 can sometimes be very formal. For instance, 去年 will be used most of the time when speaking. The words 昨春, 昨夏, 昨秋, and 昨冬 are all 書き言葉. }

\begin{center}
 \textbf{Making other Absolute Time Expressions }
\end{center}

\par{The following expressions are very important when you can't use a specific, one word time expression like the ones above. Some of these expressions are not really used and some may be more formal than others. All of these details are discussed in the word notes following the chart. }

\begin{ltabulary}{|P|P|P|P|P|P|}
\hline 

3\dothyp{}\dothyp{}\dothyp{}ago & 一昨昨+Time Word & 2\dothyp{}\dothyp{}\dothyp{}ago & 一昨 & Last\dothyp{}\dothyp{}\dothyp{} & この前の \\ \cline{1-6}

This\slash next\dothyp{}\dothyp{}\dothyp{} & 今度の & Today's\dothyp{}\dothyp{}\dothyp{} & 今日の & Next\dothyp{}\dothyp{}\dothyp{} & この次の \\ \cline{1-6}

1 after next & 翌々(の) & Day & 日 & Morning & 朝, 午前 \\ \cline{1-6}

Evening & 夕方, 夕べ, 夕暮れ, 晩, 夕 & Night & 夜, 夜間 & Week & 週(間) \\ \cline{1-6}

Month & 月, 月間 & Year & 年, 年(間) &  &  \\ \cline{1-6}

\end{ltabulary}

\par{\textbf{Word Notes }: }
 
\par{1. 夜間, 週間, 月間, and 年間 are all formal and more likely to be used in the written language. 週間 and 年間 are common in the spoken language, but they are still more formal. }

\par{2. 夕べ most frequently means "last night" where it is more common than 昨夜. \hfill\break
\hfill\break
3. 夕 is essentially the same as 夕方, but it's often used in set phrases. 夕暮れ is very commonly used, and as the character 暮 suggests, it especially refers to the time when the sun is actually going down. Lastly, 晩 is the closest equivalent to the generic English word "evening" and is a commonly used word. }
 
\par{4. You may use ${\overset{\textnormal{よく}}{\text{翌}}}$ ~ for "next" for expressions that use ${\overset{\textnormal{らい}}{\text{来}}}$ ~. Frequency of use depends on the expression. The difference between 翌週 and 来週 would be the same as the difference between "the next week" and "next week" in English. This is the same for the other expressions. }

\par{5. 翌々 is usually used without a の. You can use it with days, months, and years. Something like 翌々2016年 would it equate to "2016, the year two years from now". The word is not quite that common, and you are more likely to see it in writing. }

\par{6. 一昨々 has the reading いっさくさく. It is not used in the spoken language, and neither are other Sino-Japanese phrases made with it such as ${\overset{\textnormal{いっさくさくじつ}}{\text{一昨々日}}}$ and ${\overset{\textnormal{いっさくさくしゅう}}{\text{一昨々週}}}$ . For 一昨々日 and 一昨々年, there are the native readings さきおととい and さきおととし respectively, and they're actually used. So, say you wanted to say Monday three days ago, you would need to say ${\overset{\textnormal{さきおととい}}{\text{一昨々日}}}$ の ${\overset{\textnormal{げつようび}}{\text{月曜日}}}$ . }

\par{7. 一昨 for "2\dothyp{}\dothyp{}\dothyp{}ago" is read as いっさく and is rare. This is mainly due to the fact that it is Sino-Japanese. When it is used, it is appended without the particle の to a time phrase. So, you get phrases like 一昨 ${\overset{\textnormal{みっか}}{\text{3日}}}$ meaning "the third, which was two days ago". }

\par{8. If you want to realistically go past two units of time either direction (past or future), it is more practical to say something like 5年前 (five years ago) or ${\overset{\textnormal{みっかまえ}}{\text{3日前}}}$ (three days ago). Of course, sometimes one of these cooler phrases like さきおととい (= 3日前). Nevertheless, #+ Time Phrase +  前(before)\slash  ${\overset{\textnormal{ご}}{\text{後}}}$ (after) is more frequently used once you leave the sphere of the common words like きのう, きょう, あした, etc. }

\begin{center}
 \textbf{Examples }
\end{center}

\par{1. 昨夜、10 ${\overset{\textnormal{じ}}{\text{時}}}$ に ${\overset{\textnormal{ね}}{\text{寝}}}$ た。(More common) \hfill\break
I went to sleep at 10 last night. }

\par{2. アニメクラブは7時から ${\overset{\textnormal{れいじ}}{\text{零時}}}$ までです。 \hfill\break
Anime club is from 7 to midnight. }
 
\par{3. きのう仕事を休みました。 \hfill\break
I took a break from work yesterday. }

\par{\textbf{漢字 Note }: Words like きょう and きのう are frequently written in ひらがな. }
 
\par{4. 彼女は ${\overset{\textnormal{}}{\text{一昨年}}}$ ${\overset{\textnormal{う}}{\text{生}}}$ まれました。 \hfill\break
She was born the year before last. }
 
\par{5. 翌日の ${\overset{\textnormal{じゅぎょう}}{\text{授業}}}$ の ${\overset{\textnormal{よしゅう}}{\text{予習}}}$ をする。 \hfill\break
To prepare for the next day's lessons. }

\par{\textbf{Classroom Note }: You should always 予習 before your Japanese class to get more out of what your teacher says. }
 
\par{6. 今朝はたいそう寒かったですね。 \hfill\break
This morning was quite cold, wasn't it? }
 
\par{7. この前の火曜日におもしろい ${\overset{\textnormal{えいが}}{\text{映画}}}$ を見ましたよ。 \hfill\break
I saw an interesting movie last Tuesday! }
 
\par{8. 明日の朝8時に ${\overset{\textnormal{お}}{\text{起}}}$ こしてください。 \hfill\break
Please wake me up at 8 o'clock tomorrow morning. }
 
\par{${\overset{\textnormal{しゅんじ}}{\text{9. その人は瞬時}}}$ に(して) ${\overset{\textnormal{さる}}{\text{猿}}}$ になった。 \hfill\break
That person became a monkey in an instant. }
 
\par{10. 「 ${\overset{\textnormal{ことし}}{\text{今年}}}$ の ${\overset{\textnormal{ふゆ}}{\text{冬}}}$ は ${\overset{\textnormal{さむ}}{\text{寒}}}$ くなるでしょうか」「 ${\overset{\textnormal{てんきよほう}}{\text{天気予報}}}$ を見ていないので寒いかどうか分かりません」 \hfill\break
“Will this winter be cold?” “I don't know whether or not it's going to be cold in the weather forecast." }
      
\section{Times}
 
\begin{center}
\textbf{Times (Frequency): The Counters ~ }\textbf{度 VS ~ }\textbf{回 }
\end{center}
 
\par{ Many Japanese speakers will say that ${\overset{\textnormal{いちど}}{\text{一度}}}$ and ${\overset{\textnormal{いっかい}}{\text{一回}}}$ are the same, but they actually differentiate them. First, consider the following sentence where they are completely interchangeable. }
 
\par{11. 私はあの本を\{2度・2回\}読んだ。 \hfill\break
I read that book twice. }
 
\par{12. \{2度・2回\}目 ${\overset{\textnormal{ふわた}}{\text{の不渡}}}$ り \hfill\break
Second bouncing\slash non-payment }
 
\par{ Despite the fact that they both count the number of times an action occurs, there are instances where you choose them liberally. One restriction is that ~度 can't be used with ${\overset{\textnormal{だい}}{\text{第}}}$ ~ or ${\overset{\textnormal{ぜん}}{\text{全}}}$ ~. ~度 is also not used with decimals. }
 
\par{13. 6回 ${\overset{\textnormal{れんぞく}}{\text{連続}}}$ で ${\overset{\textnormal{さんか}}{\text{参加}}}$ している。(6回 \textrightarrow  6度 X) \hfill\break
I've been participated six times consecutively. }
 
\par{14. 今年は ${\overset{\textnormal{まつ}}{\text{祭}}}$ りが2度ある。(2度 \textrightarrow  2回 ?) \hfill\break
This year, the festival will come twice. }
 
\par{15. 第4回の ${\overset{\textnormal{かんけいかくりょうきょうぎ}}{\text{関係閣僚協議}}}$ (第4度 X) \hfill\break
The fourth relations cabinet conference }
 
\par{16. 平均週3.7回 \hfill\break
An average of 3.7 times a week }
 
\par{17. ${\overset{\textnormal{ぜん}}{\text{全}}}$ 20回のセミナー \hfill\break
All 20 seminars }
 
\par{18. 2度の ${\overset{\textnormal{たいけん}}{\text{体験}}}$ をよい経験として ${\overset{\textnormal{い}}{\text{生}}}$ かす。(2回 ?) \hfill\break
To use the second experience as a good lesson. }
 
\par{ When the number is more than 10, some of these restrictions go away. It's also interesting to note that ~度目 is used twice as much as ~回目. Remember that example with 6回連続? Something like 5年連続100度目 would be completely fine. For instances where they are interchangeable, 回 is more common. However, 度 is particularly common with the number 2. Also, as the number gets larger, 回 is less frequent. So, if a headline were to have "third non-payment" in it, odds are that it would have 3回. If a company somehow did this for the 15th time, we would expect to see 度. }
 
\par{ When counting the frequency\slash repetition of an action in a particular time frame, use ~回. This is in terms of years, days, etc. and doesn't take on details such as minutes, etc. In such case, you'd expect 度. Something like 月に三度 is possible, but this doesn't show frequency. }
 
\par{19. 毎年、この村では祭りを3回行います。 \hfill\break
We have festivals three times annually in this village. }
 
\par{ ~度 would count repetition that is uncertain and or irregular. Other expressions that deal with series or segmenting statistically prefer ~回. One last thing to consider is that ~度 is often used in regards to things that are hard to predict and ~回 is preferred overwhelmingly when the number of times of something can be known beforehand. }
 
\par{20. 10度目の ${\overset{\textnormal{ゆうしょう}}{\text{優勝}}}$ を ${\overset{\textnormal{めざ}}{\text{目指}}}$ す。 \hfill\break
To aim for the tenth victory. }
 
\par{21. 3年 ${\overset{\textnormal{れんぞく}}{\text{連続}}}$ 3度 ${\overset{\textnormal{め}}{\text{目}}}$ の ${\overset{\textnormal{ゆうしょう}}{\text{優勝}}}$ \hfill\break
 The third consecutive victory in three years }

\par{\textbf{Frequency Note }: Remember that ~度目 is generally twice as common as ~回. }
 
\par{22. 4度目の ${\overset{\textnormal{ふっかつ}}{\text{復活}}}$ は難しい。 \hfill\break
A fourth restoration is difficult. }
 
\par{23. 3回めのセミナー \hfill\break
The third seminar }

\par{24. 6回目の ${\overset{\textnormal{かくじっけん}}{\text{核実験}}}$ \hfill\break
 The sixth nuclear tests after restart }

\begin{center}
------------------------------------------------------------------- 
\end{center}

\begin{center}
Next Lesson \textrightarrow  第33課: 時, 間, \& 内 
\end{center}
    