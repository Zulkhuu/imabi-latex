    
\chapter{Might}

\begin{center}
\begin{Large}
第91課: Might: かもしれない 
\end{Large}
\end{center}
 
\par{ I might do this. That might be the case. As you can see, this lesson is about how to say might as in these contexts in Japanese. Be sure to say this phrase with pitch rising on か. }
      
\section{かもしれない}
 
\par{ かもしれない means "might" as in there is a possibility of something. The predicate that it follows should be in the plain form, but だ is deleted when directly before it. It may also be seen as やも知れない or contracted to かも in casual speech. It is typically just written in かな. }

\par{ Phrases like ひょっとして (maybe\slash possibly) and もしかしたら・もしかすると・もしかして (perhaps\slash possibly) are almost always used with かもしれない. }

\par{1. まだ ${\overset{\textnormal{}}{\text{家}}}$ へ ${\overset{\textnormal{}}{\text{帰}}}$ っていなかったかもしれない。 \hfill\break
He might have not been going home yet. }

\par{2. あんたにとっては、 ${\overset{\textnormal{ささい}}{\text{些細}}}$ なことかもけどな、でも ${\overset{\textnormal{}}{\text{俺}}}$ には ${\overset{\textnormal{}}{\text{大切}}}$ な ${\overset{\textnormal{}}{\text{問題}}}$ だぞ。(Masculine; Vulgar) \hfill\break
For you, it may like be trivial, but to me it's a like a big deal! }

\par{3. 彼はもう ${\overset{\textnormal{}}{\text{帰}}}$ ったかもな。 \hfill\break
He might have already come home. }

\par{4. ${\overset{\textnormal{にんげん}}{\text{人間}}}$ が ${\overset{\textnormal{}}{\text{全}}}$ ての ${\overset{\textnormal{せきゆ}}{\text{石油}}}$ を ${\overset{\textnormal{}}{\text{使}}}$ いきってしまう ${\overset{\textnormal{}}{\text{時}}}$ が\{ ${\overset{\textnormal{おとず}}{\text{訪}}}$ れる・ ${\overset{\textnormal{}}{\text{来}}}$ る\}かもしれません。 \hfill\break
The time may come when man will have used up all oil. }

\par{5. なるほど、 ${\overset{\textnormal{}}{\text{君}}}$ のいう ${\overset{\textnormal{}}{\text{通}}}$ りかもしれんね。 \hfill\break
Well then, you may be right. }

\par{6. 彼はひょっとしてまだ ${\overset{\textnormal{}}{\text{外}}}$ にいるでしょうか。 \hfill\break
Might he still be outside? }

\par{7. ひょっとしたら ${\overset{\textnormal{}}{\text{彼女}}}$ はここへ ${\overset{\textnormal{}}{\text{来}}}$ るだろう。 \hfill\break
She will possibly come here. }

\par{8. 前の ${\overset{\textnormal{しけん}}{\text{試験}}}$ が難しかったなら、 ${\overset{\textnormal{こんど}}{\text{今度}}}$ の試験も難しいかもしれません。 \hfill\break
If the previous exam was difficult, the next exam may also be difficult. }

\par{9. それだけの意味はあろうかもしれない。 \hfill\break
There might just be that much meaning. \hfill\break
By 堀辰雄. }

\par{\textbf{Grammar Note }: かもしれない\{だろう・でしょう\} would be weird because of doubling more or less the same thing, but it turns out that reversing this as such in the example is possible. However, this speech style is older. }

\par{10. もしかしたら ${\overset{\textnormal{}}{\text{彼}}}$ は ${\overset{\textnormal{}}{\text{気}}}$ が ${\overset{\textnormal{}}{\text{変}}}$ わるかもしれません。 \hfill\break
He might change his mind. }

\par{11. ひょっとすると ${\overset{\textnormal{}}{\text{雨}}}$ に ${\overset{\textnormal{あ}}{\text{遭}}}$ うかもしれません。 \hfill\break
Chances are that we'll meet rain. }

\par{\textbf{Alternative Note }: Possible as in "possible to do" is expressed with the adjective ${\overset{\textnormal{かのう}}{\text{可能}}}$ な. }

\par{${\overset{\textnormal{}}{\text{12. 計画}}}$ の ${\overset{\textnormal{じっこう}}{\text{実行}}}$ が ${\overset{\textnormal{}}{\text{可能}}}$ だ。 \hfill\break
The plan's execution is possible. }
    