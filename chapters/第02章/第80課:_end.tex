    
\chapter{End}

\begin{center}
\begin{Large}
第80課: End: ~終わる・終える, ~やむ, \& ~上がる・上げる 
\end{Large}
\end{center}
 
\par{ These endings will greatly aid in expanding your Japanese knowledge. }
      
\section{~終わる・終える}
 
\par{ "To end" is either ${\overset{\textnormal{お}}{\text{終}}}$ わる or 終える. The grammar is a bit confusing. Originally, \textbf{終わる was strictly intransitive and 終える was transitive }. Remember, intransitive verbs happen with no active agent but transitive verbs have a direct object . }

\par{ Transitive verbs are used with \textbf{を }, but intransitive verbs are typically used with \textbf{が }. 終える may be used as a transitive verb, but it is \textbf{rare }as an intransitive verb. The chart below will show the three main patterns with these verbs. }

\begin{ltabulary}{|P|P|P|}
\hline 

Particle & Verb & Meaning \\ \cline{1-3}

が & 終わる & Event ends naturally \\ \cline{1-3}

を & 終わる & To not end intently \\ \cline{1-3}

を & 終える & To end intently \\ \cline{1-3}

\end{ltabulary}

\par{1. ${\overset{\textnormal{じゅぎょう}}{\text{授業}}}$ が終わった。 \hfill\break
Class ended. }

\par{\textbf{Word Note }: クラスが終わった is wrong because クラス, contrary to how it's used in English, is used in reference to the group\slash body of the class. }

\par{${\overset{\textnormal{どうろ}}{\text{2. 道路}}}$ はここで終わっている。 \hfill\break
The road ends here. }

\par{\textbf{Tense Note }: Notice the use of the progressive tense in the sentence above. ~ている is used to show a continuous action. The road has and will continue to end there unless if something happened. Therefore, the non-past tense would be incorrect. }

\par{3. 先生は ${\overset{\textnormal{こうぎ}}{\text{講義}}}$ を終えましたか。 \hfill\break
Has the teacher concluded his lecture? }

\par{4. ${\overset{\textnormal{しっぱい}}{\text{失敗}}}$ に終わる。 \hfill\break
To end in failure. }

\par{5. 何時に終わりますか。 \hfill\break
At what time will it end? }

\par{6. 何時に終えますか。 \hfill\break
At what time will (they) end it? }

\par{7. それはどのように終わりましたか。 \hfill\break
How did to end up = How did it go? }

\par{8. その映画が死で終わる。 \hfill\break
That movie ends in death. }

\par{9. 昨夜の演奏は歌って終わった。 \hfill\break
Last night's performance ended with singing. }

\par{10. ${\overset{\textnormal{しあい}}{\text{試合}}}$ は引き分けに終わった。 \hfill\break
The game ended in a tie. }

\par{11. 動物の権利活動家らが抗議を終えましたが、反対の声をあげつづけるそうです。 \hfill\break
The animal rights activists ending the protest, but they are to continue raising voices of opposition. }

\par{12. ぼくは試験が全部終わったよ。ぼくの人生も終わったかもしれない。 \hfill\break
I'm finished with all of my exams! My life may have ended as well. }

\par{13. これで ${\overset{\textnormal{いっしょう}}{\text{一生}}}$ を終えるんだと思ってました。 \hfill\break
I thought that I would end my life with this. }

\par{\textbf{Word Note }: There are several words for life. ${\overset{\textnormal{せいかつ}}{\text{生活}}}$ for way of life, ${\overset{\textnormal{じんせい}}{\text{人生}}}$ for "human life", ${\overset{\textnormal{いのち}}{\text{命}}}$ for the force that keeps us alive, and 一生 is one's "whole life". }

\par{14. ${\overset{\textnormal{しゅうり}}{\text{修理}}}$ を終えましたか。 \hfill\break
Did you finish the repairs? }

\par{15. ${\overset{\textnormal{しょうがい}}{\text{生涯}}}$ が終わる。 = 死ぬ \hfill\break
To die. }

\par{\textbf{Definition Note }: This usage of the verb 終わる shows that one's life or role comes to an end. }

\par{16. 生涯を終える。 \hfill\break
To end one's life. }

\par{\textbf{Nuance Note }: This does not show intent. Otherwise, it would refer to suicide which it does not. }

\par{17. 宿題をおえたぞ。 (Masculine) \hfill\break
I finished my homework. }

\par{18. ${\overset{\textnormal{かいぎ}}{\text{会議}}}$ を終わる。 \hfill\break
To end a meeting (halfway). }

\par{19. 会議を終える。 \hfill\break
To \emph{end }a meeting. }

\par{20.  会議が終わる。 \hfill\break
The meeting ends. }

\par{\textbf{漢字 Note }: 了 and 卒 (which signifies graduation) may also be used instead of 終, but they're uncommon. }

\begin{center}
 \textbf{~終わる・終える }
\end{center}

\par{ In compounds ~終わる shows that an action ends or is completed. ~終わる is used with intransitive and transitive verbs, and it is the initial verb that decides the transitivity of the compound verb phrase. }

\par{ When you see ~終わる with transitive verbs, it shows that the subject of the sentence has finished something naturally or unintentionally in the sense that there is not necessarily a motive or goal for completion. For instance, you may finish paying off your student loan to college eleven years out of the fact. If you wanted to say you finished paying off your loan, you would use Ex. 21. }

\par{ ~終える signifies that one \textbf{ends }an action, and it must only be used with verbs of volition. Also, ~終える is more frequently seen in the written language, no doubt because one's intent is more powerfully expressed with it. }

\par{ These endings can't be with condition verbs like いる or できる. They can't attach to verbs of change or movement because there is no end point. }

\par{21. ローンを ${\overset{\textnormal{はら}}{\text{払}}}$ い終わりました。 \hfill\break
I finished paying off the loan. }

\par{22. 宿題をし終えて、外へ遊びに行った。 \hfill\break
I finished my homework and went outside to play. }

\par{23. ${\overset{\textnormal{とりひき}}{\text{取引}}}$ を終える。 \hfill\break
To close (as in the stock market). }

\par{24. その本を読み終えましたか。 \hfill\break
Did you finish reading that book? }

\par{25. 早くやり終える。 \hfill\break
To get it over with. }

\par{26. 彼は歌い終わりました。 \hfill\break
He finished singing. }

\par{27. 小説を書き終えた。 \hfill\break
I ended writing the novel. }

\par{28. 彼が意見を言い終わるか終わらないうちに、あの人は邪魔をして口を挟んだ。 \hfill\break
Before he could finish saying his opinion, that person interjected himself. }
      
\section{~やむ}
 
\par{${\overset{\textnormal{}}{\text{}}}$ む means "to end\slash stop" and is intransitive. Its counterpart that is used for personal action is 止める. In compounds やむ is only used with verbs that occur naturally. For anything else, use やむ or やめる (the transitive form of “to stop”) as independent verbs. }

\par{29. 雨が降り止んだ。 \hfill\break
雨がやんだ。 (More common) \hfill\break
It stopped raining. }

\par{30. ${\overset{\textnormal{おんがく}}{\text{音楽}}}$ が止みました。 \hfill\break
The music stopped. }

\par{31. なにかをやめることが必要かもしれません。 \hfill\break
It might be necessary to quit\slash stop (doing) something. }

\begin{center}
\textbf{To Quit Doing }
\end{center}

\par{"To quit\slash stop doing" is made with a nominalizer plus やめる. Change in habit itself is shown by the negative with "ようになる". }

\par{32. 彼は勉強するのをやめた。 \hfill\break
He stopped studying. }

\par{33. 彼はタバコを ${\overset{\textnormal{す}}{\text{吸}}}$ わないようになった。 \hfill\break
He stopped smoking. }

\par{34. 私はそれを表示できないようになってしまいました。 \hfill\break
I ended up not being able to display it. }

\par{\textbf{Meaning Note }: ~てしまう = To end up }
      
\section{~上がる・上げる}
 
\par{  上がる and 上げる are verbal pairs to express raising and lifting. Of course, there are many usages of these two words. 上がる is intransitive and 上げる is transitive. These broad meanings will make sense once you see how they are applied to actual compounds. }

\begin{ltabulary}{|P|P|}
\hline 

~上がる & ~上げる \\ \cline{1-2}

An action ends & To do exhaustively\slash to a limit \\ \cline{1-2}

An action is completed & To do an action completely \\ \cline{1-2}

 & Adds a sense of humility \\ \cline{1-2}

 & To show an explicit wording \\ \cline{1-2}

\end{ltabulary}
 
\par{35. 彼女は歌い上げました。 \hfill\break
She sang to the top of her voice. }

\par{36. ${\overset{\textnormal{しめい}}{\text{氏名}}}$ を読み上げる。 \hfill\break
To read out a name. }
 
\par{37. 何とぞよろしくお ${\overset{\textnormal{ねが}}{\text{願}}}$ い ${\overset{\textnormal{もう}}{\text{申}}}$ し上げます。 \hfill\break
I kindly ask for you to remember me on my behalf. }
 
\par{38. 本をすっかり書き上げた。 \hfill\break
I completely wrote up the book. }

\par{39. ${\overset{\textnormal{ぜんぶかぞ}}{\text{全部数}}}$ えあげたのか。(A little rough) \hfill\break
Did you count it all up? }
 
\par{40. 空が ${\overset{\textnormal{は}}{\text{晴}}}$ れ上がった。 \hfill\break
The sky cleared up. }

\par{41. ${\overset{\textnormal{や}}{\text{焼}}}$ きあがる。(Never transitive) \hfill\break
To finish burning. }
 
\par{42. いい食事が出来上がりました。 \hfill\break
A great meal is ready. }

\par{43. ${\overset{\textnormal{まちなか}}{\text{街中}}}$ が ${\overset{\textnormal{ふる}}{\text{震}}}$ え上がった。 \hfill\break
The town shuddered. }
    