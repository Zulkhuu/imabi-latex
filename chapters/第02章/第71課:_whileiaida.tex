    
\chapter{Absolute Time IV While I}

\begin{center}
\begin{Large}
第71課: Absolute Time IV: While I: 間 
\end{Large}
\end{center}
 
\par{ This lesson will be about the word 間. }
      
\section{間}
 
\par{${\overset{\textnormal{}}{\text{}}}$ literally means "between". This can be used in a physical or spacial sense. However, what we want to focus on here is how this is applied to temporal situations. 間 with time phrases means "while". The actions in the first and second clause respectively aren't necessarily happening at the same time. However, the latter action does happen within the time frame of the first. }

\par{ Following nouns, 間 should always be used with の because 間 itself is a noun. When following verbs to mean "while", the verb should be in the ~ている pattern for our purposes now. As you will see, you can attach 間 to nouns of condition, adjectives, and verbs. }

\par{\textbf{Particle Note }: If the events are simultaneously occurring, you should use the particle ながら instead. }

\par{\textbf{Tense Note }: For this section, we will consider only ~ている for when using verbs. We will later see in this lesson how to use ~ていた and return to the question of how to use ”Non-past + 間に". }

\begin{center}
 \textbf{Examples }
\end{center}
 
\par{1.食べている ${\overset{\textnormal{あいだ}}{\text{間}}}$ は、テレビを見ません。 \hfill\break
I don't watch TV while I eat. }
 
\par{2. しばらくの間、 ${\overset{\textnormal{どうが}}{\text{動画}}}$ を見ていた。 \hfill\break
I was watching a video\slash movie\slash clip for a while. }
 
\par{3. 宿題をしている間、ピザを食べた。 \hfill\break
I ate pizza while I did my homework. }
 
\par{4. ロンドンに ${\overset{\textnormal{たいざい}}{\text{滞在}}}$ している間にシェイクスピアの ${\overset{\textnormal{せいたんち}}{\text{生誕地}}}$ を ${\overset{\textnormal{おとず}}{\text{訪}}}$ れます。 \hfill\break
I'll visit Shakespeare's birthplace during my stay in London. }
 
\par{5. 夜の間に ${\overset{\textnormal{かざむ}}{\text{風向}}}$ きは東に変わった。 \hfill\break
The wind shifted to the east during the night. }
 
\par{\textbf{Reading Note }: 風向き may also be read as かぜむき. However, this is not as common and some speakers may think it is incorrect. }
 
\par{6. \{雨がザーザー ${\overset{\textnormal{ふ}}{\text{降}}}$ ってる・ ${\overset{\textnormal{どしゃぶ}}{\text{土砂降}}}$ りの\}間、中にいた。(Casual) \hfill\break
I was inside while it was pouring rain. }

\par{7. ${\overset{\textnormal{こっかい}}{\text{国会}}}$ は ${\overset{\textnormal{なつ}}{\text{夏}}}$ の間は ${\overset{\textnormal{きゅうかい}}{\text{休会}}}$ になりました。 \hfill\break
Congress adjourned for the summer. }
 
\par{8. しばらくの間、 ${\overset{\textnormal{であし}}{\text{出足}}}$ がよかった。 \hfill\break
There was a good turn-out for some time. }
 
\par{9. 二人の間を ${\overset{\textnormal{さ}}{\text{裂}}}$ くのはだめだよ! (The physical sense of 間) \hfill\break
Separating people is wrong! }
 
\par{\textbf{Grammar Note }: ~間に is used with instantial verbs while ~間 is used with durational verbs. }
 
\begin{center}
 \textbf{~ていた間に }
\end{center}
 
\par{ This form of the pattern does exist, but you should always pay close attention to differences when changing tense. For instance, ~た間に is \textbf{always bad }. Yet, regardless of whether ~ている or ~ていた is used, "A+間に+B" still shows that A and B were \textbf{during }the same time period. And, oddly, ~た is bad, ~ていた is OK. }
 
\par{10. ${\overset{\textnormal{かれし}}{\text{彼氏}}}$ が晩ご飯の ${\overset{\textnormal{したく}}{\text{支度}}}$ をして\{いる・いた\}間に、宿題をすませた。 \hfill\break
I finished my homework while my boyfriend was preparing dinner. }
 
\par{11. 子供が ${\overset{\textnormal{ひるね}}{\text{昼寝}}}$ をして\{いる・いた\}間に、本を読み終えた。 \hfill\break
While the kids were taking a nap, I finished reading \{a\slash the\} book. }
 
\par{ Though there isn't a  fundamental change in meaning, one slight nuance difference that you can get is that ~ていた間に is used in situations when the speaker treats A as already being a past event. }
 
\par{\textbf{Tense Notes }: }
 
\par{1. ~た間に is illogical because Action A would have then ended before the duration of Action B ever begins. You would be forcing an "after" statement on what should be a "while" statement. }
 
\par{2. There are cases where the non-past can be used with 間 which you will see later in this lesson. }

\begin{center}
 \textbf{~間: When to Use ~かん }
\end{center}

\par{ This suffix is not that easy to use. It depends largely on context and what time phrase you are attaching it to. }

\par{ For seconds and minutes, you'll see that there is not much of a difference whether you use ~間 or not. However, because it does demonstrate a "duration of time" more explicitly, it is a bit more formal. }

\par{12. 5 ${\overset{\textnormal{びょう}}{\text{秒}}}$ (間)待って! \hfill\break
Wait five seconds! }

\par{13. 5分(間)待って! \hfill\break
Wait five minutes! }

\par{14. 5時間待って! \hfill\break
Wait five hours! }

\par{15. ${\overset{\textnormal{じゅっ・じっ}}{\text{十}}}$ 週に 間待って! \hfill\break
Wait ten weeks! }

\par{16. ${\overset{\textnormal{ご}}{\text{5}}}$ ${\overset{\textnormal{かげつ}}{\text{ヶ月}}}$ (間)待って! \hfill\break
Wait five months! }

\par{17. 5年(間)待って! \hfill\break
Wait five years! }

\par{ Things get a little more tricky when you deal with ~前に. }

\par{十秒前に  〇    十分前に 〇 ${\overset{\textnormal{とおかまえ}}{\text{十日前}}}$ に 〇 ${\overset{\textnormal{じゅっかげつまえ}}{\text{十ヶ月前}}}$ に 〇  十年前に 〇 }

\par{ Only 週間 and 時間 can be used with ~前に. #週待つ is usually unnatural. However, when 週 is used in the context of 1週目 (first week), you can say things such as 5週待って! or 6週前に始まった. 時間 is acceptable presumably because it is not the same thing as ~時. ~時 refers to "o' clock" while ~時間 refers to "hours". So, although #時前に and #時間前に are possible, they're not the same.  }
    