    
\chapter{Transitivity I}

\begin{center}
\begin{Large}
第65課: Transitivity I: Different Transitive \& Intransitive Forms 
\end{Large}
\end{center}
 
\par{ Most verbs are usually either transitive or intransitive. Transitivity is a concept that helps explain what does what, what does what to whom, and similar questions related to action or state. }

\par{ T he Japanese terms for transitive verbs ( ${\overset{\textnormal{たどうし}}{\text{他動詞}}}$ ) and intransitive verbs ( ${\overset{\textnormal{じどうし}}{\text{自動詞}}}$ ) are defined below. As transitivity classifications don't always match with their English equivalent, the Japanese terms will be preferred in this discussion. }

\begin{ltabulary}{|P|P|}
\hline 

 \textbf{他動詞 }&  \emph{An action done by "someone\slash thing" on something or someone else (direct object). }\\ \cline{1-2}

 \textbf{自動詞 }&  \emph{An action or state that has no active agent and }\emph{no direct object. }\\ \cline{1-2}

\end{ltabulary}
\emph{}
\par{\textbf{Definition Note }: An "agent" in grammar is, simply put, the "doer" of an action. However, just because a verb has "no active agent," this doesn't mean that the verb has "no agent" at all. There are some verbs like ${\overset{\textnormal{く}}{\text{暮}}}$ ら す which do have an agent, but the agent is "non-active". After all, "people" do "live" out their lives. However, "living" is not the same thing as "cooking a meal" or "driving a car." As opposed to having a non-active agent, "cooking" and "driving" have active agents acting upon a direct object. }

\par{As alluded to above, some verbs don't have the same transitivity as their English equivalents. For example, ${\overset{\textnormal{}}{\text{分}}}$ かる (to understand) and ${\overset{\textnormal{}}{\text{要}}}$ る (to need) are intransitive in Japanese even though their English equivalents are transitive verbs. The reason for these discrepancies is that the two languages describe things from different angles in these instances. More will be discussed about this later. }

\begin{center}
\textbf{Rules of Thumb }
\end{center}

\par{ を with 他動詞 always marks the direct object of a sentence. If it is used with a 自動詞, it essentially means "through". }

\par{1. パソコンを捨てる。 (捨てる = 他動詞) \hfill\break
To throw away a PC. }

\par{2. 道を歩く。 歩く = 自動詞 \hfill\break
To walk through the street. }

\par{ If you don't see を and the subject is marked with が, it is often safe to assume that the verb is a 自動詞. However, the direct object could be dropped. So, you must take that into consideration. Is it logical for the verb in question to have a direct object? }

\par{3. 鳥が歌っている。   (歌う = 他動詞) \hfill\break
The bird is singing. }

\par{\textbf{Transitivity Note }: 歌う is a transitive verb because birds sing songs. There is intrinsically always a direct object implied. }

\par{4. 車が ${\overset{\textnormal{と}}{\text{停}}}$ ま った。 (停まる = 自動詞) \hfill\break
The car stopped. }
      
\section{自他動詞: Intransitive \& Transitive Verbs}
 
\par{ T here are upwards of 300 of these verb pairs in Japanese. Most fit nicely into four broad categories with various sub-types. Derivation becomes convoluted very quickly, however. Morphology doesn't always make a language easier. It just explains where things come from and why. As you go through the types of transitivity verb pairs, don't feel that you must memorize every detail being shown. After all, not even the best Japanese scholars have this completely figured out. }

\begin{center}
\textbf{\emph{Type 1: ある and おる = Intransitive }} 
\end{center}

\par{ About a fourth of pairs have an intransitive pair with an r or y suffix usually preceded by either the vowel \slash a\slash  or \slash o\slash  to the root of a verb (except for Type 1d below). The transitive form may end in -u, -eru, or -ru based on type. The sub-types will be listed by frequency. Frequency here is not frequency in type 1 alone but for ALL verb pairs of all types. }

\begin{ltabulary}{|P|P|P|P|P|P|}
\hline 

Type 1 & Frequency & Intransitive & Example & Transitive & Example \\ \cline{1-6}

a & ~20\% & √ -(w)ar-u & 上がる (to rise) & √-Ø-eru & 上げる (to raise) \\ \cline{1-6}

b & ~3\% &  √-ar-u &  ${\overset{\textnormal{ふさ}}{\text{塞}}}$ がる & √- Ø-u & 塞(ふさ)ぐ (to block) \\ \cline{1-6}

c &  ~1\% &  √-(w)ar-eru & 分かれる (to divide\slash branch) & √- Ø-eru & 分ける (to split) \\ \cline{1-6}

d &  ~.70\% & √-[y]-eru & 見える (to be in sight) & √- Ø-ru & 見る (to see) \\ \cline{1-6}

e & ~.33\% & √- o[y]-eru & 聞こえる (to be heard) & √- Ø-u & 聞く (to hear) \\ \cline{1-6}

f & ~.33\% & √-or-u &  ${\overset{\textnormal{つ}}{\text{積}}}$ も る (to pile) \hfill\break
& √- Ø-u & 積む (to pile) \\ \cline{1-6}

g & ~.33\% & √-or-eru &  ${\overset{\textnormal{う}}{\text{埋}}}$ も れる (to be covered) \hfill\break
& √- Ø-eru & 埋める (to bury\slash cover) \\ \cline{1-6}

\end{ltabulary}

\par{\textbf{Chart Note }:  Ø is used here to stand for "nothing" that takes the place for where a morpheme (meaning component) would otherwise be located to indicate transitivity explicitly. }

\par{ Even though there are seven sub-types, many patterns can be observed. The \slash ar-\slash  and \slash or-\slash  you see actually relate to the verbs ある and おる.The [y] in parentheses is actually silent and is etymologically the same as the \slash r\slash  in these derivations. }

\begin{center}
\textbf{\emph{Type 2: S = Transitive }}
\end{center}

\par{ These verbs all have a transitive form with a stem that ends in s. What else makes up the stem or what follows varies, but this characterizes 30\% of all pairs in Japanese. Frequency, again, refers to frequency among all verb pairs. }

\begin{ltabulary}{|P|P|P|P|P|P|}
\hline 

Type 2 & Frequency & Intransitive & Example & Transitive & Example \\ \cline{1-6}

a & ~14\% & √- Ø-eru &  ${\overset{\textnormal{ひ}}{\text{冷}}}$ え る (to get chilly) \hfill\break
& √-(y)as-u & 冷やす (to chill) \\ \cline{1-6}

b & ~10\% & √- Ø-u &  ${\overset{\textnormal{ち}}{\text{散}}}$ る (to scatter) \hfill\break
& √-(w)as-u & 散らす (to scatter) \\ \cline{1-6}

c & ~2\% & √- Ø-iru &  ${\overset{\textnormal{ひ}}{\text{干}}}$ る (to dry up) \hfill\break
& √-os-u &  ${\overset{\textnormal{ほ}}{\text{干}}}$ す (to dry) \hfill\break
\\ \cline{1-6}

d & ~1.3\% & √- Ø-iru & 生きる (to live) & √-as-u & 生かす (to keep alive) \\ \cline{1-6}

e & ~1\% & √- Ø-ru & 着る* (to wear) & √-s-eru & 着せる (to clothe) \\ \cline{1-6}

f & ~.70\% & √- Ø-eru &  ${\overset{\textnormal{ふく}}{\text{膨}}}$ れ る (to swell) \hfill\break
& √-as-eru & 膨らせる (to swell) \\ \cline{1-6}

g & ~.33\% & √- Ø-u &  ${\overset{\textnormal{およ}}{\text{及}}}$ ぶ (to extend) \hfill\break
& √-os-u & 及ぼす (to affect) \\ \cline{1-6}

h & ~.33\% & √- Ø-iru &  ${\overset{\textnormal{ほころ}}{\text{綻}}}$ びる (to come apart) \hfill\break
& √-as-eru & 綻ばせる (to break into) \\ \cline{1-6}

i & ~.33\% & √- Ø-iru &  ${\overset{\textnormal{つ}}{\text{尽}}}$ きる (to run out) \hfill\break
& √-us-u & 尽くす (to exhaust) \\ \cline{1-6}

\end{ltabulary}

\par{*: Although it is technically transitive, its meaning is more like an intransitive verb and this is important for other uses of the verb. \hfill\break
\textbf{Word Note }: 綻ばせる is "to break into" as in "to break into a smile. }

\begin{center}
 \textbf{\emph{Type 3: Once the Same Long Ago }}
\end{center}

\par{ Type 2 and Type 3 are extremely similar. Put together, they indicate that at one point, many transitivity verb pairs probably derived from a single verb which could function as either an intransitive or a transitive verb. This would be just how most English verbs work. As for Type 3, its verbs have morphemes expressed after the root indicating transitivity. Unsurprisingly, the "r\slash y-s" pattern is used for this. }

\begin{ltabulary}{|P|P|P|P|P|P|}
\hline 

Type 3 & Frequency & Intransitive & Meaning & Transitive & Meaning \\ \cline{1-6}

a & ~8\% & √-r-u &  ${\overset{\textnormal{あま}}{\text{余}}}$ る (to be plenty) \hfill\break
& √-s-u & 余す (to spare over) \\ \cline{1-6}

b & s ~6\% & √-r-eru &  ${\overset{\textnormal{あらわ}}{\text{現}}}$ れ る (to appear) \hfill\break
& √-s-u & 現す (to show\slash appear) \\ \cline{1-6}

c & r ~1\% & √-r-u & 乗る (to ride) & √-s-eru & 乗せる (to pick up) \\ \cline{1-6}

d & u ~.70\% & √-[y]-eru &  ${\overset{\textnormal{こ}}{\text{越}}}$ え る (to go over) \hfill\break
& √-s -u \hfill\break
& 越す (to go over) \\ \cline{1-6}

e & t ~.33\% & √-r-iru &  ${\overset{\textnormal{た}}{\text{足}}}$ り る (to suffice) \hfill\break
& √-s-u & 足す (to add) \\ \cline{1-6}

\end{ltabulary}

\begin{center}
 \textbf{\emph{Type 4: Change in Verb Class }}
\end{center}

\par{ For the above types, a morpheme of some sort is used to go from one transitivity to another. Adding a morpheme to the root of the verb frequently results in a verb that's in a different verb class than its counterpart. For instance, 上がる is a 五段 verb and 上げる is an 一段 verb. The addition of "-ar" to the root of these verbs "ag-" was all that was needed to cause this difference. }

\par{ For the remaining 25\% of verbs, change in verb class alone is what's responsible for transitivity change. Nothing is added to the roots of these verbs to change transitivity. In times past, the basic verb form for these verb pairs looked identical. The only distinguishing aspect they had was having different conjugations. This indicates that these verbs may be remnants of a far older process to derive transitivity pairs. Ironically, however, this type of verb pairs is split into two polar opposite sub-types. This is where memorization becomes especially important. }

\begin{ltabulary}{|P|P|P|P|P|P|}
\hline 

Type 4 & Frequency & Intransitive & Example & Transitive & Example \\ \cline{1-6}

 a &  ~16\% & √ - Ø-u &  ${\overset{\textnormal{あ}}{\text{開}}}$ く (to open) \hfill\break
& √ - Ø-eru &  開ける (to open) \\ \cline{1-6}

b & ~10\% & √- Ø-eru &  ${\overset{\textnormal{わ}}{\text{割}}}$ れ る (to crack) \hfill\break
& √- Ø-u & 割る (to crack) \\ \cline{1-6}

\end{ltabulary}

\par{\textbf{Exceptions Note: }There are several exceptions that are important to learn. Those will be found as examples in this lesson. }

\par{ \textbf{Examples }}

\par{ Sometimes knowing which verb form to use is not easy. For instance, suppose you have a cat named ニャ子ちゃん. To tell her to hide because a dog is coming, you would say かくれて, not かくして. This is because there is only an implied subject (the cat) and an action (hiding). There is no direct object. Had you wanted to say "hide yourself", you would use the transitive form かくす and get ${\overset{\textnormal{み}}{\text{身}}}$ を かくして. }

\par{The chart below has a handful of the most important pairs for you to learn. }

\begin{ltabulary}{|P|P|P|P|P|P|P|P|}
\hline 

意義 & 他動詞 & 意義 & 自動詞 & 意義 & 他動詞 & 意義 & 自動詞 \\ \cline{1-8}

To break & 壊す & To be broken & 壊れる & To open & 開ける & To open & 開く \\ \cline{1-8}

 To raise & 育てる \hfill\break
& To be raised \hfill\break
& 育つ \hfill\break
& To close & 閉める & To close & 閉まる \\ \cline{1-8}

To put out & 消す & To disappear & 消える & To flow & 流す & To flow & 流れる \\ \cline{1-8}

To drop & 落とす & To fall & 落ちる & To stand & 立てる & To stand & 立つ \\ \cline{1-8}

To stop & 止める & To stop & 止まる & To change & 変える & To change & 変わる \\ \cline{1-8}

To start\slash begin & 始める & To begin & 始まる & To reveal & 現す & To appear & 現れる \\ \cline{1-8}

To add & 加える & To take part\slash join & 加わる & To (a)wake &  ${\overset{\textnormal{さ}}{\text{覚}}}$ ま す \hfill\break
& To wake up & 覚める \\ \cline{1-8}

To leave & 残す & To remain & 残る & To rotate & 回す & To rotate & 回る \\ \cline{1-8}

\end{ltabulary}

\par{\textbf{Transitivity Notes }: }

\par{1. 開く read as あく is an intransitive verb. \hfill\break
2. 開ける read as あける is only transitive and is used to open many things. There are also other spellings of this verb depending on usage. When read as ひらける,  it is a more literary, intransitive verb for enlightenment, development, and things being opened up. For instance, you can say 目の前に海がひらけた (the sea opened up in front of our eyes) and 文明がひらけた (civilization has developed). }

\par{\textbf{Meaning Notes }: }

\par{1. 消える may also mean "to fade", "to go off\slash out", "to pass", "to vanish", etc. 消す may also mean "to turn off", "to remove", "to erase", "to vanish", "to extinguish", etc. \hfill\break
2. Etymologically, 立つ and 立てる are the same words as 建つ (to be built) and 建てる (to build) respectively. Translation of たつ and たてる, which have other spellings depending on usage, varies. However, you should get the theme of what they mean. \hfill\break
3. 止める read as とめる means "stop" as in "to halt".  It can be spelled as 停める (to halt a vehicle), 留める (to restrain; to hold (custody); to leave an impression), and  泊める (to accommodate\slash lodge). When read as やめる, it means "stop" as in stopping a condition\slash action. It can be spelled as 辞める (to quit a job). The intransitive form of やめる is やむ. }

\par{\textbf{Spelling Note }: Many of these words have additional meanings and additional spellings. Knowing each and every meaning of these verb pairs and the spellings to go with them is too overwhelming for now. }
\textbf{Examples }5. ${\overset{\textnormal{はた}}{\text{旗}}}$ を ${\overset{\textnormal{}}{\text{上}}}$ げる。 \hfill\break
To raise a flag.  
\par{${\overset{\textnormal{}}{\text{6. 車}}}$ が ${\overset{\textnormal{とつぜんと}}{\text{突然止}}}$ まった。 \hfill\break
The car suddenly stopped. }

\par{\textbf{漢字 Note }: 停まる = 止まる. The first is more indicative of a temporary stop. }
 
\par{${\overset{\textnormal{}}{\text{7. 弟}}}$ がカメラを ${\overset{\textnormal{こわ}}{\text{壊}}}$ しました。 \hfill\break
My younger brother broke the camera. }
 
\par{8. ぼくの新しいカメラが ${\overset{\textnormal{}}{\text{壊}}}$ れました。 \hfill\break
My new camera broke. }
 
\par{${\overset{\textnormal{}}{\text{9. 川}}}$ が ${\overset{\textnormal{}}{\text{}}}$ ${\overset{\textnormal{みずうみ}}{\text{湖}}}$ に ${\overset{\textnormal{}}{\text{流}}}$ れている。 \hfill\break
The river is flowing into the lake. }
 
\par{10. タクシーを ${\overset{\textnormal{}}{\text{止}}}$ めなかったの。 \hfill\break
You didn't stop a taxi? }
 
\par{11. ドアを ${\overset{\textnormal{}}{\text{開}}}$ けました。 \hfill\break
I opened the door. }
 
\par{12. ${\overset{\textnormal{じどう}}{\text{自動}}}$ ドア が ${\overset{\textnormal{}}{\text{開}}}$ きました。 \hfill\break
The automatic door opened. }
 
\par{13. いつ ${\overset{\textnormal{}}{\text{始}}}$ まりますか。 \hfill\break
When will it start? }

\par{14. ${\overset{\textnormal{き}}{\text{気}}}$ が ${\overset{\textnormal{}}{\text{変}}}$ わったかわからない。 \hfill\break
I don't know if [his\slash her] mind changed. }
 
\par{${\overset{\textnormal{}}{\text{15. 妹}}}$ がテレビをつけました。 \hfill\break
My little sister turned on the TV. }
 
\par{16. テレビがつきました。 \hfill\break
The TV came on. }

\par{17. ${\overset{\textnormal{なみだ}}{\text{涙}}}$ を ${\overset{\textnormal{}}{\text{流}}}$ す。 \hfill\break
To shed tears. }
 
\par{${\overset{\textnormal{}}{\text{18. 土}}}$ が ${\overset{\textnormal{ち}}{\text{血}}}$ で ${\overset{\textnormal{そ}}{\text{染}}}$ まった。 \hfill\break
The ground was dyed with blood. }

\par{19. ${\overset{\textnormal{しゅくだい}}{\text{宿題}}}$ を ${\overset{\textnormal{}}{\text{始}}}$ める。 \hfill\break
To start one's homework. }

\par{20. ${\overset{\textnormal{でんき}}{\text{電気}}}$ を ${\overset{\textnormal{け}}{\text{消}}}$ しました。 \hfill\break
I turned off the lights. }
 
\par{${\overset{\textnormal{}}{\text{21. 電気}}}$ が ${\overset{\textnormal{き}}{\text{消}}}$ えました。 \hfill\break
The lights [turned\slash went] off. }
 
\par{${\overset{\textnormal{}}{\text{22. \{話・話題\}}}}$ を ${\overset{\textnormal{}}{\text{変}}}$ える。 \hfill\break
To change topics. }
 
\par{${\overset{\textnormal{}}{\text{23. 彼}}}$ は ${\overset{\textnormal{はら}}{\text{腹}}}$ が ${\overset{\textnormal{}}{\text{立}}}$ った。(Set Phrase) \hfill\break
He got angry. }

\par{24. バラの木が ${\overset{\textnormal{う}}{\text{植}}}$ わっ ている。 \hfill\break
There are rose bushes planted. }

\par{25. バラの木を植えました。 \hfill\break
I planted a rose bush. }

\par{\textbf{Usage Note }: Sometimes a certain verb in a transitivity pair will not be frequently used. For instance, some speakers hardly ever use 植わる. }

\begin{center}
\textbf{More Practice }
\end{center}

\par{ The following sentences will have you choose one of two options. Use additional information and the contents of the lesson to help you choose the right answer. }

\par{\textbf{Questions }: }

\par{1. オレンジが(落ちる・落とす)。      Drop \hfill\break
2. 図書館の前に人が( ${\overset{\textnormal{なら}}{\text{並}}}$ ぶ・並べる)。    Line up \hfill\break
3. ${\overset{\textnormal{にちじ}}{\text{日時}}}$ を(決まる・決める)。        Decide \hfill\break
4. ${\overset{\textnormal{えだ}}{\text{枝}}}$ を(折れる・折る)。          Bend \hfill\break
5. 勉強を(続く・続ける)。           Continue \hfill\break
6. 草を( ${\overset{\textnormal{も}}{\text{燃}}}$ える・燃やす)。         Burn \hfill\break
7. 週末を(過ぎる・過ごす)。        Spend \hfill\break
8. 猫を(助かる・助ける)。         Save \hfill\break
9. 木が(倒れる・倒す)。          Fall down \hfill\break
10. 仕事が( ${\overset{\textnormal{ふ}}{\text{増}}}$ える・増やす)。        Increase \hfill\break
11. 数が( ${\overset{\textnormal{へ}}{\text{減}}}$ る・減らす)。          Decrease \hfill\break
12. 子どもを(育つ・育てる)。        Raise \hfill\break
13. トイレを(流れる・流す)。        Flush \hfill\break
14. ${\overset{\textnormal{かど}}{\text{角}}}$ を左に( ${\overset{\textnormal{ま}}{\text{曲}}}$ がる・曲げる)。       Turn \hfill\break
15. ひげを( ${\overset{\textnormal{の}}{\text{伸}}}$ びる・伸ばす)。        Grow long }

\par{\textbf{Curriculum Note }: Later on in a series of lessons, we will learn about verbs that do not change depending on transitivity. These verbs must be looked at on an individual basis, but because they each require a considerable amount of attention due to their complexity, this will conclude our studies of transitivity for now. }
      
\section{Exercises}
 
\par{Questions: }

\par{1. 落ちる  2. 並ぶ 3. 決める 4. 折る 5. 続ける 6. 燃やす 7. 過ごす 8. 助ける 9. 倒れる 10. 増える \hfill\break
11. 減る  12. 育てる 13. 流す 14. 曲がる 15. 伸ばす }
    