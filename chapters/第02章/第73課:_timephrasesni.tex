    
\chapter{Time Phrases + に}

\begin{center}
\begin{Large}
第73課: Time Phrases + に 
\end{Large}
\end{center}
 
\par{ When temporal nouns can be used with に and why is a matter that boggles the best minds of Japanese grammarians. It turns out that a lot of this decision depends on the more specific properties of groups of similar words and quirks on an individual basis. As the student, you will need to learn about what is certain so that you know when this relation exists or not. }
      
\section{Getting a Sense of the Problem}
 
\par{ In large part, many temporal nouns are not used with the particle に. For instance, just as you don\textquotesingle t say “at yesterday”, Japanese don\textquotesingle t say きのうに. Why, then, are there sentences like the following? }

\par{1. それを明日(に)することになっています。 \hfill\break
It has been decided that we will do that tomorrow. }

\par{2. 約束はいつにしましょうか。 \hfill\break
When shall we carry out the appointment? }

\par{ On the flip side, none of these sentences need に after the temporal nouns (which are in bold). Yet, as is the case with expressing a particular absolute point in time, に shows up obligatorily. So, if the context in which Ex. 1 was uttered were serious and particular, に would become obligatory. In reverse, if the contexts in which Ex. 2 was uttered were less serious and particular, に would become ungrammatical. }

\par{3. 私は8時に学校に行きます。 \hfill\break
I will go to school at 8 o\textquotesingle  clock. }

\par{4. どうぞお先に行ってください。 \hfill\break
Please go before (me). }

\par{5. 彼は、今年2013年(に)亡くなりました。〇 \hfill\break
He died this year, 2013. }

\par{6a. 彼は今年に亡くなりました。X\slash 〇 \hfill\break
6b. 彼は今年亡くなりました。〇 \hfill\break
He died this year. }

\par{\textbf{Sentence Note }: The grammaticality of 6a is based on whether there is context that would make 今年 more absolute. If the preceding sentence were say a question such as 「彼はいつ亡くなられたのですか」, it would become grammatical to say it. }

\par{ Just from a few examples, you may already be getting a sense at when に shows up. But, we still have to think of more definitive ways to explain all of this. }
      
\section{Non-temporal Usages of に}
 
\par{ The easiest usages of に that really aren't sources of any confusion is when rather than being part of a time phrase, に is a predicate phrase for some other reason. }

\par{7. ${\overset{\textnormal{いぜん}}{\text{以前}}}$ にもまして、彼女の ${\overset{\textnormal{かんこくご}}{\text{韓国語}}}$ は ${\overset{\textnormal{じょうたつ}}{\text{上達}}}$ している。 \hfill\break
Her Korean is improving more than before. }
 
\par{8. 昨日に引き続き今日は暑苦しい。 \hfill\break
Today is sultry continuing from yesterday. }
 
\par{9. 明日にかけて広い ${\overset{\textnormal{はんい}}{\text{範囲}}}$ で ${\overset{\textnormal{くも}}{\text{曇}}}$ るでしょう。 \hfill\break
It will be cloudy in a large area into tomorrow. }
 
\par{10. 仕事を来年に持ち ${\overset{\textnormal{こ}}{\text{越}}}$ さないほうがいい。 \hfill\break
It's best not to prolong work to the next year. }
 
\par{11. あさってにしよう。 \hfill\break
Let's do it the day after tomorrow. }
 
\par{12. 今日になってやっと ${\overset{\textnormal{あいけん}}{\text{愛犬}}}$ と ${\overset{\textnormal{さいかい}}{\text{再会}}}$ した! \hfill\break
After all this time, I finally united today with my beloved dog! }
 
\par{13. 明日に ${\overset{\textnormal{ひか}}{\text{控}}}$ えた最後の集会 \hfill\break
Last assembly postponed to tomorrow }
 
\par{14. きのう今日に始まったことではない。 (Set Phrase) \hfill\break
This is not a first. }
 
\par{15. 夜になっても、寝ないで。 \hfill\break
Don't even sleep at night. }
 
\par{ にでも, にも, and には are also significantly different and are not in the scope of the problem of this lesson. However, example sentences will be given. }
 
\par{16. 明日にでも仕事を ${\overset{\textnormal{や}}{\text{辞}}}$ めたい。 \hfill\break
I'd like to quit my job even tomorrow. }
 
\par{17. 明日にもプレゼントが ${\overset{\textnormal{とど}}{\text{届}}}$ きますよ。 \hfill\break
The present will be delivered possibly even tomorrow. }

\par{18. 奇跡 ${\overset{\textnormal{}}{\text{}}}$ の ${\overset{\textnormal{たまもの}}{\text{賜物}}}$ は今日にも当てはまるのですか。 \hfill\break
Do miraculous gifts apply to today's world? }
 
\par{19. 明日には明日の風が吹く。 \hfill\break
Tomorrow's wind will blow tomorrow. }
      
\section{Absolute, Relative, Intermediate, Referential}
 
\par{ First we need to take a bird\textquotesingle s eye view on the characteristics of temporal nouns. }

\par{ Representative temporal nouns that do not take に for time include 今日, 明日, and 昨日. This is typically explained by their relativity to now. These nouns have a range of time associated with them, and there is no specific point in time like 4:56 being expressed. They can be used with things like ~じゅう (throughout) and ~のあいだ, but they can\textquotesingle t be used with something like ~ごろ. }

\par{20. ${\overset{\textnormal{きょうじゅう}}{\text{今日中}}}$ に ${\overset{\textnormal{じゅんび}}{\text{準備}}}$ が出来ているはずです。 \hfill\break
Preparations should be finished during today. \hfill\break
21. ${\overset{\textnormal{あしたじゅう}}{\text{明日中}}}$ にやります。 \hfill\break
I will do it by (the end of) tomorrow. }

\par{ Representative temporal nouns that do take に for time include ~時, ~日, and ~月 phrases. These are most indicative of absolute time expressions. They consequently can\textquotesingle t be used with ~中 ~のあいだ. They do, however, work with 頃. }

\par{22. 何時ごろに寝るの? \hfill\break
At what time do you go to sleep? }

\par{23. 8時30分ごろに電話をください。 \hfill\break
Please call around 8:30. }

\par{24. 7月11日ごろに ${\overset{\textnormal{つゆあ}}{\text{梅雨明}}}$ けした。 \hfill\break
The rainy season ended around June 11. }

\par{ There are times when they aren\textquotesingle t used with に, but as is shown below, this is the case when you are showing which point in time a phenomenon in its entirety occurs, which differs greatly from when \textbf{an }action\slash event is done. It can also be said that without に, the time phrase becomes more emphasized. }

\par{25. この ${\overset{\textnormal{ちょうさせん}}{\text{調査船}}}$ は、先月 ${\overset{\textnormal{げじゅん}}{\text{下旬}}}$ にオーストラリアを出発して ${\overset{\textnormal{なんきょくかい}}{\text{南極海}}}$ で ${\overset{\textnormal{きこうへんどう}}{\text{気候変動}}}$ の ${\overset{\textnormal{かんそく}}{\text{観測}}}$ に ${\overset{\textnormal{あ}}{\text{当}}}$ たっていましたが、今月24日、 ${\overset{\textnormal{あつ}}{\text{厚}}}$ い ${\overset{\textnormal{こおり}}{\text{氷}}}$ に ${\overset{\textnormal{こうろ}}{\text{航路}}}$ を ${\overset{\textnormal{はば}}{\text{阻}}}$ まれ ${\overset{\textnormal{こうこう}}{\text{航行}}}$ できなくなりました。 \hfill\break
The research vessel left Australia late last month and was heading for measuring climate change in the Antarctic Sea, but its course was obstructed by thick ice this month on the 24 th , and it is now unable to pass through. }

\par{ Of course, there are those that are in between absolute and relative time phrases—intermediates—that at times can be with に and at other times can\textquotesingle t. Examples include the season words 春, 夏, 秋, and 冬. These can be used with ~中, ~の間, and ~ごろ. So, when they act more like relative time phrases, you can expect them to not be with に. }

\par{26. ${\overset{\textnormal{けんた}}{\text{健太}}}$ は一年生の間だけ家から ${\overset{\textnormal{つうがく}}{\text{通学}}}$ したが、二年になった春、 ${\overset{\textnormal{りょう}}{\text{寮}}}$ に住むことになった。 \hfill\break
Kenta commuted to school from home only while he was a first year student, but the spring he became a second year student, he began living in the dorms. }

\par{27. わたしの最初の小説を昭和50年の夏に出版しました。 \hfill\break
I published my first novel in the summer of Showa Year 50. }

\par{28. 去年の秋にバイトから首になった。 \hfill\break
I got my job cut from my part time job last fall. }

\par{ So, what are referential time phrases? As listed earlier, words like 翌日 are similar to words like 明日, these words are not rooted down in the present. Tomorrow is quite different than “the next day” in terms of usage. }

\par{29. 翌週会う約束をした。 \hfill\break
I made an appointment to meet the next week. }

\par{30. 翌日に家に帰る。 \hfill\break
To return home the next day. }

\par{31. ぼくは新しいケイタイを買って、その翌日に ${\overset{\textnormal{な}}{\text{失}}}$ くしちゃったんです。 \hfill\break
I bought a new cell-phone and lost it the next day. }
      
\section{The Heart of the Problem}
 
\par{ What we really want to know is if there is any relation between に and words like 今日, 明日, and 昨日. Consider 明日. Although it may be a relative time expression, it specifically refers to the day after "today." This quality makes these relative time phrases behave like demonstratives. Because of this, though, they don't usually go together well with actual demonstratives unless if the statement is figurative in some way. }

\par{32a. この今日、わたしは学校に行きます。X \hfill\break
32b. 今日、わたしは学校に行きます。〇 \hfill\break
I will go to school today. }

\par{33. その明日を目指す。 (Figurative) \hfill\break
To shoot for that tomorrow. }

\par{34. 自分はその明日 ${\overset{\textnormal{けいむしょ}}{\text{刑務所}}}$ へ行って... (Literary) \hfill\break
I went to the jail the next day\dothyp{}\dothyp{}\dothyp{} }

\par{35. それでは、彼は、その明日の ${\overset{\textnormal{かち}}{\text{価値}}}$ を ${\overset{\textnormal{ちょうみん}}{\text{町民}}}$ に知らせてまわっているのではないか。(やや ${\overset{\textnormal{こふう}}{\text{古風}}}$ ) \hfill\break
So then, is he not going about notifying the townspeople of the prices as of tomorrow? }

\par{36. この毎年 X \hfill\break
This every year X }

\par{ The same cannot be said for words like 日 and 夜, which are used frequently with demonstratives. Though they may be relative, their position on a timeline is not nearly as fixed as the words above. }

\par{37. その夜、彼は ${\overset{\textnormal{いがん}}{\text{胃癌}}}$ で ${\overset{\textnormal{な}}{\text{亡}}}$ くなりました。 \hfill\break
That night, he died from stomach cancer. }

\par{38. あの日々を忘れないでください。 \hfill\break
Do not forget about those days. }

\par{39. この年まで続く。 \hfill\break
To continue into this year. }
    