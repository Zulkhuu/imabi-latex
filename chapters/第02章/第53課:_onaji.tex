    
\chapter{Adjectives}

\begin{center}
\begin{Large}
第53課: Adjectives: Onaji 同じ 
\end{Large}
\end{center}
 
\par{ The adjective for same is \emph{onaji }同じ. As simple as this may be, conjugating it is not. It is for this reason that it has been avoided. Although we haven't looked into why things are the way they are, this is one instance where etymology is rather important to understand how it works. }
      
\section{Not the Same as Other Adjectives}
 
\par{ Long ago, \emph{onaji }was like all the other adjectives. It wasn\textquotesingle t the only one that ended in \slash ji\slash . All the other ones, though, evolved to end in \slash jii\slash  instead. It is for this reason we have adjectives like \emph{susamajii }凄まじい (fierce\slash tremendous) and \emph{mutsumajii }睦まじい (harmonious). Had it remained as onashi like it was even further back in time, it might be just like all the other adjectives, but now its conjugations are hybrid between adjectives and adjectival nouns. }

\begin{ltabulary}{|P|P|P|}
\hline 

Form & Plain Speech & Polite Speech \\ \cline{1-3}

Non-Past & 同じだ \hfill\break
\emph{Onaji da }& 同じです \hfill\break
\emph{Onaji desu }\\ \cline{1-3}

Past & 同じだった \hfill\break
\emph{Onaji datta }& 同じでした \hfill\break
\emph{Onaji deshita }\\ \cline{1-3}

Negative & 同じ\{では・じゃ\}ない \hfill\break
\emph{Onaji [de wa\slash ja] nai }& 同じ\{では・じゃ\}ないです \hfill\break
\emph{Onaji [de wa\slash ja] nai desu }\hfill\break
同じ\{では・じゃ\}ありません \hfill\break
\emph{Onaji [de wa\slash ja] arimasen }\\ \cline{1-3}

Negative Past & 同じ\{では・じゃ\}なかった \hfill\break
\emph{Onaji [de wa\slash ja] nakatta }& 同じ\{では・じゃ\}なかったです \hfill\break
\emph{Onaji [de wa\slash ja] nakatta desu }同じ\{では・じゃ\}ありませんでした \hfill\break
\emph{Onaji [de wa\slash ja] arimasendeshita }
\\ \cline{1-3}

Before Nouns & 同じ \hfill\break
\emph{Onaji }&  \\ \cline{1-3}

Before Nominalizer \emph{No }の & 同じな \hfill\break
\emph{Onaji na }&  \\ \cline{1-3}

Adverbial Form & 同じ \hfill\break
\emph{Onaji }\hfill\break
同じく \hfill\break
\emph{Onajiku }\hfill\break
同じに \hfill\break
\emph{Onaji ni }\hfill\break
同じように \hfill\break
\emph{Onaji yō ni }&  \\ \cline{1-3}

 \emph{Te }て Form & 同じで \hfill\break
\emph{Onaji de }& 同じでして \hfill\break
\emph{Onaji deshite }\\ \cline{1-3}

\end{ltabulary}

\par{ Although its conjugations are more or less an amalgam of adjectives and adjectival nouns, some of its forms deserve extra addition, and because this is the case, we will spend a little more time seeing how its various forms are used. }

\begin{center}
\textbf{Typical Conjugations }
\end{center}

\par{1. ${\overset{\textnormal{けっか}}{\text{結果}}}$ は ${\overset{\textnormal{おな}}{\text{同}}}$ じだったよ。 \hfill\break
 \emph{Kekka wa onaji datta yo. }\hfill\break
The results were the same. }

\par{2. ずっと ${\overset{\textnormal{おな}}{\text{同}}}$ じじゃないですか。 \hfill\break
 \emph{Zutto onaji ja nai desu ka? \hfill\break
 }Is it not always the same? }

\par{3. ${\overset{\textnormal{にほんご}}{\text{日本語}}}$ と ${\overset{\textnormal{おな}}{\text{同}}}$ じですね。 \hfill\break
 \emph{Nihongo to onaji desu ne. }\hfill\break
It\textquotesingle s the same as Japanese, huh. }

\par{4. これは ${\overset{\textnormal{にんげん}}{\text{人間}}}$ と ${\overset{\textnormal{おな}}{\text{同}}}$ じで、 ${\overset{\textnormal{いちばんおお}}{\text{一番多}}}$ いのは、ガンです。 \hfill\break
 \emph{Kore wa ningen to onaji de, ichiban ōi no wa, gan desu. }\hfill\break
This is the same as humans, and the most common is cancer. }

\par{5. ${\overset{\textnormal{きのう}}{\text{昨日}}}$ とおんなじだよ。 \hfill\break
 \emph{Kinō to on\textquotesingle naji da yo. }\hfill\break
It\textquotesingle s the same as yesterday. }

\par{\textbf{Pronunciation Note }: \emph{Onaji }may also emphatically be pronounced as \slash on\textquotesingle naji\slash . }

\begin{center}
\textbf{Before Nouns }
\end{center}

\par{6. ${\overset{\textnormal{かれ}}{\text{彼}}}$ は、 ${\overset{\textnormal{いち}}{\text{1}}}$ ${\overset{\textnormal{ねんかん}}{\text{年間}}}$ くらい ${\overset{\textnormal{おな}}{\text{同}}}$ じ ${\overset{\textnormal{ふく}}{\text{服}}}$ を ${\overset{\textnormal{き}}{\text{着}}}$ ていました。 \hfill\break
 \emph{Kare wa, ichinenkan kurai onaji fuku wo kite imashita. }\hfill\break
He wore the same clothes for about a year. }

\par{7. ほぼ ${\overset{\textnormal{おな}}{\text{同}}}$ じ ${\overset{\textnormal{いみ}}{\text{意味}}}$ です。 \hfill\break
 \emph{Hobo onaji imi desu. }\hfill\break
It has almost the same meaning. }

\par{8. ${\overset{\textnormal{ぼく}}{\text{僕}}}$ たちは ${\overset{\textnormal{おな}}{\text{同}}}$ じ ${\overset{\textnormal{がっこう}}{\text{学校}}}$ に ${\overset{\textnormal{かよ}}{\text{通}}}$ っています。(Male Speech) \hfill\break
 \emph{Bokutachi wa onaji gakkō ni kayotte imasu. }\hfill\break
We go to the same school. }

\par{9. ${\overset{\textnormal{みなおな}}{\text{皆同}}}$ じ ${\overset{\textnormal{あな}}{\text{穴}}}$ のムジナだな。 \hfill\break
Mina onaji ana no mujina da na. \hfill\break
They are all birds of a feather. }

\par{\textbf{Phrase Note }: The Japanese expression for this is literally “badgers of the same hole.” }

\par{10. ${\overset{\textnormal{わたし}}{\text{私}}}$ たちは ${\overset{\textnormal{おな}}{\text{同}}}$ じ ${\overset{\textnormal{とし}}{\text{年}}}$ に ${\overset{\textnormal{う}}{\text{生}}}$ まれました。 \hfill\break
 \emph{Watashitachi wa onaji toshi umaremashita. \hfill\break
 }We were born in the same year. }

\par{11. ${\overset{\textnormal{わたし}}{\text{私}}}$ は ${\overset{\textnormal{おな}}{\text{同}}}$ い ${\overset{\textnormal{どし}}{\text{年}}}$ の ${\overset{\textnormal{ひと}}{\text{人}}}$ と ${\overset{\textnormal{けっこん}}{\text{結婚}}}$ しています。 \hfill\break
 \emph{Watashi wa onaidoshi no hito to kekkon shite imasu. }\hfill\break
I\textquotesingle m married to someone of the same age. }

\par{\textbf{Phrase Note }: \emph{Onaidoshi }同い年 is a set phrase meaning “the same age.” }

\begin{center}
\textbf{Before Nominalizer \emph{No }の }
\end{center}

\par{ When the particle \emph{no }の is used in turning what precedes into a noun phrase for whatever reason, if \emph{onaji }同じ is what\textquotesingle s directly before it, \emph{na }な must intervene. We\textquotesingle ve already seen how \emph{no\slash n desu (ka) }の・んです(か) is used in sentences in seeking an answer, and we learned how \emph{na }な is needed with adjectival nouns. In this case, \emph{onaji }同じ functions like any other adjectival noun. }

\par{12. ${\overset{\textnormal{おんしつ}}{\text{音質}}}$ は、 ${\overset{\textnormal{シーディー}}{\text{CD}}}$ と ${\overset{\textnormal{おな}}{\text{同}}}$ じなのです。 \hfill\break
 \emph{Onshitsu wa, shiidii to onaji na no desu. }\hfill\break
That\textquotesingle s because the sound quality is the same as a CD. }

\par{13. ${\overset{\textnormal{なぜ}}{\text{何故}}}$ 、 ${\overset{\textnormal{わくせい}}{\text{惑星}}}$ の ${\overset{\textnormal{こうてんほうこう}}{\text{公転方向}}}$ は ${\overset{\textnormal{おな}}{\text{同}}}$ じなんですか。 \hfill\break
 \emph{Naze, wakusei no kōten hōkō wa onaji na n desu ka? }\hfill\break
Why is it that the direction of revolution for the planets is the same? }

\par{14. ${\overset{\textnormal{ふくそう}}{\text{服装}}}$ が ${\overset{\textnormal{おな}}{\text{同}}}$ じなので、 ${\overset{\textnormal{みわ}}{\text{見分}}}$ けがつきにくいですね。 \hfill\break
 \emph{Fukusō ga onaji na node, miwake ga tsukinikui desu ne. }\hfill\break
Their clothes are the same, so it\textquotesingle s hard to distinguish them, huh. }

\par{\textbf{Particle Note }: The particle \emph{node }ので is composed of the particle \emph{no }の and the \emph{te }て form of \emph{da }だ. Put together, \emph{node }のでmeans “because.” }

\begin{center}
\textbf{Adverbial Form: \emph{Onaji ni }同じに }
\end{center}

\par{15. ${\overset{\textnormal{わたしたち}}{\text{私達}}}$ の ${\overset{\textnormal{けっか}}{\text{結果}}}$ と ${\overset{\textnormal{どう}}{\text{同}}}$ じになりました。 \hfill\break
 \emph{Watashitachi no kekka to onaji ni narimashita. }\hfill\break
(Their results) became the same as our results. }

\par{16. ${\overset{\textnormal{ぜんぶ}}{\text{全部}}}$ ${\overset{\textnormal{おな}}{\text{同}}}$ じに ${\overset{\textnormal{み}}{\text{見}}}$ えます。 \hfill\break
 \emph{Zembu onaji ni miemasu. }\hfill\break
They all look the same. }

\par{17. ${\overset{\textnormal{じついん}}{\text{実印}}}$ と ${\overset{\textnormal{ぎんこういん}}{\text{銀行印}}}$ を ${\overset{\textnormal{どう}}{\text{同}}}$ じにすることはできるんですか。 \hfill\break
 \emph{Jitsuin to ginkōin wo onaji ni suru koto wa dekiru n desu ka? \hfill\break
 }Is it possible to have one\textquotesingle s registered seal the same as one\textquotesingle s bank transaction seal? }

\begin{center}
\textbf{Adverbial Form: \emph{Onaji yō ni }同じように }
\end{center}

\par{ This form has a meaning of “similarly\slash like.” }

\par{18. スパゲティをレシピと ${\overset{\textnormal{おな}}{\text{同}}}$ じように ${\overset{\textnormal{つく}}{\text{作}}}$ った。 \hfill\break
 \emph{Supageti wo reshipi to onaji yō ni tsukutta. }\hfill\break
I made spaghetti like the recipe. }

\par{19. ${\overset{\textnormal{ねこ}}{\text{猫}}}$ からは ${\overset{\textnormal{しゅうい}}{\text{周囲}}}$ が ${\overset{\textnormal{ひと}}{\text{人}}}$ と ${\overset{\textnormal{おな}}{\text{同}}}$ じように ${\overset{\textnormal{み}}{\text{見}}}$ えますか。 \hfill\break
 \emph{Neko kara wa shūi ga hito to onaji yō ni miemasu ka? }\hfill\break
Do one\textquotesingle s surroundings look the same from a cat\textquotesingle s perspective? }

\par{20. ${\overset{\textnormal{ふつう}}{\text{普通}}}$ の ${\overset{\textnormal{ひと}}{\text{人}}}$ と ${\overset{\textnormal{おな}}{\text{同}}}$ じように ${\overset{\textnormal{しごと}}{\text{仕事}}}$ が ${\overset{\textnormal{でき}}{\text{出来}}}$ ません。 \hfill\break
 \emph{Futsū no hito to onaji yō ni shigoto ga dekimasen. }\hfill\break
I can\textquotesingle t work like normal people. }

\par{21. ${\overset{\textnormal{かこ}}{\text{過去}}}$ と ${\overset{\textnormal{おな}}{\text{同}}}$ じように ${\overset{\textnormal{せってい}}{\text{設定}}}$ してください。 \hfill\break
 \emph{Kako to onaji yō ni settei shite kudasai. }\hfill\break
Please have settings be like they were in the past. }

\par{22. ${\overset{\textnormal{さる}}{\text{猿}}}$ たちはいつもと ${\overset{\textnormal{おな}}{\text{同}}}$ じように ${\overset{\textnormal{あそ}}{\text{遊}}}$ んでいます。 \hfill\break
 \emph{Sarutachi wa itsu mo to onaji yō ni asonde imasu. }\hfill\break
The monkeys are playing like they always do. }

\begin{center}
\textbf{Adverbial Form: A + \emph{Onajiku }同じく + B }
\end{center}

\par{ The form \emph{onajiku }同じく is reminiscent of when it was more like a regular adjective. In typical speech and writing, it isn't used that much, but when it is used, its meaning is most similar to "ditto" and "same as." }

\par{23. ${\overset{\textnormal{かれ}}{\text{彼}}}$ と ${\overset{\textnormal{おな}}{\text{同}}}$ じく ${\overset{\textnormal{わたし}}{\text{私}}}$ もイギリス ${\overset{\textnormal{しゅっしん}}{\text{出身}}}$ です。 \hfill\break
 \emph{Kare to onajiku watashi mo igirisu shusshin desu. \hfill\break
 }Ditto to him, I too am from England. }

\par{24. ${\overset{\textnormal{きのう}}{\text{昨日}}}$ と ${\overset{\textnormal{おな}}{\text{同}}}$ じく ${\overset{\textnormal{きょう}}{\text{今日}}}$ も ${\overset{\textnormal{あめ}}{\text{雨}}}$ だ。 \hfill\break
 \emph{Kinō to onajiku kyō mo ame da. }\hfill\break
Same as yesterday, today is also rain. }

\par{25a. ${\overset{\textnormal{き}}{\text{軌}}}$ を ${\overset{\textnormal{おな}}{\text{同}}}$ じくする。 \hfill\break
 \emph{Ki wo onajiku suru. }\hfill\break
25b. ${\overset{\textnormal{き}}{\text{軌}}}$ を ${\overset{\textnormal{いつ}}{\text{一}}}$ にする \hfill\break
 \emph{Ki wo itsu ni suru. }\hfill\break
To have the same way of doing. }

\par{26a. ${\overset{\textnormal{こころざし}}{\text{志}}}$ を ${\overset{\textnormal{おな}}{\text{同}}}$ じくする。 \hfill\break
 \emph{Kokorozashi wo onajiku suru. }\hfill\break
26b. ${\overset{\textnormal{たにん}}{\text{他人}}}$ と ${\overset{\textnormal{おな}}{\text{同}}}$ じように ${\overset{\textnormal{かん}}{\text{感}}}$ じる。 \hfill\break
 \emph{Tanin to onaji yō ni kanjiru. }\hfill\break
To be like-minded. }

\begin{center}
\textbf{Adverbial Form: \emph{Onaji }同じ }
\end{center}

\par{ When used in conjunction with the particle \emph{nara }なら, which creates “if” statements when making suggestions, \emph{onaji }同じ becomes an adverb meaning “anyhow” and is interchangeable with \emph{dōse }どうせ. }

\par{27. ${\overset{\textnormal{おな}}{\text{同}}}$ じ ${\overset{\textnormal{か}}{\text{買}}}$ うなら、 ${\overset{\textnormal{じもと}}{\text{地元}}}$ で ${\overset{\textnormal{か}}{\text{買}}}$ った ${\overset{\textnormal{ほう}}{\text{方}}}$ がいい。 \hfill\break
 \emph{Onaji kau nara, jimoto de katta hō ga ii. }\hfill\break
Anyhow, if you\textquotesingle re going to buy, it\textquotesingle s best to buy local. }

\par{\textbf{Grammar Note }: Verb + \emph{-ta hō ga ii }た方がいい is a grammatical pattern meaning “it\textquotesingle s best to…” }

\par{28. ${\overset{\textnormal{おな}}{\text{同}}}$ じ ${\overset{\textnormal{う}}{\text{売}}}$ るなら、 ${\overset{\textnormal{すこ}}{\text{少}}}$ しでも ${\overset{\textnormal{とく}}{\text{得}}}$ したい。 \hfill\break
 \emph{Onaji uru nara, sukoshi demo toku shitai. }\hfill\break
Anyhow, if I'm going to sell it, I want to get at least some profit. }

\par{\textbf{Grammar Note }: Verb + \emph{-tai (to omou) }たい(と思う)is a grammatical pattern meaning “want to…” }

\par{29. ${\overset{\textnormal{おな}}{\text{同}}}$ じお ${\overset{\textnormal{かね}}{\text{金}}}$ がかかるなら、 ${\overset{\textnormal{じぶん}}{\text{自分}}}$ の ${\overset{\textnormal{いえ}}{\text{家}}}$ を ${\overset{\textnormal{た}}{\text{建}}}$ てたいと ${\overset{\textnormal{おも}}{\text{思}}}$ う。 \hfill\break
 \emph{Onaji okane ga kakaru nara, jibun no ie wo tatetai to omou. }\hfill\break
Anyhow, if it's going to cost money, I\textquotesingle d want to build my own house. }

\begin{center}
\textbf{\emph{Dō- }同 } 
\end{center}

\par{ You will also discover a lot of words that start with the character 同 when read as \emph{dō }in Sino-Japanese compounds. Most of these words are literary, but this is not always the case. For example, the word for sympathy is \emph{dōjō }同情, which is simply the characters for “same” and “emotion” put together. }

\par{30. ${\overset{\textnormal{じょうき}}{\text{上記}}}$ 、 ${\overset{\textnormal{りようきやく}}{\text{利用規約}}}$ に ${\overset{\textnormal{どうい}}{\text{同意}}}$ します。 \hfill\break
 \emph{J }\emph{ōki, riyō kiyaku ni dōi shimasu. } \hfill\break
You agree to the above user agreement. }

\par{31. ${\overset{\textnormal{ゆかり}}{\text{由香里}}}$ さんの ${\overset{\textnormal{かれし}}{\text{彼氏}}}$ と ${\overset{\textnormal{わたし}}{\text{私}}}$ の ${\overset{\textnormal{おっと}}{\text{夫}}}$ は ${\overset{\textnormal{どういつじんぶつ}}{\text{同一人物}}}$ に ${\overset{\textnormal{ちが}}{\text{違}}}$ いありません! \hfill\break
 \emph{Yukari-san no kareshi to watashi no otto wa doitsu jimbutsu ni chigai arimasen! }\hfill\break
Yukari\textquotesingle s boyfriend and my husband are no doubt the exact same individual! }

\par{32. ${\overset{\textnormal{ぜんかい}}{\text{前回}}}$ と ${\overset{\textnormal{どうよう}}{\text{同様}}}$ に ${\overset{\textnormal{じっけん}}{\text{実験}}}$ を ${\overset{\textnormal{かいし}}{\text{開始}}}$ しました。 \hfill\break
 \emph{Zenkai to d }\emph{ōy }\emph{ō ni jikken wo kaishi shimashita. }\hfill\break
We began the experiment similarly to last time. }

\par{33. ${\overset{\textnormal{えいご}}{\text{英語}}}$ の ${\overset{\textnormal{どうおんいぎご}}{\text{同音異義語}}}$ を ${\overset{\textnormal{おし}}{\text{教}}}$ えてください。 \hfill\break
 \emph{Eigo no dō\textquotesingle on igigo wo oshiete kudasai. } \hfill\break
Please teach me English homophones. }

\par{34. ${\overset{\textnormal{かんごし}}{\text{看護師}}}$ は ${\overset{\textnormal{いしゃ}}{\text{医者}}}$ と ${\overset{\textnormal{どうとう}}{\text{同等}}}$ ですか。 \hfill\break
 \emph{Kangoshi wa isha to dōtō desu ka? }\hfill\break
Are nurses equivalent to doctors? }

\par{35. ${\overset{\textnormal{ひさいしゃ}}{\text{被災者}}}$ に ${\overset{\textnormal{どうじょう}}{\text{同情}}}$ します。 \hfill\break
 \emph{Hisaisha ni dōjō shimasu. }\hfill\break
I sympathize with disaster victims. }

\par{36. ${\overset{\textnormal{とうなん}}{\text{東南}}}$ アジアの ${\overset{\textnormal{くにぐに}}{\text{国々}}}$ も ${\overset{\textnormal{りんごくどうし}}{\text{隣国同士}}}$ は ${\overset{\textnormal{なか}}{\text{仲}}}$ が ${\overset{\textnormal{わる}}{\text{悪}}}$ いですか。 \hfill\break
 \emph{T }\emph{ōnan ajia no kuniguni mo ringoku dōshi wa naka ga warui desu ka? \hfill\break
 }Why is it that neighboring countries among Southeast Asian nations have such bad relations? }

\par{37. ${\overset{\textnormal{どうし}}{\text{同志}}}$ よ、 ${\overset{\textnormal{い}}{\text{行}}}$ くぞ。 \hfill\break
 \emph{Dōshi yo, iku zo! }\hfill\break
Comrades, let\textquotesingle s go! }

\par{\textbf{Phrase Note }: In Mandarin Chinese, 同志 means "homosexual." However, in Japanese, it is a somewhat uncommon word meaning “comrade.” }

\par{38. ${\overset{\textnormal{わたし}}{\text{私}}}$ は ${\overset{\textnormal{どうせい}}{\text{同性}}}$ と ${\overset{\textnormal{つ}}{\text{付}}}$ き ${\overset{\textnormal{あ}}{\text{合}}}$ っています。 \hfill\break
 \emph{Watashi wa dōsei to tsukiatte imasu. \hfill\break
 }I\textquotesingle m dating someone of the same sex. }

\par{39. ${\overset{\textnormal{わたし}}{\text{私}}}$ も ${\overset{\textnormal{まった}}{\text{全}}}$ く ${\overset{\textnormal{どうかん}}{\text{同感}}}$ です。 \hfill\break
 \emph{Watashi mo mattaku dōkan desu. } \hfill\break
I completely agree\slash feel exactly the same way. }

\par{ 40. ${\overset{\textnormal{がいこくじん}}{\text{外国人}}}$ の ${\overset{\textnormal{どうりょう}}{\text{同僚}}}$ と ${\overset{\textnormal{けんか}}{\text{喧嘩}}}$ になった。 \hfill\break
 \emph{Gaikokujin no dōryō to kenka ni natta. } \hfill\break
I got into an argument with a foreign colleague. }
    