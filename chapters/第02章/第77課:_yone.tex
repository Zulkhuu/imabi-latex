    
\chapter{語尾 I}

\begin{center}
\begin{Large}
第77課: 語尾 I: よ \& ね 
\end{Large}
\end{center}
 
\par{ ${\overset{\textnormal{ごび}}{\text{語尾}}}$ are fundamental to Japanese. In a broad sense, they're any ending at the end of a sentence, but we'll use 語尾 interchangeably with ${\overset{\textnormal{しゅうじょし}}{\text{終助詞}}}$ (final particles). The two most important ones are よ and ね. At their most basic understanding, よ emphasizing emotion, judgment, or assertion, and ね seeks agreement and\slash or a response from the addressee(s). }
 
\par{ These particles are best understand in regards to intonation and role of necessity. So, this lesson will first see why we sometimes have to use them and discuss what the actual uses are based on intonation. }
      
\section{よ \& ね}
 
\begin{center}
\textbf{自然さ }
\end{center}
 
\par{ よ is perhaps less important than ね because its absence does not cause a sentence to sound unnatural as often. As (over)use can sound rude to superiors, its use is not often positively taught. When you use よ, you are presenting something that the listener is supposed to then recognize. For instance, s entences like 「はい分かりましたよ」 and 「私がやりますよ」 can easily be interpreted as complaining, especially the first one, and at the very least, you could be showing frustration. You are telling the person that he should know that you already know. }
 
\par{  Of course, there are cases in which よ's absence is natural or unnatural. For instance, if you are telling something to someone that is beneficial to that person, the use of よ is natural. However, one must always consider the status of the listener and one's tone of voice. }
 
\par{1. おせんべい、どうぞ。美味しいですよ。 \hfill\break
Here are some rice crackers. They're delicious. }
 
\par{2. もう、帰っていいですよ。 \hfill\break
You can go home already.  }

\par{ Not using よ in these two sentences would show a lack of consideration towards the listener. }

\par{ Consider 「さあ、始めますよ」. The lack of よ doesn\textquotesingle t really change the meaning, but if you were to ask Japanese people whether 「さあ、始めるよ」 or 「さあ、始める」 is natural, they would say the former. Here\textquotesingle s another example. }

\par{3. 「 ${\overset{\textnormal{じゅぎょう}}{\text{授業}}}$ 、終わった?」「うん、終わったよ」 \hfill\break
"Has your class ended" "Yeah, it has" }

\par{ The lack of よ here would be very unnatural. }
 
\par{ Knowing how to use 語尾 is exceptionally hard if we were to only look at polite speech. Dialectical variation also makes them very difficult to acquire correctly. As 語尾 are essential to proper conversation, however, at any given level regardless of their frequency in any given speech style, we will take the time in this lesson to look at よ and ね in depth. }
 
\begin{center}
\textbf{Intonation }
\end{center}
 
\par{ When you are trying to tell someone that a response is needed, you see a slight rise in intonation in よ. It is important to note that よ never actually has an intonation ↑. Even when someone is shouting, their pitch is going in a direction more similar to ↗ because the vowel gets lengthened. This is one reason why this particle is allowed in many speech styles. }

\par{4. ${\overset{\textnormal{かみ}}{\text{髪}}}$ に何かついてますよ。 ↗ \hfill\break
Something's stuck in your hair. }
 
\par{5. それ、 ${\overset{\textnormal{ちょう}}{\text{超}}}$ ${\overset{\textnormal{から}}{\text{辛}}}$ いよ。 ↗ \hfill\break
That's gonna be hot (spicy). }

\par{ Giving new information in this way implies that the listener doesn't already know. So, you must always balance sounding pushy with being helpful. Of course, one's relation to the person must be taken into account for familiarity is a factor to the use of よ. Always know your audience, and realize that these usages are controlled by your control of tone. }
 
\par{ Sometimes this intonation is in fact used simply to show familiarity and closeness to someone. This clearly shows how よ is not just the equivalent of a verbal exclamation point.  }
 
\par{6. 本を持ってきてくれた? \hfill\break
うん、持ってきたよ。↗ \hfill\break
Did you bring the book for me? \hfill\break
Yeah, I brought it. }

\par{ A lowering ↘ in pitch can do one of four things. }

\begin{enumerate}

\item Criticize ignorance. 
\item Show some sort of strong emotion like surprise, disappointment or deep impression. 
\item Tell one\textquotesingle s emotional state with exclamation. 
\item Expressing one\textquotesingle s intent. 
\end{enumerate}
 
\par{\textbf{Usage Note }: These usages can easily be seen directed at oneself in ${\overset{\textnormal{ひと}}{\text{独}}}$ り ${\overset{\textnormal{ごと}}{\text{言}}}$ . }
 
\par{7. 今日は ${\overset{\textnormal{つか}}{\text{疲}}}$ れてるから、まっすぐ帰るよ。↘ \hfill\break
I'm tired today, so I'm going straight home. }
 
\par{8. がんばれよ。↘ \hfill\break
Come on. (Somewhat disappointed tone) }
 
\par{9. リーさんが ${\overset{\textnormal{てつだ}}{\text{手伝}}}$ ってくれたおかげで、早く終わったよ。↘ \hfill\break
Thanks to Lee-san helping, I got finished early. }
 
\par{10. やっぱりその本が見つからなかったよ。↘ \hfill\break
I unsurprisingly didn't find the book. }
 
\par{ If you want to add a kiddish nuance to things, you can alter this intonation to ↑↓, which means elongation of the particle. You could see this spelled as よう or よー. The latter spelling is more colloquial. Whether it is kiddish or not depends on context, but it certainly adds emotional appeal. }
 
\par{11. 曇ってるから、何にも見えないよう。↑↓ \hfill\break
Ahhh, we can't seen anything because it's cloudy\dothyp{}\dothyp{}\dothyp{}. }
 
\par{12. あいつがまたメールしたよ(う)。↑↓ \hfill\break
That guy's sent another e-mail\dothyp{}\dothyp{}\dothyp{}.. }

\begin{center}
 \textbf{ね }
\end{center}
 
\par{ When saying something which is deemed to be mutually understood, ね\textquotesingle s intonation goes up like ↑. This may also be used in response to an inquiry as well as for rejection. }

\par{13. ${\overset{\textnormal{あつ}}{\text{暑}}}$ いね。↑ \hfill\break
It's hot, huh. }
 
\par{14. 今日は人がよくいらっしゃいましたね。↑ \hfill\break
A lot of people came today, haven't they. }
 
\par{15. そうだね!↑ \hfill\break
Yeah, that's true! }
 
\par{16. 日本は今、何時? \hfill\break
 ええと、昼2時くらいだね。↑ \hfill\break
What time is it now in Japan? \hfill\break
Uh, it's about 2 in the afternoon. }
 
\par{17. なんとか ${\overset{\textnormal{たの}}{\text{頼}}}$ むよ。↘ \hfill\break
${\overset{\textnormal{}}{\text{嫌}}}$ だね。↑ \hfill\break
I need your help somehow. \hfill\break
Ah, I don't wanna. }
 
\par{ When used to seek confirmation or approval, ね\textquotesingle s intonation is ↗. }
 
\par{18. これが納豆だね。↗ \hfill\break
 \emph{This }is natto, right? }
 
\par{19. このペン、ちょっと借りるね。↗ \hfill\break
Let me borrow this pen just a moment, k? }

\par{20. ${\overset{\textnormal{とうちゃく}}{\text{到着}}}$ ${\overset{\textnormal{じかん}}{\text{時間}}}$ は午後4時50分ですね。↗ \hfill\break
The arrival time will be 4:50 PM, right? }
 
\par{ The patterns ↘ and ↑↓ are very similar in meaning for ね。 Going up and down in pitch doesn't make one sound more childish. Nor is the speaker seeking sympathy in either case. }
 
\par{ Both patterns may be used to show collective recognition with an added sense of emotion. In this case, the listener is not expected to respond. It\textquotesingle s possible to view such statements as being blurted out with not even much thought. This also applies to responding to inquiry. What is clear is that there is a heightened sense of emotion. }
 
\par{21. 着いたね。↑ \hfill\break
着いたね。↘・↑↓ \hfill\break
We've arrived. }
 
\par{ None of these intonations particularly means that the train had any problem, but the last two would certainly make sense if there were a problem. }
 
\par{22. そうですねえ。↘ \hfill\break
Yeah\dothyp{}\dothyp{}\dothyp{}.. }
 
\par{ This sentence shows some hesitation. This should make sense because a typical そうですね↑ is a response to having been sought agreement, but you\textquotesingle re adding more emotional emphasis, which would in this situation equate to not being so on board. }
 
\begin{center}
\textbf{Combinations }
\end{center}
 
\par{ In compounds with the two at the end, the pitch goes down. The first combination below is わよ, which is simply a feminine form of the ↘ よ. よね, on the other hand, is a combination which has a little bit more complicated status. There are more examples, but we will stick to these two for now. }
 
\par{23. このケーキ、美味しいわよ。 \hfill\break
This cake is delicious! }

\par{24. ${\overset{\textnormal{おか}}{\text{可笑}}}$ しいよね。 \hfill\break
That's so strange, no kidding? }
 
\par{ ~よね is relatively new. This combination particle has come about from a need to bring a close to one\textquotesingle s own opinion but see agreement\slash emotional appeal at the same. In contrast to a typical そうですね, そうですよね is used in anticipation of having people around you more involved. For those who think it is correct, they claim it shows more familiarity in colloquial speech and is used under the assumption that agreement is already had ( ${\overset{\textnormal{きょうゆう}}{\text{共有}}}$ ${\overset{\textnormal{にんしき}}{\text{認識}}}$ ). }
 
\begin{center}
\textbf{Using よ and ね with のだ }
\end{center}
 
\par{ のだ\textquotesingle s basic function is to get someone to understand something by bringing it to their attention. Adding ね has the tone seek sympathy as you try to have the listener understand things. Adding よ greatly emphasizes that you want to notify the listener of something. Repeating ~んだよover and over again, though, can lead you to sound verbose. How exactly overuse leads to unnaturalness in this situation is not quite clear, but it is likely due to the functions of んだ and よ being redundant to the point you aren\textquotesingle t actually using them to mean what they should, which leads to misuse. }

\par{${\overset{\textnormal{}}{\text{25. 彼}}}$ は ${\overset{\textnormal{}}{\text{銀行}}}$ に ${\overset{\textnormal{}}{\text{勤}}}$ めてるんだよね。(Not questioning) \hfill\break
He works in a bank, right? }

\par{${\overset{\textnormal{}}{\text{26. 一体何}}}$ があったんだよ?(Coarse; 男性語) \hfill\break
What the heck happened? }

\begin{center}
 \textbf{誤用 }
\end{center}

\par{ There are mistaken usages of よ and ね that even cause problems for native speakers. One of the best examples is when clerks try saying よろしいですよ to give permission but be cordial and respectful at the same time actually does the opposite and sounds rude. This stems from よ\textquotesingle s use of marking criticism. It also has to deal with how よろしい can be used, too. Because superiors can use it to show recognition or permission to underlings, using it in this way would be arrogant. }

\par{26. ${\overset{\textnormal{ぶちょう}}{\text{部長}}}$ :「よろしい。私が考えよう」 \hfill\break
Alright. I'll think about it. }

\par{ The replacement phrase for よろしいですよ would be はい、 ${\overset{\textnormal{しょうち}}{\text{承知}}}$ しました or はい、分かりました. }

\begin{center}
 \textbf{More Usages \& Examples }
\end{center}

\par{ You may also see ね used as a filler word with a ↘ intonation. }

\par{27. あのね、 ${\overset{\textnormal{ぼく}}{\text{僕}}}$ 、これがね、ほしいの。 \hfill\break
Um, I, I really want this you see. }

\par{${\overset{\textnormal{}}{\text{28. 明日}}}$ はね、 ${\overset{\textnormal{つごう}}{\text{都合}}}$ がいいんだ。 \hfill\break
Um, tomorrow is going to be convenient. }

\par{ When used after the て form to show a light, casual command with a ↑↓ intonation, ね can sound somewhat feminine.  It may also be used by both men and women when speaking to children or when trying to be cute.  }

\par{29. ここで ${\overset{\textnormal{}}{\text{待}}}$ ってね。 \hfill\break
Wait here, OK? }

\par{${\overset{\textnormal{}}{\text{30. 知}}}$ らないね。 \hfill\break
I don't know. }

\par{\textbf{Sentence Note }: This is a response to inquiry, and the tone can be most light with a ↘ intonation. }

\par{ かね is a combination that is \emph{often }heard by old speakers and by speakers of certain dialects, though it is not limited to these speakers. It is also often treated as a gender neutral form of かな (I wonder). }

\par{${\overset{\textnormal{}}{\text{31. 来}}}$ てくれるかね。 \hfill\break
Could you come over? }

\par{32. そんな ${\overset{\textnormal{}}{\text{手}}}$ にだれが ${\overset{\textnormal{}}{\text{乗}}}$ りますかね。 \hfill\break
I wonder who would drive in such transportation.  }

\par{${\overset{\textnormal{}}{\text{33. 彼女}}}$ に ${\overset{\textnormal{}}{\text{会}}}$ いたいよ。↗ \hfill\break
I want to see her! }

\par{${\overset{\textnormal{}}{\text{34. 早}}}$ くしないと ${\overset{\textnormal{おく}}{\text{遅}}}$ れますよ。↘  (Scolding) \hfill\break
You'll be late if you don't hurry. }

\par{35. ${\overset{\textnormal{じょうし}}{\text{上司}}}$ ${\overset{\textnormal{}}{\text{が思}}}$ いつきでものを ${\overset{\textnormal{}}{\text{言}}}$ うからまいるよ。↗ \hfill\break
My boss talks off the top of his head. }

\par{${\overset{\textnormal{}}{\text{36. 泣}}}$ かなくてもいいのよ。(Feminine)  ↗↘ \hfill\break
It's OK not to cry. }

\par{\textbf{Usage Note }: Final particles, again, can be used as filler words at the end of dependent clauses to add emotion. This doesn't apply as much to よ, but it can happen. The usage of these particles like this is not proper in truly formal polite situations. You might hear your boss use particles like ね in this fashion, but you wouldn't talk that way to him. So, this, like anything in Japanese that is sensitive to situation, you should do some investigation before attempting. }

\par{ Another use of よ is to call out for someone. We often see this is great pleads which we don't get to see in every day situations much but might in literature, dramas, etc. }

\par{37. 神よ、助けてくれ! \hfill\break
Oh God, save me! }

\par{\textbf{蝶よ花よ \&  月よ星よ }}

\par{ These phrases are used as adverbs followed by the particle と. Although literally "butterflies and flowers!" and "moon and stars!" respectively, they are used idiomatically to mean "bringing up like a princess" and "extremely loving and praising" respectively. The origin of these phrases is questionable, but that's not really important. }

\par{38. 娘を蝶よ花よと育てた。 \hfill\break
I raised my daughter like a princess. }

\par{39. 月よ星よと眺める。 \hfill\break
To gaze with great admiration.  }
    