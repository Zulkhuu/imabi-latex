    
\chapter{The Giving Verbs}

\begin{center}
\begin{Large}
第74課: The Giving Verbs 
\end{Large}
\end{center}
 
\par{ Japanese has specific verbs that define the relationship between the giver and recipient. Their usages with て are so important and particularly difficult that they will be covered separately. }
      
\section{The 3 Personal Giving Verbs}
 
\par{ The main verbs of giving are あげる, くれる, and もらう. They also have different forms. }

\begin{ltabulary}{|P|P|P|}
\hline 

Vulgar & Regular & Respectful \\ \cline{1-3}

やる & あげる & 差し上げる \\ \cline{1-3}

 & くれる & 下さる \\ \cline{1-3}

 & もらう & いただく \\ \cline{1-3}

\end{ltabulary}

\par{\textbf{Conjugation Note }: The ${\overset{\textnormal{れんようけい}}{\text{連用形}}}$ of ${\overset{\textnormal{}}{\text{下}}}$ さる has a sound change of り \textrightarrow  い with -ます. }
 
\par{あげる shows oneself giving something. It may also be used to show the giving of things by people in one's in-group to other (close) in-group members. This does not mean, though, that this should ever be used for when someone, even if they are in your in-group, gives something to you. }
 
\par{1. 父は母に時計をあげました。私も母にプレゼントをあげました。 \hfill\break
My father gave my mother a watch. I too gave my mother a present. }
 
\par{2a. 父は私にコンピューターゲームをあげました。X \hfill\break
2b. 父は私にコンピューターゲームをくれました。〇 \hfill\break
My dad gave me a computer game. }
 
\par{\textbf{Definitions Note }: あげる can also mean "to rise". Spelling depends on how it's used. やる may be a casual form of "to do", and it can also mean "to send; dispatch", "to kill", "to show (movie)", "to drink (alcohol)", "to suffer from", etc. }
 
\par{くれる means an outsider or a less close member gives something to the speaker or to a member of one's in-group. So, it is when someone gives something to either you or someone in your in group. It may also be used to describe either you or someone in your in-group doing something unpleasant to an opposing person. This is similar to "to let someone have\dothyp{}\dothyp{}\dothyp{}". }
 
\par{もらう and its variant means "to receive; get".  It may also mean "to welcome someone\slash something into the family". いただく is always said when receiving food, especially at the beginning of a meal. }
 
\par{\textbf{Particle Note }: \emph{For "to give", }\emph{に is used to mark the recipient while for "to receive", the giver is marked by }\emph{に. For the latter, }\emph{から may be used instead. } \emph{から, unlike }\emph{に, may also be used with "receive" aside from an actual person. }}
 
\par{3. 私は彼にプレゼントをあげた。 \hfill\break
I gave him a gift. }

\par{4. ${\overset{\textnormal{いっぱい}}{\text{一杯}}}$ やらないか。 \hfill\break
How about a drink? }

\par{5. ${\overset{\textnormal{しば}}{\text{芝}}}$ に水をやる。 \hfill\break
To water the lawn. }

\par{6. ${\overset{\textnormal{かとう}}{\text{加藤}}}$ さんは、 ${\overset{\textnormal{むすこ}}{\text{息子}}}$ に犬をプレゼントしてあげました。 \hfill\break
Mr. Kato gave a dog to his son as a present. }

\par{7. ${\overset{\textnormal{しゃちょう}}{\text{社長}}}$ さんは私に ${\overset{\textnormal{しょうかいじょう}}{\text{紹介状}}}$ を下さいました。 \hfill\break
The company president gave me a letter of introduction. }
 
\par{8. 犬に肉をやった。 \hfill\break
I gave meat to the\slash a dog. }

\par{9. ${\overset{\textnormal{むすめ}}{\text{娘}}}$ はお ${\overset{\textnormal{きゃく}}{\text{客}}}$ さまに ${\overset{\textnormal{はなたば}}{\text{花束}}}$ を ${\overset{\textnormal{さ}}{\text{差}}}$ し上げました。 \hfill\break
My daughter gave a bouquet to the guests. }
 
\par{10. やるせない \hfill\break
Helpless\slash miserable }
 
\par{11. アイディアをもらう。 \hfill\break
To borrow an idea. }
 
\par{12. この ${\overset{\textnormal{しょうぶ}}{\text{勝負}}}$ は ${\overset{\textnormal{われわれ}}{\text{我々}}}$ がもらった。 \hfill\break
We took this match. }
      
\section{Impersonal Giving Verbs}
 
\par{ There are other verbs in Japanese that mean "to give" but are not at a personal level. \textbf{受ける }is used in several situations for things received "outside of favor", and \textbf{与える }is generally used with things given "outside of a personal level". }

\begin{ltabulary}{|P|P|}
\hline 

受ける (Receive) \hfill\break
& 与える (Give) \hfill\break
\\ \cline{1-2}

Figurative, with things such as a ball or event. & Used in respect to animals. \\ \cline{1-2}

Used with things such as a job, education, etc. & Used when granting things. \\ \cline{1-2}

"To take\slash receive" an exam; often 受験する & Used with prizes. \\ \cline{1-2}

Used with bringing harm. &  \\ \cline{1-2}

\end{ltabulary}

\par{\textbf{Word Note }: ${\overset{\textnormal{}}{\text{受}}}$ ける may also be ${\overset{\textnormal{}}{\text{受}}}$ け ${\overset{\textnormal{}}{\text{止}}}$ める. It is used in the sense of "to take" as in "to take the news", "to take the events badly", etc. }
 
\par{${\overset{\textnormal{}}{\text{13. 彼}}}$ は ${\overset{\textnormal{しんこく}}{\text{深刻}}}$ に ${\overset{\textnormal{}}{\text{受}}}$ け ${\overset{\textnormal{}}{\text{止}}}$ めたね。 \hfill\break
He took it seriously, didn't he? }
 
\par{${\overset{\textnormal{}}{\text{14. 有名高校}}}$ を ${\overset{\textnormal{めざ}}{\text{目指}}}$ して ${\overset{\textnormal{なんかん}}{\text{難関}}}$ の ${\overset{\textnormal{しけん}}{\text{試験}}}$ を ${\overset{\textnormal{}}{\text{受}}}$ けた。 \hfill\break
I took a highly competitive exam aiming for the prestigious high school. }
 
\par{${\overset{\textnormal{}}{\text{15. 彼}}}$ は ${\overset{\textnormal{くび}}{\text{首}}}$ に ${\overset{\textnormal{けいしょう}}{\text{軽傷}}}$ を ${\overset{\textnormal{}}{\text{受}}}$ けた。 \hfill\break
He received a smaller injury on the neck. }

\par{16. ${\overset{\textnormal{はんけつ}}{\text{判決}}}$ を ${\overset{\textnormal{}}{\text{受}}}$ ける。 \hfill\break
To receive a sentence. }

\par{17. ${\overset{\textnormal{まず}}{\text{貧}}}$ しい ${\overset{\textnormal{}}{\text{人々}}}$ に ${\overset{\textnormal{}}{\text{与}}}$ える。 \hfill\break
Give to the poor. }

\par{18. ${\overset{\textnormal{たいふう}}{\text{台風}}}$ が ${\overset{\textnormal{}}{\text{与}}}$ えた ${\overset{\textnormal{そんがい}}{\text{損害}}}$ が ${\overset{\textnormal{}}{\text{多}}}$ い。 \hfill\break
There is a lot of damage caused (given) by the typhoon. }

\par{19. ${\overset{\textnormal{きょか}}{\text{許可}}}$ を ${\overset{\textnormal{}}{\text{与}}}$ えます。 \hfill\break
I will give permission. }
 
\par{20. いい ${\overset{\textnormal{せいせき}}{\text{成績}}}$ を ${\overset{\textnormal{}}{\text{与}}}$ えた。 \hfill\break
I gave (him) a good grade. }

\par{21. ${\overset{\textnormal{けんこうしんだん}}{\text{健康診断}}}$ を ${\overset{\textnormal{}}{\text{受}}}$ ける。 \hfill\break
To have health checks. }
    