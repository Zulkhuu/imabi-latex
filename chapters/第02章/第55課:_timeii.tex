    
\chapter{Counters IV (Time Part II)}

\begin{center}
\begin{Large}
第55課: Counters IV (Time: Part II): 時, 時間, 分, 秒, 晩, 夜 
\end{Large}
\end{center}
 
\par{ In part two of our coverage on temporal counters, we\textquotesingle ll learn how to tell time and understand a little bit more about time in general. }

\begin{center}
\textbf{Counters Covered in This Lesson }
\end{center}

\par{1.       - \emph{jikan }時間 \hfill\break
2.       - \emph{ji }時 \hfill\break
3.       - \emph{fun(kan) }分(間) \hfill\break
4.       - \emph{byō(kan) }秒(間) \hfill\break
5.       - \emph{ban }晩 \hfill\break
6.       - \emph{ya }\slash - \emph{yo }夜 }
      
\section{Hours}
 
\begin{center}
\textbf{- \emph{jikan }時間 }
\end{center}

\par{ The counter - \emph{jikan }時間 is used to count the number of hours. Do note that 4 must be read as “ \emph{yo }よ” with this counter. }

\begin{ltabulary}{|P|P|P|P|P|P|P|P|}
\hline 

1 & いちじかん & 2 & にじかん & 3 & さんじかん & 4 & よじかん \\ \cline{1-8}

5 & ごじかん & 6 & ろくじかん & 7 &  \textbf{ななじかん }\hfill\break
しちじかん & 8 & はちじかん \\ \cline{1-8}

9 & くじかん & 10 & じゅうじかん & 14 & じゅうよじかん & ? & なんじかん \\ \cline{1-8}

\end{ltabulary}

\par{\hfill\break
1. ${\overset{\textnormal{はねだくうこう}}{\text{羽田空港}}}$ に ${\overset{\textnormal{い}}{\text{行}}}$ くのに ${\overset{\textnormal{よ}}{\text{4}}}$ ${\overset{\textnormal{じかん}}{\text{時間}}}$ かかりました。 \hfill\break
 \emph{Haneda K }\emph{ūk }\emph{ō ni iku no ni yojikan kakarimashita. \hfill\break
 }It took four hours to go to Haneda Airport. }

\par{2. ${\overset{\textnormal{はくぶつかん}}{\text{博物館}}}$ で ${\overset{\textnormal{なんじかんす}}{\text{何時間過}}}$ ごしましたか。 \hfill\break
 \emph{Hakubutsukan de nanjikan sugoshimashita? \hfill\break
 }How many hours did you spend at the museum? }

\par{3. スタートから ${\overset{\textnormal{はち}}{\text{8}}}$ ${\overset{\textnormal{じかんけいか}}{\text{時間経過}}}$ しています。 \hfill\break
 \emph{Sut }\emph{āto kara hachijikan keika shite imasu. \hfill\break
 }Eight hours have passed since the start (of this). }

\par{4. ${\overset{\textnormal{まいにち}}{\text{毎日}}}$ 、 ${\overset{\textnormal{じゅう}}{\text{10}}}$ ${\overset{\textnormal{じかん}}{\text{時間}}}$ の ${\overset{\textnormal{べんきょう}}{\text{勉強}}}$ を ${\overset{\textnormal{いち}}{\text{1}}}$ ${\overset{\textnormal{ねんかんつづ}}{\text{年間続}}}$ けました。 \hfill\break
 \emph{Mainichi, j }\emph{ūjikan no benky }\emph{ō wo ichinenkan tsuzukemashita. \hfill\break
 }I have continued my ten-hour studies every day for a year. }

\par{5. ${\overset{\textnormal{ぼく}}{\text{僕}}}$ は ${\overset{\textnormal{まいしゅう}}{\text{毎週}}}$ 、アイヌ ${\overset{\textnormal{ご}}{\text{語}}}$ を ${\overset{\textnormal{じゅう}}{\text{10}}}$ ${\overset{\textnormal{じかん}}{\text{時間}}}$ \{くらい・ほど\} ${\overset{\textnormal{べんきょう}}{\text{勉強}}}$ しています。 \hfill\break
 \emph{Boku wa maish }\emph{ū, Ainugo wo j }\emph{ūjikan [kurai\slash hodo] benky }\emph{ō shite imasu. \hfill\break
 }I\textquotesingle m studying Ainu for about ten hours every week. }

\par{6. ${\overset{\textnormal{さくや}}{\text{昨夜}}}$ は ${\overset{\textnormal{ひさ}}{\text{久}}}$ しぶりに ${\overset{\textnormal{はち}}{\text{8}}}$ ${\overset{\textnormal{じかんね}}{\text{時間寝}}}$ ました! \hfill\break
 \emph{Sakuya wa hisashiburi ni hachijikan nemashita! \hfill\break
 }Last night, I slept eight years for the first time in a while! }

\par{7. あのファミレスも ${\overset{\textnormal{にじゅうよ}}{\text{24}}}$ ${\overset{\textnormal{じかんえいぎょう}}{\text{時間営業}}}$ をやめたんですね。 \hfill\break
 \emph{Ano famiresu mo nij }\emph{ūyojikan mo eigy }\emph{ō wo yameta n desu ne. \hfill\break
 }That family restaurant also quit their twenty-four service, huh. }

\begin{center}
\textbf{- \emph{ji }時 }
\end{center}

\begin{ltabulary}{|P|P|P|P|P|P|}
\hline 

0:00 & れいじ (零時) & 1:00 & いちじ & 2:00 & にじ \\ \cline{1-6}

3:00 & さんじ & 4:00 & よじ & 5:00 & ごじ \\ \cline{1-6}

6:00 & ろくじ & 7:00 &  \textbf{しちじ \hfill\break
 }ななじ & 8:00 & はちじ \\ \cline{1-6}

9:00 & くじ & 10:00 & じゅうじ & 11:00 & じゅういちじ \\ \cline{1-6}

12:00 & じゅうにじ & 13:00 & じゅうさんじ & 14:00 & じゅうよじ \\ \cline{1-6}

15:00 & じゅうごじ & 16:00 & じゅうろくじ & 17:00 &  \textbf{じゅうしちじ \hfill\break
 }じゅうななじ \\ \cline{1-6}

18:00 & じゅうはちじ & 19:00 & じゅうくじ & 20:00 & にじゅうじ \\ \cline{1-6}

21:00 & にじゅういちじ & 22:00 & にじゅうにじ & 23:00 & にじゅうさんじ \\ \cline{1-6}

24:00 & にじゅうよじ & ? & なんじ &  &  \\ \cline{1-6}

\end{ltabulary}

\par{ In Japan, military time is frequently used. Although phrases for “A.M” and “P.M” exist, we\textquotesingle ll first see how 0:00 through 24:00 are expressed with the counter - \emph{ji }時. Do note that 4 must be read as “ \emph{yo }よ” with this counter. }

\par{ As 0:00 and 24:00 are synonymous, speakers will typically default to 0:00 to refer to midnight. In colloquial speech, the words \emph{zero }ゼロ and \emph{nij }\emph{ūyon }にじゅうよん often replace the above expressions to refer to midnight. Above, you\textquotesingle ll notice that \emph{nanaji }ななじ and \emph{j }\emph{ūnanaji }じゅうななじ are possible readings for 7:00 and 17:00 respectively. These readings are dialectical at best, or pronunciations only appropriate when making oneself as clear as possible, but they are not viewed as the ‘correct\textquotesingle  readings of these phrases. }

\par{8. ${\overset{\textnormal{わたし}}{\text{私}}}$ たちは ${\overset{\textnormal{かぞく}}{\text{家族}}}$ ${\overset{\textnormal{よ}}{\text{4}}}$ ${\overset{\textnormal{にん}}{\text{人}}}$ で ${\overset{\textnormal{はち}}{\text{8}}}$ ${\overset{\textnormal{じ}}{\text{時}}}$ に ${\overset{\textnormal{こうえん}}{\text{公園}}}$ に ${\overset{\textnormal{い}}{\text{行}}}$ きました。 \hfill\break
 \emph{Watashitachi wa kazoku yonin de hachiji ni k }\emph{ōen ni ikimashita. \hfill\break
 }We went to the park at eight o\textquotesingle  clock as a family of four. }

\par{9. ${\overset{\textnormal{じゅうよ}}{\text{14}}}$ ${\overset{\textnormal{じ}}{\text{時}}}$ に ${\overset{\textnormal{お}}{\text{起}}}$ こしてください。 \hfill\break
 \emph{J }\emph{ūyoji ni okoshite kudasai. \hfill\break
 }Wake me up at 2 P.M. }

\par{12. ${\overset{\textnormal{いま}}{\text{今}}}$ 、 ${\overset{\textnormal{なんじ}}{\text{何時}}}$ ですか。 \hfill\break
 \emph{Ima, nanji desu ka? \hfill\break
 }What time is it now? }

\par{13. ほとんどの ${\overset{\textnormal{しゃいん}}{\text{社員}}}$ が ${\overset{\textnormal{じゅうしち}}{\text{17}}}$ ${\overset{\textnormal{じ}}{\text{時}}}$ に ${\overset{\textnormal{たいしゃ}}{\text{退社}}}$ した。 \hfill\break
 \emph{Hotondo no shain ga j }\emph{ūshichiji ni taisha shita. \hfill\break
 }Most of the company employees left work at 5 P.M. }

\par{14. ${\overset{\textnormal{まいあさ}}{\text{毎朝}}}$ 、 ${\overset{\textnormal{エヌエイチケー}}{\text{NHK}}}$ ${\overset{\textnormal{そうごう}}{\text{総合}}}$ テレビの ${\overset{\textnormal{しち}}{\text{7}}}$ ${\overset{\textnormal{じだい}}{\text{時台}}}$ のニュースを ${\overset{\textnormal{み}}{\text{見}}}$ ます。 \hfill\break
 \emph{Maiasa, enueichik }\emph{ē s }\emph{ōg }\emph{ō terebi no shichiji-dai no ny }\emph{ūsu wo mimasu. \hfill\break
 }Every morning, I watch the news from the 7 o\textquotesingle  clock hour on NHK General TV. }

\par{\textbf{Pronunciation Note }: NHK stands for “ \emph{Nippon H }\emph{ōs }\emph{ō Ky }\emph{ōkai }日本放送協会” and it is Japan\textquotesingle s national public broadcast organization. Many speakers alternatively pronounce NHK as “eneichikē.” }

\par{15. ${\overset{\textnormal{れいじ}}{\text{零時}}}$ にシャワーを ${\overset{\textnormal{あ}}{\text{浴}}}$ びました。 \hfill\break
 \emph{Reiji ni shaw }\emph{ā wo abimashita. \hfill\break
 }I took a shower at midnight? }

\par{16. ${\overset{\textnormal{なんじ}}{\text{何時}}}$ に ${\overset{\textnormal{かいてん}}{\text{開店}}}$ しますか。 \hfill\break
 \emph{Nanji ni kaiten shimasu ka? \hfill\break
 }At what time do you open? }

\par{17. バスは ${\overset{\textnormal{なんじ}}{\text{何時}}}$ に ${\overset{\textnormal{つ}}{\text{着}}}$ きますか。 \hfill\break
 \emph{Basu wa nanji ni tsukimasu ka? \hfill\break
 }At what time does the bus arrive? }

\par{18. ${\overset{\textnormal{く}}{\text{9}}}$ ${\overset{\textnormal{じ}}{\text{時}}}$ に ${\overset{\textnormal{あさ}}{\text{朝}}}$ ご ${\overset{\textnormal{はん}}{\text{飯}}}$ を ${\overset{\textnormal{た}}{\text{食}}}$ べて、 ${\overset{\textnormal{じゅうに}}{\text{12}}}$ ${\overset{\textnormal{じ}}{\text{時}}}$ に ${\overset{\textnormal{ひる}}{\text{昼}}}$ ご ${\overset{\textnormal{はん}}{\text{飯}}}$ 、 ${\overset{\textnormal{じゅうしち}}{\text{17}}}$ ${\overset{\textnormal{じ}}{\text{時}}}$ に ${\overset{\textnormal{はや}}{\text{早}}}$ めの ${\overset{\textnormal{ゆうしょく}}{\text{夕食}}}$ を ${\overset{\textnormal{と}}{\text{取}}}$ ることにしています。 \hfill\break
 \emph{Kuji ni asagohan wo tabete, j }\emph{ūniji ni hirugohan, j }\emph{ūshichiji ni hayame no y }\emph{ūshoku wo toru koto ni shite imasu. }\hfill\break
I\textquotesingle m trying to eat breakfast at nine, lunch at twelve, and take an earlier dinner at five. }

\par{19. ${\overset{\textnormal{わたし}}{\text{私}}}$ はたいてい ${\overset{\textnormal{しち}}{\text{7}}}$ ${\overset{\textnormal{じ}}{\text{時}}}$ に ${\overset{\textnormal{がっこう}}{\text{学校}}}$ に ${\overset{\textnormal{い}}{\text{行}}}$ きます。 \hfill\break
 \emph{Watashi wa taitei shichiji ni gakk }\emph{ō ni ikimasu. \hfill\break
 }I generally go to school at seven. }

\par{20. 私は16時から晩ご飯の支度とお風呂の掃除をして、18時に晩ご飯を食べます。 \hfill\break
Watashi wa jūrokuji kara bangohan no shitaku to ofuro no sōji wo shite, jūhachiji ni bangohan wo tabemasu. \hfill\break
I do preparations for dinner and clean the bath at four and then eat dinner at six. }

\begin{center}
\textbf{A.M \& P.M }
\end{center}

\par{ A.M and P.M are replaced by six expressions that come first in a time phrase. They may either be used as nouns or adverbs. As nouns, they refer to the time of day. }

\begin{ltabulary}{|P|P|P|P|}
\hline 
 
  5:00~11:00 
 &    \emph{Gozen }午前\slash  \emph{Asa }朝 
 &   12 PM 
 &    \emph{Hiru }昼 
 \\ \cline{1-4} 
 
  11:00~24:00 
 &    \emph{Gogo }午後 
 &   17:00~24:00 
 &    \emph{Yūgata }夕方 
 \\ \cline{1-4} 
 
  18:00~24:00 
 &    \emph{Yoru }夜 
 &   12 AM~4 AM 
 &    \emph{Shin\textquotesingle ya }深夜 
 \\ \cline{1-4} 
 
\end{ltabulary}

\par{\textbf{Word Note }: This process can be simplified by simply using \emph{gozen }午前 for a.m and \emph{gogo }午後 for p.m. }

\par{21. ${\overset{\textnormal{きょう}}{\text{今日}}}$ 、 ${\overset{\textnormal{わたし}}{\text{私}}}$ は ${\overset{\textnormal{あさ}}{\text{朝}}}$ ${\overset{\textnormal{しち}}{\text{7}}}$ ${\overset{\textnormal{じ}}{\text{時}}}$ に ${\overset{\textnormal{お}}{\text{起}}}$ きました。 \hfill\break
 \emph{Ky }\emph{ō, watashi wa asa shichiji ni okimashita. \hfill\break
 }I woke up today at seven in the morning. }

\par{22. グラウンドに ${\overset{\textnormal{ひる}}{\text{昼}}}$ ${\overset{\textnormal{じゅうに}}{\text{12}}}$ ${\overset{\textnormal{じ}}{\text{時}}}$ に ${\overset{\textnormal{しゅうごう}}{\text{集合}}}$ してください。 \hfill\break
 \emph{Guraundo ni hiru j }\emph{ūniji ni sh }\emph{ūg }\emph{ō shite kudasai. \hfill\break
 }Please gather at the sports ground at twelve noon. }

\par{23. ${\overset{\textnormal{ゆうがた}}{\text{夕方}}}$ ${\overset{\textnormal{はち}}{\text{8}}}$ ${\overset{\textnormal{じ}}{\text{時}}}$ に ${\overset{\textnormal{きたく}}{\text{帰宅}}}$ しました。 (Written language) \hfill\break
 \emph{Y }\emph{ūgata hachiji ni kitaku shimashita. \hfill\break
 }I returned home at eight in the evening. }

\par{24. ${\overset{\textnormal{わたし}}{\text{私}}}$ は ${\overset{\textnormal{ごぜん}}{\text{午前}}}$ ${\overset{\textnormal{はち}}{\text{8}}}$ ${\overset{\textnormal{じ}}{\text{時}}}$ に ${\overset{\textnormal{しゅっしゃ}}{\text{出社}}}$ しました。 \hfill\break
 \emph{Watashi wa gozen hachiji ni shussha shimashita. \hfill\break
 }I went\slash came to work at 8 A.M. }

\par{25. ${\overset{\textnormal{ごご}}{\text{午後}}}$ ${\overset{\textnormal{じゅう}}{\text{10}}}$ ${\overset{\textnormal{じ}}{\text{時}}}$ になりました。 \hfill\break
 \emph{Gozen j }\emph{ūji ni narimashita. \hfill\break
 }It\textquotesingle s 10 P.M. }

\par{26. ${\overset{\textnormal{よる}}{\text{夜}}}$ ${\overset{\textnormal{く}}{\text{9}}}$ ${\overset{\textnormal{じほうそう}}{\text{時放送}}}$ です! \hfill\break
 \emph{Yoru kuji h }\emph{ōs }\emph{ō desu! \hfill\break
 }This is a 9 P.M. broadcast! }

\par{27. ${\overset{\textnormal{わたし}}{\text{私}}}$ は ${\overset{\textnormal{しんや}}{\text{深夜}}}$ ${\overset{\textnormal{さん}}{\text{3}}}$ ${\overset{\textnormal{じ}}{\text{時}}}$ に ${\overset{\textnormal{ね}}{\text{寝}}}$ て、 ${\overset{\textnormal{あさ}}{\text{朝}}}$ ${\overset{\textnormal{はち}}{\text{8}}}$ ${\overset{\textnormal{じ}}{\text{時}}}$ に ${\overset{\textnormal{お}}{\text{起}}}$ きる ${\overset{\textnormal{せいかつ}}{\text{生活}}}$ を ${\overset{\textnormal{つづ}}{\text{続}}}$ けています。 \hfill\break
 \emph{Watashi wa shin\textquotesingle ya sanji ni nete, asa hachiji ni okiru seikatsu wo tsuzukete imasu. \hfill\break
 }I continue my life of sleeping at 3 A.M. and waking up at 8 A.M. }
      
\section{-fun(kan) 分(間)}
 
\par{ When counting minutes in the sense of duration, you use - \emph{fun(kan) }分(間). If you are expressing minutes as in time, you only use - \emph{fun }分. As such, below you\textquotesingle ll only see - \emph{fun }分 reflected in the chart. Just know that you can add - \emph{kan }間 so long as you\textquotesingle re not expressing the time. }

\begin{ltabulary}{|P|P|P|P|P|P|P|P|}
\hline 

1 & いっぷん & 2 & にふん & 3 & さんぷん & 4 & よんぷん \\ \cline{1-8}

5 & ごふん & 6 & ろっぷん & 7 & ななふん & 8 &  \textbf{はっぷん \hfill\break
 }はちふん \\ \cline{1-8}

9 & きゅうふん & 10 &  \textbf{じゅっぷん \hfill\break
 }じっぷん & 100 & ひゃっぷん & ? & なんぷん \\ \cline{1-8}

\end{ltabulary}

\par{\hfill\break
28. ${\overset{\textnormal{とうちゃく}}{\text{到着}}}$ は、 ${\overset{\textnormal{あす}}{\text{明日}}}$ の ${\overset{\textnormal{あさ}}{\text{朝}}}$ ${\overset{\textnormal{ろく}}{\text{6}}}$ ${\overset{\textnormal{じ}}{\text{時}}}$ ${\overset{\textnormal{よんじゅうご}}{\text{45}}}$ ${\overset{\textnormal{ふん}}{\text{分}}}$ です。 \hfill\break
 \emph{T }\emph{ōchaku wa, asu no asa rokuji yonj }\emph{ūgofun desu. \hfill\break
 }My\slash our arrival will be at 6:45 tomorrow morning. }

\par{29. ${\overset{\textnormal{しょたいめん}}{\text{初対面}}}$ ${\overset{\textnormal{さんじゅっ}}{\text{30}}}$ ${\overset{\textnormal{ふん}}{\text{分}}}$ で ${\overset{\textnormal{けっこん}}{\text{結婚}}}$ を ${\overset{\textnormal{き}}{\text{決}}}$ めた ${\overset{\textnormal{ふうふ}}{\text{夫婦}}}$ もいますよ。 \hfill\break
 \emph{Shotaimen sanjuppun de kekkon wo kimeta f }\emph{ūfu mo imasu yo. \hfill\break
 }There are also couples who decided to marry just from meeting each other for the first in thirty minutes. }

\par{30. ${\overset{\textnormal{じゅっ}}{\text{10}}}$ ${\overset{\textnormal{ふん}}{\text{分}}}$ でお腹の ${\overset{\textnormal{しぼう}}{\text{脂肪}}}$ を ${\overset{\textnormal{も}}{\text{燃}}}$ やすワークアウトをご ${\overset{\textnormal{しょうかい}}{\text{紹介}}}$ します。 \hfill\break
 \emph{Juppun de onaka no shib }\emph{ō wo moyasu wakuauto wo go-sh }\emph{ōkai shimasu. \hfill\break
 }I will introduce to you a workout to burn stomach fat in ten minutes. }

\par{31. ${\overset{\textnormal{にじゅっ}}{\text{20}}}$ ${\overset{\textnormal{ふん}}{\text{分}}}$ ( ${\overset{\textnormal{かん}}{\text{間}}}$ ) ${\overset{\textnormal{ま}}{\text{待}}}$ ちました。 \hfill\break
 \emph{Nijuppun(kan) machimashita. \hfill\break
 }I waited for twenty minutes. }

\par{32. ${\overset{\textnormal{よていどお}}{\text{予定通}}}$ り、 ${\overset{\textnormal{じゅうごじ}}{\text{15}}}$ : ${\overset{\textnormal{じゅうよんぷん}}{\text{14}}}$ に ${\overset{\textnormal{しながわえき}}{\text{品川駅}}}$ に ${\overset{\textnormal{とうちゃく}}{\text{到着}}}$ しました。 \hfill\break
 \emph{Yotei-dōri, jūgoji jūyonpun ni Shinagawa-eki ni tōchaku shimashita. \hfill\break
 }As planned, I\slash we arrived at Shinagawa Station at 3:14 P.M. }

\begin{center}
\textbf{Quarter and Half }
\end{center}

\par{ Like English, there are other phrases that can accompany “o\textquotesingle clock” phrases that roughly indicate minutes. For instance, using - \emph{han }半 refers to “half past.” There are also phrases that equate to “quarter past” and “quarter to.” However, unlike English, they simply indicate that a certain time has either already past or hasn\textquotesingle t past respectively. }
 
\begin{ltabulary}{|P|P|}
\hline 
 
  Half Past (\dothyp{}\dothyp{}\dothyp{}:30) 
 &   - \emph{han }半 
 \\ \cline{1-2} 
 
  (Quarter) Past (\dothyp{}\dothyp{}\dothyp{}:15) 
 &   - \emph{sugi }過ぎ 
 \\ \cline{1-2} 
 
  (Quarter) To (\dothyp{}\dothyp{}\dothyp{}:15) 
 &   - \emph{mae }前 
 \\ \cline{1-2} 
 
\end{ltabulary}
  \hfill\break
33. ${\overset{\textnormal{いま}}{\text{今}}}$ 、 ${\overset{\textnormal{ごぜん}}{\text{午前}}}$ ${\overset{\textnormal{しち}}{\text{7}}}$  ${\overset{\textnormal{じ}}{\text{時}}}$ \{ ${\overset{\textnormal{さんじゅっ}}{\text{30}}}$ ${\overset{\textnormal{ぷん}}{\text{分}}}$ ・ ${\overset{\textnormal{はん}}{\text{半}}}$ \}です。  \emph{\emph{Ima, gozen shichiji [sanjuppun\slash han] desu. } }It is 7:30 A.M. now.    34. ${\overset{\textnormal{ごご}}{\text{午後}}}$ ${\overset{\textnormal{さん}}{\text{3}}}$ ${\overset{\textnormal{じす}}{\text{時過}}}$ ぎに ${\overset{\textnormal{さつえい}}{\text{撮影}}}$ しました。  \emph{\emph{Gogo sanji-sugi ni satsuei shimashita. } }I photographed\slash filmed it a little past 3 P.M.    35. ${\overset{\textnormal{こども}}{\text{子供}}}$ の ${\overset{\textnormal{ね}}{\text{寝}}}$ る ${\overset{\textnormal{じかん}}{\text{時間}}}$ 、 ${\overset{\textnormal{よる}}{\text{夜}}}$ ${\overset{\textnormal{じゅう}}{\text{10}}}$ ${\overset{\textnormal{じまえ}}{\text{時前}}}$ に ${\overset{\textnormal{かえ}}{\text{帰}}}$ ってこないで。  \emph{Kodomo no neru jikan, yoru j }\emph{ūji-mae ni kaette konaide. \hfill\break
 }Don\textquotesingle t come home right before ten when the kids go to sleep.        
\section{-byō(kan) 秒(間)}
 
\par{ When counting seconds in the sense of duration, you use - \emph{byō(kan) }秒(間). If you are expressing seconds as in time, you only use - \emph{byō }秒. As such, below you\textquotesingle ll only see - \emph{byō }秒 reflected in the chart. Just know that you can add - \emph{kan }間 so long as you\textquotesingle re not expressing the time. }

\begin{ltabulary}{|P|P|P|P|P|P|P|P|}
\hline 

1 & いちびょう & 2 & にびょう & 3 & さんびょう & 4 & よんびょう \\ \cline{1-8}

5 & ごびょう & 6 & ろくびょう & 7 & ななびょう & 8 & はちびょう \\ \cline{1-8}

9 & きゅうびょう & 10 & じゅうびょう & 20 & にじゅうびょう & ? & なんびょう \\ \cline{1-8}

\end{ltabulary}

\par{\hfill\break
36. ${\overset{\textnormal{いっぷん}}{\text{一分}}}$ でも ${\overset{\textnormal{いちびょう}}{\text{一秒}}}$ でもいいから。 \hfill\break
 \emph{Ippun demo ichiby }\emph{ō demo ii kara. \hfill\break
 }I don\textquotesingle t care if it\textquotesingle s just for a minute or even a second. }

\par{37. ${\overset{\textnormal{さんじゅう}}{\text{30}}}$ ${\overset{\textnormal{びょうかんまたた}}{\text{秒間瞬}}}$ き ${\overset{\textnormal{きんし}}{\text{禁止}}}$ ! \hfill\break
 \emph{Sanj }\emph{ūby }\emph{ō matataki kinshi! \hfill\break
 }Blinking prohibited for 30 seconds! }

\par{38. トンネルを ${\overset{\textnormal{つうか}}{\text{通過}}}$ するのに ${\overset{\textnormal{じゅう}}{\text{10}}}$ ${\overset{\textnormal{びょう}}{\text{秒}}}$ かかりました。 \hfill\break
 \emph{Ton\textquotesingle neru wo ts }\emph{ūka suru no ni j }\emph{ūbyō kakarimashita. \hfill\break
 }It took ten seconds to pass through the tunnel. }

\par{39. ${\overset{\textnormal{あと}}{\text{後}}}$ ${\overset{\textnormal{ひゃく}}{\text{100}}}$ ${\overset{\textnormal{びょう}}{\text{秒}}}$ の ${\overset{\textnormal{あいだ}}{\text{間}}}$ に ${\overset{\textnormal{なんこ}}{\text{何個}}}$ の ${\overset{\textnormal{つ}}{\text{積}}}$ み ${\overset{\textnormal{き}}{\text{木}}}$ を ${\overset{\textnormal{かたづ}}{\text{片付}}}$ けられるか ${\overset{\textnormal{ため}}{\text{試}}}$ してみてください。 \hfill\break
 \emph{Ato hyakuby }\emph{ō no aida ni nanko no tsumiki wo katazukerareru ka tameshite mite kudasai. \hfill\break
 }Try seeing how many building blocks you can put up in the 100 seconds left. }
      
\section{Evenings}
 
\begin{center}
\textbf{- \emph{ban }晩 }
\end{center}

\par{ - \emph{ban }晩 is used to count nights, but it is only used for the following numbers. Typically, this counter is used to count nights as in “stays.” As such, you ought to replace it with - \emph{haku }泊 when not literally counting nights once you pass 2. }

\begin{ltabulary}{|P|P|P|P|P|P|P|P|}
\hline 

1 & ひとばん & 2 & ふたばん & 3 & みばん & ? & いくばん \\ \cline{1-8}

\end{ltabulary}

\par{\textbf{Usage Note }: \emph{Miban }みばん is only used in the set phrase \emph{mikka miban }三日三晩, which means “three days and three nights.” }

\par{40. ${\overset{\textnormal{ひなんじょ}}{\text{避難所}}}$ に ${\overset{\textnormal{ひとばんと}}{\text{一晩泊}}}$ まりました。 \hfill\break
 \emph{Hinanjo ni hitoban tomarimashita. \hfill\break
 }I stayed one night at the shelter. }

\par{41. ${\overset{\textnormal{こんかい}}{\text{今回}}}$ は ${\overset{\textnormal{かぞく}}{\text{家族}}}$ ${\overset{\textnormal{ろく}}{\text{6}}}$ ${\overset{\textnormal{にん}}{\text{人}}}$ で ${\overset{\textnormal{ふたばんと}}{\text{二晩泊}}}$ まりました。 \hfill\break
 \emph{Konkai wa kazoku rokunin de futaban tomarimashita. \hfill\break
 }This time, we stayed two nights as a family of six. }

\par{42. ${\overset{\textnormal{ふたばんてつや}}{\text{二晩徹夜}}}$ したよ。 \hfill\break
 \emph{Futaban tetsuya shita yo. \hfill\break
 }I stayed up for two nights. }

\par{43. ${\overset{\textnormal{ふたばん}}{\text{二晩}}}$ で ${\overset{\textnormal{いた}}{\text{痛}}}$ みもなくなった。 \hfill\break
 \emph{Futaban de itami mo nakunatta. \hfill\break
 }The pain also went away in two nights. }

\par{44. ${\overset{\textnormal{つよ}}{\text{強}}}$ いショックを ${\overset{\textnormal{う}}{\text{受}}}$ けて ${\overset{\textnormal{ひとばん}}{\text{一晩}}}$ で ${\overset{\textnormal{しらが}}{\text{白髪}}}$ になった。 \hfill\break
 \emph{Tsuyoi shokku wo ukete hitoban de shiraga ni natta. \hfill\break
 }I received a great shock and became gray-haired in a night. }

\par{45. ${\overset{\textnormal{ひとばん}}{\text{一晩}}}$ で ${\overset{\textnormal{ねつ}}{\text{熱}}}$ が ${\overset{\textnormal{さ}}{\text{下}}}$ がりました。 \hfill\break
 \emph{Hitoban de netsu ga sagarimashita. \hfill\break
 }My fever went down in one night. }

\begin{center}
\textbf{- \emph{ya }\slash - \emph{yo }夜 }
\end{center}

\par{ The word for night is \emph{yoru }夜. To count nights in a literal fashion, you use the counter - \emph{ya }夜. For 1 and 2, there is also the counter - \emph{yo }夜; however, it is limited to the written language and seldom used in the spoken language. }

\begin{ltabulary}{|P|P|P|P|P|P|P|P|}
\hline 

1 & いちや \hfill\break
ひとよ & 2 & にや \hfill\break
ふたよ & 3 & さんや & 4 & よんや \\ \cline{1-8}

5 & ごや & 6 & ろくや & 7 & ななや & 8 & はちや \\ \cline{1-8}

9 & きゅうや & 10 & じゅうや & 11 & じゅういちや & ? & なんや \\ \cline{1-8}

\end{ltabulary}

\par{\hfill\break
46. ${\overset{\textnormal{いちや}}{\text{一夜}}}$ が ${\overset{\textnormal{あ}}{\text{明}}}$ けた。 \hfill\break
 \emph{Ichiya ga aketa. \hfill\break
 }The following morning dawned. \hfill\break
Literally: One night ended. }

\par{47. ${\overset{\textnormal{だんせい}}{\text{男性}}}$ は ${\overset{\textnormal{ゆき}}{\text{雪}}}$ が ${\overset{\textnormal{つ}}{\text{積}}}$ もった ${\overset{\textnormal{あな}}{\text{穴}}}$ の ${\overset{\textnormal{そこ}}{\text{底}}}$ で ${\overset{\textnormal{ご}}{\text{5}}}$ ${\overset{\textnormal{やす}}{\text{夜過}}}$ ごした。 \hfill\break
 \emph{Dansei wa yuki ga tsumotta ana no soko de goya sugoshita. \hfill\break
 }The man spent five nights at the bottom of a whole covered by snow. }

\par{48. ${\overset{\textnormal{かた}}{\text{硬}}}$ いベッドマットで ${\overset{\textnormal{さん}}{\text{3}}}$ ${\overset{\textnormal{や}}{\text{夜}}}$ ${\overset{\textnormal{す}}{\text{過}}}$ ごした ${\overset{\textnormal{けいけん}}{\text{経験}}}$ があります。 \hfill\break
 \emph{Katai beddomatto de san\textquotesingle ya sugoshita keiken ga arimasu. \hfill\break
 }I\textquotesingle ve had the experience of spending three nights on a hard bedmat. \hfill\break
 \hfill\break
49. ${\overset{\textnormal{たび}}{\text{旅}}}$ の ${\overset{\textnormal{にもつ}}{\text{荷物}}}$ を ${\overset{\textnormal{せいり}}{\text{整理}}}$ して ${\overset{\textnormal{じゅう}}{\text{10}}}$ ${\overset{\textnormal{や}}{\text{夜}}}$ 過ごした部屋を後にした。 \hfill\break
 \emph{Tabi no nimotsu wo seiri shite j }\emph{ūya sugoshita heya wo ato ni shita. \hfill\break
 }I gathered up my luggage to my journey and left the room I had spent ten nights in. }

\par{ 50. ${\overset{\textnormal{あき}}{\text{秋}}}$ の ${\overset{\textnormal{ひとよ}}{\text{一夜}}}$ をゆったりと ${\overset{\textnormal{す}}{\text{過}}}$ ごしてみませんか。 \hfill\break
 \emph{Aki no hitoyo wo yuttari to sugoshite mimasen ka? \hfill\break
 }Why not spend an autumn night at ease? }
    