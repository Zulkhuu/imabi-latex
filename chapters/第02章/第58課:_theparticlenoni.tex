    
\chapter{The Particle のに}

\begin{center}
\begin{Large}
第58課: The Particle のに 
\end{Large}
\end{center}
 
\par{ This particle is very similar in grammar to ので. However, it serves a different function. }
      
\section{のに}
 
\par{ ~のに shows that the latter statement differs from the former, "although". Again, the second clause has to be contrary to what is expected from the first clause. If not, ~が or ~ても should be used. It is not uncommon to see it used as a final particle. \hfill\break
}

\begin{ltabulary}{|P|P|P|}
\hline 

Nouns & 犬 \textbf{な }のに & Must use copula \\ \cline{1-3}

Adjectives & 新し \textbf{い }のに & Must not use copula \\ \cline{1-3}

Adjectival Nouns & 簡単なのに & Must use copula \\ \cline{1-3}

Verbs & 食べ \textbf{る }のに & Must not use copula \\ \cline{1-3}

\end{ltabulary}

\par{\textbf{Grammar Note }: ~だのに is used by some speakers and is viewed as being rather dialectical. }

\par{1. あら、2 ${\overset{\textnormal{}}{\text{杯}}}$ も ${\overset{\textnormal{}}{\text{食}}}$ べたのに、またお ${\overset{\textnormal{か}}{\text{代}}}$ わり? (Feminine) \hfill\break
My goodness, you've already had two helpings, but you want another one? }

\par{${\overset{\textnormal{}}{\text{2. 急}}}$ いでいるのに、バスが来なくて、 ${\overset{\textnormal{}}{\text{困}}}$ りました。 \hfill\break
Although I was in a hurry, the bus didn't come, so I was troubled. }

\par{${\overset{\textnormal{}}{\text{3. 彼}}}$ はとても ${\overset{\textnormal{}}{\text{年}}}$ を ${\overset{\textnormal{}}{\text{取}}}$ っているのに ${\overset{\textnormal{}}{\text{元気}}}$ ですよ。 \hfill\break
Although he's very old, he's healthy! }

\par{4. あれだけがんばって教えたのに私の生徒たちはみなこの前の試験に合格しませんでした。 \hfill\break
Despite after all I did, all my students didn't pass the last exam. }

\par{5. ( ${\overset{\textnormal{しゅうい}}{\text{周囲}}}$ の) ${\overset{\textnormal{はんたい}}{\text{反対}}}$ をおして怪我をしているのに、 ${\overset{\textnormal{たたか}}{\text{戦}}}$ い続けた。 \hfill\break
Despite voices of opposition around me and my injuries, I continued to fight. }

\par{6. その服、すごくぼろぼろなのに、まだ着てるんだね。 \hfill\break
Although those clothes are really worn-out, you're still wearing them, aren't you. }

\par{7. 姉は75歳になるのに、まだフランス語の家庭教師をやっている。 \hfill\break
Although my older sister is turning 75, she still gives French tutorials (to students). }

\par{\textbf{Spelling Note }: せっかく can be written as 折角. }

\par{8. 食べないの? せっかく作ったのに。 \hfill\break
You're not going to eat it? Even though I went through all the trouble of making it? }

\par{9. 早く雨が上がるといいなあ。〇 \hfill\break
早く雨が上がるといいのに。△ \hfill\break
It would be good for the rain to let up quickly\dothyp{}\dothyp{}\dothyp{} }

\begin{center}
 \textbf{~たらいいのに }
\end{center}

\par{ ~たらいいのに shows regret. It is often used with the potential. It can also be used in proposal of ideas. It is normally translated as "if only". }

\par{10. 彼ともう一度会えたらいいのに。 \hfill\break
I wish I was able to meet with him one more time. }

\par{11. (お)金持ちだったらいいのになあ。 \hfill\break
It would be nice if I were rich. }

\par{12.  空を飛べたらいいなあ。 \hfill\break
Wouldn't it be nice if I could fly… }

\begin{center}
\textbf{~ばいいのに }
\end{center}

\par{ ~ばいいのに・ばいいのだが・ば\{なあ・ねえ\} are like "although it would be good if you". This suggests that the person is doing the contrary. "You" doesn't have to be the only subject. It could be an "it". }

\par{13. そんなに日本語を勉強したいなら、日本へ行けばいいのに。 \hfill\break
If you really want to study Japanese so badly, then it'd be good for you to just go to Japan. }

\par{14. 雨がやめばいいのだが。 \hfill\break
Although it would be nice for the rain to end\dothyp{}\dothyp{}\dothyp{} }

\par{15. 彼はもう少しゴルフがうまければなあ。 \hfill\break
If only he could be a little better at golf. }

\par{16. 「このごろ ${\overset{\textnormal{ふと}}{\text{太}}}$ っちゃって\dothyp{}\dothyp{}\dothyp{}」「もっと ${\overset{\textnormal{うんどう}}{\text{運動}}}$ (を)すればいいのに。テニスとかバスケとか(して)はどう?」 \hfill\break
“I've been getting fat recently” “It would be nice if you exercised more. How about doing tennis, basketball, or something?” }

\par{17. 「漢字が覚えられないんです」「 ${\overset{\textnormal{じしょ}}{\text{辞書}}}$ を引けばいいのに」 \hfill\break
“I can't remember Kanji” “It would be nice if you used a dictionary”. }

\par{18. 「漢字が覚えられないんです」「毎日読めばいいのに」 \hfill\break
“I can't remember Kanji” “It would be nice if you read”. }

\par{19. ${\overset{\textnormal{まえむ}}{\text{前向}}}$ きに考えればいいのに。 \hfill\break
It would be nice if you thought positively. }

\par{20. 「 ${\overset{\textnormal{しけん}}{\text{試験}}}$ の ${\overset{\textnormal{てん}}{\text{点}}}$ が悪くて\dothyp{}\dothyp{}\dothyp{}」「もっと勉強すればいいのに」 \hfill\break
“I did bad on my exam” “It would be nice if you studied more”. }

\par{\textbf{Nuance Note }: ~ばいいのに is a little cold. Try using nicer patterns to suggest things like ~たらどうですか and  ~たらいいと ${\overset{\textnormal{おも}}{\text{思}}}$ います. You can also add context to make the suggestion more 柔らかい "soft". }

\par{21. 「お金がなくて、 ${\overset{\textnormal{こま}}{\text{困}}}$ ってるんだ」 「仕事を探せばいいのに。最低賃金でも ${\overset{\textnormal{いちもん}}{\text{一文}}}$ なしより(かは)いいだろう」 \hfill\break
"I'm troubled because I don't have any money" "It would be nice if you got a job. Minimum wage is better than being penniless". }

\par{22. 一日40時間あればいいのに。 \hfill\break
I wish there was 40 hours in a day. }

\par{23. 雨が ${\overset{\textnormal{はげ}}{\text{激}}}$ しく降ればよかったのに。 \hfill\break
I wish it would have poured. }
    