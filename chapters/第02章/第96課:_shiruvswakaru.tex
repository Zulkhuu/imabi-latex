    
\chapter{知る VS 分かる}

\begin{center}
\begin{Large}
第96課: 知る VS 分かる 
\end{Large}
\end{center}
 
\par{ To understand the difference between 知る and 分かる, you have to consider a lot of things. First, there is the basic definition of both and how they differ. There are also fundamental grammatical differences. Let\textquotesingle s get started. }
      
\section{Differentiating the Two}
 
\par{ First, consider the following basic definitions of 知る and 分かる. }

\begin{itemize}

\item 知る shows the acquisition of information, knowledge, experience, and in the progressive form, it may show that one has such and or remember a matter. 
\item 分かる shows true understanding at the level of comprehending the true\slash actual state of things. It can also mean that one identifies\slash establishes something as so and or that one has discerned something. 
\end{itemize}

\par{ Just from this introductory explanation, we can see that 分かる implies a more serious state of comprehension than知る. }

\par{ Consider the difference between よく知らない人 and よく分からない人. The first sounds like it\textquotesingle s a person you\textquotesingle ve not actually been acquainted with and or a mysterious individual from somewhere. The latter sounds like a mysterious person but one you\textquotesingle ve been acquainted with, or perhaps the person has changed somehow. So, when you go from 未知 (not yet known) to 既知 (already known), you use 知る. Reaching the true essence of this 既知の事実 is within the realm of 分かる. 分かる also implies an effect not involving one\textquotesingle s volition. }

\par{1. その時に、ラーメンの味を知った。 \hfill\break
I was then acquainted with the taste of ramen. }

\par{2. ああ、分かった。 \hfill\break
Ah, I get it. }

\par{3. 出席するかどうか分かりません。 \hfill\break
I don't know whether I will attend or not. }

\par{4. 検索してみれば分かるよ。 \hfill\break
You'll understand once you look it up. }

\par{5. 歌詞の意味なんて分からないように出来てるよ。 \hfill\break
The meaning of lyrics are crafted so that you don't understand them. }

\par{6. 最初から読まないと分からないんだから、悪い日本語だよ。 \hfill\break
If you can't understand it without reading from the beginning, it's bad Japanese. }

\par{7. 日本語でも何言ってるか分からなくて理解できなかった。 \hfill\break
I couldn't comprehend it because I didn't even know what was being said even in Japanese. }

\par{8. ${\overset{\textnormal{だんな}}{\text{旦那}}}$ が分かってないのに録音してるからって適当に受け答えして帰ってきちゃったの。 \hfill\break
Although my husband couldn't understand, he recorded and came home after giving replies. }

\par{9. わからない単語があったから辞書を引いたのに、日本語でも英語でも分からないから、笑っちゃった。 \hfill\break
Despite looking up words that I didn't know, I laughed with the fact that I don't get the Japanese or the   English. }

\par{10a. 考えれば、分かります。 〇 \hfill\break
10b. 考えれば、知ります。 X \hfill\break
If you think about it, you'll understand. }

\par{11. ゴルフやったことないから何のことか分からない。 \hfill\break
I've never played golf, so I don't know what that is. }

\par{12. ググってフェアウェーとグリーンの違いを知った。 \hfill\break
I learned of the difference between a fairway and a green when I googled. }

\par{13. 青信号っていう言葉が出来上がっちゃってるから青じゃない場合になんて言ったらいいか分からないや。 \hfill\break
Since we have this phrase "aoshingo", even when the light is not blue, I don't know what's best to say. }

\par{14. 彼女はその人が日本人であることも知っていたはずなんだけど。 \hfill\break
Though she should have known that that person was Japanese. }

\par{15. これは本当かどうか分からないけど、ベトナムでも犬を食べる習慣があって、彼らはペットの犬も食べるらし  い。 \hfill\break
I don't know whether this is true or not, but it seems that there is a tradition even in Vietnam of eating dogs and that they also eat dogs meant as pets. }

\par{16. アメリカではタコも ${\overset{\textnormal{めずら}}{\text{珍}}}$ しい?知らなかった! \hfill\break
Octopus is also rare in America? I had no idea! }

\par{ 分かる is intransitive in Japanese and so you must use が to mark what in English would be the direct object. However, you do occasionally see を分かる due to Western influence. Nevertheless, other things like わかりたい, わかられる, and わかりえる are wrong with only 分かりたい being sometimes acceptable as a more emphatic way of saying知りたい, which to some speakers is still just wrong. }

\par{17a. 日本語が分かりたい。? \hfill\break
17b. 日本語を理解したい。〇 \hfill\break
I want to know Japanese. }

\par{18. 分かりたいなら、よく調べなさい。 \hfill\break
If you really want to know, go look into it. }

\par{ 知る can be conjugated into forms such as 知りたい, 知られる, and 知り得る, but it has a tense restriction that 分かる does not have. Asking 知りますか would be very weird. It would mean that you\textquotesingle re asking whether someone is responsive to information, which would be very weird to ask another human being. You can, though, use 分かりますか to ask about comprehension. 知っていますか is perfectly fine for asking whether someone knows something. So, if you were to ask 「大統領を知っていますか」, the person is asking whether you know anything about the president. 知っていない is almost always wrong, but saying it is always impossible is a fallacy that will be addressed in depth shortly in this lesson. }

\par{19. もっと深く知りたい。 \hfill\break
I want to know deeper about it. }

\par{20. ${\overset{\textnormal{ちょうせん}}{\text{朝鮮}}}$ ${\overset{\textnormal{はんとう}}{\text{半島}}}$ の ${\overset{\textnormal{じじょう}}{\text{事情}}}$ をよく知ることが大切です。 \hfill\break
It's important to know well of the conditions in the Korean Peninsula. }

\par{ Finally, it's important to quickly think of the difference between 知っている and 分かっている. }

\begin{itemize}

\item 知っている: Shows that you are in the state of keeping hold of some knowledge about something. 
\item 分かっている: Shows that you are in the state of knowing the true essence of something. 
\end{itemize}

\par{21. 知っていますが、今はちょっと分かりません。 \hfill\break
I know of it, but I don't actually know it at the moment. }

\par{22. 差別と区別すら解ってない。 \hfill\break
(He\slash someone) doesn't even understand the difference between "discrimination" and       "differentiating". }
      
\section{知らない VS 知っていない}
 
\par{ The standard negative form of 知っている is 知らない. However, 知っていない does exist. 知っていない often appears whenever the affirmative and negative of 知っている are contrasted. 知っていない is possible when the negative is being used for affirmative conjecture, which we've seen already before with things like じゃないか, 遊びに行かない? It's also possible in the following contrast. }

\par{知っている + ば \textrightarrow  知っていなければ = If you don't know }

\par{知らない + ば \textrightarrow  知らなければ = If you don't realize }

\par{ Essentially, 知らない denotes attention to a static condition of not knowing whereas 知っていない denotes attention to the completeness of knowing in the negative sense. 知っていない is inconstant and denotes an objective view from the outside in regards to a lack of knowledge, which is exactly why students are rightfully told they are wrong when they try to apply it as meaning "I don't know". 知らない involves denoting a lack of knowledge from the inward perspective of the thing at hand. }

\par{ There are still instances when 知っていない is seen, but it is almost completely avoided. The single instance where it is in imperative is still used, but many simply avoid it as there are synonymous phrases like 理解できない and 腑に落ちていない. Derivatives are most common, which is the most important thing to keep in mind. }

\par{23a. 知っていても知っていなくても 〇 \hfill\break
23b. 知っていても知らなくても 〇 \hfill\break
Even if you do or do not know }

\par{24. そのことを知っているか否かではない。 \hfill\break
It's not whether [they] know or do not know that. }

\par{\textbf{Grammar Note }: Notice how in this situation the use of 知っていない is avoided by using 否. }

\par{25a. 知っていようが知っていなかろうが 〇 \hfill\break
25b. 知っていようが知らなかろうが 〇 \hfill\break
Whether you know or not }

\par{26. この問題は、日本語を知らなければ答えられない。?? \hfill\break
Intended: If you don't know Japanese, you can't answer this question. }

\par{27. この試験に合格するためには、日本語をよく知らなければならない。X \hfill\break
Intended: In order to pass this exam, you must know Japanese well. }

\par{ Otherwise, 知っていない is probably a mistake or dialectical because there are dialects where the same thing is acceptable. }

\par{28. 「田中君はこのことについて、何か知っていたか」「いや、まだ何も知っていませんでした」 〇 \hfill\break
"What did Tanaka know about this? "No, he still didn't know anything." }
    