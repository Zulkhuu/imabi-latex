    
\chapter{The Particles が \& けれど}

\begin{center}
\begin{Large}
第76課: The Particles が \& けれど 
\end{Large}
\end{center}
 
\par{ These are perhaps the most used conjunctions in Japanese. }
      
\section{The Conjunctive Particle が}
 
\par{ The conjunctive particle が may give the \textbf{premise }of a conversation or indicate a contradiction. The two clauses of the sentence should be the same in politeness. }

\par{${\overset{\textnormal{}}{\text{1. 目}}}$ が ${\overset{\textnormal{}}{\text{赤}}}$ いが、どうしたの。(Casual) \hfill\break
Your eyes are red\dothyp{}\dothyp{}\dothyp{}what's the matter? \hfill\break
\hfill\break
2. テレビで聞いたが。 \hfill\break
I heard it on TV, but\dothyp{}\dothyp{}\dothyp{} }

\par{${\overset{\textnormal{}}{\text{3. 高}}}$ いが、 ${\overset{\textnormal{しつ}}{\text{質}}}$ は ${\overset{\textnormal{}}{\text{悪}}}$ い。 \hfill\break
It's expensive, but the quality is bad. }

\par{4. ちょっとすみませんが、 ${\overset{\textnormal{あんないじょ}}{\text{案内所}}}$ はどこですか。 \hfill\break
Excuse me one moment, but where is the information center? }

\par{5. すみませんが、 ${\overset{\textnormal{しお}}{\text{塩}}}$ を取っていただけませんか。 \hfill\break
Excuse me, but could you pass me the salt? }

\par{6. ${\overset{\textnormal{やす}}{\text{安}}}$ いことは ${\overset{\textnormal{}}{\text{安}}}$ いが、 ${\overset{\textnormal{}}{\text{質}}}$ が ${\overset{\textnormal{}}{\text{悪}}}$ い。 \hfill\break
A cheap thing is cheap, but the quality is bad. }
7. 寒いんですが、 ${\overset{\textnormal{か}}{\text{掛}}}$ け ${\overset{\textnormal{ぶとん}}{\text{布団}}}$ をもう ${\overset{\textnormal{いちまい}}{\text{一枚}}}$ かしてくださいませんか。 \hfill\break
Since it is cold, could you lend me another quilt? 
\par{8. すみませんが、私の ${\overset{\textnormal{しつもん}}{\text{質問}}}$ に ${\overset{\textnormal{こた}}{\text{答}}}$ えてくださいませんか。 \hfill\break
I'm sorry, but could you please answer to my questions? }

\par{9. 質はいいが、 ${\overset{\textnormal{ねだん}}{\text{値段}}}$ も高い。 \hfill\break
The quality is good, but the price is high. }

\par{10. ${\overset{\textnormal{えいが}}{\text{映画}}}$ を ${\overset{\textnormal{}}{\text{見}}}$ たが、 ${\overset{\textnormal{おもしろ}}{\text{面白}}}$ かった。 \hfill\break
I watched a movie, and it was interesting. }

\par{11. 「日本語のコースでは、漢字をいくつか習いますか」「いくつ習うか分かりませんが、もう漢字が読めますから、 私は気にしません」 \hfill\break
“How many Kanji do you learn in the Japanese course?” “I don't know how many you learn, but since I can already read Kanji, I don't care”. \hfill\break
\hfill\break
12. 本を読んだが、つまらなかった。 \hfill\break
I read it, and it was dull. }

\par{13. 札幌へは行きますが、函館へは行きません。 \hfill\break
I'll go to Sapporo, but I won't go to Hakodate. }

\par{14. ${\overset{\textnormal{ま}}{\text{負}}}$ けるに負けたが、悲しくない。 \hfill\break
\{I\slash we\} indeed lost, but I'm not sad. }
 
\par{15. 聞くには聞いたが、まだよく聞こえない。 \hfill\break
I've listened to it and listened, but I still can't hear it well. }

\par{16. 高田くんは頭がいいが、冷たい人だ。 \hfill\break
Takata is smart, but he's a cold person. }

\par{\textbf{Sentence Note }: As is, this sentence is rather harsh. To critique someone less harshly in Japanese, you should mention the bad quality first and then say something positive. }

\par{17. ${\overset{\textnormal{はな}}{\text{鼻}}}$ が ${\overset{\textnormal{}}{\text{高}}}$ いが、においができない。 \hfill\break
His nose is long, but he can't smell. }

\par{\textbf{Word Choice Note }: 高い is "long" for human noses; long for other noses like elephants' is ${\overset{\textnormal{}}{\text{長}}}$ い. }

\par{\textbf{漢字 Note }: The 漢字 spelling for におい in this situation is 匂い, which is very common. }

\par{18. ちょっとお ${\overset{\textnormal{ねが}}{\text{願}}}$ いがあるんですが。 \hfill\break
I have a little request. }
 
\par{\textbf{Usage Note }: This sentence shows how んですが can be used to introduce a topic politely, and it's followed by something that needs permission, whether it be a request or invitation. As this sentence shows, what follows が doesn't always have to be mentioned. Something that could have followed it here would be いいでしょうか. け(れ)ど could have been used instead of が. }
      
\section{The Conjunctive Particle けれど}
 
\par{ けれど, more polite as けれども and casual as けど, means "although\slash but", similarly to が. It may also be used to mean "but" at the end of a sentence, but it may also just soften a statement to be more polite and indirect. Even as a conjunctive particle, it may just simply connect phrases without a sense of contrast. }
 
\par{18. 彼は若いけれど、考えのしっかりした人です。 \hfill\break
Although he's young, he's a sound thinking person. }
 
\par{19. 漢字は書けないけれど、読むことはできます。 \hfill\break
Although I can't write Kanji, I can read them. }
 
\par{20. ちょっとテレビがうるさくて勉強できないんだけど、音を小さくしてくれない?(Casual) \hfill\break
The TV is a little loud and I can't study, so could you turn it down? }
 
\par{21. 益田ですけれども。 \hfill\break
This is Masuda. }
 
\par{22. もうすこし日本語がよく分かるといいのだけれども。 \hfill\break
But it would be nice if I understood Japanese a little bit better. }

\par{23. ${\overset{\textnormal{とざん}}{\text{登山}}}$ に行きたいんですけど、どこかいいところありませんか。 \hfill\break
I want to go mountain climbing. Could you tell me of any good places? }
 
\par{24. この ${\overset{\textnormal{ちほう}}{\text{地方}}}$ は寒くないと聞いたけれども、本当に毎日冷え ${\overset{\textnormal{こ}}{\text{込}}}$ みますよ。 \hfill\break
Although I heard that it wasn't cold in this region, it actually gets chilly every day. }
 
\par{25. それは ${\overset{\textnormal{りっぱ}}{\text{立派}}}$ だけれど、そうなっていない。 \hfill\break
That's great, but it's not turning out as such. }

\par{26. ${\overset{\textnormal{へんじ}}{\text{返事}}}$ するにはするけど、もうちょっと待ってね。(Soft; probably a female speaker) \hfill\break
I'll give you the answer, certainly, but could you wait a minute? }
 
\par{27. そうすっけど、本当にだいじょうぶ? \hfill\break
I'll do so, but is it really OK? }
 
\par{\textbf{Contraction Note }: すっけど = するけど. る \textrightarrow  っ is common in really causal speech in situations like this where it is before a consonant like k. Thus, するから may be seen as すっから. }
 
\par{\textbf{Variant Note }: けども is a slightly old and somewhat dialectical variant. }
 
\par{\textbf{Phrase Note }: こう言ってはなんだ\{が・けど\} is a vague way of saying that one might be saying too much. }
    