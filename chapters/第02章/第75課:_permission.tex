    
\chapter{~て(も)いい}

\begin{center}
\begin{Large}
第75課: ~て(も)いい: Permission 
\end{Large}
\end{center}
 
\par{ This lesson will be about ~て(も)いい. This is an essential part of many sentences for asking and giving permission to someone to do something. }
      
\section{~ていい}
 
\par{ ~て(も)いい is used to ask for permission. As speech level changes this phrase, this is a good time to review these processes. Remember that も makes the sentence more indirect, thus making it more polite. The particle adds the meaning of "even if I". }

\begin{ltabulary}{|P|P|P|P|P|P|}
\hline 

Plain &  て(も)いい &  Polite &  て(も)いいですか &  Very Polite &  てもいいでしょうか \\ \cline{1-6}

 Formal &  てもよろしいですか &  Very Formal &  てもよろしいでしょうか  &  &  \\ \cline{1-6}

\end{ltabulary}

\par{ The addition of the particle も emphasizes the sense of "if you do X, that's OK", and it is often used to give a softening effect. This is why it is usually seen when asking questions as to be more polite. }

\begin{center}
 \textbf{Examples }
\end{center}

\par{1. お ${\overset{\textnormal{かあ}}{\text{母}}}$ ちゃん! ${\overset{\textnormal{こんやえいが}}{\text{今夜映画}}}$ を ${\overset{\textnormal{み}}{\text{観}}}$ ていい? \hfill\break
Mom! Can I see a movie tonight? }

\par{2. すみません、タバコを吸ってもいいですか。 \hfill\break
Excuse me, is it alright if I smoke? }

\par{3. あのう、お隣に座ってもよろしいですか。 \hfill\break
Mm, is it fine for me to sit by you? }

\par{4. ${\overset{\textnormal{はこ}}{\text{箱}}}$ を ${\overset{\textnormal{あ}}{\text{開}}}$ けてよろしいですか。 \hfill\break
May I open the box? }
 
\par{5. たくさんありますから、 ${\overset{\textnormal{ぜんぶ}}{\text{全部}}}$ を ${\overset{\textnormal{こうせい}}{\text{校正}}}$ しなくてもいいです。 \hfill\break
Because there is a lot of them, it's OK if you don't proofread them all. }

\par{6. ${\overset{\textnormal{はい}}{\text{入}}}$ ってよろしいでしょうか。 \hfill\break
May I come in? }

\par{7. ${\overset{\textnormal{くろだしゃちょう}}{\text{黒田社長}}}$ 、 ${\overset{\textnormal{わたしたち}}{\text{私達}}}$ はこちらに ${\overset{\textnormal{すわ}}{\text{座}}}$ ってもよろしいですか。 \hfill\break
President Kuroda, may we sit down here? }
 
\par{${\overset{\textnormal{}}{\text{8. 彼}}}$ はこちらに ${\overset{\textnormal{すわ}}{\text{座}}}$ ってもよろしいでしょうか。 \hfill\break
May he sit down here? }

\par{9. ${\overset{\textnormal{しゃちょう}}{\text{社長}}}$ 、この ${\overset{\textnormal{}}{\text{本}}}$ をお ${\overset{\textnormal{か}}{\text{借}}}$ りしてもよろしいでしょうか。(Humble Speech) \hfill\break
President, may I borrow this book? }

\par{10. こちらにご記入をいただいてよろしいですか。 \hfill\break
May I have you fill this out here? }

\par{\textbf{Sentence Note }: In honorifics, it is important to lessen the effect of pushing a request on someone. So,~てよろしい is seen here asking for permission to make a request to the listener. }

\begin{center}
 \textbf{許可 Not 依頼 }
\end{center}

\par{ ~ていい asks for permission, not request. In any case, the listener could still say no. If you were to use ~ていい with 借りる (to borrow), the listener is the person you're seeking permission to borrow. The listener is not the person lending you the money (貸す人ではない!). We can surmise that the only people that would say something like お金を貸してください are bank deposit salesmen or loan sharks, but friends wouldn't use it among each other. }

\par{11a. 先生に〇を習ってもいいですか?  X \hfill\break
11b. 先生、〇を教えていただけませんか。〇 \hfill\break
Sensei, may you teach me 〇? }

\par{12a. 10万円を借りてもよろしいですか。  X \hfill\break
12b. 10万円を貸していただけませんか。〇 \hfill\break
Could I please have you lend me 100 thousand yen? }

\par{13. 友達から借りてもいいですか。 \hfill\break
Is it alright if I borrow from a friend? }

\par{14. 細かいのがないので、ちょっと10円\{借りてもいい X・貸してくれない\}? \hfill\break
I don't have small change, so could I borrow 10 yen? }

\par{\textbf{Sentence Note }: In Ex. 14, the difference between the English "borrow" and Japanese 借りる is apparent. Asking to borrow something from the listener always involves the verb 貸す instead. }

\begin{center}
 \textbf{~て結構 }
\end{center}

\par{Another equivalent phrase ~[て・で] ${\overset{\textnormal{けっこう}}{\text{結構}}}$ . It shows that something is fine but may not be entirely satisfactory. }
 
\par{15. どれでも ${\overset{\textnormal{}}{\text{結構}}}$ だ。 \hfill\break
Anything's fine. }

\par{\textbf{Particle Note }: As you should know by now で+も results in the particle でも. }
    