    
\chapter{After I}

\begin{center}
\begin{Large}
第62課: After I 
\end{Large}
\end{center}
 
\par{ There is not one phrase for "after," and there are minute differences that you need to know about the various ways to say it. So, pay attention as always. }
      
\section{後}
 
\par{ "Verb A + た ${\overset{\textnormal{あと}}{\text{後}}}$ (で) + Verb B" states that B happens after A. However, there are several other similar expressions in Japanese. So, pay close attention to the differences discussed later on. }

\par{ As for this basic pattern, the verb before 後 must be in the past tense. Yet, ~ていた後 is ungrammatical because it needs to show the end of an action. Other grammatical issues that the example sentences and subsequent explanation will show is that whether to use で, に, or no particle is somewhat difficult. However, it's important to see examples first. }

\par{1a. ${\overset{\textnormal{ざっし}}{\text{雑誌}}}$ を読んでいた後で、アニメを見た。 X \hfill\break
1b. 雑誌を読んだ後で、アニメを見た。〇 \hfill\break
I watched anime after reading a magazine. }

\par{2a. ${\overset{\textnormal{ひゃっかてん}}{\text{百貨店}}}$ があった後で、 ${\overset{\textnormal{えき}}{\text{駅}}}$ ができた。X \hfill\break
2b. 百貨店\{があった・の\}後に、駅ができた。〇 \hfill\break
A train station was built where a department store had been. }

\par{3. 昨日は ${\overset{\textnormal{じゅぎょう}}{\text{授業}}}$ の ${\overset{\textnormal{あと}}{\text{後}}}$ 、 ${\overset{\textnormal{うち}}{\text{家}}}$ に帰らないで ${\overset{\textnormal{えいが}}{\text{映画}}}$ を ${\overset{\textnormal{み}}{\text{観}}}$ に行った。〇 \hfill\break
Yesterday I went to go see a movie instead of going home after class. }

\par{\textbf{漢字 Note }: 観 \emph{can }be used instead of 見 when one is watching a program of some sort such as a movie, a TV show, play, drama, Family Guy (I love this show), etc. }

\par{4. 学校のあと、ふと ${\overset{\textnormal{め}}{\text{目}}}$ まい目まいがした。〇 \hfill\break
I felt dizzy all of a sudden after school. }

\par{\textbf{Spelling Notes }: 後 is occasionally written in ひらがな as the character has several readings and some speakers just like to write it in ひらがな to disambiguate as habit. Lastly, 目まい may seldom fully be written in 漢字 as 眩暈・目眩. }

\par{ Ex. 2a is ungrammatical because ある is existential, and to clearly show the location in which the train station has now taken the place of the department store, you need to use the particle に. Even so, verbs of condition can be used with 後 as long as the "end" of the state is clearly instantiated correctly. }

\par{5. しばらくギリシャにいた後、トルコへ ${\overset{\textnormal{わた}}{\text{渡}}}$ った。 \hfill\break
After being in Greece for a while, I crossed over to Turkey. }

\par{6. ${\overset{\textnormal{たいへん}}{\text{大変}}}$ だった後は、 ${\overset{\textnormal{よ}}{\text{良}}}$ いことがあるのかな。 \hfill\break
I wonder if there can be anything good after things being so terrible. }

\par{ Many other particles can follow 後. You should not be surprised to a sentence where に, は, や, と, or から follow it. After all, it is a noun. When showing the condition after a change, then 後は is perfect to use. Verb B after 後は should deal with duration or condition. }

\par{7. 勉強をしたあとは、 ${\overset{\textnormal{つか}}{\text{疲}}}$ れて何もできない。(X) \textrightarrow  (後で) \hfill\break
After studying, I'm tired and can't do anything. }

\par{ Ex. 7 is not plausible to some speakers as being unable to do anything due to studying isn't a state of change over any serious duration of time. Studying is usually not long in duration and is realistically a temporary state of action. Now, if you are exaggerating things, then it would be more OK. }

\par{8. ${\overset{\textnormal{けんがい}}{\text{県外}}}$ に ${\overset{\textnormal{}}{\text{っ}}}$ ${\overset{\textnormal{こ}}{\text{越}}}$ した後は、 ${\overset{\textnormal{ゆうじん}}{\text{友人}}}$ と会う ${\overset{\textnormal{きかい}}{\text{機会}}}$ が ${\overset{\textnormal{へ}}{\text{減}}}$ った。 \hfill\break
Since moving out of the prefecture, chances to meet friends have decreased. }

\par{\textbf{Word Note }: 友人 is more frequently used in the written language. }

\par{9. Kポップを ${\overset{\textnormal{き}}{\text{聴}}}$ いた後は、 ${\overset{\textnormal{きも}}{\text{気持}}}$ ちがよくなります。 \hfill\break
I feel better after having listened to K-pop. }

\par{\textbf{Pronunciation Note }: Pronounce K as ケイ. }

\par{ 後は can also be used to show that something is done after another action habitually. It's also very important to have this form when you want to conjoin multiple 後 expressions for topicalization\slash emphasis. }

\par{10. ${\overset{\textnormal{}}{\text{}}}$ ${\overset{\textnormal{は}}{\text{歯}}}$ を ${\overset{\textnormal{みが}}{\text{磨}}}$ いた後はすぐ寝ます。 \hfill\break
I go right to sleep after I brush my teeth. }

\par{11. ${\overset{\textnormal{はたら}}{\text{働}}}$ いた後や勉強した後はカラオケ! \hfill\break
Karaoke after working and or studying! }

\par{ 後に can be used in a sense of time, and in doing so it is more punctual. But, there are many more examples where it is used in the sense of location as well. }

\par{12. ${\overset{\textnormal{ぼく}}{\text{僕}}}$ は食べた後(に)歯を ${\overset{\textnormal{みが}}{\text{磨}}}$ いた。 \hfill\break
I brushed my teeth after I ate. }

\par{13. ${\overset{\textnormal{ともだち}}{\text{友達}}}$ が帰った後には、ゴミがたくさん落ちてた。(Casual) \hfill\break
There was a lot of trash dropped after my friends went home. }

\par{14. 花が ${\overset{\textnormal{さ}}{\text{咲}}}$ いた後に、 ${\overset{\textnormal{み}}{\text{実}}}$ ができて、その中に ${\overset{\textnormal{たね}}{\text{種}}}$ ができる。 \hfill\break
After a flower blooms, a fruit forms, and inside that seeds form. }

\par{15. ${\overset{\textnormal{そつぎょう}}{\text{卒業}}}$ (を)した後に、仕事を ${\overset{\textnormal{さが}}{\text{探}}}$ します。 \hfill\break
After graduating, I'll look for a job. }

\par{16. 子供が寝た後に、映画を見ました。 \hfill\break
I watched a movie after my child went to bed. }

\par{17. 昼ご飯の後に、その町を ${\overset{\textnormal{けんぶつ}}{\text{見物}}}$ しました。 \hfill\break
We went sightseeing in the town after lunch. }

\par{ As 後 is a noun, ~後だ can also be seen at the end of a sentence or clause. So, it could be that で in 後で is the て Form of だ, which will show in context and intonation. You can also see あとのN. As far as 後から is concerned, it uses the physical sense of the word. から means "from". So, imagine that there is something that comes about from the end of an action A. }

\par{18. ${\overset{\textnormal{けっか}}{\text{結果}}}$ が分かるのは、 ${\overset{\textnormal{ちょうさ}}{\text{調査}}}$ の後だ。 \hfill\break
We'll know the results after the investigation. }

\par{19. 子供が帰ったあとからはま~るい大きなお ${\overset{\textnormal{つきさま}}{\text{月様}}}$ ! \hfill\break
From after the children have gone home, the round, big moon comes out! \hfill\break
From the lyrics of 夕焼け小焼け. }

\begin{center}
\textbf{~後に }
\end{center}

\par{ ~ ${\overset{\textnormal{ご}}{\text{後}}}$ に also means "later". This attaches to Sino-Japanese words involving time. It is sometimes formal such as in Ex. 22 and 23, but there are also instances in which it is quite normal like when it is paired with 週間 as in Ex. 20 and 21. }

\par{20. 三時間後に ${\overset{\textnormal{はとやまそうりだいじん}}{\text{鳩山総理大臣}}}$ は ${\overset{\textnormal{なりたくうこう}}{\text{成田空港}}}$ に ${\overset{\textnormal{とうちゃく}}{\text{到着}}}$ します。 \hfill\break
Prime Minister Hatoyama will arrive in Narita Airport in 3 hours. }

\par{\textbf{Person Note }: Yukio Hatoyama (鳩山由紀夫) was the Prime Minister of Japan from 2009 to 2010. }

\par{21. 2週間後に帰ります。 \hfill\break
I'll return in two weeks. }

\par{22. ${\overset{\textnormal{しょくご}}{\text{食後}}}$ に ${\overset{\textnormal{くすり}}{\text{薬}}}$ を飲む。 \hfill\break
To take medicine after a meal. }

\par{\textbf{Sentence Note }: Ex. 22 would be most expected of a doctor to say or write. }

\par{23. ${\overset{\textnormal{きたくご}}{\text{帰宅後}}}$ にメイクを ${\overset{\textnormal{お}}{\text{落}}}$ とす ${\overset{\textnormal{じょせい}}{\text{女性}}}$  \hfill\break
Women who take off their make-up after arriving home }

\par{\textbf{Word Note }: 帰宅後 = 家に帰った後 }

\begin{center}
 \textbf{~のち }
\end{center}

\par{ 後 may also be read as のち. のち may be used in the same sense as 後 and may be used just like ~後に in the sense of ~あとに and ~ごに. However, it is more formal and is most likely to be in the written language. のちに may also be used in the sense of "henceforth". }

\par{24. それをずっとのちに知りました。 \hfill\break
I knew of that long afterwards. }

\par{25. ${\overset{\textnormal{さん}}{\text{三}}}$ ${\overset{\textnormal{か}}{\text{ヶ}}}$ ${\overset{\textnormal{げつ}}{\text{月}}}$  ${\overset{\textnormal{ご・のち}}{\text{後}}}$ に ${\overset{\textnormal{りだつ}}{\text{離脱}}}$ 。 \hfill\break
Defecting after three months. }

\par{26. 大学を卒業した ${\overset{\textnormal{のち}}{\text{後}}}$ 、 ${\overset{\textnormal{えんげきかい}}{\text{演劇界}}}$ に入った。 \hfill\break
After graduating, I entered the world of theatre. }

\par{27. ${\overset{\textnormal{は}}{\text{晴}}}$ れのち ${\overset{\textnormal{くも}}{\text{曇}}}$ り \hfill\break
Fine, cloudy later }

\par{\textbf{Sentence Note }: Ex. 27 is an example of common messages you might find watching a Japanese weather report. }
      
\section{~てから}
 
\par{ \textbf{~てから shows what you do after something else }. This goes in hand when showing the starting point of something. In this case, it's the starting point of the new action \emph{after }the previous one. So, once the first action is established, then that sets up the starting point for the next action. This pattern is very good for showing sequence. }

\par{28. ${\overset{\textnormal{ゆうしょく}}{\text{夕食}}}$ を ${\overset{\textnormal{}}{\text{食}}}$ べてから ${\overset{\textnormal{ね}}{\text{寝}}}$ る前の ${\overset{\textnormal{じかん}}{\text{時間が}}}$ 長すぎる。 \hfill\break
The time before I go to bed after eating dinner is too long. }

\par{29. ${\overset{\textnormal{わか}}{\text{別}}}$ れてから ${\overset{\textnormal{}}{\text{会}}}$ っていない。 \hfill\break
I haven't met her since we split up. }

\par{${\overset{\textnormal{}}{\text{30. 私}}}$ は ${\overset{\textnormal{}}{\text{勉強}}}$ を ${\overset{\textnormal{}}{\text{始}}}$ めてから、 ${\overset{\textnormal{}}{\text{多}}}$ くのことを ${\overset{\textnormal{}}{\text{学}}}$ びました。 \hfill\break
I've learned a lot of things since I started studying. }

\par{31. よく ${\overset{\textnormal{}}{\text{考}}}$ えてから ${\overset{\textnormal{こた}}{\text{答}}}$ えます。 \hfill\break
I will answer after some thought. }

\par{${\overset{\textnormal{}}{\text{32. 日本}}}$ に ${\overset{\textnormal{}}{\text{来}}}$ てからもう10 ${\overset{\textnormal{}}{\text{年}}}$ になります。 \hfill\break
It has already been ten years since I came to Japan. }

\par{${\overset{\textnormal{}}{\text{33. 日本}}}$ に ${\overset{\textnormal{}}{\text{来}}}$ てからずっと ${\overset{\textnormal{}}{\text{日本語}}}$ の ${\overset{\textnormal{}}{\text{勉強}}}$ をしています。 \hfill\break
I've been studying Japanese ever since I came to Japan. }

\par{${\overset{\textnormal{}}{\text{34. 高校}}}$ を ${\overset{\textnormal{そつぎょう}}{\text{卒業}}}$ してから、 ${\overset{\textnormal{}}{\text{大学}}}$ に ${\overset{\textnormal{}}{\text{行}}}$ かないで ${\overset{\textnormal{}}{\text{仕事}}}$ をする ${\overset{\textnormal{}}{\text{人}}}$ も ${\overset{\textnormal{}}{\text{多}}}$ い。 \hfill\break
There are also a lot of people that work instead of going to college after graduating high school. }

\par{\textbf{Phrase Note }: ${\overset{\textnormal{にゅうがく}}{\text{入学}}}$ する = "to be admitted", which can also be used in reference to middle and high school in Japan. So, using this instead of ${\overset{\textnormal{}}{\text{大学}}}$ に ${\overset{\textnormal{}}{\text{行}}}$ く may not be a smart idea if you don't have the context. Now, 大学に入学する would still mean "to be admitted into college", but it doesn't describe going to four years of college like the above sentence does. }

\par{35. ${\overset{\textnormal{はい}}{\text{入}}}$ ってから ${\overset{\textnormal{}}{\text{出}}}$ るまで、 ${\overset{\textnormal{こわ}}{\text{怖}}}$ かった。 \hfill\break
I was scared from when I entered until I left. }

\par{\textbf{Particle Note }: Other particles like は and の can follow ~てから. You can even see だ after it. }

\par{\textbf{Tense Note: }You should not use this pattern with ~ている.So, ~ていてから X. }

\begin{center}
 \textbf{~後で VS  ~てから }
\end{center}

\par{ There are many instances where using ~後で can be unnatural or even ungrammatical.  At other times, they're interchangeable. Consider the following examples. }

\par{36. 買い物を\{してから・した後で\}映画を見ました。 \hfill\break
I watched a movie after shopping. }

\par{37. ${\overset{\textnormal{きっぷ}}{\text{切符}}}$ を買っ\{てから 〇・た後に 〇・た後で △\}ポケットの中に入れた。 \hfill\break
After I bought the ticket, I put it inside my pocket. }

\par{\textbf{Grammar Note }: 後で is unnatural because Action B happens right after Action A as a consequence of it. The same problem exists for Ex. 38. }

\par{38. ちょっと口をすすい\{でから 〇 ・た後に 〇・だ後で △\}歯を磨く。 \hfill\break
To brush one's teeth after rinsing one's mouth a little. }

\par{\textbf{漢字 Note }: すすぐ may rarely be written as 漱ぐ in this context. }

\par{39. 卒業し\{てから 〇・た後で X\}、彼には会っていない。 \hfill\break
I haven't met with him since graduating. }

\par{\textbf{Grammar Note }: You should never use ない as the latter condition (state\slash action B) with 後で. So, in the sense of "since", you should use ~てから rather than ~後で. }

\par{40. ${\overset{\textnormal{あき}}{\text{秋}}}$ が来\{てから 〇・た後で X\}、 ${\overset{\textnormal{きゅう}}{\text{急}}}$ に ${\overset{\textnormal{ひとどお}}{\text{人通}}}$ りが少なくなった。 \hfill\break
Street traffic suddenly got deserted since fall came. }
    