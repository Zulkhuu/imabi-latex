    
\chapter{Want \& Feeling II}

\begin{center}
\begin{Large}
第100課: Want \& Feeling II: ~がる 
\end{Large}
\end{center}
 
\par{ ~がる is an extremely important suffix to overcome restrictions on person with phrases of emotional\slash physical state in Japanese. }
      
\section{The Suffix ~がる}
 
\par{ Adjectival phrases of emotion in Japanese cannot take third person subjects. Meaning, you cannot say things like. }

\par{1a. 鈴木さんは失恋して悲しいです。X \hfill\break
1a. Suzuki is heartbroken and sad. }

\par{ To overcome this grammatical restriction, the suffix ~がる is needed. }

\par{1b. 鈴木さんは失恋して悲しがっています。 \hfill\break
1b. Suzuki seems sad from being heartbroken. }

\par{ To conjugate, you simply attach ~がる to the stem of adjective regardless of class. Although we haven't gotten into what ~がる means, it is conveniently similar to "seem" in English. There are important differences that we'll get to later, so pay close attention! }

\begin{ltabulary}{|P|P|P|P|}
\hline 

Sad & 悲しい & To seem sad & 悲しがる \\ \cline{1-4}

Happy & 嬉しい & To seem happy & 嬉しがる \\ \cline{1-4}

Embarrassing & 恥ずかしい \hfill\break
& To seem embarrassed & 恥ずかしがる \hfill\break
\\ \cline{1-4}

Scared & 怖い & To seem scared & 怖がる \\ \cline{1-4}

New & 新しい & To be fond of new things & 新しがる \hfill\break
\\ \cline{1-4}

Rare & 珍しい & To seem rare & 珍しがる \hfill\break
\\ \cline{1-4}

Want to\dothyp{}\dothyp{}\dothyp{} & ~たい & To seem to want & ~たがる \\ \cline{1-4}

\end{ltabulary}

\par{\textbf{Conjugation Note }: Be careful to note confuse ~たかった and ~たがった or ほしかった and ほしがった, especially when listening as it may be tricky. Note that ~がる behaves like any other verb phrase. So, if ~ている is needed, you'll need to use ~がっている. }

\par{\textbf{Usage Note }: It's also interesting to note that there are many adjectival phrases that ~がる is never used with for whatever reason. Some include 好きだ, 涼しい、軽い, きれいだ, etc. One restriction is that it can't be used with adjectives that show no emotional aspect of someone. There is also a subjective\slash objective aspect to adjectives that mustn't be ignored. When you "like" someone, that isn't a one time thing. But, いやだ could be temporary. This is one reason why you can say 嫌がる but not 好きがる. }

\par{ Although this suffix is one method of getting around third person restrictions on emotion phrase. }

\par{2. わたしがアイスクリームを食べたがると、母が食べさせてくれた。 \hfill\break
Whenever I wanted to eat ice cream, my mother would let me eat it. }

\par{\textbf{Phrase Note }: An easy example of debunking the claim that ~がる is used only in third person is the pattern ~がり屋, which is used to describe people's natures. So, if you're hot-nature, you're an 暑がり屋. }

\par{ Using 食べたい would present an even worse grammatical error, which is using the conjunctive particle と (conditional form) with an adjective. So, what does ~がる mean? In most cases, it assesses the outward appearance or overall knowledge of something and relates the situation with the internal state of the person in question. }

\par{3. 遼太朗は顔には出さなかったが、心の中では悔しがっていた。 \hfill\break
Ryotaro didn't let out, but he was regretting inside. }

\par{ It may also be the case that it is used to help show an internal state not shown outwardly, and this can be referring to an attitude being floated by the person in question or an attitude grasped by someone else. }

\par{4. 遼太朗は表面上は悔しがったが、心の中では喜んでいた。 \hfill\break
Ryotaro was regretting outwardly, but he was rejoicing inside. }

\begin{center}
 \textbf{Examples }
\end{center}

\par{5. 俺だけが面白がってたのか。 \hfill\break
Was I the only one who thought (that) was funny. }

\par{Note: This example shows how ~がる may be used to show a situation different than everything else. }

\par{6. 弟はおもちゃを見ると、いつもほしがる。 \hfill\break
My little brother always wants a toy when he sees it. }

\par{Note: This example shows how ~がる may be used to show someone being unreasonable. }

\par{7. 人はみな他人の事情を知りたがる。 \hfill\break
People want to know about everyone else's situation. }

\par{8. 生徒は新しい単語や言い回しを知りたがりますが、まずは基本を教えることが大事です。 \hfill\break
Students want to know new words and expressions, but it's important to first teach the basics. }

\par{9. 寒がり ${\overset{\textnormal{や}}{\text{屋}}}$ だから、カナダの ${\overset{\textnormal{く}}{\text{暮}}}$ らしは ${\overset{\textnormal{つら}}{\text{辛}}}$ かった。 \hfill\break
Because I'm cold-nature, life in Canada was harsh. }

\par{10. ${\overset{\textnormal{あつ}}{\text{暑}}}$ がり屋だから、テキサスの暮らしはひどく辛かったよ! \hfill\break
Because I'm hot-nature, life in Texas was extremely harsh! }

\par{11. べつにすまながらなくてもいい。 \hfill\break
You really don't have to feel sorry. \hfill\break
From 海辺のカフカ by 村上春樹. }

\par{12. 子供が食事を食べたがらない。 \hfill\break
My child doesn't want to eat dinner. }

\par{${\overset{\textnormal{}}{\text{13. 彼女}}}$ はアイスクリームを ${\overset{\textnormal{}}{\text{食}}}$ べたがっている。 \hfill\break
She wants to eat ice cream. }

\par{14. ${\overset{\textnormal{だれ}}{\text{誰}}}$ でも ${\overset{\textnormal{かみ}}{\text{神}}}$ を ${\overset{\textnormal{しん}}{\text{信}}}$ じたがります。 \hfill\break
Everyone wants to believe in God. }

\par{\textbf{Grammar Note }: ~たがる is used here because this is a general statement that may not be 100\% true. ~がる gives a sense that someone\slash  feels or thinks that way. }

\par{参照: https:\slash \slash www.lang.nagoya-u.ac.jp\slash nichigen\slash menu7\_ folder\slash symposium\slash pdf\slash 8\slash 06.pdf }

\begin{center}
 \textbf{Aside from }\textbf{~がる }
\end{center}

\par{ Of course, there are other things you can use. For instance, ~そうだ after the stems of adjectives means "seems\dothyp{}\dothyp{}\dothyp{}.". You may use かもしれない (might) or だろう・でしょう (probably). Of course, there are many other things. At the individual phrase level, however, you will see that some things should just never be used for oneself and others that should only be used in reference to others. }

\par{15a. 私は怒っている。 △ \hfill\break
15b. 私は腹が立っている。〇 \hfill\break
I'm angry. }

\par{16. 彼女は私のことを怒っていた。 \hfill\break
She was angry at me. }

\par{17. いや、(彼は)僕のことを怒ってるかもね。(Casual) \hfill\break
Well, (he) might actually be angry at me. }

\par{18. 山田さんはとても悲しそうですね。 \hfill\break
Yamada-san seems so sad, doesn't she? }

\begin{center}
 \textbf{Attribute: No Restrictions }
\end{center}

\par{ Words that refer to someone's wants, feelings, or likes don't need another pattern when used as an attribute\slash general knowledge (総合的な知識). Thankfully there are situations where the first\slash second and third person grammatical distinctions aren't necessary. This also goes for with situations reflecting on the past where something is already known. }

\par{19. 母が好きな食べ物は寿司です。 \hfill\break
The food my mother likes is sushi. }

\par{20. 若い人たちは海外に出たいのだ。 \hfill\break
Young people want to go overseas. }

\par{21. 弟はそのごろ韓国に行きたかったです。 \hfill\break
My younger brother wanted to go to Korea then. \hfill\break
\hfill\break
22. 参加したい人は図書館に行ってください。 \hfill\break
Those that want to participate, please go to the library. }

\par{23. 日本では ${\overset{\textnormal{うらな}}{\text{占}}}$ い ${\overset{\textnormal{し}}{\text{師}}}$ に ${\overset{\textnormal{うんせい}}{\text{運勢}}}$ をみてもらいたい人が多いみたいです。 \hfill\break
In Japan, it seems that a lot of people want their fortunes read by fortune-tellers. }

\par{24. すぐやりたがる人はいやだ。 \hfill\break
People that want to immediately do it are a pain. }
 
\par{\textbf{Grammar Note }: This doesn't give you a 100\% free pass to not use ~がる when the situation calls for it. }
Practice (2): \hfill\break
\hfill\break
1. She is afraid of spiders. \hfill\break
2. I am very sad. \hfill\break
3. The teacher is afraid of ghosts. \hfill\break
4. She hates going to the countryside. \hfill\break
5. He seemed very happy. \hfill\break
Practice (2): \hfill\break
\hfill\break
1. She is afraid of spiders. \hfill\break
2. I am very sad. \hfill\break
3. The teacher is afraid of ghosts. \hfill\break
4. She hates going to the countryside. \hfill\break
5. He seemed very happy. \hfill\break
    