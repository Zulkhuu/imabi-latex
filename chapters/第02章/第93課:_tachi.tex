    
\chapter{Pluralization II}

\begin{center}
\begin{Large}
第93課: Pluralization II: ~たち 
\end{Large}
\end{center}
   Japanese is well known for not having an explicitly built-in distinction between the singular and plural forms. This does not mean that Japanese people cannot distinguish between these two concepts. Rather, the way this contrast manifests in Japanese is rather complex. Of course, there are many instances in which the distinction is vague to say the least. Needless to say, it is fair to say that a level of cognitive awareness is needed to not get overly confused. \hfill\break
 
\par{Note: It\textquotesingle s not fair to say that English perfectly distinguishes singular and plural forms, though. After all, we have words like “fish” and deer” that are both. So, thinking about how we overcome such ambiguity in English with these words may help you understand what is going on in Japanese. }
      
\section{~タチの使い方}
 
\par{ Overall, there are several methods of pluralization or structures resembling pluralization in Japanese. Consider the following: }

\begin{enumerate}

\item Noun Repetition \hfill\break
Ex. 山々, 木々, 神々, Etc. 
\item Suffixes \hfill\break
-達: 男達、子供達、学生達 \hfill\break
-等: 子供等、友人等、彼等、彼女等 \hfill\break
-ども: 鬼ども、若造ども、悪漢ども、犬ども、野郎ども \hfill\break
-がた: あなたがた 
\item Adding qualifiers to noun phrases: \hfill\break
色々な人, 様々な場所, たくさんの外国人 
\item Prefixes to Sino-Japanese Phrase \hfill\break
多文化, 多民族, 諸言語 
\item Implied by Certain Verbs and Adjectives \hfill\break
集まる, 散る, 片付ける, 多い 
\item Counter phrases \hfill\break
本が10冊ある, 女性が5人いる 
\end{enumerate}

\par{ This lesson will focus on ~たち as it is arguably the most difficult to master as its necessity is also something that is not straightforward. If it were simply a plural marker, we could just end there, but it isn't. }

\begin{center}
 \textbf{タチの性質(たち) }
\end{center}

\par{ Historically, ~たち started out as a suffix for the nobility and the gods, but from this historical note alone, we know that it has never been just for simple pluralization purposes. In fact, when it is used today, whether it is used with nouns of living things or non-living things, it is OK not to use it. However, using ~たち often shows a sense of empathy. }

\par{1. 男性(たち)が4人銃で撃たれて病院に搬送された。 \hfill\break
Four men were shot and taken to the hospital. }

\par{2. 色々な人(たち)が物理学学会に出席した。 \hfill\break
A lot of interesting people attended the physics convention. }

\par{3. テキサス州の州都、オースティン市には数多くのホームレスの人(たち)がいる。 \hfill\break
In Austin, the state capital of Texas, there are many homeless people. }

\par{ In the next set of sentences, though, ~たち is not optional. }

\par{4. 私たちは今年も、全国各地から集まってきました。 \hfill\break
We have gathered this year as well from all over the country. }

\par{5. 池田さんたち3人は連帯意識が強い。 \hfill\break
Ikeda and his group of three have a strong common bond. }

\par{6. あなたたちがいないと、私は生きていけないのです。 \hfill\break
I can't live on without you all. }

\par{ The plural meanings of these phrases cannot be supplemented by the basic phrases themselves without ~たち. For phrases like 池田さんたち, one is no longer just using a typical third person pronoun phrase, but as Makino (1996) has stated, but a part of one's 'uchi'. Thus, the term ウチ人称 becomes appropriate. }

\par{ We continue to see that an emphatic role is present in ~たち\textquotesingle s use. It may be clearly obligatory with pronouns, but why? Is it because we have a keen emotional awareness towards the number of pronouns? This is likely the case, and this factor appears to be what controls when ~たち is used regardless of what kind of noun is being used. }

\begin{center}
\textbf{Example Sentences }
\end{center}

\par{ Now it's time to look at example sentences. To be fair and objective, all example sentences will be from Google searches. The reason for this is that example phrases in isolation may be deemed ungrammatical without context, and native speakers themselves may not agree with whether any given expression is used or not. }

\par{7. 30年の眠りから ${\overset{\textnormal{さ}}{\text{覚}}}$ めた木々たち \hfill\break
These trees which have awoken from a 30 year slumber }

\par{8. ${\overset{\textnormal{ふぞろ}}{\text{不揃}}}$ いのリンゴたち \hfill\break
Uneven apples }

\par{9. そこの ${\overset{\textnormal{かたすみ}}{\text{片隅}}}$ に咲いているバラたちも ${\overset{\textnormal{か}}{\text{枯}}}$ れていくのでしょう。 \hfill\break
Those roses which have bloomed in the corner will also surely fade away. }

\par{10. 遠くの星々たちは誰にも知られずとも ${\overset{\textnormal{かがや}}{\text{輝}}}$ く。 \hfill\break
The stars far away will shine despite being not being known by anyone. }

\par{11. 私の住む京の地で、様々な場所から ${\overset{\textnormal{とら}}{\text{捉}}}$ えた ${\overset{\textnormal{さくら}}{\text{桜}}}$ たちの ${\overset{\textnormal{ひょうじょう}}{\text{表情}}}$ をご ${\overset{\textnormal{らん}}{\text{覧}}}$ ください。 \hfill\break
Look at the expressions of cherry blossoms I took from various places at my home place, Kyoto. }

\par{12. ${\overset{\textnormal{なら}}{\text{奈良}}}$ ${\overset{\textnormal{こうえん}}{\text{公園}}}$ の ${\overset{\textnormal{しか}}{\text{鹿}}}$ ${\overset{\textnormal{たち}}{\text{達}}}$ は ${\overset{\textnormal{だいたん}}{\text{大胆}}}$ に道路を ${\overset{\textnormal{せんりょう}}{\text{占領}}}$ して過ごしている! \hfill\break
The deer in Nara Park have boldly taken over the roads! }

\par{13. ${\overset{\textnormal{きつね}}{\text{狐}}}$ たちの ${\overset{\textnormal{みや}}{\text{宮}}}$ へ \hfill\break
To the temple of foxes }

\par{14. ${\overset{\textnormal{たぬき}}{\text{狸}}}$ たちの ${\overset{\textnormal{ていこう}}{\text{抵抗}}}$ と ${\overset{\textnormal{はいぼく}}{\text{敗北}}}$ の物語です。 \hfill\break
This is a story about the resistance and defeat of the tanuki. }

\par{15. 猫たちの ${\overset{\textnormal{しゅうだん}}{\text{集団}}}$ がやってきました。 \hfill\break
A group of cats have arrived. }

\par{16. ${\overset{\textnormal{すずめ}}{\text{雀}}}$ たちは ${\overset{\textnormal{いっせい}}{\text{一斉}}}$ に鳴き始めた。 \hfill\break
The sparrows began to cry at once. }

\par{17. 我が ${\overset{\textnormal{や}}{\text{家}}}$ の ${\overset{\textnormal{にわとり}}{\text{鶏}}}$ たちを ${\overset{\textnormal{と}}{\text{撮}}}$ ってみました。 \hfill\break
I took pictures of the chickens at my home. }

\par{18. 春にやって来て ${\overset{\textnormal{はんしょく}}{\text{繁殖}}}$ する鳥たちを ${\overset{\textnormal{なつどり}}{\text{夏鳥}}}$ と呼びます。 \hfill\break
We call birds who come to reproduce in the spring “summer birds”. }

\par{19. 蝉たちが死にかけている。 \hfill\break
The cicadas are dying. }

\par{20. 強風に巻き上げられた ${\overset{\textnormal{ほたる}}{\text{蛍}}}$ たちが光の水しぶきのように ${\overset{\textnormal{ふ}}{\text{降}}}$ り ${\overset{\textnormal{そそ}}{\text{注}}}$ ぐ。 \hfill\break
The fireflies caught up in the gale wind are raining down like a mist of light. }

\par{21. 池の ${\overset{\textnormal{こい}}{\text{鯉}}}$ たちが元気に泳いでいます。 \hfill\break
The coy fish in the pond are swimming lively. }

\par{22. 高級な日本酒を飲みながら、思い ${\overset{\textnormal{ぞんぶん}}{\text{存分}}}$ 魚たちを見ることができます。 \hfill\break
You can watch the fish to one\textquotesingle s heart\textquotesingle s desire as you drink high class sake. }

\par{23. 絵のような雲たち \hfill\break
Group of clouds which resemble a painting }

\par{24. この雪たちのように、 ${\overset{\textnormal{と}}{\text{溶}}}$ けてなくなってしまって欲しい思い出たち。 \hfill\break
These memories which I wish would melt and disappear like this snow. }

\par{ A lot of other nouns may require a lot of personification to work. A built-up sense of empathy is needed in such a situation. }

\par{25. あたしが大好きな本たちを紹介しますわ。 \hfill\break
I'm going to introduce you to my precious books! }

\par{参照文献:認知世界の窓としての日本語の複数標示‐たち by 牧野成一 }
    