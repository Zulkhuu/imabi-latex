    
\chapter{The Particle で II}

\begin{center}
\begin{Large}
第90課: The Particle で II 
\end{Large}
\end{center}
 
\par{ This lesson will finish our coverage on the most fundamental uses of the particle で. They are really different from the one's you've already learned. So, try not to confuse them with each other! This lesson will conclude with an exercise that will challenge you to use all your knowledge of the combined four uses covered in this lesson and the previous lesson on で. }
      
\section{The Particle で Continued}
 
\par{1. で shows the basis of an action\slash event not of one's volition--”due to". It's used with nouns that indicate things such as natural phenomena, events, illnesses, etc.  The verb should not show any will of someone. This is because nothing and no one can cause natural phenomena. }

\par{1. 彼は ${\overset{\textnormal{しんふぜん}}{\text{心不全}}}$ で ${\overset{\textnormal{}}{\text{死}}}$ んだ。 \hfill\break
He died due to heart failure. }

\par{2. ${\overset{\textnormal{ひがいしゃ}}{\text{被害者}}}$ は ${\overset{\textnormal{しゅっけつたりょう}}{\text{出血多量}}}$ で ${\overset{\textnormal{}}{\text{死}}}$ にました。 \hfill\break
The victim died from loss of blood. }

\par{3. その ${\overset{\textnormal{かじ}}{\text{火事}}}$ はマッチでの ${\overset{\textnormal{ひあそ}}{\text{火遊}}}$ びが ${\overset{\textnormal{げんいん}}{\text{原因}}}$ で ${\overset{\textnormal{お}}{\text{起}}}$ こりました。 \hfill\break
The fire was caused by playing with matches. }

\par{4. ${\overset{\textnormal{ぶし}}{\text{武士}}}$ は ${\overset{\textnormal{けが}}{\text{怪我}}}$ で ${\overset{\textnormal{}}{\text{死}}}$ んだ。 \hfill\break
The warrior died from his wounds. }

\par{${\overset{\textnormal{}}{\text{5. 学校}}}$ はクリスマスで ${\overset{\textnormal{}}{\text{休}}}$ みになりました。 \hfill\break
The school has closed for Christmas\slash the holidays. }

\par{${\overset{\textnormal{}}{\text{6. 寒}}}$ さで ${\overset{\textnormal{ふる}}{\text{震}}}$ える。 \hfill\break
To shake from the cold. }

\par{${\overset{\textnormal{}}{\text{7. 私}}}$ は ${\overset{\textnormal{}}{\text{宿題}}}$ で ${\overset{\textnormal{}}{\text{忙}}}$ しいです。 \hfill\break
I'm busy with homework. }

\par{8. ${\overset{\textnormal{よろこ}}{\text{喜}}}$ びで ${\overset{\textnormal{われ}}{\text{我}}}$ を ${\overset{\textnormal{}}{\text{忘}}}$ れました。 \hfill\break
I forgot myself from joy. }

\par{\textbf{Pronoun Note }: 我 is used here like a set phrase. Normally, you just don't get to use 我 whenever you want. }

\par{9. 悲しい気持ちでいっぱいです。〇\slash △ \hfill\break
I'm filled with sad emotions. }

\par{\textbf{Naturalness Note }: Not all speakers like this phrase, but 〇〇(という)気持ちでいっぱい is becoming very common these days. Speakers who find this phrase unnatural would replace it with something like 悲しいです or とても悲しく思っています. }

\par{2. で shows an extent which may create juncture. Juncture deals with a point in time or place. However, it is not definite in nature as the particle に. This does not mean phrases like 一秒で are impossible. The purpose of に is to show exact time. Think of the difference as "The ice fully melted in 3 hours, 4 minutes, and 32 seconds" vs "The ice fully melted in three hours". Juncture could also be used to show at what point something happens. The vagueness of this comment is on purpose. For instance, you'd use で to show at what temperature something boils or melts. }

\par{Translation isn't really important to focus on, but it usually translates to "at" or "for". As for other specific instances this で can be used in, it can show summation, which helps with stating prices like in Ex. 15 (this sentence would be said by say a person who knows you rather than a clerk). }

\par{ \textbf{Examples }}

\par{10. ${\overset{\textnormal{せかい}}{\text{世界}}}$ で ${\overset{\textnormal{}}{\text{一番高}}}$ い ${\overset{\textnormal{}}{\text{山}}}$ はエベレストです。 \hfill\break
The tallest mountain in the world is Mount Everest. }

\par{${\overset{\textnormal{}}{\text{11. 母}}}$ は ${\overset{\textnormal{はたち}}{\text{二十歳}}}$ で ${\overset{\textnormal{けっこん}}{\text{結婚}}}$ した。 \hfill\break
My mother got married at age 20. }

\par{${\overset{\textnormal{}}{\text{12. 明日}}}$ でお ${\overset{\textnormal{わか}}{\text{別}}}$ れだ。 \hfill\break
I leave you tomorrow. }

\par{${\overset{\textnormal{}}{\text{13. 自分}}}$ で ${\overset{\textnormal{}}{\text{考}}}$ えてください。 \hfill\break
Please think for yourself. }

\par{14. チームで ${\overset{\textnormal{たんとう}}{\text{担当}}}$ するのが ${\overset{\textnormal{す}}{\text{好}}}$ きじゃない。 \hfill\break
I don\textquotesingle t like managing with the team. }

\par{15. ${\overset{\textnormal{ぜんぶ}}{\text{全部}}}$ で ${\overset{\textnormal{}}{\text{500}}}$ ${\overset{\textnormal{}}{\text{円}}}$ ですよ。 \hfill\break
It's 500 yen with everything. }

\par{16. 10 ${\overset{\textnormal{}}{\text{時}}}$ で ${\overset{\textnormal{へいてん}}{\text{閉店}}}$ です。 \hfill\break
We will close at 10. }

\par{${\overset{\textnormal{}}{\text{17a. 彼}}}$ は ${\overset{\textnormal{}}{\text{百歳}}}$ で ${\overset{\textnormal{な}}{\text{亡}}}$ くなりました。 \hfill\break
17b. 彼は ${\overset{\textnormal{きょうねん}}{\text{享年}}}$ 100 ${\overset{\textnormal{さい}}{\text{歳}}}$ でした。 \hfill\break
He passed away at 100 years old. }

\par{${\overset{\textnormal{}}{\text{18. 水}}}$ は0 ${\overset{\textnormal{ど}}{\text{℃}}}$ で ${\overset{\textnormal{こお}}{\text{凍}}}$ る。 \hfill\break
Water freezes at 0℃. }

\par{19. 一秒で分かった! \hfill\break
I found out in a second! }

\par{\textbf{Practice }: Translate the following. If it's in English translate it into Japanese (polite speech). If it's in Japanese, translate it in English. \hfill\break
\hfill\break
1. In a week it becomes summer vacation. \hfill\break
2. あなたの ${\overset{\textnormal{とけい}}{\text{時計}}}$ では ${\overset{\textnormal{}}{\text{何時}}}$ ですか。 \hfill\break
3. You make butter with milk. \hfill\break
4. その ${\overset{\textnormal{}}{\text{本}}}$ を3 ${\overset{\textnormal{}}{\text{千円}}}$ で ${\overset{\textnormal{}}{\text{買}}}$ いました。 \hfill\break
5. ガラスで ${\overset{\textnormal{}}{\text{指}}}$ を ${\overset{\textnormal{}}{\text{切}}}$ る。 \hfill\break
6. 2つで500 ${\overset{\textnormal{}}{\text{円}}}$ です。 \hfill\break
7. Can you do it in 20 minutes? \hfill\break
8. ${\overset{\textnormal{}}{\text{学校}}}$ に ${\overset{\textnormal{}}{\text{自転車}}}$ で ${\overset{\textnormal{かよ}}{\text{通}}}$ う。 \hfill\break
9. He broke it due to carelessness. \hfill\break
10. Please write in pen. \hfill\break
11. 10 ${\overset{\textnormal{}}{\text{時}}}$ になった。 \hfill\break
12. To end with sadness. \hfill\break
13. ${\overset{\textnormal{}}{\text{雨}}}$ で ${\overset{\textnormal{ちゅうし}}{\text{中止}}}$ になりました。 \hfill\break
14. ${\overset{\textnormal{}}{\text{雪}}}$ でバスが ${\overset{\textnormal{}}{\text{来}}}$ ませんでした。 \hfill\break
15. I went by myself. }
      
\section{The Conjunction で}
 
\par{ The conjunction で is a contraction of それで, which utilizes usage 4 from above from the sense of juncture (connecting sentences) in a sense related to reasoning. This makes it very similar to the particle ので, which may also be found starting a sentence in なので. The use of なので in this manner is relatively new, and a lot of people think it is wrong. So, keep this in mind as well. }

\par{ However, unlike で, the speaker is not intending on simply responding and or trying to change the topic. なので is "so" as in "so, this happened" because of what is stated before it.  }

\par{20. で、 ${\overset{\textnormal{}}{\text{君}}}$ は ${\overset{\textnormal{だいじょうぶ}}{\text{大丈夫}}}$ ? \hfill\break
So, are you alright? }

\par{21. ${\overset{\textnormal{みな}}{\text{皆}}}$ その ${\overset{\textnormal{ほうあん}}{\text{法案}}}$ に ${\overset{\textnormal{どうい}}{\text{同意}}}$ しませんでした。なので、 ${\overset{\textnormal{せいじか}}{\text{政治家}}}$ は ${\overset{\textnormal{だいあん}}{\text{代案}}}$ を ${\overset{\textnormal{だ}}{\text{出}}}$ しました。 \hfill\break
Everyone didn't agree to the bill. So, the politicians proposed an alternative plan. }

\par{22. で、どうなったん(っ)すか。(Colloquial) \hfill\break
So, what happened? }
      
\section{Key}
 
\par{Practice }

\par{1. ${\overset{\textnormal{あといっしゅうかん}}{\text{後一週間}}}$ で ${\overset{\textnormal{}}{\text{夏休}}}$ みになります。 \hfill\break
2. What's the time on your watch? \hfill\break
3. バターはミルクで ${\overset{\textnormal{}}{\text{作}}}$ ります。 \hfill\break
4. I bought that book for 3,000 yen. \hfill\break
5. To cut a finger on glass. \hfill\break
6. It is 500 yen for two. \hfill\break
7. 20 ${\overset{\textnormal{ぷん}}{\text{分}}}$ でできますか。 \hfill\break
8. I commute to school by bicycle. \hfill\break
9. ${\overset{\textnormal{}}{\text{彼}}}$ は ${\overset{\textnormal{ふちゅうい}}{\text{不注意}}}$ でそれを ${\overset{\textnormal{こわ}}{\text{壊}}}$ しました。 \hfill\break
10. ペンで ${\overset{\textnormal{}}{\text{書}}}$ いてください。 \hfill\break
11. It became 10 o' clock. \hfill\break
12. ${\overset{\textnormal{}}{\text{悲}}}$ しみで ${\overset{\textnormal{}}{\text{終}}}$ わる。 \hfill\break
Note: Remember that the predicate function is the 終止形. -ます would change the definition. \hfill\break
13. It was postponed due to the rain. \hfill\break
14. The bus didn't come due to (the) snow. }

\par{15. ${\overset{\textnormal{}}{\text{一人}}}$ で ${\overset{\textnormal{}}{\text{行}}}$ きました。  }
    