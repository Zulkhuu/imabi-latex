    
\chapter{Counters VI}

\begin{center}
\begin{Large}
第98課: Counters VI: 羽, 部, 戸, 発, 軒, 人前, 丁, 本, 点, 泊, 通, 尾, 駅, 億, \& 兆 
\end{Large}
\end{center}
 
\par{ In this lesson on counters, we will learn about fifteen more counters used very frequently. With a total of fifteen more counters in your repertoire, you\textquotesingle ll be closer to properly counting most of what you\textquotesingle d need to count in daily conversation. }
 
\begin{center}
\textbf{Counters Covered in this Lesson }
\end{center}
 
\par{1.       - \emph{wa }羽 \hfill\break
2.       - \emph{bu }部 \hfill\break
3.       - \emph{ko }戸 \hfill\break
4.       - \emph{hatsu }発 \hfill\break
5.       - \emph{ken }軒 \hfill\break
6.       - \emph{nimmae }人前 \hfill\break
7.       - \emph{chō }丁 \hfill\break
8.       - \emph{hon }本 \hfill\break
9.       - \emph{ten }点 \hfill\break
10.   - \emph{haku }泊 \hfill\break
11.   - \emph{tsū }通 \hfill\break
12.   - \emph{bi }尾 \hfill\break
13.   - \emph{eki }駅 \hfill\break
14.   - \emph{oku }億 \hfill\break
15.   - \emph{chō }兆 }
 Counter expressions, when there are multiple readings, are listed from most to least common.  Options are put in bold to mark the most commonly used form whenever there is ambiguity concerning this.       
\section{More Sino-Japanese Counters}
 
\begin{center}
\textbf{- \emph{wa }羽 }
\end{center}

\par{ The counter - \emph{wa }羽 counts birds and rabbits. This includes large birds and small birds that don\textquotesingle t fly such as ostriches and penguins. However, large birds are sometimes counted with - \emph{tō }頭. }

\par{ Rabbits can alternatively be counted with - \emph{hiki }匹 but can be counted with - \emph{wa }羽 due to being treated as rabbits in traditional Buddhist beliefs so that their meat along with poultry wouldn\textquotesingle t be outlawed. }

\begin{ltabulary}{|P|P|P|P|P|P|P|P|}
\hline 

1 & いちわ & 2 & にわ & 3 &  \textbf{さんわ \hfill\break
}\textbf{ }さんば & 4 &  \textbf{よんわ \hfill\break
}\textbf{ }よんば \\ \cline{1-8}

5 & ごわ & 6 &  \textbf{ろっぱ \hfill\break
}\textbf{ }ろくわ & 7 &  \textbf{ななわ \hfill\break
}\textbf{ }しちわ & 8 &  \textbf{はちわ \hfill\break
}\textbf{ }はっぱ 
\\ \cline{1-8}

9 & きゅうわ & 10 &  \textbf{じゅっぱ \hfill\break
}\textbf{ }じゅうわ \hfill\break
じっぱ & 100 &  \textbf{ひゃっぱ \hfill\break
}\textbf{ }ひゃくわ & 1000 &  \textbf{せんば \hfill\break
}\textbf{ }せんわ \\ \cline{1-8}

10000 &  \textbf{いちまんば \hfill\break
}\textbf{ }いちまんわ & ? &  \textbf{なんわ \hfill\break
}\textbf{ }なんば &  &  &  &  \\ \cline{1-8}

\end{ltabulary}

\par{\textbf{Reading Note }: There are some expressions with set readings that utilize one reading over other options (Ex . ${\overset{\textnormal{せんばづる}}{\text{千羽鶴}}}$ : String of 1000 paper cranes). }

\par{1. ニワトリ ${\overset{\textnormal{あ}}{\text{合}}}$ わせて ${\overset{\textnormal{ろくじゅう}}{\text{68}}}$ ${\overset{\textnormal{わ}}{\text{羽}}}$ が ${\overset{\textnormal{し}}{\text{死}}}$ んでいるのが ${\overset{\textnormal{み}}{\text{見}}}$ つかりました。 \hfill\break
 \emph{Niwatori awasete rokujūhachiwa ga shinde iru no ga mitsukarimashita. \hfill\break
 }A total of 68 chickens were found dead. }

\par{\textbf{Spelling Note }: \emph{Niwator }i is occasionally spelled as 鶏. \hfill\break
 \hfill\break
2. ${\overset{\textnormal{でんせん}}{\text{電線}}}$ に ${\overset{\textnormal{とり}}{\text{鳥}}}$ が ${\overset{\textnormal{じゅっ}}{\text{10}}}$ ${\overset{\textnormal{ぱ}}{\text{羽}}}$ ${\overset{\textnormal{と}}{\text{止}}}$ まっていました。 \hfill\break
 \emph{Densen ni tori ga juppa tomatte imashita. \hfill\break
 }Ten birds were stopped on the electric lines. }

\par{3. カモメが ${\overset{\textnormal{ろっ}}{\text{6}}}$ ${\overset{\textnormal{ぱ}}{\text{羽}}}$ ${\overset{\textnormal{なら}}{\text{並}}}$ んで ${\overset{\textnormal{と}}{\text{止}}}$ まっていました。 \hfill\break
 \emph{Kamome ga roppa narande tomatte imashita. \hfill\break
 }There were six seagulls stopped and lined up together. }

\par{\textbf{Spelling Note }: Kamome is occasionally spelled as 鴎・鷗. }

\par{4. ${\overset{\textnormal{たてものよこ}}{\text{建物横}}}$ の ${\overset{\textnormal{ほり}}{\text{堀}}}$ には、アヒルが ${\overset{\textnormal{さん}}{\text{3}}}$ ${\overset{\textnormal{わ}}{\text{羽}}}$ いました。 \hfill\break
 \emph{Tatemono yoko no hori ni wa, ahiru ga sanwa imashita. \hfill\break
 }There were three ducks in the ditch on the side of the building. }

\par{\textbf{Spelling Note }: \emph{Ahiru }is only seldom spelled as 家鴨. }

\begin{center}
\textbf{- \emph{bu }部 }
\end{center}

\par{ The counter - \emph{bu }部 counts copies of printed materials such as magazines and newspapers. It may also count packets of paper. }

\begin{ltabulary}{|P|P|P|P|P|P|P|P|}
\hline 

1 & いちぶ & 2 & にぶ & 3 & さんぶ & 4 & よんぶ \\ \cline{1-8}

5 & ごぶ & 6 & ろくぶ & 7 & ななぶ & 8 & はちぶ \\ \cline{1-8}

9 & きゅうぶ & 10 & じゅうぶ & 100 & ひゃくぶ & ? & なんぶ \\ \cline{1-8}

\end{ltabulary}

\par{\textbf{Phrase Note }: \emph{Ichibu }一部 is commonly used to also mean “one part\slash portion.” }

\par{5. ${\overset{\textnormal{いちにち}}{\text{一日}}}$ で ${\overset{\textnormal{せん}}{\text{1000}}}$ ${\overset{\textnormal{ぶ}}{\text{部}}}$ ${\overset{\textnormal{う}}{\text{売}}}$ りました。 \hfill\break
 \emph{Ichinichi de sembu urimashita. \hfill\break
 }I sold 1000 copies in a day. }

\par{6. ${\overset{\textnormal{りれき}}{\text{履歴}}}$ の ${\overset{\textnormal{いちぶ}}{\text{一部}}}$ だけ ${\overset{\textnormal{さくじょ}}{\text{削除}}}$ した。 \hfill\break
 \emph{Rireki no ichibu dake sakujo shita. \hfill\break
 }I deleted only a part of my history. }

\par{7. パンフレットを ${\overset{\textnormal{じゅう}}{\text{10}}}$ ${\overset{\textnormal{ぶ}}{\text{部}}}$ ${\overset{\textnormal{おく}}{\text{送}}}$ りました。 \hfill\break
 \emph{Panfuretto wo jūbu okurimashita. \hfill\break
 }I sent ten copies of the pamphlet. }

\begin{center}
\textbf{- \emph{ko }戸 }
\end{center}

\par{ The counter - \emph{ko }戸 counts houses. }

\begin{ltabulary}{|P|P|P|P|P|P|P|P|}
\hline 

1 & いっこ & 2 & にこ & 3 & さんこ & 4 & よんこ \\ \cline{1-8}

5 & ごこ & 6 & ろっこ & 7 & ななこ & 8 &  \textbf{はっこ \hfill\break
}\textbf{ }はちこ \\ \cline{1-8}

9 & きゅうこ & 10 &  \textbf{じゅっこ \hfill\break
}\textbf{ }じっこ & 100 & ひゃっこ & ? & なんこ \\ \cline{1-8}

\end{ltabulary}

\par{\hfill\break
8. ${\overset{\textnormal{じゅっ}}{\text{10}}}$ ${\overset{\textnormal{こ}}{\text{戸}}}$ はまだ ${\overset{\textnormal{あ}}{\text{空}}}$ いています。 \hfill\break
 \emph{Jukko wa mada aite imasu. \hfill\break
 }Ten homes are still empty. }

\par{9. およそ ${\overset{\textnormal{ろくじゅっ}}{\text{60}}}$ ${\overset{\textnormal{こ}}{\text{戸}}}$ が ${\overset{\textnormal{ていでん}}{\text{停電}}}$ していました。 \hfill\break
 \emph{Oyoso rokujukko ga teiden shite imashita. \hfill\break
 }Approximately sixty homes had experienced a power outage. }

\begin{center}
\textbf{- \emph{hatsu }発 }
\end{center}

\par{ The counter - \emph{hatsu }発 counts bullets, explosions, attacks, sneezes, and anything that can colloquially be described as explosive. }

\begin{ltabulary}{|P|P|P|P|P|P|P|P|}
\hline 

1 & いっぱつ & 2 & にはつ & 3 & さんぱつ & 4 & よんぱつ \\ \cline{1-8}

5 & ごはつ & 6 & ろっぱつ & 7 & ななはつ & 8 & はっぱつ \\ \cline{1-8}

9 & きゅうはつ & 10 &  \textbf{じゅっぱつ \hfill\break
}\textbf{ }じっぱつ & 100 & ひゃっぱつ & ? & なんぱつ \\ \cline{1-8}

\end{ltabulary}

\par{\hfill\break
10. アメリカ ${\overset{\textnormal{ぐん}}{\text{軍}}}$ が ${\overset{\textnormal{げんばく}}{\text{原爆}}}$ を ${\overset{\textnormal{に}}{\text{2}}}$ ${\overset{\textnormal{はつ}}{\text{発}}}$ ${\overset{\textnormal{とうか}}{\text{投下}}}$ した。 \hfill\break
 \emph{Amerika-gun ga gembaku wo nihatsu tōka shita. \hfill\break
 }The U.S army dropped two nuclear bombs. }

\par{11. ${\overset{\textnormal{かれ}}{\text{彼}}}$ は ${\overset{\textnormal{たま}}{\text{弾}}}$ を ${\overset{\textnormal{じゅっ}}{\text{10}}}$ ${\overset{\textnormal{ぱつう}}{\text{発撃}}}$ った。 \hfill\break
 \emph{Kare wa tama wo juppatsu utta. \hfill\break
 }I shot ten bullets. }

\par{12. あのアライグマを ${\overset{\textnormal{に}}{\text{2}}}$ 、 ${\overset{\textnormal{さん}}{\text{3}}}$ ${\overset{\textnormal{ぱつ}}{\text{発}}}$ ${\overset{\textnormal{け}}{\text{蹴}}}$ った。 \hfill\break
 \emph{Ano araiguma wo ni, sampatsu ketta. \hfill\break
 }I kicked that raccoon two, three times. }

\par{13. くしゃみを ${\overset{\textnormal{ご}}{\text{5}}}$ ${\overset{\textnormal{はつ}}{\text{発}}}$ した ${\overset{\textnormal{あと}}{\text{後}}}$ で、 ${\overset{\textnormal{はなぢ}}{\text{鼻血}}}$ を ${\overset{\textnormal{だ}}{\text{出}}}$ した。 \hfill\break
 \emph{Kushami wo gohatsu shita ato de, hanaji wo dashita. \hfill\break
 }I had a nosebleed after sneezing three times. }
\textbf{- \emph{ken }軒 }
\par{  The counter - \emph{ken }軒 counts individual houses that are not connected to each other. This is different from - \emph{ko }戸, which generically counts homes. }

\begin{ltabulary}{|P|P|P|P|P|P|P|P|}
\hline 

1 & いっけん & 2 & にけん & 3 &  \textbf{さんけん \hfill\break
}\textbf{ }さんげん & 4 & よんけん \\ \cline{1-8}

5 & ごけん & 6 & ろっけん & 7 & ななけん & 8 & はっけん \\ \cline{1-8}

9 & きゅうけん & 10 &  \textbf{じゅっけん \hfill\break
}\textbf{ }じっけん & 100 & ひゃっけん & ? &  \textbf{なんけん \hfill\break
}\textbf{ }なんげん \\ \cline{1-8}

\end{ltabulary}

\par{1 4. ${\overset{\textnormal{さん}}{\text{3}}}$ ${\overset{\textnormal{けん}}{\text{軒}}}$ に ${\overset{\textnormal{いっ}}{\text{1}}}$ ${\overset{\textnormal{けん}}{\text{軒}}}$ が ${\overset{\textnormal{あ}}{\text{空}}}$ き ${\overset{\textnormal{や}}{\text{家}}}$ になっている。 \hfill\break
\emph{Sanken ni ikken ga akiya ni natte iru. \hfill\break
}One in three houses are vacant. }

\par{15. ${\overset{\textnormal{きのう}}{\text{昨日}}}$ の ${\overset{\textnormal{じしん}}{\text{地震}}}$ で、 ${\overset{\textnormal{じゅっ}}{\text{10}}}$ ${\overset{\textnormal{けん}}{\text{軒}}}$ ${\overset{\textnormal{ちゅう}}{\text{中}}}$ ${\overset{\textnormal{よん}}{\text{4}}}$ ${\overset{\textnormal{けん}}{\text{軒}}}$ ${\overset{\textnormal{こわ}}{\text{壊}}}$ れました。 \hfill\break
 \emph{Kinō no jishin de, jukken-chū yonken kowaremashita. \hfill\break
 }Four out of ten houses were destroyed in yesterday\textquotesingle s earthquake. }

\par{16. ${\overset{\textnormal{しせんおおじしん}}{\text{四川大地震}}}$ で ${\overset{\textnormal{なな}}{\text{7}}}$ ${\overset{\textnormal{まんけんいじょう}}{\text{万軒以上}}}$ が ${\overset{\textnormal{とうかい}}{\text{倒壊}}}$ して ${\overset{\textnormal{ふた}}{\text{2}}}$ ${\overset{\textnormal{り}}{\text{人}}}$ が ${\overset{\textnormal{しぼう}}{\text{死亡}}}$ しました。 \hfill\break
 \emph{Shisen Ōjishin de nanamanken ijō ga tōkai shite futari ga shibō shimashita. \hfill\break
 }Over 70,000 houses collapsed and two people died in the Great Sichuan Earthquake. }

\begin{center}
\textbf{- \emph{nimmae }人前 }
\end{center}

\par{ The counter - \emph{nimmae }人前 counts food portions. Despite beginning with 人, its readings are all regular. }

\begin{ltabulary}{|P|P|P|P|P|P|P|P|}
\hline 

1 & いちにんまえ & 2 & ににんまえ & 3 & さんにんまえ & 4 & よにんまえ \\ \cline{1-8}

5 & ごにんまえ & 6 & ろくにんまえ & 7 &  \textbf{しちにんまえ \hfill\break
}\textbf{ }ななにんまえ & 8 & はちにんまえ \\ \cline{1-8}

9 &  \textbf{きゅうにんまえ \hfill\break
}\textbf{ }くにんまえ & 10 & じゅうにんまえ & 100 & ひゃくにんまえ & ? & なんにんまえ \\ \cline{1-8}

\end{ltabulary}

\par{\hfill\break
17. ${\overset{\textnormal{さん}}{\text{3}}}$ ${\overset{\textnormal{にんまえ}}{\text{人前}}}$ の ${\overset{\textnormal{りょう}}{\text{量}}}$ を ${\overset{\textnormal{つく}}{\text{作}}}$ りました。 \hfill\break
 \emph{San\textquotesingle nimmae no ryō wo tsukurimashita. \hfill\break
 }I made three portions worth. }

\par{18. ${\overset{\textnormal{ぎょうざ}}{\text{餃子}}}$ を ${\overset{\textnormal{じゅう}}{\text{10}}}$ ${\overset{\textnormal{にんまえか}}{\text{人前買}}}$ ってきました。 \hfill\break
 \emph{Gyōza wo jūnimmae katte kimashita. \hfill\break
 }I\textquotesingle ve come back with ten portions of gyoza. }

\par{19. (お)そばを ${\overset{\textnormal{に}}{\text{2}}}$ ${\overset{\textnormal{にん}}{\text{人}}}$ ${\overset{\textnormal{まえちゅうもん}}{\text{前注文}}}$ しました。 \hfill\break
 \emph{(O-)soba wo ninimmae chūmon shimashita. \hfill\break
 }I ordered two portions of soba. }

\par{\textbf{Spelling Note }: \emph{Soba }is only seldom spelled as 蕎麦. }

\begin{center}
\textbf{- \emph{chō }丁 }
\end{center}

\par{ The counter - \emph{chō }丁 counts the following items: Tofu, guns, konnyaku, servings at a restaurant (especially ramen), scissors, candles on a candlestick, saws, town blocks, violins, guitars, ink sticks, palanquins, photo attempts,  rickshaws, etc. }

\begin{ltabulary}{|P|P|P|P|P|P|P|P|}
\hline 

1 & いっちょう & 2 & にちょう & 3 & さんちょう & 4 & よんちょう \\ \cline{1-8}

5 & ごちょう & 6 & ろくちょう & 7 & ななちょう & 8 & はっちょう \\ \cline{1-8}

9 & きゅうちょう & 10 &  \textbf{じゅっちょう \hfill\break
}\textbf{ }じっちょう & 100 & ひゃくちょう & ? & なんちょう \\ \cline{1-8}

\end{ltabulary}

\par{\hfill\break
20. ${\overset{\textnormal{けいさつ}}{\text{警察}}}$ は ${\overset{\textnormal{けんじゅう}}{\text{拳銃}}}$ ${\overset{\textnormal{ろく}}{\text{6}}}$ ${\overset{\textnormal{ちょう}}{\text{丁}}}$ と ${\overset{\textnormal{じつだん}}{\text{実弾}}}$ ${\overset{\textnormal{にひゃくごじゅっ}}{\text{250}}}$ ${\overset{\textnormal{ぱつ}}{\text{発}}}$ を ${\overset{\textnormal{おうしゅう}}{\text{押収}}}$ しました。 \hfill\break
 \emph{Keisatsu wa kenjū rokuchō to jitsudan nihyakugojuppatsu wo ōshū shimashita. \hfill\break
 }The police confiscated six handungs and 250 live bullets. }

\par{21. ${\overset{\textnormal{わたし}}{\text{私}}}$ は ${\overset{\textnormal{きょう}}{\text{今日}}}$ 、 ${\overset{\textnormal{とうふ}}{\text{豆腐}}}$ を3 ${\overset{\textnormal{ちょう}}{\text{丁}}}$ も ${\overset{\textnormal{た}}{\text{食}}}$ べました! \hfill\break
 \emph{Watashi wa kyō, tofu wo sanchō mo tabemashita! \hfill\break
 }I ate three things of tofu today! }

\par{22. こんにゃく ${\overset{\textnormal{いっ}}{\text{1}}}$ ${\overset{\textnormal{ちょう}}{\text{丁}}}$ をタオルにくるみました。 \hfill\break
 \emph{Kon\textquotesingle nyaku itchō wo taoru ni kurumimashita. \hfill\break
 }I tucked one block of konnyaku into a towel. }

\par{\textbf{Spelling Note }: \emph{Kon\textquotesingle nyaku }is only seldom spelled as 蒟蒻. }

\par{23. ハサミは ${\overset{\textnormal{いっ}}{\text{1}}}$ ${\overset{\textnormal{ちょう}}{\text{丁}}}$ も ${\overset{\textnormal{も}}{\text{持}}}$ っていません。 \hfill\break
 \emph{Hasami wa itchō mo motte imasen. \hfill\break
 }I don\textquotesingle t own a single pair of scissors. }

\par{24. ノコギリを ${\overset{\textnormal{に}}{\text{2}}}$ ${\overset{\textnormal{ちょう}}{\text{丁}}}$ ${\overset{\textnormal{か}}{\text{買}}}$ ってきました。 \hfill\break
 \emph{Nokogiri wo nichō katte kimashita. \hfill\break
 }I bought two saws. }

\par{\textbf{Spelling Note }: \emph{Nokogiri }is occasionally spelled as 鋸. }

\begin{center}
\textbf{- \emph{hon }本 } 
\end{center}

\par{ The counter - \emph{hon }本 counts long, thin items: pencils, string, laces, cans, roads, movies, train lines, phone calls, home runs, rivers, moves in martial arts, etc. }

\begin{ltabulary}{|P|P|P|P|P|P|P|P|}
\hline 

1 & いっぽん & 2 & にほん & 3 & さんぼん & 4 & よんほん \\ \cline{1-8}

5 & ごほん & 6 & ろっぽん & 7 & ななほん & 8 & はっぽん \\ \cline{1-8}

9 & きゅうほん & 10 &  \textbf{じゅっぽん \hfill\break
}\textbf{ }じっぽん & 100 & ひゃっぽん & ? & なんぼん \\ \cline{1-8}

\end{ltabulary}

\par{\hfill\break
25. ${\overset{\textnormal{えんぴつ}}{\text{鉛筆}}}$ を ${\overset{\textnormal{さん}}{\text{3}}}$ ${\overset{\textnormal{ぼんと}}{\text{本取}}}$ ってください。 \hfill\break
 \emph{Empitsu wo sambon totte kudasai. \hfill\break
 }Please three pencils. }

\par{26. ${\overset{\textnormal{みちいっぽんへだ}}{\text{道一本隔}}}$ てた ${\overset{\textnormal{てら}}{\text{寺}}}$ に ${\overset{\textnormal{さんぱい}}{\text{参拝}}}$ した。 \hfill\break
 \emph{Michi ippon hedateta tera ni sampai shita. \hfill\break
 }I visited a temple a road away. }

\par{27. アニメを ${\overset{\textnormal{じゅっ}}{\text{10}}}$ ${\overset{\textnormal{ほんみ}}{\text{本見}}}$ ました。 \hfill\break
 \emph{Anime wo juppon mimashita. \hfill\break
 }I watched ten anime. }

\par{28. ${\overset{\textnormal{かれ}}{\text{彼}}}$ は ${\overset{\textnormal{ゆび}}{\text{指}}}$ を ${\overset{\textnormal{ご}}{\text{5}}}$ ${\overset{\textnormal{ほんお}}{\text{本折}}}$ った。 \hfill\break
 \emph{Kare wa yubi wo gohon otta. \hfill\break
 }He broke five fingers. }

\begin{center}
\textbf{\emph{-ten }点 }\hfill\break

\end{center}

\par{ The counter - \emph{ten }点 counts points, commodities, or merchandise. }

\begin{ltabulary}{|P|P|P|P|P|P|P|P|}
\hline 

1 & いってん & 2 & にてん & 3 & さんてん & 4 & よんてん \\ \cline{1-8}

5 & ごてん & 6 & ろくてん & 7 & ななてん & 8 & はってん \hfill\break
はちてん \\ \cline{1-8}

9 & きゅうてん & 10 &  \textbf{じゅってん \hfill\break
}\textbf{ }じってん & 100 & ひゃくてん & ? & なんてん \\ \cline{1-8}

\end{ltabulary}

\par{\hfill\break
29. ${\overset{\textnormal{かれ}}{\text{彼}}}$ は ${\overset{\textnormal{じゅってんまんてん}}{\text{十点満点}}}$ (を) ${\overset{\textnormal{と}}{\text{取}}}$ りました。 \hfill\break
 \emph{Kare wa juttenmanten (wo) torimashita. \hfill\break
 }He got a perfect ten. }

\par{30. ${\overset{\textnormal{じゅう}}{\text{10}}}$ ${\overset{\textnormal{まんえん}}{\text{万円}}}$ の ${\overset{\textnormal{しょうひん}}{\text{商品}}}$ を ${\overset{\textnormal{つき}}{\text{月}}}$ (に) ${\overset{\textnormal{ひゃく}}{\text{100}}}$ ${\overset{\textnormal{てんう}}{\text{点売}}}$ っています。 \hfill\break
 \emph{J }\emph{ūman\textquotesingle en no sh }\emph{ōhin wo tsuki (ni) hyakuten utte imasu. \hfill\break
 }I sell 100 pieces of merchandise worth 100,000 yen a month. }

\par{31. ノート ${\overset{\textnormal{ご}}{\text{5}}}$ ${\overset{\textnormal{さつ}}{\text{冊}}}$ と ${\overset{\textnormal{まんねんひつ}}{\text{万年筆}}}$ ${\overset{\textnormal{に}}{\text{2}}}$ ${\overset{\textnormal{ほん}}{\text{本}}}$ と ${\overset{\textnormal{け}}{\text{消}}}$ しゴム ${\overset{\textnormal{さん}}{\text{3}}}$ ${\overset{\textnormal{こ}}{\text{個}}}$ 、 ${\overset{\textnormal{あ}}{\text{合}}}$ わせて ${\overset{\textnormal{じゅっ}}{\text{10}}}$ ${\overset{\textnormal{てん}}{\text{点}}}$ のお ${\overset{\textnormal{か}}{\text{買}}}$ い ${\overset{\textnormal{あ}}{\text{上}}}$ げですね。 \hfill\break
 \emph{Nōto gosatsu to man\textquotesingle nenhitsu nihon to keshigomu sanko, awasete jutten no o-kaiage desu ne? \hfill\break
 }You're buying five notebooks, two fountain pens, and three erasers for a total of ten items, yes? }

\begin{center}
\textbf{- \emph{haku }泊 }
\end{center}

\par{ The counter - \emph{haku }拍 counts nights of a stay. }

\begin{ltabulary}{|P|P|P|P|P|P|P|P|}
\hline 

1 & いっぱく & 2 & にはく & 3 & さんぱく & 4 & よんぱく \\ \cline{1-8}

5 & ごはく & 6 & ろっぱく & 7 & ななはく & 8 & はっぱく \\ \cline{1-8}

9 & きゅうはく & 10 &  \textbf{じゅっぱく \hfill\break
}\textbf{ }じっぱく & 100 & ひゃっぱく & ? & なんぱく \\ \cline{1-8}

\end{ltabulary}

\par{\hfill\break
32. ${\overset{\textnormal{ぜいべつ}}{\text{税別}}}$ で ${\overset{\textnormal{いっ}}{\text{1}}}$ ${\overset{\textnormal{はく}}{\text{泊}}}$ ${\overset{\textnormal{じゅうご}}{\text{15}}}$ ${\overset{\textnormal{まん}}{\text{万}}}$ ウォンでした。 \hfill\break
 \emph{Zeibetsu de ippaku jūgomanwon deshita. \hfill\break
 }It was 150,000 won for one stay excluding tax. }

\par{33. ${\overset{\textnormal{りょうきん}}{\text{料金}}}$ は ${\overset{\textnormal{かんりひ}}{\text{管理費}}}$ など ${\overset{\textnormal{すべ}}{\text{全}}}$ て ${\overset{\textnormal{こ}}{\text{込}}}$ みで ${\overset{\textnormal{いっ}}{\text{1}}}$ ${\overset{\textnormal{はく}}{\text{泊}}}$ ${\overset{\textnormal{いち}}{\text{1}}}$ ${\overset{\textnormal{めいさま}}{\text{名様}}}$ (につき)1 ${\overset{\textnormal{まんえん}}{\text{万円}}}$ です。 \hfill\break
 \emph{Ryōkin wa kanrihi nado subete komi de ippaku ichimei-sama ni tsuki ichiman\textquotesingle en desu. \hfill\break
 }The charge is 10,000 yen per person for one with with all-fees-in-price including management costs. }

\par{34. ツインルーム ${\overset{\textnormal{ひと}}{\text{1}}}$ ${\overset{\textnormal{へや}}{\text{部屋}}}$ で ${\overset{\textnormal{さん}}{\text{3}}}$ ${\overset{\textnormal{ぱく}}{\text{泊}}}$ です。 \hfill\break
 \emph{Tsuin rūmu hitoheya de sampaku desu. \hfill\break
 }Three nights in 1 twin-sized room. }

\par{35. フランスのマルセイユに ${\overset{\textnormal{ご}}{\text{5}}}$ ${\overset{\textnormal{ぱく}}{\text{泊}}}$ しました。 \hfill\break
 \emph{Furansu no maruseiyu ni gohaku shimashita. \hfill\break
 }I stayed five nights in Merceille, France. }

\begin{center}
\textbf{- \emph{tsū }通 }
\end{center}

\par{ The counter - \emph{tsū }通 counts letters as in \emph{tegami }手紙, e-mails, notes, and written reports. }

\begin{ltabulary}{|P|P|P|P|P|P|P|P|}
\hline 

1 & いっつう & 2 & につう & 3 & さんつう & 4 & よんつう \\ \cline{1-8}

5 & ごつう & 6 & ろくつう & 7 & ななつう & 8 &  \textbf{はちつう \hfill\break
}\textbf{ }はっつう \\ \cline{1-8}

9 & きゅうつう & 10 &  \textbf{じゅっつう \hfill\break
}\textbf{ }じっつう & 100 & ひゃくつう & ? & なんつう \\ \cline{1-8}

\end{ltabulary}

\par{\hfill\break
36. ${\overset{\textnormal{おな}}{\text{同}}}$ じ ${\overset{\textnormal{ないよう}}{\text{内容}}}$ の ${\overset{\textnormal{ぶんしょ}}{\text{文書}}}$ を ${\overset{\textnormal{さん}}{\text{3}}}$ ${\overset{\textnormal{つうおく}}{\text{通送}}}$ りました。 \hfill\break
 \emph{Onaji naiyō no bunsho wo santsū okurimashita. \hfill\break
 }I sent three documents with the same content. }

\par{37. ${\overset{\textnormal{ライン}}{\text{LINE}}}$ を ${\overset{\textnormal{いっ}}{\text{1}}}$ ${\overset{\textnormal{つうおく}}{\text{通送}}}$ りました。 \hfill\break
 \emph{Rain wo ittsū okurimashita. \hfill\break
 }I sent one LINE (message). }

\par{38. ${\overset{\textnormal{へんじ}}{\text{返事}}}$ の ${\overset{\textnormal{てがみ}}{\text{手紙}}}$ を ${\overset{\textnormal{に}}{\text{2}}}$ ${\overset{\textnormal{つうおく}}{\text{通送}}}$ りました。 \hfill\break
 \emph{Henji no tegami wo nitsū okurimashita. \hfill\break
 }I sent two letters in reply. }

\par{39. ${\overset{\textnormal{じゅしん}}{\text{受信}}}$ メールを ${\overset{\textnormal{なんつう}}{\text{何通}}}$ か ${\overset{\textnormal{け}}{\text{消}}}$ した。 \hfill\break
 \emph{Jushin mēru wo nantsū-ka keshita. \hfill\break
 }I deleted several received e-mails\slash texts. }

\begin{center}
\textbf{- \emph{bi }尾 }
\end{center}

\par{ The counter - \emph{bi }尾 counts small fish and shrimp. It is also used in industries to refer to high-quality fish or simply when fish go from being caught to now being treated as ‘food.\textquotesingle  }

\begin{ltabulary}{|P|P|P|P|P|P|P|P|}
\hline 

1 & いちび & 2 & にび & 3 & さんび & 4 & よんび \\ \cline{1-8}

5 & ごび & 6 & ろくび & 7 & ななび & 8 & はちび \\ \cline{1-8}

9 & きゅうび & 10 & じゅうび & 100 & ひゃくび & ? & なんび \\ \cline{1-8}

\end{ltabulary}

\par{\hfill\break
40. ${\overset{\textnormal{さかな}}{\text{魚}}}$ の ${\overset{\textnormal{なまえ}}{\text{名前}}}$ を ${\overset{\textnormal{ご}}{\text{5}}}$ ${\overset{\textnormal{びか}}{\text{尾書}}}$ いてください。 \hfill\break
 \emph{Sakana no namae wo gobi kaite kudasai. \hfill\break
 }Please write the names of five fish. }

\par{41. ${\overset{\textnormal{こんかい}}{\text{今回}}}$ は ${\overset{\textnormal{とくだい}}{\text{特大}}}$ (サイズ)の ${\overset{\textnormal{れいとう}}{\text{冷凍}}}$ むき ${\overset{\textnormal{えび}}{\text{海老}}}$ ${\overset{\textnormal{ろく}}{\text{6}}}$ ${\overset{\textnormal{び}}{\text{尾}}}$ ${\overset{\textnormal{つか}}{\text{使}}}$ いました。 \hfill\break
 \emph{Konkai wa tokudai (saizu) no reitō mukiebi rokubi tsukaimashita. \hfill\break
 }This time, I used six extra-large shelled shrimp. }

\par{42. ${\overset{\textnormal{がいこくさん}}{\text{外国産}}}$ のウナギも20尾売れました。 \hfill\break
 \emph{Gaikokusan no unagi mo nijūbi uremashita. \hfill\break
 }Twenty foreign-produced eels also sold. }

\par{\textbf{Spelling Note }: \emph{Unagi }is often spelled as 鰻. }

\par{43. メバルの ${\overset{\textnormal{ちぎょ}}{\text{稚魚}}}$ を( ${\overset{\textnormal{いっ}}{\text{1}}}$ ) ${\overset{\textnormal{せんび}}{\text{千尾}}}$ ほど ${\overset{\textnormal{かわ}}{\text{川}}}$ に ${\overset{\textnormal{ほうりゅう}}{\text{放流}}}$ した。 \hfill\break
 \emph{Mebaru no chigyo wo (is)sembi hodo kawa ni hōryū shita. \hfill\break
 }(They\slash we) stocked into the river around a thousand rockfish juveniles. }

\par{\textbf{Spelling Note: } \emph{Mebaru }is only seldom spelled as 眼張. }

\begin{center}
\textbf{Counting Fish }
\end{center}

\par{ Counting fish is no easy task. Dead or alive, most fish are counted with - \emph{hiki }匹. However, when fish are lined up from head to tail fin, then they\textquotesingle re counted with - \emph{hon }本. If referring to slices of fish, then you use - \emph{mai }枚. When presented as \emph{sashimi }刺し身, however, you then use - \emph{kire }切れ. When fish are dried out, they\textquotesingle re counted with - \emph{mai }枚. When fish are gutted with their heads and spines taken out, they\textquotesingle re counted with - \emph{chō }丁. \hfill\break
}

\par{ Cuts of fish cut into rectangular shapes are counted with - \emph{saku }冊 (which uses native number for 1 and 2). Simply from their appearance, certain fish are counted with - \emph{hon }本 by default for being long: amberjack  ( \emph{buri }ブリ・鰤), Pacific saury ( \emph{samma }サンマ・秋刀魚), bonito ( \emph{katsuo }カツオ・鰹), tuna ( \emph{maguro }マグロ・鮪), etc. Flat fish are counted with - \emph{mai }枚:  flounder ( \emph{hirame }ヒラメ・平目), right-eye flounder ( \emph{karei }カレイ・鰈), etc. Similarly, very thin fish are often counted with - \emph{jō }条: halfbeak ( \emph{sayori }サヨリ 細魚), icefish ( \emph{shirauo }シラウオ ・白魚), etc. \hfill\break
}

\par{ Eel ( \emph{unagi }ウナギ・鰻) are counted with either - \emph{hiki }匹 or - \emph{hon }本. However, when broiled into \emph{kabayaki }蒲焼, you use - \emph{mai }枚. When served on a skewer, you\textquotesingle d count them with - \emph{kushi }串 (which uses native numbers for 1-4, 7). This works for other fish as well. Although seldom done so today, very large fish may be counted with - \emph{kon }喉. This character means “throat” and in this context, it refers to large fish being carried whilst hanging from rope through their throats. Then, moreover you have - \emph{bi }尾, which is used for fish being used as bait, high-class fish, or fish when (being sold and or used) for cooking. }

\par{\textbf{Lesson Note }: For counters mentioned here that have not been formally introduced, it is okay not to memorize them until they are properly discussed. }

\begin{center}
\textbf{- \emph{eki }駅 }
\end{center}

\par{\emph{ Eki }駅 means “stations” and also happens to be the counter for stations. }

\begin{ltabulary}{|P|P|P|P|P|P|P|P|}
\hline 

1 & ひとえき & 2 & ふたえき & 3 & さんえき & 4 & よんえき \\ \cline{1-8}

5 & ごえき & 6 & ろくえき & 7 & ななえき & 8 & はちえき \\ \cline{1-8}

9 & きゅうえき & 10 & じゅうえき & 11 & じゅういちえき & ? & なんえき \\ \cline{1-8}

\end{ltabulary}

\par{\hfill\break
44. ${\overset{\textnormal{わたし}}{\text{私}}}$ は ${\overset{\textnormal{なごやえき}}{\text{名古屋駅}}}$ から ${\overset{\textnormal{ふた}}{\text{2}}}$ ${\overset{\textnormal{えきはな}}{\text{駅離}}}$ れたところに ${\overset{\textnormal{す}}{\text{住}}}$ んでいます。 \hfill\break
 \emph{Watashi wa Nagoya-eki kara futaeki hanareta tokoro ni sunde imasu. \hfill\break
 }I live somewhere two stations away from Nagoya Station. }

\par{45. ${\overset{\textnormal{ジェイアール}}{\text{JR}}}$ は ${\overset{\textnormal{ことし}}{\text{今年}}}$ 、 ${\overset{\textnormal{さん}}{\text{3}}}$ ${\overset{\textnormal{えき}}{\text{駅}}}$ を ${\overset{\textnormal{はいし}}{\text{廃止}}}$ した。 \hfill\break
 \emph{Jeiāru wa kotoshi, san\textquotesingle eki wo haishi shita. \hfill\break
 }JR went away with three stations this year. }

\par{46. ${\overset{\textnormal{さいたまけん}}{\text{埼玉県}}}$ の ${\overset{\textnormal{にんき}}{\text{人気}}}$ の ${\overset{\textnormal{ひゃく}}{\text{100}}}$ ${\overset{\textnormal{えき}}{\text{駅}}}$ に ${\overset{\textnormal{げんてい}}{\text{限定}}}$ しています。 \hfill\break
 \emph{Saitama-ken no ninki no hyakueki ni gentei shite imasu. \hfill\break
 }It's limited to the 100 stations popular in Saitama Prefecture. }

\begin{center}
\textbf{\emph{-oku }億 } 
\end{center}

\par{ The numbers 10, 100, 1,000, 10,000 are technically counters themselves. After 10,000, each number for ever fourth power is also treated as a counter. The next such unit after 10,000 is for 100 million, which is - \emph{oku }億. }

\begin{ltabulary}{|P|P|P|P|}
\hline 

100 million & いちおく & 1 billion & じゅうおく \\ \cline{1-4}

10 billion & ひゃくおく & 100 billion & (いっ)せんおく \\ \cline{1-4}

? & なんおく &  &  \\ \cline{1-4}

\end{ltabulary}

\par{\hfill\break
47. ${\overset{\textnormal{にほん}}{\text{日本}}}$ は ${\overset{\textnormal{じんこうきぼ}}{\text{人口規模}}}$ を ${\overset{\textnormal{いち}}{\text{1}}}$ ${\overset{\textnormal{おくにん}}{\text{億人}}}$ に ${\overset{\textnormal{いじ}}{\text{維持}}}$ することを ${\overset{\textnormal{めざ}}{\text{目指}}}$ している。 \hfill\break
 \emph{Nihon wa jinkō kibo wo ichiokunin ni iji suru koto wo mezashite iru. \hfill\break
 }Japan is aiming to maintain its population scale at 100 million people. }

\par{48. ${\overset{\textnormal{ちゅうごくご}}{\text{中国語}}}$ は、 ${\overset{\textnormal{じゅう}}{\text{10}}}$ ${\overset{\textnormal{おくにん}}{\text{億人}}}$ に ${\overset{\textnormal{つう}}{\text{通}}}$ じる ${\overset{\textnormal{げんご}}{\text{言語}}}$ のひとつです。 \hfill\break
 \emph{Chūgokugo wa, jūokunin ni tsūjiru gengo no hitotsu desu. \hfill\break
 }Chinese is one language that is understood by a billion people. }

\begin{center}
\textbf{\emph{-chō }兆 }
\end{center}

\par{ The next power to be treated as a counter is - \emph{chō }兆, which stands for trillion. }

\begin{ltabulary}{|P|P|P|P|P|P|P|P|}
\hline 

1 & いっちょう & 2 & にちょう & 3 & さんちょう & 4 & よんちょう \\ \cline{1-8}

5 & ごちょう & 6 &  \textbf{ろくちょう \hfill\break
}ろっちょう & 7 & ななちょう & 8 & はっちょう \\ \cline{1-8}

9 & きゅうちょう & 10 &  \textbf{じゅっちょう \hfill\break
}\textbf{ }じっちょう & 100 & ひゃくちょう & Quad. & (いっ)せんちょう \\ \cline{1-8}

? & なんちょう &  &  &  &  &  &  \\ \cline{1-8}

\end{ltabulary}

\par{\hfill\break
49. ${\overset{\textnormal{びようせいけい}}{\text{美容整形}}}$ の ${\overset{\textnormal{せかいしじょう}}{\text{世界市場}}}$ は ${\overset{\textnormal{いっ}}{\text{1}}}$ ${\overset{\textnormal{ちょうえん}}{\text{兆円}}}$ に ${\overset{\textnormal{たっ}}{\text{達}}}$ しています。 \hfill\break
 \emph{Biyōseikei no sekai shijō wa itchōen ni tasshite imasu. \hfill\break
 }The world cosmetic plastic surgery market has reached one trillion yen. }

\par{50. ${\overset{\textnormal{ひと}}{\text{人}}}$ の ${\overset{\textnormal{ちょうない}}{\text{腸内}}}$ には、 ${\overset{\textnormal{ひゃく}}{\text{100}}}$ ${\overset{\textnormal{ちょうひきいじょう}}{\text{兆匹以上}}}$ の ${\overset{\textnormal{きん}}{\text{菌}}}$ が ${\overset{\textnormal{す}}{\text{棲}}}$ んでいます。 \hfill\break
 \emph{Hito no chōnai ni wa, hyakuchōhiki ijō no kin ga sunde imasu. \hfill\break
 }There are over 100 trillion bacteria living inside the human intestines. }

\par{\textbf{Counter Note }: Incidentally, despite not being animals, bacteria are also frequently counted with - \emph{hiki }匹. }
    