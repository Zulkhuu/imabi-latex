    
\chapter{こそあど IV}

\begin{center}
\begin{Large}
第70課: こそあど IV: More こそあど 
\end{Large}
\end{center}
 
\par{ This lesson will go over many similar こそあど. Many are interrelated with each other, and so nuance differences will be a major take-out from this lesson. }
      
\section{Interrelated こそあど}
 
\par{ There are many こそあど phrases used to describe what something is like. Of course, the "what" is replaced with a こそあど. So, remember that you can't use こそあど if what you're referencing cannot be inferred from context. The words below are the basic phrases for "like\dothyp{}\dothyp{}\dothyp{}". The first four are the most important. }

\begin{ltabulary}{|P|P|P|P|P|P|}
\hline 

 & Like this & Like that & Like that* & What sort of\dothyp{}\dothyp{}\dothyp{} & Any \\ \cline{1-6}

Attribute & こんな & そんな & あんな & どんな & いかな** \\ \cline{1-6}

End of a Sentence & このようだ & そのようだ & あのようだ & どのようだ &  \\ \cline{1-6}

\end{ltabulary}

\par{*: "Like that" involves understand basic differences between それ and あれ. そんな sounds less distant than あんな in terms of feeling towards whatever style is being referred to. This is in regards to both the speaker and listener. そんな may even be used as an interjection similar to "no way!". }

\par{**: いかな is a contraction of いかなる. Both words are not common and are essentially 書き言葉. The latter is far more common in writing. You may never see いかな, but it finishes the series of demonstratives here. }

\par{ As you can see, these are all contractions involving the noun よう (様). This word literally means "appearance". These abbreviated forms are more casual. The attribute forms may actually be used at the end of dependent and independent clauses, but it is not as frequent for them to be in that position. These forms may even be followed by particles such as ので and のに. }

\par{\textbf{Variant Phrases } }

\par{ Various other methods can be used to make "like\dothyp{}\dothyp{}\dothyp{}" phrases. For instance, you can add the filler word 風 read as ふう, which also means "style", to the attribute forms mentioned above. Or, you can use a series of adverbial こそあど which simply start with the sounds but have long vowels instead. To use them as attribute phrases, you either use した, いう, or いった. }

\begin{ltabulary}{|P|P|P|P|P|P|}
\hline 

Attribute & 文尾* & Attribute & 文尾 & Attribute & 文尾 \\ \cline{1-6}

 こんな風\{の・な\} &  こんな風だ & こうした & こうしたものだ &  こういう(風\{の・な\}) & こう(いう風)だ \\ \cline{1-6}

 そんな風\{の・な\} &  そんな風だ & そうした & そうしたものだ &  そういう(風\{の・な\}) & そう(いう風)だ \\ \cline{1-6}

 あんな風\{の・な\} & あ んな風だ & ああした & ああしたものだ &  ああいう(風\{の・な\}) & ああ(いう風)だ \\ \cline{1-6}

 どんな風\{の・な\} &  どんな風だ & どうした & どうしたものか &  どういう(風\{の・な\}) & どう(いう風)か \\ \cline{1-6}

\end{ltabulary}

\par{*: 文尾 = End of the sentence. }

\par{1. The ~いう series is more objective than the first series with or without 風. }

\par{2. The addition of 風 is even more common in the spoken language. Traditionally, the first column is bad Japanese. Another issue is the use of な. の is much more common as it is traditionally correct. However, な is acceptable to most speakers. If in doubt, choose の. }

\par{3. When you use いった instead of いう, rather than grouping similar things together, you're describing them separately. }

\par{4. どういった sounds like you're interrogating someone on something whereas どのような is formal with no such nuance.  }

\begin{center}
\textbf{Examples }
\end{center}

\par{1. ああいう ${\overset{\textnormal{とけい}}{\text{時計}}}$ \hfill\break
A watch like that }

\par{2. どんな ${\overset{\textnormal{こうざ}}{\text{講座}}}$ を ${\overset{\textnormal{う}}{\text{受}}}$ けていますか。 \hfill\break
What kind of course are you taking? }

\par{3. そんな ${\overset{\textnormal{ほうほう}}{\text{方法}}}$ のことは ${\overset{\textnormal{いちど}}{\text{一度}}}$ も ${\overset{\textnormal{}}{\text{聞}}}$ かなかった。 \hfill\break
I never heard of a method like that. }

\par{4. こういう ${\overset{\textnormal{じじょう}}{\text{事情}}}$ なので \hfill\break
Such being the case }

\par{5. ${\overset{\textnormal{はいご}}{\text{背後}}}$ にどういう ${\overset{\textnormal{は}}{\text{葉}}}$ っぱ(の色)がくるかが ${\overset{\textnormal{だいじ}}{\text{大事}}}$ です。 \hfill\break
What kind of color the leaves will come in the background is important. }

\par{\textbf{Grammar Note }: Remember that the particle か in situations like this is used to make relative clauses. }

\par{6. これはどういう ${\overset{\textnormal{いみ}}{\text{意味}}}$ ですか。 \hfill\break
What does this mean? }

\par{7. ${\overset{\textnormal{こうえん}}{\text{公園}}}$ にはそのような ${\overset{\textnormal{ことり}}{\text{小鳥}}}$ がたくさんいます。 \hfill\break
There are many such birds in the park. }

\par{8. キャンディーやアイスクリーム、私はこういう甘いものが好きです。 \hfill\break
I like sweet things like candy and ice cream. (Groups things similar together) \hfill\break
\hfill\break
9. キャンディーやアイスクリーム、私はこういった甘いものが好きです。 \hfill\break
I like sweet things like candy and like ice cream. (Mentions things as separate examples) }

\par{10. \{こういう 〇・このような △・こうした\}色の服が欲しい。 \hfill\break
I want clothes of this color. }

\par{11. あの人はどんなマナー違反をしていますか。 \hfill\break
What sort of breaches of manner is that person committing? }

\par{\textbf{Culture Note }: Japan is a very 規則正しい society. Violating socially understood rules of behavior is bad, but in Japan, a lot of things that are not thought as being bad etiquette in America happen to be frowned upon in Japan. Some of the worst things people complain about in Japan include the following. }

\begin{ltabulary}{|P|P|}
\hline 

ポイ捨て (Littering) & 平気で遅刻 (Coming late all fine) \\ \cline{1-2}

電車で化粧をする (Doing make-up on the train) & あいさつしない (Not greeting) \\ \cline{1-2}

大声で話す (Speaking loudly) & ヘッドホンの音漏れ (Hearing sound from headphones) \\ \cline{1-2}

\end{ltabulary}

\begin{center}
\textbf{Adverbial Forms }
\end{center}

\par{ We have yet to address that these forms can be made into adverbs. The interrogative columns will not be mentioned. It's important to note that the final column is frequently used for showing method. }

\begin{ltabulary}{|P|P|P|P|}
\hline 

こんな \textrightarrow  こんなに & こう & こういう風に & こうやって \\ \cline{1-4}

そんな \textrightarrow  そんなに & そう & そういう風に & そうやって \\ \cline{1-4}

あんな \textrightarrow  あんなに & ああ* & ああいう風に & ああやって \\ \cline{1-4}

\end{ltabulary}

\par{*: ああ in this sense is limited to specific phrases such as ああでも and ああいえば. }

\begin{center}
\textbf{Examples }
\end{center}
 
\par{12. そんなに ${\overset{\textnormal{}}{\text{急}}}$ がないでね。(Familiar) \hfill\break
Don't be in such a hurry. }
 
\par{13. そうは ${\overset{\textnormal{}}{\text{思}}}$ いません。(Contrastive) \hfill\break
I don't think so. }
 
\par{14. こんなに ${\overset{\textnormal{おそ}}{\text{遅}}}$ いとは ${\overset{\textnormal{}}{\text{知}}}$ らなかった。 \hfill\break
I didn't realize how late it was. }

\par{15. そういう風に英語を勉強します。 \hfill\break
I study English that sort of way. }

\par{16. 映像をこうやって作ります。 \hfill\break
You create a clip like this. }

\begin{center}
\textbf{いかにも } 
\end{center}

\par{ いかにも shows us that the adverbial series may also be followed by the emphatic も. Early, it was said that forms of いか are not used in the spoken language, but this is the one exception. So, this is actually an important word. いかにも is similar in meaning to "indeed". }
 
\par{17. いかにも ${\overset{\textnormal{とくい}}{\text{得意}}}$ そうに \hfill\break
With evident pride }
 
\par{18. いかにもその ${\overset{\textnormal{とお}}{\text{通}}}$ りだよ。 \hfill\break
You can say that again! }

\begin{center}
 \textbf{+(い)ら }
\end{center}

\par{ ~いら can be added to the こそあど for place to create "\dothyp{}\dothyp{}\dothyp{}area\slash neighborhood" phrases. So, ここいら = This area. You may drop the い in these phrases. You may see ~辺 attached to ~ら. So, ここら辺 is "these surroundings". }

\par{ An important set phrase with this grammar is ~かそこいら, which can show approximated value\slash amount. }

\par{19.13歳かそこいらの ${\overset{\textnormal{わかぞう}}{\text{若僧}}}$ 。 \hfill\break
A young person around 13 years old. }

\par{20. ${\overset{\textnormal{ほすう}}{\text{歩数}}}$ は千歩かそこらだ。 \hfill\break
The number of steps is around 1000. }

\par{21. 一時間かそこら待った。 \hfill\break
I waited for about an hour. }

\par{ Another interesting phrase is そんじょそこらの, which is a very emphatic form of そこら, which is in turn a form of そこいら. This is often used with a tangible noun afterward.  It points out something as not being of the ordinary, but the word is still rather vague and ambiguous in regards to degree. }

\par{22. あの会社はそんじょそこらの人じゃ受からないよ。 \hfill\break
No ordinary person would get through at that company. }

\par{23. 彼女はそんじょそこらの女とは違う。 \hfill\break
She's quite different from just the ordinary woman. }

\par{24. そんじょそこらのチョコレートでは満足しない。 \hfill\break
I'm not satisfied with just ordinary chocolate. }
    