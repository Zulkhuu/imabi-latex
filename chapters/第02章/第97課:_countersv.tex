    
\chapter{Counters V}

\begin{center}
\begin{Large}
第97課: Counters V: 文字, クラス, 品, 桁, 切れ, 玉, 房, 株, 袋, 箱, 棟, 組み, 皿, 束, \& 通り 
\end{Large}
\end{center}
 
\par{ In previous lessons, we learned about the native numbers that are still used a lot in Modern Japanese. As a recap, here are these numbers once more. }

\begin{ltabulary}{|P|P|P|P|P|P|P|P|}
\hline 

1 &  \emph{Hito }ひと & 2 &  \emph{Futa }ふた & 3 &  \emph{Mi }み & 4 &  \emph{Yo(n) }よ(ん) \\ \cline{1-8}

5 &  \emph{Itsu }いつ & 6 &  \emph{Mu }む & 7 &  \emph{Nana }なな & 8 &  \emph{Ya }や \\ \cline{1-8}

9 &  \emph{Kokono }ここの & 10 &  \emph{Tō }とお & ? &  \emph{Iku }いく &  &  \\ \cline{1-8}

\end{ltabulary}
 
\par{\textbf{Usage Note }: The use of \emph{iku }- いく to create “how many…” phrases with native counters is limited to set phrases and\slash or neo-classical grammar. As such, you will not see it in the charts of this lesson. }
 
\par{ When you strip away - \emph{tsu }つ, you get the actual numbers. These numbers are used most with native vocabulary. However, they are not restricted to native vocabulary. Oddly enough, though, for the counters they\textquotesingle re used with, they\textquotesingle re hardly used to their full extent. Many counters only use the numbers for 1-3 or even just 1-2. This isn\textquotesingle t all that surprising considering how the native number system is practically limited to 1-10.  In this lesson, you will formally be introduced to counters that involve native numbers for the first time. The ones to be covered in this lesson are as follows. }
 
\begin{center}
\textbf{Counters Covered in this Lesson }
\end{center}
 
\par{1.       – \emph{(mo)ji }(文)字 \hfill\break
2.       - \emph{kurasu }クラス \hfill\break
3.       - \emph{shina }\slash - \emph{hin }品 \hfill\break
4.       - \emph{keta }桁 \hfill\break
5.       - \emph{kire }切れ \hfill\break
6.       - \emph{tama }玉 \hfill\break
7.       - \emph{fusa }房 \hfill\break
8.       - \emph{kabu }株 \hfill\break
9.       - \emph{fukuro }\slash - \emph{tai }袋 \hfill\break
10.   - \emph{hako }箱 \hfill\break
11.   - \emph{mune }\slash - \emph{t }\emph{ō }棟 \hfill\break
12.   - \emph{kumi }組み \hfill\break
13.   - \emph{sara }皿 \hfill\break
14.   - \emph{taba }\slash - \emph{soku }束 \hfill\break
15.   - \emph{t }\emph{ōri }通り }
 
\par{\textbf{Lesson Note }: Just as in prior lessons on counters, reading charts will have readings listed from most to least used. The most used variant will be in bold. }
      
\section{Native Numbers + Counters}
 
\begin{center}
\textbf{- \emph{moji }文字 \& - \emph{ji }字 }
\end{center}

\par{ If you are instructed to write a report within 300 characters, you will likely use a word processor and see how many characters—with or without spaces—are in what you\textquotesingle ve written. This would include punctuation marks and any other non-letter symbols. If you were instead instructed to write a report within 300 letters, you would interpret this to mean not including non-letter symbols. Therefore, that extra comma won\textquotesingle t make you go over like it would in a Twitter post. }

\par{ In Japanese, this same distinction is made by having two separate counters for “characters” and “letters.” They also conveniently correspond to the actual words for these things. To count characters, you use - \emph{ji }字. To count letters \emph{Kanji }漢字 and \emph{Kana }かな, you use - \emph{moji }文字. Of these two counters, only - \emph{moji }文字 uses more native numbers by using them for 1, 2, 4, and 7. }

\par{- \emph{ji }字 }

\begin{ltabulary}{|P|P|P|P|P|P|P|P|}
\hline 

1 & いちじ & 2 & にじ & 3 & さんじ & 4 & よんじ \\ \cline{1-8}

5 & ごじ & 6 & ろくじ & 7 & ななじ & 8 & はちじ \\ \cline{1-8}

9 & きゅうじ & 10 & じゅうじ & 100 & ひゃくじ & ? & なんじ \\ \cline{1-8}

\end{ltabulary}

\par{- \emph{moji }文字 }

\begin{ltabulary}{|P|P|P|P|P|P|P|P|}
\hline 

1 & ひともじ & 2 & ふたもじ & 3 & さんもじ & 4 & よんもじ \\ \cline{1-8}

5 & ごもじ & 6 & ろくもじ & 7 & ななもじ & 8 & はちもじ \\ \cline{1-8}

9 & きゅうもじ & 10 & じゅうもじ & 100 & ひゃくもじ & ? & なんもじ \\ \cline{1-8}

\end{ltabulary}

\par{\hfill\break
1. ${\overset{\textnormal{じすうせいげん}}{\text{字数制限}}}$ を ${\overset{\textnormal{いち}}{\text{1}}}$ ${\overset{\textnormal{まんじ}}{\text{万字}}}$ に ${\overset{\textnormal{かくだい}}{\text{拡大}}}$ しました。 \hfill\break
 \emph{Jisū seigen wo ichimanji ni kakudai shimashita. \hfill\break
 }I\slash we expanded the character limit to 10,000 characters. }

\par{2. その ${\overset{\textnormal{ないよう}}{\text{内容}}}$ を ${\overset{\textnormal{えいご}}{\text{英語}}}$ の ${\overset{\textnormal{ひゃく}}{\text{100}}}$ ${\overset{\textnormal{じ}}{\text{字}}}$ で ${\overset{\textnormal{まと}}{\text{纏}}}$ めてください。 \hfill\break
 \emph{Sono naiyō wo eigo no hyakuji de matomete kudasai. \hfill\break
 }Please summarize the content in 100 English characters. }

\par{3. ${\overset{\textnormal{えいご}}{\text{英語}}}$ では、 ${\overset{\textnormal{ようび}}{\text{曜日}}}$ や ${\overset{\textnormal{つき}}{\text{月}}}$ を ${\overset{\textnormal{さん}}{\text{3}}}$ ${\overset{\textnormal{もじ}}{\text{文字}}}$ で ${\overset{\textnormal{しょうりゃく}}{\text{省略}}}$ して ${\overset{\textnormal{か}}{\text{書}}}$ くことがあります。 \hfill\break
 \emph{Eigo de wa, yōbi ya tsuki wo sammoji wo shōryaku shite kaku koto ga arimasu. \hfill\break
 }In English, you often abbreviate the days of the week and the months to three letters when writing. }

\par{4. ${\overset{\textnormal{いっ}}{\text{1}}}$ ${\overset{\textnormal{ぷん}}{\text{分}}}$ で ${\overset{\textnormal{よ}}{\text{読}}}$ める ${\overset{\textnormal{さんびゃく}}{\text{300}}}$ ${\overset{\textnormal{もじ}}{\text{文字}}}$ \hfill\break
 \emph{Ippun de yomeru sambyakumoji \hfill\break
 }300 letters that can be read in a minute }
 \textbf{- \emph{kurasu }クラス }\hfill\break
 The counter - \emph{kurasu }クラス is used to count school classes. Oddly enough, it usually utilizes the native words for 1 and 2, which is unlike the other counters we\textquotesingle ve encountered thus far. However, as is the case with many counters with which these native numbers are possible, they\textquotesingle re not imperative with クラス.  
\begin{ltabulary}{|P|P|P|P|P|P|P|P|}
\hline 

1 &  \textbf{ひとくらす }\hfill\break
いちくらす & 2 &  \textbf{ふたくらす }\hfill\break
にくらす & 3 & さんくらす & 4 & よんくらす \\ \cline{1-8}

5 & ごくらす & 6 &  \textbf{ろっくらす }\hfill\break
ろくくらす & 7 & ななくらす & 8 & はちくらす \\ \cline{1-8}

9 & きゅうくらす & 10 &  \textbf{じゅっくらす }\hfill\break
じっくらす & ? & なんくらす &  &  \\ \cline{1-8}

\end{ltabulary}

\par{\hfill\break
5. ${\overset{\textnormal{みな}}{\text{皆}}}$ さんのお ${\overset{\textnormal{こ}}{\text{子}}}$ さんが ${\overset{\textnormal{かよ}}{\text{通}}}$ う ${\overset{\textnormal{しょうがっこう}}{\text{小学校}}}$ は ${\overset{\textnormal{なん}}{\text{何}}}$ クラスありますか。 \hfill\break
 \emph{Mina-san no o-ko-san ga kayou sh }\emph{ōgakk }\emph{ō wa nankurasu arimasu ka? \hfill\break
 }How many classes do the elementary schools that everyone\textquotesingle s children go to have? }

\par{6. ${\overset{\textnormal{わたし}}{\text{私}}}$ は ${\overset{\textnormal{しりつこうこう}}{\text{私立高校}}}$ ですが、 ${\overset{\textnormal{じゅうご}}{\text{15}}}$ クラスあります。 \hfill\break
 \emph{Watashi wa shiritsu k }\emph{ōk }\emph{ō desu ga, j }\emph{ūgokurasu arimasu. \hfill\break
 }I (go to) a public school, but there are\slash I have fifteen classes. }

\par{\textbf{Counter Note }: When referring to individual home rooms, the counter - \emph{gakkyū }学級 may be used instead. There are no sound changes with this counter. }

\begin{center}
\textbf{- \emph{hin }\slash - \emph{shina }品(ひん・しな) }
\end{center}

\par{ The counter - \emph{hin }\slash - \emph{shina }品counts “dishes of food.” For the most part, either set of readings are permissible. However, in the restaurant business, the - \emph{shina }しな readings are preferred. This is largely because \emph{ippin }一品 (1 dish\slash item) is homophonous with \emph{ippin }逸品 (rarity\slash excellent article). In set phrases, however, there is no flexibility as to which set of readings is used. \hfill\break
\hfill\break
・- \emph{hin }ひん }

\begin{ltabulary}{|P|P|P|P|P|P|P|P|}
\hline 

1 & いっぴん & 2 & にひん & 3 &  \textbf{さんぴん }\hfill\break
さんひん & 4 & よんひん \\ \cline{1-8}

5 & ごひん & 6 & ろっぴん & 7 & ななひん & 8 &  \textbf{はっぴん }\hfill\break
はちひん \\ \cline{1-8}

9 & きゅうひん & 10 &  \textbf{じゅっぴん }\hfill\break
じっぴん & 100 & ひゃっぴん & ? & なんぴん \\ \cline{1-8}

\end{ltabulary}

\par{・- \emph{shina }しな }

\begin{ltabulary}{|P|P|P|P|P|P|P|P|}
\hline 

1 & ひとしな & 2 & ふたしな & 3 & みしな & 4 &  \textbf{よしな }\hfill\break
よんしな \\ \cline{1-8}

5 &  \textbf{ごしな }\hfill\break
いつしな & 6 &  \textbf{ろくしな }\hfill\break
むしな & 7 & ななしな & 8 &  \textbf{はっしな }\hfill\break
はちしな \hfill\break
やしな \\ \cline{1-8}

9 &  \textbf{きゅうしな }\hfill\break
ここのしな & 10 &  \textbf{じゅっしな }\hfill\break
じっしな \hfill\break
としな & 100 & ひゃくしな & ? &  \textbf{なんしな }\\ \cline{1-8}

\end{ltabulary}

\par{\textbf{Reading Note }: - \emph{shina }品 is one of a handful of counters where many speakers remain acutely aware of the traditional readings for 1-10. These speakers tend to use native numbers for the entirety of the 1-10 series. However, they are in the minority. For most speakers, native numbers are not used after 4 with exception to 7. }

\par{7. これとこれを ${\overset{\textnormal{ふたしな}}{\text{二品}}}$ ${\overset{\textnormal{くだ}}{\text{下}}}$ さい。 \hfill\break
 \emph{Kore to kore wo futashina kudasai. \hfill\break
 }Please give me these two. }

\par{8. お ${\overset{\textnormal{とうふりょうり}}{\text{豆腐料理}}}$ を ${\overset{\textnormal{じゅっ}}{\text{10}}}$ 品 ${\overset{\textnormal{つく}}{\text{作}}}$ りました。 \hfill\break
 \emph{O-tōfu ryōri wo juppin\slash jusshina tsukurimashita. } \hfill\break
I made 10 tofu dishes. }

\par{9. ${\overset{\textnormal{いっぴんりょうり}}{\text{一品料理}}}$ を ${\overset{\textnormal{ちゅうもん}}{\text{注文}}}$ しました。 \hfill\break
 \emph{Ippin ryōri wo chūmon shimashita. \hfill\break
 }I ordered à la carte. }

\begin{center}
\textbf{- \emph{keta }桁 }
\end{center}

\par{ The counter - \emph{keta }桁 counts numerical digits. }

\begin{ltabulary}{|P|P|P|P|P|P|P|P|}
\hline 

1 & ひとけた & 2 & ふたけた & 3 &  \textbf{みけた }\hfill\break
さんけた & 4 &  \textbf{よけた }\hfill\break
よんけた \\ \cline{1-8}

5 & ごけた & 6 & ろっけた & 7 & ななけた & 8 &  \textbf{はっけた }\hfill\break
はちけた \\ \cline{1-8}

9 & きゅうけた & 10 &  \textbf{じゅっけた \hfill\break
}じっけた & 100 & ひゃっけた & ? & なんけた \\ \cline{1-8}

\end{ltabulary}

\par{\hfill\break
10. ${\overset{\textnormal{した}}{\text{下}}}$ ${\overset{\textnormal{み}}{\text{3}}}$ ${\overset{\textnormal{けた}}{\text{桁}}}$ は ${\overset{\textnormal{しーごーろく}}{\text{456}}}$ です。 \hfill\break
 \emph{Shita miketa wa shii gō roku desu. \hfill\break
 }The last three digits are 456. }

\par{\textbf{Pronunciation Note }: When reading out numbers, 4 and 5 are typically pronounced with long vowels as seen in Ex. 10. }

\par{11. ${\overset{\textnormal{ねだん}}{\text{値段}}}$ を ${\overset{\textnormal{ひとけた}}{\text{一桁}}}$ ( ${\overset{\textnormal{み}}{\text{見}}}$ ) ${\overset{\textnormal{まちが}}{\text{間違}}}$ えました。 \hfill\break
 \emph{Nedan wo hitoketa (mi)machigaemashita. \hfill\break
 }I mistook the price by one digit. }

\par{12. ${\overset{\textnormal{ふたけたまんえん}}{\text{二桁万円}}}$ の ${\overset{\textnormal{とけい}}{\text{時計}}}$ の ${\overset{\textnormal{よ}}{\text{良}}}$ さを ${\overset{\textnormal{じっかん}}{\text{実感}}}$ しました。 \hfill\break
 \emph{Futaketaman\textquotesingle en no tokei no yosa wo jikkan shimashita. \hfill\break
 }I really felt the merit of a watch in the six-figure range. }

\par{\textbf{Word Note }: Because \emph{man }万 means “ten thousand,” \emph{futaketaman }二桁万 refers to any value between 100,000 and 999,999. }

\begin{center}
\textbf{- \emph{kire }切れ }
\end{center}

\par{ The counter - \emph{kire }切れ counts slices. }

\begin{ltabulary}{|P|P|P|P|P|P|P|P|}
\hline 

1 & ひときれ & 2 & ふたきれ & 3 &  \textbf{みきれ }\hfill\break
さんきれ & 4 &  \textbf{よんきれ }\hfill\break
よきれ \\ \cline{1-8}

5 & ごきれ & 6 & ろっきれ & 7 & ななきれ & 8 &  \textbf{はちきれ \hfill\break
}はっきれ \\ \cline{1-8}

9 & きゅうきれ & 10 &  \textbf{じゅっきれ \hfill\break
}じっきれ & 100 & ひゃっきれ & ? & なんきれ \\ \cline{1-8}

\end{ltabulary}

\par{\hfill\break
13. サンドイッチにチーズを ${\overset{\textnormal{ふた}}{\text{2}}}$ ${\overset{\textnormal{き}}{\text{切}}}$ れ ${\overset{\textnormal{はさ}}{\text{挟}}}$ みます。 \hfill\break
 \emph{Sandoitchi ni chiizu wo futakire hasamimasu. \hfill\break
 }(I\textquotesingle ll) put two slices of cheese in the sandwich. }

\par{14. ${\overset{\textnormal{ちょうなん}}{\text{長男}}}$ にケーキを ${\overset{\textnormal{ひと}}{\text{1}}}$ ${\overset{\textnormal{き}}{\text{切}}}$ れ ${\overset{\textnormal{き}}{\text{切}}}$ り ${\overset{\textnormal{わ}}{\text{分}}}$ けてあげました。 \hfill\break
 \emph{Chōnan ni kēki wo hitokire kiriwakete agemashita. \hfill\break
 }I cut my oldest son a piece of cake. }

\par{15. ${\overset{\textnormal{さけ}}{\text{鮭}}}$ を ${\overset{\textnormal{じゅっ}}{\text{10}}}$ ${\overset{\textnormal{き}}{\text{切}}}$ れの ${\overset{\textnormal{き}}{\text{切}}}$ り ${\overset{\textnormal{み}}{\text{身}}}$ にして、パック ${\overset{\textnormal{い}}{\text{入}}}$ りにしました。 \hfill\break
 \emph{Sake wo jukkire no kirimi ni shite, pakku-iri ni shimashita. }\hfill\break
I cut the salmon into ten slices and packed them. }

\begin{center}
\textbf{- \emph{tama }玉 }
\end{center}

\par{ The counter - \emph{tama }玉 is used to count spherical food items such as tomatoes ( \emph{tomato }トマト), heads of lettuce ( \emph{retasu }レタス), melons (メロン), watermelons ( \emph{suika }スイカ), garlic ( \emph{nin\textquotesingle niku }ニンニク), cabbage ( \emph{kyabetsu }キャベツ), onions ( \emph{tamanegi }玉葱), portions of noodles such as udon ( \emph{udon }うどん), etc. }

\begin{ltabulary}{|P|P|P|P|P|P|P|P|}
\hline 
 
  1 
 &   ひとたま 
 &   2 
 &   ふたたま 
 &   3 
 &   さんたま・みたま 
 &   4 
 &   よんたま・よたま 
 \\ \cline{1-8} 
 
  5 
 &   ごたま 
 &   6 
 &   ろくたま 
 &   7 
 &   ななたま 
 &   8 
 &   はちたま 
 \\ \cline{1-8} 
 
  9 
 &   きゅうたま 
 &   10 
 &   \textbf{じゅったま }じったま 
 &   100 
 &   ひゃくたま 
 &   ? 
 &   なんたま 
\\ \cline{1-8}

\end{ltabulary}

\par{\hfill\break
16. ${\overset{\textnormal{おお}}{\text{大}}}$ きめのメロンを ${\overset{\textnormal{ひと}}{\text{1}}}$ ${\overset{\textnormal{たま}}{\text{玉}}}$ ${\overset{\textnormal{か}}{\text{買}}}$ いました。 \hfill\break
 \emph{Ōkime no meron wo hitotama kaimashita. \hfill\break
 }I bought one large-sized melon. }

\par{17. あの ${\overset{\textnormal{ぞう}}{\text{象}}}$ はスイカを ${\overset{\textnormal{ふた}}{\text{2}}}$ ${\overset{\textnormal{たま}}{\text{玉}}}$ ${\overset{\textnormal{まる}}{\text{丸}}}$ ごと ${\overset{\textnormal{た}}{\text{食}}}$ べましたよ。 \hfill\break
 \emph{Ano zō wa suika wo futatama marugoto tabemashita yo. \hfill\break
 }That elephant ate two watermelons whole. }

\par{\textbf{Spelling Note }: \emph{Suika }is frequently spelled as 西瓜. }

\par{18. ${\overset{\textnormal{なま}}{\text{生}}}$ うどんを ${\overset{\textnormal{ご}}{\text{5}}}$ ${\overset{\textnormal{たま}}{\text{玉}}}$ ${\overset{\textnormal{か}}{\text{買}}}$ いました。 \hfill\break
 \emph{Nama udon wo gotama kaimashita. \hfill\break
 }I bought five things of fresh udon. }

\par{\textbf{Spelling Note }: \emph{Udon }is only seldom spelled as 饂飩. }

\par{19. ハワイ ${\overset{\textnormal{さん}}{\text{産}}}$ のパパイヤを ${\overset{\textnormal{み}}{\text{3}}}$ ${\overset{\textnormal{たま}}{\text{玉}}}$ ${\overset{\textnormal{う}}{\text{売}}}$ りました。 \hfill\break
 \emph{Hawai-san no papaiya wo mitama urimashita. \hfill\break
 }I sold three papayas from Hawaii. }

\begin{center}
\textbf{- \emph{fusa }房 }
\end{center}

\par{ The counter - \emph{fusa }房 counts produce that are in bunches like grapes ( \emph{budō }ブドウ) or bananas ( \emph{banana }バナナ). }

\begin{ltabulary}{|P|P|P|P|P|P|P|P|}
\hline 

1 & ひとふさ & 2 & ふたふさ & 3 &  \textbf{みふさ }\hfill\break
さんふさ & 4 & よんふさ \\ \cline{1-8}

5 & ごふさ & 6 & ろくふさ & 7 & ななふさ & 8 & はちふさ \\ \cline{1-8}

9 & きゅうふさ & 10 &  \textbf{じゅうふさ }\hfill\break
とふさ & 100 & ひゃくふさ & ? & なんふさ \\ \cline{1-8}

\end{ltabulary}

\par{\hfill\break
20. ${\overset{\textnormal{みせ}}{\text{店}}}$ に、 ${\overset{\textnormal{ひと}}{\text{1}}}$ ${\overset{\textnormal{ふさ}}{\text{房}}}$ に ${\overset{\textnormal{はっ}}{\text{8}}}$ ${\overset{\textnormal{ぽん}}{\text{本}}}$ ${\overset{\textnormal{つ}}{\text{付}}}$ いたバナナを ${\overset{\textnormal{ひと}}{\text{1}}}$ ${\overset{\textnormal{ふさ}}{\text{房}}}$ ${\overset{\textnormal{ひゃく}}{\text{100}}}$ ${\overset{\textnormal{えん}}{\text{円}}}$ で ${\overset{\textnormal{にじゅう}}{\text{20}}}$ ${\overset{\textnormal{ふさ}}{\text{房}}}$ ${\overset{\textnormal{しい}}{\text{仕入}}}$ れました。 \hfill\break
 \emph{Mise ni, hitofusa ni happon tsuita banana wo hitofusa hyakuen de nijūfusa shiiremashita. \hfill\break
 }For the store, I procured 20 bunches of bananas for 100 yen per bunch with 8 bananas in each bunch. }

\par{21. ${\overset{\textnormal{うつわ}}{\text{器}}}$ にミカンを ${\overset{\textnormal{ふた}}{\text{2}}}$ ${\overset{\textnormal{ふさ}}{\text{房}}}$ ${\overset{\textnormal{い}}{\text{入}}}$ れました。 \hfill\break
 \emph{Utsuwa ni mikan wo futafusa iremashita. \hfill\break
 }I put two bunches of mandarin oranges into the bowl. \hfill\break
 \hfill\break
\textbf{Spelling Note }: \emph{Mikan }is occasionally spelled as 蜜柑. }

\par{22. ${\overset{\textnormal{あお}}{\text{青}}}$ いブドウを ${\overset{\textnormal{み}}{\text{3}}}$ ${\overset{\textnormal{ふさ}}{\text{房}}}$ ${\overset{\textnormal{れいぞうこ}}{\text{冷蔵庫}}}$ に ${\overset{\textnormal{い}}{\text{入}}}$ れました。 \hfill\break
 \emph{Aoi budō wo mifusa reizōko ni iremashita. }\textbf{\hfill\break
 }I put three bunches of green grapes into the refrigerator. }

\par{\textbf{Spelling Note }: \emph{Budō }is only seldom spelled as 葡萄. }

\begin{center}
\textbf{- \emph{kabu }株 }
\end{center}

\par{ The counter - \emph{kabu }株 can count heads of produce such as cabbage ( \emph{kyabetsu }キャベツ), Chinese cabbage ( \emph{hakusai }白菜), broccoli ( \emph{burokkorii }ブロッコリー), cauliflower ( \emph{karifurawā }カリフラワー), or bok choy ( \emph{chingensai }チンゲン菜), spinach ( \emph{hōrensō }ほうれん草), Japanese mustard spinach ( \emph{komatsuna }小松菜), etc. connected at the stem, or stocks as in the stock exchange ( \emph{kabushiki shijō }株式市場). }

\begin{ltabulary}{|P|P|P|P|P|P|P|P|}
\hline 
 
  1 
 &   ひとかぶ 
 &   2 
 &   ふたかぶ 
 &   3 
 &   \textbf{みかぶ }\hfill\break
さんかぶ 
 &   4 
 &   \textbf{よんかぶ }\hfill\break
よかぶ 
 \\ \cline{1-8} 
 
  5 
 &   ごかぶ 
 &   6 
 &   \textbf{ろっかぶ }\hfill\break
ろくかぶ 
 &   7 
 &   ななかぶ 
 &   8 
 &   \textbf{はちかぶ }\hfill\break
はっかぶ 
 \\ \cline{1-8} 
 
  9 
 &   きゅうかぶ 
 &   10 
 &   \textbf{じゅっかぶ }\hfill\break
じっかぶ 
 &   100 
 &   ひゃっかぶ 
 &   ? 
 &   なんかぶ 
 \\ \cline{1-8} 
 
\end{ltabulary}

\par{\hfill\break
23. ${\overset{\textnormal{ひと}}{\text{1}}}$ ${\overset{\textnormal{かぶ}}{\text{株}}}$ ${\overset{\textnormal{はち}}{\text{8}}}$ ${\overset{\textnormal{まんえん}}{\text{万円}}}$ の ${\overset{\textnormal{かぶ}}{\text{株}}}$ を ${\overset{\textnormal{ろっ}}{\text{6}}}$ ${\overset{\textnormal{かぶか}}{\text{株買}}}$ いました。 \hfill\break
 \emph{Hitokabu hachiman\textquotesingle en no kabu wo rokkabu kaimashita. \hfill\break
 }I bought six stocks for 80,000 yen a piece. }

\par{24. ${\overset{\textnormal{じかせい}}{\text{自家製}}}$ の ${\overset{\textnormal{たんたんめん}}{\text{坦々麺}}}$ にチンゲン ${\overset{\textnormal{さい}}{\text{菜}}}$ を ${\overset{\textnormal{ふた}}{\text{2}}}$ ${\overset{\textnormal{かぶ}}{\text{株}}}$ ${\overset{\textnormal{はい}}{\text{入}}}$ れて ${\overset{\textnormal{た}}{\text{食}}}$ べました。 \hfill\break
 \emph{Jikasei no tantanmen ni chingensai wo futakabu irete tabemashita. \hfill\break
 }I added two things of bok choy into the homemade dandan noodles I ate. }

\par{25. ほうれん ${\overset{\textnormal{そう}}{\text{草}}}$ を ${\overset{\textnormal{み}}{\text{3}}}$ ${\overset{\textnormal{かぶ}}{\text{株}}}$ ${\overset{\textnormal{しゅうかく}}{\text{収穫}}}$ しました。 \hfill\break
 \emph{Hōrensō wo mikabu shūkaku shimashita. \hfill\break
 }I harvested three bundles of spinach. }

\begin{center}
\textbf{- \emph{fukuro }\slash - \emph{tai }袋 }
\end{center}

\par{ The counter - \emph{fukuro }\slash - \emph{tai }袋 counts bagfuls. The former reading is more colloquial, but the latter reading is used as the proper unit in industrial\slash business settings. This counter is frequently used with words like tea ( \emph{ocha }お茶), cement ( \emph{semento }セメント), wheat flower ( \emph{komugiko }小麦粉), etc. }

\par{・ - \emph{fukuro }ふくろ }

\begin{ltabulary}{|P|P|P|P|P|P|P|P|}
\hline 

1 & ひとふくろ & 2 & ふたふくろ & 3 & みふくろ & 4 & よふくろ \\ \cline{1-8}

5 &  \textbf{ごふくろ }\hfill\break
いつふくろ & 6 &  \textbf{ろくふくろ }\hfill\break
ろっぷくろ & 7 & ななふくろ & 8 &  \textbf{はちふくろ }\hfill\break
はっぷくろ \\ \cline{1-8}

9 & きゅうふくろ & 10 &  \textbf{じゅっぷくろ }\hfill\break
じっぷくろ \hfill\break
とふくろ & 100 &  \textbf{ひゃっぷくろ }\hfill\break
ひゃくふくろ & ? & なんふくろ \\ \cline{1-8}

\end{ltabulary}

\par{・- \emph{tai }たい }

\begin{ltabulary}{|P|P|P|P|P|P|P|P|}
\hline 

1 & いったい & 2 & にたい & 3 & さんたい & 4 & よんたい \\ \cline{1-8}

5 & ごたい & 6 &  \textbf{ろくたい }\hfill\break
ろったい & 7 & ななたい & 8 &  \textbf{はちたい }\hfill\break
はったい \\ \cline{1-8}

9 & きゅうたい & 10 &  \textbf{じゅったい \hfill\break
}じったい & 100 &  \textbf{ひゃくたい }\hfill\break
ひゃったい & ? & なんたい \\ \cline{1-8}

\end{ltabulary}

\par{\hfill\break
26. ${\overset{\textnormal{いた}}{\text{炒}}}$ め ${\overset{\textnormal{たまねぎ}}{\text{玉葱}}}$ を ${\overset{\textnormal{ひとふくろい}}{\text{一袋入}}}$ れました。 \hfill\break
 \emph{Itame-tamanegi wo hitofukuro iremashita. \hfill\break
 }I put in a bagful of sautéed onions. }

\par{27. ポテチを ${\overset{\textnormal{ひとふくろた}}{\text{一袋食}}}$ べました。 \hfill\break
 \emph{Potechi wo hitofukuro tabemashita. \hfill\break
 }I ate a bag of potato chips. }

\par{28. ${\overset{\textnormal{ど}}{\text{土}}}$ のうを ${\overset{\textnormal{ひゃく}}{\text{100}}}$ 袋くらい ${\overset{\textnormal{た}}{\text{貯}}}$ めています。 \hfill\break
 \emph{Don }\emph{ō wo hyakutai\slash hyakufukuro kurai tamete imasu. \hfill\break
 }We have about 100 sandbags stored. }

\par{\textbf{Spelling Note }: \emph{Donō }is seldom spelled as 土嚢. }

\begin{center}
\textbf{- \emph{hako }箱 }
\end{center}

\par{ The counter - \emph{hako }箱 counts boxes. It itself is the noun for box. Its readings are in flux, but focus on the ones that are most common. This counter is one of a handful of native counters with which native speakers are more likely to be acutely aware of the traditional readings for it. The native series is possible for 1-10, but it is rare for a speaker to follow the entire series. }

\begin{ltabulary}{|P|P|P|P|P|P|P|P|}
\hline 

1 & \textbf{ }ひとはこ \hfill\break
いっぱこ & 2 &  \textbf{ふたはこ }\hfill\break
にはこ & 3 &  \textbf{さんぱこ }\hfill\break
\textbf{さんはこ }\hfill\break
みはこ \hfill\break
さんばこ & 4 &  \textbf{よんはこ }\hfill\break
よんぱこ \hfill\break
よはこ \\ \cline{1-8}

5 &  \textbf{ごはこ }\hfill\break
いつはこ & 6 &  \textbf{ろっぱこ }\hfill\break
ろくはこ \hfill\break
むはこ & 7 & ななはこ & 8 & はっぱこ \hfill\break
はちはこ \hfill\break
やはこ \\ \cline{1-8}

9 &  \textbf{きゅうはこ }\hfill\break
ここのはこ & 10 &  \textbf{じゅっぱこ }じゅうはこ \hfill\break
じっぱこ \hfill\break
とはこ 
& 100 &  \textbf{ひゃっぱこ }\hfill\break
ひゃくはこ & ? &  \textbf{せんぱこ }\hfill\break
せんはこ \\ \cline{1-8}

1万 &  \textbf{いちまんはこ }\hfill\break
いちまんぱこ &  &  &  &  &  &  \\ \cline{1-8}

\end{ltabulary}

\par{\hfill\break
29. ${\overset{\textnormal{す}}{\text{好}}}$ きな ${\overset{\textnormal{ほん}}{\text{本}}}$ を一箱に ${\overset{\textnormal{つ}}{\text{詰}}}$ め ${\overset{\textnormal{こ}}{\text{込}}}$ みました。 \hfill\break
 \emph{Suki na hon wo ippako\slash hitohako ni tsumekomimashita. \hfill\break
 }I crammed the books I liked into one box. }

\par{30. ${\overset{\textnormal{かれ}}{\text{彼}}}$ は ${\overset{\textnormal{いちにち}}{\text{一日}}}$ にタバコを\{2箱・2パック\}吸っている。 \hfill\break
 \emph{Kare wa ichinichi ni tabako wo [futahako\slash nipakku }\emph{] sutte iru. \hfill\break
 }He smokes\slash two packs of cigarettes a day. }

\par{\textbf{Counter Notes }: The counter - \emph{pakku }パック counts packs. For the number 1, it can be used with the forms “ \emph{ichi }いち,” “ \emph{ip }いっ,” “ \emph{hito }ひと,” and “ \emph{wan }ワン.” For 2, it can be used with “ \emph{ni }に,” “ \emph{futa }ふた,” and “ \emph{tsū }ツ―.” For other numbers, it behaves just like the counter - \emph{pēji }ページ. Also, the counter phrase \emph{ichinichi }一日 uses the counter - \emph{nichi }日 which counts days. This counter will be looked at later in IMABI. }

\par{31. ${\overset{\textnormal{ひ}}{\text{引}}}$ っ ${\overset{\textnormal{こ}}{\text{越}}}$ しで ${\overset{\textnormal{だん}}{\text{段}}}$ ボール ${\overset{\textnormal{ばこ}}{\text{箱}}}$ を ${\overset{\textnormal{じゅっ}}{\text{10}}}$ ${\overset{\textnormal{ぱこ}}{\text{箱}}}$ くらい ${\overset{\textnormal{はっそう}}{\text{発送}}}$ しました。 \hfill\break
 \emph{Hikkoshi de damb }\emph{ōrubako wo juppako kurai hassō shimashita. \hfill\break
 }I shipped about ten cardboard boxes in moving. }

\begin{center}
\textbf{- \emph{mune }\slash - \emph{t }\emph{ō }棟 }
\end{center}

\par{ The counter - \emph{mune }\slash - \emph{t }\emph{ō }棟 counts buildings. Historically, only the former reading existed. Nowadays, many people use the latter reading when counting very large structures like high-rises and use the former for any smaller sized building. This, though, is not set in stone. In the news, only - \emph{mune }むね is used. This practice is commonly seen in all kinds of formal situations. In the spoken language, - \emph{mune }むね is still largely the reading of choice. However, a decent minority still use - \emph{t }\emph{ō }とう instead, especially when referring to large buildings. }

\par{・- \emph{mune }むね }

\begin{ltabulary}{|P|P|P|P|P|P|P|P|}
\hline 

1 & ひとむね & 2 & ふたむね & 3 & さんむね & 4 & よんむね \\ \cline{1-8}

5 & ごむね & 6 & ろくむね & 7 & ななむね & 8 & はちむね \\ \cline{1-8}

9 & きゅうむね & 10 & じゅうむね & 100 & ひゃくむね & ? & なんむね \\ \cline{1-8}

\end{ltabulary}

\par{・- \emph{t }\emph{ō }とう }

\begin{ltabulary}{|P|P|P|P|P|P|P|P|}
\hline 

1 & いっとう & 2 & にとう & 3 & さんとう & 4 & よんとう \\ \cline{1-8}

5 & ごとう & 6 & ろくとう & 7 & ななとう & 8 &  \textbf{はちとう }\hfill\break
はっとう \\ \cline{1-8}

9 & きゅうとう & 10 &  \textbf{じゅっとう \hfill\break
}じっとう & 100 &  \textbf{ひゃくとう \hfill\break
}ひゃっとう & ? & なんとう \\ \cline{1-8}

\end{ltabulary}

\par{\hfill\break
32. ${\overset{\textnormal{じゅうたく}}{\text{住宅}}}$ ${\overset{\textnormal{ひと}}{\text{1}}}$ ${\overset{\textnormal{むね}}{\text{棟}}}$ が ${\overset{\textnormal{ぜんしょう}}{\text{全焼}}}$ しました。 \hfill\break
 \emph{J }\emph{ūtaku hitomune ga zensh }\emph{ō shimashita. \hfill\break
 }One home burned down. }

\par{33. マンションが10棟 ${\overset{\textnormal{た}}{\text{立}}}$ ち ${\overset{\textnormal{なら}}{\text{並}}}$ んでいる。 \hfill\break
 \emph{Manshon ga j }\emph{ūmune\slash jutt }\emph{ō tachinarande iru. \hfill\break
 }Ten apartment complexes are lined in a row. }

\par{34. ${\overset{\textnormal{きくかわ}}{\text{菊川}}}$ はオフィスビル ${\overset{\textnormal{さん}}{\text{3}}}$ ${\overset{\textnormal{むね}}{\text{棟}}}$ を ${\overset{\textnormal{しょゆう}}{\text{所有}}}$ している。 \hfill\break
 \emph{Kikukawa wa ofisu biru sammune wo shoy }\emph{ū shite iru. \hfill\break
 }Kikukawa owns three office buildings. }

\begin{center}
\textbf{\emph{-kumi }組み }
\end{center}

\par{ The counter \emph{-kumi }組み counts sets of something as well as what “class” one is in in school. In Japan, homerooms will be referred to by \#組. For this latter meaning, it is spelled without み. This spelling is also used when it counts “groups of people.” }

\begin{ltabulary}{|P|P|P|P|P|P|P|P|}
\hline 

1 & ひとくみ & 2 & ふたくみ & 3 &  \textbf{さんくみ }\hfill\break
みくみ & 4 & よんくみ \\ \cline{1-8}

5 & ごくみ & 6 & ろっくみ & 7 & ななくみ & 8 &  \textbf{はっくみ }\hfill\break
はちくみ \\ \cline{1-8}

9 & きゅうくみ & 10 &  \textbf{じゅっくみ \hfill\break
}じっくみ & 100 & ひゃっくみ & ? & なんくみ \\ \cline{1-8}

\end{ltabulary}

\par{\hfill\break
35. コーヒー ${\overset{\textnormal{ぢゃわん}}{\text{茶碗}}}$ ${\overset{\textnormal{ひとく}}{\text{一組}}}$ みを ${\overset{\textnormal{こうにゅう}}{\text{購入}}}$ しました。 \hfill\break
 \emph{K }\emph{ōhiijawan hitokumi wo k }\emph{ōny }\emph{ū shimashita. \hfill\break
 }I purchased a coffee cup set. }

\par{\textbf{Reading Note }: 茶碗 can also be read as “ \emph{chawan }ちゃわん.” }

\par{\textbf{Spelling Note }: \emph{K }\emph{ōhii }is only seldom spelled as 珈琲. }

\par{36. ${\overset{\textnormal{じゅっ}}{\text{10}}}$ チームを ${\overset{\textnormal{さん}}{\text{3}}}$ ${\overset{\textnormal{くみ}}{\text{組}}}$ に ${\overset{\textnormal{わ}}{\text{分}}}$ けました。 \hfill\break
 \emph{Jutchiimu wo sankumi\slash mikumi ni wakemashita. }\hfill\break
I split ten teams into three groups. }

\par{\textbf{Counter Note }: The counter - \emph{chiimu }チーム counts “teams.” For 1, you can use “ \emph{it }いっ” or “ \emph{hito }ひと.” For 2, you can use “ \emph{ni }に” or “ \emph{futa }ふた.” For other numbers, it behaves like - \emph{t }\emph{ō }頭. }

\par{37. ${\overset{\textnormal{わたし}}{\text{私}}}$ は ${\overset{\textnormal{さん}}{\text{3}}}$ ${\overset{\textnormal{くみ}}{\text{組}}}$ です。 \hfill\break
 \emph{Watashi wa sankumi desu. \hfill\break
 }I\textquotesingle m in Class 3. }

\begin{center}
\textbf{- \emph{sara }皿 }
\end{center}

\par{ The counter - \emph{sara }皿 is used to count plates of food. When plates don\textquotesingle t have any food on them, however, the counter - \emph{mai }枚 should be used instead. Incidentally, the word for plate is \emph{sara }皿. }

\begin{ltabulary}{|P|P|P|P|P|P|P|P|}
\hline 

1 & ひとさら & 2 & ふたさら & 3 &  \textbf{みさら }\hfill\break
さんさら & 4 & よんさら \\ \cline{1-8}

5 & ごさら & 6 & ろくさら & 7 & ななさら & 8 &  \textbf{はっさら }\hfill\break
はちさら \\ \cline{1-8}

9 & きゅうさら & 10 &  \textbf{じゅっさら \hfill\break
}じっさら & 100 & ひゃくさら & ? & なんさら \\ \cline{1-8}

\end{ltabulary}

\par{\hfill\break
38. ${\overset{\textnormal{わたし}}{\text{私}}}$ は ${\overset{\textnormal{あぶ}}{\text{炙}}}$ りサーモンの ${\overset{\textnormal{にぎ}}{\text{握}}}$ り ${\overset{\textnormal{ずし}}{\text{寿司}}}$ を ${\overset{\textnormal{じゅうろく}}{\text{16}}}$ ${\overset{\textnormal{さらた}}{\text{皿食}}}$ べきりました。 \hfill\break
 \emph{Watashi wa aburisāmon no nigirizushi wo j }\emph{ūrokusara tabekirimashita. \hfill\break
 }I completely ate 16 plates of seared salmon nigirizushi. }

\par{39. きのう、 ${\overset{\textnormal{ひゃく}}{\text{100}}}$ ${\overset{\textnormal{えん}}{\text{円}}}$ の ${\overset{\textnormal{かいてんずし}}{\text{回転寿司}}}$ を ${\overset{\textnormal{はっ}}{\text{8}}}$ ${\overset{\textnormal{さらた}}{\text{皿食}}}$ べましたよ。 \hfill\break
 \emph{Kin }\emph{ō, hyakuen no kaitenzushi wo hassara tabemashita yo. \hfill\break
 }I ate eight plates of 100-yen conveyor belt sushi yesterday. }

\par{40. ${\overset{\textnormal{かいてんずしてん}}{\text{回転寿司店}}}$ で ${\overset{\textnormal{すし}}{\text{寿司}}}$ を ${\overset{\textnormal{なんさらた}}{\text{何皿食}}}$ べることが ${\overset{\textnormal{おお}}{\text{多}}}$ いですか。 \hfill\break
 \emph{Kaitenzushi-ten de sushi wo nansara taberu koto ga }\emph{ōi desu ka? \hfill\break
 }How many plates of sushi do you eat the most at conveyor belt sushi restaurants? }

\par{41. お ${\overset{\textnormal{も}}{\text{持}}}$ ち ${\overset{\textnormal{かえ}}{\text{帰}}}$ りの ${\overset{\textnormal{かた}}{\text{方}}}$ はエビ ${\overset{\textnormal{てんまき}}{\text{天巻}}}$ を ${\overset{\textnormal{じゅっ}}{\text{10}}}$ \{ ${\overset{\textnormal{さらぶん}}{\text{皿分}}}$ ・ ${\overset{\textnormal{かん}}{\text{貫}}}$ \} ${\overset{\textnormal{ちゅうもん}}{\text{注文}}}$ しました。 \hfill\break
 \emph{O-mochikaeri no kata wa ebiten-maki wo [jussara-bun\slash jukkan] chūmon shimashita. \hfill\break
 }The person ordering takeout ordered 10 plates worth\slash 10 pieces of shrimp tempura rolls. }

\par{\textbf{Counter Note }: The counter - \emph{kan }貫 counts pieces of sushi. Its sound changes are the same as - \emph{kabu }株. }

\begin{center}
\textbf{- \emph{taba }束 }
\end{center}

\par{ The counter - \emph{taba }束 counts leafy vegetables ( \emph{yōsairui }葉菜類\slash  \emph{ha(mono)yasai }葉(物)野菜\slash  \emph{nappa }菜っ葉) that are in bundles. It may also be read as - \emph{soku }そく, but this reading tends to be used to indicate quantities of a hundred for vegetables, bamboo ( \emph{take }竹), rice plants ( \emph{ine }稲), firewood ( \emph{maki }\slash  \emph{takigi }薪), etc. This comes from bundles being treated as units of ten and - \emph{soku }そく being ten of those bundles. Not all speakers know this, but this distinction is still used in several industries. Because different places can assign different numerical values to how much a 束 is, it is usually always stated how much one is. }

\par{・- \emph{taba }たば }

\begin{ltabulary}{|P|P|P|P|P|P|P|P|}
\hline 

1 & ひとたば & 2 & ふたたば & 3 &  \textbf{みたば }\hfill\break
さんたば & 4 & よんたば \\ \cline{1-8}

5 & ごたば & 6 & ろくたば & 7 & ななたば & 8 & はちたば \\ \cline{1-8}

9 & きゅうたば & 10 &  \textbf{じゅったば }\hfill\break
じったば & 100 & ひゃくたば & ? & なんたば \\ \cline{1-8}

\end{ltabulary}

\par{・- \emph{soku }そく }

\begin{ltabulary}{|P|P|P|P|P|P|P|P|}
\hline 

1 & いっそく & 2 & にそく & 3 &  \textbf{さんぞく }\hfill\break
さんそく & 4 & よんそく \\ \cline{1-8}

5 & ごそく & 6 & ろくそく & 7 & ななそく & 8 & はっそく \\ \cline{1-8}

9 & きゅうそく & 10 &  \textbf{じっそく }\hfill\break
じゅっそく & 100 & ひゃくそく & ? &  \textbf{なんぞく }\hfill\break
なんそく \\ \cline{1-8}

\end{ltabulary}

\par{\hfill\break
42. このお ${\overset{\textnormal{みそしる}}{\text{味噌汁}}}$ ${\overset{\textnormal{いっ}}{\text{1}}}$ ${\overset{\textnormal{ぱい}}{\text{杯}}}$ で、ほうれん ${\overset{\textnormal{そう}}{\text{草}}}$ ${\overset{\textnormal{ひと}}{\text{1}}}$ ${\overset{\textnormal{たばぶん}}{\text{束分}}}$ の ${\overset{\textnormal{てつぶん}}{\text{鉄分}}}$ が ${\overset{\textnormal{と}}{\text{摂}}}$ れます。 \hfill\break
 \emph{Kono o-misoshiru ippai de, hōrensō hitotaba-bun no tetsubun ga toremasu. \hfill\break
 }You can get the iron content of a bundle of spinach with one cup of this miso soup. }

\par{43. ${\overset{\textnormal{ぎょうむよう}}{\text{業務用}}}$ の ${\overset{\textnormal{あじつ}}{\text{味付}}}$ け ${\overset{\textnormal{のり}}{\text{海苔}}}$ を ${\overset{\textnormal{ひゃく}}{\text{100}}}$ ${\overset{\textnormal{たば}}{\text{束}}}$ ${\overset{\textnormal{はっちゅう}}{\text{発注}}}$ しました。 \hfill\break
 \emph{Gyōmu-yō ajitsuke nori wo hyakusoku hatchū shimashita. \hfill\break
 }I put in an order for 100 bundles of business use, seasoned nori. }

\par{44. ${\overset{\textnormal{なが}}{\text{長}}}$ ネギ ${\overset{\textnormal{に}}{\text{2}}}$ ${\overset{\textnormal{ほん}}{\text{本}}}$ と、 ${\overset{\textnormal{みずな}}{\text{水菜}}}$ ${\overset{\textnormal{ひと}}{\text{1}}}$ ${\overset{\textnormal{たば}}{\text{束}}}$ 、 ${\overset{\textnormal{なべ}}{\text{鍋}}}$ つゆ ${\overset{\textnormal{ひと}}{\text{1}}}$ ${\overset{\textnormal{ふくろ}}{\text{袋}}}$ 、ポン ${\overset{\textnormal{ず}}{\text{酢}}}$ ${\overset{\textnormal{すこ}}{\text{少}}}$ しで ${\overset{\textnormal{つく}}{\text{作}}}$ りました。 \hfill\break
 \emph{Naganegi nihon to, mizuna hitotaba, nabetsuyu hitofukuro, ponzu sukoshi de tsukurimashita. \hfill\break
 }I made it with two Japanese leeks, one bundle of potherb mustard, a bag of hot pot soup, and a little ponzu sauce. }

\par{\textbf{Counter Note }: Long items like leeks are counted with - \emph{hon }本, which will looked in greater detail later in IMABI. }

\par{45. ${\overset{\textnormal{した}}{\text{下}}}$ の ${\overset{\textnormal{しゃしん}}{\text{写真}}}$ のお ${\overset{\textnormal{ひた}}{\text{浸}}}$ しで ${\overset{\textnormal{しゅんぎく}}{\text{春菊}}}$ ${\overset{\textnormal{み}}{\text{3}}}$ ${\overset{\textnormal{たばぶん}}{\text{束分}}}$ です! \hfill\break
 \emph{Shita no shashin no ohitashi de shungiku mitaba-bun desu! }\hfill\break
In the ohitashi in the picture below, there is three bundles worth of edible chrysanthemum! }

\par{\textbf{Culture Note }: \emph{Ohitashi }is boiled greens like spinach, mustard spinach, edible chrysanthemum, etc. }

\begin{center}
\textbf{- \emph{tōri }通り }
\end{center}

\par{ The counter - \emph{tōri }通り counts ways. The phrase \emph{hitotōri }ひととおり is not only used to mean “one way\slash method,” but it also means “in general\slash generally\slash roughly.” }

\begin{ltabulary}{|P|P|P|P|P|P|P|P|}
\hline 

1 & ひととおり & 2 &  \textbf{ふたとおり \hfill\break
}\textbf{ }にとおり & 3 &  \textbf{さんとおり \hfill\break
}\textbf{ }みとおり & 4 & よんとおり \\ \cline{1-8}

5 & ごとおり & 6 & ろくとおり & 7 & ななとおり & 8 & はちとおり \\ \cline{1-8}

9 & きゅうとおり & 10 &  \textbf{じゅっとおり \hfill\break
}\textbf{ }じっとおり & 100 & ひゃくとおり & ? & なんとおり \\ \cline{1-8}

\end{ltabulary}

\par{\hfill\break
46. ${\overset{\textnormal{ひつよう}}{\text{必要}}}$ な ${\overset{\textnormal{どうぐ}}{\text{道具}}}$ は ${\overset{\textnormal{ひととお}}{\text{一通}}}$ り ${\overset{\textnormal{そろ}}{\text{揃}}}$ っています。 \hfill\break
 \emph{Hitsuyō na dōgu wa hitotōri sorotte imasu. \hfill\break
 }There\textquotesingle s a full lineup of needed tools\slash instruments. }

\par{47. ${\overset{\textnormal{だいがくじゅけん}}{\text{大学受験}}}$ が ${\overset{\textnormal{ひととお}}{\text{一通}}}$ り ${\overset{\textnormal{お}}{\text{終}}}$ わりました。 \hfill\break
 \emph{Daigaku juken ga hitotōri owarimashita. \hfill\break
 }I\textquotesingle m generally done with taking my college exams. }

\par{48. ${\overset{\textnormal{よ}}{\text{読}}}$ みが ${\overset{\textnormal{ふた}}{\text{2}}}$ ${\overset{\textnormal{とお}}{\text{通}}}$ りある ${\overset{\textnormal{じゅくご}}{\text{熟語}}}$ がたくさん ${\overset{\textnormal{そんざい}}{\text{存在}}}$ します。 \hfill\break
 \emph{Yomi ga futatōri aru jukugo ga takusan sonzai shimasu. \hfill\break
 }There are many compound words that have two readings. }

\par{49. ${\overset{\textnormal{さん}}{\text{3}}}$ ${\overset{\textnormal{とお}}{\text{通}}}$ りの ${\overset{\textnormal{だんせい}}{\text{男性}}}$ の ${\overset{\textnormal{さそ}}{\text{誘}}}$ い ${\overset{\textnormal{かた}}{\text{方}}}$ を ${\overset{\textnormal{しょうかい}}{\text{紹介}}}$ していきます。 \hfill\break
 \emph{Santōri no dansei no sasoikata wo shōkai shite ikimasu. \hfill\break
 }I am going to introduce to you three ways to lure men. }

\par{ 50. ${\overset{\textnormal{く}}{\text{組}}}$ み ${\overset{\textnormal{あ}}{\text{合}}}$ わせは ${\overset{\textnormal{なんとお}}{\text{何通}}}$ りあるか ${\overset{\textnormal{おし}}{\text{教}}}$ えてください。 \hfill\break
 \emph{Kumiawase wa nantōri aru ka oshiete kudasai. \hfill\break
 }Please tell me how many combinations there are. }
    