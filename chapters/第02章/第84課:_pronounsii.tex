    
\chapter{Pronouns II}

\begin{center}
\begin{Large}
第84課: Pronouns II 
\end{Large}
\end{center}
 
\par{ Pronouns in Japanese are not easy to use, and it is truly impossible to explain how each may be used. Given that there are endless facets of use to these words, pay attention to what is mentioned. If you take anything as an absolute, you are not giving the material fair justice in interpretation. That's not to say everything is up for grabs. Rather, the fluidity in pronoun usage must be viewed as fluid rather than being defined by strict perimeters. }

\par{ As a quick grammar note, Japanese pronouns (代名詞) act a lot like regular nouns, and one way we can prove that is simply from the fact that we're talking about so many of them. One thing that you cannot do with them, for instance, is repeat them over and over again in the same sentence. It's not just unnatural to do so, but it results in incorrect Japanese. }

\par{1. 私は私の部屋で私のパソコンで私の友達とチャットしていました。X \hfill\break
私は部屋でパソコンで友達とチャットしていました。〇 \hfill\break
I was talking with my friend with my computer in my room. }

\par{ The other instances of first person are assumed as defaults unless situation says otherwise, in which case, you should use 自分 instead or some other pronoun other than first pronoun for the other instances for there to be a reason for qualifying. You will learn that although there are especially many first and second person pronouns to choose from, you should try to avoid using them whenever possible. }

\par{ This is because Japanese is a pro-drop language: it drops such pronouns unless there is a need to emphasize the information these pronouns give. For instance, topicalization, contrast, pointing out things are legitimate reasons to use them. Using them so much in a single assignment because you're trying to make a certain word count and are ignoring proper Japanese grammar is not legitimate.  }
      
\section{1st Person Pronouns}
 
\par{\textbf{わたくし }: This is a respectful pronoun and has been numerously defined by government guidelines as the proper first person pronoun of choice for polite speech and honorifics. However, the less polite but more neutral form わたし has surpassed it in frequency. In 漢字 both are 私. わたくし can sound pompous outside of honorifics. This, though, is due to a flouting of speech styles, not intrinsic in the word itself. }

\par{\textbf{わたし }: Overuse of this may sound slightly feminine. This is because the trend for men using it is going down, and given the business world esteems proper use of honorifics, わたくし can potentially receive more currency. }

\par{\textbf{あっし }: In the working class dialect of Tokyo Bay, you also hear あっし, which ultimately comes from わたくし. }

\par{\textbf{わし }: This is casual and used by men over 50 and by old people in general in some regions. As you can see, it is a contraction of わたし. It is sometimes written in 漢字 as 儂. }

\par{\textbf{あた(く)し }: These are generally deemed to be feminine and causal. Considering the history of traditional feminine speech, this is somewhat ironic. It is certain, though, that the use of these pronouns in polite contexts makes a woman look less educated, despite how unfair it may seem from the use of a feminine word. They may also appear in 下町 speech of Tokyo by both men and women. }

\par{\textbf{あたい }: This is found in rougher casual speech of female speakers. }

\par{\textbf{うち }: This is often used by women in casual and polite settings; however, there are many regions in Japan where this is a commonplace pronoun used by both genders. }

\par{\textbf{Usage Note }: First person pronouns are often dropped and assumed in context. Don't overuse them because it can make you sound egotistic. Don't assume that the situations discussed are the only ones. Tone and one's identity can change their implications a lot. }

\par{\textbf{Practice (1) }: }

\par{Determine which first person pronoun the following people will most likely use. }

\par{1. A 50 year old man at home. \hfill\break
2. A foreigner visiting Japan. \hfill\break
3. A teacher speaking to her principal. \hfill\break
4. A female speaking to her boyfriend. \hfill\break
5. A woman you see in an alley. }

\par{ ${\overset{\textnormal{ぼく}}{\text{僕}}}$ is the general pronoun for guys of any age. It is actually used by both genders in song. It may also be used to call for a boy child. 僕, due to its sense of humility, is appropriate in polite speech. ${\overset{\textnormal{おれ}}{\text{俺}}}$ asserts masculinity and ${\overset{\textnormal{おれさま}}{\text{俺様}}}$ is a very pompous variant. おら and おいら are slang, non-vulgar, country-like variants. ${\overset{\textnormal{わがはい}}{\text{我輩}}}$ is old-fashioned and pompous. }

\par{Another old-fashioned pronoun is ${\overset{\textnormal{われ}}{\text{我}}}$ . It is normally used in set expressions. The form ${\overset{\textnormal{わ}}{\text{我}}}$ が is used in formal situations to mean "our". Ex. "our country" is 我が ${\overset{\textnormal{}}{\text{国}}}$ . Finally, ${\overset{\textnormal{ちん}}{\text{朕}}}$ is equivalent to the "royal we" used by the Emperor. Other archaic pronouns can be found in Lesson 216  . }

\par{\textbf{Practice (2) }: }

\par{Determine which first person pronoun the following people will most likely use. }

\par{1. A boy child in kindergarten. \hfill\break
2. A teenager showing off to his friends. \hfill\break
3. A 20 year old male being incredibly ostentatious. \hfill\break
4. Someone sounding archaic. \hfill\break
5. The Emperor and Empress. }

\par{\textbf{Curriculum Note }: We'll study dialectical and classical pronouns much later on. }

\par{2. あたしは ${\overset{\textnormal{かんごふ}}{\text{看護婦}}}$ です。 \hfill\break
I am a nurse. }

\par{\textbf{Word Note }: ${\overset{\textnormal{かんごし}}{\text{看護師}}}$ is the gender neutral word for nurse. }

\par{3. 俺は ${\overset{\textnormal{おとこ}}{\text{男}}}$ だ。 \hfill\break
I am a man. }

\par{4. 僕は ${\overset{\textnormal{うた}}{\text{歌}}}$ う。 \hfill\break
I sing. }

\par{5. 我輩は ${\overset{\textnormal{ねこ}}{\text{猫}}}$ である。 \hfill\break
I am a cat. }

\par{\textbf{Sentence Note }: The sentence above is the title of a famous novel. }

\par{6. わしは大阪の ${\overset{\textnormal{しゅっしん}}{\text{出身}}}$ じゃ。(Old person) \hfill\break
I was born in Osaka. }

\par{\textbf{Phrase Notes }: To indicate where you are born, add の出身 or ${\overset{\textnormal{う}}{\text{生}}}$ まれ after the name of your birthplace. "Is" is usually だ in plain speech, but じゃ is commonly used by old people. }
      
\section{2nd Person}
 
\par{ "You" is considered \textbf{rude }because Japanese people \emph{normally address each other by name or title. }So, instead of "Is this your pen?", you'll hear something like "Is this Kim's pen?". When being polite, attach -さん to the name. }

\par{There are a lot of second person pronouns to choose from. The most neutral is あなた. It is direct, so use a name if possible. It can also mean "dear" when used by women to their spouses. あんた is a rude variant in Standard Japanese when talking to someone, but it can be seen in casual situations. There are a lot of dialects in which it is the standard second person pronoun. So, know your listener. }

\par{ A teacher with 50 kids wouldn't refer to her pupils by name, though. Instead, she would use あなたたち, the plural form of you. However, when you should be polite to an audience, use みなさん (everyone). }

\par{ ${\overset{\textnormal{きみ}}{\text{君}}}$ is a common casual "you” and it can be used towards subordinates and peers. It is also used by men to their lovers. However, this last usage may not be acceptable by some women. It can be seen by some as belittling while by some as intimate. From guy to guy, it may be seen as rude. }

\par{お ${\overset{\textnormal{まえ}}{\text{前}}}$ and おめー are often used in vulgar situations by guys. However, they may still be seen used in just casual situations. They are actually extremely commonly used by men in close company as a general word to use with no such stigma. Things are really different when you know someone personally. }

\par{ Most other second person pronouns are \textbf{very rude or old-fashioned }but with various degrees of nuance: てめー (this is from てまえ but the non-contracted form is not used anymore in regular conversation), きさま (貴様), おのれ (originally a first person reflexive pronoun meaning oneself, which it still means in set phrases), (お)ぬし (not that common), われ(most people don't use it aside from actual delinquent speech, which was the basis for pop-culture usage such as in movies and manga), etc. You just don't use these rude pronouns to people. They're just like fighting words. }

\par{\textbf{History Note }: It's important to note that 貴様 and お前 used to be very honorific terms, and the former stayed slightly honorific up until the early Edo Period. おぬし (お主) also has honorific origin. }

\par{7. お前は ${\overset{\textnormal{ま}}{\text{負}}}$ ける。 \hfill\break
You will lose. }

\par{8. あなたに ${\overset{\textnormal{あ}}{\text{会}}}$ います。 \hfill\break
I'll meet with you. }

\par{${\overset{\textnormal{やまもと}}{\text{9. 山本}}}$ さんのペンですか。(P) \hfill\break
Is this your pen Mr. Yamamoto? }

\par{10. あんたはずるい。(Casual; rude) \hfill\break
You're sly. }

\par{\textbf{Vocabulary Note }: おたく is a second person pronoun that shows light respect towards someone of relatively equal status. It can also mean "nerd" and "someone's house (honorific)". }

\par{11. お宅(、)どちらさん? \hfill\break
Who are you? \hfill\break
\hfill\break
12. コンピューターおたく \hfill\break
Computer nerd }

\par{13. お名前は何ですか。 \hfill\break
What is your name? }

\par{\textbf{Phrasing Note }: あなたの名前は何ですか is grammatically possible, but the first is more natural. }

\par{\textbf{Archaism Note }: There is an archaic second person pronoun 爾汝 used in the phrase ${\overset{\textnormal{じじょ}}{\text{爾汝}}}$ の ${\overset{\textnormal{まじ}}{\text{交}}}$ わり, which refers to using words like お前 and きさま mutually with those you are close to. It is not the kind of phrase most people know about. It's just interesting to know. }
      
\section{3rd Person}
 
\par{ The third person pronouns in Japanese that you should know are the following. Japanese actually prefers to avoid the words for he and she as they are more frequently used to refer to boyfriend and girlfriend. That does not mean, though, that you won't seem them clearly used to mean "he" and "she" as this is extremely common in translated works, of which there are many, and you will still hear the pronoun meanings used in conversation. Circumstances should make the intended meaning obvious. }

\begin{ltabulary}{|P|P|P|P|P|P|}
\hline 

He & 彼 & She & 彼女 & That person & あの人 \\ \cline{1-6}

That man & あの男 & That woman & あの女 & That person (F) & あの方 \\ \cline{1-6}

\end{ltabulary}

\par{\textbf{Word Note }: For "that woman" and "that man", add の人 in formal contexts. Or, you may be inclined to say things like あの男性(の方)or  あの女性(の方) for man and woman respectively. Japanese also tends to use titles to address others. So, if "he" is the clerk at the bank and you are referring to him, you would say 銀行員さん instead of 彼. }

\par{14. 彼は ${\overset{\textnormal{へん}}{\text{変}}}$ な男だ。 \hfill\break
He is a weird man. }

\par{15. 彼女はばかものだ。(PL) \hfill\break
She is stupid. }

\par{16. あの ${\overset{\textnormal{かた}}{\text{方}}}$ は ${\overset{\textnormal{}}{\text{水川}}}$ さんのお ${\overset{\textnormal{かあ}}{\text{母}}}$ さんです。 \hfill\break
That person over there is Mrs. Mizukawa's mother. }

\par{17. あの人はだれですか?(Polite) \hfill\break
Who is that person over there? }
      
\section{Keys}
 
\par{Practice (1) }

\par{1. わし \hfill\break
2. わたし \hfill\break
3. わたくし \hfill\break
4. あたし \hfill\break
5. あたい }
Practice (2) 
\par{1. ぼく \hfill\break
2. おれ \hfill\break
3. おれさま \hfill\break
4. われ \hfill\break
5. ちん }
     