    
\chapter{The Particle し}

\begin{center}
\begin{Large}
第69課: The Particle し 
\end{Large}
\end{center}
 
\par{ This particle is especially important for daily conversation. So, make sure you understand it before leaving this lesson! }

\par{\textbf{Curriculum Note }: The し seen in phrases such as 過ぎ去りし日 is not the particle し. Rather, it is the attribute form of a classical ending for past event. Thus, it will not be discussed in this lesson.  }
\textbf{Curriculum Note }: The し seen in phrases like 過ぎ去りし日 is not the particle し. It is a form of a past tense auxiliary in Classical Japanese  . So, it will not be discussed in this lesson. \hfill\break
\textbf{Curriculum Note }: The し seen in phrases like 過ぎ去りし日 is not the particle し. It is a form of a past tense auxiliary in Classical Japanese  . So, it will not be discussed in this lesson. \hfill\break
\textbf{Curriculum Note }: The し seen in phrases like 過ぎ去りし日 is not the particle し. It is a form of a past tense auxiliary in Classical Japanese  . So, it will not be discussed in this lesson. \hfill\break
\textbf{Curriculum Note }: The し seen in phrases like 過ぎ去りし日 is not the particle し. It is a form of a past tense auxiliary in Classical Japanese  . So, it will not be discussed in this lesson. \hfill\break
\textbf{Curriculum Note }: The し seen in phrases like 過ぎ去りし日 is not the particle し. It is a form of a past tense auxiliary in Classical Japanese  . So, it will not be discussed in this lesson. \hfill\break
\textbf{Curriculum Note }: The し seen in phrases like 過ぎ去りし日 is not the particle し. It is a form of a past tense auxiliary in Classical Japanese  . So, it will not be discussed in this lesson. \hfill\break
      
\section{使い方}
 
\par{ し is used to emphatically list similar situations and is after the ${\overset{\textnormal{しゅうしけい}}{\text{終止形}}}$ of a verb or adjective. }

\begin{ltabulary}{|P|P|P|P|P|P|P|P|}
\hline 

動詞 & 吸 う し & 形容詞 & 高 い し & 形容動詞 & 簡単 だ し & だ & バカ だ し \\ \cline{1-8}

\end{ltabulary}

\par{ It's often used in descriptions from a certain viewpoint that links certain aspects of a particular subject. It can list good things and bad things and also show contrast. }

\par{1. お金もないし、車もない(し)。 \hfill\break
I don't have money or a car. }

\par{2. 彼女は顔も美しいし、背も高いし、性格もいい(し)。 \hfill\break
She has a beautiful face, she's tall, and she has a good personality\dothyp{}\dothyp{}\dothyp{} }

\par{3. 彼氏は、タバコも吸\{う・います\}し、お酒も飲みます。 \hfill\break
My boyfriend smokes and drinks. }

\par{4. その映画は見たいし、お金はないし、どうしよう。 \hfill\break
I want to watch that movie, but I don't have the money\dothyp{}\dothyp{}\dothyp{}what am I going to do\dothyp{}\dothyp{}\dothyp{} }

\par{\textbf{Politeness Note }: し may follow ~ます, but it is required in the final verb of a polite statement. Other verbs in the sentence may optionally have ~ます as well. }

\par{ し is most frequently seen in sentences like Ex. 1 and 2 showing parallel situations (並列). Note that the particle も shows up in this situation instead of が  (Ex. 1 and 2) or  を (Ex. 3). }

\par{ Sentences of contrast (対比) is very similar to listing bad situations as they are both bad in the sense that they contradict with each other and that's causing the speaker grief. Here, we see that the particle after the nouns is は. After all, it has the function for showing contrast. Contrast does not require the phrases be in the negative. }

\par{5. ${\overset{\textnormal{は}}{\text{歯}}}$ は ${\overset{\textnormal{いた}}{\text{痛}}}$ いし、 ${\overset{\textnormal{はいしゃ}}{\text{歯医者}}}$ には行きたくないし、 ${\overset{\textnormal{こま}}{\text{困}}}$ ったよ。 \hfill\break
My teeth hurt, but I don't want to go to the doctor, I'm in a rut! }

\par{6. 台風は来るし、仕事はあるし、本当に困ったよ。  \hfill\break
The typhoon's coming, but I got my job. And so, I'm really at a loss.  }

\par{ Usually, the use of し regardless of how it is used leads to showing some sort of reason. So, you often see ~し~(し)で and ~し~から. A difference between this  し and から is that even when there is one thing stated, more parallel (or contrasting) things are implied with し. This is \textbf{not so }with から! Thus, they can occur with each other. }

\par{7. お金もあるし、プリウスでも買おうか。 \hfill\break
We have the money, so why don't we buy a Prius? }

\par{8. 今日は日曜(日)だし ${\overset{\textnormal{てんき}}{\text{天気}}}$ がいいから、 ${\overset{\textnormal{さんぽ}}{\text{散歩}}}$ に行きましょう。 \hfill\break
Today is Sunday, and the weather is fine. So, let's go out for a walk. }

\par{9. ${\overset{\textnormal{しょく}}{\text{職}}}$ は ${\overset{\textnormal{うしな}}{\text{失}}}$ うし、 ${\overset{\textnormal{つま}}{\text{妻}}}$ とは ${\overset{\textnormal{わか}}{\text{別}}}$ れるし(で)、彼はひどく ${\overset{\textnormal{げんき}}{\text{元気}}}$ がない。 \hfill\break
Losing his job, splitting up with his wife, he is really depressed. }

\par{10. どうせ暇ですし(本当にやりたい、等) ≈ どうせ暇ですから。 \hfill\break
At any rate, I'm free. So\dothyp{}\dothyp{}\dothyp{} }

\par{ If, however, you are showing any sort of comparison or contrast that comes about as a consequence of X, then you are stating things in terms of a ~ば conditional. The only difference between ~ば and ~し in this case, aside from obvious conjugation differences (~ば goes after the いぜん形), the former is a conditional. And so, "if X is such, then Y must also be such". The such can be a comparison or a contrast, and this relation always implies reason. }

\par{11. 金もなければ暇もない。 \hfill\break
If one has no money, then one has no free time. }

\par{12. 頭も良ければ、スポーツも出来る。 \hfill\break
If you are smart, you can also do sports. }

\par{13. 頭のいい人もいれば、(頭の)悪い人もいる。 \hfill\break
If there are smart people, there are also dumb people. }

\par{\textbf{Grammar Note }: The deletion of 頭の is more natural, but the addition of it would be the sentence more parallel. }

\par{14. 歩いても歩いても緑もなければ、獲物もいない。 \hfill\break
No matter if you walk and walk, if there are no greens, there is no prey. }

\par{15. 晴れの日もあれば、雨の日もあれば、曇りの日もありますよ。 \hfill\break
If there are clear days, and if there are rainy days, there are also cloudy days. }

\par{\textbf{Grammar Note }: Having two conditional phrases with the same Y final phrase is OK! }

\par{16. 風も吹けば雨も降る、寒い日もあれば、暑い日もある。 \hfill\break
If there are days in which the wind blows with the rainfall, then there are also days that are hot. }

\par{\textbf{Grammar Note }: Notice how this pattern is compounded. }

\par{17. 先日は、[風も吹けば雨も降る]で、すごく大変でした。 \hfill\break
The other day was one of those days when if the wind blows it rains, so it was really awful. }

\par{\textbf{Grammar Note }: Sometimes, you may see entire phrases treated as a set phrase like this. }

\par{ Another thing to note is how the particle し appears as a final particle. This is very colloquial, and although the particle overall is 話し言葉的, this particular usage would not be appropriate for 書き言葉. Also, in really くだけた speech, we even see ~しね. This is often used when there is but one thing mentioned and し is almost like a filler word to be less direct. }

\par{18. 今日は天気も悪いしね。 \hfill\break
And today, the weather's bad. }

\par{19. 寒いと、風邪も引きますし・・・。 \hfill\break
If it's cold, you'll get a cold. }

\par{20. 風邪も引くし、熱も出るしね。 \hfill\break
You'll get a cold and a fever, you know. }

\par{21. 「あなたはいつアメリカに帰りますか」「今のところ、帰るつもりはないんです。帰っても景気が悪い(です)しね」 \hfill\break
"When will you return home to America?" "At this point, I don't plan to return. Even if I went home, the economy is bad and such" }

\par{\textbf{More Examples }}

\par{22. 新しい ${\overset{\textnormal{ようふく}}{\text{洋服}}}$ はほしいし、お金はないし、学生の ${\overset{\textnormal{ふところ}}{\text{懐}}}$ は ${\overset{\textnormal{さび}}{\text{寂}}}$ しいなあ。 \hfill\break
I want to buy new clothes, but I have no money, a student's finances are depressing. }

\par{23. 彼の体は大きいし、力も強い。 \hfill\break
His body is large, and he is very strong. }

\par{24. 田中さんもそう言いましたし、鈴木さんもそう言いました。 \hfill\break
Mr. Tanaka said so, and Mr. Suzuki said so too. }

\par{25. 風は強いし、雪は ${\overset{\textnormal{ふ}}{\text{降}}}$ り出したし、今日は出かけるのはやめるわよ。(Feminine) \hfill\break
It's very windy, it's started to snow, so I'm going to quit on going out today. }

\par{26. ${\overset{\textnormal{きぶん}}{\text{気分}}}$ が悪いし忙しいから出かけられませんね。 \hfill\break
Since I'm sick and busy, I won't go out, OK? }

\par{27. 彼はまだ若いんだし、いくらでもやり ${\overset{\textnormal{なお}}{\text{直}}}$ せるね。 \hfill\break
He's still young, so he has as many times as he wants to redo things. }

\par{28. 駅からは遠いしバスもないしで ${\overset{\textnormal{たいへん}}{\text{大変}}}$ です。 \hfill\break
It's terrible that we're too far from the train station and that there aren't any buses. }

\par{29. 畑中先生は ${\overset{\textnormal{ねっしん}}{\text{熱心}}}$ だし、まじめだし、たくさんの ${\overset{\textnormal{けいけん}}{\text{経験}}}$ もあります。 \hfill\break
Hatanaka Sensei is earnest, diligent, and has a lot of experience. }

\par{30. 駅から遠いし、車でも来られないし、この ${\overset{\textnormal{みせ}}{\text{店}}}$ はとっても ${\overset{\textnormal{ふべん}}{\text{不便}}}$ だよ。 \hfill\break
This store is very inconvenient since it's far from the train station and you can't even get to it by car. }

\par{\textbf{Grammar\slash Curriculum Note }: ~られない makes the negative potential form for 一段 verbs and 来る. }

\par{2. ~\{では・でも\}あるまいし expresses a light sense of contempt meaning "it's not as if\dothyp{}\dothyp{}\dothyp{}". If the two parts are parallel, then this pattern stresses reason. If the two parts are contrasting, then the phrase means ~ないのに. }

\par{ This usage is one of the only usages in which the auxiliary ~まい is still even used. Its meaning is just like ~ないだろう. Even though it is usually old-fashioned, it is even found in colloquial conversations in this expression. However, if the first clause is lengthy, there is no particle deletion before ~まい. Though you may see it in conversation, its use is still following down even in this usage, and so あるまい is usually replaced by ない. }

\par{31. ${\overset{\textnormal{ふゆ}}{\text{冬}}}$ でも\{ない・ない\}し、 ${\overset{\textnormal{あつ}}{\text{厚}}}$ いシャツを着てるんだから、暑くないでしょう。(ちょっと ${\overset{\textnormal{くだ}}{\text{砕}}}$ けた言い方) \hfill\break
It's not as if it's even winter, and since you're wearing a thick shirt, aren't you hot? }

\par{32. 彼は、選手になれるわけじゃ\{ない・あるまい\}し、 悪くはない。 \hfill\break
Although it's not like he can become an athlete, he's not that bad. }

\par{33. ${\overset{\textnormal{たにん}}{\text{他人}}}$ じゃ\{ない・あるまい\}し、 ${\overset{\textnormal{みずくさ}}{\text{水臭}}}$ いじゃん?  (砕けた言い方) \hfill\break
It's not as if he's a stranger, so isn't he stand-offish? }

\par{\textbf{Contraction Note }: Remember that じゃ is the contraction of では. }

\par{\textbf{参照 }: http:\slash \slash web.ydu.edu.tw\slash ~uchiyama\slash conv\slash shishi.htm }
    