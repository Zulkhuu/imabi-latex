    
\chapter{Relative Time}

\begin{center}
\begin{Large}
第64課: Relative Time 
\end{Large}
\end{center}
 
\par{ Relative time phrases differ from absolute time phrases in that they do not refer to specific instances of time. Like their absolute counterparts, relative time phrases may function as either nouns or adverbs. Typically, their part of speech mirrors what they would be in English. However, this is not always the case. }
      
\section{Important Phrases}
 
\par{ Of the many relative time phrases in Japanese, the ones below are the most important to get down. Some of them involve grammar you've already encountered. For instance, よく comes from the adjective よい. Other phrases involve prefixes, suffixes, and particles that make them more or less set phrases. }

\begin{ltabulary}{|P|P|P|P|P|P|}
\hline 

Long ago &  ${\overset{\textnormal{むかし}}{\text{昔}}}$ & At one time & かつて & Before and after &  ${\overset{\textnormal{ぜんご}}{\text{前後}}}$ \\ \cline{1-6}

The other day & ${\overset{\textnormal{}}{\text{}}}$ 
& Previously & 先に & Already & もう \\ \cline{1-6}

Mostly & たいてい \hfill\break
& Later & 後で & No longer & もはや \\ \cline{1-6}

Always & いつも & At once & すぐに & Usually & 普通 \\ \cline{1-6}

Frequently & 度々 & Sometimes & 時々 \hfill\break
& Often \hfill\break
& よく \hfill\break
\\ \cline{1-6}

Again & また & Again; twice &  ${\overset{\textnormal{ふたた}}{\text{再}}}$ び & For a first & はじめて \\ \cline{1-6}

Seldom &  ${\overset{\textnormal{めった}}{\text{滅多}}}$ に & Never & ぜんぜん & Never & 決して \\ \cline{1-6}

A while ago & さっき & Every other & ~おきに & At about & ~ごろ \\ \cline{1-6}

\end{ltabulary}

\par{\textbf{Usage Notes }: }

\par{1. ~前後 may mean "back and forth" when not used with time. \hfill\break
2. (の) ${\overset{\textnormal{ころ}}{\text{頃}}}$ means "as; when" and is similar to 時. \hfill\break
3. 昔に is not frequently used, but it is seen in compound expressions such as はるか昔に (in the remote past). \hfill\break
4. ふたたびに is incorrect. }

\begin{center}
\textbf{Examples }
\end{center}

\par{1. 私たちは ${\overset{\textnormal{めった}}{\text{滅多}}}$ に\{ ${\overset{\textnormal{くちげんか}}{\text{口喧嘩}}}$ ${\overset{\textnormal{こうろん}}{\text{・口論}}}$ \}しません。 \hfill\break
We seldom quarrel. }

\par{2. 時々スプーンとフォークで食べます。 \hfill\break
I sometimes eat with a spoon and fork. }

\par{3. 彼女は ${\overset{\textnormal{いちにち}}{\text{一日}}}$ おきに ${\overset{\textnormal{しゅっきん}}{\text{出勤}}}$ する。 \hfill\break
She goes to work every other day. }

\par{4. ${\overset{\textnormal{じこ}}{\text{事故}}}$ は日曜日の ${\overset{\textnormal{さんじごろ}}{\text{三時頃}}}$ に ${\overset{\textnormal{お}}{\text{起}}}$ こった。 \hfill\break
The accident happened on Sunday at about 3:00. }

\par{5. お(お)よそ五時間の ${\overset{\textnormal{たび}}{\text{旅}}}$ となります。 \hfill\break
It'll turn into around a five hour trip. }

\par{6. 8月の ${\overset{\textnormal{なか}}{\text{半}}}$ ばに学校に ${\overset{\textnormal{もど}}{\text{戻}}}$ ります。 \hfill\break
I will go back to school in the middle of August. }

\par{7. 彼は9時前後にここに ${\overset{\textnormal{とうちゃく}}{\text{到着}}}$ するでしょう。 \hfill\break
He'll probably arrive here at around nine o'clock. }

\par{8. 僕は子供のころから英語に ${\overset{\textnormal{きょうみ}}{\text{興味}}}$ があった。 \hfill\break
I had an interest in England since I was a kid. }

\par{9. 彼を子供のころから知っている。 \hfill\break
I have known him since childhood. }

\par{10. このごろアメリカへ ${\overset{\textnormal{りょこう}}{\text{旅行}}}$ する人が多い。 \hfill\break
A lot of people are traveling to America nowadays. }
      
\section{Used To}
 
\par{ "Used to" is usually expressed with one of two patterns. The pattern ~たものだ is used to reminisce or confirm a past experience with a deep sense of emotion. かつて shows what something "used to be," \textbf{not }what one used "to do". かつて when used as noun means "former." }

\begin{ltabulary}{|P|P|P|}
\hline 

1. & Used to be\slash was once & 嘗(かつ)て+~た \\ \cline{1-3}

2. & Used to do & よく+~た+~ものだ \\ \cline{1-3}

\end{ltabulary}

\begin{center}
\textbf{Examples }
\end{center}

\par{11. 火曜日にはいつも友だちと ${\overset{\textnormal{つ}}{\text{釣}}}$ りに行ったものだ。 \hfill\break
I used to always like go fishing with my friends on Tuesdays. }

\par{12. かつてないほどの ${\overset{\textnormal{えんだか}}{\text{円高}}}$ だ。 \hfill\break
The strong yen is better than ever. }

\par{\textbf{Word Note }: The opposite of 円高 is 円安. The suffixes ~高 and ~安 may be used with any world currency such as the ドル (dollar), the ユーロ (euro), and the ウォン (won). }

\par{13. かつて、 ${\overset{\textnormal{だいにじせかいたいせん}}{\text{第二次世界大戦}}}$ にドイツはイタリアと ${\overset{\textnormal{どうめいこく}}{\text{同盟国}}}$ でありました。 (Formal 書き言葉) \hfill\break
Germany used to be an ally of Italy in World War II. }

\par{14a. かつて住んでいたところはこちらです。(Literary) \hfill\break
14b. 以前こちらに住んでいました。(Spoken) \hfill\break
The place I used to live at is here. }

\par{15. こんな ${\overset{\textnormal{じけん}}{\text{事件}}}$ はいまだかつてなかった。 \hfill\break
We have not yet had this kind of case ever before. }

\par{16. かつて彼らは ${\overset{\textnormal{どうりょう}}{\text{同僚}}}$ だった。 \hfill\break
They were once colleagues. }

\par{17. 彼はかつて ${\overset{\textnormal{そしき}}{\text{組織}}}$ の ${\overset{\textnormal{いちいん}}{\text{一員}}}$ であった。(書き言葉) \hfill\break
He used to be a member of the organization. }

\par{18. \{かつて・旧来\}の友  (Somewhat lyrical) \hfill\break
A former friend }

\par{\textbf{Orthography Note }: Although rare, かつて may be written in 漢字 as 嘗て. }

\begin{center}
\textbf{Customary Action:  ~ていた }
\end{center}

\par{ Customary action may also be expressed with ~ていた. This also equates to "used to." }

\par{19. 毎朝ジョギングをしていました。 \hfill\break
I used to jog every morning. }

\par{20. 中学校の時、毎晩9時に ${\overset{\textnormal{ね}}{\text{寝}}}$ ていた。 \hfill\break
I used to go to bed every night at nine when I was in junior high. }
    