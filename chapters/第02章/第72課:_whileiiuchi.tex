    
\chapter{Absolute Time V While II}

\begin{center}
\begin{Large}
第72課: Absolute Time V: While II: うち \& Others 
\end{Large}
\end{center}
 
\par{ In this lesson we will learn all about うち. うち is another word for "while," and it will be important for us to compare it with あいだ and other less common "while" phrases. }
      
\section{うち}
 
\par{ うち can attach to verbs, adjectives, and nouns of state\slash condition (夜・留守 るす ). However, it must \textbf{never }be used with the past tense. When used to mean "while", うち must be followed by a particle. So, "A + うち + Ø” is impossible despite that "A + 間 + Ø” is possible. }

\par{ There is some interchangeability between ~間に and ~うちに.  Due to the existence of ~ないうちに, there are three time variables in this lesson: A, B, and C. The basic interpretation of ~うちに is Time A ends after Time C takes place. ”Time", here, replaces the words "state" and or "action" as either kind of word is plausible. In the case of ~ないうちに, Time C occurs before Time B ever begins. }

\par{ To show what forms these patterns may take, look at the following table. Example words will be used to better show what these forms look like. Again, these two patterns can be used with verbs, adjectives, and nouns, but they all involve state of being or action. }

\begin{ltabulary}{|P|P|P|}
\hline 

 & うちに & 間に \\ \cline{1-3}

Verb & するうちに \hfill\break
しているうちに & する間に \hfill\break
している間に \hfill\break
してい \textbf{た }間に \\ \cline{1-3}

形容詞 & すずしいうちに & すずしい間に \\ \cline{1-3}

形容動詞 & 元気なうちに & 元気な間に \\ \cline{1-3}

Noun & 留守のうちに & 留守の間に \\ \cline{1-3}

\end{ltabulary}

\par{\textbf{漢字 Note }: うちに is usually spelled in ひらがな when used in a temporal sense, but it still may be spelled in 漢字 as 内に. }

\par{ In either case, both ~うちに and ~間に involve a Time C realizing within the frame of Time A. However, for ~間に, Time C starts and ends anytime during Time A. ~うちに means that by the time Time A ends, Time C is realized. If not, there would be bad consequences. In either case, Time C doesn't start until Time A starts. For ~ないうちに, then, Time C is realized before Time B ever happens because Time B is unfavorable. Take for example Ex. 1. }

\par{1. ${\overset{\textnormal{わか}}{\text{若}}}$ い(うちに 〇・間に X\}英語を勉強した ${\overset{\textnormal{ほう}}{\text{方}}}$ がいいよ。 \hfill\break
It's best that you study English while you're young. }

\par{\textbf{Grammar Note }: ~た方がいい = It's best that (you)\dothyp{}\dothyp{}\dothyp{} }

\par{\textbf{Clarification Note }: Because 内 literally comes from indoors\slash within, you are literally saying that you're doing C in the time frame of A. So, you can't do C before A because it's within and not outside A. }

\par{ 若い間に itself is a possible phrase. You can do many things while you're young, but learning languages is something that becomes incredibly difficult by age 12 for most people. So, there is a suggestion in Ex. 1 that if you want to learn English, you better do so by the time Time A runs out. If this favorable condition is so intrinsic to the statement, you cannot replace ~うちに with ~間に. }
2. スープがまだ熱い\{うちに 〇・間に X\}飲むのが好きです。 \hfill\break
I like eating my soup while it's still hot. 
\par{\textbf{Word Note }: The Japanese use the verb “to drink” instead of “to eat” for soup. }

\par{3. ${\overset{\textnormal{けいさつ}}{\text{警察}}}$ が来ないうちに、 ${\overset{\textnormal{に}}{\text{逃}}}$ げるから、 ${\overset{\textnormal{しんぱい}}{\text{心配}}}$ ないよ。(Casual) \hfill\break
I'll run away before the police come, so no worries. }

\par{ In Ex. 3, the main point is escaping before the cops make it there (before Time B realizes). Because this so important, you cannot use ~間に. In fact, ~ない間に is usually unnatural in general. However, when it is possible to imagine the end point of 〇〇ない, ~ない間に is possible. }

\par{4. ${\overset{\textnormal{つま}}{\text{妻}}}$ が帰ってこない\{うちに・間に\}、部屋を ${\overset{\textnormal{かた}}{\text{片}}}$ づけました。 \hfill\break
I cleaned up the room in the time my wife wasn't home yet. }

\par{\textbf{Nuance Note }: Using うちに would suggest not cleaning the room before your wife comes home would be bad. }

\par{5. 知らない\{間に・うちに\}、 ${\overset{\textnormal{ねむ}}{\text{眠}}}$ ってしまっていた。 \hfill\break
I was accidentally sleeping when I didn't know it. }

\par{ Again, it is possible for 間 to not be followed by a particle, but this is not possible for うち. Using ~うちに with まで would be illogical because Time C ends before A does. まで below has the events be simultaneous to the end, which is possible with ~間に. }

\par{6. 家に帰る \textbf{までの間 }、雨につかまってしまった。 \hfill\break
I was caught in the rain till I got home. }

\par{\textbf{Grammar Note }: Here, ~てしまった shows that you were \emph{unfortunately\slash regretfully }caught in the rain. }

\par{ Now it is time to delve more into how they are different. }

\begin{center}
\textbf{Using "While" with No Time Line }
\end{center}

\par{ Both ~間に and ~うちに may be used with verbs like 行く and 帰る despite the fact that they have no timeline as they're instantial verbs. The reason for this is there is one implied via the route needed to do the actions. After all, they are motion verbs. So, instead of 帰っているうちに, you say 帰るうちに. This does not mean, though, using ~うちに and ~間に here interchangeably result in the same meaning. }

\par{7. 家へ帰る間に、 ${\overset{\textnormal{かさ}}{\text{傘}}}$ を忘れてきたのに気づいた。 \hfill\break
While going home, I noticed that I had come without my umbrella. \hfill\break
\hfill\break
8. 家へ帰るうちに、傘を忘れてきたのに気づいた。 \hfill\break
Before getting home, I noticed that I had come without my umbrella. }

\par{\textbf{漢字 Note }: 気づく may be written in 漢字 as 気付く, but this is not that common anymore. }

\par{\textbf{Grammar Note }: ~の nominalizes the verb expressions preceding it. 気づく can also be 気がつく. }

\begin{center}
\textbf{Non-Past: ~間に 〇 but ~うちに X }
\end{center}

\par{ Before we talk about what happens when you use ~ている with these patterns, let's look into restrictions involving the non-past tense, which is a form you will less likely encounter. If even \emph{after the realization of Time C there is the potential that Time A continues for a long time or indefinitely }, you \textbf{can use ~間に }because its Time C just has to occur within Time A begins and ends. You \textbf{cannot use ~うちに }because it implies that Time A will soon stop or stop once Time C is realized. Otherwise, there would not be the added nuance of consequences if things don't go as planned. }

\par{9a. ${\overset{\textnormal{かれし}}{\text{彼氏}}}$ が晩ご ${\overset{\textnormal{はん}}{\text{飯}}}$ の ${\overset{\textnormal{したく}}{\text{支度}}}$ をする間に、宿題を ${\overset{\textnormal{ぜんぶす}}{\text{全部済}}}$ ませてしまった。 〇 \hfill\break
9b. 彼氏が晩ご飯の支度をするうちに、宿題を全部済ませてしまった。X \hfill\break
I finished all of my homework while my boyfriend prepared dinner.  }

\par{\textbf{Grammar Note }: Here, ~てしまった indicates the full completion of an action. }

\begin{center}
\textbf{Non-Past: ~間に X but ~うちに 〇 }
\end{center}

\par{ However, if there is the potential that Time A stops \emph{due to }Time C realizing, you can use  ~うちに because Time C realizes before Time A ends, but you cannot use ~間に. This is because although the time frame for ~間に is broad and wide so long Time A keeps on going, you cannot use it if Time A ends abruptly due to Time C as Time A doesn't come to a proper close. For instance, say you're drinking and intended for the event to start and end at some point. We'll call this Time A. During Time A, though, you get really red and then you redden once more later on, but this time, Time C, it's serious to the point you have to stop your drinking Time A. This is when you can't use ~間に and have to use ~うちに. }

\par{10. 酒をごくごくと飲む\{〇 うちに・X 間に\}、また顔が ${\overset{\textnormal{}}{\text{真っ赤}}}$ になってた。(Casual) \hfill\break
My face got completely red again while I was gulping down sake\slash liquor\slash alcohol. }

\begin{center}
\textbf{Non-Past: ~間に 〇 \& ~うちに 〇 }
\end{center}

\par{ Conversely and almost contrary to what has been said in the last two sections, if Time C realizes and there is both the potential that Time A continues on later or that Time A is suspended, you can use either or. When you use ~間に, though, you imply that Time A didn't just end there because of Time C realizing. Or, if you do stop Time A, you can still go back to it later. This note of interchangeability refers to simply replacing one for the other and getting a correct sentence because the context could go either way depending on what you use. So, if you used ~うちに, it would sound like Time A got suspended because of the realization of Time C. Thus, the nuances of the two apply whenever they show up. }

\par{11. 暖かい\{うちに・間に\}、ちょっと ${\overset{\textnormal{さんぽ}}{\text{散歩}}}$ に出かけます。 \hfill\break
I'll go out for a little walk while it's warm. }

\par{12. 駅へ行く\{うちに・間に\}、家に ${\overset{\textnormal{かさ}}{\text{傘}}}$ を忘れてきたのに気がついた。 \hfill\break
While I was going to the train station, I realized that I had left my umbrella at home. }

\par{13. その新聞は、夜の\{うちに・間に\}、 ${\overset{\textnormal{くば}}{\text{配}}}$ られてくるんですね。 \hfill\break
That's because the newspaper is delivered during the night. }

\par{\textbf{Grammar Note }: 配られる is the passive form of 配る (to deliver\slash distribute). }

\begin{center}
\textbf{~ている + ~間に・うちに } 
\end{center}

\par{ So, what's the difference when you use ~ている and just the non-past tense for both these patterns? In the case of ~ている, the time line involving Time A and Time C is quite long. So, there is ample time for Time C to end before Time A, allowing ~うちに to be grammatical in situations it wouldn't be with the non-past due to high probably that Time C wouldn't be able to end in time before the conclusion of Time A. }

\par{14. 赤ちゃんが寝てる\{うちに・間に\}、新聞読むよ。(Casual) \hfill\break
I'll read the newspaper while the baby is sleeping. }

\par{15. 彼氏が晩ご飯の支度をして\{いるうちに・いる間に\}、宿題を全部済ませてしまった。 \hfill\break
I finished all of my homework while my boyfriend was preparing dinner. }

\begin{center}
\textbf{More on ~ないうちに }
\end{center}

\par{ First, consider the following examples. }

\par{16. \{まだ覚えている・忘れない\}うちに宿題をします。 \hfill\break
I'll do my homework \{while I remember\slash before I forget\}. }

\par{17. \{ ${\overset{\textnormal{すず}}{\text{涼}}}$ しい・ ${\overset{\textnormal{あつ}}{\text{暑}}}$ くならない\}うちに、夏休みの宿題をしてしまいました。 \hfill\break
I completed all of my summer break homework \{while it was cool\slash before it got hot\}. }

\par{${\overset{\textnormal{}}{\text{18. \{明るい・暗}}}$ くならない\}うちに帰りなさい。(Stern) \hfill\break
Come home \{while it's light\slash before it gets dark\}. }

\par{ As said earlier, ~ないうちに shows that Time C is realized before Time B realizes. You can reword these sentences by replacing ~ないうちに with ~間に. ~ないうちに can be used to paraphrase ~間に phrases so long as there is a complete opposite of Time A or a possible Time B and that Time B, if it does occur, happens after Time C is done. So, if it gets dark, that's OK so long as you get home in time. Thus, we can also say that Time B may happen after Time A (when the sun goes away) as long Time C happens beforehand. It's important to note, though, that ~ないうちに inherently implies a misfortune if things don't go as planned, which is not the case at all with ~間に. }

\par{ ~ないうちに can be the only correct version of a sentence. For instance, in Exs. 19 and 20, see how the opposite isn't possible with either うち  or 間. Shots are preventative, so it's logical that ~ないうちに makes sense. }

\par{19. 病気にならないうちに、 ${\overset{\textnormal{よぼうちゅうしゃ}}{\text{予防注射}}}$ をしておいた方がいいですよ。〇 \hfill\break
It's best that you get immunizations in advance before you get ill. }

\par{20. ${\overset{\textnormal{じょうぶ}}{\text{丈夫}}}$ な\{うち・間\}に、予防注射をしておいた方がいいですよ。X \hfill\break
It's best that you get immunizations in advance while you're sturdy. }

\par{\textbf{Grammar Note }: ~ておく = To do in advance. }

\par{21. ${\overset{\textnormal{か}}{\text{蚊}}}$ が\{〇 入らない・X 出ている\}うちに、 ${\overset{\textnormal{かや}}{\text{蚊帳}}}$ を ${\overset{\textnormal{つ}}{\text{吊}}}$ った。 \hfill\break
I hanged up the mosquito net \{〇 before\slash X while\} the mosquitoes got in. }

\par{\textbf{漢字 Note }: 吊 is not a common character, so don't worry about remembering it now. }

\par{ Remember, when you can visualize a beginning and end point, ~ない間に can be used, but all of the facts leading up to now hold. }

\par{21. ${\overset{\textnormal{だれ}}{\text{誰}}}$ もいない\{うちに・間に\}、机を ${\overset{\textnormal{かたづ}}{\text{片付}}}$ けた。 \hfill\break
While there was no one, I fixed up the desks. }

\begin{center}
\textbf{~ないうちに VS ~まえに }
\end{center}

\par{ Although their translations may have you think they are the same, they're not. Remember that in "Bない+うちに+C", C is a positive situation, and the realization of B is feared. This is not implied in ~まえに. Rather, in this case, B hasn't realized yet, but its potential to is right in front of one's eyes. }

\par{22. 走ったが、家に\{着かないうちに・着くまえに\}、雨につかまった。 \hfill\break
I ran, but I got hit by the rain before I got home. }

\par{23. 妻が\{起きないうちに・起きるまえに\}、家を出た。 \hfill\break
I left the house before my wife woke up. }

\par{24. 雨が ${\overset{\textnormal{ふ}}{\text{降}}}$ \{る前に・らないうちに\}、帰りましょう。 \hfill\break
Let's go home before it rains. }

\par{\textbf{Grammar Note }: ~ましょう attaches to the ${\overset{\textnormal{れんようけい}}{\text{連用形}}}$ of a verb to make the polite volitional (let's) form. }

\begin{center}
\textbf{~すきに \& ~まに? }
\end{center}

\par{ If you were to come across these expressions, you would find them very similar to ~うちに. However, there are important differences. }

\par{  ~\{る・ない・た\}すきに shows that one does what one wishes to do during an opportunity that opens up (from the negligence of someone). }

\par{ ~まに is used to show that one does something when a chance arises and the time until that chance is lost. Thus, it has interchangeability with ~あいだに. It just has the added sense of taking advantage. }

\par{25. 先生のいない\{うち・あいだ・ま・すき\}に、すばやくホワイトボードを消してしまった。 \hfill\break
I erased the white board quickly while the teacher wasn't there. }

\par{26. 今だ!こっちを見ない\{〇 うち・〇 すき・X ま・ X 間\}に、逃げて! \hfill\break
Now! Run away while they don't look here! }

\par{27. 夜\{が・の\}涼しい\{〇 うち・〇 間・X ま・X すき\}に、部屋をきれいにした。 \hfill\break
I cleaned my room while the night was cool. }

\par{\textbf{漢字 Note }: This すき can be written as 隙. }

\begin{center}
\textbf{~間(は) \& ~うちは }
\end{center}

\begin{ltabulary}{|P|}
\hline 

X+あいだは+Z \\

X+うちは+Z \\

\end{ltabulary}

\par{ Z shows a line of time, and with the contrasting は, a \textbf{contrasting situation }to X is either explicitly stated or hinted at. Although you can use ~間 whether there is a contrasting situation or not, in order to use ~うちは, there has to be a contrasting situation. }

\par{28. 彼氏が晩ご飯の支度する\{あいだ(は)・うちは\}、宿題をしていた。 \hfill\break
I was doing my homework while my boyfriend was preparing homework (but not at other times). }

\par{29a. 授業のあいだ(は)、ずっと ${\overset{\textnormal{いねむ}}{\text{居眠}}}$ りしていた。 〇 \hfill\break
29b. 授業のうちは、ずっと居眠りしていた。X \hfill\break
I napped all through class. }

\par{30. コーヒーって若いうちはあまり飲まないほうがいいんですか。 \hfill\break
Is it best to not drink a lot of coffee while you're young? }

\begin{center}
\textbf{~うちが }
\end{center}

\par{ Although any instance of this can be rephrased to avoid it, it shouldn't surprise you that this is possible given that うち is a noun. In this case all this is showing is "within A". }

\par{31. 何といっても、若いうちが花だよ。 \hfill\break
No matter what they say, youth is the flower of life. }

\par{32. お酒は、ほろ ${\overset{\textnormal{よ}}{\text{酔}}}$ いのうちが ${\overset{\textnormal{さいこう}}{\text{最高}}}$ だと言われる。 \hfill\break
It's said that alcohol is the best while tipsy. }

\par{\textbf{Grammar Note }: 言われる is the passive form of 言う and is translated as "it's said". }

\begin{center}
\textbf{Other Meanings of うち }
\end{center}

\par{ Aside from meaning "while", うち has other usages. All these meanings can be written as 内, but this is only the case today when used in a spacial sense. }

\par{1. Indoors; interior; middle \hfill\break
2. One's home, family, spouse, in-group \hfill\break
3. In an abstractly established perimeter. \hfill\break
4. First person pronoun most often used by women. }

\begin{center}
\textbf{Examples }
\end{center}

\par{33. うちの車や。(関西弁; feminine) \hfill\break
It's my car. }

\par{34. ${\overset{\textnormal{ひゃくにんのこ}}{\text{百人残}}}$ ったうちの一人が ${\overset{\textnormal{はんにん}}{\text{犯人}}}$ だぞ。 (Masculine; vulgar) \hfill\break
One of these remaining 100 people is the criminal! }

\par{35. あのドアは内に ${\overset{\textnormal{む}}{\text{向}}}$ かって ${\overset{\textnormal{あ・ひら}}{\text{開}}}$ きます。 \hfill\break
That door over there opens inward. }

\par{36. また、近いうちに。 \hfill\break
Let's meet again soon. }

\par{37. ${\overset{\textnormal{うちがわ}}{\text{内側}}}$ から ${\overset{\textnormal{かぎ}}{\text{鍵}}}$ がかかってしまった。 \hfill\break
It was locked from the inside. }

\par{\textbf{Grammar Note }: ~てしまう = To accidentally do something. }

\par{38. ${\overset{\textnormal{うん}}{\text{運}}}$ も ${\overset{\textnormal{じつりょく}}{\text{実力}}}$ のうちである。 \hfill\break
Luck is also in one's true ability. }
    