    
\chapter{The Particle の II}

\begin{center}
\begin{Large}
第89課: The Particle の II 
\end{Large}
\end{center}
 
\par{ This lesson will be about two interesting usages of の. The first will be about its unique role in attribute clauses in which it is sometimes interchangeable with が. We will then learn how の can even be used like a noun! }
      
\section{In Attribute Clauses}
 
\par{ の may be used to mark the thing or person carrying out an action that is the attribute of something. In other words, it may replace が. This is \emph{only seen in attributive expressions }. }

\par{1. 雪の ${\overset{\textnormal{}}{\text{降}}}$ る夜 \hfill\break
A snowy night }

\par{2. 書類の入ったかばんは? \hfill\break
(What about) the bag with the documents? }

\par{3. 本当も本当 、 ${\overset{\textnormal{うそいつわ}}{\text{嘘偽}}}$ りのない ${\overset{\textnormal{}}{\text{話}}}$ だよ。 \hfill\break
It's really really true, it's a story without any lies. }

\par{\textbf{Particle Note }: Sometimes の is required when there is still a sense of possession. This is the case for Exs. 4 and 5. }

\par{4. 公園で子供の遊んでいる声が聞こえる。 \hfill\break
I can hear the voices of kids playing in the park. }

\par{5. 地面の緩い状態が続いています。 \hfill\break
The ground's instability continues. }

\par{\textbf{Grammar Note }: Japanese usually allows the use of either が or を in an attribute clause, and length doesn't have much to do with the decision unless the phrase is excessively long, in which case が is often obligatory. One case, though, in which が is clearly obligatory is when a dummy noun is used to attach to an otherwise independent clause. }

\par{ Think of patterns involving こと. You can't say ~XのYことができる. Below, [] brackets enclose attribute phrases. The nouns they modify will be right outside the right bracket.  }

\par{6. [[北部の人口 \textbf{の }少ない]自治体など作業 \textbf{が }完了した]ところから[順に開票結果 \textbf{が }発表される]見通しです。 \hfill\break
With the northern municipalities with little population completing operation, voting results are planned to be announced one by one. \hfill\break
From NHK on September 9, 2014. }

\par{7. 独立への賛成が反対を上回れば、投票率に関係なく、[スコットランドのイギリスからの独立が決まる]ことになり、 イギリスの公共放送BBCは終夜、開票速報番組を放送するなど、関心の高さを示しています。 \hfill\break
From NHK on September 9, 2014. \hfill\break
In disregard to voter turnout, if supporters for independence surpass the opposition, the decision for whether Scotland receives independence from England is made, and the British public broadcasting network BBC is showing its heightened interest in the matter by airing a voting result program all night. }

\par{\textbf{Practice (1) }: Translate the following. You may use a dictionary. }

\par{1. A mother that likes video games. 2. A rainy day. 3. Her pencil. 4. Apartment in New York. }

\par{ \textbf{The Nominalizer }\textbf{の }}

\par{ It may also be used as a nominalizer. 読むのが ${\overset{\textnormal{す}}{\text{好}}}$ きだ = I like reading; の makes 読む a noun. This usage of の is very interesting grammatically as it is treated like a dumby noun. So, it may be best to not think of it as a particle in this context. }

\par{8. 私は泳ぐのが好きです。 \hfill\break
I like to swim\slash I like swimming. }

\par{9. 私は歌うのが好きではありません。 \hfill\break
I don't like to sing. }

\par{10. それを買うのはだめだ。 \hfill\break
Buying that is useless. }

\par{\textbf{Practice (2) }: Translate the following. You may use a dictionary. }

\par{1. I don't like movies. \hfill\break
2. I don't like to go to the movie theater. \hfill\break
3. The blue one. }

\par{\textbf{One }}

\par{ の can be used to make indefinite pronouns by nominalizes things. With things other than the non-past form a verb, the result of の-nominalization is an indefinite pronoun. An indefinite pronoun has the same properties as a noun. So, although we're using a particle to make it, other particles may follow including の. Now, the doubling of particles is usually unnatural. However, it is sometimes inevitable. }

\par{11. 私が買ったの \hfill\break
The one I bought }

\par{12. あれは ${\overset{\textnormal{}}{\text{緑色}}}$ のです。 \hfill\break
That over there is a green one. }

\par{13. この ${\overset{\textnormal{かさ}}{\text{傘}}}$ はだれのですか。 \hfill\break
Whose umbrella is this? }

\par{14. どっちの本が ${\overset{\textnormal{ぼく}}{\text{僕}}}$ のですか。 \hfill\break
Which book is mine?  }

\par{15. 刺身は\{新しい・ ${\overset{\textnormal{しんせん}}{\text{新鮮}}}$ な\}のがおいしい。 \hfill\break
As for sashimi, fresh ones are delicious. }

\par{\textbf{Grammar Note }: If you were to have something nominalized with の and then made possessive, you get two の in a row. }

\par{16. 白っぽいのの二種類がある。 \hfill\break
They have two kinds of the whitish ones. }

\par{17. 足が臭いのを治す方法を教えてくれませんか。 \hfill\break
Could you tell me of ways to fix my feet from stinking? }

\par{18. 死んだのが馬鹿だったということは間違いない。 \hfill\break
There is no doubt that the one who died was an idiot.   }

\par{11. この ${\overset{\textnormal{かさ}}{\text{傘}}}$ はだれのですか。 \hfill\break
Whose umbrella is this? }

\par{12. どっちの本が ${\overset{\textnormal{ぼく}}{\text{僕}}}$ のですか。 \hfill\break
Which book is mine?  }

\par{11. この ${\overset{\textnormal{かさ}}{\text{傘}}}$ はだれのですか。 \hfill\break
Whose umbrella is this? }

\par{12. どっちの本が ${\overset{\textnormal{ぼく}}{\text{僕}}}$ のですか。 \hfill\break
Which book is mine?  }
      
\section{Keys}
 
\par{Practice (1) }

\par{1. ビデオゲームの好きな ${\overset{\textnormal{ははおや}}{\text{母親}}}$ 。 \hfill\break
2. 雨の降る日; 雨降りの日。 \hfill\break
3. 彼女の ${\overset{\textnormal{えんぴつ}}{\text{鉛筆}}}$ 。 \hfill\break
4. ニューヨークのアパート。 }

\par{Practice (2) }

\par{1. ${\overset{\textnormal{えいが}}{\text{映画}}}$ が好きじゃない。 \hfill\break
2. ${\overset{\textnormal{えいがかん}}{\text{映画館}}}$ に行くのが好きじゃない。 \hfill\break
3. 青 }
    