    
\chapter{Counters III (Time Part I)}

\begin{center}
\begin{Large}
第54課: Counters III (Time: Part I): 日, 週間, 月, 年, Etc. 
\end{Large}
\end{center}
 
\par{ In this lesson, we will learn about temporal counters. Before doing so, though, we will start off by learning the days of the week, which unfortunately don\textquotesingle t involve counters. }

\begin{center}
\textbf{Counters Covered in This Lesson }
\end{center}

\par{1.       - \emph{ka }日 \hfill\break
2.       - \emph{nichi(kan) }日(間) \hfill\break
3.       -s \emph{hūkan }週間 \hfill\break
4.       - \emph{gatsu }月 \hfill\break
5.       - \emph{kagetsu(kan) }ヶ月(間) \hfill\break
6.       - \emph{tsuki }月 \hfill\break
7.       - \emph{nen(kan) }年(間) \hfill\break
8.       - \emph{kanen }ヶ年 \hfill\break
9.       - \emph{nensei }年生 \hfill\break
10.   - \emph{gakunen }学年 }
      
\section{Yōbi 曜日: The Days of the Week}
 
\par{ The days of the week, \emph{yōbi } ${\overset{\textnormal{}}{\text{曜日}}}$ , in Japanese each start with the first character of seven luminous bodies in our solar system in the following order:  the Sun, moon, Mars, Mercury, Jupiter, Venus and Saturn. }

\begin{ltabulary}{|P|P|P|P|}
\hline 
 
  Sunday 
 &    \emph{Nichiyōbi }日曜日 
 &   Monday 
 &    \emph{Getsuyōbi }月曜日 
 \\ \cline{1-4} 
 
  Tuesday 
 &    \emph{Kayōbi }火曜日 
 &   Wednesday 
 &    \emph{Suiyōbi }水曜日 
 \\ \cline{1-4} 
 
  Thursday 
 &    \emph{Mokuyōbi }木曜日 
 &   Friday 
 &    \emph{Kin\textquotesingle yōbi }金曜日 
 \\ \cline{1-4} 
 
  Saturday 
 &    \emph{Doyōbi }土曜日 
 &   What day (of the week)? 
 &   \emph{Nan\textquotesingle yōbi }何曜日 
 \\ \cline{1-4} 
 
\end{ltabulary}

\par{\textbf{Reading Note }: Theoretically, if you were to want to talk about hypothetical days of the week, you would read 何曜日as “ \emph{naniyōbi },” but aside from this, it\textquotesingle s always read as above. }

\par{ All days of the week may be abbreviated by dropping the final - \emph{bi } ${\overset{\textnormal{}}{\text{日}}}$ . In abbreviated writing, you will very frequently see days of the week written with only the initial character. They can even be combined to make saying things like “Thursday and Friday” easier. Meaning, instead of writing \emph{mokuyōbi to kin\textquotesingle yōbi } ${\overset{\textnormal{}}{\text{木曜日}}}$ と ${\overset{\textnormal{}}{\text{金曜日}}}$ , you can simply write \emph{mokukin }${\overset{\textnormal{}}{\text{木金}}}$ . This is usually done in the context of scheduling. }

\par{1. ${\overset{\textnormal{いっしゅうかん}}{\text{一週間}}}$ の ${\overset{\textnormal{ま}}{\text{真}}}$ ん ${\overset{\textnormal{なか}}{\text{中}}}$ の ${\overset{\textnormal{ひ}}{\text{日}}}$ は ${\overset{\textnormal{なんようび}}{\text{何曜日}}}$ ですか。 \hfill\break
 \emph{Isshūkan no man\textquotesingle naka no hi wa nan\textquotesingle yōbi desu ka? \hfill\break
 }What is the day of the week that is in the middle of the week? }

\par{2. ${\overset{\textnormal{にちようび}}{\text{日曜日}}}$ の ${\overset{\textnormal{しんぶん}}{\text{新聞}}}$ には ${\overset{\textnormal{りょこうとくしゅう}}{\text{旅行特集}}}$ の ${\overset{\textnormal{きじ}}{\text{記事}}}$ が ${\overset{\textnormal{の}}{\text{載}}}$ っています。 \hfill\break
 \emph{Nichiyōbi no shimbun ni wa ryokō tokushū no kiji ga notte imasu. \hfill\break
 }There is a travel supplement article in the Sunday newspaper. }

\par{3. ${\overset{\textnormal{かようび}}{\text{火曜日}}}$ から ${\overset{\textnormal{もくようび}}{\text{木曜日}}}$ まで ${\overset{\textnormal{はたら}}{\text{働}}}$ きます。 \hfill\break
 \emph{Kayōbi kara mokuyōbi made hatarakimasu. \hfill\break
 }I work from Tuesday to Thursday. }

\par{4. ${\overset{\textnormal{あした}}{\text{明日}}}$ は ${\overset{\textnormal{なんよう}}{\text{何曜}}}$ ( ${\overset{\textnormal{び}}{\text{日}}}$ )ですか? \hfill\break
 \emph{Ashita wa nan\textquotesingle yō(bi) desu ka? \hfill\break
 }What day is it tomorrow? }

\par{5. ${\overset{\textnormal{きょう}}{\text{今日}}}$ は ${\overset{\textnormal{なんようび}}{\text{何曜日}}}$ ですか。 \hfill\break
 \emph{Kyō wa nan\textquotesingle yōbi desu ka? \hfill\break
 }What's today? }

\par{6. ${\overset{\textnormal{もくようび}}{\text{木曜日}}}$ は ${\overset{\textnormal{すいようび}}{\text{水曜日}}}$ の ${\overset{\textnormal{まえ}}{\text{前}}}$ ですか、 ${\overset{\textnormal{あと}}{\text{後}}}$ ですか? \hfill\break
 \emph{Mokuyōbi wa suiyōbi no mae desu ka, ato desu ka? \hfill\break
 }Is Thursday before or after Wednesday? }

\par{7. ${\overset{\textnormal{あさって}}{\text{明後日}}}$ は ${\overset{\textnormal{どよう}}{\text{土曜}}}$ です。 \hfill\break
 \emph{Asatte wa doyō desu. \hfill\break
 }The day after tomorrow is Saturday. }

\par{8. なぜ ${\overset{\textnormal{にちようび}}{\text{日曜日}}}$ に ${\overset{\textnormal{がくせい}}{\text{学生}}}$ は ${\overset{\textnormal{がっこう}}{\text{学校}}}$ に ${\overset{\textnormal{い}}{\text{行}}}$ かないんですか。 \hfill\break
 \emph{Naze nichiyōbi ni gakusei wa gakkō ni ikanai n desu ka? \hfill\break
 }Why don't students go to school on Sunday? }

\par{9. ${\overset{\textnormal{すいようび}}{\text{水曜日}}}$ に ${\overset{\textnormal{き}}{\text{来}}}$ てください。 \hfill\break
 \emph{Suiyōbi ni kite kudasai. \hfill\break
 }Please come on Wednesday. }

\par{10. ${\overset{\textnormal{にちようび}}{\text{日曜日}}}$ は ${\overset{\textnormal{いちしゅうかん}}{\text{一週間}}}$ の ${\overset{\textnormal{はじ}}{\text{初}}}$ めの ${\overset{\textnormal{ひ}}{\text{日}}}$ で、 ${\overset{\textnormal{どようび}}{\text{土曜日}}}$ は ${\overset{\textnormal{お}}{\text{終}}}$ わりの ${\overset{\textnormal{ひ}}{\text{日}}}$ だ。 \hfill\break
 \emph{Nichiyōbi wa isshūkan no hajime no hi de, doyōbi wa owari no hi da. \hfill\break
 }Sunday is the first day of the week, and Saturday is the last day. }

\par{\textbf{Exercises }: }

\par{1. Try to use the Japanese days of the week rather than in English. Get others involved to help you remember them. }

\par{2. Write out the days of the week in order 10 times. }
      
\section{Hinichi 日にち: Days of the Month}
 
\par{ You would expect the days of the month to be the same as how you count days in general. Although this is somewhat true, several days of the month utilize native words, hybrids of Sino-Japanese and native parts, and set readings. 1-10 are especially complicated. }
 
\begin{ltabulary}{|P|P|P|P|P|P|}
\hline 
 
  1日 
 &    \textbf{ついたち }\hfill\break
\textbf{ }\emph{Tsuitachi }
 &   2日 
 &    \textbf{ふつか \hfill\break
 }\emph{Futsuka }
 &   3日 
 &    \textbf{みっか \hfill\break
 }\emph{Mikka }
 \\ \cline{1-6} 
 
  4日 
 &    \textbf{よっか \hfill\break
 }\emph{Yokka }
 &   5日 
 &    \textbf{いつか \hfill\break
 }\emph{Itsuka }
 &   6日 
 &    \textbf{むいか \hfill\break
 }\emph{Muika }
 \\ \cline{1-6} 
 
  7日 
 &    \textbf{なのか \hfill\break
 }\emph{Nanoka }
 &   8日 
 &    \textbf{ようか \hfill\break
 }\emph{Yōka }
 &   9日 
 &    \textbf{ここのか \hfill\break
 }\emph{Kokonoka }
 \\ \cline{1-6} 
 
  10日 
 &    \textbf{とおか \hfill\break
 }\emph{Tōka }
 &   11日 
 &   じゅういちにち \hfill\break
 \emph{Jūichinichi }
 &   12日 
 &   じゅうににち \hfill\break
 \emph{Jūninichi }
 \\ \cline{1-6} 
 
  13日 
 &   じゅうさんにち \hfill\break
 \emph{Jūsan'nichi }
 &   14日 
 &    \textbf{じゅうよっか \hfill\break
 }\emph{Jūyokka }
 &   15日 
 &   じゅうごにち \hfill\break
 \emph{Jūgonichi }
 \\ \cline{1-6} 
 
  16日 
 &   じゅうろくにち \hfill\break
 \emph{Jūrokunichi }
 &   17日 
 &    \textbf{じゅうしちにち \hfill\break
 }\emph{Jūshichinichi }
 &   18日 
 &   じゅうはちにち \hfill\break
 \emph{Jūhachinichi }
 \\ \cline{1-6} 
 
  19日 
 &    \textbf{じゅうくにち \hfill\break
 }\emph{Jūkunichi }
 &   20日 
 &    \textbf{はつか \hfill\break
 }\emph{Hatsuka }
 &   21日 
 &   にじゅういちにち \hfill\break
 \emph{Nijūichinichi }
 \\ \cline{1-6} 
 
  22日 
 &   にじゅうににち \hfill\break
 \emph{Nijūninichi }
 &   23日 
 &   にじゅうさんにち \hfill\break
 \emph{Nijūsan'nichi }
 &   24日 
 &   \textbf{ }\textbf{にじゅうよっか \hfill\break
 }\emph{Nijūyokka }
 \\ \cline{1-6} 
 
  25日 
 &   にじゅうごにち \hfill\break
 \emph{Nijūgonichi }
 &   26日 
 &   にじゅうろくにち \hfill\break
 \emph{Nijūrokunichi }
 &   27日 
 &    \textbf{にじゅうしちにち \hfill\break
 }\emph{Nijūshichinichi }
 \\ \cline{1-6} 
 
  28日 
 &   にじゅうはちにち \hfill\break
 \emph{Nijūhachinichi }
 &   29日 
 &    \textbf{にじゅうくにち \hfill\break
 }\emph{Nijūkunichi }
 &   30日 
 &   さんじゅうにち \hfill\break
 \emph{Sanjūnichi }
 \\ \cline{1-6} 
 
  31日 
 &   さんじゅういちにち \hfill\break
 \emph{Sanjūichinichi }
 &   何日 
 &   なんにち \hfill\break
 \emph{Nan'nichi }
 &     &     \\ \cline{1-6} 
 
\end{ltabulary}
 
\par{\textbf{Reading Notes }: }
 
\par{1. \emph{Tsuitachi }ついたち is a contraction of \emph{tsukitachi }つきたち, which means “the start of the month.” The 1st may also be \emph{ippi }いっぴ in the business world. \hfill\break
2. The native counter for days is - \emph{ka }日 . It is seen in several of these phrases. }
 
\par{3. 7日 is traditionally read as “ \emph{nanuka }なぬか” and is still somewhat used. }
 
\par{4. The last day of a \emph{ }month is called “ \emph{misoka } ${\overset{\textnormal{}}{\text{晦・三十日}}}$ .” New Year's Eve is “ \emph{Ōmisoka } ${\overset{\textnormal{}}{\text{大晦日}}}$ . \hfill\break
5. The first three days of the New Year is called “ \emph{sanganichi } ${\overset{\textnormal{}}{\text{三}}}$ が ${\overset{\textnormal{}}{\text{日}}}$ .” }
 
\par{\textbf{Particle Note }: The particle \emph{ni }に usually follows these phrases whenever “in\slash on” would be used in English. However, it can frequently be omitted. }
 
\par{10. ${\overset{\textnormal{きょう}}{\text{今日}}}$ は ${\overset{\textnormal{なんにち}}{\text{何日}}}$ ですか。 \hfill\break
 \emph{Kyō wa nan\textquotesingle nichi desu ka? \hfill\break
 }What day is it? }
 
\par{11. ${\overset{\textnormal{なの}}{\text{7}}}$ ${\overset{\textnormal{か}}{\text{日}}}$ の ${\overset{\textnormal{あさ}}{\text{朝}}}$ (に) ${\overset{\textnormal{とうちゃく}}{\text{到着}}}$ しました。 \hfill\break
 \emph{Nanoka no asa (ni) tōchaku shimashita. \hfill\break
 }I arrived on the morning of the seventh. }
 
\par{12. ${\overset{\textnormal{きゅうしゅう}}{\text{九州}}}$ では ${\overset{\textnormal{いつ}}{\text{5}}}$ ${\overset{\textnormal{か}}{\text{日}}}$ からたくさん ${\overset{\textnormal{あめ}}{\text{雨}}}$ が ${\overset{\textnormal{ふ}}{\text{降}}}$ っています。 \hfill\break
 \emph{Kyūshū de wa itsuka kara takusan ame ga futte imasu. \hfill\break
 }In Kyushu, a lot of rain has been falling since the fifth. }
 
\par{13. ${\overset{\textnormal{じゅういち}}{\text{11}}}$ ${\overset{\textnormal{にち}}{\text{日}}}$ に、 ${\overset{\textnormal{うみ}}{\text{海}}}$ に ${\overset{\textnormal{う}}{\text{浮}}}$ いている ${\overset{\textnormal{ぞう}}{\text{象}}}$ を ${\overset{\textnormal{かい}}{\text{海}}}$ ${\overset{\textnormal{ぐん}}{\text{軍}}}$ の ${\overset{\textnormal{ふね}}{\text{船}}}$ が ${\overset{\textnormal{み}}{\text{見}}}$ つけました。 \hfill\break
 \emph{Jūichinichi ni, umi ni uite iru zō wo gun ga mitsukemashita. \hfill\break
 }On the eleventh, a navy vessel found an elephant floating in the ocean. }
14. ${\overset{\textnormal{よこはまこう}}{\text{横浜港}}}$ で ${\overset{\textnormal{じゅうよっ}}{\text{14}}}$ 日(に) ${\overset{\textnormal{ひあり}}{\text{ヒアリ}}}$ が ${\overset{\textnormal{ごひゃっ}}{\text{500}}}$ ${\overset{\textnormal{ぴき}}{\text{匹}}}$ くらい ${\overset{\textnormal{み}}{\text{見}}}$ つかりました。 \hfill\break
 \emph{Yokohama-kō de jūyokka (ni) hiari ga gohyappiki kurai mitsukarimashita. \hfill\break
 }About 500 fire ants were found on the fourteenth in the Port of Yokohama.       
\section{Nissū 日数: Number of Days}
 
\par{ Days are counted with - \emph{nichi(kan) }日(間). - \emph{kan }間 is not obligatory but is used to prevent ambiguity except with the number 1. Very rarely, you\textquotesingle ll see it used with - \emph{kan }間 in things like forms. Incidentally, except for “one day,” all day phrases that were exceptional above are used instead with little exception. }

\begin{ltabulary}{|P|P|P|P|P|P|}
\hline 

1 & いちにち & 2 & ふつか(かん) & 3 & みっか(かん) \\ \cline{1-6}

4 & よっか(かん) & 5 & いつか(かん) & 6 & むいか(かん) \\ \cline{1-6}

7 &  \textbf{なのか(かん) }\hfill\break
なぬか(かん) & 8 & ようか(かん) & 9 & ここのか(かん) \\ \cline{1-6}

10 & とおか(かん) & 14 & じゅうよっか(かん) & 17 &  \textbf{じゅうしちにち(かん) }\hfill\break
じゅうななにち(かん) \\ \cline{1-6}

19 & じゅうくにち(かん) & 20 & はつか(かん) & 24 & にじゅうよっか(かん) \\ \cline{1-6}

27 &  \textbf{にじゅうしちにち(かん) }\hfill\break
にじゅうななにち(かん) & 29 & にじゅうくにち(かん) & 30 & さんじゅうにち(かん) \\ \cline{1-6}

50 & ごじゅうにち(かん) & 100 & ひゃくにち(かん) & ? & なんにち(かん) \\ \cline{1-6}

\end{ltabulary}
 
\par{\hfill\break
15. このソフトウェアを ${\overset{\textnormal{さんじゅう}}{\text{30}}}$ ${\overset{\textnormal{にちかんむりょう}}{\text{日間無料}}}$ で ${\overset{\textnormal{つか}}{\text{使}}}$ いました。 \hfill\break
 \emph{Kono sofutowea wo sanjūnichikan muryō de tsukaimashita. \hfill\break
 }I used this software for free for 30 days. }
 
\par{16. ヨーロッパを ${\overset{\textnormal{とお}}{\text{10}}}$ ${\overset{\textnormal{かかんりょこう}}{\text{日間旅行}}}$ しました。 \hfill\break
 \emph{Yōroppa wo tokakan ryokō shimashita. }\hfill\break
I traveled Europe for ten days. }
 
\par{17. ${\overset{\textnormal{みやこじま}}{\text{宮古島}}}$ に ${\overset{\textnormal{みっ}}{\text{3}}}$ ${\overset{\textnormal{かかんりょこう}}{\text{日間旅行}}}$ しました。 \hfill\break
 \emph{Miyako-jima ni mikkakan ryokō shimashita. \hfill\break
 }I traveled to Miyako-jima for three days. }
 
\par{18. ${\overset{\textnormal{はいたつ}}{\text{配達}}}$ には ${\overset{\textnormal{ふつ}}{\text{2}}}$ ${\overset{\textnormal{か}}{\text{日}}}$ ( ${\overset{\textnormal{あいだ}}{\text{間}}}$ )かかります。 \hfill\break
 \emph{Haitatsu ni wa futsuka(kan) kakarimasu. \hfill\break
 }It takes two days for delivery. }
 
\par{\textbf{Grammar Note }: - \emph{kan }間 is frequently omitted when the verb \emph{kakaru }かかる is used. This verb, in the sense of time, indicates how long something takes to transpire. }
 
\par{19. ${\overset{\textnormal{ねつ}}{\text{熱}}}$ が ${\overset{\textnormal{むい}}{\text{6}}}$ ${\overset{\textnormal{かかんつづ}}{\text{日間続}}}$ きました。 \hfill\break
 \emph{Netsu ga muikakan tsuzukimashita. \hfill\break
 }The\slash my fever continued for six days. }
 
\par{20. ${\overset{\textnormal{なんにち}}{\text{何日}}}$ ( ${\overset{\textnormal{あいだ}}{\text{間}}}$ ) ${\overset{\textnormal{たいざい}}{\text{滞在}}}$ する ${\overset{\textnormal{よてい}}{\text{予定}}}$ ですか。 \hfill\break
 \emph{Nan\textquotesingle nichi(kan) taizai suru yotei desu ka? \hfill\break
 }How long will you stay? }
      
\section{-shūkan 週間: Weeks}
 
\par{  The word for “week” is \emph{shū(kan) }週(間). To say “this week,” you use the word \emph{konshū }今週. For “last week,” you use \emph{senshū }先週. Like their English counterparts not being used with the prepositions “in” or “on,” neither \emph{konshū }今週 nor \emph{senshū }先週 are typically used with the particle \emph{ni }に. To count weeks, you use the word for week in the form - \emph{shūkan }週間 as a counter. }

\begin{ltabulary}{|P|P|P|P|P|P|}
\hline 

1 & いっしゅうかん & 2 & にしゅうかん & 3 & さんしゅうかん \\ \cline{1-6}

4 & よんしゅうかん & 5 & ごしゅうかん & 6 & ろくしゅうかん \\ \cline{1-6}

7 & ななしゅうかん & 8 &  \textbf{はちしゅうかん }\hfill\break
はっしゅうかん & 9 & きゅうしゅうかん \\ \cline{1-6}

10 &  \textbf{じゅっしゅうかん \hfill\break
 }じっしゅうかん & 100 & ひゃくしゅうかん & ? & なんしゅうかん \\ \cline{1-6}

\end{ltabulary}
 
\par{\hfill\break
21. ${\overset{\textnormal{いっしゅうかん}}{\text{一週間}}}$ は ${\overset{\textnormal{なのか}}{\text{七日}}}$ ( ${\overset{\textnormal{かん}}{\text{間}}}$ )です。 \hfill\break
 \emph{Isshūkan wa nanoka(kan) desu. \hfill\break
 }A week is seven days. }
 
\par{22. ${\overset{\textnormal{しごと}}{\text{仕事}}}$ でモスクワに ${\overset{\textnormal{しゅっちょう}}{\text{出張}}}$ して、(そこで)2週間過ごしました。 \hfill\break
 \emph{Shigoto de Mosukuwa ni shutchō shite, (soko de) nishūkan sugoshimashita. \hfill\break
 }I went on a business trip to Moscow for work, and I spent two weeks (there). }
 
\par{23. ${\overset{\textnormal{こんしゅう}}{\text{今週}}}$ は ${\overset{\textnormal{にっしょく}}{\text{日食}}}$ が ${\overset{\textnormal{お}}{\text{起}}}$ きます。 \hfill\break
 \emph{Konshū wa nisshoku ga okimasu. \hfill\break
 }This week, there\textquotesingle s going to be a solar eclipse. }
 
\par{24. ${\overset{\textnormal{せんしゅう}}{\text{先週}}}$ も ${\overset{\textnormal{てんき}}{\text{天気}}}$ が ${\overset{\textnormal{よ}}{\text{良}}}$ かったです。 \hfill\break
 \emph{Senshū mo tenki ga yokatta desu. \hfill\break
 }The weather was also good last week. }
 
\par{25. ${\overset{\textnormal{はつ}}{\text{20}}}$ ${\overset{\textnormal{か}}{\text{日}}}$ から ${\overset{\textnormal{まるいっしゅうかんゆき}}{\text{丸一週間雪}}}$ が( ${\overset{\textnormal{ふ}}{\text{降}}}$ り) ${\overset{\textnormal{つづ}}{\text{続}}}$ きました。 \hfill\break
 \emph{Hatsuka kara maru-isshūkan yuki ga (furi)tsuzukimashita. }\hfill\break
Snow continued to fall for a whole week since the twentieth. }
 
\par{26. 明日から3 ${\overset{\textnormal{しゅうかん}}{\text{週間}}}$ お ${\overset{\textnormal{ねが}}{\text{願}}}$ いします。 \hfill\break
 \emph{Asu\slash ashita kara sanshūkan onegai shimasu. \hfill\break
 }For three weeks from tomorrow, please. }
 
\par{\textbf{Reading Note }: \emph{Asu }is slightly politer than \emph{ashita }, which makes it very likely in a sentence like this. }
      
\section{Tsuki no Yobikata 月の呼び方: The Months}
 
\par{ The names of the months ( \emph{tsuki }月) are simply expressed with the number of the month plus the counter - \emph{gatsu }月. \hfill\break
}

\begin{ltabulary}{|P|P|P|P|P|P|}
\hline 

January & いちがつ \hfill\break
 \emph{Ichigatsu }& February & にがつ \hfill\break
 \emph{Nigatsu }& March & さんがつ \hfill\break
 \emph{Sangatsu }\\ \cline{1-6}

April & しがつ \hfill\break
 \emph{Shigatsu }& May & ごがつ \hfill\break
 \emph{Gogatsu }& June & ろくがつ \hfill\break
 \emph{Rokugatsu }\\ \cline{1-6}

July & しちがつ \hfill\break
 \emph{Shichigatsu }& August & はちがつ \hfill\break
 \emph{Hachigatsu }& September & くがつ \hfill\break
 \emph{Kugatsu }\\ \cline{1-6}

October & じゅうがつ \hfill\break
 \emph{Jū }gatsu & November & じゅういちがつ \hfill\break
 \emph{Jūichigatsu }& December & じゅうにがつ \hfill\break
 \emph{Jūnigatsu }\\ \cline{1-6}

? & なんがつ \hfill\break
 \emph{Nangatsu }&  &  &  &  \\ \cline{1-6}

\end{ltabulary}
 
\par{\textbf{Reading Note }: As you can see, April and September use the not so frequently used variants of 4 and 9. You should also note that July is \emph{shichigatsu }しちがつ instead of \emph{nanagatsu }なながつ, which is only dialectical. }
 
\par{27. ${\overset{\textnormal{にほん}}{\text{日本}}}$ では ${\overset{\textnormal{しんがっき}}{\text{新学期}}}$ は ${\overset{\textnormal{し}}{\text{4}}}$ ${\overset{\textnormal{がつ}}{\text{月}}}$ に ${\overset{\textnormal{はじ}}{\text{始}}}$ まります。 \hfill\break
 \emph{Nihon de wa shingakki wa shigatsu ni hajimarimasu. \hfill\break
 }In Japan, new semesters begin in April. }
 
\par{28. ${\overset{\textnormal{わたし}}{\text{私}}}$ は ${\overset{\textnormal{いち}}{\text{1}}}$ ${\overset{\textnormal{がつ}}{\text{月}}}$ ${\overset{\textnormal{じゅうく}}{\text{19}}}$ ${\overset{\textnormal{にち}}{\text{日}}}$ に ${\overset{\textnormal{おおさか}}{\text{大阪}}}$ に ${\overset{\textnormal{いん}}{\text{引}}}$ っ ${\overset{\textnormal{こ}}{\text{越}}}$ しました。 \hfill\break
 \emph{Watashi wa ichigatsu jūkunichi ni Ōsaka ni hikkoshimashita. \hfill\break
 }I moved to Osaka on January 19 th . }
 
\par{29. ${\overset{\textnormal{かのじょ}}{\text{彼女}}}$ は ${\overset{\textnormal{ご}}{\text{5}}}$ ${\overset{\textnormal{がつ}}{\text{月}}}$ の ${\overset{\textnormal{お}}{\text{終}}}$ わりにカナダに ${\overset{\textnormal{む}}{\text{向}}}$ かいました。 \hfill\break
 \emph{Kanojo wa gogatsu no owari ni Kanada ni mukaimashita. \hfill\break
 }She headed for Canada at the end of May. }
 
\par{30. ${\overset{\textnormal{ぼく}}{\text{僕}}}$ の ${\overset{\textnormal{たんじょうび}}{\text{誕生日}}}$ は ${\overset{\textnormal{じゅういち}}{\text{11}}}$ ${\overset{\textnormal{がつ}}{\text{月}}}$ ${\overset{\textnormal{じゅうろく}}{\text{16}}}$ ${\overset{\textnormal{にち}}{\text{日}}}$ です。 \hfill\break
 \emph{Boku no tanjōbi wa jūichigatsu jūrokunichi desu. \hfill\break
 }My birthday is on November 16 th . }
31. クリスマスは ${\overset{\textnormal{じゅうに}}{\text{12}}}$ ${\overset{\textnormal{がつ}}{\text{月}}}$ ${\overset{\textnormal{にじゅうご}}{\text{25}}}$ ${\overset{\textnormal{にち}}{\text{日}}}$ ですね。 \hfill\break
 \emph{Kurisumasu wa jūnigatsu nijūgonichi desu ne? \hfill\break
 }Christmas is on December 25 th , right?       
\section{The Suffix -jun 旬}
 
\par{ - \emph{jun }旬 means "tens days of a month" and is used in the following phrases. }

\begin{ltabulary}{|P|P|P|}
\hline 
 
  上旬 
 &   じょうじゅん \hfill\break
\emph{J }\emph{ōjun }
 &   First ten days of the month 
 \\ \cline{1-3} 
 
  中旬 
 &   ちゅうじゅん \hfill\break
 \emph{Ch }\emph{ūjun }
 &   Middle ten days of the month 
 \\ \cline{1-3} 
 
  下旬 
 &   げじゅん \hfill\break
 \emph{Gejun }
 &   Last ten days of the month 
 \\ \cline{1-3} 
 
\end{ltabulary}

\par{\hfill\break
32. ${\overset{\textnormal{さん}}{\text{3}}}$ ${\overset{\textnormal{がつじょうじゅん}}{\text{月上旬}}}$ に ${\overset{\textnormal{はっぴょう}}{\text{発表}}}$ します。 \hfill\break
 \emph{Sangatsu jōjun ni happyo shimasu. \hfill\break
 }We will announce it in the first ten days of next month. }

\par{33. ${\overset{\textnormal{く}}{\text{9}}}$ ${\overset{\textnormal{がつちゅうじゅん}}{\text{月中旬}}}$ に ${\overset{\textnormal{しゅっちょう}}{\text{出張}}}$ することになりました。 \hfill\break
 \emph{Kugatsu chūjun ni shutchō suru koto ni narimashita. \hfill\break
 }I am to go on a business trip in mid-September. }

\par{34. ${\overset{\textnormal{じゅういち}}{\text{11}}}$ ${\overset{\textnormal{がつげじゅん}}{\text{月下旬}}}$ に ${\overset{\textnormal{はつばい}}{\text{発売}}}$ した。 \hfill\break
 \emph{Jūichigatsu gejun ni hatsubai shita. \hfill\break
 }it went on sale in the latter part of November. }
      
\section{Period of Months}
 
\begin{center}
\textbf{- \emph{kagetsu(kan) }ヶ月(間) }
\end{center}

\par{ To generically count months, you must use the counter - \emph{kagetsu(kan) }ヶ月(間). }

\begin{ltabulary}{|P|P|P|P|P|P|}
\hline 

1 & いっかげつ(かん) & 2 & にかげつ(かん) & 3 & さんかげつ(かん) \\ \cline{1-6}

4 & よんかげつ(かん) & 5 & ごかげつ(かん) & 6 & ろっかげつ(かん) \\ \cline{1-6}

7 & ななかげつ(かん) & 8 &  \textbf{はちかげつ(かん) }\hfill\break
はっかげつ(かん) & 9 & きゅうかげつ(かん) \\ \cline{1-6}

10 &  \textbf{じゅっかげつ(かん) \hfill\break
 }じっかげつ(かん) & 100 & ひゃっかげつ(かん) & ? & なんかげつ(かん) \\ \cline{1-6}

\end{ltabulary}

\par{\hfill\break
35. ${\overset{\textnormal{いっ}}{\text{一}}}$ ${\overset{\textnormal{か}}{\text{ヶ}}}$ ${\overset{\textnormal{げつ}}{\text{月}}}$ は ${\overset{\textnormal{やく}}{\text{約}}}$ ${\overset{\textnormal{よん}}{\text{4}}}$ ${\overset{\textnormal{しゅうかん}}{\text{週間}}}$ です。 \hfill\break
 \emph{Ikkagetsu wa yaku-yonshūkan desu. \hfill\break
 }There are about four weeks in a month. }

\par{36. ${\overset{\textnormal{に}}{\text{2}}}$ ${\overset{\textnormal{か}}{\text{ヶ}}}$ ${\overset{\textnormal{げつぶん}}{\text{月分}}}$ を ${\overset{\textnormal{はら}}{\text{払}}}$ いました。 \hfill\break
 \emph{Nikagetsu-bun wo haraimashita. \hfill\break
 }I paid two months\textquotesingle  worth. }

\par{37. ${\overset{\textnormal{さん}}{\text{3}}}$ ${\overset{\textnormal{か}}{\text{ヶ}}}$ ${\overset{\textnormal{げつ}}{\text{月}}}$ ( ${\overset{\textnormal{かん}}{\text{間}}}$ ) ${\overset{\textnormal{ともだち}}{\text{友達}}}$ に ${\overset{\textnormal{あ}}{\text{会}}}$ っていなかった。 \hfill\break
 \emph{Sankagetsu(kan) tomodachi ni atte inakatta. \hfill\break
 }I hadn\textquotesingle t met my friends in three months. }

\par{38. ${\overset{\textnormal{ご}}{\text{5}}}$ ${\overset{\textnormal{か}}{\text{ヶ}}}$ ${\overset{\textnormal{げつかんおも}}{\text{月間思}}}$ いっきり ${\overset{\textnormal{きん}}{\text{筋}}}$ トレしています。 \hfill\break
 \emph{Gokagetsu(kan) omoikkiri kintore shite imasu. \hfill\break
 }I\textquotesingle ve been weight training wholeheartedly for five months. }

\begin{center}
\textbf{- \emph{tsuki }月 }
\end{center}

\par{ In the spoken language, month periods are often expressed with native phrases for 1-4 as seen below with the counter - \emph{tsuki }月. Historically, this counter could be used with more numbers. For instance, there is an old phrase, \emph{totsuki tōka }十月十日, which refers to day 10 of the 10 th month to refer to how long pregnancy typically is. Outside of such set phrases, though - \emph{tsuki }月 is not outside the numbers 1-4. }

\begin{ltabulary}{|P|P|P|P|P|P|P|P|}
\hline 

1 & ひとつき & 2 & ふたつき & 3 & みつき & 4 & よつき \\ \cline{1-8}

\end{ltabulary}

\par{\hfill\break
39. ${\overset{\textnormal{ひとつき}}{\text{一月}}}$ ${\overset{\textnormal{さんぜん}}{\text{3000}}}$ ${\overset{\textnormal{えん}}{\text{円}}}$ です! \hfill\break
 \emph{Hitotsuki sanzen\textquotesingle en desu! \hfill\break
 }It\textquotesingle s 3,000 yen a month! }

\par{40. ふた ${\overset{\textnormal{つき}}{\text{月}}}$ でも、み ${\overset{\textnormal{つき}}{\text{月}}}$ でも ${\overset{\textnormal{ま}}{\text{待}}}$ ちます。 \hfill\break
 \emph{Futatsuki demo, mitsuki demo machimasu. \hfill\break
 }I will wait, whether it be two months or even three months. }

\par{41. ${\overset{\textnormal{みっか}}{\text{三日}}}$ でも ${\overset{\textnormal{みつき}}{\text{三月}}}$ でも ${\overset{\textnormal{かま}}{\text{構}}}$ いませんよ。 \hfill\break
 \emph{Mikka demo mitsuki demo kamaimasen yo. \hfill\break
 }I don\textquotesingle t mind if it\textquotesingle s three days or three months. }

\par{42. ${\overset{\textnormal{やちん}}{\text{家賃}}}$ は\{ひと ${\overset{\textnormal{つき}}{\text{月}}}$ ・ ${\overset{\textnormal{いっ}}{\text{1}}}$ ${\overset{\textnormal{か}}{\text{ヶ}}}$ ${\overset{\textnormal{げつ}}{\text{月}}}$ \}いくらですか。 \hfill\break
 \emph{Yachin wa [hitotsuki\slash ikkagetsu] ikura desu ka? \hfill\break
 }How much is rent per month? }

\par{43. ${\overset{\textnormal{つきづき}}{\text{月々}}}$ の ${\overset{\textnormal{えんじょ}}{\text{援助}}}$ は\{ ${\overset{\textnormal{に}}{\text{2}}}$ ${\overset{\textnormal{か}}{\text{ヶ}}}$ ${\overset{\textnormal{げつ}}{\text{月}}}$ ・ふた ${\overset{\textnormal{つき}}{\text{月}}}$ \} ${\overset{\textnormal{とどこお}}{\text{滞}}}$ っています。 \hfill\break
 \emph{Tsukizuki no enjo wa [nikagetsu\slash futatsuki] todokotte imasu. \hfill\break
 }(My) monthly support is two months behind. }
      
\section{Period of Years}
 
\begin{center}
\textbf{- \emph{nen(kan) }年(間) }
\end{center}

\par{ The counter for year is - \emph{nen(kan) }年(間). This counter is also one of the handful of counters in which 4 must be read as " \emph{yo }よ." }

\begin{ltabulary}{|P|P|P|P|P|P|}
\hline 

1 & いちねん(かん) & 2 & にねん(かん) & 3 & さんねん(かん) \\ \cline{1-6}

4 & よねん(かん) & 5 & ごねんかん & 6 & ろくねん(かん) \\ \cline{1-6}

7 &  \textbf{ななねん(かん) \hfill\break
}\textbf{ }しちねん(かん) & 8 & はちねん(かん) & 9 &  \textbf{きゅうねん(かん) \hfill\break
}\textbf{ }くねん(かん) \\ \cline{1-6}

10 & じゅうねん(かん) & 14 & じゅうよねん(かん) & 100 & ひゃくねん(かん) \\ \cline{1-6}

\end{ltabulary}

\par{ When used to refer to what year it is, - \emph{kan }間 is never used. Generally, there are no differences with how numbers are counted when used to refer to what year it is. However, it is worth noting that 2007 and 2009 are typically read as “ \emph{nisenshichinen }にせんしちねん” and “ \emph{nisenkunen }にせんくねん” respectively, although “ \emph{nisen\textquotesingle nananen }にせんななねん” and “ \emph{nisenkyūnen }にせんきゅうねん” respectively are also common. The same goes for 2017, 2019, etc. }

\par{44. ${\overset{\textnormal{たいわんご}}{\text{台湾語}}}$ を ${\overset{\textnormal{さん}}{\text{3}}}$ ${\overset{\textnormal{ねんかんべんきょう}}{\text{年間勉強}}}$ しています。 \hfill\break
 \emph{Taiwango wo san\textquotesingle nenkan benkyō shite imasu. \hfill\break
 }I\textquotesingle ve been studying Taiwanese for three years. }

\par{45. マレー(シア) ${\overset{\textnormal{ご}}{\text{語}}}$ を ${\overset{\textnormal{しゅうとく}}{\text{習得}}}$ するのに ${\overset{\textnormal{ご}}{\text{5}}}$ ${\overset{\textnormal{ねん}}{\text{年}}}$ ( ${\overset{\textnormal{かん}}{\text{間}}}$ )かかりました。 \hfill\break
 \emph{Marē(shia)go wo shūtoku suru no ni gonen(kan) kakarimashita. \hfill\break
 }It took me five years to acquire Malay. }

\par{46. ${\overset{\textnormal{わたし}}{\text{私}}}$ は ${\overset{\textnormal{みなみ}}{\text{南}}}$ アフリカに ${\overset{\textnormal{じゅう}}{\text{10}}}$ ${\overset{\textnormal{ねん}}{\text{年}}}$ ( ${\overset{\textnormal{かん}}{\text{間}}}$ ) ${\overset{\textnormal{す}}{\text{住}}}$ んでいました。 \hfill\break
 \emph{Watashi wa Minami Afurika ni jūnen(kan) sunde imashita. \hfill\break
 }I once lived in South Africa for ten years. }

\par{47. ${\overset{\textnormal{ほしょう}}{\text{保証}}}$ は ${\overset{\textnormal{ご}}{\text{5}}}$ ${\overset{\textnormal{ねんかん}}{\text{年間}}}$ です。 \hfill\break
 \emph{Hoshō wa gonenkan desu. \hfill\break
 }The warranty is for five years. }

\par{48. ${\overset{\textnormal{かれ}}{\text{彼}}}$ は ${\overset{\textnormal{はち}}{\text{8}}}$ ${\overset{\textnormal{ねんかんせんせい}}{\text{年間先生}}}$ をしています。 \hfill\break
 \emph{Kare wa hachinenkan sensei wo shite imasu. }\hfill\break
He has been a teacher for eight years. }

\par{49. ${\overset{\textnormal{ひづけ}}{\text{日付}}}$ は ${\overset{\textnormal{にせんじゅうしちねん}}{\text{2017}}}$ ${\overset{\textnormal{ねん}}{\text{年}}}$ ${\overset{\textnormal{し}}{\text{4}}}$ ${\overset{\textnormal{がつ}}{\text{月}}}$ ${\overset{\textnormal{にじゅうに}}{\text{22}}}$ ${\overset{\textnormal{にち}}{\text{日}}}$ です。 \hfill\break
 \emph{Hizuke wa nisenj }\emph{ūshichinen shigatsu nij }\emph{ūninichi desu. \hfill\break
 }The date is April 22 nd , 2017. }

\par{50. ${\overset{\textnormal{ことし}}{\text{今年}}}$ は ${\overset{\textnormal{へいせい}}{\text{平成}}}$ ${\overset{\textnormal{にじゅうく}}{\text{29}}}$ ${\overset{\textnormal{ねん}}{\text{年}}}$ です。 \hfill\break
 \emph{Kotoshi wa Heisei nij }\emph{ūkunen desu. \hfill\break
 }This year is Heisei Year 29. }

\par{\textbf{Culture Note }: In Japan, alongside the Western date system, there is traditional date system that utilizes the eras of emperors. The current era is called the Heisei Period ( \emph{Heisei Jidai }平成時代). }

\begin{center}
\textbf{- \emph{kanen }ヶ年 }
\end{center}

\par{ The counter - \emph{kanen }ヶ年 translates as “over…years” and is frequently employed in legislation and business settings. However, \emph{ikkanen }1ヶ年 is infrequently used because “one year” is typically too short for major projects to take place. }

\begin{ltabulary}{|P|P|P|P|P|P|P|P|}
\hline 

1 & いっかねん △ & 2 & にかねん & 3 & さんかねん & 4 & よんかねん \\ \cline{1-8}

5 & ごかねん & 6 & ろっかねん & 7 & ななかねん & 8 &  \textbf{はちかねん \hfill\break
}\textbf{ }はっかねん \\ \cline{1-8}

9 & きゅうかねん & 10 &  \textbf{じゅっかねん \hfill\break
}\textbf{ }じっかねん & 100 & ひゃっかねん & ? & なんかねん \\ \cline{1-8}

\end{ltabulary}

\par{\hfill\break
51. ${\overset{\textnormal{ご}}{\text{5}}}$ ${\overset{\textnormal{か}}{\text{ヶ}}}$ ${\overset{\textnormal{ねんけいかく}}{\text{年計画}}}$ を ${\overset{\textnormal{じっし}}{\text{実施}}}$ しました。 \hfill\break
 \emph{Gokanen keikaku wo jisshi shimashita. \hfill\break
 }We implemented a five-year plan. }

\par{52. ${\overset{\textnormal{じゅっ}}{\text{10}}}$ ${\overset{\textnormal{か}}{\text{ヶ}}}$ ${\overset{\textnormal{ねん}}{\text{年}}}$ の ${\overset{\textnormal{せんりゃく}}{\text{戦略}}}$ を ${\overset{\textnormal{じつげん}}{\text{実現}}}$ しました。 \hfill\break
 \emph{Jukkanen no senryaku wo jitsugen shimashita. \hfill\break
 }We actualized a 10-year strategy. }

\par{53. ${\overset{\textnormal{あら}}{\text{新}}}$ たな ${\overset{\textnormal{さん}}{\text{3}}}$ ${\overset{\textnormal{か}}{\text{ヶ}}}$ ${\overset{\textnormal{ねんすうちもくひょう}}{\text{年数値目標}}}$ を ${\overset{\textnormal{はっぴょう}}{\text{発表}}}$ しました。 \hfill\break
 \emph{Aratana sankanen suchi mokuhyō wo happyō shimashita. \hfill\break
 }We announced a new three-year target value. }

\par{54. ${\overset{\textnormal{われわれ}}{\text{我々}}}$ は ${\overset{\textnormal{に}}{\text{2}}}$ ${\overset{\textnormal{か}}{\text{ヶ}}}$ ${\overset{\textnormal{ねん}}{\text{年}}}$ にわたる ${\overset{\textnormal{けんきゅう}}{\text{研究}}}$ に ${\overset{\textnormal{ちゃくしゅ}}{\text{着手}}}$ しました。 \hfill\break
 \emph{Wareware wa nikanen ni wataru kenkyū ni chakushu shimashita. \hfill\break
 }We embarked a two-year long study. }

\par{55. ${\overset{\textnormal{じんせい}}{\text{人生}}}$ ${\overset{\textnormal{ごじゅっ}}{\text{50}}}$ ${\overset{\textnormal{か}}{\text{ヶ}}}$ ${\overset{\textnormal{ねんけいかく}}{\text{年計画}}}$ を ${\overset{\textnormal{た}}{\text{立}}}$ てました。 \hfill\break
 \emph{Jinsei gojukkanen keikaku wo tatemashita. \hfill\break
 }I made a fifty-year life plan. }

\begin{center}
\textbf{- \emph{nensei }年生 }
\end{center}

\par{ The counter - \emph{nensei }年生is used to indicate what grade in school someone is in. }

\begin{ltabulary}{|P|P|P|P|P|P|P|P|}
\hline 

1 & いちねんせい & 2 & にねんせい & 3 & さんねんせい & 4 & よねんせい \\ \cline{1-8}

5 & ごねんせい & 6 & ろくねんせい & 7 & ななねんせい & 8 & はちねんせい \\ \cline{1-8}

9 & きゅうねんせい & 10 & じゅうねんせい & ? & なんねんせい &  &  \\ \cline{1-8}

\end{ltabulary}

\par{  It is important to note that after 6, the only time one would use this counter would be when referencing the American school system which is divided into 12 grade levels. Although the Japanese education system is also divided into 12 grades, what they\textquotesingle re called is not the same. }

\begin{ltabulary}{|P|P|P|}
\hline 

Grade 1 &  \emph{Shōgakkō ichinen }小学校1年 &  \emph{Shō-Ichi }小1 \\ \cline{1-3}

Grade 2 &  \emph{Shōgakkō ninen }小学校2年 &  \emph{Shō-ni }小2 \\ \cline{1-3}

Grade 3 &  \emph{Shōgakkō san\textquotesingle nen }小学校3年 &  \emph{Shō-san }小3 \\ \cline{1-3}

Grade 4 &  \emph{Shōgakkō yonen }小学校4年 &  \emph{Shō-yon }小4 \\ \cline{1-3}

Grade 5 &  \emph{Shōgakkō gonen }小学校5年 &  \emph{Shō-go }小5 \\ \cline{1-3}

Grade 6 &  \emph{Shōgakkō rokunen }小学校6年 &  \emph{Shō-roku }小6 \\ \cline{1-3}

Grade 7 &  \emph{Chūgakkō ichinen }中学校1年 &  \emph{Chū-ichi }中1 \\ \cline{1-3}

Grade 8 &  \emph{Chūgakkō ninen }中学校2年 &  \emph{Chū-ni }中2 \\ \cline{1-3}

Grade 9 &  \emph{Chūgakkō san\textquotesingle nen }中学校3年 &  \emph{Chū-san }中3 \\ \cline{1-3}

Grade 10 &  \emph{Kōkō ichinen }高校1年 &  \emph{Kō-ichi }高1 \\ \cline{1-3}

Grade 11 &  \emph{Kōkō ninen }高校2年 &  \emph{Kō-ni }高2 \\ \cline{1-3}

Grade 12 &  \emph{Kōkō san\textquotesingle nen }高校3年 &  \emph{Kō-san }高3 \\ \cline{1-3}

\end{ltabulary}

\par{\hfill\break
56. ${\overset{\textnormal{いち}}{\text{1}}}$ ${\overset{\textnormal{ねんせい}}{\text{年生}}}$ がたくさん ${\overset{\textnormal{き}}{\text{来}}}$ ました。 \hfill\break
 \emph{Ichinensei ga takusan kimashita. \hfill\break
 }A lot of first graders came. }

\par{57. ${\overset{\textnormal{ぼく}}{\text{僕}}}$ は ${\overset{\textnormal{きょねんちゅうがく}}{\text{去年中学}}}$ ${\overset{\textnormal{に}}{\text{2}}}$ ${\overset{\textnormal{ねんせい}}{\text{年生}}}$ でした。 \hfill\break
 \emph{Boku wa kyonen chūgaku ninensei deshita. \hfill\break
 }I was in eighth grade last year. }

\par{58. ${\overset{\textnormal{かのじょ}}{\text{彼女}}}$ は ${\overset{\textnormal{いま}}{\text{今}}}$ ハーバード ${\overset{\textnormal{だいがく}}{\text{大学}}}$ の ${\overset{\textnormal{よ}}{\text{4}}}$ ${\overset{\textnormal{ねんせい}}{\text{年生}}}$ です。 \hfill\break
 \emph{Kanojo wa ima Hābādo Daigaku no yonensei desu. \hfill\break
 }She is now a senior at Harvard. }

\begin{center}
\textbf{- \emph{gakunen }学年 }
\end{center}

\par{\emph{ Gakunen }学年 means “academic year” and can also be used as a counter to count them. }

\begin{ltabulary}{|P|P|P|P|P|P|P|P|}
\hline 

1 & いちがくねん & 2 & にがくねん & 3 & さんがくねん & 4 & よんがくねん \\ \cline{1-8}

5 & ごがくねん & 6 & ろくがくねん & 7 &  \textbf{なながくねん \hfill\break
}\textbf{ }しちがくねん & 8 & はちがくねん \\ \cline{1-8}

9 & きゅうがくねん & 10 & じゅうがくねん & 11 & じゅういちがくねん & ? & なんがくねん \\ \cline{1-8}

\end{ltabulary}

\par{\hfill\break
59. ${\overset{\textnormal{いち}}{\text{1}}}$ ${\overset{\textnormal{がくねん}}{\text{学年}}}$ で ${\overset{\textnormal{がくしゅう}}{\text{学習}}}$ する ${\overset{\textnormal{ないよう}}{\text{内容}}}$ を ${\overset{\textnormal{してい}}{\text{指定}}}$ しました。 \hfill\break
 \emph{Ichigakunen de gakushū suru naiyō wo shitei shimashita. \hfill\break
 }We designated what material is to be studied in one academic year. }

\par{ 60. ${\overset{\textnormal{に}}{\text{2}}}$ ${\overset{\textnormal{がくねんさ}}{\text{学年差}}}$ の ${\overset{\textnormal{きょうだい}}{\text{兄弟}}}$ がいます。 \hfill\break
 \emph{Nigakunen-sa no kyōdai ga imasu. \hfill\break
 }I have siblings two grades apart (from me). }
    