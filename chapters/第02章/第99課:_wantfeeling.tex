    
\chapter{Want \& Feeling I}

\begin{center}
\begin{Large}
第99課: Want \& Feeling I: ~たい \& ほしい 
\end{Large}
\end{center}
 
\par{ Your wants and feelings are different from those of your friends'. This seemingly obvious difference is distinguished in Japanese. In English, both "Steve seems mad" and "Steve is mad" are grammatically correct and commonly used, but the same cannot be said for Japanese. }

\par{ In this lesson, we'll learn how to express first and second person wants, meaning you'll be able to say things like "I want to go the park" and "Do you want to?" }
      
\section{~たい: Want to do\dothyp{}\dothyp{}\dothyp{}}
 
\par{ ~たい is used to show one's own want to do something, to ask about someone's wants, or to show what people \textbf{in general }want. Because this is about wanting to do something, though, the verb needs to express an action that involves willpower of the person in question. }

\par{\textbf{漢字 Note }: ~度い is an old and rare 漢字 spelling of ~たい. You may find it if you read older stuff. }

\begin{center}
 \textbf{Conjugating ~たい }
\end{center}

\par{ As is the case for most endings, it attaches normally to the 連用形 of verbs.  }

\begin{ltabulary}{|P|P|P|P|P|P|P|}
\hline 

一段 & 食 \textbf{べ }たい & I want to eat & 見たい & I want to see & 信じたい & I want to believe \\ \cline{1-7}

五段 & 飲 \textbf{み }たい & I want to drink & 行きたい & I want to go & 知りたい & I want to know \\ \cline{1-7}

する &  \textbf{し }たい & I want to do & 見物したい & I want to sightsee & 買い物したい & I want to shop \\ \cline{1-7}

来る &  \textbf{来 }たい & I want to come &  &  &  &  \\ \cline{1-7}

\end{ltabulary}

\par{ ~たい conjugates as a 形容詞. Although it may be strange to associate volition to act as an adjectival phrase, "want" is still a state, and so this is how you can rationalize this oddity of grammar. }

\begin{ltabulary}{|P|P|P|}
\hline 

 & Plain Speech & Polite Speech \\ \cline{1-3}

Non-Past & ~たい & ~たいです \\ \cline{1-3}

Negative & ~たくない & ~たくないです・~たくありません \\ \cline{1-3}

Past & ~たかった & ~たかったです \\ \cline{1-3}

Negative Past & ~たくなかった & ~たくなかったです・~たくありませんでした \\ \cline{1-3}

\end{ltabulary}

\begin{center}
 \textbf{が Or を? }
\end{center}

\par{ Whenever a verb has a direct object, we know to mark it with を. However, because ~たい makes a verb an adjective regardless of whether it has a direct object or not, a direct object can consequently be marked with either が or を. Whenever を is used, though, a lot more emphasis is placed on one's volition. Of course, if we're referencing different kinds of verbs like travel verbs which would use に・へ instead, neither would be applicable. }

\begin{ltabulary}{|P|P|P|}
\hline 

 & が OK? & を OK? \\ \cline{1-3}

飲みたい & ○ & ○ \\ \cline{1-3}

見たい & ○ & ○ \\ \cline{1-3}

行きたい & x & x \\ \cline{1-3}

成りたい & x & x \\ \cline{1-3}

\end{ltabulary}

\par{1.  コーヒーを飲みたくない。 \hfill\break
I don't want to drink coffee. }

\par{2. 邪魔になりたくありません。 \hfill\break
I don't want to become a bother. }

\par{3. ピザ\{が・を\}食べたいです。 \hfill\break
I want to eat pizza. }

\par{4. キス(が・を)したいだけ! \hfill\break
I just want to kiss! }

\par{\textbf{Particle Note }: As demonstrated, が and をcan be dropped in casual speech. }

\par{${\overset{\textnormal{}}{\text{5. 映画}}}$ を ${\overset{\textnormal{み}}{\text{観}}}$ に ${\overset{\textnormal{}}{\text{行}}}$ きたいです。 \hfill\break
I want to go see a movie. }

\par{\textbf{Particle Note }: In this sentence, を cannot be replaced by が because it is part of the verb phrase 映画を観に行く which is being modified by ~たい as a whole. }

\begin{center}
 \textbf{Second Person }
\end{center}

\par{ Second person is normally used in questions. Things like ~たいでしょう could be used to indirectly express second person want without sounding rude or too direct. }

\par{6. 東京に行きたいですか。 \hfill\break
Do you want to go to Tokyo? }

\par{7. 日本に旅行したいですか。 \hfill\break
Do you want to travel to Japan? }

\par{8. 何\{が・を\}食べたい? (Plain) \hfill\break
What do you want to eat? }

\begin{center}
\textbf{~たい\{の・ん\}ですが: Asking for Advice }
\end{center}

\par{ When you are asking for advice on something you would like to do, it is best to use ~たい with ~のですが・んですが to be more polite. It is also more indirect, and what may or may not follow is something like 何がいいでしょうか. }

\par{9. お ${\overset{\textnormal{みやげ}}{\text{土産}}}$ を買いたいんですが、何がいいでしょうか。 \hfill\break
I'd like to buy a souvenir, but what would be good? }

\par{10. テレビを買いたいんですが、どちらがいいでしょうか。 \hfill\break
I'd like to buy a TV, but which would be good? }

\par{11. 温泉に行きたいのですが、どこがいいでしょうか。 \hfill\break
I'd like to go to a hot spring, but where is good? }
${\overset{\textnormal{おや}}{\text{親}}}$ は ${\overset{\textnormal{}}{\text{私}}}$ が ${\overset{\textnormal{}}{\text{大学}}}$ へ ${\overset{\textnormal{}}{\text{行}}}$ ってほしいと ${\overset{\textnormal{}}{\text{思}}}$ っています。 \hfill\break
My parents are wishing for me to go to college. \hfill\break
      
\section{欲しい: Wanting Something}
 
\par{ When showing first person "want for something," use the adjective ほしい, alternatively sometimes written in 漢字 as 欲しい. We'll learn later about how to express someone else's want for something, but just as reminder, below are ほしい's basic conjugations. }

\begin{ltabulary}{|P|P|}
\hline 

Plain Speech & ほしい \\ \cline{1-2}

Negative & ほしくない \\ \cline{1-2}

Past & ほしかった \\ \cline{1-2}

Negative Past & ほしくなかった \\ \cline{1-2}

\end{ltabulary}

\par{\textbf{Particle Note }: Although informal and somewhat improper, ~をほしい is also sometimes seen. }

\par{\textbf{Person Note }: As was the case  for ~たい, showing second person want for something is normally used in the form of question. }

\par{${\overset{\textnormal{}}{\text{12. 新}}}$ しいプリンターがほしいです。 \hfill\break
I want a new printer. }

\par{13. この本、ほしいでしょう? \hfill\break
You want this book, right? }

\par{14. このCDはどうしてもほしいから、買ってくれない? \hfill\break
I want this CD no matter what, so could you get it for me? }
      
\section{~てほしい: Wanting Something Done by\dothyp{}\dothyp{}\dothyp{}}
 
\par{ ~てほしい shows that you "want something done\slash to happen". The person you want to do the action for you is marked by the particle に. }
${\overset{\textnormal{}}{\text{15.}}}$ \textbf{に }きれいでいてほしい。 \hfill\break
I want \textbf{her }to stay beautiful. 
\par{16. ( \textbf{あなたに })毎日チャレンジしてほしい。 \hfill\break
I would like \textbf{for you }to challenge yourselves every day. }

\par{17. ( \textbf{僕に })山田さんと話してほしいんですか。 \hfill\break
Do you want \textbf{me }to talk with Mr. Yamada? }

\par{${\overset{\textnormal{}}{\text{18. 何}}}$ かアドバイスしてほしいのですが\dothyp{}\dothyp{}\dothyp{} \hfill\break
We'd like you to give us some sort of advice, but\dothyp{}\dothyp{}\dothyp{} }

\begin{center}
~てほしかった 
\end{center}

\par{ When ほしい is in the past tense, it shows regret that something didn't happen. }

\par{19. 彼氏にだけは分かってほしかった。 \hfill\break
I only wanted my boyfriend to understand. }

\par{20. もっと早く来てほしかった。 \hfill\break
I wanted you to have come earlier. }

\par{ In showing someone's wish for something to happen, ほしい is followed by ~と思う or ~と思っている with the latter imperative for third person. }

\par{21. ${\overset{\textnormal{おや}}{\text{親}}}$ は私に大学へ行ってほしいと思っています。 \hfill\break
My parents are wishing for me to go to college. }

\par{22. 復活してほしいと思う。 \hfill\break
I wish for it to revive. }

\begin{center}
 ~てもらいたい・いただきたい 
\end{center}

\par{ ~てほしい, even when polite, is not the most polite way to tell someone that you would like them do to something for you. In this case you are implying that you are to be receiving a favor, and therefore, ~てもらいたい (~ていただきたい being more honorific) would be most appropriate. }

\par{23. 銀行へ行ってきてもらいたいんですが。 \hfill\break
I would like you to go to the bank. }

\par{24. 先生に参加していただきたいんですが。 \hfill\break
We'd like you to attend, Professor. }

\par{25. 彼らは私に出て行ってもらいたいと思っています。 \hfill\break
They would like for me to leave. }

\par{26. 私たちに何か ${\overset{\textnormal{ちゅうこく}}{\text{忠告}}}$ してほしいのですが。 \hfill\break
We would like you to give us some advice. }

\begin{center}
 \textbf{With the Negative }
\end{center}

\par{ For ほしい, you may see ~ないでほしい or ~ほしくない. The first gives an explicit request. ~てほしくない just states implicitly that you don't want something to be done or happen. Consequently, the latter sounds much softer. ~ないでほしい places emphasis on what you don't want to hear, see, etc. The first is like "I want you to not\dothyp{}\dothyp{}\dothyp{}" and the second is like “I don't want\dothyp{}\dothyp{}\dothyp{}to\dothyp{}\dothyp{}\dothyp{}”. }

\par{27. ${\overset{\textnormal{せいじか}}{\text{政治家}}}$ にならないでほしい。 \hfill\break
I want you to not become a politician. }

\par{28. 誰にもできたと知らせないでほしい。 \hfill\break
I want you to not inform anyone that I was able to do it. }

\par{29. 雨が降ってほしくない。 \hfill\break
I don't want it to rain. }

\par{30. 家族がいることを忘れてほしくないわ。(Feminine) \hfill\break
I don't want you to forget that you have a family. }

\begin{center}
\textbf{More Examples }
\end{center}

\par{31a. そんなことまでしたくはないよ。 \hfill\break
31b. そんなに ${\overset{\textnormal{}}{\text{遠}}}$ くまで ${\overset{\textnormal{}}{\text{行}}}$ きたくはないよ。(With the literal definition of "to go") \hfill\break
I don't want to go that far. }

\par{${\overset{\textnormal{}}{\text{32. 彼}}}$ らは ${\overset{\textnormal{とざん}}{\text{登山}}}$ しに ${\overset{\textnormal{}}{\text{行}}}$ ったかが ${\overset{\textnormal{}}{\text{知}}}$ りたい。 \hfill\break
I want to know whether they went mountain climbing. }

\par{33. イギリスまでこの ${\overset{\textnormal{}}{\text{手紙}}}$ を ${\overset{\textnormal{}}{\text{出}}}$ したいんです。 \hfill\break
I want to mail these letters to England. }

\par{34. もっと自由が欲しい。 \hfill\break
I want more freedom. }

\par{35. これに ${\overset{\textnormal{}}{\text{何}}}$ を ${\overset{\textnormal{}}{\text{書}}}$ きたいんですか。 \hfill\break
What do you want to write on this? }

\par{${\overset{\textnormal{}}{\text{36. 彼}}}$ に ${\overset{\textnormal{}}{\text{政治家}}}$ になってもらいたい。 \hfill\break
I want him to be a politician. }

\par{${\overset{\textnormal{}}{\text{37. 彼女}}}$ に ${\overset{\textnormal{}}{\text{来}}}$ てもらいたくない。 \hfill\break
I don't want her coming. }

\par{38. ${\overset{\textnormal{ふつうよきん}}{\text{普通預金}}}$ を ${\overset{\textnormal{}}{\text{始}}}$ めたいんですが。 \hfill\break
I'd like to start an ordinary savings account. }

\par{\textbf{Culture Note }: Money may be deposited and withdrawn anytime you like with a ${\overset{\textnormal{ふつうよきん}}{\text{普通預金}}}$ . ${\overset{\textnormal{ていきよきん}}{\text{定期預金}}}$ is a fixed deposit and a savings deposit is a ${\overset{\textnormal{}}{\text{積}}}$ み ${\overset{\textnormal{た}}{\text{立}}}$ て ${\overset{\textnormal{よきん}}{\text{預金}}}$ . }

\par{39. ${\overset{\textnormal{ほうこくしょ}}{\text{報告書}}}$ は ${\overset{\textnormal{こんど}}{\text{今度}}}$ はもうすこし ${\overset{\textnormal{}}{\text{早}}}$ く ${\overset{\textnormal{}}{\text{出}}}$ してほしいのよ。 ${\overset{\textnormal{}}{\text{聞}}}$ いてる? (Feminine) \hfill\break
I want you to turn in your report a little earlier next time, you hear? }

\par{40. 5 ${\overset{\textnormal{ごまんさんぜんえんお}}{\text{万3千円降}}}$ ろしたいんですが。 \hfill\break
I want to withdraw 53,000 yen. }

\par{41. 私たちは ${\overset{\textnormal{さんぱく}}{\text{三泊}}}$ したいです。 \hfill\break
We want to stay three nights. }

\par{42. もし私が ${\overset{\textnormal{}}{\text{金持}}}$ ちなら、 ${\overset{\textnormal{しろ}}{\text{城}}}$ を買いたい。 \hfill\break
If I were rich, I would want to buy a castle. }

\par{43. ソウルに ${\overset{\textnormal{}}{\text{行}}}$ ってみたい。 \hfill\break
I want to try to go to Seoul. }

\par{\textbf{Grammar Note }: ~てみたい should be used instead of ~たい when you are trying to say that you want to attempt to do something. }

\par{44a. クリントン前大統領と ${\overset{\textnormal{}}{\text{会}}}$ いたいです。 \hfill\break
I want to meet with Former President Clinton. \hfill\break
44b. クリントン前大統領に ${\overset{\textnormal{}}{\text{会}}}$ ってみたいです。 \hfill\break
I would like to try and meet Former President Clinton. }

\par{It wouldn't be surprising for another politician very close to Former President Clinton to say this. On the other hand, if they're on opposing sides and proximity isn't granted, ~てみたい may still be better. }
    