    
\chapter{The Particle だけ}

\begin{center}
\begin{Large}
第85課: The Particle だけ 
\end{Large}
\end{center}
 
\par{ In a nutshell, the adverbial particle だけ shows limitation meaning "just\slash only." This lesson will teach you how it is used to mean \emph{just }this. }
      
\section{The Adverbial Particle だけ}
 
\par{ だけ means "just\slash only" and shows the extent\slash limit of something.  だけ can also be after \textbf{verbs, adjectives, etc }that are in the れんたいけい. So, いやなだけ not いやだけ. This is because it's actually from the noun 丈(たけ), which means "length". }

\begin{ltabulary}{|P|P|P|P|}
\hline 

Noun + だけ & 形容詞 + だけ & 形容動詞 + だけ & Verb + だけ \\ \cline{1-4}

手 + だけ \textrightarrow  手だけ & ほしい + だけ \textrightarrow  ほしいだけ & 好き + だけ \textrightarrow  好きなだけ & 話す+ だけ \textrightarrow  話すだけ \\ \cline{1-4}

\end{ltabulary}

\par{\textbf{Particle Note }: が and を are optional \textbf{after }だけ. Don\textquotesingle t put them before it. }

\par{1. テストのことを思うだけで ${\overset{\textnormal{ふあん}}{\text{不安}}}$ になった。 \hfill\break
I became uneasy just at the thought of the test. }
 
\par{2. 赤いリンゴだけ五つください。 \hfill\break
Please give me five red apples only. }
 
\par{3. 君だけを愛している。 \hfill\break
I only love you. }

\par{4. 私は ${\overset{\textnormal{えいご}}{\text{英語}}}$ だけ(を)勉強します。 \hfill\break
I only study English. }
 
\par{5. 人は ${\overset{\textnormal{がいけん}}{\text{外見}}}$ だけでは分かりません。 \hfill\break
You can't tell a person just by looks. }

\par{6. ${\overset{\textnormal{こんかい}}{\text{今回}}}$ だけは ${\overset{\textnormal{みのが}}{\text{見逃}}}$ してください。 \hfill\break
Please overlook just this one time. }

\par{7. ${\overset{\textnormal{ひとり}}{\text{一人}}}$ だけ\{です・います\}。 \hfill\break
There is only 1 person. }

\par{8. ほしいだけ ${\overset{\textnormal{も}}{\text{持}}}$ つ。 \hfill\break
To carry just what one wants. }

\par{9. 彼は、この前の ${\overset{\textnormal{しゅうまつ}}{\text{週末}}}$ に ${\overset{\textnormal{せんたく}}{\text{洗濯}}}$ だけしたのよ。(Feminine) \hfill\break
He only did the laundry last weekend! }
 
\par{10. 彼女は、この前の週末に ${\overset{\textnormal{そうじ}}{\text{掃除}}}$ だけしたんだよ。(Masculine) \hfill\break
She only did cleaning last weekend! }

\begin{center}
\textbf{だけで(は)なく }
\end{center}

\par{ だけで\{(は)・じゃ\}なく means "not only". }

\par{11. ${\overset{\textnormal{ねこ}}{\text{猫}}}$ がいるだけでなく、 ${\overset{\textnormal{}}{\text{犬}}}$ もいます。 \hfill\break
Not only are there cats, but there are also dogs. }

\par{12a. あの ${\overset{\textnormal{へや}}{\text{部屋}}}$ は広いだけでなく、とても明るくて美しいです。(ちょっと不自然) \hfill\break
12b. あの部屋は広くて明るいです。(もっと自然) \hfill\break
That room is not only spacious, but it's also very bright and beautiful. }

\begin{center}
\textbf{だけ(のこと)だ }
\end{center}
 
\par{ だけ(のこと)だ states that there is nothing more than something. So, it can be translated as "\dothyp{}\dothyp{}\dothyp{}is no more than". With the inclusion of のこと, the statement is more forceful. }

\par{13. ${\overset{\textnormal{かぜ}}{\text{風邪}}}$ を ${\overset{\textnormal{ひ}}{\text{引}}}$ いただけだよ。 \hfill\break
I only caught a cold. }

\par{14. ${\overset{\textnormal{かいしゃ}}{\text{会社}}}$ が ${\overset{\textnormal{はさん}}{\text{破産}}}$ しただけのことだ。 \hfill\break
It's just that the company went bankrupt. }

\par{15. 見ているだけです。 \hfill\break
I'm just looking. }

\par{16. ${\overset{\textnormal{ひざ}}{\text{膝}}}$ を ${\overset{\textnormal{す}}{\text{擦}}}$ り ${\overset{\textnormal{む}}{\text{剥}}}$ いただけ。 \hfill\break
I only grazed my knee. }

\par{\textbf{Speech Level Note }: The deletion of だ makes だけ a final particle. It makes the sentence \textbf{less blunt }. だけです is perfectly fine for polite speech. }

\begin{center}
 \textbf{With こそあど }
\end{center}

\par{With これ, それ, あれ, and どれ, だけ translates as "much". }

\par{18. ${\overset{\textnormal{べんり}}{\text{便利}}}$ なサイトは \textbf{これだけ }です。 \hfill\break
 \textbf{These }are the \textbf{only }useful sites. }

\par{19. \textbf{これだけ }の ${\overset{\textnormal{きっぷ}}{\text{切符}}}$ を ${\overset{\textnormal{あつ}}{\text{集}}}$ めていた。 \hfill\break
I've collected \textbf{this many }tickets. }

\par{20. それだけなの?(A little feminine) \hfill\break
Is that all? }

\par{21. その ${\overset{\textnormal{にゅうもんしょ}}{\text{入門書}}}$ は本当に高いが、それだけの ${\overset{\textnormal{かち}}{\text{価値}}}$ がある。 \hfill\break
The beginner's book is really expensive, but it has the worth. }

\par{22. どれだけ ${\overset{\textnormal{くる}}{\text{苦}}}$ しくても僕は ${\overset{\textnormal{はな}}{\text{離}}}$ しはしない。 \hfill\break
No matter how hard it is, I won't let go. }

\begin{center}
 \textbf{Verb+だけ+Verb }
\end{center}

\par{  だけ shows a limit, and in this expression you make it even more clear that you are not expecting, wanting, or doing any more. }
 
\par{23. このことは両親にも話すだけは話しておいた方がいい。 \hfill\break
As for this, it's best that you at the most talk to your parents. }
 
\par{24. やるだけはやったんだから、静かに結果を待ちましょう。 \hfill\break
I'll wait quietly for the results since I did what I had to do. }
 
\par{25. 言うだけ言ったらすっきりした。 \hfill\break
I feel good now that I've said (what I had to say). }
 
\par{26. まあ聞くだけ聞いてくれ。 \hfill\break
Come on, at least listen to what I have to say. }

\begin{center}
----------------------------------------------------------------------------------- 
\end{center}

\begin{center}
 \textbf{Potential + だけ }
\end{center}

\par{  With the potential form of the verb, it can be translated in this way as "as much as\dothyp{}\dothyp{}\dothyp{}". }
 
\par{27. できるだけして。 \hfill\break
Do as much as possible. }

\par{${\overset{\textnormal{}}{\text{28. 私}}}$ は ${\overset{\textnormal{}}{\text{走}}}$ れるだけ ${\overset{\textnormal{}}{\text{走}}}$ りました。 \hfill\break
I ran as much as I could. }

\par{\textbf{なるべく VS }\textbf{できるだけ }}
 
\par{Both mean "if possible\slash as (much) as possible". なるべく is (somewhat) formal. できるだけ can be followed by の. To use it with なるべく , you have to use the rare なるたけ\slash なるだけ. }
 
\par{29. なるべくタバコをすうな。 \hfill\break
As much as possible\slash if possible, do not smoke. }
 
\par{\textbf{Grammar Note }: な in the above sentence creates the negative imperative. }
 
\par{30. なるべく ${\overset{\textnormal{ごご}}{\text{午後}}}$ に ${\overset{\textnormal{}}{\text{来}}}$ てほしいです。 \hfill\break
I would like you to come during the afternoon if possible. }
 
\par{31. できるだけ ${\overset{\textnormal{}}{\text{多}}}$ くの ${\overset{\textnormal{}}{\text{金}}}$ が ${\overset{\textnormal{ひつよう}}{\text{必要}}}$ ! \hfill\break
I need as much money as possible! }

\par{32. できるだけ ${\overset{\textnormal{}}{\text{早}}}$ く ${\overset{\textnormal{}}{\text{帰}}}$ ってください。 \hfill\break
Please come home as soon as possible. }
      
\section{ただ}
 
\par{ As an adverb, ただ means "just what you're doing". For example, "you're just hoping that you get a good grade in Japanese class". Other synonyms include ${\overset{\textnormal{もっぱ}}{\text{専}}}$ ら and ひたすら. Or, it may stress that there's "nothing else" or that there is "merely" something. Other words for this include たった and わずかに. }

\par{  ただ\dothyp{}\dothyp{}\dothyp{}だけ means "mere(ly)", but だけ isn\textquotesingle t necessary. ${\overset{\textnormal{}}{\text{専}}}$ ら ≒ "entirely", so there is no need for だけ.  Neither is there a need for it with ひたすら. ほんの ${\overset{\textnormal{}}{\text{僅}}}$ か\dothyp{}\dothyp{}\dothyp{}だけ = "just a few". たった comes from ただ to mean "mere" and is frequently used with だけ. }

\begin{center}
\textbf{Examples } 
\end{center}
 
\par{33. たった ${\overset{\textnormal{}}{\text{一}}}$ つ \hfill\break
Just one }
 
\par{34. たった ${\overset{\textnormal{}}{\text{今}}}$ ${\overset{\textnormal{}}{\text{消}}}$ えたんだよ。 \hfill\break
It disappeared just now! }
 
\par{35. ただいま \hfill\break
I'm home! \hfill\break
 \hfill\break
 \textbf{Culture Note }: Whenever you come home, you say ただいま. People there will response with お帰り(なさい) "welcome home". The added part makes it polite. }
 
\par{37. もっぱらの ${\overset{\textnormal{うわさ}}{\text{噂}}}$ だよ。 \hfill\break
It's widely rumored. }
 
\par{38. ただの ${\overset{\textnormal{りゆう}}{\text{理由}}}$ で \hfill\break
With just a mere excuse }
 
\par{39. それは ${\overset{\textnormal{たん}}{\text{単}}}$ に ${\overset{\textnormal{ていど}}{\text{程度}}}$ の ${\overset{\textnormal{もんだい}}{\text{問題}}}$ じゃないか? \hfill\break
Isn't that just a problem of degree? }
 
\par{\textbf{Word Note }: ${\overset{\textnormal{}}{\text{単}}}$ に means "purely\slash simply\slash merely" and is also often used with だけ. }
 
\par{${\overset{\textnormal{}}{\text{40. 単}}}$ に ${\overset{\textnormal{しけん}}{\text{私見}}}$ を ${\overset{\textnormal{の}}{\text{述}}}$ べただけだよ。 \hfill\break
I merely stated my own opinion. }
 
\par{\textbf{Nuance Note }: Simply as in "easily" can be ${\overset{\textnormal{たんじゅん}}{\text{単純}}}$ に, ${\overset{\textnormal{}}{\text{簡単}}}$ に, ${\overset{\textnormal{}}{\text{単}}}$ に, etc. with just slight differences. The first shows a heavy emphasis on simplifying a process down. The second shows more so the easy, and the latter emphasizes on the extent of the matter. }
 
\par{${\overset{\textnormal{}}{\text{41. 東京}}}$ までたった10キロです。 \hfill\break
It's just 10 kilometers to Tokyo. }
 
\par{42. あられがほんのわずか ${\overset{\textnormal{ふ}}{\text{降}}}$ っただけだ。(Somewhat written style) \hfill\break
There was only a light amount of hail that fell. }
 
\par{43. ほんの ${\overset{\textnormal{}}{\text{一度}}}$ だけ \hfill\break
Just one time }
 
\par{\textbf{Word Note }: ほんの is an attributive phrase that means "mere". ほんの ${\overset{\textnormal{}}{\text{少}}}$ し means "just a little". The word gives the sense that there really is nothing else beyond it. It can also be written as ${\overset{\textnormal{}}{\text{本}}}$ の. }
 
\par{44. ただの ${\overset{\textnormal{かぜ}}{\text{風邪}}}$ でしょう。 \hfill\break
It's probably just a cold. }
 
\par{\textbf{Word Note }: ${\overset{\textnormal{}}{\text{風邪}}}$ means "cold" but comes from and is pronounced just like ${\overset{\textnormal{}}{\text{風}}}$ (wind). }
 
\par{45 [徒・只] \textbf{より }高いものはない。 \hfill\break
Nothing costs more \textbf{than }what is given to us. }

\par{${\overset{\textnormal{}}{\text{46. 道}}}$ はただひとつ。 \hfill\break
There is but one way to live. }

\par{47. ${\overset{\textnormal{び}}{\text{美}}}$ はただ ${\overset{\textnormal{かわいちまい}}{\text{皮一枚}}}$ 。 \hfill\break
Beauty is but skin-deep. }

\par{ ただでは in a negative sentence means that something serious is to happen. }
 
\par{48. ただではおかないぞ! \hfill\break
You'll pay for this! }
    