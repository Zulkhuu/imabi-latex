    
\chapter{のだ, から, \& ので}

\begin{center}
\begin{Large}
第57課: のだ, から, \& ので 
\end{Large}
\end{center}
 
\par{ Showing reason is a very complex matter in Japanese, but these three items, two of which are related to each other, will give you a basic understanding of how to do this. }
      
\section{のだ}
 
\par{ のだ \textbf{decisively shows reason }. It may strongly emphasize an \emph{exclamation, reasoning, cause, grounds, or desire }. It is used when, for instance, you are responding to having been asked for a reason or explanation. のだ allows the speaker to not become detached in conversation . However, it can't be used to just state a fact. It's going to show reasoning of some sort. }

\par{ You can see it as のだ (more literary), んだ (spoken), のである (literary), and のでございます (honorific form).  It must follow the ${\overset{\textnormal{れんたいけい}}{\text{連体形}}}$ of an adjective or verb. Therefore, with nouns you'll use なのだ, which uses the copula. }

\begin{ltabulary}{|P|P|P|P|}
\hline 

Noun & けいようし & けいようどうし & Verb \\ \cline{1-4}

犬 \textbf{な }のだ & 新しいのだ & だめ \textbf{な }のだ & 見るのだ \\ \cline{1-4}

\end{ltabulary}

\par{ のか・[のですか・んですか] is the question form, which inquires \textbf{why }and is perfect for asking for a reason or explanation. In certain situations it may not be appropriate because of tone. の alone can also make a question in casual speech. It shows reason with a falling intonation in feminine speech. As expected, のだ may be used to create a blunt question. }

\begin{center}
 \textbf{Tenses }
\end{center}

\par{ The following chart lists the following \emph{potential }conjugations involving tense and or negation for ~のだ. The chart below shows these forms for when ~のだ is used with a nominal phrase. For anything else, simply take away any existing な. As will be discussed in more detail, this chart does not imply each form below is equally used and natural in any circumstance. It is important to not go and run with this but be cautious as to what is frequently used and when. }

\begin{ltabulary}{|P|P|P|}
\hline 

 & Plain & Polite \\ \cline{1-3}

 \textbf{Non-past }& ~なのだ & ~なのです \\ \cline{1-3}

 \emph{Past }& ~なのだった & ~なのでした \\ \cline{1-3}

 \textbf{Past }& ~だったのだ & ~だったのです \\ \cline{1-3}

 \textbf{\emph{Negative }}& ~なのじゃない & ~なのじゃないです \\ \cline{1-3}

 \textbf{Negative }& ~じゃないのだ & ~じゃないのです \\ \cline{1-3}

 \emph{Negative Past }& ~なのじゃなかった & ~なのじゃなかったです \\ \cline{1-3}

 \textbf{Negative Past }& ~じゃなかったのだ & ~じゃなかったのです \\ \cline{1-3}

\end{ltabulary}

\par{ This chart is for the sake of having all of these basic forms represented. It does not imply that all of these phrases above could even work with any given noun such as 犬. All of these forms are grammatically correct, though some of them would need a lot of context to be natural, and the language for some of them may have to be more old-fashioned to work. }

\par{ Patterns italicized are rarer and literary. If bold and italicized, it means that it is rare\slash older as shown, but it may be frequently used with ~ん instead of ~の. ~のだった is like a rather forceful and also explanatory "it so happened that". }

\par{ ~のじゃない, just like ~じゃない, is often used to ask for verification, but in this sense you are wanting explanation. Even so, you'd be hard pressed to hear ~のじゃない? Instead, you would hear ~んじゃない? If you wanted the の, ~のではないか would be your next best option. However, this form is more indicative of 書き言葉 or very serious utterances. You should separate this from the のじゃない that comes about from using の as a dummy noun and then following it with the copula. }

\par{ ~のじゃなかった is like "it wasn't that". Again, in actually speaking this form as well is usually んじゃなかった. This is the case for all of these forms. }

\par{ ん is 話し言葉的 and の is 書き言葉的. This statement is for this grammatical circumstance, but it is also true that ん is more colloquial in nature where ever it shows up in Japanese. This statement does not also deny the potential of の being used in these forms at all in the spoken language. As you will see, ~のです and ~のですか do get used in careful politeness speech, but ~んです and ~んですか are certainly more prevalent. }

\begin{center}
 \textbf{Examples }
\end{center}

\par{1. 負けなかったんだ。 \hfill\break
I didn't lose. }

\par{2. もうすこし安いのはないんですか。 \hfill\break
Do you not have a cheaper one? }

\par{3. 行かないんじゃない? \hfill\break
You\textquotesingle re not going, right? }

\par{4. 行くんじゃない? \hfill\break
Aren't you going? }

\par{5. あの水は飲むんじゃなかった。 \hfill\break
That water was not to drink. }

\par{6. この ${\overset{\textnormal{つぎ}}{\text{次}}}$ の ${\overset{\textnormal{どようび}}{\text{土曜日}}}$ に来るんですか。 \hfill\break
Will you be coming next Saturday? }

\par{7. ${\overset{\textnormal{かぜ}}{\text{風邪}}}$ を ${\overset{\textnormal{ひ}}{\text{引}}}$ いたんだ。 \hfill\break
I caught a cold. (Reason) }

\par{8. コーヒーは ${\overset{\textnormal{えんりょ}}{\text{遠慮}}}$ して おきます。もうすぐ ${\overset{\textnormal{ね}}{\text{寝}}}$ るんです。 \hfill\break
I won't drink coffee now. I am going to bed soon. }

\par{9. すみません、 ${\overset{\textnormal{の}}{\text{乗}}}$ り ${\overset{\textnormal{おく}}{\text{遅}}}$ れたのですが。 \hfill\break
Sorry, I missed my ride (vehicle). }

\par{10. ${\overset{\textnormal{つゆ}}{\text{梅雨}}}$ はいつ終わるんですか。 \hfill\break
When is the rainy season over? }

\par{11. 「どうして納豆がすきなんだ?」「おいしいんだ」 \hfill\break
"Why do you like natto?" "It's delicious." }

\par{12. 友だちが待っているんです。 \hfill\break
My friend is waiting. }

\par{\textbf{Particle Note }: The particle ので is actually just the て form of this. It shows specific reasoning and is translated as "because\slash since." }
      
\section{から}
 
\par{ The most important usage of the conjunctive particle から is to connect clauses to mean "because". It shows reason and may \emph{solicit a response, suggest something, or any means that appeals to the listener }. If the sentence is polite, the clause it modifies should be too. }

\par{から doesn't have to be used after a dependent clause to show reason. It may also be at the end of a sentence. When at the end of the sentence and followed by だ, you can make a strong assertion. Without a copula, though, you show reason with more emotional appeal. Regardless, if から is used with nouns or 形容動詞, it \textbf{must }be preceded by the copula! }

\par{ から goes after the 終止形. So, you should use it after verbs, adjectives, or a copula phrase. }

\begin{ltabulary}{|P|P|P|P|P|P|P|P|P|}
\hline 

だ &  \textbf{だ }から & で \textbf{す }から & 形容詞 & い \textbf{い }から & 形容動詞 & 静か \textbf{だ }から & 動詞 & す \textbf{る }から \\ \cline{1-9}

\end{ltabulary}

\par{ In Standard Japanese, you cannot say なから. However, it does exist in certain dialects. For the purposes of Tokyo speech, it is deemed incorrect. }

\begin{center}
 \textbf{Examples }
\end{center}

\par{ For some of these examples, a question would be stated before the response in order to sound perfectly natural. }

\par{13. ${\overset{\textnormal{じかん}}{\text{時間}}}$ があるから、 ${\overset{\textnormal{きゅうけい}}{\text{休憩}}}$ をしない? \hfill\break
Since we have time, how about taking a break? }

\par{14. ${\overset{\textnormal{きかい}}{\text{機械}}}$ が ${\overset{\textnormal{こわ}}{\text{壊}}}$ れたからです。 \hfill\break
It's because the machine broke. }

\par{15. 明日、 ${\overset{\textnormal{しけん}}{\text{試験}}}$ があるからです。 \hfill\break
It's because I have an exam tomorrow. }

\par{16. ${\overset{\textnormal{かんたん}}{\text{簡単}}}$ だからだ。 \hfill\break
It's because it's easy. }

\par{17. あいつが馬鹿だからだな。   (Masculine) \hfill\break
It's because he's an idiot, you know. }

\begin{center}
 \textbf{のだから }
\end{center}

\par{ から can mean "because" and のだ strongly shows reason. Thus, んだから has it that the listener knows the "why". The speaker, then, \textbf{reasserts }. If the listener doesn't know the "why", then it's extremely pushy. A clause following のだから often expresses the speaker's judgment, intent, wish, etc. }

\par{18. やべーから、早くしろ!時間(が)ねーんだよ。すでに遅れてんだから、早よ準備しろ!(Casual; vulgar) \hfill\break
We're screwed, so hurry up! We don't have time! We're already late, so get ready already! }

\par{ ~たら makes a hypothetical. ねーんだよ means the same thing as ないんだよ. It is just more vulgar. ~てんだ is a contraction of ~ているんだ. }
      
\section{ので}
 
\par{ ので, んで in casual speech, is used to show reason or cause. Unlike から, it must only be used in this way. So, you should not use it with an imperative of any sort. Also unlike から, ので is used to show reason of present circumstance where there is no control in the matter on your part. It is objective in nature, and so it's also weaker. Appropriate translations include "since", and "because". }

\par{ Unlike から, ので follows the 連体形 of adjectives, verbs, and copula phrases. }

\begin{ltabulary}{|P|P|P|P|P|P|P|P|P|P|}
\hline 

だ &  \textbf{な }ので & です &  \textbf{です }ので* & 形容詞 & よ \textbf{い }ので & 形容動詞 & 静か \textbf{な }ので & 動詞 & す \textbf{る }ので \\ \cline{1-10}

\end{ltabulary}

\par{\textbf{Grammar Notes }: \hfill\break
*1: ですので is technically incorrect Japanese, but it is has become frequently used to the point that is now correct Japanese. \hfill\break
*2: Although improper i n Standard Japanese, だので is actually seen, but it's dialectical. }

\par{ This rule, though, is relatively new in Japanese, and it really has no bearing over ので. In fact, you can also find ~ますので. This, however, will be considered ungrammatical by more speakers than ですので. Why this is the case is not clear, but it probably has to deal with part of speech and the formal nature of ので. If someone doesn't like ~ますので, the person is probably from West Japan. }

\begin{center}
 \textbf{Examples }
\end{center}

\par{19. 今日は暖かいので、桜も満開になるでしょう。 \hfill\break
Since it is warm today, the cherry blossoms will likely be in full bloom. }

\par{20. 宿題をし終わったので、外へ遊びに行った。 \hfill\break
Since I finished my homework, I went to play outside. }

\par{21. その料理は思ったよりおいしくなかったので、 ${\overset{\textnormal{ひとくち}}{\text{一口}}}$ しか食べませんでした。 \hfill\break
Since that dish wasn't as a good as I thought it would be, I didn't eat more than a bite. }

\par{21. 彼が ${\overset{\textnormal{びじゅつかん}}{\text{美術館}}}$ に行きたいというので、 ${\overset{\textnormal{あんない}}{\text{案内}}}$ してあげた。 \hfill\break
Because he said he wanted to go to an art museum, I took him around. }

\par{22. あまりに寒いので、ストーブをつけた。 \hfill\break
Since it was so cold, I turned on the heater. }

\par{23. 日本語がうまく(しゃべれるように)なりたいので、 ${\overset{\textnormal{いっしょうけんめいべんきょう}}{\text{一生懸命勉強}}}$ します。 \hfill\break
As I want to become good at Japanese, I'll study with all my might. }

\par{24.「 ${\overset{\textnormal{つつ}}{\text{包}}}$ みは ${\overset{\textnormal{こうくうびん}}{\text{航空便}}}$ でいくらですか」「600グラムですので、1500円になります」 \hfill\break
“How much will it cost (to send) for the package?” “Since it is 600 grams, it will be 1500 yen”. }

\par{ When at the end of a sentence, ので is essentially a softer version of のだ. }
25. すまない、もう ${\overset{\textnormal{}}{\text{先約が}}}$ ${\overset{\textnormal{}}{\text{}}}$ あるんで。 \hfill\break
Sorry, I've got a commitment. 
\par{26. 終電に乗りたいので。 \hfill\break
I need to catch the last train. }

\par{\textbf{Sentence Note }: Ex. 26 is in response to a question. }
「 ${\overset{\textnormal{つつ}}{\text{包}}}$ みは ${\overset{\textnormal{こうくうびん}}{\text{航空便}}}$ でいくらですか」「600グラムですので、1500円になります」 \hfill\break
“How much will it cost (to send) for the package?” “Since it is 600 grams, it will be 1500 yen”. \hfill\break
    