    
\chapter{Adnominal Adjectives}

\begin{center}
\begin{Large}
第60課: Adnominal Adjectives: \slash i\slash  vs \slash na\slash  
\end{Large}
\end{center}
 
\par{ The first thing that you might be wondering what an adnominal adjective is. The word "adnominal" simply means \textbf{something that modifies a noun }. In which case, all adjectives are adnominal. However, \emph{adnominal adjectives }in the realm of Japanese grammar refer to \textbf{adjectival phrases that don\textquotesingle t conjugate and solely modify the noun they attach to }and are not interpreted as predicates modifying a noun. First, let\textquotesingle s see quickly what\textquotesingle s meant by this. }

\par{i. ${\overset{\textnormal{やま}}{\text{山}}}$ の ${\overset{\textnormal{ちい}}{\text{小}}}$ さい ${\overset{\textnormal{とり}}{\text{鳥}}}$ たちを ${\overset{\textnormal{かんしょう}}{\text{観賞}}}$ しています。 \hfill\break
 \emph{Yama no chiisai toritachi wo kansh }\emph{ō shite imasu. }\hfill\break
I\textquotesingle m admiring the small birds of the mountain. }

\par{ii. あの ${\overset{\textnormal{やま}}{\text{山}}}$ の ${\overset{\textnormal{ちい}}{\text{小}}}$ さな ${\overset{\textnormal{まち}}{\text{町}}}$ に ${\overset{\textnormal{す}}{\text{住}}}$ んでいます。 \hfill\break
 \emph{Ano yama no chiisana machi ni sunde imasu. }\hfill\break
I live in the small town on that mountain. }

\par{iii. ${\overset{\textnormal{にほんじん}}{\text{日本人}}}$ は ${\overset{\textnormal{ほんとう}}{\text{本当}}}$ に ${\overset{\textnormal{たいかく}}{\text{体格}}}$ \{が・の\} ${\overset{\textnormal{ちい}}{\text{小}}}$ さい ${\overset{\textnormal{じんしゅ}}{\text{人種}}}$ なんですか。 \hfill\break
 \emph{Nihonjin wa hont }\emph{ō ni taikaku [ga\slash no] chiisai jinshu na n desu ka? }\hfill\break
Are Japanese really a race whose build is small? }

\par{\textbf{Particle Note }: Either \emph{ga }が or \emph{no }の can be used to mark what would be the subject if the dependent clause were being used as an independent clause. }

\par{ As you can see, \emph{chiisai }小さい may be part of a predicate modifying a noun as in ii., but doesn\textquotesingle t always have to as is the case in i. Phrases that utilize \emph{no }の are actually adnominal minus how the particle is used in iii.  As for \emph{chiisana }小さな, it is grammatically akin to the particle \emph{no }の as it too only modifies whatever noun follows it. Unlike \emph{no }の, though, it must be right next to the noun and can\textquotesingle t float away like it does in i. and ii. with \emph{yama no }山の. }

\par{iv. \{ ${\overset{\textnormal{ちい}}{\text{小}}}$ さい・ ${\overset{\textnormal{ちい}}{\text{小}}}$ さな\} ${\overset{\textnormal{とき}}{\text{時}}}$ から ${\overset{\textnormal{だいす}}{\text{大好}}}$ きでした。 \hfill\break
 \emph{[Chiisai\slash chiisana] toki kara daisuki deshita. }\hfill\break
I loved it since I was little. }

\par{ As you can see, \emph{chiisana }小さな shows that adnominal adjectives can in fact at times function as predicates modifying a noun. Interestingly enough, \emph{chiisai toki }小さい時 is only ever interpreted as “when one is\slash was little” and it is \emph{chiisana }小さな that can be interpreted as that or “small time” depending on context. It\textquotesingle s just that \emph{chiisana }小さな is used a lot more in front of nouns. }

\par{\textbf{Terminology Note }: Adnominal adjectives are called \emph{rentaishi }連体詞 in Japanese. Remember that adnominal adjectives simply means adjectives that don't conjugate. The reason why they're so important to point out specifically is to know how they differ in nuance and usage with other adjectives. }

\par{\textbf{Curriculum Note }: This will be the first installment of coverage on these adjectival phrases. This lesson, however, will only be those that end in \slash na\slash . Other kinds of adnominal nouns will be discussed in later lessons. }
      
\section{\slash i\slash  \& \slash na\slash  Pairs}
 
\par{ The real questions that you probably have now is how does the adnominal adjective \emph{chiisana }小さな differ with the regular \emph{chiisai }小さい, and how many other of these pairs of phrases exist. In the chart below, you will see that the third column details whether the form can be used as a regular adjectival noun by including \emph{da }だ, which is of course used when the adjectival noun is the predicate. }

\begin{ltabulary}{|P|P|P|}
\hline 

Meaning & Ending in \slash i\slash  & Ending in \slash na\slash  \\ \cline{1-3}

Warm & 暖かい \hfill\break
\emph{Atatakai }& 暖か\{な・だ\} \hfill\break
\emph{Atataka [na\slash da] }\\ \cline{1-3}

Square & 四角い \hfill\break
\emph{Shikakui }& 四角\{な・だ\} \hfill\break
\emph{Shikaku [na\slash da] }\\ \cline{1-3}

Soft & 柔らかい \hfill\break
\emph{Yawarakai }& 柔らか\{な・だ\} \hfill\break
\emph{Yawaraka [na\slash da] }\\ \cline{1-3}

Fine\slash minute & 細かい \hfill\break
\emph{Komakai }& 細か\{な・だ\} \hfill\break
\emph{Komaka [na\slash da] }\\ \cline{1-3}

Small & 小さい \hfill\break
\emph{Chiisai } & 小さな \hfill\break
\emph{Chiisana }\\ \cline{1-3}

Large & 大きい \hfill\break
\emph{Ōkii }& 大きな \hfill\break
\emph{Ōkina }\\ \cline{1-3}

Strange & 可笑しい \hfill\break
\emph{Okashii }& 可笑しな \hfill\break
\emph{Okashina }\\ \cline{1-3}

New & 新しい \hfill\break
\emph{Atarashii }& 新たな \hfill\break
\emph{Aratana }\\ \cline{1-3}

\end{ltabulary}

\par{ All of the \slash na\slash  forms in the right column are examples of adnominal adjectives. Even if any of these words do have other conjugations, those conjugations would be incredibly limited in scope and would need to be studied separately. }

\par{1. か ${\overset{\textnormal{ぼそ}}{\text{細}}}$ い ${\overset{\textnormal{こえ}}{\text{声}}}$ はだんだん ${\overset{\textnormal{おお}}{\text{大}}}$ きくなっている。(Use of 大きい) \hfill\break
 \emph{Kabosoi koe wa dandan }\emph{ōkiku natte iru. }\hfill\break
The fragile voice is becoming gradually larger. }

\par{2. ${\overset{\textnormal{おどろ}}{\text{驚}}}$ くほど ${\overset{\textnormal{ちい}}{\text{小}}}$ さな ${\overset{\textnormal{すあし}}{\text{素足}}}$ が ${\overset{\textnormal{あらわ}}{\text{露}}}$ になった。 (Use of 小さな) \hfill\break
 \emph{Odoroku hodo chiisana suashi ga arawa ni natta. \hfill\break
 }Surprisingly small bare feet became exposed. }

\par{\textbf{Grammar Note }: This is an example of 小さな being used as a predicate; however, it is still juxtaposed right before the noun it modifies. Only the degree of the sense of "small" is being modified, and that is via the adverbial phrase 驚くほど. Otherwise, adnominal adjectives do not have the same free range as other adjectives to be used in complex predicate phrases that modify nouns. }

\par{3. ${\overset{\textnormal{ちい}}{\text{小}}}$ さな ${\overset{\textnormal{こえ}}{\text{声}}}$ で ${\overset{\textnormal{はな}}{\text{話}}}$ しかける。(Use of 小さな) \hfill\break
 \emph{Chiisana koe de hanashikakeru. \hfill\break
 }To talk to someone in a small voice. }

\par{4. ${\overset{\textnormal{やわ}}{\text{柔}}}$ らかかった。 ${\overset{\textnormal{あたた}}{\text{暖}}}$ かかった。(Use of 柔らかい \& 暖かい) \hfill\break
 \emph{Yawarakakakatta. Atatakakatta. }\hfill\break
It was soft. It was warm. }

\par{5. ${\overset{\textnormal{かわ}}{\text{皮}}}$ を ${\overset{\textnormal{む}}{\text{剥}}}$ いて ${\overset{\textnormal{こま}}{\text{細}}}$ かく ${\overset{\textnormal{き}}{\text{切}}}$ りました。(Use of 細かい) \hfill\break
 \emph{Kawa wo muite komakaku kirimashita. }\hfill\break
I peeled the skin and finely cut it up. }

\par{6. ${\overset{\textnormal{おお}}{\text{大}}}$ きい ${\overset{\textnormal{きぼう}}{\text{希望}}}$ の ${\overset{\textnormal{くも}}{\text{雲}}}$ が ${\overset{\textnormal{わ}}{\text{湧}}}$ いている。(Use of 大きい) \hfill\break
 \emph{Ōkii kib }\emph{ō no kumo ga waite iru. }\hfill\break
Large clouds of hope are gushing forth. }

\par{7. ${\overset{\textnormal{おお}}{\text{大}}}$ きな ${\overset{\textnormal{きぼう}}{\text{希望}}}$ の ${\overset{\textnormal{くも}}{\text{雲}}}$ が ${\overset{\textnormal{わ}}{\text{湧}}}$ いている。(Use of 大きな) \hfill\break
 \emph{Ōkina kib }\emph{ō no kumo ga waite iru. }\hfill\break
Clouds of great hope are gushing forth. }
      
\section{Nuance Restrictions}
 
\par{ When \slash i\slash  and \slash na\slash  pairs exist, the ones that end in \slash i\slash  are naturally objective, typically used with concrete nouns, but are not limited to literal interpretations, and the ones that end in \slash na\slash  are naturally subjective, typically used with abstract nouns, and almost always limited to literal yet emotional interpretations. }

\par{ However, this is not all you have to consider. Some forms do have nuances the others don\textquotesingle t. You also can\textquotesingle t just choose which form you want in a set phrase. After all, set phrases are set for a reason. With that being said, we will now focus on what exactly these nuance restrictions are. }

\begin{center}
\textbf{\emph{Chiisai\slash Chiisana }小さい・小さな \emph{Ōkii\slash  }\emph{Ōkina }大きい・大きな }
\end{center}

\par{ Generally speaking, \emph{chiisana }小さな and \emph{ōkina }大きな are only used to indicate physical size but with a subjective twist. Only \emph{chiisai }小さい and \emph{ōkii }大きい may be used to indicate small\slash large monetary values, and they may even be used to mean “old” and “young” in the context of age among siblings. }

\par{8. ${\overset{\textnormal{よんひゃく}}{\text{400}}}$ ${\overset{\textnormal{えん}}{\text{円}}}$ ですか。すみません、\{ ${\overset{\textnormal{おお}}{\text{大}}}$ きい ○・ ${\overset{\textnormal{おお}}{\text{大}}}$ きな X\}のしかないんです。 \hfill\break
\emph{Yonhyakuen desu ka? Sumimasen, [ }\emph{ōkii ○\slash  }\emph{ōkina X] no shika nai n desu. }\hfill\break
It\textquotesingle s 400 yen? I\textquotesingle m sorry, but I only have large ones (bills). }

\par{9. ${\overset{\textnormal{せかいいちちい}}{\text{世界一小}}}$ さい ${\overset{\textnormal{しへい}}{\text{紙幣}}}$ は ${\overset{\textnormal{なん}}{\text{何}}}$ ですか。 \hfill\break
\emph{Sekai\textquotesingle ichi chiisai shihei wa nan desu ka? }\hfill\break
What is the world\textquotesingle s smallest paper bill? }

\par{10. ${\overset{\textnormal{おお}}{\text{大}}}$ きくなったね。 \hfill\break
\emph{Ōkiku natta ne. }\hfill\break
My, you\textquotesingle ve grown. }

\par{11. ${\overset{\textnormal{ちい}}{\text{小}}}$ さい ${\overset{\textnormal{ころ}}{\text{頃}}}$ からずっとそう ${\overset{\textnormal{おも}}{\text{思}}}$ っていました。 \hfill\break
\emph{Chiisai kara koro zutto s }\emph{ō omotte imashita. }\hfill\break
I\textquotesingle ve always thought so since I was little. }

\par{12. ${\overset{\textnormal{ちゅうおう}}{\text{中央}}}$ に ${\overset{\textnormal{ちい}}{\text{小}}}$ さなテーブルがありました。 \hfill\break
\emph{Ch }\emph{ūō ni chiisana t }\emph{ēburu ga arimashita. }\hfill\break
There was a small table in the center. }

\par{13. ${\overset{\textnormal{せかい}}{\text{世界}}}$ で ${\overset{\textnormal{いちばんおお}}{\text{一番大}}}$ きい ${\overset{\textnormal{たてもの}}{\text{建物}}}$ は ${\overset{\textnormal{なん}}{\text{何}}}$ ですか。 \hfill\break
\emph{Sekai de ichiban }\emph{ōkii tatemono wa nan desu ka? }\hfill\break
What is the largest building in the world? }

\par{14. ${\overset{\textnormal{おお}}{\text{大}}}$ きな ${\overset{\textnormal{せいふ}}{\text{政府}}}$ を ${\overset{\textnormal{もと}}{\text{求}}}$ める ${\overset{\textnormal{ひと}}{\text{人}}}$ たちが ${\overset{\textnormal{かなら}}{\text{必}}}$ ずどこの ${\overset{\textnormal{くに}}{\text{国}}}$ にもいる。 \hfill\break
\emph{Ōkina seifu wo motomeru hitotachi ga kanarazu doko no kuni ni mo iru. }\hfill\break
There are always people who seek big government in any country. }

\begin{center}
\textbf{\emph{Okashii }おかしい vs \emph{Okashina }おかしな }
\end{center}

\par{\emph{ Okashii }おかしい is generally used in positive connotations in the sense of “funny” whereas \emph{okashina }おかしな is generally used in negative connotations in the sense of “weird\slash suspicious\slash odd.” This, of course, is only a rule of thumb, but it generally holds true. However, it is important to note that this distinction really has nothing to do with being objective or subjective or being concrete or abstract. }

\par{15. おかしな ${\overset{\textnormal{かお}}{\text{顔}}}$ をする。 \hfill\break
\emph{Okashina kao wo suru. \hfill\break
}To make a strange\slash suspicious face. }

\par{16. おかしい ${\overset{\textnormal{かお}}{\text{顔}}}$ をする。 \hfill\break
\emph{Okashii kao wo suru. }\hfill\break
To make a strange\slash funny face. }

\par{17. おかしい ${\overset{\textnormal{はなし}}{\text{話}}}$ ですね。 \hfill\break
\emph{Okashii hanashi desu ne. }\hfill\break
What a strange thing to say. }

\par{18. おかしな ${\overset{\textnormal{こうどう}}{\text{行動}}}$ を ${\overset{\textnormal{と}}{\text{取}}}$ る。 \hfill\break
\emph{Okashina k }\emph{ōd }\emph{ō wo toru. }\hfill\break
To take strange\slash suspicious action(s). }

\begin{center}
\textbf{\emph{Atarashii }新しい vs \emph{Aratana }新たな }
\end{center}

\par{ Although \emph{aratana }新たな was grouped as an adnominal adjective above, the adverbial form \emph{aratani }新たに does exist, and it is actually quite common. The difference between \emph{atarashii }新しい and \emph{aratana }新たな is that the former is more suitable for objective\slash concrete contexts whereas is most suitable for (potentially) subjective\slash obscure. The latter form also happens to be primarily used in writing and has what can be best described as a ‘cool\textquotesingle  nuance. }

\par{ That isn\textquotesingle t to say \emph{aratana }新たな and \emph{aratani }新たに are never used in the spoken language. To the contrary, because they are ‘cool\textquotesingle  sounding with great emotional undertones, they are used extensively in advertisement and the like. }

\par{19. ${\overset{\textnormal{あたら}}{\text{新}}}$ しい ${\overset{\textnormal{かぐ}}{\text{家具}}}$ を ${\overset{\textnormal{か}}{\text{買}}}$ いました。 \hfill\break
\emph{Atarashii kagu wo kaimashita. }\hfill\break
I bought new furniture. }

\par{20. ${\overset{\textnormal{あら}}{\text{新}}}$ たな ${\overset{\textnormal{ぼうけん}}{\text{冒険}}}$ が ${\overset{\textnormal{はじ}}{\text{始}}}$ まる! \hfill\break
\emph{Aratana b }\emph{ōken ga hajimaru! }\hfill\break
A new adventure will become! }

\par{21. ${\overset{\textnormal{あたら}}{\text{新}}}$ しい ${\overset{\textnormal{かんが}}{\text{考}}}$ えに ${\overset{\textnormal{と}}{\text{取}}}$ って ${\overset{\textnormal{か}}{\text{代}}}$ わる。 \hfill\break
\emph{Atarashii kangae ni tottekawaru. \hfill\break
}To replace with a new idea. }

\par{22. ${\overset{\textnormal{けつい}}{\text{決意}}}$ を ${\overset{\textnormal{あら}}{\text{新}}}$ たにしました。 \hfill\break
\emph{Ketsui wo aratani shimashita. }\hfill\break
I\textquotesingle ve renewed my resolution. }

\begin{center}
\textbf{\emph{Yawarakai }柔らかい vs \emph{Yawarakana }柔らかな }
\end{center}

\par{ \emph{Yawarakana }柔らかな is only capable of referring to literal softness of the five senses. However, \emph{yawarakai }柔らかい may be used more broadly despite being less subjective. For instance, when used in the phrase \emph{yawarakai hon }柔らかい本, it can refer to erotica. However, other phrases are more common for this. }

\par{ As you may have already noticed, whenever forms are not interchangeable, it is the set phrases and obscure instances that only seem to matter.  Of course, everything distinguishing \slash i\slash  and \slash na\slash  forms still apply. }

\par{23. \{ ${\overset{\textnormal{やわ}}{\text{柔}}}$ らかい・ ${\overset{\textnormal{やわ}}{\text{柔}}}$ らかな\} ${\overset{\textnormal{かんしょく}}{\text{感触}}}$ に ${\overset{\textnormal{め}}{\text{目}}}$ を ${\overset{\textnormal{みひら}}{\text{見開}}}$ く。 \hfill\break
\emph{[Yawarakai\slash yawarakana] kanshoku ni me wo mihiraku. }\hfill\break
To open one\textquotesingle s eyes to a tinder sensation. }

\par{24. ${\overset{\textnormal{やわ}}{\text{柔}}}$ らかな ${\overset{\textnormal{ひざ}}{\text{日差}}}$ しを ${\overset{\textnormal{かん}}{\text{感}}}$ じる。 \hfill\break
\emph{Yawarakana hizashi wo kanjiru. }\hfill\break
To feel gentle sunlight. }

\par{25. ${\overset{\textnormal{やわ}}{\text{柔}}}$ らかな ${\overset{\textnormal{ものごし}}{\text{物腰}}}$ で ${\overset{\textnormal{せっ}}{\text{接}}}$ する。 \hfill\break
\emph{Yawarakana monogoshi de sessuru. }\hfill\break
To look after\slash deal with a gentle demeanor. }

\par{26. ${\overset{\textnormal{やわ}}{\text{柔}}}$ らかい(お) ${\overset{\textnormal{もち}}{\text{餅}}}$ を ${\overset{\textnormal{つく}}{\text{作}}}$ った。 \hfill\break
\emph{Yawarakai (o)mochi wo tsukutta. }\hfill\break
I made soft rice cake. }

\par{27. ${\overset{\textnormal{やわ}}{\text{柔}}}$ らかな(お) ${\overset{\textnormal{もち}}{\text{餅}}}$ を ${\overset{\textnormal{た}}{\text{食}}}$ べませんか。 \hfill\break
\emph{Yawarakana (o)mochi wo tabemasen ka? \hfill\break
}Why not have some soft rice cake? }
    