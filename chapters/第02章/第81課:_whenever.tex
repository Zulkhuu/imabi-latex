    
\chapter{Whenever}

\begin{center}
\begin{Large}
第81課: Whenever: 毎~, ~毎に, \& ~おきに 
\end{Large}
\end{center}
 
\par{ This lesson is about speech modals that describe repetition of circumstances. Though in some respects, there are striking commonalities among these expressions, pay very close attention to detail as all information in this lesson is relevant in further understanding Japanese time phrases. }
      
\section{Every~}
 
\par{ Though English has the single word "every\dothyp{}\dothyp{}\dothyp{}", Japanese does not. Rather, there are 毎~ and ~毎に, but their combinations, exact readings, and exact nuances cause issues. }

\begin{ltabulary}{|P|P|P|P|P|P|P|P|P|}
\hline 

 & 朝 & 晩 & 夜 & 日 & 時間 & 週間 & 月間 & 年間 \\ \cline{1-9}

毎~ &  \textbf{まいあさ }\hfill\break
まいちょう \hfill\break
& まいばん & まいよ & まいにち & まいじ & まいしゅう & まいげつ \hfill\break
 \textbf{まいつき }& まいねん \hfill\break
 \textbf{まいとし }\\ \cline{1-9}

~毎に & あさごとに & ひとばんごとに \hfill\break
ゆうごとに & よごとに & 1日ごとに \hfill\break
ひごとに & (1)じかんごとに & (1)しゅうごとに \hfill\break
& いっかげつごとに \hfill\break
つきごとに & (1)ねんごとに \\ \cline{1-9}

\end{ltabulary}

\par{\textbf{Usage Notes }: }

\par{1. You may also use these expressions with the words for the seasons. ~ごとに is on the decline in general, so you may not get chances to hear it very often. It also can be seen after verbs, but this has also declined in usage. }

\begin{ltabulary}{|P|P|P|P|P|}
\hline 

 & 春 & 夏 & 秋 & 冬 \\ \cline{1-5}

毎~ & まいはる \hfill\break
まいしゅん (書き言葉) & まいなつ \hfill\break
まいか (書き言葉; Rare) & まいあき \hfill\break
まいしゅう (書き言葉) & まいふゆ \hfill\break
まいとう (書き言葉; Rare) \\ \cline{1-5}

~毎に & はるごとに & なつごとに & あきごとに & ふゆごとに \\ \cline{1-5}

\end{ltabulary}

\par{ \hfill\break
2. まいちょう is rare and very formal. It is more likely to be in writing. \hfill\break
3.  ~毎に can also be translated as "X by X". So, 1時間ごとに = "hour by hour". \hfill\break
4. "Night after night" can be ${\overset{\textnormal{よ}}{\text{夜}}}$ な ${\overset{\textnormal{よ}}{\text{夜}}}$ な. A similar expression is ${\overset{\textnormal{あさ}}{\text{朝}}}$ な ${\overset{\textnormal{ゆう}}{\text{夕}}}$ な meaning "day and evening" in a repetitive sense. These phrases are not usually used in the spoken language. \hfill\break
5. "Month by month" can also be ${\overset{\textnormal{つきづき}}{\text{月々}}}$ . \hfill\break
6. 毎~ must never be used with non-temporal phrases. \hfill\break
7. 毎年 and 毎月 are usually read as まいとし and まいつき respectively. }

\begin{center}
\textbf{Examples }
\end{center}

\par{1. 毎朝散歩に出かけます。 \hfill\break
I go out on a walk every morning. }

\par{2. 毎時50キロメートルの ${\overset{\textnormal{そくど}}{\text{速度}}}$ \hfill\break
50 kilometers per hour }

\par{3. ${\overset{\textnormal{はは}}{\text{母}}}$ は ${\overset{\textnormal{}}{\text{毎朝散歩}}}$ します。 \hfill\break
My mom walks every morning. }

\par{4. 毎日どこでご飯を食べますか。 \hfill\break
Where do you eat every day? }

\par{5. 日本人とアメリカ人は ${\overset{\textnormal{まいにちさんかいしょくじ}}{\text{毎日三回食事}}}$ をするのが ${\overset{\textnormal{ふつう}}{\text{普通}}}$ だ。 \hfill\break
Japanese and Americans usually eat three times every day. }

\par{Practice: }

\par{1.Try making a sentence with ことあるごとに (With every little thing) \hfill\break
2. Translate ことあるごとに、彼を思い出すんだよ。 }

\par{\textbf{Using ~ごとに }}

\par{ ~ごとに has several usages. Other than phrases such as ことあるごとに, it is hardly ever used after verbs. However, it is always very old-fashioned when after verbs to mean “whenever” (See ~度に below). There are four distinct situations that you may see ~ごとに used in. }

\par{1. For every constant A (place, point of time, or person), a repetition B occurs. }

\par{2. For every variable A (place\slash person), a different thing exists. }

\par{3. It shows that something takes place once every given amount of time. }

\par{4. As you repeat a certain constant A, a change progresses. (Rarely seen with verbs) }

\begin{center}
 \textbf{Examples }
\end{center}

\par{6. ${\overset{\textnormal{しゅくじつ}}{\text{祝日}}}$ \{ごと・のたび\}に、家族みんなでドライブに出かけることにしてますよ。(話し言葉) \hfill\break
The whole family goes out on a drive \{every\slash whenever there is a\} holiday. }

\par{\textbf{Speech Style Note }: ~のたびに would make it more 話し言葉的. }

\par{7. その ${\overset{\textnormal{きぎょう}}{\text{企業}}}$ ${\overset{\textnormal{}}{\text{グ}}}$ ループは、会社ごとに、それぞれの ${\overset{\textnormal{しゃぜ}}{\text{社是}}}$ と ${\overset{\textnormal{しゃそく}}{\text{社則}}}$ があるから、よく注意してください。 \hfill\break
As for that corporation group, every company has its own policies and rules. So, please pay close  attention. }

\par{8. 一雨ごとに、もっと暖かくなる。 \hfill\break
With each rain, it gets warmer. }

\par{9. ことあるごとに ${\overset{\textnormal{はんたい}}{\text{反対}}}$ するのはだめだよ。 \hfill\break
It's bad for you to be against with every little thing. }

\par{10. 四年ごとに ${\overset{\textnormal{うるうどし}}{\text{閏年}}}$ になって、オリンピックが開かれるということは ${\overset{\textnormal{じょうしき}}{\text{常識}}}$ だ。 \hfill\break
It's common sense that every four years there is a leap year, and the Olympics are held. }

\par{\textbf{Grammar Note }: 開かれる is the passive form of the 五段 verb 開く. \hfill\break
 \textbf{漢字 Note }: 閏 is not a 常用漢字. So, you aren't responsible for it at this time. }

\begin{center}
 \textbf{Every Time }
\end{center}

\par{ For "every time", ${\overset{\textnormal{まいど}}{\text{毎度}}}$ , ${\overset{\textnormal{まいかい}}{\text{毎回}}}$ , ${\overset{\textnormal{まいじ}}{\text{毎次}}}$ , ${\overset{\textnormal{たび}}{\text{度}}}$ に, and ${\overset{\textnormal{つど}}{\text{都度}}}$ exist. The first two are different in that the former is stronger and seen in set phrases like 毎度ありがとうございます. For when you want to say you attend every time, you need to use 毎回 instead. ${\overset{\textnormal{}}{\text{毎次}}}$ is quite formal is used in the sense of "whenever" but is not used in the spoken language. }

\par{11. 六時間 ${\overset{\textnormal{ごと}}{\text{毎}}}$ に ${\overset{\textnormal{くすり}}{\text{薬}}}$ を ${\overset{\textnormal{ふくよう}}{\text{服用}}}$ しなくてはいけません。 \hfill\break
You have to take medicine every 6 hours of time. }

\par{12. その都度 ${\overset{\textnormal{しようりょう}}{\text{使用料}}}$ を ${\overset{\textnormal{はら}}{\text{払}}}$ う。 \hfill\break
I pay the usage fee each and every time. }

\par{13. 毎回 ${\overset{\textnormal{さんか}}{\text{参加}}}$ します。 \hfill\break
I participate each time. }

\par{14. ${\overset{\textnormal{まいじみっせつ}}{\text{毎次密接}}}$ に ${\overset{\textnormal{れんらく}}{\text{連絡}}}$ を取ります。 \hfill\break
Each time I'll be in close touch. }

\par{15. やる度に ${\overset{\textnormal{しっぱい}}{\text{失敗}}}$ する。 \hfill\break
To fail whenever one tries. }

\par{16. 彼は週毎に4万円の ${\overset{\textnormal{かせ}}{\text{稼}}}$ ぎがあります。 \hfill\break
He has earnings of 40,000 yen a week. }

\par{\textbf{Frequency Note }: Although 月ごとに and 週ごとに are correct, they are typically replaced with 月に and 週に respectively. }

\par{\textbf{関西弁 Note }: 毎度大きに is a very important phrase from 関西弁 meaning the same thing as 毎度ありがとうございます. }
      
\section{~おきに}
 
\begin{center}
 \textbf{With Time }
\end{center}

\par{ ~おきに is equivalent to "every other". Like all of these other expressions, something is repeated in a given interval. However, as you can see in the translation, it is not quite the same. }

\par{17. 2日おきに = Every other two days  VS    2日ごとに = Every second day \hfill\break
●○○●○○●○○●\dothyp{}\dothyp{}\dothyp{}  ●○●○●○●\dothyp{}\dothyp{}\dothyp{} }

\par{18. 3日おきに  = 4日ごとに  Every other three days = every fourth day \hfill\break
●○○○●○○○●\dothyp{}\dothyp{}\dothyp{}   ●○○●○○●\dothyp{}\dothyp{}\dothyp{} }

\par{19. 彼女は ${\overset{\textnormal{いちにち}}{\text{一日}}}$ おきに ${\overset{\textnormal{しゅっきん}}{\text{出勤}}}$ する。 \hfill\break
She goes to work every other day }

\begin{center}
 \textbf{With Distance }
\end{center}

\par{ For distance, ~ごとにshows precise repetition of distance. So, if you are lining bottles and place them 5メートルごとに, you start from the midpoint from one bottle and place the next bottle five meters from that point. If you place them 5メートルおきに, you are measuring the space in between each bottle, so you measure from the edges of each bottle. }

\par{20. 部屋は5分おきに ${\overset{\textnormal{ゆ}}{\text{揺}}}$ れた。 \hfill\break
The room shook every other five minutes. }
    