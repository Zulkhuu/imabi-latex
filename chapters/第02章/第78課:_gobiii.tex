    
\chapter{語尾 II}

\begin{center}
\begin{Large}
第78課: 語尾 II: わ, な, と, の, さ, ぞ, ぜ, \& え 
\end{Large}
\end{center}
 
\par{  ${\overset{\textnormal{ごび}}{\text{語尾}}}$ , as we have learned, are interjectory particles that show all sorts of \textbf{emotions }. }
      
\section{The Final Particle わ}
 
\par{ わ is more \textbf{colloquial }--used in casual speech--than よ with gender restrictions based on tone. If \textbf{high }, it is associated with standard \textbf{feminine speech }. If low, it is associated with ${\overset{\textnormal{かんさいべん}}{\text{関西弁}}}$ in which both men and women use it frequently. It may also list things with exclamation; this isn't necessarily feminine. }
 
\par{${\overset{\textnormal{}}{\text{1. 本当}}}$ に ${\overset{\textnormal{こま}}{\text{困}}}$ ったわ。 \hfill\break
I'm really troubled. }
 
\par{2. まあ、 ${\overset{\textnormal{すてき}}{\text{素敵}}}$ だわ。 \hfill\break
Well, that's great. }
 
\par{${\overset{\textnormal{}}{\text{3. 行}}}$ くわよ。 \hfill\break
I'm going! }
 
\par{\textbf{Usage Note }: ~わよ is \textbf{very feminine }. Extremely feminine and extremely masculine expressions are in somewhat of a decline in favor of more gender neutral ones. They are also not going to be used in polite speech. }
 
\par{4. バンが ${\overset{\textnormal{こわ}}{\text{壊}}}$ れるわ、 ${\overset{\textnormal{}}{\text{乗}}}$ り ${\overset{\textnormal{}}{\text{上}}}$ げるわ、 ${\overset{\textnormal{さんざん}}{\text{散々}}}$ な ${\overset{\textnormal{}}{\text{一日}}}$ だった。 \hfill\break
The van broke, I got stranded, and I've had a day! }
 
\par{\textbf{Particle Note }: The sentence above is a great example of the usage of particles as filler words after clauses mentioned in the last section. }

\par{5. あとの事は何れ東京へ出たら、逢った上で話を付けらあ。 \hfill\break
I'll talk about that once I've gone and met (with him). \hfill\break
From 門 by 夏目漱石. }

\par{\textbf{Contraction Note }: ~らあ is a contraction of ~るわ and is equivalent to ~るなぁ in this context. }
 
\par{\textbf{Historical Note }: わ is an evolved form of は. }
      
\section{The Final Particle な}
 
\par{1. な is for the most part the masculine version of ね. However, females have begun to use it as well, particularly those that are seen as being stronger than the average woman. This should not be used in polite situations. }

\par{6. ${\overset{\textnormal{おれ}}{\text{俺}}}$ な、 ${\overset{\textnormal{こんど}}{\text{今度}}}$ な、 ${\overset{\textnormal{}}{\text{勝}}}$ つんだ。 \hfill\break
I, uh, this time, will win. }
 
\par{7. あのな、こっちではな、 ${\overset{\textnormal{}}{\text{酒}}}$ はいけないんだ。(説教的な言い方) \hfill\break
Um, you mustn't drink sake here, you know. }

\par{8. 懐かしいなあ。 \hfill\break
How nostalgic. }
 
\par{9. いやだな。 \hfill\break
What a pain! }

\par{10. 君が来ないと、つまらないなあ。 \hfill\break
If you don't come, it'll be boring. }

\par{11. ${\overset{\textnormal{だれ}}{\text{誰}}}$ か ${\overset{\textnormal{}}{\text{来}}}$ たな。 \hfill\break
Someone came, didn't they? }
 
\par{${\overset{\textnormal{}}{\text{12. 罪人}}}$ はお ${\overset{\textnormal{}}{\text{前}}}$ だな。 \hfill\break
The sinner is you, isn't it! }

\par{2. Placed after the ${\overset{\textnormal{しゅうしけい}}{\text{終止形}}}$ (the end form) of a verb to show a strong negative command. }
 
\par{13. そんなことをするな。 \hfill\break
Don't do such a thing! }
 
\par{${\overset{\textnormal{}}{\text{14. 行}}}$ くな! \hfill\break
Don't go! }
 
\par{15. すんな! \hfill\break
Don't do it! \hfill\break
 \hfill\break
\textbf{Contraction Note }: ~るな is often contracted to ~んな. Even if a verb doesn't end in the sound, ん is often inserted before な. So, things ${\overset{\textnormal{}}{\text{話}}}$ すんな are becoming more common. Again, don't use these contractions in polite situations. }
 
\par{3. Used as a contraction of ~なさい,. Don't confuse this with Usage 2! }
 
\par{${\overset{\textnormal{}}{\text{16. 行}}}$ きな! \hfill\break
Go! }
 
\par{17. やってみな! \hfill\break
Try it! }
 
\par{18. ちょっと こ っちに ${\overset{\textnormal{}}{\text{来}}}$ な。 \hfill\break
Come here for a moment. }
      
\section{The Final Particle と}
 
\par{ There are three very different usages of the final particle と. }
 
\begin{enumerate}
 
\item In a question with a high intonation, it presents a demanding      question. This is used in casual or vulgar speech.  
\item In っと it is used to      declare something with a light tone. It is used in plain speech.  
\item とさ is used with      polite endings in story-telling.  
\end{enumerate}
 
\begin{center}
\textbf{Examples } 
\end{center}

\par{19. 何だと!?(Vulgar\slash casual) \hfill\break
WHAT!? }
 
\par{20. いやだと? \hfill\break
You say no!? }
 
\par{21. 知らねーっと。(Vulgar; slang) \hfill\break
I don't know. }
 
\par{22. 二人で ${\overset{\textnormal{しあわ}}{\text{幸}}}$ せに ${\overset{\textnormal{く}}{\text{暮}}}$ らしましたとさ。 \hfill\break
And they lived happily ever after. \hfill\break
 \hfill\break
23. やめたいだと? \hfill\break
You want to quit!? }
 
\par{\textbf{Grammar Note }: だと is allowed here because it is acting as a final particle. }
      
\section{The Final Particle の}
 
\par{ The final particle の is becoming more unisex in appeal, but guys should be careful with their tone of voice. As said earlier, it can show a decisiveness\slash confirmation, reasoning with a low pitch, and question with a high pitch. Patterns that are particularly feminine include (な)のよ, のね, and の!. }
 
\par{24. やりたくないの!(Feminine) \hfill\break
I don't want to do it! }
 
\par{25. 休みなのよ。(Feminine) \hfill\break
It's a break! }
 
\par{${\overset{\textnormal{}}{\text{26. 宿題}}}$ (を) ${\overset{\textnormal{}}{\text{忘}}}$ れたの? \hfill\break
You forgot your homework? }
 
\par{27. ああ、そうだったの。(A little feminine) \hfill\break
Ah, is that so? }
 
\par{28. やはりだめだったのね。(Feminine) \hfill\break
It was bad as expected, wasn't it? }
 
\par{${\overset{\textnormal{}}{\text{29. 明日来}}}$ ないの。 \hfill\break
You're not coming tomorrow? }
 
\par{${\overset{\textnormal{}}{\text{30. 仕事}}}$ があったんじゃないの? \hfill\break
Didn't you have work? }
 
\par{31. 強い ${\overset{\textnormal{}}{\text{男}}}$ は ${\overset{\textnormal{}}{\text{泣}}}$ かないの。 \hfill\break
Strong guys don't cry. }
 
\par{32. おお、まことか、よく来たの。(Old person) \hfill\break
Oh, it's you Makoto; good of you to come. }
 
\par{\textbf{Speech Style Note }: The last sentence sounds like it came from an old person. よく来たな is a more masculine yet more common way to say it. よく来たね is more gender neutral. As you can see, it's different from the other sentences above. }
      
\section{The Final Particle さ}
 
\par{  さ is a signature feature of Tokyo speech and is the Japanese equivalent to the overuse of "like" in English. Don't use this in polite speech. This is very casual, and the usage is rather random based on region and age. The people most likely to use it are young people in and around the capital area ( ${\overset{\textnormal{しゅとけん}}{\text{首都圏}}}$ ). }
 
\par{${\overset{\textnormal{}}{\text{33. 東京}}}$ は ${\overset{\textnormal{}}{\text{本当}}}$ にきれいさ。 \hfill\break
Tokyo is like really pretty. }
 
\par{${\overset{\textnormal{}}{\text{34. 明日}}}$ はさ、 ${\overset{\textnormal{}}{\text{土曜}}}$ だからさ、 ${\overset{\textnormal{}}{\text{休}}}$ みなんだ。 \hfill\break
Tomorrow is, um, cause it\textquotesingle s Saturday, we're going to be on break. }
 
\par{${\overset{\textnormal{}}{\text{35. 何}}}$ さ、あんなやつめ! \hfill\break
What the, that guy! }
 
\par{\textbf{Grammar Note }: さ is typically not placed after だ as the first example demonstrates. However, many speakers don't follow this. }
      
\section{The Particle ぞ}
 
\par{ ぞ is very casual and at times harsh. ぞ, which is associated with masculine speech, creates a casual yet strong assertive emphasis. It may also be used to make a rhetorical question. Its role as being strongly masculine is in decline, but it is still used frequently, just not with as much of the gender baggage. You should not use this in polite speech. }
 
\par{36 ${\overset{\textnormal{}}{\text{. 何}}}$ か ${\overset{\textnormal{へん}}{\text{変}}}$ だぞ。 \hfill\break
Something's strange. }
 
\par{${\overset{\textnormal{}}{\text{37. 今}}}$ しないと、 ${\overset{\textnormal{しっぱい}}{\text{失敗}}}$ するぞ。 \hfill\break
If you don't do it now, you'll fail! }
 
\par{38. とにかく ${\overset{\textnormal{}}{\text{家}}}$ へ ${\overset{\textnormal{}}{\text{帰}}}$ るぞ。 \hfill\break
Anyway, I'm going home. }
 
\par{39. 勉強しなかったから、テストに ${\overset{\textnormal{ごうかく}}{\text{合格}}}$ しなかったんだぞ。 \hfill\break
It's because you didn't study that you didn't pass your test! }
 
\par{40a. あれは ${\overset{\textnormal{なにもの}}{\text{何者}}}$ ぞ。 (Very old-fashioned) \hfill\break
40b. あいつは何者だ? (More natural) \hfill\break
Who is that? }
 
\par{\textbf{Phrasing Note }: Using ぞ in making a rhetorical question is rather archaic. }
      
\section{The Final Particle ぜ}
 
\par{ ぜ rudely and or forcefully pushes an idea. Due to this, it is only appropriate in casual conversation and should never be used with your superiors (in a typical situation). Only if you and your superior(s) are drunk should you ever use it; that is unless you're quitting. }
 
\par{ ぜ has historically been extremely masculine, but it is now not completely out of the norm for female speakers to use it among themselves. This, though, may make them look unfeminine. Strong female bodybuilders can use ぜ just like a rough playing boy child in a schoolyard. }
 
\par{ It is because of these reasons why experience in hearing them used is the best way to truly know the full realm of their usage. }
 
\par{${\overset{\textnormal{}}{\text{41. 雪}}}$ だぜ。 \hfill\break
It's snow! }
 
\par{${\overset{\textnormal{}}{\text{42. 行}}}$ くぜ! \hfill\break
Let's go! }
 
\par{\textbf{Origin Note }: ぜ is the contraction of ぞえ. }
      
\section{The Final Particle え}
 
\par{ Although no longer common, え is used to either to call for someone or push an idea for questioning something. It's often in the pattern ぞえ. You might hear something with this in it in some old-fashioned drama. }
 
\par{${\overset{\textnormal{}}{\text{43. 行}}}$ くぞえ。 \hfill\break
Let's go! }
 
\par{44. お ${\overset{\textnormal{かみ}}{\text{上}}}$ さんえ (Old-fashioned) \hfill\break
Missus! }
 
\par{\textbf{Spelling Note }: ぞえ and ぞぇ are both common spellings, and the pattern is still common among many speakers. It's just that using the particle in any other situation is very uncommon. }
    