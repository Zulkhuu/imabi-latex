    
\chapter{Pluralization}

\begin{center}
\begin{Large}
第92課: Pluralization 
\end{Large}
\end{center}
 
\par{ In English and most other Indo-European languages, grammatical number is expressed with inflections such as –s. However, Japanese grammar does not have grammatical number. This fact is very hard for beginners to grasp, but given now that you have had more experience in Japanese, it is time to look at this serious issue in greater detail. }

\par{\textbf{Terminology Note }: Indo-European (インド・ヨーロッパ祖語) simply refers to the ancestral language that ties English with most languages from Europe to India. }
      
\section{Lack of Grammatical Number}
 
\par{ However, the fact that Japanese doesn't have grammatical number doesn't mean that Japanese lacks the ability to differentiate between something that is singular and something that is plural. After all, if you use quantity qualifiers such as 多くの or 少しの, there isn't any problem in telling whether someone is referring to just one thing or more than one of something. }

\par{1. 花がたくさん咲いている。 \hfill\break
There are a lot of flower \textbf{s }blooming. }

\par{ Noun classification differs considerably between English and Japanese. In English, when we sense that something is more than 1, we attribute that to being plural and unconsciously use a form of –s, ignoring instances where the singular and plural form are the same and other such irregularities. All Japanese nouns are 全部 ${\overset{\textnormal{ちゅうしょう}}{\text{抽象}}}$ 名詞. So, there is a vague feeling in comparing it to English. So, if you were to want to say 2 pens, you would say 2本のペン. There is no reason to ever have any inflection on ペン. }

\par{ In English, there are also nouns that are countable and others that are not countable. For example, chalk isn't countable. “Chalks” would refer to different kinds of chalk, not there being two pieces of chalk. In Japanese, no such distinctions are made. There are counter phrases that help bridge this gap. }

\par{ Grammatical number in English also has a direct correlation with numbers themselves. In Christianity, Judaism, and Islam, there is but one God. However, in Japanese, you have the well-known yet often misunderstood phrase ${\overset{\textnormal{やまと}}{\text{大和}}}$ には ${\overset{\textnormal{やおよろず}}{\text{八百万}}}$ の神がある. This is not saying that there is some exact number of 神 in Japanese, but that there is a lot. 8, 9, 10, and 10,000 were often used in the ancient period to refer to a lot. }

\par{ Thus, the items that do express plurality in Japanese are different from grammatical number in the sense that added nuances are attributed to particular endings, and these particular endings are used with particular nouns. }

\par{\textbf{Form Note }: There are rare instances in Japanese that hint that Japanese did have a clearer distinction of plurality in some words. For instance, ひ becoming か in counters after 1. 1 day used to be ひとひ. Then, 2 days is ふつか. Notice the change? The same thing goes for people. You get ひとり but then ふたり, みたり, よたり. This とり \textrightarrow  たり is of the same vein as ひ \textrightarrow  か. In fact, there's even an old phrase 日並べて (かなべて) that refers to days one after another. }
      
\section{~たち・達}
 
\par{ More frequently written in 漢字 in literature, the suffix ~たち is the closest thing in Japanese to the “-s” in English. However, it typically indicates a group of things. For instance, when you say 石田君たち, 石田君 is pointed out as the head of a group of a multitude of people, which could be his family or his group of friends. Thus, correct interpretation of a phrase with ~たち is heavily sensitive to context. }

\par{2. 私たちは賛成します。 \hfill\break
We agree. }

\par{3. 山田君たち、頭が悪いね。 \hfill\break
Yamada's group isn't smart. }

\par{4. 理事たちは、大島理事長排斥に手をつけると言った。 \hfill\break
The directors said that they would start work on the boycott of Board Chairman Ohshima. \hfill\break
From 混声の森 (下) by 松本清張. }

\par{5. 数々のすばらしいヒット曲をまだまだたくさんの方たちに届け続けてほしかったです。 \hfill\break
I wanted [her] to still continue delivering many wonderful hit songs to a lot of people. \hfill\break
From the words of the singer 八代亜紀 upon hearing the death of singer 藤圭子. }

\par{ Another related issue is something like 犬たち. Although this could very well mean “dogs”, depending on context, it could also comprehensively refer to a multitude of all sorts of (similar) animals. }

\par{\textbf{Grammar Note }: Unless if something that could be viewed as belonging in a group like a boat in a fleet, plural suffixes are rarely ever used with inanimate nouns. }

\par{\textbf{Word Note }: 友達, note the voicing of ~たち, can still be used even if you are referring to just one friend. This shows how fragile the nuances of these endings are. This is all possible due to the lack of grammatical number. This is of the same vein as 子供, which will be touched on later in this lesson. However, 友たち is possible, which would refer to there being a plurality in friends. }

\par{\textbf{Classical Japanese Note }: ~達 started out as a very respectful suffix used with 神・貴人 “aristocrat” ( ${\overset{\textnormal{きんだち}}{\text{公達}}}$ = Kings; children of nobles; nobleman\slash noblemen). It eventually lessened to being used with light respect, and in Modern Japanese it is no longer honorific. Notice how the ambiguity in plurality has existed when the suffix is voiced. }
      
\section{~ら・等}
 
\par{ ~ \textbf{ら }is casual, so it should not be used with わたくし. However, it does get used with わたしら. This, though, is not that common in 標準語. For the most part, both ~ら and \textbf{~ }たち attach to basically any pronoun or person noun. \hfill\break
\hfill\break
 They are also used a lot with 3rd person pronouns, but 彼女ら is oddly not near as common as 彼女たち. To make things more weird, 彼ら is far more common than 彼たち. For nouns of person, ら is the ideal choice in formalized speech. Nothing or たち would be more common in more spoken-language speech. }

\par{6. 私らには分かりません。 \hfill\break
We don't understand. }

\par{7. 夏は、私らみたいな、いわゆる家出少女が溢れる季節だ。 \hfill\break
Summer is a season where so-called runaway girls like us overflow. \hfill\break
From 冷たい誘惑 by 乃南アサ. }

\par{8. 僕らは強くないさ。 \hfill\break
We aren't strong. }

\par{ ~ら also attaches to これ, それ, and あれ. However, it is not actual obligatory to express these words in a plural sense. }

\par{9. これらをXと呼びます。 \hfill\break
We call these X. }

\par{10. これらの手掛かりを繋ぎ合わせてみよう。 \hfill\break
Let's try to link these clues together. }

\par{\textbf{漢字 Note }: This suffix is also sometimes spelled as 等. }
      
\section{~ども・共}
 
\par{ Rarely written in 漢字, the suffix ~ども is typically used as a humble plural marker for pronouns. For instance, わたくしども is quite common for “we” in this group sense. It may also be used to refer to groups condescendingly. So, you can see it with わたくし, やつ, 手前, etc. It should not be used 僕, あなた or titles such as ${\overset{\textnormal{せんせい}}{\text{先生}}}$ . }

\par{11. 私どもにお任せください。 \hfill\break
Please leave it to us. }

\par{12. 野郎ども  (You have seen nothing) \hfill\break
Bastards }

\par{13. 自分のことを男は「俺」ということが多いが、俺共は「オレドー・オイドー」になり単数に使われることも多い。 \hfill\break
Men often call themselves "ore", but "us" becomes, "oredō\slash oidō", and these are used a lot for the singular. }

\par{\textbf{Dialect Note }: This sentence comes from 大分方言 by 高田和彦. 俺共 is a rather interesting combination, but 俺 is being used in a more casual sense while being qualified with 共 to be somewhat humble. In this dialect, the dialectical form of 俺共 ends up becoming singular again, which happens in 子供 too. }

\par{14. 「 ${\overset{\textnormal{むらびと}}{\text{邑人}}}$ どもにはわたしが ${\overset{\textnormal{なが}}{\text{永}}}$ の病で ${\overset{\textnormal{とうじ}}{\text{湯治}}}$ に来たといつわっていたのだ。...」 \hfill\break
The villagers are lying that I\textquotesingle ve come to get a hot-spring cure for an eternal illness. \hfill\break
From ${\overset{\textnormal{かるのみこ}}{\text{軽王子}}}$ と ${\overset{\textnormal{そとおりひめ}}{\text{衣通姫}}}$ by 三島由紀夫. }

\par{ In older Japanese, ~ども can be seen with 事. Thus, you get 事ども. Another well studied example of ~ども is 子供・子ども. The character for it in 子供 is あて字. However, again, the problem about 子供 is that it is usually used in the singular sense. Thus, the suffix has lost its original meaning. Thus, you often see 子どもたち. }
      
\section{~がた・方}
 
\par{\textbf{ ~がた }is a very polite plural suffix that denotes a high status. One of the most common examples is 方々, from which it came. Thus, it is often spelling in 漢字 as 方. It is \textbf{not }compatible with first person pronouns, degrading second person pronouns, or titles that suggest a lower status. It may be used with words such as 先生 meaning "teacher" and ${\overset{\textnormal{みなさま}}{\text{皆様}}}$ meaning "everyone". }
      
\section{Pronouns and Pluralization}
 
\par{ The following pronouns in their plural forms are listed from most formal to least formal. }

\begin{ltabulary}{|P|P|P|P|P|P|P|P|P|}
\hline 

1st & 私共 & 我々 & 我ら & わたくしたち & 私[達・ら] & 僕[達・ら] & あたし[たち・ら] & 俺[たち・ら] \\ \cline{1-9}

2nd & 皆様(がた) & 皆さん & み(ん)な & あなたがた & あなたたち & 君[達・ら] & あんた[たち・ら] & お前[たち・ら] \\ \cline{1-9}

3rd & 彼達 & 彼女達 & 彼ら & 彼女ら &  &  &  &  \\ \cline{1-9}

\end{ltabulary}

\par{\textbf{Speech Note }: みんな is casual. みな also has the form 皆々様, which is a more emphatic yet very honorific form used a lot in New Years cards, 年賀状, and other situations when honorific speech is expected. }

\par{15. 私たちは ${\overset{\textnormal{こういち}}{\text{高一}}}$ です。 \hfill\break
We are 10th graders. }

\par{16. 俺らはいる。 \hfill\break
We're here. }

\par{17. 我々の ${\overset{\textnormal{もんだい}}{\text{問題}}}$ です。 \hfill\break
It's our problem. }

\par{18, 皆々様のご多幸をお祈り申し上げます。(Very Humble). \hfill\break
I pray that you all may be very fortunate (in the new year). }

\par{19. 自分らの手で書いたらしく、下手な字で、行も乱れていた。 \hfill\break
It seemed to be written by them, and the handwriting was bad, and the lines were messed up. \hfill\break
From 混声の森 (下) by 松本清張. }

\par{\textbf{Word Note }: The use of 自分 refers to a person in the story being accused of a relationship with another person in the academy, and so the reason why the author chose to pluralize 自分 was to refer to the two as a group in a condescending way, though the post being talked about was written by one person. }

\par{20. その点われわれも同感です。あるいは、ほかの理事の方に逆効果を与えるかも分かりませんな。 \hfill\break
We are sympathetic as well with that point. Perhaps, it may give the opposite effect to the other directors. }

\par{\textbf{Grammar Note }: Here, 理事の方 is being used in the plural sense, and this is obvious from context. So, again, remember that although there are plural suffixes in Japanese, you must be aware that in situations such as this and in many instances, using them is not necessary and could at the very worst be wrong. }
    