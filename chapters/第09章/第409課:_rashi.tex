    
\chapter{The Auxiliary Verb ~らし}

\begin{center}
\begin{Large}
第409課: The Auxiliary Verb ~らし 
\end{Large}
\end{center}
 
\par{ As you would imagine just by the name of this auxiliary that this is the predecessor to ~らしい. Just as it is used now, ~らし is an evidential modal that shows supposition based on good reason and or objective evidence. This is why it is translated as "it appears that". In many instances, it is much like ~に違いない. Aside from notes concerned examples and abbreviations, this explanation right here pretty much covers this item. }
      
\section{~らし}
 
\par{ As mentioned earlier, this is an evidential. Evidentials are phrases found in the world's languages that define where information comes from. There are arguably more obvious evidentials in Japanese, of which are all classified as auxiliary verbs. One thing to note is that ~らし has special conjugation rules in Classical Japanese. }

\begin{ltabulary}{|P|P|P|P|P|P|}
\hline 

未然形 & 連用形 & 終止形 & 連体形 & 已然形 & 命令形 \\ \cline{1-6}

X & X & らし & らし & らし & X \\ \cline{1-6}

\end{ltabulary}

\par{\textbf{Base Note }: The 已然形 is used when there is a bound particle. This is how you can distinguish it from the 終止形 and 連体形. }

\par{ It would be a mistake to use らしき, and it would be a mistake to use it as a 形容詞. So, in some ways, the usage of this has evolved to become more variable in Modern Japanese. However, one thing that could be done with ~らし that cannot be done with ~らしい is that with ラ変 conjugating items, the ~る in the 連体形 may be dropped when ~らし is attached. }

\begin{ltabulary}{|P|P|P|P|P|P|P|P|P|P|}
\hline 

けり+らし & \textrightarrow  & けるらし & \textrightarrow  & けらし & あり+らし & \textrightarrow  & あるらし & \textrightarrow  & あらし \\ \cline{1-10}

なり+らし & \textrightarrow  & なるらし & \textrightarrow  & ならし & 新し + らし & \textrightarrow  & 新しかるらし & \textrightarrow  & 新しからし \\ \cline{1-10}

\end{ltabulary}

\begin{center}
\textbf{Examples }
\end{center}

\par{1. 暮去者 小倉乃山尓 鳴鹿者 今夜波不鳴 寐宿家良思母 (原文) \hfill\break
夕されば小倉の山に鳴く鹿は今夜は鳴かず い寝にけらしも \hfill\break
When it comes night, the deer that always cry at Mt. Ogura are not crying tonight. They must have         fallen asleep. \hfill\break
From the 万葉集. }

\par{2. 百年に ${\overset{\textnormal{ひととせ}}{\text{一年}}}$ たらぬつくも髪我を恋ふらし面影に見ゆ。 \hfill\break
The old woman almost 99 years old is definitely missing me. In my eyes I vividly see her image. \hfill\break
From the 伊勢物語. }
 
\par{3. この川に ${\overset{\textnormal{もみぢ}}{\text{紅葉}}}$ 葉流る奥山の雪消の水ぞ今まさるらし。 \hfill\break
As far as the colored leaves flowing in this river, it now definitely appears that the water from the             snow melting from the remote mountains is increasing. \hfill\break
From the 古今和歌集. }
 
\par{4. 白雲のこのかたにしもおりゐるは天つ風こそ吹きて来ぬらし。 \hfill\break
The white clouds descending and trailing this direction definitely appears to be because of the wind         blowing from the heavens. \hfill\break
From the 大鏡. }

\par{5. ${\overset{\textnormal{つついつついづつ}}{\text{筒井筒井筒}}}$ にかけしまろがたけ過ぎにけらしな妹見ざるまに。 \hfill\break
When comparing statures with me and the round well curb, it definitely appears that I have surpassed     it in height…during the time I haven't met you. \hfill\break
From the 伊勢物語. }
 
\par{6. これに稲つみたるをや、いな船といふならし。 \hfill\break
It appears that they call this boat that rice is loaded in the “rice boat”. \hfill\break
From the 奥の細道. }

\par{7. 竜田川 もみぢ葉ながる 神奈備の 三諸の山に しぐれ降るらし。 \hfill\break
The colored leaves flow in the Tatsuta River. Based on this, it appears that a scattered shower fell at      the holy Mimoro Mountain upstream. \hfill\break
From the 拾遺集. }
8. 久方之 天芳山 此夕 霞霏微 春立下 (原文) \hfill\break
ひさかたの天の香具山 このゆふべ霞たなびく春たつらしも。 \hfill\break
This evening on Heavenly Kaguyama, the mist is trailing. It seems that spring has arrived. \hfill\break
From the 万葉集. 
\par{9. 和我多妣波 比左思久安良思 許能安我家流 伊毛我許呂母能 阿可都久見礼婆 (原文) \hfill\break
我が旅は久しくあらしこの我が着る妹が衣の垢つく見れば(万葉集・一五-三六六七) \hfill\break
It appears that my trip has gone long. When you look at the robe of my wife that I\textquotesingle m wearing being         covered in dirt… \hfill\break
From the 万葉集. }

\par{10. 春過而 夏来良之 白妙能 衣乾有 天之香来山 (原文) \hfill\break
春過ぎて夏来たるらし  白栲の衣ほしたり  天の香具山 \hfill\break
It definitely appears that spring has passed and summer has arrived. \hfill\break
They are drying the white hemp robes at Heavenly Kaguyama. \hfill\break
From the 万葉集. }

\par{11. 春過ぎて夏来にけ(る)らし。 \hfill\break
Spring has passed and summer appears to have definitely arrived. \hfill\break
\hfill\break
\textbf{Sentence Note }: This sentence is a later variant of the same phrase found in the 万葉集. }

\par{Exercises: Correctly put together the following adjectives with らし. }

\par{1. 重し \hfill\break
2. 寒し \hfill\break
3. 熱し \hfill\break
4. 美し \hfill\break
5. をかし \hfill\break
\hfill\break
Translate the following into English and Modern Japanese. }

\par{1. 蝉の声聞こゆ。夏来たるらし。 }
     