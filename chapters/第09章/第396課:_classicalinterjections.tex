    
\chapter{Interjections}

\begin{center}
\begin{Large}
第396課: Interjections 
\end{Large}
\end{center}
 
\par{In Classical Japanese there are a lot of interjections that are no longer used today. However, they are by no means difficult words to remember. You can even find some that are still used today. "Aa!", for example, is pretty universal. }

\par{\textbf{Curriculum Note }: This lesson will not cover interjectory particles. They will be discussed later when we look at the wide range of final particles in Classical Japanese. }
      
\section{Classical Japanese Interjections}
 
\par{In these example sentences, notice how the interjections are used in the sentence. The interjections will be pointed out in bold. }

\par{1. \textbf{いな }、さもあらず。 \hfill\break
No, that's not the way. \hfill\break
From the 竹取物語. }
 
\par{\textbf{Word Note }: いな can still be used as an interjection today; however, it is usually accompanied with slightly older speech patterns such as the auxiliary verbs ~ぬ and ~まい. }

\par{2. \textbf{のうのう }われをも舟に乗せて賜はり給へ。 \hfill\break
Hey you there, please allow me to board the boat as well. \hfill\break
From the Noh Play 隅田川. }

\par{\textbf{Word Note }: のうのう is onomatopoeic and is still seen as a 擬声語 meaning "carelessly". It is used in this context to get the attention of the boat, perhaps being careless for not seeing him wanting to get on board? }
 
\par{\textbf{Kanji Note }: 舟 denotes a smaller vessel. }

\par{3. \textbf{あな }、うらやまし。 \hfill\break
My! How enviable. \hfill\break
From the 徒然草. }
 
\par{\textbf{Word Note }: あな is still used as an interjection in some parts of Japan. }

\par{4. \textbf{しかしか }、さ ${\overset{\textnormal{はべ}}{\text{侍}}}$ りことなり。 \hfill\break
Yes, yes, that is quite so. \hfill\break
From the 大鏡. }
 
\par{\textbf{Word Note }: しかしか is the modern equivalent of そうそう. }

\par{5. ${\overset{\textnormal{むき}}{\text{無期}}}$ ののちに「 \textbf{えい }」といらへたりければ \hfill\break
Since he responded "yes" after an indefinitely long time \hfill\break
From the 宇治拾遺物語. }
 
\par{\textbf{Word Note }: えい can still be used in this fashion. However, it is normally spelled as ええ because えい is used as an interjection when you're doing something with a lot of resolution. }

\par{6. \textbf{あな }いみじのおもとたちや。 \hfill\break
Ah! What wonderful women. \hfill\break
From the 枕草子. }

\par{7. \textbf{これ }乗せて行け、具して行け。 \hfill\break
Hey, go pick them up and take them along! \hfill\break
From the 平家物語. }
 
\par{\textbf{Word Note }: これ can still be used in this same fashion. }

\par{8. \textbf{やや }、もの ${\overset{\textnormal{まう}}{\text{申}}}$ さむ。 \hfill\break
Hello, I have something that I would like to speak about. \hfill\break
From the 大鏡. }
 
\par{\textbf{Word Note }: やや is like today's もしもし. }

\par{9. \textbf{すは }、しつることを。 \hfill\break
There, we did it! \hfill\break
From the 平家物語. }

\par{10. \textbf{あつばれ }、よからうかたきがな。 \hfill\break
Ah, it would be nice if there were a good opponent. \hfill\break
From the 平家物語. }

\par{11. \textbf{いかに }夢かうつつか。 \hfill\break
Well, is this a dream or is this reality? \hfill\break
From the 平家物語. }

\par{12. \textbf{いで }、いと ${\overset{\textnormal{きやう}}{\text{興}}}$ ある事いふ ${\overset{\textnormal{らうしや}}{\text{老者}}}$ かな。 \hfill\break
Well, aren't you old men that say such interesting things! \hfill\break
From the 大鏡. }

\par{13. \textbf{あはや }と目をかけて飛んで掛かるに \hfill\break
Wow, just when I had my eyes on it and sprang\dothyp{}\dothyp{}\dothyp{} \hfill\break
From the 平家物語. }
 
\par{\textbf{Word Note }: This interjection survives as the adverb あわやと which can be found in expressions like "あわやというところで" meaning "just in time". }
 
\par{14. 「 \textbf{さはれ }、道にても」などといひて、みな乗りぬ。 \hfill\break
He said something like, "Still, but even in the road (isn't it fine)?", and everyone ended up riding. \hfill\break
From the 枕草子. }

\par{15. \textbf{すはや }宮こそ ${\overset{\textnormal{なんと}}{\text{南都}}}$ へ落ちさせ給ふなれ。 \hfill\break
There! It looks like the palace is to flee to the southern capital (Nara). \hfill\break
From the 平家物語. }

\par{16. \textbf{まことや }、 ${\overset{\textnormal{ほふりん}}{\text{法輪}}}$ は近ければ \hfill\break
Yes that's so, and since Hourin (Temple) is close, \hfill\break
From the 平家物語. }

\par{17. \textbf{なんでふ }、さやうのあそび者は、人の ${\overset{\textnormal{め}}{\text{召}}}$ ししたがうてこそ ${\overset{\textnormal{まゐ}}{\text{参}}}$ れ。 \hfill\break
No kidding, that prostitute is called upon by people and comes. \hfill\break
From the 平家物語. }
      
\section{Exercises}
 
\par{1. What is an interjection? }

\par{2. いさ = いいえ. Translate the following into English or Modern Japanese. }

\par{いさ、人のにくしとおもひたりしが }

\par{3. を was an interjection meaning "はい". Make a sentence with it. }

\par{4. あなめでたや = ? }

\par{5. いざ is in Modern Japanese too. What does it mean, and how is it used? }
    