    
\chapter{The Auxiliary Verb ~ず II}

\begin{center}
\begin{Large}
第388課: The Auxiliary Verb ~ず II 
\end{Large}
\end{center}
  ~ず is still used infrequently in Modern Japanese, howbeit typically in archaic expressions and set phrases. In this lesson, we will finally learn all about the auxiliary verb ~ず so that we can better understand Classical Japanese!       
\section{The Auxiliary Verb -ず}
 
\par{ The auxiliary verb ~ず follows the 未然形 to show the negative. This is the \emph{easiest }part about it. ~ず has \textbf{three sets of bases }. This is the \emph{hardest }part about it. Throughout this lesson we'll study ~ず by using the chart below. The base sets will be referred to by their respective 連用形. }

\begin{ltabulary}{|P|P|P|P|}
\hline 

未然形 & な & X & ざら \\ \cline{1-4}

連用形 & に & ず & ざり \\ \cline{1-4}

終止形 & X & ず & X \\ \cline{1-4}

連体形 & ぬ & X & ざる \\ \cline{1-4}

已然形 & ね & X & ざれ \\ \cline{1-4}

命令形 & X & X & ざれ \\ \cline{1-4}

\end{ltabulary}

\par{ The first base sequence of the negative auxiliary verb was な-, に-, X, ぬ-, ね-, and X. The な-未然形 and the に-連用形 faded away and the 連用形 was replaced with ~ず. This in turn became the 終止形. A third set of bases was created with the fusion of ~ず with あり. The base sets are interchangeable. For example, ~ぬ and ~ざる are both common. }

\par{ Temporally speaking, although ず ultimately comes from にす, Old Japanese already had ず as the終止形. The third column most definitely appears later on in Middle Japanese. }
      
\section{The に- Bases}
 
\par{ The な-未然形 was used in the speech modal ~なくに, which meant "because\dothyp{}\dothyp{}\dothyp{}isn't". It was also a 語尾 meaning "how\dothyp{}\dothyp{}\dothyp{}isn't?". }
 
\par{1. み ${\overset{\textnormal{}}{\text{山}}}$ には ${\overset{\textnormal{まつ}}{\text{松}}}$ の ${\overset{\textnormal{}}{\text{雪}}}$ だに ${\overset{\textnormal{き}}{\text{消}}}$ えなくに ${\overset{\textnormal{みやこ}}{\text{都}}}$ は ${\overset{\textnormal{のべ}}{\text{野辺}}}$ の ${\overset{\textnormal{わかなつ}}{\text{若菜摘}}}$ みけり。 \hfill\break
As even the snow on the pine trees hasn't disappeared in the mountains, the (people in the) capital        gathered young greens from the fields! }

\par{From the 古今和歌集. }
 
\par{\textbf{Nuance Note }: The prefix み- in this poem is used as a beautifier. }
 
\par{The に-連用形 was used in the \textbf{Nara Period }(奈良時代) in the same way as the modern ~なくて. }
 
\par{${\overset{\textnormal{}}{\text{2.}}}$ 去方乎不知 舎人者迷惑 \hfill\break
${\overset{\textnormal{}}{\text{行}}}$ くへを ${\overset{\textnormal{}}{\text{知}}}$ らに ${\overset{\textnormal{とねり}}{\text{舎人}}}$ はまどふ \hfill\break
Not knowing the way to go, the low-ranking officials working for the nobility wander about. \hfill\break
From the 万葉集. }
 
\par{\textbf{Culture Note }: A とねり was a low-ranking official who worked for the nobility. }
 
\par{3. たづきを知らに \hfill\break
Not knowing the livelihood }
 
\par{The 連体形 and the 已然形 weren't lost like the 未然形 and 連用形. }
 
\par{${\overset{\textnormal{}}{\text{4. 人}}}$ こそ ${\overset{\textnormal{}}{\text{知}}}$ らね。 \hfill\break
Though people don't know. }
 
\par{\textbf{Grammar Note }: The particle こそ has the following verb be in the 已然形 in Classical Japanese. }

\par{5. ${\overset{\textnormal{みやこ}}{\text{京}}}$ には ${\overset{\textnormal{}}{\text{見}}}$ えぬ ${\overset{\textnormal{}}{\text{鳥}}}$ なれば \hfill\break
Since it was a bird not seen in the capital \hfill\break
From the 伊勢物語. }
 
\par{\textbf{Grammar Note }: The 連体形 is used because the verb ${\overset{\textnormal{}}{\text{見}}}$ ゆ, the Classical form of 見える, is used as negative participle. }

\par{6. ${\overset{\textnormal{やえむぐらしげ}}{\text{八重葎茂}}}$ れる ${\overset{\textnormal{やど}}{\text{宿}}}$ の ${\overset{\textnormal{さび}}{\text{寂}}}$ しきに ${\overset{\textnormal{}}{\text{人}}}$ こそ ${\overset{\textnormal{}}{\text{見}}}$ えね ${\overset{\textnormal{}}{\text{秋}}}$ は ${\overset{\textnormal{}}{\text{来}}}$ にけり \hfill\break
Although you could see no person in the loneliness of the lodge thickened with mixed weeds, autumn       ended up coming. \hfill\break
From the 新日本古典文学大系. }
 
\par{\textbf{Grammar Note }: This sentence is in the 連体形 because of the particle こそ. Don't mind other Classical grammar items. }
 
\par{7. いとやむごとなき ${\overset{\textnormal{きは}}{\text{際}}}$ にはあらぬがすぐれて ${\overset{\textnormal{}}{\text{時}}}$ めき ${\overset{\textnormal{たま}}{\text{給}}}$ ふありけり \hfill\break
The person didn't have very high regards in status and found favor in particular. \hfill\break
From the 源氏物語. }
 
\par{\textbf{Grammar Note }: The 連体形 is used with conjunctive particles such as が. }
 
\par{${\overset{\textnormal{}}{\text{8. 人}}}$ の ${\overset{\textnormal{}}{\text{歌}}}$ の ${\overset{\textnormal{かえ}}{\text{返}}}$ しとすべきをえ ${\overset{\textnormal{よ}}{\text{詠}}}$ み ${\overset{\textnormal{え}}{\text{得}}}$ ぬほども ${\overset{\textnormal{こころもと}}{\text{心許}}}$ なし \hfill\break
You should respond quickly to a person's poem, and when you're unable to write one, it (the                    situation) is precarious. \hfill\break
From the 枕草子. }
 
\par{\textbf{Grammar Note }: え\dothyp{}\dothyp{}\dothyp{}-ず is the negative potential. The 連体形, 読み得ぬ, modifies the noun ほど. The particle を in this sentence is interchangeable with the particle が. }

\par{9. ${\overset{\textnormal{ほふし}}{\text{法師}}}$ ばかり ${\overset{\textnormal{うらや}}{\text{羨}}}$ ましからぬものはあらじ \hfill\break
Nothing\slash none is probably as unenviable as a priest. \hfill\break
From the 徒然草. }
      
\section{The ず- Bases}
 
\par{ How did ~ず come from n-sounds? The incomplete first column of bases is the primary reason for the evolution that brought forth three sets of bases. To make a 終止形, the 連用形 combined with す, the Classical する, to create ~にす which became ~んす which became ~んず which became ~ず. ~ず's 連用形 and 終止形 are still very important in Modern Japanese grammar. }
 
\par{${\overset{\textnormal{}}{\text{10. 連絡}}}$ が ${\overset{\textnormal{}}{\text{取}}}$ れず、 ${\overset{\textnormal{}}{\text{心配}}}$ しました。 \hfill\break
Without having contact, I got worried. }
 
\par{\textbf{Grammar Note }: The above sentence uses ~ず in the 連用形. }
 
\par{${\overset{\textnormal{}}{\text{11. 分}}}$ らず ${\overset{\textnormal{}}{\text{屋 (Set Phrase)}}}$ \hfill\break
An obstinate person. }

\par{12. ${\overset{\textnormal{}}{\text{君}}}$ や ${\overset{\textnormal{こ}}{\text{来}}}$ し ${\overset{\textnormal{}}{\text{行}}}$ きけむ ${\overset{\textnormal{}}{\text{思}}}$ ほえず \hfill\break
Did you come? Did I even go? I can't think naturally. \hfill\break
From the 伊勢物語. }

\par{13. ${\overset{\textnormal{しんと}}{\text{新都}}}$ はいまだ ${\overset{\textnormal{な}}{\text{成}}}$ らず。 \hfill\break
The new capital has still not materialized. \hfill\break
From the 方丈記. }

\par{14. ${\overset{\textnormal{じひ}}{\text{慈悲}}}$ の ${\overset{\textnormal{}}{\text{心}}}$ なからむは ${\overset{\textnormal{じんりん}}{\text{人倫}}}$ に ${\overset{\textnormal{あら}}{\text{非}}}$ ず \hfill\break
If someone didn't have a heart of compassion, he wouldn't be human. \hfill\break
From the 徒然草. }
 
\par{\textbf{Grammar Note }: ~ず follows the verb あり (ある in Modern Japanese) normally. }

\par{15. ${\overset{\textnormal{あ}}{\text{飽}}}$ かずして ${\overset{\textnormal{わか}}{\text{別}}}$ るる ${\overset{\textnormal{そで}}{\text{袖}}}$ の ${\overset{\textnormal{しらたま}}{\text{白玉}}}$ \hfill\break
 The white jewel of your sleeve which parted from you despite being unsatisfied. \hfill\break
From the 古今和歌集. }

\par{16. ${\overset{\textnormal{あくしよ}}{\text{悪所}}}$ におちて ${\overset{\textnormal{し}}{\text{死}}}$ にたからず \hfill\break
I don't want to die falling into a dangerous place. \hfill\break
From the 平家物語. }
 
\par{17. つゆまどろまれず \hfill\break
I couldn't doze off at all. \hfill\break
From the 日本古典文学大系. }
 
\par{${\overset{\textnormal{}}{\text{18. 昔}}}$ のごとくにもあらず \hfill\break
It's not even like it was in the past. \hfill\break
From the 大和物語. }

\par{19. ゆく ${\overset{\textnormal{}}{\text{川}}}$ の ${\overset{\textnormal{}}{\text{流}}}$ れは ${\overset{\textnormal{た}}{\text{絶}}}$ えずしてしかももとの ${\overset{\textnormal{みづ}}{\text{水}}}$ にあらず \hfill\break
The flowing of the passing water is endless at the same; moreover, it is not the original water. \hfill\break
From the 徒然草. }
 
\par{${\overset{\textnormal{}}{\text{20. 住}}}$ まずして ${\overset{\textnormal{たれ}}{\text{誰}}}$ かさとらむ \hfill\break
Without living there, who would understand (the pleasure)? \hfill\break
From the 大和物語. }

\par{21. ${\overset{\textnormal{あ}}{\text{逢}}}$ はずして ${\overset{\textnormal{こよひあ}}{\text{今宵明}}}$ けなば ${\overset{\textnormal{}}{\text{春}}}$ の ${\overset{\textnormal{}}{\text{日}}}$ の ${\overset{\textnormal{}}{\text{長}}}$ くや ${\overset{\textnormal{}}{\text{人}}}$ をつらしと ${\overset{\textnormal{}}{\text{思}}}$ はん \hfill\break
Without meeting, and then the evening ending up dawn, I will surely think of you as long as a spring         day as being cruel. \hfill\break
From the 古今和歌集. }
      
\section{The ざり- Bases}
 
\par{ The ざり- bases came from the fusion of ず- and the supplementary verb あり. The ざり- bases are interchangeable with the other sets but become prevalent in later Classical texts. }
 
\par{At the point of inception, the first and second base sets were essential one as "X, ず-, ず, ぬ, ね-, X". Both 連用形 were acceptable with ざり- being preferred. The 連体形 were completely interchangeable. As for the 已然形, ね- was preferred when used with bound particles and ざれ- was preferred when used with an ending. }
 
\par{In the 室町時代 the ぬ-連体形 became an interchangeable variant of the 終止形. ~ぬ eventually replaced ~ず as the negative auxiliary verb and evolved into ~ん in the West and into ~ない in the East. When this happened, all of the bases were scrubbed for 形容詞 bases. }
 
\begin{center}
\textbf{Examples } 
\end{center}

\par{22. つひにゆく ${\overset{\textnormal{}}{\text{道}}}$ とはかねて ${\overset{\textnormal{}}{\text{聞}}}$ きしかど ${\overset{\textnormal{きのふけふ}}{\text{昨日今日}}}$ とは ${\overset{\textnormal{}}{\text{思}}}$ はざりしを。 \hfill\break
I had heard before of the path we will eventually go, but I didn't think that it would be yesterday or             today! \hfill\break
From the 古今和歌集. }
 
\par{${\overset{\textnormal{}}{\text{23. 絶}}}$ えざる ${\overset{\textnormal{}}{\text{不安}}}$ \hfill\break
Anxiety that won't cease }
 
\par{${\overset{\textnormal{}}{\text{24. 何}}}$ も ${\overset{\textnormal{}}{\text{書}}}$ かん。 \hfill\break
I won't write anything. }

\par{25. ${\overset{\textnormal{ゆめ}}{\text{夢}}}$ と ${\overset{\textnormal{し}}{\text{知}}}$ りせば ${\overset{\textnormal{さ}}{\text{覚}}}$ めざらましを \hfill\break
If I had known that it was a dream, I would've never woken up. \hfill\break
From the 古今和歌集. }
 
\par{${\overset{\textnormal{}}{\text{26. 山高}}}$ きと ${\overset{\textnormal{}}{\text{知}}}$ らず。 \hfill\break
To not know the mountain is tall. }
 
\par{${\overset{\textnormal{}}{\text{27. 怖}}}$ いものもあらず。 \hfill\break
To not even have things one is scared of. }

\par{28. 自らの信条を吐露せざれ。 \hfill\break
Express one's beliefs. }
      
\section{Speech Modals with ~ず}
 
\par{\textbf{~ずに \& ~ずにはいられない }}

\par{ ~ずに and ~ずにはいられない mean "without" and "cannot help but\dothyp{}\dothyp{}\dothyp{}" respectively and are formed with the ず-連用形. }
 
\par{${\overset{\textnormal{}}{\text{29. 勉強}}}$ せずに、 ${\overset{\textnormal{}}{\text{受験}}}$ したから、 ${\overset{\textnormal{}}{\text{落第}}}$ してしまった。 \hfill\break
Because I took the exam without studying, I ended up failing it. }
 
\par{\textbf{Grammar Note }: You must use the せ-未然形 of する for all Classical Japanese auxiliary verbs. }
 
\par{${\overset{\textnormal{}}{\text{29. 心配}}}$ せずにはいられない。 \hfill\break
I couldn't help but worry. }

\par{30. ${\overset{\textnormal{こ}}{\text{懲}}}$ らしめてやらずにはおかない。 \hfill\break
I couldn't help but give him punishment. }

\par{31. 霞立 長春日乃 晩家流 和豆肝之良受 \hfill\break
${\overset{\textnormal{かすみた}}{\text{霞立}}}$ つ ${\overset{\textnormal{}}{\text{長}}}$ き ${\overset{\textnormal{はるひ}}{\text{春日}}}$ の ${\overset{\textnormal{く}}{\text{暮}}}$ れにける わづきも ${\overset{\textnormal{}}{\text{知}}}$ らず \hfill\break
I also don't know of the moon setting in the overshadowing long spring day. \hfill\break
From the 万葉集. }

\par{32. ${\overset{\textnormal{おこ}}{\text{怒}}}$ らせずにはすまないでしょう。 \hfill\break
You probably can't help but get angry, right? }
 
\par{${\overset{\textnormal{}}{\text{33. 泣}}}$ き ${\overset{\textnormal{}}{\text{出}}}$ さずにはすまさない。 \hfill\break
I cannot help but cry profusely. }
 
\par{\textbf{Variation Note }: ~ずにはいられない variants include ~ずにはすまない, ~ずにはおかない, and ~ずにはすまさない. }
 
\par{\textbf{~ }\textbf{ずは・ずば・ずんば }}
 
\par{ The main usage of ~ずは is to show negative hypothesis and is equivalent to しないなら in Modern Japanese. So, ~ずは means "if\dothyp{}\dothyp{}\dothyp{}had not". In the late 奈良時代 it was also used like ~ずに to mean "without\dothyp{}\dothyp{}\dothyp{}". In the medieval era ~ずは was seen as ~ず(ん)ば. }

\par{34. ${\overset{\textnormal{こけつ}}{\text{虎穴}}}$ に ${\overset{\textnormal{い}}{\text{入}}}$ らずんば ${\overset{\textnormal{こし}}{\text{虎子}}}$ を ${\overset{\textnormal{え}}{\text{得}}}$ ず。 \hfill\break
Nothing ventured, nothing gained. \hfill\break
Literally: Without entering the tiger's den, you won't get the tiger's cub. }

\par{35. ${\overset{\textnormal{こうふく}}{\text{降伏}}}$ せずば、 ${\overset{\textnormal{}}{\text{命}}}$ はない。 \hfill\break
If you don't surrender, you will not live. }

\par{36. ${\overset{\textnormal{きじ}}{\text{雉}}}$ も ${\overset{\textnormal{}}{\text{鳴}}}$ かずば ${\overset{\textnormal{う}}{\text{撃}}}$ たれまい。 \hfill\break
If the green pheasant doesn't cry, let's not shoot. }
 
\par{37. まろ ${\overset{\textnormal{かうし}}{\text{格子上}}}$ げずば ${\overset{\textnormal{}}{\text{道}}}$ なくてげにえ ${\overset{\textnormal{い}}{\text{入}}}$ り ${\overset{\textnormal{こ}}{\text{來}}}$ ざらまし \hfill\break
If I hadn't raised the lattice, there would have been no path and it would have truly not have been             able to come in. \hfill\break
From the 源氏物語. }
 
\par{${\overset{\textnormal{}}{\text{38. 行}}}$ かずばなるまい。 \hfill\break
We have no choice but to go. \hfill\break
 \hfill\break
\textbf{Grammar Note }: ~ずばなるまい is the predecessor of ~なければならない. }
 
\par{${\overset{\textnormal{}}{\text{39. 火}}}$ に燒かむに燒けずはこそ ${\overset{\textnormal{まこと}}{\text{真}}}$ ならめ \hfill\break
When we try to burn it by fire and it cannot burn, it's definitely the real thing. \hfill\break
From the 竹取物語. }
 
\par{\textbf{~ }\textbf{ずて }}
 
\par{The speech modal ~ずて means "not\dothyp{}\dothyp{}\dothyp{}" and is equivalent to ~ないで. }
 
\par{40. 麻都我延乃 都知尓都久麻掾 布流由伎乎 美受弖也伊毛我 許母里乎流良牟 \hfill\break
 ${\overset{\textnormal{}}{\text{松}}}$ が ${\overset{\textnormal{}}{\text{枝}}}$ の ${\overset{\textnormal{}}{\text{土}}}$ に ${\overset{\textnormal{}}{\text{着}}}$ くまで ${\overset{\textnormal{}}{\text{降}}}$ る ${\overset{\textnormal{}}{\text{雪}}}$ を ${\overset{\textnormal{}}{\text{見}}}$ ずてや ${\overset{\textnormal{いも}}{\text{妹}}}$ が ${\overset{\textnormal{こも}}{\text{隠}}}$ り ${\overset{\textnormal{を}}{\text{居}}}$ るらむ \hfill\break
How could you not see the snow that falls up to the earth of the pine tree branches and stay indoors       dear? \hfill\break
From the 万葉集. }
 
\par{41. 比等未奈能 美良武麻都良能 多麻志末乎 美受弖夜和礼波 故飛都々遠良武 \hfill\break
 ${\overset{\textnormal{}}{\text{人皆}}}$ の ${\overset{\textnormal{}}{\text{見}}}$ らむ ${\overset{\textnormal{}}{\text{松浦}}}$ の ${\overset{\textnormal{}}{\text{玉島}}}$ を ${\overset{\textnormal{}}{\text{見}}}$ ずてや ${\overset{\textnormal{われ}}{\text{我}}}$ は ${\overset{\textnormal{こ}}{\text{恋}}}$ ひつつ ${\overset{\textnormal{を}}{\text{居}}}$ らむ \hfill\break
Not looking at Matsu'ura no Tamashima that everyone is said to see, I was homesick. \hfill\break
From the 万葉集. }

\par{------------------------------------------------------ }

\par{Next Lesson \textrightarrow  第142課: The Auxiliary Verbs -き \& -けり  }

\par{------------------------------------------------------ }

\par{Next Lesson \textrightarrow  第142課: The Auxiliary Verbs -き \& -けり  }

\par{------------------------------------------------------ }

\par{Next Lesson \textrightarrow  第142課: The Auxiliary Verbs -き \& -けり  }

\par{------------------------------------------------------ }

\par{Next Lesson \textrightarrow  第142課: The Auxiliary Verbs -き \& -けり  }
      
\section{Exercises}
 
\par{1. Illustrate the bases of -ず. }

\par{2. What base does -ず follow? }

\par{3.  Create a sentence using the first column of bases of -ず. Use the example sentences in this lesson to model your sentence. }

\par{4. How did -ず come about? }

\par{5. Create a sentence with -ず being in the 連用形 or 終止形. }

\par{6. Create a sentence with a speech modal utilizing -ず. }

\par{7. Create a sentence using the third column of bases of -ず. }

\par{8. Create a sentence using -ぬ, the evolved form of -ず. }

\par{9. Create a sentence with -ずんば. }
    