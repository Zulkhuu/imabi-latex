    
\chapter{The Copula Verbs にあり(なり) \& とあり(たり)}

\begin{center}
\begin{Large}
第380課: The Copula Verbs にあり(なり) \& とあり(たり) 
\end{Large}
\end{center}
 
\par{ The copula なり may be archaic, but this hasn't stopped anyone from using it from time to time. You will also see another copula verb, たり, that was used to follow nominals. Also, unlike the modern である, all of the bases are used for the copula verb \textbf{s }in Classical Japanese. }
      
\section{にあり(なり)}
 
\par{ The copula verb なり comes from the fusion of the case particle に and the ラ-変 irregular verb あり "to be". The bases of the copula verb as as follows. }

\begin{ltabulary}{|P|P|P|P|P|P|}
\hline 

未然形 \hfill\break
& 連用形 & 終止形 & 連体形 & 已然形 & 命令形 \\ \cline{1-6}

なら- & なり・に- & なり & なる & なれ- & なれ \\ \cline{1-6}

\end{ltabulary}

\par{Before we learn about the differences between the two 連用形 bases, we will take a look at the six bases of conjugation. }

\par{\textbf{未然形 }: The 未然形 is the " \emph{irrealis form }" and is associated with endings that indicate actions that have \textbf{not }yet taken place: negation, desire, and hypothesis. }

\par{\textbf{連用形 }: The 連用形 is the " \emph{continuative form }" and is used with endings that indicate actions that are \textbf{in the process }of being carried out and the verb is either taken or taking place. This is why the 連用形 is used with conjunctive particles such as て and auxiliary verbs that express tense and politeness. }

\par{\textbf{終止形 }: The 終止形 is the " \emph{terminal form }" and is used to mark the end of a complete sentence. Actions are in the present if and only if there is not a temporal noun that tells otherwise . The 終止形 may still be followed by final and interjectory particles. }

\par{\textbf{連体形 }:  The 連体形 is the " \emph{attributive form }" is used when you want to use something as a \textbf{participial }when modifying a nominal phrase with an inflectional part of speech. So, as all endings have a 連体形, any state of time and purpose is possible. It is also used as the nominalized form of a verb. }

\par{\textbf{已然形 }: The 已然形 is the " \emph{hypothetical form }" but is viewed as the "perfective form" in Classical Japanese because it is with conjunctive particles such as ば to show reason. Thus, unlike now the 已然形 typically describes actions that have \textbf{already occurred }. }

\par{\textbf{命令形 }: The 命令形 is the " \emph{imperative\slash command form }": this is pretty much explanatory. }

\par{\textbf{Using なり }}

\par{ なり's main usage is making declarative sentences. In the なる-連体形 it either shows location or name. When showing location, it indicates the place or direction that something is "located" at. This equivalent to にある. When showing name, it is equivalent to the Modern という. This last usage appeared in the 漢文訓読 style of writing during the 江戸時代. }

\par{1. 吉野尓有 夏實之河乃 川余杼尓 鴨曾鳴成 山影尓之弖 \hfill\break
吉野にある 夏実の ${\overset{\textnormal{かは}}{\text{河}}}$ の ${\overset{\textnormal{かはよど}}{\text{川淀}}}$ に 鴨そ鳴くなる  山影にして \hfill\break
In the river pool of the Natsumi River in Yoshino, you can hear ducks cry in the shadowy mountains. \hfill\break
From the 万葉集. }
 
\par{\textbf{Orthography Note }: Notice in the 万葉仮名 that basic words such as 山影, 鴨, and 川 are written semantically whereas almost everything else is written phonetically. }
 
\par{2. この 西なる ${\overset{\textnormal{いへ}}{\text{家}}}$ には ${\overset{\textnormal{なにびと}}{\text{何人}}}$ の住むぞ。 \hfill\break
What kind of person lives there as for this house located in the west? \hfill\break
From the 源氏物語. }
 
\par{\textbf{Morphology Notes }: この is not viewed as a single word in Classical Japanese. The particle ぞ is used as a \textbf{rhetorical }final particle. }
 
\par{ The なり-連用形 is typically used, but there are two major instances where it isn't. With the particle て, に is used. }

\par{3. ${\overset{\textnormal{たれ}}{\text{誰}}}$ もいまだ都なれぬほどにて え見つけず。 \hfill\break
It was a time when no one was still used to the capital and none could find it. \hfill\break
From the 更級日記. }
 
\par{\textbf{Grammar Notes }: }

\par{1. なれぬ is the verb なれる "to get used to" with ~ず in the ぬ-連体形. \hfill\break
2. なり is again seen in the に-連用形 with て to show parallelism and is then followed by the pattern for negative potential in Classical Japanese, え\dothyp{}\dothyp{}\dothyp{}-ず. }
 
\par{4. 父はなほ人にて 母なん ${\overset{\textnormal{ふぢはら}}{\text{藤原}}}$ なりける。 \hfill\break
The father was a normal person, and the mother was a Fujiwara. \hfill\break
From the 伊勢物語. }
 
\par{\textbf{Grammar Notes }: }

\par{1. In this sentence, にて is the に-連用形 of なり with て, and なりける is the なり-連用形 of なり with the recollection auxiliary verb けり in the ける-連体形. \hfill\break
2. Although sentences should end in the 終止形, there has always been a trend to end with the 連体形 for special emphasis which eventually lead to the fusion of the two bases. }
 
\par{ Unlike Modern Japanese, conjugation is not as straightforward, and there are more auxiliary verbs. The primary thing that you should notice in the example sentences in this lesson is how the copula verbs are used in terms of \textbf{meaning }and \textbf{bases }and what they are used with. With that in mind, consider the following additional examples. }

\par{5. 天地之 分時従 神左備手 高貴寸 駿河有 布士能高嶺乎 天原 振放見者 度日之 陰毛隠比 照月乃 光毛不見 ${\overset{\textnormal{あめのつち}}{\text{天地}}}$ の別れし時ゆ ${\overset{\textnormal{かみ}}{\text{神}}}$ さびて 高く ${\overset{\textnormal{たふと}}{\text{貴}}}$ き ${\overset{\textnormal{するが}}{\text{駿河}}}$ なる ${\overset{\textnormal{ふじ}}{\text{富士}}}$ の ${\overset{\textnormal{たかね}}{\text{高嶺}}}$ を ${\overset{\textnormal{あま}}{\text{天}}}$ の原振り ${\overset{\textnormal{さ}}{\text{放}}}$ け見れば 渡る日の影も隠らひ \hfill\break
照る月の光も見えず \hfill\break
Looking afar into the wide sky at Mount Fuji of Suruga, which has been a highly exalted and divine           peak since the division of heaven and earth, you cannot even see the shining moon light for even the     shadow of the passing sun is hidden. \hfill\break
From the 万葉集. }
 
\par{\textbf{Word Notes }: }

\par{1. There is no Sino-Japanese word in this poem. Although read as てんち today, 天地 was read with 訓読み as あめつち. \hfill\break
2. The particle ゆ is an old case particle meaning "from". \hfill\break
3. 天の原 is a 雅語. \hfill\break
4. 振りさく is a somewhat archaic verb equivalent to 遥かに仰ぐ. \hfill\break
5. The verb 隠らふ, 未然形 of 隠る (the Classical form of 隠れる) + the auxiliary verb -ふ, shows that the place is repetitively covered by clouds. }
 
\par{6. 我が敵は我にあり。 \hfill\break
My enemy is myself. }
 
\par{7. たましきの ${\overset{\textnormal{みやこ}}{\text{都}}}$ のうちに ${\overset{\textnormal{むね}}{\text{棟}}}$ を ${\overset{\textnormal{なら}}{\text{並}}}$ べ ${\overset{\textnormal{いらか}}{\text{甍}}}$ を ${\overset{\textnormal{あらそ}}{\text{爭}}}$ へる高き ${\overset{\textnormal{いや}}{\text{賤}}}$ しき人のすまひは ${\overset{\textnormal{よよ}}{\text{世々}}}$ を ${\overset{\textnormal{へ}}{\text{經}}}$ て ${\overset{\textnormal{つ}}{\text{盡}}}$ きせぬものなれど  これをまことかと ${\overset{\textnormal{たづ}}{\text{尋}}}$ ぬれば 昔ありし ${\overset{\textnormal{いへ}}{\text{家}}}$ は稀なり。 \hfill\break
In the jewel-strewn capital the houses lined up with the abodes of the noble and base competing with their roof tiles pass the generations everlastingly; however, when you ask if this is true, houses (that appear) from long ago are (actually) rare. \hfill\break
From the 方丈記 }
 
\par{\textbf{Word Note }: たましき is an epithet word that embellishes 都 which is used here to refer to 京都. }

\par{\textbf{Particle Note }: なれど is the 已然形 of なり with the conjunctive particle ど. }
 
\par{\textbf{Historical Note }: なり became だ as such. にて gave rise to で by first dropping the vowel "i", which led to voicing of で. As voiced sounds were believed to have had pre-nasalization, it would have been treated as a single mora. Thus, で was born. Once あり changed to ある, である was formed. Then, である contracted to だ in the East and じゃ in the West. When 東京 became the capital, 東京弁 became standard and so did だ. }

\par{ As for です, etymology is a little unclear. It's clear that the で must have derived from にて, but the latter element has been interpreted in two primary ways. One theory of thought suggests that the predecessor of です was でございます, which is easy for Modern Japanese speakers and learners alike to understand. If one knew that でござんす was used a lot in the Edo Period in East Japan and that dialects in this region of Japan are known for truncating long phrases drastically, this relation is quite logical. }

\par{ Another theory proposes that the predecessor is actually で候. 候, which is rendered as そうろう in Modern Orthography and さうらふ in Old Orthography, was a formal auxiliary used primarily in texts now understandably called 候文. There are truncated forms of 候 such as そろう. However, there needs to be a reason for why it is further truncated and why a vowel change happened after the truncation. Vowel changes on the individual word level are not uncommon in Japanese dialects. So, this is not impossible. However, the fact that です has always been associated with the "spoken language" makes the connection implausible from a social linguistic standpoint. }
      
\section{The Copula Verb たり}
 
\par{ The copula たり, from the particle と + あり, appeared in the 平安時代 and was often used to emphasize the qualities of someone. It originally appeared in 漢文 and eventually become more influential in 軍記物 (warrior narratives), 説話 (anecdotes), and 和漢混淆 (Japanese-Chinese mixed style). The bases of たり are: }

\begin{ltabulary}{|P|P|P|P|P|P|}
\hline 

未然形 & 連用形 & 終止形 & 連体形 & 已然形 & 命令形 \\ \cline{1-6}

たら- & たり・と- & たり & たる & たれ- & たれ \\ \cline{1-6}

\end{ltabulary}

\par{The 連用形 to is seen when used to make adverbial phrases. This is still this case in Modern Japanese with words such as 堂々と (magnificently). }

\par{Like なり, たり may be used for showing declaration. There are also four important patterns that involve たり when it is used in the 連体形. }

\begin{ltabulary}{|P|P|P|}
\hline 

\dothyp{}\dothyp{}\dothyp{}たると\dothyp{}\dothyp{}\dothyp{}たるとを問(と)はず \hfill\break
& No matter which regardless of & Like どちらであっても関係(かんけい)なく in Modern Japanese. \hfill\break
\\ \cline{1-3}

\dothyp{}\dothyp{}\dothyp{}.たるや \hfill\break
& Speaking of\dothyp{}\dothyp{}\dothyp{} \hfill\break
& Strongly emphasizes something \hfill\break
\\ \cline{1-3}

\dothyp{}\dothyp{}\dothyp{}.たる者(もの) \hfill\break
& those who are (in the capacity of) & Describes someone with the exquisite qualities of something regardless of how long he or she was as such. \\ \cline{1-3}

何(なん)たる & What & Either used in astonishment or just as 何である. \hfill\break
\\ \cline{1-3}

\end{ltabulary}
  
\begin{center}
\textbf{Examples } 
\end{center}

\par{8. ${\overset{\textnormal{けうし}}{\text{教師}}}$ たる者が ${\overset{\textnormal{わいろ}}{\text{賄賂}}}$ を ${\overset{\textnormal{えうきう}}{\text{要求}}}$ するとは! \hfill\break
A person in the capacity of a teacher seeking a bribe! }

\par{9. 団体、または、公人たると私人たるとを問わず ${\overset{\textnormal{じゆんしゆ}}{\text{遵守}}}$ すべき。 \hfill\break
(This) body, again, regardless of whether you are an officeholder or a private citizen, you should              comply. \hfill\break
 \hfill\break
 \textbf{Modern Application Note }: This example is an example how Classical Japanese grammar can still be applied today. So, the second line for かな shows it in 歴史仮名遣い simply for exposure. }

\par{10. その姿たるや、実に悲惨なものだった。 \hfill\break
Speaking of the appearance (of it), it truly was a pitiful thing. }

\par{11. 君たれども ${\overset{\textnormal{おみ}}{\text{臣}}}$ たれども、だかひにこころざし深く ${\overset{\textnormal{へだ}}{\text{隔}}}$ つる思ひのなきは \hfill\break
Whether you are the lord or the vassal, if it were not feelings that mutually one's motives are deeply         distant \hfill\break
From the 十訓抄 (The Miscellany of Ten Maxims) \hfill\break
 \hfill\break
 \textbf{Nuance Change Note }: 君 in Classical Japanese means "lord" and not the informal you as it does today. The opposite of 君 is 臣which is an old term that is equivalent to the modern ${\overset{\textnormal{かしん}}{\text{家臣}}}$ , which is less degrading. }

\par{12. しかるを ${\overset{\textnormal{ただもり}}{\text{忠盛}}}$ ${\overset{\textnormal{びぜんかみ}}{\text{備前守}}}$ たりし時 \hfill\break
Well, when Tadamori was the Governor of Bizen \hfill\break
From the 平家物語. \hfill\break
 \hfill\break
 \textbf{Name Note }: 備前 was a former province of Japan and is now a part of Okayama Prefecture. \hfill\break
 \hfill\break
13. その ${\overset{\textnormal{ゆゑ}}{\text{故}}}$ はいにしへ清盛公 未だ安芸守たりしとき \hfill\break
The reason is that long ago when Lord Kiyomori was still the Governor of Aki \hfill\break
From the 平家物語. \hfill\break
 \hfill\break
 \textbf{Name Note }: 安芸 is the former name of the 広島県. \hfill\break
 \hfill\break
14. かの ${\overset{\textnormal{りうかいりつし}}{\text{隆海律師}}}$ の ${\overset{\textnormal{いを}}{\text{魚}}}$ つりの ${\overset{\textnormal{わらは}}{\text{童}}}$ とありけるとき \hfill\break
When the Buddhist Priest Ryuukai was a child working in fishery \hfill\break
From the 今昔物語. }

\par{15. ${\overset{\textnormal{いつか}}{\text{五日}}}$ のあかつきに せうとたる人 ほかより来て \hfill\break
On the dawn of the fifth day people that were brothers came from other places \hfill\break
From the 蜻蛉日記. }

\par{16. 清盛 ${\overset{\textnormal{ちやくなん}}{\text{嫡男}}}$ たるによって ${\overset{\textnormal{そのあと}}{\text{其跡}}}$ をつぐ \hfill\break
Since Kiyomori was the first son, he inherited that position. \hfill\break
From the 平家物語. \hfill\break
 \hfill\break
 \textbf{Word Note }: 跡 is used to refer to inheritance, which is why it is translated as "position". }

\par{\textbf{HISTORICAL NOTE }: Once the copula だ became the main copula verb, both なり and たり fell out of use. たり, as mentioned earlier, is sometimes used in modern contexts as the auxiliary verb -たる in some lasting patterns. }

\par{\textbf{Archaism Note }: There are still instances where the auxiliary copula verb たり is still used in Modern Japanese. As it is more emphatic, it often serves a role in formal yet serious situations. One instance is 日本よ国家たれ! (Japan, be a nation!) and its derivations. }
      
\section{Exercises}
 
\par{1. Make a chart of the bases of the copula なり. }

\par{2. Make a chart of the bases of the copula たり. }

\par{3. Create a sentence with たるや. }

\par{4. In the following example sentence, find the copula verb and tell what base it is in. }

\par{はやても龍の吹かするなり \hfill\break
はやてもりうのふかするなり }

\par{Note: The copula verb is used like the Modern Japanese "no da" in the sentence above. }

\par{5.  When did たり first appear and in what kind of writing? }

\par{6. How is 何たる used? }
    