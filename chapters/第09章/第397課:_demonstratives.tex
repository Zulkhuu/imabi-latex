    
\chapter{Demonstratives}

\begin{center}
\begin{Large}
第397課: Demonstratives 
\end{Large}
\end{center}
 
\par{ Demonstrative words can't be simply defined as starting with these four sounds in Classical Japanese. So, we will simply refer to them as 指示詞, which means "demonstratives". }
      
\section{The Four Dimensions}
 
\par{There are four different types of 指示詞. This is just like in Modern Japanese; it's just appearance that doesn't make this clear sometimes. These four types differ in terms of distance because they are, again, demonstratives. }

\begin{enumerate}

\item Close to the speaker 
\item Close to the listener(s) 
\item Far away from both the speaker and listener(s) 
\item Indeterminate 
\end{enumerate}
 However, it is never this simple. For example, when referring to something that is not concretely there, there are three words for that: その, あの, and かの. The first is used when something is unfamiliar to the listener, the second is used when both are familiar, and the third is used when neither the speaker nor the listener are familiar with what's being talked about. We see little details like this in Classical Japanese as well.  \textbf{Conditions of 指示詞 } No single theory can completely explain the usages of 指示詞. Distance is not the only factor. 我 refers to oneself rather than like そなた  to mean "you" because of the distance factor, but distance alone doesn't explain the rather complex usages of そ and あ words.   First, you must look at the sphere of influence of the speaker and listener. Say you're taunting someone to hit you. You tell them to hit you "here". The person you're taunting responds, "hit you where, here?" "Yes, there". In the first instance, ここ would be used. Speaker B would use どこ then ここ. You would then respond with そこ. ここ shows one's realm of control. ここ used by Speaker B shows his\slash her realm of influence, the place where you're going to be punched. If you turn your back around, you're going to be hit somewhere on your backside. Now, it is out of your control and you're probably going to get what you deserve.    If that was the only condition, any abstract application of これ or それ would be ungrammatical. However, this is not the case at all. Thus, a dimension of experience must be accounted for. こ and あ words relate to direct experience whereas それ is indirect (in respect to the listener's knowledge or sense). This explains the usage of "that" mentioned earlier.   This leads to even more things to consider. Essentially, there are two kinds of usages of 指示詞. They are either based on the scene (現場指示) or on context (文脈指示), with the last being the hardest one to interpret.  
\begin{ltabulary}{|P|P|P|}
\hline 

 &  \textbf{現場指示 }& \textbf{文脈指示 }\\ \cline{1-3}

こ & What the speaker has influence over. & As if it is managed by the speaker. \\ \cline{1-3}

そ & Creates an indirect experiential scene. & Necessarily indirectly experienced object. \\ \cline{1-3}

あ & No influence over something \textbf{now }. & Direct experience by the speaker in the \textbf{past }. \\ \cline{1-3}

\end{ltabulary}
      
\section{指示代名詞}
 
\par{As the title says, this section is about demonstrative pronouns. This section will be dominated by a chart with notes and examples after it. More Japanese is to be implemented in charts, so get ready. }

\begin{ltabulary}{|P|P|P|P|P|P|P|}
\hline 

 & こ & そ & あ & か (方向・場所・他人) & 不定称 & わ ( \textbf{自分 }) \\ \cline{1-7}

限定 & こ(れ) & そ(れ) & あ(れ) & か(れ) & たれ (人) \hfill\break
いづれ (もの) & わ(れ) \\ \cline{1-7}

場所 & ここ & そこ & あそこ \hfill\break
あしこ \hfill\break
あすこ \hfill\break
あこ & かそこ \hfill\break
かしこ & いづこ \hfill\break
いづく &  \\ \cline{1-7}

場所周辺 & ここら \hfill\break
ここいら & そこら \hfill\break
そこいら & あ(そ)こら \hfill\break
あすこら \hfill\break
あそこいら &  & どこら \hfill\break
どこいら &  \\ \cline{1-7}

指定 & この & その & あの & かの & いづれの & われの \hfill\break
わが \\ \cline{1-7}

限定指定 & これの & それの & あれの & かれの & どれの &  \\ \cline{1-7}

方向 & こち & そち & あち &  & いづち \hfill\break
いづかた &  \\ \cline{1-7}

方向周辺 & こちら & そちら & あちら &  &  &  \\ \cline{1-7}

指定方向 & このち & そのち & あのち & かのち & どのち &  \\ \cline{1-7}

時 &  &  &  &  & いつ &  \\ \cline{1-7}

尊称 & こなた & そなた & あなた & かなた & どなた &  \\ \cline{1-7}

蔑称   & こやつ & そやつ & あやつ & かやつ & どやつ &  \\ \cline{1-7}

\end{ltabulary}

\par{\textbf{Usage Notes }: }

\par{1. The れ in the first row is not seen most often in older works. \hfill\break
2. The use of か words for 3rd person is limited. We all know that 彼 means "he" in Modern Japanese, but this is a rather recent development. Words such as かれ and かなた could originally also refer to other people. \hfill\break
3. -ら is like "-abouts".  Before the arrival of どこ, いづこ・いづく was just used for the indefinite column. \hfill\break
4. The second person pronoun 其(し) also existed. \hfill\break
5. Of course, not all of these originate at the same time, and not all of them have survived to the present. All of this comes with the course of language. However, when you see one of these, you should have a pretty good idea what it's being used for. Again, some of these are ancient. \hfill\break
6. The words かすか and かそけし share origin with かしこ. \hfill\break
7. In Modern Japanese これの, それの, あれの, and どれの are not used as alternatives to この, その, etc. They are very limited. かれの VS かの is more productive. The difference can be ascertained from the chart. The first is now used to mean "his". The second is used to mean "that" which is distant\slash unfamiliar to both the speaker and listener(s). \hfill\break
8.  この, その, etc., were deemed as two words put together, a demonstrative word + の. \hfill\break
9. こなた、そなた、Etc. could also be used as direction words. \hfill\break
10. For those that can be used as pronouns, the 1st person pronouns such as こちら can refer not necessarily to oneself but someone in one's in-group just like in Modern Japanese. So, rather, you can view the demonstratives こちら and そちら referring to one's in-group and out-group respectively. \hfill\break
11. Of course, there are other demonstratives in Japanese. Another includes 遠近(をちこち), which is equivalent to あちらこちら. }

\begin{center}
 \textbf{Examples }
\end{center}

\par{1. \textbf{あの }${\overset{\textnormal{をのこ}}{\text{男}}}$ 、 \textbf{こち }${\overset{\textnormal{よ}}{\text{寄}}}$ れ。 \hfill\break
Man over there, come here! \hfill\break
From the 更級日記. }

\par{2. \textbf{わが }${\overset{\textnormal{いもうと}}{\text{妹}}}$ ども \hfill\break
My own sisters \hfill\break
From the 源氏物語. }

\par{3. ${\overset{\textnormal{し}}{\text{知}}}$ らず、 ${\overset{\textnormal{う}}{\text{生}}}$ まれ ${\overset{\textnormal{し}}{\text{死}}}$ ぬる ${\overset{\textnormal{ひと}}{\text{人}}}$ 、 \textbf{いづかた }より ${\overset{\textnormal{き}}{\text{来}}}$ たりて、 \textbf{いづかた }へか ${\overset{\textnormal{さ}}{\text{去}}}$ る。 \hfill\break
I know not where people who are born and die come and go. \hfill\break
From the 方丈記. }
 
\par{4. 原文:篭毛與 美篭母乳 布久思毛與 美夫君志持 \textbf{此 }岳尓 菜採須兒 家吉閑名 告<紗>根 虚見津 山跡乃國者 押奈戸手 \textbf{吾 }許曽居 師<吉>名倍手 \textbf{吾 }己曽座 \textbf{我 }<許>背齒 告目 家呼毛名雄母 \hfill\break
訓読: ${\overset{\textnormal{こ}}{\text{篭}}}$ もよ み ${\overset{\textnormal{こも}}{\text{篭持}}}$ ち ${\overset{\textnormal{ふくし}}{\text{堀串}}}$ もよ み ${\overset{\textnormal{ぶくしも}}{\text{堀串持}}}$ ち \textbf{この }${\overset{\textnormal{おか}}{\text{岳}}}$ に ${\overset{\textnormal{なつみ}}{\text{菜摘}}}$ ます ${\overset{\textnormal{こ}}{\text{児}}}$  ${\overset{\textnormal{いへき}}{\text{家聞}}}$ かな ${\overset{\textnormal{の}}{\text{告}}}$ らさね そらみつ ${\overset{\textnormal{やまと}}{\text{大和}}}$ の ${\overset{\textnormal{くに}}{\text{国}}}$ は おしなべて \textbf{${\overset{\textnormal{わ}}{\text{我}}}$ れ }こそ ${\overset{\textnormal{ゐ}}{\text{居}}}$ れ しきなべて \textbf{ ${\overset{\textnormal{われ}}{\text{我}}}$ れ }こそ ${\overset{\textnormal{いま}}{\text{座}}}$ せ \textbf{${\overset{\textnormal{わ}}{\text{我}}}$ れ }こそば ${\overset{\textnormal{の}}{\text{告}}}$ らめ ${\overset{\textnormal{いへ}}{\text{家}}}$ をも ${\overset{\textnormal{な}}{\text{名}}}$ をも \hfill\break
Hey, young girl with the basket, the wonderful basket, and the hand shovel the wonderful hand shovel picking grasses, tell me where you live but not its name. For I rule all of the land of Yamato filled with the spirits of the Gods. Though I rule everything, show me where you\textquotesingle re from, your house and your name. \hfill\break
From the 万葉集. }

\par{5. \textbf{かの }花は失せにけるは。 \hfill\break
That flower has ended up disappearing! \hfill\break
From the 枕草子. }

\par{\textbf{Word Note }: か words are weaker in their demonstrative nature, and かの can be interpreted as being "usual\slash certain", which would make this sentence all the more exclamatory. }

\par{6. \textbf{誰か }知らまし。 \hfill\break
Who would know? \hfill\break
From the 古今和歌集. }

\par{7. \textbf{それ }を見れば、三寸ばかりなる人。 \hfill\break
When he looked at it, it was a person of around three sun. \hfill\break
From the 竹取物語. }

\par{\textbf{Word Notes }: A 寸 is approximately three centimeters. Also note that it is often easy to translate それ as "it", but remember in Japanese that these demonstrative words have a dimension of distance in their interpretation. }

\par{8. 生きとし生けるもの、 \textbf{いづれか }歌を詠まざりける。 \hfill\break
Of all living things, which does not recite poetry? \hfill\break
From the 古今和歌集. }

\par{\textbf{Word Note }: 生きとし生けるもの is actually still used as a set phrase today. }

\par{9. 「 \textbf{これ }なむ都鳥」と言ふを聞きて \hfill\break
Hearing him say, "This is the capital bird"\dothyp{}\dothyp{}\dothyp{} \hfill\break
From the 伊勢物語. }

\par{10. \textbf{こち }へおはいりあそばせ。 \hfill\break
Please come in here. \hfill\break
From the 菅原伝授手習鑑. }

\par{11. \textbf{そこ }に日を暮らしつ。 \hfill\break
They ended up passing the day there. \hfill\break
From the 更級日記. }

\par{\textbf{Word Note }: Even self-reflexive pronouns like おのれ・おのが are demonstrative pronouns. }
      
\section{Demonstrative Adverbs}
 
\par{Demonstrative adverbs have much more variation. Today, there is こう, そう, ああ, and どう. However, way back when, there was かく, さ, しか, と, and いか along with many other expressions based off of them, many of which are still used today. }

\begin{center}
 \textbf{Expressions }
\end{center}

\begin{ltabulary}{|P|P|P|P|}
\hline 

漢字 & 意味 & 変化 & 表現 \\ \cline{1-4}

斯く & こう & かく \textrightarrow  かう \textrightarrow  こう & 
\par{かくして = こうして }

\par{かく+ある \textrightarrow  かかる = こういう }
\\ \cline{1-4}

然 & そう & さ \textrightarrow  さう \textrightarrow  そう & 
\par{さほど = それほど }

\par{さにあらず = そうではない }

\par{さ+ある \textrightarrow  さる = とある・或る }

\par{さ+あらば \textrightarrow  さらば = そうしたら;さようなら }

\par{さ+あれば \textrightarrow  されば = それでは;だから }
\\ \cline{1-4}

しか & そう &  & 
\par{しかして = そうして }

\par{しか+あるに \textrightarrow  しかるに = そうであるのに }

\par{しか+あるべく \textrightarrow  しかるべく = 適当に }

\par{しか+あれども \textrightarrow  しかれども = そうであるけれども }

\par{しか+あらば \textrightarrow  しからば = それなら }

\par{しか+あれば \textrightarrow  しかれば = そうであるので }
\\ \cline{1-4}

と & ああ &  & 
\par{とかく = ああだったりこうだったり }

\par{とにもかくにも = ああでもこうでも; いずれにしても }
\\ \cline{1-4}

いか & どう &  & 
\par{いかに = どのように、どう }

\par{いかばかり = どれほど }
\\ \cline{1-4}

\end{ltabulary}

\par{  Not all phrases are simple translations to modern Japanese. Some are worded differently today. There are also many phrases with 指示副詞 + あり in Classical Japanese. Some of these become 連体詞, 接続詞, and other 副詞. }

\begin{center}
\textbf{Examples } 
\end{center}

\par{12. いかばかり恥ずかしう、かたはらいたくも候ふらむ。 \hfill\break
How embarrassing and awkward it must have been. \hfill\break
From the 平家物語. }

\par{13. 命は天に在り。然れば時を待つのみ。 \hfill\break
Life is in heaven. So, we must only wait for that time. \hfill\break
By 福沢諭吉. }

\par{14. しかるに禄いまだに賜はらず。 \hfill\break
Even so, we still haven't received our rewards. \hfill\break
From the 竹取物語. }
    