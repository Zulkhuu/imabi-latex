    
\chapter{Combination Particles with もの}

\begin{center}
\begin{Large}
第406課: Combination Particles with もの 
\end{Large}
\end{center}
 
\par{ Just like in Modern Japanese, there are combination particles with もの and they are all conjunctive particles. The particles to be discussed in this lesson are ものを, ものから, ものの, and ものゆゑ(に). }
      
\section{ものを, ものから, ものの, ものゆゑ(に)}
 
\par{ Although some have come and gone or changed in nuance, back in Classical Japanese, these four combination particles did the same thing. So, don't let this be a reason to be confused. All of these particles follow the 連体形. they all may show a concession based on existing conditions. So, they mean "although". Though ものを and ものの survive today with modified nuances, you have to think in more simplistic terms for this. That should be easy enough to do. }

\par{1. 鳴く声も聞こえぬものの悲しきは忍びに燃ゆる蛍なりけり。 \hfill\break
Although I couldn't hear their cries, the sad things were the fireflies hidden and burning. \hfill\break
From the 詞花集. }

\par{2. 都出でて君に会はむと来しものを来しかひもなく別れぬるかな。 \hfill\break
Although I left the capital, there was no avail in thinking about meeting you, and we ended up parting. \hfill\break
From the 土佐日記. }

\par{3. いつはりと思ふ物から今さらにたがまことをか我はたのまむ \hfill\break
Although I think that (that person's word) is a lie, who's truth am I to entrust in now? \hfill\break
From the 古今和歌集. }

\par{4. 頼まぬものの恋ひつつぞ経る。 \hfill\break
Though I rely on you, I pass time longing for you. \hfill\break
From the 伊勢物語. \hfill\break
\hfill\break
5. ${\overset{\textnormal{うつせみ}}{\text{空蝉}}}$ の世の人ごとのしげければわすれぬもののかれぬべらなり。 \hfill\break
The rumors of the world are so annoying; though I'll never forget, I may naturally leave them behind. \hfill\break
From the 古今和歌集. }

\par{6. 雖念 知僧裳無跡 知物乎 奈何幾許 吾戀渡 (原文) \hfill\break
思へどもしるしも無しと知るものを何かここだく我が恋ひ渡る \hfill\break
Although I think of you to know avail, why is that I still long for you like this? \hfill\break
From the 万葉集. }
 
\par{7. 春の野に若菜つまむと来しものを散りかふ花に道はまどひぬ。 \hfill\break
I came for young greens in the spring field, but scattering flowers blew and obscured my path . \hfill\break
From the 古今和歌集. }
 
\par{8. 将来云毛 不来時有乎 不来云乎 将来常者不待 不来云物乎 (原文) \hfill\break
来むと言ふも 来ぬ時あるを 来じと言ふを 来むとは待たじ 来じと言ふものを \hfill\break
You say you'll come and don't, but I wait not for you to say you won't since you may as you won't. \hfill\break
From the 万葉集. }
9. 和我由恵尓 於毛比奈夜勢曽 秋風能 布可武曽能都奇 安波牟母能由恵 (原文) \hfill\break
我が故に思ひな痩せそ秋風の吹かむその月逢はむものゆゑ \hfill\break
Please don't grow thin thinking of me as I shall moon which the autumn wind does blow. \hfill\break
From the 万葉集. 
\par{10. 毎年尓 来喧毛能由恵 霍公鳥 聞婆之努波久 不相日乎於保美 毎年謂之等之乃波 (原文) \hfill\break
年のはに 来鳴くものゆゑ ${\overset{\textnormal{ほととぎす}}{\text{霍公鳥}}}$ 聞けばしのはく 逢はぬ日を多み \hfill\break
Every year when I hear the cries of the cuckoo as they come singing, I get this unexplainable                 feeling. There are too many days I can't meet them. \hfill\break
From the 万葉集. }

\par{11. 月は在明にて光をさまれるものから、不二の峰幽かに見えて。 \hfill\break
The moon was that of dawn, and since the light was in check, Mount Fuji's peak appeared faint. \hfill\break
From the 奥の細道. }

\par{12. ふるさとにあらぬ物からわがために人の心のあれてみゆらむ。 \hfill\break
You may say a man's heart is not in his hometown, but why is it that his heart goes to ruin for me? \hfill\break
From the 古今和歌集. }

\par{\textbf{Historical Note }: }

\par{1. ものを is a combination of the noun もの and the interjectory--not the case particle--を. This explains why even in Modern Japanese it can be seen as a final particle showing a concessive exclamatory statement. Though not normally the case anymore, it did at one point become a causal connector similar to the conjunctive が during the 江戸時代. Nowadays, it shows concession with a sense of the speaker's discomfort and or dissatisfaction. }

\par{13. よど河のよどむとひとは見るらめど流れてふかき心あるものを。 \hfill\break
(You probably think that I haven't been coming and going a while), \hfill\break
although just like that Yodo River's current is deep I too in my heart think much of you with no end. \hfill\break
From the 古今和歌集. }

\par{14. 雀の子を、 ${\overset{\textnormal{いぬき}}{\text{犬君}}}$ が逃しつる、 ${\overset{\textnormal{ふせご}}{\text{伏籠}}}$ の中に、こめたりつるものを。 \hfill\break
Inuki let the baby sparrow escape even though I had kept it in the cage! \hfill\break
From the 源氏物語. }

\par{15. かたみこそ今はあだなれこれなくばわするる時もあらましものを。 \hfill\break
This memento is a resentment that has hurt me even up till now. If I just didn't have it, I would                 have time to forget that person. \hfill\break
From the 古今和歌集. }

\par{2. ものから came into existence during the 奈良時代 was one of the first phrases where から became a true conjunctive particle. As から became used even more heavily in the spoken language, it eventually made people confused to the point that it turned into a particle with similar meaning to ので. Today you would just use から or ので. }

\par{16. 心をぞわりなき物と思ひぬる見るものからや恋しかるべき。 \hfill\break
I thought that emotion was irrational. Since I've met you, I feel such longing; is this supposed to be? \hfill\break
From the 古今和歌集. }

\par{3. ものの got to survive unscathed into Modern Japanese. It's not used frequently, but you see it often in literature. }

\par{4. ものゆゑ(に) has such a good ring to it, yet the only place it has is in 古典. It too could be used to show reason like から. In fact, it can be seen doing this way back in the 万葉集. }
    