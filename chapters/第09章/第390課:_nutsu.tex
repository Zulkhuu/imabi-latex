    
\chapter{The Auxiliary Verbs ~ぬ \& ~つ}

\begin{center}
\begin{Large}
第390課: The Auxiliary Verbs ~ぬ \& ~つ 
\end{Large}
\end{center}
 
\par{No, this lesson is not about the negative auxiliary verb ~ぬ that follows the 未然形. Rather, this lesson is about auxiliary verbs that show the 完了形 (perfective), ~ぬ and ~つ. }

\par{The perfective refers to an action that has been realized or completed . ~ぬ and ~つ  are parallel in basically every way and have the same usages. What makes  them different is that ~ぬ is used intransitively and ~つ is used transitively. They both follow the 連用形. }
      
\section{The Auxiliary Verb ~ぬ}
 
\par{The auxiliary verb ~ぬ shows that an action not only has been completed but was also of natural occurrence. It may also show certainty where it is often followed by other auxiliary verbs that show intention or speculation such as ~べし (should). ~ぬ may also show back-and-forth parallel action like the particle たり today. }

\par{~ぬ conjugations as a ナ変 verb. Its bases are below. }

\begin{ltabulary}{|P|P|P|P|P|P|}
\hline 

未然形 & 連用形 & 終止形 & 連体形 & 已然形 & 命令形 \\ \cline{1-6}

な & に & ぬ & ぬる & ぬれ & ね \\ \cline{1-6}

\end{ltabulary}

\begin{center}
 \textbf{Examples }
\end{center}

\par{1. みな ${\overset{\textnormal{くれなゐ}}{\text{紅}}}$ の ${\overset{\textnormal{あうぎ}}{\text{扇}}}$ の日いだしたるが、 ${\overset{\textnormal{しらなみ}}{\text{白波}}}$ の ${\overset{\textnormal{うへ}}{\text{上}}}$ にただよひ、 ${\overset{\textnormal{うき}}{\text{浮}}}$ きぬ ${\overset{\textnormal{しづ}}{\text{沈}}}$ みぬ ${\overset{\textnormal{ゆ}}{\text{揺}}}$ られければ \hfill\break
Since the fan with the sun drawn with gold in the crimson land floated on top of the white wave and           swayed while floating and sinking \hfill\break
From the 平家物語. }

\par{2. ${\overset{\textnormal{みね}}{\text{嶺}}}$ の ${\overset{\textnormal{さくら}}{\text{櫻}}}$ は散りはてぬらむ。 \hfill\break
The cherry blossoms at the peak have probably completely scattered. \hfill\break
From the 新古今和歌集. }

\par{3. ${\overset{\textnormal{ふね}}{\text{舟}}}$ こぞりて泣きにけり。 \hfill\break
Everyone at once ended up crying in the boat. \hfill\break
From the 伊勢物語. }

\par{\textbf{Grammar Note }: ~にけり is a common combination with ~ぬ. Crying is intransitive, but it is also something that people do. }

\par{4. ${\overset{\textnormal{さか}}{\text{盛}}}$ りにならば、 ${\overset{\textnormal{ようばう}}{\text{容貌}}}$ もかぎりなくよく ${\overset{\textnormal{かみ}}{\text{髪}}}$ もいみじく長くなりなむ。 \hfill\break
When I reach my prime, my features will be exceedingly good, and my hair will also no doubt become        extremely long. \hfill\break
From the 更級日記. }

\par{5. 晴れたる空は ${\overset{\textnormal{よる}}{\text{夜}}}$ に ${\overset{\textnormal{い}}{\text{入}}}$ りて雨となりぬ。 \hfill\break
The bright sky went into night, and it ended up raining. \hfill\break
By 田山花袋. }

\par{6. けぶりあふにやあらむ、 ${\overset{\textnormal{きよみ}}{\text{清見}}}$ が ${\overset{\textnormal{せき}}{\text{関}}}$ の波も高くなりぬべし。 \hfill\break
I wonder if the ocean spray will raise a lot. The waves in Kiyomigaseki will surely get high. \hfill\break
From the 更級日記. }

\par{7. ${\overset{\textnormal{は}}{\text{果}}}$ たし ${\overset{\textnormal{はべ}}{\text{侍}}}$ りぬ。 \hfill\break
It was finally carried out. \hfill\break
From the 徒然草. }

\par{You should expect ~ぬ to follow intransitive verbs, passive auxiliary verbs such as ~る and ~らる. There is no volition as everything occurs naturally on its own, the true definition of being intransitive. Keep this in mind because ~つ will share ~ぬ's three usages but differ in how it is used. }
      
\section{The Auxiliary Verb ~つ}
 
\par{As stated earlier, ~つ is the same as ~ぬ with the exception that it is used with transitive verbs. ~つ generally indicates deliberate action. Like ~ぬ, it can be in various patterns to show confidence. As they show confidence, they are never followed by negative auxiliary verbs. ~つ does survive today as a conjunctive particle in set phrases such as 矯めつ眇めつ. ~つ has a 下二段 conjugation and has the following bases. }

\begin{ltabulary}{|P|P|P|P|P|P|}
\hline 

未然形 & 連用形 & 終止形 & 連体形 & 已然形 & 命令形 \\ \cline{1-6}

て & て & つ & つ & つれ & てよ \\ \cline{1-6}

\end{ltabulary}

\begin{center}
 \textbf{Examples }
\end{center}

\par{8. そこに日を暮らしつ。 \hfill\break
They ended up passing the day there. \hfill\break
From the 更級日記. }

\par{9. ${\overset{\textnormal{そうづ}}{\text{僧都}}}$ 、乗っては降りつ、降りては乗っつ。 \hfill\break
The priest went aboard, fell off, and came aboard. \hfill\break
From the 平家物語. }

\par{10. 年ごろ思ひつること \hfill\break
What I had been thinking for years \hfill\break
From the 徒然草. }

\par{11. とまれかうまれ、とく ${\overset{\textnormal{やぶ}}{\text{破}}}$ りてん。 \hfill\break
Whatever the case may be, I will definitely tear it up. \hfill\break
From the 土佐日記. \hfill\break
 \hfill\break
\textbf{Contraction Note }: とまれかうまれ is a contraction of とまれかくまれ. }

\par{12. この事をばまづ言ひてん。 \hfill\break
This should end up being said above all. \hfill\break
From the 徒然草. }

\par{13. 命限りつと思ひ ${\overset{\textnormal{まど}}{\text{惑}}}$ はる。 \hfill\break
I couldn't help but panic thinking, "my life has definitely come to an end". \hfill\break
From the 更級日記. }

\par{14. なよ竹のかぐや ${\overset{\textnormal{ひめ}}{\text{姫}}}$ とつけつ。 \hfill\break
He ended up naming her Shining Princess of Supple Bamboo. \hfill\break
From the 竹取物語. }

\par{15. これにて ${\overset{\textnormal{ロシア}}{\text{魯西亜}}}$ より帰り来んまでの ${\overset{\textnormal{つひ}}{\text{費}}}$ えをば支えつべし。 \hfill\break
With this, he should end up sustaining the wasteful expenses up to coming home from Russia. \hfill\break
By 鷗外. }
      
\section{Exercises}
 
\par{1. Give the bases of -つ. }

\par{2. Give the bases of -ぬ. }

\par{3. What is different about -ぬ and -つ. }

\par{4. How would you express the perfective in Modern Japanese? }

\par{5. What would be a good definition of -てん・てむ? }

\par{6. Explain the elements of the example sentence 果たし侍りぬ. }

\par{7. How is the usage of emphasis and certainty enumerated more clearly in respect to -ぬ and -つ? }

\par{8. Back-and-forth action began to be a declining usage of -ぬ and -つ. What takes that role today? Give an example. }

\par{9. Translate the following. }

\par{雪のうちに春はきにけり。 }

\par{10. 日も暮れぬ。What function of what auxiliary verb was used? }

\par{11. 散りぬ and 散らしつ differ how? }

\par{12. -ぬ and -つ generally follow verbs based on transitivity. But, there are some verbs such as 出づ (いづ) "to leave" and 寝ぬ (いぬ) "to sleep" where both are OK. However, there is a nuance difference. What would this be? }
    