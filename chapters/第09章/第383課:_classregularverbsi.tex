    
\chapter{Regular Verbs I}

\begin{center}
\begin{Large}
第383課: Regular Verbs I: 四段 
\end{Large}
\end{center}
 
\par{Verbs are by far the most difficult part of learning Classical Japanese. For those that have excelled in controlling the inflectional parts of speech in Modern Japanese, there will be a lot of initial struggle in getting used to Classical Japanese conjugation. }

\par{One of the most obvious differences about verbs in Classical Japanese is that there are more verbal classes than there are in Modern Japanese. In Modern Japanese there are 上一段, 下一段, 五段 , サ変, and カ変 verbs. In Classical Japanese there are 上一段, 下一段, 上二段, 下二段, 四段, サ変, カ変, ラ変, and ナ変 verbs. }

\par{These verbal classes all differ in how their bases look and in the frequency that they appear. Later, we will learn how these 9 verbal classes condensed to the 5 that exist today. For the mean time, we will begin coverage of Classical Japanese verbs with the 四段 class, the predecessor of the modern 五段 class. }

\par{We are looking at the Japanese language over a period of 1000+ years. It would be anachronistic if not completely wrong to assume verbs never changed classes. For now, it is up to you to learn how to identify the class of a verb, conjugate any given verb into its bases, and recognize the modern equivalent of a given verb that you have. The material will be given to you in order to figure this out. }
      
\section{四段活用動詞}
 
\par{四段 verbs are the most common kind of verbs in Classical Japanese, and it is still so today. 四段 verbs are called such because they have a four-grade conjugation. Meaning, they have four different vowel grades. In other words, out of the six bases, at least one of them ends with one of four vowels. These four vowels, in other of the bases, are a, i, u, u, e, and e. Below are the bases for 四段 verbs with the verb 逢ふ. }

\begin{ltabulary}{|P|P|P|P|P|P|}
\hline 

未然形 & 連用形 & 終止形 & 連体形 & 已然形 & 命令形 \\ \cline{1-6}

逢は & 逢ひ \hfill\break
& 逢ふ & 逢ふ & 逢へ & 逢へ \\ \cline{1-6}

\end{ltabulary}
\hfill\break
1. ${\overset{\textnormal{むさし}}{\text{武蔵}}}$ の ${\overset{\textnormal{くに}}{\text{國}}}$ まで ${\overset{\textnormal{まど}}{\text{惑}}}$ ひ歩きけり。 \hfill\break
He walked and wandered about as far as Musashi Province. \hfill\break
From the 伊勢物語. 
\par{\textbf{Grammar Note }: ~けり shows recollection. }

\par{2. けふは ${\overset{\textnormal{みやこ}}{\text{都}}}$ のみぞ思ひやらるる。 \hfill\break
I found myself turning my thoughts to just the capital today. \hfill\break
From the 土佐日記. }

\par{\textbf{Grammar Notes }: }

\par{1. ぞ is used in Classical Japanese as a bound particle where it is very emphatic. \hfill\break
2. ~るる is the 連体形 of the auxiliary verb ~る, the predecessor of the Modern Japanese ~れる. }

\par{3. 風も吹きぬべし。 \hfill\break
The wind will also blow for sure. \hfill\break
From the 土佐日記. }

\par{\textbf{Grammar Note }: ~ぬ is similar to "end up" and is used frequently with ~べし (should). }

\par{4. 夏は ${\overset{\textnormal{}}{\text{郭公}}}$ を聞く。 \hfill\break
I listen to the cuckoo in the summer. }

\par{5. ${\overset{\textnormal{なんぢ}}{\text{汝}}}$ 、よく聞け。 \hfill\break
Listen well you! \hfill\break
From the 今昔物語集. }

\par{6. 我は、さやは思ふ。 \hfill\break
Do I think that way? \hfill\break
From the 徒然草. }

\par{\textbf{Grammar Note }: さ is like そのように in Modern Japanese and や is used here to make an emphatic rhetorical question. }

\par{7. 今 ${\overset{\textnormal{ひとたび}}{\text{一度}}}$ ${\overset{\textnormal{こゑ}}{\text{聲}}}$ をだに聞かせ給へ。 \hfill\break
No, let me hear your voice at least one time. \hfill\break
From the 源氏物語. }

\par{8. 鳴く鹿の聲聞く時ぞ秋はかなしき。 \hfill\break
Oh when I hear the voice of the crying deer, autumn is just sad! \hfill\break
From the 古今和歌集. }

\par{\textbf{Grammar Note }: The 連体形 may end a sentence instead of the 終止形 for emphasis. It is this that causes the confusion between the two and the inevitable fusing of them. }

\par{9. 信濃奈流 須我能安良能尓 保登等芸須 奈久許恵伎気婆 登伎須疑尓家里 \hfill\break
${\overset{\textnormal{しなの}}{\text{信濃}}}$ なる ${\overset{\textnormal{すが}}{\text{須賀}}}$ の ${\overset{\textnormal{あらの}}{\text{荒野}}}$ に ほととぎす 鳴く聲聞けば 時すぎにけり \hfill\break
In the vast Suga Plain of Shinano, when I heard the call of the cuckoo, I felt that time had passed. \hfill\break
From the 万葉集. }

\par{\textbf{Grammar Notes }: }

\par{1. The particle ば may be used like the particle to in Modern Japanese to mean "when". \hfill\break
2. The combination ~にけり is ~ぬ and ~けり put together. ~ぬ in this case shows the perfective--なってしまう. }
      
\section{Exercises}
 
\par{1. Conjugate 思ふ into its bases. }

\par{2. Conjugate 聞く into its bases. }

\par{3. Conjugate 知る with -ず. }

\par{4. Conjugate 読む with the auxiliary verb -ども. }

\par{5. Complete the phrase with the right base of the verb. }

\par{人こそ知_。 }

\par{6. What are Yodan verbs called in Modern Japanese? }
    