    
\chapter{Regular Verbs III}

\begin{center}
\begin{Large}
第385課: Regular Verbs III: 上二段 \& 下二段 
\end{Large}
\end{center}
 
\par{In this lesson we will learn about 二段 verbs. There are several single mora verbs as well as verbs that end in ゆ and づ. As the two classes of 二段 verbs look the same in the 終止形, you will have to memorize what class a verb is in, which is easy when you know what a given verb is in Modern Japanese. 上二段 and 下二段, each of which are only different in what two vowels are involved in their bases as the names suggest, have become 上一段 and 下一段 verbs respectively in Modern Japanese. }
      
\section{上二段活用動詞}
 
\par{There are a lot of 上二段 verbs in Classical Japanese. These verbs are those that end in -iru as 上一段 verbs in Japanese. Unlike Modern Japanese verbs which can only end in 9 certain morae, 二段 verbs can end in any う-vowel かな. There are three verbs in particular that are y-row verbs. }

\begin{ltabulary}{|P|P|P|P|P|P|}
\hline 

老ゆ & おゆ & 悔ゆ & くゆ & 報ゆ & むくゆ \\ \cline{1-6}

\end{ltabulary}

\par{The upper two-vowel grade conjugation for 上二段 verbs consists of the verbs i and u. Below are the bases of 上二段 verbs illustrated by the verb 落つ meaning "to fall". }

\begin{ltabulary}{|P|P|P|P|P|P|}
\hline 

未然形 & 連用形 & 終止形 & 連体形 & 已然形 & 命令形 \\ \cline{1-6}

おち & おち & おつ & おつる & おつれ & おちよ \\ \cline{1-6}

\end{ltabulary}

\begin{center}
 \textbf{Examples }
\end{center}

\par{1. 清見が関を過ぐ。 \hfill\break
We pass the barrier at Kiyomi. \hfill\break
From the 十六夜. }
 
\par{2. 木の葉の落つるも、まづ落ちて芽ぐむにはあらず。 \hfill\break
As for the falling of the tree leaves, it's not that they fall and then bud. \hfill\break
From the 徒然草. }
 
\par{3. たけき者もつひには滅びぬ。 \hfill\break
Even the brave will eventually parish. \hfill\break
From the 平家物語. }
 
\par{4. おのづから、このほど過ぎば、見直したまひてむ。 \hfill\break
Naturally, if one passes this period, one will no doubt look over it again. \hfill\break
From the 源氏物語. }
 
\par{\textbf{Grammar Note }: ~てむ is the combination of the auxiliary verb ~つ and the auxiliary verb ~む to in this situation to show speculation with confidence. }
 
\par{5. むなしく過ぐる事を惜しむべし。 \hfill\break
You should regret passing your life in vain. \hfill\break
From the 徒然草. }
 
\par{6. はやく過ぎよ。 \hfill\break
Pass it quickly! \hfill\break
From the 枕草子. }
 
\par{7. 良頼の ${\overset{\textnormal{ひやふゑのかみ}}{\text{兵衛督}}}$ と ${\overset{\textnormal{まう}}{\text{申}}}$ しし人の ${\overset{\textnormal{いへ}}{\text{家}}}$ を過ぐれば、それ ${\overset{\textnormal{さじき}}{\text{桟敷}}}$ へ渡り給ふなるべし。 \hfill\break
When we passed the house of a person named Hyouenokami of Yoshiyori, he was supposed to have      just passed the balcony. \hfill\break
From the 更級日記. }
      
\section{下二段活用動詞}
 
\par{下二段 verbs are those that now end in -える. There were even verbs like 得(う) "to get" and 経(ふ) that were only one mora long. The distinction between its bases and its stem is blurred. You can find 下二段 verbs  that end in any う-vowel かな. This at times makes it hard to distinguish them from 四段 verbs. However, 四段 verbs, just like in Modern Japanese, only end in う, く, す, つ, ぬ, む, る, ぐ, or ぶ . Below are the bases of 下二段 verbs with the verb 植う. }

\begin{ltabulary}{|P|P|P|P|P|P|}
\hline 

未然形 & 連用形 & 終止形 & 連体形 & 已然形 & 命令形 \\ \cline{1-6}

うゑ & うゑ & うう & ううる & ううれ & うゑよ \\ \cline{1-6}

\end{ltabulary}

\begin{center}
\textbf{Examples }
\end{center}

\par{8. 雨降り日暮る。 \hfill\break
The rain fell, and the sun set. \hfill\break
From the 奥の細道. }
 
\par{9. 心かしこき ${\overset{\textnormal{せきもり}}{\text{関守}}}$ 侍りと聞こゆ。 \hfill\break
They say that there is a clever barrier guard. \hfill\break
From the 枕草子. }

\par{10. ${\overset{\textnormal{いさ}}{\text{諫}}}$ めをも思ひ入れず \hfill\break
Not even trying to accept counsel \hfill\break
From the 平家物語. }
 
\par{11. 都を出でて \hfill\break
We left the capital \hfill\break
From the 平家物語. }
 
\par{\textbf{Base Note }: The bases for a verb like 出づ are 出で、いで、いづ、いづる、いづれ、and いでよ. }

\par{12. ${\overset{\textnormal{あした}}{\text{朝}}}$ に死に、 ${\overset{\textnormal{ゆふべ}}{\text{夕}}}$ に生まるるならひ \hfill\break
The pattern of dying in the morning and being born in the evening \hfill\break
From the 方丈記. }
 
\par{13. 近き火などに逃ぐる人は、「しばし」とやいふ。 \hfill\break
Do people fleeing a nearby fire say, "wait a minute"? \hfill\break
From the 徒然草. }
 
\par{14. 日も ${\overset{\textnormal{と}}{\text{疾}}}$ く暮れよかし。 \hfill\break
May the sun set soon! \hfill\break
From the 今昔物語集. }
 
\par{\textbf{Particle Note }: The particle かし creates great emphasis and frequently follows the 命令形. }
 
\par{15. かくて ${\overset{\textnormal{けふ}}{\text{今日}}}$ 暮れぬ。 \hfill\break
Thus, today came to an end. \hfill\break
From the 土佐日記. }

\par{\textbf{Grammar Note }: ~ぬ shows the perfective. }
Next Lesson \textrightarrow  第310: Irregular Verbs I: サ変 \& カ変   \hfill\break
      
\section{Exercises}
 
\par{1. What did 二段 verbs become in Modern Japanese? }

\par{2. How did 二段 verbs evolve to what they are today? }

\par{3. Give the bases for 上二段 verbs. }

\par{4. Give the bases for 下一段 verbs. }

\par{5. Give the bases for 下二段 verbs. }

\par{6. Conjugate the 下二段 verb 經 (ふ) meaning "to pass (time)" into its bases. }

\par{7. Conjugate 捨つ with the endings べし, む, and ど. }

\par{8. Conjugate 恥づ into its bases and then use it with the noun 事. }

\par{9. Categorize the following verbs. }

\begin{ltabulary}{|P|P|}
\hline 

Verb & Class \\ \cline{1-2}

呼ぶ (よぶ) &  \\ \cline{1-2}

飢う (うう) &  \\ \cline{1-2}

得 (う) &  \\ \cline{1-2}

老ゆ (おゆ) &  \\ \cline{1-2}

思う (おもう) &  \\ \cline{1-2}

過ぐ (すぐ) &  \\ \cline{1-2}

居る (ゐる) &  \\ \cline{1-2}

出づ (いづ) &  \\ \cline{1-2}

強ふ (しふ) &  \\ \cline{1-2}

告ぐ (つぐ) &  \\ \cline{1-2}

朽つ (くつ) &  \\ \cline{1-2}

見る (みる) &  \\ \cline{1-2}

\end{ltabulary}

\par{10. For the verbs mentioned in exercise 9, give the modern form of the verb. If it is already modern, say so. }
    