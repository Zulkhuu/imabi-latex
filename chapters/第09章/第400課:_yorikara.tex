    
\chapter{The Particles より \& から}

\begin{center}
\begin{Large}
第400課: The Particles より \& から 
\end{Large}
\end{center}
 
\par{ There should be little surprise about these particles. Not much has changed. Both follow the same kinds of words: nominal and things in the 連体形. They even share most usages with each other. より came first, and から was once a noun. We will begin by looking at より first. }
      
\section{The Case Particle より}
 
\par{ より can mean "than". However, it can also be used just like から to show point of origin—“from”. This can also be used to show point of transit. This usage today is quite formal, but it was far more commonplace in classical times. }

\par{1. みそかなる所なれば、門よりもえ ${\overset{\textnormal{い}}{\text{入}}}$ らで、 ${\overset{\textnormal{わらは}}{\text{童}}}$ べの踏みあけたる ${\overset{\textnormal{ついひぢ}}{\text{築地}}}$ のくづれより通ひけり。 \hfill\break
In order to not be seen and not being able to enter from the gate, he went back and forth through a       tumbled part of an earthen wall that children had tread upon and opened. \hfill\break
From the 伊勢物語. }

\par{2. ${\overset{\textnormal{おほつ}}{\text{大津}}}$ より ${\overset{\textnormal{うらど}}{\text{浦戸}}}$ をさして ${\overset{\textnormal{こ}}{\text{漕}}}$ ぎいづ。 \hfill\break
From the 土佐日記. }
 
\par{3. その人、かたちよりは心なむまさりたりける。 \hfill\break
As for that person, her heart was superior to her looks. \hfill\break
From the 伊勢物語. }

\par{4. ${\overset{\textnormal{まへ}}{\text{前}}}$ より ${\overset{\textnormal{ゆ}}{\text{行}}}$ く ${\overset{\textnormal{みづ}}{\text{水}}}$ をば、初瀬川といふなりけり。 \hfill\break
The water that passes through the front is called the Hatsuse River. \hfill\break
From the 源氏物語. }

\par{5. もとより友とする人ひとりふたりしていきけり。 \hfill\break
He went out together with one, two people that were his friends from the beginning. \hfill\break
From the 伊勢物語. }

\par{6. いづくより来つる猫ぞと見るに \hfill\break
While I was looked and thinking where the cat turned up from, \hfill\break
From the 更級日記. }

\par{7. ${\overset{\textnormal{あづまぢ}}{\text{東路}}}$ の ${\overset{\textnormal{みち}}{\text{道}}}$ の ${\overset{\textnormal{は}}{\text{果}}}$ てよりも、なほ ${\overset{\textnormal{おく}}{\text{奥}}}$ つ ${\overset{\textnormal{かた}}{\text{方}}}$ に ${\overset{\textnormal{お}}{\text{生}}}$ ひ ${\overset{\textnormal{い}}{\text{出}}}$ でたる人 \hfill\break
A person who has grown in a more remote place than the end of the road of Azuma. }
 
\par{\textbf{Word Note }: Azuma is an old name for the 関東地方. }
 
\begin{center}
\textbf{Method } 
\end{center}

\par{ It used to also be used just like で to show method. }
 
\par{8. ただ一人、 ${\overset{\textnormal{かち}}{\text{徒歩}}}$ よりまうでけり。 \hfill\break
(He) made a pilgrimage alone by foot. \hfill\break
From the 徒然草. }
 
\begin{center}
\textbf{よりほかに }
\end{center}

\par{ よりほかに is also still used to mean “except for” or “aside from”. }
 
\par{9. ひぐらしの鳴く山里の ${\overset{\textnormal{ゆふぐ}}{\text{夕暮}}}$ れは風よりほかに ${\overset{\textnormal{と}}{\text{訪}}}$ ふ人もなし。 \hfill\break
Except for the wind, no one visits on evenings when the cicadas sing in the mountain villages. \hfill\break
From the 古今和歌集. }
 
\par{\textbf{Phrasing Note }: As you can see, 人もなし was quite acceptable in Classical Japanese. }

\begin{center}
\textbf{Reason }
\end{center}

\par{ Y ou can also see より showing reason. Some people think that this is a conjunctive usage. }

\par{つはものどもあまた具して山へ登りけるよりなむ、その山をふじの山とは名付けける。 \hfill\break
The mountain was named "the mountain rich in soldiers” due to many being sent and having climbed it. \hfill\break
From the 竹取物語. }

\begin{center}
 \textbf{As Soon As }
\end{center}

\par{ より can also mean "as soon as", which is a usage that is not used today. It is just like やいなや. }

\par{10. 名を聞くより、やがて面影は推しはからるる心地するを。 \hfill\break
As soon as one hears the name, one immediately imagines the face. \hfill\break
From the 徒然草. }

\par{\textbf{Historical Note }: より was extensively used in the 平安時代. It most certainly shares origin with other particles from the ancient period which include よ, ゆ, and ゆり. }
      
\section{The Case Particle から}
 
\par{ Like より, から can also follow nominals and the 連体形. They share many functions. から can be used to show point of origin, means, passage, rapid sequence, and show reason. This reason function would eventually expand into the conjunctive particle we know and use all the time today in the medieval period. }

\par{ One other grammar point that is special about it was the combination からに, which was truly the predecessor of the conjunctive particle as this could show cause as a result of something, just like today, and also show the meaning of "even if" and rapid sequence. Today, the grammar point からには, which means "so long as". }

\begin{center}
 \textbf{Examples }
\end{center}

\par{11. 浪の花沖から咲きて散り来めり。 \hfill\break
The flowers of the waves appear to bloom offshore and then come shatter on the shore. \hfill\break
From the 古今和歌集. }

\par{12. 月夜好三 妹二相跡 直道柄 吾者雖來 夜其深去來 (原文) \hfill\break
月夜良み妹に逢はむと直道から我は来れども夜そ更けにける。 \hfill\break
A moonlit night is good. Down a straight path I came to meet my wife though night had fallen. \hfill\break
From the 万葉集. }

\par{13. 浪の音の今朝からことに聞こゆるは春の調べやあらたまるらむ \hfill\break
I wonder if the waves heard this morning have put in order the start of spring and its new tune? \hfill\break
From the 古今和歌集. }

\par{14. 惜しむから恋しき物を  白雲の立ちなむ後はなに心ちせむ \hfill\break
As soon as I regret something, I longing for it. Right after the white clouds depart, what kind of                 feeling does one have? \hfill\break
From the 古今和歌集. }

\par{15. 吹くからに秋の草木のしをるればむべ山風邪をあらしといふらむ。 \hfill\break
As soon as the wind blows, the grasses and autumn trees wither; consequentially, one can indeed         say that the mountain wind is a storm. \hfill\break
From the 古今和歌集. }

\par{\textbf{Historical Note }: から actually has its roots as a nominal phrase, and examples can be found occasionally used as such in the 万葉集. }
    