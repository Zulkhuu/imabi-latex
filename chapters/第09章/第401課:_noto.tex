    
\chapter{The Particles の \& と}

\begin{center}
\begin{Large}
第401課: The Particles の \& と 
\end{Large}
\end{center}
 
\par{ This lesson will be about the particles の and と as they were used in Classical Japanese. }
      
\section{The Case Particle の}
 
\par{ The case particle の has been interchangeable with が from the onset. Although this is so, it appears that が was originally limited to noun phrases with an emotional connection, particularly human nouns. This leaves everything else to の. This can explain why even today Japanese say 我が国, although 我の did exist at one point when this distinction faded away. }

\par{ They were both used as attribute markers and after the subject(s) of a sentence. Differentiating between these two different usages is a matter of looking at the logical relationships found in the sentence. If the next word is the object of the sentence but の is clearly after the subject, then you wouldn't interpret it as the attribute marker の. }

\begin{center}
 \textbf{Examples }
\end{center}

\par{1. 思ひあまり ${\overset{\textnormal{い}}{\text{出}}}$ でにし ${\overset{\textnormal{たま}}{\text{魂}}}$ のあるならん。 \hfill\break
It must have been my soul that ventured out in an excess of love\dothyp{}\dothyp{}\dothyp{} \hfill\break
From the 伊勢物語. }

\par{2. 山崎の橋見ゆ。 \hfill\break
The Yamazaki Bridge appeared. \hfill\break
From the 土佐日記. }

\par{3. 秋芽者 可咲有良之 吾屋戸之 淺茅之花乃 散去見者 (原文) \hfill\break
秋萩は咲くべくあらし 吾がやどの浅茅が花の散りゆく見れば \hfill\break
There is no doubt that the clovers have bloomed. If you look at the cogon grass flowers that have           scatter in my yard\dothyp{}\dothyp{}\dothyp{} \hfill\break
From the 万葉集. }

\par{4. 風のをとにぞおどろかれぬる。 \hfill\break
It is by the sound of the wind that I find myself startled. \hfill\break
From the 古今和歌集. }

\par{5. 久方のひかりのどけき春の日にしづ心なく花の散るらむ。 \hfill\break
In this spring day with the sun beaming tranquilly, why have the cherry blossoms not yet calmed? \hfill\break
From the 古今和歌集. }

\par{6. 山の ${\overset{\textnormal{は}}{\text{端}}}$ に日のかかるほど、住吉の浦を過ぐ。 \hfill\break
When the sun reached the edge of the mountain, we passed through the Sumiyoshi Bay. \hfill\break
From the 更級日記. }

\par{7. ${\overset{\textnormal{をんな}}{\text{女}}}$ の女らしからざる、男の男らしからざる、共に天然の道に背きて醜き事の ${\overset{\textnormal{ちやうじやう}}{\text{頂上}}}$ なり。 \hfill\break
Women not acting like women, men not acting like men, both contrary to nature, are the zenith of           ugly things. \hfill\break
By 幸田露伴. }

\par{8. 雪のおもしろう降りたりし ${\overset{\textnormal{あした}}{\text{朝}}}$ 、人のがり言ふべき事ありて ${\overset{\textnormal{ふみ}}{\text{文}}}$ をやるとて、雪のこと何とも言はざりし ${\overset{\textnormal{かへりごと}}{\text{返事}}}$ に、この雪いかが見ると、一筆のたまはせぬほどのひがひがしからむ人の ${\overset{\textnormal{おほ}}{\text{仰}}}$ せらるること、聞き入るべきかは。 \hfill\break
On a morning when the snow fell elegantly, I had something to say to a certain person, and in writing the letter, I in the reply to which I mentioned nothing of the snow, how must one even comply to such a request from an unrefined, perverse person who doesn't even ask about what one thinks of this snow!? \hfill\break
From the 徒然草. }

\par{ Just like in Modern Japanese, の can stand in the place of another nominal. This usage remains a very important function in Japanese and can become very context dependent quickly.  }

\par{9. 草の花なでしこ。唐のはさらなり、大和のもいとめでたし。 \hfill\break
As for grass flowers, the pink (are the best). The Chinese ones go without saying. \hfill\break
The Japanese ones are also very splendid. \hfill\break
From the 枕草子. }

\par{ A usage that is not seen any more is showing apposition, the indication that the subject of the preceded phrase and that of the following are the same. This is replaced in Modern Japanese with であって or (もの)で.  }

\par{10. 白き鳥の、 ${\overset{\textnormal{はし}}{\text{嘴}}}$ と脚と赤き、 ${\overset{\textnormal{しぎ}}{\text{鴫}}}$ の大きさなる、 ${\overset{\textnormal{みづ}}{\text{水}}}$ の ${\overset{\textnormal{うへ}}{\text{上}}}$ に遊びつつ魚を ${\overset{\textnormal{く}}{\text{食}}}$ ふ。 \hfill\break
It was a white bird, and it had a red beak and legs; this bird, which was as big as a snipe, was eating       fish as it played on top of the water. \hfill\break
From the 伊勢物語. }

\par{ Other usages that you might find is in making similes just like のような. A usage that has been limited to rarer 美化語・雅語 speech but was far more common in Classical Japanese was the usage of の after the stems of adjectives. This was far more productive. Compare the first two examples and then the rare modern examples. This grammar pattern is a relic now and was actually more common some 100 years ago in literature.   }

\par{11. 露の命、はかなきものを朝夕に生きたるかぎりあひ見てしがな。 \hfill\break
Life like dew, oh how I would like to watch and confront this vain thing till death. \hfill\break
From the 続後撰集. }

\par{12. 逢ふと見て ことぞともなく 明けぬなり はかな \textbf{の }夢の 忘れがたみや \hfill\break
I met the person (in a dream), and it was like the night broke oh too soon. That vain dream is a               memento to make me remember that person. \hfill\break
From the 新古今和歌集. }

\par{13. 永の別れ \hfill\break
Eternal separation }

\par{14. 愛しの君へ \hfill\break
To you, my dear }

\par{15. 麗しの富士 \hfill\break
Lovely Mt. Fuji }
      
\section{The Case Particle と}
 
\par{  Just like in Modern Japanese, the case particle と can be used to show actions done "with" someone as well as parallel items--"and". For the "and" function, it is often seen in the pattern XとYと rather than just XとY. It also has the citation function, and the citation verb is often dropped. Just like Modern Japanese, it shows the result of change with verbs like 成る. }

\par{16. ${\overset{\textnormal{いを}}{\text{魚}}}$ と鳥とのありさまを見よ。 \hfill\break
Look at the appearance of the fish and birds. \hfill\break
From the 方丈記. }

\par{17. 「よさり、このありつる人たまへ」と。 \hfill\break
(He) said, "take this person from a while ago (into my room) once it becomes night!". \hfill\break
From the 伊勢物語. }

\par{18. 唐土とこの國とは、言異なるものなれど \hfill\break
Although the languages of China and this country differ \hfill\break
From 土佐. }
 
\par{19. 「長き御世にもあらなん」とぞ、思ひはべる。 \hfill\break
She thought, "May she have long life". \hfill\break
From the 源氏物語. }
 
\par{20. 野干玉之夜者須柄尓此床乃比師跡鳴左右嘆鶴鴨 (原文) \hfill\break
ぬばたまの夜はすがらにこの床のひしと鳴るまで嘆きつるかも \hfill\break
All throughout the pitch-black night I sighed to the point that the floor rang sharply. \hfill\break
From the 万葉集. }
 
\par{21. 声絶たず鳴けや鶯ひととせに再びとだに来べき春かは。 \hfill\break
Cry and don't stop bush warbler! If spring went by now, there wouldn't be another time this year! \hfill\break
From the 古今和歌集. }
 
\par{\textbf{Sentence Note }: Ex. 21 shows how that even way back when と could be after an adverb like 再び in a sentence with negative implications. }
 
\par{22. 遊士跡吾者聞流乎屋戸不借吾乎還利於曽能風流士 (原文) \hfill\break
遊士とわれは聞けるを屋戸貸さず吾を還せりおその風流士 \hfill\break
Although you heard that I was a man of elegance, you didn't lend me lodging and made me go               home, saying I was a foolish, elegant man. \hfill\break
From the 万葉集. }
 
\par{\textbf{Reading Note }: 遊士 and 風流士 are both read as みやびを in this poem. }

\par{23. 一声にあくると聞けど郭公まだ深き夜の月に鳴くなり。 \hfill\break
I've been hearing that voice since dawn broke, but even still in the deep of night, the cuckoo cried           because of the moon light. \hfill\break
From the 続後拾遺. }

\par{\textbf{Usage Note }: Starting in the 鎌倉時代, と often became used after the 連体形 as in Ex. 23. }
 
\par{ と may also be used like ~のように often with verbs like 聞こゆ. You can also see it used like として. It can also be used to show a comparison.  }
 
\par{24. 笛の音のただ秋風と聞こゆ。 \hfill\break
The sound of the flute sounds just like the autumn wind. \hfill\break
From the 更級日記. }

\par{25. 奥波  来依荒礒乎  色妙乃  枕等巻而  奈世流君香聞 (原文) \hfill\break
沖つ波来寄る荒磯をしきたへの枕とまきて寝せる君かも \hfill\break
Oh you who are sleeping making a pillow of the reef coast which the offing waves are washing over \hfill\break
From the 万葉集. }
 
\par{26. 中〃尓人跡不有者酒壺二成而師鴨酒二染甞 (原文) \hfill\break
なかなかに人とあらずは酒壺に成りにてしかも酒に染みなむ \hfill\break
I would like to turn into a sake jar without being a rash person so I can be immersed in as much             sake as I want. \hfill\break
From the 万葉集. }
 
\par{\textbf{Development Note }: と+あり \textrightarrow  the copula auxiliary verb たり after the 平安時代. }

\begin{center}
 \textbf{とて }
\end{center}
 
\par{と, often in the form of とて, with a citation verb implied with とて, can be used to show objective. }

\par{27. 牟佐々婢波  木末求跡  足日木乃  山能佐都雄尓  相尓来鴨 (原文) \hfill\break
むささびは 木末求むと あしひきの 山のさつ男に あひにけるかも \hfill\break
The giant flying squirrel was aiming to climb to the tips of big trees and got found by a hunter. \hfill\break
From the 万葉集. }
 
\par{28. からすの寝所へ行くとて、みつよつ、ふたつみつなどとびいそぐさへあはれなり。 \hfill\break
Just the sight of the crows, three or four, two or three, and such, hurrying for a rest place is moving. \hfill\break
From the 枕草子. }

\par{29. 山吹はあやなな咲きそ花見むと植ゑけむ君がこよひ来なくに。 \hfill\break
The kerria mustn\textquotesingle t bloom in vain, though the one who planted and aimed to see the flowers won\textquotesingle t           come tonight. \hfill\break
From the 古今和歌集. }

\begin{center}
 \textbf{連用形 + と }
\end{center}

\par{ と can be seen after the 連用形 of a verb and then be followed again by that verb to give a meaning of "without exception". This usage is still commonly seen in the phrase 生きとし生けるもの (all living things). }

\par{30. 生きと生けるものいづれかうたをよまざりける。 \hfill\break
Every living thing, which will not recite a song? \hfill\break
From the 古今和歌集. }

\par{31. 家の中にありとあるもの、声を調へて泣かなしむ。 \hfill\break
All in the house massing their voices together bewailed. \hfill\break
From the 平家物語. }

\par{\textbf{Origin Note }: The particle と has its roots as an adverb in phrases like とある (a certain) and とかく (apt to; anyhow; this and that). }
    