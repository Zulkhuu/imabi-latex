    
\chapter{Adjectives II}

\begin{center}
\begin{Large}
第382課: Adjectives II: ナリ \& タリ 
\end{Large}
\end{center}
 
\par{ Adjectives in Classical Japanese look similar to their modern counterparts, but the differences in the usage of bases is quite different. ナリ型形容動詞 are cognate to Modern Japanese 形容動詞. タリ型形容動詞 are cognate to 連体詞 that end in ~たる are adverbs that are end in to. }
      
\section{ナリ型形容動詞}
 
\par{ 形容動詞 can be created by adding ~か,  ~やか, and ~らか to more morphemes that for the most part all appear to have been nominal phrases. The original nouns are normally not used, but these adjectives make up the majority of 形容動詞 in Modern Japanese and ナリ型形容動詞 in Classical Japanese. }

\par{ Another important ending used to make 形容動詞 is ~げ. Of course, there are other 形容動詞 that don't end in these sounds. In fact, these endings are only for native words. So, there are a ton of Sino-Japanese words that can be used adjectivally, just as in Modern Japanese. }

\par{One of the main differences that you will notice about ナリ型形容動詞 is that the bases are different and that the adjectives are used differently altogether. Rather than saying ばかじゃない you would say ばかならず.  The bases for 形容動詞 all have a ラ変 conjugation. This means that they have the same conjugation as the irregular Classical Japanese verb あり. Below are the bases. }

\begin{ltabulary}{|P|P|P|P|P|P|}
\hline 

未然形 & 連用形 & 終止形 & 連体形 & 已然形 & 命令形 \\ \cline{1-6}

なら & なり・に & なり & なる & なれ & なれ \\ \cline{1-6}

\end{ltabulary}

\par{Similar to 形容詞, the に-連用形 is used with conjunctive particles and may be used adverbially. Since all but one base is a ラ-変 conjugate, the なる-連体形 may follow nominals whereas the かる・しかる-連体形 cannot. }

\par{\textbf{Curriculum Note }: Tenses and politeness for Classical Japanese have not been covered yet. }

\begin{center}
 \textbf{Examples }
\end{center}

\par{1. 三、四 ${\overset{\textnormal{ちやう}}{\text{町}}}$ を吹きまくる ${\overset{\textnormal{あひだ}}{\text{間}}}$ に、こもれる ${\overset{\textnormal{いへ}}{\text{家}}}$ ども、 ${\overset{\textnormal{おほ}}{\text{大}}}$ きなる小さきも、一つとして破れざるはなし。 \hfill\break
There was not a single house big or small not destroyed in the fire that blew away three, four towns. \hfill\break
From the 方丈記. }

\par{\textbf{Grammar Notes }: }

\par{1. ども is used frequently to pluralize items in Classical Japanese. \hfill\break
2. The 連体形 can be used to show nominalization. \hfill\break
3. ざる is the 連体形 of the negative auxiliary verb -ず. }

\par{2. 心も ${\overset{\textnormal{しづ}}{\text{静}}}$ かならず。 \hfill\break
The heart is unsettled. \hfill\break
From the 徒然草. }

\par{3. あまねく ${\overset{\textnormal{くれなゐ}}{\text{紅}}}$ なる中に \hfill\break
Throughout the completely red sky \hfill\break
From the 方丈記. }

\par{4. 心おのづから静かなれば \hfill\break
Since the heart was naturally quite, \hfill\break
From the 徒然草. \hfill\break
 \hfill\break
 \textbf{Particle Note }: The particle ば is seen after the 已然形 in Classical Japanese to mean "since". }

\par{5. 風 ${\overset{\textnormal{はげ}}{\text{烈}}}$ しく吹きて 静かならざらし夜 ${\overset{\textnormal{いぬ}}{\text{戌}}}$ の時 ${\overset{\textnormal{ばかり}}{\text{許}}}$  ${\overset{\textnormal{みやこ}}{\text{都}}}$ の ${\overset{\textnormal{たつみ}}{\text{東南}}}$ より火 ${\overset{\textnormal{い}}{\text{出}}}$ で来て ${\overset{\textnormal{いぬゐ}}{\text{西北}}}$ に至る。 \hfill\break
In the night at around eight o' clock where the wind blew fiercely and did not settle down, a fire came       from the southeast of the capital and went northwest. \hfill\break
From the 方丈記. }

\par{\textbf{Contraction Note }: The modern contraction rules of Modern Japanese are not reflected throughout the Classical Japanese era. }

\par{\textbf{Auxiliary Note }: The auxiliary verb ~き, in the し-連体形, shows retrospective past. }

\par{\textbf{Word Note }: いぬゐ is an old word for northwest as well is たつみ for southeast. }

\par{6. 心は ${\overset{\textnormal{えん}}{\text{縁}}}$ にひかれて移るものなれば ${\overset{\textnormal{のど}}{\text{閑}}}$ かならでは 道は ${\overset{\textnormal{ぎやう}}{\text{行}}}$ じ ${\overset{\textnormal{がた}}{\text{難}}}$ し。 \hfill\break
Since one's heart is something that can be dragged and changed into another, if one's mind is not           quite, it is difficult to practice. \hfill\break
From the 徒然草. }

\par{\textbf{Contraction Note }: ~では is the contraction of ~ずには. }

\par{7.つれづれなるままに \hfill\break
Just as at one's leisure \hfill\break
From the 徒然草. }

\par{8. ${\overset{\textnormal{なかみかどきやうごく}}{\text{中御門京極}}}$ のほどより大きなる ${\overset{\textnormal{つじかぜ}}{\text{辻風}}}$ 起こりて \hfill\break
A large wind from Nakamikado Kyougoku occurred, and \hfill\break
From the 方丈記. }

\par{\textbf{Meaning Notes }: }

\par{1. ほど as a noun in Classical Japanese can mean "direction". \hfill\break
2. Remember that the particle より can be used to mean "from". }

\par{9. いみじく静かに ${\overset{\textnormal{おほやけ}}{\text{公}}}$ に ${\overset{\textnormal{おんふみ}}{\text{御文}}}$ たてまつり給ふ。 \hfill\break
She offered the letter to the emperor very quietly. \hfill\break
From the 竹取物語. }

\par{10. これをまことか ${\overset{\textnormal{たづ}}{\text{尋}}}$ ぬれば、昔ありし家は ${\overset{\textnormal{まれ}}{\text{稀}}}$ なり。 \hfill\break
When I asked if this was true, I (was told that) old houses were rare. \hfill\break
From the 方丈記. }

\par{\textbf{Classical Form Note }: The verb 尋ねるis 尋ぬ in Classical Japanese. }

\par{11. なかなか ${\overset{\textnormal{やうが}}{\text{様変}}}$ はりて ${\overset{\textnormal{ゆう}}{\text{優}}}$ なるかたもはべり。 \hfill\break
It was rather eccentric, and it was also elegant. \hfill\break
From the 方丈記. }

\par{\textbf{敬語 Note }: はべり is the honorific form of あり (to be) and is used here as an honorific variant of the copula. }

\par{\textbf{Evolution Note }: Not all ナリ型形容動詞 became 形容動詞 in Modern Japanese. Some became -なる ending 連体詞. }
      
\section{タリ型形容動詞}
 
\par{ There are not that many タリ型形容動詞 in Classical Japanese, and there are not that many that have survived as 連体詞 in Modern Japanese. タリ型形容動詞 come from Sino-Japanese compounds. These kind of adjectives were rarely ever seen in 平安時代 works, but they became very frequent in military tales and 江戸時代 Chinese-style writing. Below are the bases for タリ型形容動詞. }

\begin{ltabulary}{|P|P|P|P|P|P|}
\hline 

未然形 & 連用形 & 終止形 & 連体形 & 已然形 & 命令形 \\ \cline{1-6}

たら & たり・と & たり & たる & たれ & たれ \\ \cline{1-6}

\end{ltabulary}

\par{\textbf{Base Note }: These bases are also ラ-変 conjugate, and the 連用形 are used just as before with the other adjectival classes. }

\begin{center}
 \textbf{Examples }
\end{center}

\par{12. さつさつとよをわたるべし。 \hfill\break
You should get along in the world refreshingly. \hfill\break
From the 副王百羽. }

\par{13. ${\overset{\textnormal{りやうふうさつさつ}}{\text{涼風颯々}}}$ たりし ${\overset{\textnormal{よ}}{\text{夜}}}$ なか ${\overset{\textnormal{ば}}{\text{半}}}$ に \hfill\break
In the middle of the night with a refreshing cool breeze \hfill\break
From the 平家物語. }

\par{14. ${\overset{\textnormal{ばうばう}}{\text{茫々}}}$ とした ${\overset{\textnormal{だいさうげん}}{\text{大草原}}}$ \hfill\break
 An expansive large grassland }

\par{15. ${\overset{\textnormal{わうきう}}{\text{王宮}}}$ の体を見るに、 ${\overset{\textnormal{ぐわいくわくべうべう}}{\text{外郭渺々}}}$ として、 ${\overset{\textnormal{そのうちくわうくわう}}{\text{其内曠々}}}$ たり。 \hfill\break
Looking at the state of the lord's palace, the outer walls appear unending, and the grounds within           are expansive. \hfill\break
From the 平家物語. }

\par{16. ${\overset{\textnormal{すで}}{\text{已}}}$ に ${\overset{\textnormal{もうろう}}{\text{朦朧}}}$ たり。 \hfill\break
It is already misty. }

\par{17. ${\overset{\textnormal{くわうりやう}}{\text{荒涼}}}$ たるその景色 \hfill\break
The bleak landscape \hfill\break
From 或る女 by 有島武郎. }
 
\par{18. かの ${\overset{\textnormal{しげふぢまんまん}}{\text{滋藤漫々}}}$ たる ${\overset{\textnormal{かいしやう}}{\text{海上}}}$ を ${\overset{\textnormal{ゑんけん}}{\text{遠見}}}$ して \hfill\break
The aforementioned Shigefuji looked afar over the vast sea. \hfill\break
From the 平家物語. }
      
\section{Exercises}
 
\par{1. Conjugate 索々(さくさく)たり meaning "rustling" into its bases. }

\par{2. What did most ナリ型形容動詞 become. Give an example. }

\par{3. How do you decide what 連用形 to use in respect to adjectives? }

\par{4. Conjugate 清(きよ)らなり meaning "beautiful" into its bases. }
    