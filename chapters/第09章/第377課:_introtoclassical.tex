    
\chapter{Introduction to Classical Japanese}

\begin{center}
\begin{Large}
第377課: Introduction to Classical Japanese 
\end{Large}
\end{center}
 
\par{ What is Japanese? Where did it come from? How has it changed? These are the three questions that you should ask yourself as you learn about Classical Japanese. The term "Classical Japanese" as used here is very broad. Essentially, it's referring to \emph{anything written in Japanese from the万葉集 until the development of pre-modern Tokyo Dialect }. This spans many centuries and language stages. As such, you should not take anything for granted. Always assume that things will not be exactly the same. }

\par{ Going backwards in time from the present, before Early Modern Japanese, there was Late Middle Japanese, and before that there was Early Middle Japanese. Before Middle Japanese there was Old Japanese. The history of Japanese before this time can never be definitively known because no written materials or speakers from that time have survived to the present. However, what can be done is a deep study of the Japanese lexicon to see where Japanese might possibly descend from. }

\par{ Ever since Western-style linguistic studies have been introduced in Japan, it has been assumed for several decades that Japanese should be classified as an Altaic language, which should then share origin with the Uralic languages. Other languages posited to be in this family include Finnish and Hungarian. }

\par{ However, it is almost unanimously believed by most linguists that such connections are unsubstantiated, leaving Japanese to be classified as an isolate. What is certain from etymological analysis is that Japanese has taken a lot of words from various groups as it has developed, and it continues to do so. This will be a challenge for you as you study Classical Japanese as it is the lexical (word) changes that will cause the most headaches, not the grammatical differences. }

\par{In showing interest in wanting to learn about Classical Japanese, you have decided to go far beyond the typical understanding of Japanese known by most of its native speakers. Most people take their native languages for granted. The great history and culture that makes it what it is often gets neglected. This section assumes that you have a decent background in Modern Japanese. So, once you are acquainted with different spellings and pronunciations, you will be quickly given lines of text from actual works. Childishly small sentences are rarely preserved, so you are going to have to be acquainted with \emph{what people did write }. }

\par{\textbf{Elements of いまび: 古典語 } }

\par{ The information introduced is for advanced students of Modern Japanese to learn about Classical Japanese to experience a deeper understanding in literary texts. It also provides the basis to study dialectology as modern dialects evolved differently from Classical Japanese. }

\par{ Sentences will be in 万葉仮名 if the text was originally in it. It will then be followed by a 漢字-かな line with Historical Kana Orthography . For all other texts, examples will be written with 漢字-かな. Although 変体仮名 were actually used, as they cannot be typed, we just have to keep their existence in mind. As for the use of 旧字体 , if they are used, you should realize that they are. As for punctuation, artificial punctuation is put in place. \textbf{Romanization will be seldom used }. You should have a pretty strong knowledge in 漢字 by now. If you don't, you shouldn't be reading about Classical Japanese in the first place. You can see 表外字 and 外字 in classical works. As the readings will always be given, you should not fret over this. }

\par{\textbf{HISTORICAL PERIODS }}

\par{To be fully ready, you need to become familiar with the major historical periods of Japan as well as major Classical Japanese works. }

\begin{ltabulary}{|P|P|P|}
\hline 

奈良時代 \hfill\break
& Nara Period \hfill\break
& 710-784 \\ \cline{1-3}

平安時代 \hfill\break
& Heian Period \hfill\break
& 794-1185 \\ \cline{1-3}

鎌倉時代 & Kamakura Period \hfill\break
& 1185-1333 \\ \cline{1-3}

南北朝時代 \hfill\break
& North and South Courts Period \hfill\break
& 1336-1392 \\ \cline{1-3}

室町時代 \hfill\break
& Muromachi Period \hfill\break
& 1392-1573 \\ \cline{1-3}

戦国時代 \hfill\break
& Warring States Period \hfill\break
& 1477-1573 \\ \cline{1-3}

江戸時代 \hfill\break
& Edo Period \hfill\break
& 1600-1867 \\ \cline{1-3}

明治時代 \hfill\break
& Meiji Period \hfill\break
& 1868-1912 \\ \cline{1-3}

\end{ltabulary}

\par{\textbf{Historical Note }: Pay attention when notes are given that indicate time period. In order to prevent anachronistic mistakes, it is crucial that you take this seriously. }

\par{\textbf{IMPORTANT CLASSICAL WORKS }}

\begin{ltabulary}{|P|P|P|P|}
\hline 

 & 歴史仮名遣い & 現代仮名遣い & Date(s) \\ \cline{1-4}

万葉集 & まんえふしふ & まんようしゅう & 456-760 \\ \cline{1-4}

古事記 & こじき & こじき & 712 \\ \cline{1-4}

竹取物語 \hfill\break
& たけとりものがたり & たけとりものがたり & 890 \\ \cline{1-4}

古今和歌集 & こきんわかしふ & こきんわかしゅう & 905 \\ \cline{1-4}

伊勢物語 \hfill\break
& いせものがたり & いせものがたり & 905 \\ \cline{1-4}

土佐日記 \hfill\break
& とさにつき & とさにっき & 935 \\ \cline{1-4}

大和物語 \hfill\break
& やまとものがたり & やまとものがたり & 951 \\ \cline{1-4}

宇津保物語 \hfill\break
& うつほものがたり & うつほものがたり & 967-984 \\ \cline{1-4}

落窪物語 \hfill\break
& おちくぼものがたり & おちくぼものがたり & Late 10th Century \hfill\break
\\ \cline{1-4}

蜻蛉日記 \hfill\break
& かげろふにつき & かげろうにっき & 974-995 \\ \cline{1-4}

大鏡 \hfill\break
& おほかがみ & おおかがみ & 850-1025 \\ \cline{1-4}

枕草子 \hfill\break
& まくらのさうし & まくらのそうし & 1000-1017 \\ \cline{1-4}

映画物語 \hfill\break
& えいぐわものがたり & えいがものがたり & 1028-1037 \\ \cline{1-4}

源氏物語 \hfill\break
& げんじものがたり & げんじものがたり & 1008-1021 \\ \cline{1-4}

更級日記 \hfill\break
& さらしなにつき & さらしなにっき & 1060 \\ \cline{1-4}

今昔物語集 \hfill\break
& こんじやくものがたりしふ & こんじゃくものがたりしゅう & Early 12th Century \hfill\break
\\ \cline{1-4}

堤中納言物語 \hfill\break
& つつみちゆうなごんものがたり & つつみちゅうなごんものがたり & Late 12th Century \\ \cline{1-4}

小倉百人一首 \hfill\break
& をぐらひやくにんいつしゆ & おぐらひゃくにんいっしゅ & Late 12th Century \hfill\break
\\ \cline{1-4}

新古今和歌集 \hfill\break
& しんこきんわかしふ & しんこきんわかしゅう & 1205 \\ \cline{1-4}

宇治拾遺物語 \hfill\break
& うぢしふゐものがたり & うじしゅういものがたり & 1212-1221 \\ \cline{1-4}

平家物語 \hfill\break
& へいけものがたり & へいけものがたり & 1219-1243 \\ \cline{1-4}

徒然草 & つれづれぐさ & つれづれぐさ & 1331 \\ \cline{1-4}

浮世物語 \hfill\break
& うきよものがたり \hfill\break
& うきよものがたり & 1661 \\ \cline{1-4}

奥の細道 \hfill\break
& おくのほそみち & おくのほそみち & 1702 \\ \cline{1-4}

\end{ltabulary}

\par{These are not the only texts where examples will be pulled from. It is just that these documents are the most important. }

\par{\textbf{Conclusion }}

\par{Japanese is still Japanese. A lot of the vocabulary and syntax of Classical Japanese are readily understandable. Yet, there are also a lot more native words as well as vocabulary that has fallen into archaic usage. The parts of speech are the same, but the way that they are used is quite different in many instances, especially inflectional items. We will spend most of our time studying the inflectional parts of speech. Just looking at a single passage of the 源氏物語 may overwhelm you, but if you truly go through it slowly, you will still find things in common with the present. }

\par{As this section assumes that you know a lot of Modern Japanese, it is also imperative that you make sure that you are confident in your skills. }
    