    
\chapter{The Particle ば}

\begin{center}
\begin{Large}
第404課: The Particle ば 
\end{Large}
\end{center}
 
\par{ The particle ば in Classical Japanese is far from what it is like today. In Modern Japanese we know it as one of the "if" expressions, in fact the strongest of the four options. How and when did it get this way? }

\par{ To look at the history of this particle we first need to address the obvious. If you only knew Japanese grammar and orthography at a bare minimum, you would probably thing that ば and は look so much like each other that they had to have a common origin. Your hunch would be right. It is unclear when this voicing occurred and made the two particles contrastive, but if we consider their usages in the larger spectrum of the history of the Japanese language, there are more similarities. After all, は can be contrasting structures as well as hypotheticals, ~てはいけない. }

\par{ Focusing on everything starting from the written record, we will divide the usages of ba according to what base was involved. Unlike any ending you've seen, this particle went after two different bases, and it took centuries for only one to be the victor, the 已然形. This particle in Classical Japanese could be after the 未然形 and the 已然形. Both were conjunctive, but they clearly reflected the original purposes of the two bases respectively. }
      
\section{After the 未然形}
 
\par{ You would be surprised to know that there are a handful of archaisms that utilize this grammar structure. とあらば is not just some random typo. This reflects the original usage of showing hypothesis. A hypothetical statement just like a negative statement refers to an event that is not realized. The only difference between the two is simply nuance. So, if you view the 未然形 as "the base for unrealization", things should begin to make a lot more sense. }

\par{ As stated, you often see this pattern in old expressions, proverbs, etc. These are the obvious places to look for how things were once used constructively. You can't simply replace 急げば with 急がば without someone telling you that you're either wrong or extremely old-fashioned. }

\begin{center}
 \textbf{Examples }
\end{center}

\par{1. いつわりのなき世なりせばいか許人の事の葉うれしからまし。 \hfill\break
If this were a world without falsehood, how pleasing would people's words (like yours) be! \hfill\break
From the 古今和歌集. }

\par{2. 若し ${\overset{\textnormal{とみ}}{\text{急}}}$ の事有らば、この袋の口を解きたまへ。 \hfill\break
If there is an emergency, undo the mouth of this bag. \hfill\break
From the 古事記. }

\par{3. 盛りにならば、 ${\overset{\textnormal{かたち}}{\text{容貌}}}$ も限りなくよく、髪もいみじく長くなりなむ。 \hfill\break
If I reach my prime, my looks will be without comparison, and my hair will become very long. \hfill\break
From the 更級日記. }

\par{4. 徳をつかんと思はば、すべからく、まづその心づかひを ${\overset{\textnormal{しゆぎやう}}{\text{修行}}}$ すべし。 \hfill\break
If you intend to obtain virtue, you ought to definitely first discipline the heart. \hfill\break
From the 徒然草. }

\par{5. 君為 手力労 織在衣服斜 春去 何色 摺者吉 \hfill\break
君がため ${\overset{\textnormal{たぢから}}{\text{手力}}}$ 疲れ織りたる衣ぞ春さらばいかなる色に ${\overset{\textnormal{す}}{\text{摺}}}$ りてば ${\overset{\textnormal{よ}}{\text{好}}}$ けむ。 \hfill\break
This is a kimono that I've threaded by my hands for you. In spring, what color shall we dye it in? \hfill\break
From the 万葉集。 }

\par{\textbf{Base Note }: よけ is what an ancient く-stem 未然形 would have looked like for adjectives. The conditional here would be translated with the particle たら today. This kind of usage where the conditional and temporal usage to be discussed seem to be mixed would eventually lead them to become one by the 室町時代. }

\par{6. 笑わば笑え。 \hfill\break
If you're going to laugh, laugh! \hfill\break
\hfill\break
7. 死なば諸共 \hfill\break
We die together. \hfill\break
Literally: If we are today, all together. }

\par{8. 遠慮会釈もあらばこそ  (Archaism using the strong negative pattern ~あらばこそ =あろうはずもなく) \hfill\break
Indeed not even with the thought of consideration for the others\dothyp{}\dothyp{}\dothyp{} }
      
\section{After the  已然形}
 
\par{ When after the 已然形, ば shows causation (since) or means "when". This is still a part of the properties of the conditional ば. When something will happen, then something Y is going to generally happen. }

\par{9. 命長ければ、辱多し。 \hfill\break
When you're life is long\slash if you live long, you have a lot of shame. \hfill\break
From the 徒然草. }

\par{10. 老いぬればさらぬ別れのありといへばいよいよ見まくほしき君かな。 \hfill\break
Since they say when we grow old an unavoidable parting awaits, I want to see you all the more! \hfill\break
From the 伊勢物語. }

\par{11. はべる所の焼け侍りにければ \hfill\break
Since the place where I was ended up burning \hfill\break
From the 枕草子. }
    