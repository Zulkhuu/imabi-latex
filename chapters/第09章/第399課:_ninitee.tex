    
\chapter{The Particles に, にて, \& へ}

\begin{center}
\begin{Large}
第399課: The Particles に, にて, \& へ 
\end{Large}
\end{center}
 
\par{ You can't just assume that particles are completely the same in Classical Japanese. Although all three of these particles are very similar to what they are in Modern Japanese, there are quite a few things that make a big difference. For example, where's で? It evolves from にて. So, way back when, に showed all of the functions now allotted to just で! }
      
\section{に}
 
\par{Just as in Modern Japanese, に has a lot of usages! In Classical Japanese it has both case and conjunctive usages. Let's start with what's different. }
 
\par{1. It can be used like the particles で and にて to show the location of an action. It may also show method by which something is done as well as cause. All of these usages have been given to で in Modern Japanese. In Modern Japanese you would be counted wrong for this. However, for a very long time up until and even a whiles after the contraction of にて to で was made, に was still used in this way. }
 
\par{1. この ${\overset{\textnormal{かはぎぬ}}{\text{皮衣}}}$ は、火に焼かむに焼けずはこそ、 ${\overset{\textnormal{まこと}}{\text{真}}}$ ならめ。 \hfill\break
When we try to burn this leather robe with fire and it doesn't burn, then it's probably the real thing. \hfill\break
From the 竹取物語. }
 
\par{\textbf{Particle Note }: Remember the 連体形の準体法? This examples why に is after 焼かむ. The implied noun is 時. }
 
\par{2. 在京荒有家尓 一宿者 益旅而 可辛苦 (原文) \hfill\break
都なる荒れたる家にひとり寝ば旅にまさりて苦しかるべし \hfill\break
Sleeping by myself in my ruined home in the capital is perhaps more trying than sleeping in travel. \hfill\break
From the 万葉集. }
 
\par{3. 烏玉之 夜之深去者 久木生留 清河原尓 知鳥數鳴 (原文) \hfill\break
ぬばたまの夜の更けゆけば久木 ${\overset{\textnormal{お}}{\text{生}}}$ ふる清き ${\overset{\textnormal{かはら}}{\text{川原}}}$ に千鳥しば鳴く \hfill\break
As the pitch black night grew late, the plovers constantly chirped at the pure riverbank where the             hisaki grows. \hfill\break
From the 万葉集. }
 
\par{\textbf{Word Note }: It's not exactly known what plant the 久木 was. }
 
\par{2. The conjunctive particle に is interchangeable with the conjunctive particle を. As we learned in that lesson, this に can be used to show causation, concession, as well as sequence. The usages in order in Modern Japanese are ので, のに, and が respectively. }
 
\par{4. 命のあるものを見るに、人ばかり久しきはなし。 \hfill\break
When you look at things that have life, nothing (lives) as long as people. \hfill\break
From the 徒然草. }
 
\par{3. So, what are all of those usages in Modern Japanese that are in Classical Japanese? The most important usage still in this time is doing all of what we learned way back in Lesson 19. Although most of this is the same, some things have been expanded upon, so don't just glance through it. }
 
\par{\emph{ }\emph{に \textbf{compensates the establishment of an action or state }. }\emph{に shows deep establishment. In doing so, it plays this role in many crucial situations. It is for the most part equivalent to "at". The following five points describe }\emph{に very well. }}
 
\begin{itemize}
 
\item \emph{Shows the place of where a state (of action) deep      establishment. It may show the \textbf{place of existence or occurrence }or \textbf{person      of possession }( }\emph{彼には子どもがいる). } 
\item \emph{Shows \textbf{destination\slash direction\slash target }! It's the \textbf{indirect      object }marker as well. } 
\item \emph{Shows the \textbf{effect, condition, state, or goal of an      action }. Verbs like "to become" come to mind. This may also      show pretext. For example, "you make your hands \textbf{as }a      pillow". It is also the "as" of comparison and      capacity. } 
\item \emph{Shows the \textbf{standard of action, condition, or state }.      Ex. "to lack \textbf{in }creativity". } 
\item \emph{Shows \textbf{what brings about some sort of measure, feelings,      situation, or work }. }  
\end{itemize}
 
\par{More specifically, other applications of all of this that you have to keep in mind are how it is used with the passive and causative to mean "by". Of course, you can't forget how it is used to show time of action. }

\begin{center}
\textbf{Examples }
\end{center}

\par{5. ${\overset{\textnormal{あした}}{\text{朝}}}$ に死に、夕べに生まるるならひ。 \hfill\break
The pattern of dying in the morning and being born in the evening. \hfill\break
From the 方丈記. }
 
\par{6. 葦北乃 野坂乃浦従 船出為而 水嶋尓将去 浪立莫勤 (原文) \hfill\break
葦北の野坂の浦ゆ船出して水嶋に行かむ浪立つなゆめ \hfill\break
Let's set sail from Ashikita's Nozaka Inlet to Mizushima. Waves, never roll! \hfill\break
From the 万葉集. }
 
\par{7. 狩りに往にけり。 \hfill\break
He went hunting (with falcons). \hfill\break
From the 伊勢物語. }
 
\par{8. 白波の上にただよひ \hfill\break
Since it floated on top of the white waves \hfill\break
From the 平家物語. }
 
\par{\textbf{Historical Note }: Remember that the copula verb なり comes from にあり. }
 
\par{9. まことかと聞きて見つれば、言の葉を飾れる玉の枝にぞありける。 \hfill\break
He listened, wondering if it was real, and when he looked, he found that to his surprise that it was a          jeweled branch decorated with words! \hfill\break
From the 竹取物語. }
 
\par{10. 妹之家毛 継而見麻思乎 山跡有 大嶋嶺尓 家母有猿尾 (原文) \hfill\break
${\overset{\textnormal{}}{\text{妹}}}$ が家も継ぎて見ましを大和なる ${\overset{\textnormal{おほしま}}{\text{大島}}}$ の ${\overset{\textnormal{ね}}{\text{嶺}}}$ に ${\overset{\textnormal{いへ}}{\text{家}}}$ もあらましを \hfill\break
(If we can't meet), at least it would be nice to see your house. If only my house were on top of the       peak of the great island of Yamato(, then I could always see your house). \hfill\break
From the 万葉集. }
 
\par{11. 京に入り立ちてうれし。 \hfill\break
We were happy to enter the capital. \hfill\break
From the 土佐日記. }

\par{12. ${\overset{\textnormal{よど}}{\text{淀}}}$ みに浮かぶうたかたは \hfill\break
Bubbles floating in a pool (of water) \hfill\break
From the 方丈記. }
      
\section{にて}
 
\par{This particle can follow both nominals and the 連体形. It is the ancestor of で. It is the combination of the case particle に and the conjunctive particle て. Just like で can be used to indicate place of action, means of action, what something is made of, age, cause, and condition. In showing condition it is often like として. When used as a conjunctive particle, it is just like ので. Remember that there was no need to have something nominalize verbs because of 連体形の準体法. で first appears in the Late 平安時代. However, it was probably in local vernaculars much earlier. Whatever you do, don't confuse it with the て form of なり. }

\par{13. 後生でだに悪道へおもむかんずる事のかなしさよ。 \hfill\break
Ah, the sadness of heading toward hell even in the next life! \hfill\break
From the 平家物語. }

\par{14. 芝の上にて飲みたるもをかし。 \hfill\break
Drinking on top of the grass is also charming. \hfill\break
From the 徒然草. }

\par{15. 應還 時者成来 京師尓而 誰手本乎可 吾将枕 (原文) \hfill\break
帰るべく時はなりけり都にて誰が手本をか我が枕かむ \hfill\break
I at last may go home, but whose sleeves of someone in the capital shall I sleep on for I no longer           have a wife. \hfill\break
From the 万葉集. }

\par{16. いみじう寂しげなるに、 \hfill\break
He seemed very lonesome, but \hfill\break
From the 源氏物語. }
      
\section{へ}
 
\par{  There is nothing different about this particle. One thing that you should keep in mind, though, is that originally it was meant to indicate movement going (far) away. However, it has ever since the 平安時代 been used with any direction like it is used today. The particle most likely comes from the noun 辺. }

\par{17. 住む館より出でて、舟に乗るべき所へ渡る。 \hfill\break
We left the fort where we had lived and crossed to the place where we were to board a boat. \hfill\break
From the 土佐日記. }
    