    
\chapter{Adjectives I}

\begin{center}
\begin{Large}
第381課: Adjectives I: ク \& シク 
\end{Large}
\end{center}
  Adjectives in Modern Japanese are either 形容詞 or 形容動詞. 形容詞 today are actually the simplified result of two kinds of adjectives: ク and シク ending 形容詞. In this lesson you will learn how to conjugate and use these two classes of Classical Japanese verbs and also how they are different from each other.       
\section{ク活用形容詞}
 
\par{ ク活用形容詞 get their name from the fact that their 連用形 is く-. ク adjectives are known for describing physicality. Things like shallow 浅し, black 黒し, etc. are all ク活用形容詞. So what's different about Modern 形容詞 and ク活用形容詞? First, we need to understand what Modern 形容詞 are. }

\begin{center}
 \textbf{In Modern Japanese }
\end{center}

\par{In Modern Japanese 形容詞 are all adjectives that end in ~い and that can naturally be conjugated without assistance from だ. The bases of 形容詞 are: }

\begin{ltabulary}{|P|P|P|P|P|P|}
\hline 

未然形 \hfill\break
& 連用形 & 終止形 & 連体形 & 已然形 & 命令形 \\ \cline{1-6}

かろ & く・かり & い & い & けれ & かれ \\ \cline{1-6}

\end{ltabulary}

\par{The かろ-未然形 is used with the volitional ~う . In actually, ~う attaches to the から-未然形, the true 未然形. Remember, the combination あ+う becomes pronounced as "おお". When spelling reform took place, the spelling became おう. The two 連用形 are actually the \textbf{remnants }of an entire two base system from Classical Japanese. The 終止形 and 連体形 are the \textbf{same }in Modern Japanese, and the 命令形 is almost \textbf{never }used. }

\begin{center}
 \textbf{In Classical Japanese }
\end{center}

\par{In Classical Japanese ク活用動詞 circumference only adjectives that end in ~し, that can naturally be conjugated into two sets of bases without the assistance of the copula なり, and that have meaning that is specific to physicality. There are two bases of ク活用動詞: く stem and かり stem. The two sets of bases of ク活用動詞 are: }

\begin{ltabulary}{|P|P|P|}
\hline 

 & く stem \hfill\break
& かり stem \hfill\break
\\ \cline{1-3}

未然形 \hfill\break
& X & から \hfill\break
\\ \cline{1-3}

連用形 \hfill\break
& く \hfill\break
& かり \hfill\break
\\ \cline{1-3}

終止形 \hfill\break
& し \hfill\break
& かり* \\ \cline{1-3}

連体形 \hfill\break
& き \hfill\break
 & かる \\ \cline{1-3}

已然形 \hfill\break
& けれ \hfill\break
& かれ* \\ \cline{1-3}

命令形 & X & かれ \hfill\break
\\ \cline{1-3}

\end{ltabulary}

\par{The く stem bases are the original bases, and かり stem bases are あり + the く-連用形. When this finally gave  way in the 平安時代 to produce the かり stem bases, bases became  available for 形容詞 with a ラ変 irregular  conjugation--あり was an irregular verb. So, before the 平安時代, only く stem conjugations existed. As most of Classical Japanese  literature was written from the 平安時代 and afterward, you need to know how they differed. }

\par{The く stem bases are with conjunctive  particles such as て and do as well as with nominals, but the かり stem  bases are not. \textbf{The かり stem bases are used with auxiliary verbs }, but  the く stem bases are not. Before the 平安時代, あり was used to  compensate the limitations of the く stem bases, the true reason why the かり stem bases formed. }

\par{The から-未然形 is used for negation, volition, or desire . Before the 平安時代, you had to use ある in the negative after the く-連用形. In Classical Japanese the 終止形 and 連体形 are different. As for the 已然形, it is viewed as the perfective form because it is used with endings such as ば. ば wasn't used like today until the 室町時代 and centuries went by before it became only used like today. In Classical Japanese the 命令形 is actually used. }

\par{*: The かり stem 終止形 and 已然形 are \textbf{only }used for two adjectives which are considered exceptions because of this. These adjectives, 多し and 無し, which may also be referred to as 多かり and 無かり,are seen with these かり stem conjugations often in writings from the 平安時代 onward in poetry, tales, and other sorts of writing. }

\begin{center}
\textbf{Summary }
\end{center}

\begin{enumerate}

\item Modern 形容詞 end in ~い and account for all adjectives that can naturally conjugate  whereas ク活用形容詞 end in ~し and account for adjectives of physicality that can naturally conjugate . 
\item The bases are significantly different between Modern 形容詞 and ク活用形容詞. 
\item The 終止形 and 連体形 are not the same in Classical Japanese. 
\item The 未然形, 已然形, and 命令形 are used slightly different between the two eras. 
\end{enumerate}
\textbf{使用例 } \hfill\break

\par{1. 所も変はらず 人も多かれど いにしへ見し人は ${\overset{\textnormal{にさんじふにん}}{\text{二三十人}}}$ が中にわづかにひとりふたりなり。 \hfill\break
Places too don't change, and although there are many people, of the 20 or 30 people that I have seen in the past, there are (now) merely one or two. \hfill\break
From the 方丈記. }
 
\par{\textbf{Grammar Note }: The particle が before なか is used in Classical Japanese as the possessive case particle. }

\par{2. ${\overset{\textnormal{いらか}}{\text{甍}}}$ を争へる高き ${\overset{\textnormal{いや}}{\text{賤}}}$ しき人のすまひは \hfill\break
The houses of both the noble and the base competing with their roof tiles \hfill\break
From the 方丈記. }
 
\par{3. 消えずといへども ${\overset{\textnormal{ゆふ}}{\text{夕}}}$ を待つ事なし。 \hfill\break
Even if (the dew) doesn't vanish, there is never a case where it waits for evening. \hfill\break
From the 方丈記. }
 
\par{4. 目も ${\overset{\textnormal{あ}}{\text{當}}}$ てられぬこと多かり。 \hfill\break
There are many things that one couldn't cast one's eyes on. \hfill\break
From the 方丈記. }
 
\par{5. この吹く風は よきかたの風なり。 \hfill\break
This wind that is blowing is wind from a good direction. \hfill\break
From the 竹取物語. }
 
\par{6. よからぬ物 \hfill\break
Things that aren't good }
 
\par{7. 上は ${\overset{\textnormal{さやまき}}{\text{鞘巻}}}$ の黑く ${\overset{\textnormal{ぬ}}{\text{塗}}}$ りたりけるが \hfill\break
As for the surface, it was varnished black with a sayamaki. \hfill\break
From the 平家物語. \hfill\break
 \hfill\break
\textbf{Definition Note }: A 鞘巻 is a part of a short sword. }
 
\par{8. 善かれ。 \hfill\break
Be good! }

\par{\textbf{Archaism Note }: Sometimes old forms of adjectives show up in Modern Japanese phrases. For instance, the usage of the 連体形 as a nominal form is very common in set phrases such as proverbs even today. }

\par{9. 弱きを助け、強きを挫く。 \hfill\break
Help the weak, and crush the strong. }
      
\section{シク活用形容詞}
 
\par{  シク活用形容詞  in Modern Japanese are 形容詞 that \textbf{end in しい or じい }. These adjectives are typically subjective. These also tend to have different meanings than now. For example, をかし in Classical Japanese is equivalent to "zestful", but its modern form おかしい means "funny". The bases of シク活用形容詞  are the same as ク活用形容詞 with the only difference being that -し is present. This, of course, is -じ whenever the adjective ends in -じ. }

\begin{ltabulary}{|P|P|P|P|P|P|P|}
\hline 

Stem & 未然形 & 連用形 & 終止形 & 連体形 & 已然形 & 命令形 \\ \cline{1-7}

く & X & しく & し & しき & しけれ & X \\ \cline{1-7}

かり & しから & しかり & X & しかる & X & しかれ \\ \cline{1-7}

\end{ltabulary}

\par{Common シク活用形容詞 in Classical Japanese include the following. }

\begin{ltabulary}{|P|P|P|P|}
\hline 

いみじ & Extreme & 同じ (おなじ) & Same \\ \cline{1-4}

嬉し (うれし) & Happy & ゆかし & Attractive \\ \cline{1-4}

正し (ただし) & Correct & 美し (うつくし) & Pretty \\ \cline{1-4}

\end{ltabulary}

\par{\textbf{Grammar Note }: 同じ has two alternative 連体形 aside from the かり-stem 同じかる, おなじ and おなじき. The first is used in 和文 (classical) texts while the second is used in 漢文. }

\begin{center}
\textbf{使用例 } 
\end{center}

\par{10. ゆめうれしからず。 \hfill\break
I would not be happy at all. \hfill\break
From the 土佐物語. }
 
\par{11. また、これに同じかるべし。 \hfill\break
Again, it should be the same as this. \hfill\break
From the 徒然草. }
 
\par{12. さりければ ${\overset{\textnormal{おとど}}{\text{大臣}}}$ いときよらに ${\overset{\textnormal{すをがさね}}{\text{蘇芳襲}}}$ などきたまうて ${\overset{\textnormal{きさき}}{\text{后}}}$ の宮にまゐりたまうて ${\overset{\textnormal{ゐん}}{\text{院}}}$ の ${\overset{\textnormal{おほむせうそこ}}{\text{御消息}}}$ のいとうれしく 侍りてかくいろゆるされて侍こと などきこえ給ふ。 \hfill\break
With that the reason, the minister was very elegant and handsome, and he wore suoagasane and came to the palace of the empress and said he was very happy to receive a message from the retired emperor and that he was allowed to wear the restricted colors. \hfill\break
From the 大和物語. \hfill\break
 \hfill\break
 \textbf{Cultural Note }: 蘇芳襲 were layered clothes with light brown on front and dark red underneath. }
 
\par{13. いみじううれしかりしものかな。 \hfill\break
What an extremely pleasant thing it was! \hfill\break
From the 枕草子. \hfill\break
 \hfill\break
\textbf{Contraction Note }: いみじう is a contraction of いみじく. }
 
\par{14. 柳などをかしこそさらなれ それもまだまゆにこもりたるはをかし。 \hfill\break
It's needless to say that the (blossoming of the) willow tree is zestful, and the willow tree still covered in a cocoon(-like bud) is also zestful. \hfill\break
From the 枕草子. }
 
\par{15. おもしろく咲きたる ${\overset{\textnormal{さくら}}{\text{櫻}}}$ を長く折りて 大きなる ${\overset{\textnormal{かめ}}{\text{瓶}}}$ にさしたるこそをかしけれ。 \hfill\break
Picking up beautifully blossomed cherry blossoms and placing them in a big vase is precisely zestful. \hfill\break
From the 枕草子. }
 
\par{16. うれしきもの  まだ見ぬ物語の一を見て いみじうゆかしとのみ思ふが 残り見いでたる。 \hfill\break
Pleasing things. You looked at a volume of a story you still hadn't read, and just as you were thinking that it was only extremely attractive, you found the remaining volume. \hfill\break
From the 枕草子. }
 
\par{17. 同じ心ならん人としめやかに物語して をかしき事も 世の ${\overset{\textnormal{はかな}}{\text{儚}}}$ き事も うらなく言ひ慰まんこそうれしかるべき \hfill\break
にさる人あるまじければ つゆ違はざらんと向ひゐたらんは たゞひとりある心地やせん。 \hfill\break
If there is a person of the same heart, I will solemnly talk (with that person) about the zestful things, the vain things, and if we certainly comfort (each other) without hiding things it will be a happy thing. Yet, without there being supposed to be such a person, in order to not differ even slightly with one and other, if we confront each other, we will have the feeling of one. \hfill\break
From the 徒然草. }
 
\par{18. 同じき三年 ${\overset{\textnormal{だざいふ}}{\text{太宰大}}}$ 弐になる。 \hfill\break
He became the Daini of the Dazaifu in the same third year. \hfill\break
From the 平家物語. }

\begin{center}
\textbf{Important Contraction Rules }
\end{center}

\par{The k in the き・しき-連体形 and the k in the く・しく-連用形 is sometimes dropped in Classical Japanese. As for the contraction in the 連体形, this is how the modern form of 形容詞 developed. The contraction in the 連用形 is maintained in many dialects today.  As for the 連用形 contraction, be cautious in pronunciation as 歴史仮名遣い applies. }
      
\section{Exercises}
 
\par{1. Conjugate the シク活用形容詞  口惜し(くちをし) meaning "regrettable" into its bases. }

\par{2. Conjugate the ク活用形容詞  青し(あおし) meaning "blue" into its bases. }

\par{3. What do most ク活用形容詞 describe? }

\par{4. What do most シク活用動詞 describe? }

\par{5. Illustrate the difference between the 終止形 and 連体形 and show what they became in Modern Japanese for both types of adjectives. }
    