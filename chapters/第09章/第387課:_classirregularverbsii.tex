    
\chapter{Irregular Verbs II}

\begin{center}
\begin{Large}
第387課: Irregular Verbs II: ラ変 \& ナ変 
\end{Large}
\end{center}
 
\par{There are four basic ラ変 irregular verbs in Classical Japanese: あり, をり, はべり, and いまそがり. And, there are two ナ変 irregular verbs: 死ぬ and 往ぬ. }
      
\section{ラ行変格動詞}
 
\par{ ラ変 verbs that we have already seen before include あり, をり, and はべり before. あり is the primary ラ変 verb and is also a supplementary verb that tends to fuse with other things. をり is an alternate variant of ゐる. はべり, a contraction of ${\overset{\textnormal{は}}{\text{這}}}$ いあり, means "to serve a superior" and is very honorific. It too can be used as a supplementary verb. Another ラ変 verb is いまそがり, a contraction of ${\overset{\textnormal{いま}}{\text{座}}}$ す ${\overset{\textnormal{か}}{\text{処}}}$ あり, which means "to be" and is very honorific. }
 
\par{These verbs end in い rather than う in the 終止形 to signify a state of existence and or continuing state as do ある and いる in Modern Japanese. These verbs became 四段 verbs during the 鎌倉時代 when the 連体形 began to replace the 終止形 altogether. }
 
\par{あり means "to be" and does not necessarily have the animate v. inanimate object distinction as its modern form ある does. As it is the most common of the four ラ変 verbs, we will use it to demonstrate the bases of ラ変 irregular verbs. Below are its bases. }

\begin{ltabulary}{|P|P|P|P|P|P|}
\hline 

未然形 & 連用形 & 終止形 & 連体形 & 已然形 & 命令形 \\ \cline{1-6}

あら & あり & あり & ある & あれ & あれ \\ \cline{1-6}

\end{ltabulary}

\par{Notice how the conjugations for ラ変 irregular verbs are almost identical to that of 四段 verbs as both have a four-vowel grade. }

\begin{center}
\textbf{Examples }
\end{center}

\par{1. ${\overset{\textnormal{よしかげ}}{\text{義景}}}$ は、 ${\overset{\textnormal{きりど}}{\text{切戸}}}$ の ${\overset{\textnormal{わき}}{\text{脇}}}$ にかしこまりてぞはべりける。 \hfill\break
Yoshikage waited by the side of the side door and was ready to be of service. \hfill\break
From the 増鏡. }

\par{2. その ${\overset{\textnormal{みかど}}{\text{帝}}}$ のみこたかい子と ${\overset{\textnormal{まう}}{\text{申}}}$ すいまそがりけり。 \hfill\break
There was a child of that emperor called Takaiko. \hfill\break
From the 伊勢物語. }

\par{3. 月の都 の 人にて、 ${\overset{\textnormal{ちちはは}}{\text{父母}}}$ あり。 \hfill\break
I am a person of the capital of the moon, and I have a father and mother. \hfill\break
From the 竹取物語. }

\par{4. この ${\overset{\textnormal{くに}}{\text{國}}}$ にある物 \hfill\break
Something that exists in this country \hfill\break
From the 竹取物語. }

\par{5. いづくにもあれ \hfill\break
Where it may be \hfill\break
From the 徒然草. }

\par{6. さすがに住む人のあればなるべし。 \hfill\break
As expected, it is definitely because there are people living there. \hfill\break
From the 徒然草. }

\par{7. 深き ${\overset{\textnormal{ゆゑ}}{\text{故}}}$ あらん。 \hfill\break
There is a probably a deep reason. \hfill\break
From the 徒然草. }

\par{8. 昔 ${\overset{\textnormal{をとこ}}{\text{男}}}$ ありけり。 \hfill\break
There was a man long ago. \hfill\break
From the 伊勢物語. }

\par{9. 世の中にある人 \hfill\break
A person who exists in the world \hfill\break
From the 古今和歌集. }

\par{10. かの白く咲けるをなん「 ${\overset{\textnormal{ゆふがほ}}{\text{夕顏}}}$ 」と ${\overset{\textnormal{まう}}{\text{申}}}$ しはべる。 \hfill\break
That flower blooming white is called the "evening face". \hfill\break
From the 源氏物語. }

\par{11. ${\overset{\textnormal{ふたり}}{\text{二人}}}$ して打たんには、 ${\overset{\textnormal{はべ}}{\text{侍}}}$ りなむや。 \hfill\break
If two people were to hit (the dog), would it still live? \hfill\break
From the 枕草子. }

\par{12. ${\overset{\textnormal{じんりん}}{\text{人倫}}}$ にあらず。 \hfill\break
One would not be human. \hfill\break
From the 徒然草. }

\par{13. かかる所に住む人、心に思ひ ${\overset{\textnormal{のこ}}{\text{殘}}}$ すことはあらじかし。 \hfill\break
A person living in this kind of place would probably not have anything to regret. \hfill\break
From the 源氏物語. }

\par{14. 立礼杼毛 居礼杼毛 \hfill\break
立てれども ${\overset{\textnormal{を}}{\text{居}}}$ れども \hfill\break
Whether standing or sitting \hfill\break
From the 万葉集. }

\par{\textbf{Modern Remnant Note }: You can still find あり as is in a few set phrases like ${\overset{\textnormal{ぼうちゅうかん}}{\text{忙中閑}}}$ あり which means "to enjoy a moment of relief through one's work". }

\par{\textbf{Further Study Note }: Other ラ変 verbs formed by the fusing of あり with other things to consider include ${\overset{\textnormal{さ}}{\text{然}}}$ り "to be like" and ${\overset{\textnormal{しか}}{\text{然}}}$ り "to be so". }
      
\section{ナ行変格動詞}
 
\par{The two ナ変 verbs in Classical Japanese are 死ぬ and 往ぬ, which mean "to die" and "to go" respectively. Their bases are just like 四段 verbs with exception to the 連体形 and the 已然形. In fact, this class eventually dissimulate quickly starting in the 江戸時代. We will use the verb 死ぬ to illustrate the bases. }

\begin{ltabulary}{|P|P|P|P|P|P|}
\hline 

未然形 & 連用形 & 終止形 & 連体形 & 已然形 & 命令形 \\ \cline{1-6}

しな & しに & しぬ & しぬる & しぬれ & しね \\ \cline{1-6}

\end{ltabulary}

\begin{center}
 \textbf{Examples }
\end{center}

\par{15. 死なば ${\overset{\textnormal{いつしよ}}{\text{一所}}}$ で死なん。 \hfill\break
If we are going to die, let's die in one place. \hfill\break
From the 平家物語. }

\par{16. 北の方、 ${\overset{\textnormal{にく}}{\text{憎}}}$ し、とく死ねかしと思ふ。 \hfill\break
The principal wife thought, "Despicable, die quickly!" \hfill\break
From the 落窪の細道. }

\par{17. ${\overset{\textnormal{ほのほ}}{\text{焰}}}$ にまぐれてたちまちに死ぬ。 \hfill\break
They are engulfed in the flames and immediately die. \hfill\break
From the 方丈記. }

\par{18. 水におぼれて死なば死ね。 \hfill\break
If we're going to drown in the water, let's die! \hfill\break
From the 平家物語. }

\par{19. 「 ${\overset{\textnormal{い}}{\text{往}}}$ ね」といひければ \hfill\break
Since she said, "Leave", \hfill\break
From the 大和物語. }
      
\section{Exercises}
 
\par{1. What is the significance of ラ変 verbs ending in い? }

\par{2. What differentiates ラ変 and 四段 verbs? }

\par{3. What differentiates ナ変 and 四段 verbs? }

\par{4. When did the ナ変 class begin to disappear? }

\par{5. What is いまそがり a contraction of? }

\par{6. Give the bases of 居り. }

\par{7. Create a simple sentence with 死ぬ. }

\par{8. What is the only 五段 verb still used today that ends in -ぬ? }

\par{9. Give the bases for 往ぬ. }

\par{10. When did the 終止形 and the 連体形 begin to fuse for あり? }
    