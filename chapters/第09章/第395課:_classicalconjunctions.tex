    
\chapter{Conjunctions}

\begin{center}
\begin{Large}
第395課: Conjunctions 
\end{Large}
\end{center}
 
\par{This lesson will hopefully be much easier than the syntax we've been looking at thus far. Conjunctions are pretty much used the same as they are today in Classical Japanese. Some have changed in spelling, appearance, or have been replaced by other words. The most difficult ones will more than likely be the ones that you have never seen before. Try not to let this get the best of you though. }
      
\section{接続詞}
 
\par{ Conjunctions, again, are \textbf{dependent non-inflected }words that connect things together. Conjunctions in Japanese can be categorized by the following six main types. By no means is the exclusive, and you may find that some words can have more than one interpretation depending on how it is used. }

\begin{ltabulary}{|P|P|P|}
\hline 

Function & 漢字 & かな \\ \cline{1-3}

Parallelism & 並行 & へいれつ \\ \cline{1-3}

Alternation & 代替 & だいがえ \hfill\break
\\ \cline{1-3}

Addition & 添加 & てんか \\ \cline{1-3}

Change & 転換 & てんかん \\ \cline{1-3}

Concession & 逆接 & ぎやくせつ \\ \cline{1-3}

Sequence & 連続 & れんぞく \\ \cline{1-3}

\end{ltabulary}

\par{Many conjunctions in Japanese originate from different grammatical parts of speech. Many of these include some sort of verb or adverb which are fixated to a particle. Then, they are treated as one word. We have already studied adverbs such as されど (however). されど comes from the ラ変 verb さり, which comes from the contraction of さあり. さり means "to be that way". The conjunctive particle do means "though", and together it expresses the word "however; nevertheless; be that as it may". }

\par{This next section will involve examples. The conjunction(s) in the examples will be made bold for identification along with their definitions in the translation. This will help you realize how they are supposed to be used in context. }
      
\section{Conjunctions in Context}
 
\par{1. 枝の長さ七尺、 \textbf{あるいは }${\overset{\textnormal{ろくしやく}}{\text{六尺}}}$ 。 \hfill\break
The length of the branch is \textbf{either }seven \textbf{or }six shaku. \hfill\break
From the 徒然草. }

\par{\textbf{Meaning Note }: A 尺 is equivalent to 30.3 cm. }

\par{2. ゆくの ${\overset{\textnormal{かは}}{\text{河}}}$ の流れは絶えずして、 \textbf{しかも }もとの ${\overset{\textnormal{みづ}}{\text{水}}}$ にあらず。 \hfill\break
The flowing of the passing water is endless, yet \textbf{moreover }it is not the original water. \hfill\break
From the 方丈記. }

\par{3. \textbf{さらば }ゆるさむ。 \hfill\break
 \textbf{Then }, I will let you go. \hfill\break
From the 紫式部日記. }

\par{4. まして \textbf{その外 }、数へ知るに及ばず。 \hfill\break
 \textbf{Besides }, you can't much less count (the houses). \hfill\break
From the 方丈記. }

\par{5. 力を尽くしたること少なからず。 \textbf{しかるに }、 ${\overset{\textnormal{ろく}}{\text{禄}}}$ いまだ ${\overset{\textnormal{たま}}{\text{賜}}}$ はらず。 \hfill\break
The strength we expended wasn't small. Even so, we still haven't received our stipends. \hfill\break
From the 竹取物語. }

\par{6. ${\overset{\textnormal{すざくいん}}{\text{朱雀院}}}$ \textbf{ならびに }村上の ${\overset{\textnormal{おん}}{\text{御}}}$ をぢにをはします。 \hfill\break
He was the uncle of the retired Suzaku Emperor and Emperor Murakami. \hfill\break
From the 大鏡. }

\par{7. \textbf{さて }${\overset{\textnormal{ふゆが}}{\text{冬枯}}}$ れのけしきこそ、秋にはをさをさおとるまじけれ。 \hfill\break
 \textbf{Well }, a withered winter landscape would not certainly be at all inferior to that of autumn! \hfill\break
From the 徒然草. }

\par{8. ${\overset{\textnormal{よど}}{\text{淀}}}$ みに ${\overset{\textnormal{うか}}{\text{浮}}}$ ぶうたかたは、 \textbf{かつ }消え、 \textbf{かつ }結びて、久しくとどまりたる ${\overset{\textnormal{ためし}}{\text{例}}}$ なし。 \hfill\break
As for the bubble afloat on the pool, on the one hand, just as one thinks it disappears it reappears,          and there is never a case where it is ever the same shape.| \hfill\break
From the 方丈記. }

\par{9. ${\overset{\textnormal{みちのく}}{\text{陸奥}}}$ のしのぶもぢずり誰 \textbf{ゆゑに }乱れそめにしわれならなくに \hfill\break
Like the clothing pattern "Shinobumojizuri" weaved in Michinoku, \textbf{ }whom is it \textbf{hence }that it has begun       to be tussled, although it is not because of me. \hfill\break
From the 百人一首. }

\par{10. そも、 ${\overset{\textnormal{まゐ}}{\text{参}}}$ りたる人ごとに山へ登りしは、 ${\overset{\textnormal{なにごと}}{\text{何事}}}$ かありけん、ゆかしかりしかど、神へ参るこそ ${\overset{\textnormal{ほんい}}{\text{本意}}}$ なれと思ひて、山までは見ずとぞ言ひける。 \hfill\break
"Even so, all the people coming to worship climbing the mountain thought 'did something happen? We were eager to know, but worshiping the god(s) is our primary goal', and they did see up to the mountain" said (the priest). \hfill\break
From the 徒然草. }
      
\section{Examples}
 
\par{1. Create a sentence with しかし (however) }

\par{2. Create a sentence with しかして (moreover) }

\par{3. Create a sentence with したがって (therefore; as a result). }

\par{4. なほ is just like its modern なお. What two usages does it have, which are classified as different parts of speech? Look through this lesson and see which was used. Then, differentiate it with a simple example of the other in Classical Japanese to the best of your ability. }

\par{5. Translate the following sentence into English. }

\par{はかばかしき後見しなければ、事あるときは、なほ拠り所なく心細げなり。 \hfill\break
はかばかしきうしろみしなければ、ことあるときは、なほよりどころなくこころぼそげなり。 }

\par{Hints: The particle し provides emphasis. Also, なければ in Classical Japanese means "since not have". }
    