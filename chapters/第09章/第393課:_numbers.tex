    
\chapter{Numbers XI}

\begin{center}
\begin{Large}
第393課: Numbers XI: ひふみ 
\end{Large}
\end{center}
 
\par{ Before Sino-Japanese words were introduced to 和 (ancient Japan), there were only native numbers. Counters still existed, but some of them looked different. Counters were limited to native numbers before the flood of Sino-Japanese words into Japanese. }

\par{ Although Sino-Japanese numbers and counters have been used for much of written history, it has only been the case fairly recently counter phrases became so difficult to read. Sino-Japanese numbers were used with Sino-Japanese counters. So, even 400 would have been しひゃく, which can be seen in Modern Japanese in the set phrases ${\overset{\textnormal{しひゃくよしゅう}}{\text{四百余州}}}$ (all of China; literally "approximately 400 provinces") and ${\overset{\textnormal{しひゃくしびょう}}{\text{四百四病}}}$ (every kind of disease; literally "four hundred four" diseases) }
      
\section{Native Numbers 0~99,999}
 
\par{Many people are surprised to find out that native numbers could actually reach from 0~99,999. Whether it was actually used that far is debatable and highly unlikely, but the patterns are there to reach that high. }

\par{The number 0 was represented with the word なし, literally meaning "nothing",  and 99,999 would be ここのよろず ここのほ ここのそ あまり ここの. Of course, when \textbf{counting things }, these numbers had to be used with native counters. The basic native counter to count things as well as the basic numbers can be summarized as follows. }

\begin{ltabulary}{|P|P|P|}
\hline 

 & \textbf{Number }& \textbf{Counter Phrase }\\ \cline{1-3}

0 & なし & なし \\ \cline{1-3}

1 & ひと & ひとつ \\ \cline{1-3}

2 & ふた & ふたつ \\ \cline{1-3}

3 & み(い) & み(い)つ \\ \cline{1-3}

4 & よ(を) & よ(を)つ \\ \cline{1-3}

5 & いつ & いつつ \\ \cline{1-3}

6 & む(ゆ) & む(ゆ)つ \\ \cline{1-3}

7 & なな & ななつ \\ \cline{1-3}

8 & や(わ) & や(わ)つ \\ \cline{1-3}

9 & ここの & ここのつ \\ \cline{1-3}

10 & とを & とを \\ \cline{1-3}

11 & とをあまりひと & とをあまりひとつ \\ \cline{1-3}

20 & はた & はたち \\ \cline{1-3}

30 & みそ & みそぢ \\ \cline{1-3}

40 & よそ & よそぢ \\ \cline{1-3}

50 & いそ & いそぢ \\ \cline{1-3}

60 & むそ & むそぢ \\ \cline{1-3}

70 & ななそ & ななそぢ \\ \cline{1-3}

80 & やそ & やそぢ \\ \cline{1-3}

90 & ここのそ & ここのそぢ \\ \cline{1-3}

100 & もも & もも \\ \cline{1-3}

200 & ふたほ & ふたほ \\ \cline{1-3}

300 & みほ & みほ \\ \cline{1-3}

400 & よほ & よほ \\ \cline{1-3}

500 & いつほ & いつほ \\ \cline{1-3}

600 & むほ & むほ \\ \cline{1-3}

700 & ななほ & ななほ \\ \cline{1-3}

800 & やほ & やほ \\ \cline{1-3}

900 & ここのほ & ここのほ \\ \cline{1-3}

1000 & ち & ち \\ \cline{1-3}

2000 & ふたち & ふたち \\ \cline{1-3}

3000 & みち & みち \\ \cline{1-3}

4000 & よち & よち \\ \cline{1-3}

5000 & いつち & いつち \\ \cline{1-3}

6000 & むち & むち \\ \cline{1-3}

7000 & ななち & ななち \\ \cline{1-3}

8000 & やち & やち \\ \cline{1-3}

9000 & ここのち & ここのち \\ \cline{1-3}

10000 & よろづ & よろづ \\ \cline{1-3}

20000 & ふたよろづ & ふたよろづ \\ \cline{1-3}

30000 & みよろづ & みよろづ \\ \cline{1-3}

40000 & よよろづ & よよろづ \\ \cline{1-3}

50000 & いよろづ & いよろづ \\ \cline{1-3}

60000 & むよろづ & むよろづ \\ \cline{1-3}

70000 & ななよろづ & ななよろづ \\ \cline{1-3}

80000 & やよろづ & やよろづ \\ \cline{1-3}

90000 & ここのよろづ & ここのよろづ \\ \cline{1-3}

\end{ltabulary}

\par{\textbf{Origin Notes }: You can skip this if you have no interest in etymology. For those that find it interesting, have fun. }

\par{1. Without a counter, numbers were originally disyllabic words. }

\par{2. The sounds in parentheses would drop out and or stay in some words. For instance, むゆか (six days; sixth day of the month), survives in Modern Japanese as むいか. }

\par{3. It is posited by some that 5 was originally just い and that つ was added to make it disyllabic, but the sense of the word being two morphemes was lost, which gave way for いつつ to be formed. }

\par{4. A rather interesting theory as to the ultimate origin of many Japanese numbers is that 2, 6, 8, and 10 descend from Austronesian languages, and because the pairs <1 2>, <3 6>, <4 8>, and <5 10> appeared to have interchangeable function in the ancient period and the origins of these pairs coincide with each other, it's very plausible. If you go back to these numbers, you can still see that these pairs resemble each other, with only the last pair being more different than the rest. }

\par{ Below is a chart summarizing this theory. This theory is more generous to meaning changes. }

\begin{ltabulary}{|P|P|P|P|}
\hline 

 & 日本語の数詞 & 南祖の対応形 & 意味 \\ \cline{1-4}

1 & ɸi \emph{tə }& *i \emph{sa }& 1 \\ \cline{1-4}

2 & ɸu \emph{ta }& *i \emph{sa }& 2 \\ \cline{1-4}

3 & mi < *mi \emph{y }< *mi \emph{yi }& *te \emph{lu }& 3 \\ \cline{1-4}

4 & yə < *y \emph{əw }< *yə \emph{wə }& *em \emph{pat }& 4 \\ \cline{1-4}

5 & itu < * \emph{yi }& * \emph{li }ma & 5 \\ \cline{1-4}

6 & mu, mu \emph{i }, mu \emph{yu }& *te \emph{lu }& 3 \\ \cline{1-4}

7 &  \emph{nana }(鼻音重複形) & *i \emph{sa }& 1 \\ \cline{1-4}

8 & ya <.*ya \emph{w }< *ya \emph{wa }& *em \emph{pat }& 4 \\ \cline{1-4}

9 &  \emph{kəkənə }& * \emph{genep }& 数が揃った \\ \cline{1-4}

10 &  \emph{təwo }& * \emph{sa( }ŋ \emph{)pul }uq \hfill\break
& 1つの10 \\ \cline{1-4}

20 &  \emph{ɸata }& * \emph{pasa }ŋ & 対 \\ \cline{1-4}

百 & - \emph{ɸo, momo }(鼻音重複形) & * \emph{pu }luq & 10 \\ \cline{1-4}

千 &  \emph{ti }& * \emph{ri }bu & 千 \\ \cline{1-4}

万 &  \emph{yərədu }& * \emph{(n)teReb }& 万, 豊か \\ \cline{1-4}

\end{ltabulary}

\par{\textbf{Chart Note }: 南祖 stands for Southern Austronesian reconstruction used in this theory. }

\par{\textbf{Transcription Note }: y is not written in IPA as j. }

\par{\textbf{Citation Note }: This is from Pg 171 of 川本崇雄\textquotesingle s 南から来た日本語. }

\par{5. There's also the problem of Japanese-related numerals being found in recovered 高麗語, which would have been spoken in the Korean peninsula. This further complicates the problem, but that's what happens when all your evidence goes away into the abyss of history. }

\par{6. For those that think Japanese is Altaic in origin, the number 4 is their friend because they think よ comes from dər. 7 is akin to the Proto-Manchu word *nadan 8 is then akin to the Manchu word japkun. The biggest problem that these people have is that they can't get anywhere close to accounting for the entire number system. Despite lack of evidence and hard to follow sound correspondences, the previous theory covers far more territory. }

\par{ \textbf{Counters }}

\par{As far as counters are concerned, the exceptions found in Modern Japanese with Sino-Japanese endings are the same for the most part. For example, the counter for people is -人(にん) for Sino-Japanese numbers and ~人(たり) for native numbers. ~人 is ~人(り) for ひと (1). In Modern Japanese, the native counter is normally only used for 1-2, but in Classical Japanese there was no such restriction. Native counters can be used with any native number, except 0 of course and likewise Sino-Japanese counters with Sino-Japanese numbers. }

\par{There are some counters that are noticeably \textbf{different }. For example, the counter for year is とせ instead of とし. The counter for days was ~か too, but it was used with any number. For example, 100 days was ももか. The exceptions with ~か were minus small changes. っ found in some can't be attested throughout history because っ wasn't written until much later. But, they were probably pronounced with them. 6 days (of the month) and 7 days (of the month) in particular were むゆか and なぬか instead of むいか and なのか respectively. }

\begin{center}
\textbf{Examples }
\end{center}

\par{1. また、 ${\overset{\textnormal{ぢしよう}}{\text{治承}}}$ ${\overset{\textnormal{よとせ}}{\text{四年}}}$ ${\overset{\textnormal{みなづき}}{\text{水無月}}}$ のころ、にはかに ${\overset{\textnormal{みやこ}}{\text{都}}}$ ${\overset{\textnormal{うつ}}{\text{遷}}}$ りはべりき。 \hfill\break
Again, around the sixth month and fourth year of the Jishou Era, there was a sudden capital move. \hfill\break
From the 方丈記. }
 
\par{2. おほかた、この ${\overset{\textnormal{みやこ}}{\text{京}}}$ の ${\overset{\textnormal{はじ}}{\text{初}}}$ めを聞けることは、 ${\overset{\textnormal{さが}}{\text{嵯峨}}}$ の ${\overset{\textnormal{てんわう}}{\text{天皇}}}$ の ${\overset{\textnormal{おんとき}}{\text{御時}}}$ 、都と定まりにけるより後、すでに四百 ${\overset{\textnormal{よさい}}{\text{余歳}}}$ を \hfill\break
経たり。 \hfill\break
To begin with, as for hearing about the rise of Heiankyou, it has already been four hundred years           since the capital was determined in the reign of Emperor Saga. \hfill\break
From the 方丈記. }

\par{3. ${\overset{\textnormal{じふいちにち}}{\text{十一日}}}$ の月もかくれなむとすれば \hfill\break
When the moon of the eleventh day was about to disappear \hfill\break
From the 伊勢物語. }

\par{4. 八雲 やくも \hfill\break
 Eight\slash a lot of clouds }
 
\par{5. 同じき ${\overset{\textnormal{なぬかのひ}}{\text{七日}}}$ \hfill\break
On the same seventh day of the month \hfill\break
From the 平家物語. }
 
\par{6. 一坏乃 濁酒乎 可飲有良師 \hfill\break
${\overset{\textnormal{ひとつき}}{\text{一杯}}}$ の ${\overset{\textnormal{にご}}{\text{濁}}}$ れる酒を飲むべくあるらし。 \hfill\break
It seems better to drink a cup of cloudy sake. \hfill\break
From the 万葉集. }
 
\par{7. 二人して打たんには、 ${\overset{\textnormal{はべ}}{\text{侍}}}$ りなむや。 \hfill\break
If two people hit (the dog), would it still live? \hfill\break
From the 枕草子. }

\par{8. ${\overset{\textnormal{むゆか}}{\text{六日}}}$ 、きのふのごとし。 \hfill\break
The sixth. The same as yesterday. \hfill\break
From the 土佐日記. }

\par{9. ${\overset{\textnormal{つかさ}}{\text{官}}}$ ・ ${\overset{\textnormal{くらゐ}}{\text{位}}}$ に ${\overset{\textnormal{おも}}{\text{思}}}$ ひをかけ、 ${\overset{\textnormal{しゅくん}}{\text{主君}}}$ のかげを頼むほどの人は、 ${\overset{\textnormal{ひとひ}}{\text{一日}}}$ なりとも ${\overset{\textnormal{と}}{\text{疾}}}$ く ${\overset{\textnormal{うつ}}{\text{移}}}$ ろはむとはげみ、時を失ひ世に ${\overset{\textnormal{あま}}{\text{余}}}$ れて ${\overset{\textnormal{ご}}{\text{期}}}$ する所なきものは、 ${\overset{\textnormal{うれ}}{\text{愁}}}$ へながら止まり ${\overset{\textnormal{を}}{\text{居}}}$ り。 \hfill\break
People who pinned their hopes on people of rank and relied on their favor, even for a day quickly try   to use their energies to move, and people who were left behind by society and had nothing to hope for stay put and complain. \hfill\break
From the 方丈記. }

\par{\textbf{Reading Note }: 一日 could also be read as ひとひ in the ancient period. Although つひたち and いちにち existed as well, this reading could also be used to mean "the other day" or "all day". The latter reading is actually acceptable today in more so written Japanese, but it is normally replaced with 終日 or 一日中.  }
       
\section{Exercises}
 
\par{1. Give the Sino-Japanese numbers from 1-10 in 歴史仮名遣い. }

\par{2. Give the native numbers from 1-10 in 歴史仮名遣い. }

\par{3. Show a counter phrase different from Modern Japanese. }

\par{4. What is 101 in the native set of numbers? }
    