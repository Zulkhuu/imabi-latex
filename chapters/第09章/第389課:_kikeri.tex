    
\chapter{The Auxiliary Verbs ~き \& ~けり}

\begin{center}
\begin{Large}
第389課: The Auxiliary Verbs ~き \& ~けり 
\end{Large}
\end{center}
 
\par{The auxiliary verb ~き indicates (personal) past and shows direct recollection. However, the auxiliary verb ~けり indicates hearsay past and shows transmitted recollection. }
      
\section{The Auxiliary Verb -き}
 
\par{~き indicates a past that is distant and separate from the present; ~けり looks retrospectively at the past. This, though, is speculative and not necessarily the case in each instance. However, this is typically how it is described. So, this site will go with the traditional explanation. }

\par{~き has an irregular conjugation and follows the 連用形; however, when used with サ変 or カ変 verbs, it may instead follow the 未然形. Its bases are: }

\begin{ltabulary}{|P|P|P|P|P|P|}
\hline 

未然形 & 連用形 & 終止形 & 連体形 & 已然形 & 命令形 \\ \cline{1-6}

せ & X & き & し & しか & X \\ \cline{1-6}

\end{ltabulary}

\par{\textbf{Conjugation Note }: ~き neither has a 連用形 nor a 命令形, and it is rarely ever seen in the 未然形. }

\begin{center}
\textbf{Examples } 
\end{center}

\par{1. ${\overset{\textnormal{きやう}}{\text{京}}}$ より下りし時に、みな人、子どもなかりき。 \hfill\break
When we left the capital, no one had children. \hfill\break
From the 土佐日記. }

\par{2. つひにゆく道とはかねて聞きしかど ${\overset{\textnormal{きのふけふ}}{\text{昨日今日}}}$ とは思はざりしを。 \hfill\break
I had heard before about the path that we go in the end, but I didn't think that it would be yesterday or     today. \hfill\break
From the 古今和歌集. }

\par{3. ${\overset{\textnormal{かたとき}}{\text{片時}}}$ の ${\overset{\textnormal{あひだ}}{\text{間}}}$ とて、かの國よりまうで ${\overset{\textnormal{こ}}{\text{來}}}$ しかども、かく、この國にはあまたの年を ${\overset{\textnormal{へ}}{\text{經}}}$ ぬるになむありける。 \hfill\break
Thinking that it would be momentary, I came from that country, but I ended up like this in this country       spending several years. \hfill\break
From the 竹取物語. }

\par{\textbf{Particle Note }: The bound particle なむ in the example above is used for emphatic purposes. }

\par{4. ${\overset{\textnormal{おに}}{\text{鬼}}}$ のやうなるもの ${\overset{\textnormal{い}}{\text{出}}}$ で來て殺さむとしき。 \hfill\break
Demon like things came out and tried to kill us. \hfill\break
From the 竹取物語. }

\par{5. ${\overset{\textnormal{ひとよ}}{\text{一夜}}}$ のうちに ${\overset{\textnormal{ちりはい}}{\text{塵灰}}}$ となりにき。 \hfill\break
In one night, (the buildings) ended up becoming dust and ashes. \hfill\break
From the 方丈記. }

\par{6. よべもすずろに起きあかしてき。 \hfill\break
Somehow or other, I also got up last night and ended up staying up until dawn. \hfill\break
From the 源氏物語. }

\par{7. 和歌の ${\overset{\textnormal{じやうず}}{\text{上手}}}$ 、 ${\overset{\textnormal{くわんげん}}{\text{管弦}}}$ の道にもすぐれ ${\overset{\textnormal{たま}}{\text{給}}}$ へりき。 \hfill\break
He had been a master of poetry, and he had also excelled in music. \hfill\break
From the 大鏡. }

\par{8. きし ${\overset{\textnormal{かたゆ}}{\text{方行}}}$ く ${\overset{\textnormal{すゑ}}{\text{末}}}$ も知らず海にまぎれむとしき。 \hfill\break
Not knowing the direction I had come from nor where I was going, it seemed that I'd get lost. \hfill\break
From the 竹取物語. }

\par{9. 雨のいたく降りしかば、え ${\overset{\textnormal{まゐ}}{\text{參}}}$ らずなりにき。 \hfill\break
Since the rain fell hard, I wasn't able to visit. \hfill\break
From the 大和物語. }

\par{10. この寺にありし ${\overset{\textnormal{げんじ}}{\text{源氏}}}$ の ${\overset{\textnormal{きみ}}{\text{君}}}$ こそおはしたなれ。 \hfill\break
They say that Lord Genji, who was at this temple, came. \hfill\break
From the 源氏物語. }

\par{\textbf{Fossilization Note }: You can still see this auxiliary verb in a few expressions in Modern Japanese such as in ${\overset{\textnormal{あ}}{\text{在}}}$ りし日 which means "the olden days". Another one is 聞きしに ${\overset{\textnormal{まさ}}{\text{勝}}}$ る which means "to go beyond one's expectations". }
      
\section{The Auxiliary Verb -けり}
 
\par{~けり follows the 連用形 and has a ラ-変 conjugation. As said before, it is important in showing hearsay past and direct past (recollection perspective). ~けり also has an exclamatory element to it where it can be viewed as the modern だったな. This, in turn, can show discovery. Both of these functions can overlap, especially in poetry. Its bases are: }

\begin{ltabulary}{|P|P|P|P|P|P|}
\hline 

未然形 & 連用形 & 終止形 & 連体形 & 已然形 \hfill\break
& 命令形 \\ \cline{1-6}

けら & X & けり & ける & けれ & X \hfill\break
\\ \cline{1-6}

\end{ltabulary}

\par{\textbf{Conjugation Note }: The bases are limited for the same reasons the bases for ~き are. }

\begin{center}
 \textbf{Examples }
\end{center}

\par{11. 行かずなりにけり。 \hfill\break
I ended up not going. \hfill\break
From the 伊勢物語. }

\par{12. ${\overset{\textnormal{か}}{\text{枯}}}$ れ ${\overset{\textnormal{えだ}}{\text{枝}}}$ に ${\overset{\textnormal{からす}}{\text{烏}}}$ のとまりけり秋の ${\overset{\textnormal{く}}{\text{暮}}}$ れ。 \hfill\break
A crow has stopped on a withered branch, an autumn evening! \hfill\break
From 芭蕉. }

\par{13. むかし、をとこ、身はいやしくて、いとになき人を思ひかけたりけり。 \hfill\break
They say that long ago there was a man who was of low status who fell in love with a person of high        status. \hfill\break
From the 伊勢物語. }

\par{14. このをとこ、かいまみてけり。 \hfill\break
This man ended up looking through the hedge (at the sisters). \hfill\break
From the 伊勢物語. }

\par{15. その人の名忘れにけり。 \hfill\break
I ended up forgetting that person's name. \hfill\break
From the 伊勢物語. }

\par{16. 「ここやいどこ」と問ひければ、「土佐の ${\overset{\textnormal{とまり}}{\text{泊}}}$ 」と言ひけり。 \hfill\break
When I asked, "Where is this place", they said, "The landing at Tosa". \hfill\break
From the 土佐日記. }

\par{17. その人、ほどなく ${\overset{\textnormal{う}}{\text{失}}}$ せにけりと聞き ${\overset{\textnormal{はべ}}{\text{侍}}}$ りし。 \hfill\break
I heard people say that person passed away before long. \hfill\break
From the 徒然草. }

\par{18. 見渡せば ${\overset{\textnormal{やなぎさくら}}{\text{柳櫻}}}$ をこきまぜて ${\overset{\textnormal{みやこ}}{\text{都}}}$ ぞ春の ${\overset{\textnormal{にしき}}{\text{錦}}}$ なりける。 \hfill\break
When I looked across the willow and cherry blossom trees mix together, the capital was a brocade of       spring! \hfill\break
From the 古今和歌集. }

\par{19. わたりし時は ${\overset{\textnormal{みづ}}{\text{水}}}$ ばかり見えし田どもも、みな刈りはててけり。 \hfill\break
I found that even the rice fields that had appeared filled with water when I came here had all been           completely harvested. \hfill\break
From the 更級日記. }

\par{20. 今は昔、 ${\overset{\textnormal{たけとり}}{\text{竹取}}}$ の ${\overset{\textnormal{おきな}}{\text{翁}}}$ といふものありけり。 \hfill\break
Now in the distant past, it is said that there was a person called "Old Man the Bamboo Cutter". \hfill\break
From the 竹取物語. }

\par{21. うちおどろきたれば、夢なりけり。 \hfill\break
When I suddenly woke up, I realized that it was just a dream! \hfill\break
From the 更級日記. }

\par{22. 昔、 ${\overset{\textnormal{をとこ}}{\text{男}}}$ ありけり。身はいやしながら、母なむ宮なりける。 \hfill\break
It is said that long ago there was a man. While his status was low, his mother was a princess. \hfill\break
From the 伊勢物語. }

\par{23. まことかと聞きて見つれば、 ${\overset{\textnormal{こと}}{\text{言}}}$ の ${\overset{\textnormal{は}}{\text{葉}}}$ を ${\overset{\textnormal{かざ}}{\text{飾}}}$ れる玉の枝にぞありける。 \hfill\break
He listened, wondering if it was real, and when he looked closely, he discovered that it was a         jeweled branch. \hfill\break
From the 竹取物語. }

\par{24. ${\overset{\textnormal{にんな}}{\text{仁和}}}$ のみかど、みこにおはしましける時に、人にわかなたまひける ${\overset{\textnormal{おほむ}}{\text{御}}}$ うた。 \hfill\break
An honorable poem from when Emperor Ninna was a prince and bestowed young herbs on a         person. \hfill\break
From the 古今和歌集. }

\par{25. 野を見れば春めきにけり。 \hfill\break
When I look at the wild field, it has become spring-like. \hfill\break
From the 拾遺集. }

\par{26. ${\overset{\textnormal{こよひ}}{\text{今夜}}}$ は ${\overset{\textnormal{じふごや}}{\text{十五夜}}}$ なりけり。 \hfill\break
Tonight is the fifteenth night! \hfill\break
From the 源氏物語. }

\par{27. 鳴く ${\overset{\textnormal{こゑ}}{\text{声}}}$ 、 ${\overset{\textnormal{ぬえ}}{\text{鵺}}}$ にぞ似たりける。 \hfill\break
The crying voice resembled that of the nue. \hfill\break
From the 平家物語. }

\par{28. ${\overset{\textnormal{あうくわ}}{\text{櫻花}}}$ 咲き染めにけり。 \hfill\break
The cherry blossoms have blossomed and tinted. }

\par{29. その根のありければ、きりくひの ${\overset{\textnormal{そうじやう}}{\text{僧正}}}$ といひけり。 \hfill\break
Since that tree stump was there, people called him "Archbishop Tree Stump". \hfill\break
From the 徒然草. }

\par{30. 一來 ${\overset{\textnormal{ほふしうちじ}}{\text{法師打死}}}$ にしてんげり。 \hfill\break
Priest Ichirai ended up dying in battle. \hfill\break
From the 平家物語. }

\par{31. 在原なりける ${\overset{\textnormal{をのこ}}{\text{男}}}$ 。 \hfill\break
The man who was of the Ariwara Clan. }

\par{32. ${\overset{\textnormal{たなばた}}{\text{七夕}}}$ まつるこそなまめかしけれ。 \hfill\break
Celebrating Tanabata is indeed elegant. \hfill\break
From the 徒然草. }

\par{33. ${\overset{\textnormal{くちなは}}{\text{蛇}}}$ をば ${\overset{\textnormal{おほゐがは}}{\text{大井川}}}$ に流してけり。 \hfill\break
He ended up throwing the snake into the Ooi River. \hfill\break
From the 徒然草. }

\par{\textbf{Historical Note }: ~けり came from the combination of the ~き and あり. }

\par{\textbf{Modern Remnant }: ~けり survives in poetry today. The particle け actually comes from ~けり. It can also be seen in けりをつける meaning "to put an end to". It can also strength the verb 因る in よりけり to mean "depend on". }

\par{34. ことと次第によりけりだ。 \hfill\break
That all depends. }

\par{35. 話は条件によりけりだよ。 \hfill\break
The discussion all depends on the conditions. }
      
\section{Exercises}
 
\par{1.Describe the difference between -き and -けり. }

\par{2. What base(s) does -き follow? }

\par{3. What base(s) does -けり follow? }

\par{4. Create a simple sentence or phrase with -き. }

\par{5. Create a simple sentence or phrase with -けり. }

\par{6. List the bases for -き. }

\par{7. List the bases for -けり. }

\par{8. The 未然形 is seldom used for both き and けり. Why? }

\par{9. The 連用形 and 命令形 are not used for both き and けり. Why? }

\par{10. こしかば, what base of the verb and auxiliary verb is used in this expression? }
    