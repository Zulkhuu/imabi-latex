    
\chapter{Historical かな Orthography}

\begin{center}
\begin{Large}
第378課: Historical かな Orthography 
\end{Large}
\end{center}
 
\par{ This lesson will introduce you to historical かな usage and what it tells us about Japanese. The language has been changing ever since its inception. Many words\slash phrases have been or are homophonous. This is a primary source of confusion in understanding historical orthography correctly. }

\par{\textbf{Curriculum Note }: IPA (International Phonetic Alphabet) spelling will be used in this lesson to transcribe Japanese words. There will also be corresponding Japanese text. }
      
\section{Historical Kana Orthography}
 
\par{ Historical Kana usage ( ${\overset{\textnormal{れきしかなづか}}{\text{歴史仮名遣}}}$ い) was the orthography system used up until Kana simplification in the early 1900s. Historical Kana orthography typically refers to Kana spelling standards from when they were first established in the Heian Period to the early 1900s. Spelling variation and bad spellers during all of this time does cause orthography to be even more difficult to understand. }

\par{ In addition to odd spellings with existing Kana, in original texts, you will also encounter obsolete Kana, which are called 変体仮名・異体仮名. However, due to the fact that they can't be typed without the aid of special programs, they are typically ignored in studying Historical Kana Orthography. }

\begin{center}
 \textbf{The Traditional 五十音図 }
\end{center}

\par{ Below is the traditional 五十音図. This, again, does not include variant Kana that would have existed for these basic sounds. However, see if you can find Kana in this chart that have become obsolete and are not shown in modern renderings of this table. Lastly, as pronunciation has changed over the centuries, no sound correspondences will be given in this chart. }

\begin{ltabulary}{|P|P|P|P|P|}
\hline 

あ & い & う & え & お \\ \cline{1-5}

か & き & く & け & こ \\ \cline{1-5}

さ & し & す & せ & そ \\ \cline{1-5}

た & ち & つ & て & と \\ \cline{1-5}

な & に & ぬ & ね & の \\ \cline{1-5}

は & ひ & ふ & へ & ほ \\ \cline{1-5}

ま & み & む & め & も \\ \cline{1-5}

や &  & ゆ &  & よ \\ \cline{1-5}

ら & り & る & れ & ろ \\ \cline{1-5}

わ & ゐ &  & ゑ & を \\ \cline{1-5}

\end{ltabulary}

\par{\textbf{Chart Note }: Notice that ゐ and ゑ are obsolete. They will be looked at more closely later in this lesson. }
      
\section{Vowel Pronunciation:上古日本語~現代日本語}
 
\begin{center}
 \textbf{Old Japanese }
\end{center}

\par{${\overset{\textnormal{まんようがな}}{\text{万葉仮名}}}$ is the predecessor of かな. This is the basis, then, for the Kana spellings to come once subsets of the 漢字 that composed 万葉仮名 were simplified to create Kana. As far as what 万葉仮名 can tell us about Old Japanese (上古日本語; 347~794 A.D?) pronunciation, to most scholars, the system suggests that during and some time before the Nara Period, Japanese had approximately 8 vowels. As this would simplify into the 5 vowel system of today, the original vowel structure would leave an impression on the language's vocabulary and burgeoning spelling system. }

\par{ Below is Poem 707 from the 万葉集, a compilation of poems in Old Japanese that begin the literary history of Japanese. Notice also that unsimplified Kanji are used. Using resources if necessary, try matching the Kanji with the Kana associated with them. In going through this example, consider the following questions. }

\par{Question 1: What can you discern from this one example how 万葉仮名 was used? \hfill\break
Question 2: Are there any apparent archaic grammar structures that you can point out? \hfill\break
Question 3: What Kanji are unsimplified? How can you tell? }

\par{万葉仮名: 思 遣 爲便乃不知 者 片垸之   底曽吾者    戀成尓家類 注土垸之中 \hfill\break
歴史仮名遣い: おもひやる すべのしらねば かたもひの そこにぞあれは こひなりにける  Potential Pronunciation: Omopiyaru subenos(h)iraneba katamopino soko ni zo are pa kopi narinikeru  Note: This does not take into account different o and e vowels that more than likely existed at the time. }
 
\begin{center}
 \textbf{Middle Japanese }
\end{center}

\par{ It is generally believed that the vowels "e" and "o" up until Early Middle Japanese (中古日本語; 794~1185 A.D?) were pronounced as [je] and [wo]. This is demonstrated in transliterations done in the mid-1500s by the first foreigners from Europe to transcribe Japanese. Consider place name spellings such as Yedo for Edo, the original name of Tokyo. }

\par{ This is also verified by remote dialects in various parts of Japan. There are still 東北 dialects that retain the sound [je]. For instance, in ${\overset{\textnormal{つがるご}}{\text{津軽語}}}$ , which has evolved so much that it is often deemed its own language, you see examples such as はやい \textrightarrow  はーいぇ [ha:ye]. Though this can be accounted for the vowel combination \slash a.i\slash  contracting to \slash e\slash , therefore producing \slash ye\slash , it is not an isolated example. }
 
\par{ In Middle Japanese, the vowel combination \slash a.u\slash  contracted to [ɔː], which is like the o in the American English pronunciation of "bore". This suggests that there were still two o sounds for quite some time despite what spelling might suggest. It's just that one was preserved in a diphthong, which would then simplify to [o:]. }

\par{ The differences between Middle and Modern Japanese pronunciation can be summarized with this table. The arrows indicate what the original pronunciation(s) have become in Modern Japanese. As we do not have time machines, transcriptions for Middle Japanese are merely highly probably guesses on how they did originally sound. }

\begin{ltabulary}{|P|P|P|P|P|}
\hline 

え = \slash je\slash ,  \slash e\slash  \textrightarrow  [e] & えう = \slash e.u\slash  \textrightarrow  [jo ː] & お = \slash wo\slash , \slash o\slash  \textrightarrow  [o] & あう =\slash  ɔ ː \slash  \textrightarrow  [o:] & おう =\slash o.u \slash  \textrightarrow  [o:] \\ \cline{1-5}

\end{ltabulary}

\par{\textbf{Other Vowel Notes }: }

\par{1. The vowel a was likely more like the o in "stock" rather than the a in "car" as it is today for much of Japanese history. }
 
\par{2. It is also not certain whether u was ever compressed in the same way it is in Standard Modern Japanese pronunciation before modern times.  }

\par{\textbf{Long Vowels }}

\par{ Long vowel contrast in Modern Japanese is extremely important, but it's important to realize that at one point in Japanese history, there was no such system in the sense that there is today. Many long vowel spellings are simplifications. These unsimplified spellings tell us what previous pronunciations were and what changes occurred to lead to modern pronunciations. }

\par{ For instance, in the chart below, you see the old spellings of long vowel combinations. It also shows probably pronunciations of them. }

\begin{ltabulary}{|P|P|P|P|P|}
\hline 

Combination & Pronunciation at Inception & Late Middle Japanese & Today & Modern Spelling \\ \cline{1-5}

あふ & [apu] ? & [ ɔ ː] & [o:] & おう \\ \cline{1-5}

あう & [a.u] ? & [ ɔ ː] & [o:] & おう \\ \cline{1-5}

おふ & [opu] ? & [(w)o:] & [o:] & おう \\ \cline{1-5}

おう & [o.u] ? & [(w)o:] & [o:] & おう \\ \cline{1-5}

うふ & [upu] ? & [u:] & [ɯ:] & うう \\ \cline{1-5}

ゆふ & [jupu] ? & [ju:] & [jɯ:] & ゆう \\ \cline{1-5}

いふ & [ipu] ? & [ju:] & [jɯ:] & ゆう \\ \cline{1-5}

いう & [i.u] ? & [ju:] & [jɯ:] & ゆう* \\ \cline{1-5}

えふ & [epu] ? & [jo:] & [jo ː] & よう \\ \cline{1-5}

えう & [e.u] ? & [jo:] & [jo ː] & よう \\ \cline{1-5}

\end{ltabulary}

\par{\textbf{Chart Note }: Again, vowel pronunciation in particular environments in far older stages of Japanese is largely uncertain. Thus, the first column of pronunciations is greatly simplified. For clarity, the second column starts at the 室町時代. }

\par{*: The one exception is the verb 言う, which is spelled in a somewhat old-fashioned way although it was originally spelled as いふ. }

\par{  Sino-Japanese words greatly influenced Japanese pronunciation. The Chinese languages from which these words came did affect the vowel structure of Japanese. To what extent the Japanese assimilated these sounds to the existing phonology system and to what extent these words had on later developments to Japanese phonology are uncertain. Nevertheless, it is safe to say that thanks to Chinese, Modern Japanese has heavy short-long consonant and vowel contrasts. }
      
\section{Consonant Pronunciation: 上古日本語~現代日本語}
 
\par{ Consonants tend to not change much over time, but over the course of 2,000 years, significant shifts can be seen. Unless there is significant evidence for consonants having different pronunciations in Old Japanese than Early Modern Japanese, no column will be provided for Old Japanese. As for the Modern Japanese columns in the charts, know that the pronunciations chosen are based off of Standard Japanese pronunciation, and these pronunciations may not reflect the pronunciation of natives from other parts of Modern Japan. }

\par{\textbf{S-Sounds }}

\begin{ltabulary}{|P|P|P|P|P|}
\hline 

 & Old Japanese & Early Middle Japanese & Late Middle Japanese & Modern Japanese \\ \cline{1-5}

さ & [ʃa\slash sa] & [s a] & [s a] & [sa] \\ \cline{1-5}

し & [ʃi\slash si] & [ ʃi] & [ ɕ i] & [ ɕ i] \\ \cline{1-5}

す & [ʃu\slash su] & [su] & [su] & [sɯ] \\ \cline{1-5}

せ & [ʃe\slash se] & [ ʃe] & [ ɕ e] & [se] \\ \cline{1-5}

そ & [ʃo\slash so] & [so] & [so] & [so] \\ \cline{1-5}

\end{ltabulary}

\par{\textbf{Symbol Note: }ʃ is the sh-sound found in English. }

\begin{ltabulary}{|P|P|P|P|P|P|P|P|}
\hline 

Word & Meaning & Old Spelling & New Spelling & Word & Meaning & Old Spelling & New Spelling \\ \cline{1-8}

最終 & Final & さいしふ & さいしゅう & 鈴 & Bell & すず & すず \\ \cline{1-8}

然う斯う & This and that & さうかう & そうこう & 底 & Bottom & そこ & そこ \\ \cline{1-8}

\end{ltabulary}
 \textbf{T-Sounds }
\begin{ltabulary}{|P|P|P|P|P|}
\hline 

 & Old Japanese & Early Middle Japanese & Late Middle Japanese \hfill\break
& Modern Japanese \\ \cline{1-5}

た & [ta] & [t a] & [t a] & [ta] \\ \cline{1-5}

ち & [ti] & [t ʃi] & [t ɕ i] & [t ɕ i] \\ \cline{1-5}

つ & [tu] & [t s u]  \hfill\break
& [t s u] & [t s ɯ] \\ \cline{1-5}

て & [te] & [te] & [te] & [te] \\ \cline{1-5}

と & [to] & [to] & [to] & [to] \\ \cline{1-5}

\end{ltabulary}

\par{ In Old Japanese, we see that there were no variant pronunciations of t. The modern pronunciations emerge after the Nara Period. }

\par{ The only significant note of Old Kana Orthography that contrasts Modern Kana Orthography in respect to t-sounds is that つ at one point started to represent the 促音 (っ) with the onset of contractions that have led to the existence of the 促音. However, distinguishing it from a つ standing for "tsu" is difficult. }

\par{ The size distinction between つ and っ was not standardized until simplification. ち・ひ・り \textrightarrow  つ was first, and there is evidence that people did pronounce this contraction as "tsu" rather than making the next sound a double consonant. Eventually, though, this contraction did result in double consonants. Knowing what things have become, then, becomes essential. }

\begin{ltabulary}{|P|P|P|P|P|P|P|P|}
\hline 

Word & Meaning & Old Spelling & New Spelling & Word & Meaning & Old Spelling & New Spelling \\ \cline{1-8}

日記 & Diary & につき & にっき & 手 & Hand & て & て \\ \cline{1-8}

血汐 & Blood & ちしほ & ちしお & 東京 & Tokyo & とうきやう & とうきょう \\ \cline{1-8}

\end{ltabulary}

\par{\textbf{H-sounds }}

\begin{ltabulary}{|P|P|P|P|P|}
\hline 

 & Old Japanese & Early Middle Japanese & Late Middle Japanese \hfill\break
& Modern Japanese \\ \cline{1-5}

は & [pa] & [ ɸa] & [ha] & [ha] \\ \cline{1-5}

ひ & [pi] & [ ɸi] & [hi\slash  çi] & [çi] \\ \cline{1-5}

ふ & [pu] & [ ɸu] & [hu?\slash  ɸu] & [ɸɯ] \\ \cline{1-5}

へ & [pe] & [ ɸe] & [he] & [he] \\ \cline{1-5}

ほ & [po] & [ ɸo] & [ho] & [ho] \\ \cline{1-5}

\end{ltabulary}

\par{ [ ɸ] came from Old Japanese \slash p\slash , the ancient pronunciation. [ɸ] would then became \slash h\slash , except in the sound ふ . However, it is not certain whether  ふ  ever did become \slash hu\slash  in any variety of Japanese or not. }
 
\par{By Late Middle Japanese, when these sounds were inside of a word, \slash ɸ\slash  beca me [w]. An exception to this is  へ , which had become [(j)e] by this time but would have at one point been pronounced [we]. Another thing to know is that \slash wu\slash  never existed, so \slash ɸ\slash  would have been dropped entirely. See  笑ふ  below. }

\begin{ltabulary}{|P|P|P|P|P|}
\hline 

Word & Meaning & Old Spelling & New Spelling & L.M.J Pronunciation \\ \cline{1-5}

川 & River & かは & かわ & [kawa] \\ \cline{1-5}

宵 & Evening & よひ & よい & [jo(w)i] \\ \cline{1-5}

笑ふ & To laugh & わらふ & わらう & [wa ɽau] \\ \cline{1-5}

潮 & Tide & しほ & しお & [ ɕi(w)o] \\ \cline{1-5}

縄 & Rope & なは & なわ & [nawa] \\ \cline{1-5}

帰る & To go home & かへる & かえる & [ka(y)eru] \\ \cline{1-5}

顔 & Face & かほ & かお & [ka(w)o] \\ \cline{1-5}

\end{ltabulary}

\par{\textbf{Historical Notes }: }

\par{1. As Early Modern Japanese came closer to fruition, え , ゑ , and へ became e altogether and the distinctions that they once had become either dialectical or obsolete. }

\par{2. It is not clear to what extent and when the sounds wi, we, and wo were used. So, this is why most w's are in parentheses in the Late Middle Japanese pronunciation column. }

\par{3. Medial w's originating from \slash ɸ\slash  which came from \slash p\slash  are only preserved with [wa]. Thus, 川 is still [kawa] and not [ka:].  }

\par{4. There are some compound words like 河原 [kawara], which is a contraction of kawa and hara with the medial h of hara being pronounced as a w. For most compound verbs, though, the は行 \textrightarrow  わ行 sound change has disappeared. }

\begin{center}
 \textbf{Palatal-Sounds }
\end{center}

\par{ There are no small y-sound かな. So, all of the palatal sounds that are now written with it were originally spelled with full-sized かな. Whether or not a y-かな was used or not depends on the sound changes leading to long vowels mentioned at the beginning of this lesson. }

\begin{ltabulary}{|P|P|P|P|P|P|P|P|}
\hline 

Word & Meaning & Old Spelling & New Spelling & Word & Meaning & Old Spelling & New Spelling \\ \cline{1-8}

行者 & Devotee & ぎやうじや & ぎょうじゃ & 客 & Guest & きやく & きゃく \\ \cline{1-8}

勅 & Decree & ちよく & ちょく & 屏風 & Folding Screen & びやうぶ & びょうぶ \\ \cline{1-8}

憂鬱 & Depression & いううつ & ゆううつ & 万葉集 & Man'youshuu & まんえうしふ & まんようしゅう \\ \cline{1-8}

\end{ltabulary}
 \textbf{R-sounds }\textbf{\hfill\break
}
\begin{ltabulary}{|P|P|P|P|P|}
\hline 

ら & り & る & れ & ろ \\ \cline{1-5}

\end{ltabulary}

\par{ The exact pronunciation of r is not certain. Even in Modern Japanese, there are plenty of varying pronunciations, none of which are used to contrast meaning in any words. But, it is fair to assume that the r-sounds of older stages of Japanese had all or some of the allophones, \slash ɽ\slash , \slash  ɾ\slash  , \slash  ɺ\slash , and \slash r\slash , found in Modern Japanese. }
 
\par{\textbf{W-sounds }}
 
\par{It is not certain whether the exact manner of pronunciation of \slash w\slash  in older stages of Japanese was the same as it is today. However, it is certain that the ワ行 has had significant change over the centuries. }

\begin{ltabulary}{|P|P|P|P|}
\hline 

 & Early Middle Japanese & Late Middle Japanese & Modern Japanese \\ \cline{1-4}

わ & [wa] & [wa] & [wa] \\ \cline{1-4}

ゐ & [wi] & [(w)i] & [i]* \\ \cline{1-4}

ゑ & [we] & [(y)e] & [e]* \\ \cline{1-4}

を & [wo] & [wo] & [(w)o] \\ \cline{1-4}

\end{ltabulary}

\par{*: ゐ and ゑ have been substituted in Modern Japanese with い and え respectively. They are only seen in purposely old-fashioned spellings, especially in place and personal names.  }

\par{を started many words in Classical Japanese and it is not certain if it was pronounced as wo or o. ゑ was in many words and was pronounced as [(j)e]. The y sound is retained depending on how far back in time you go. Words with ゐ , as the chart implies, were probably pronounced as wi until Late Middle Japanese. }

\begin{ltabulary}{|P|P|P|P|P|P|P|P|}
\hline 

Word & Meaning & Old Spelling & New Spelling & Word & Meaning & Old Spelling & New Spelling \\ \cline{1-8}

声 & Voice & こゑ & こえ & 井戸 & Well & ゐど & いど \\ \cline{1-8}

笑む & To smile & ゑむ & えむ & 参る & To come & まゐる & まいる \\ \cline{1-8}

男 & Man & をとこ & おとこ & 鷲 & Eagle & わし & わし \\ \cline{1-8}

\end{ltabulary}

\par{\textbf{The Uvular ん }}

\par{The uvular nasal consonant N' appeared in the Heian Period due to the influence of Sino-Japanese words. This sound in Modern Japanese has various pronunciations depending on what it is followed by. It is often mispronounced by students as [n], which ignores this property of ん along with the fact that it is moraic. So, it is pronounced with the same amount of time as any other かな. }

\par{ In Classical Japanese, ん was most likely pronounced across the board as a syllabic [m]. It was written as either む or ん, with the original pronunciation for both being [mu]. む was either [mu] or [m] depending on context. Over time. [m] would change into the various pronunciations in Modern Japanese (Lesson 193). }

\par{ It is believed that the origin of [N] was caused by sound changes in words like emperor てんわう, which would have likely been pronounced as [temuwa.u] at one point. Then, it would become [te.m.wɔ] and then eventually [t en:oː]. }

\par{\textbf{Sokuon 促音 }}

\par{Double consonants started to appear in Late Middle Japanese and were either never written out or written with a full-sized つ. Below is a Modern Japanese example of Old Orthography. }

\par{いまだ夢多くして、異國の文學にのみ心を奪はれて居つたその頃の私に、或日この古い押し花のにほひのするやうな奥ゆかしい日記の話をしてくだすつたのは松村みね子さんであつた。 \hfill\break
It was Matsumura Mineko who one day told me, who was just enchanted in foreign literature and still had many dreams, of this elegant and scented like an old pressed flower diary. \hfill\break
From 姨捨記 by 堀辰雄. }

\par{\textbf{Voiced Sounds }}

\par{\textbf{G-Sounds }}

\begin{ltabulary}{|P|P|P|P|}
\hline 

 & Early Middle Japanese & Late Middle Japanese \hfill\break
& Modern Japanese \\ \cline{1-4}

が & [ga] & [ (ⁿ)ɡa] & [ga] \\ \cline{1-4}

ぎ & [gi] & [ (ⁿ)ɡi] & [gi] \\ \cline{1-4}

ぐ & [gu] & [ (ⁿ)ɡu] & [gɯ] \\ \cline{1-4}

げ & [ge] & [ (ⁿ)ɡe] & [ge] \\ \cline{1-4}

ご & [g]o & [ (ⁿ)ɡo] & [go] \\ \cline{1-4}

\end{ltabulary}

\par{\textbf{Chart Note }: The superscript n before g in the Late Middle Japanese columns denotes pre-nasalization. \hfill\break
}

\par{\textbf{Pronunciation Note }: G within words in Modern Japanese may alternatively be pronounced as [ ŋ]. }

\par{\textbf{Z-Sounds }}

\begin{ltabulary}{|P|P|P|P|}
\hline 

 & Early Middle Japanese \hfill\break
& Late Middle Japanese \hfill\break
& Modern Japanese \\ \cline{1-4}

ざ & [za] & [ (ⁿ) za] & [za] \\ \cline{1-4}

じ & [ ʒi] & [ (ⁿ) ʑi] & [ʑi] \\ \cline{1-4}

ず & [zu] & [ (ⁿ) zu] & [zɯ] \\ \cline{1-4}

ぜ & [ ʒe] & [ (ⁿ) ʑe] & [ze] \\ \cline{1-4}

ぞ & [zo] & [ (ⁿ) zo] & [zo] \\ \cline{1-4}

\end{ltabulary}

\par{\textbf{Symbol Note }: ʒ stands for the English j sound. }

\par{\textbf{D-Sounds }}

\begin{ltabulary}{|P|P|P|P|}
\hline 

 & Early Middle Japanese \hfill\break
& Late Middle Japanese \hfill\break
& Modern Japanese \\ \cline{1-4}

だ & [da] & [ (ⁿ)da] & [da] \\ \cline{1-4}

ぢ & [d ʒi] & [ (ⁿ)d ʑi] & [(d)ʑi] \\ \cline{1-4}

づ & [dzu] & [ (ⁿ)dzu] & [(d)zɯ] \\ \cline{1-4}

で & [de] & [ (ⁿ)de] & [de] \\ \cline{1-4}

ど & [do] & [ (ⁿ)do] & [do] \\ \cline{1-4}

\end{ltabulary}

\par{ Distinguishing Yotsugana ( 四つ仮名), じ・ぢ・ず・づ, was very important in Classical Japanese pronunciation. Though their pronunciations have slightly changed over the centuries, it is not until Modern Japanese that we see the great simplification in their pronunciations, which has rendered じ and ぢ and ず and づ  respectively homophonous. }

\begin{ltabulary}{|P|P|P|P|P|P|P|P|}
\hline 

Word & Meaning & Old Spelling & New Spelling & Word & Meaning & Old Spelling & New Spelling \\ \cline{1-8}

火事 & Fire & くわ \textbf{ぢ }& かじ & 恥 & Embarrassment & は \textbf{ぢ }& はじ \\ \cline{1-8}

人知 & Knowledge &  \textbf{じ }んち & じんち & 数 & Number & か \textbf{ず }& かず \\ \cline{1-8}

上手 & Good at &  \textbf{じ }やうず & じょうず & 鬘 & Wig & か \textbf{づ }ら & かずら \\ \cline{1-8}

続く & To continue & つ \textbf{づ }く & つづく & 地盤 & Stronghold &  \textbf{ぢ }ばん & じばん \\ \cline{1-8}

\end{ltabulary}

\par{\textbf{B-Sounds }}

\par{B-sounds are the same as in Modern Japanese. However, there is evidence that there was pre-nasalization before b in Late Middle Japanese. So, as you should have already noticed by now, voiced sounds were more than likely systematically pronounced with pre-nasalization. }

\par{\textbf{KW\slash GW-Sounds }}

\begin{ltabulary}{|P|P|P|P|P|}
\hline 

Old Spelling & Early Middle Japanese & Late Middle Japanese \hfill\break
& Modern Japanese & New Spelling \\ \cline{1-5}

くわ・ぐわ & [kwa\slash gwa] & [kɰa\slash gɰa] & [ka] & か \\ \cline{1-5}

くゐ・ぐゐ & [kwi\slash gwi] & [ki\slash gi] & [ki] & き \\ \cline{1-5}

くゑ・ぐゑ & [kwe\slash gwe] & [ke\slash ge] & [ke] & け \\ \cline{1-5}

くを・ぐを & [kwo\slash gwo] & [ko\slash go] & [ko] & こ \\ \cline{1-5}

\end{ltabulary}

\par{As you can see, only kwa and gwa remained in Late Middle Japanese. Even these sounds would disappear by Modern Japanese in most dialects, which is why these unique spellings were ultimately erased. }

\par{ For a relatively extremely example, consider the following words with the same Modern Japanese pronunciation but different traditional 仮名 spellings. }

\begin{ltabulary}{|P|P|P|P|P|P|}
\hline 

公道 & こうだう & Public road; justice & 行動 & かうどう & Action \\ \cline{1-6}

講堂 & くわうだう & Auditorium & 高堂 & かうだう & Your beautiful home \\ \cline{1-6}

坑道 & かうだう & Tunnel & 香道 & かうだう & Incense smelling ceremony \\ \cline{1-6}

黄道 & くわうだう & Ecliptic & 黄銅 & くわうだう & Brass \\ \cline{1-6}

\end{ltabulary}
       
\section{Exercises}
 
\par{1. Show instances where pronunciation of かな differ from Modern Japanese in Classical Japanese. }

\par{2. What かな become obsolete in 現代仮名遣い. }

\par{3. What role does 四つ仮名 play in Classical Japanese? }

\par{4. What role does 四つ仮名 play in Modern Japanese? }

\par{5. What rules were scrubbed in the change from Historical to Modern Kana Orthography? }

\par{6. How were palatalized sounds written in Classical Japanese? }

\par{7. What long vowel combinations in 歴史仮名遣い were actually long palatalized sounds? }

\par{8. Research what the historical spellings were for 後胤, 行員, 公印, and 光陰. }
    