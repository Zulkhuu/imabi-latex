    
\chapter{Adverbs}

\begin{center}
\begin{Large}
第394課: Adverbs 
\end{Large}
\end{center}
 
\par{Adverbs, as you will find, are pretty much how they are in Modern Japanese. Many patterns, however, have changed or have lost currency over time. Nevertheless, don't let this get in the way of you understanding them. }
      
\section{副詞}
 
\par{The Japanese way to classify adverbs involves four categories. \hfill\break
}

\begin{ltabulary}{|P|P|}
\hline 

Condition Adverbs \hfill\break
& With verbs of condition. \\ \cline{1-2}

Degree Adverbs & Shows degree. \\ \cline{1-2}

Syntax Agreement Adverbs \hfill\break
& Agrees with conjugation\slash particles. \\ \cline{1-2}

Instruction Adverbs \hfill\break
& Shows instruction. \\ \cline{1-2}

\end{ltabulary}

\par{Condition adverbs are used with verbs whereas degree adverbs are used  with adjectives. These are the easiest to use and to recognize. Consider とても (very) as an example. }

\par{When an adverb requires a sentence to be in a certain  conjugation, it will always be a syntax agreement adverb. Another word for this is "correlated". Correlated adverbs are found in very common combinations. T he instruction adverbs are similar, and we will look at them later. }
\textbf{Where Adverbs Come From }\hfill\break

\par{This, though, causes a lot of ambiguity on what constitutes an adverb  as far as where they come from. All adverbs in Japanese come from  another part of speech. }

\begin{itemize}

\item \textbf{True Adverbs }: These adverbs are those that are purely adverbs and are found as such in the dictionary. 
\item \textbf{Varying Part of Speech }:  Some adverbs are used adverbially, but are classified as other parts of  speech when used differently. 
\item \textbf{From Grammatical Nouns }:  Most of these nouns are temporal nouns that describe time such as "now"   and "today". They can either be used as nouns or adverbs. Counter   phrases are also great examples. 
\item \textbf{From Adjectives }: The 連用形 of adjectives can also be used as adverbs. For ナリ型形容動詞  you use the に-連用形, for タリ型形容動詞 you use the と-連用形, for ク活用形容詞 you use  the く-連用形, and for シク活用形容詞 you use the しく-連用形. These adverbs are normally translated with  "-er" or  "-ly". All adjectives may be changed into adverbs in Japanese  whereas  this is not the case in English. 
\item \textbf{From the Gerund of Verbs }: Verbs in the gerund can sometimes be used to make adverbs too. For example, 初めて can be used to mean "for the first time. 
\item \textbf{From Suffixes }: Suffixes such as すがら in 夜もすがら (all night) can make adverbs. 
\item \textbf{Onomatopoeia }:  There is a lot of variety in onomatopoeia just as in Modern Japanese. \hfill\break

\end{itemize}
      
\section{Correlated Adverbs}
 
\par{The majority of this lesson will be about what is not the same as in Modern Japanese. You already know the 連用形 of adjectives can be used adjectivally, and since the process is the same, you just need examples of Classical Japanese sentences with them. Many of the most common Japanese adverbs in Modern Japanese are common in Classical Japanese too. }

\par{\textbf{Correlated Adverb Combination Categories }}

\begin{itemize}

\item Negative 
\item Prohibition 
\item Interrogative 
\item Hypothetical 
\item Desiderative 
\item Appropriateness 
\item Suppositional 
\end{itemize}

\par{\textbf{Warning Note }: Some of the endings will not be familiar, but just focus on the adverbs. It will help you tremendously if you can remember the full patterns though. }

\par{\textbf{Set-Up Note }: In order for better memorization, categories will be mixed. They will be labeled. }

\par{ \textbf{Important Common Combinations (Not Exhaustive) }}

\begin{ltabulary}{|P|P|P|}
\hline 

Pattern & Type & Meaning \\ \cline{1-3}

え\dothyp{}\dothyp{}\dothyp{}ず & Negative & Cannot \\ \cline{1-3}

 努(ゆめ)\dothyp{}\dothyp{}\dothyp{}な; ゆめ\dothyp{}\dothyp{}\dothyp{}ず & Prohibition; negative & Absolutely not; by no means \hfill\break
\\ \cline{1-3}

など\dothyp{}\dothyp{}\dothyp{}連体形 & Interrogative & Why? \\ \cline{1-3}

たとひ\dothyp{}\dothyp{}\dothyp{}とも & Hypothetical & Even if \\ \cline{1-3}

願はく\dothyp{}\dothyp{}\dothyp{}む & Desiderative & What I wish for is \\ \cline{1-3}

たへて\dothyp{}\dothyp{}\dothyp{}ず & Negative & Not at all \\ \cline{1-3}

よも\dothyp{}\dothyp{}\dothyp{}じ & Negative & It can't be \\ \cline{1-3}

なんぞ\dothyp{}\dothyp{}\dothyp{}連体形 & Interrogative & Why?; what kind of? \\ \cline{1-3}

いかばかり\dothyp{}\dothyp{}\dothyp{}らむ & Suppositional & How\dothyp{}\dothyp{}\dothyp{}it must be \\ \cline{1-3}

な\dothyp{}\dothyp{}\dothyp{}そ・ぞ & Prohibition & Do not! \\ \cline{1-3}

さだめて…む & Without a doubt & No doubt \\ \cline{1-3}

\end{ltabulary}
\textbf{Examples }1. ${\overset{\textnormal{すべか}}{\text{須}}}$ らく約束は守るべし。 \hfill\break
You ought to keep your promise(s).  
\par{2. 安志比紀乃 夜麻保登等藝須 奈騰可伎奈賀奴 (原文) \hfill\break
あしひきの山ほととぎすなどか来鳴かぬ。 \hfill\break
Why doesn't the mountain cuckoo come and sing? \hfill\break
From the 万葉集. }
 
\par{3. けだし至言なり。 \hfill\break
I dare so it's a wise saying. }
 
\par{4. 山主者 盖雖有 吾妹子之 将結標乎 人将解八方 (原文) \hfill\break
${\overset{\textnormal{やまもり}}{\text{山守}}}$ はけだしありとも、 ${\overset{\textnormal{わぎもこ}}{\text{我妹子}}}$ が結ひけむ ${\overset{\textnormal{しめ}}{\text{標}}}$ を人解かめやも \hfill\break
Even if by some chance the mountain guard were to be there, I wonder if a person could untie the       sign you (my daughter) fastened? \hfill\break
From the 万葉集. }
 
\par{5. なほ奥つ方に ${\overset{\textnormal{お}}{\text{生}}}$ ひ ${\overset{\textnormal{い}}{\text{出}}}$ でたる人、いかばかりかはあやしかりけむを。 \hfill\break
As a person who grew up in a far region of the country, how queer I must have been! \hfill\break
From the 更級日記. }
 
\par{6. いかで月を見ではあらん。 \hfill\break
How is it that you not look at the moon? \hfill\break
From the 竹取物語. }
 
\par{7. 念仏に勝る事 ${\overset{\textnormal{さうら}}{\text{候}}}$ ふまじとはなど ${\overset{\textnormal{まう}}{\text{申}}}$ し給はぬぞ。 \hfill\break
Why don't you say that it is unlikely that it is superior to chanting the name of Buddha? \hfill\break
From the 徒然草. }
 
\par{8. 月な見給ひそ。 \hfill\break
Do not look at the moon. \hfill\break
From the 竹取物語. }
 
\par{9. 春の鳥な鳴きそ鳴きそ。 \hfill\break
Do not sing spring bird, do not sing. \hfill\break
By 北原白秋. }
 
\par{10. 誰もいまだ都慣れぬほどにて、え見つけず。 \hfill\break
Since it was a time when no one was yet accustomed to the capital, they weren't able to find it. \hfill\break
From the 更級日記. }
 
\par{11. むべなるかな。 \hfill\break
It is quite plausible. }
 
\par{12. あの国の人を、え戦はぬなり。 \hfill\break
It is said that one cannot fight the people of that country. \hfill\break
From the 竹取物語. }
 
\par{13. 願はくは幸多からんことを。 \hfill\break
What I wish for is a lot of happiness. }
 
\par{\textbf{Grammar Note }: ~む is often contracted to ~ん in the 連体形. }

\par{14. ${\overset{\textnormal{たけ}}{\text{猛}}}$ き心つかふ人もよもあらじ。 \hfill\break
A person with a bold heart could not possibly exist. \hfill\break
From the 竹取物語. }
 
\par{15. 昔ながらつゆ変はることなきも、めでたきことなり。 \hfill\break
Though there is not a thing at all changed from the past, it is still auspicious. \hfill\break
From an unnamed author. }
 
\par{16. ゆめ疑うことなかれ \hfill\break
Absolutely do not doubt. }
 
\par{17. さて ${\overset{\textnormal{ふゆが}}{\text{冬枯}}}$ れのけしきこそ、秋にはをさをさ劣るまじけれ。 \hfill\break
Well, a withered winter landscape would not certainly be at all inferior to that of autumn! \hfill\break
From the 徒然草. }
      
\section{More Adverbs}
 
\par{This section will serve to show you more examples of adverbs used in Classical Japanese contexts. You should be very familiar with these adverbs, and if you are not, you can probably find them in Modern Japanese texts as well. }

\par{18. いかにせまし。 \hfill\break
What should I do? \hfill\break
From the 堤中納言物語 }

\par{19. 毎度ただ得失なく、この一矢に定むべしと思へ。 \hfill\break
Each time do not think of hitting or missing, just think that you will certainly hit if with a single arrow. }

\par{20. え読みえぬほども心もとなし。 \hfill\break
When you cannot compose a poem, it is nerve-racking. \hfill\break
From the 枕草子. }

\par{21. その沢にかきつばたいとおもしろく咲きたり。 \hfill\break
In that swamp, irises were blooming very beautifully. \hfill\break
From the 伊勢物語. }

\par{22. なべて心柔らかに、情けあるゆゑに、人の言ふほどのこと、けやけくいなびがたくて、よろづえ言ひ放たず、 \hfill\break
心弱くことうけしつ。 \hfill\break
Since in all respects they are gentle in heart and have compassion, it is clearly hard for them to deny someone's favor, and without being able to assert all the circumstances, they ended up timidly took it upon themselves. \hfill\break
From the 徒然草. }

\par{23. 夜もすがら月を眺む。 \hfill\break
To gaze at the moon all night. }

\par{24. ほのぼのと春こそ空に来にけらし。 \hfill\break
Spring seems to have come faintly to the sky. \hfill\break
From the 新古今和歌集. }

\par{25. ちと承らばや。 \hfill\break
I would like to hear a little bit (about it). \hfill\break
From the 徒然草. }

\par{26. いつしか咲かなむ。 \hfill\break
I want the plum tree to bloom quickly! \hfill\break
From the 更級日記. }

\par{ \textbf{Even More Adverbs }}

\begin{ltabulary}{|P|P|P|P|P|P|}
\hline 

漸う & やうやう & Gradually & いとど & いとど & All the most \\ \cline{1-6}

最も & もつとも & The most & 全て & すべて & All \\ \cline{1-6}

極めて & きはめて & Extremely & 軈て & やがて & Presently; immediately \\ \cline{1-6}

終日 & ひねもす & From morning to night & 恐らく & おそらく & No doubt \\ \cline{1-6}

縦しや & よしや & For example & 定めて & さだめて & No doubt \\ \cline{1-6}

\end{ltabulary}
       
\section{Exercises}
 
\par{1. いかが\dothyp{}\dothyp{}\dothyp{}べき means "how (should one)?". Create a simple sentence with this. }

\par{2. Create a simple sentence with an adverb from an adjective. Make sure your verb or adjective is correctly }

\par{3. What kind of adverb would さばかり (that much) be? }

\par{4. Find an example in Modern Japanese with the adverb いと "very". }

\par{5. な\dothyp{}\dothyp{}\dothyp{}そ is replaced by what in Modern Japanese? }

\par{6. え泳がず would mean? }

\par{7. Translate the following into either English or Modern Japanese. }

\par{たとひ耳鼻こそ切れ失(う)すとも }

\par{8. Translate the following into either English or Modern Japanese. }

\par{願はくは桜のもとにて死なむ。 }

\par{9. Translate the following into Modern Japanese. }

\par{とても寒し。 }
    