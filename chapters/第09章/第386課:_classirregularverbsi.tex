    
\chapter{Irregular Verbs I}

\begin{center}
\begin{Large}
第386課: Irregular Verbs I: サ変 \& カ変 
\end{Large}
\end{center}
 
\par{The verbs する and 来る have always been irregular. The first thing to know is that they were once す and 来 respectively. Their meaning have changed little, so we don't have to worry about much semantic change. }
      
\section{サ行変格動詞}
 
\par{The primary verb in this irregular verbal class is す. Its bases are by no means normal, and it can be used in both transitive and intransitive contexts. As a transitive verb, it's like "to do". In intransitive contexts, it's like "to exist" or "to occur". }

\par{The other verb in this class is おはす, which is a very honorific verb meaning "to be\slash go\slash come". Like in Modern Japanese, す may attach to (Sino-)Japanese nouns to create 'new' verbs. At times, it may be voiced as ず. When the 終止形 and 連体形 became the same for verbs, ずる and じる became possible 終止形 for these "ず-すverbs". The verbs that end in -っする when する attaches to a (Sino-)Japanese word ending in する may also have つ contracted in Classical Japanese. However, it is debatable when the contraction started to reflect in the spoken language. }

\par{Originally, す was the only サ変 verb. As it is such a dynamic word, it just had to be added to other words to create even more verbs. So, when did す become する? This change started to take place as early as the end of the 平安時代; both す and する hence began being used as interchangeable 終止形. The bases for サ変  verbs are: }

\begin{ltabulary}{|P|P|P|P|P|P|}
\hline 

未然形 & 連用形 & 終止形 & 連体形 & 已然形 & 命令形 \\ \cline{1-6}

せ & し & す & する & すれ & せよ \\ \cline{1-6}

\end{ltabulary}

\par{\textbf{Bases Notes }: }

\par{1. The alternative 未然形 and 命令形 that are present in Modern Japanese are due to the evolution of the bases. \hfill\break
2. As you can see, the 連用形, 連体形, and 已然形 are the same. When the 未然形 became し, it caused the 命令形 to take the form of しろ, which is essentially like the 命令形 of an 一段 verb. せい comes from the contraction of せよ itself. }

\begin{center}
 \textbf{Examples }
\end{center}

\par{1. ${\overset{\textnormal{けふ}}{\text{今日}}}$ にのぼりて ${\overset{\textnormal{みやづか}}{\text{宮仕}}}$ へをせよ。 \hfill\break
Go to the capital and serve for the imperial court! \hfill\break
From the 大和物語. }

\par{2. 昔、 ${\overset{\textnormal{これたか}}{\text{惟喬}}}$ の ${\overset{\textnormal{みこ}}{\text{親王}}}$ と ${\overset{\textnormal{まう}}{\text{申}}}$ す親王おはしましけり。 \hfill\break
Long ago, there was a prince named Prince Koretaka. \hfill\break
From the 伊勢物語. }

\par{3. 昔の人の ${\overset{\textnormal{そで}}{\text{袖}}}$ の ${\overset{\textnormal{か}}{\text{香}}}$ ぞする。 \hfill\break
It had the scent of the sleeve of a person from the past (whom I knew very well). \hfill\break
From the 古今和歌集. }

\par{4. 死なむ藥も何かはせむ。 \hfill\break
What would I do with death preventing medicine? \hfill\break
From the 竹取物語. }

\par{5. ${\overset{\textnormal{おに}}{\text{鬼}}}$ のやうなるもの ${\overset{\textnormal{い}}{\text{出}}}$ で ${\overset{\textnormal{き}}{\text{來}}}$ て殺さむとしき。 \hfill\break
Things that looked like demons came out and tried to kill us. \hfill\break
From the 竹取物語. }

\par{6. いとどあはれとご ${\overset{\textnormal{らん}}{\text{覽}}}$ じて。 \hfill\break
(The emperor) looked with increasing pity (on the Kiritsubo consort). \hfill\break
From the 源氏物語. }

\par{7. 人々あまた ${\overset{\textnormal{こゑ}}{\text{聲}}}$ して ${\overset{\textnormal{く}}{\text{來}}}$ なり。 \hfill\break
It sounds as if a lot of people are coming and raising their voices. \hfill\break
From the 宇治拾遺物語. }

\par{\textbf{Grammar Note }: The auxiliary verb ~なり shows aural supposition in this example. }

\par{8. ${\overset{\textnormal{つまど}}{\text{妻戸}}}$ をやはらかい ${\overset{\textnormal{はな}}{\text{放}}}$ つ音すなり。 \hfill\break
It sounds like (someone) quietly opened up the chamber doors. \hfill\break
From the 堤中納言物語. }

\par{9. 秋の野に ${\overset{\textnormal{ひとまつ}}{\text{人松}}}$ ${\overset{\textnormal{むし}}{\text{蟲}}}$ の聲すなり。 \hfill\break
You can hear the voice of the pine cricket while waiting for someone in the autumn fields. \hfill\break
From the 古今和歌集. }

\par{\textbf{Grammar Note }: It is common to have two homophonous words intended at the same time. This is the case with the word まつ in the above example sentence. }

\par{10. われ ${\overset{\textnormal{あさ}}{\text{朝}}}$ ごと ${\overset{\textnormal{ゆふ}}{\text{夕}}}$ ごとに見る ${\overset{\textnormal{たけ}}{\text{竹}}}$ の中におはするにて知りぬ。 \hfill\break
I know because she was inside the bamboo I had looked at each and every morning and evening. \hfill\break
From the 竹取物語. }

\par{11. ${\overset{\textnormal{ながあめ}}{\text{镸雨}}}$ 、例の年よりもいたくして。 \hfill\break
The long rains were more intense than in usual years. \hfill\break
From the 源氏物語. }
      
\section{カ行変格動詞}
 
\par{The only verb throughout the history of the Japanese language that has been a カ変 verb is 来る. In Classical Japanese, its 終止形 was 来. For the most part, its translation is on the lines of "to come" or "to visit". The bases for 来 are: }

\begin{ltabulary}{|P|P|P|P|P|P|}
\hline 

未然形 & 連用形 & 終止形 & 連体形 & 已然形 & 命令形 \\ \cline{1-6}

こ & き & く & くる & くれ & こ(よ) \\ \cline{1-6}

\end{ltabulary}

\par{The bases for the most part are very similar to what they are in Modern Japanese. The contraction of 来よ, of course, lead to the modern 来い-命令形. }

\par{12. かの ${\overset{\textnormal{もろこしぶね}}{\text{唐船}}}$ ${\overset{\textnormal{き}}{\text{來}}}$ けり。 \hfill\break
That Chinese boat came. \hfill\break
From the 竹取物語. }
 
\par{13. 春 ${\overset{\textnormal{く}}{\text{來}}}$ れば ${\overset{\textnormal{かり}}{\text{雁}}}$ かへるなり。 \hfill\break
Since spring has come, I hear the wild geese returning. \hfill\break
From the 古今和歌集. }
 
\par{14. ひとびとたえずとぶらひにく。 \hfill\break
People endlessly came to visit. }
 
\par{\textbf{Grammar Note }: You may have noticed that the 終止形 can be used to represent the past tense. This is actually also so in Modern Japanese, although the practice is not that common outside of literature. So long as the \textbf{context }has temporal words that suggest anything but the present or future tense, auxiliary verbs are not necessary to show the preterit. }
 
\par{15. 山の方より人あまた來る音す。 \hfill\break
A sound was made from many people coming from the direction of the mountain. \hfill\break
From the 更級日記. }
 
\par{16. 秋風吹かむをりぞ ${\overset{\textnormal{こ}}{\text{來}}}$ むとする。 \hfill\break
I'll come when the autumn winds blow. \hfill\break
From the 枕草子. }
 
\par{17. 「その ${\overset{\textnormal{こ}}{\text{兒}}}$ こち ${\overset{\textnormal{ゐ}}{\text{率}}}$ て ${\overset{\textnormal{こ}}{\text{來}}}$ よ。」 \hfill\break
Bring that kid here! \hfill\break
From the 大和物語. }

\par{18. ${\overset{\textnormal{たつ}}{\text{龍}}}$ の ${\overset{\textnormal{くび}}{\text{頚}}}$ の取りえずは ${\overset{\textnormal{かへ}}{\text{歸}}}$ り ${\overset{\textnormal{く}}{\text{來}}}$ な。 \hfill\break
If you are unable to obtain the jewel from the dragon's neck, do not come back home! \hfill\break
From the 竹取物語. }
 
\par{\textbf{Grammar Note }: ~ずは is like しないなら and may also be seen as ~ずば or ~ずんば. }
 
\par{19. かた時の ${\overset{\textnormal{あひだ}}{\text{間}}}$ とて、かの ${\overset{\textnormal{くに}}{\text{國}}}$ よりまうで ${\overset{\textnormal{こ}}{\text{來}}}$ しかども \hfill\break
Thinking that it would be but for a short time, I came from that country, but\dothyp{}\dothyp{}\dothyp{} \hfill\break
From the 竹取物語. }
 
\par{20. 春過而 夏来良之 \hfill\break
春過ぎて夏來たるらし。 \hfill\break
It seems that spring has passed and summer has come. \hfill\break
From the 万葉集. }
 
\par{21. 世にある物ならば、この國にも持てまうで ${\overset{\textnormal{き}}{\text{來}}}$ なまし。 \hfill\break
If it were something that existed in the world, they would've brought it and come to this country also. \hfill\break
From the 竹取物語. }
 
\par{\textbf{Grammar Note }: ~まし, which follows the 未然形, shows counter-factual speculation in this example. }
      
\section{Exercises}
 
\par{1. Compare and contrast the bases of す and する. }

\par{2. Compare and contrast the bases of 来 and 来る. }

\par{3. What would 信じる be in Classical Japanese and what would its bases be? }

\par{4. True or False: The す-終止形 and する were interchangeable from as early as the 奈良時代. }

\par{5. Conjugate the following verbs into their bases. }

\par{愛す  拝す  おはす  来  音す }

\par{6. True or False: す and する are only limited to being attached to Sino-Japanese nouns to create 'new' verbs. }

\par{7. Create a simple sentence with す. You may model your sentence after any of the examples given in this lesson. }

\par{8. Why has the verb "to do" been so dynamic throughout the history of the Japanese language? Is this not also the case in other languages? }

\par{9. What are the conditions needed to use the 終止形 as the preterite? }

\par{10. Conjugate す and 来 with the auxiliary verbs -む, -て, and -べし. }
    