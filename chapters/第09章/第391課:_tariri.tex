    
\chapter{The Auxiliary Verbs ~たり \& ~り}

\begin{center}
\begin{Large}
第391課: The Auxiliary Verbs ~たり \& ~り 
\end{Large}
\end{center}
 
\par{The auxiliary verbs ~き and ~けり are about recollection. ~ぬ and ~つ are about the perfective. The auxiliary verbs ~たり and ~り are about results, continuation, and the past tense. ~たり is the ancestor of ~た, the all for one ending that replaced all of these endings. Although ~り came first, we will start this discussion by talking about the most important of two endings, ~たり. }
      
\section{The Auxiliary Verb ~たり}
 
\par{~たり comes from ~てあり. In Modern Japanese, there is a difference between ~ てある, ~ている , and ~た. However, in Classical Japanese, ~たり served all three of these purposes. ~たり first appeared way back in the 奈良時代, when the ending -り was prevalent for the same functions ~たり has. So, why did ~たり come to be? It is because ~り could only follow 四段 and サ変 verbs. As no such restriction was placed on ~たり, it led to its prevalence after the 奈良時代, resulting in ~た dominating over several other endings relating to the past. }

\par{~た began being used as early as the 鎌倉時代. However, it took until the 江戸時代 for the continuative and resultative functions of ~たり to be replaced by ~てある and ~ている. Now, let's go into more detail about what these functions are. }

\par{First and foremost, the bases of ~たり, which follow a ラ変 conjugation, are as follows. }

\begin{ltabulary}{|P|P|P|P|P|P|}
\hline 

未然形 & 連用形 & 終止形 & 連体形 & 已然形 & 命令形 \\ \cline{1-6}

たら & たり & たり & たる & たれ & たれ \\ \cline{1-6}

\end{ltabulary}

\par{ \textbf{The Six Functions of ~たり }}

\begin{enumerate}

\item \textbf{Continuative }: This is "-ing" in English and ~ている in Modern Japanese. 
\item \textbf{Resultative }: This is "has been" in English and ~てある in Modern Japanese. This indicates that someone has already done something and that the result continues. 
\item \textbf{Past Tense }: This is "-ed" in English and ~た in Modern Japanese. 
\item \textbf{Back-and-forth }: This is classified as a conjunctive particle usage in Modern Japanese. 
\item \textbf{Future realization }: "when\dothyp{}\dothyp{}\dothyp{}occurs". 
\item \textbf{Urgent Command }: This is now also carried out by ~た in Modern Japanese. 
\end{enumerate}

\begin{center}
 \textbf{Examples }
\end{center}

\par{1. みな ${\overset{\textnormal{ひと}}{\text{人}}}$ はおもき ${\overset{\textnormal{よろひ}}{\text{鎧}}}$ のうへに、おもき物を負うたり、 ${\overset{\textnormal{いだ}}{\text{抱}}}$ ひたりして ${\overset{\textnormal{い}}{\text{入}}}$ ればこそ ${\overset{\textnormal{しづ}}{\text{沈}}}$ め。 \hfill\break
On top of their heavy armor, they were bearing heavy things on their backs and holding them in their     hands, so when they entered the water they sank. \hfill\break
From the 平家物語. }

\par{\textbf{Sound Change Note }: Two noticeable sound changes can be found in this sentence. ~たり follows 負う instead of 負ひ-, which is still common for ~た after similar verbs in Western Japan, and it follows 抱ひ- instead of 抱き-. This sound change is still used for all 五段 verbs that end in く in Modern Japanese. Also, the particle こそ makes the verb of the clause\slash sentence it is in in the 已然形. }

\par{2. 橋をひいたぞ。 \hfill\break
They pulled up the bridge! \hfill\break
From the 平家物語. }

\par{3. いづくにもあれ、しばし旅立ちたるこそ、目 ${\overset{\textnormal{さ}}{\text{覚}}}$ むる心ちすれ。 \hfill\break
When one is on a trip for a while, it doesn't matter where as one will certainly feel being awakened. \hfill\break
From the 徒然草. }

\par{4. ${\overset{\textnormal{うへ}}{\text{上}}}$ は ${\overset{\textnormal{さやまき}}{\text{鞘巻}}}$ の黒く塗りたりけるが \hfill\break
As for the surface, it was varnished black with a sayamaki. \hfill\break
From the 平家物語. }

\par{5. おもしろく咲きたる ${\overset{\textnormal{さくら}}{\text{櫻}}}$ を長く ${\overset{\textnormal{を}}{\text{折}}}$ りて 大きなる ${\overset{\textnormal{かめ}}{\text{瓶}}}$ にさしたるこそをかしけれ \hfill\break
Picking up beautifully blossomed cherry blossoms and placing them in a big vase is precisely zestful. \hfill\break
From the 枕草子. }

\par{6. ${\overset{\textnormal{うり}}{\text{瓜}}}$ にかきたる ${\overset{\textnormal{ちご}}{\text{稚児}}}$ の ${\overset{\textnormal{かほ}}{\text{顔}}}$ \hfill\break
 The face of a child scratched onto a melon \hfill\break
From the Makura no Soushi.  }

\par{7. さあ、どいたり、どいたり。(Old-fashioned Modern Japanese example) \hfill\break
Well, move it, move it! }
      
\section{The Auxiliary Verb ~り}
 
\par{As ~り is much older than ~たり, it never evolved in usage to be able to show parallel action nor future realization. ~り comes from あり, which attached itself to the 連用形 of verbs. This caused a merging of "-i+a" which lead to "e". It was then treated as an auxiliary verb that followed the 已然形・命令形 of 四段 verbs as well as the 未然形 of サ変 verbs. It was very common in works from the 奈良時代 such as the 万葉集. }

\par{~り wasn't completely abandoned however. It still survived in Japanese-Chinese style writing. ~り has the same resultative, continuative, and past tense functions as ~り. It's just that its lack of versatility in the kind of verbs that could follow is what caused its demise overall. The bases of ~り are: }

\begin{ltabulary}{|P|P|P|P|P|P|}
\hline 

未然形 & 連用形 & 終止形 & 連体形 & 已然形 & 命令形 \\ \cline{1-6}

ら & り & り & る & れ & れ \\ \cline{1-6}

\end{ltabulary}

\par{8. 事すでに ${\overset{\textnormal{ちようでふ}}{\text{重畳}}}$ せり。 \hfill\break
This has already piled up. \hfill\break
From the 平家物語. }

\par{9. おごれる心 \hfill\break
Proud heart \hfill\break
From the 平家物語. }

\par{10. 生ける ${\overset{\textnormal{しかばね}}{\text{屍}}}$ \hfill\break
Living corpse }

\par{11. ${\overset{\textnormal{こぞ}}{\text{去年}}}$ 焼けて、 ${\overset{\textnormal{ことし}}{\text{今年}}}$ 作れり。 \hfill\break
(A house) burned last year, and this year it has been built again. \hfill\break
From the 方丈記. }

\par{12. ${\overset{\textnormal{こうゑん}}{\text{公園}}}$ の木の ${\overset{\textnormal{あひだ}}{\text{間}}}$ に ${\overset{\textnormal{ことり}}{\text{小鳥}}}$ をあそべるをながめてしばし ${\overset{\textnormal{いこ}}{\text{憩}}}$ ひけるかな。 \hfill\break
Oh how I rested for a while looking at the small beard playing in the trees in the park. \hfill\break
By 石川啄木. }

\par{13. 金野乃 美草苅葺 屋杼礼里之 兎道乃宮子能 借五百礒所念 (原文) \hfill\break
秋の野の、み草 ${\overset{\textnormal{か}}{\text{刈}}}$ り ${\overset{\textnormal{ふ}}{\text{葺}}}$ き、宿れりし、 ${\overset{\textnormal{うぢ}}{\text{宇治}}}$ の ${\overset{\textnormal{みやこ}}{\text{宮処}}}$ の、 ${\overset{\textnormal{かりいほ}}{\text{仮廬}}}$ し思ほゆ。 \hfill\break
I remember the thatching of the roof with grass only from the autumn field of the temporary inn we         stayed at in the capital. \hfill\break
From the 万葉集. }

\par{14. 宇良々々尓 照流春日尓 比婆理安我里 情悲毛 (原文) \hfill\break
うらうらに、照れる ${\overset{\textnormal{はるひ}}{\text{春日}}}$ に、 ${\overset{\textnormal{ひばり}}{\text{雲雀}}}$ 上がり、心悲しも。 \hfill\break
On a spring day when the sun is shining gently, my heart is sad though the skylark dances above. \hfill\break
From the 万葉集. }

\par{\textbf{Spelling Note }: Notice the usage of 々 above in how the first doubled 宇 and the second doubled 良. }

\par{15. 人を遣りて見するに、おほかた逢へる者なし。 \hfill\break
He sent a man and made him look, but there wasn't a single person who had met (the demon). \hfill\break
From the 徒然草. }

\par{\textbf{Modern Remnant Note }: This item can be found alive in the set phrase 至れり尽くせり, which means "most gracious", using the particle の to be used as an attribute. }

\par{16. 至れり尽くせりの持て成し \hfill\break
Most gracious hospitality }
      
\section{Exercises}
 
\par{1. Give the bases of -たり. }

\par{2. Give the bases of -り. }

\par{3. What is the difference between the copula verb たり and the auxiliary verb -たり? Illustrate. }

\par{4. Why did -り get replaced by -たり. }

\par{5. Why is it that you think -た replaced all of the other endings related to the past? }

\par{6. Think of the functions of -たり and -り. How are they differentiated in modern grammar? }

\par{7. Translate the following. }

\par{吉野の里に降れる白雪 }

\par{8. Detail the sound changes discussed in this lesson. }

\par{9. What are -ぬ and -つ used to represent? }
    