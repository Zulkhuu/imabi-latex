    
\chapter{Nouns \& Pronouns}

\begin{center}
\begin{Large}
第392課: Nouns \& Pronouns 
\end{Large}
\end{center}
 Nouns and pronouns are probably the most familiar and easy aspects of Classical Japanese. There are two main sources of vocabulary in Classical Japanese:  Sino-Japanese words and native Japanese words. Native words are much  more common, and it is only as time progresses that Sino-Japanese words  begin to make a big chunk of the lexicon. Keep this in mind.       
\section{Nouns}
 
\par{There are still four kinds of nouns in Classical Japanese as there is in Modern Japanese: proper, common, numerical, and nominal. }

\begin{ltabulary}{|P|P|}
\hline 

Proper \hfill\break
& The names of people and places. \\ \cline{1-2}

Common & Represent the human experience by representing objects and ideas. \\ \cline{1-2}

Numerical \hfill\break
& Numbers and counter phrases, of which are slightly different from today. \hfill\break
\\ \cline{1-2}

Nominal & Nouns with a nominalizing effect due to weakened literal meaning. \\ \cline{1-2}

\end{ltabulary}

\par{Note: Spelling and pronunciation are often different, of these differences time period places an important role. Also, common nominal nouns that Classical Japanese and Modern Japanese share are: こと, ため, ところ, もの, やう, よし, ゆゑ, and わけ. }

\par{As numbers are going to be discussed separately in Lesson 146, the following chart will only have examples of the other three kinds of nouns. }

\begin{ltabulary}{|P|P|P|P|P|P|}
\hline 

故 (ゆゑ) & Therefore & 江戸 (えど) \hfill\break
& Edo & 源氏 (げんじ) \hfill\break
& Genji \\ \cline{1-6}

無常 (むじやう) \hfill\break
& Impermanence & 雨 (あめ) \hfill\break
& Rain \hfill\break
& 水 (みづ) \hfill\break
& Water \\ \cline{1-6}

小野小町 (おののこまち) \hfill\break
& Ono no Komachi & 少納言 (せうなごん) \hfill\break
& Shounagon & 聲 (こゑ) \hfill\break
& Voice \\ \cline{1-6}

三笠の山 (みかさのやま) \hfill\break
& Mount Mikasa & 夕顔 (ゆふがほ) \hfill\break
& Evening face & 頚 (くび) \hfill\break
& Neck \\ \cline{1-6}

\end{ltabulary}
\textbf{Examples }1. \textbf{${\overset{\textnormal{ほど}}{\text{程}}}$ }${\overset{\textnormal{へ}}{\text{経}}}$ て \hfill\break
After time has passed \hfill\break
From the 徒然草. 
\par{2. いつしか \textbf{${\overset{\textnormal{うめ}}{\text{梅}}}$ }咲かむ。 \hfill\break
I wish that the plum would bloom as soon as possible. \hfill\break
From the 更級日記. }

\par{3. \textbf{月 }は \textbf{${\overset{\textnormal{ありあけ}}{\text{在明}}}$ }にて \hfill\break
The moon was that of dawn, \hfill\break
From the 奥の細道. }

\par{4. いざ ${\overset{\textnormal{たま}}{\text{給}}}$ へ、 \textbf{${\overset{\textnormal{いづも}}{\text{出雲}}}$ }${\overset{\textnormal{をが}}{\text{拝}}}$ みに \hfill\break
Well, please come, in order to make a pilgrimage to Izumo. \hfill\break
From the 徒然草. }
      
\section{Pronouns}
 
\par{There are quite a few pronouns in Classical Japanese that are no longer used in Modern Japanese, at least not in modern contexts. Below are the pronouns of Classical Japanese categorized by person with notes on how they are used. }

\begin{ltabulary}{|P|P|P|P|}
\hline 

Person? & 漢字 & 歴史仮名遣い & Usage \\ \cline{1-4}

First & 我・吾 & われ & Refers to oneself. \hfill\break
\\ \cline{1-4}

First & 我・吾 & わ & Original first person pronoun. Seen a lot in 我が. \hfill\break
\\ \cline{1-4}

First & 我輩 & わがはい & A haughty word that literally means "my fellows" but normally refers to just oneself. \\ \cline{1-4}

First & 某 & それがし & Used by 江戸時代 samurai. \\ \cline{1-4}

First & 妾 & わらは & A humble I. It was often used by samurai women. \hfill\break
\\ \cline{1-4}

First & 豫・余 & よ & A little pompous. Used only in the singular form. \\ \cline{1-4}

First & 朕 & ちん & Used only by the Emperor. \\ \cline{1-4}

First & 仇家人 & あだかど & Like おたく, it refers to one's house, but it is used to refer to oneself. \\ \cline{1-4}

First\slash Second & 己 & おのれ & Either a humble I or a hostile second person pronoun. \hfill\break
\\ \cline{1-4}

First &  麿 & まろ & Used since the 平安時代. \\ \cline{1-4}

First & あっし & あっし & Used a lot by men in the feudal period; slang. \\ \cline{1-4}

First & 拙者 & せっしゃ & Used by samurai and upper class merchants in the feudal age in humiliation. \\ \cline{1-4}

First & 僕 & やつがれ & A humble I. \\ \cline{1-4}

First & みな(人) & みな(ひと) & Everyone \\ \cline{1-4}

First & 身共 & みども & We; similar to 我ら. \\ \cline{1-4}

First & 吾人 & ごじん & We \\ \cline{1-4}

Second & 君 & きみ & Lord; compare to how it is used now. \hfill\break
\\ \cline{1-4}

Second & 御前 & おまえ & Honorific you; compare to how it is used now. \hfill\break
\\ \cline{1-4}

Second & 貴様 & きさま & Honorific you to nobility; compare to how it is used now. \\ \cline{1-4}

First\slash Second & 手前 & てまえ & A humble I or a somewhat condescending you. \\ \cline{1-4}

Second & 御主 & おぬし & Used towards people below one's status. \hfill\break
\\ \cline{1-4}

Second & 其方 & そなた・そのはう & Used towards people below one's status. \\ \cline{1-4}

Second & 汝 & なんぢ & Thou \\ \cline{1-4}

Third & 彼 & かれ & He \\ \cline{1-4}

Third & 彼女 & かのじよ & She \\ \cline{1-4}

Third & 誰 & た(れ) & Who \\ \cline{1-4}

\end{ltabulary}

\par{\textbf{Usage Note }: Second person pronouns are more frequently used in Classical Japanese as their usages were honorific rather than insulting. This, of course, changed over time. So, it is still the case that a person's name was better to use. }

\par{\textbf{Historical Note }: Other pronouns like 私・わたくし ( from the word private) and あなた (literally that direction) can still be seen. 俺 could also be used as a second person pronoun, and ore itself was neutral and it was the second person pronoun usage that was used contemptuously. This is because ore is a contraction of 己! }

\par{\textbf{Demonstratives }: Demonstratives such as こちら, そちら, and あちら (comparable to こなた, そなた, and あなた) are used as pronouns in Classical Japanese as well. In fact, これ can also mean "that person" and 彼 could also mean "that thing". }

\par{\textbf{Pluralization }: 我, 君, お前, 貴様, 手前, 彼, and 彼女 are commonly pluralized. The same suffixes for pluralization are used, but ~ ども is more common. }

\par{5. かれは、人の許し聞こえざりしに。 \hfill\break
No one gave their approval of that person. \hfill\break
From the 源氏物語. }

\par{6. 君や ${\overset{\textnormal{こ}}{\text{来}}}$ し ${\overset{\textnormal{われ}}{\text{我}}}$ や ${\overset{\textnormal{い}}{\text{行}}}$ きけむ思ほえず。 \hfill\break
Did you come? Did I go? I don't remember. \hfill\break
From the 伊勢物語. }

\par{7. みな ${\overset{\textnormal{よろひ}}{\text{鎧}}}$ の袖をぞぬらしける。 \hfill\break
Everyone wet the sleeves of their armor! \hfill\break
From the 平家物語. }

\par{8. ${\overset{\textnormal{たれ}}{\text{誰}}}$ もいまだ都慣れぬほどにて、え見つけず。 \hfill\break
Since it was a time when no one was yet used to the capital, they were not able to find it. \hfill\break
From the 更級日記. }

\par{9. これはみかたぞ。 \hfill\break
I am an ally! \hfill\break
From the 平家物語. }

\par{10. わが宿は道もなきまで ${\overset{\textnormal{あ}}{\text{荒}}}$ れにけり。 \hfill\break
My lodge has grown wild to the point where there is no longer a road. \hfill\break
From the 古今和歌集. }

\par{11. ${\overset{\textnormal{ごじん}}{\text{吾人}}}$ の関知するところにあらず。 \hfill\break
It is not of our concern. }

\par{12. 君ならで誰にか見せん。 \hfill\break
If not to you, to whom would I show? \hfill\break
From the 古今和歌集. }

\par{13. 君も臣もさわがせ給ふ。 \hfill\break
Both the lord and the retainers panicked. \hfill\break
From the 平家物語. }

\par{14. 私が二階にゐることを必ずいふまいぞ。 \hfill\break
Never say that I am on the second floor. \hfill\break
From the 冥土の飛脚. }

\par{15. ${\overset{\textnormal{なんぢ}}{\text{汝}}}$ 殺すなかれ。 \hfill\break
Thou shalt not kill. }

\par{16. ${\overset{\textnormal{それがし}}{\text{某}}}$ にお任せください。 \hfill\break
Allow me to do this. }
      
\section{Exercises}
 
\par{1. 汝 was once なむち. Show how it changed over time. }

\par{2. How is 貴様 different from what it was in Classical Japanese? }

\par{3. How is お前 different from what it was in Classical Japanese? }

\par{4. 俺 comes from what? }

\par{5. How are demonstratives used as pronouns? }
    