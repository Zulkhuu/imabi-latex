    
\chapter{Regular Verbs II}

\begin{center}
\begin{Large}
第384課: Regular Verbs II: 上一段 \& 下一段 
\end{Large}
\end{center}
 
\par{上一段 and 下一段 verbs are very large verbal classes in Modern Japanese. In Classical Japanese, however, there are only around 11 of them when excluding compound verbs. As we will see in 第258課, the 二段 classes collapsed and became 一段 verbs. }
      
\section{上一段活用動詞}
 
\par{上一段 verbs are called such because they have an upper one-grade conjugation. In other words, their stem ends in い. Let's use the verb 見る to illustrate the bases. }

\begin{ltabulary}{|P|P|P|P|P|P|}
\hline 

未然形 & 連用形 & 終止形 & 連体形 & 已然形 & 命令形 \\ \cline{1-6}

み & み & みる & みる & みれ & みよ \\ \cline{1-6}

\end{ltabulary}

\par{\textbf{Form Note }: Notice how the 命令形 is みよ and not みろ. This is because みろ was used in Eastern Japan and did not become standard until 東京弁 became the basis for 標準語. }

\par{ All of these verbs are compounds or different meanings of the words ひる, いる, ゐる, きる, みる, and にる. There are less than 20 of such verbs. }

\begin{ltabulary}{|P|P|P|P|P|P|}
\hline 

鑄る & いる & To cast & 射る & いる & To shoot \\ \cline{1-6}

沃る & いる & To dowse & 着る & きる & To wear \\ \cline{1-6}

煮る & にる \hfill\break
& To boil \hfill\break
& 似る & にる & To resemble \\ \cline{1-6}

簸る & ひる & To winnow & 干る & ひる & To dry up \\ \cline{1-6}

嚏る & ひる & To sneeze & 試みる & こころみる & To attempt \\ \cline{1-6}

見る & みる & To see & 顧みる & かえりみる & To look back \\ \cline{1-6}

後ろ見る & うしろみる & To look after & 用ゐる & もちゐる & To use \\ \cline{1-6}

居る & ゐる & To sit & 率る & ゐる & To lead \\ \cline{1-6}

\end{ltabulary}

\begin{center}
 \textbf{Examples }
\end{center}

\par{1. 春は ${\overset{\textnormal{ふぢなみ}}{\text{藤波}}}$ を見る。 \hfill\break
We look at the waves of wisteria in the spring. \hfill\break
From the 方丈記. }
 
\par{2. 月な ${\overset{\textnormal{みたま}}{\text{見給}}}$ ひそ。 \hfill\break
Don't look at the moon. \hfill\break
From the 竹取物語. }
 
\par{\textbf{Grammar Notes }: }

\par{1. The pattern な\dothyp{}\dothyp{}\dothyp{}~そ makes a negative command. そ is the non-voiced form of ぞ and was commonly used from the 奈良時代 to the平安時代初期. \hfill\break
2. ~給ふ is an honorific supplementary verb. Its usage is extremely common in Classical Japanese. }
 
\par{3. また、いさゝかおぼつかなく覚えて、頼むにもあらず、頼まずもあらで、案じゐたる人あり。 \hfill\break
Again, there are those that can think somewhat uneasily, and it's not even trusting, and even without       trusting, there are still people that think uneasily. \hfill\break
From the 徒然草. }
 
\par{\textbf{Grammar Notes }: }

\par{1. ~たる in this sentence functions like the modern pattern ~ている. This usage is still common in certain dialects today. \hfill\break
2. ~で is the contraction of ~ずに. }
 
\par{4. まして、明らかならん人の、 ${\overset{\textnormal{まど}}{\text{惑}}}$ へる我等を見んこと、 ${\overset{\textnormal{てのひら}}{\text{掌}}}$ の ${\overset{\textnormal{うへ}}{\text{上}}}$ の物を見んが如し。 \hfill\break
Much less, as far clear-headed people, as for trying to do things such as look at our bewildered               selves, it is just like looking at something in the palm of one's hands. \hfill\break
From the 徒然草. }
 
\par{\textbf{Grammar Note }: ~ん is the contraction of the volitional auxiliary verb ~む, the predecessor of ~よ(う). }

\par{5. ${\overset{\textnormal{いを}}{\text{魚}}}$ と鳥とのありさまを見よ。 \hfill\break
Look at the appearance of fish and birds. \hfill\break
From the 方丈記. }
 
\par{6. その ${\overset{\textnormal{さは}}{\text{澤}}}$ のほとりの木の陰におりゐて、 \hfill\break
They got (their horses) and sat down in the shade of a tree on the edge of the marsh, and they\dothyp{}\dothyp{}\dothyp{} \hfill\break
From the 伊勢物語. }
      
\section{下一段活用動詞}
 
\par{There is only one 下一段 verb in Classical Japanese. The single verb is 蹴(け)る meaning "to kick". What is odd is that it is a 五段 verb in Modern Japanese. Its bases are shown below and is followed by example sentences. }

\begin{ltabulary}{|P|P|P|P|P|P|}
\hline 

未然形 & 連用形 & 終止形 & 連体形 & 已然形 & 命令形 \\ \cline{1-6}

け & け & ける & ける & けれ & けよ \\ \cline{1-6}

\end{ltabulary}

\par{\textbf{Base Note }: The 命令形 is けよ because the r in the modern 命令形 was and still is a feature of Eastern Japanese dialects. As Classical Japanese is predominantly written in Western Dialects, it wasn't until the capital was moved to 東京 when the shift occurred. }

\par{7. かの ${\overset{\textnormal{てんやく}}{\text{典薬}}}$ の ${\overset{\textnormal{すけ}}{\text{助}}}$ は ${\overset{\textnormal{け}}{\text{蹴}}}$ られたりしを \hfill\break
That assistant director of the Medical Ministry was kicked, and \hfill\break
From the 落窪物語. }
 
\par{\textbf{Grammar Note }: The particle を in this sentence is a final particle that shows exclamation. }
 
\par{8. この ${\overset{\textnormal{しり}}{\text{尻}}}$ 蹴よ。 \hfill\break
Kick these buttocks! \hfill\break
From the 宇治拾遺物語. }

\par{9. ${\overset{\textnormal{かしら}}{\text{頭}}}$ 蹴わられ \hfill\break
Their heads were kicked and split open, and \hfill\break
From the 平家物語. }

\par{10. ${\overset{\textnormal{まり}}{\text{鞠}}}$ を蹴る事か。 \hfill\break
Do you mean kick the football? \hfill\break
From the 浮世物語. }

\par{11. 尻蹴んとする ${\overset{\textnormal{すまふ}}{\text{相撲}}}$ \hfill\break
The sumo wrestler trying to kick the buttocks \hfill\break
From the 宇治拾遺物語. }
 
\par{12. 蹴れども \hfill\break
Even though you don't kick }
 
\par{13. さと寄りて ${\overset{\textnormal{いつそく}}{\text{一足}}}$ づつ蹴る。 \hfill\break
He quickly approached and kicked with each foot. \hfill\break
From the 落窪物語. }

\par{14. ${\overset{\textnormal{まりこがわ}}{\text{円子川}}}$ 蹴ればぞ波はあがりける。 \hfill\break
When I kicked through Mariko River, a wave rose. \hfill\break
From the 源平盛衰記. }
      
\section{Exercises}
 
\par{1. List the 上一段 verbs in Classical Japanese. }

\par{2. What is the only 下一段 verb in Classical Japanese. }

\par{3. Where is the source for the plethora of 一段 verbs in Modern Japanese? }

\par{4. Conjugate the verb 用ゐる into its bases. }

\par{5. Contrast the bases of 蹴る in Modern Japanese to Classical Japanese. }

\par{6. Conjugate 蹴る and 居る with the auxiliary verb む・ん. }

\par{7. Conjugate 見る with the auxiliary verb けり. }
    