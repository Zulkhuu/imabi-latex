    
\chapter{Parts of Speech \& Basic Syntax}

\begin{center}
\begin{Large}
第379: Parts of Speech \& Basic Syntax 
\end{Large}
\end{center}
 
\par{ Way back in Lesson 7, we learned about the ten major aspects that define Modern Japanese. These principles come into play in Classical Japanese as well. However, this does not mean that Japanese did not behave differently. As you will continue to see throughout the lessons to come, a lot of change has occurred in Japanese over the two thousand or so year it has been written down. }
      
\section{The 10 Major Aspects in Classical Japanese}
 
\par{ Classical Japanese is referred to as 文語 or 大和言葉. These terms, though, are not exactly the same. The first refers to older Japanese prose and can refer to words that may still be used in Modern Japanese but in very literary contexts. For the purpose of these lessons, it will refer to texts written in Japanese before the 明治時代. 大和言葉 often refers to the language before the influx of loanwords from Chinese, but it may also refer to "native words". To be more linguistically correct, the terms Old Japanese, Early Middle Japanese, Late Middle Japanese, Early Modern Japanese, and Modern Japanese will be used to more accurately define which language an example sentence is written in. }

\par{1: The basic word order of Classical Japanese is SOV. Other aspects of word order are more or less the same as in Modern Japanese. }

\par{2: Just like Modern Japanese, the ordering of phrases in a sentence is based upon a hierarchy of importance determined by the speaker. This is why, just like Modern Japanese, Classical Japanese is said to be a macro-to-micro language. }

\par{1. おほかた  ものの音には ${\overset{\textnormal{ふえ}}{\text{笛}}}$  ${\overset{\textnormal{ひちりき}}{\text{篳篥}}}$  ${\overset{\textnormal{つね}}{\text{常}}}$ に聞きたきは ${\overset{\textnormal{びわ}}{\text{琵琶}}}$  ${\overset{\textnormal{わごん}}{\text{和琴}}}$ \hfill\break
With the sounds of most things are the (elements of the) flute and hichiriki. What I always want to hear   are the and koto. \hfill\break
From the 徒然草. }

\par{3: Classical Japanese relatively has more inflectional endings than Modern Japanese. A single auxiliary verb in Modern Japanese may correspond to several different ones in Classical Japanese. The use of bases has also changed slightly for some endings. Also, the overall frequency of all the bases is much higher in Classical Japanese than today. }

\par{4: The eleven parts of speech of Modern Japanese are slightly more complex in Classical Japanese. Nouns tend to have little change. However, the original pronunciation, meaning, and frequency of usage may be different on an individual bases. For instance, pronouns continue to come and gone in the language. }

\par{Verbs and adjectives are more complex as there are nine and four classes respectively instead of five and two respectively as there is in Modern Japanese. As for the other parts of speech, similar changes in form, meaning, and usage have changed on a case by case basis. Otherwise, the overall structure of Japanese has remained constant over time. }

\begin{ltabulary}{|P|P|P|P|}
\hline 

Part of Speech \hfill\break
&  & Inflectional? & Lesson(s) \\ \cline{1-4}

Nouns & 名詞 \hfill\break
& No & 145 \\ \cline{1-4}

Pronouns & 代名詞 \hfill\break
& No & 145 \\ \cline{1-4}

Verbs & 動詞 \hfill\break
& Yes & 133, 136-140 \\ \cline{1-4}

Adjectives & 形容詞 \hfill\break
& Yes & 134 \\ \cline{1-4}

Adjectival Verbs \hfill\break
& 形容動詞 \hfill\break
& Yes & 135 \\ \cline{1-4}

Particles & 助詞 \hfill\break
& No & 151-160? \\ \cline{1-4}

Auxiliary Verbs \hfill\break
& 助動詞 \hfill\break
& Yes & 141-144, 163-168?, 172-177?, 184-190? \\ \cline{1-4}

Interjections & 感動詞 \hfill\break
& No & 149 \\ \cline{1-4}

Conjunctions & 接続詞 \hfill\break
& No & 148 \\ \cline{1-4}

Attributives \hfill\break
& 連体詞 \hfill\break
& No & 171? \\ \cline{1-4}

Adverbs & 副詞 & No & 147 \\ \cline{1-4}

\end{ltabulary}

\par{5: In regards to etymology, the majority of words in Classical Japanese are of native origin. However, when you begin to turn back the clock of time, you start running into problems as to what is truly native or not. For general purposes in this curriculum, traditional classifications of words will be looked at first. If any important doubt of a word's etymology status is important to a discussion, it will be appropriately noted. }

\par{ Foreign words, which at this time would have primarily been from Chinese in the form of the thousands of Sino-Japanese words, mainly come into play in Middle Japanese. However, they were not completely absent in Old Japanese texts. Sino-Japanese words would over time heavily influence the phonology of Japanese as well as the grammar in some respects. }

\par{6: Dialects have always existed, and they are present even in the 万葉集. As there is only a limited amount of written material and because the birthplaces of the authors of the poems are largely unknown, there isn't much to say about dialectical features of Old Japanese as there is for Modern Japanese. However, attested dialectical differences will frequently become important in future discussions.  . }

\par{The speech levels of Classical Japanese have a different feel than in Modern Japanese. Honorifics becomes more important and more organized as time goes by as well as vulgar speech. }

\par{7: There are no articles nor number in Japanese. }

\par{8: There are words in Classical Japanese just as in Modern Japanese that were used by a particular sex. }

\par{9: Quality in emotion and sound certainly mirror to what is the case today. Dialects as well as isolated villages likely had distinct pitch as well as intonation patterns. }

\par{10: Punctuation is inconsistent and different depending on time period. }
      
\section{Exercises}
 
\par{1. What does every sentence in Classical Japanese have? }

\par{2. What is the word order of Classical Japanese? }

\par{3. What is the language of Japan in the Classical Period referred to as? }

\par{4. How are the parts of speech in Classical Japanese more different and more difficult than in Modern Japanese? }

\par{5.  幽玄 (いうげん) "mystery and depth" is what kind of word in terms of etymology? How frequent is its category in Classical Japanese? }

\par{6.  The auxiliary verb ず becomes ぬ when attached before a nominal phrase. What does this demonstrate? }

\par{7. Below is a simple Classical Japanese sentence. What is different? }

\par{山いと高し \hfill\break
やまいとたかし \hfill\break
Modern Japanese: 山がとても高い。 }

\par{8. The Modern Japanese form of the Meireikei ends in -ろ, but the Classical Japanese form of the Meireikei ends in -よ. Modern Standard Japanese is based off of the Tokyo Dialect. Most Classical Japanese text is based off of the dialect region of the Kansai Region. If told that these differences have always been the case, what does this mean overall? }
    