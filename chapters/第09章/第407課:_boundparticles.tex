    
\chapter{Bound Particles}

\begin{center}
\begin{Large}
第407課: Bound Particles  
\end{Large}
\end{center}
 
\par{ Bound particles (係助詞) might as well be called emphatic particles. Most of them have become adverbial and or final particles. Their functions range from emphatic to interrogative usages, and this is often determined by careful attention to context. }

\par{ Bound particles require that the final verb phrase in their clause to be in a particular form. Most require the final verb phrase to be in the 連体形 because, as we've seen, that can make verbs function as nominalized phrases in Classical Japanese. }

\par{ Though the particles discussed in this lesson may not be exhaustive in accounting for all such particles in Classical Japanese, it will enable you to have a decent understanding of the some times very complex grammar rules that go with them. }

\par{ The particles to be discussed are the following: }

\begin{ltabulary}{|P|P|P|P|}
\hline 

Particle & The Fixed Base & Function(s) & Final Usages? \\ \cline{1-4}

は & 終止形 & Topic Marker & Yes \\ \cline{1-4}

も & 終止形 & Parallel listing & Yes \\ \cline{1-4}

や(は) & 連体形 & Doubt\slash question & Yes \\ \cline{1-4}

か(は) & 連体形 & Doubt\slash question & Yes \\ \cline{1-4}

ぞ & 連体形 & Emphasis & Yes \\ \cline{1-4}

なむ・なん & 連体形 & Emphasis & Yes \\ \cline{1-4}

こそ & 已然形 & Strong emphasis & Yes \\ \cline{1-4}

\end{ltabulary}

\par{\textbf{Grammar Note }: These particles have such a strong role in binding sentences in a certain form that even in their absence, when other elements that would be used in concert with them are showing doubt or a rhetorical question, as is an important usage for particles like ぞ, なむ, や, and か, the sentence may still be seen bound in the 連体形. No wonder the 終止形 and 連体形 merged. }
      
\section{は}
 
\par{ The particle は is as you should certainly know is extremely important in Modern Japanese. However, there was a point in time where it was not used, and it has taken centuries for the がVSは rivalry to get to its present form. In fact, the thing that rivaled は, which shows known information, was も, which could present new information. Where did this particle come from? It likely has origin with the particle ば, but it is certain that by the 平安時代 its usages had become greatly varied and increasingly more complex. }

\par{ So, we know that it marks the topic. This topic is emphasized and often made distinct from other things, which is why we still often translate it into English with "as for". This topic is what the explanation\slash predicate is going to be about. It can be used with parallel items to show comparison and or contrast. Of course, it still has the role of highlighting the negative. }

\par{ As you saw from above, it can also be used with other bound particles, and it can also be used as a final particle. That's right. Although modern orthography changes doesn't make this obvious, the adverbial は and final particle わ of Modern Japanese are from the same thing. When it comes at the end, the verb has to go to the 連体形,which is the common trait for bound particles. }

\par{\textbf{Grammar Note }: Remember from previous lessons that は can attach to an adjective 連用形 or -ず to create a hypothetical situation, and ず+は often makes ず(ん)ば. Also remember that を+は = をば. }

\begin{center}
 \textbf{Examples }
\end{center}

\par{1. 古京はすでに荒れて、新都はいまだ成らず。 \hfill\break
The old capital has already been ravaged, and the new capital has still not been materialized. \hfill\break
From the 方丈記. }

\par{2. 人妻尓 言者誰事 酢衣乃 此紐解跡 言者孰言 (原文) \hfill\break
人妻に言ふは ${\overset{\textnormal{た}}{\text{誰}}}$ が ${\overset{\textnormal{こと}}{\text{言}}}$ さ衣のこの紐解けと言ふは誰が言 \hfill\break
Who calls for me, a married woman? Who says I untie the cords of my cloth and sleep together? \hfill\break
From the 万葉集. }

\par{3. 其願空しかるべくは、道にて死ぬべし。 \hfill\break
If such a request proves hopeless, I shall die en route! \hfill\break
From the 平家物語. }

\par{4. なにもなにも小さきものはみなうつくし。 \hfill\break
Of everything small things are particularly adorable. \hfill\break
From the 枕草子. }

\par{5. まろ ${\overset{\textnormal{かうし}}{\text{格子上}}}$ げずば道なくてげにえ ${\overset{\textnormal{い}}{\text{入}}}$ り来ざらまし。 \hfill\break
If I hadn't raised the lattice, there would've been no path and I would have not been able to come. \hfill\break
From the 源氏物語. }
 
\par{6. 法師ばかり羨ましからぬものはあらじ。 \hfill\break
There is probably no less enviable than a monk. \hfill\break
From the 徒然草. }
 
\par{7. 火に焼かむに焼けずはこそ ${\overset{\textnormal{まこと}}{\text{真}}}$ ならめ \hfill\break
When we try to burn it by fire and it cannot burn, it's definitely the real thing. \hfill\break
From the 竹取物語. }
 
\par{8. かの花は失せにけるは。 \hfill\break
That flower has ended up disappearing! \hfill\break
From the 枕草子. }

\par{9. とまらしなよもの時雨のふるさとゝなりにしならの霜のくちはゝ (原文) \hfill\break
とまらじな四方の時雨の古郷となりにしならの霜の朽ち葉は \hfill\break
It won\textquotesingle t stop. In Nara the showers pour every which way, and the oak leaves fall and rot in the dew. \hfill\break
From the 院百首. \hfill\break
\hfill\break
\textbf{Word Note }: [降る里 and 古里] and [奈良 and 楢] are overlapped together respectively. }

\par{10. 古里となりにし奈良の都にも色はかはらず花は咲きけり。 \hfill\break
Even in the old capital Nara, without change in color, the flowers had bloomed. \hfill\break
From the 古今和歌集. }

\par{11. 悪所に落ちては死にたからず。 \hfill\break
We don't want to die falling into a bad spot. \hfill\break
From the 平家物語. }

\par{12. 居明而 君乎者将待 奴婆珠能吾黒髪尓 霜者零騰文 (原文) \hfill\break
居明かして君をば待たむ ${\overset{\textnormal{ぬばたま}}{\text{烏珠}}}$ の吾が黒髪に霜は降るとも \hfill\break
I will wait as is till dawn, no matter if the dew falls into my black hair. \hfill\break
From the 万葉集. }

\par{\textbf{Base Note }: Although we say that it is paired with the 終止形, there are sentences with it that are in the 命令形, where it is cut off in the 連用形, or ends in a nominal (体言止め). This is all still the case in Modern Japanese. }
      
\section{も}
   The particle も is like "too\slash also" and can show listing of things that are new information. Or, it shows that what it follows is not limited in the situation to it alone. In being such, も can be seen in adverbial phrases showing an outer limit or minimal desire, which it still does today.  13. 山川 水陰生 山草 不止妹  所念鴨 (原文) \hfill\break
山河の ${\overset{\textnormal{みかげ}}{\text{水陰}}}$ に生ふる ${\overset{\textnormal{やますげ}}{\text{山菅}}}$ の ${\overset{\textnormal{や}}{\text{止}}}$ まずも妹がおもほゆるかも \hfill\break
Like the mountain sedge growing on the waterside in the mountain, I think of you nonstop. \hfill\break
From the 万葉集.  14. 限りなく遠くも来にけるかな。 \hfill\break
How far we have come endlessly! \hfill\break
From the 伊勢物語.  15. 熟田津尓 船乗世武登 月待者 潮毛可奈比沼 今者許藝乞菜 (原文) \hfill\break
熟田津に船乗りせむと月待てば潮もかなひぬ、今は漕ぎ出でな \hfill\break
The tide too has improved while I waited for the moon while boarding in Nikita Harbor. I now sail!  From the 万葉集. \hfill\break
16. 長き御世にもあらなん May (she) have long life! \hfill\break
From the 源氏物語.  17. 月だにも雲のいづくに夏の夜のやみはあやなし曙の空。 \hfill\break
Even the moon somewhere in the clouds colors the darkness of the summer night--daybreak. \hfill\break
From 順徳院.  18. 京極の屋の南むきに、今も二本侍るめり。 \hfill\break
There are apparently two trees still on the south side of the Kyogoku residence. \hfill\break
From the 徒然草.  19. 花ちれる水のまにまにとめくれば山には春もなくなりにけり。 \hfill\break
When the flowers scattered afloat the river, spring ended went away in the mountains. \hfill\break
From the 古今和歌集.   も can be very emotive, and it was used as a final particle in the 奈良時代.  20. 冨等登藝須  奈保毛奈賀那牟  母等都比等  可氣都々母等奈  安乎祢之奈久母 (原文) \hfill\break
霍公鳥 なほも鳴かなむ 本つ人 かけつつもとな 吾を音し泣くも \hfill\break
Oh cuckoo cry still; I think of the person passed away and in that vestige I cry and weep. \hfill\break
From the 万葉集. 
\par{21. 樂浪乃  國都美神乃  浦佐備而  荒有京  見者悲毛 (原文) \hfill\break
楽浪の 国つ御神の うらさびて 荒れたる都 見れば悲しも \hfill\break
The heart of the kami of Sasanami has withered, and the ravaged capital ruins are sad to behold. \hfill\break
From the 万葉集. }
 When in もぞ and もこそ, it can show that something can't be--するといけない.  22. 烏などもこそ、見つくれ。 \hfill\break
It must not be that something like a crow would find it. \hfill\break
From the 源氏物語.  23. 玉の緒よ 絶えなば絶えね ながらへば 忍ぶることの 弱りもぞする。 \hfill\break
If this life shall cease, may it cease. If I live on like this, my secret love will be known to all. \hfill\break
From the 新古今和歌集.       
\section{や(は)}
 \hfill\break
 や(は) can show doubt and (rhetorical) question. When in the middle of a sentence it no doubts binds the final verb in the 連体形. However, as will be evident in the examples, the 終止形 can indeed be used when や is in the final position of a clause.  24. 龍田河もみぢみだれて流るめり渡らば錦なかや絶えなむ The Tatsuda River seems to flow with colored leaves. If I cross, the brocade may be severed in half. \hfill\break
From the 古今和歌集.  25. 忘哉 語 意遣 雖過不過 猶戀 \hfill\break
忘るやと物語りして心遣り過ぐせど過ぎずなほ恋ひにけり \hfill\break
Wondering if I would forget, I avert my mind with gossip, but I still can\textquotesingle t go by my love (for you). \hfill\break
From the 万葉集.  26. 冬ながら空より花の散りくるは雲のあなたは春にやあるらむ。 \hfill\break
As for flowers falling from the sky while it's winter, does that mean spring is beyond the clouds? \hfill\break
From the 古今和歌集.  27. 宵のまに身を投げはつる夏虫は燃えてや人に逢ふと聞きけむ。 \hfill\break
I wonder if the summer bugs that burned in the night heard they could meet their loved ones by             throwing themselves into the flames? \hfill\break
From the 伊勢物語.  28. 秋の田の穂の上を照らす稲妻の光の間にも我や忘るる。 \hfill\break
I will not forget you even when lightning brightens the tops of the ears of rice in the autumn field. \hfill\break
From the 古今和歌集.  29. さくら花春くははれる年だにも人の心に飽かれやはせぬ。 \hfill\break
Cherry blossom, even in years when spring is long, would you scatter without having satisfied man?       Bloom to your heart's desire! \hfill\break
From the 古今和歌集.  \textbf{Grammar Note }: やは\dothyp{}\dothyp{}\dothyp{}~ぬ = Why don't you\dothyp{}\dothyp{}\dothyp{} (showing want of the speaker as well)  30. 光やあると見るに、蛍ばかりの光だになし。 \hfill\break
When she looked wondering if there were light, there wasn't even the light of a firefly. \hfill\break
From the 竹取物語.  31. 古 人尓和礼有哉  樂浪乃  故京乎  見者悲寸 (原文) \hfill\break
古の 人に我れあれや 楽浪の 古き都を 見れば悲しき \hfill\break
Aren't I not a person of the old; when I look at the old capital of Sasanami, I\textquotesingle m sad without reason. \hfill\break
From the 万葉集.  \textbf{Grammar Note }: や can also be after 已然形 to show a rhetorical question directed at oneself.  32. うちはへて春はさばかりのどけきを花の心や何いそぐらむ \hfill\break
Although spring is calm like this every day, why do the hearts of the flowers hurry and try to scatter? From the 後撰和歌集.  
\par{\textbf{Word Note }: うちはえて comes from a verb with the meaning of stringing along a rope and is used here adverbially to express continuing to be so. }
33. 多知之奈布  伎美我須我多乎  和須礼受波  与能可藝里尓夜  故非和多里奈無 (原文) \hfill\break
立ちしなふ君が姿を忘れずは世の限りにや恋い渡りなむ。 \hfill\break
Without forgetting your great appearance will certainly go on loving you all my life. \hfill\break
From the 万葉集.  \textbf{Grammar Note }: This なむ is the 未然形 of the auxiliary ~ぬ + the volitional auxiliary ~む.  34. 保等登藝須  許々尓知可久乎  伎奈伎弖余  須疑奈<无>能知尓  之流志安良米夜母 (原文) \hfill\break
霍公鳥 ここに近くを 来鳴きてよ 過ぎなむ後に 験あらめやも \hfill\break
Cuckoo come cry closer. Now that after you pass by, is this not an omen? \hfill\break
From the 万葉集.  
\par{\textbf{Grammar Note }: ~めやも is a combination of the auxiliary ~む, which shows conjecture in this pattern, and the bound particles や and も. The pattern is used to rhetorically bring up conjecture or well with an added sense of exclamation and is equivalent to ~だろうか、いや、そんなことはないなあ. }

\par{\textbf{Variant Note }: This can be seen as やも in the 奈良時代. }

\par{35. 士也母 空応有 萬代尓 語続可 名者不立之而 (原文) \hfill\break
士やも 空しかるべき 万代に 語り継ぐべき 名は立てずして \hfill\break
Should a man end in vain? Without his name passed down for generations? \hfill\break
From the 万葉集. }

\par{\textbf{Warning Note }: Do not confuse this with the interjectory particle や. }
      
\section{か(は)}
 \hfill\break
 The bound particle か(は) may also show doubt and (rhetorical) question. かは was also originally かも in the 奈良時代. When in the middle or at the end of a sentence, it requires that the final verb be in the 連体形.  36. 幾世へてのちか忘れむ散りぬべき野辺の秋萩みがく月夜を \hfill\break
Will you forget years later? (Show) the moonlit night upon the autumn clover covered field. \hfill\break
From the 後撰和歌集.  37. いづれの山か天に近き。 \hfill\break
Which mountain is closest to the heavens? \hfill\break
From the 竹取物語.  38. 何をか奉らむ。 \hfill\break
What shall I give you? \hfill\break
From the 更級日記.  39. 安之我良乃  夜敝也麻故要弖  伊麻之奈波  多礼乎可伎美等  弥都々志努波牟 (原文) \hfill\break
足柄の 八重山越えて いましなば 誰れをか君と 見つつ偲はむ \hfill\break
When you pass over the eight folds of mountains in Ashigara, who will we look at to remember you? \hfill\break
From the 万葉集.  40. くひなのたたくなど、心細からぬかは。 \hfill\break
At such times when the marsh hen knocks, are people free of uncertainty? \hfill\break
From the 徒然草.  41. 何事ぞや。童べと、腹だち給へるか。 \hfill\break
What happened? Did you quarrel with the children? \hfill\break
From the 源氏物語.       
\section{ぞ}
  The particle ぞ is quite emphatic as it is in Modern Japanese and originally started out as そ. It often shows anxiety and when it is at the end of a sentence, depending on the context, it can either show a strong declaration or a strong question. It is often at the end of a sentence preceded by the citation particle と with a verb such as 言ふ implying that the said verb is there in the 連体形 as it would otherwise stipulate given that it is a bound particle.  \textbf{Particle Note }: ぞ frequently follows case, adverbial, and conjunctive particles to emphasize their roles. \textbf{Examples } 42. 人はいさ心もしらずふるさとは花ぞ昔の香ににほひける \hfill\break
You know not the feelings of others, but at home, the flowers smell the same as in the past. \hfill\break
From the 古今和歌集.  43. みな鎧の袖をぞぬらしける。 \hfill\break
Everyone wet the sleeves of their armor! \hfill\break
From the 平家物語.  44. なく雁のねをのみぞ聞くをぐら山霧たちはるる時しなければ \hfill\break
All I hear are the cries of geese! That's because the mist never clears at Mount Ogura. \hfill\break
From the 新古今和歌集.  45. 吉野なる夏実の川の川淀に鴨ぞ鳴くなる山影にして \hfill\break
In the pool of the Natsumi River in Yoshino, the ducks are singing hidden in the mountain\textquotesingle s shadow. \hfill\break
From the 万葉集.  46. 門よくさしてよ。雨もぞ降る。 \hfill\break
Close the gate well! It might rain! \hfill\break
From the 徒然草.  47. 見る人もなき山里の桜花ほかの散りなむのちぞ咲かまし。 \hfill\break
Cherry blossoms, in the mountain village with no one to see you, if you could bloom after all else             has, that would be good\dothyp{}\dothyp{}\dothyp{} \hfill\break
From the 古今和歌集.  48. 人所見 表結 人不見 裏紐開 戀日太 (原文) \hfill\break
人に見ゆる上は結びて人に見えぬ ${\overset{\textnormal{したひも}}{\text{下紐}}}$ 開けて ${\overset{\textnormal{こ}}{\text{恋}}}$ ふる日ぞ多き \hfill\break
Doing the cord of the coat that people can see but opening the cord of the underclothes that people       can\textquotesingle t see, the days that I long are many! \hfill\break
From the 万葉集.  49. あれは、何ものぞ。 \hfill\break
Who is that? \hfill\break
From the 平家物語.  50. 川霧のふもとをこめて立ちぬれば空にぞ秋の山は見えける。 \hfill\break
Since the river midst has shrouded the base, the mountain of autumn appeared to float in the sky. \hfill\break
From 宇治拾遺物語.  51. 飼ひける犬の、暗けれど主を知りて飛び付きたりけるとぞ。 \hfill\break
They say that the dog that he had raised,  though it was dark, knew his master jumped on him. From the 徒然草.  52. 恋しとはたが名づけけむことならむ死ぬとぞただに言ふべかりける。 \hfill\break
"Longing" is surely a word someone named. "To ie", though, should have never been coined. \hfill\break
From the 古今和歌集.  \textbf{Historical Note }: ぞ actually started out as a final particle but acquired bound particle usages in becoming more emphatic.       
\section{なむ・なん}
   The particle なむ・なん shows emphasis and like ぞ it can be after the citation と with a verb in the 連体形 abbreviated such as 言ふ. The particle first started out in the form なも. It quickly became associated with casual speech and was frequently used in dialogues and stories.  53. 橋を八つわたせるによりてなむ八橋といひける。 \hfill\break
It is exactly because they laid eight bridges that they call it Eight Bridges. \hfill\break
From the 伊勢物語.  54. 物なん心ぼそくおぼゆる。 \hfill\break
Everything seems so depressing. \hfill\break
From the 竹取物語.  55. もののあはれも知らずなりゆくなん、浅ましき。 \hfill\break
Sure enough, one becomes no longer aware of the pathos of things. How base! \hfill\break
From the 徒然草.  56. ただ今なむ、聞きつけ侍る。 \hfill\break
I heard about it just now! \hfill\break
From the 源氏物語. \hfill\break
\textbf{Particle Note }: Although the bound particle なむ is rarely seen at the end of a sentence, when it is, it should not be confused with the desiderative final particle なむ, which means "I wish that\dothyp{}\dothyp{}\dothyp{}" and attaches to the 未然形. Perhaps it is this exact confusion that resulted in it rarely being at the end. When no verb is at the end of a sentence, though, it appears. After all, the source of confusion, connecting to a verb, is deleted.  57. 舞人を宿せる仮屋より出で来たりけるとなん。 \hfill\break
Sure enough, it's said that (the fire) came from the temporary dwelling housing the dancers. \hfill\break
From the 方丈記.        
\section{こそ}
   こそ creates strong emphasis and when in the middle of the sentence it binds the end into the 已然形. It's often at the end of a sentence, but it is normally not directly preceded by a verb, so the form that is preceded by it is determined from the rest of the sentence. This can get tricky, which is the lovely trait that bound particles have when they're not in the middle of a sentence.   Like the others, it can be at the end of a sentence where the final verb is dropped, leaving it to imply the verb that is dropped. It can also show hailing after a person's name. This is just an application of its overall use of showing strong emphasis. This, strong emphasis, though, may at times show a concession. This can be determined when the two clauses, the one with こそ and the one that follows, seem to be counter to each other. You wouldn't, then, interpret the particle as meaning "certainly" but "although". Even still, the same degree of emphasis is maintained.  \textbf{Exercises } 58. 大かた、萬のしわざは止めて、暇あるこそ、めやすく、あらまほしけれ。 \hfill\break
It is generally desirable and seemly to quit one's myriad of doings and have leisure. \hfill\break
From the 徒然草.  59. さらにこそ信ぜられね。 \hfill\break
I absolutely cannot believe it! \hfill\break
From the 大鏡.  60. みつ潮のながれひる間を逢ひがたみみるめの浦に夜をこそ待て。 \hfill\break
Since meeting is hard until the tide dries, I will wait through the night to meet you at Mirume Bay. \hfill\break
From the 古今和歌集.  61. 敵にあふてこそ死にたけれ。悪所に落ちては死にたからず。 \hfill\break
We want to die facing the enemy! We don't want to die falling in a bad spot. \hfill\break
From the 平家物語.  62. これは龍のしわざにこそありけれ。 \hfill\break
This was definitely the doing of a dragon. \hfill\break
From the 竹取物語.     