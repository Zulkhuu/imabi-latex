    
\chapter{The Particles と, とも, ど, \& ども}

\begin{center}
\begin{Large}
第405課: The Particles と, とも, ど, \& ども 
\end{Large}
\end{center}
 
\par{ Although the title may suggest that this lesson is going to be very complicated, it'll actually be quite easy. You are essentially looking at only two different topics. }
      
\section{The Conjunctive と(も)}
 
\par{ These are both conjunctive particles that follow the 終止形 of verbs but the 連用形 of adjectives, and the shorter version is a rarer alternative and should not be confused with the conditional conjunctive particle と in Modern Japanese. Both show hypothetical concession and equate to "though" or "even if". The usage of this particle became quite restricted in the spoken language after the 江戸時代, but even in Modern Japanese, there are instances like 少なくとも (at the least) and 遅くとも (at the latest) where it is still quite productive. However, given that the shortened form was already uncommon in Classical Japanese, it is unlikely that you'll ever see it in Modern Japanese. \hfill\break
}

\par{1. 玉藻苅 海未通女等 見尓将去 船梶毛欲得 浪高友 (原文) ${\overset{\textnormal{}}{\text{}}}$ ${\overset{\textnormal{たまも}}{\text{玉藻}}}$ 刈る ${\overset{\textnormal{あま}}{\text{海人}}}$ をとめども見に行かむ ${\overset{\textnormal{ふなかぢ}}{\text{船楫}}}$ もがも波高くとも \hfill\break
If only there were a boat or turret to go see the fishermen's daughters harvesting seaweed. \hfill\break
From the 万葉集. }

\par{2. ${\overset{\textnormal{ちとせ}}{\text{千年}}}$ を過ぐすとも、一夜の夢の心ちこそせめ。 (原文) \hfill\break
Even if a thousand years were to pass, it would undoubtedly feel like a dream of a single night. \hfill\break
From the 徒然草. }

\par{3. 消えずはありとも \hfill\break
Even if they're still there and not yet disappeared\dothyp{}\dothyp{}\dothyp{} \hfill\break
From the 古今和歌集. }

\par{4. よろづ代 ${\overset{\textnormal{ふ}}{\text{経}}}$ とも色はかはらじ。 \hfill\break
Even if ten thousand generations passed, its color would probably not change. \hfill\break
From the 古今和歌集. }

\par{5. さはありとも、などか宮仕へをしたまはざらむ。 \hfill\break
Even so, why aren't you doing your court service? \hfill\break
From the 竹取物語. }

\par{6. 花の色は霞にこめて見せずとも香をだにぬすめ春の山風 \hfill\break
Although there's no help but hiding the flower colors with mist, at least steal their scent! \hfill\break
From the 古今和歌集. }

\par{7. 穂に出でたりとかひやなからん。 \hfill\break
Even if it were to come into ear, it would probably be to no avail. \hfill\break
From the 蜻蛉日記. }

\par{8. さと人のことは夏ののしげくともかれ行くきみにあはざらめやは。 \hfill\break
Even if the rumors of the villagers were to grow rampant like the summer grass, would I not be able       to stand seeing you who runs away from (those rumors)? \hfill\break
From the 古今和歌集. }

\par{9. 尓保杼里乃 於吉奈我河波半 多延奴等母 伎美尓可多良武 己等都奇米也母 (原文) \hfill\break
にほ鳥の息長川は絶えぬとも君に語らむ言尽きめやも \hfill\break
Even if the Okinaga River were to cease, would the words I want to speak to you ever run out!? \hfill\break
From the 万葉集. \hfill\break
\hfill\break
\textbf{Noun Note }: にほ鳥・鳰鳥 means "grebe" and is an epithet word. }

\par{10. わが身は女なりとも敵の手にはかかるまじ。 \hfill\break
Though I may be a woman, I shall not fall into the hands of the enemy. \hfill\break
From the 平家物語. }

\par{\textbf{Word Note }: It was also common to see the verb 見る shortened to just 見 with と(も). }

\par{11. 徃廻 雖見将飽八 名寸隅乃 船瀬之濱尓 四寸流思良名美   (原文) 往き還り見とも飽かめやも名寸隅の船瀬の浜にしきる白波 Going and coming, no matter how much I see, will I ever get tired of seeing them, the white wave incessantly crashing into the Nakisumi Harbor's bay? From the 万葉集. }
      
\section{The Conjunctive ど(も)}
 
\par{ Although it still exists in Modern Japanese, the conjunctive ど(も), which attaches to the 已然形 of a conjugatable part of speech and is quite productive just as much in either variant, it is largely replaced in Modern Japanese with け(れ)ど(も), which is clearly  けり+ど(も). This also means "although" but it creates a direct concession from existing conditions. So, it is never hypothetical. }

\par{12. 残るといへども明日に枯れぬ。 \hfill\break
Although they say that it remains, it ends up withering in the morning sun. \hfill\break
From the 方丈記. }

\par{13. かくて明けゆく空の気色、昨日にかはりたりとは見えねど、ひきかへめづらしき心地ぞする。 \hfill\break
As for the sky's appearance such as this with the day breaking, although one can't see that it has           really changed since yesterday, one can certainly feel refreshed at heart. \hfill\break
From the 徒然草. }

\par{14. ${\overset{\textnormal{つる}}{\text{鶴}}}$ は、いとこちたきさまなれど、鳴く声の ${\overset{\textnormal{くもゐ}}{\text{雲井}}}$ まで聞こゆる、いとめでたし。 \hfill\break
Though the crane has a very overbearing appearance, it is very admirable that its singing voice can    be heard as far as the clouds. \hfill\break
From the 枕草子. }

\par{15. 男、血の涙を流せども、とどむるよしなし。 \hfill\break
Though the man shed tears of blood, there was no way to keep her (alive). \hfill\break
From the 伊勢物語. }

\par{16. よきほどにて出で給ひぬれど、なほ事ざまの優におぼえて、物のかくれよりしばし見ゐたるに、妻戸を今少し \hfill\break
おしあけて、月見るけしきなり。 \hfill\break
At the right moment, the person came out, though I was in awe of the home, I looked in cover and          saw the lady of the house from the gazing at the moon with the door opened. \hfill\break
From the 徒然草. }

\par{\textbf{Historical Note }: As far as frequency is concerned, starting in the 平安時代 the particle ど became associated with women's literature and ども became associated with 漢文-style writing. By the 鎌倉時代, ど essentially disappeared leaving only ども, which by the  室町時代 was almost completely replaced with けれども as is the case to this day. }
    