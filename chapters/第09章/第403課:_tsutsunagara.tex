    
\chapter{The Particles つつ \& ながら}

\begin{center}
\begin{Large}
第403課: The Particles つつ \& ながら  
\end{Large}
\end{center}
 
\par{ These particles even in antiquity were still almost the same. So, hopefully this will be one of the easiest lessons concerning Classical Japanese. }
      
\section{The Conjunctive Particle つつ}
 
\par{ The conjunctive particle つつ follows the 連用形 of a verb to show repetitive or continuous action or simultaneous action. This latter usage is shared with ながら, which eventually replaces it altogether in the spoken language by the 室町時代. You may sometimes see it combined with the perfective ぬ (につつ). This is still used in some schools of poetry. Also be aware that this can be seen at the end of a sentence where an auxiliary like たり or り would be to give a big emotional pull. This is seen a lot in 和歌. }

\par{1. 春花能 宇都路布麻泥尓 相見祢婆 月日餘美都追 伊母麻都良牟曽 (原文) \hfill\break
春花のうつろふまでに相見ねば月日よみつつ妹待つらむぞ \hfill\break
Since we won\textquotesingle t meet till the spring flowers have scattered, my wife will wait as she counts the days. \hfill\break
From the 万葉集. }

\par{2. 於保吉宇美能 美奈曽己布可久 於毛比都〃 毛婢伎奈良之思 須我波良能佐刀 (原文) \hfill\break
大き海の水底深く思ひつつ裳引きならしし菅原の里 \hfill\break
Oh Sugawara Village with the foot of the mountains drawn and flat, I adore as deep as the bottom of       a great sea. \hfill\break
From the 万葉集. }

\par{3. 野山にまじりて竹を取りつつ、よろづの事に使ひけり。 \hfill\break
Repeatedly entering the fields and hills and gathering bamboo, he used it for thousands of things. \hfill\break
From the 竹取物語. }

\par{4. 多都多夜麻 見都〃古要許之 佐久良波奈 知利加須疑奈牟 和我可敝流刀尓 (原文) \hfill\break
龍田山見つつ越え来し桜花散りか過ぎなむわが帰るとに \hfill\break
I wonder if the cherry blossoms I came to see as I cross Tatsuta Mountain will fall while I\textquotesingle m not home. \hfill\break
From the 万葉集. }

\par{5. 角障経   石村毛不過   泊瀬山   何時毛将超   夜者深去通都 (原文) \hfill\break
つのさはふ ${\overset{\textnormal{いはれ}}{\text{磐余}}}$ も過ぎず泊瀬山いつかも越えむ夜は更けにつつ \hfill\break
Still having not passed Iware, and though the night is already going to get late, I wonder when I\textquotesingle ll             cross Hatsuse Mountain? \hfill\break
From the 万葉集. }
 
\par{\textbf{Word Note }: つのさはふ is an epithet word for 磐余. }
      
\section{The Conjunctive Particle ながら}
 
\par{ The particle ながら comes from the combination of the ancient attribute marker な, which is still seen fossilized in some words such as 水底 from above, and から, which was once a noun as has been demonstrated from passages from the 万葉集. }

\par{ Like つつ, it can be used to show simultaneous action, but in doing so it might also show a concessive connection. ながら can and does follow nominals often. In doing so, it shows that something is in the same state as it has been. The not-so-common meaning of "all" came about during the 平安時代. }

\par{\textbf{Orthography Note }: You can sometimes see this particle spelled as 乍(ら). }

\begin{center}
 \textbf{Examples }
\end{center}

\par{1. 波利夫久路 應婢都々氣奈我良 佐刀其等邇 天良佐比安流氣騰 比等毛登賀米授 (原文) \hfill\break
針袋帯び続けながら里ごとに照らさひ歩けど人もとがめず \hfill\break
Lowering a needle bag on the waist, I walked and showed off through each town, and no one could       find fault. \hfill\break
From the 万葉集. }

\par{2. さざなみや滋賀の都は荒れにしを昔ながらの山桜かな。 \hfill\break
The capital of Shiga has been ravaged, but the mountain cherry blossoms are blowing just as in the   distant past! \hfill\break
From the 千載集. }

\par{3. 身はいやしながら、母なむ宮なりける。 \hfill\break
Though his position was low, his mother was a princess. \hfill\break
From the 伊勢物語. }

\par{4. 膝元に置きつつ、食ひながら文をも読みけり。 \hfill\break
Placing it at the foot of his knees, he ate while lecturing over the texts. \hfill\break
From the 徒然草. }

\par{5. 取りつきながら、いたう睡りて、落ちぬべき時に目を醒ます事、度々なり。 \hfill\break
While clinging, he went into a deep sleep, and he frequently woke right as he was about to fall down. \hfill\break
From the 徒然草 }
    