    
\chapter{Direction Intransitives + を}

\begin{center}
\begin{Large}
第151課: Direction Intransitives + を: 向く 
\end{Large}
\end{center}
 
\par{ There are a handful of verbs in Japanese that handle direction. Of these, the first verbs that come to mind are words like 行く(to go), 来る (to come), 歩く (to walk), and 走る (to run). The particles that are naturally associated with these verbs are に and へ, but most intrinsically に. The use of these particles indicate that these verbs are 自動詞  (intransitive verbs) as opposed to 他動詞 (transitive verbs). }
 
\par{ However, it is not the case that all direction verbs are intransitive. For example, consider the following verbs: }
 
\par{・指す (to point) }
 
\par{1. コンパスの ${\overset{\textnormal{ししん}}{\text{指針}}}$ が ${\overset{\textnormal{ほくせい}}{\text{北西}}}$ を ${\overset{\textnormal{さ}}{\text{指}}}$ した。 \hfill\break
The compass needle pointed northwest. }
 
\par{・指差す (to point at) }
 
\par{2. ${\overset{\textnormal{かのじょ}}{\text{彼女}}}$ は ${\overset{\textnormal{げんかんさき}}{\text{玄関先}}}$ に ${\overset{\textnormal{お}}{\text{置}}}$ かれた ${\overset{\textnormal{かびん}}{\text{花瓶}}}$ を ${\overset{\textnormal{ゆびさ}}{\text{指差}}}$ した。 \hfill\break
She pointed at the vase placed at the front door. }
 
\par{・見る (to see\slash look) }
 
\par{3. ちゃんと ${\overset{\textnormal{まえ}}{\text{前}}}$ を ${\overset{\textnormal{み}}{\text{見}}}$ て ${\overset{\textnormal{ちゅうい}}{\text{注意}}}$ しながら ${\overset{\textnormal{ある}}{\text{歩}}}$ いてください。 \hfill\break
Please walk while paying attention by properly looking in front of you. }
 
\par{・捜す (to search) }
 
\par{4. ${\overset{\textnormal{こうえん}}{\text{公園}}}$ の ${\overset{\textnormal{はじ}}{\text{端}}}$ っこのほうを ${\overset{\textnormal{さが}}{\text{捜}}}$ すと ${\overset{\textnormal{み}}{\text{見}}}$ つかるかもしれない。 \hfill\break
You might find it if you search the edge of the park. }
 
\par{・向ける (to turn towards) \hfill\break
5. ${\overset{\textnormal{じじつ}}{\text{事実}}}$ と ${\overset{\textnormal{どうり}}{\text{道理}}}$ に ${\overset{\textnormal{せ}}{\text{背}}}$ を ${\overset{\textnormal{む}}{\text{向}}}$ ける。 \hfill\break
To turn one\textquotesingle s back on facts and logic. }
 
\par{・目指す (to aim\slash head for) }
 
\par{6. ${\overset{\textnormal{よとう}}{\text{与党}}}$ も ${\overset{\textnormal{やとう}}{\text{野党}}}$ も ${\overset{\textnormal{おおすじおな}}{\text{大筋同}}}$ じ方向を ${\overset{\textnormal{めざ}}{\text{目指}}}$ している。 \hfill\break
The ruling party and the opposition party are both aiming roughly in the same direction. }
 
\par{ These six verbs demonstrate that although verbs of direction may mostly be intransitive, there are a few handfuls that are in fact transitive. What this demonstrates is that “direction” can at times be conceptualized as the “object” of a verb rather than just a destination point. }
 
\par{ In this lesson, we will learn about how the particle を is used with one intransitive verb of direction in particular: 向く. Although it has the transitive form 向ける, 向く is frequently seen paired with を. The verb itself has a handful of meanings, each presenting clarity as to how particles work with it. }
 
\par{ Other semantically similar verbs behave similarly to 向く, and so upon fully studying 向く and its important derivatives, we\textquotesingle ll spend some time looking at this grammar phenomenon with other examples. }
      
\section{The Verb 向く \& Its Derivatives}
 
\begin{center}
\textbf{を向く }
\end{center}

\par{ The verb 向く is one of only three verbs of direction that seemingly behave as transitive verbs with the case particle を despite being intrinsically intransitive. The remaining two verbs are 振り向く (to turn around) and 振り返る (to look\slash think back (on), but both of these verbs overlap semantically with 向く. The thought process as to why を is allowed with these verbs comes from the fact that the action carried out—shift in direction—isn\textquotesingle t so much an objective change of state, but rather a subjective action done by the agent; it is this direction of sense, if you will, that allows 向くto behave like a transitive verb as most transitive verbs in Japanese imply an active agent causing the action in question to occur. By proxy, this affects 振り向くand 振り返る as we will also look at in further detail. }

\par{ を向く can be used even with non-volitional agents. In other words, even if the doer of the action isn\textquotesingle t necessarily doing the act of turning out of its own volition, the particle を is still overwhelmingly used. When the agent is overtly purposely doing the act of turning, then を becomes obligatory. }

\par{ For を向く to be grammatical, the “object” must be one that can be conceptualized as a direction word. If it can\textquotesingle t, it will need to take のほう (the direction of). }

\par{7. ${\overset{\textnormal{め}}{\text{目}}}$ が ${\overset{\textnormal{うえ}}{\text{上}}}$ を ${\overset{\textnormal{む}}{\text{向}}}$ いている ${\overset{\textnormal{とき}}{\text{時}}}$ は、 ${\overset{\textnormal{かんが}}{\text{考}}}$ えをまとめている ${\overset{\textnormal{とき}}{\text{時}}}$ です。 \hfill\break
When one\textquotesingle s eyes are facing up, it\textquotesingle s when one is gathering one\textquotesingle s thoughts. }

\par{8. ${\overset{\textnormal{おも}}{\text{重}}}$ さで ${\overset{\textnormal{せんぷうき}}{\text{扇風機}}}$ が ${\overset{\textnormal{した}}{\text{下}}}$ を ${\overset{\textnormal{む}}{\text{向}}}$ いてしまいます。 \hfill\break
The fan faces downward due to its weight. }

\par{9. ${\overset{\textnormal{した}}{\text{下}}}$ を ${\overset{\textnormal{む}}{\text{向}}}$ くな。 \hfill\break
Don\textquotesingle t look down. }

\par{\textbf{Grammar Note }: The を seen in Exs. 7-9 can be interpreted as the を seen with intransitive verbs of movement, making its usage obligatory in situations such as theses. }

\par{10. ナマケモノは ${\overset{\textnormal{まうし}}{\text{真後}}}$ ろを ${\overset{\textnormal{む}}{\text{向}}}$ いたまま、 ${\overset{\textnormal{きのぼ}}{\text{木登}}}$ りができる。 \hfill\break
Sloths can climb trees whilst still facing right behind them. }

\par{11. ${\overset{\textnormal{ほんどの}}{\text{本殿}}}$ が ${\overset{\textnormal{うみ}}{\text{海}}}$ を ${\overset{\textnormal{む}}{\text{向}}}$ いて ${\overset{\textnormal{た}}{\text{建}}}$ っている。 \hfill\break
The main shrine is built towards the sea. }

\par{12. ${\overset{\textnormal{いしゃ}}{\text{医者}}}$ のほうを ${\overset{\textnormal{む}}{\text{向}}}$ いてではなく、 ${\overset{\textnormal{かんじゃ}}{\text{患者}}}$ さんのほうを ${\overset{\textnormal{む}}{\text{向}}}$ いて ${\overset{\textnormal{しごと}}{\text{仕事}}}$ をしたい。 \hfill\break
I want to work turned towards the patients and not towards the physician. \hfill\break
 \hfill\break
13. ${\overset{\textnormal{ぼく}}{\text{僕}}}$ はふいと ${\overset{\textnormal{えみこ}}{\text{笑美子}}}$ さんのほうを ${\overset{\textnormal{む}}{\text{向}}}$ いて ${\overset{\textnormal{ことば}}{\text{言葉}}}$ をかけた。 \hfill\break
I suddenly turned toward Emiko and spoke to her. }

\par{14. ${\overset{\textnormal{こくばん}}{\text{黒板}}}$ を ${\overset{\textnormal{む}}{\text{向}}}$ いて ${\overset{\textnormal{き}}{\text{聞}}}$ くことが ${\overset{\textnormal{ちゅうしん}}{\text{中心}}}$ の ${\overset{\textnormal{じゅぎょう}}{\text{授業}}}$ では、 ${\overset{\textnormal{こどもたち}}{\text{子供達}}}$ の ${\overset{\textnormal{しゅうちゅう}}{\text{集中}}}$ も ${\overset{\textnormal{ながつづ}}{\text{長続}}}$ きしません。 \hfill\break
In classes centered around listening whilst facing the board, the children\textquotesingle s concentration won\textquotesingle t last long either. }

\begin{center}
\textbf{The Transitive Verb 向ける } \hfill\break

\end{center}

\par{ When moving something into a certain direction, however, the direction becomes an indirect object, thus requiring the transitive form 向ける to be used. }

\par{15. ${\overset{\textnormal{けんたろう}}{\text{謙太郎}}}$ が、 ${\overset{\textnormal{ひだり}}{\text{左}}}$ の ${\overset{\textnormal{せんぷうき}}{\text{扇風機}}}$ を ${\overset{\textnormal{した}}{\text{下}}}$ に、 ${\overset{\textnormal{みぎ}}{\text{右}}}$ の ${\overset{\textnormal{せんぷうき}}{\text{扇風機}}}$ を ${\overset{\textnormal{みぎ}}{\text{右}}}$ に ${\overset{\textnormal{む}}{\text{向}}}$ けた。 \hfill\break
Kentaro pointed the fan on the left downward and the fan on the right to the right. }

\par{16. ${\overset{\textnormal{ひと}}{\text{人}}}$ ばかり ${\overset{\textnormal{うつ}}{\text{写}}}$ るので、カメラレンズを ${\overset{\textnormal{うえ}}{\text{上}}}$ に ${\overset{\textnormal{む}}{\text{向}}}$ けた。 \hfill\break
Because only people would be in the picture, I pointed the camera lens upward. }

\par{ It is also the case that を向ける can follow direction words, and when it does, it implies explicit active will on the part of the agent in turning that said direction. }

\par{17. ${\overset{\textnormal{みな}}{\text{皆}}}$ が ${\overset{\textnormal{ひだり}}{\text{左}}}$ を ${\overset{\textnormal{む}}{\text{向}}}$ いたとき、あなただけ ${\overset{\textnormal{みぎ}}{\text{右}}}$ を ${\overset{\textnormal{む}}{\text{向}}}$ けるか。 \hfill\break
Only you turn to the right when everyone turns to the left? }

\par{\textbf{Sentence Note }: In Ex. 17 above, the first clause uses を向く with no implied nuance of the subject “everyone” truly purposely orienting to the left, but the second clause uses を向ける, which does imply that the subject “you” are purposely orienting yourself to the right. }

\begin{center}
\textbf{に向く } \hfill\break

\end{center}

\par{ It goes without saying that the original particle that has always been paired with 向く has been に. This make senses as even in English, the word “toward(s)” usually follows “to turn.” The destination\slash direction of the orientation of the subject is emphasized with the use of に, and consequently, there isn\textquotesingle t an implied active agent. Lastly, the state that に向く describes must be one that will or is ongoing. }

\par{18. ${\overset{\textnormal{しんげつ}}{\text{新月}}}$ の ${\overset{\textnormal{ころ}}{\text{頃}}}$ は、 ${\overset{\textnormal{ちきゅう}}{\text{地球}}}$ の ${\overset{\textnormal{たいよう}}{\text{太陽}}}$ に ${\overset{\textnormal{て}}{\text{照}}}$ らされた ${\overset{\textnormal{めん}}{\text{面}}}$ が、ほぼ ${\overset{\textnormal{つき}}{\text{月}}}$ のほうに ${\overset{\textnormal{む}}{\text{向}}}$ いている。 \hfill\break
In a new moon, the side of the Earth that is illuminated by the Sun is roughly pointed towards the Moon. }

\par{\textbf{Sentence Note }: In Ex. 18, the side of the Earth illuminated by the sun is not purposely oriented toward the moon. This statement is a simply fact of observation being made about a natural phenomenon. }

\par{19. ${\overset{\textnormal{わたし}}{\text{私}}}$ の ${\overset{\textnormal{かんしん}}{\text{関心}}}$ はいつも ${\overset{\textnormal{みらい}}{\text{未来}}}$ に ${\overset{\textnormal{む}}{\text{向}}}$ いている。 \hfill\break
My interests are always directed to the future. }

\par{20. ${\overset{\textnormal{しゅじんこう}}{\text{主人公}}}$ の ${\overset{\textnormal{あし}}{\text{足}}}$ の ${\overset{\textnormal{む}}{\text{向}}}$ くところには ${\overset{\textnormal{ほのお}}{\text{炎}}}$ が ${\overset{\textnormal{も}}{\text{燃}}}$ え ${\overset{\textnormal{ひろ}}{\text{広}}}$ がっていた。 \hfill\break
Flames spread where the protagonist\textquotesingle s feed headed. }

\par{\textbf{Grammar Note }: The phrase 足が向く, alternatively seen as 足の向く when modifying a noun phrase as is the case in Ex. 20, uses the particle が before 向く only because no direction-noun is stated. If there were, it would be marked withに or へ. }

\par{21. ${\overset{\textnormal{しごとがら}}{\text{仕事柄}}}$ 、 ${\overset{\textnormal{ほんや}}{\text{本屋}}}$ に ${\overset{\textnormal{い}}{\text{行}}}$ くと、 ${\overset{\textnormal{びじゅつ}}{\text{美術}}}$ の ${\overset{\textnormal{ほん}}{\text{本}}}$ のコーナーに ${\overset{\textnormal{あし}}{\text{足}}}$ が ${\overset{\textnormal{む}}{\text{向}}}$ く。 \hfill\break
Because of my work, when I go to a book store, my feet head for the fine arts corner. }

\par{22. ${\overset{\textnormal{いえ}}{\text{家}}}$ の ${\overset{\textnormal{む}}{\text{向}}}$ きが ${\overset{\textnormal{すこ}}{\text{少}}}$ し ${\overset{\textnormal{なな}}{\text{斜}}}$ めに ${\overset{\textnormal{む}}{\text{向}}}$ いている。 \hfill\break
The aspect of the house is slightly tilted. }

\par{ There is a particular usage of に向く that is unique to it and not shared with を向く—or へ向く which is to be showcased next—is being synonymous with ~に適している, “to be suited\slash apt\slash fit for…” }

\par{23. アナウンサーに ${\overset{\textnormal{む}}{\text{向}}}$ いている ${\overset{\textnormal{ひと}}{\text{人}}}$ の ${\overset{\textnormal{とくちょう}}{\text{特徴}}}$ は ${\overset{\textnormal{なに}}{\text{何}}}$ でしょうか。 \hfill\break
What are the characteristics of someone who\textquotesingle s fit for announcing. }

\par{24. うちの ${\overset{\textnormal{むすこ}}{\text{息子}}}$ は物書きに ${\overset{\textnormal{む}}{\text{向}}}$ いていない。 \hfill\break
My son is not cut out for writing. }

\par{25. ${\overset{\textnormal{だれ}}{\text{誰}}}$ しも、 ${\overset{\textnormal{いま}}{\text{今}}}$ の ${\overset{\textnormal{しごと}}{\text{仕事}}}$ で ${\overset{\textnormal{おも}}{\text{思}}}$ うような ${\overset{\textnormal{けっか}}{\text{結果}}}$ が ${\overset{\textnormal{で}}{\text{出}}}$ ないと、「 ${\overset{\textnormal{じぶん}}{\text{自分}}}$ はこの ${\overset{\textnormal{しごと}}{\text{仕事}}}$ に ${\overset{\textnormal{む}}{\text{向}}}$ いていないのでは」という ${\overset{\textnormal{ぎもん}}{\text{疑問}}}$ が ${\overset{\textnormal{あたま}}{\text{頭}}}$ をよぎったりする。 \hfill\break
Whenever one doesn\textquotesingle t get the results that one thought at one\textquotesingle s current job, the question as to whether “one is suited for the job” will cross anyone\textquotesingle s mind. }

\par{26. ${\overset{\textnormal{かれ}}{\text{彼}}}$ の ${\overset{\textnormal{たいけい}}{\text{体型}}}$ は ${\overset{\textnormal{すいえい}}{\text{水泳}}}$ に ${\overset{\textnormal{む}}{\text{向}}}$ いている。 \hfill\break
His build is suited for swimming. }

\begin{center}
\textbf{へ向く } \hfill\break

\end{center}

\par{ へ向く is largely synonymous with the first sense of に向く, but it is especially when one wishes to express a change in orientation that is heading away from an original position. }

\par{27. ${\overset{\textnormal{なえぎ}}{\text{苗木}}}$ が ${\overset{\textnormal{たいよう}}{\text{太陽}}}$ の ${\overset{\textnormal{ほう}}{\text{方}}}$ へ ${\overset{\textnormal{む}}{\text{向}}}$ いてぐんぐんと ${\overset{\textnormal{せいちょう}}{\text{成長}}}$ している。 \hfill\break
The saplings are growing steadily toward the sun. }

\par{28. ${\overset{\textnormal{きみ}}{\text{君}}}$ はどっちの ${\overset{\textnormal{ほうがく}}{\text{方角}}}$ へ ${\overset{\textnormal{あし}}{\text{足}}}$ が ${\overset{\textnormal{む}}{\text{向}}}$ くかね。 \hfill\break
I wonder which direction you\textquotesingle ll head for. }

\par{29. ${\overset{\textnormal{ちゅうごく}}{\text{中国}}}$ の ${\overset{\textnormal{しせん}}{\text{視線}}}$ は ${\overset{\textnormal{いま}}{\text{今}}}$ 、 ${\overset{\textnormal{べいこく}}{\text{米国}}}$ へ ${\overset{\textnormal{む}}{\text{向}}}$ いている。 \hfill\break
China\textquotesingle s gaze is currently pointed toward America. }

\par{30. いつも ${\overset{\textnormal{がい}}{\text{外}}}$ へ ${\overset{\textnormal{む}}{\text{向}}}$ いている ${\overset{\textnormal{いしき}}{\text{意識}}}$ を ${\overset{\textnormal{うちがわ}}{\text{内側}}}$ に ${\overset{\textnormal{む}}{\text{向}}}$ け、 ${\overset{\textnormal{しごと}}{\text{仕事}}}$ のストレスを ${\overset{\textnormal{やわ}}{\text{和}}}$ らげましょう。 \hfill\break
Point your awareness which is always pointed outward inward and alleviate your work stress. }

\par{\textbf{Grammar Note }: Ex. 30 demonstrates how へ向く, and by proxy に向く are suitable for when 向く is used with abstract subjects. を向く would, in fact, be incorrect. }

\begin{center}
\textbf{Differences Matter } \hfill\break

\end{center}

\par{ Given that を向く, に向く, へ向く, and  ~を(~に)向ける aren\textquotesingle t exactly the same, it\textquotesingle s only natural that they can all occur at the same time. Ex. 19 is an example of all four of these forms used in tandem. }

\par{31. ${\overset{\textnormal{じょうはんしん}}{\text{上半身}}}$ を ${\overset{\textnormal{みぎ}}{\text{右}}}$ に ${\overset{\textnormal{む}}{\text{向}}}$ けて、この ${\overset{\textnormal{みぎ}}{\text{右}}}$ に ${\overset{\textnormal{む}}{\text{向}}}$ いた ${\overset{\textnormal{じかん}}{\text{時間}}}$ が ${\overset{\textnormal{なが}}{\text{長}}}$ いほど、インパクトで ${\overset{\textnormal{からだ}}{\text{体}}}$ が ${\overset{\textnormal{しょうめん}}{\text{正面}}}$ を ${\overset{\textnormal{む}}{\text{向}}}$ いている ${\overset{\textnormal{じかん}}{\text{時間}}}$ は ${\overset{\textnormal{いっしゅん}}{\text{一瞬}}}$ で、 ${\overset{\textnormal{すばや}}{\text{素早}}}$ く ${\overset{\textnormal{ひだり}}{\text{左}}}$ へ ${\overset{\textnormal{む}}{\text{向}}}$ いていくというのがポイントです。 \hfill\break
The point is to point your upper body to the right, and the longer it\textquotesingle s pointed to the right, the time the body points to the front at impact becomes instantaneous, at which point the body swiftly faces leftward. }

\begin{center}
\textbf{The Intransitive Verb 向かう }
\end{center}

\par{ Another verb that derives from 向く is 向かう. It\textquotesingle s a combination of 向くand the archaic auxiliary verb ふ, which is used to express a continuous state. 向かう means “to face” or “to go towards.” Even still, there is a subtle nuance that once one “faces” or “heads toward” X that the state will last for a certain length of time, or that the change in orientation will have a measurable duration. This verb is solely intransitive and either takes the particles に or へ, but never を. }

\par{32. ${\overset{\textnormal{おおがた}}{\text{大型}}}$ の ${\overset{\textnormal{たいふう}}{\text{台風}}}$ ${\overset{\textnormal{じゅうご}}{\text{15}}}$ ${\overset{\textnormal{ごう}}{\text{号}}}$ が ${\overset{\textnormal{にほん}}{\text{日本}}}$ へ ${\overset{\textnormal{む}}{\text{向}}}$ かっている。 \hfill\break
Large-scale Typhoon \#15 is heading toward Japan. }

\par{33. ${\overset{\textnormal{ふゆ}}{\text{冬}}}$ に ${\overset{\textnormal{む}}{\text{向}}}$ かっているのに、なんていいお ${\overset{\textnormal{てんき}}{\text{天気}}}$ ! \hfill\break
Although we\textquotesingle re heading to winter, what great weather it is (today)! }

\par{34. ${\overset{\textnormal{にゅういん}}{\text{入院}}}$ のブッシュ ${\overset{\textnormal{もとだいとうりょう}}{\text{元大統領}}}$ 、 ${\overset{\textnormal{かいほう}}{\text{快方}}}$ に ${\overset{\textnormal{む}}{\text{向}}}$ かう \hfill\break
Hospitalized Former President Bush Getting Better }
      
\section{The Suffixes ~向き \& ~向け}
 
\par{ The suffixes ~向き and ~向け are very similar, but there are a few subtle differences that ultimately make the latter far more commonly used. Don't let commonality, though, confuse you into never using the former as it is necessary in its own circumstances, which are detailed first below. }

\begin{center}
 \textbf{The Suffix ~向き }\hfill\break

\end{center}

\par{ The first usage of ~向き is to mean “facing” when attached to literal direction words such as 東 (east) and 南 (south). }

\par{35. ${\overset{\textnormal{しょっきだな}}{\text{食器棚}}}$ にお ${\overset{\textnormal{さら}}{\text{皿}}}$ やコップを ${\overset{\textnormal{はい}}{\text{入}}}$ れる ${\overset{\textnormal{とき}}{\text{時}}}$ 、 ${\overset{\textnormal{したむ}}{\text{下向}}}$ きに ${\overset{\textnormal{お}}{\text{置}}}$ いた ${\overset{\textnormal{ほう}}{\text{方}}}$ がいいんでしょうか。 \hfill\break
Is it best to place themes bottom up when putting up plates and cups in the cupboard? }

\par{36. せっかく ${\overset{\textnormal{あたら}}{\text{新}}}$ しい ${\overset{\textnormal{いえ}}{\text{家}}}$ に ${\overset{\textnormal{す}}{\text{住}}}$ むのなら、 ${\overset{\textnormal{あか}}{\text{明}}}$ るい ${\overset{\textnormal{みなみむ}}{\text{南向}}}$ きがいい。 \hfill\break
If you\textquotesingle re going to go through the trouble of living in a new home, one facing the bright southerly direction would be best. }

\par{37. ${\overset{\textnormal{わたし}}{\text{私}}}$ のように ${\overset{\textnormal{いま}}{\text{今}}}$ ネガティブな ${\overset{\textnormal{しこう}}{\text{思考}}}$ をしている ${\overset{\textnormal{ほう}}{\text{方}}}$ も、 ${\overset{\textnormal{まえむ}}{\text{前向}}}$ きになりたいと ${\overset{\textnormal{おも}}{\text{思}}}$ ったことがあるはずです。 \hfill\break
People who, like me, think negatively now should have had a moment where they wanted to become positive. }

\par{\textbf{Phrase Note }: 前向き literally means “facing forward” and is usually used in the sense of being “positive\slash proactive.” }

\par{ Another usage of ~向き is being equivalent to ~に適した (suited\slash apt\slash fit for…). This usage is also seen in the phrase 向き不向き, which means “being cut out for certain things and not for others.” It is implied that the suitability is naturally so. }

\par{38. ${\overset{\textnormal{あんしん}}{\text{安心}}}$ して ${\overset{\textnormal{く}}{\text{暮}}}$ らせる ${\overset{\textnormal{がくせいむ}}{\text{学生向}}}$ きの ${\overset{\textnormal{ぶっけん}}{\text{物件}}}$ を ${\overset{\textnormal{かずおお}}{\text{数多}}}$ く ${\overset{\textnormal{あつか}}{\text{扱}}}$ っています。 \hfill\break
We handle a vast number of properties suited for students to be able to live with peace of mind. }

\par{39. この ${\overset{\textnormal{しょうせつ}}{\text{小説}}}$ は ${\overset{\textnormal{ぐうぜん}}{\text{偶然}}}$ にも、 ${\overset{\textnormal{にほんご}}{\text{日本語}}}$ を ${\overset{\textnormal{べんきょう}}{\text{勉強}}}$ している ${\overset{\textnormal{がいこくじんむ}}{\text{外国人向}}}$ きでもある。 \hfill\break
This novel is coincidentally also suited for foreigners who are studying Japanese. }

\par{40. これは ${\overset{\textnormal{かんこくじんむ}}{\text{韓国人向}}}$ きの ${\overset{\textnormal{からくち}}{\text{辛口}}}$ ビールです。 \hfill\break
This is a spicy beer suited for Koreans. }

\par{41. ${\overset{\textnormal{しごと}}{\text{仕事}}}$ に ${\overset{\textnormal{む}}{\text{向}}}$ き ${\overset{\textnormal{ふむ}}{\text{不向}}}$ きってあるんですか。 \hfill\break
Is there such thing as some things being cut or not cut out for you in jobs? }

\begin{center}
\textbf{The Suffix ~向け } \hfill\break

\end{center}

\par{ Similarly, ~向け is used to indicate that something has been tailored towards something\slash someone. Essentially, it indicates a target. For instance, 日本向けの商品 means “merchandise tailored\slash targeted for Japan.” }

\par{42. ${\overset{\textnormal{しょしんしゃむ}}{\text{初心者向}}}$ けのIMABIの ${\overset{\textnormal{きょうかしょ}}{\text{教科書}}}$ を ${\overset{\textnormal{か}}{\text{買}}}$ いたい。 \hfill\break
I wish I could buy an IMABI textbook made for beginners. }

\par{43. ${\overset{\textnormal{にほん}}{\text{日本}}}$ に ${\overset{\textnormal{す}}{\text{住}}}$ んでいた ${\overset{\textnormal{とき}}{\text{時}}}$ は、大人向けの ${\overset{\textnormal{きょうようばんぐみ}}{\text{教養番組}}}$ をいつも ${\overset{\textnormal{み}}{\text{見}}}$ ていた。 \hfill\break
When I lived in Japan, I always watched educational programs made for adults. }

\par{ 44. ${\overset{\textnormal{すげもとそうりだいじん}}{\text{菅元総理大臣}}}$ はいくつかの ${\overset{\textnormal{ぜんこくむ}}{\text{全国向}}}$ けテレビ ${\overset{\textnormal{えんぜつ}}{\text{演説}}}$ をしました。 \hfill\break
Prime Minister Kan made several nation-wide televised speeches. }
      
\section{Other Mentions: 振り向く, 振り返る, \& 注目する}
 
\begin{center}
\textbf{振り向く } \hfill\break

\end{center}

\par{ In the sense of “to draw interest in…”, 振り向く is used with the particle に. Otherwise, in the sense of meaning “to look back at,” it is always used with the particle を. }

\par{45. ${\overset{\textnormal{きょうみ}}{\text{興味}}}$ のあることには ${\overset{\textnormal{ふ}}{\text{振}}}$ り ${\overset{\textnormal{む}}{\text{向}}}$ いて ${\overset{\textnormal{め}}{\text{目}}}$ を ${\overset{\textnormal{あ}}{\text{合}}}$ わせますが、 ${\overset{\textnormal{きょうみ}}{\text{興味}}}$ のないことには ${\overset{\textnormal{ふ}}{\text{振}}}$ り ${\overset{\textnormal{む}}{\text{向}}}$ きもしないことは、 ${\overset{\textnormal{にちじょう}}{\text{日常}}}$ でもよくあることです。 \hfill\break
Giving attention and making eye contact with things that are not interesting but not giving a bit of attention to things that aren\textquotesingle t interesting is something that often happens in the ordinary. }

\par{46. あなたの ${\overset{\textnormal{す}}{\text{好}}}$ きな ${\overset{\textnormal{ひと}}{\text{人}}}$ がどんな ${\overset{\textnormal{ひと}}{\text{人}}}$ に振り ${\overset{\textnormal{む}}{\text{向}}}$ くのかリサーチしましょう。 \hfill\break
Research the kind of person the person you like gives attention to! }

\par{47. あの ${\overset{\textnormal{ひ}}{\text{日}}}$ の ${\overset{\textnormal{あさ}}{\text{朝}}}$ 、 ${\overset{\textnormal{むすこ}}{\text{息子}}}$ を ${\overset{\textnormal{がっこう}}{\text{学校}}}$ に ${\overset{\textnormal{おく}}{\text{送}}}$ り ${\overset{\textnormal{だ}}{\text{出}}}$ したとき、こちらを振り ${\overset{\textnormal{む}}{\text{向}}}$ いた ${\overset{\textnormal{かお}}{\text{顔}}}$ がまさか ${\overset{\textnormal{さいご}}{\text{最後}}}$ になるとは ${\overset{\textnormal{おも}}{\text{思}}}$ いもしませんでした。 \hfill\break
That morning when I sent off my son to school, I had no idea that the face he gave me when looking back would be his last. }
 \textbf{振り返る }\textbf{}   に振り返る is used in the sense of “to look back at” in a literal sense due to something (noises, etc.) whereas を振り返る is used in a figurative sense as in “to think back on.”  
\par{48. ${\overset{\textnormal{はいご}}{\text{背後}}}$ の ${\overset{\textnormal{きみょう}}{\text{奇妙}}}$ な ${\overset{\textnormal{ものおと}}{\text{物音}}}$ に ${\overset{\textnormal{ふ}}{\text{振}}}$ り ${\overset{\textnormal{かえ}}{\text{返}}}$ った。 \hfill\break
I turned around at the strange sounds in the back. }

\par{49. ${\overset{\textnormal{おどろ}}{\text{驚}}}$ きながら ${\overset{\textnormal{ばくはつおん}}{\text{爆発音}}}$ に ${\overset{\textnormal{ふ}}{\text{振}}}$ り ${\overset{\textnormal{かえ}}{\text{返}}}$ った。 \hfill\break
I turned around surprised at the noise of the explosion. }

\par{50. ${\overset{\textnormal{ごじゅう}}{\text{50}}}$ ${\overset{\textnormal{ねんいじょう}}{\text{年以上}}}$ の ${\overset{\textnormal{むかし}}{\text{昔}}}$ を ${\overset{\textnormal{ふ}}{\text{振}}}$ り ${\overset{\textnormal{かえ}}{\text{返}}}$ る。 \hfill\break
To think back on olden times over fifty years ago. }

\begin{center}
\textbf{注目する } \hfill\break

\end{center}

\par{ In Standard Japanese grammar, 注目する (to notice) is used with the particle に. It is synonymous with 目を向ける, which makes it clear why it would take に. However, because it is semantically very similar to other verbs like 見る (to see\slash look), 注視する (to gaze steadily), and  監視する (to monitor) which all take を, some speakers do happen to say を注目する. It is important to reiterate, though, that に注目する is still the true, correct form. }

\par{51. ${\overset{\textnormal{がめんひだりした}}{\text{画面左下}}}$ \{に・△ を\} ${\overset{\textnormal{ちゅうもく}}{\text{注目}}}$ する。 \hfill\break
Notice the bottom left of the screen. }

\par{52. ${\overset{\textnormal{こんご}}{\text{今後}}}$ どうなるか\{に・△ を\} ${\overset{\textnormal{ちゅうもく}}{\text{注目}}}$ してください。 \hfill\break
Please pay attention to what becomes of it from now on. }

\par{53. ${\overset{\textnormal{まいにち}}{\text{毎日}}}$ ツイート\{に・△ を\} ${\overset{\textnormal{ちゅうもく}}{\text{注目}}}$ してください。 \hfill\break
Please pay attention to tweets every day. }

\par{54. ${\overset{\textnormal{いりょうぎょうかい}}{\text{医療業界}}}$ の ${\overset{\textnormal{な}}{\text{成}}}$ り ${\overset{\textnormal{ゆ}}{\text{行}}}$ きを\{ ${\overset{\textnormal{ちゅうし}}{\text{注視}}}$ ・△ ${\overset{\textnormal{ちゅうもく}}{\text{注目}}}$ \}する。 \hfill\break
To observe the development of the medical care industry. }

\par{55. ${\overset{\textnormal{がめん}}{\text{画面}}}$ の ${\overset{\textnormal{みぎした}}{\text{右下}}}$ \{に・△を\} ${\overset{\textnormal{ちゅうもく}}{\text{注目}}}$ してないよね。 \hfill\break
You aren\textquotesingle t paying attention to the bottom right of the screen, huh. }
    