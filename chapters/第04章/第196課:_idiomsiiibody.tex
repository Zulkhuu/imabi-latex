    
\chapter{Idioms III}

\begin{center}
\begin{Large}
第196課: Idioms III: The Body 
\end{Large}
\end{center}
 
\par{ This lesson will introduce you to a lot of important idioms that involve parts of the body. }
      
\section{Examples}
 
\par{1. マンガで目の保養になりました。 \hfill\break
Literal: Seeing the manga book became a recreation of the eyes. \hfill\break
Seeing the manga book was a feast for my eyes. }
 
\par{2. 待つほかに手はなかった。 \hfill\break
Literal: There wasn't a hand but to wait. \hfill\break
There was nothing that I could do but wait for him. }
 
\par{3. 汚い手を使う。 \hfill\break
Literal: To use a dirty hand. \hfill\break
To hit below the belt. }
 
\par{4. もし手に余るようなことがあれば \hfill\break
Literal: If there is something that is too much for your hands. \hfill\break
If there is something that is beyond your control }
 
\par{5. 先生のお言葉が今でも耳に残っています。 \hfill\break
Literal: What my teacher said is still even now remaining in my ear. \hfill\break
What my teacher said is still even now lingering in my ear. }
 
\par{6. 彼は大統領の右腕として知られています。 \hfill\break
Literal: He is known as the president\textquotesingle s right arm. \hfill\break
He is known as the president\textquotesingle s right hand man. }
 
\par{7. そのアイスティーは僕の口に合わなかった。 \hfill\break
Literal: The ice tea doesn\textquotesingle t go well with my mouth. \hfill\break
The ice tea doesn\textquotesingle t fit my tastes. }
 
\par{8. 借金で首が回らない。 \hfill\break
Literal: One\textquotesingle s neck can\textquotesingle t turn with debt. \hfill\break
To be eaten up with debt. }
 
\par{9. 彼は首を振った。 \hfill\break
Literal: He nodded his neck. \hfill\break
He nodded his head. }

\par{10. このケーキはおいしくて顎が落ちそうです。 \hfill\break
This cake is so delicious that my jaws are dropping. }

\par{11. 顎を出してしまった。 \hfill\break
Literal: I showed by chin. \hfill\break
I got exhausted. }
 
\par{12a. 首をやる。 \hfill\break
12b. 首を切る。(To cut the neck). \hfill\break
Literal: To kill the neck \hfill\break
To be hanged. }
 
\par{13. 職務怠慢で首にした。 \hfill\break
Literal: To be necked due to neglect of duty. \hfill\break
To be fired due to neglect of duty. }
 
\par{14. 今は手が塞がっている。 \hfill\break
Literal: My hands are occupied now. \hfill\break
My hands are full now. }
 
\par{15. いつまでも親の脛を齧るつもりだそうだ。 \hfill\break
Literal: It sounds like she plans to gnaw at her parents' shin forever. \hfill\break
It sounds like she intends to sponge off her parents forever. }
 
\par{16. 日本語の腕を磨くために日本に引っ越しました。 \hfill\break
Literal: I moved to Japan in order to polish my Japanese arm. \hfill\break
I moved to Japan in order to polish my Japanese skills. }
 
\par{17. 頭が重くなった。 \hfill\break
Literal: My head got heavy. \hfill\break
I got a headache. }
 
\par{18. お小僧は大人になって頭を丸めた。 \hfill\break
Literal: The boy became an adult and rounded his head. \hfill\break
The boy became an adult and then became a Buddhist priest. }
 
\par{19. あなたにはとても頭が上がりません。 \hfill\break
Literal: I do not really raise my head to you. \hfill\break
I have a great esteem for you. }
 
\par{20. その問題に頭を抱えている。 \hfill\break
Literal: I\textquotesingle m carrying my head in the problem. \hfill\break
I\textquotesingle m racking my mind over the problem. }
 
\par{21. 解決のため頭を搾る。 \hfill\break
Literal: To squeeze one\textquotesingle s head for a solution. \hfill\break
To think hard on something in order to reach a solution. }
 
\par{22. 頭を冷やせよ。 \hfill\break
Literal: Cool your head. \hfill\break
Cool it. }
 
\par{23. 名案がぱっと頭に浮かんだ。 \hfill\break
Literal: A good idea suddenly floated into my mind. \hfill\break
A good idea flashed before my mind. }
 
\par{24. 疑念が頭を擡げてきた。 \hfill\break
Literal: Suspicion raised a head. \hfill\break
Suspicion reared its head. }
 
\par{25. 頭にきた。 \hfill\break
Literal: It came in (my) mind. \hfill\break
That makes me angry! }

\par{26. 彼は、何をやっても足が地につかない。 \hfill\break
Literally: No matter what he does, his feet won't stick to the ground. \hfill\break
No matter what he does, he can't stick to anything. }
 
\par{27. 彼は頭が鈍くて理解が遅い。 \hfill\break
Literal: His mind is dull and he\textquotesingle s comprehension is slow. \hfill\break
He\textquotesingle s dull-minded and slow at comprehending. }

\par{28. 足が重い。 \hfill\break
Literal: Feet are heavy. \hfill\break
To have lead feet. }

\par{29. 彼女は気が多すぎるね。 \hfill\break
Literal: Her spirit is too much. \hfill\break
She's too fickle. }
 
\par{30. とにかく頭を下げてこい。 \hfill\break
Literal: Anyways come over and lower your head. \hfill\break
Just go and apologize. }
 
\par{31. 将棋の腕が上がったな。 \hfill\break
Literal: Your shogi arm has risen, hasn\textquotesingle t it? \hfill\break
Your skills in shogi have improved, haven\textquotesingle t they? }

\par{ Below is a handful of some of the most common idioms of the body. Hundreds more exist, but this can keep you busy. The more you learn, the more you see how these phrases are constructed and about the true meanings of the words in Japanese. }

\begin{ltabulary}{|P|P|P|P|}
\hline 

頭が荒い & To breath hard & 頭が固まる & To have a fixed idea \\ \cline{1-4}

頭が切れる & To be quick thinking & 頭を抱える & To rack over \\ \cline{1-4}

顔が厚い & To be impudent & 顔が売れる & To be popular \\ \cline{1-4}

顔が利く & To be influential & 顔を貸す & To meet\dothyp{}\dothyp{}\dothyp{}wishes \\ \cline{1-4}

顔を潰す & To blight someone's dignity & 鼻であしらう & To snub someone \\ \cline{1-4}

鼻が凹む & To be put down & 鼻で笑う & To snicker \\ \cline{1-4}

耳が肥える & To have an ear of & 耳に立つ & To strike one's ear \\ \cline{1-4}

口が上がる & To become eloquent & 口が開く & To make a beginning \\ \cline{1-4}

唇を奪う & To steal a kiss & 歯が浮く & To be nauseating \\ \cline{1-4}

歯が立つ & To be edible; in one's reach & 舌が伸びる & To exaggerate \\ \cline{1-4}

舌を返す & To change one's tune & 目が散る & To be diverted \\ \cline{1-4}

目が出る & The die is cast & 目が届く & To keep an eye on \\ \cline{1-4}

額に汗する & To do with all one's might & 額を集める & To confer together \\ \cline{1-4}

首が危ない & To be in grave danger & 肩が怒る & To get worked up \\ \cline{1-4}

腕を鳴らす & To gain recognition & 腕を引く & To make a solemn vow \\ \cline{1-4}

指を折る & To make a vow & 爪を研ぐ & To prepare for a fight \\ \cline{1-4}

手が切れる & To fall out with & 手が冴える & To be skilled \\ \cline{1-4}

手が焼ける & To be troublesome & 手が笑う & To lose control of hands \\ \cline{1-4}

胸が決まる & To decide to & 胸が焦げる & To be impatient; pine for \\ \cline{1-4}

胸が裂ける & To be heart-broken & 胴が据わる & To be resolute \\ \cline{1-4}

臍で笑う & To be preposterous & 腹がある & To have an agenda \\ \cline{1-4}

腹がいる & To vent anger & 胆が抜ける & To be scared stiff \\ \cline{1-4}

胆が冷える & To be scared to death & 心が動く & To be interested \\ \cline{1-4}

心が変わる & To be unfaithful & 心が腐る & To be corrupted \\ \cline{1-4}

腰が落ち着く & To take root & 腰を上げる & To take action \\ \cline{1-4}

足を洗う & To make a new start & 足を払う & To trip up \\ \cline{1-4}

踵を巡らす & To retrace steps & 血が通う & To be kindhearted \\ \cline{1-4}

脈が上がる & To pass away; lose hope & 脈を見る & To test viability \\ \cline{1-4}

骨がある & To have fortitude & 肌で感じる & To have first-hand experience \\ \cline{1-4}

筋を言う & To split hairs & 筋を書く & To have in mind \\ \cline{1-4}

体が空く & To be vacant & 体が続く & To be in good health \\ \cline{1-4}

\end{ltabulary}
    