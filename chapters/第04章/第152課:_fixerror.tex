    
\chapter{Fix\slash Error}

\begin{center}
\begin{Large}
第152課: Fix\slash Error: ~忘れる, ~直す, 間違える, \& ~誤る 
\end{Large}
\end{center}
 
\par{ As the title suggests, this lesson will be about compound verb endings for fixing or making errors of judgment. }
\textbf{}      
\section{~忘れる}
 
\par{${\overset{\textnormal{}}{\text{}}}$ 忘れる, which is an 一段 verb, means "to forget".  In compound verbs it means "to forget to\dothyp{}\dothyp{}\dothyp{}". This implies that you forgot to do something that you needed to do . If this is not what you wish to imply, ~のを忘れる should be used. Both are often used with ~てしまう, which means "to accidentally\slash end up doing". }
 
\par{1. テレビを消し忘れた。 \hfill\break
I forgot to turn off the TV. }
 
\par{2. ケーキを買い忘れました。 \hfill\break
I forgot to buy the cake. }
 
\par{3a. プリンターを消し忘れて会社を出た。 \hfill\break
3b. プリンターのスイッチを切り忘れて会社を出た。 \hfill\break
I left the company without turning off the printer. }
 
\par{4a. 宿題をし忘れてしまった。(ちょっと不自然) \hfill\break
4b. 宿題(を)するのを忘れた。 (もっと自然) \hfill\break
4c. 宿題を忘れた。(一番自然) \hfill\break
I accidentally forgot to do my homework. }
 
\par{\textbf{Naturalness Note }: The notes on naturalness are most relevant when there is no preceding context. If you were to tell your Japanese teacher, for instance, that you forgot to do your homework, it would be more appropriate to use 4a. }

\par{5. ${\overset{\textnormal{かし}}{\text{歌詞}}}$ を忘れるのは ${\overset{\textnormal{は}}{\text{恥}}}$ ずかしいじゃん。(Casual) \hfill\break
Isn't it embarrassing to forget your lyrics? }
 
\par{6. すっかり忘れたよ。 \hfill\break
I completely forgot. }
 
\par{7. 彼に電話するのをすっかり忘れてた。(Casual) \hfill\break
I'd completely forgotten to call him. }

\par{8. ${\overset{\textnormal{たき}}{\text{滝}}}$ の写真を ${\overset{\textnormal{と}}{\text{撮}}}$ るのを忘れた。 \hfill\break
I forgot to take a picture of the waterfall. }
      
\section{~直す}
 
\par{ 直す is used in a variety of situations to represent that one is curing or fixing something. In compound verbs it expresses that one is "doing something over again". }

\par{9. 電話をかけ直す。 \hfill\break
To call back. }

\par{10. 計画を ${\overset{\textnormal{ね}}{\text{練}}}$ り直した。 \hfill\break
I planned it over again. \hfill\break
 \hfill\break
11. やりなおしたんだ。 \hfill\break
I redid it. }

\par{\textbf{Orthography Note }: ~直す is often left in かな. }
      
\section{~間違える}
 
\par{  ${\overset{\textnormal{まちが}}{\text{間違}}}$ える is the transitive form of ${\overset{\textnormal{まちが}}{\text{間違}}}$ う and means "to confuse\slash mistake". The intransitive 間違う is sometimes used nowadays to function as 間違える, but when this happens, it tends to be in very casual speech. In compound verbs, ~間違える is used to mean "messed up on\dothyp{}\dothyp{}\dothyp{}". }

\par{12. 自分の名前を書き間違えました。 \hfill\break
I made a mistake in writing my name. }

\par{13. 言い間違えた。 \hfill\break
I made a mistake in speaking. }

\par{14. 歌うのを間違えた。 \hfill\break
I mistakenly sang the wrong song. }

\par{15. 歌い間違えた。 \hfill\break
I messed up on the song. }

\par{\textbf{Phrase Note }: Another similar phrase would be 歌で ${\overset{\textnormal{しっぱい}}{\text{失敗}}}$ した meaning "failed at the song". }

\par{16. 彼は僕を彼の弟と見間違えた。 \hfill\break
He mistook me for his little brother. }
      
\section{~誤る}
 
\par{   ${\overset{\textnormal{あやま}}{\text{誤}}}$ る means "to mistake" and is in compound verbs to show that one "mis-\dothyp{}\dothyp{}\dothyp{}s" something. }
 
\par{17. 彼は私を兄と見誤りました。 \hfill\break
He mistook me for my older brother. }
 
\par{18. 記事を読み誤ったんじゃないか。 \hfill\break
Didn't you misread the article? }
    