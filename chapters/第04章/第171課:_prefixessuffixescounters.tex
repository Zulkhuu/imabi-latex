    
\chapter{Counters VII Prefixes\slash Suffixes with Counters}

\begin{center}
\begin{Large}
第171課: Counters VII: Prefixes\slash Suffixes with Counters: 何~, 幾~, 数~, ~数, 半~, ~半, ~余, \& ~余り 
\end{Large}
\end{center}
 
\par{ In this lesson, rather than learn about new counters, we\textquotesingle ll study several important appendages added to counters themselves. You've already seen them at play at least once before, so this lesson should serve as a very informative yet well deserved break from learning more counters. }

\begin{center}
\textbf{Prefixes\slash Suffixes Covered }
\end{center}

\par{1. \emph{Nan }- 何~ \hfill\break
2. \emph{Iku }- 幾~ \hfill\break
3. \emph{Sū }- 数~ \hfill\break
4.       - \emph{sū }~数 \hfill\break
5. \emph{Han }- 半~ \hfill\break
6.       - \emph{han }~半 \hfill\break
7. \emph{-yo }- 余 \hfill\break
8.       - \emph{amari }余り }

\par{\textbf{Curriculum Note }: This lesson will be moved earlier into the curriculum in the next lesson reordering. }
      
\section{How many\dothyp{}\dothyp{}\dothyp{}}
 
\begin{center}
\textbf{\emph{Nan- }何~ }
\end{center}

\par{ Aside from certain temporal phrases where it may also function as “what…” \emph{nan }- 何 is typically used with counters to express “how…” as in quantity. As a recap of this, consider the following examples. }

\par{1. ${\overset{\textnormal{きにゅうも}}{\text{記入漏}}}$ れは、 ${\overset{\textnormal{いちにち}}{\text{一日}}}$ (に) ${\overset{\textnormal{なんけん}}{\text{何件}}}$ ありますか。 \hfill\break
 \emph{Kinyūmore wa, ichinichi (ni) nanken arimasu ka? \hfill\break
 }How many omissions are there a day? }

\par{2. ${\overset{\textnormal{だいあん}}{\text{代案}}}$ は ${\overset{\textnormal{なんぜん}}{\text{何千}}}$ もある。 \hfill\break
 \emph{Daian wa nanzen mo aru. \hfill\break
 }There are thousands of alternate plans. }

\par{3. ${\overset{\textnormal{きょねん}}{\text{去年}}}$ は ${\overset{\textnormal{ねんがじょう}}{\text{年賀状}}}$ を ${\overset{\textnormal{なんつうおく}}{\text{何通送}}}$ りましたか。 \hfill\break
 \emph{Kyonen wa nengajō wo nantsū okurimashita ka? \hfill\break
 }How many New Year\textquotesingle s cards did you send last year? }

\par{4. ${\overset{\textnormal{たまご}}{\text{卵}}}$ は ${\overset{\textnormal{いちにち}}{\text{一日}}}$ (に) ${\overset{\textnormal{なんこ}}{\text{何個}}}$ まで ${\overset{\textnormal{た}}{\text{食}}}$ べてよいのか ${\overset{\textnormal{し}}{\text{知}}}$ っていますか。 \hfill\break
 \emph{Tamago wa ichinichi (ni) nanko made tabete yoi no ka shitte imasu ka? \hfill\break
 }Do you know how many eggs are okay to eat a day? }

\par{5. ${\overset{\textnormal{ぎんこうこうざばんごう}}{\text{銀行口座番号}}}$ は ${\overset{\textnormal{なんけた}}{\text{何桁}}}$ でしょうか。 \hfill\break
 \emph{Ginkō kōza bangō wa nanketa deshō ka? \hfill\break
 }How many digits are in a bank account number? }

\par{6. ラーメンの ${\overset{\textnormal{か}}{\text{替}}}$ え ${\overset{\textnormal{だま}}{\text{玉}}}$ は ${\overset{\textnormal{さいこう}}{\text{最高}}}$ で ${\overset{\textnormal{なんたまちゅうもん}}{\text{何玉注文}}}$ したことがありますか。 \hfill\break
 \emph{Rāmen no kaedama wa saikō de nantama chūmon shita koto ga arimasu ka? \hfill\break
 }How many second servings of ramen have you ordered at the most? }

\par{7. ${\overset{\textnormal{なんだい}}{\text{何台}}}$ かの ${\overset{\textnormal{くるま}}{\text{車}}}$ が ${\overset{\textnormal{こうさてん}}{\text{交差点}}}$ で ${\overset{\textnormal{と}}{\text{止}}}$ まっている。 \hfill\break
 \emph{Nandaika no kuruma ga kōsaten de tomatte iru. \hfill\break
 }Several cars are stopped at the intersection. }

\par{8. ピザハットの ${\overset{\textnormal{エール}}{\text{L}}}$ サイズのピザ ${\overset{\textnormal{いち}}{\text{1}}}$ ${\overset{\textnormal{まい}}{\text{枚}}}$ は ${\overset{\textnormal{なんき}}{\text{何切}}}$ れでしょうか。 \hfill\break
 \emph{Pizahatto no ēru saizu no piza ichimai wa nankire deshō ka? \hfill\break
 }How many slices is a single large-sized pizza from Pizza Hut? \hfill\break
 \hfill\break
9. ヤギは ${\overset{\textnormal{なんとう}}{\text{何頭}}}$ 飼っているんですか。 \hfill\break
 \emph{Yagi wa nantō katte iru n desu ka? \hfill\break
 }How many goats are you raising? }

\par{\textbf{Spelling Note }: \emph{Yagi }is seldom spelled as 山羊. }

\par{10. ${\overset{\textnormal{いち}}{\text{1}}}$ ${\overset{\textnormal{にち}}{\text{日}}}$ に ${\overset{\textnormal{さいこう}}{\text{最高}}}$ で ${\overset{\textnormal{なんぽある}}{\text{何歩歩}}}$ きましたか。そして、 ${\overset{\textnormal{なんじかん}}{\text{何時間}}}$ くらい ${\overset{\textnormal{ある}}{\text{歩}}}$ きましたか。 \hfill\break
 \emph{Ichinichi ni saikō de nampo arukimashita ka? Soshite, nanjikan kurai arukimashita ka? \hfill\break
 }How many steps did you walk a day at most? Also, about how many hours did you walk? }

\begin{center}
\textbf{幾~ }
\end{center}

\par{ The native equivalent of \emph{nan }- 何 is \emph{iku }- 幾. In Modern Japanese, it is largely limited to the written language and song lyrics. Its use is also limited to only a handful of phrases. As such, you must learn each one on an individual basis. }

\par{11. ${\overset{\textnormal{ただ}}{\text{正}}}$ しい ${\overset{\textnormal{みち}}{\text{道}}}$ を ${\overset{\textnormal{あゆ}}{\text{歩}}}$ んでいる ${\overset{\textnormal{もの}}{\text{者}}}$ は ${\overset{\textnormal{いくにん}}{\text{幾人}}}$ かいる。 \hfill\break
 \emph{Tadashii michi wo ayunde iru mono wa ikuninka iru. \hfill\break
 }There are few who are walking down the right path. }

\par{12. ${\overset{\textnormal{いくせん}}{\text{幾千}}}$ の ${\overset{\textnormal{よる}}{\text{夜}}}$ を ${\overset{\textnormal{こ}}{\text{越}}}$ えて ${\overset{\textnormal{さが}}{\text{探}}}$ し ${\overset{\textnormal{つづ}}{\text{続}}}$ けた。 \hfill\break
 \emph{Ikusen no yoru wo koete sagashitsuzuketa. \hfill\break
 }I continued to search past thousands of nights. }

\par{13. ${\overset{\textnormal{かのじょ}}{\text{彼女}}}$ は ${\overset{\textnormal{ほしぞら}}{\text{星空}}}$ を ${\overset{\textnormal{みあ}}{\text{見上}}}$ げて ${\overset{\textnormal{いくばん}}{\text{幾晩}}}$ も ${\overset{\textnormal{いくばん}}{\text{幾晩}}}$ も ${\overset{\textnormal{す}}{\text{過}}}$ ごした。 \hfill\break
 \emph{Kanojo wa hoshizora wo miagete ikuban mo ikuban mo sugoshita. \hfill\break
 }She spent evening after evening looking up at the starry sky. }

\par{14. ${\overset{\textnormal{いくにち}}{\text{幾日}}}$ も ${\overset{\textnormal{たたか}}{\text{戦}}}$ い ${\overset{\textnormal{つづ}}{\text{続}}}$ けた。 \hfill\break
 \emph{Ikunichi mo tatakaitsuzuketa. \hfill\break
 }I continued to fight for days. }

\par{15. ${\overset{\textnormal{いくえ}}{\text{幾重}}}$ にも ${\overset{\textnormal{かさ}}{\text{重}}}$ なる ${\overset{\textnormal{くも}}{\text{雲}}}$ を ${\overset{\textnormal{なが}}{\text{眺}}}$ めていた。 \hfill\break
 \emph{Ikue ni mo kasanaru kumo wo nagamete ita. \hfill\break
 }I was gazing up at multiple-layered clouds. }
      
\section{数~ \& ~数}
 
\par{ The prefix \emph{sū }- 数~ attaches to all sorts of counters to indicate “several…” The number implied by this prefix is based largely on context and personal intuition. It generally refers to at least 2-10. }

\par{16. キャッシュ ${\overset{\textnormal{さくじょ}}{\text{削除}}}$ に ${\overset{\textnormal{すうふん}}{\text{数分}}}$ かかりました。 \hfill\break
 \emph{Kyasshu sakujo ni sūfun kakarimashita. \hfill\break
 }it took several minutes to delete the cache. }

\par{17. ${\overset{\textnormal{ディーブイディー}}{\text{DVD}}}$ を ${\overset{\textnormal{すうじゅうまいか}}{\text{数十枚買}}}$ いました。 \hfill\break
 \emph{Diibuidii wo sūmai kaimashita. \hfill\break
 }I bought several DVDs. }

\par{18. ${\overset{\textnormal{つき}}{\text{月}}}$ に ${\overset{\textnormal{すうひゃくこう}}{\text{数百個売}}}$ れている ${\overset{\textnormal{しょうひん}}{\text{商品}}}$ もあります。 \hfill\break
 \emph{Tsuki ni sūhyakko urete iru shōhin mo arimasu. \hfill\break
 }We also have products that sell several hundred a month. }

\par{19. ${\overset{\textnormal{なま}}{\text{生}}}$ ビールを ${\overset{\textnormal{すうはいの}}{\text{数杯飲}}}$ みました。 \hfill\break
 \emph{Namabiiru wo sūhai nomimashita. \hfill\break
 }I drank several glasses of draught beer. }

\par{20. ${\overset{\textnormal{すうまんにん}}{\text{数万人}}}$ の ${\overset{\textnormal{しみん}}{\text{市民}}}$ が ${\overset{\textnormal{どうろ}}{\text{道路}}}$ を ${\overset{\textnormal{う}}{\text{埋}}}$ め ${\overset{\textnormal{つ}}{\text{尽}}}$ くした。 \hfill\break
 \emph{Sūman\textquotesingle nin no shimin ga dōro wo umetsukushita. \hfill\break
 }Several tens of thousands of citizens filled up the road. }

\par{21. ${\overset{\textnormal{すうむね}}{\text{数棟}}}$ の ${\overset{\textnormal{じゅうたく}}{\text{住宅}}}$ が ${\overset{\textnormal{しゃめん}}{\text{斜面}}}$ を ${\overset{\textnormal{すべ}}{\text{滑}}}$ り ${\overset{\textnormal{お}}{\text{落}}}$ ちかけている。 \hfill\break
 \emph{Sūmune no jūtaku ga shamen wo suberiochikakete iru. \hfill\break
 }Several residences are slipping off the slope. }

\par{22. ${\overset{\textnormal{さわ}}{\text{沢}}}$ が ${\overset{\textnormal{あふ}}{\text{溢}}}$ れて ${\overset{\textnormal{ふくすう}}{\text{複数}}}$ の ${\overset{\textnormal{じゅうたく}}{\text{住宅}}}$ が ${\overset{\textnormal{ゆかした}}{\text{床下}}}$ まで ${\overset{\textnormal{しんすい}}{\text{浸水}}}$ している。 \hfill\break
 \emph{Sawa ga afurete fukusū no jūtaku ga yukashita made shinsui shite iru. \hfill\break
 }The marsh overflowed and several residences are now inundated up beneath the floor. }

\par{23. ${\overset{\textnormal{われわれ}}{\text{我々}}}$ は ${\overset{\textnormal{ふくすうにん}}{\text{複数人}}}$ で ${\overset{\textnormal{ひと}}{\text{1}}}$ つのシステムの ${\overset{\textnormal{かいはつ}}{\text{開発}}}$ を ${\overset{\textnormal{すす}}{\text{進}}}$ めています。 \hfill\break
 \emph{Wareware wa fukusūnin de hitotsu no shisutemu no kaihatsu wo susumete imasu. \hfill\break
 }We are furthering the development of one system with several people. }

\par{24. ${\overset{\textnormal{ふくすうめい}}{\text{複数名}}}$ にメールを ${\overset{\textnormal{そうしん}}{\text{送信}}}$ した ${\overset{\textnormal{さい}}{\text{際}}}$ 、 ${\overset{\textnormal{つうち}}{\text{通知}}}$ メールは ${\overset{\textnormal{ふくすうにんぶんとど}}{\text{複数人分届}}}$ きます。 \hfill\break
 \emph{Fukusūmei ni mēru wo sōshin shita sai, tsūchi mēru wa fukusūnin-bun todokimasu. \hfill\break
 }When you send an e-mail to several people, you will receive that amount of people\textquotesingle s worth of notification e-mails. }

\par{25. ここ ${\overset{\textnormal{すうじつ}}{\text{数日}}}$ とても ${\overset{\textnormal{あつ}}{\text{暑}}}$ いです。 \hfill\break
 \emph{Koko sūjitsu totemo atsui desu. }\hfill\break
These past few days have been really hot. }

\par{ \textbf{Reading Note }: Note that 数日 is read as “ \emph{sūjitsu }” rather than “ \emph{sūnichi }.” }

\begin{center}
\textbf{~数 }
\end{center}

\par{ When - \emph{sū }数 is attached after a counter, it expresses “number of…” Do not confuse this with the prefix \emph{sū }- 数~ from above. This suffix can essentially be used with any counter. }

\par{26. ${\overset{\textnormal{たてもの}}{\text{建物}}}$ の ${\overset{\textnormal{かいすう}}{\text{階数}}}$ を ${\overset{\textnormal{ひょうげん}}{\text{表現}}}$ する ${\overset{\textnormal{い}}{\text{言}}}$ い ${\overset{\textnormal{かた}}{\text{方}}}$ で、アメリカでは ${\overset{\textnormal{いっ}}{\text{1}}}$ ${\overset{\textnormal{かい}}{\text{階}}}$ を「first floor」、 ${\overset{\textnormal{に}}{\text{2}}}$ ${\overset{\textnormal{かい}}{\text{階}}}$ を「second floor」と ${\overset{\textnormal{い}}{\text{言}}}$ いますが、イギリスでは ${\overset{\textnormal{いっ}}{\text{1}}}$ ${\overset{\textnormal{かい}}{\text{階}}}$ を「ground floor」、 ${\overset{\textnormal{に}}{\text{2}}}$ ${\overset{\textnormal{かい}}{\text{階}}}$ を「first floor」と ${\overset{\textnormal{い}}{\text{言}}}$ います。 \hfill\break
 \emph{Tatemono no kaisū wo hyōgen suru iikata de, Amerika de wa ikkai wo “first floor,” nikai wo “second floor” to iimasu ga, Igirisu de wa ikkai wo “ground floor,” nikai wo “first floor” to iimasu. \hfill\break
 }For phrases that express number of floors in a building, in America "ikkai" is called the "first floor" and "nikai" is called the "second floor," but in England "ikkai" is called the "ground floor” and “nikai” is called the “first floor.” }

\par{27. ${\overset{\textnormal{いち}}{\text{1}}}$ ${\overset{\textnormal{にち}}{\text{日}}}$ で ${\overset{\textnormal{こうつうじこ}}{\text{交通事故}}}$ が ${\overset{\textnormal{お}}{\text{起}}}$ こる ${\overset{\textnormal{けんすう}}{\text{件数}}}$ は、 ${\overset{\textnormal{ぜんこく}}{\text{全国}}}$ で ${\overset{\textnormal{なんけん}}{\text{何件}}}$ ぐらいあるんですか。 \hfill\break
 \emph{Ichinichi de kōtsū jiko ga okoru kensū wa, zenkoku de nanken gurai aru n desu ka? \hfill\break
 }About how many traffic accident cases are there nationwide a day? }

\par{28. スタート ${\overset{\textnormal{ちてん}}{\text{地点}}}$ から ${\overset{\textnormal{もくてきち}}{\text{目的地}}}$ までの ${\overset{\textnormal{ほすう}}{\text{歩数}}}$ を ${\overset{\textnormal{かぞ}}{\text{数}}}$ えました。 \hfill\break
 \emph{Sutāto chiten kara mokutekichi made no hosū wo kazoemashita. \hfill\break
 }I counted the number of steps from my starting point to my destination. }

\par{29. マイナンバーは ${\overset{\textnormal{なんけた}}{\text{何桁}}}$ になるんですか。 \hfill\break
 \emph{Mainambā wa nanketa ni naru n desu ka? \hfill\break
 }How many digits will “my number” be? }

\par{\textbf{Culture Note }: \emph{Mainambā }マイナンバー, also known as \emph{kojin bangō }個人番号 (individual number), is a 12-digit ID number issued to all citizens and (foreign) residents of Japan for taxation purposes. }

\par{30. ${\overset{\textnormal{ちゅうかん}}{\text{中間}}}$ テストの ${\overset{\textnormal{てんすう}}{\text{点数}}}$ が ${\overset{\textnormal{わる}}{\text{悪}}}$ かった。 \hfill\break
 \emph{Chūkan tesuto no tensū ga warukatta. \hfill\break
 }My mid-term test score was bad. }

\par{31. ボールの ${\overset{\textnormal{のこ}}{\text{残}}}$ り ${\overset{\textnormal{こすう}}{\text{個数}}}$ をあまり ${\overset{\textnormal{き}}{\text{気}}}$ にしなくていいですよ。 \hfill\break
 \emph{Bōru no nokori kosū wo amari ki ni shinakute ii desu yo. \hfill\break
 }You don't need to worry so much about how many remaining balls you have. }

\par{32. ${\overset{\textnormal{しょうかき}}{\text{消火器}}}$ の ${\overset{\textnormal{たいようねんすう}}{\text{耐用年数}}}$ を ${\overset{\textnormal{かくにん}}{\text{確認}}}$ して ${\overset{\textnormal{くだ}}{\text{下}}}$ さい。 \hfill\break
 \emph{Shōkaki no taiyō nensū wo kakunin shite kudasai. \hfill\break
 }Please verify the life of the fire extinguishers. }

\par{33. ${\overset{\textnormal{にほん}}{\text{日本}}}$ での ${\overset{\textnormal{ぶすう}}{\text{部数}}}$ が ${\overset{\textnormal{ひゃく}}{\text{100}}}$ ${\overset{\textnormal{まんぶ}}{\text{万部}}}$ を ${\overset{\textnormal{とっぱ}}{\text{突破}}}$ した。 \hfill\break
 \emph{Nihon de no busū ga hyakumambu wo toppa shita. \hfill\break
 }The number of copies in Japan has broken through a million. \hfill\break
34. タバコの ${\overset{\textnormal{ほんすう}}{\text{本数}}}$ を ${\overset{\textnormal{じょじょ}}{\text{徐々}}}$ に ${\overset{\textnormal{へ}}{\text{減}}}$ らしていく ${\overset{\textnormal{ほうほう}}{\text{方法}}}$ で ${\overset{\textnormal{きんえん}}{\text{禁煙}}}$ に ${\overset{\textnormal{せいこう}}{\text{成功}}}$ した ${\overset{\textnormal{ひと}}{\text{人}}}$ はいる。 \hfill\break
 \emph{Tabako no honsū wo jojo ni herashite iku hōhō de kin\textquotesingle en ni seikō shita hito wa iru. \hfill\break
 }There are people who have successfully quit smoking by using the method of gradually decreased the number of cigarettes they have. }

\par{35. ${\overset{\textnormal{さいきん}}{\text{最近}}}$ 、 ${\overset{\textnormal{しんせつじゅうたく}}{\text{新設住宅}}}$ の ${\overset{\textnormal{こすう}}{\text{戸数}}}$ が ${\overset{\textnormal{ふ}}{\text{増}}}$ えている。 \hfill\break
 \emph{Saikin, shinsetsu jūtaku no kosū ga fuete iru. \hfill\break
 }Recently, the number of new residences has been increasing. }

\par{36. ${\overset{\textnormal{いちばんかくすう}}{\text{一番画数}}}$ の ${\overset{\textnormal{おお}}{\text{多}}}$ い ${\overset{\textnormal{かんじ}}{\text{漢字}}}$ は ${\overset{\textnormal{なに}}{\text{何}}}$ ですか。 \hfill\break
 \emph{Ichiban kakusū no ōi kanji wa nan desu ka? \hfill\break
 }What Kanji has the most number of strokes? }

\par{37. ${\overset{\textnormal{しへい}}{\text{紙幣}}}$ の ${\overset{\textnormal{まいすう}}{\text{枚数}}}$ を ${\overset{\textnormal{かくにん}}{\text{確認}}}$ してください。 \hfill\break
 \emph{Shihei no maisū wo kakunin shite kudasai. \hfill\break
 }Please verify the number of bills. }

\par{38. ダチョウの ${\overset{\textnormal{とうすう}}{\text{頭数}}}$ は ${\overset{\textnormal{まいとしへ}}{\text{毎年減}}}$ っています。 \hfill\break
 \emph{Dachō no tōsū wa maitoshi hette imasu. \hfill\break
 }The number of ostriches decreases every year. }

\par{Spelling Note: \emph{Dachō }may seldom be spelled as 駝鳥. }

\par{39. ${\overset{\textnormal{ぼく}}{\text{僕}}}$ を ${\overset{\textnormal{あたまかず}}{\text{頭数}}}$ に ${\overset{\textnormal{い}}{\text{入}}}$ れないで。 \hfill\break
 \emph{Boku wo atamakazu ni irenaide. \hfill\break
 }Don\textquotesingle t add me in the headcount. }

\par{\textbf{Sentence Note }: Though not related necessary to the suffix - \emph{sū }数, it is important to know that 頭数 has two different meanings and readings for each respectively as is demonstrated in Exs. 38 and 39. }

\par{ Interestingly enough, when paired with the counter - \emph{nin }人, ~数 undergoes a sound change and becomes either - \emph{zu }or - \emph{zū }. Either reading is fine in the phrases 人数 creates, as is demonstrated below. }

\par{40. なんとか ${\overset{\textnormal{にんずう}}{\text{人数}}}$ を ${\overset{\textnormal{そろ}}{\text{揃}}}$ えました。 \hfill\break
 \emph{Nantoka ninzū wo soroemashita. \hfill\break
 }We somehow managed to gather many people. }

\par{41. ${\overset{\textnormal{たにんずう}}{\text{多人数}}}$ で ${\overset{\textnormal{りよう}}{\text{利用}}}$ するには ${\overset{\textnormal{さいてき}}{\text{最適}}}$ ! \hfill\break
 \emph{Taninzū de riyō suru ni wa saiteki! \hfill\break
 }It\textquotesingle s most suitable for use with a large number of people! \hfill\break
 \hfill\break
42. ${\overset{\textnormal{おおにんずう}}{\text{大人数}}}$ でポケモンをゲットしに ${\overset{\textnormal{い}}{\text{行}}}$ きました。 \hfill\break
 \emph{Ōninzū de pokemon wo getto shi ni ikimashita. \hfill\break
 }I went to catch Pokemon with a lot of people. }

\par{ 43. ${\overset{\textnormal{こにんずう}}{\text{小人数}}}$ でレイドボスを ${\overset{\textnormal{たお}}{\text{倒}}}$ すコツを ${\overset{\textnormal{おぼ}}{\text{覚}}}$ えました。 \hfill\break
 \emph{Koninzū de reido bosu wo taosu kotsu wo oboemashita. \hfill\break
 }I learned the tricks to taking down a raid boss with a small amount of people. }
      
\section{半~ \& ~半}
 
\par{ The prefix \emph{han }- 半 indicates “half” of something and is limited to counters that measure some sort of increment, whether it be a period of time or quantity of something. It is important to note, however, that it does have one peculiar restriction. This restriction is on how to say “half a week.” You would think \emph{hanshū }半週 would be used. However, this is not the case for most speakers. Instead, phrases like \emph{mikka }三日 (three days) or \emph{yokka }四日 (four days) would be used instead. }

\par{44. ${\overset{\textnormal{はんぶん}}{\text{半分}}}$ に ${\overset{\textnormal{き}}{\text{切}}}$ ってください。 \hfill\break
 \emph{Hambun ni kitte kudasai. \hfill\break
 }Please cut it in half. }

\par{45. さて、 ${\overset{\textnormal{はん}}{\text{半}}}$ ${\overset{\textnormal{か}}{\text{ヶ}}}$ ${\overset{\textnormal{げつ}}{\text{月}}}$ ぶりの ${\overset{\textnormal{こうしん}}{\text{更新}}}$ です! \hfill\break
 \emph{Sate, hankagetsu-buri no kōshin desu! \hfill\break
 }Alright now, this will be a half-month belated update! }

\par{46. ドーナッツを ${\overset{\textnormal{はん}}{\text{半}}}$ ダース ${\overset{\textnormal{か}}{\text{買}}}$ いました。 \hfill\break
 \emph{Dōnattsu wo handāsu kaimashita. \hfill\break
 }I bought half a dozen of donuts. }

\par{47. ${\overset{\textnormal{しょうにんずう}}{\text{少人数}}}$ の ${\overset{\textnormal{しょくば}}{\text{職場}}}$ に ${\overset{\textnormal{はい}}{\text{入}}}$ って ${\overset{\textnormal{はんとしみまん}}{\text{半年未満}}}$ の ${\overset{\textnormal{もの}}{\text{者}}}$ です。 \hfill\break
 \emph{Shōninzū no shokuba ni haitte hantoshi miman no mono desu. \hfill\break
 }I am someone who entered a small work-place and have been there for under half a year. }

\par{48. ${\overset{\textnormal{ことし}}{\text{今年}}}$ ${\overset{\textnormal{いち}}{\text{1}}}$ ${\overset{\textnormal{がつまつ}}{\text{月末}}}$ までにおよそ ${\overset{\textnormal{はんすう}}{\text{半数}}}$ の ${\overset{\textnormal{よんひゃくごじゅうよん}}{\text{454}}}$ ${\overset{\textnormal{しせつ}}{\text{施設}}}$ から ${\overset{\textnormal{かいとう}}{\text{回答}}}$ を ${\overset{\textnormal{え}}{\text{得}}}$ ました。 \hfill\break
 \emph{Kotoshi ichigatsu-matsu made ni oyoso hansū no yonhyakugojūyon shisetsu kara kaitō wo emashita. \hfill\break
 }Before the end of January of this year, we had received responses from approximately half of the 454 facilities. }

\begin{center}
\textbf{~半 }
\end{center}

\par{ The suffix - \emph{han }半 is added to time phrases to indicate “and a half.” }

\par{49. ${\overset{\textnormal{わたし}}{\text{私}}}$ は ${\overset{\textnormal{きょう}}{\text{今日}}}$ から ${\overset{\textnormal{いっしゅうかんはんだんじき}}{\text{一週間半断食}}}$ を ${\overset{\textnormal{おこな}}{\text{行}}}$ います。 \hfill\break
 \emph{Watashi wa kyō kara isshūkan-han danjiki wo okonaimasu. \hfill\break
 }I will start a week and a half long fast today. }

\par{ 50. ${\overset{\textnormal{わたし}}{\text{私}}}$ は ${\overset{\textnormal{てんきん}}{\text{転勤}}}$ でバンクーバー ${\overset{\textnormal{し}}{\text{市}}}$ で ${\overset{\textnormal{いちねん}}{\text{一年}}}$ ( ${\overset{\textnormal{かん}}{\text{間}}}$ ) ${\overset{\textnormal{はんす}}{\text{半過}}}$ ごしました。 \hfill\break
 \emph{Watashi wa tenkin de Bankūbā-shi de ichinen(kan)-han sugoshimashita. \hfill\break
 }I spent a year and a half in Vancouver due to a job transfer. }
      
\section{~余 \& ~余り}
 
\par{\emph{ Yo }- 余 goes in between a number and counter to mean “more than.” Technically, it is a suffix which attaches to numbers which is then followed by a counter. This phrase is used largely in the written language. In the spoken language, - \emph{amari }余り is used instead, which is a suffix that follows counter phrases. }

\par{51. ${\overset{\textnormal{そうぎょう}}{\text{創業}}}$ ${\overset{\textnormal{ひゃく}}{\text{100}}}$ ${\overset{\textnormal{よねん}}{\text{余年}}}$ ! \hfill\break
 \emph{Sōgyō hyaku-yo-nen! }\hfill\break
Over 100 years since [its\slash our] establishing! }

\par{52. この ${\overset{\textnormal{しちじゅう}}{\text{70}}}$ ${\overset{\textnormal{よねんかん}}{\text{余年間}}}$ 、 ${\overset{\textnormal{せんそう}}{\text{戦争}}}$ はありませんでした。 \hfill\break
 \emph{Kono shichijū-yo-nenkan, sensō wa arimasen deshita. \hfill\break
 }There has not been any war over these past seventy-odd years. }

\par{53. ${\overset{\textnormal{へいしごまんよにん}}{\text{兵士五万余人}}}$ がいた。 \hfill\break
 \emph{Heishi goman-yo-nin ga ita. \hfill\break
 }There were over fifty thousand soldiers. }

\par{54. ${\overset{\textnormal{よんじゅう}}{\text{40}}}$ ${\overset{\textnormal{まんにんあま}}{\text{万人余}}}$ りが ${\overset{\textnormal{かんせん}}{\text{感染}}}$ している。 \hfill\break
 \emph{Yoman\textquotesingle nin-amari ga kansen shite iru. \hfill\break
 }Over four hundred thousand people are infected. }

\par{ 55. 彼は6ヶ月余りの戦闘の末に戦死した。 \hfill\break
 \emph{Kare wa rokkagetsu-amari no sentō no sue ni senshi shita. \hfill\break
 }He died in battle at the end of over six months of combat. }
    