    
\chapter{Counters IX}

\begin{center}
\begin{Large}
第195課: Counters IX: Counters with ヶ・ヵ 
\end{Large}
\end{center}
 
\par{ 个 is the most commonly used measure word in Mandarin Chinese. This character is a simplified version of 箇, derived by taking one half of the 竹 radical. In Japanese,  个 is further reduced to ケ, which is then normally shrunk to ヶ. Appearance-wise, this glyph looks like a small Katakana ケ. However, the Katakana ケ actually derives from 介. The glyph ヶ is also treated as a variant of 個. }

\par{ When used in counter phrases, ヶ is pronounced as “ka.” In government documents and news broadcasts, it is alternatively written as か. In other forms of publications such as the news or official documents at the workplace, it may also alternatively be written as カ or even ヵ. }

\par{ The purpose of ヶ is to count things, but it can\textquotesingle t do this on its own. It is found as part of a handful of counters, all of which are frequently used. These counters will be the focus of this lesson. }

\par{・ヶ月(間)\slash カ月(間)\slash か月(間)\slash ヵ月(間) \hfill\break
・ヶ年\slash カ年\slash か年\slash ヵ年 \hfill\break
・ヶ条\slash カ条\slash か条\slash ヵ条 \hfill\break
・ヶ所\slash カ所\slash か所\slash ヵ所 \hfill\break
・ヶ国\slash カ国\slash か国\slash ヵ国 \hfill\break
・ヶ国語\slash カ国語\slash か国語\slash ヵ国語 \hfill\break
・言語 }

\par{ As you can see, we have already familiarized ourselves with the first two counters in Lesson 49. }

\par{ Lastly, after we have learned about these counters, we will conclude the lesson by learning about an exclusive purpose ヶ, alternatively written as a full-sized ケ, has in place names. }

\par{\textbf{Spelling Notes }: \hfill\break
1. The following outlets use か月: 読売新聞, NHK, 日本テレビ, and テレビ東京. \hfill\break
2. The following outlets use カ月: 朝日新聞, 毎日新聞, 日本経済新聞, 産経新聞, テレビ朝日, and フジテレビ. \hfill\break
3. The non-simplified spelling 箇 may also be seen in the spellings of the counters introduced in this lesson. }

\par{ \textbf{Reading Notes }: \hfill\break
1. Because ヶ is also treated as an abbreviated form of 個, it can also stand for the counter 個. This, though, is only done based on personal preference. \hfill\break
2. At times in literature, カタカナ is sometimes used in 送り仮名. When this is the case, ケmay be shrunk to ヶ with no change in meaning or pronunciation. Meaning, it would be read as “ke” and would need to be distinguished from the ヶ discussed in this lesson. This use of a literal small ヶ can also be seen used in dialectical spelling. }
      
\section{Counters with ヶ・カ・か・ヵ}
 
\begin{center}
 \textbf{ヶ月(間)\slash カ月(間)\slash か月(間)\slash ヵ月(間) }
\end{center}
 
\par{ This counter is used to count a period of months. }
 
\begin{ltabulary}{|P|P|P|P|P|P|}
\hline 
 
  1 
 &   いっかげつ(かん) 
 &   2 
 &   にかげつ(かん) 
 &   3 
 &   さんかげつ(かん) 
 \\ \cline{1-6} 
 
  4 
 &   よんかげつ(かん) 
 &   5 
 &   ごかげつ(かん) 
 &   6 
 &   ろっかげつ(かん) 
 \\ \cline{1-6} 
 
  7 
 &   ななかげつ(かん) 
 &   8 
 &   はちかげつ(かん) \hfill\break
はっかげつ(かん) 
 &   9 
 &   きゅうかげつ(かん) 
 \\ \cline{1-6} 
 
  10 
 &   じゅっかげつ(かん) \hfill\break
じっかげつ(かん) 
 &   100 
 &   ひゃっかげつ(かん) 
 &   ? 
 &   なんかげつ(かん) 
\\ \cline{1-6}

\end{ltabulary}
 
\par{\hfill\break
1. ${\overset{\textnormal{はち}}{\text{8}}}$ か ${\overset{\textnormal{げつ}}{\text{月}}}$ で ${\overset{\textnormal{やく}}{\text{約}}}$ ${\overset{\textnormal{ご}}{\text{5}}}$ ${\overset{\textnormal{ばい}}{\text{倍}}}$ になった ${\overset{\textnormal{けいさん}}{\text{計算}}}$ だ。 \hfill\break
This calculation shows it\textquotesingle s become approximately five times greater in eight months. }
 
\par{2. ${\overset{\textnormal{いじょうきしょう}}{\text{異常気象}}}$ が ${\overset{\textnormal{すう}}{\text{数}}}$ ${\overset{\textnormal{か}}{\text{カ}}}$ ${\overset{\textnormal{げつ}}{\text{月}}}$ でグリーンアノールを ${\overset{\textnormal{きゅうそく}}{\text{急速}}}$ に ${\overset{\textnormal{しんか}}{\text{進化}}}$ させた。 \hfill\break
Extreme weather has made the green anole to rapidly evolve in a few months. }
 
\par{3. レギュラーガソリン、 ${\overset{\textnormal{さん}}{\text{3}}}$ か ${\overset{\textnormal{げつ}}{\text{月}}}$ ぶりに ${\overset{\textnormal{ねさ}}{\text{値下}}}$ がり \hfill\break
First Price Decline for Regular Gasoline in Three Months }
 
\par{4. 「 ${\overset{\textnormal{ほうてい}}{\text{法定}}}$ ${\overset{\textnormal{じゅうに}}{\text{12}}}$ ${\overset{\textnormal{か}}{\text{ヶ}}}$ ${\overset{\textnormal{げつてんけん}}{\text{月点検}}}$ 」って ${\overset{\textnormal{かなら}}{\text{必}}}$ ず ${\overset{\textnormal{じゅ}}{\text{受}}}$ けなければいけないのでしょうか。 \hfill\break
Must you always take the “Statutory 12-Month Inspection”? }
 
\par{5. ${\overset{\textnormal{こんかい}}{\text{今回}}}$ はこの ${\overset{\textnormal{ご}}{\text{5}}}$ ${\overset{\textnormal{か}}{\text{ヶ}}}$ ${\overset{\textnormal{げつ}}{\text{月}}}$ のダイエット ${\overset{\textnormal{せいかつ}}{\text{生活}}}$ を ${\overset{\textnormal{ふ}}{\text{振}}}$ り ${\overset{\textnormal{かえ}}{\text{返}}}$ ってみたいと ${\overset{\textnormal{おも}}{\text{思}}}$ います。 \hfill\break
This time, I would like to look back at my dieting life for these past five months. }
 
\begin{center}
\textbf{ヶ年\slash カ年\slash か年\slash ヵ年 }
\end{center}
 
\par{ This counter translates as “over…years” and is frequently employed in legislation and business settings. }
 
\begin{ltabulary}{|P|P|P|P|P|P|P|P|}
\hline 
 
  1 
 &   いっかねん △ 
 &   2 
 &   にかねん 
 &   3 
 &   さんかねん 
 &   4 
 &   よんかねん 
 \\ \cline{1-8} 
 
  5 
 &   ごかねん 
 &   6 
 &   ろっかねん 
 &   7 
 &   ななかねん 
 &   8 
 &    \textbf{はちかねん \hfill\break
 }はっかねん 
 \\ \cline{1-8} 
 
  9 
 &   きゅうかねん 
 &   10 
 &    \textbf{じゅっかねん \hfill\break
 }じっかねん 
 &   100 
 &   ひゃっかねん 
 &   ? 
 &   なんかねん 
\\ \cline{1-8}

\end{ltabulary}
 
\par{\hfill\break
6. この ${\overset{\textnormal{さん}}{\text{3}}}$ か ${\overset{\textnormal{ねんじっしけいかくしょ}}{\text{年実施計画書}}}$ に ${\overset{\textnormal{もと}}{\text{基}}}$ づいて ${\overset{\textnormal{よさん}}{\text{予算}}}$ を ${\overset{\textnormal{へんせい}}{\text{編成}}}$ します。 \hfill\break
(We) will compile the budget based on this three-year implementation plan. }
 
\par{7. ${\overset{\textnormal{どうろせいび}}{\text{道路整備}}}$ の ${\overset{\textnormal{ご}}{\text{5}}}$ か ${\overset{\textnormal{ねんけいかく}}{\text{年計画}}}$ が ${\overset{\textnormal{けってい}}{\text{決定}}}$ された。 \hfill\break
The five-year road maintenance plan was decided upon. }
 
\par{8. ${\overset{\textnormal{くに}}{\text{国}}}$ の ${\overset{\textnormal{ちょっかつこうじ}}{\text{直轄工事}}}$ では、 ${\overset{\textnormal{に}}{\text{2}}}$ ${\overset{\textnormal{か}}{\text{カ}}}$ ${\overset{\textnormal{ねんこくさい}}{\text{年国債}}}$ の ${\overset{\textnormal{かつよう}}{\text{活用}}}$ によって ${\overset{\textnormal{しんきこうじけいやくけんすう}}{\text{新規工事契約件数}}}$ が ${\overset{\textnormal{ふ}}{\text{増}}}$ えた。 \hfill\break
In government-controlled construction, the number of new construction contracts increased due to the application of two-year government bonds. }
 
\par{9. この ${\overset{\textnormal{ろっ}}{\text{6}}}$ ${\overset{\textnormal{か}}{\text{ヵ}}}$ ${\overset{\textnormal{ねんけいかく}}{\text{年計画}}}$ の ${\overset{\textnormal{ひげんじつせい}}{\text{非現実性}}}$ について ${\overset{\textnormal{せつめい}}{\text{説明}}}$ します。 \hfill\break
I will explain the impracticality of this six-year plan. }
 
\par{10. ${\overset{\textnormal{せいふ}}{\text{政府}}}$ は、がん ${\overset{\textnormal{けんきゅう}}{\text{研究}}}$ ${\overset{\textnormal{じっ}}{\text{10}}}$ か ${\overset{\textnormal{ねんせんりゃく}}{\text{年戦略}}}$ を ${\overset{\textnormal{さくてい}}{\text{策定}}}$ した。 \hfill\break
The government settled on the ten-year strategy for cancer research. }
 
\begin{center}
\textbf{ヶ条\slash カ条\slash か条\slash ヵ条 }
\end{center}
 
\par{ This counter counts the number of articles\slash clauses\slash sections in an official document. It may also be used to mean “points” as in “10 points to a happy marriage.” The noun for article\slash clause is 箇条, which is written with 箇 instead of with any of the abbreviated variants of the character. }
 
\begin{ltabulary}{|P|P|P|P|P|P|}
\hline 
 
  1 
 &   いっかじょう 
 &   2 
 &   にかじょう 
 &   3 
 &   さんかじょう 
 \\ \cline{1-6} 
 
  4 
 &   よんかじょう 
 &   5 
 &   ごかじょう 
 &   6 
 &   ろっかじょう 
 \\ \cline{1-6} 
 
  7 
 &   ななかじょう 
 &   8 
 &    \textbf{はちかじょう }\hfill\break
はっかじょう 
 &   9 
 &   きゅうかじょう 
 \\ \cline{1-6} 
 
  10 
 &    \textbf{じゅっかじょう }\hfill\break
じっかじょう 
 &   100 
 &   ひゃっかじょう 
 &   ? 
 &   なんかじょう 
 \\ \cline{1-6} 
 
\end{ltabulary}
   
\par{11. ${\overset{\textnormal{しあわ}}{\text{幸}}}$ せになる ${\overset{\textnormal{ひゃっ}}{\text{100}}}$ か ${\overset{\textnormal{じょう}}{\text{条}}}$ \hfill\break
The 100 Points to Becoming Happy }
 
\par{12. がんを防ぐための12か条 \hfill\break
12 Point Guide to Preventing Cancer }
 
\par{13. ${\overset{\textnormal{じょうほう}}{\text{情報}}}$ セキュリティ ${\overset{\textnormal{ご}}{\text{5}}}$ か ${\overset{\textnormal{じょう}}{\text{条}}}$ \hfill\break
Five-Point Information Security }
 
\par{14. ${\overset{\textnormal{ほうかぼうし}}{\text{放火防止}}}$ ${\overset{\textnormal{ご}}{\text{5}}}$ か ${\overset{\textnormal{じょう}}{\text{条}}}$ を ${\overset{\textnormal{じっせん}}{\text{実践}}}$ しましょう。 \hfill\break
Let\textquotesingle s practice the five points to preventing arson. }
 
\par{15. ${\overset{\textnormal{ぼうりょくだんたいおう}}{\text{暴力団対応}}}$ ${\overset{\textnormal{じゅうに}}{\text{12}}}$ か ${\overset{\textnormal{じょう}}{\text{条}}}$ \hfill\break
Twelve Points to Handling Gangs }
 
\par{16. マイナンバー ${\overset{\textnormal{ほう}}{\text{法}}}$ は ${\overset{\textnormal{ごじゅうなな}}{\text{57}}}$ ${\overset{\textnormal{か}}{\text{ヶ}}}$ ${\overset{\textnormal{じょう}}{\text{条}}}$ あります。 \hfill\break
The My Number Law has 57 clauses. }
 
\par{17. ${\overset{\textnormal{けんぽう}}{\text{憲法}}}$ は ${\overset{\textnormal{ぜんぶ}}{\text{全部}}}$ で ${\overset{\textnormal{ひゃくろくじゅうきゅう}}{\text{169}}}$ ${\overset{\textnormal{か}}{\text{ヶ}}}$ ${\overset{\textnormal{じょう}}{\text{条}}}$ あります。 \hfill\break
The constitution has 169 clauses in total. }
 
\par{18. ${\overset{\textnormal{けんり}}{\text{権利}}}$ や ${\overset{\textnormal{じゆう}}{\text{自由}}}$ を ${\overset{\textnormal{さだ}}{\text{定}}}$ めた ${\overset{\textnormal{じょうぶん}}{\text{条文}}}$ は ${\overset{\textnormal{じゅういっ}}{\text{11}}}$ ${\overset{\textnormal{か}}{\text{ヶ}}}$ ${\overset{\textnormal{じょう}}{\text{条}}}$ あります。 \hfill\break
The text has 11 clauses that establish rights and freedoms. }
 
\par{19. ${\overset{\textnormal{げんこうけんぽう}}{\text{現行憲法}}}$ は ${\overset{\textnormal{ぜんぶ}}{\text{全部}}}$ で ${\overset{\textnormal{ひゃくさん}}{\text{103}}}$ ${\overset{\textnormal{か}}{\text{ヶ}}}$ ${\overset{\textnormal{じょう}}{\text{条}}}$ あって、そのうち ${\overset{\textnormal{ほそく}}{\text{補足}}}$ が ${\overset{\textnormal{よん}}{\text{4}}}$ ${\overset{\textnormal{か}}{\text{ヶ}}}$ ${\overset{\textnormal{じょう}}{\text{条}}}$ あります。 \hfill\break
The current constitution has 103 clauses in total, and of those there are four complementary notes. }
  \textbf{ヶ所\slash カ所\slash か所\slash ヵ所 }\hfill\break
  This counter is used to count places\slash locations. As a noun, 箇所 means “passage\slash part.”   
\begin{ltabulary}{|P|P|P|P|P|P|}
\hline 
 
  1 
 &   いっかしょ 
 &   2 
 &   にかしょ 
 &   3 
 &   さんかしょ 
 \\ \cline{1-6} 
 
  4 
 &   よんかしょ 
 &   5 
 &   ごかしょ 
 &   6 
 &   ろっかしょ 
 \\ \cline{1-6} 
 
  7 
 &   ななかしょ 
 &   8 
 &    \textbf{はちかしょ }\hfill\break
はっかしょ 
 &   9 
 &   きゅうかしょ 
 \\ \cline{1-6} 
 
  10 
 &    \textbf{じゅっかしょ }\hfill\break
じっかしょ 
 &   100 
 &   ひゃっかしょ 
 &   ? 
 &   なんかしょ 
 \\ \cline{1-6} 
 
\end{ltabulary}
 
\par{\hfill\break
20. ${\overset{\textnormal{さん}}{\text{3}}}$ ${\overset{\textnormal{ねんまえ}}{\text{年前}}}$ の ${\overset{\textnormal{はち}}{\text{8}}}$ ${\overset{\textnormal{がつ}}{\text{月}}}$ ${\overset{\textnormal{はつ}}{\text{20}}}$ ${\overset{\textnormal{か}}{\text{日}}}$ に ${\overset{\textnormal{ひろしましない}}{\text{広島市内}}}$ の ${\overset{\textnormal{ひゃくろくじゅうろっ}}{\text{166}}}$ か ${\overset{\textnormal{しょ}}{\text{所}}}$ で ${\overset{\textnormal{どせきりゅう}}{\text{土石流}}}$ や ${\overset{\textnormal{がけくず}}{\text{崖崩}}}$ れが ${\overset{\textnormal{はっせい}}{\text{発生}}}$ し、 ${\overset{\textnormal{ななじゅうなな}}{\text{77}}}$ ${\overset{\textnormal{にん}}{\text{人}}}$ が ${\overset{\textnormal{しぼう}}{\text{死亡}}}$ しました。 \hfill\break
77 people died in the debris flows and the landslides that occurred in 166 places within Hiroshima City three years ago on August 20 th . }
 
\par{21. ${\overset{\textnormal{に}}{\text{2}}}$ 、 ${\overset{\textnormal{さん}}{\text{3}}}$ ${\overset{\textnormal{か}}{\text{ヶ}}}$ ${\overset{\textnormal{しょ}}{\text{所}}}$ ${\overset{\textnormal{あやま}}{\text{誤}}}$ りがあります。 \hfill\break
There are two or three areas with mistakes. }
 
\par{22. ${\overset{\textnormal{みなお}}{\text{見直}}}$ された ${\overset{\textnormal{ひなんじょ}}{\text{避難所}}}$ は ${\overset{\textnormal{いっ}}{\text{1}}}$ か ${\overset{\textnormal{しょ}}{\text{所}}}$ もありません。 \hfill\break
There is not a single shelter that has been re-examined. }
 
\par{23. ${\overset{\textnormal{さん}}{\text{3}}}$ か ${\overset{\textnormal{しょいじょう}}{\text{所以上}}}$ の ${\overset{\textnormal{いりょうきかん}}{\text{医療機関}}}$ に ${\overset{\textnormal{つういん}}{\text{通院}}}$ している ${\overset{\textnormal{こうれいしゃ}}{\text{高齢者}}}$ の ${\overset{\textnormal{きゅう}}{\text{9}}}$ ${\overset{\textnormal{わり}}{\text{割}}}$ が、 ${\overset{\textnormal{まんせいしっかん}}{\text{慢性疾患}}}$ の ${\overset{\textnormal{くすり}}{\text{薬}}}$ を ${\overset{\textnormal{ご}}{\text{5}}}$ ${\overset{\textnormal{しゅるいいじょうしょほう}}{\text{種類以上処方}}}$ されている。 \hfill\break
Nine-tenths of elderly patients being regularly treated at over three medial institutions are being prescribed over five kinds of chronic illness medication. }
 
\par{24. ポケストップを ${\overset{\textnormal{じゅっ}}{\text{10}}}$ か ${\overset{\textnormal{しょれんぞく}}{\text{所連続}}}$ で ${\overset{\textnormal{まわ}}{\text{回}}}$ りました。 \hfill\break
I visited ten Pokestops in a row. }
  \textbf{ヶ国\slash カ国\slash か国\slash ヵ国 } \hfill\break
 This counter is used to count countries\slash nations.   
\begin{ltabulary}{|P|P|P|P|P|P|}
\hline 
 
  1 
 &   いっかこく 
 &   2 
 &   にかこく 
 &   3 
 &   さんかこく 
 \\ \cline{1-6} 
 
  4 
 &   よんかこく 
 &   5 
 &   ごかこく 
 &   6 
 &   ろっかこく 
 \\ \cline{1-6} 
 
  7 
 &   ななかこく 
 &   8 
 &    \textbf{はちかこく }\hfill\break
はっかこく 
 &   9 
 &   きゅうかこく 
 \\ \cline{1-6} 
 
  10 
 &    \textbf{じゅっかこく }\hfill\break
じっかこく 
 &   100 
 &   ひゃっかこく 
 &   ? 
 &   なんかこく 
\\ \cline{1-6}

\end{ltabulary}
 
\par{\hfill\break
25. ${\overset{\textnormal{げんざい}}{\text{現在}}}$ 、 ${\overset{\textnormal{オーイーシーディー}}{\text{OECD}}}$ の ${\overset{\textnormal{かめいこく}}{\text{加盟国}}}$ は ${\overset{\textnormal{さんじゅうご}}{\text{35}}}$ か ${\overset{\textnormal{こく}}{\text{国}}}$ となっています。 \hfill\break
Currently, there are thirty-five member states in the OECD. }
 
\par{26. ${\overset{\textnormal{せかい}}{\text{世界}}}$ の ${\overset{\textnormal{くに}}{\text{国}}}$ の ${\overset{\textnormal{かず}}{\text{数}}}$ は ${\overset{\textnormal{なん}}{\text{何}}}$ ${\overset{\textnormal{か}}{\text{カ}}}$ ${\overset{\textnormal{こく}}{\text{国}}}$ ? \hfill\break
What is the number of countries in the world? }
 
\par{27. パレスチナに ${\overset{\textnormal{かん}}{\text{関}}}$ しては、 ${\overset{\textnormal{にっぽんせいふ}}{\text{日本政府}}}$ は ${\overset{\textnormal{こっかしょうにん}}{\text{国家承認}}}$ していないが、すでに ${\overset{\textnormal{やく}}{\text{約}}}$ ${\overset{\textnormal{ひゃくさんじゅっ}}{\text{130}}}$ か ${\overset{\textnormal{こく}}{\text{国}}}$ がパレスティナを ${\overset{\textnormal{しょうにん}}{\text{承認}}}$ している。 \hfill\break
Regarding Palestine, although the Japanese government does not recognize it as a state, there are already approximately 130 nations that have recognized Palestine. }
 
\par{28. アジア ${\overset{\textnormal{さん}}{\text{3}}}$ ${\overset{\textnormal{か}}{\text{カ}}}$ ${\overset{\textnormal{こく}}{\text{国}}}$ に ${\overset{\textnormal{またが}}{\text{跨}}}$ る ${\overset{\textnormal{こうずい}}{\text{洪水}}}$ で ${\overset{\textnormal{やく}}{\text{約}}}$ ${\overset{\textnormal{さんびゃく}}{\text{300}}}$ ${\overset{\textnormal{にん}}{\text{人}}}$ が ${\overset{\textnormal{しぼう}}{\text{死亡}}}$ した。 \hfill\break
Approximately 300 people have died in the flooding that has spanned three Asian countries. }
 
\par{29. ${\overset{\textnormal{せかい}}{\text{世界}}}$ ${\overset{\textnormal{じゅっ}}{\text{10}}}$ ${\overset{\textnormal{か}}{\text{カ}}}$ ${\overset{\textnormal{こくいじょう}}{\text{国以上}}}$ で ${\overset{\textnormal{あいよう}}{\text{愛用}}}$ されている。 \hfill\break
It is habitually used in over ten countries in the world. }
 
\par{ In set phrases, especially those related to history and international politics, nations may be counted with 国. }
 
\par{30. オーストリアとカナダは ${\overset{\textnormal{にこくかんかんけい}}{\text{二国間関係}}}$ を ${\overset{\textnormal{むす}}{\text{結}}}$ んでいる。 \hfill\break
Australia and Canada have bilateral relations. }
 
\par{31. アメリカとの ${\overset{\textnormal{に}}{\text{二}}}$ ${\overset{\textnormal{こく}}{\text{国}}}$ ${\overset{\textnormal{かん}}{\text{間}}}$ ${\overset{\textnormal{じ}}{\text{自}}}$ ${\overset{\textnormal{ゆう}}{\text{由}}}$ ${\overset{\textnormal{ぼう}}{\text{貿}}}$ ${\overset{\textnormal{えき}}{\text{易}}}$ ${\overset{\textnormal{きょう}}{\text{協}}}$ ${\overset{\textnormal{てい}}{\text{定}}}$ は ${\overset{\textnormal{じつ}}{\text{実}}}$ ${\overset{\textnormal{げん}}{\text{現}}}$ するのか。 \hfill\break
Will a bilateral free trade agreement realize with America? }
 
\par{32. バルト ${\overset{\textnormal{さんごく}}{\text{三国}}}$ は、バルト ${\overset{\textnormal{かい}}{\text{海}}}$ の ${\overset{\textnormal{とうがん}}{\text{東岸}}}$ 、フィンランドの ${\overset{\textnormal{みなみ}}{\text{南}}}$ に ${\overset{\textnormal{なんぼく}}{\text{南北}}}$ に ${\overset{\textnormal{なら}}{\text{並}}}$ ぶ ${\overset{\textnormal{みっ}}{\text{3}}}$ つの ${\overset{\textnormal{くに}}{\text{国}}}$ を ${\overset{\textnormal{さ}}{\text{指}}}$ す。 \hfill\break
The Baltic States refers to the three countries lined up north to south to the south of Finland on the east coast of the Baltic Sea. }
 
\par{\textbf{Grammar Note }: Notice how “the three countries” is expressed with 3つの国. This is another means of counting countries that is occasionally used. }
 
\begin{center}
\textbf{ヶ国語\slash カ国語\slash か国語\slash ヵ国語 }
\end{center}
 
\par{ This counter is used to count languages. However, as is implied by the spelling, this usually only refers to languages that are national languages. Because of this, counting languages with 言語, the word for “language,” is deemed most appropriate whenever national status is not a concern with the languages one is counting. }
 
\par{・かこくご }
 
\begin{ltabulary}{|P|P|P|P|P|P|}
\hline 
 
  1 
 &   いっかこくご 
 &   2 
 &   にかこくご 
 &   3 
 &   さんかこくご 
 \\ \cline{1-6} 
 
  4 
 &   よんかこくご 
 &   5 
 &   ごかこくご 
 &   6 
 &   ろっかこくご 
 \\ \cline{1-6} 
 
  7 
 &   ななかこくご 
 &   8 
 &    \textbf{はちかこくご \hfill\break
 }はっこくご 
 &   9 
 &   きゅうかこくご 
 \\ \cline{1-6} 
 
  10 
 &    \textbf{じゅっかこくご }\hfill\break
じっかこくご 
 &   100 
 &   ひゃっかこくご 
 &   ? 
 &   なんかこくご 
\\ \cline{1-6}

\end{ltabulary}

\par{・げんご }
 
\begin{ltabulary}{|P|P|P|P|P|P|}
\hline 
 
  1 
 &   いちげんご 
 &   2 
 &   にげんご 
 &   3 
 &   さんげんご 
 \\ \cline{1-6} 
 
  4 
 &   よんげんご 
 &   5 
 &   ごげんご 
 &   6 
 &   ろくげんご 
 \\ \cline{1-6} 
 
  7 
 &   ななげんご 
 &   8 
 &   はちげんご 
 &   9 
 &   きゅうげんご 
 \\ \cline{1-6} 
 
  10 
 &   じゅうげんご 
 &   100 
 &   ひゃくげんご 
 &   ? 
 &   なんげんご 
\\ \cline{1-6}

\end{ltabulary}
 
\par{\hfill\break
33. ${\overset{\textnormal{に}}{\text{二}}}$ ${\overset{\textnormal{か}}{\text{ヶ}}}$ ${\overset{\textnormal{こくご}}{\text{国語}}}$ を ${\overset{\textnormal{はな}}{\text{話}}}$ す ${\overset{\textnormal{ひと}}{\text{人}}}$ のことをバイリンガルと ${\overset{\textnormal{よ}}{\text{呼}}}$ びますが、 ${\overset{\textnormal{さん}}{\text{三}}}$ ${\overset{\textnormal{か}}{\text{ヶ}}}$ ${\overset{\textnormal{こくご}}{\text{国語}}}$ の ${\overset{\textnormal{わしゃ}}{\text{話者}}}$ をトライリンガルもしくはトリリンガルと ${\overset{\textnormal{よ}}{\text{呼}}}$ びます。 \hfill\break
We call people who speak two languages bilingual, but we call speakers of three languages trilingual. }
 
\par{34. ${\overset{\textnormal{に}}{\text{2}}}$ ${\overset{\textnormal{げんごへいき}}{\text{言語併記}}}$ の ${\overset{\textnormal{ひょうしき}}{\text{標識}}}$ や ${\overset{\textnormal{かんばん}}{\text{看板}}}$ などが ${\overset{\textnormal{せっち}}{\text{設置}}}$ されている。 \hfill\break
Bilingual signs and billboards are installed. }
 
\par{35. ヘルシンキ ${\overset{\textnormal{しない}}{\text{市内}}}$ には ${\overset{\textnormal{さんじゅっ}}{\text{30}}}$ か ${\overset{\textnormal{こくご}}{\text{国語}}}$ を ${\overset{\textnormal{おし}}{\text{教}}}$ える ${\overset{\textnormal{がっこう}}{\text{学校}}}$ もある。 \hfill\break
There is also a school in Helsinki that teaches thirty languages. }
 
\par{36. この ${\overset{\textnormal{ちいき}}{\text{地域}}}$ では ${\overset{\textnormal{じゅうご}}{\text{15}}}$ ${\overset{\textnormal{げんご}}{\text{言語}}}$ が ${\overset{\textnormal{しよう}}{\text{使用}}}$ されている。 \hfill\break
There are fifteen languages used in this region. }
 
\par{37. ${\overset{\textnormal{えいご}}{\text{英語}}}$ ・フランス ${\overset{\textnormal{ご}}{\text{語}}}$ ・オランダ ${\overset{\textnormal{ご}}{\text{語}}}$ の ${\overset{\textnormal{さん}}{\text{3}}}$ か ${\overset{\textnormal{こくご}}{\text{国語}}}$ を ${\overset{\textnormal{おし}}{\text{教}}}$ える ${\overset{\textnormal{がっこう}}{\text{学校}}}$ となりました。 \hfill\break
The school has become a school that teaches the three languages English, French, and Dutch. }
 
\par{38. ${\overset{\textnormal{にっぽんせいふ}}{\text{日本政府}}}$ からの ${\overset{\textnormal{じょうほうばんぐみ}}{\text{情報番組}}}$ ( ${\overset{\textnormal{ご}}{\text{5}}}$ か ${\overset{\textnormal{こくごほうそう}}{\text{国語放送}}}$ )をご ${\overset{\textnormal{しょうかい}}{\text{紹介}}}$ します。 \hfill\break
Here is an introduction to an information program (five-language broadcast) from the Japanese government. }
 
\par{39. ${\overset{\textnormal{かれ}}{\text{彼}}}$ はプログラミング ${\overset{\textnormal{げんご}}{\text{言語}}}$ も ${\overset{\textnormal{ふく}}{\text{含}}}$ めて、 ${\overset{\textnormal{じゅう}}{\text{10}}}$ ${\overset{\textnormal{げんごはな}}{\text{言語話}}}$ せるらしい。 \hfill\break
He apparently knows ten languages, including also programming languages. }
 
\par{40. ${\overset{\textnormal{エスオーブイ}}{\text{SOV}}}$ ${\overset{\textnormal{がた}}{\text{型}}}$ が ${\overset{\textnormal{いちばんおお}}{\text{一番多}}}$ く ${\overset{\textnormal{ごひゃくろくじゅうご}}{\text{565}}}$ ${\overset{\textnormal{げんご}}{\text{言語}}}$ 、 ${\overset{\textnormal{つ}}{\text{次}}}$ いで ${\overset{\textnormal{エスブイオー}}{\text{SVO}}}$ ${\overset{\textnormal{がた}}{\text{型}}}$ が ${\overset{\textnormal{よんひゃくはちじゅうはち}}{\text{488}}}$ ${\overset{\textnormal{げんご}}{\text{言語}}}$ である。 \hfill\break
The SOV pattern is the one with the most languages with 565, following it is the SVO pattern with 488. }
      
\section{Place Names: ヶ・ケ}
 
\par{ When ヶ, alternatively spelled as ケ, is used in places names, especially those found in East Japan, it stands in place of a classical usage of the particle が, which is to mark the possessive case. In this sense, it is equivalent to the particle の. Not all place names with this が are written with these glyphs, though. Sometimes it\textquotesingle s just written as が・ガ. }
 
\begin{ltabulary}{|P|P|P|P|P|P|}
\hline 
  
 Sekigahara &  関ケ原 
 &  Hatogaya &  鳩ヶ谷 
 &  Tsutsujigaoka & つつじヶ丘 \\ \cline{1-6} 
 
 Chigasaki &  茅ヶ崎 
 &  Shichigahama & 七ヶ浜 &  Aogashima &  青ヶ島 
 \\ \cline{1-6} 
 
 Yatsugatake &  八ヶ岳 
 &  Kamagaya &  鎌ケ谷 
 &  Kasumigaseki &  霞が関 
 \\ \cline{1-6} 
 
\end{ltabulary}
      