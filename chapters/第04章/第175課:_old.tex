    
\chapter{Old}

\begin{center}
\begin{Large}
第175課: Old 
\end{Large}
\end{center}
 
\par{ The word "old" has many usages. English speakers often disregard the multifaceted usages of English words when trying to express something in Japanese. To focus on the word "old" in particular, c onsider the following phrases which all have some phrase that translates as "old." }

\par{1. 年とった政治家は堪え性がない。 \hfill\break
Old politicians don't have endurance. }

\par{2. 年老いた犬を介護し、最後を看取る。 \hfill\break
To care for an old dog and watch its final moments. }

\par{3. 義父が \textbf{老 }人ホームに入りたがらない。 \hfill\break
My father-in-law does not want to enter the nursing home. }

\par{4. 老人を尊敬しないといけないことは肝に銘じている。 \hfill\break
Respecting the old\slash elderly is engraved in me. }

\par{5. カシの老木を伐採する。 \hfill\break
To cut down old oak trees. }

\par{6. 20歳になると、成人になる。 \hfill\break
You become an adult when you turn twenty years old. }

\par{7. 夜が更けた。 \hfill\break
The night grew old. }

\par{8. お子さんはおいくつですか。 \hfill\break
How old is your child? }

\par{9. 古い木造の家が全焼した。 \hfill\break
An\slash the old, wooden house completely burned. }

\par{10. 昔からの慣習を重んじる。 \hfill\break
To respect old traditions. }

\par{11. 前の彼氏にばったり出会った。 \hfill\break
I came across my old boyfriend. }
      
\section{Old (Age)}
 
\par{ One problem English speakers have with Japanese is Japanese's lack of a simple pattern such as "\# old" that can apply for anything. Depending on the object, the phrasing options are different. }

\par{ Have you ever thought about the age of inanimate objects? There is nothing wrong in saying something like Ex. 12a, but Ex. 12b is also OK. }

\par{12a. 地球の年齢は何歳ですか? \hfill\break
12b. 地球の年齢は何年ですか。 \hfill\break
How old is the age of the Earth? }

\par{ The answer, however, would always be 46億年. Using 歳・才 would be incorrect.  If your dog is 11, he\slash she is 11歳, not 11年. This is because dogs are alive and \emph{like }humans. }

\par{ Your apartment complex may be a 5年物, not 5歳. You may also live in a 5年前に建てた家 or 5年前に建った家 depending on context. With this in mind, consider the following examples. }

\par{13. 100年前に建てられたホテル \hfill\break
A hotel that was built 100 years ago }

\par{14. 三百年前に建った家だ。 \hfill\break
This is\slash it's a house built 300 years ago. }

\par{15. 築50年のマンションに住む。 \hfill\break
To live in an apartment of 50 years. }

\par{\textbf{Phrase Note }: 築#年 specifically shows how long ago a piece of architecture was constructed. }

\par{16. 10年もののワイン \hfill\break
Ten year wine }

\par{17. 年代もののブドウ酒 \hfill\break
Old\slash aged wine }

\par{\textbf{Word Note }: The phrase 古いワイン exists as well. }

\par{18. これは50年ものワインだよ。 \hfill\break
This is a fifty year old wine. }

\par{\textbf{Particle Note }: The もの is 物. の can be dropped in this context. It is more traditional\slash correct with it. }

\par{19. 50年前の写真 \hfill\break
A photograph form fifty years ago }

\par{20. 30年前(の)車両 \hfill\break
Train cars from 30 years ago }

\par{\textbf{Particle Note }: The particle の may also be omitted here as well. }

\par{ When you have something like 〇〇年齢は何歳, 〇〇の樹齢は何歳・何年, you get to use 歳 in the question. If you were to show a \emph{difference }in age, 歳 is possible. Though, when the tree is far older than a human could ever be, ~歳 becomes impractical. Even so, 木の年齢差 is shown with ~年. ~歳違いの〇〇 or ~歳違っている, though, is common for animals and plants. It should be clear by now that everything is subject to pragmatic issues. }

\par{  Applying ~歳 to something non-human personifies the object. We're fine with animals because we're animals. We're sometimes fine with plants because we have our tree lovers in Japan too. We're OK using it in a question when we specifically use the phrases 年齢・樹齢, but unless you have a very specific environment, this isn't applied to something like a house. }

\par{21. 生後10か月の赤ちゃん \hfill\break
10 month baby }

\par{22. ほまれくんが10か月(歳)になりました。 \hfill\break
Homare's now 10 months! }

\par{ In Ex. 22, some speakers would add 歳 to equate the mile stones of months for their child in the same way they as if they were to turn one or two. }

\par{ 日齢 exists, but it's the age of something born\slash birthed. So, it can work for humans or even eggs. We can even go smaller by considering 時(間)齢 (hour age), 分齢 (minute age), and 秒齢 (second age). These are certainly not used in the spoken language, but we can get around this. }

\par{23. 14日齢卵 \hfill\break
14 day old eggs }

\par{24a. 彼の母親は、彼がわずか5時間だった齢に亡くなった。△ \hfill\break
24b. 彼の母親は、彼が生まれてわずか5時間後に亡くなった。 \hfill\break
His mother died when he was only five hours old\slash His mother died only five hours after he was born. }

\par{\textbf{自然さ Note }: 19b is more natural than 19a. Japanese does not like having to express age in things so small and would rather avoid it by using phrases such as "after" if possible. }

\par{25. 海外在住の人が日本で里帰り出産する場合は、海外に戻るのにゼロ歳の赤ちゃんのパスポートが必要です。 \hfill\break
In the case of people abroad who come home to give birth in Japan, a passport is needed for the zero year old child for returning overseas.  }

\par{\textbf{Phrase Note }: Japanese has adopted ゼロ歳 and it is frequently used. You can also see 生後ゼロ歳ゼロヶ月. }

\par{26. 分裂して3秒後の菌 \hfill\break
Three second old bacteria \textrightarrow  bacteria three seconds after splitting }

\par{27. 13日培養された菌 \hfill\break
Thirteen day old bacteria culture \textrightarrow  Bacteria cultured for thirteen days }

\par{\textbf{Phrasing Note }: If English can avoid the phrase, then you can use that to help you understand how Japanese avoids "old". }

\par{ This, however, wouldn't be a word that you would just use in conversation. The spoken language must have ways to go around this. With this in mind, consider the following. These examples show how Japanese takes close detail to the total semantic context. }

\par{28. 1日経過したパン \hfill\break
Day old bread }

\par{29. 3日経ったウニ \hfill\break
Three day old sea urchin(s) }

\par{30. 産卵から5日たった卵をゆでる。 \hfill\break
To boil five day old eggs since being laid. }

\par{31. 生後5日の赤ちゃん \hfill\break
Five day old baby }

\par{32. 作ってから2日目のカレー \hfill\break
Curry one made two days ago }

\par{33. 妻の車は買ってから5年たっている。 \hfill\break
My wife's car is five years old\slash of five years. \textrightarrow  It has been five years since my wife bought her car. }
    