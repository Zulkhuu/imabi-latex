    
\chapter{Adjectives}

\begin{center}
\begin{Large}
第199課: Adjectives: Other Forms 
\end{Large}
\end{center}
 
\par{ There is still a bit more that you need to know about adjectival expressions in Japanese. There are older conjugations that you need to get used to as well as two other classes of adjectives with syntactic restrictions to them that the other two classes don't. }
      
\section{形容詞: The Original 連体形}
 
\par{ Using the original 連体形 for 形容詞 is very limited in Modern Japanese. As you would imagine, fossilized use in set phrases will be the most likely place you find this. It's also the case that literary titles from the West often have older style Japanese. Unless in set phrases, it is most likely the case that the places you find this is in literature. And, there is a good chance that the context may very well be in Classical Japanese.  }

\begin{ltabulary}{|P|P|}
\hline 

 & 連体形 \\ \cline{1-2}

形容詞 ending in い & き \\ \cline{1-2}

形容詞 ending in しい・じい & しき・じき \\ \cline{1-2}

\end{ltabulary}

\begin{center}
\textbf{Examples } 
\end{center}

\par{1. 熱き海  (Literary; classical) \hfill\break
The warm seas }

\par{2. 熱い海 \hfill\break
Hot seas }
 
\par{\textbf{Translation Note }: Of course, "hot" is a better translation of any form of 熱い. However, to fit the style of 熱き海, warm is used instead. This phrase especially makes sense in contexts like in the following. }
 
\par{3. 70度以上の熱い海に生息する。 \hfill\break
To live in hot seas over 70℃. }

\par{4. この素晴らしき世界 \hfill\break
What a wonderful world }

\par{5. 我が良き友よ \hfill\break
My good friend! }

\par{6. 悪しき者は火と硫黄の池に投げ込まれた。 \hfill\break
The evil ones were thrown into the lake of fire and sulfur. }

\par{7. それであなたがた理解ある人々よ、わたしに聞け、神は断じて悪を行うことなく、全能者は断じて不義を行うことはない。神は人のわざに従ってその身に報い、各々の道に従って、その身に振りかからせられる。誠に神は悪しき事を行われない。全能者は裁きをまげられない。だれか全世界を彼に負わせた者があるか。神がもしその霊をご自分に取り戻し、その息をご自分に取り集められるならば、全ての肉は共に滅び、人は塵に帰るであろう。 \hfill\break
Therefore, hear me, you men of understanding: far be it from God that he should do wickedness, and from the Almighty that he should do wrong. For according to the work of a man he will repay him, and according to his ways he will make it befall him. Of a truth, God will not do wickedly, and the Almighty will not pervert justice. Who gave him charge over the earth, and who laid on him the whole world? If he should set his heart to it and gather to himself his spirit and his breath, all flesh would perish together and man would return to dust. \hfill\break
From ヨブ記 第三四章一〇~一五節 }

\par{8. 強気を砕く。 \hfill\break
To crush the strong. }

\par{\textbf{Grammar Note }: It was also possible to use the 連体形 of adjectives as nominal phrases in older Japanese. This is still seen in set phrases or purposely old-fashioned statements like Ex. 8. }
      
\section{Adjectival Elements in Sino-Japanese Compounds}
 
\par{ Have you noticed many compounds with translations with both an adjective and a noun? Many adjectival phrases that came into the language as "adjective + noun" were turned into nominal phrases in Japanese. Some become 形容動詞, but some don't. Why is this? }

\par{ It's interesting to consider how 明るい and 暗い are both adjectives, but 明暗 is not. In fact, 明暗な is ungrammatical.You can say something like the following. }

\begin{center}
\textbf{Examples } 
\end{center}

\par{9. 明暗の対比 \hfill\break
Light and dark contrast }

\par{ Of course, there are times when you add two adjectival 漢字 and get a 形容動詞. }

\par{10. 善良な市民 \hfill\break
Good citizen }

\par{\textbf{漢字 Note }: 善 and 良 may be used to spell いい. }

\par{11. 日本の \textbf{希少 }な野生水生生物 \hfill\break
Japan's scarce wild aquatic organisms }

\par{\textbf{漢字 Note }: 希 is a simplified spelling of 稀, which is in 稀な read as まれ to mean "rare". 少 is in 少ない. }

\par{ The most interesting examples are things like 良法 (good method). These words tend to always be formal and 書き言葉. After all, this is a foreign construction. However, there are still plenty more examples that are commonly used words. }

\par{ There is no morphology on the adjectival morpheme (meaning unit) 良 to function as an adjective. Thus, \{よい・いい\}方法 would be more practical in speaking. Below are more examples of such words. }

\par{12. \textbf{${\overset{\textnormal{だん}}{\text{暖}}}$ }${\overset{\textnormal{かい}}{\text{海}}}$ に ${\overset{\textnormal{す}}{\text{棲}}}$ む ${\overset{\textnormal{さめ}}{\text{鮫}}}$  \hfill\break
A shark that lives in warm seas }

\par{\textbf{漢字 Note }: 棲 and 鮫 are not 常用漢字. }

\par{13. ${\overset{\textnormal{しがいせん}}{\text{外線}}}$ による ${\overset{\textnormal{えいきょう}}{\text{影響}}}$ を ${\overset{\textnormal{ふせぐ}}{\text{防ぐ}}}$ 。 \hfill\break
To prevent effects from ultraviolet rays. }

\par{14. ${\overset{\textnormal{せいぶつ}}{\text{生物}}}$ の ${\overset{\textnormal{たようせい}}{\text{様性}}}$ を ${\overset{\textnormal{たも}}{\text{保}}}$ つべきだ。 \hfill\break
We should protect biological diversity. }

\par{15. ${\overset{\textnormal{ぜつめつ}}{\text{絶滅}}}$ ${\overset{\textnormal{きぐ}}{\text{危惧}}}$ の ${\overset{\textnormal{しょうすう}}{\text{数}}}$ ${\overset{\textnormal{げんご}}{\text{言語}}}$ を守るプロジェクトを開始する。 \hfill\break
To start a project for protecting endangered minority languages. }

\par{16. ${\overset{\textnormal{りょうやく}}{\text{薬}}}$ は口に ${\overset{\textnormal{にが}}{\text{苦}}}$ し。 \hfill\break
Good medicine is bitter to the taste. }

\par{17. 彼には ${\overset{\textnormal{ぜんあく}}{\text{}}}$ の ${\overset{\textnormal{かんねん}}{\text{観念}}}$ がない。 \hfill\break
He cannot tell right from wrong. }

\par{18. ${\overset{\textnormal{ぜんい}}{\text{意}}}$ VS ${\overset{\textnormal{あくい}}{\text{意}}}$ \hfill\break
Good intent vs malice }
      
\section{ナル形容動詞}
 
\par{ These adjectives never made the complete jump to modern 形容動詞. All modern ones come from this class from Classical Japanese. There are still several that are used a lot, but they still often have a formal feeling simply because of their grammatically restrictive use. Their old base set comes from the base set of the old copula verb なり, and they may show up in old proverbs and set phrases, but they are not necessary to know in order to use them. For completeness, they are provided below.  }

\begin{ltabulary}{|P|P|P|P|P|P|}
\hline 

未然形 & 連用形 & 終止形 & 連体形 & 已然形 & 命令形 \\ \cline{1-6}

なら- & なり-・に & なり & なる & なれ- & なれ \\ \cline{1-6}

\end{ltabulary}

\par{\textbf{Usage Note }: You can also use the なる-連体形 of current 形容動詞 and other attributive expressions in more neo-classical or formal texts. For instance, you might see 次なる instead of 次の. }

\begin{center}
\textbf{Examples } 
\end{center}

\par{19a. ${\overset{\textnormal{しんでん}}{\text{神殿}}}$ は ${\overset{\textnormal{せい}}{\text{聖}}}$ なる地であるはずです。 \hfill\break
19b. 神殿は ${\overset{\textnormal{しんれい}}{\text{神霊}}}$ な場所であるはずです。(More common) \hfill\break
A temple is supposed to be a holy place. }

\par{20. ${\overset{\textnormal{じゅしょう}}{\text{受賞}}}$ を ${\overset{\textnormal{おお}}{\text{大}}}$ いなる ${\overset{\textnormal{よろこ}}{\text{喜}}}$ びとする。 \hfill\break
To treat receiving the prize as a great joy. }

\par{21. 私たちはいくつかの ${\overset{\textnormal{せい}}{\text{聖}}}$ なる ${\overset{\textnormal{ち}}{\text{地}}}$ を ${\overset{\textnormal{おとず}}{\text{訪}}}$ れました。 \hfill\break
We visited some holy sites. }

\par{22. 七は聖なる数です。 \hfill\break
7 is a holy number. }

\par{23. ${\overset{\textnormal{さら}}{\text{更}}}$ なる ${\overset{\textnormal{しえん}}{\text{支援}}}$ を求めています。 \hfill\break
We are still seeking more aid. }

\par{24a. いかな(る)時でも (ちょっと古風) \hfill\break
24b. どんな時でも  (もっと自然) \hfill\break
Any time }

\par{${\overset{\textnormal{}}{\text{25a. 単}}}$ なる ${\overset{\textnormal{うわさ}}{\text{噂}}}$ にすぎない。 \hfill\break
25b. ただの噂にすぎない。 \hfill\break
It doesn't pass being a mere rumor. }

\par{26. 仕事が ${\overset{\textnormal{おお}}{\text{大}}}$ いに ${\overset{\textnormal{はかど}}{\text{捗}}}$ った。 \hfill\break
My job has made good headway. }

\par{\textbf{Word Note }: 大いに comes from 大いなる, which happens to retain its adverbial form. }
      
\section{タル形容動詞}
 
\par{ As mentioned earlier in the introduction of this lesson, there is a defunct class of adjectival verbs in Japanese called タル形容動詞. As the name suggests, their attribute base is タル. In Modern Japanese the bases are typically limited to the と-連用形 and the たる-連体形. The と-連用形 can make adverbs. Most are in decline. Their attribute base can be replaced with とした. Some have acquired other legitimate attributive forms. For instance, you can use 主な and 主たる (principal\slash main). }

\par{27a. 主たる理由はこれです。(古風) \hfill\break
27b. 主な理由はこれです。(もっと自然) \hfill\break
The main reason is this. }

\par{28. 全然たる狂人 \hfill\break
An absolute maniac }

\par{29. \{名立たる・有名な\}観光地 \hfill\break
A famous tourist spot }

\par{30. ${\overset{\textnormal{めんぜん}}{\text{面前}}}$ での ${\overset{\textnormal{ちょうしょう}}{\text{嘲笑}}}$ は ${\overset{\textnormal{ぶじょく}}{\text{侮辱}}}$ の ${\overset{\textnormal{さい}}{\text{最}}}$ たるものだ。 \hfill\break
Scorn in one's presence is the extremity of insult. }

\par{31a. 最たる例 (古風) \hfill\break
31b. 最も顕著な例  (自然) \hfill\break
Prime example }

\par{32. 人間の活動の最たるもの \hfill\break
The prime thing to human activity }

\par{33. ${\overset{\textnormal{どうどう}}{\text{堂々}}}$ \{たる・とした\} ${\overset{\textnormal{すがた}}{\text{姿}}}$ \hfill\break
 A magnificent figure }

\par{34. ${\overset{\textnormal{だんこ}}{\text{断固}}}$ \{たる・とした\}決意 \hfill\break
Resolute determination }

\par{35. ${\overset{\textnormal{たんたん}}{\text{淡々}}}$ \{たる・とした\} ${\overset{\textnormal{くちょう}}{\text{口調}}}$ で話す。 \hfill\break
To speak in a cool tone. }

\par{36. ${\overset{\textnormal{ばくぜん}}{\text{漠然}}}$ とした不安 \hfill\break
Vague anxiety }

\par{37. ${\overset{\textnormal{じゅんぜん}}{\text{純然}}}$ たる銀行 \hfill\break
Pure and simple bank }
    