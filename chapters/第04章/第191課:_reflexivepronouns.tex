    
\chapter{Pronouns III}

\begin{center}
\begin{Large}
第191課: Pronouns III: 再帰代名詞 
\end{Large}
\end{center}
 
\par{ Japanese 再帰代名詞 \emph{equate }to English "reflexive pronouns". However, unlike the reflexive pronouns of English and other European languages, these Japanese words are far more complex. Usages vary and are heavily reliant on context for correct interpretation. }

\par{ The first problem that you have to come to grips with is the sheer number of "reflexive pronouns" in Japanese: 自分、自分自身、自身、自己、自体、自ら、自ずから、己、and 各々. Though this list may appear exhaustive, it does not include words with reflexive elements in them or dialectical variants. }

\par{\textbf{漢字 Note }: There are other words with 自- with a meaning of "own". Ex. 自宅 = "One's house". }
      
\section{自分(じぶん)}
 
\par{ 自分 causes linguists on both sides of the Pacific to constantly write on the so-called long-distance reflexive pronouns of Japanese. If you ever look into this matter in greater academic depth, this term will come up. However, even without any linguistic knowledge, with what is to be discussed, this term will inevitably hone things together in your mind. }

\begin{center}
 \textbf{Source of Confusion }
\end{center}

\par{ 自分 has \textbf{no exact }English equivalent. Most textbooks say it means "(one)self", "myself", "yourself", "himself", "herself", "themselves", "itself", etc. depending on context. This, though, is oversimplified. How do you know which is meant with just \textbf{a }sentence? What if there are multiple things 自分 could refer to? }

\par{ This is where long-distance reflexive properties and knowledge of clause structures and particles come into play. For starters, consider the following sentence: }

\par{1a. 自分が馬鹿者だって知らないんだ。 }

\par{ From an English perspective where an anaphor, a "-self" word, must have an antecedent (something before them) to refer to, 自分 having no subject to refer to and being marked by が as the subject is problematic! As, additional information can be easily dropped in a Japanese sentence, long-distance co-referencing comes into play. }

\par{ 自分 can refer to a subject mentioned earlier. In the alteration of the prevision example, it is clear that 自分 = 彼女ら自身 (they themselves). }

\par{1b. 彼女らは自分が馬鹿者だって知らないんだ。 }

\begin{center}
 \textbf{Scenarios with Multiple Potential Readings }
\end{center}

\par{ There are some cases where you need the full discourse to conclude the meaning of 自分. Even with context, in sentences such as below, there really are multiple, variable readings that you can derive and not be wrong. In such situations, even for native speakers, explaining may be the only way to negate ambiguity as seen below. }

\par{2. 岸田さんは、「小田原さんが『自分は頭がいい』と言った」と言った。 }

\par{It could be that Odawara was referring to Kishida. Or, he could have been just talking about himself. Regardless of what you think is the most probable interpretation, without any other information, it's not certain. }

\begin{center}
 \textbf{Single Interpretation Scenarios }
\end{center}

\par{ Consider the following with only one possible interpretation. Regardless whether there is ambiguity in what 自分 refers to or not or how long the sentence is, it still refers to something specific. }

\par{3. 畑中先生は憲太にとって自分の親のような存在だった。 \hfill\break
Hatanaka Sensei was to Kenta like his own parent. }

\par{4. 清美は、実際としては、春彦に何を白状させたというのでなく、ただ自分の推測でものを言っているだけだ。 \hfill\break
As for Kiyomi, it's not what did she make Haruhiko confess; it's just her own conjecture. }

\par{ In Ex. 3, 畑中先生 is still the topic. She has an existence of [憲太にとって自分の親のような]. We know that 自分 doesn't refer to her because にとって provides an explanatory sense to the situation, negating the other possible reference, 憲太. Without 憲太にとって, 自分 would refer to 畑中先生. }

\par{ In Ex. 4, 自分 refers to 清美. This is because [春彦から何を白状させたという] is embedded in the nominal phrase marked by の, which could be replaced by any other noun phrase you'd like. }

\begin{center}
 \textbf{Unlikely Ambiguous Scenarios }
\end{center}

\par{ In this final example, it would be odd for the command to be for the listener to retrospect on what the \emph{speaker }has done. However, this reading could exist given enough context. However, if this is all that is said, you can safely assume that the only plausible reading is the one given in translation. }

\par{5. 君は、自分のことを反省しなさい。 \hfill\break
Look back on what you yourself have done. }

\par{ One could potentially bring up counterexamples that defy minor details of what has been said thus far, but it is clear that context decides. Even if this context still doesn't solve everything, just like how we deal with ambiguity in English, you either make the most logical assumption of what it means or seek clarification if possible. }

\begin{center}
\textbf{More Examples }
\end{center}

\par{6. 自分でやりなさい。 \hfill\break
Do it yourself. }

\par{7a. 自分自身が自分であるために \hfill\break
7b. 自分が自分自身であるために   (もっと自然) \hfill\break
In order for one to be oneself }

\par{8. 若いうちに自分のしたいことをしておくといい。 \hfill\break
It is good for you to do what you want to do while you're still young. }

\par{9a. 多くの人は、自分の体を大切にすべきだと主張している。(もっと自然な言い方) \hfill\break
9b. 自分の体を大切にすべきだと主張する人は多い。(あまり使われていない言い方) \hfill\break
A lot of people stress that everyone should take good care of themselves. }

\par{10. 彼女は鏡で長いこと自分を見つめていた。 \hfill\break
She was staring at herself in the mirror for a long time. }

\par{11.\{各人・自分\}の責任において \hfill\break
At one's own risk }

\par{\textbf{Meaning Note }: It is not proper in honorifics to use 自分 to mean "I", which is extremely common in places like the 関西弁. This colloquial usage appears to have risen from the practice of soldiers having to refer to themselves as such. It can also confusingly still be used to refer to the listener. }

\par{12. 自分にやらせてください。 \hfill\break
Please let me do it. }

\par{13. この事実にたいして君が自分達を如何ように裁いてくれても自分たちは勿論甘受する。 \hfill\break
To this truth, no matter how you judge us, we will of course put up with it. \hfill\break
From 友情 by 武者小路実篤. }

\par{\textbf{Grammar Note }: As the sentence shows, 自分達 also exists. }

\begin{center}
 \textbf{自分自身 }
\end{center}

\par{ 自分自身 is a more potent way of saying "oneself". You can see 自身 after other pronouns. So, you can get 私自身、彼自身、彼女自身, etc.  Don't confuse 自身 with 自信, which means self-confidence. }

\par{14. 自分自身のことをしろ。 \hfill\break
Do your own thing! }

\par{15. 自分自身の言葉で言ったほうがいいんじゃない? \hfill\break
Isn't it best for you to say it in your own words? }

\par{16. ぼくは自信なんかない。 \hfill\break
I don't have any confidence (in myself). }
      
\section{自己}
 
\par{ 自己 = Self. It is used in many phrases and is the choice for translating instances of "self-" in English. }

\par{17. 自己紹介させていただきます。 \hfill\break
Allow me to introduce myself. }

\par{18. その選手は、自己記録を更新しましたよ。 \hfill\break
That athlete beat her own record! }

\begin{ltabulary}{|P|P|P|P|P|P|}
\hline 

自己防衛 & Self-defence & 自己批判 & Self-criticism & 自己矛盾 & Self-contradiction \\ \cline{1-6}

自己紹介 & Self-introduction & 自己犠牲 & Self-sacrifice & 自己採点 & Self-rating \\ \cline{1-6}

自己評価 & Self-evaluation & 自己中心 & Self-centered & 自己主張 & Self-assertion \\ \cline{1-6}

\end{ltabulary}
      
\section{自体}
 
\par{ This word refers to "itself", but it can also seldom refer to one's own body or be used as an adverb meaning on the lines of そもそも. }

\par{19. 考えそれ自体は、悪くありません。 \hfill\break
The idea itself isn't bad. }

\par{20. それ自体は毒じゃない。 \hfill\break
It itself isn't a poison. }

\par{21. 家自体が古い。 \hfill\break
The house itself is old. }

\par{22. 口答えすること自体そもそも間違いだろう。 \hfill\break
Talking back in of itself in the first place was a mistake, no? }
      
\section{自ら \& 自ずから}
 
\par{ As you would imagine, these are native reflexive pronouns. The first is read as みずから. The second is read as おのずから. So, be very careful about that. They're not used in any colloquial sense like 自分. However, the overall problems of interpretation are still very relevant. In an adverbial sense they can be used like 自分で. In this case, though, there is no particle involved. These words have the same nuance effect as 自分自身. }

\par{23a. 天は自ら助くるものを助く。 (Proverb) \hfill\break
23b. 天は\{自ら・自分(自身)を\}助けるものを助ける。(Modern Japanese Equivalent) \hfill\break
Heaven helps those who help themselves. }

\par{24. 彼自ら東京へ行った。 \hfill\break
He went to Tokyo himself. }

\par{25. その事実は自ずから明らかだ。 \hfill\break
The fact speak for themselves. }

\par{\textbf{Nuance Note }: 自ずから is often seen as 自ずと and both are used in a sense that the situation and hand is naturally so. }
      
\section{己 \& 各々}
 
\par{ 己 has always been an important reflexive pronoun in Japanese. Although now it can be used as a second person pronoun slur, you still see it in a lot of old, set phrases. The root is おの-, which can be seen doubled in the expression 各々 (each and every). Although there are a lot of possible set and at times very archaic expressions that use these words, only truly relevant examples will be brought up. }

\par{26. 己を知れ。 \hfill\break
Know thyself. }

\par{27. 己の説くところを励行せよ。 \hfill\break
Practice what you preach. }

\par{28. 己が誰なのか知れ。 \hfill\break
Know who you are. }

\par{29a. 己をもって他人を律するな。 \hfill\break
29a. 自分の基準で他人を律するな。(もっと自然) \hfill\break
Don't just others by yourself. }

\par{30. 容疑者がおのおの違った説明をした。 \hfill\break
The suspects each gave a different story. }

\par{31. どこの学校でも新入生がそうであるように、私は毎日新鮮な気持ちで通いながらも、とりとめのない思いがしていた。知り人は鶴川であった。どうしても鶴川とばかり話すようになる。それでは折角新らしい世界へ出て来た意味がないのを、鶴川のほうでも感じているらしく、数日たつうちに、休み時間にはわざと二人が離れて、 \textbf{お のがじし }新らしい友を開拓しようとした。しかし吃りの私には、そういう勇気もなかったので、鶴川の友が増えるにつれ、私はますます孤りになった。 \hfill\break
Like a new student in any school, while I went with fresh feelings each day, there was something I couldn't bring myself to. Tsurukawa was my acquaintance. I was to no matter what only talk with Tsurukawa. With even Tsurukawa sensing that there was no meaning to coming into this long awaited new world, in a matter of days, we purposely separated during break to each find friends. But, with my stutter, as Tsurukawa's friends grew in number, I became ever more alone. \hfill\break
From 金閣寺 by 三島由紀夫. }

\par{\textbf{Word Note }: おのがじし is an archaism meaning それぞれに. }

\par{\textbf{Spelling Note }: In Ex. 31, there are some interesting spellings. 独り is written instead as 孤り. Also, 新しい is written as 新らしい. This is because 送り仮名 usage was not standardized at the time. Lastly, although its not exceptional, 吃り is read as どもり.  }
    