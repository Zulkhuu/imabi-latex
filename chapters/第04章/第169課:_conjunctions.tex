    
\chapter{Conjunctions}

\begin{center}
\begin{Large}
第169課: Conjunctions 
\end{Large}
\end{center}
 
\par{ Conjunctions ${\overset{\textnormal{せつぞくし}}{\text{接続詞}}}$ connect sentences together in Japanese. They don't normally connect clauses, however. This is quite unlike English, which often doesn't like some conjunctions being used at the beginning of a sentence. Instead, Japanese does a good job distinguishing between conjunctions, which is the topic of this lesson, and conjunctive particles. }

\par{ Some conjunctive phrases are made of multiple phrases. This makes things a bit more complicated, especially when things look very similar minus one thing. Also be aware that all of the conjunctions may not be expressed with either conjunctions or conjunctive particles and thus will not be mentioned in this lesson. }

\par{ In this this lesson conjunctions are labeled with the following terms. This lesson does not aim to teach you all conjunctive phrases in Japanese, but you will definitely learn what they are, how they are used, and plenty to practice with. }

\begin{ltabulary}{|P|P|P|P|}
\hline 

Function &  &  & Abbreviation \\ \cline{1-4}

Parallelism & 並行 & へいこう & 並 \\ \cline{1-4}

Alternation & 代替 & だいがえ \hfill\break
& 代 \\ \cline{1-4}

Addition & 添加 & てんか & 添 \\ \cline{1-4}

Change & 転換 & てんかん & 転 \\ \cline{1-4}

Concession & 逆接 & ぎゃくせつ & 逆 \\ \cline{1-4}

Sequence & 連続 & れんぞく & 連 \\ \cline{1-4}

\end{ltabulary}
      
\section{Single Word Conjunctions}
 
\begin{ltabulary}{|P|P|P|P|P|P|}
\hline 

 転 & さて & Now & 添 & 加えて & Moreover \\ \cline{1-6}

逆 & しかし & However & 逆 & 一方 & On the other hand \\ \cline{1-6}

添 & しかも & Moreover & 連 & 従って & Therefore \\ \cline{1-6}

代 & 即ち & In other words & 逆 & ただし \hfill\break
& Provided \hfill\break
\\ \cline{1-6}

添 & そして & And & 並 & 及び & And \\ \cline{1-6}

転 & そもそも & In the first place & 添 & なお & Still \\ \cline{1-6}

添 & 且つ & Also & 連 & よって & Thus \\ \cline{1-6}

\end{ltabulary}

\par{\textbf{Usage Notes }: }
 
\par{1. さて is used to change the topic of conversation. It may also be an interjection similar to "well" in English. \hfill\break
2. しかし is not used as frequently as the English equivalent "however". It is used, first and foremost, to contrast two different things. \hfill\break
3. When (その) ${\overset{\textnormal{いっぽう}}{\text{一方}}}$ (では)  is used with ${\overset{\textnormal{たほう}}{\text{他方}}}$ (では) before it, in which case その may never precede it, the interpretation changes to "on the one hand". This extended pattern is not likely to be used in the spoken language. \hfill\break
4. Due to it looking like しかし, しかも is often misused by students. It is just like そのうえに. \hfill\break
5. そして shows that something additionally happens. }

\begin{center}
\textbf{Examples } 
\end{center}
 
\par{1a. あるいは ${\overset{\textnormal{}}{\text{本当}}}$ かもしれません。 \hfill\break
 ${\overset{\textnormal{おそ}}{\text{1b. 恐}}}$ らく(それが) ${\overset{\textnormal{}}{\text{本当}}}$ かもしれません。(More natural) \hfill\break
Perhaps that's true. }
 
\par{${\overset{\textnormal{}}{\text{2. 明朝十時集合}}}$ 。ただし ${\overset{\textnormal{}}{\text{雨}}}$ の ${\overset{\textnormal{}}{\text{場合}}}$ は ${\overset{\textnormal{}}{\text{中止}}}$ 。 \hfill\break
A 10 o\textquotesingle clock meeting tomorrow, but cancellation in case of rain. }

\par{3. しかも、あられが降ってるんだよ。 \hfill\break
Besides, it's hailing! }

\par{\textbf{漢字 Note }: あられ may rarely be spelled as霰 }

\par{4. そりゃそもそもの始まりだった。 \hfill\break
That's all it was to begin with. }

\par{5. 命は天に ${\overset{\textnormal{あ}}{\text{在}}}$ り。 ${\overset{\textnormal{さ}}{\text{然}}}$ らばただ時を待つのみ。(Old-fashioned) \hfill\break
Life is in heaven. So, we just wait for the time. }

\par{6a. よく学び ${\overset{\textnormal{か}}{\text{且}}}$ つよく遊ぶ。(Not an imperative; Set phrase; old-fashioned) \hfill\break
6b. よく学びよく遊べ。(Imperative) \hfill\break
Study well and play well. }

\par{7. お金は ${\overset{\textnormal{すなわ}}{\text{即}}}$ ち幸福と考える。 \hfill\break
To think of money, in other words, happiness. }

\par{8. しかし、 ${\overset{\textnormal{けいき}}{\text{景気}}}$ はまだ ${\overset{\textnormal{かいふく}}{\text{回復}}}$ しない。 \hfill\break
However, the economy hasn't recovered. }

\par{9. \{さて・さあ\}、始めよう。 \hfill\break
Well, let's begin. }

\par{10. だったら、 ${\overset{\textnormal{てつだ}}{\text{手伝}}}$ おう。 \hfill\break
If that's the case, I'll help. }

\par{11. ${\overset{\textnormal{ただ}}{\text{但}}}$ し雨の場合は ${\overset{\textnormal{えんき}}{\text{延期}}}$ 。 \hfill\break
However, it will be postponed in the case of rain. }

\par{12. 彼女は ${\overset{\textnormal{かじん}}{\text{歌人}}}$ であり、かつ小説家であります。 \hfill\break
She is a tanka poet, and she is also a novelist. }

\par{13. あいつはそもそも殺すつもりはなかった。 \hfill\break
He didn't have an intention of killing in the first case. }

\par{14. 0 ${\overset{\textnormal{たい}}{\text{対}}}$ 0の ${\overset{\textnormal{きんこう}}{\text{均衡}}}$ を ${\overset{\textnormal{やぶ}}{\text{破}}}$ る。 \hfill\break
To break a tie of 0-0. }

\par{\textbf{Word Note }: 対 is the equivalent of "versus". }

\par{15. ${\overset{\textnormal{ま}}{\text{先}}}$ ず日本へ行きました。そしていろいろなところへ行きました。 \hfill\break
First of all, I went to Japan, and I went to a lot of different places. }

\par{16. よって ${\overset{\textnormal{くだん}}{\text{件}}}$ の ${\overset{\textnormal{ごと}}{\text{如}}}$ し。(Set phrase; Formal) \hfill\break
Therefore, it is as the aforementioned statement. }
      
\section{Multiple Word Conjunctions}
 
\begin{ltabulary}{|P|P|P|P|P|P|}
\hline 

 連 & そのうえ & Besides & 連 & そのうち & Some day \\ \cline{1-6}

代 & または & Or & 添 & ところで & By the way \\ \cline{1-6}

連 & 何故なら & Because & 連 & そればかりか & Besides \\ \cline{1-6}

逆 & だから & Because & 連 & さもないと & Otherwise \\ \cline{1-6}

添 & それで & And so & 連 & それから & Then \\ \cline{1-6}

代 & それとも & Or & 連 & それなら & If so \\ \cline{1-6}

添 & こうして & With this & 添 & そうして & With that \\ \cline{1-6}

代 & もしくは & Or & 逆 & それどころか & Rather \\ \cline{1-6}

並 & 並びに & Both\dothyp{}\dothyp{}\dothyp{}and\dothyp{}\dothyp{}\dothyp{} & 逆 & それでも & Nevertheless \\ \cline{1-6}

連 & それ故 & Therefore, thus & 連 & それにしても & Even so \\ \cline{1-6}

連 & 故に & Accordingly & 連 & それにつけても & Anyway \\ \cline{1-6}

逆 & だが & But & 連 & それはさておき & By the way \\ \cline{1-6}

連 & それに & Moreover &  &  &  \\ \cline{1-6}

\end{ltabulary}

\par{17. それゆえ、 ${\overset{\textnormal{じゅうげきせん}}{\text{銃撃戦}}}$ の ${\overset{\textnormal{ししゃ}}{\text{死者}}}$ は ${\overset{\textnormal{}}{\text{百人以上}}}$ に ${\overset{\textnormal{のぼ}}{\text{上}}}$ ります。 \hfill\break
Therefore, the casualties from the shoot-out will climb to over 100. }

\par{18a. ${\overset{\textnormal{ゆ}}{\text{行}}}$ く ${\overset{\textnormal{}}{\text{川}}}$ の ${\overset{\textnormal{}}{\text{流}}}$ れは ${\overset{\textnormal{た}}{\text{絶}}}$ えず ${\overset{\textnormal{あふ}}{\text{溢}}}$ れ ${\overset{\textnormal{}}{\text{出}}}$ している ${\overset{\textnormal{}}{\text{故}}}$ にもとの ${\overset{\textnormal{}}{\text{水}}}$ ではない。 \hfill\break
${\overset{\textnormal{}}{\text{18b. 行}}}$ く ${\overset{\textnormal{}}{\text{川}}}$ の ${\overset{\textnormal{}}{\text{流}}}$ れは ${\overset{\textnormal{}}{\text{絶}}}$ えずして、しかももとの ${\overset{\textnormal{}}{\text{水}}}$ にあらず。(Original Classical version) \hfill\break
The flow of a passing river endlessly flows; hence, it is not the original water. }
 
\par{19. その ${\overset{\textnormal{か}}{\text{代}}}$ わりに、 ${\overset{\textnormal{}}{\text{本}}}$ を ${\overset{\textnormal{}}{\text{買}}}$ った。 \hfill\break
Instead, I bought a book. }
 
\par{20. ところでお ${\overset{\textnormal{}}{\text{仕事}}}$ は? \hfill\break
By the way, your job is? }
 
\par{21. それにつけても ${\overset{\textnormal{}}{\text{思}}}$ い ${\overset{\textnormal{}}{\text{出}}}$ すのは ${\overset{\textnormal{}}{\text{古}}}$ き ${\overset{\textnormal{}}{\text{良}}}$ き ${\overset{\textnormal{}}{\text{時代}}}$ だ。 \hfill\break
Anyways, that reminds me of the good old days. }

\par{22. ${\overset{\textnormal{しんごなら}}{\text{新語並}}}$ びに ${\overset{\textnormal{がいらいご}}{\text{外来語}}}$ に ${\overset{\textnormal{かん}}{\text{関}}}$ する ${\overset{\textnormal{しりょう}}{\text{資料}}}$ を ${\overset{\textnormal{しら}}{\text{調}}}$ べる。 \hfill\break
To examine data about both neologisms and foreign expressions. }
 
\par{23. そういえば、 ${\overset{\textnormal{}}{\text{久実}}}$ さんはどうしてるんだろう? \hfill\break
Now that I think of it, I wonder what Kumi is doing? }
 
\par{${\overset{\textnormal{}}{\text{24. 日本語}}}$ には ${\overset{\textnormal{どうおんいぎご}}{\text{同音異義語}}}$ が ${\overset{\textnormal{}}{\text{多}}}$ い。\{ ${\overset{\textnormal{}}{\text{故}}}$ に・それゆえ・よって・そういうわけで・そのようなわけで・従って・このた め・そのため・だから・このことから\}、 ${\overset{\textnormal{}}{\text{漢字}}}$ で ${\overset{\textnormal{}}{\text{書}}}$ く。 \hfill\break
Japanese has a lot of homophones. Therefore, you write with Kanji. }
 
\par{\textbf{Historical Note }: The introduction of 漢字 caused Japanese to have a lot of homophones. }
 
\par{\textbf{Word Note }: From the single example above, there are a lot of possible conjunctions out there that relatively mean the same thing. However, what are their exact differences? }
 
\begin{itemize}
 
\item ${\overset{\textnormal{}}{\text{故}}}$ に      shows that due to the fact it's after, the following is as effect.  
\item それ ${\overset{\textnormal{}}{\text{故}}}$ に      shows that the stated matter is reason for the next case stated after it.  
\item よって states that      the previous sentence as the reason or evidence.  
\item そういわけで・そのようなわけで = With that reason  
\item 従って = Therefore;      so; consequently.  
\item このため・これゆえに =      Points that as the goal or reason.  
\item そのため = このため. This is a rare      occasion with これ are      interchangeable それ.  
\item だから = So;      because  
\item このことから ≒ With this  
\end{itemize}
 
\par{ Even in English, there are several interchangeable but slightly different phrases that can be used. The specifics and impromptu nature of speech at a given situation is the ultimate determining factor. Some of these are more formal or casual then the others. So, that has a lot to do with which is used. }
 
\par{25a. だから ${\overset{\textnormal{}}{\text{言}}}$ わないことじゃない。 \hfill\break
25b. だから ${\overset{\textnormal{}}{\text{言}}}$ わんこっちゃない。(Slang\slash very casual) \hfill\break
I told you so. }
 
\par{26. その ${\overset{\textnormal{けっか}}{\text{結果}}}$ 、 ${\overset{\textnormal{しけん}}{\text{試験}}}$ に ${\overset{\textnormal{}}{\text{受}}}$ かった。 \hfill\break
Because of that, I passed the exam. }

\par{27. それはさておき、東京に引っ ${\overset{\textnormal{こ}}{\text{越}}}$ すんだ。 \hfill\break
By the way, I'm moving to Tokyo. }

\par{28. それにしても\{どれも・いずれも\} ${\overset{\textnormal{けっ}}{\text{決}}}$ して ${\overset{\textnormal{かんぺき}}{\text{完璧}}}$ じゃないね。 \hfill\break
Even so, nothing is perfect, you know? }
 
\par{29. しかしながら、 ${\overset{\textnormal{よさん}}{\text{予算}}}$ がかかりすぎる。 \hfill\break
However, it is too much for the budget. }

\par{30. ${\overset{\textnormal{われおも}}{\text{我思}}}$ う、 ${\overset{\textnormal{}}{\text{故}}}$ に ${\overset{\textnormal{}}{\text{我}}}$ あり。 \hfill\break
I think; therefore,I am. }
 
\par{31. そればかりか ${\overset{\textnormal{}}{\text{動物}}}$ も ${\overset{\textnormal{}}{\text{殺}}}$ された。 \hfill\break
Besides that, even the animals were killed. }
 
\par{32. それどころか、もう ${\overset{\textnormal{}}{\text{20歳}}}$ です。 \hfill\break
Rather, he's already twenty. }
 
\par{33. その ${\overset{\textnormal{}}{\text{結果試験}}}$ に ${\overset{\textnormal{}}{\text{落}}}$ ちた。 \hfill\break
And thus, I failed the exam. }
 
\par{34. さもないと ${\overset{\textnormal{}}{\text{警察}}}$ を ${\overset{\textnormal{}}{\text{呼}}}$ ぶぞ。 \hfill\break
If you don't, I'll call the police. }
 
\par{${\overset{\textnormal{}}{\text{35. 度々}}}$ インフルエンザに市民の半分もしくは ${\overset{\textnormal{ぜんいん}}{\text{全員}}}$ かかって ${\overset{\textnormal{とこ}}{\text{床}}}$ につくこともある。 \hfill\break
Often, (the city) also has times where half or all of the citizens are down with influenza. }

\par{36. それなら、いつも ${\overset{\textnormal{}}{\text{学校}}}$ に ${\overset{\textnormal{おく}}{\text{遅}}}$ れるのはどういうわけですか。 \hfill\break
If that's the case, how is it that you're always late to school? }

\begin{center}
 \textbf{それで, それに, それから, \& そして }
\end{center}

\par{  Many people confuse それで and それに. それで shows that what was stated before is the reason or cause for what follows while それに shows another additional fact or situation. Also, if the previous facts were positive in nature, so should the additional information. This is the same for negative things too. そこで is also similar to それで, but it is specifically used to when you know in detail the reason for what follows. The previous context is very concrete, and this is not always the case with それで. そこで can also be used to mean さて, and this is something それで never means. }

\par{ Other related conjunctions include それから and そして. The former is used to mean "after that\slash then" showing chronological order of events. The latter is the generic "and". }
 
\par{${\overset{\textnormal{}}{\text{37. 韓国語}}}$ は ${\overset{\textnormal{}}{\text{面白}}}$ いです。それに、 ${\overset{\textnormal{やく}}{\text{役}}}$ に ${\overset{\textnormal{}}{\text{立}}}$ ちます。 \hfill\break
Korean is interesting. Moreover, it's beneficial. }

\par{38. 「ロッテリアは値段も安いし、おいしいんです」「それで人が多いんですね」 \hfill\break
"Lotteria is cheap and delicious" "So, that's why it's crowded." }
 
\par{${\overset{\textnormal{}}{\text{39. きのう風邪}}}$ を引きました。それで、 ${\overset{\textnormal{}}{\text{今日学校}}}$ を ${\overset{\textnormal{}}{\text{休}}}$ んだんです。 \hfill\break
I caught a cold yesterday. So, I stayed away from school today. }
 
\par{${\overset{\textnormal{}}{\text{40. 昨日}}}$ は ${\overset{\textnormal{じゅぎょう}}{\text{授業}}}$ のあと ${\overset{\textnormal{}}{\text{公園}}}$ へ ${\overset{\textnormal{}}{\text{行}}}$ きました。それから、 ${\overset{\textnormal{}}{\text{三時間}}}$ ぐらい ${\overset{\textnormal{}}{\text{友達}}}$ と ${\overset{\textnormal{}}{\text{話}}}$ をしました。 \hfill\break
Yesterday, I went to the park after class. Then, I talked for about three hours with friends. }
 
\par{${\overset{\textnormal{}}{\text{41. 私}}}$ は ${\overset{\textnormal{}}{\text{おととし五月}}}$ に ${\overset{\textnormal{そつぎょう}}{\text{卒業}}}$ しましたが、それからずっと ${\overset{\textnormal{}}{\text{仕事}}}$ を ${\overset{\textnormal{}}{\text{探}}}$ しています。 \hfill\break
I graduated in May two years ago, but I've been searching for a job ever since. }
 
\par{42. 「このごろどうですか。 ${\overset{\textnormal{}}{\text{忙}}}$ しいですか」「 ${\overset{\textnormal{}}{\text{宿題}}}$ がたくさんありますし、それに、 ${\overset{\textnormal{ひま}}{\text{暇}}}$ な ${\overset{\textnormal{}}{\text{時間}}}$ はほとんどありません」 \hfill\break
"How have you been lately? Are you busy?" "I have a lot of homework, and on top of that, I barely            have any free time". }
 
\par{43. 「あなたのアパートは、どんなアパートですか」「 ${\overset{\textnormal{}}{\text{私}}}$ のアパートはきれいなところですし、それに ${\overset{\textnormal{やちん}}{\text{家賃}}}$ が ${\overset{\textnormal{}}{\text{安}}}$ いの          で、 ${\overset{\textnormal{}}{\text{住}}}$ みやすいです」 \hfill\break
“What kind of apartment do you have?” “My apartment is a pretty place, and since the rent is cheap,         it's easy to live there”. }
 
\par{\textbf{Phrase Note }: When using それに, all the parts of the sentence must either have positive or negative connotations but never both! }
 
\par{44. 「あなたの ${\overset{\textnormal{}}{\text{住}}}$ んでいる ${\overset{\textnormal{}}{\text{町}}}$ はどんな ${\overset{\textnormal{}}{\text{町}}}$ ですか」「 ${\overset{\textnormal{}}{\text{人口}}}$ が ${\overset{\textnormal{}}{\text{少}}}$ なく ${\overset{\textnormal{}}{\text{美}}}$ しい ${\overset{\textnormal{}}{\text{町}}}$ です。それに、 ${\overset{\textnormal{ふんいき}}{\text{雰囲気}}}$ のよいところです」 \hfill\break
“What kind of town is the town that you live in?” “It's a small, beautiful town. Moreover, it is a local             with a good atmosphere”. \hfill\break
 \hfill\break
 \textbf{Word Note }: ${\overset{\textnormal{}}{\text{人}}}$ can replace ${\overset{\textnormal{}}{\text{人口}}}$ above, but it is not the best choice of the two. }
 
\par{45. 「 ${\overset{\textnormal{}}{\text{日本語}}}$ の ${\overset{\textnormal{}}{\text{勉強}}}$ はどうですか」「 ${\overset{\textnormal{}}{\text{宿題}}}$ が ${\overset{\textnormal{}}{\text{簡単}}}$ ですし、それに ${\overset{\textnormal{}}{\text{日本人}}}$ と ${\overset{\textnormal{}}{\text{毎日会話}}}$ しております」 \hfill\break
“How are your Japanese studies?” “My homework is easy, and I talk to Japanese people every day”. }

\begin{center}
 \textbf{読み物: アイヌ語を守ろう! }
\end{center}

\par{ This is an example of a small speech in Japanese. Read through the text and answer the questions that follow. Conjunctions will be in bold. No English will be given. You are free to use previous lessons and dictionary resources to understand the text. }

\par{皆様、こんにちは。〇〇と ${\overset{\textnormal{もう}}{\text{申}}}$ します。今日は、「アイヌ語を守ろう」というテーマについて、発表させていただきます。 }
 
\par{私は大学で言語学を専攻しており、特に東アジアの言語学について研究しております。その中で、日本の文化の一部であるアイヌ語が消滅の ${\overset{\textnormal{きき}}{\text{危機}}}$ にあるということを知りました。言語学を専攻している私にとって、言語の消滅は ${\overset{\textnormal{むし}}{\text{無視}}}$ することの出来ない問題です。 \textbf{また }、言語は人間のごとく生きているというのが私の考えです。アイヌ語の命が ${\overset{\textnormal{た}}{\text{絶}}}$ たれたならば、 \textbf{いわば }、使われなくなってしまったのなら、もう二度とアイヌ語が復活できなくなってしまうと考え、今回の発表のテーマに ${\overset{\textnormal{いた}}{\text{至}}}$ りました。 }
 
\par{\textbf{先ず }、簡単にアイヌ語について説明させていただきます。アイヌ語とは、日本の北海道などで、話されている少数言語です。多くのアイヌ人は、現代では、両親からアイヌ語を教わっていません。アイヌ語の教室が開設されていても、ほとんどの人が学ぼうとしない、というのが現状です。このように、消滅の危機にあるアイヌ語ですが、私はアイヌ語を保たなければならないと思います。 }
 
\par{\textbf{では }、どうしてアイヌ語を保つ必要があるのでしょうか。それはアイヌ語の ${\overset{\textnormal{どくじせい}}{\text{独自性}}}$ にあります。アイヌ人の歴史や経験が記号化されたものがアイヌ語です。 \textbf{例えば }、「神」という日本語はアイヌ語の「カムイ」から出来ました。これは、大昔に、神道の ${\overset{\textnormal{がいねん}}{\text{概念}}}$ がなかった日本語が ${\overset{\textnormal{こんぽんてき}}{\text{根本的}}}$ な神道の概念を持つアイヌ語の ${\overset{\textnormal{えいきょう}}{\text{影響}}}$ を受けたことに由来します。 \textbf{${\overset{\textnormal{ほか}}{\text{他}}}$ }\textbf{にも }、「ラッコ」や「トナカイ」、「くま」などアイヌ語に影響を受けた日本語が少なからず存在します。このように、アイヌ語の独自性が日本の文化に ${\overset{\textnormal{ふく}}{\text{含}}}$ まれていることは明らかなのです。アイヌ語も日本語と同じく ${\overset{\textnormal{そんちょう}}{\text{尊重}}}$ されるべきではないでしょうか? }
 
\par{現在、アイヌ語のネイティブスピーカーのほとんどが ${\overset{\textnormal{こうれいしゃ}}{\text{高齢者}}}$ になってしまっているので、会話を ${\overset{\textnormal{ほぞん}}{\text{保存}}}$ するといった方法でアイヌ語を守ろうとしている人がいるそうです。これらのテープを使って、未来の ${\overset{\textnormal{ねっしん}}{\text{熱心}}}$ な生徒のために ${\overset{\textnormal{べんきょう}}{\text{勉強}}}$ ${\overset{\textnormal{きょうざい}}{\text{教材}}}$ を作ることができるようになればいいと思います。アイヌ語が消滅する可能性は ${\overset{\textnormal{いぜん}}{\text{依然}}}$ として高いものの、こういった ${\overset{\textnormal{こうかてき}}{\text{効果的}}}$ な ${\overset{\textnormal{かつどう}}{\text{活動}}}$ のため、近いうちに消滅してしまう可能性も、少しずつですが、減ってきているようです。これから多くの日本人がアイヌ語の尊さに気づき、アイヌ語が ${\overset{\textnormal{けいしょう}}{\text{継承}}}$ されていくことを願います。 }
 
\par{これで私の発表を終わります。ご ${\overset{\textnormal{せいちょう}}{\text{静聴}}}$ どうもありがとうございました。何かご質問などありますでしょうか? }

\par{Questions: }

\par{1. What is the theme of this speech? \hfill\break
2. Why is Ainu important to protect according to this person? \hfill\break
3. What are some examples of Ainu influence in Japanese? \hfill\break
4. What has become of the Ainu speaking population? \hfill\break
5. What is being doing to protect Ainu? \hfill\break
6. What is the presenter majoring in? \hfill\break
7. What is language linked to? \hfill\break
8. Is it still likely Ainu will die out? }
    