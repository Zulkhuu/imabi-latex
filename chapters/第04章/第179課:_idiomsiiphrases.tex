    
\chapter{Idioms II}

\begin{center}
\begin{Large}
第179課: Idioms II: Basic Expressions 
\end{Large}
\end{center}
 
\par{ An idiom (慣用句) is a set expression that diverts to some degree from the literal definition from which it originally derives. }

\par{It is very easy to comprehend an idiom when said in one's native language. When a Japanese person thinks or hears a given set phrase, their mind naturally thinks of the idiomatic intent from its literal approach. So, no matter how deviant a phrase is from the literal arrangement of words, a connection is still visible and very much shows the cultural reasoning why a phrase means what it does idiomatically. But, a definition of a said idiom from another language is a different story and requires a two step process. }

\par{1. Understanding the literal meaning of the phrase \hfill\break
2. Understanding what is actually trying to be said }

\par{Idioms is by far the most difficult aspect of the Japanese language. This is further compounded by the sheer number of idioms that exist in the language. On the spot translation without a culture translation can lead to a horrible chain of events caused by tripping into the 'idiomatic cultural divide'. }

\par{Many idioms in Japanese derive from unique indigenous concepts such as martial arts and an array of ceremonies--tea, etc. The sheer number, though, of the idioms that exist can be reduced by realizing a few key concepts. }

\par{Many idioms are only different from each other by a single synonymous and interchangeable particle or word. Understanding particles is an important element in understanding idioms and constructing them. At this point in IMABI, this should not be a problem. If it is a problem, you will have many problems as particle usage can greatly alter the meaning intended in an idiomatic expression. }

\par{It is also important to realize that many idioms are only off by transitivity orientation. The nuance given off is really the only thing, then, that typically changes. First person is normally shown with transitive expressions and intransitive expressions show some sort of observation. }

\par{Something that is hard to decipher, especially in text, is the sense that an idiom is being used in. Here, context decides. For example, if you go to a jail and set everyone free, you would say 自由にした. But, if you were instead an interrogator of the prisoners, you would have most likely meant that you had them all at your mercy. }

\par{Lastly, do not be completely overwhelmed by the expressions that are basically completely different from their literal meanings. You will just have to sit, think, study, and learn them. }

\par{ In the chart below there are some of the many common idioms that are used in Japanese. Take note of their particle usages, what kind of noun and verb combinations that are made, and determine the degree of 'idiomaticity' there is between the literal and idiomatic definitions. Following the chart, there will be several example sentences to give you the cultural background to better apply them in your speech. }
      
\section{Idioms}
 
\begin{ltabulary}{|P|P|P|P|}
\hline 

 &  & Meaning & Literal Meaning \\ \cline{1-4}

火に油を注ぐ & ひにあぶらをそそぐ & Add fuel to the fire & Pour grease in a fire \\ \cline{1-4}

転ばぬ先の杖 & ころばぬさきのつえ & Look before you leap & Twig of point that doesn't fall \\ \cline{1-4}

金看板を掛ける & きんかんばんをかける & To assume importance & To hang a billboard with gold \hfill\break
\\ \cline{1-4}

草を結ぶ & くさをむすぶ & To return a favor & To bind grass \\ \cline{1-4}

声が詰まる & こえがつまる & To speak in a choked voice & For the voice to choke \\ \cline{1-4}

声を曇らす & こえをくもらす & To falter out & To cloud the voice \\ \cline{1-4}

声を呑む & こえをのむ & To swallow one's words & To swallow one's voice \\ \cline{1-4}

声を立てる & こえをたてる & To cry out & To raise one's voice \\ \cline{1-4}

暮しに困る & くらしにこまる & To be in financial trouble & To be troubled in livelihood \\ \cline{1-4}

暮しを立てる & くらしをたてる & To make a living & To raise a living \\ \cline{1-4}

芸が立つ & げいがたつ & To be a master of the arts & For skill to stand \\ \cline{1-4}

芸は身を助く & げいはみをたすく & Art brings bread & A skill will save yourself \\ \cline{1-4}

蔵が建つ & くらがたつ & To become a millionaire & A storage house to be built \\ \cline{1-4}

車に切る & くるまに切る & To cut clockwise & To cut in a car \\ \cline{1-4}

癖を直す & くせをなおす & To break a habit & To fix a habit \\ \cline{1-4}

記録に載る & きろくにのる & To be recorded & To appear in the records \\ \cline{1-4}

義理がある & ぎりがある & To be bound by duty & To have duty \\ \cline{1-4}

看板を下す & かんばんをおろす & To close down shop & To take down a billboard \\ \cline{1-4}

肝胆を出す & かんたんをいだす & To do with devotion & To show one's inner being \\ \cline{1-4}

慣例を残す & かんれいをのこす & To set precedent & To leave behind a precedent \\ \cline{1-4}

勘定を留める & かんじょうをとめる & To run up bills & To pile up bills \\ \cline{1-4}

嘴が黄色い & くちばしがきいろい & To be immature & For the beak to be yellow \\ \cline{1-4}

嘴を入れる & くちばしをいれる & To interfere with & To put in one's beak \\ \cline{1-4}

嘴を鳴らす & くちばしをならす & To babble about \hfill\break
& To sound one's beak \\ \cline{1-4}

気が焦る & きがあせる & To be impatient & For the mind to be in a hurry \\ \cline{1-4}

気が荒い & きがあらい & To be quarrelsome & For the mind to be violent \\ \cline{1-4}

気が進む & きがすすむ & To feel like doing & For the mind to advance \\ \cline{1-4}

気が軽い & きがかるい & To be sociable \hfill\break
& For the mind to be light \\ \cline{1-4}

気が座る & きがすわる & To be at ease & For the mind to sit \\ \cline{1-4}

気が置ける & きがおける & To feel ill at ease & For the mind to be place-able \hfill\break
\\ \cline{1-4}

気がそれる & きがそれる & To be distracted & For the mind to divert \\ \cline{1-4}

気が多い & きがおおい & To be fickle & To have many minds \\ \cline{1-4}

気が大きい & きがおおきい & To be generous & To have a big mind \\ \cline{1-4}

気が腐る & きがくさる & To be dejected & For one's mind to rot \\ \cline{1-4}

気に持つ & きにもつ & To weigh on one's mind & To hold in one's mind \\ \cline{1-4}

雲を凌ぐ & くもをしのぐ & To rise over the clouds & To gain advantage of clouds \\ \cline{1-4}

組みになる & くみになる & To join forces & To become a group \\ \cline{1-4}

組みを選ぶ & くみをえらぶ & To choose sides & To choose groups \\ \cline{1-4}

訓練が行届く & くんれんがいきとどく & To be well-trained & For training to be well-kept \\ \cline{1-4}

先がある & さきがある & To have potential & To have a future \\ \cline{1-4}

先に立つ & さきにたつ & To be in the lead & To stand ahead \\ \cline{1-4}

先を読む & さきをよむ & To look into the future & To read the future \\ \cline{1-4}

気が戻る & きがもどる & To be turned off & For the mind to return \\ \cline{1-4}

酒に痛む & さけにいたむ & To get dead drunk & To ache in liquor \\ \cline{1-4}

酒に回される & さけにまわされる & To lose oneself to liquor & To be winded in liquor \\ \cline{1-4}

酒を使う & さけをつかう & To be under the influence & To use liquor \\ \cline{1-4}

肉が落ちる & にくがおちる & To lose weight & For meat to drop \\ \cline{1-4}

匙を投げる & さじをなげる & To throw in the towel & To throw a spoon \\ \cline{1-4}

鯖を読む & さばをよむ & To cheat in counting & To read the mackerel \\ \cline{1-4}

最期を遂げる & さいごをとげる & To die a pitiful death & To achieve one's latter end \hfill\break
\\ \cline{1-4}

策に富む & さくにとむ & To be resourceful & To be rich in measures \\ \cline{1-4}

策を弄する & さくをろうする & To use artifice & To play with measures \\ \cline{1-4}

工夫を凝らす & くふうをこらす & To work out a plan & To concentrate devices \\ \cline{1-4}

気に留める & きにとめる & To keep in mind & To keep in the mind \\ \cline{1-4}

災難を免れる & さいなんをまぬかれる & To avoid a disaster & To avoid misfortune \\ \cline{1-4}

財布を叩く & さいふをたたく & To empty one's purse & To hit one's purse \\ \cline{1-4}

財布を満たす & さいふをみたす & To fill one's purse & To fill one's purse \\ \cline{1-4}

構想を練る & こうそうをねる & To rack one's brains & To draw up a framework \\ \cline{1-4}

座を冷ます \hfill\break
& ざをさます & To ruin the mood & To cool the seat \\ \cline{1-4}

才に溺れる & さいにおぼれる & To rely heavily on talent & To drown in ability \\ \cline{1-4}

我を忘れる & われをわすれる & To get carried away & To forget oneself \\ \cline{1-4}

割符が合う & わりふがあう & To meet eye to eye & To match tallies \\ \cline{1-4}

草鞋を剥ぐ & わらじをはぐ & To end one's journey & To take off one's sandals \hfill\break
\\ \cline{1-4}

利に走る & りにはしる & To be eager to make profit & To run to profits \\ \cline{1-4}

巧言を用いる & こうげんをもちいる & To flatter & To use flatter \\ \cline{1-4}

涎が出る & よだれがでる & To be delicious & To begin to drool \hfill\break
\\ \cline{1-4}

運に任せる & うんにまかせる & Trust to Providence & To entrust in destiny \\ \cline{1-4}

気前がいい & きまえがいい & To have an open hand & To have good generosity \\ \cline{1-4}

歓心を買う & かんしんをかう & To buy favor & To buy favor \\ \cline{1-4}

口火となる & くちびとなる & To trigger something & To become the spark \\ \cline{1-4}

口火を切る & くちびをきる & To spark something & To cut the spark \\ \cline{1-4}

攻撃を防ぐ & こうげきをふせぐ & To defend against an attack & To prevent an attack \\ \cline{1-4}

看板が泣く & かんばんがなく & Not true to one's name & For the billboard to cry \\ \cline{1-4}

冠を曲げる & かんむりをまげる & To take offense & To bend a crown \\ \cline{1-4}

気が急く & きがせく & To feel hard pressed & For the mind to be hurried \\ \cline{1-4}

犠牲を払う & ぎせいをはらう & To make a sacrifice & To pay a sacrifice \\ \cline{1-4}

気は心 & きはこころ & It's the thought that counts & The mind's the heart \\ \cline{1-4}

気が重い & きがおもい & To feel depressed & For the mind to be heavy \\ \cline{1-4}

気が散る & きがちる & To be distracted & For the mind to be scattered \\ \cline{1-4}

気が乗らない & きがのらない & To not be in the mood & For the mind to not be riding \\ \cline{1-4}

火事に遭う & かじにあう & To be in a fire & To encounter a fire \\ \cline{1-4}

口車に乗る & くちぐるまにのる & To be cajoled into something & To ride a cajoler \\ \cline{1-4}

傘に乗る & かさにのる & To be carried away & To ride an umbrella \\ \cline{1-4}

過去に生きる & かこにいきる & To live in the past & To live in the past \\ \cline{1-4}

仇を成す & あだをなす & To make enemies & To give birth to enemies \\ \cline{1-4}

案に落つ & あんにおつ & To go according to plan & To fall into a plan \\ \cline{1-4}

左右に托する & さゆうにたくする & To dodge an issue & To make excuses left and right \\ \cline{1-4}

授業を休む & じゅぎょうをやすむ & To miss a class & To take a rest from class \\ \cline{1-4}

処置に窮する & しょちにきゅうする & To be at a loss & For a measure to be a loss \\ \cline{1-4}

塵界を脱する & じんかいをだっする & To retire from the world & To get out of a dirty world \hfill\break
\\ \cline{1-4}

気を引く & きをひく & To rouse excitement & To draw in minds \\ \cline{1-4}

奇跡を現す & きせきをあらわす & To achieve a miracle & To reveal a miracle \\ \cline{1-4}

期待に添う & きたいにそう & To meet expectations & To live up to expectations \\ \cline{1-4}

隙に乗じる & すきにじょうじる & To catch off guard & To take advantage of gaps \\ \cline{1-4}

逃げを打つ & にげをうつ & To attempt to escape & To hit an escape \\ \cline{1-4}

気を揉む & きをもむ & To be anxious about & To worry the mind \\ \cline{1-4}

恥を掻く & はじをかく & To be ashamed \hfill\break
& To scratch one's shame \\ \cline{1-4}

鼻薬を嗅がせる & はなぐすりをかがせる & To offer a bribe & To make\dothyp{}\dothyp{}\dothyp{}smell nasal spray \\ \cline{1-4}

幅に成る & はばになる & To gain prestige & To become width \\ \cline{1-4}

百計が尽きる & ひゃっけいがつきる & To be at the end of the rope & To exhaust all means \\ \cline{1-4}

顰蹙を買う & ひんしゅくをかう & To be frowned upon & To buy frowning on \\ \cline{1-4}

不信を抱く & ふしんをいだく & To have a suspicion & To hold a distrust \\ \cline{1-4}

管を巻く & くだをまく & To blurt out something & To wind a pipe \\ \cline{1-4}

風致を害する & ふうちをがいする & To spoil the view & To damage the scenic beauty \\ \cline{1-4}

武を争う & ぶをあらそう & To struggle for supremacy & To fight martial affairs \\ \cline{1-4}

風呂を落とす & ふろをおとす & To empty a bathtub & To drop a bathtub \\ \cline{1-4}

狐が落ちる & きつねがおちる & To come to one's senses & The fox falls \\ \cline{1-4}

狐に摘まされる & きつねにつままされる & To be baffled & To be caught by the fox \\ \cline{1-4}

不平を並べる & ふへいをならべる & To whine over & To line up dissatisfaction \\ \cline{1-4}

不評を買う & ふひょうをかう & To lose popularity & To buy a bad reputation \\ \cline{1-4}

枕を重ねる & まくらをかさねる & To sleep together regularly & To stack up pillows \\ \cline{1-4}

枕を砕く & まくらをくだく & To fret over & To break a pillow \hfill\break
\\ \cline{1-4}

身を誤る & みをあやまる & To go astray & To misjudge the body \\ \cline{1-4}

見切りで買う & みきりでかう & To buy at a reduced price & To buy with abandonment \\ \cline{1-4}

虫の居所が悪い & むしのいどころがわるい & To be in a bad mood & The the bug is in a bad spot. \\ \cline{1-4}

虫が良すぎる & むしがよすぎる & To ask for too much & For a bug to be too good \\ \cline{1-4}

肘鉄砲を食う & ひじでっぽうをくう & To get snubbed & To eat at a rebuff \\ \cline{1-4}

馬力がある & ばりきがある & To have stamina & To have horse power \\ \cline{1-4}

日が浅い & ひがあさい & To be only recent & For the day to be shallow \\ \cline{1-4}

日を消す & ひをけす & To spend one's time & To erase the day \\ \cline{1-4}

火を被る & ひをかぶる & To be overcome with grief & To wear fire \\ \cline{1-4}

舞台を踏む & ぶたいをふむ & To make one's debut & To step on stage \\ \cline{1-4}

平気を装う & へいきをよそおう & To keep one's head & To put on calmness \\ \cline{1-4}

気を負う & きをおう & To be eager & To bear the mind \\ \cline{1-4}

芸がない & げいがない & To be good for nothing & To have no art\slash skill \\ \cline{1-4}

声を殺す & こえをころす & To talk in a whisper & To kill one's voice \\ \cline{1-4}

\end{ltabulary}
      
\section{Example Sentences}
 
\par{ Despite that idiomatic phrases are typically stand-alone phrases that can be and are understood in isolation, it is helpful to see context with these phrases. Do not be confused with syntax as nothing out of the ordinary was shown. If you must, get familiar with the literal definitions to think of the phrases. }

\par{1. おやつで腹の虫を抑えたらどうや。(Dialectical) \hfill\break
What do you think about easing your emotions with a little snack? }

\par{2. なんとなくあいつ、虫が好かねぇ。(Vulgar) \hfill\break
For some reason I just don't like that guy. }

\par{3. うちの家はよく日が当たります。 \hfill\break
My house gets a lot of sun. }

\par{4. 火のないところに煙はたたぬ。(Slightly old-fashioned) \hfill\break
There's no smoke without fire. }

\par{5. 日をみるより明らかな問題だと強調しております。(Humble) \hfill\break
I'm stressing that it is a problem as clear as day. }

\par{6. 社会で幅をきかせている。 \hfill\break
To be having a big influence and becoming prestigious in society. }

\par{7. 看板を下ろすとは廃業して店をたたむということである。(改まった) \hfill\break
”Kanban wo orosu" is to shut down a shop in discontinuing a business. }

\par{8. 塵界を脱して逃れた方がましだ。 \hfill\break
I'd rather retire and run away from the hustle and bustle of this world. }

\par{9. 5歳ほど鯖を読むのにFacebookアカウントを作る子供が多いそうです。 \hfill\break
There are supposedly a lot of kids that make a Facebook account that edge their age by a little 5 years. }

\par{10. 彼はいつも自分の思い通りじゃないと気がすまない嫌いがある。 \hfill\break
He has the tendency to always want his way. }

\par{11. コンピューターがついていると気が散ってちっとも勉強できないんだ。 \hfill\break
I can't study at all when the computer is on. }
    