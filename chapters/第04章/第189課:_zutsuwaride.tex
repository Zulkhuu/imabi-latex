    
\chapter{ずつ \& わりで}

\begin{center}
\begin{Large}
第189課: ずつ \& わりで 
\end{Large}
\end{center}
 
\par{ Allocation and proportion is a little tricky. Although these expressions aren't quite different from their English equivalents, there are a few differences that you will need to pay extra attention to. }
      
\section{The Adverbial Particle ずつ}
 
\par{ ずつ is a rather straightforward particle that splits things up into groups, creating ratios. Say you have X number of kids and Y number of mothers to watch the kids. Say there are 100 kids, but you want a chaperon every four kids. To tell the mothers this, you could say 子供4人に1人ずつ、お母さんがついてください。You can do the math as to how many mothers there are. }

\par{ In 1, of all the 100 kids now in groups of 10, two mothers are allotted to each. }

\par{1. 子供10人に、2ずつ、お母さんがついてください? \hfill\break
May two mothers follow with every ten kids? }

\par{ Think of it as grouping things from the start of a line\slash process until the end. It is implied that the allocation is repeated so that the intended ratio is carried out in turn. }

\par{2. この20冊の雑誌をひとりひとりに一冊ずつ配ってください。 \hfill\break
Please pass out to each and every person one of these twenty magazines. }

\par{ In English, it is more common for someone to say “give three pieces of paper to each kid” than “give three pieces of paper to every ten kids”. We want to make things grammatically singular in number and say “to each \textbf{group }of ten kids”. Saying group is not that necessary in Japanese because ずつ does that already. }

\par{ You can, though, paraphrase ずつ out of the sentence. Compare and contrast the following sets of sentences. }

\par{3a. カードはひとり15枚ずつですよ。残りは、ここに伏せておきましょう。 \hfill\break
3b. ひとりあたり15枚になるように、カードをみんなに配っていきます。 \hfill\break
3a. It'll be fifteen cards to each person. We'll put the rest face down here. \hfill\break
3b. So there will be fifteen cards for each person, we'll pass out the cards to everyone. }

\par{4a. 教科書は2冊ずつ、小説は10冊ずつ、雑誌は20冊ずつ、3本の ${\overset{\textnormal{ひも}}{\text{紐}}}$ で ${\overset{\textnormal{くく}}{\text{括}}}$ ってください。 \hfill\break
4b. 1 ${\overset{\textnormal{くく}}{\text{括}}}$ りにつき、教科書が2冊、小説が10冊、雑誌が20冊になるように、3本の紐で ${\overset{\textnormal{しば}}{\text{縛}}}$ ってください。 \hfill\break
4a. Bind up textbooks in groups of two, novels in groups of ten, and magazines in groups of twenty with three cords. \hfill\break
4b. To make bundles, bind so that textbooks are in sets of two, novels in ten, and magazines in twenty with three cords. }

\par{ As you can see, all ずつ does is mark how much the identified recipient(s) are going to get. Some expressions that you will constantly see include 少しずつ (little by little) and わずかずつ (similar to the first but smaller in degree). }

\par{5. それらを二個ずつ ${\overset{\textnormal{そろ}}{\text{揃}}}$ える。 \hfill\break
To arrange them in twos. }

\par{6. 子供が ${\overset{\textnormal{あめ}}{\text{飴}}}$ を二つずつもらいます。 \hfill\break
Kids will receive two candies each. }

\par{7. 少しずつ食べた方がいい。 \hfill\break
It's best to eat a little bit at a time. }

\par{8. 彼女は少しずつ回復しました。 \hfill\break
She recovered little by little. }

\par{9. 2枚ずつ下さい。 \hfill\break
Two sheets each please. }

\par{10. 一人ずつバスに乗りなさい。 \hfill\break
Please enter the bus one by one. }

\par{11. 毎日数ドルずつ ${\overset{\textnormal{たくわ}}{\text{貯}}}$ えた。 \hfill\break
He put aside a few dollars every day. }

\par{12. 昨日は ${\overset{\textnormal{かわ}}{\text{河}}}$ の ${\overset{\textnormal{すいい}}{\text{水位}}}$ が少しずつ上がっていた。 \hfill\break
The river water level rose little by little yesterday. }

\par{13. これらの単語を ${\overset{\textnormal{ひとこと}}{\text{一言}}}$ ずつ覚えてください。 \hfill\break
Please learn these words one word at a time. }

\par{14. 私は事の ${\overset{\textnormal{しんそう}}{\text{真相}}}$ が少しずつ分かってきました。 \hfill\break
I came to understand the bottom of it little by little. }

\par{\textbf{Orthography Note }: づつ is also correct but old-fashioned. }
      
\section{わりで}
 
\par{ ~わりで has some interchangeability withずつ. Rather than being involved with the flow of work or time, this pattern just shows a rate\slash proportion. So, if the numbers of the whole situation are not certainly known, you can\textquotesingle t use ~わりで. In spoken speech, however, ずつ and ~わりで are often omitted out of the sentence. }

\par{15. 1時間に10マイル\{のわりで・ずつ\}、いつ目的地に着きますか。 \hfill\break
At ten miles an hour, when will you arrive at your destination? }

\par{ ずつ still gets used in math texts, but because it needs context to be understood clearly, it is usually limited to the spoken language. ~わりで, on the other hand, clearly states things in more mathematical terms, so it is more indicative of the written language. }
    