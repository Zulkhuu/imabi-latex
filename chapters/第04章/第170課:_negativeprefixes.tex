    
\chapter{Negative Prefixes}

\begin{center}
\begin{Large}
第170課: Negative Prefixes: 未, 無, 非, 否, \& 不 
\end{Large}
\end{center}
 
\par{ Prefixes are cumbersome in Japanese as they are in English. In English when we think of negative prefixes, we think of un-, non-, a-, in-, de-, etc. If distinguishing between all these is difficult, then you might be relieved to know that Japanese isn\textquotesingle t as complex in this regard. Although there is some truth in having to learn when they\textquotesingle re used on a case-by-case basis, differentiating between 未, 無, 非, 否, and 不 is not all that troublesome. }

\par{ Although there are no rules that determine which of these prefixes are to be used, corresponding these endings with the following Japanese and English keywords will help you tremendously in understanding how they're used. }

\begin{ltabulary}{|P|P|P|}
\hline 

未 & まだ…ない & Incompletion \\ \cline{1-3}

無 & ない・存在しない & Absence \\ \cline{1-3}

非 & しない・ではない & Unjustifiability \\ \cline{1-3}

否 & 同意しない & Noncompliance \\ \cline{1-3}

不 & ではない & Simple Negation \\ \cline{1-3}

\end{ltabulary}

\par{Another thing to keep in mind about these prefixes is that although they are being referred to as prefixes, it is not a 100\% guarantee that what they attach to can be used in isolation as independent words. This is because they are being viewed as prefixes based on how they are used from a Chinese language perspective. For instance, anxiety in Japanese is 不安. 安, however, is not used as the antonym of 不安.  Instead, the antonym of 不安 is 安心. These are simply quirks that you will need to become accustomed to as you learn words with these prefixes. }
      
\section{未: Incompletion}
 
\par{ Simply put, this prefix indicates incompletion. Meaning, something isn\textquotesingle t quite so yet, but it will (most likely) be so in the future. }

\par{1. ${\overset{\textnormal{みじゅく}}{\text{未熟}}}$ なだけに、かえって ${\overset{\textnormal{ひがい}}{\text{被害}}}$ を起こす可能性が高いのです。 \hfill\break
Precisely because (X) is inexperienced, the possibility of him causing damage is all the more high. }

\par{2. きのう、 ${\overset{\textnormal{みこうかい}}{\text{未公開}}}$ の映画を ${\overset{\textnormal{み}}{\text{観}}}$ ました。 \hfill\break
I watched an unreleased movie yesterday. }

\par{3. ${\overset{\textnormal{みかいけつ}}{\text{未解決}}}$ の ${\overset{\textnormal{じけん}}{\text{事件}}}$ を ${\overset{\textnormal{まと}}{\text{纏}}}$ めました。 \hfill\break
I have compiled unresolved cases. }

\par{4. ${\overset{\textnormal{むすこ}}{\text{息子}}}$ が ${\overset{\textnormal{みせいねんいんしゅ}}{\text{未成年飲酒}}}$ をしたということで ${\overset{\textnormal{けいさつ}}{\text{警察}}}$ に ${\overset{\textnormal{つ}}{\text{連}}}$ れていかれました。 \hfill\break
My son was taken by police for underage drinking. }

\par{5. もしも ${\overset{\textnormal{みぼうじん}}{\text{未亡人}}}$ になったら、 ${\overset{\textnormal{く}}{\text{暮}}}$ らし ${\overset{\textnormal{かた}}{\text{方}}}$ は今までと違うのか。 \hfill\break
If you were to become a widow, how would the way you live differ from what it is now? }

\par{6. ${\overset{\textnormal{みけいけんしゃ}}{\text{未経験者}}}$ ばかりのオフィスで働いています。 }

\par{7. 人間は ${\overset{\textnormal{だれ}}{\text{誰}}}$ もが ${\overset{\textnormal{みかんせい}}{\text{未完成}}}$ だ。 \hfill\break
All people are incomplete. }

\par{8. もしかしたら ${\overset{\textnormal{みかくにんひこうぶったい}}{\text{未確認飛行物体}}}$ を見たかもしれない。 \hfill\break
I may have possibly seen an unidentified flying object. }

\par{9. ${\overset{\textnormal{たいけんばん}}{\text{体験版}}}$ を ${\overset{\textnormal{つう}}{\text{通}}}$ じて、たくさんの ${\overset{\textnormal{みはっぴょう}}{\text{未発表}}}$ のポケモンがリークされてしまった。 \hfill\break
Many unrevealed Pokemon were accidentally leaked through the demo version. }

\par{10. ${\overset{\textnormal{みかいはつしじょう}}{\text{未開発市場}}}$ に乗り出す。 \hfill\break
To set out in undeveloped markets. }

\par{11. 10歳 ${\overset{\textnormal{みまん}}{\text{未満}}}$ の子供を ${\overset{\textnormal{たいしょう}}{\text{対象}}}$ としています。 \hfill\break
We are targeting children under ten. }

\par{12. わたしは ${\overset{\textnormal{みこん}}{\text{未婚}}}$ ですが、子供が欲しいです。 \hfill\break
I'm unmarried, but I want a child. }
      
\section{無: Absence}
 
\par{ Attaches to things that are nouns (in Chinese and as an effect can be viewed as nouns in Japanese). It may result in an adjective or adverb on a case-by-case basis, though. This prefix indicates total lack. It typically only attaches to Sino-Japanese words. When dealing with non-Sino-Japanese words, ~なし is preferred (ex. 底なし = bottomless). }

\par{13. ${\overset{\textnormal{むじんとう}}{\text{無人島}}}$ に住みたいと思っています。 \hfill\break
I want to live on an uninhabited island. }

\par{14. ${\overset{\textnormal{かいいんとうろく}}{\text{会員登録}}}$ は ${\overset{\textnormal{むりょう}}{\text{無料}}}$ ですか。 \hfill\break
Is membership registration free? }

\par{15. 一人で泣いたって無意味だよ。 \hfill\break
It's meaningless to cry by yourself. }

\par{16. 可能性は ${\overset{\textnormal{かいむ}}{\text{皆無}}}$ に ${\overset{\textnormal{ひと}}{\text{等}}}$ しい。 \hfill\break
The likelihood is near nothing. }

\par{17. ${\overset{\textnormal{よなか}}{\text{夜中}}}$ に ${\overset{\textnormal{むすう}}{\text{無数}}}$ の ${\overset{\textnormal{ほしぼし}}{\text{星々}}}$ を ${\overset{\textnormal{なが}}{\text{眺}}}$ める。 \hfill\break
To gaze at the endless stars in the middle of the night. }

\par{18. ${\overset{\textnormal{むろん}}{\text{無論}}}$ 、その通りです。 \hfill\break
Of course, that's exactly the case. }

\par{19. ${\overset{\textnormal{むし}}{\text{無視}}}$ されたくないです。 \hfill\break
I don't like being ignored. }

\par{20. ジンバブエは ${\overset{\textnormal{むほうちたい}}{\text{無法地帯}}}$ のソマリアよりも ${\overset{\textnormal{ちあん}}{\text{治安}}}$ が悪いと聞いています。 \hfill\break
I'm hearing that public order in Zimbabwe is worse that lawless Somalia. }

\par{21. ${\overset{\textnormal{むしょく}}{\text{無職}}}$ の男性が ${\overset{\textnormal{たいほ}}{\text{逮捕}}}$ されました。 \hfill\break
An unemployed male was arrested. }

\par{22. 彼女は ${\overset{\textnormal{むひょうじょう}}{\text{無表情}}}$ な顔をして ${\overset{\textnormal{だま}}{\text{黙}}}$ っていた。 \hfill\break
Her face was expressionless as she stayed quiet. }

\par{23. ${\overset{\textnormal{むいしき}}{\text{無意識}}}$ に ${\overset{\textnormal{は}}{\text{歯}}}$ を ${\overset{\textnormal{く}}{\text{食}}}$ いしばる ${\overset{\textnormal{くせ}}{\text{癖}}}$ を治したい。 \hfill\break
I want to fix my habit of unconsciously grinding my teeth. }

\par{24. ${\overset{\textnormal{むかんしん}}{\text{無関心}}}$ な相手に ${\overset{\textnormal{たい}}{\text{対}}}$ しては ${\overset{\textnormal{いか}}{\text{怒}}}$ りを感じません。 \hfill\break
I don't feel anger towards those who are uninterested. }

\begin{center}
\textbf{In 当て字 }
\end{center}

\par{ Although not an example of the prefix, you will also see the character 無 found in the 熟字訓 reading of the native word for “fig,” which is イチジク. This is because figs, despite having many stamen and pistils, they are not visible from the outside, thus the spelling  無花果. }
      
\section{非: Unjustifiability}
 
\par{ This prefix indicates that something is not belonging to a certain state of being. It is used to show unjustifiability. Meaning, whatever it is referring to, there is a state in which it ought to be but this is the negation of it being so. More emotion, as an effect, can be seen in many examples of it. }

\par{不合理:  Simple negation of something being rational. }

\par{非合理: Emotional appeal to something being irrational\slash illogical. }

\par{ It usually attaches to nouns but may result in adjectives depending on the phrase. It can attach to Sino-Japanese, native, and foreign words. }

\par{25. ${\overset{\textnormal{ひじょうしき}}{\text{非常識}}}$ な妻に息子を育てさせるのは絶対に嫌です。 \hfill\break
I absolutely don't want to let my senseless wife raise our son. }

\par{26. ${\overset{\textnormal{ひわ}}{\text{非割}}}$ り ${\overset{\textnormal{こ}}{\text{込}}}$ み ${\overset{\textnormal{がたゆうせんどじゅん}}{\text{型優先度順}}}$ サービス \hfill\break
Non-preemptive priority service }

\par{27. ${\overset{\textnormal{ひ}}{\text{非}}}$ アフリカ ${\overset{\textnormal{けい}}{\text{系}}}$ の ${\overset{\textnormal{げんだいじん}}{\text{現代人}}}$ が、DNAの1~4\%をネアンデルタール人から受け ${\overset{\textnormal{つ}}{\text{継}}}$ いでいる。 \hfill\break
Non-African descent modern humans inherit 1~4\% of their DNA from Neanderthals. }

\par{28. 現実の世界から非現実の世界に入っていってしまう。 \hfill\break
To end up entering an unreal world from the real world. }

\par{29. ${\overset{\textnormal{ひかがくてき}}{\text{非科学的}}}$ な ${\overset{\textnormal{はつげん}}{\text{発言}}}$ が大嫌いだ。 \hfill\break
I hate nonscientific remarks. }

\par{30. このページは ${\overset{\textnormal{ひひょうじ}}{\text{非表示}}}$ になっています。 \hfill\break
This page does not display. }

\par{31. これからは ${\overset{\textnormal{ひこん}}{\text{非婚}}}$ の男性が増えるでしょう。 \hfill\break
Men who aren't married (and don't wish to) will increase from now on. }

\par{32. ${\overset{\textnormal{へいじつ}}{\text{平日}}}$ は ${\overset{\textnormal{ひじょうきんしょくいん}}{\text{非常勤職員}}}$ として働いています。 \hfill\break
On week days I work as a part-time worker. }

\par{33. ${\overset{\textnormal{じょうおん}}{\text{常温}}}$ で ${\overset{\textnormal{こたい}}{\text{固体}}}$ の ${\overset{\textnormal{ひきんぞくげんそ}}{\text{非金属元素}}}$ といえば、ケイ素などあります。 \hfill\break
In speaking of non-metal elements that are solid at room temperature, there is silicon. }
      
\section{否: Noncompliance}
   This prefix is not near as common as any of the previous ones, but when it is used, it indicates a meaning of noncompliance. \hfill\break

\par{34. その ${\overset{\textnormal{じじつ}}{\text{事実}}}$ を ${\overset{\textnormal{ひてい}}{\text{否定}}}$ することはできないし、否定しても意味がない。 \hfill\break
You can't even deny the truth, and even if you were to, there would be no point. }

\par{35. ${\overset{\textnormal{ぎかい}}{\text{議会}}}$ に ${\overset{\textnormal{ひけつ}}{\text{否決}}}$ されたらどうなるのか。 \hfill\break
What happens if it's rejected by the assembly? }

\par{36. ${\overset{\textnormal{ひこく}}{\text{被告}}}$ は ${\overset{\textnormal{ようぎ}}{\text{容疑}}}$ を ${\overset{\textnormal{ひにん}}{\text{否認}}}$ しています。 \hfill\break
The defendant is denying the allegation. }
      
\section{不: Simple Negation}
 
\par{ This prefix attaches to adjectives and verbs to show simple negation. It may not be absolute negation, though. For instance, it may indicate stagnation in some sense. It typically follows nouns to negate an action, and it typically follows 形容動詞 to negate a state. }

\par{37. 人は、気分がいいときは健康な食べ物を、ストレスを感じているときは不健康な食べ物を選ぶ ${\overset{\textnormal{けいこう}}{\text{傾向}}}$ がある。 \hfill\break
People have the tendency of choosing healthy foods when they feel good and unhealthy foods when they're under stress. }

\par{38. ${\overset{\textnormal{さきほど}}{\text{先程}}}$ お送りしたメールに ${\overset{\textnormal{ふび}}{\text{不備}}}$ がありました。 \hfill\break
There was a fault in the e-mail that I had just sent you. }

\par{39. ${\overset{\textnormal{あくさい}}{\text{悪妻}}}$ は百年の ${\overset{\textnormal{ふさく}}{\text{不作}}}$ 。 \hfill\break
A bad wife leads to 100 years of failure. }

\par{40. 今年は ${\overset{\textnormal{ふきょう}}{\text{不況}}}$ のため ${\overset{\textnormal{さいよう}}{\text{採用}}}$ を ${\overset{\textnormal{ひか}}{\text{控}}}$ えます。 \hfill\break
We are holding off on hiring this year due to recession. }

\par{41. ${\overset{\textnormal{ろうがん}}{\text{老眼}}}$ は不便だね。 \hfill\break
Farsightedness due to old age is inconvenient, you know. }

\par{42. 海外で ${\overset{\textnormal{ゆくえふめい}}{\text{行方不明}}}$ になる。 \hfill\break
To go missing overseas. }

\begin{center}
 \textbf{不 +Native Words }
\end{center}

\begin{center}
不 has been around long enough to follow a select number of native words. 
\end{center}

\par{43. ${\overset{\textnormal{ふゆ}}{\text{不行}}}$ き ${\overset{\textnormal{とど}}{\text{届}}}$ きの点は許してください。 \hfill\break
Please forgive the carelessness. }

\par{44. ${\overset{\textnormal{てう}}{\text{手打}}}$ ちで ${\overset{\textnormal{ふぞろ}}{\text{不揃}}}$ いの ${\overset{\textnormal{めん}}{\text{麺}}}$ が多い。 \hfill\break
There are many uneven noodles made by hand. }

\par{45. なぜ ${\overset{\textnormal{ふまじめ}}{\text{不真面目}}}$ な人ほど、 ${\overset{\textnormal{しゅっせ}}{\text{出世}}}$ するのか。 \hfill\break
Why is it that the less serious someone is the more he advances in life? }

\begin{center}
\textbf{不・無(ぶ): Ill-representation }
\end{center}

\par{ This is yet another prefix which can be written as either 不 or 無 with no true rules as to when you should use which aside from standard convention on a case-by-case basis. Most examples of this involve 形容動詞 as equivalents of  the English prefixes “mis-“ and “ill-.” }

\par{46. ${\overset{\textnormal{てさき}}{\text{手先}}}$ が ${\overset{\textnormal{ぶきよう}}{\text{不器用}}}$ すぎる。 \hfill\break
I'm too clumsy with my hands. }

\par{47. ${\overset{\textnormal{ぶあいそう}}{\text{無愛想}}}$ な顔をしていると、相手も無愛想になる。 \hfill\break
Whenever you have a blunt look on your face, your opponent will also become blunt. }

\par{48. 顔が ${\overset{\textnormal{ぶさいく}}{\text{不細工}}}$ な人と付き合えますか。 \hfill\break
Are you capable of dating someone with an ugly face? }
    