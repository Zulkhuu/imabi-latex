    
\chapter{Onomatopoeia I}

\begin{center}
\begin{Large}
第155課: Onomatopoeia I: Giongo 擬音語  
\end{Large}
\end{center}
 
\par{ Japanese has a lot of onomatopoeic words that not only describe sound but also physical and mental states. Unlike English, they are more numerous and found in all sorts of speech for reasons you will learn in this lesson. These words are often hard to translate, but don't let this be a problem for you. }
      
\section{擬音語}
 
\par{ The definition of a ${\overset{\textnormal{}}{\text{擬音語}}}$ is directly tied to what it's used with. Most onomatopoeic expressions in Japanese have several usages, and not all might fall into the same category. Because we are only dealing with 擬音語 in this lesson, we will not see usages of any word introduced that fall out of this category. }

\begin{center}
 \textbf{Common 擬音語 }
\end{center}

\begin{ltabulary}{|P|P|P|P|}
\hline 

ぺこぺこ(と・に) & Being hungry & がらがら(と・に) & Rattling \\ \cline{1-4}

どきどき(と) & Heart beating & しくしく(と) & Silently (weeping) \hfill\break
\\ \cline{1-4}

ぺらぺら(と) & Fluently & ざーざー(と) & Raining very hard \\ \cline{1-4}

\end{ltabulary}
 
\begin{enumerate}
 
\item \textbf{ぺこぺこ }: With と it refers to being servile or dented.  
\item \textbf{がらがら }: When  followed by と, it refers to the sound      of something solid crashing, a tire      revolving, gargling, clattering, or rattling.  
\item \textbf{ぺらぺら(と) }: This may also be the sound of flipping pages.  
\item \textbf{ざーざー }: It can also refer to the sound of radio static. 
\end{enumerate}
  
\par{\textbf{Part of Speech Note }: Some verbs are based off of onomatopoeia. Ex.  はためく (to flutter) }

\par{\textbf{Voicing Note }: Voiced onomatopoeia often have a more serious or dramatic tone to them versus their very similar non-voiced counterparts. They are often antonymous. For instance, からから can refer to clattering, but がらがら can refer to something solid crashing or really loud clattering (at the least). }

\begin{center}
 \textbf{Various Realization }
\end{center}

\par{ Let's say that a common property of onomatopoeic expressions is that there is a root. This root can be doubled and result in something like しくしく.  Now, not all onomatopoeia will have as many possible forms as others. So, you should learn onomatopoeia one at a time, but you can always look to see if a certain form exists. }

\par{ To look at the wide variety of things that can happen, we will use コロ (sound of something rolling) as an example. }

\begin{ltabulary}{|P|P|P|}
\hline 

コロッ(と) & Insertion of ッ after root & Looks like it's going to roll \\ \cline{1-3}

コロン & Insertion of ン after root & Bounces back and rolls \\ \cline{1-3}

コロリ & Insertion of リ after root & Rolls once and stops \\ \cline{1-3}

コロコロ & Duplication of root & Rolls in succession \\ \cline{1-3}

コロンコロン & Duplication of root + ン & Rebounds with more momentum while rolling \\ \cline{1-3}

コロリコロリ & Duplication of root + リ & Intermittent rolling \\ \cline{1-3}

\end{ltabulary}

\par{\textbf{Derivation Note }: There are cases when a ッ may be inserted inside the root, but this can't happen here because the consonant inside the root is r. }

\par{ Of course, there can always be other words derived from onomatopoeia. Please note that you always have your irregularities. Sometimes different forms have different nuances, although always related. This does not include non-onomatopoeic words with repeating elements. This is really just something you have to mess around with and test the limits of. }
      
\section{Examples}
 
\par{${\overset{\textnormal{}}{\text{1. 日本語}}}$ がぺらぺらですね。 \hfill\break
You speak Japanese very fluently, don't you? }

\par{2. ${\overset{\textnormal{しゃりん}}{\text{車輪}}}$ はくるくる ${\overset{\textnormal{かいてん}}{\text{回転}}}$ した。 \hfill\break
The wheels turned around. }

\par{3. しとしとと ${\overset{\textnormal{}}{\text{雨}}}$ が ${\overset{\textnormal{ふ}}{\text{降}}}$ る。 \hfill\break
To drizzle. }

\par{4. 冷蔵庫の中をごそごそあさる。 \hfill\break
To feel through the refrigerator. }

\par{5. お ${\overset{\textnormal{なか}}{\text{腹}}}$ がぺこぺこだよ。 \hfill\break
I am very hungry! }

\par{6. ゴロゴロと ${\overset{\textnormal{かみなり}}{\text{雷}}}$ が ${\overset{\textnormal{}}{\text{鳴}}}$ っている。 \hfill\break
Thunder is rumbling. }

\par{${\overset{\textnormal{}}{\text{7. 大}}}$ きな ${\overset{\textnormal{}}{\text{木}}}$ がどさっと ${\overset{\textnormal{}}{\text{倒}}}$ れた。 \hfill\break
A large tree thudded down. }

\par{8. ばたんと ${\overset{\textnormal{し}}{\text{閉}}}$ める。 \hfill\break
To shut with a bang. }

\par{9. しんとした森 \hfill\break
A silent forest }

\par{10. ${\overset{\textnormal{しずく}}{\text{滴}}}$ がぽたぽたと落ちていた。 \hfill\break
The drops were plopping down. }
 
\par{11. 雨が ${\overset{\textnormal{やね}}{\text{屋根}}}$ をパラパラと ${\overset{\textnormal{う}}{\text{打}}}$ っていた。 \hfill\break
The rain was pattering on the roof. }
 
\par{12. ぐつぐつ(と) ${\overset{\textnormal{に}}{\text{煮}}}$ る。 \hfill\break
To simmer. }

\par{13. ざわざわ(と)する \hfill\break
To hum }

\par{14. カブトムシがカサカサと草むらを動いている。 \hfill\break
Beetles are rustling through the grass thickets. }

\par{15. カブトムシが空をぶんぶんと飛んでいる。 \hfill\break
Beetles are buzzing through the air. }

\par{16. 風がぴゅうぴゅうと ${\overset{\textnormal{ふ}}{\text{吹}}}$ く、寒い日でした。 \hfill\break
It was a cold day with the wind really blowing. }
 
\par{\textbf{Grammar Note }: The last example shows how a verbal expression can be used as an attribute when another attribute is used at the same time. Notice the use of the comma. }

\begin{center}
\textbf{Saying } 
\end{center}

\begin{ltabulary}{|P|P|P|P|P|P|}
\hline 

To harp & くだくだ(と)いう \hfill\break
くどくど(と)いう & To nag & がみがみ(と)する & To be fluent & ぺらぺら(と) \\ \cline{1-6}

To murmur & ぶつぶつ(と)いう & To buzz & がやがや & To be outspoken & ぽんぽん(と)いう \\ \cline{1-6}

To chatter & ぺちゃくちゃ(と)しゃべる \hfill\break
べらべら(と)しゃべる & To scold & がんがん(と)いう & To swallow & ぼそぼそ(と)いう \\ \cline{1-6}

To whisper & ひそひそ(と)いう & To grunt & ぶうぶう(と)いう & Noisily & わいわい(と) \\ \cline{1-6}

\end{ltabulary}

\begin{center}
\textbf{Eating \& Drinking }
\end{center}

\begin{ltabulary}{|P|P|P|P|}
\hline 

To gulp & ごくごく(と)飲む \hfill\break
がぶがぶ(と)飲む \hfill\break
ぐっと飲む & To guzzle & がつがつ(と)食べる \\ \cline{1-4}

Crunchy & こりこり(と)する & Scraping; hard to the teeth & ごりごり(と) \\ \cline{1-4}

To gobble & ぱくぱく(と)食べる & To suck & ちゅうちゅう(と)吸う \\ \cline{1-4}

To swallow & ごくり(と)飲む \hfill\break
ごくん(と)飲む & To gnaw \hfill\break
& がりがり(と)かじる \\ \cline{1-4}

\end{ltabulary}

\begin{center}
\textbf{Laughter }
\end{center}

\begin{ltabulary}{|P|P|P|P|P|P|}
\hline 

To sneer \hfill\break
& せせら ${\overset{\textnormal{}}{\text{笑}}}$ う & To chuckle \hfill\break
& くすくす(と)笑う & Laughing \hfill\break
Sounds \hfill\break
& あはは: いひひ \hfill\break
うふふ: えへへ \hfill\break
おほほ: ははは \hfill\break
ひひひ: くっくっ \\ \cline{1-6}

To cackle & けらけら(と)笑う \hfill\break
& To guffaw \hfill\break
& げらげら笑う \hfill\break
&  &  \\ \cline{1-6}

\end{ltabulary}
\hfill\break
\textbf{Laughing Sounds Notes }: Each of the different options in the lower right hand corner have different nuances. The first is a rather happy laugh; the second is more sinister; the third is a feminine chuckle; the fourth is used when you're embarrassed; the fifth gives an impression of rich people; the sixth is a loud laugh; the seventh is like the second; the last is a stifled laughter.       
\section{With 来る}
 
\par{ With onomatopoeia and と, 来る shows some sort of reaction. This may be a physical or an emotional reaction. Whatever the case may be, the verb still keeps its sense of "to come." }

\par{15. わさびが ${\overset{\textnormal{はな}}{\text{鼻}}}$ につんと来た。 \hfill\break
Wasabi got in my nose big time. }

\par{16. ${\overset{\textnormal{せいでんき}}{\text{静電気}}}$ がびりっと来る。 \hfill\break
For static electricity to shock you. }

\par{17. ぼくにはしっくり来ない。 \hfill\break
It doesn't fit well with me. }

\par{18. ぴったり来る ${\overset{\textnormal{}}{\text{音楽}}}$ \hfill\break
Agreeable music }

\par{19. カチンと来る。 \hfill\break
To get angry. }
    