    
\chapter{Idioms I}

\begin{center}
\begin{Large}
第168課: Idioms I: 気 
\end{Large}
\end{center}
 
\par{ There are tons of idioms with 気. There is no simple definition of 気. It can refer to one's spirit, mind, disposition, mood, intention, feeling, attention, interest, etc. It can also refer to the atmosphere or the essence of something. All of these have a common theme. Once you begin seeing expressions with it, this will become much clearer. }
      
\section{The Idioms}
 
\par{ The literal interpretations of idioms gives us an insight on how truly different Japanese words things in comparison to English. Don't let this, though, make you unable to understand them. Keep in mind that 気 is just a normal noun. There's nothing really different from it than words in English with lots of usages or those found in many set phrases. }

\par{ Don't view this list as bunch of set phrases whose literal meanings are just bizarre. This is certainly not how the Japanese view them. In every language, there are idiomatic phrases that have deviated so far from the original meanings of the words that they are composed of that you have to treat them as separate items in one's vocabulary. But, for the most part, the Japanese really think of 気 expressions with the literal interpretations provided in the right column. }

\par{ Also, idioms are not equal in idiomacy (level of being idiomatic). Some may be very similar to the English phrasing minus a word here and there. The Japanese itself in 気 phrases may be far more abstract than others. For instance, 気が重い is more straightforward than 気をそろえる. Keep all of this in mind as you look at this list and the example sentences that follow. }

\begin{ltabulary}{|P|P|P|}
\hline 

 & Meaning & Literal Meaning \\ \cline{1-3}

気を張る & To pay attention to & To stretch the mind \\ \cline{1-3}

気を使う & To fuss about; attend to & To use the mind \\ \cline{1-3}

気を回す & To be suspicious & To spin the mind \\ \cline{1-3}

気をそろえる & To pull together & To line up one's feelings \\ \cline{1-3}

気を通す & To have the sense to & To carry through the mind \\ \cline{1-3}

気が付く & To notice & To attach to the mind (intrans.) \\ \cline{1-3}

気を付ける & To be careful & To attach to the mind (trans.) \\ \cline{1-3}

気になる & To be on one's mind & To become in mind \\ \cline{1-3}

気に入る & To like\slash be fond of & To enter the mind \\ \cline{1-3}

気が焦る & To be impatient & For the mind to be in a hurry \\ \cline{1-3}

気が荒い & To be quarrelsome & For the mind to be violent \\ \cline{1-3}

気が進む & To feel like doing something & For the mind to advance \\ \cline{1-3}

気が軽い & To be sociable \hfill\break
& For the mind to be light \\ \cline{1-3}

気が座る & To feel relieved\slash be at ease & For the mind to sit \\ \cline{1-3}

気がそれる & To be distracted & For the mind to divert \\ \cline{1-3}

気が多い & To be fickle & To have many minds \\ \cline{1-3}

気が大きい & To be generous & To have a big mind \\ \cline{1-3}

気が腐る & To feel dispirited & For one's mind to rot \\ \cline{1-3}

気に留める & To keep in mind & To keep in the mind \\ \cline{1-3}

気前がいい & To be generous & To have good generosity \\ \cline{1-3}

気が急く & To be in a hurry\slash feel under pressure & For the mind to be hurried \\ \cline{1-3}

気は心 & It's the thought that counts & The mind's the heart \\ \cline{1-3}

気が重い & To feel depressed & For the mind to be heavy \\ \cline{1-3}

気が散る & To be distracted & For the mind to be scattered \\ \cline{1-3}

気が乗らない & To not be in the mood & For the mind to not be riding \\ \cline{1-3}

気を引く & To attract someone's affection & To draw in minds \\ \cline{1-3}

気をもむ & To fret & To worry the mind \\ \cline{1-3}

\end{ltabulary}
  
\section{}
 
\par{Despite that idiomatic  phrases are typically stand-alone phrases that can be and are understood  in isolation, it is helpful to see context with these phrases. Do not  be confused with syntax as nothing out of the ordinary was shown. If you  must, get familiar with the literal definitions to think of the  phrases. }

\par{1. 彼は僕と \textbf{気の合う }友人です。 \hfill\break
He is a congenial friend to me. }

\par{2. あいつに従う \textbf{気 }はない。 \hfill\break
I have no intentions to obey him. }

\par{3. 気を悪くしないで。 \hfill\break
No hard feelings. }

\par{4. あの子 \textbf{に気がある }の?(Casual) \hfill\break
Do you have a fancy for her? }

\par{5a. もし俺の言うことを聞く気があんなら助けてやろう。(Really casual) \hfill\break
5b. もし俺の言う通りにするなら助けてやろう。 \hfill\break
If he intends to listen what I'm going to say, I'll help him. }

\par{6. 今度の\{催し・イベント\}のこと(を)考えると気が重くなっちゃうの。(A little feminine) \hfill\break
Whenever I think of the coming event, I get depressed. }

\par{7. \{気が狂いそうな・頭がおかしい\}やつじゃん。(Casual) \hfill\break
Isn't he a crazy guy? }

\par{8. 遊ぶ \textbf{気がしない }子供は存在するはずがない。(Somewhat old-fashioned) \hfill\break
There shouldn't exist children that don't feel like playing. }

\par{9. 鈴木さんは \textbf{気前のよい }寄贈者でいらっしゃいます。(Honorific) \hfill\break
Mr. Suzuki is a generous contributor. }

\par{10. 気は心。 \hfill\break
It's the thought that counts. }

\par{11. 彼はいつも自分の思い通りじゃないと \textbf{気がすまない }嫌いがある。 \hfill\break
He has the tendency to always want his way. }

\par{12. コンピューターがついてると \textbf{気が散って }ちっとも勉強できないんだ。(Casual) \hfill\break
I can't study at all when the computer is on. }
    