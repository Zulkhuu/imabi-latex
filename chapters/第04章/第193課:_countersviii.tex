    
\chapter{Counters VIII}

\begin{center}
\begin{Large}
第193課: Counters VIII: 基, 滴, 票, 件,行, 画, 種(類), 脚, 着, 膳, 貫, 対, 男, \& 女 
\end{Large}
\end{center}
 
\par{ In this installation on counters, you will learn about yet another fifteen counters to expand your knowledge on how to count even more things. }

\begin{center}
\textbf{Counters to be Covered }
\end{center}
 
\par{1.       ~基 \hfill\break
2.       ~滴 \hfill\break
3.       ~票 \hfill\break
4.       ~件 \hfill\break
5.       ~行(ぎょう・こう) \hfill\break
6.       ~画 \hfill\break
7.       ~種(類) \hfill\break
8.       ~脚 \hfill\break
9.       ~着 \hfill\break
10.   ~膳 \hfill\break
11.   ~貫 \hfill\break
12.   ~輪 \hfill\break
13.   ~対 \hfill\break
14.   ~男 \hfill\break
15.   ~女 }
      
\section{Intermediate Counters}
 
\begin{center}
\textbf{~基 }
\end{center}
 
\par{ The counter ~ ${\overset{\textnormal{き}}{\text{基}}}$ counts a variety of things that are installed and or placed somewhere. Common things it is used for include dams, elevators, grave(stone)s, wreaths, reactors, CPUs, large machinery, lanterns, etc. }

\begin{ltabulary}{|P|P|P|P|P|P|P|P|}
\hline 

1 & いっき & 2 & にき & 3 & さんき & 4 & よんき \\ \cline{1-8}

5 & ごき & 6 & ろっき & 7 & ななき & 8 & はっき \\ \cline{1-8}

9 & きゅうき & 10 &  \textbf{じゅっき \hfill\break
 }じっき & 100 & ひゃっき & ? & なんき \\ \cline{1-8}

\end{ltabulary}
 
\par{\hfill\break
1. この ${\overset{\textnormal{まち}}{\text{町}}}$ は ${\overset{\textnormal{ことし}}{\text{今年}}}$ 、 ${\overset{\textnormal{ふうりょくはつでんき}}{\text{風力発電機}}}$ を ${\overset{\textnormal{じゅっ}}{\text{10}}}$ ${\overset{\textnormal{き}}{\text{基}}}$ ${\overset{\textnormal{せっち}}{\text{設置}}}$ した。 \hfill\break
This town installed ten wind turbine generators this year. }
 
\par{2. ダムを ${\overset{\textnormal{いっ}}{\text{1}}}$ ${\overset{\textnormal{き}}{\text{基}}}$ ${\overset{\textnormal{けんせつ}}{\text{建設}}}$ するのに ${\overset{\textnormal{にじゅう}}{\text{20}}}$ ${\overset{\textnormal{ねんいじょう}}{\text{年以上}}}$ かかります。 \hfill\break
It takes over twenty years to construct one dam. }
 
\par{3. ${\overset{\textnormal{こうせいのう}}{\text{高性能}}}$ の ${\overset{\textnormal{シーピーユー}}{\text{CPU}}}$ を ${\overset{\textnormal{に}}{\text{2}}}$ ${\overset{\textnormal{き}}{\text{基}}}$ ${\overset{\textnormal{つ}}{\text{積}}}$ んでいます。 \hfill\break
It has two high-performance CPUs loaded. }
 
\begin{center}
\textbf{~滴 }
\end{center}
 
\par{ The counter ~ ${\overset{\textnormal{てき}}{\text{滴}}}$ is used to count drops of liquid. }
 
\begin{ltabulary}{|P|P|P|P|P|P|P|P|}
\hline 

1 & いってき & 2 & にてき & 3 & さんてき & 4 & よんてき \\ \cline{1-8}

5 & ごてき & 6 & ろくてき & 7 & ななてき & 8 & はってき \\ \cline{1-8}

9 & きゅうてき & 10 &  \textbf{じゅってき }\hfill\break
じってき & 100 & ひゃくてき & ? & なんてき \\ \cline{1-8}

\end{ltabulary}
 
\par{\hfill\break
4. ${\overset{\textnormal{みず}}{\text{水}}}$ が ${\overset{\textnormal{いっ}}{\text{1}}}$ ${\overset{\textnormal{てき}}{\text{滴}}}$ も ${\overset{\textnormal{はい}}{\text{入}}}$ っていないんですよ。 \hfill\break
There\textquotesingle s not a single drop of water in it. }
 
\par{5. ラー ${\overset{\textnormal{ゆ}}{\text{油}}}$ を ${\overset{\textnormal{じゅっ}}{\text{10}}}$ ${\overset{\textnormal{てき}}{\text{滴}}}$ かけて ${\overset{\textnormal{た}}{\text{食}}}$ べてみました。 \hfill\break
I tried eating with 10 drops of Chinese chili oil on it. }
 
\par{6. ${\overset{\textnormal{ゆび}}{\text{指}}}$ にオイルをつけて、 ${\overset{\textnormal{ほ}}{\text{頬}}}$ っぺたに ${\overset{\textnormal{なじ}}{\text{馴染}}}$ ませてください。 \hfill\break
Put the oil on your finger and then thoroughly blend it into your cheeks. }
 
\begin{center}
~ \textbf{票 }
\end{center}
 
\par{ The counter ~ ${\overset{\textnormal{ひょう}}{\text{票}}}$ counts votes. The word for vote also happens to be 票. }

\begin{ltabulary}{|P|P|P|P|P|P|P|P|}
\hline 

1 & いっぴょう & 2 & にひょう & 3 &  \textbf{さんぴょう }\hfill\break
さんびょう \hfill\break
さんひょう & 4 & よんひょう \\ \cline{1-8}

5 & ごひょう & 6 & ろっぴょう & 7 & ななひょう & 8 &  \textbf{はっぴょう }\hfill\break
はちひょう \\ \cline{1-8}

9 & きゅうひょう & 10 &  \textbf{じゅっぴょう \hfill\break
 }じっぴょう & 100 & ひゃっぴょう & ? &  \textbf{なんぴょう \hfill\break
}なんびょう \hfill\break
なんひょう \\ \cline{1-8}

\end{ltabulary}

\par{\hfill\break
7. ${\overset{\textnormal{しんじん}}{\text{新人}}}$ の ${\overset{\textnormal{だんせいこうほしゃ}}{\text{男性候補者}}}$ が ${\overset{\textnormal{いっ}}{\text{1}}}$ ${\overset{\textnormal{ぴょう}}{\text{票}}}$ も ${\overset{\textnormal{かくとく}}{\text{獲得}}}$ できなかった。 \hfill\break
The rookie male candidate was unable to acquire a single vote. }
 
\par{8. ${\overset{\textnormal{さいしゅうとうひょう}}{\text{最終投票}}}$ では、リオデジャネイロが ${\overset{\textnormal{ろくじゅうろく}}{\text{66}}}$ ${\overset{\textnormal{ひょうかくとく}}{\text{票獲得}}}$ した。 \hfill\break
In the final vote, Rio de Janeiro acquired 66 votes. }
 
\par{9. ${\overset{\textnormal{わたなべし}}{\text{渡辺氏}}}$ が ${\overset{\textnormal{ご}}{\text{5}}}$ ${\overset{\textnormal{まん}}{\text{万}}}$ ${\overset{\textnormal{よんせんろっぴゃくさんじゅうに}}{\text{4632}}}$ ${\overset{\textnormal{ひょうかくとく}}{\text{票獲得}}}$ しました。 \hfill\break
Mr. Watanabe acquired 54,632 votes. }
 
\begin{center}
\textbf{~件 }
\end{center}
 
\par{ The counter ${\overset{\textnormal{けん}}{\text{件}}}$ counts situations\slash incidents. This can extend to accidents, bugs (IT), cases (of a disease, etc.), inquiries, applications, search results, responses, news articles, etc. }

\begin{ltabulary}{|P|P|P|P|P|P|P|P|}
\hline 

1 & いっけん & 2 & にけん & 3 & さんけん & 4 & よんけん \\ \cline{1-8}

5 & ごけん & 6 & ろっけん & 7 & ななけん & 8 &  \textbf{はちけん }\hfill\break
はっけん \\ \cline{1-8}

9 & きゅうけん & 10 &  \textbf{じゅっけん \hfill\break
 }じっけん & 100 & ひゃっけん & ? & なんけん \\ \cline{1-8}

\end{ltabulary}
 
\par{\hfill\break
10. ${\overset{\textnormal{しゅとけんない}}{\text{首都圏内}}}$ では、 ${\overset{\textnormal{ことしじゅう}}{\text{今年中}}}$ に ${\overset{\textnormal{じんしんじこ}}{\text{人身事故}}}$ が ${\overset{\textnormal{さんじゅうよん}}{\text{34}}}$ ${\overset{\textnormal{けんはっせい}}{\text{件発生}}}$ しています。 \hfill\break
In greater Tokyo, there have been 34 accidents resulting in injury\slash death during this year. }
 
\par{11. ${\overset{\textnormal{かながわけん}}{\text{神奈川県}}}$ でも、 ${\overset{\textnormal{どうよう}}{\text{同様}}}$ の ${\overset{\textnormal{ひがい}}{\text{被害}}}$ が ${\overset{\textnormal{にじゅっ}}{\text{20}}}$ ${\overset{\textnormal{けん}}{\text{件}}}$ くらい ${\overset{\textnormal{はっせい}}{\text{発生}}}$ しています。 \hfill\break
Even in Kanagawa Prefecture, there have been about 20 similar cases of damage. }
 
\par{12. 検出したバグはすでに ${\overset{\textnormal{ひゃっ}}{\text{100}}}$ ${\overset{\textnormal{けん}}{\text{件}}}$ を ${\overset{\textnormal{こ}}{\text{超}}}$ えています。 \hfill\break
Bugs detected have already surpassed 100. }
 
\begin{center}
\textbf{~行(ぎょう・こう) }
\end{center}
 
\par{ When read as ぎょう, ~行 is used to count lines of text. When read as こう, it is used to count banks. However, the phrase ${\overset{\textnormal{いっこう}}{\text{一行}}}$ can also be used to mean “group of people\slash line of…” }
 
\par{~ぎょう }

\begin{ltabulary}{|P|P|P|P|P|P|P|P|}
\hline 

1 & いちぎょう & 2 & にぎょう & 3 & さんぎょう & 4 & よんぎょう \\ \cline{1-8}

5 & ごぎょう & 6 & ろくぎょう & 7 & ななぎょう & 8 & はちぎょう \\ \cline{1-8}

9 & きゅうぎょう & 10 & じゅうぎょう & 100 & ひゃくぎょう & ? & なんぎょう \\ \cline{1-8}

\end{ltabulary}

\par{\hfill\break
~こう }

\begin{ltabulary}{|P|P|P|P|P|P|P|P|}
\hline 

1 & いっこう & 2 & にこう & 3 & さんこう & 4 & よんこう \\ \cline{1-8}

5 & ごこう & 6 & ろっこう & 7 & ななこう & 8 &  \textbf{はちこう }\hfill\break
はっこう \\ \cline{1-8}

9 & きゅうこう & 10 &  \textbf{じゅっこう }\hfill\break
じっこう & 100 & ひゃっこう & ? & なんこう \\ \cline{1-8}

\end{ltabulary}
 
\par{\hfill\break
13. スペイン ${\overset{\textnormal{こくゆう}}{\text{国有}}}$ の ${\overset{\textnormal{に}}{\text{2}}}$ ${\overset{\textnormal{こう}}{\text{行}}}$ が ${\overset{\textnormal{がっぺい}}{\text{合併}}}$ することで ${\overset{\textnormal{ごうい}}{\text{合意}}}$ しました。 \hfill\break
Two of Spain\textquotesingle s national-owned banks have agreed to merge. }
 
\par{14. ${\overset{\textnormal{かみなり}}{\text{雷}}}$ に ${\overset{\textnormal{おどろ}}{\text{驚}}}$ いた ${\overset{\textnormal{いっこう}}{\text{一行}}}$ の ${\overset{\textnormal{うま}}{\text{馬}}}$ が ${\overset{\textnormal{ぼうそう}}{\text{暴走}}}$ して ${\overset{\textnormal{おんなひとり}}{\text{女一人}}}$ を ${\overset{\textnormal{ふ}}{\text{踏}}}$ み ${\overset{\textnormal{ころ}}{\text{殺}}}$ した。 \hfill\break
A troupe of horses scared by thunder rampaged and trampled one woman to death. }
 
\par{15. ${\overset{\textnormal{に}}{\text{2}}}$ ${\overset{\textnormal{ぎょうめ}}{\text{行目}}}$ の ${\overset{\textnormal{もじすう}}{\text{文字数}}}$ を教えてください。 \hfill\break
Please tell me the number of characters in the second line. }
 
\par{16. ${\overset{\textnormal{まるまるしいっこう}}{\text{〇〇氏一行}}}$ が ${\overset{\textnormal{ちゅうごくとうほくぶ}}{\text{中国東北部}}}$ へ ${\overset{\textnormal{しさつ}}{\text{視察}}}$ (し)に ${\overset{\textnormal{い}}{\text{行}}}$ きました。 \hfill\break
Mr. \#\#\textquotesingle s entourage went to observe Northeast China. }
 
\par{17. ${\overset{\textnormal{にせんはち}}{\text{2008}}}$ ${\overset{\textnormal{ねん}}{\text{年}}}$ には、アメリカの ${\overset{\textnormal{きんゆうきかん}}{\text{金融機関}}}$ のうち ${\overset{\textnormal{すく}}{\text{少}}}$ なくとも ${\overset{\textnormal{いっ}}{\text{1}}}$ ${\overset{\textnormal{こう}}{\text{行}}}$ が ${\overset{\textnormal{はたん}}{\text{破綻}}}$ してしまった。 \hfill\break
In 2008, of the financial institutions in America, at least one bank went into bankruptcy. }
 
\begin{center}
\textbf{~画 }
\end{center}
 
\par{ The counter ~ ${\overset{\textnormal{かく}}{\text{画}}}$ counts the number of strokes in a 漢字. It can also be used to count plots\slash lots of land. For this latter meaning, it is usually seen as 区画. }
 
\par{~かく }

\begin{ltabulary}{|P|P|P|P|P|P|P|P|}
\hline 

1 & いっかく & 2 & にかく & 3 & さんかく & 4 & よんかく \\ \cline{1-8}

5 & ごかく & 6 & ろっかく & 7 & ななかく & 8 &  \textbf{はっかく \hfill\break
 }はちかく \\ \cline{1-8}

9 & きゅうかく & 10 &  \textbf{じゅっかく \hfill\break
 }じっかく & 11 & じゅういっかく & ? & なんかく \\ \cline{1-8}

\end{ltabulary}

\par{~くかく }

\begin{ltabulary}{|P|P|P|P|P|P|P|P|}
\hline 

1 & いっくかく & 2 & にくかく & 3 & さんくかく & 4 & よんくかく \\ \cline{1-8}

5 & ごくかく & 6 & ろっくかく & 7 & ななくかく & 8 &  \textbf{はちくかく \hfill\break
 }はっくかく \\ \cline{1-8}

9 & きゅうくかく & 10 &  \textbf{じゅっくかく \hfill\break
}じっくかく & 100 & ひゃっくかく & ? & なんくかく \\ \cline{1-8}

\end{ltabulary}
 
\par{\hfill\break
18. 「ヲ」は ${\overset{\textnormal{さん}}{\text{3}}}$ ${\overset{\textnormal{かく}}{\text{画}}}$ で ${\overset{\textnormal{か}}{\text{書}}}$ きます。 \hfill\break
“ヲ” is written with 3 strokes. }
 
\par{19. ${\overset{\textnormal{かんこく}}{\text{韓国}}}$ の「 ${\overset{\textnormal{かん}}{\text{韓}}}$ 」を ${\overset{\textnormal{じゅうなな}}{\text{17}}}$ ${\overset{\textnormal{かく}}{\text{画}}}$ で ${\overset{\textnormal{か}}{\text{書}}}$ くって ${\overset{\textnormal{し}}{\text{知}}}$ ってる? \hfill\break
Did you know that the “kan” in “Kankoku” is written with 17 strokes? }
 
\par{20. このホテルはバンクーバーの ${\overset{\textnormal{かんこうちいき}}{\text{観光地域}}}$ から ${\overset{\textnormal{やく}}{\text{約}}}$ ${\overset{\textnormal{じゅうご}}{\text{15}}}$ ${\overset{\textnormal{くかくはな}}{\text{区画離}}}$ れた ${\overset{\textnormal{ばしょ}}{\text{場所}}}$ にあります。 \hfill\break
This hotel is at a place 15 blocks removed from the tourist area of Vancouver. }
 
\par{21. ${\overset{\textnormal{ぶんじょうち}}{\text{分譲地}}}$ の ${\overset{\textnormal{いっ}}{\text{1}}}$ ( ${\overset{\textnormal{く}}{\text{区}}}$ ) ${\overset{\textnormal{かく}}{\text{画}}}$ を ${\overset{\textnormal{こうにゅう}}{\text{購入}}}$ する ${\overset{\textnormal{よてい}}{\text{予定}}}$ です。 \hfill\break
I intend to purchase one section of a lot. }
 
\begin{center}
\textbf{~種(類) }
\end{center}
 
\par{ The counter ~ ${\overset{\textnormal{しゅ}}{\text{種}}}$ ( ${\overset{\textnormal{るい}}{\text{類}}}$ )counts species\slash kinds. }

\begin{ltabulary}{|P|P|P|P|P|P|}
\hline 

1 & いっしゅ(るい) & 2 & にしゅ(るい) & 3 & さんしゅ(るい) \\ \cline{1-6}

4 & よんしゅ(るい) & 5 & ごしゅ(るい) & 6 & ろくしゅ(るい) \\ \cline{1-6}

7 & ななしゅ(るい) & 8 &  \textbf{はちしゅ(るい) \hfill\break
 }はっしゅるい & 9 & きゅうしゅ(るい) \\ \cline{1-6}

10 &  \textbf{じゅっしゅ(るい) \hfill\break
}じっしゅ(るい) & 100 & ひゃくしゅ(るい) & ? & なんしゅ(るい) \\ \cline{1-6}

\end{ltabulary}
 
\par{\hfill\break
22. ${\overset{\textnormal{しはら}}{\text{支払}}}$ いプランが ${\overset{\textnormal{さん}}{\text{3}}}$ ${\overset{\textnormal{しゅるい}}{\text{種類}}}$ あります。 \hfill\break
There are three types of payment plans. }
 
\par{23. この ${\overset{\textnormal{に}}{\text{2}}}$ ${\overset{\textnormal{しゅ}}{\text{種}}}$ ( ${\overset{\textnormal{るい}}{\text{類}}}$ )の ${\overset{\textnormal{ぞう}}{\text{象}}}$ の ${\overset{\textnormal{かせき}}{\text{化石}}}$ は ${\overset{\textnormal{ほっかいどう}}{\text{北海道}}}$ でも ${\overset{\textnormal{み}}{\text{見}}}$ つかっている。 \hfill\break
Fossils of these two species of elephant have also been found in Hokkadio. }
 
\par{24. ${\overset{\textnormal{にほんこゆう}}{\text{日本固有}}}$ の ${\overset{\textnormal{けんしゅ}}{\text{犬種}}}$ は ${\overset{\textnormal{ぜんぶ}}{\text{全部}}}$ で ${\overset{\textnormal{ろく}}{\text{6}}}$ ${\overset{\textnormal{しゅ}}{\text{種}}}$ あります。 \hfill\break
There is a total of six species of dug unique to Japan. }
 
\begin{center}
\textbf{~脚 }
\end{center}
 
\par{ The counter ~ ${\overset{\textnormal{きゃく}}{\text{脚}}}$ counts chairs, desks, trays with legs, and seats. It can also be the “pod” in 一脚 (mono-pod).” Outside the home, however, these things frequently counted with ~台 or some other counter. }

\begin{ltabulary}{|P|P|P|P|P|P|P|P|}
\hline 

1 & いっきゃく & 2 & にきゃく & 3 & さんきゃく & 4 & よんきゃく \\ \cline{1-8}

5 & ごきゃく & 6 & ろっきゃく & 7 & ななきゃく & 8 & はっきゃく \\ \cline{1-8}

9 & きゅうきゃく & 10 &  \textbf{じゅっきゃく \hfill\break
 }じっきゃく & 100 & ひゃっきゃく & ? & なんきゃく \\ \cline{1-8}

\end{ltabulary}

\par{\hfill\break
25. オフィスチェアが\{ ${\overset{\textnormal{よん}}{\text{4}}}$ ${\overset{\textnormal{こ}}{\text{個}}}$ ・ ${\overset{\textnormal{よん}}{\text{4}}}$ ${\overset{\textnormal{きゃく}}{\text{脚}}}$ \}あります。 \hfill\break
There are four office chairs. }
 
\par{26. ${\overset{\textnormal{お}}{\text{折}}}$ り ${\overset{\textnormal{たた}}{\text{畳}}}$ み ${\overset{\textnormal{しき}}{\text{式}}}$ (の) ${\overset{\textnormal{いす}}{\text{椅子}}}$ が\{ ${\overset{\textnormal{さん}}{\text{3}}}$ ${\overset{\textnormal{こ}}{\text{個}}}$ ・ ${\overset{\textnormal{さん}}{\text{3}}}$ ${\overset{\textnormal{きゃく}}{\text{脚}}}$ \}あります。 \hfill\break
There are three fold-up chairs. }
 
\par{27. ${\overset{\textnormal{てっせい}}{\text{鉄製}}}$ の ${\overset{\textnormal{つくえ}}{\text{机}}}$ が\{ ${\overset{\textnormal{に}}{\text{2}}}$ ${\overset{\textnormal{だい}}{\text{台}}}$ ・ ${\overset{\textnormal{ふた}}{\text{2}}}$ つ・ ${\overset{\textnormal{に}}{\text{2}}}$ ${\overset{\textnormal{きゃく}}{\text{脚}}}$ \}あります。 \hfill\break
There are two steel desks. }
 
\par{28. ${\overset{\textnormal{さんきゃく}}{\text{三脚}}}$ を ${\overset{\textnormal{つか}}{\text{使}}}$ ったことがありますか。 \hfill\break
Have you ever used a tripod? }
 
\par{\textbf{Word Note }: In the phrase 三脚, the literal meaning of “leg” for the character 脚 is used. }
 
\begin{center}
\textbf{~着 }
\end{center}
 
\par{ Depending on the clothing, you will either count it with ~ ${\overset{\textnormal{ちゃく}}{\text{着}}}$ or ~枚.  Suits, dresses, coats, and jackets that cover the whole body are counted with ~着. Shirts, blouses, sweaters, skirts, pants, underwear, casual jackets, one-pieces, etc. are counted with ~枚. }

\begin{ltabulary}{|P|P|P|P|P|P|P|P|}
\hline 

1 & いっちゃく & 2 & にちゃく & 3 & さんちゃく & 4 & よんちゃく \\ \cline{1-8}

5 & ごちゃく & 6 & ろくちゃく & 7 & ななちゃく & 8 & はっちゃく \\ \cline{1-8}

9 & きゅうちゃく & 10 &  \textbf{じゅっちゃく }\hfill\break
じっちゃく & 100 & ひゃっちゃく & ? & なんちゃく \\ \cline{1-8}

\end{ltabulary}
 
\par{\hfill\break
29. ${\overset{\textnormal{しゃかいじん}}{\text{社会人}}}$ でもスーツを ${\overset{\textnormal{いっ}}{\text{1}}}$ ${\overset{\textnormal{ちゃく}}{\text{着}}}$ しか ${\overset{\textnormal{も}}{\text{持}}}$ ってない ${\overset{\textnormal{ひと}}{\text{人}}}$ はたくさんいるでしょう。 \hfill\break
There are plenty of people who may be working-adults yet have but one suit. }
 
\par{30. ${\overset{\textnormal{ゆかた}}{\text{浴衣}}}$ は ${\overset{\textnormal{たぶん}}{\text{多分}}}$ ${\overset{\textnormal{さん}}{\text{3}}}$ ${\overset{\textnormal{ちゃくも}}{\text{着持}}}$ ってると ${\overset{\textnormal{おも}}{\text{思}}}$ います。 \hfill\break
I think I probably have three yukatas. }
 
\par{31. ${\overset{\textnormal{みな}}{\text{皆}}}$ さんは、 ${\overset{\textnormal{しごとよう}}{\text{仕事用}}}$ のスーツは ${\overset{\textnormal{なんちゃくも}}{\text{何着持}}}$ ってるんですか。 \hfill\break
How many business suits does everyone have? }
 
\begin{center}
\textbf{~膳 }
\end{center}
 
\par{ The counter ~ ${\overset{\textnormal{ぜん}}{\text{膳}}}$ counts bowlfuls of rice or pairs of chopsticks. }

\begin{ltabulary}{|P|P|P|P|P|P|P|P|}
\hline 

1 & いちぜん & 2 & にぜん & 3 & さんぜん & 4 & よんぜん \\ \cline{1-8}

5 & ごぜん & 6 & ろくぜん & 7 & ななぜん & 8 & はちぜん \\ \cline{1-8}

9 & きゅうぜん & 10 & じゅうぜん & 100 & ひゃくぜん & ? & なんぜん \\ \cline{1-8}

\end{ltabulary}

\par{\hfill\break
32. ${\overset{\textnormal{わたし}}{\text{私}}}$ は ${\overset{\textnormal{さん}}{\text{3}}}$ ${\overset{\textnormal{ぜん}}{\text{膳}}}$ のご ${\overset{\textnormal{はん}}{\text{飯}}}$ を ${\overset{\textnormal{たい}}{\text{平}}}$ らげた。 \hfill\break
I ate up three bowfuls of rice. }
 
\par{33. ${\overset{\textnormal{はし}}{\text{箸}}}$ を ${\overset{\textnormal{に}}{\text{2}}}$ ${\overset{\textnormal{ぜん}}{\text{膳}}}$ ください。 \hfill\break
Two (pairs of) chopsticks, please. }
 
\par{34. ${\overset{\textnormal{しゅしょく}}{\text{主食}}}$ は、 ${\overset{\textnormal{いち}}{\text{1}}}$ ${\overset{\textnormal{にち}}{\text{日}}}$ ${\overset{\textnormal{に}}{\text{2}}}$ ${\overset{\textnormal{ぜん}}{\text{膳}}}$ のご ${\overset{\textnormal{はん}}{\text{飯}}}$ と ${\overset{\textnormal{たまご}}{\text{卵}}}$ や ${\overset{\textnormal{さかな}}{\text{魚}}}$ 、お ${\overset{\textnormal{にく}}{\text{肉}}}$ や ${\overset{\textnormal{とうふ}}{\text{豆腐}}}$ など ${\overset{\textnormal{たんぱくしつ}}{\text{蛋白質}}}$ の ${\overset{\textnormal{おお}}{\text{多}}}$ いものを ${\overset{\textnormal{かなら}}{\text{必}}}$ ず ${\overset{\textnormal{せっしゅ}}{\text{摂取}}}$ することが ${\overset{\textnormal{たいせつ}}{\text{大切}}}$ です。 \hfill\break
For one\textquotesingle s staple food, it is important that you always intake two bowlfuls of rice and foods rich in protein such as  eggs, fish, meat, tofu, etc. a day. }
 
\begin{center}
\textbf{~貫 }
\end{center}
 
\par{ The counter ~ ${\overset{\textnormal{かん}}{\text{貫}}}$ has three usages, two of which are obsolete. The first two involve units of currency and weight that are no longer used today but continue to be used when talking about or speaking as if one were in a period in which these units were in use. Otherwise, it is used to count pieces of sushi. Of course, ~個 is also appropriate. Interestingly enough, nearly half of all speakers treat 1貫 as 2個 rather than 1個. This is because sushi is usually dished out as two pieces. However, this practice itself is called 2貫付け, indicating that traditionally 1貫 refers to 1 piece. Because of this confusion, some speakers will either view each increment as referring to either one or two pieces each. }

\begin{ltabulary}{|P|P|P|P|P|P|P|P|}
\hline 

1 & いっかん & 2 & にかん & 3 & さんかん & 4 & よんかん \\ \cline{1-8}

5 & ごかん & 6 & ろっかん & 7 & ななかん & 8 &  \textbf{はっかん }\hfill\break
はちかん \\ \cline{1-8}

9 & きゅうかん & 10 &  \textbf{じゅっかん }\hfill\break
じっかん & 100 & ひゃっかん & ? & なんかん \\ \cline{1-8}

\end{ltabulary}
 
\par{\hfill\break
35. ${\overset{\textnormal{すし}}{\text{寿司}}}$ は ${\overset{\textnormal{なんかん}}{\text{何貫}}}$ ぐらい ${\overset{\textnormal{た}}{\text{食}}}$ べられますか。 \hfill\break
How many pieces of sushi can you eat? }
 
\par{36. ${\overset{\textnormal{いっ}}{\text{1}}}$ ${\overset{\textnormal{かん}}{\text{貫}}}$ って ${\overset{\textnormal{なんこ}}{\text{何個}}}$ ですか? \hfill\break
How much is 1 kan (of sushi)? }
 
\par{37. ${\overset{\textnormal{おお}}{\text{大}}}$ トロ ${\overset{\textnormal{じゅっ}}{\text{10}}}$ ${\overset{\textnormal{かん}}{\text{貫}}}$ ${\overset{\textnormal{た}}{\text{食}}}$ べても ${\overset{\textnormal{せんにひゃく}}{\text{1200}}}$ ${\overset{\textnormal{えん}}{\text{円}}}$ ! \hfill\break
It\textquotesingle s only 1200 yen even if you have 10 pieces of fat under-belly (of tuna)! }
 
\begin{center}
\textbf{~輪 }
\end{center}
 
\par{ The counter ~ ${\overset{\textnormal{りん}}{\text{輪}}}$ either counts flowers (in bloom) or the number of wheels (on a vehicle). When counting flowers, some speakers prefer to use ~本 whenever a single plant has more than one bud that blooms. }

\begin{ltabulary}{|P|P|P|P|P|P|P|P|}
\hline 

1 & いちりん & 2 & にりん & 3 & さんりん & 4 & よんりん \\ \cline{1-8}

5 & ごりん & 6 & ろくりん & 7 & ななりん & 8 & はちりん \\ \cline{1-8}

9 & きゅうりん & 10 & じゅうりん & 100 & ひゃくりん & ? & なんりん \\ \cline{1-8}

\end{ltabulary}
 
\par{\hfill\break
38. この ${\overset{\textnormal{ひごぜん}}{\text{日午前}}}$ ${\overset{\textnormal{じゅういち}}{\text{11}}}$ ${\overset{\textnormal{じ}}{\text{時}}}$ ごろ、 ${\overset{\textnormal{はな}}{\text{花}}}$ が ${\overset{\textnormal{ご}}{\text{5}}}$ ${\overset{\textnormal{りんさ}}{\text{輪咲}}}$ いているのを ${\overset{\textnormal{かくにん}}{\text{確認}}}$ しました。 \hfill\break
Today at around 11 AM, (I\slash we) verified that five flowers have bloomed. }
 
\par{39. ${\overset{\textnormal{おも}}{\text{思}}}$ い ${\overset{\textnormal{かえ}}{\text{返}}}$ せば、 ${\overset{\textnormal{ちい}}{\text{小}}}$ さいころに ${\overset{\textnormal{たし}}{\text{確}}}$ かに ${\overset{\textnormal{さんりんしゃ}}{\text{三輪車}}}$ に ${\overset{\textnormal{の}}{\text{乗}}}$ ったことないなあ。 \hfill\break
Now that I look back, I don\textquotesingle t think I ever rode a tricycle when I was little. }
 
\par{40. ${\overset{\textnormal{さくねん}}{\text{昨年}}}$ は ${\overset{\textnormal{いち}}{\text{1}}}$ ${\overset{\textnormal{りん}}{\text{輪}}}$ しか ${\overset{\textnormal{かいか}}{\text{開花}}}$ しなかった。 \hfill\break
Not one flower bloomed last year. }
 
\begin{center}
\textbf{~対 }
\end{center}
 
\par{ The counter ~ ${\overset{\textnormal{つい}}{\text{対}}}$ counts two things that make up a set. Be aware that some kinds of sets like footwear or teams are still counted with their unique counters and thus would not otherwise be counted with ~対. }

\begin{ltabulary}{|P|P|P|P|P|P|P|P|}
\hline 

1 & いっつい & 2 & につい & 3 & さんつい & 4 & よんつい \\ \cline{1-8}

5 & ごつい & 6 & ろくつい & 7 & ななつい & 8 & はっつい \\ \cline{1-8}

9 & きゅうつい & 10 &  \textbf{じゅっつい \hfill\break
}じっつい & 100 & ひゃくつい & ? & なんつい \\ \cline{1-8}

\end{ltabulary}

\par{\hfill\break
41. ウサギは ${\overset{\textnormal{せっし}}{\text{切歯}}}$ を ${\overset{\textnormal{ろっ}}{\text{6}}}$ ${\overset{\textnormal{ほん}}{\text{本}}}$ ( ${\overset{\textnormal{うえさゆう}}{\text{上左右}}}$ ${\overset{\textnormal{に}}{\text{2}}}$ ${\overset{\textnormal{つい}}{\text{対}}}$ ・ ${\overset{\textnormal{したさゆう}}{\text{下左右}}}$ ${\overset{\textnormal{いっ}}{\text{1}}}$ ${\overset{\textnormal{つい}}{\text{対}}}$ ) ${\overset{\textnormal{も}}{\text{持}}}$ っている。 \hfill\break
Rabbits have six incisors (two pairs to the left and right on the top and one pair to the left and right on the bottom). }
 
\par{42. ${\overset{\textnormal{さいぼう}}{\text{細胞}}}$ は ${\overset{\textnormal{ちちおや}}{\text{父親}}}$ 、 ${\overset{\textnormal{ははおやゆらい}}{\text{母親由来}}}$ の ${\overset{\textnormal{せんしょくたい}}{\text{染色体}}}$ を ${\overset{\textnormal{ひと}}{\text{1}}}$ つずつ、 ${\overset{\textnormal{いっ}}{\text{1}}}$ ${\overset{\textnormal{ついも}}{\text{対持}}}$ っている。 \hfill\break
Cells carry one pair of chromosomes, one derived from both one\textquotesingle s father and mother. }
 
\par{43. ${\overset{\textnormal{けんし}}{\text{犬歯}}}$ は ${\overset{\textnormal{せんい}}{\text{繊維}}}$ の ${\overset{\textnormal{おお}}{\text{多}}}$ いものを ${\overset{\textnormal{ひ}}{\text{引}}}$ き ${\overset{\textnormal{さ}}{\text{裂}}}$ くための ${\overset{\textnormal{は}}{\text{歯}}}$ で ${\overset{\textnormal{じょうげ}}{\text{上下}}}$ ${\overset{\textnormal{いっ}}{\text{1}}}$ ${\overset{\textnormal{つい}}{\text{対}}}$ ずつ ${\overset{\textnormal{よん}}{\text{4}}}$ ${\overset{\textnormal{ほん}}{\text{本}}}$ あります。 \hfill\break
Canines are teeth meant for slicing fibrous foods and there are four of them, a pair both top and both. }
 
\par{44. ${\overset{\textnormal{のうしんけい}}{\text{脳神経}}}$ ${\overset{\textnormal{じゅうに}}{\text{12}}}$ ${\overset{\textnormal{つい}}{\text{対}}}$ のうち ${\overset{\textnormal{のうかんぶ}}{\text{脳幹部}}}$ より ${\overset{\textnormal{で}}{\text{出}}}$ る ${\overset{\textnormal{のうしんけい}}{\text{脳神経}}}$ は ${\overset{\textnormal{じゅっ}}{\text{10}}}$ ${\overset{\textnormal{つい}}{\text{対}}}$ あります。 \hfill\break
Of the twelve pairs of cranial nerves, there are ten pairs of cranial nerves that come out from the brain stem. }
 
\begin{center}
\textbf{~男 }
\end{center}
 
\par{ The counter ~ ${\overset{\textnormal{なん}}{\text{男}}}$ counts sons and is typically used only up till five. The phrases for “first son” and “second son” are set phrases. However, when used to just count offspring in general, 1 and 2 are expressed as usual. }

\begin{ltabulary}{|P|P|P|P|P|P|P|P|P|P|}
\hline 

1 &  \textbf{ちょうなん(長男) \hfill\break
 }いちなん & 2 & じなん(次男) & 3 & さんなん & 4 & よんなん & 5 & ごなん \\ \cline{1-10}

\end{ltabulary}
 
\par{\hfill\break
45. ${\overset{\textnormal{むしょく}}{\text{無職}}}$ の ${\overset{\textnormal{よんじゅう}}{\text{42}}}$ ${\overset{\textnormal{さい}}{\text{歳}}}$ の ${\overset{\textnormal{だんせい}}{\text{男性}}}$ が、 ${\overset{\textnormal{しゅうしんちゅう}}{\text{就寝中}}}$ の ${\overset{\textnormal{にじゅうに}}{\text{22}}}$ ${\overset{\textnormal{さい}}{\text{歳}}}$ の ${\overset{\textnormal{ちょうなん}}{\text{長男}}}$ を ${\overset{\textnormal{おの}}{\text{斧}}}$ で ${\overset{\textnormal{さつがい}}{\text{殺害}}}$ しようとした ${\overset{\textnormal{うたが}}{\text{疑}}}$ いで ${\overset{\textnormal{たいほ}}{\text{逮捕}}}$ された。 \hfill\break
An unemployed male aged 42 was arrested under the suspicion that he attempted to murder his eldest son, aged 22, with an axe as he slept. }
 
\par{46. ${\overset{\textnormal{わたし}}{\text{私}}}$ は ${\overset{\textnormal{なぜ}}{\text{何故}}}$ 、 ${\overset{\textnormal{じなん}}{\text{次男}}}$ を ${\overset{\textnormal{あい}}{\text{愛}}}$ せないのか、 ${\overset{\textnormal{まいにち}}{\text{毎日}}}$ そのことを ${\overset{\textnormal{かんが}}{\text{考}}}$ えて ${\overset{\textnormal{かっとう}}{\text{葛藤}}}$ していました。 \hfill\break
I have thought and been conflicted every day about why it is that I can\textquotesingle t love my second son. }
 
\par{47. ${\overset{\textnormal{ちょうなん}}{\text{長男}}}$ 、 ${\overset{\textnormal{じなん}}{\text{次男}}}$ とも ${\overset{\textnormal{いえ}}{\text{家}}}$ を ${\overset{\textnormal{つ}}{\text{継}}}$ ぐとかそういうことしたくないと ${\overset{\textnormal{い}}{\text{言}}}$ い、 ${\overset{\textnormal{さんなん}}{\text{三男}}}$ が ${\overset{\textnormal{あとつ}}{\text{跡継}}}$ ぎになりました。 \hfill\break
The eldest and second son both stated that they didn\textquotesingle t want to be the successor of the family or anything of the such, and so the third son became the successor. }
  \textbf{~女 } \hfill\break
 The counter ~ ${\overset{\textnormal{じょ}}{\text{女}}}$ counts daughters and is typically used only up till five. The phrases for “first daughter” and “second daughter” are set phrases. However, when used to just count offspring in general, 1 and 2 are expressed as usual.  
\begin{ltabulary}{|P|P|P|P|P|P|P|P|P|P|}
\hline 

1 &  \textbf{ちょうじょ(長女) \hfill\break
 }いちじょ & 2 & じじょ(次女) & 3 & さんじょ & 4 & よんじょ & 5 & ごじょ \\ \cline{1-10}

\end{ltabulary}
 
\par{\hfill\break
48. ${\overset{\textnormal{わたし}}{\text{私}}}$ は ${\overset{\textnormal{おっと}}{\text{夫}}}$ から ${\overset{\textnormal{ちょうじょ}}{\text{長女}}}$ の ${\overset{\textnormal{ディーエヌエイ}}{\text{DNA}}}$ ${\overset{\textnormal{かんてい}}{\text{鑑定}}}$ を ${\overset{\textnormal{もと}}{\text{求}}}$ められました。 \hfill\break
I was asked by my husband to have a DNA test done for our eldest daughter. }
49. ${\overset{\textnormal{わたし}}{\text{私}}}$ は ${\overset{\textnormal{ちょうじょ}}{\text{長女}}}$ が ${\overset{\textnormal{ご}}{\text{5}}}$ ${\overset{\textnormal{さい}}{\text{歳}}}$ のときに ${\overset{\textnormal{じじょ}}{\text{次女}}}$ を ${\overset{\textnormal{う}}{\text{産}}}$ みました。 \hfill\break
I gave birth to my second daughter when my eldest daughter was five years old. \hfill\break
 \hfill\break
50. ${\overset{\textnormal{かぞく}}{\text{家族}}}$ は ${\overset{\textnormal{つま}}{\text{妻}}}$ との ${\overset{\textnormal{あいだ}}{\text{間}}}$ に ${\overset{\textnormal{いち}}{\text{1}}}$ ${\overset{\textnormal{なん}}{\text{男}}}$ ${\overset{\textnormal{さん}}{\text{3}}}$ ${\overset{\textnormal{じょ}}{\text{女}}}$ がいます。 \hfill\break
The family has one son and three daughters between him and his wife.  
\begin{ltabulary}{|P|P|P|P|P|P|P|P|}
\hline 

1 &  & 2 &  & 3 &  & 4 &  \\ \cline{1-8}

5 &  & 6 &  & 7 &  & 8 &  \\ \cline{1-8}

9 &  & 10 &  & 100 &  & ? &  \\ \cline{1-8}

\end{ltabulary}
 
\begin{ltabulary}{|P|P|P|P|P|P|P|P|}
\hline 

1 &  & 2 &  & 3 &  & 4 &  \\ \cline{1-8}

5 &  & 6 &  & 7 &  & 8 &  \\ \cline{1-8}

9 &  & 10 &  & 100 &  & ? &  \\ \cline{1-8}

\end{ltabulary}
 
\begin{ltabulary}{|P|P|P|P|P|P|P|P|}
\hline 

1 &  & 2 &  & 3 &  & 4 &  \\ \cline{1-8}

5 &  & 6 &  & 7 &  & 8 &  \\ \cline{1-8}

9 &  & 10 &  & 100 &  & ? &  \\ \cline{1-8}

\end{ltabulary}
 
\begin{ltabulary}{|P|P|P|P|P|P|P|P|}
\hline 

1 &  & 2 &  & 3 &  & 4 &  \\ \cline{1-8}

5 &  & 6 &  & 7 &  & 8 &  \\ \cline{1-8}

9 &  & 10 &  & 100 &  & ? &  \\ \cline{1-8}

\end{ltabulary}
     