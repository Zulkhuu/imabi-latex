    
\chapter{~てある II}

\begin{center}
\begin{Large}
第184課: ~てある II: With を 
\end{Large}
\end{center}
 
\par{ In Japanese textbooks, ~を~てある is often described as being incorrect. Although it is very easy to misuse it, this is simply not the case. Historically speaking, it is a relatively recent speech pattern. It derived, however, from the duplicate nature of ~てある itself. It can show completion of preparation and motivation. For the former, we see that を shows up a lot. Now, it is time to learn a little bit more about this mysterious grammar point. }
      
\section{~を~てある ?}
 
\par{ ~を~てある is an odd grammar point as, overall, ~が~てある is more common. Statistically speaking, が is still far more common than を, an there are very few grammatical circumstances when only the latter is correct. These situations, one would imagine, would probably be questionable phrases anyway. }

\par{ ~を~てある stems from emphasizing the completion of preparation by an agent. In this case, the agent can clearly be oneself. This is unlike ~が~てある which implies an active agent purposefully doing something, but the nuance doesn't go beyond that. The involvement of the agent with the resultant state is much higher. This makes this grammar only applicable to 他動詞. However, we'll see later on that for a few rare 自動詞, this similar nuance of ~てある may be allowed. }

\par{\textbf{Particle Note }: For examples with は instead of を below, the underlining case particle is still を, but because of other factors, は is more natural. }

\par{1. 英語はもう予習してあるから、大丈夫だ。 \hfill\break
I've already prepared for English, so I'm OK. }

\par{2. 伊藤さんに来月の予定を話してありますか? \hfill\break
Have you gotten talks done for next month's plans to Ito-san? }

\par{3. 漢字を調べてありますか? \hfill\break
Have you gotten done researching the Kanji? }

\par{4. 予約をしてあります。 \hfill\break
I have made the reservation. }

\par{5. はい、もう飛行機のチケットを買ってあります。 \hfill\break
Yes, I've already bought plane tickets. }

\par{6. 子供がいたずらするといけないから、コンセントを抜いてある。 \hfill\break
The kids can't mess with the outlet, so I've unplugged it. }

\par{7. ご飯を作ってある。 \hfill\break
I've made dinner. }

\par{8. いつ地震が起こるかわからないので、災害セットを用意してある。 \hfill\break
I don't know when an earthquake will hit, so I've prepared a disaster kit. }

\par{9. 貴重品はロッカーに預けてある。 \hfill\break
I've stored by valuables in a locker. }

\par{10. 朝から洗濯物を外に干してあるのでそろそろ乾いただろう。 \hfill\break
I've had the laundry drying outside since the morning, so they should be dry soon. }

\par{11. 前に水漏れが起きたので、ここの水道は水を止めてある。 \hfill\break
There was a water leak earlier, so I've stopped the water supply here. }

\par{12. 外に置く予定なので、錆止めスプレーをかけてある。  \hfill\break
I plan to place it outside, so I've sprayed with anti-rusting spray. }

\par{ There are some phrases that are just incompatible with ~を~てある for various semantic conflicts. }

\par{13. 授業が始めてある \textrightarrow  授業を始めてある X \textrightarrow  授業が始まっている 〇 \hfill\break
Class has been started. }

\par{\textbf{自動詞+~てある }}

\par{ ~てある can seldom be seen after intransitive verbs. It is fair to say that these intransitive verbs take on \textbf{contextual }qualities as transitive verbs. Again, think 準備. }

\par{14. 明日の試験に備えて、たくさん寝てある。 \hfill\break
I\textquotesingle ve slept a lot in preparation for the exam tomorrow. }

\par{\textbf{Sentence Note }: The speaker having slept is has had\slash needed to do so. The use of 寝てある is heavily dictated by context, and even though it is possible, it is hardly ever used and a lot of people would still think it's wrong. 寝ておく would be far much better. }

\par{ But, why is 寝てある possible if ~てある should only be used with transitive verbs? I t just so happens that ${\overset{\textnormal{すいみん}}{\text{睡眠}}}$ を取る and ${\overset{\textnormal{きゅうよう}}{\text{休養}}}$ を取る exist, which mean "to take a rest". Thus, though 寝る may be intransitive, it's not hard to think of it in transitive terms. Fo r intransitive verbs that involve a subject doing some sort of motion\slash exercise, through relying on context, ~てある can be used to show the building up of such effect\slash prior preparation. }

\par{15. このコースは4回、5回\{泳いである 〇・泳いだ ◎・泳いでおいた ◎\}から、 ${\overset{\textnormal{とまど}}{\text{戸惑}}}$ うことはない。 \hfill\break
I've swam the course 4, 5 times, so there is nothing to be perplexed about. }

\par{\textbf{Phrase Note }: Similar non-English likes usages of ある can be seen in phrases like ことがある, which is used to show that one has done something before when used with the past tense of a verb. This, though, should not be confused with 自動詞+てある. }

\par{\textbf{Grammar Note }: This grammar is very similar to ~ておく. As we've seen, it is prescriptively the only correct phrase for preparatory action with intransitives. ~ておく can be used with all verbs of volition for it shows the agent is doing something in preparation for benefit. ~てある preparation is done out of necessity or command and often comes about from an accumulation of effect. }
      
\section{Comparing ~てある, ~ている, \& ~ておく}
 
\par{ Now that we have seen some particle issues with ~てある and how it means a lot like ~ておく, we need to expand this conversation to make larger connections between not just these two patterns but also with ~ている. First, consider the following information in the chart below. }

\begin{ltabulary}{|P|P|}
\hline 

~ておく & Action in advance or current state maintenance in thinking of a later benefit. \\ \cline{1-2}

~ている & In the continuation of an action\slash the continuation of a state after a \hfill\break
change\slash custom and or experience\slash unchanging. \\ \cline{1-2}

~てある & Self-confidence or condition in which handling has \hfill\break
finished by a certain necessity\slash obligation\slash command. \\ \cline{1-2}

\end{ltabulary}

\par{ In the case of ~ている・いた and ~てある・あった, the base time simply goes from present to past, but the time that Verb A expresses respectively doesn\textquotesingle t change. But, ~ておく・おいた shows a strange behavior. Without sliding the base time, it has the time frame of Verb A to go from “unrealized” to “already realized”. This is because of the original nature of the verb 置く. }

\par{ Consider this. 行きます and 食べます express an action that is unrealized at the present base time. On the other hand, 行きました and 食べました show that the action has formerly realized at the present base time. This is just like ~ておく and ~ておいた. }

\par{ Again, consider 呼んでおく and 呼んでおいた. The Verb A 呼ぶ is unrealized in the former and realized beforehand in the latter. However, in the both cases, the base time is the present. In the first, you are going to do something in preparation for a certain circumstance. In the latter, you have made those preparations, but the circumstance has not occurred. Thus, the base time is the same. }

\par{ ある and いる take such strong grammatical burdens that they lose much of their independent properties; however, as the previous section concerning the verb 置く, おく\textquotesingle s original properties play a direct role in the usage of ~ておく. }

\par{ \textbf{~ておく, }\textbf{~てある, \& }\textbf{~ている Interchangeability }}

\par{ There is interchangeability among these items when Verb A is already realized in the present base time. Of course, the nuances are not the same. However, this doesn't negate the interchangeability. }

\par{1. XがYをAて\{ある・おいた・いる\}: Handling finished by X }

\par{16. 教科書は借りて\{あった・おいた・いた\}が、結局、読まなかった。 \hfill\break
I had borrowed the textbook, but in the end, I didn't read it. }

\par{17. 夕食は用意して\{ある・おいた・いる\}から、食べていってね。 \hfill\break
Dinner has been prepared, so go eat. }

\par{18. 明日、日本語の試験でしょう?ちゃんと勉強して\{ある・おいた・いる\}のか。 \hfill\break
Tomorrow\textquotesingle s the Japanese exam, right? Have you gotten your studying done? }

\par{19. 今日という日のために買い込んで\{あった・おいた・いた\}んだよ。 \hfill\break
I have had it bought up for a day such as today. }

\par{2. XがYをAて\{ある・いる・おく\}: A result of an action done by X continues }

\par{20. 思い出のある ${\overset{\textnormal{きちょうひん}}{\text{貴重品}}}$ を、 ${\overset{\textnormal{す}}{\text{捨}}}$ てずに残して\{ある・いる・おく\}。 \hfill\break
Keeping one's valuables with memories instead of throwing them away. }

\par{ Of course, even in this situation, if there is a meaning of handling being finished, ~ておく becomes ~ておいた. Also, again, though one translation may be given for each example, remember that the translations are only taking advantage of English ambiguity and that you should still keep in mind the meaning differences discussed earlier. }

\par{\textbf{Further Differentiating }\textbf{~ておく, }\textbf{~ている, \& }\textbf{~てある }}

\par{ Consider the following sentence where the three patterns could be used and think how the meaning of the sentence changes. }

\par{21. あのカップルは、よく ${\overset{\textnormal{けんか}}{\text{喧嘩}}}$ するから、席を離して\{ある・おいた・いる\}。 \hfill\break
Because that couple fights a lot, their sets [have been kept\slash were\slash were made]separate. }

\par{\textbf{席を離してある }: Shows the result of inevitably dealing with the problem by separating their seats apart from each other. The visual effect isn\textquotesingle t to the point of 「席が離してある」, but since the two end up going up to each other when they take their eyes off each other, a sense of effort in keeping their seats apart is felt, and a dynamic image is presented. }

\par{\textbf{席を離しておいた }: Rather than inevitability but by necessity, you are informing someone after dealing with it with benefit as being the goal. }

\par{\textbf{席を離している }: Shows that one is maintaining the situation after initially dealing with it. }

\par{ It\textquotesingle s fair to say that both ~てある and ~ておく have not a particular meaning of showing preparation, but rather, their interpretation is based on context. This is because the strongest feeling of “preparation” is when something \textbf{has }been done beforehand. }

\begin{ltabulary}{|P|P|P|P|P|P|}
\hline 

パターン & 場面 & 視覚効果 & 処理の意志 & 持続の意図 & 原因・目的 \\ \cline{1-6}

~ている & 静的 & X & X & X & X \\ \cline{1-6}

~てある & 動的 & △ & 処理済 & 〇 & 必要性など \\ \cline{1-6}

~ておく & 動的 & X & 〇 & 〇 & 都合のよさ \\ \cline{1-6}

~ておいた & 静的 & X & 処理済 & ? & 都合のよさ \\ \cline{1-6}

\end{ltabulary}

\par{\textbf{Terminology Notes }: ${\overset{\textnormal{しかくこうか}}{\text{視覚効果}}}$ = Visual Effect; 処理の意志 = Volition of Handling; 持続の意図 = Intentions of Continuation; 静的 = Static; 動的 = Dynamic; ${\overset{\textnormal{ず}}{\text{済}}}$ (み) = Resolved; Completed }

\par{\textbf{With Transitivity Verb Pairs } }

\par{ With all of these patterns and then adding transitivity into the problem, there can be at times six possible options. Of course, options in Japanese are never 100\% synonymous, but in such instances where things happen to be very similar, you definitely need to understand the differences. }

\par{22a. そのテレビがついているのは、お昼のニュースを見るためですよ。 \hfill\break
22b. そのテレビをつけているのは、お昼のニュースを見るためですよ。 \hfill\break
22c. そのテレビがつけてあるのは、お昼のニュースを見るためですよ。 \hfill\break
22d. そのテレビをつけてあるのは、お昼のニュースを見るためですよ。 \hfill\break
22e. そのテレビをつけておくのは、お昼のニュースを見るためですよ。 \hfill\break
22f. そのテレビをつけておいたのは、お昼のニュースを見るためですよ。 }

\par{ All of these sentence show the condition after the TV is turned on. However, \#2 also has the possible reading of currently turning on the TV. }

\par{ Imagine the situation is someone asking why the TV is on. In such a situation, the speaker could use the aforementioned sentences to respond in the following ways. }

\par{\textbf{テレビがついている }: The TV is just on. There is no image of who may have turned it on. Even if the speaker were the one that turned it on, from this statement, you wouldn't know. }

\par{\textbf{テレビをつけている }: A nonspecific someone has put the TV on. Even if the speaker turned it on, the speaker asks as if he\slash she doesn't know, and the phrase simply infers that someone willfully turned the TV on. }

\par{\textbf{テレビがつけてある }: Someone not the speaker has gone through the trouble of turning the TV on. The person that turned it on felt a necessity or duty to turn it on for the purpose of news watching. Even if the speaker were the one to turn on the TV, the speaker is not telling that to the listener. }

\par{\textbf{テレビをつけてある }: The speaker has gone through the trouble of turning on the TV. \hfill\break
Even if it is not the speaker, someone managing the TV with a responsibility to deal with it did. }

\par{\textbf{テレビをつけておく }: The speaker or a particular person maintain the state of the TV being turned on. The one who turned on the TV saw benefit in watching the news and is the one trying to maintain the state of the TV turned on for that purpose. }

\par{\textbf{テレビをつけておいた }: The speaker or a particular person went through the trouble of turning on the TV, but it is uncertain where from that point that individual is trying to maintain the state of the TV being on for the purpose of watching the news. In other words, it doesn't even show whether the TV is still on or not.  }

\par{\textbf{Summarization of the Usages of ~ておく, ~ている, \& ~てある } }

\begin{ltabulary}{|P|P|P|}
\hline 

Pattern & Time Relation & Usage(s) \\ \cline{1-3}

が + 自動詞 + ている & Realized beforehand & Condition after a natural change \\ \cline{1-3}

を + 他動詞 + ている & Realized beforehand \hfill\break
Same time frame & Condition\slash effect after you make a change \hfill\break
In the act of doing something \\ \cline{1-3}

が + 他動詞 + てある & Realized beforehand & Condition after a change is made purposely by someone \\ \cline{1-3}

を + 他動詞 + てある & Realized beforehand & Condition after a change is made by the speaker\slash agent. \\ \cline{1-3}

を + 他動詞 + ておく & Unrealized & Will in favorably changing a circumstance. \hfill\break
Will in maintaining a favorable position \\ \cline{1-3}

を + 他動詞 + ておいた & Realized beforehand & Effect\slash condition one changes into a favorable outcome. \hfill\break
The condition caused may or may not end up favorable. \\ \cline{1-3}

\end{ltabulary}

\par{The chart above details the differences that have been discussed up to this point between these three expressions. Now, the next chart shows when they \textbf{mustn't }ever be interchanged with each other. These are listed in the chart above as well. }

\begin{ltabulary}{|P|P|P|}
\hline 

Pattern & Time Relation & Usage(s) \\ \cline{1-3}

を + 他動詞 + ている & Same time frame & Progressive action \\ \cline{1-3}

を + 他動詞 + ておく & Unrealized & Preparing in advance \\ \cline{1-3}

を + 他動詞 + ておいた & Already realized & When the outcome isn't beneficial \hfill\break
When preparations for the future have been completed \\ \cline{1-3}

\end{ltabulary}

\par{\textbf{Grammar Notes }: }

\par{1. As we've seen before, when ~ておく is used in showing custom or repeated action, interchangeability is made as it then shows what has previously happened before. }

\par{2. When ~ている is used to show a natural change, then interchangeability is impossible. Remember the TV example? TVs are turned on by people most of the time. So, テレビがついている can be rephrased and still make valid yet different sentences. A sentence like the following can\textquotesingle t be rephrased. }

\par{23. お酒、すっかり冷めてるんじゃない?温めましょう。〇 \hfill\break
The sake\textquotesingle s completely chilled down? Let\textquotesingle s heat it up. }

\par{3. However, if the speaker wants to emphasize that something is not just a natural change, given that this is logical to assume, 自動詞+ている \textrightarrow  他動詞+ている is allowed. }

\par{24. 肩もだいぶ\{凝ってる ◎・凝らしてる\}ね。 \hfill\break
My shoulder\textquotesingle s also quite stiff. }
    