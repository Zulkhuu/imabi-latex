    
\chapter{The Particle しも}

\begin{center}
\begin{Large}
第183課: The Particle しも 
\end{Large}
\end{center}
  This particle is very limited in use, but it is not that difficult.       
\section{The Adverbial Particle しも}
 
\par{ しも is essentially an emphatic し. However, its use in a sentence reflects a more productive use of し itself. Though we often see し in the spoken language today, the particle has existed for a long time, and the combination of the emphatic し and emphatic も has been around just as long. Although しも has survived along with し, its usage is primarily restricted to the following phrases. Notice how it is designated to nominal (or nominalized) phrases or after adverbial phrases. }

\par{ Aside from the last phrase 折しも, all of these phrases are used in negative sentences. This just goes to show you how many restrictions are on its use, and it's no surprise that most of these phrases are most frequently used in 書き言葉. }

\begin{ltabulary}{|P|P|P|P|}
\hline 

誰しも & Everyone, anyone (very emphatic) & ~ならまだしも & It\textquotesingle s one thing, but\dothyp{}\dothyp{}\dothyp{} \\ \cline{1-4}

必ずしも & Not necessarily & ~なきにしもあらず & It's not to say that\dothyp{}\dothyp{}\dothyp{}won't \\ \cline{1-4}

折しも & Just then &  &  \\ \cline{1-4}

\end{ltabulary}

\par{\textbf{Phrase Note }: ${\overset{\textnormal{かなら}}{\text{必}}}$ ずしも is often paired with ~とは ${\overset{\textnormal{かぎ}}{\text{限}}}$ らない ending the sentence. }

\par{\textbf{Speech Style Note }: ~なきにしもあらず and 折しも are especially 書き言葉. }

\par{\textbf{Variant Note }: A rarer variant of 折しも is 時しも. This essentially does not show up in Modern Japanese works, but it does show up sometimes in Early Modern Japanese works. Meaning wise, 折 and 時 mean the same thing here. }

\par{1. 誰しも ${\overset{\textnormal{じごく}}{\text{地獄}}}$ へ ${\overset{\textnormal{お}}{\text{落}}}$ ちるのは ${\overset{\textnormal{こわ}}{\text{怖}}}$ い。 \hfill\break
Everyone is afraid of going to hell. }

\par{2. ${\overset{\textnormal{せいりょく}}{\text{勢力}}}$ はそれ ${\overset{\textnormal{じたい}}{\text{自体}}}$ では必ずしも ${\overset{\textnormal{こうふく}}{\text{幸福}}}$ をもたらすとは限らない。 \hfill\break
Power, in itself, doesn't necessarily bring happiness. }

\par{3. 折しも、地震が起きました。 \hfill\break
Just at that time, the earthquake occurred. }

\par{4. 折しも、雪崩が発生し、登山者の二人は行方不明となった。依然として行方不明のままである。 \hfill\break
Just then, the avalanche was sparked, and the two mountain climbers went missing. They are still to this moment unaccounted for. }

\begin{center}
\textbf{~ならまだしも }
\end{center}

\par{ This phrase can be used after nouns, verbs, and adjectives. For verbs and adjectives, you attach it to the 終止形. For 形容動詞, simply add after the stem. }

\par{5. 1日か2つかならまだしも、10日も無断欠勤だなんて、許されないものだし、非常識だ。 \hfill\break
It's one thing to be 1 or 2 days, but an over ten day unexcused leave is intolerable and against common sense. }

\par{6. 新鮮ならまだしも、変色して黒ずんでいる果物を誰が買うものなのか。   \hfill\break
Being fresh is one thing, but who would ever by fruit that's discolored and black? }
7. 日本語ならまだしも、 ${\overset{\textnormal{えいご}}{\text{英語}}}$ なんて全く全然分からないよ。 \hfill\break
Japanese is one thing, but I absolutely don't understand English at all! 
\par{8. 寒いだけならまだしも、お腹が空いてきた。  \hfill\break
If it were just cold, that would be one thing, but I've gotten hungry. }

\par{9. 事情を説明しに来るならまだしも、顔さえ見せない。 \hfill\break
Coming to explain the situation is one thing, but (he) won't even show his face. }

\par{10. 一度ならまだしも、ここまで10回までその言葉を間違えて書いたんですよ。 \hfill\break
Once is fine, but you've written the word incorrectly ten times now. }

\par{11. まだしも死んだ方が良い。 \hfill\break
It would be best to just die. }

\begin{center}
\textbf{~なきにしもあらず } 
\end{center}

\par{ This is a double negative phrase which functions as a positive expression, and it ultimately has the meaning of 有り得る. Although it is a predicate phrase, it is still followed by the copula. Remember thatしも is here to show emphasis (強調). It is seldom used in the spoken language, but it can still show up. }

\par{12. あの子はまだ ${\overset{\textnormal{のぞ}}{\text{望}}}$ みはなきにしもあらずだ。 \hfill\break
It's not to say that the kid doesn't have (any) hope. }

\par{13. 台風が接近しているので、雨が降る事もなきにしもあらずなので、傘をお忘れなく。 \hfill\break
The typhoon is approaching, so don't forget your umbrella because it's not like it couldn't rain. }

\par{14. 急にカメラが壊れることもなきにしもあらずですよ。 \hfill\break
It's not to say that your camera won't suddenly break down. }

\par{15. 後数分で事故などで死ぬ事もなきにしもあらずだからだ。 \hfill\break
That's because it's not the case that you won't die in an accident or something a few minutes later. }
 
\begin{center}
\textbf{With Other Particles? }
\end{center}
 
\par{16. 男たちは縁側で将棋に興じている。街路樹のプラタナスの葉ずれ。ああいうのをしも、人間の文化といわずし て、何というのだろう。 \hfill\break
The men are amusing themselves with shogi on the veranda while plane trees rustle on the sides of the road. What would you call this if not human culture? \hfill\break
By 田辺聖子in 古川柳おちぼひろい. }
 
\par{\textbf{Grammar Note }: The particle しも used to be more versatile in the past. In the example above, the particle is used after を. This is very rare now, but it is not ungrammatical. }

\begin{center}
\textbf{必ず VS }\textbf{きっと VS }\textbf{ ${\overset{\textnormal{ぜったい}}{\text{絶対}}}$ \textbf{( }\textbf{に) }}
\end{center}

\par{ So, given that you have now seen these three similar words for quite a while, you're probably wondering how they're different. There is overlap. So, focus not only on the differences but also the commonalities. }

\par{\textbf{ きっと }:  }

\begin{itemize}
 
\item Based on observation, the probability of something happening is      high.  It may show strong determination or show strong will towards      the listener(s). It is like "surely". It may be seen at the      beginning, middle, or at the end of a sentence.  
\item It can also mean "certainly", synonymous with ${\overset{\textnormal{たし}}{\text{確}}}$ かに and ${\overset{\textnormal{うたが}}{\text{疑}}}$ いなく. Therefore it isn't used in a question.      In a command it is like "without fail", just like 必ず.  
\item きっと\dothyp{}\dothyp{}\dothyp{}する = "to be      sure to\dothyp{}\dothyp{}\dothyp{}".  
\item きっと\dothyp{}\dothyp{}\dothyp{}だ = "have      to be"  
\item きっと\dothyp{}\dothyp{}\dothyp{}に ${\overset{\textnormal{ちが}}{\text{違}}}$ いない = "must be\dothyp{}\dothyp{}\dothyp{}".  
\end{itemize}
It can also show sternness. 
\par{\textbf{必ず }: }

\begin{itemize}

\item Without a doubt, something will be done or happen. It is far      more firm. Probability is 100\%. It is like "always", and in      meaning so it makes a general noun the subject. 
\item It can mean "surely" just like きっと and be seen      in the same locations in the sentence. 
\item 必ずや is an even      more emphatic form. 
\item "Necessarily" as in something is inevitable,      interchangeable with ${\overset{\textnormal{ひつぜんてき}}{\text{必然的}}}$ に. 
\item In a command, it means "by all means\slash without      fail". 
\item Some commands with the pattern 必ず\dothyp{}\dothyp{}\dothyp{}する may mean      "be sure to\slash make certain". 
\item 必ずしも~ない      completely negates something and is equivalent to 絶対そう\dothyp{}\dothyp{}\dothyp{}とは限らない. 
\item 必ず is much more      serious than きっと despite that      they're used in similar environments. 
\end{itemize}

\par{\textbf{絶対(に) }: }

\begin{itemize}

\item No matter what 
\item Positively\slash definitely 
\item With the negative it means "never". 
\item It is often used with phrases that mean "must",      "will" and "would" to be similar to "on not account",      clearly in a negative sentence. 
\item Unlike the others, it is more constructive and can be used as      an noun and used as an attribute as 絶対の・ ${\overset{\textnormal{ぜったいてき}}{\text{絶対的}}}$ な to mean      "absolute\slash indispensable". 
\end{itemize}

\par{Some of these things feature grammar points that we haven't studied yet, but you should know the overall usage of these three words. }
17. 絶対に ${\overset{\textnormal{かくしん}}{\text{確信}}}$ があります。 I'm absolutely sure. 
\par{18. 絶対にそうだ。 \hfill\break
There's no doubt about it. }

\par{${\overset{\textnormal{}}{\text{19. 絶対零度を測定する}}}$ \hfill\break
To record absolute zero }

\par{20. 彼女は必ずしも忙しくない。 \hfill\break
She's not always busy. }

\par{${\overset{\textnormal{}}{\text{21. 戦争}}}$ は必ず起こる。 \hfill\break
War will inevitably occur. }

\par{22. 必ず ${\overset{\textnormal{やくそく}}{\text{約束}}}$ を ${\overset{\textnormal{まも}}{\text{守}}}$ ってください。 \hfill\break
Do not fail to keep your word\slash promise. }

\par{${\overset{\textnormal{}}{\text{23. 明日中}}}$ にはきっと ${\overset{\textnormal{うかが}}{\text{伺}}}$ います。 \hfill\break
I will certainly come sometime tomorrow. }

\par{24. 絶対的な ${\overset{\textnormal{けんりょく}}{\text{権力}}}$ を握る。 \hfill\break
To grab absolute power }

\par{25. 僕らはきっと勝つ。 \hfill\break
We will surely win. }

\par{26. きっとだ、 ${\overset{\textnormal{まちが}}{\text{間違}}}$ いない。 \hfill\break
I'll be bound. }

\par{27. それは絶対だめだよ。 \hfill\break
That'll never do. }

\par{28. そんなことを絶対にしてはいけません。 \hfill\break
You must never do something like that. }
    