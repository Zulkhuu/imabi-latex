    
\chapter{Interjections}

\begin{center}
\begin{Large}
第200課: Interjections  
\end{Large}
\end{center}
 
\par{ There are several terms for "interjection". They may be called ${\overset{\textnormal{かんどうし}}{\text{感動詞}}}$ , ${\overset{\textnormal{かんとうし}}{\text{間投詞}}}$ , or ${\overset{\textnormal{かんたんし}}{\text{感嘆詞}}}$ . Interjections may be outbursts of emotion while others are simply greetings ( ${\overset{\textnormal{あいさつ}}{\text{挨拶}}}$ ). Some are nasty pejoratives, others simply internet slang. Due to this diversity, we are going to study interjections with the following categories. }

\begin{ltabulary}{|P|P|}
\hline 

Emotional Interjections & 感動詞 \\ \cline{1-2}

Interjections of Yelling & かけ声の間投詞 \\ \cline{1-2}

Interjections of Response & 応答の間投詞 \\ \cline{1-2}

Interjections of Appeal & 呼びかけの間投詞 \\ \cline{1-2}

Interjections of Salutation & 挨拶の間投詞 \\ \cline{1-2}

Slang Interjections \hfill\break
& 俗語的な間投詞 \\ \cline{1-2}

Onomatopoeic Interjections \hfill\break
& 擬声語的な間投詞 \\ \cline{1-2}

\end{ltabulary}

\par{ This lesson won't show every interjection nor every variant, but this should hopefully help find about the most important ones. }
      
\section{Emotional Interjections}
 
\begin{ltabulary}{|P|P|P|}
\hline 

Ah & ああ & It can be used to show surprise, grief, joy, etc. and also as a light response. In the latter case it's often just あ. \\ \cline{1-3}

Oh & おう・おお & Used when moved (emotionally) by something, sudden remember something, or agree with something. \\ \cline{1-3}

Thank you & どうもありがとう・おおきに・サンキュー・三Q & 大きに is famous for being from Kansai Dialects. サンキュー comes from English and is used in casual settings. The second spelling is common in internet slang. \\ \cline{1-3}

Oh my & おや(おや) & It shows a little distrust when met with something unexpected. \\ \cline{1-3}

My, my & おやまあ &  \\ \cline{1-3}

Huh & まったく &  \\ \cline{1-3}

Goodness & やれ(やれ) & Used when relieved, tired, dejected, and bewildered. \\ \cline{1-3}

\end{ltabulary}
      
\section{Interjections of Yelling}
  
\begin{ltabulary}{|P|P|P|}
\hline 

 ああ・わあ & Ah & わあ is often used by females but not limited to them. The adverb わあと also exists. \\ \cline{1-3}

よいしょ &  & Sound of when you are lifting up something heavy. \\ \cline{1-3}

えいや &  & Sound one makes when pulling something with strength. \\ \cline{1-3}

おい & Hey! & Comes from Satsuguu Dialect for punishing people. \\ \cline{1-3}

おっと & Oops & Used mainly by middle-aged people and older. \\ \cline{1-3}

乾杯 & Cheers &  \\ \cline{1-3}

きゃー & Eek &  \\ \cline{1-3}

ふん & Pssh &  \\ \cline{1-3}

しまった & Dang\slash damn &  \\ \cline{1-3}

図星 & Bull's-eye &  \\ \cline{1-3}

畜生 & Damn you &  \\ \cline{1-3}

はくしょん & Achoo &  \\ \cline{1-3}

ファック & F*** & This is not used like in English frequency-wise, and it's not a given that the speaker knows it is a bad word in English. \\ \cline{1-3}

ヤッホー & Yo-ho &  \\ \cline{1-3}

もう & Jeez &  \\ \cline{1-3}

すごい & Awesome &  \\ \cline{1-3}

やった & # did it! &  \\ \cline{1-3}

\end{ltabulary}
 Notes: 1. The first shows how っ can make more emphasis, and this can be used for other adjectives. The latter contraction, which can also occur for 形容詞 ending in おい, あい, うい, and いい to えー, which is done in standard speech by men in casual situations. It depends on the region if women get to use this contraction or not. \textbf{Examples } \textbf{やんや }の喝采になってしまった。 \hfill\break
It ended up becoming a thunderous applause. \textbf{ }
\begin{ltabulary}{|P|P|P|}
\hline 

 & Notes \\ \cline{1-3}

Ah & ああ・わあ & わあ is not limited to females. It can even be in わあと with verbs dealing with crying and joy. \\ \cline{1-3}

Yo-ho & よいしょ &  \\ \cline{1-3}

Ugh & えいや &  \\ \cline{1-3}

There! & それ・そら & The latter is more so dialectical and depends on the speaker. This pattern can be applied to things like これ as well. \\ \cline{1-3}

Shut up \hfill\break
& うるさい・だまれ・だまって & There are many other ways to "shut up" that differ in grammar and politeness. \\ \cline{1-3}

Hey! & おい・こら &  \\ \cline{1-3}

Ouch! & いたっ・いたい・いてぇ &  \\ \cline{1-3}

Oops & おっと &  \\ \cline{1-3}

Have it your way! & 勝手にしろ &  \\ \cline{1-3}

Good luck! & 頑張って &  \\ \cline{1-3}

Cheers! & 乾杯 &  \\ \cline{1-3}

Eek & きゃー &  \\ \cline{1-3}

\dothyp{}\dothyp{}\dothyp{}stinks & くさい・くせー &  \\ \cline{1-3}

S%イ# & くそ・ふん &  \\ \cline{1-3}

Dαムni\% & しまった &  \\ \cline{1-3}

Awesome & すごい・すげー &  \\ \cline{1-3}

Stop! & ストップ &  \\ \cline{1-3}

Bull\textquotesingle s-eye & 図星 &  \\ \cline{1-3}

Da\#\% you & 畜生 &  \\ \cline{1-3}

Voila & 出来上がり &  \\ \cline{1-3}

Achoo & ハクション &  \\ \cline{1-3}

F*\#K & ファック &  \\ \cline{1-3}

Jeez & もう &  \\ \cline{1-3}

Yes; hurray \hfill\break
& やった &  \\ \cline{1-3}

Yoohoo & ヤッホー &  \\ \cline{1-3}

\end{ltabulary}

\begin{ltabulary}{|P|P|P|}
\hline 

 & Notes \\ \cline{1-3}

Ah & ああ・わあ & わあ is not limited to females. It can even be in わあと with verbs dealing with crying and joy. \\ \cline{1-3}

Yo-ho & よいしょ &  \\ \cline{1-3}

Ugh & えいや &  \\ \cline{1-3}

There! & それ・そら & The latter is more so dialectical and depends on the speaker. This pattern can be applied to things like これ as well. \\ \cline{1-3}

Shut up \hfill\break
& うるさい・だまれ・だまって & There are many other ways to "shut up" that differ in grammar and politeness. \\ \cline{1-3}

Hey! & おい・こら &  \\ \cline{1-3}

Ouch! & いたっ・いたい・いてぇ &  \\ \cline{1-3}

Oops & おっと &  \\ \cline{1-3}

Have it your way! & 勝手にしろ &  \\ \cline{1-3}

Good luck! & 頑張って &  \\ \cline{1-3}

Cheers! & 乾杯 &  \\ \cline{1-3}

Eek & きゃー &  \\ \cline{1-3}

\dothyp{}\dothyp{}\dothyp{}stinks & くさい・くせー &  \\ \cline{1-3}

S%イ# & くそ・ふん &  \\ \cline{1-3}

Dαムni\% & しまった &  \\ \cline{1-3}

Awesome & すごい・すげー &  \\ \cline{1-3}

Stop! & ストップ &  \\ \cline{1-3}

Bull\textquotesingle s-eye & 図星 &  \\ \cline{1-3}

Da\#\% you & 畜生 &  \\ \cline{1-3}

Voila & 出来上がり &  \\ \cline{1-3}

Achoo & ハクション &  \\ \cline{1-3}

F*\#K & ファック &  \\ \cline{1-3}

Jeez & もう &  \\ \cline{1-3}

Yes; hurray \hfill\break
& やった &  \\ \cline{1-3}

Yoohoo & ヤッホー &  \\ \cline{1-3}

\end{ltabulary}
      
\section{Interjections of Response}
 \hfill\break

\begin{ltabulary}{|P|P|P|}
\hline 

 &  & Notes \\ \cline{1-3}

Ah & ああ・あ &  \\ \cline{1-3}

Uh; um & あの(う) &  \\ \cline{1-3}

Well & じゃあ・さあ・まあ & See below \\ \cline{1-3}

Um & えーと & えーと is a very casual spelling. ええと and えっと are regular spellings. \\ \cline{1-3}

No & い(い)え・いや・ううん・や(ー)だ & やだ is quite casual and is more like "that's bad". \\ \cline{1-3}

Yes & はい・ええ・うん &  \\ \cline{1-3}

OK & OK・オーケー &  \\ \cline{1-3}

Eh? & え(っ) & May show doubt when surprised at something or when you want something to be repeated. \\ \cline{1-3}

Oh, yes? & へえ & Shows surprise at something. \\ \cline{1-3}

Sorry & ・ ごめんなさい \hfill\break
・申しわけありません \hfill\break
・すみません &  \\ \cline{1-3}

Uh-huh & しか &  \\ \cline{1-3}

Yeah & そうですね & This is even said when you will then show disagreement. In that case, it would be followed with something like そうかといって. \\ \cline{1-3}

Not at all \hfill\break
& とんでもない & See below. \\ \cline{1-3}

What? & 何 & Of course, it has variants like 何だ. \\ \cline{1-3}

Indeed & なるほど & Used in agreement or recognition of what someone said. \\ \cline{1-3}

Let me see & はて・はてな & Used when one is suspecting or thinking about something. \\ \cline{1-3}

Not really \hfill\break
& 別に & Maybe accompanied by the verb that is omitted. \\ \cline{1-3}

No use arguing & 問答無用 &  \\ \cline{1-3}

Rodger & 了解 &  \\ \cline{1-3}

Wow & あら & Often used by women. \\ \cline{1-3}

Of course \hfill\break
& もちろん &  \\ \cline{1-3}

Hey there; I say \hfill\break
& これは(これは) &  \\ \cline{1-3}

\end{ltabulary}
 \textbf{さあ VS まあ VS じゃあ }  Though all translated with relatively the same word "well", these three interjections aren't exactly the same, and they're used in different situations for different emotive effects.   さあ can be used to urge or invite someone to do something. It may be used to bolster oneself or even used when one is confused or troubled. It can also be the well in "well, that's done".   まあ is quite different. Women often use it whenever they're shocked, relieved, etc. In general, however, it can be used to suggest that something is not completely all right, but that it will suffice for now. Or, it can be used to pacify someone and tell them to do something for the time being.   じゃあ ・じゃ・では is like "ok now". It can be used like それでは as "then", which can also be してみるお. It can even be used to switch over to another conversation.   \textbf{とんでもない }   It can be used to completely reject another's idea and is equivalent to めっそうもない.It can also be used like とほうもない, which means "outrageous\slash absurd". It can be seen as "no kidding!". It's often used in a negative fashion.   In polite speech it can be more easily interpreted as "it's nothing". Or, it can be used in the sense of "don't mention it". Of course, a lot of this has to deal with context and tone of voice. For its meaning "unthinkable", it is interchangeable with とんだ・思いのほかの, which are both attributive expressions.        
\section{Interjections of Appeal}
 
\begin{ltabulary}{|P|P|}
\hline 

Sh! & しーっ \\ \cline{1-2}

I'm leaving & 行ってきます \\ \cline{1-2}

Have a nice day & 行ってらっしゃい \\ \cline{1-2}

Wait & ちょっと \\ \cline{1-2}

Please & お願いします・どうぞ \\ \cline{1-2}

Pretty please & かなりしてください \\ \cline{1-2}

Hello (on the telephone) & もしもし \\ \cline{1-2}

Good job & あっぱれ・グッジョブ \\ \cline{1-2}

Excuse me \hfill\break
& 失礼します \\ \cline{1-2}

Help & 助けて \\ \cline{1-2}

I'm home & ただいま \\ \cline{1-2}

\end{ltabulary}
      
\section{Interjections of Salutation}
 
\begin{ltabulary}{|P|P|}
\hline 

So long & あばよ \\ \cline{1-2}

Welcome home & お帰りなさい \\ \cline{1-2}

Congratulations & おめでとう(ございます) \\ \cline{1-2}

Happy birthday & (お)誕生日おめでとう(ございます) \\ \cline{1-2}

Good morning \hfill\break
& おはようございます \\ \cline{1-2}

Good afternoon \hfill\break
& 今日は \\ \cline{1-2}

Good evening \hfill\break
& 今晩は \\ \cline{1-2}

Good night \hfill\break
& お休みなさい \\ \cline{1-2}

How are you?; farewell & ご機嫌よう \\ \cline{1-2}

Farewell & さよ(う)なら・さらば \\ \cline{1-2}

See you later & また明日・あとで \\ \cline{1-2}

What's up? & どう \\ \cline{1-2}

Bye & ばい・バイバイ \\ \cline{1-2}

Long time no see & 久しぶりですね・久々・調子はどう? \\ \cline{1-2}

Merry Christmas & メリークリ(スマス) \\ \cline{1-2}

Yo!; Wow! & やあ \\ \cline{1-2}

Nice to meet you & どうぞよろしく・初めまして \\ \cline{1-2}

\end{ltabulary}
      
\section{Slangish Interjections}
  
\begin{ltabulary}{|P|P|}
\hline 

 What's up & おっす \\ \cline{1-2}

Here it comes & キタ \\ \cline{1-2}

Nice to meet you \hfill\break
& 4649 (Internet) \hfill\break
\\ \cline{1-2}

Lol & 笑・wwww (Internet)・ワロタ \\ \cline{1-2}

\end{ltabulary}
 
\par{\textbf{Usage Note }: Words that come from the infamous site 2ちゃんねる are deemed to be used by people with low intelligence. So, avoid using ワロタ. }
      
\section{Onomatopoeic Interjections}
 
\begin{ltabulary}{|P|P|}
\hline 

Ook & ウキー \\ \cline{1-2}

Caw & かー \\ \cline{1-2}

Ook ook \hfill\break
& きーきー \\ \cline{1-2}

Yuck & げっ \\ \cline{1-2}

Ribbit ribbit & げろげろ \\ \cline{1-2}

Cock-a-doodle-doo & こけこっこー \\ \cline{1-2}

Purr; meow \hfill\break
& ごろごろ・にゃあ \\ \cline{1-2}

Baa baa & めーめー \\ \cline{1-2}

\end{ltabulary}
    