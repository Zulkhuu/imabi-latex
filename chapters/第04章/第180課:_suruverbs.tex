    
\chapter{Nouns \textrightarrow  Verbs with する}

\begin{center}
\begin{Large}
第180課: Nouns \textrightarrow  Verbs with する 
\end{Large}
\end{center}
 
\par{ Numerous verbs are made with する attached to nouns. する is primarily seen after ${\overset{\textnormal{かんごめいし}}{\text{漢語名詞}}}$ , nouns that are Sino-Japanese (Chinese based) in origin or composition, for this. However, する is not limited to these nouns. It is also seen after native words and recent loanwords. }

\begin{ltabulary}{|P|P|P|P|}
\hline 

To study & 勉強する & To buy and sell & 売買する \\ \cline{1-4}

To eat and drink & 飲食する & To point out & 指図する \\ \cline{1-4}

To oversleep & 寝坊する & To exchange (money) & 両替する \\ \cline{1-4}

To love (sexual relation) & 恋する & To sweat & 汗する \\ \cline{1-4}

To copy & コピーする & To sign & サインする \\ \cline{1-4}

To design & デザインする & To cancel & キャンセルする \\ \cline{1-4}

\end{ltabulary}

\par{  When する attaches to these words, it loses its literal meaning and places a grammatical function instead. In this sense, we say that it has become grammaticalized. }

\par{ There are problematic restrictions on this usage of する. Even though ${\overset{\textnormal{きょうきゅう}}{\text{供給}}}$ (supply) is the opposite of ${\overset{\textnormal{じゅよう}}{\text{需要}}}$ (demand), only 供給 can be followed by する. It turns out that verbal-like nouns get する verb forms, but others not so verbal don't. This lesson will investigate what exactly the patterns are to figuring this out. Particles are also confusing. Some nouns become verbalized by adding ~する, ~をする, ~にする, ~になる, a combination of these. This does not consider the particle to be used before the resultant verb phrase. }

\par{\textbf{Curriculum Note }: The level of this lesson is currently under evaluation. Once a decision has been made, reading aids will be added to suit the level chosen. }
      
\section{Initial Problems}
 
\par{ Our first problem is when to use ~をする. Many students learn about 練習(を)する (to practice ) and 勉強(を)する (to study). But, extending ~をする haphazardly to any する verb isn't smart. There are situations in which only ~する is right. In other situations, ~をする is obligatory. Thus, ~する would be ungrammatical.  For other nouns, neither is correct and other means of making them verbal, if possible, must be used. }

\par{ One thing you have to be careful of is grammatical context. These judgments are based for non-casual speech and when the phrase is used without any other attribute phrases. For instance, テニスする is used in casual speech because を is unconditionally dropped in casual speech. }

\begin{ltabulary}{|P|P|P|P|}
\hline 

To play tennis & テニス\{をする 〇・する △\} & To play baseball & 野球\{をする 〇  ・する △\} \\ \cline{1-4}

To mean\slash imply & 意味\{する 〇・をする\} X & To imply & 含意\{する 〇・をする X\} \\ \cline{1-4}

To consult & 参考\{する X・をする X・にする 〇\} & To be in a daze & 夢中\{する X ・をする X・になる 〇\} \\ \cline{1-4}

\end{ltabulary}

\par{\textbf{Clarification Note }: 意味をする may be used if you have ~という意味 or ~のような意味. However, the verb 意味する is not quite the same as just to define. Rather, it is more like to mean as in implication. You can't take this chart and run with it. Like always, be careful of understanding the entire picture. This includes knowing any nuance differences with familiar looking phrases. So, when we focus on this 意味する as a separate vocabulary word and try to insert a を, you would get を意味をする. It may also mean "to stand for", but this is still slightly different than ~という意味をする(〇〇). }

\par{\textbf{Particle Note }: Look how になる・する appear to save the day. }

\par{When the particle を is required, する is interpreted literally . This する is sometimes called a 重動詞 (heavy verb) in contrast to when it is just a grammatical item. Notice that the words that only take をする are specific activities: 強盗, テニス, 野球. Are these nouns then made verbs? No, they're obligatorily marked by を and are thus functioning as direct objects. }

\par{ For activity nouns that are broad in meaning such as 勉強 and 練習, we see that を is optional. When を is used, these nouns usually have some sort of attribute, making them more specific. }

\par{1. 日本語の勉強をする。 \hfill\break
To study Japanese. \hfill\break
Literally: To do Japanese studies. }

\par{ For those that obligatorily only take する, they all represent a state\slash situation. More examples of these nouns include 刺殺 (stabbing to death), 集中 (concentration), 信用 (trust), 逆転 (reversal). }

\par{ When する cannot be used at all, the noun lacks a strong verbal aspect. How do you know this? There must be certain conditions to look at. Ignoring set phrases that may break the rules, there are a set of tests to determine whether a noun has a high 動詞性 (verb nature) and can then take する. }
      
\section{The Tests}
 
\par{ These tests predict what nouns have verb forms. You already know what nouns that take する look like and come from. So, these tests are simply here to help you more definitively figure out whether a noun has one or not. However, if you don't understand why things are, you open yourself up to making unnecessary mistakes. }

\begin{center}
\textbf{Test 1: Aspect Modifier Test }
\end{center}

\par{ Time phrases like “until" (まで) in Japanese set aspect limitations on verbs. They make verbs have a punctual end. If a noun is able to agree with these time phrases, it should have a high 動詞性 and thus have a verb form. When we do this test on the following native noun, it passes. }

\par{2. 明日までの 仕上がり   合格: 動詞性が高い \textrightarrow  V.P  明日までに仕上がる \hfill\break
Completion by tomorrow  To complete by tomorrow }

\par{ If you were to replace 仕上がり with 本 or 部屋, which are both Sino-Japanese nouns, they fail. It turns out that there is no such thing as 本する or 部屋する, validating this test. }

\par{ There is a drawback to this test. Noun forms of the verbs 痛む (to feel pain) and 信頼する (to trust\slash confide) come out negative. For these verbs, time parameters are not a part of their meanings. It\textquotesingle s not correct Japanese to say something like 夜間以内の痛み (pain within night time). 夜に痛む is fine, though. We've just hit a separate roadblock in the meaning of these words. }

\par{ This test enables us in theory to prove that nouns with a temporal end implied should have verb forms. So, in dealing with such 漢語名詞, they should have する verb forms. }

\par{ 3. 完了 (Completion) \textrightarrow  完了する 〇  完成 (Completion\slash perfection) \textrightarrow  完成する \hfill\break
 4. 計算 (Calculation) \textrightarrow  計算する 〇  算数 (arithmetic)  \textrightarrow  算数する X\slash △ }

\begin{center}
\textbf{Test 2: Aspect Ending Test }
\end{center}

\par{ Nouns that describe a コト (event) and can then be followed by time endings such as ~後 (after) ~中 (under\slash during) should have verb forms. This test is immediately problematic because some words that can be used with one such ending can't be used with another. All of these words, though, can take ~中. So, according to the test, they should be able to take する, and they do. }

\begin{ltabulary}{|P|P|P|P|}
\hline 

採用 & Adoption (of something) & 活動 & Activity \\ \cline{1-4}

研究 & Research & 募集 & Recruitment \\ \cline{1-4}

建設 & Construction & 仮眠 & Doze \\ \cline{1-4}

\end{ltabulary}

\par{ Like the last test, this test shows that nouns that express モノ cannot have verb forms. However, it accidentally invalidates nouns with verb forms that can\textquotesingle t take both ~後 and ~中 for other reasons. Such nouns include 感動 (feeling\slash movement), 緊張 (nervousness), 静止 (stillness), 信用 (trust\slash faith), 信頼. They express particular states that can\textquotesingle t be viewed as having exact beginnings, midpoints, or ends. }

\begin{center}
\textbf{Test 3: State Aspect Test }
\end{center}

\par{ Nouns such as 痛み represent states. However, the state unlike that of a building has the potential of changing over time. In this sense, such nouns are more verb-like. When we use 一時的な (temporary), we can quickly eliminate many nouns that can\textquotesingle t take する. }

\par{5. 一時的な\{緊張 〇・保存 (Preservation) 〇・在庫 (Stock (supply)) 〇・健康 (health) X・精神 (spirit\slash mind) X\} }

\par{ Words like 健康 and 精神 also fail previous aspect tenses. They have no salient relation with time. Although nouns like 緊張, 混乱 (chaos), and 流行 (fashion\slash trend) may not necessarily have defined time frames, these stages are subject to change, and change takes time. }

\begin{center}
\textbf{Test 4: Intent\slash State of Action with Adjective Modifiers Test }
\end{center}

\par{ There are still some する verbs not accounted for. For nouns with which intention and state can be expressed with an adjectival modifier, it is more verb-like and should take する. Those that can't show intention in this way are less verb-like and shouldn\textquotesingle t take する. }

\par{6a. 彼女は一生懸命に調査を行ないました。〇 \hfill\break
6b. 彼女は一生懸命な調査を行ないました。〇 \hfill\break
 \hfill\break
 Though slightly different, we know that she carried out a full-hearted investigation. We know her intent and state of action. So, 調査する should be a possible verb form of 調査, which it is. }

\par{ Let\textquotesingle s look at similar noun pairs in which one can take する and one can't. So, one can take certain intent\slash state adjectival phrases but the other cannot. If one can\textquotesingle t, it shouldn't take する. }

\par{7. 意識的な参考 X  意識的な参照 〇  異常な熱中 〇  異常な夢中 X \hfill\break
Conscious consultation  Conscious reference  Abnormal zeal  Abnormal daze }

\par{ Nouns with Xs don't have する verb forms, thus validating the test. Though in English verb forms of these word aren't bad, if intent\slash volition isn't in the noun in Japanese, the noun fails the test. This is why 需要 doesn't have a する verb form. It fails this test. }

\par{ This test is great for nouns that would have transitive verb form. It doesn't work for intransitive verbs, which have no volition. In which case, you would use the previous tests. }

\begin{center}
\textbf{Final Comments }
\end{center}

\par{ With these four tests, you should be able to correctly assume whether a noun can take する or not. Beware of influence from your native language(s) that may make you accidentally say things like 独立になる, which should be 独立する (to become independent). Independence like marriage presumably lasts for some time. 独立 along with 結婚, 安心, etc. pass Test 3 and thus have する verbs. If a noun fails this test, ~になる, ~にする, or both might be used instead. }

\par{ As final examples, consider 瀕死 (to become moribund) and 中毒 (intoxication\slash poisoning). 瀕死する fails all tests. When verbal, you say 瀕死になる, as expected. Both 瀕死 and 中毒 don't make sense with “temporary”. They describe things that simply happen\slash come to be. It would be weird to say that something was moribund for several days or someone was intoxicated for two hours. }

\par{\textbf{Word Note }: 中毒する does exist in the transitive sense of “to poison”, which then passes Test 4. }

\par{\textbf{Particle Note }: The particle used before the する verb is determined by the semantics of the verb phrase itself. This is not the topic of this lesson. The topic of this lesson is what sort of nouns become する verbs and which ones need を. }

\par{ For instance, を敬礼する is ungrammatical because the する verb 敬礼する means "to salute", and because you "salute to something", you use the particle に with it. Though not using "to" is alright in English, this is not English, so you are going to have to consider how 敬礼する is used and not assume it's like its English equivalents. }
      
\section{Examples}
 
\par{ After a long time studying grammar, we will end this section with example sentences with even more する verbs. Though not part of the exercises, try to figure out which tests these verbs passed to be acceptable. }

\par{8. ${\overset{\textnormal{こんばん}}{\text{今晩}}}$ 、 ${\overset{\textnormal{よやく}}{\text{予約}}}$ したスミスです。 \hfill\break
I am Smith with a reservation for tonight. }

\par{9. ${\overset{\textnormal{にほん}}{\text{日本}}}$ に ${\overset{\textnormal{りゅうがく}}{\text{留学}}}$ します。 \hfill\break
I will study abroad in Japan. }

\par{10. ${\overset{\textnormal{かいぎ}}{\text{会議}}}$ を ${\overset{\textnormal{きゅうし}}{\text{休止}}}$ する。 \hfill\break
To adjourn\slash pause a meeting. }

\par{11. ${\overset{\textnormal{あ}}{\text{会}}}$ う ${\overset{\textnormal{やくそく}}{\text{約束}}}$ をキャンセルする。 \hfill\break
To cancel an appointment. }
      
\section{Different Kinds of する Verbs}
 
\par{ Long time ago, する was す. After certain words, it was voiced as ず, which changed to the interchangeable ~じる・ずる that differ only in speech style and conjugation. The tests taught in this lesson still work for these verbs. Only their conjugations are tricky. }

\par{ ~ずる came before ~じる, but the latter is more common. ~ずる is mainly found in literature, but because they do occasionally show up, you at least need to know that you are looking at a form of する and that it has a more modern equivalent, ~じる. }

\begin{ltabulary}{|P|P|P|P|P|P|P|}
\hline 

 & みぜんけい & れんようけい & しゅうしけい & れんたいけい & いぜんけい & めいれいけい \\ \cline{1-7}

~じる & じ- & じ- & じる & じる- & じれ- & じろ・じよ \\ \cline{1-7}

~ずる & ぜ・じ- & じ- & ず(る) & ずる- & ずれ- & ぜよ \\ \cline{1-7}

\end{ltabulary}

\par{\textbf{Base Note }: Like する, ぜ is used with older endings. So, you don't need to know it now. }

\par{ Another sound change with する occurs when it attaches to single character Sino-Japanese nouns that end in つ. This つ, then, subsequently contracts to  っ. The chart below illustrates the bases of ${\overset{\textnormal{はっ}}{\text{発}}}$ する (to emit). }

\begin{ltabulary}{|P|P|P|P|P|P|P|}
\hline 

動詞 & みぜんけい & れんようけい & しゅうしけい & れんたいけい & いぜんけい & めいれいけい \\ \cline{1-7}

発する & 発し・発せ・発さ- & 発し- & 発す(る) & 発する- & 発すれ- & しろ・せよ \\ \cline{1-7}

\end{ltabulary}

\par{\textbf{Base Note: }し- takes ~ない and the volitional as expected, but さ- can also take ~ない aside from being used with the passive and causative. せ-, is of course used with older endings that you should not worry about at this point. It\textquotesingle s just that if you see する verbs in these forms and see that the vowel in the base has changed to an “e”, you at least know what base you\textquotesingle re looking at. This alone can help you know what the ending is being used for. }

\begin{center}
\textbf{More Examples } 
\end{center}

\begin{ltabulary}{|P|P|P|P|P|P|}
\hline 

To emit & 発する & To presume & 察する & To believe & 信\{じる・ずる\} \\ \cline{1-6}

To feel; sense & 感\{じる・ずる\} & To anticipate & 先ん\{じる・ずる\} & To value & 重ん\{じる・ずる\} \\ \cline{1-6}

To memorize & 諳ん\{じる・ずる\} &  &  &  &  \\ \cline{1-6}

\end{ltabulary}
\hfill\break
\textbf{Word Note }: Notice that じる・ずる do not only follow Sino-Japanese words. When following native words, the native root is followed by ん, which then じる・ずる follows.     