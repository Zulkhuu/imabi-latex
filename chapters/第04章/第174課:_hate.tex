    
\chapter{Hate}

\begin{center}
\begin{Large}
第174課: Hate 
\end{Large}
\end{center}
 
\par{ We have gone through all the phrases for liking and loving, but now it's time to go over hatred. Many of the phrases will look very similar, and unlike the previous phrases for liking and loving, there isn't much nuance splitting to worry about. However, like always, you should pay attention to detail. }
      
\section{Hatred}
 
\par{\textbf{(大)嫌いだ }}

\par{The most basic word for "to hate" in Japanese is 嫌いだ. 嫌い implies that you hold a bad impression of something, and the disdain that you hold is a reaction to this. You can't use 嫌いだ to reject a request. }
 
\par{1. エッセイを書くのが嫌いだよ。 \hfill\break
I hate writing essays. }
 
\par{2. 毎日走らなきゃいけないけど、ホントに嫌いだよ。とっても ${\overset{\textnormal{つら}}{\text{辛}}}$ いんだ。いつも ${\overset{\textnormal{あせ}}{\text{汗}}}$ かいてて、体中が ${\overset{\textnormal{ぬ}}{\text{濡}}}$ れちゃうし、 ${\overset{\textnormal{くさ}}{\text{臭}}}$ くなっちゃうよ。 \hfill\break
I have to run every day, but I really hate it. It's really tough, and I'm always sweating. My whole body gets drenched, and I end up smelling. }
 
\par{\textbf{Part of Speech Note }:  The expression comes from the verb 嫌う, which means "to detest", but unlike it, 嫌いだ and 大嫌いだ are used as 形容動詞. }
 
\par{大嫌い is the extreme of hating, as is implied by the prefix 大-. This is an emphatic version of 嫌いだ. Both come from the verb 嫌う. }
 
\par{3. 大嫌いなスポーツはありますか。 \hfill\break
Are there any sports that you hate? }
 
\par{4. 嫌いな人のことで ${\overset{\textnormal{なや}}{\text{悩}}}$ む。 \hfill\break
To fret over people who you hate. }
 
\par{5. スカンクが嫌いなの。(Feminine) \hfill\break
I hate skunks. }
 
\par{6. 好き嫌い \hfill\break
Likes and dislikes }
 
\par{7. 平気で ${\overset{\textnormal{うそ}}{\text{嘘}}}$ つく人が大嫌いなの。(Feminine) \hfill\break
I hate people that can just flat out lie in your face. }
 
\par{8. 彼女は野球が大嫌いです。 \hfill\break
She really hates baseball. }
 
\par{\textbf{Word Note }: ベースボール is the same thing as 野球, but it not used nearly as often. }
 
\par{ 嫌いがある is often used to show bad tendency. See more examples in Lesson 201. }
 
\par{9. 人前で鼻くそをほじる嫌いがある女性と付き合ったことある? \hfill\break
Have you ever dated a girl who picks her nose in public? }
 
\par{10. 蛙を棒で叩く嫌いがある、そのいたずらっ子はまたほかの小学生の目の前で蛙を殺したばっかりなんだよ。どうしたらいいのかさっぱり分からんよ。 \hfill\break
That bad kid that likes to hit frogs with sticks just killed a frog in front of the other elementary kids. I have no idea what we're going to do with him. }
 
\par{ 嫌う can be translated as "to hate". This hate is a very strong dislike, and although 嫌いだ comes from it, 嫌う is still more powerful. This hate implies that you don't want to deal with whatever or whoever you hate. }
 
\par{11. 人は何故 ${\overset{\textnormal{むじゅん}}{\text{矛盾}}}$ を嫌うのでしょう。 \hfill\break
Why do people despise contradiction? }
 
\par{12. いじめっ子を嫌うとしても、何の役にも立たないでしょう。 \hfill\break
Even if you were to hate the bully, that does nothing, no? }
 
\par{13. 僕を嫌わないで。 \hfill\break
Don't hate me. }
 
\par{ On the flip side, though, it can even refer to animals and plants disliking something. So, vampires hating the sunlight would be a great example. When used with non-living things, it shows that two things are not compatible such as fire and water. }
 
\par{14. 本は ${\overset{\textnormal{しっけ}}{\text{湿気}}}$ を嫌う。 \hfill\break
Books hate moisture. }

\par{15. ${\overset{\textnormal{きゅうけつき}}{\text{吸血鬼}}}$ は日光を嫌う。 \hfill\break
Vampires hate sunlight. }
 
\par{ It is also found in the phrase 嫌わず, which uses the old negative auxiliary verb ~ず, which functions as ~ないで here. In this phrase, you just do something without even giving a dime about the place or person you're dealing with. }
 
\par{17. 中国では所嫌わず ${\overset{\textnormal{つば}}{\text{唾}}}$ を ${\overset{\textnormal{は}}{\text{吐}}}$ いてしまうと、逮捕されるそうだ。 \hfill\break
They say that if you spit carelessly anywhere in China that you get arrested. }
 
\par{\textbf{Etymology Note }: 嫌う potentially comes from the verb 切る + the Old Japanese auxiliary verb ふ, which was very similar to ~ている. It survives in many words, but this one is not completely verified. However, it makes perfect sense. Proof for this can be found in the fact that its original meaning was 除き去る. Exclusion is closely related to \textbf{cutting }off. It also used to mean to separate\slash distinguish and avoid. This is still seen rarely in Modern Japanese in ~の嫌いなく. }
 
\begin{center}
 \textbf{いやだ }
\end{center}
 
\par{ いやだ, written in 漢字 as  嫌だ or 厭だ, is very similar to 嫌いだ, but there are considerable differences. いやだ refers to situations in which you reject a person's request or invitation or a certain circumstance. }
 
\par{18. こんな時に行くなんていやだよ。 \hfill\break
No way am I going at a time like this. }
 
\par{19. 「先輩、先に帰ってもいいですか?」「いやだよ」 \hfill\break
"Senpai, is it alright if I go home first?" "No way". }
 
\par{20. 息子がいやなことばかりするから、 どうしつけたらいいか分かりません。 \hfill\break
Because my son keeps on doing bad things, I don't know how to discipline him. }
 
\par{21. 嫌だから、やめてちょうだい。 \hfill\break
No, that's bad. Please quit it. }
 
\par{22. 猫に ${\overset{\textnormal{さわ}}{\text{触}}}$ られるのもいやになってきた。 \hfill\break
I've gotten to the point that it's awful to be touched by cats. }
 
\par{23. いやな気分になっても、平気でやりつづけなさい。 \hfill\break
Continue to work calmly even if you feel uncomfortable. }
 
\par{ When you use いやだ in not accepting a certain person or thing, it is far stronger than 嫌いだ. For example, consider the following. }
 
\par{24. 健太くんなんか嫌いだよ。 \hfill\break
I really don't like Kenta. }
 
\par{25. 健太くんなんかいやだよ。 \hfill\break
Kenta is just no good. }
 
\par{The first sentence does show hatred towards Kenta, but the latter sentence is to the point that you don't even want to recognize his existence. It's almost like a euphemism for wishing he'd no longer live. }
 
\begin{center}
\textbf{嫌がる・嫌がらせ }
\end{center}
 
\par{You can't just use 嫌だ for referring to what someone else hates\slash dislikes. Instead, you have to couple this with the auxiliary 〜がる. The nominal form of the causative form of 嫌がる, 嫌がらせ, actually means "nuisance\slash annoyance". 嫌がらせをする happens to mean "to annoy (someone)". }
 
\par{26. 他人がいやがる仕事だけ引き受ける人、かわいそうだな。おれにもそんなことできないけどね。 \hfill\break
People that only handle jobs that everyone else dislikes are pitiful, aren't they? I sure couldn't do any of it. }
 
\par{27. まず嫌がらせをした人の気持ちをよく考えればいいと思います。 \hfill\break
I think you need to first think really hard about the feelings of the person you annoyed. }

\begin{center}
\textbf{だめだ }
\end{center}
 
\par{ This phrase is meant to show that something is impossible or incompatible in light of the circumstances. It's often objective, although objectivity is not necessarily something natives think about when they use this word. }
 
\par{28. そもそも意味が重複したら、だめでしょうか。 \hfill\break
Is it really bad in the first place for the meaning to be doubled? }
 
\par{29. 「先輩、お先に帰ってもいいですか?」「ううん、だめだよ。まだ仕事あるから。」 \hfill\break
"Senpai, is it alright if I go home first?" "No, that won't work because we still have things to do". }
 
\par{30. 健太なんかだめだよ。 \hfill\break
Kenta is bad. }
 
\par{\textbf{Sentence Note }: This sentence is a qualification of some attribute to Kenta, not necessarily that you hate him. }
 
\begin{center}
\textbf{憎む } 
\end{center}
 
\par{Rarely spelled as 悪む and the primary verbal form of 憎い, this verb shows that you think something or someone is detestable. This verb can also be used to show abhorrence to an abstract thing such as an ideology or war. }
 
\par{31. 大統領を憎む。 \hfill\break
To detest the president. }
 
\par{32. 共産主義を憎む。 \hfill\break
To detest communism. }
 
\par{33. 戦争を憎む。 \hfill\break
To detest war. }

\par{34. ${\overset{\textnormal{こいがたき}}{\text{恋敵}}}$ を憎むことが当然でしょう。 \hfill\break
It's only natural to detest a love rival. }
 
\par{35. 罪を憎んで、人を憎まず。(Set Phrase) \hfill\break
I hate sin, not people. }
 
\begin{center}
\textbf{憎らしい }
\end{center}
 
\par{ 憎らしい points out the (condition of the) person that makes you mad. Ironically, it is not always the case that this word has negative connotations. }
 
\par{36. 若者の砕けたスラングだらけの話し方が憎らしい。 \hfill\break
The broken down, slang filled speech of young people is detestable. }
 
\par{37. 愛犬を死なせた、あの人が憎らしい。 \hfill\break
That person who let my beloved dog die is detestable. }
 
\par{38. 憎らしい口を利く。 \hfill\break
To say hateful things. }
 
\par{39. 彼女はあまりにもかわいくて憎らしくなってきた。(More positive than negative) \hfill\break
She's just so cute that it's gotten to me. }
 
\begin{center}
\textbf{憎い }
\end{center}
 
\par{Unlike the opposite of love, 憎い expresses emotion of feeling displeasure, irritation, envy, etc. towards\slash about someone. It is you yourself who feels this discomfort. It also happens to have the old meaning of "ugly", which is now typically handled by 醜い except in rare circumstances. }
 
\par{40. 不正を許す大臣が憎い。 \hfill\break
I hate prime ministers who allow injustice. }

\par{41. ${\overset{\textnormal{ぼうず}}{\text{坊主}}}$ 憎けりゃ ${\overset{\textnormal{けさ}}{\text{袈裟}}}$ まで憎い。(Set Phrase) \hfill\break
He who hates Peter harms his dogs. \hfill\break
Literally: If you hate a bonze, you also hate his kesa. }
 
\par{42. 妻がとっても憎いもんだ。 \hfill\break
I really hate my wife. }
 
\par{43. あんた、憎いこと言うね。(Ironic) \hfill\break
You really do say some provoking things. }
 
\par{44. 憎き〇〇の豚どもを日本から追い出そう。(Racist) \hfill\break
Let's drive all those ugly XX pigs out of Japan! }
 
\begin{center}
 \textbf{憎々しい・憎(っ)たらしい }
\end{center}
 
\par{These words are stronger versions of 憎らしい. }
 
\par{45. 彼は憎々しげにあのうるさい猫を ${\overset{\textnormal{け}}{\text{蹴}}}$ ったが、すぐその後、あの猫が ${\overset{\textnormal{けが}}{\text{怪我}}}$ で死んで、ひどく ${\overset{\textnormal{こうかい}}{\text{後悔}}}$ した。 \hfill\break
He viciously kicked that annoying cat, but shortly afterwards, that cat died from its injuries, and he was filled with remorse. }
 
\par{46. あの人はね、とっても憎たらしいよ。 \hfill\break
That person just really infuriates me. }
 
\begin{center}
\textbf{憎しみ }
\end{center}
 
\par{ Rather than using 憎み, which is a word but not used as a noun, as the nominal form of 憎む, 憎しみ is typically used. The verb form 憎しむ did exist at one point, but it is no longer used. }
 
\par{47. 憎しみを覚える。 \hfill\break
To feel hatred\slash enmity. }
 
\par{48. 憎しみの炎を燃やす。 \hfill\break
To fuel the flames of hatred. }
 
\begin{center}
 \textbf{気に\{食・喰\}わない }
\end{center}
 
\par{ This 気 idiom just shows that you just can't stomach something. It is rather colloquial, so there are plenty of instances you can use it in speaking. }
 
\par{49. 見くびられるの、気に食わない。 \hfill\break
I can't stand being looked down at (by others). }
 
\par{50. あいつ、ちっとも気にくわねーよ。 \hfill\break
I just can't stand that guy. }
 
\begin{center}
\textbf{嫌悪 }
\end{center}
 
\par{ This word is "hatred" by definition, and it is rather cruel hatred. People tend to not use this word correctly, and one of the most egregious misuses happens to be the following. }
 
\par{51. 嫌悪感を感じる X\slash △ \hfill\break
To feel hatred. }
 
\par{The problem is that the phrase is a 重複表現. Double phrases are almost always frowned upon in Japanese, and this is especially bad. however, all you have to do to make this phrase correct is replace 感じる with another verb like 覚える and 持つ. }
 
\par{52. 分かりづらいことばかり言う人に嫌悪感が覚える。 \hfill\break
To feel disgust towards people who only say things that are difficult to understand. }
 
\par{53. 自己嫌悪に ${\overset{\textnormal{おちい}}{\text{陥}}}$ る。 \hfill\break
To despise oneself. }
 
\par{54. 自分の母親の殺人に対して嫌悪を持つのは ${\overset{\textnormal{にんじょう}}{\text{人情}}}$ というものだ。 \hfill\break
It's only human nature to hold hatred against the murderer of your own mother. }
 
\par{55. 世界中の人々は ${\overset{\textnormal{ざんこく}}{\text{残酷}}}$ さを嫌悪すべきだ。 \hfill\break
Everyone in the world should abhor cruelty. }
 
\begin{center}
\textbf{毛嫌いする } 
\end{center}
 
\par{  ${\overset{\textnormal{けぎら}}{\text{毛嫌}}}$ いする is commonly used in the spoken language. This word, though, has the particular nuance of hating something for no particular reason. }

\par{56. ${\overset{\textnormal{なま}}{\text{怠}}}$ け者を毛嫌いしたほうがいい。 \hfill\break
It's best to hate lazy people. }
 
\par{57. インテリーを毛嫌いするのですか。 \hfill\break
Do you detest intellectuals? }
 
\par{58. わけもなく村上春樹さんの小説を毛嫌いする人はバカだね。 \hfill\break
Those that hate Murakami Haruki's novels for no reason are stupid, aren't they? }
 
\par{\textbf{Person Note }: 村上春樹 is one of the current most renowned authors of Japanese literature. }
 
\begin{center}
\textbf{不愉快 }
\end{center}
 
\par{ Although not necessarily hatred, 不愉快, the antonym of 愉快, shows that something is not pleasant at all and can put you in a bad mood. The reason why it is mentioned is because in contexts like the last example, it really is akin to "hate". Most of the time, it is typically equivalent to "disgusted" and "unpleasant". This word is also used frequently in the spoken language. }
 
\par{59. ちょっと不愉快な思いをした。 \hfill\break
I was a bit disgusted. }
 
\par{60. 不愉快な現実を認めなければなりません。 \hfill\break
You must recognize the unpleasant reality. }
 
\par{61. 不愉快な人と話すのはいやだね。 \hfill\break
Speaking to unpleasant people is awful, isn't it? }
 
\par{62. 本当に不愉快なやつだね、黒田君は。 \hfill\break
He's really a pain, that Kuroda. }
 
\begin{center}
\textbf{憎悪 } 
\end{center}
 
\par{ This word is rather literary, but it shows a very violent hatred, which is why it might be left more so to writing because of its potency. }
 
\par{63. 憎悪に満ちた目で睨む。 \hfill\break
To stare down with eyes full of revulsion. }
 
\par{64. 人種差別を憎悪する。 \hfill\break
To hate racial discrimination. }
 
\par{65. 憎悪の炎を燃やす。 \hfill\break
To fuel the flames of abhorrence. }
 
\par{66. 公開の場で憎悪の感情を抑えなくてはいけない。 \hfill\break
You must control your feelings of hatred in public. }
 
\par{67. 憎悪に ${\overset{\textnormal{か}}{\text{駆}}}$ られてはならない。 \hfill\break
You can't get caught up in anger. }
 
\par{68. 憎悪で狂ったように人を ${\overset{\textnormal{なぐ}}{\text{殴}}}$ ったり蹴ったりしてしまう。 \hfill\break
To end up beating up and kicking people in a fit of mad rage. }
 
\begin{center}
\textbf{忌む }
\end{center}
 
\par{This is by all means a literary word, but one usage other than "abhor" that this word has is "to be taboo", which is quite unlike the rest although semantically related. }
 
\par{69. 忌むべき者の ${\overset{\textnormal{かがりび}}{\text{篝火}}}$ \hfill\break
Bonfire of the damned }
 
\par{70. 国民は ${\overset{\textnormal{かくしん}}{\text{革新}}}$ を忌む方がよい。 \hfill\break
It is best for the citizens to abhor the notion of revolution. }
 
\par{71. 日本人が死に通じるとして四を忌むことはアメリカ人でも知っている。 \hfill\break
Even Americans know about the Japanese hating the number four because it correlates to death. }
 
\par{\textbf{Word Note }:  忌み嫌う is also possible and means "to detest\slash abhor", too. Just view it as a combination of 忌む and 嫌う. }
 
\begin{center}
 \textbf{厭う }
\end{center}
 
\par{ ${\overset{\textnormal{いと}}{\text{厭}}}$ う means "to begrudge", but its negative form means "willing". Since 〜ない is used, this positive definition might be a surprise. This is the negative of “to begrudge”. So, it is literally "to not begrudge in". This is normally spelled in かな. It is also important to note that this word is very literary. In fact, any word with 厭 is going to be uncommon and literary. }
 
\par{72. 彼女は手を ${\overset{\textnormal{さ}}{\text{差}}}$ し出すこともいとわない。 \hfill\break
She is willing to lend a hand. }
 
\par{73. 彼は ${\overset{\textnormal{ようせい}}{\text{要請}}}$ に ${\overset{\textnormal{おう}}{\text{応}}}$ じることをいといません。 \hfill\break
He is willing to answer to our requests. }
 
\par{74. 世をいとうな。 \hfill\break
Hate not the world. }
 
\par{75a. お体をご自愛ください。 \hfill\break
75b. お体おいといください。 \hfill\break
Please be careful to not cause yourself any harm. }
 
\begin{center}
\textbf{厭わしい }
\end{center}
 
\par{ Like its verb form 厭う, 厭わしい is very literary. Its meaning is similar to it as it means "detestable\slash deplorable". The following sentence would be a good way to intelligently insult someone with class in one\textquotesingle s word choice. }
 
\par{76. あの顔を見ることさえ厭わしい。 \hfill\break
Just looking at that face is deplorable. }
 
\begin{center}
\textbf{厭悪 }
\end{center}
 
\par{ This is a very literary word meaning "detestation". The first sentence shows just how complicated a context with this word may be. It is important to note, though, that in reality when a word of hatred is used in a piece of literature, the surrounding context is likely to have more hate related words. }
 
\par{77. 双方ともに、自らの側に絶対不可分にして圧倒的な正義のあることをつねづね主張し、当然の嫌悪を抱いて相手の ${\overset{\textnormal{そこし}}{\text{底知}}}$ れぬ ${\overset{\textnormal{じゃあく}}{\text{邪悪}}}$ に、その野望、冷酷さ、 ${\overset{\textnormal{はいしん}}{\text{背信}}}$ 、 ${\overset{\textnormal{いんぼう}}{\text{陰謀}}}$ の数々に ${\overset{\textnormal{えんお}}{\text{厭悪}}}$ の ${\overset{\textnormal{まなざ}}{\text{眼差}}}$ しを向けていた。 \hfill\break
I always insisted on what was completely mutually inseparable from my own side and overwhelmingly righteous, and I naturally held hatred and gazed upon my enemy's bottomless evil, that ambition, the cruelty, the treachery, and the endless numbers of conspiracy in abhorrence. \hfill\break
From 智慧の林檎. }
 
\par{78. 厭悪の眼差し \hfill\break
Gaze of detestation }
    