    
\chapter{Adjectives}

\begin{center}
\begin{Large}
第198課: Adjectives: Sound Changes 
\end{Large}
\end{center}
 
\par{ Although the contraction rules described in this lesson are discussed under the guise of slang speech, they are in fact more so representative of a widespread phonological phenomenon in many dialects in Japan. This is, Japanese tends to not like different vowels right next to each other. }

\par{ In many dialects (mostly northern and eastern) including casual Tokyo-ben 東京弁, \slash ai\slash , \slash oi\slash , and \slash ii\slash  at the end of adjectives contract to [ē] in slang. Less commonly, those that end in \slash ui\slash  may be found contracted to [ē] or [ii]. This phenomenon has existed in some capacity in Japanese for quite some time. In fact, outside of adjectives, you\textquotesingle ll find plenty of words contracted with the same premise. For instance, the course words for “you”— \emph{omae }お前 and \emph{temae }手前—are very frequently pronounced as \emph{omē }おめー and \emph{temē }てめー instead. Again, this is by no mistake. }
      
\section{How Sound Changes Affect Adjectives}
 Returning to how this applies to adjectives, in Standard Japanese there is no instance of the syllables \slash ye\slash  and \slash we\slash , at least in native vocabulary, which means if the sound change \slash ai\slash , \slash oi\slash , \slash ii\slash  \textrightarrow  [ē] causes either of those two syllables to form, \slash y\slash  and \slash w\slash  just drop altogether. An example of this is the adjectives tsuyoi 強い (strong). Its contracted form is tsuē, not tsuyē. In the examples below, you will find examples of these contracted adjectives in their glory. Unsurprisingly, this will be a wonderful opportunity to familiarize with some of the most iconic dialectical phrases in Japanese. 1. あの服、ほんまにええねん。(=あの服、本当にいいんだよ) Ano fuku, homma ni ē nen. Those clothes are really good. 2. ええやないか! (=いいじゃないか) Ē ya nai ka! Isn\textquotesingle t it great? \hfill\break
3. ええんちゃうか? (=いいんじゃないだろうか?) Ē n chau ka? Is it not great? Grammar Note: ちゃう is a contraction of 違う and is used in many Western dialects to ask if something is in fact not the case while still seeking a positive affirmation. 4. あのサメ、怖えぇぇ! (=あのサメ、怖い!) Ano same, koē…! That shark is scary! 5. すげーだろ! (=すごいだろう!) Sugē daro! Isn\textquotesingle t that cool! 6. めっちゃ眠てー! (=とっても眠たい!) Metcha nemutē! I\textquotesingle m really sleepy! 7. 冷蔵庫も何にもねえ! (=冷蔵庫も何もない!) Reizōko mo nan\textquotesingle ni mo nē! There\textquotesingle s nothing in the fridge! 8. うっせーな、てめー。 (=うるさい、お前) Ussē na, temē. Shut up, you. \hfill\break
Word Note: As this example shows, some words can be further contracted—urusai \textrightarrow  ussē. This sound change also affects grammatical endings such as –nai ない and –tai たい 9. 弱ぇ奴は着る服も選べねェ! (=弱い奴は着る服も選べない!) Yoē yatsu wa kiru fuku mo erabenē! Weak guys can\textquotesingle t even choose their own clothes to wear! 10. たくさん食べてーなぁ。 (=たくさん食べたいなぁ) Takusan tabetē nā. Man, I wanna eat a lot. 11. くだらねーことでいちいち突っかかってくんじゃねーよ。(=下らないことでいちいち突っかかっていくんじゃないよ) Kudaranē koto de ichi\textquotesingle ichi tsukkakatteku n ja nē yo. You can\textquotesingle t be charging at me with every little stupid thing. 12. つまんねーこと聞くなよ。(=つまらないことを聞くなよ) Tsuman\textquotesingle nē koto kiku na yo. Don\textquotesingle t ask absurd things. One must understand that the productivity of this sound change is not applied to all adjectives. There are plenty of idiosyncrasies and specialized variations depending on the adjective. 小さい  Chiisai  ちっちゃい・ちっちぇー Chitchai\slash chitchē  寒い Samui  さみい・さめー Samii\slash samē 悪い Warui  わりい Warii  暑い・熱い Atsui  あちい・あちぇー Achii\slash achē 安い Yasui  やせえ Yasē  まずい Mazui  ま(っ)ぜー Ma(z)zē \hfill\break
\hfill\break
\hfill\break
Another dialectical phenomenon that has an even wider distribution in Japan is dropping the final –i of adjectives and replacing it with a glottal stop, which is optionally spelled with っ. This is frequently employed when talking to oneself or when the adjective in question is not necessarily directed toward anyone. In other words, it has an expletive nature to it. 13. やばっ、逃げろ! Yaba…nigero! Crap…run! 14. 痛っ! Ita…! Ouch…! 15. このお湯、熱っ! Kono oyu, atsu…! This (bath) water\textquotesingle s hot! 16. くそ寒っ! Kuso samu…! It\textquotesingle s freaking cold…! 17. うわっ、臭っ! Uwa…, kusa…! Dang…it smells…! Another phenomenon that's not so much dialectical as it is emphatic, rather than stopping with a glottal stop, emphasis can be added to an adjective in casual speech by dropping I and elongating the preceding vowel. In fact, the final \slash i\slash  doesn\textquotesingle t have to be dropped for this to work. 18. 外、寒ー。 Soto, samū. Outside\textquotesingle s co-o-old! 19. すごーい! Sugōi! Coool! 20. 胃が痛ーい! I ga itāi! My stomach hurrts! Returning to how this applies to adjectives, in Standard Japanese there is no instance of the syllables \slash ye\slash  and \slash we\slash , at least in native vocabulary, which means if the sound change \slash ai\slash , \slash oi\slash , \slash ii\slash  \textrightarrow  [ē] causes either of those two syllables to form, \slash y\slash  and \slash w\slash  just drop altogether. An example of this is the adjectives tsuyoi 強い (strong). Its contracted form is tsuē, not tsuyē. In the examples below, you will find examples of these contracted adjectives in their glory. Unsurprisingly, this will be a wonderful opportunity to familiarize with some of the most iconic dialectical phrases in Japanese. 1. あの服、ほんまにええねん。(=あの服、本当にいいんだよ) Ano fuku, homma ni ē nen. Those clothes are really good. 2. ええやないか! (=いいじゃないか) Ē ya nai ka! Isn\textquotesingle t it great? \hfill\break
3. ええんちゃうか? (=いいんじゃないだろうか?) Ē n chau ka? Is it not great? Grammar Note: ちゃう is a contraction of 違う and is used in many Western dialects to ask if something is in fact not the case while still seeking a positive affirmation. 4. あのサメ、怖えぇぇ! (=あのサメ、怖い!) Ano same, koē…! That shark is scary! 5. すげーだろ! (=すごいだろう!) Sugē daro! Isn\textquotesingle t that cool! 6. めっちゃ眠てー! (=とっても眠たい!) Metcha nemutē! I\textquotesingle m really sleepy! 7. 冷蔵庫も何にもねえ! (=冷蔵庫も何もない!) Reizōko mo nan\textquotesingle ni mo nē! There\textquotesingle s nothing in the fridge! 8. うっせーな、てめー。 (=うるさい、お前) Ussē na, temē. Shut up, you. \hfill\break
Word Note: As this example shows, some words can be further contracted—urusai \textrightarrow  ussē. This sound change also affects grammatical endings such as –nai ない and –tai たい 9. 弱ぇ奴は着る服も選べねェ! (=弱い奴は着る服も選べない!) Yoē yatsu wa kiru fuku mo erabenē! Weak guys can\textquotesingle t even choose their own clothes to wear! 10. たくさん食べてーなぁ。 (=たくさん食べたいなぁ) Takusan tabetē nā. Man, I wanna eat a lot. 11. くだらねーことでいちいち突っかかってくんじゃねーよ。(=下らないことでいちいち突っかかっていくんじゃないよ) Kudaranē koto de ichi\textquotesingle ichi tsukkakatteku n ja nē yo. You can\textquotesingle t be charging at me with every little stupid thing. 12. つまんねーこと聞くなよ。(=つまらないことを聞くなよ) Tsuman\textquotesingle nē koto kiku na yo. Don\textquotesingle t ask absurd things. One must understand that the productivity of this sound change is not applied to all adjectives. There are plenty of idiosyncrasies and specialized variations depending on the adjective. 小さい  Chiisai  ちっちゃい・ちっちぇー Chitchai\slash chitchē  寒い Samui  さみい・さめー Samii\slash samē 悪い Warui  わりい Warii  暑い・熱い Atsui  あちい・あちぇー Achii\slash achē 安い Yasui  やせえ Yasē  まずい Mazui  ま(っ)ぜー Ma(z)zē \hfill\break
\hfill\break
\hfill\break
Another dialectical phenomenon that has an even wider distribution in Japan is dropping the final –i of adjectives and replacing it with a glottal stop, which is optionally spelled with っ. This is frequently employed when talking to oneself or when the adjective in question is not necessarily directed toward anyone. In other words, it has an expletive nature to it. 13. やばっ、逃げろ! Yaba…nigero! Crap…run! 14. 痛っ! Ita…! Ouch…! 15. このお湯、熱っ! Kono oyu, atsu…! This (bath) water\textquotesingle s hot! 16. くそ寒っ! Kuso samu…! It\textquotesingle s freaking cold…! 17. うわっ、臭っ! Uwa…, kusa…! Dang…it smells…! Another phenomenon that's not so much dialectical as it is emphatic, rather than stopping with a glottal stop, emphasis can be added to an adjective in casual speech by dropping I and elongating the preceding vowel. In fact, the final \slash i\slash  doesn\textquotesingle t have to be dropped for this to work. 18. 外、寒ー。 Soto, samū. Outside\textquotesingle s co-o-old! 19. すごーい! Sugōi! Coool! 20. 胃が痛ーい! I ga itāi! My stomach hurrts!   Returning to how this applies to adjectives, in Standard Japanese there is no instance of the syllables \slash ye\slash  and \slash we\slash , at least in native vocabulary, which means if the sound change \slash ai\slash , \slash oi\slash , \slash ii\slash  \textrightarrow  [ē] causes either of those two syllables to form, \slash y\slash  and \slash w\slash  just drop altogether. An example of this is the adjectives \emph{tsuyoi }強い (strong). Its contracted form is \emph{tsuē }, not \emph{tsuyē }. In the examples below, you will find examples of these contracted adjectives in their glory. Unsurprisingly, this will be a wonderful opportunity to familiarize with some of the most iconic dialectical phrases in Japanese.  
\par{1. あの服、ほんまにええねん。(=あの服、本当にいいんだよ) \hfill\break
 \emph{Ano fuku, homma ni ē nen. }\hfill\break
Those clothes are really good. }
 
\par{2. ええやないか! (=いいじゃないか) \hfill\break
 \emph{Ē ya nai ka! }\hfill\break
Isn\textquotesingle t it great? }
 
\par{3. ええんちゃうか? (=いいんじゃないだろうか?) \hfill\break
 \emph{Ē n chau ka? }\hfill\break
Is it not great? }
 
\par{\textbf{Grammar Note }: ちゃう is a contraction of 違う and is used in many Western dialects to ask if something is in fact not the case while still seeking a positive affirmation. }
 
\par{4. あのサメ、怖えぇぇ! (=あのサメ、怖い!) \hfill\break
 \emph{Ano same, koē…! }\emph{ }\hfill\break
That shark is scary! }
 
\par{5. すげーだろ! (=すごいだろう!) \hfill\break
 \emph{Sugē daro! }\hfill\break
Isn\textquotesingle t that cool! }
 
\par{6. めっちゃ眠てー! (=とっても眠たい!) \hfill\break
 \emph{Metcha nemutē! }\hfill\break
I\textquotesingle m really sleepy! }
 
\par{7. 冷蔵庫も何にもねえ! (=冷蔵庫も何もない!) \hfill\break
 \emph{Reizōko mo nan\textquotesingle ni mo nē! }\hfill\break
There\textquotesingle s nothing in the fridge! }
 
\par{8. うっせーな、てめー。 (=うるさい、お前) \hfill\break
 \emph{Ussē na, temē. }\hfill\break
Shut up, you. \hfill\break
 \hfill\break
 \textbf{Word Note }: As this example shows, some words can be further contracted— \emph{urusai }\textrightarrow  \emph{ussē }. }
 
\par{This sound change also affects grammatical endings such as \emph{–nai }ない and \emph{–tai }たい }
 
\par{9. 弱ぇ奴は着る服も選べねェ! (=弱い奴は着る服も選べない!) \hfill\break
 \emph{Yoē yatsu wa kiru fuku mo erabenē! }\hfill\break
Weak guys can\textquotesingle t even choose their own clothes to wear! }
 
\par{10. たくさん食べてーなぁ。 (=たくさん食べたいなぁ) \hfill\break
\emph{Takusan tabetē nā. } \hfill\break
Man, I wanna eat a lot. }
 
\par{11. くだらねーことでいちいち突っかかってくんじゃねーよ。(=下らないことでいちいち突っかかっていくんじゃないよ) \hfill\break
 \emph{Kudaranē koto de ichi\textquotesingle ichi tsukkakatteku n ja nē yo. }\hfill\break
You can\textquotesingle t be charging at me with every little stupid thing. }
 
\par{12. つまんねーこと聞くなよ。(=つまらないことを聞くなよ) \hfill\break
 \emph{Tsuman\textquotesingle nē koto kiku na yo. }\hfill\break
Don\textquotesingle t ask absurd things. }
 
\begin{center}
\textbf{Idiosyncrasies } 
\end{center}

\par{ One must understand that the productivity of this sound change is not applied to all adjectives. There are plenty of idiosyncrasies and specialized variations depending on the adjective. }

\begin{ltabulary}{|P|P|P|P|}
\hline 

小さい \emph{Chiisai }& ちっちゃい・ちっちぇー \hfill\break
\emph{Chitchai\slash chitch }\emph{ē }& 寒い Samui & さみい・さめー \emph{Samii\slash sam }\emph{ē }\\ \cline{1-4}

悪い \emph{Warui }& わりい \emph{Warii }& 暑い・熱い Atsui & あちい・あちぇー \emph{Achii\slash ach }\emph{ē }\\ \cline{1-4}

安い \emph{Yasui }& やせえ \emph{Yas }\emph{ē }& まずい Mazui & ま(っ)ぜー \emph{Ma(z)z }\emph{ē }\\ \cline{1-4}

\end{ltabulary}

\par{ Another dialectical phenomenon that has an even wider distribution in Japan is dropping the final –i of adjectives and replacing it with a glottal stop, which is optionally spelled with っ. This is frequently employed when talking to oneself or when the adjective in question is not necessarily directed toward anyone. In other words, it has an expletive nature to it. }
 
\par{13. やばっ、逃げろ! \hfill\break
 \emph{Yaba…nigero! }\hfill\break
Crap…run! }
 
\par{14. 痛っ! \hfill\break
 \emph{Ita…! }\hfill\break
Ouch…! }
 
\par{15. このお湯、熱っ! \hfill\break
 \emph{Kono oyu, atsu…! }\hfill\break
This (bath) water\textquotesingle s hot! }
 
\par{16. くそ寒っ! \hfill\break
 \emph{Kuso samu…! }\hfill\break
It\textquotesingle s freaking cold…! }
 
\par{17. うわっ、臭っ! \hfill\break
 \emph{Uwa…, kusa…! }\hfill\break
Dang…it smells…! }
 
\begin{center}
\textbf{Vowel Elongation in Stem }
\end{center}

\par{ Another phenomenon that's not so much dialectical as it is emphatic, rather than stopping with a glottal stop, emphasis can be added to an adjective in casual speech by dropping I and elongating the preceding vowel. In fact, the final \slash i\slash  doesn\textquotesingle t have to be dropped for this to work. }
 
\par{18. 外、寒ー。 \hfill\break
 \emph{Soto, samū. }\hfill\break
Outside\textquotesingle s co-o-old! }
 
\par{19. すごーい! \hfill\break
 \emph{Sugōi! }\hfill\break
Coool! }
 
\par{20. 胃が痛ーい! \hfill\break
 \emph{I ga itāi! }\hfill\break
My stomach hurrts! }
    