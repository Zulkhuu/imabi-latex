    
\chapter{~慣れる, ~合う・合わせる, \& ~切る}

\begin{center}
\begin{Large}
第153課: ~慣れる, ~合う・合わせる, \& ~切る 
\end{Large}
\end{center}
 
\par{ This lesson continues on with intermediate compound verb endings. \hfill\break
}

\par{\textbf{Grammar Note }: 来る can only be used with the following patterns: 来 \textbf{慣れる }(to be used to coming to) and 来 \textbf{合わせる }(to make one's appearance). If you think about the meanings of these phrases as they are discussed, this will make sense to you. }
      
\section{~慣れる}
 
\par{  ${\overset{\textnormal{な}}{\text{慣}}}$ れる, an 一段   verb, means "to get used to" and may also mean "to domesticate", which is in essence animals getting used to human control. 慣れる is intransitive and is used as a compound ending to show an inclination to liking or being accustomed to something. }

\par{1. この街には住み慣れました。 \hfill\break
I got used to living in this town. }

\par{2. 彼は都市の生活に慣れることができなかった。 \hfill\break
He couldn't get used to living in the city. }

\par{3. 通学に慣れるのは ${\overset{\textnormal{じゅうよう}}{\text{重要}}}$ です。 \hfill\break
It is important that you become used to commuting to school. }

\par{4. コンピューターは使い慣れるのは時間がかかります。 \hfill\break
It takes time to get used to the computer. }

\par{5. 彼は ${\overset{\textnormal{たびな}}{\text{旅慣}}}$ れている。 \hfill\break
He's well-traveled. }

\par{6a. 日本料理\{は、・を\}食べ慣れました。 \hfill\break
6b. 日本料理に慣れました。(もっと自然) \hfill\break
I've gotten used to eating Japanese food. }

\par{7. 私は ${\overset{\textnormal{ひとまえ}}{\text{人前}}}$ で話すことに慣れていません。 \hfill\break
I am not used to speaking in front of people. }

\par{8. 日本の ${\overset{\textnormal{てんこう}}{\text{天候}}}$ に慣れる。 \hfill\break
To get used to Japanese weather. }

\par{9. まだ明るさに慣れてねー。( ${\overset{\textnormal{くだ}}{\text{砕}}}$ けた言い方) \hfill\break
(My eyes) are still not adjusted to the light. }

\par{10. ${\overset{\textnormal{ちょうあい}}{\text{寵愛}}}$ に慣れるのはだめですよ。 \hfill\break
It is no good to be over-familiar with attention. }

\par{\textbf{Orthography Note }: なれる may be written as 狎れる when specifically referring to domestication. }
      
\section{~合う・合わせる}
 
\par{ 合う shows how something "fits". In compounds it shows "to\dothyp{}\dothyp{}\dothyp{}with each other". So, there is someone else with you doing the same action. Essentially, ~合う shows people doing something with each other--reciprocal action. The object(s) of the sentence must be similar. }

\par{ The causative form 合わせる may show that one makes something into one, looks into differences, reciprocally does, or does something all of a sudden. }

\par{11. ロミオとジュリエットは愛し合っていた。 \hfill\break
Romeo and Juliet loved each other. }

\par{12. ${\overset{\textnormal{さそ}}{\text{誘}}}$ い合わせる。 \hfill\break
To invite each other. }

\par{13. 彼らは宿題を ${\overset{\textnormal{てつだ}}{\text{手伝}}}$ い合った。 \hfill\break
They helped each other with homework. }

\par{14. ${\overset{\textnormal{はんこうげんば}}{\text{犯行現場}}}$ に居合わせました。 \hfill\break
I was present at the crime scene. }

\par{15. 何時に待ち合わせようか? \hfill\break
What time shall we meet? }

\par{16. 彼と彼女は君が ${\overset{\textnormal{よ}}{\text{代}}}$ を歌い合っていた。? \hfill\break
彼と彼女は君が代を一緒に歌っていた。〇 \hfill\break
He and she sang Kimigayo  with each other. }

\par{\textbf{Culture Note }: ${\overset{\textnormal{きみ}}{\text{君}}}$ が ${\overset{\textnormal{よ}}{\text{代}}}$ is the national anthem ( ${\overset{\textnormal{こっか}}{\text{国歌}}}$ ) of Japan. }

\par{17. 二人で ${\overset{\textnormal{せいしょ}}{\text{聖書}}}$ を読み上げ合いました。 \hfill\break
The two read aloud the bible with each other. }

\par{18. 彼らは ${\overset{\textnormal{おも}}{\text{想}}}$ いを打ち明け合うだろう。 \hfill\break
They will probably confide their feelings with each other. }

\par{19. ケーキを一緒に\{食べましょう 〇・ 食べ合いましょう X\}。 \hfill\break
Let's eat cake together. }

\par{20. 乗り合わせた乗客 \hfill\break
Fellow passengers }

\par{21. 計算を読み合わせる。 \hfill\break
To read out and compare calculations. }

\par{22. このドレスはあたしにぴったり\{(と)合いますか・ですか\}。 (Feminine) \hfill\break
Does this dress fit perfectly with me? }

\par{23. 申し合わせた通りにしましょう。 \hfill\break
Let's do as arranged. }

\par{24. お ${\overset{\textnormal{たが}}{\text{互}}}$ いに引き合わせる。 \hfill\break
To introduce to each other. }

\par{25. 警察に問い合わせましたか。 \hfill\break
Did you check with the police? }

\par{26. ${\overset{\textnormal{しょるい}}{\text{書類}}}$ を ${\overset{\textnormal{と}}{\text{綴}}}$ じ合わせましたか。 \hfill\break
Have you bound the documents together? }

\par{27. 布を ${\overset{\textnormal{ぬ}}{\text{縫}}}$ い合わせる。 \hfill\break
To sew together. }

\par{28. ${\overset{\textnormal{きずぐち}}{\text{傷口}}}$ を縫い合わせてもらった。 \hfill\break
I had my wound(s) sewn up. }

\par{29. 各種 ${\overset{\textnormal{づ}}{\text{詰}}}$ め ${\overset{\textnormal{あ}}{\text{合}}}$ わせキャンディーを ${\overset{\textnormal{か}}{\text{買}}}$ いに ${\overset{\textnormal{い}}{\text{行}}}$ った。 \hfill\break
I went to buy mixed candies. }

\par{30. それらの糸は ${\overset{\textnormal{よ}}{\text{縒}}}$ り合わせ ${\overset{\textnormal{がた}}{\text{難}}}$ い。 \hfill\break
These threads are hard to twist together. }

\par{31. 悲しみを分かち合う。 \hfill\break
To share in the sadness. }

\par{\textbf{Word Note }: 分かち合う is typically interchangeable with 分け合う, but it is generally more 文語的. However, in the case of "sharing" non-physical items such as sadness, you should use 分かち合う. }

\par{32. 家族の都合が合うのは今日だけだ。 \hfill\break
Today is only when my convenience matches with my family's. }

\par{\textbf{Word Note }: 都合が合う is unavoidable in contexts like these, but it is generally not liked when referring to just one's convenience. In which case, you should say 都合がいい. }

\par{ There is also another ending that shows mutual action: ~違える. This sometimes confusingly has a meaning of ~間違える. This is because the verb 違える itself has the following meanings: to not have something be the same; to mess up; to be against a contract; to injure one's muscles. You have to essentially learn on a case by case basis. }

\par{33. 靴を履き違える。 \hfill\break
To mix up shoes. }

\par{34. 約束を違える。 \hfill\break
To break one's promise. }

\par{\textbf{Reading Note }: This 違える may be read as either ちがえる or たがえる. }

\par{35. 寝違えて、首筋が痛い。 \hfill\break
I got a crick in my neck, and now it hurts. }

\par{36. 社長と刺し違える。 \hfill\break
To expose the company president and receive repercussions from him\slash her. }

\par{37. 一審を差し違える。 \hfill\break
To overturn the first match decision (and give the win to the other opponent). }
      
\section{~切る}
 
\par{  ${\overset{\textnormal{き}}{\text{切}}}$ る means "to cut" and may be used literally and figuratively. ~切る shows that "something is done completely". }

\par{38. 彼は読書に ${\overset{\textnormal{ひた}}{\text{浸}}}$ り切っていた。 \hfill\break
He was completely engrossed in reading. }

\par{39. 紙をはさみで切る。 \hfill\break
To cut paper with scissors. }

\par{\textbf{漢字 Note }: 鋏 is not uncommonly used to spell はさみ. }

\par{40. 彼はかの歌手の全ての歌を歌い切った。 \hfill\break
He sang all of that singer's songs completely. }

\par{41. いつかは地球に残っている石油は全て使い切ってしまうだろう。 \hfill\break
The remaining oil on the earth will probably one day end up completely used. }

\par{42. ${\overset{\textnormal{ひも}}{\text{紐}}}$ を切る。 \hfill\break
To cut a string. }

\par{43. 電話を切る。 \hfill\break
To hang up the phone. }

\par{44. \{手・関係\}を切る。 \hfill\break
To break a relationship. }

\par{45. 彼がイギリス ${\overset{\textnormal{かいきょう}}{\text{海峡}}}$ の ${\overset{\textnormal{きょり}}{\text{距離}}}$ を泳ぎ切りました。 \hfill\break
He completely swam the (entire) distance of the English Channel. }

\par{46. 水泳選手は、日本海を一緒に泳ぎきりました。 \hfill\break
The swimmers swam across the Sea of Japan together. }

\par{47. ジョーンズ先生は来月までに ${\overset{\textnormal{やく}}{\text{約}}}$ 10冊読み切っているでしょう。 \hfill\break
Ms. Jones will have probably read around 10 books by next month. }
    