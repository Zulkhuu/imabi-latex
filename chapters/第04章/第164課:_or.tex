    
\chapter{Or}

\begin{center}
\begin{Large}
第164課: Or 
\end{Large}
\end{center}
 
\par{ This lesson is about the several ways to say "or," but it is not exhaustive. Nevertheless, it will provide you what you need to know to understand how to express "or" correctly in Japanese. }
      
\section{Or}
 
\begin{center}
  \textbf{か・それか・それとも }
\end{center}

\par{\textbf{} か can also list noun phrases to mean "or." When you start a new sentence to list things with "or," you use それか. If you start a new sentence in listing with "or" but are making a question, you use それとも. か functions for both situations when you don't make a new sentence. }

\par{1. すしか ${\overset{\textnormal{}}{\text{さしみ}}}$ を ${\overset{\textnormal{}}{\text{食}}}$ べるつもりです。 \hfill\break
I intend to eat sushi or sashimi. }

\par{\textbf{Grammar Note }: か is not necessary after sashimi. Adding it would be old-fashioned. }

\par{2. すしを ${\overset{\textnormal{}}{\text{食}}}$ べる。それか、フランス ${\overset{\textnormal{}}{\text{料理}}}$ を ${\overset{\textnormal{}}{\text{食}}}$ べる。 \hfill\break
Eat sushi, or eat French cuisine. }

\par{3. すしを食べる? それとも、フランス料理を食べるの? \hfill\break
Will you have sushi? Or, will you have French cuisine? }

\par{${\overset{\textnormal{}}{\text{4. 手}}}$ で ${\overset{\textnormal{}}{\text{書}}}$ くか。それとも、タイプするか。 \hfill\break
Will you write it by hand? Or, will you type it? }

\par{${\overset{\textnormal{}}{\text{5. 料理}}}$ をする。それか、 ${\overset{\textnormal{}}{\text{掃除}}}$ をする。 \hfill\break
I'll cook. Or, I'll do the cleaning. }

\par{\textbf{Practice }: Translate the following. You may use a dictionary. }

\par{1. A nation like Japan or China. \hfill\break
2. I'll eat fish. Or, I'll eat pizza. \hfill\break
3. Will you go next month? Or, will you go next week? \hfill\break
4. He is around 20 years old. \hfill\break
5. Is he smart? }

\begin{center}
 \textbf{か(どうか) }
\end{center}

\par{ "Whether (or not)" is "か(どう)か".  どう may only be added when the embedded question doesn't have a question word like 何 in it.  When just か is used, the "not" is implied. }

\par{6. ${\overset{\textnormal{かれ}}{\text{彼}}}$ が ${\overset{\textnormal{おとうと}}{\text{弟}}}$ さんかどうかは ${\overset{\textnormal{あや}}{\text{怪}}}$ しい。 \hfill\break
It is doubtful that he is his little brother. }
 
\par{7. 彼女がパーティーに来るかどうか(を)知っていますか。 \hfill\break
Do you know if she is coming to the party or not? }
 
\par{8. 明日までに宿題ができるかどうかわかんない。 \hfill\break
I don't know whether or not I'll be able to finish my homework by tomorrow. }
 
\par{9. 「 ${\overset{\textnormal{あした}}{\text{明日}}}$ はいい ${\overset{\textnormal{てんき}}{\text{天気}}}$ でしょうか」「あの、いい天気かどうか \textbf{分かりません }」 \hfill\break
“Is tomorrow's weather going to be good?” “Uh, I don't know if it's going to be good weather or not”. }
 
\par{10. 先生が学校に行った \textbf{か }は ${\overset{\textnormal{ふめい}}{\text{不明}}}$ だ。 \hfill\break
It is unknown \textbf{whether }the teacher went to the school. }
 
\par{11. だれ(だ)か分からない。 \hfill\break
I don't know who he is. }
 
\par{\textbf{Grammar Note }: The だ isn't necessary when followed by か inside a subordinate clause. }
 
\par{\textbf{Variant Note }: かいなか is a very formal variant of かどうか. }
 
\begin{center}
\textbf{A }\textbf{かA }\textbf{ないか: No More Than }
\end{center}
 
\par{ This usage is \textbf{exclusively used with counter phrases }. As the examples below suggest, A is the same verbal expression in the affirmative and negative form respectively. }
 
\par{12. 2千円するかしないかだ。 \hfill\break
It costs no more than 2,000 yen. }

\par{13. ${\overset{\textnormal{ほすう}}{\text{歩数}}}$ は千歩行くか行かないかだ。 \hfill\break
The number of steps is no more than 1000. }

\par{14. 彼の ${\overset{\textnormal{}}{\text{年}}}$ のころは50 ${\overset{\textnormal{さい}}{\text{歳}}}$ になるかならないかだ。 \hfill\break
He's no more than fifty years old. }
      
\section{Literary "Or" Phrases}
 
\begin{center}
 \textbf{または }
\end{center}

\par{ または, rarely written as 又は, is used in situations such as when you want to express tolerance\slash allowance of either options presented or when out of two things you take one and get rid of the other. }

\par{15. 電話または電報で知らせる。 \hfill\break
To inform by either phone or telegram. }

\par{16. 肉または魚の料理を準備する。 \hfill\break
To either prepare meat or fish.  }

\par{17. A \textbf{または }Bに〇を ${\overset{\textnormal{}}{\text{付}}}$ ける。 \hfill\break
To put a 〇 to A or B. }

\par{\textbf{Reminder Note }: As for \textbf{または }, it gives the sense that either is fine. }

\begin{center}
\textbf{もしくは }
\end{center}

\par{ もしくは, rarely written in 漢字 as 若しくは, is used in limited situations where you choose something out of several options. }

\par{18. 万年筆もしくはボールペンで書くこと \hfill\break
Writing with a fountain pen or a ballpoint pen. }

\par{${\overset{\textnormal{}}{\text{19. 手紙}}}$ \textbf{もしくは }${\overset{\textnormal{}}{\text{電話}}}$ で ${\overset{\textnormal{}}{\text{連絡}}}$ する。 \hfill\break
To contact via letter or phone. }

\par{\textbf{Remember Note }: もしくは should only be used if the options are not significantly different. }

\begin{center}
\textbf{あるいは }
\end{center}

\par{ あるいは, rarely in 漢字 as 或(い)は, is used in situations where you are showing that things are alternate or both simultaneous, but it is not normally used in showing permission\slash allowance. }

\par{20. インタビューの結果を口頭発表、あるいは論文の形で報告する。 \hfill\break
To either report the interview results by oral representation or essay form. }

\par{${\overset{\textnormal{}}{\text{21. 東京}}}$ \textbf{あるいは }ソウルのような ${\overset{\textnormal{}}{\text{首都}}}$ \hfill\break
A capital city like Tokyo or Seoul. \hfill\break
\hfill\break
Note: As for \textbf{あるいは }, the items must be of the same kind. }

\begin{center}
 \textbf{ないし }
\end{center}

\par{ ないし, rarely spelled in 漢字 as 乃至, doesn't merely suggest A or B but A and B and what's in between. This is quite different from the other options. So, pay attention to this. }

\par{${\overset{\textnormal{}}{\text{22. 北}}}$ \textbf{ないし }${\overset{\textnormal{}}{\text{北東}}}$ \hfill\break
North or northeast }

\par{\textbf{Etymology Note }: As the characters show, this is not a combination of the negative ない and the particle し. }

\par{\textbf{Reminder Note }: All of these are rather formal and literary and would be replaced by か in the spoken language. }
    