    
\chapter{The Body}

\begin{center}
\begin{Large}
第160課: The Body 
\end{Large}
\end{center}
 
\par{ This lesson will be about the parts of the body. }
      
\section{The Body}
  
\par{  Many phrases relate to the body. Below is a chart of the most important parts of the body.  \hfill\break
\hfill\break
}

\begin{ltabulary}{|P|P|P|P|P|P|}
\hline 

Blood & 血 & ち \hfill\break
& Blood vessel & 血管 & けっかん \hfill\break
\\ \cline{1-6}

Hand & 手 & て \hfill\break
& Arm & 腕 & うで \hfill\break
\\ \cline{1-6}

Finger & 指 & ゆび & Wrist & 手首 & てくび \hfill\break
\\ \cline{1-6}

Ankle & 足首 & あしくび \hfill\break
& Heel & 踵 & かかと \hfill\break
\\ \cline{1-6}

Shoulder & 肩 & かた \hfill\break
& Nail & 爪 & つめ \hfill\break
\\ \cline{1-6}

Shin & 脛 & すね \hfill\break
& Knee & 膝 & ひざ \hfill\break
\\ \cline{1-6}

Calf & 脹脛 & ふくらはぎ \hfill\break
& Thigh & 腿 & もも \hfill\break
\\ \cline{1-6}

Thumb & 親指・拇 & おやゆび \hfill\break
& Index finger & 人差し指 & ひとさしゆび \hfill\break
\\ \cline{1-6}

Middle finger & 中指 & なかゆび \hfill\break
& Ring finger & 薬指 & くすりゆび \hfill\break
\\ \cline{1-6}

Pinky & 小指 & こゆび \hfill\break
& Palm & 手の平 & てのひら \hfill\break
\\ \cline{1-6}

Intestines & 腸 & ちょう \hfill\break
& Bellybutton & 臍 & へそ \hfill\break
\\ \cline{1-6}

Head & 頭 & あたま & Hair & 髪(の毛) & かみ(のけ) \\ \cline{1-6}

Forehead & 額 & ひたい \hfill\break
& Throat & 喉 & のど \hfill\break
\\ \cline{1-6}

Brain & 脳 & のう \hfill\break
& Tongue & 舌 & した \hfill\break
\\ \cline{1-6}

Eyes & 目、瞳 & め、ひとみ \hfill\break
& Eyeball & 眼球 & がんきゅう \hfill\break
\\ \cline{1-6}

Eyelash & 睫毛 & まつげ \hfill\break
& Eyebrow & 眉(毛) & まゆ(げ) \hfill\break
\\ \cline{1-6}

Ear & 耳 & みみ \hfill\break
& Eardrum & 鼓膜 & こまく \hfill\break
\\ \cline{1-6}

Skull & 頭蓋骨 & ずがいこつ \hfill\break
& Nose & 鼻 & はな \hfill\break
\\ \cline{1-6}

Nose hair & 鼻毛 & はなげ \hfill\break
& Cheek & 頬 & ほお・ほほ \\ \cline{1-6}

Mouth & 口 & くち \hfill\break
& Teeth & 歯 & は \hfill\break
\\ \cline{1-6}

Gums & 歯茎 & はぐき \hfill\break
& Lips & 唇 & くちびる \hfill\break
\\ \cline{1-6}

Jaw & 顎・頤 & あご \hfill\break
& Chin & 下顎 & したあご \hfill\break
\\ \cline{1-6}

Face & 顔 & かお \hfill\break
& Neck & 首、頸部 & くび \\ \cline{1-6}

Spine & 脊柱、背骨 & せきちゅう、せぼね \hfill\break
& Back & 背(中) & せ(なか) \hfill\break
\\ \cline{1-6}

Chest & 胸 & むね \hfill\break
& Esophagus & 食道 & しょくどう \hfill\break
\\ \cline{1-6}

Stomach & 胃 & い \hfill\break
& Internal organ & 内臓 & ないぞう \hfill\break
\\ \cline{1-6}

Bowel & 腸 & はらわた \hfill\break
& Liver & 肝臓 & かんぞう \hfill\break
\\ \cline{1-6}

Pancreas & 膵臓 & すいぞう \hfill\break
& Appendix & 盲腸 & もうちょう \hfill\break
\\ \cline{1-6}

Small intestine & 小腸 & しょうちょう & Large intestine & 大腸 & だいちょう \hfill\break
\\ \cline{1-6}

Abdomen & 腹 & はら \hfill\break
& Heart & 心臓 & しんぞう \hfill\break
\\ \cline{1-6}

Gall bladder & 胆のう \hfill\break
& たんのう \hfill\break
& Lungs & 肺(臓) & はい(ぞう) \hfill\break
\\ \cline{1-6}

Kidney & 腎臓 & じんぞう \hfill\break
& Bone & 骨 & ほね \hfill\break
\\ \cline{1-6}

Pelvis & 骨盤 & こつばん \hfill\break
& Leg; foot & 足 (foot)、脚 (leg) & あし \\ \cline{1-6}

\end{ltabulary}
 
\par{\textbf{Word Notes }: }
 
\par{1. To say "left" or "right X", all you need to do is add 左 and 右 respectively. So, right leg would be 右足. }

\par{2. 手 sometimes refers to full arm. }

\begin{center}
\textbf{Examples } 
\end{center}
 Blood 血 ち \hfill\break
Blood vessel 血管 けっかん \hfill\break
Hand 手 て \hfill\break
Arm 腕 うで \hfill\break
Finger 指 ゆび Wrist 手首 てくび \hfill\break
Ankle 足首 あしくび \hfill\break
Heel 踵 かかと \hfill\break
Shoulder 肩 かた \hfill\break
Nail 爪 つめ \hfill\break
Shin 脛 すね \hfill\break
Knee 膝 ひざ \hfill\break
Calf 脹脛 ふくらはぎ \hfill\break
Thigh 腿 もも \hfill\break
Thumb 親指・拇 おやゆび \hfill\break
Index finger 人差し指 ひとさしゆび \hfill\break
Middle finger 中指 なかゆび \hfill\break
Ring finger 薬指 くすりゆび \hfill\break
Pinky 小指 こゆび \hfill\break
Palm 手の平 てのひら \hfill\break
Intestines 腸 ちょう \hfill\break
Bellybutton 臍 へそ \hfill\break
Head 頭 あたま Hair 髪(の毛) かみ(のけ) Forehead 額 ひたい \hfill\break
Throat 喉 のど \hfill\break
Brain 脳 のう \hfill\break
Tongue 舌 した \hfill\break
Eyes 目、瞳 め、ひとみ \hfill\break
Eyeball 眼球 がんきゅう \hfill\break
Eyelash 睫毛 まつげ \hfill\break
Eyebrow 眉(毛) まゆ(げ) \hfill\break
Ear 耳 みみ \hfill\break
Eardrum 鼓膜 こまく \hfill\break
Skull 頭蓋骨 ずがいこつ \hfill\break
Nose 鼻 はな \hfill\break
Nose hair 鼻毛 はなげ \hfill\break
Cheek 頬 ほお・ほほ Mouth 口 くち \hfill\break
Teeth 歯 は \hfill\break
Gums 歯茎 はぐき \hfill\break
Lips 唇 くちびる \hfill\break
Jaw 顎・頤 あご \hfill\break
Chin 下顎 したあご \hfill\break
Face 顔 かお \hfill\break
Neck 首、頸部 くび Spine 脊柱、背骨 せきちゅう、せぼね \hfill\break
Back 背(中) せ(なか) \hfill\break
Chest 胸 むね \hfill\break
Esophagus 食道 しょくどう \hfill\break
Stomach 胃 い \hfill\break
Internal organ 内臓 ないぞう \hfill\break
Bowel 腸 はらわた \hfill\break
Liver 肝臓 かんぞう \hfill\break
Pancreas 膵臓 すいぞう \hfill\break
Appendix 盲腸 もうちょう \hfill\break
Small intestine 小腸 しょうちょう Large intestine 大腸 だいちょう \hfill\break
Abdomen 腹 はら \hfill\break
Heart 心臓 しんぞう \hfill\break
Gall bladder 胆のう \hfill\break
たんのう \hfill\break
Lungs 肺(臓) はい(ぞう) \hfill\break
Kidney 腎臓 じんぞう \hfill\break
Bone 骨 ほね \hfill\break
Pelvis 骨盤 こつばん \hfill\break
Leg; foot 足 (foot)、脚 (leg) あし 
\par{1. のどが ${\overset{\textnormal{いた}}{\text{痛}}}$ い。 \hfill\break
To have a sore throat. }
 
\par{3. ドアに\{ ${\overset{\textnormal{}}{\text{指・親指\}}}}$ を ${\overset{\textnormal{はさ}}{\text{挟}}}$ む。 \hfill\break
To smash one's [finger\slash thumb] in a door. }

\par{5. ${\overset{\textnormal{のう}}{\text{能}}}$ あるタカは ${\overset{\textnormal{つめ}}{\text{爪}}}$ を ${\overset{\textnormal{かく}}{\text{隠}}}$ す。(Idiomatic) \hfill\break
Still waters run deep. \hfill\break
Literally: A hawk with talent hides its talons. }
 
\par{${\overset{\textnormal{}}{\text{7. 顔}}}$ が ${\overset{\textnormal{き}}{\text{利}}}$ く。 \hfill\break
To have many contacts. \hfill\break
Literally: The face effective. }
 
\par{9. のどが ${\overset{\textnormal{つ}}{\text{詰}}}$ まる。 \hfill\break
To choke. }
 
\par{${\overset{\textnormal{}}{\text{11. 耳}}}$ が ${\overset{\textnormal{}}{\text{遠}}}$ い。 \hfill\break
To be hard of hearing. }
 
\par{${\overset{\textnormal{}}{\text{13. 耳}}}$ に ${\overset{\textnormal{のこ}}{\text{残}}}$ る。 \hfill\break
To tingle in the ears. }
 
\par{${\overset{\textnormal{}}{\text{15. 耳}}}$ を ${\overset{\textnormal{ふさ}}{\text{塞}}}$ ぐ。 \hfill\break
To close one's ears. }
 
\par{${\overset{\textnormal{}}{\text{17. 鼻}}}$ をほじる。 \hfill\break
To pick one's nose. }

\par{19. よく ${\overset{\textnormal{}}{\text{舌}}}$ が ${\overset{\textnormal{}}{\text{回}}}$ る。 \hfill\break
To be quite eloquent. }

\par{${\overset{\textnormal{}}{\text{21. 胸}}}$ に ${\overset{\textnormal{}}{\text{閉}}}$ まる。 \hfill\break
To keep to oneself. }

\par{23a. 錠を服用する。X \hfill\break
23b. 錠剤を服用する。 \hfill\break
To take a pill. \hfill\break
\textbf{\hfill\break
Vocabulary Note }: ~錠 is the counter for pills. }

\par{25. 予約をしてもらえますか。 \hfill\break
Could you make an appointment for me? }

\par{\textbf{Grammar Note }: Using the negative form is OK here as well or a more honorific final form like いただけませんか. }
 2. のどが ${\overset{\textnormal{かわ}}{\text{渇}}}$ いた。 \hfill\break
I'm thirsty. 
\par{${\overset{\textnormal{}}{\text{4. 手}}}$ を ${\overset{\textnormal{あ}}{\text{挙}}}$ げる。 \hfill\break
To surrender. \hfill\break
Literally: To raise one's hands\slash arms. }

\par{${\overset{\textnormal{}}{\text{6. 手}}}$ が ${\overset{\textnormal{た}}{\text{足}}}$ りない。 \hfill\break
To be shorthanded. }

\par{8. ${\overset{\textnormal{ひざ}}{\text{膝}}}$ の ${\overset{\textnormal{さら}}{\text{皿}}}$ \hfill\break
 Kneecap }

\par{${\overset{\textnormal{}}{\text{10. 地}}}$ に ${\overset{\textnormal{}}{\text{膝}}}$ を ${\overset{\textnormal{つ}}{\text{突}}}$ く。 \hfill\break
To go to the ground on one's knees. }

\par{${\overset{\textnormal{}}{\text{12. 顔}}}$ を ${\overset{\textnormal{つぶ}}{\text{潰}}}$ す。 \hfill\break
To embarrass. \hfill\break
Literally: To crush one's face. }

\par{${\overset{\textnormal{}}{\text{14. 鼻}}}$ がいい。 \hfill\break
Have a good sense of smell. }

\par{${\overset{\textnormal{}}{\text{16. 鼻}}}$ が ${\overset{\textnormal{}}{\text{高}}}$ い。 \hfill\break
To be proud. }

\par{${\overset{\textnormal{}}{\text{18. 鼻}}}$ であしらう。 \hfill\break
To snub. }

\par{20. のどを ${\overset{\textnormal{とお}}{\text{通}}}$ らない。 \hfill\break
To have no appetite. }

\par{22. 胃腸が\{強い・弱い\}。 \hfill\break
To have a [strong\slash weak] system. }

\par{24. こういう ${\overset{\textnormal{むいしき}}{\text{無意識}}}$ の時間は ${\overset{\textnormal{だいのうへんえんけい}}{\text{大脳辺縁系}}}$ の ${\overset{\textnormal{かいば}}{\text{海馬}}}$ が ${\overset{\textnormal{つかさど}}{\text{司}}}$ っている。 \hfill\break
Such unconscious time is controlled by the             hippocampus of the limbic system. \hfill\break
By 池田 ${\overset{\textnormal{きよひこ}}{\text{清彦}}}$ , a biologist, in the commentary for 冷たい ${\overset{\textnormal{ゆうわく}}{\text{誘惑}}}$  \emph{Cold Temptations }by ${\overset{\textnormal{のなみ}}{\text{乃南}}}$ アサ. }
 
\par{\textbf{Anatomy Note }: If you don't know what the limbic system or the hippocampus is, the 漢字 tell you they're about the brain. }

\begin{center}
 \textbf{凝る Versus 懲りる }
\end{center}
 
\par{ These words are often confused because they sound similar, and sometimes when not written in 漢字, they resemble each other. The first is a 五段 verb and the second is a 一段 verb. The following chart describes their most important usages. }

\begin{ltabulary}{|P|P|P|P|}
\hline 

凝る (こる) & 使用例 & 懲りる (こりる) & 使用例 \\ \cline{1-4}

To be stiff & 肩が凝っている。 & To learn a lesson & 彼女の運転には懲りた。 \\ \cline{1-4}

To be addicted to & 釣りに凝る。 &  &  \\ \cline{1-4}

To be intricate with & 凝りすぎた文体 &  &  \\ \cline{1-4}

\end{ltabulary}

\par{-------------------------------------------------------------------------------- }

\begin{center}
\textbf{会話 1 }
\end{center}

\par{26. お母さん: あら、セッス、顔色がよくないわね。 \hfill\break
セス: ええ、実は、今朝から何となく寒気がして、それに頭が痛くて。。。 \hfill\break
お母さん: 熱はあるの? \hfill\break
セス: さあ、分かりません。でも、あるかもしれません。 \hfill\break
お母さん: そうねえ。顔がちょっと赤いわねえ。風邪かしら。お医者さんに診察をもらったほうがいいよ。電話をかけてみるから。  }

\par{(電話を切る) }

\par{セス: どうもありがとうございました。 \hfill\break
お母さん: いいえ、でもよかったね、ちょうど空いてて。 }

\par{ There are several new things in this short little conversation. The mother uses a lot of feminine expressions, which haven't really been covered up to this point. あら is an expression like "oh my!" and is often used by female speakers. Sentence endings like わね, の (making a question), and かしら (I wonder) are all feminine. }

\begin{center}
\textbf{Grammar Points }
\end{center}

\par{~かもしれません = Might  ~たほうがいい = It's best to ~てみる = To try to\dothyp{}\dothyp{}\dothyp{} }

\begin{center}
\textbf{New Vocab }
\end{center}

\begin{ltabulary}{|P|P|P|}
\hline 

顔色 & かおいろ & Complexion \\ \cline{1-3}

何となく & なんとなく & Somehow or other \\ \cline{1-3}

お医者さん & おいしゃさん & Doctor \\ \cline{1-3}

診察 & しんさつ & Examination (medical) \\ \cline{1-3}

\end{ltabulary}

\begin{center}
\textbf{会話 2 }
\end{center}

\par{27. セス、医者に行く。 }

\par{医者: どうしましたか。 \hfill\break
セス: あのう、今朝から体がだるくなって、さっきのども痛くなって。 \hfill\break
医者: そうですか。 \hfill\break
セス: それに、ものを飲み込むとき、痛いんです。 \hfill\break
医者: いけませんね。食欲は? \hfill\break
セス: あまりありません。 \hfill\break
医者: じゃ、ちょっとのどを見てみましょう。大きく口を開けてください。ああ、やはりずいぶん赤いですね。風邪ですよ。薬を出しますから、一週間飲んでみてください。それから、一日に何回か、うがいをしてください。早く治ると思いますよ。 \hfill\break
セス: はい、分かりました。食事はどうしたらいいでしょうか。 \hfill\break
医者: そうですね。あのう、まあ、軟らかいものだけにしておいたらどうですか。おかゆぐらいにするんですが。 \hfill\break
セス: ありがとうございました。 \hfill\break
医者: お大事に。 }

\begin{center}
\textbf{Grammar Points }
\end{center}

\par{1. ~ておく: Used to show action in advance or preparation of something. \hfill\break
2. ~たらいいでしょうか = What do I do about? \hfill\break
3. ~たらどうですか = How about doing\dothyp{}\dothyp{}\dothyp{}? \hfill\break
4. ~ぐらい is used here to show a minimal limit. \hfill\break
5. ~てみましょう = Let's try to\dothyp{}\dothyp{}\dothyp{} Here it is used to more indirect and polite to the patient. }

\par{\textbf{Culture Note }: お大事に is only said to those that are in need of care. }

\begin{ltabulary}{|P|P|P|}
\hline 

怠い & だるい & To feel sluggish \\ \cline{1-3}

飲み込む & のみこむ & To swallow \\ \cline{1-3}

食欲 & しょくよく & Appetite \\ \cline{1-3}

何回か & なんかいか & Several times \\ \cline{1-3}

\end{ltabulary}

\begin{ltabulary}{|P|P|P|}
\hline 

治る & なおる & To heal \\ \cline{1-3}

嗽をする & うがいをする & To gargle \\ \cline{1-3}

軟らかい & やわらかい & Soft \\ \cline{1-3}

お大事に & おだいじに & Take care \\ \cline{1-3}

\end{ltabulary}
     