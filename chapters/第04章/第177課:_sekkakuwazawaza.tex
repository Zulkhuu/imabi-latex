    
\chapter{せっかく \& わざわざ}

\begin{center}
\begin{Large}
第177課: せっかく \& わざわざ 
\end{Large}
\end{center}
 
\par{ These are mixed up all the time by students. Maybe it's because わざわざ is just so fun to say? }
      
\section{せっかく \& わざわざ}
 
\par{ せっかく shows something that is done through great trouble or time to reach a certain point or condition. It is an adverbial noun, so it is possible for it to act as a noun. Again, there is some sort of big cost implied, and it is that cost being put into good use that's being insisted on. }

\par{ There are several grammatical environments that せっかく can be used in. }

\begin{itemize}

\item せっかくの: Used as a noun with の to be an attribute. 
\item せっかく\dothyp{}\dothyp{}\dothyp{}のだから: Combined with のだから to stress hard efforts that are to be well known by the listener. 
\item せっかく\dothyp{}\dothyp{}\dothyp{}のに: This is often used in declining something from someone, assuming the person went through the trouble. However, depending on the situation, it can be used sardonically. It can also be used by the person who went through the hard efforts. Or it could be used to show a shame. 
\item せっかくだから: Used by someone excepting an opportunity from the person that went through the trouble. 
\item せっかくだが: Used in declining something from someone who went through the trouble. 
\end{itemize}

\par{ These are simply applications of せっかく with other patterns. Of course, you should always be aware of how polite you should be to someone. }

\begin{center}
\textbf{Examples } 
\end{center}

\par{1. せっかく日本語を習ったんだから、日本語をぺらぺらと話すことができるようになりたいです。 \hfill\break
Since I've taken the trouble to study Japanese, I want to become able to speak Japanese fluently. }

\par{\textbf{Nuance Note }: んだから implies that the person you're speaking to is well aware of your Japanese skills. }

\par{2. せっかくの苦労も ${\overset{\textnormal{あだ}}{\text{仇}}}$ になった。 \hfill\break
All my troubles have been done for nothing. }

\par{3. せっかくの苦労が水の ${\overset{\textnormal{あわ}}{\text{泡}}}$ になった。 \hfill\break
All my pains were in vain. }

\par{4. せっかくのご ${\overset{\textnormal{こうい}}{\text{厚意}}}$ ですから、お受けしましょう。 \hfill\break
For your great kindness, I accept. }

\par{5. せっかく習った日本語は忘れないようにしましょう。 \hfill\break
Please try not to forget Japanese, which you've taken the trouble to learn. }

\par{6. せっかく日本へ行っても、日本人と話さないと、日本語は上手になりません。 \hfill\break
Even if you make the efforts to go to Japan, if you don't speak with Japanese people, your Japanese won't get better. }

\par{7. せっかくアメリカへ来たんだから、しばらく ${\overset{\textnormal{たいざい}}{\text{滞在}}}$ してください。 \hfill\break
Since you've gone through the trouble to come to America, please stay for a while. }

\par{8. せっかくエッセイを書いたのに、文が全部消えてしまったよ。 \hfill\break
Although I spent so much energy writing my essay, it got completely erased! }

\par{9. せっかくやれるなら、やってほしいんだが。 (失礼な言い方) \hfill\break
If you could do it, I'd like you to, but\dothyp{}\dothyp{}\dothyp{} }

\par{\textbf{Orthography Note }: せっかく can be written in 漢字 as 折角 or 切角 }

\begin{center}
 \textbf{せっかく VS わざわざ }
\end{center}

\par{ せっかく gives a feeling that something should not be wasted because one has put a lot of effort into it. It can be used as a nominal, and it is often used in refusing a request. わざわざ, on the other hand, can only be used as an adverb and describes that there was no necessity to go through the trouble. It is often used out of care for a person. }

\par{10. お忙しいところをわざわざ来ていただいて、すみません。 \hfill\break
I'm sorry for having you come all the way while you're busy. }

\par{11. 焼き肉を食べにわざわざ北見市まで行きました。 \hfill\break
I went all the way to Kitami to eat yakiniku. }

\par{12. せっかくですがお ${\overset{\textnormal{かま}}{\text{構}}}$ いなく。 \hfill\break
Thank you, but don't trouble. }

\par{13. せっかく作ったんですから、食べてみてください。 \hfill\break
Since I went through the trouble to make it, please try to eat it. }

\par{14. わざわざ取りに帰らなくてもいいですよ。 \hfill\break
It's alright for you to not go all the way home to get it. }

\par{15. せっかくの休みなのに、仕事をしなけりゃいけないさ。(すごく砕けた言い方) \hfill\break
Although it's a long-awaited holiday, I have to work. }

\par{16a. 多忙なところをわざわざお越し頂きありがとうございました。(もっと自然) \hfill\break
16b. 多忙なところを押して来てもらってありがとうございました。 \hfill\break
Thank you very much for coming for me although you are quite busy. }
    