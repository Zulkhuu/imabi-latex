    
\chapter{The Auxiliary Verb ~ず I}

\begin{center}
\begin{Large}
第190課: The Auxiliary Verb ~ず I 
\end{Large}
\end{center}
 
\par{ ~ず is a Classical Japanese auxiliary verb that is still infrequently used in Modern Japanese. Its use in Modern Japanese is old-fashioned, but it is often used within sentences for poetic effects. ~ず is also in a several grammatical structures and set phrases. Set phrases, after all, is where you can expect to find archaisms in any language. Given its archaic status, it is mainly seen in 書き言葉. }
      
\section{The Auxiliary Verb ~ず}
 
\par{ The function of ~ず is to show negation. This means that it is going to follow the 未然形 of verbs. However, because it is older, it follows the original 未然形. So, for verbs whose 未然形 may have changed over the centuries, it is still the original one that is used with ~ず. }

\par{ Though extremely infrequent and almost entirely limited to set phrases, given that it has the potential to be attached to the 未然形 of adjectives, the following chart will show how to conjugate with it for both verbs and adjectives. }

\begin{ltabulary}{|P|P|P|P|}
\hline 

Class & Example Verb & 未然形 & 未然形+ず \\ \cline{1-4}

一段 Verbs & 見る &  \textbf{見 }- & 見ず \\ \cline{1-4}

五段 Verbs & 学ぶ & 学 \textbf{ば }- & 学ばず \\ \cline{1-4}

サ変 Verb & する &  \textbf{せ }- & せず \\ \cline{1-4}

カ変 Verb & 来る & 来( \textbf{こ })- & 来ず \\ \cline{1-4}

形容詞 & 少ない & 少な \textbf{から }- & 少なからず \\ \cline{1-4}

形容動詞 (Most Rare) & 華麗だ & 華麗 \textbf{なら }- & 華麗ならず \\ \cline{1-4}

\end{ltabulary}

\par{  ~ず also has three sets of bases, and as a consequence, a rather complicated history behind them. However, luckily for you, only the 連体形 ~ぬ and ~ざる and the 連用形・終止形-ず are ever used today. }

\par{ The 連体形 ~ぬ is used quite a lot, although it is typically limited to set phrases or more literary settings. It can also be used as the 終止形. ~ん comes from this usage. However, ~ない is said to derive from ~なふ, which came into being in northern Japanese dialects during the classical period. The ざる-連体形 is even more limited, and it is typically only seen in set phrases. However, it does find itself in the phrase ~ざるを得ない. }
 
\par{~ずに is just like ~ないで. ~ず as the 連用形 can also function like ~なくて. However, the form ~ずに is going to always be equivalent to ~ないで. }

\begin{center}
 \textbf{Examples }
\end{center}

\par{1. 連絡が取れ \textbf{ず }、心配しました。(Somewhat formal) \hfill\break
Without having contact, I got worried. }

\par{\textbf{Grammar Note }: The above sentence uses ~ず in the 連用形. }

\par{2. 分から \textbf{ず }屋 (Set Phrase) \hfill\break
An obstinate person. }

\par{3. 人知れ \textbf{ず }焦がれる。 \hfill\break
To inwardly yearn for. \hfill\break
\hfill\break
\textbf{Word Note }: 人知れず literally means "without it being known to people". }

\par{4. 新都はいまだ成ら \textbf{ず }。(Classical) \hfill\break
The new capital is still not completed. \hfill\break
From the 方丈記. }

\par{5. 絶え \textbf{ざる }不安     (Archaic\slash old-fashioned) \hfill\break
Anxiety that won't cease }

\par{6. 見ざる聞かざる言わざる・見猿聞か猿言わ猿 (Set Phrase) \hfill\break
See no evil, hear no evil, speak no evil. }

\par{\textbf{Phrase Note }: This expression is usually turned into a pun about monkeys, especially the ones that enter 温泉 in Japan. }

\par{7a. やる瀬ぬ X \hfill\break
7b. やる瀬(が)ない 〇 \hfill\break
Helpless }

\par{\textbf{Grammar Note }: やるせぬ comes from an over-generalization of ~ぬ. }

\par{8. 彼は知らずに側にいた。 \hfill\break
He was by my side without knowing it. }

\par{9. 生き急がずに、ときには立ち止まってあたりを見回すのもいいよ。 \hfill\break
It's also good to just stop once and awhile and look around and not live one's life so fast. }

\par{10. 何も書かん。 (Slang; dialectical) \hfill\break
I won't write anything. }

\par{11. それは開いた口が塞がらぬことだ。 (ちょっと古風) \hfill\break
That's a jaw-dropping situation. }

\par{12. \textbf{取りも直さず }やりくりが大変になるということだ。(Set phrase) \hfill\break
Which is to say, managing will become challenging. }

\par{13. 怖いものもあら \textbf{ず }。(Archaic) \hfill\break
To not even have things one is scared of. }

\par{14a. 好まし \textbf{からざる }状態 (古風な書き言葉) \hfill\break
14b. 好まし \textbf{くない }状態 (話し言葉) \hfill\break
An unfavorable situation }

\par{15. 瀧野は金彌の片足を拾い上げて、土踏まずを摑んだ。 \hfill\break
Takino picked up one of Kanaya's legs and gripped her foot's arch. \hfill\break
From 童謡 by 川端康成. }

\par{16. 雪を積らせ \textbf{ぬ }ためであろう、 ${\overset{\textnormal{ゆぶね}}{\text{湯槽}}}$ から ${\overset{\textnormal{あふ}}{\text{溢}}}$ れる湯を ${\overset{\textnormal{にわか}}{\text{俄}}}$ づくりの ${\overset{\textnormal{みぞ}}{\text{溝}}}$ で宿の壁沿いにめぐらせてあるが、玄関先では浅い ${\overset{\textnormal{せんすい}}{\text{泉水}}}$ のように ${\overset{\textnormal{ひろ}}{\text{拡}}}$ がっていた。 \hfill\break
It must be because they don't allow snow to pile up, but the hot water flowing from the tubs was made to encircle the inn along the walls in a make-shift ditch, and it stretched like shallow spring water at the entrance. \hfill\break
From 雪国 by 川端康成. }

\par{\textbf{Grammar Note }: せぬ is not しない in this sentence. Rather, 積らせぬ is a negative form of the causative form of 積る. }

\par{17. それにしても、さっき、吉田と向き合った自分は、いかにも惨めに見えたことだろうと思うと、言いしれ \textbf{ぬ }侘しさが心を冷やしていく。 \hfill\break
Even so, as I thought how wretched I, who had been face to face with Yoshida, a while ago, indescribable dreariness cooled my mind off.. \hfill\break
From 冷たい誘惑 by 乃南アサ. }

\par{18. このため石井委員長は、「 ${\overset{\textnormal{ないかく}}{\text{内閣}}}$ が取った ${\overset{\textnormal{そち}}{\text{措置}}}$ は ${\overset{\textnormal{さんぎいんよさんいいんかい}}{\text{参議院予算委員会}}}$ を ${\overset{\textnormal{ぐろう}}{\text{愚弄}}}$ するものであると同時に、 ${\overset{\textnormal{けんぽう}}{\text{憲法}}}$ の ${\overset{\textnormal{せいしん}}{\text{精神}}}$ に ${\overset{\textnormal{かえ}}{\text{反}}}$ しており、 ${\overset{\textnormal{だん}}{\text{断}}}$ じて ${\overset{\textnormal{ようにん}}{\text{容認}}}$ するわけにはいかず、予算委員長として政府に ${\overset{\textnormal{もうせい}}{\text{猛省}}}$ を ${\overset{\textnormal{うなが}}{\text{促}}}$ す」と批判したうえで、委員会を ${\overset{\textnormal{きゅうけい}}{\text{休憩}}}$ にしました。 \hfill\break
As so, Chairman Ishi\textquotesingle i put the committee on recess upon criticizing that “the measures that the cabinet took mock the House of Chancellors Budget Committee and at the same time are against the spirit of the constitution, and being absolutely impossible to approve, as the budget chairman, I urge to the government to reconsider seriously”. \hfill\break
From NHK 区 ${\overset{\textnormal{わ}}{\text{割}}}$ り法案 参院否決と見なす動議提出   2013年6月24日 }
\textbf{}
\par{\textbf{Part of Speech Note }: Set expressions like 土踏まず have gone from being verbal in nature to referring to the none that it represents this. This word means "arch of the foot". }

\par{19. だが、さしのべられる手に応えた \textbf{向こうみず }がいた。 \hfill\break
But, there was a foolhardy person that responded to the hand extended. \hfill\break
From 野生の風 by 村山由佳. }

\par{\textbf{Word Note }: In the sentence above, 向こう見ず is used as a noun, and in other contexts it can act as a 形容動詞. }

\par{20. だから心 \textbf{ならずも }、最後の大事な仕事を片づけることができなかったのだ。 \hfill\break
Against his heart's intentions, however, he couldn't finish the final, important job. \hfill\break
From 海辺のカフカ by 村上春樹. }

\par{\textbf{Phrase Note }: ならずも is a combination なり, the Classical copula, + ず + も, which also follows ず in other set phrases such as 図らずも (unexpectedly). }
 
\par{21. 君はなんといってもただの未成熟な、寸 \textbf{足らず }の幻想にすぎないわけだからね。 \hfill\break
No matter what you may say, it's because you're no more than a lacking, immature illusion. \hfill\break
From 海辺のカフカ by 村上春樹. }

\par{\textbf{Phrase Note }: 寸足らず is a set phrase that functions as a 形容動詞 or attribute that takes の. 寸 is a traditional unit of measurement equivalent to 3.03 cm. ~足らず is a suffix that means "just under", and it gives a strong sense of lacking as the verb 足る in the negative suggests. In this sentence, 寸足らず stresses the inferiority of the addressee. }

\par{22. 村は ${\overset{\textnormal{ちんじゅ}}{\text{鎮守}}}$ の ${\overset{\textnormal{すぎばやし}}{\text{杉林}}}$ の ${\overset{\textnormal{かげ}}{\text{陰}}}$ に半ば隠れているが、自動車で十分足らずの停車場の ${\overset{\textnormal{とうか}}{\text{灯火}}}$ は、寒さのためぴいんぴいんと音を立てて ${\overset{\textnormal{こぼ}}{\text{毀}}}$ れそうに ${\overset{\textnormal{まばた}}{\text{瞬}}}$ いていた。 \hfill\break
The village was half hidden by its Shinto shrine grove, and the lamplight from the train station under ten minutes away by car flickered and made a great noise due to the cold like it was going to give way. \hfill\break
From 雪国 by 川端康成. }

\par{23. 細かい手の ${\overset{\textnormal{きよう}}{\text{器用}}}$ なさばきは耳に覚えていず、ただ音の感情が分かる程度の島村は、 ${\overset{\textnormal{こまこ}}{\text{駒子}}}$ にはちょうどよい聞き手なのであろう。 \hfill\break
Her fine, clever hand movements couldn't be felt, but Shimura, who could only understand the emotion to the sound, was surely just the right listener to Komako. \hfill\break
From 雪国 by 川端康成. }

\par{\textbf{Grammar Note }: Typically, いる + ず is おらず, which utilizes the humble form to prevent the phrase from sounding odd, but the form いず still exists. This use is rather rare, so it is best for you to not use it. However, it does exist. So, don't get confused when it does appear. }

\par{24. 宴会では半玉が太鼓を叩いて踊らねばならぬので、その五六人がこの家へ入らぬ日はなく、時折の滞在の間に、瀧野は彼女等の顔を見覚えてしまっていた。 \hfill\break
Since young geisha's have to beat taiko and dance at banquets, there wasn't a day those five, six people didn't come in this house, and while they stayed occasionally, and Takino ended up remembering their faces. \hfill\break
From 童謡 by 川端康成. }

\par{25. この家へ来なければお茶を引いていると芸者の言うのが、満更お世辞でないほど、町へ来るお客の大半はここ一軒に取られてしまうのだとすると、女中の言葉も全く信じられぬではなかった。 \hfill\break
If you suppose that the majority of the guests that come to the town are taken in this one place to the point that what the geisha said about if she didn't come to this house that she was on break is not completely flattering, one couldn't completely disbelieve the maid's words too. \hfill\break
From 童謡 by 川端康. }

\par{\textbf{Grammar Note }: Notice how the 連体形-ぬ is functioning as a nominal without the aid of の. }

\par{26. わが信ずる運命を ${\overset{\textnormal{たむ}}{\text{手向}}}$ け、死によって ${\overset{\textnormal{かた}}{\text{互}}}$ みの悲劇を理解しようがために、こりずまに剣を交わして戦ったのだ。 \hfill\break
Offering the destiny I believed in, I incorrigibly fought and crossed swords in order to understand the mutual tragedy by death. \hfill\break
From ${\overset{\textnormal{かるのみこ}}{\text{軽王子}}}$ と ${\overset{\textnormal{そとおりひめ}}{\text{衣通姫}}}$ by 三島由紀夫. }

\par{\textbf{Grammar Note }: The ま in こりずま is a suffix used to show a certain condition. }

\par{27. ${\overset{\textnormal{と}}{\text{問}}}$ うは ${\overset{\textnormal{いっ}}{\text{一}}}$ ${\overset{\textnormal{とき}}{\text{時}}}$ の ${\overset{\textnormal{はじ}}{\text{恥}}}$ ${\overset{\textnormal{と}}{\text{問}}}$ わぬは ${\overset{\textnormal{まつ}}{\text{末}}}$ ${\overset{\textnormal{だい}}{\text{代}}}$ の ${\overset{\textnormal{はじ}}{\text{恥}}}$ 。(Proverb) \hfill\break
It is better to ask and be embarrassed than not ask and never know. }

\par{28. 小六が帰りがけに茶の間を覗いたら、御米は何にも \textbf{しずに }、長火鉢に倚り掛かっていた。 \hfill\break
As Koroku was about to go home, he took a glance around the living room, and Oyome was leaning   against an oblong brazier doing nothing. \hfill\break
From 門 by 夏目漱石. }

\par{\textbf{Dialect Note }: We can assume that しずに is an old, dialectical variant of せずに. }

\par{29. きつからず緩からずで休憩をするところを見つけられず、辿った道を振り返った。 \hfill\break
Though (the journey was) neither harsh or smooth, having not been able to find a place to rest, I turned around to where I had come.  }

\par{30. 天を恨みず人を尤めず \hfill\break
To not spite heaven and fault no one; to recover by reflecting on oneself }

\par{\textbf{Conjugation Note }: 恨む・怨む used to be a 上二段 verb. What this means is that the bases of conjugation alternated between i and u. So, the 未然形 would have been うらみ. Because it looked like a 五段 verb, it eventually became one. Thus, in Modern Japanese, 怨まず exists. However, set phrases cannot be changed. }
      
\section{Speech Modals with -ず}
 
\par{\textbf{~ずにはいられない }}

\par{ ~ずにはいられない means "can  not help but\dothyp{}\dothyp{}\dothyp{}" Variants of ~ずにはいられない include ~ずにはすまない, ~ずにはおかない, and ~ずにはすまさない. }

\par{31. 勉強せずに、受験したから、落第しちゃった。 \hfill\break
Because I took the exam without studying, I ended up failing it. }

\par{\textbf{Grammar Note }: You must use the 未然形-せ- of する. }

\par{32. 心配せずにはいられない。 \hfill\break
I couldn't help but worry. }

\par{33. 懲らしめてやらずにはおかない。 \hfill\break
I couldn't help but give him punishment. }

\par{34. 怒らせずにはすまないでしょう。 \hfill\break
You probably can't help but get angry, right? }

\par{35. 泣き出さずにはすまさない。 \hfill\break
I can't help but cry profusely. }

\par{ \textbf{~ずと }}

\par{ ~ずと is a variant of ないでも and is often seen in the set phrase 言わずと知(し)れた. It is classical, so it would only be seen in such set phrases in the spoken language. Otherwise, it would be limited to writing styles and situations suitable for 古語的な書き言葉. }

\par{36. 言わずと知れたことだ。 \hfill\break
It's an obvious thing. }

\par{37. 文句を言わずと話を聞け。 \hfill\break
Listen even if you don't argue. }

\begin{center}
 \textbf{~に過ぎず }
\end{center}

\par{ This is merely a form of ~に過ぎない, which is used to show that something doesn't even pass a certain extent. }
\hfill\break

\par{38. したがって、現代かなづかい論者の一人である吉川幸次郎博士が、日本語は発音をそのままに表記し得ることを大きな特徴とし、かつその表記法の歴史は、この特徴を生かしつつ発展して来たと述べていることを引用 し、発音をそのままに表記し得ることは、とりもなおさず表音式であり、表音文字=表音式であって、現代語に基づくことと表音式との差は五十歩百歩に過ぎず、現代かなづかいが、もともと表音的性格を有している仮名の線を、さらに徹底せしめたものであってみれば、現代かなづかいは表音式だと言い切ってさしつかえないと思うと述べている福田発言は、正しい判断と言ってよい。 \hfill\break
Hence, Professor Kojiro Yoshikawa, one advocate of Modern Kana Orthography, cites that Japanese has had the great characteristic of being transcribed phonetically and the history of its orthography has developed this characteristic and capitalized on it, which is to say that being able to transcribe the language's pronunciation phonetically is a phonetic system (phonetic characters = a phonetic system), and with there being a scant difference between basing it on the Modern Language and a phonetic system , Fukuda's statement which states his belief that "if Modern Kana Orthography tries to be something that furthermore completes the line of Kana which originally possesses a phonetic nature, Modern Japanese Orthography without objection should be called a phonetic system" may very well be a correct judgment. }

\par{From 国語国字の根本問題 By 渡部晋太郎. }

\par{ The main reason for choosing such a long sentence is that it shows how sentences in Japanese can get larger and larger the same way they can in English. Also, there are several uses of ~ず. }

\par{39. あやつは一介の庶民にすぎぬものだ。(ちょっと古風) \hfill\break
He is no more than a mere commoner. }

\begin{center}
 \textbf{~に忍びず }
\end{center}

\par{  ~に忍びず means "cannot stand to". It is synonymous to ~ に耐えられない. ~に忍びない is another more common form. }
 
\par{40. 見るに忍びぬ ${\overset{\textnormal{さんじょう}}{\text{惨状}}}$ だった。 \hfill\break
It was a sight too gruesome to stand seeing. }
 
\par{41. 物を捨てるのは忍びない。 \hfill\break
I can't stand throwing things away. }
 
\par{42. 捨てるには忍ばないと、彼はポケットに ${\overset{\textnormal{い}}{\text{入}}}$ れた。 \hfill\break
Not able to stand throwing it away, he put it in his pocket. }

\begin{center}
 \textbf{~ずじまい }
\end{center}

\par{ This is a particularly odd expression. As you can imagine by looking at it, it is simply expressing a circumstance of ending up not doing something. This is actually occasionally used in the spoken language, and it is often treated as creative language. It is rather unique that voicing occurs with しまい. }

\par{43. 食べてしまおう、言えずじまい。 \hfill\break
I'll eat (them) all, not able to say a word. \hfill\break
From Chocolate Prayer by DIV }

\begin{center}
\textbf{~ならいざ知らず }
\end{center}

\par{ This pattern means that if X is the case, something might be so, but since the circumstances are completely different, so are the results. It often follows words of extremes like 神、大昔、赤ん坊、 ヒマラヤ、etc. What follows is something opposite of it, showing a sense of dissatisfaction or astonishment. }

\par{44. ${\overset{\textnormal{やす}}{\text{安}}}$ いホテルならいざしらず、 ${\overset{\textnormal{いちりゅう}}{\text{一流}}}$ ホテルでこんなにサービスが ${\overset{\textnormal{わる}}{\text{悪}}}$ いなんてとても ${\overset{\textnormal{しん}}{\text{信}}}$ じられないよね。 \hfill\break
I don\textquotesingle t know about cheap hotels, but I can\textquotesingle t even believe how awful this service is from a first-class hotel. }

\par{45. ${\overset{\textnormal{さん}}{\text{3}}}$ ${\overset{\textnormal{さい}}{\text{歳}}}$ の ${\overset{\textnormal{こども}}{\text{子供}}}$ ならいざいしらず、 ${\overset{\textnormal{おとな}}{\text{大人}}}$ がこんなことを ${\overset{\textnormal{し}}{\text{知}}}$ らないなんておかしい。 \hfill\break
I don\textquotesingle t know about  a three year old, but it\textquotesingle s strange that an adult doesn\textquotesingle t know something like this. }

\par{46. ${\overset{\textnormal{しんじん}}{\text{新人}}}$ ならいざ ${\overset{\textnormal{し}}{\text{知}}}$ らず、 ${\overset{\textnormal{きみ}}{\text{君}}}$ があんなことをするなんて。 \hfill\break
I don\textquotesingle t know about newcomers, but you doing something like that… }

\par{47. ${\overset{\textnormal{ちい}}{\text{小}}}$ さい ${\overset{\textnormal{こども}}{\text{子供}}}$ ならいざ ${\overset{\textnormal{し}}{\text{知}}}$ らず、お ${\overset{\textnormal{としごろ}}{\text{年頃}}}$ の ${\overset{\textnormal{こどもたち}}{\text{子供達}}}$ にはつまらないかもしれない。 \hfill\break
I don\textquotesingle t know about small children, but (this) might be boring to older children. }

\par{48. プロ ${\overset{\textnormal{せんしゅ}}{\text{選手}}}$ ならいざ ${\overset{\textnormal{し}}{\text{知}}}$ らず、アマチュア ${\overset{\textnormal{せんしゅ}}{\text{選手}}}$ には ${\overset{\textnormal{むり}}{\text{無理}}}$ でしょう。 \hfill\break
I don\textquotesingle t know about pro-athletes, but it\textquotesingle s probably too much for an amateur athlete. }
    