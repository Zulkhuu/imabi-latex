    
\chapter{擬声語 III}

\begin{center}
\begin{Large}
第158課: 擬声語 III: 擬態語・擬情語 
\end{Large}
\end{center}
 
\par{ 擬態語 represent states and 擬情語 represent emotion(al states). These are intertwined with each other, and they are often tied to some sort of sound. Just like before, different spellings and nuances based on context are to be expected. However, the best thing that you can do to learn Japanese onomatopoeic expressions is see them how they're being used. }
      
\section{擬態語・擬情語}
 
\par{ To begin, we will look at a chart of common 擬態語 and 擬情語. Some notes that we have already seen before will be repeated in context of this lesson as reinforcement of what you already know with new material. }

\begin{ltabulary}{|P|P|P|P|}
\hline 

Tired, exhausted & くたくた(と・に) & Irritated & いらいら(と) \\ \cline{1-4}

Refreshed & すっきり(と) & Fixedly & じっと \\ \cline{1-4}

Round and round & ぐるぐる(と) & Firmly; fixedly \hfill\break
& ぐっと \\ \cline{1-4}

Relieved & ほっと & Nervous; excitedly & わくわく(と) \\ \cline{1-4}

Resolutely; tightly; firmly; steadily \hfill\break
& しっかり(と) & Furiously & ぷんぷん(と) \\ \cline{1-4}

Restlessly & そわそわ(と) & Astonished & びっくり(と) \\ \cline{1-4}

In a mess & めちゃくちゃ & Drenched & びっしょり(と) \\ \cline{1-4}

Glistening & ぴかぴか(と) & Radiantly & きらきら(と) \\ \cline{1-4}

Watery & べちゃべちゃ & Sneeze & はくしょん \\ \cline{1-4}

Rough & ざらざら(と・に) & Stickily & ねばねば(と) \\ \cline{1-4}

Ecstatically; vacantly & うっとり(と) & Dejected & がっかり \\ \cline{1-4}

\end{ltabulary}

\begin{itemize}

\item \textbf{くたくた }: Exhausted, worn down,  boiled to a mush, or even wordy. 
\item \textbf{いらいら(と) }: It may be irritating as in emotions are as in the body. 
\item \textbf{すっくり(と) }:  Being refreshed, clean-cut, straightforward, complete, and even neat. 
\item \textbf{しっかり(と) }: It can also mean reliable and enough. 
\item \textbf{ぷんぷん(と) }: It can also refer to a strong smell in a negative fashion. 
\end{itemize}

\par{\textbf{Part of Speech Note }: Some verbs are based off of onomatopoeia. Ex. きらめく (to sparkle\slash radiate). }

\par{\textbf{Voicing Note }: Voiced onomatopoeia often have a more serious or dramatic tone to them versus their very similar non-voiced counterparts. They are often antonymous. For example, さらさら can be smooth but ざらざら is rough. }

\par{ As you can see, there are some very similar patterns going on. Many onomatopoeic expressions in Japanese are the result of a doubled element(s). We have expressions like こそこそ(と) (stealthily) where double-morae element is doubled. These in particular are subject to having many variants. For instance, you can say こそっと or こっそり instead.  Note that the insertion of the っ is prevented when the resulting double consonant is not one that is allowed in Japanese. }

\par{ Please note that you always have your irregularities. Sometimes different forms have different nuances, although always related. This does not include non-onomatopoeic words with repeating elements. This is really just something you have to mess around with and test the limits of. }
      
\section{Examples}
 
\par{1. めらめらともえている ${\overset{\textnormal{ほのお}}{\text{炎}}}$ \hfill\break
A flaring flame }

\par{2. ちんちん。 \hfill\break
Beg! (To a dog) }

\par{3. だらだらとした ${\overset{\textnormal{とうろん}}{\text{討論}}}$ \hfill\break
A lengthy debate }

\par{4. じとじとした ${\overset{\textnormal{}}{\text{部屋}}}$ \hfill\break
Humid\slash damp room }

\par{5. がっかりした顔 \hfill\break
A dejected face }

\par{6. ${\overset{\textnormal{おんがく}}{\text{音楽}}}$ にうっとりする。 \hfill\break
To be enchanted by music. }

\par{7. はらはらして ${\overset{\textnormal{}}{\text{待}}}$ つ。 \hfill\break
To wait in great suspense. }

\par{8. ばらばらに ${\overset{\textnormal{こわ}}{\text{壊}}}$ す。 \hfill\break
To break into pieces. }

\par{9. でれでれにする。 \hfill\break
To be love-stricken. }

\par{10. にょろにょろと ${\overset{\textnormal{は}}{\text{這}}}$ い ${\overset{\textnormal{まわ}}{\text{回}}}$ る。 \hfill\break
To slither about. }

\par{11. ${\overset{\textnormal{あに}}{\text{兄}}}$ は今日プンプンしてる。(Casual) \hfill\break
My old brother is in a bad mood today. }

\par{12. ぎらぎら ${\overset{\textnormal{}}{\text{光}}}$ る ${\overset{\textnormal{たいよう}}{\text{太陽}}}$ \hfill\break
A glaring sun }

\par{${\overset{\textnormal{}}{\text{13a. 雨}}}$ の ${\overset{\textnormal{}}{\text{中}}}$ をはるばる ${\overset{\textnormal{}}{\text{来}}}$ る。 \hfill\break
${\overset{\textnormal{}}{\text{13b. 雨}}}$ の ${\overset{\textnormal{}}{\text{中}}}$ をわざわざ ${\overset{\textnormal{}}{\text{来}}}$ る。 \hfill\break
Come all the way through the rain. }

\par{\textbf{Sentence Note }: 13a infers that you never stopped on the way, and 13b infers that you took the trouble to come that far. }

\par{14. からからにする。 \hfill\break
To dry up. }

\par{15. ずきずきと ${\overset{\textnormal{いた}}{\text{痛}}}$ む。 \hfill\break
To throb in pain. }

\par{16. ぎっしり ${\overset{\textnormal{つ}}{\text{詰}}}$ まった \hfill\break
Packed; tight; heavy }

\par{${\overset{\textnormal{}}{\text{17. 雨}}}$ でぐっしょりとぬれた。 \hfill\break
I got soaked by the rain. }

\par{18. ぺこりと ${\overset{\textnormal{あたま}}{\text{頭}}}$ を ${\overset{\textnormal{さ}}{\text{下}}}$ げる。 \hfill\break
To bob one's head. }

\par{19. ぼんやりとした ${\overset{\textnormal{ひとかげ}}{\text{人影}}}$ \hfill\break
A vague figure. }

\par{20. がっしりとした ${\overset{\textnormal{}}{\text{男}}}$ \hfill\break
A well-built man }

\par{21. (あなた)の ${\overset{\textnormal{}}{\text{日本語}}}$ の ${\overset{\textnormal{のうりょく}}{\text{能力}}}$ はめきめきと ${\overset{\textnormal{じょうたつ}}{\text{上達}}}$ していますね。 \hfill\break
Your Japanese skills are remarkably improving. }

\par{22. チューインガムが ${\overset{\textnormal{くつ}}{\text{靴}}}$ の ${\overset{\textnormal{そこ}}{\text{底}}}$ にぺたっとくっついた。 \hfill\break
Chewing gum stuck to the bottom of my shoe. }

\par{23. ぐずぐずする。 \hfill\break
To be slow at doing. }

\par{24. まるまるとした ${\overset{\textnormal{よ}}{\text{酔}}}$ っ ${\overset{\textnormal{ぱら}}{\text{払}}}$ い \hfill\break
A plump drunkard }

\par{25. ${\overset{\textnormal{けむり}}{\text{煙}}}$ がもくもくと ${\overset{\textnormal{}}{\text{上}}}$ がる。 \hfill\break
For smoke to rise. }

\par{26. ぱっくりと裂ける。 \hfill\break
To split open. }

\par{27. とげがちくちく(と)する \hfill\break
Thorns are prickly }

\par{28. 背中がぞくぞく(と)する。 \hfill\break
For one's back to chill. }

\par{29. ふつふつ(と)沸く \hfill\break
To boil out. }

\par{30. 彼はまあまあ ${\overset{\textnormal{やさ}}{\text{優}}}$ しい。 \hfill\break
He is relatively nice. }

\par{\textbf{Part of Speech Note }: まあまあ can also be seen as an interjection meaning "now, now" or "my, my". Many adverbial phrases have varying parts of speech depending on usage. }

\par{31. ${\overset{\textnormal{びん}}{\text{瓶}}}$ はすっかり ${\overset{\textnormal{から}}{\text{空}}}$ だ。 \hfill\break
The bottle is quite\slash completely empty. }

\par{32. もうすっかりよくなりましたか。 \hfill\break
Have you become quite well already? \hfill\break
\hfill\break
33. そっと ${\overset{\textnormal{かた}}{\text{肩}}}$ を ${\overset{\textnormal{だ}}{\text{抱}}}$ いた。 \hfill\break
I gently hugged his shoulder. }

\par{\textbf{Reading Note }: 抱く is either read as だく or いだく. The first shows physical embrace. The latter shows the bearing of thoughts, feelings, etc. }

\begin{center}
\textbf{Eating \& Drinking }
\end{center}

\begin{ltabulary}{|P|P|P|P|}
\hline 

To gnaw & がりがり(と)かじる & To eat heartily & もりもり(と)食べる \\ \cline{1-4}

To gulp whole & がぶりと飲み込む \hfill\break
& To bite into & ぱ(っ)くり(と)食べる \\ \cline{1-4}

To bite fiercely & がぶり(と)かむ &  &  \\ \cline{1-4}

\end{ltabulary}

\begin{center}
\textbf{Laughter }
\end{center}

\begin{ltabulary}{|P|P|P|P|}
\hline 

To smile & にっこり(と)笑う & To smile & にこにこ(と)笑う \\ \cline{1-4}

To sneer \hfill\break
& せせら ${\overset{\textnormal{}}{\text{笑}}}$ う & To have a broad grin & にた\{っと・りと・にた(と)\}笑う \\ \cline{1-4}

To smirk & にや\{っと・りと・にや(と)\}笑う &  &  \\ \cline{1-4}

\end{ltabulary}
\hfill\break

\begin{center}
\textbf{Saying } 
\end{center}

\begin{ltabulary}{|P|P|P|P|P|P|}
\hline 

To harp & くだくだ(と)いう \hfill\break
くどくど(と)いう & To nag & がみがみ(と)する & To be fluent & ぺらぺら(と) \\ \cline{1-6}

To murmur & ぶつぶつ(と)いう & To buzz & がやがや & To be outspoken & ぽんぽん(と)いう \\ \cline{1-6}

To chatter & ぺちゃくちゃ(と)しゃべる \hfill\break
べらべら(と)しゃべる & To scold & がんがん(と)いう & To swallow & ぼそぼそ(と)いう \\ \cline{1-6}

To whisper & ひそひそ(と)いう & To grunt & ぶうぶう(と)いう & Noisily & わいわい(と) \\ \cline{1-6}

\end{ltabulary}

\begin{center}
 \textbf{Eating \& Drinking }
\end{center}

\begin{ltabulary}{|P|P|P|P|}
\hline 

To gulp & ごくごく(と)飲む \hfill\break
がぶがぶ(と)飲む \hfill\break
ぐっと飲む & To guzzle & がつがつ(と)食べる \\ \cline{1-4}

To gnaw & がりがり(と)かじる & To eat heartily & もりもり(と)食べる \\ \cline{1-4}

Crunchy & こりこり(と)する & Scraping; hard to the teeth & ごりごり(と) \\ \cline{1-4}

To gobble & ぱくぱく(と)食べる & To suck & ちゅうちゅう(と)吸う \\ \cline{1-4}

To swallow & ごくり(と)飲む \hfill\break
ごくん(と)飲む & To bite into & ぱ(っ)くり(と)食べる \\ \cline{1-4}

To bite fearcly & がぶり(と)かむ & To gulp whole & がぶりと飲み込む \\ \cline{1-4}

\end{ltabulary}
    