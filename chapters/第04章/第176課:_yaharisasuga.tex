    
\chapter{やはり \& さすが}

\begin{center}
\begin{Large}
第176課: やはり \& さすが 
\end{Large}
\end{center}
 
\par{ These two adverbial expressions are referred to as phrases that don't have good English equivalents. Although this is true, the main thing that is brushed aside is how to differentiate between them. Though they are different enough to the point that that shouldn't really be an issue, it's best to make sure that you know when to use them. }
      
\section{やはり}
 
\par{ やはり, also やっぱり, やっぱし, and やっぱ in slang, means "just as one thought", "as of\slash still yet\slash now". It may also suggest a feeling of returning back to one's original idea or motives. }

\par{1. やっぱ(り)、これでは変だよ。 \hfill\break
This here is weird as I thought. }

\par{2. やはり想像した通りの家です。 \hfill\break
This is the house just as I had imagined. }

\par{3. やっぱり殺されちゃった。 \hfill\break
I knew that he was killed. }

\par{4. やっぱしだめ。 \hfill\break
It's bad in the end. }

\par{5. やっぱり来たのね。 \hfill\break
Here you come again. }

\par{6. 「夏休みはどうしたんですか」「日本語を勉強しました」「じゃあ、たくさん勉強できたでしょう」 \hfill\break
「ええ、でも、夏はやっぱり暇な時間がたくさんあると思いました」 \hfill\break
「そうですね。ところで、きのうはどんなレストランへ行ったんですか」 \hfill\break
「メキシコ風料理屋へ行きました」「おいしかったですか」 \hfill\break
「ええ、でも、私は韓国人だから、やっぱり味が韓国料理店のとすごく違うと思いました」 \hfill\break
"How was your summer break?". "I studied Japanese". "Well, so you got to do a lot of studying, right?". "Yes, but, I thought that there was a lot of free time". "True, by the way, what kind of restaurant did you go to yesterday?". "I went to a Mexican food restaurant". "Was it good?". "Yes, but since I'm Korean, I thought that the flavor was quite different from a Korean restaurant". }
      
\section{さすが}
 
\par{ さすが is just one of those words that gives a lot of trouble because it doesn't have an English equivalent. さすが shows something \textbf{\emph{good is as expected }}. It can also be used with the negative to show that although one thought one's expectations would come through, things don't pan out so. This pattern is often used with ~だけあって. }

\par{7. さすがだな。 \hfill\break
Just as expected. }

\par{8. 時は金なりとはさすがによく言えている。 \hfill\break
It is indeed said that time is money. }

\par{\textbf{Grammar Note }: とは quotes a set phrase. In this set phrase, なり is simply a classical copular verb. }

\par{9. さすがは彼女だ。 \hfill\break
It's just like her to. }

\par{10. 田中さんは、プロだけあって、さすがにホームランが打てますよ。 \hfill\break
Since Tanaka is a pro, he can hit home runs as expected. }

\par{11. 有名なレストランだけあって、さすがに予約をしても、いつも客が長い列を作っていて、席に案内されるのに長時間がかかるのです。 \hfill\break
Given that it's a famous restaurant, even when you get a reservation, there is always a line of customers, and it takes a long time to get seated. }

\par{12. 彼は \textbf{さすがは }大統領です。 \hfill\break
He is worthy to be the president. }

\par{13. \textbf{さすがの }僕もそこまでは言えませんよ。 \hfill\break
Even I can't say to that extent. }

\par{\textbf{Variant Note }: さすがの = さしもの. However, the latter is rare and is seen in literature. }

\par{\textbf{Orthography Note }: さすが can be seldom seen written in 漢字 as 流石. }
    