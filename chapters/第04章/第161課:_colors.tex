    
\chapter{Colors}

\begin{center}
\begin{Large}
第161課: Colors 
\end{Large}
\end{center}
 
\par{ Each society has its own customs about colors. Japanese is now based on a seven color (七色) scheme, but Japanese has not always been this way. This will become quite apparent as we look at individual colors. }

\par{ A lot can be said about the nuances of each color, how to be more specific about shades and hues, and what sort of idiomatic expressions can be made with color. By now you should have already learned the basic colors through example sentences thus far. So, this lesson is more about knowing exact details about colors, and it will be expanded over time. }
      
\section{Colors}
 
\par{ There is a noun and an adjectival form to each color. Also, there are additional Sino-Japanese variants for many colors, which can be made adjectival by adding の. }

\begin{ltabulary}{|P|P|P|P|P|P|}
\hline 

Color & Native & Adjective & Non-native & Adjective & Kango \\ \cline{1-6}

Blue & 青(あお) & 青い &  &  & 青色(せいしょく) \\ \cline{1-6}

Red & 赤(あか) & 赤い &  &  & 赤色(せきしょく) \\ \cline{1-6}

Yellow & 黄(き) & 黄色(きいろ)い \hfill\break
黄(色)の &  &  \hfill\break
& 黄色(こうしょく) \\ \cline{1-6}

Green & 緑(みどり) & 緑の &  &  & 緑色(みどりいろ・りょくしょく) \\ \cline{1-6}

Black & 黒(くろ) & 黒い &  &  & 黒色(こくしょく) \\ \cline{1-6}

Brown &  &  &  & 茶色(ちゃいろ) & 茶色(ちゃいろ)い \hfill\break
茶色(ちゃいろ)の \\ \cline{1-6}

White & 白(しろ) & 白い &  &  & 白色(はくしょく) \\ \cline{1-6}

Purple & 紫(むらさき) & 紫の &  &  & 紫色(むらさきいろ) \\ \cline{1-6}

Grey &  &  &  &  & 灰色(はいいろ・かいしょく) \\ \cline{1-6}

Pink &  &  & ピンク & ピンクの &  \\ \cline{1-6}

Orange &  &  & オレンジ & オレンジの &  \\ \cline{1-6}

Gold & 金色(かねいろ) &  &  &  & 金色(きんいろ・こんじき・きんしょく) \\ \cline{1-6}

Silver & 銀色(しろがねいろ) &  &  &  & 銀色(ぎんいろ・ぎんしょく) \\ \cline{1-6}

Beige &  &  & ベージュ & ベージュの &  \\ \cline{1-6}

Tan & 渋色(しぶいろ) &  &  & 渋色の &  \\ \cline{1-6}

Turquoise &  &  & ターコイズ & ターコイズの &  \\ \cline{1-6}

Vermilion &  &  &  &  & 朱色(しゅいろ・しゅしょく) \\ \cline{1-6}

Indigo &  &  & インディゴ \hfill\break
インジゴ \hfill\break
&  & 藍色(あいいろ・らんしょく) \\ \cline{1-6}

\end{ltabulary}

\par{\textbf{Usage Notes }: }

\par{1. All colors of native or foreign origin may be followed by ${\overset{\textnormal{}}{\text{色}}}$ . Certain colors should be used with ${\overset{\textnormal{}}{\text{色}}}$ . For example, ${\overset{\textnormal{}}{\text{灰}}}$ just means "ashes". Gold and silver must have ${\overset{\textnormal{}}{\text{色}}}$ in order to not be confused with the actual elements. Words such as ターコイズ and オレンジ can be understood to mean the color, but they really refer to the objects. Without ${\overset{\textnormal{}}{\text{色}}}$ , ${\overset{\textnormal{}}{\text{緑}}}$ may mean "greenery". ${\overset{\textnormal{}}{\text{紫}}}$ as a noun may refer to soy sauce in sushi restaurants. }

\par{2. The native versions of gold and silver are rare. In fact, the one for gold is never used but in 白銀色, 銅色, and the like. }

\par{3. Most of the Sino-Japanese readings are rare. }

\par{4. ${\overset{\textnormal{}}{\text{青}}}$ , not ${\overset{\textnormal{}}{\text{緑}}}$ , is the color used for streetlights for "green". ${\overset{\textnormal{}}{\text{青}}}$ may also mean "greens" in expressions like ${\overset{\textnormal{あおものいちば}}{\text{青物市場}}}$ meaning "vegetable market". It is also the color for pale face, youth, and freshness in plants, coolness, the sea, and even the color of moonlight and evening mist. It may also refer to black as in a horse's coat. }

\par{5. 茶色 is "brown" instead of "green" because when tea was first introduced to Japan, it would be shipped to elites in hardened, steamed form. From then, it would be cut up and boiled in hot water and drank. As "green tea" variants would come later, the color of the original tea drinks became the Japanese word for brown. }

\par{\textbf{Light and Dark Colors }}

\par{Light colors are expressed by using ${\overset{\textnormal{うす}}{\text{薄}}}$ い or ${\overset{\textnormal{うす}}{\text{薄}}}$ ~. For dark colors, use ${\overset{\textnormal{こ}}{\text{濃}}}$ い. Now, there will be colors that are light or dark variants of a general color. 淡い is "light\slash faint" and its antonym is also 濃い. }

\par{1. 淡い黄色の葉っぱがありますよね。 \hfill\break
There are light yellow flowers, aren't there? }

\par{2. この花の色が薄い。 \hfill\break
This flower's color is pale\slash weak\slash thin. }

\par{3. 高橋さんの車はその濃い緑色のですね。 \hfill\break
Mr. Takahashi's car is that dark green one, right? }

\par{4. ${\overset{\textnormal{うめ}}{\text{梅}}}$ の ${\overset{\textnormal{うすべに}}{\text{薄紅}}}$ \hfill\break
 The light crimson of a plum }

\par{\textbf{Mixed Colors }}

\par{Colors may be put together to make things such as "white-black" and "yellow-green". The resultant expressions may be read with either 訓読み or 音読み. 音読み are rarer and reserved for the spoken language with exception to \#8. }

\par{5. 黄緑(きみどり・おうりょく) \hfill\break
Yellow-green \hfill\break
\hfill\break
6. 黒白(くろしろ・こくはく) \hfill\break
Black and white }

\par{7. 青緑(あおみどり・せいりょく) \hfill\break
Blue-green\slash aqua }

\par{8. 金茶(きんちゃ) \hfill\break
Golden brown }

\begin{center}
\textbf{Interesting Phrases }
\end{center}

\par{${\overset{\textnormal{}}{\text{赤字}}}$ and ${\overset{\textnormal{}}{\text{黒字}}}$ are deficit and surplus respectfully. They are words that typically confuse students, although there are similar expressions in English. For example, you can say "to be in the red". When you get a failing grade, you can say you got an ${\overset{\textnormal{あかてん}}{\text{赤点}}}$ . However, ${\overset{\textnormal{こくてん}}{\text{黒点}}}$ refers to sunspots. So, it is not the opposite of 赤点. Instead, the antonym for 赤点 is either ${\overset{\textnormal{きゅうだいてん}}{\text{及第点}}}$ or ${\overset{\textnormal{ごうかくてん}}{\text{合格点}}}$ . }

\par{ It is also important to note that the phrases 肌が\{白い・黒い\} are NOT used for nationality\slash race. Rather, they refer to someone's complexion. }
    