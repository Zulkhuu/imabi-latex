    
\chapter{Adjective Nominalization II}

\begin{center}
\begin{Large}
第166課: Adjective Nominalization II: ~く 
\end{Large}
\end{center}
 
\par{ The adjectival form ending in く, for a very small amount of adjectives, can be used to create nominalized expressions. }
      
\section{~く \textrightarrow  Noun Phrase}
 
\par{ Words such as 近く, 遠く, 多く, 古く, and 早く are often used as nouns. Such words and phrases involving them are almost always dealing with time and space. Even so, their antonyms are not necessarily applicable with this grammar. For instance, 古く can be used nominally, but 若く cannot. }

\par{ There are three basic things to know about these phrases that have already been mentioned. To be clear, they are: }

\begin{itemize}

\item All of these end in く. 
\item You can view this usage as coming from the deletion of a noun phrase that would be before the adjective in question. 
\item These phrases are usually about time and space. 
\end{itemize}

\par{ These words often take the particles へ, から, に, and まで, but they are almost never used with other particles like が and を. If が and を are to be used, the phrase in question must be fully nominalized. Though there are particles that they can often be used with, this does not mean you can always use them together. }

\par{ Consider the following example. }

\par{1a. 友里が遠くに行ってしまった。 \hfill\break
1b. 友里が遠\{い・くの\}場所に行ってしまった。 \hfill\break
Yuri went far away. }

\par{2. 空高くを目指す。 \hfill\break
To aim up for the skies. }

\par{3. ${\overset{\textnormal{つばめ}}{\text{燕}}}$ が空高くを飛んでいた。 \hfill\break
The swallows were flying up above. }

\par{4. その ${\overset{\textnormal{びん}}{\text{瓶}}}$ が遠くから海に浮かんできた。 \hfill\break
The bottle came and floated (here) from afar on the sea. }

\par{5. 友里が船橋駅の近くに住んでいる。 \hfill\break
Yuri lives near Funabashi Station. }

\par{6. その土地の広くに伝えられた。 \hfill\break
(It) was spread throughout the land. }

\par{7. こんな遅くにどうしたの? \hfill\break
What's wrong this late (at night)? }

\par{8. ついに目的の谷に ${\overset{\textnormal{たど}}{\text{辿}}}$ り着いた。勢いをつけて ${\overset{\textnormal{しょうこひん}}{\text{証拠品}}}$ を深くに投げ捨てた。 \hfill\break
I finally arrived at the valley, my destination. I then threw the evidence deep down the valley. }

\par{9.セスが遅くに出て行った。午前2時を回っていた。 \hfill\break
Seth left when it was late out. It was past 2 A.M. }

\par{ Several words only work well when used right after a noun, creating very commonly used phrases. }

\par{5. 友里が \textbf{朝早くから夜遅くまで }働いていた。 \hfill\break
Yuri was working from early in the morning until late at night. }

\par{6. \textbf{地中奥深くまで }沈み込む。 \hfill\break
To sink deep into the ground. }

\par{7. ${\overset{\textnormal{あんこう}}{\text{鮟鱇}}}$ は海の \textbf{底深くに }生息している。 \hfill\break
Angler fish live deep below on the sea floor. }

\par{8. ${\overset{\textnormal{だんな}}{\text{旦那}}}$ は \textbf{朝早くに }目覚めた。 \hfill\break
My husband woke up early in the morning. }

\par{9.私はほとんど毎日、 \textbf{夜遅くに }帰宅します。 \hfill\break
I come home late in the night almost every day. }

\par{14. 太平洋の東側にフィリピン ${\overset{\textnormal{かいこう}}{\text{海溝}}}$ がある。百合が \textbf{その深くに }${\overset{\textnormal{もぐった}}{\text{潜った}}}$ 。 \hfill\break
The Philippine Trench is on the east side of the Pacific Ocean. There, Yuri dove deep down. }

\par{ Though these words have been described as involving the nominalization of adjectives through the連用形, it is actually more probable that these are merely the 省略形 (abbreviated form) of statements with these adjectives in the 連体形 followed by some time\slash place noun. The evidence for this is the ungrammatically of Ex. 15. This example shows that you cannot fully exploit them as nouns unless like in Ex. 2. }

\par{15a. 月が古くについて語る。 X \hfill\break
15b. ${\overset{\textnormal{むかし}}{\text{昔}}}$ の月について語る。◯ \hfill\break
To tell about the ancient moon. }

\par{16. きょう、近くの大学を見学しました。 \hfill\break
Today, I observed a nearby college. }

\par{ To use these phrases in isolation, there has to be a clear reference to time or space. In Ex. 17, この噂 refers to a time that acts as the referent for 古く. This point of reference is abbreviated out of the phrase, but without such as phrase existing, you get incorrect sentences like 18a. }

\par{17. この ${\overset{\textnormal{うわさ}}{\text{噂}}}$ は古くからある。 \hfill\break
This rumor is ancient. }

\par{18a. 優里が古くを振り返った。X \hfill\break
18b. 優里が過去を振り返った。◯ \hfill\break
Yuri looked back at the past. }

\par{19. その歴史は古くまで ${\overset{\textnormal{さかのぼ}}{\text{遡}}}$ る。 \hfill\break
The history goes way back. }

\begin{center}
\textbf{多く: The Exceptional Word } 
\end{center}

\par{ 多くrefers to quantity and can be used freely like any other noun. It does not follow the rules discussed above. }

\par{20. 市民の多くから信任を得る。 \hfill\break
To receive trust from most of the townspeople. }

\par{21. ${\overset{\textnormal{けつえき}}{\text{血液}}}$ の ${\overset{\textnormal{りゅうしゅつ}}{\text{流出}}}$ を止める機能は動物の多くに ${\overset{\textnormal{そな}}{\text{備}}}$ わっている。 \hfill\break
The function of stopping blood flow is found in a lot of animals. }

\par{22. ${\overset{\textnormal{ひょうざん}}{\text{氷山}}}$ がすべて ${\overset{\textnormal{と}}{\text{溶}}}$ けてしまったら、 ${\overset{\textnormal{なんきょく}}{\text{南極}}}$ ${\overset{\textnormal{たいりく}}{\text{大陸}}}$ に住む動物の多くが ${\overset{\textnormal{ぜつめつ}}{\text{絶滅}}}$ してしまう。 \hfill\break
If all the icebergs were to melt, most of the animals living in Antarctica would go extinct. }

\par{23. インターネットによる ${\overset{\textnormal{しんさつ}}{\text{診察}}}$ ${\overset{\textnormal{よやく}}{\text{予約}}}$ が多くの病院で導入されている。 \hfill\break
Reserving medical examinations online is being introduced to many hospitals. }

\par{24. 実験結果の多くにこの ${\overset{\textnormal{けいこう}}{\text{傾向}}}$ が見られる。 \hfill\break
This trend is seen in most of the experiment results. }

\begin{center}
\textbf{若く \& 浅く: X }
\end{center}

\par{ 若く and 浅く, despite being related to time and space, cannot be used nominally. Yet, their antonyms 古く and 深く can. However, you can still use them in other ways. }

\par{25. ワニに ${\overset{\textnormal{く}}{\text{食}}}$ いちぎられたレイヨウの ${\overset{\textnormal{しがい}}{\text{死骸}}}$ が水中深くまで沈んだ。 \hfill\break
The antelope carcass ripped to shreds by the alligator sunk deep below the water. }

\par{26a. ${\overset{\textnormal{てんぷく}}{\text{転覆}}}$ した船が浅くに浮いている。X \hfill\break
26b. 転覆した船が浅く浮いている。? \hfill\break
26c. 転覆した船が少し沈んでいる。〇 \hfill\break
26d. 転覆した船が沈みかけている。 \hfill\break
The capsized boat is teetering. }

\par{\textbf{Sentence Note }: Imagine a boat capsized and teetering above and below the surface. Usually, 26c or 26d would be used to describe this, but 26b is not out of the question. Unlike the rest, it is vague as to whether the boat is permanently jutting out of the water or is fully sunken directly below the water but not deep down. }

\par{27. ${\overset{\textnormal{するど}}{\text{鋭}}}$ い ${\overset{\textnormal{ぼう}}{\text{棒}}}$ で ${\overset{\textnormal{つ}}{\text{突}}}$ かれて死んだカエルがかなり深いところまで沈んでいる。 \hfill\break
The frog that died from being stabbed with a sharp rod has sunk really deep. }

\par{28. ${\overset{\textnormal{しんかいぎょ}}{\text{深海魚}}}$ は浅いところに住んでいる。 \hfill\break
Deep-sea fish are living in a shallow area. }

\par{29. ${\overset{\textnormal{あんしょう}}{\text{暗礁}}}$ に乗り上げた ${\overset{\textnormal{りょかく}}{\text{旅客}}}$ ${\overset{\textnormal{せん}}{\text{船}}}$ は、10メートルくらいの浅いところに沈んでいる。 \hfill\break
The cruise ship capsized on the coral reefs and is sunken in a shallow spot 10 meters deep. }

\par{30a. 妹の友達は若くに亡くなった。X \hfill\break
30b. 妹の友達は若く亡くなった。◯ \hfill\break
My younger sister's friend died young. }

\par{31. 若いときに亡くなる人が多くて、悲しい。 \hfill\break
It's sad that a lot of people die when they are young. }

\par{32. かなり深くまで潜った。 \hfill\break
I dived quite far down. }

\par{\textbf{Grammar Note }: Using an adverb before these words allows them to stay grammatical as if they were used with another noun before them. However, this cannot be used to expand this pattern to other adjectives. }

\begin{center}
 \textbf{近く \& 遠く }
\end{center}

\par{ The adjectives 近い and 遠い are somewhat irregular because they can essentially always be used with the particles に, へ, から, and まで in the forms 近く and 遠くrespectively. }

\par{33. 流れ着いてよ、どこか遠くへ。あなたの待つ ${\overset{\textnormal{すてき}}{\text{素敵}}}$ な場所へ探しているよ、歌えるような今は音信不通のラブソングを。 \hfill\break
Drift somewhere afar. I search for a song that I might sing, one that is now on silent, as I head to the splendid place you are waiting for me. }

\par{From DIV's ${\overset{\textnormal{ひょうりゅう}}{\text{漂流}}}$ 彼女. }

\par{34. ${\overset{\textnormal{めいおうせい}}{\text{冥王星}}}$ の近くからやってきた ${\overset{\textnormal{うちゅうじん}}{\text{宇宙人}}}$ たち、いったい何を ${\overset{\textnormal{たくら}}{\text{企}}}$ んでいるのか? \hfill\break
What on Earth are the aliens from near Pluto planning? }

\par{35. 遠くにある火山が ${\overset{\textnormal{ふんか}}{\text{噴火}}}$ して、まさか ${\overset{\textnormal{かざん}}{\text{火山}}}$ ${\overset{\textnormal{ばい}}{\text{灰}}}$ がここまで降って来るとは思わなかった。 \hfill\break
I never thought that volcanic ash would fall all the way here since the volcano is so far away. }

\par{36. 銀閣寺の近くにある ${\overset{\textnormal{せんとう}}{\text{銭湯}}}$ に行きたいです。 \hfill\break
I want to go to a public bath near Ginkakuji. }

\par{37. 近くにある公園で桜が満開です。 \hfill\break
The cherry blossoms are in full bloom at the nearby park. }

\begin{center}
 \textbf{Obligatory ~くの }
\end{center}

\par{ For 近い, 遠い, 少ない, and 多い, unless part of an entire phrase modifying a noun, they can't be used to modify a noun alone. With exception to 近い and 遠い which we'll get to later, this holds true. In this situation you must use ~くの. However, for 少ない,  you have to totally rephrase as 少なくの doesn't exist. 少しの exists, though. }

\begin{ltabulary}{|P|P|P|P|}
\hline 

近い学校 X & 遠い大学 X & 少ない人 X & 多い人 X \\ \cline{1-4}

\end{ltabulary}

\par{38. いちばん近いところ \hfill\break
The closest place }

\par{39. 近いうちに   (Set Phrase) \hfill\break
In the near future }

\par{\textbf{Grammar Note }: You can actually say 近く to mean 近いうちに. The form 近々 also exists. This just goes to show you what might be done to an adjective on an individual basis. }

\par{40. 駅に近い。 \hfill\break
It's close to the train station. }

\par{41. 駅に近いマンションに住む。 \hfill\break
To live in an apartment close to the train station. }

\par{42. 近くの学校 \hfill\break
A school nearby }

\par{43. 遠くの大学 \hfill\break
A college far away }

\par{44. 日本は ${\overset{\textnormal{じしん}}{\text{地震}}}$ が多い。 \hfill\break
There are a lot of earthquakes in Japan. }

\par{45. 人が少ない。 \hfill\break
There are few people. }

\par{46. 少ない ${\overset{\textnormal{にんずう}}{\text{人数}}}$  \hfill\break
Small amount of people }

\par{\textbf{Phrase Note }: This phrase is alright because 人数, unlike 人, is a quantity noun. Nevertheless, it's shown to show one way of how to overcome the ungrammaticality of 少ない人. }

\par{47a. 東京には少しの緑がある。 \hfill\break
47b. 東京には緑が少しある。 }

\par{ It seems that 近い and 遠い actually fall out of the problem if a modifier is implied in context. Although not appropriate for writing, using these two without restriction appears to be a feature of the spoken language among younger people. }
 
\par{48. 遠い学校に通いたくない。 \hfill\break
I don't want to go to a school that's far away. }
    