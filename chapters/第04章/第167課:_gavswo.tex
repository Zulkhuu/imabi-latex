    
\chapter{が VS を}

\begin{center}
\begin{Large}
第167課: が VS を 
\end{Large}
\end{center}
 
\par{ The difference between が and を does not stop at 私が魚を食べる kind of sentences. Though が is called the subject marker and を is called the direct object marker, there are times when they are indeed interchangeable. As to be expected, there are restrictions on when they are interchangeable. }

\par{ When we learned about the potential form, we saw how が and を can be interchangeable. For instance, you can say 日本語が話せる or 日本語を話せる for "can speak Japanese." The use of を with the potential is highly tied to personal volition. The more volition the agent has, the more likely を will be used and the less likely が is used. }

\par{1. 美恵子は自然に美しい歌詞\{が 〇・を ???\}書けた。 \hfill\break
Mieko was naturally able to write beautiful lyrics. }

\par{2. ホームラン\{が 〇・を ???\}打てた。 \hfill\break
I was able to hit a home run. }

\par{3. この漢字\{が・を\}書ける人はあまりいないでしょうね。 \hfill\break
There probably isn't a lot of people who can write this Kanji.  }

\par{ Aside from the potential form, there is interchangeability with ~たい and phrases of like and dislike (好き and 嫌い). To see if the same concept of control is at work, consider the following. }

\par{4. 私は自分のこと\{が・を\}好きになった。 \hfill\break
I've gotten to like myself. }

\par{5. 私はポケットモンスター新作\{が・を\}買いたいです。 \hfill\break
I want to buy the latest Pokemon game. }

\par{ ~を分かる has traditionally been incorrect, but it first started to appear in the late 1800s and is here to stay. The trigger for why を is being acceptable is a change in perceptible control implied by 分かる. }

\par{6.  誰もあたしの気持ちを分かるはずなんてない! \hfill\break
There's no way anyone understands my feelings! }
      
\section{In-Depth Analysis on が VS を}
 
\par{\textbf{助詞の選択 }}

\par{ Japanese grammar generally frowns upon the same particle showing up more than once in a single clause. Consequently, counterexamples involve very specific grammatical structures that trump regular judgment calls. }

\par{  I f a subject that would otherwise take が is used in a potential sentence with an object marked with が, we get XがYがPotential Verb. Of course, XはYがPotential Verb exists and is most common, but the existence of the former sentence type requires explanation. In reality, we must consider five different sentence types. The least common is V, and we'll give it an ? for being questionable. }

\begin{enumerate}

\item 友里が英語を話す。 
\item 友里が英語が話せる。 X* 
\item 友里が英語を話せる。 
\item 友里に英語が話せる。 
\item 友里に英語を話せる。(X) 
\end{enumerate}

\par{*: These sentence patterns should be thought of as underlining forms that then may change when spoken. So, in reality, as you can see later below, II is not used as is but is when the subject is marked by は. }

\par{ Here is a chart that shows the percentages of what the subject is marked with depending on whether the object is marked with either が or を. }

\begin{enumerate}

\item 
\begin{ltabulary}{|P|P|P|}
\hline 

 & Object+が & Object+を \\ \cline{1-3}

Subject & が  は   に   には    にも & が   は   に  にも  には \\ \cline{1-3}

 & 0\%  25\%     8\%     51\%     16\% & 50\%      49\%     0\%       1\%     0\% \\ \cline{1-3}

\end{ltabulary}

\end{enumerate}

\par{\textbf{Chart Note }: The subject does not like to have the same particle as the object. には and にも, which indirectly refer to the subject in terms of spontaneous action, should not be used when the object is marked with を as を highlights volition. Spontaneity is the opposite of volition. So, you'd be adding two grammatical opposites together.  }

\par{ There is no doubt that Type III is becoming more pervasive than Type II (remember that Type II essentially surfaces with は on the subject instead), but that doesn't help explain the difference between them at all. We will need to investigate what sort of other grammatical triggers aid in the decision. First, though, let's get back to the somewhat questionable Type V. }

\par{7a. 俺は涙を流せない。 \hfill\break
7b. 俺には涙を流せない。△・X \hfill\break
I can't shed a tear. }

\par{ In context, questionable grammar can be made natural. With this in mind, consider the following }

\par{8. 人間には涙が流せるし、自分は涙を流せないが、涙を流す理由は理解できた。 1 \hfill\break
Man can cry, and so though I myself cannot shed tears, I have now understood the reason for crying. }

\par{ Ex. 8 shows us several of the sentence types above in one, which are both frequently used in negative structures, but we do not see Pattern V. This gives us more proof to say that it is ungrammatical. The first part of this sentence lacks volition. The ability to cry is described initially as an innate characteristic of people that may in essence occur spontaneously. Thus, 涙を would be ungrammatical for this meaning. Because には calls for this sort of nuance, を becomes inappropriate. }

\par{ \textbf{${\overset{\textnormal{じょうたいてき}}{\text{状態的}}}$ }\textbf{動詞 }\textbf{\slash  }\textbf{State Verbs }}

\par{ One way to start thinking about が versus を is whether the verb is in regards to a state. So, we would expect transitive verbs like 理解する and 期待する to favor or mandate the use of が. If not, then が or を, but the internalized relationship between the parts of the sentence would be different. }

\par{9a. [友里は英語を話]せる。 \hfill\break
9b. [友里は英語が][話せる]。 \hfill\break
Yuri can speak English. }

\par{10a. [僕はパンを食べ]たい。 \hfill\break
10b. [僕はパンが][食べたい]。 \hfill\break
I want to eat bread. }

\par{11 . 日本語\{が・を\}話したいです。 \hfill\break
I want to speak Japanese. }

\par{12. 日本語\{が・を\}話せますか。 \hfill\break
Can you speak Japanese? }

\par{13. その文法点\{が・を\}説明できません。 \hfill\break
I can't explain that grammar point. }

\par{14. ${\overset{\textnormal{きたちょうせん}}{\text{北朝鮮}}}$ はいつでもミサイルを ${\overset{\textnormal{はっしゃ}}{\text{発射}}}$ できる状態 だ。 \hfill\break
North Korea is in the state of being able to fire a missile at any time. }

\par{ We will see again this concept of whether the ending is modifying the verb only or the entire phrase itself. Also, you may be wondering why を is used in Ex. 14 even though the subject is clearly in the state of being able to do the action. That is because of the next factor: control. }

\par{ \textbf{${\overset{\textnormal{さいだいげん}}{\text{最大限}}}$ }\textbf{のコントロール }\textbf{\slash  }\textbf{Ultimate Control } }

\par{  The next example is quite intriguing. Using が with 自分 and the like in this sort of grammar is avoided, and wrong at the worst. But, if we replace 自分 with other people nouns, the grammar doesn't change. が is highly related to spontaneity (things happening naturally), and that does help us with examples like Ex.1, but it doesn't help us here. The concept to introduce here is control. The use of を is determined by whether the speaker is able to control the action or want being expressed. }

\par{15. あいつは\{自分・他人\}を ${\overset{\textnormal{いつわ}}{\text{偽}}}$ れる男 だ。 \hfill\break
That guy is a man can deceive himself\slash others. }

\par{ This sentence demonstrates ultimate control. The subject is able to deceive so much that he himself can be victim to his own deceit. So, there are at least two restrictions to keep in mind: spontaneity and control. The former makes が obligatory, and the latter makes を obligatory. }

\par{ \textbf{~ようになる \& ~ようにする }}

\par{ Consider ~ようになる and ~ようにする. The former is intransitive and the latter is transitive. Thus, you do not see interchangeability with が and を. [] will be used to show how to view the main argument to focus on in these sentences. The content of the brackets could be replaced with other phrases, but the grammar would still be the same. }

\par{16. 私は毎日やっているうちに[自然に漢字が書けるよう]になりました。 \hfill\break
I became able to naturally write Kanji while I was studying every day. \hfill\break
\hfill\break
17. 4年生が終わるまでに常用漢字を[書けるようにします]。 \hfill\break
I will have (the students) able to write the Joyo Kanji by the time the fourth year students end. }

\par{18. 乃理子は独学でハングル\{が 〇・を?\}書けるようになった。 \hfill\break
Noriko became able to write Hangul through self-study. }

\par{19. 乃理子は独学でハングル\{が ?・を 〇\}を読めるようにした。 \hfill\break
Noriko had herself able to read Hangul through self-study. }

\par{ する has a high degree of 他動詞性 even when it is in a potential phrase. This is further seen when we use する with a potential phrase with the addition of imperative or imperative-like structures such as ~なければならない or the 命令形. }

\par{20. 運転中は、絶対にスマホ\{が ?・を 〇\}使えないようにしなくてはいけない。 \hfill\break
You must make it that you can't use your smartphone ever while you drive. }

\par{21. 引っ越しの前に、 ${\overset{\textnormal{ふよう}}{\text{不要}}}$ なもの \{が ?・を 〇\}捨てられるようにしておけ。 \hfill\break
Before moving, have it that you are able to throw away unneeded things. }

\par{  \textbf{動作主性 = Agency }}

\par{ Another way to look at this issue is agency. Control is directly tied to a sense of agency. Whenever there is a perfective aspect and\slash or emphasis on result, が becomes inappropriate. Anything related to completion is really perfect for を but not が.  }

\par{22. 中国に長く住んでいて、英語を話す機会がほとんどなくなったので、英語\{が 〇・を ??\}話せなくなった。 \hfill\break
I lived in China for a long time, and because I basically lost my chances to speak English, I became unable to speak English. }

\par{23. いつでも ${\overset{\textnormal{のら}}{\text{野良}}}$ ${\overset{\textnormal{ねこ}}{\text{猫}}}$ \{を 〇・が ?\}殺せるわけではない。 \hfill\break
You can't just be able to kill the stray cats whenever. }

\par{24. 思ったより早くエッセイを書けてよかった。 \hfill\break
I'm so glad I was able to write the essay earlier than I thought. }

\par{25. 冬の間はあの ${\overset{\textnormal{どうくつ}}{\text{洞窟}}}$ にある ${\overset{\textnormal{つらら}}{\text{氷柱}}}$ を ${\overset{\textnormal{と}}{\text{溶}}}$ かせない 。 \hfill\break
You can't melt the icicles in that cave during the winter. }

\par{26. ${\overset{\textnormal{りょう}}{\text{量}}}$ が多すぎて、ビール\{を 〇・が ?\}\{飲み ${\overset{\textnormal{ほ}}{\text{干}}}$ せなかった・飲み切れなかった\}。  \hfill\break
I was unable to douse down\slash completely drink\} all of the beer because there was just too much. }

\par{27. 山口さんがフランス語\{を 〇・が??\}話せるように、私はフランス人も招待しました。 \hfill\break
To get Yamaguchi-san able to speak French, I also invited French people. }

\par{\textbf{Sentence Note }: Another reason why が wouldn't be used in Ex. 27 is the doubling of が in the same clause. }

\par{28. 神経科学試験の前に韓国語の宿題\{を 〇・が X\}してしまいたい。 \hfill\break
I want to get my Korean homework over before my neurology exam. }

\par{29. 子供たちが来る前に、テーブル\{を 〇・が X\}片づけておきたい。 \hfill\break
I want to have the table cleared off before the children come. }

\par{\textbf{Grammar Note }: The interchangeability of が and を also applies to ~たい. Here, the argument that control is the deciding factor seems really strong. }

\par{ \textbf{When the Potential Form and Intransitive Form Look the Same }}

\par{ There are instances in which the potential form of a transitive form looks just like the intransitive form. Examples of this situation include 焼ける and 割れる. So, sentences such as パンが焼ける are naturally ambiguous. The agent may or may not be implied. In this case, the meaning difference is minor. Meaning A would be for "the bread to bake" and Meaning B would be "to be able to bake bread". The solution to distinguish them would be to say パンを焼ける instead. Though one would think avoiding ambiguity would be a good thing, not all speakers like ~を焼ける because both the default intransitive meaning and the potential meaning are intransitive. }

\par{ \textbf{自発性 Betrays Us at Times }}

\par{ が is the particle for spontaneity(自発性) . It is this sense of lack of control in something spontaneous that makes が the better choice, but for verbs that typically imply a sense of control, the particle を may be seen used nonetheless due to it being the norm rather than being a reflection of the grammatical constraints of the context in question. }

\par{30. セスは日本語の ${\overset{\textnormal{てんさい}}{\text{天才}}}$ だから 、新しいレッスン\{が・を ?\}次から次にとめどなく作れた。 \hfill\break
Because Seth is a Japanese genius, he was able create new lessons nonstop one after the other. }

\par{31. 俺はお前\{が 〇・を 〇・?\}好きで好きでたまらねーんだよ。 \hfill\break
I like you so, so much I can't stand it. }

\par{32. 僕はきのうから豚骨スープ\{が・を ?\}食べたくて、食べたくて仕方なかったので、 あの ${\overset{\textnormal{やたい}}{\text{屋台}}}$ に行ってみた。 \hfill\break
I wanted to eat tonkotsu soup so bad since yesterday, and so I went to that stand. }

\par{ Sure, you will have people say that ? is in fact X to them. After all, this is a grey zone. It all has to deal with how the speaker internalizes this concept of spontaneity versus control and how this relates to potential (可能)and desire (願望). We can see how either makes sense for a lot of verbs. In the second example here, the fact that the sentence is overall more colloquial is another factor for why を may appear. The reason why が is always right in this sort of context is because of its role as the spontaneity particle. }

\par{ \textbf{Control ≈ Will } }

\par{ If you have control over something, you usually willfully exercise this control. Because of this, we will likely never see ~を聞こえる or ~を見える because not only are we dealing with completely spontaneous actions, there is no way will (有意志性) or motor action will ever be expressed with them. Spontaneous verbs dealing with the senses are completely contrary in meaning to 他動詞. Remember, grey zones like above are where the semantic domains of two things overlap. Here, we are looking at an extremity. が would have to completely disappear for ~を聞こえる or ~を見える to ever be acceptable. This is more evidence that these verbs are indeed not potential verbs despite what textbooks often claim. }

\par{ This explanation also explains why ~をできる is not acceptable. Though, ~を+する Verb can be put into the potential with the same restrictions on が・を交替, the independent verb 出来る must be treated differently. }

\par{33. 知也は英語\{が 〇・を X\}出来る。 \hfill\break
Tomoya can speak English. }

\par{34. 知也は英語\{が 〇・を X\}出来るようになった。 \hfill\break
Tomoya became able to speak English. }

\par{35. 知也は英語\{が 〇・を X・???\}出来るようになりたいと思っている。 \hfill\break
Tomoya wants to become able to speak English. }

\par{36. 知也は英語\{が ??・を 〇\}出来るようにしたいと思っている。 \hfill\break
Tomoya wants to have himself able to speak English. }

\par{ We see again how ~ようになる and ~ようにする influence the decision and change the organization of the parts of the sentence. We expect the same things if we use ~ておく. }

\par{37. これまでに出来なかった問題をできるようにしておくことを願います。 \hfill\break
We ask that you prepare yourself to be able to answer the problems that you haven't been able to up till now. }

\par{38. 休み時間になったら 、 ${\overset{\textnormal{うわさばなし}}{\text{噂話}}}$ \{が・を\}できる ようにしておいた。 \hfill\break
We had that we cold gossip once we were in free time. }

\par{39. 学校で使っている問題集の問題が出来るようにしておくこと。 \hfill\break
Have it that you can do the problems in the problem set used at school. }

\par{40. お金の ${\overset{\textnormal{かんり}}{\text{管理}}}$ をできるようにしておくことが、自分の生活を守る上 でも大切になってくる。 2 \hfill\break
Being able to manage one's money before will become even more important than protecting one's way of life. }

\par{41. みだりに青年が ${\overset{\textnormal{うめたてち}}{\text{埋立地}}}$ に立ち入るのを ${\overset{\textnormal{ぼうし}}{\text{防止}}}$ すること ができるようにしておくこと。 \hfill\break
Have it that you can prevent youths from trespassing recklessly into the land reclamation site. }

\par{ We see that ~することができる will never become ~することできる even when you add ~ておく. Aside from this, there is a lot of variation between  が and を. Due to the independent nature of the verb 出来る, we can say that が出来る would still be more common and grammatically safer overall even with the addition of ~ておく. }

\par{ If using ~ておく worked to get を used before 出来る, then this should work for verbs like 見える. The use of ~ようにする or an imperative phrase should also influence this. }

\par{42. ${\overset{\textnormal{つうがく}}{\text{通学}}}$ ${\overset{\textnormal{ろ}}{\text{路}}}$ の近くでは、どこからでも ${\overset{\textnormal{こうつう}}{\text{交通}}}$ ${\overset{\textnormal{ひょうしき}}{\text{標識}}}$ \{が・を\}見えるようにしておきなさい。 \hfill\break
Please make sure that the traffic signs are visible from any direction near the school zone. }

\par{ \textbf{分かる }}

\par{ The use of ~を分かる is here to stay, though it has traditionally been incorrect. Again, the idea that it is English's fault is not plausible. What is certain is that this verb has come to mean 理解する and has taken on the same grammatical rules as it in colloquial\slash emphatic speech. }

\par{ Can we, though, find a Japanese route to the emergence of ~を分かる through this discussion? Yes, think about ~を分かろうとする. It would be harder to find Japanese speakers who don't like this example because the volitional pattern ~(よ)うとする is added, which adds the highly transitive する and 'control' related grammar. Similar grammar, then, should also make を is easier to use or at most obligatory. }

\par{43. 俺の気持ち\{を 〇・が X\}分かってくれ! \hfill\break
Understand my feelings! }

\par{参照: https:\slash \slash www.jpf.go.jp\slash j\slash japanese\slash survey\slash globe\slash 18\slash 08.pdf }
    