    
\chapter{Adverbs III}

\begin{center}
\begin{Large}
第154課: Adverbs III: Syntax Agreement 
\end{Large}
\end{center}
 
\par{ Syntax agreement simply describes adverbs that have specific meanings when used in a positive or negative sentence, and the adverb may specifically require being in a negative sentence. }
      
\section{Positive \& Negative}
 
\par{ Some adverbs must be used in a negative sentence. Others can be in either positive or negative sentences, but translations change. This can get quite tricky. }

\begin{ltabulary}{|P|P|P|}
\hline 

Adverb & Positive & Negative \\ \cline{1-3}

全然 & Extremely\slash a lot (Colloquial) & Not at all \\ \cline{1-3}

絶対に & Absolutely & Never \\ \cline{1-3}

あまり & Quite\slash too (あまりに Only) & Not quite\slash very (あまり Only) \\ \cline{1-3}

とても & Very & Simply cannot \\ \cline{1-3}

決して &  & Never \\ \cline{1-3}

もはや & Already & No more \\ \cline{1-3}

[すこし・ちっと]も &  & Not a bit \\ \cline{1-3}

\end{ltabulary}

\par{\textbf{Point 1 }: Examples of the colloquial usage of 全然 include 全然 ${\overset{\textnormal{だいじょうぶ}}{\text{大丈夫}}}$ (completely fine). }

\par{\textbf{Point 2 }: Examples of ${\overset{\textnormal{ぜったい}}{\text{絶対}}}$ (に) include the following. }

\par{${\overset{\textnormal{}}{\text{1. 絶対}}}$ に ${\overset{\textnormal{ゆる}}{\text{許}}}$ さない。 \hfill\break
I will never allow\slash forgive. }

\par{2. 絶対に ${\overset{\textnormal{ちが}}{\text{違}}}$ う! \hfill\break
Absolutely not! }

\par{\textbf{Point 3 }: あまり is more common in negative contexts. あんまり is a colloquial variant due to ん insertion. In positive contexts, it implies that a limit has been passed, making it similar to 非常に (very\slash greatly\slash much\slash quite). }

\par{3. あまりうまくない。 \hfill\break
I'm not really good. }

\par{4. あんまり分かんない。(Colloquial; 東京弁) \hfill\break
I don't quite understand. }

\par{5. あんまり ${\overset{\textnormal{うんどう}}{\text{運動}}}$ しません。(More spoken) \hfill\break
I don't exercise much. }

\par{\textbf{Point 4 }: With negative expressions とても means "simply cannot". とっても is a more forceful variant. }

\par{6. とても真似できない。 \hfill\break
I simply cannot mimic. }

\par{7. とても疲れた。 \hfill\break
I'm very tired. }

\par{8. 中国語はとっても難しい! \hfill\break
Chinese is very difficult! }

\par{\textbf{Point 5 }: ${\overset{\textnormal{けっ}}{\text{決}}}$ して may be casually pronounced as けして. }

\begin{center}
\textbf{More Examples }
\end{center}

\par{${\overset{\textnormal{}}{\text{9. 全然分}}}$ からない。 \hfill\break
I don't understand at all. }

\par{10. このドアは ${\overset{\textnormal{し}}{\text{閉}}}$ まらないよ。 \hfill\break
This door won't shut. }

\par{11. わたしは一切テレビを見ません。 \hfill\break
I don't watch television at all. }

\par{12. その日はちょっと\dothyp{}\dothyp{}\dothyp{} \hfill\break
That day is a little\dothyp{}\dothyp{}\dothyp{} }

\par{\textbf{Culture Note }: Japanese is indirect and so are the people that speak it. When people want to decline an invitation, they often say \dothyp{}\dothyp{}\dothyp{}はちょっと with a very reluctant tone. }

\begin{center}
 \textbf{まだ VS }\textbf{全然 VS 全く }
\end{center}

\par{In a negative sentence, まだ means "yet\slash still hasn't." 全然  and 全く both mean "not at all," and they are both not viewed as synonyms of まだ. }

\par{13. まだ雨が降っています。 \hfill\break
It's still raining. }

\par{14. 「もう書きましたか」「いいえ、まだ書いていません」 \hfill\break
"Have you written it?" "No, I haven't written it yet". }

\par{ Sleeping is difficult sometimes. We might tell our friends we didn't sleep at all last night even though we actually slept a little. Or, we may have a hard time falling asleep and try talking to someone in the meantime. In that situation, though, have you actually dozed off and failed to truly fall asleep, or have you been completely sleepless? With all of this in mind, we'll now learn how to express these situations in Japanese.  }

\par{15a. まだ ${\overset{\textnormal{ね}}{\text{寝}}}$ ていません。 \hfill\break
15b. まだぜんぜん寝ていません。 \hfill\break
15a. I still haven't slept (at all). \hfill\break
15b. I still haven't slept any. (Have slept but not enough) }

\par{16. まだ寝ない。            VS   ぜんぜん寝ない。 \hfill\break
I still won't sleep.  I won't sleep at all. }

\par{17. もう朝なの?まだ全然寝てない。 \hfill\break
It's already morning? But I still haven't slept much at all. }

\par{18. きのうは全く寝なかった。  (You didn't sleep for even a minute) }

\par{19. きのうは全然寝なかった。   (You slept a little) }

\par{ きのうは全然寝(ら)れなかった means "I couldn't sleep at all", but it sounds like you might have slept some. You might find yourself in a conversation like the following. }

\par{20. 「はぁー、きのうは全然寝(ら)れなかったよ」「本当に ${\overset{\textnormal{いっすい}}{\text{一睡}}}$ も してないの?」「いやー、寝たには寝たけど30分おきに起きちゃってさ」 \hfill\break
"Haa, I didn't sleep at all last night" "Really? You didn't sleep a bit?" "Well, I did sleep if that's what you mean, but I would wake up every thirty minutes" }

\begin{center}
 \textbf{Misconceptions on 全然 }
\end{center}

\par{ 全然 was borrowed from Chinese about three centuries ago. At the time, it roughly equated to "completely" with both positive and negative sentences. Getting closer to modern times, its meaning narrowed to only be used in negative sentences. Now, the word has changed again in casual language to mean とても. For example, you'll hear things like ぜんぜんおいしい and ぜんぜん大丈夫. The former, though, may sometimes have the nuance of "not thinking it would be delicious but turns out it is quite alright." }
    