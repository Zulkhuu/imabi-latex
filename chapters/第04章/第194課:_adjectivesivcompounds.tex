    
\chapter{Adjectives IV}

\begin{center}
\begin{Large}
第194課: Adjectives IV: Stems of Adjectives in Compounds 
\end{Large}
\end{center}
 
\par{ A very constructive means of using adjectives happens to be using the stems of adjectives in compounds. Instead of saying something like \emph{furui shimbun }古い新聞 to mean “old newspaper,” you can simply say \emph{furushimbun }古新聞. }

\par{ Before you go overboard and drop every \slash i\slash  each time you use an adjective, it\textquotesingle s important to note that all examples of this must be treated as set phrases. This is because you can\textquotesingle t assume that dropping \slash i\slash  results in a valid phrase and because you cannot be 100\% sure that nuance won\textquotesingle t change even if the resultant phrase is valid. For example, both of the following are possible, but they differ significantly in nuance. }
 
\par{\emph{Furudokei }古時計: Antique clock \hfill\break
 \emph{Furui tokei }古い時計: An old watch\slash clock }
 
\par{ The difference, as you can see, is far from subtle. Here's another instance of nuance being quite different: }
 
\par{\emph{Furuhon\textquotesingle ya }古本屋: A used book store \hfill\break
 \emph{Furui hon\textquotesingle ya }古い本屋: An old bookstore }
 
\par{ Yet again, you could unintentionally drastically change what you intend to say if you\textquotesingle re not careful. }
      
\section{Vocabulary List}
 Nouns ・新聞 Shimbun – Newspaper ・時計 Tokei – Clock\slash watch ・本屋 Hon\textquotesingle ya – Bookstore ・古新聞 Furushimbun – Old newspaper ・古時計 Furudokei – Antique clock ・古本屋 Furuhon\textquotesingle ya – Used book store ・考え方 Kangaekata – Mindset\slash way of thinking ・地区 Chiku – Sector\slash area ・お客さん Okyaku-san – Customer ・噂話 Uwasabanashi – Gossip ・前 Mae – In front ・店員 Ten\textquotesingle in – Store employee ・古着屋 Furugiya – Old clothing store ・売り場 Uriba – Sales floor ・過去  Kako – Past ・古傷 Furukizu – Old wounds ・故郷 Kokyō\slash furusato – Hometown ・時 Toki – Time\slash when ・古株 Furukabu – Old timer ・古雑誌 Furuzasshi – Old magazine(s) ・電車 Densha – Train ・気分 Kibun – Feeling\slash mood ・嬉し涙 Ureshinamida – Tears of joy ・弱虫 Yowamushi – Coward\slash weakling ・火 Hi – Fire ・弱火 Yowabi – Low flame ・とろ火 Torobi – Simmer flame ・こと Koto – Thing\slash incident\slash situation ・外交 Gaikō – Diplomacy ・弱音 Yowane – Feeble complaint ・赤字 Akaji – Deficit ・黒字 Kuroji – Surplus ・生活 Seikatsu – Life\slash livelihood ・倒産 Tōsan – Bankruptcy ・確率 Kakuritsu – Probability ・業種Gyōshu – Industry ・玄関前 Genkan-mae – In front of entryway ・赤身魚 Akamizakami – Red meat fish ・低温 Teion – Low temperature ・低時間 Teijikan – Short amount of time ・卵 Tamago – Egg ・白身 Shiromi – White meat\slash white of an egg ・黄身 Kimi – Yolk ・スーパー Sūpā – Supermarket ・八百屋 Yaoya – Greengrocer ・市場 Ichiba – Market ・高値 Takane – High price ・安値 Yasune – Low price ・本 Hon – Book ・弁護士 Bengoshi – Lawyer ・態度 Taido – Atittude ・婚活 Konkatsu – Marriage hunting ・シロクマ(白熊)Shirokuma – Polar bear ・薄着 Usugi – Light clothing ・厚着 Atsugi – Heavy clothing ・餌 Esa – Food (animal\textquotesingle s) ・量 Ryō – Amount ・男性 Dansei – Man\slash male ・女性  Josei – Woman\slash female ・厚化粧 Atsugeshō – Heavy makeup Pronouns ・俺 Ore – I (rough male speech) ・自分 Jibun – Oneself Proper Nouns ・日本 Nihon – Japan ・共産党 Kyosantō – Communist Party Adjectives ・古い Furui – Old ・古臭い Furukusai – Old-fashioned ・若い Wakai – Young ・低い Hikui – Low ・無い Nai – Not being ・辛い Tsurai – Bitter\slash tough\slash painful ・暑い Atsui – Hot (weather) Adjectival Nouns ・田舎\{の\}Inaka [no] – Rural ・他\{の\}Ta\slash hoka [no] – Other ・嫌\{な\}Iya [na] – Unpleasant ・地球上\{の\}Chikyūjō [no] – On Earth ・最強\{の\}Saikyō – Strongest ・弱腰\{な・の\}Yowagoshi [na\slash no] – Weak-kneed ・隣人\{の\}Rinjin [no] – Neighbor ・可能\{な\}Kanō [na] – Possible ・高飛車\{な\}Takabisha [na] – High-handed ・野生\{の\}Yasei [no] – Wild ・多く\{の\}Ōku – A lot\slash many Adnominal Adjectives ・ある Aru – A certain Question Words ・どうして Dōshite – Why ・何故 Naze – Why Prefixes ・最~ Sai- - -est Suffixes ・~達 -tachi – Plural marker ・~(Adj. stem +)そうだ -sō da – To seem ・~にくい -nikui – Difficult to Adverbs ・毎月 Maitsuki – Every month ・将来 Shōrai – In the future ・とっても Tottemo – Very ・必ずしも Kanarazu shimo – Not always ・ここ数日 Koko sūjitsu – The past few days ・限り Kagiri – As possible ・うちに Uchi ni – While ・ほとんど Hotondo – Hardly ・なかなか Nakanaka – Considerably\slash by no means ・いつも Itsumo – Always ・全然 Zenzen – Not at all ・一日に Ichinichi ni – In a day ・どのくらい Dono kurai – How much ・実際 Jissai – Actually ・あまり Amari – Not really ・快く Kokoroyoku – Pleasantly Number Phrases ・一人 Hitori – One person (ru) Ichidan Verbs ・掲げるKakageru – To hoist\slash carry (an article)\slash tout\slash adopt (slogan) (trans.) ・比べる Kuraberu – To compare (trans.) ・売れる Ureru – To be sold (intr.) ・寝る Neru – To sleep (intr.) (u) Godan verbs ・持つ Motsu – To hold\slash possess (trans.) ・抉る Eguru – To gouge\slash greatly perturb (trans.) ・向かう Mukau – To head  toward (intr.) ・乗る Noru – To ride (intr.) ・聞く Kiku – To hear\slash listen\slash ask (trans.) ・なる Naru – To be(come) (intr.) ・止まる Tomaru – To stop (intr.) ・吐く Haku – To vomit\slash breathe out\slash spit up (trans.) ・割る Waru – To crack\slash split  (trans.) ・濁る Nigoru – To be cloudy\slash muddy (intr.) ・出回る Demawaru – To appear (in market) (intr.) ・見つかる Mitsukaru – To find incidentally (intr.) ・取る Toru – To take (trans.) suru Verbs ・経営する Keiei suru – To operate\slash manage (trans.) ・分別する Bumbetsu suru – To separate (recycling)\slash division (trans.) ・回収する Kaishū suru – To collect\slash retrieve (trans.) ・脱出する Dasshutsu suru – To escape (intr.) ・冷凍する Reitō suru – To freeze (in freezer) (trans.) ・解凍する Kaitō suru – To thaw  (trans.) ・高望みする Takanozomi suru – To aim too high (intr.) Set Phrases ・うまくいく Umaku iku – To go well ・必要とする Hitsuyō to suru – To require ・黒字倒産 Kuroji tōsan – Insolvency due to liquidity problems ・赤旗  Akahata – Red Flag Nouns  
\par{・新聞 \emph{Shimbun }– Newspaper }
 
\par{・時計 \emph{Tokei }– Clock\slash watch }
 
\par{・本屋 \emph{Hon\textquotesingle ya }– Bookstore }
 
\par{・古新聞 \emph{Furushimbun }– Old newspaper }
 
\par{・古時計 \emph{Furudokei }– Antique clock }
 
\par{・古本屋 \emph{Furuhon\textquotesingle ya }– Used book store }
 
\par{・考え方 \emph{Kangaekata }– Mindset\slash way of thinking }
 
\par{・地区 \emph{Chiku }– Sector\slash area }
 
\par{・お客さん \emph{Okyaku-san }– Customer }
 
\par{・噂話 \emph{Uwasabanashi }– Gossip }
 
\par{・前 \emph{Mae }– In front }
 
\par{・店員 \emph{Ten\textquotesingle in }– Store employee }
 
\par{・古着屋 \emph{Furugiya }– Old clothing store }
 
\par{・売り場 \emph{Uriba }– Sales floor }
 
\par{・過去 \emph{Kako }– Past }
 
\par{・古傷 \emph{Furukizu }– Old wounds }
 
\par{・故郷 \emph{Koky }\emph{ō\slash furusato }– Hometown }
 
\par{・時 \emph{Toki }– Time\slash when }
 
\par{・古株 \emph{Furukabu }– Old timer }
 
\par{・古雑誌 \emph{Furuzasshi }– Old magazine(s) }
 
\par{・電車 \emph{Densha }– Train }
 
\par{・気分 \emph{Kibun }– Feeling\slash mood }
 
\par{・嬉し涙 \emph{Ureshinamida }– Tears of joy }
 
\par{・弱虫 \emph{Yowamushi }– Coward\slash weakling }
 
\par{・火 \emph{Hi }– Fire }
 
\par{・弱火 \emph{Yowabi }– Low flame }
 
\par{・とろ火 \emph{Torobi }– Simmer flame }
 
\par{・こと \emph{Koto }– Thing\slash incident\slash situation }
 
\par{・外交 \emph{Gaik }\emph{ō }– Diplomacy }
 
\par{・弱音 \emph{Yowane }– Feeble complaint }
 
\par{・赤字 \emph{Akaji }– Deficit }
 
\par{・黒字 \emph{Kuroji }– Surplus }
 
\par{・生活 \emph{Seikatsu }– Life\slash livelihood }
 
\par{・倒産 \emph{T }\emph{ōsan }– Bankruptcy }
 
\par{・確率 \emph{Kakuritsu }– Probability }
 
\par{・業種 \emph{Gy }\emph{ōshu }– Industry }
 
\par{・玄関前 \emph{Genkan-mae }– In front of entryway }
 
\par{・赤身魚 \emph{Akamizakami }– Red meat fish }
 
\par{・低温 \emph{Teion }– Low temperature }
 
\par{・低時間 \emph{Teijikan }– Short amount of time }
 
\par{・卵 \emph{Tamago }– Egg }
 
\par{・白身 \emph{Shiromi }– White meat\slash white of an egg }
 
\par{・黄身 \emph{Kimi }– Yolk }
 
\par{・スーパー \emph{S }\emph{ūp }\emph{ā }– Supermarket }
 
\par{・八百屋 \emph{Yaoya }– Greengrocer }
 
\par{・市場 \emph{Ichiba }– Market }
 
\par{・高値 \emph{Takane }– High price }
 
\par{・安値 \emph{Yasune }– Low price }
 
\par{・本 \emph{Hon }– Book }
 
\par{・弁護士 \emph{Bengoshi }– Lawyer }
 
\par{・態度 \emph{Taido }– Atittude }
 
\par{・婚活 \emph{Konkatsu }– Marriage hunting }
 
\par{・シロクマ(白熊) \emph{Shirokuma }– Polar bear }
 
\par{・薄着 \emph{Usugi }– Light clothing }
 
\par{・厚着 \emph{Atsugi }– Heavy clothing }
 
\par{・餌 \emph{Esa }– Food (animal\textquotesingle s) }
 
\par{・量 \emph{Ry }\emph{ō }– Amount }
 
\par{・男性 \emph{Dansei }– Man\slash male }
 
\par{・女性 \emph{Josei }– Woman\slash female }
 
\par{・厚化粧 \emph{Atsugesh }\emph{ō }– Heavy makeup }
 
\par{\textbf{Pronouns }}
 
\par{・俺 \emph{Ore }– I (rough male speech) }
 
\par{・自分 \emph{Jibun }– Oneself }
 
\par{\textbf{Proper Nouns }}
 
\par{・日本 \emph{Nihon }– Japan }
 
\par{・共産党 \emph{Kyosant }\emph{ō }– Communist Party }
 
\par{\textbf{Adjectives }}
 
\par{・古い \emph{Furui }– Old }
 
\par{・古臭い \emph{Furukusai }– Old-fashioned }
 
\par{・若い \emph{Wakai }– Young }
 
\par{・低い \emph{Hikui }– Low }
 
\par{・無い \emph{Nai }– Not being }
 
\par{・辛い \emph{Tsurai }– Bitter\slash tough\slash painful }
 
\par{・暑い \emph{Atsui }– Hot (weather) }
 
\par{\textbf{Adnominal Adjectives }}
 
\par{・ある \emph{Aru }– A certain }
    \textbf{Adverbs }
\par{・毎月 \emph{Maitsuki }– Every month }

\par{・将来 \emph{Sh }\emph{ōrai }– In the future }

\par{・とっても \emph{Tottemo }– Very }

\par{・必ずしも \emph{Kanarazu shimo }– Not always }

\par{・ここ数日 \emph{Koko s }\emph{ūjitsu }– The past few days }

\par{・限り \emph{Kagiri }– As possible }

\par{・うちに \emph{Uchi ni }– While }

\par{・ほとんど \emph{Hotondo }– Hardly }

\par{・なかなか \emph{Nakanaka }– Considerably\slash by no means }

\par{・いつも \emph{Itsumo }– Always }

\par{・全然 \emph{Zenzen }– Not at all }

\par{・一日に \emph{Ichinichi ni }– In a day }

\par{・どのくらい \emph{Dono kurai }– How much }

\par{・実際 \emph{Jissai }– Actually }

\par{・あまり \emph{Amari }– Not really }

\par{・快く \emph{Kokoroyoku }– Pleasantly }

\par{\textbf{Number Phrases }}

\par{・一人 \emph{Hitori }– One person }

\par{\textbf{\emph{(ru) Ichidan }Verbs }}

\par{・掲げる \emph{Kakageru }– To hoist\slash carry (an article)\slash tout\slash adopt (slogan) (trans.) }

\par{・比べる \emph{Kuraberu }– To compare (trans.) }

\par{・売れる \emph{Ureru }– To be sold (intr.) }

\par{・寝る \emph{Neru }– To sleep (intr.) }

\par{\textbf{\emph{(u) Godan }verbs }}

\par{・持つ \emph{Motsu }– To hold\slash possess (trans.) }

\par{・抉る \emph{Eguru }– To gouge\slash greatly perturb (trans.) }

\par{・向かう \emph{Mukau }– To head  toward (intr.) }

\par{・乗る \emph{Noru }– To ride (intr.) }

\par{・聞く \emph{Kiku }– To hear\slash listen\slash ask (trans.) }

\par{・なる \emph{Naru }– To be(come) (intr.) }

\par{・止まる \emph{Tomaru }– To stop (intr.) }

\par{・吐く \emph{Haku }– To vomit\slash breathe out\slash spit up (trans.) }

\par{・割る \emph{Waru }– To crack\slash split  (trans.) }

\par{・濁る \emph{Nigoru }– To be cloudy\slash muddy (intr.) }

\par{・出回る \emph{Demawaru }– To appear (in market) (intr.) }

\par{・見つかる \emph{Mitsukaru }– To find incidentally (intr.) }

\par{・取る \emph{Toru }– To take (trans.) }

\par{\textbf{\emph{suru }Verbs }}

\par{・経営する \emph{Keiei suru }– To operate\slash manage (trans.) }

\par{・分別する \emph{Bumbetsu suru }– To separate (recycling)\slash division (trans.) }

\par{・回収する \emph{Kaish }\emph{ū suru }– To collect\slash retrieve (trans.) }

\par{・脱出する \emph{Dasshutsu suru }– To escape (intr.) }

\par{・冷凍する \emph{Reit }\emph{ō suru }– To freeze (in freezer) (trans.) }

\par{・解凍する \emph{Kait }\emph{ō suru }– To thaw  (trans.) }

\par{・高望みする \emph{Takanozomi suru }– To aim too high (intr.) }

\par{\textbf{Set Phrases }}

\par{・うまくいく \emph{Umaku iku }– To go well }

\par{・必要とする \emph{Hitsuy }\emph{ō to suru }– To require }

\par{・黒字倒産 \emph{Kuroji t }\emph{ōsan }– Insolvency due to liquidity problems }
・赤旗 \emph{Akahata }– Red Flag  \textbf{Adjectival Nouns }
\par{・田舎\{の\} \emph{Inaka [no] }– Rural }

\par{・他\{の\} \emph{Ta\slash hoka [no] }– Other }

\par{・嫌\{な\} \emph{Iya [na] }– Unpleasant }

\par{・地球上\{の\} \emph{Chiky }\emph{ūj }\emph{ō [no] }– On Earth }

\par{・最強\{の\} \emph{Saiky }\emph{ō }– Strongest }

\par{・弱腰\{な・の\} \emph{Yowagoshi [na\slash no] }– Weak-kneed }

\par{・隣人\{の\} \emph{Rinjin [no] }– Neighbor }

\par{・可能\{な\} \emph{Kan }\emph{ō [na] }– Possible }

\par{・高飛車\{な\} \emph{Takabisha [na] }– High-handed }

\par{・野生\{の\} \emph{Yasei [no] }– Wild }

\par{・多く\{の\} \emph{Ōku }– A lot\slash many }

\par{\textbf{Prefixes }}

\par{・最~ \emph{Sai }- - -est }

\par{\textbf{Suffixes }}

\par{・~達 \emph{-tachi }– Plural marker }

\par{・~(Adj. stem +)そうだ \emph{-s }\emph{ō da }– To seem }

\par{・~にくい \emph{-nikui }– Difficult to }

\par{\textbf{Question Words }}

\par{・どうして \emph{D }\emph{ōshite }– Why }

\par{・何故 \emph{Naze }– Why }
      
\section{Stems in Compounds}
 1. 田舎の古臭い考え方を持つ。 Inaka no furukusai kangaekata wo motsu. To hold a rural, old-fashioned mindset. 2. ある地区では、毎月、古新聞と古雑誌を分別して回収している。 Aru chiku de wa, maitsuki, furushimbun to furuzasshi wo bumbetsu shite kaishū shite iru. In a certain sector, every month, (the municipality) separates and collects old newspapers and magazines. 3. 将来、古着屋を経営したいと考えています。 Shōrai, furugiya wo keiei shitai to kangaete imasu. I\textquotesingle m thinking about wanting to run an old clothing store in the future. 4. 売り場の古株そうな店員が他の若い店員達を前に、お客さんの噂話をしていたのを聞いたときもとっても嫌な気分になったんですよ。 Uriba no furukabu-sō na ten\textquotesingle in ga hoka no wakai ten\textquotesingle intachi wo mae ni, okyaku-san no uwasabanashi wo shite ita no wo kiita toki mo tottemo iya na kibun ni natta n desu yo. I was also in a really bad mood when I heard employees who seemed to be old-timers of the sales floor gossiping about customers in front of other young employees. 5. 過去の古傷をえぐってしまう。 Kako no furukizu wo egutte shimau. To accidentally perturb old wounds. 6. 故郷へ向かって電車に乗る。 Furusato e mukatte densha ni noru. To ride on a train heading for one\textquotesingle s hometown. Reading Note: 故郷 may also be read as “kokyō,” but the reading “furusato” is a combination of the adjective furui 古い (old) and the noun “sato” 里 (neighborhood). 7. 嬉し涙が止まらない。 Ureshinamida ga tomaranai. My tears of joy won\textquotesingle t stop. 8. 地球上最強の俺は、弱虫じゃないぞ! Chikyūjō saikyō no ore wa, yowamushi ja nai zo! I, the strongest man on Earth, am not a coward! 9. 弱火より弱い火のことをとろ火と言います。 Yowabi yori yowai hi no koto wo torobi to iimasu. A flame weaker than a “yowabi (low flame)” is called a “torobi.” 10. 何故日本の外交は弱腰なのか。 Naze Nihon no gaikō wa yowagoshi na no ka? Why is Japanese diplomacy weak-kneed? 11. 弱音を吐くのは必ずしもいけないんでしょうか。 Yowane wo haku no wa kanarazu shimo ikenai n deshō ka? Is making complaints always a bad thing? 12. 赤字生活から脱出したいんです。 Akaji seikatsu kara dasshutsu shitai n desu. I wish to escape my life in the red. 13. 黒字倒産になる確率は他の業種と比べて低いのです。 Kuroji tōsan ni naru kakuritsu wa ta no gyōshu to kurabete hikui no desu. The probability of insolvency due to liquidity issues compared to other industries is low. 14. ここ数日、隣人のひとりが、日本共産党の赤旗を自分の玄関前に掲げていました。 Koko sūjitsu, rinjin no hitori ga, Nihon Kyōsan-tō no akahata wo jibun no genkan-mae ni kakagete imashita. For the past couple of days, one of my neighbors has had the red flag of the Communist Party of Japan hanging in front of his entryway. 15. 冷凍した赤身魚を解凍する時は、可能な限り、低温で短時間のうちに解凍してください。 Reitō shita akamizakana wo kaitō suru toki wa, kanō na kagiri, teion de tanjikan no uchi ni kaitō shite kudasai. When thawing frozen fish with red meat, thaw at low temperature in as short amount of time as possible. 16. 卵を割ったとき、白身が濁っていたことはありませんか。 Tamago wo watta toki, shiromi ga nigotte ita koto wa arimasen ka? Has the white ever been cloudy for you when cracking an egg? 17. 黄身なしの卵がスーパーや八百屋などの市場で出回ることはほとんどないでしょう。 Kimi nashi no tamago ga sūpā ya yaoya nado no ichiba de demawaru koto wa hotondo nai deshō. Eggs without yolks probably hardly ever appear in markets like supermarkets or greengrocers. 18. 高値で売れる本はなかなか見つかりにくい。 Takane de ureru hon wa nakanaka mitsukarinikui. It is rather difficult to find books sold at a high price. 19. 株を一日の最安値で買うのは無理です。 Ichinichi no saiyasune de kau no wa muri desu. It\textquotesingle s impossible to buy stocks at the lowest price of the day. 20. 弁護士は何故いつも高飛車な態度を取るんでしょうか。 Bengoshi wa naze itsumo takabisha na taido wo toru n deshō ka? Why is it that lawyers always take a high-handed attitude? 21. 全然高望みしてないのに婚活がうまくいかなくて辛い。 Zenzen takanozomi shitenai noni konkatsu ga umaku ikanakute tsurai. Even though I\textquotesingle m not aiming too high at all, it\textquotesingle s been tough with my marriage hunting not going well. 22. 野生の白熊が一日に必要としている餌の量は、どのくらいですか。 Yasei no shirokuma ga ichinichi ni hitsuyō to shite iru esa no ryō wa, dono kurai desu ka? What is the amount of food a wild polar bear requires a day? 23. どうして薄着で寝るの? Dōshite usugi de neru no? Why do you sleep in light clothing? 24. 暑いのに厚着をする。 Atsui noni atsugi wo suru. To wear thick clothing despite it being hot. 25. 実際、多くの男性は厚化粧の女性をあまり快く思っていないようです。 Jissai, ōku no dansei wa atsugeshō no josei wo amari kokoroyoku omotte inai yō desu. 
\begin{center}
\textbf{Adjectives Aren't Created Equal }
\end{center}
Actually, many men don\textquotesingle t think very pleasantly of women with heavy makeup. 
\par{ Not all adjectives are created equally in how productive they may be used in compounds. Nonetheless, the number of examples that are commonly used is quite high. For the remainder of this lesson, you will become acquainted with plenty of examples to get a feel of what to look for as you continue to encounter more of them in your studies. }

\par{1. 田舎の古臭い考え方を持つ。 \hfill\break
 \emph{Inaka no furukusai kangaekata wo motsu. \hfill\break
}To hold a rural, old-fashioned mindset. }
 
\par{2. ある地区では、毎月、古新聞と古雑誌を分別して回収している。 \hfill\break
 \emph{Aru chiku de wa, maitsuki, furushimbun to furuzasshi wo bumbetsu shite kaishū shite iru. }\hfill\break
In a certain sector, every month, (the municipality) separates and collects old newspapers and magazines. }
 
\par{3. 将来、古着屋を経営したいと考えています。 \hfill\break
 \emph{Shōrai, furugiya wo keiei shitai to kangaete imasu. }\hfill\break
I\textquotesingle m thinking about wanting to run an old clothing store in the future. }
 
\par{4. 売り場の古株そうな店員が他の若い店員達を前に、お客さんの噂話をしていたのを聞いたときもとっても嫌な気分になったんですよ。 \hfill\break
 \emph{Uriba no furukabu-sō na ten\textquotesingle in ga hoka no wakai ten\textquotesingle intachi wo mae ni, okyaku-san no uwasabanashi wo shite ita no wo kiita toki mo tottemo iya na kibun ni natta n desu yo. \hfill\break
 }I was also in a really bad mood when I heard employees who seemed to be old-timers of the sales floor gossiping about customers in front of other young employees. }
 
\par{5. 過去の古傷をえぐってしまう。 \hfill\break
 \emph{Kako no furukizu wo egutte shimau. }\hfill\break
To accidentally perturb old wounds. }
 
\par{6. 故郷へ向かって電車に乗る。 \hfill\break
 \emph{Furusato e mukatte densha ni noru. } \hfill\break
To ride on a train heading for one\textquotesingle s hometown. }
 
\par{\textbf{Reading Note }: 故郷 may also be read as “ \emph{koky }\emph{ō },” but the reading “ \emph{furusato }” is a combination of the adjective \emph{furui }古い (old) and the noun “ \emph{sato }” 里 (neighborhood). }
 
\par{7. 嬉し涙が止まらない。 \hfill\break
 \emph{Ureshinamida ga tomaranai. }\hfill\break
My tears of joy won\textquotesingle t stop. }
 
\par{8. 地球上最強の俺は、弱虫じゃないぞ! \hfill\break
 \emph{Chikyūjō saikyō no ore wa, yowamushi ja nai zo! }\hfill\break
I, the strongest man on Earth, am not a coward! }
 
\par{9. 弱火より弱い火のことをとろ火と言います。 \hfill\break
 \emph{Yowabi yori yowai hi no koto wo torobi to iimasu. }\hfill\break
A flame weaker than a “yowabi (low flame)” is called a “torobi.” }
 
\par{10. 何故日本の外交は弱腰なのか。 \hfill\break
 \emph{Naze Nihon no gaikō wa yowagoshi na no ka? }\hfill\break
Why is Japanese diplomacy weak-kneed? }
 
\par{11. 弱音を吐くのは必ずしもいけないんでしょうか。 \hfill\break
 \emph{Yowane wo haku no wa kanarazu shimo ikenai n deshō ka? }\hfill\break
Is making complaints always a bad thing? }
 
\par{12. 赤字生活から脱出したいんです。 \hfill\break
 \emph{Akaji seikatsu kara dasshutsu shitai n desu. }\hfill\break
I wish to escape my life in the red. }
 
\par{13. 黒字倒産になる確率は他の業種と比べて低いのです。 \hfill\break
 \emph{Kuroji tōsan ni naru kakuritsu wa ta no gyōshu to kurabete hikui no desu. }\hfill\break
The probability of insolvency due to liquidity issues compared to other industries is low. }
 
\par{14. ここ数日、隣人のひとりが、日本共産党の赤旗を自分の玄関前に掲げていました。 \hfill\break
 \emph{Koko sūjitsu, rinjin no hitori ga, Nihon Kyōsan-tō no akahata wo jibun no genkan-mae ni kakagete imashita. }\hfill\break
For the past couple of days, one of my neighbors has had the red flag of the Communist Party of Japan hanging in front of his entryway. }
 
\par{15. 冷凍した赤身魚を解凍する時は、可能な限り、低温で短時間のうちに解凍してください。 \hfill\break
\emph{Reitō shita akamizakana wo kaitō suru toki wa, kanō na kagiri, teion de tanjikan no uchi ni kaitō shite kudasai. }\hfill\break
When thawing frozen fish with red meat, thaw at low temperature in as short amount of time as possible. }
 
\par{16. 卵を割ったとき、白身が濁っていたことはありませんか。 \hfill\break
 \emph{Tamago wo watta toki, shiromi ga nigotte ita koto wa arimasen ka? }\hfill\break
Has the white ever been cloudy for you when cracking an egg? }
 
\par{17. 黄身なしの卵がスーパーや八百屋などの市場で出回ることはほとんどないでしょう。 \hfill\break
 \emph{Kimi nashi no tamago ga sūpā ya yaoya nado no ichiba de demawaru koto wa hotondo nai deshō. \hfill\break
 }Eggs without yolks probably hardly ever appear in markets like supermarkets or greengrocers. }
 
\par{18. 高値で売れる本はなかなか見つかりにくい。 \hfill\break
 \emph{Takane de ureru hon wa nakanaka mitsukarinikui. }\hfill\break
It is rather difficult to find books sold at a high price. }
 
\par{19. 株を一日の最安値で買うのは無理です。 \hfill\break
 \emph{Ichinichi no saiyasune de kau no wa muri desu. }\hfill\break
It\textquotesingle s impossible to buy stocks at the lowest price of the day. }
 
\par{20. 弁護士は何故いつも高飛車な態度を取るんでしょうか。 \hfill\break
 \emph{Bengoshi wa naze itsumo takabisha na taido wo toru n deshō ka? }\hfill\break
Why is it that lawyers always take a high-handed attitude? }
 
\par{21. 全然高望みしてないのに婚活がうまくいかなくて辛い。 \hfill\break
 \emph{Zenzen takanozomi shitenai noni konkatsu ga umaku ikanakute tsurai. \hfill\break
 }Even though I\textquotesingle m not aiming too high at all, it\textquotesingle s been tough with my marriage hunting not going well. }
 
\par{22. 野生の白熊が一日に必要としている餌の量は、どのくらいですか。 \hfill\break
 \emph{Yasei no shirokuma ga ichinichi ni hitsuyō to shite iru esa no ryō wa, dono kurai desu ka? }\hfill\break
What is the amount of food a wild polar bear requires a day? }
 
\par{23. どうして薄着で寝るの? \hfill\break
 \emph{D }\emph{ōshite usugi de neru no? } \hfill\break
Why do you sleep in light clothing? }
 
\par{24. 暑いのに厚着をする。 \hfill\break
 \emph{Atsui noni atsugi wo suru. }\hfill\break
To wear thick clothing despite it being hot. }
 
\par{25. 実際、多くの男性は厚化粧の女性をあまり快く思っていないようです。 \hfill\break
 \emph{Jissai, ōku no dansei wa atsugeshō no josei wo amari kokoroyoku omotte inai yō desu. \hfill\break
 }Actually, many men don\textquotesingle t think very pleasantly of women with heavy makeup. }
    