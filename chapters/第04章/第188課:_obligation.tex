    
\chapter{Planning \& Obligation}

\begin{center}
\begin{Large}
第188課: Planning \& Obligation: つもり, はず, \& ~べきだ 
\end{Large}
\end{center}
 
\par{ In this lesson we will learn about speech modals of planning and obligation. The speech modals that we are going to cover are the following. }
      
\section{つもり}
 
\par{ The noun 積もり may show one's intent or expectation to do something. In speech modals, it is normally left in かな. }
 
\par{${\overset{\textnormal{}}{\text{1a. 今日何}}}$ をするつもり(か)? \hfill\break
${\overset{\textnormal{}}{\text{1b. 今日}}}$ は ${\overset{\textnormal{}}{\text{何}}}$ をするの? \hfill\break
What do you plan to do today? }
 
\par{${\overset{\textnormal{}}{\text{2. 車}}}$ を ${\overset{\textnormal{}}{\text{買}}}$ うつもりだ。 \hfill\break
I plan to buy a car. }
 
\par{${\overset{\textnormal{}}{\text{3. 彼女}}}$ に ${\overset{\textnormal{}}{\text{従}}}$ うつもりはない。 \hfill\break
I have no intention of obeying her. }
 
\par{4. タバコをやめるつもりはない。 \hfill\break
I am not planning on quitting smoking. }
 
\par{5. もっと ${\overset{\textnormal{}}{\text{早}}}$ く ${\overset{\textnormal{}}{\text{帰}}}$ るつもりでした。 \hfill\break
I planned on coming home earlier. }
 
\par{${\overset{\textnormal{}}{\text{6. 留学}}}$ するつもりはありますか。 \hfill\break
Do you have intentions of studying abroad? }
 
\par{${\overset{\textnormal{}}{\text{7. 女優}}}$ のつもりでいる。 \hfill\break
She is by way of being an actress. }
 
\par{8. 冗談のつもりで言ったのに、彼を怒らせてしまいました。 \hfill\break
I meant it as a joke, but I accidentally made him angry. }

\par{9. 日本へ着いたら、日本語の ${\overset{\textnormal{じてん}}{\text{辞典}}}$ を買うつもりだ。 \hfill\break
(When\slash right after) I arrive in Japan, I plan to buy a Japanese dictionary. }
 
\par{${\overset{\textnormal{}}{\text{10a. 今日}}}$ 、 ${\overset{\textnormal{}}{\text{田中}}}$ さんに ${\overset{\textnormal{}}{\text{会}}}$ う ${\overset{\textnormal{}}{\text{予定}}}$ はない。 \hfill\break
 ${\overset{\textnormal{}}{\text{10b. 今日}}}$ 、 ${\overset{\textnormal{}}{\text{田中}}}$ さんに ${\overset{\textnormal{}}{\text{会}}}$ わない。 \hfill\break
I have no intention of meeting today. }
 
\par{\textbf{Word Note }: As the two variants above show, sometimes ~つもりだ is a little unnatural.  予定だ shows an intended schedule to inform people. }
 
\par{${\overset{\textnormal{}}{\text{11. 明日}}}$ の ${\overset{\textnormal{}}{\text{朝}}}$ は ${\overset{\textnormal{}}{\text{早}}}$ く ${\overset{\textnormal{}}{\text{起}}}$ きるつもりです。 \hfill\break
I plan to wake up early tomorrow morning. }
 
\par{${\overset{\textnormal{}}{\text{12. 旅行}}}$ は、 ${\overset{\textnormal{}}{\text{三日間}}}$ ぐらいの ${\overset{\textnormal{}}{\text{予定}}}$ です。 \hfill\break
The trip is scheduled to last three days. }
 
\par{${\overset{\textnormal{}}{\text{13. 明日}}}$ は ${\overset{\textnormal{}}{\text{授業}}}$ に ${\overset{\textnormal{}}{\text{行}}}$ かないつもりです。 \hfill\break
I plan to not go to class tomorrow. }
 
\par{${\overset{\textnormal{}}{\text{14. 友}}}$ だちのつもりだが ${\overset{\textnormal{なん}}{\text{何}}}$ だか ${\overset{\textnormal{みょう}}{\text{妙}}}$ に ${\overset{\textnormal{な}}{\text{馴}}}$ れ ${\overset{\textnormal{}}{\text{馴}}}$ れしい。 \hfill\break
We intended to be friends, but for some reason we got strangely over-familiar. }
 
\par{\textbf{Word Note }: The verb ${\overset{\textnormal{もくろ}}{\text{目論}}}$ む may also be used to show planning. }
 
\par{15. あいつが ${\overset{\textnormal{}}{\text{何}}}$ を ${\overset{\textnormal{}}{\text{目論}}}$ んでるのか ${\overset{\textnormal{}}{\text{分}}}$ からねー。(Slang; Vulgar) \hfill\break
I don't know what he's scheming to do. }
 
\par{${\overset{\textnormal{}}{\text{16. 私}}}$ は ${\overset{\textnormal{}}{\text{医者}}}$ になるつもりでしたが。 \hfill\break
I planned to become a doctor, but\dothyp{}\dothyp{}\dothyp{} }
 
\par{17. どこへ行くつもりだい?(Masculine; old-fashioned) \hfill\break
Where do you think you're going? }
 
\par{\textbf{Phrase Note }: The use of ~だい in this sentence makes it a little old-fashioned. Also, it would only be used by men. }
 
\par{${\overset{\textnormal{}}{\text{18. 来週}}}$ までに4 ${\overset{\textnormal{}}{\text{章読}}}$ むつもりです。 \hfill\break
I plan to read 4 chapters by next week }
 
\par{19. そんなつもりじゃなかったよ。 \hfill\break
I didn't mean that. }
 
\par{${\overset{\textnormal{}}{\text{20. 全力}}}$ で ${\overset{\textnormal{しえん}}{\text{支援}}}$ するつもりだよ。 \hfill\break
I plan to support you with all my support. }

\par{21. ${\overset{\textnormal{ぜ}}{\text{是}}}$ が ${\overset{\textnormal{ひ}}{\text{非}}}$ でも ${\overset{\textnormal{}}{\text{留学}}}$ するつもりらしいです。 \hfill\break
He seems to plan to study abroad by all means. }
 
\par{\textbf{Usage Note }: Remember that your plans should be treated differently from another person's. }
 
\par{\textbf{Grammar Note }: 「~たつもりだ、 ~のつもりだ、 ~ているつもりだ」等 show suppositions that contrast reality. They may also show self-centered decisions, subjective impressions, etc. }

\par{22. ${\overset{\textnormal{ぬ}}{\text{抜}}}$ かりなくやったつもりだったが ${\overset{\textnormal{}}{\text{失敗}}}$ した。 \hfill\break
I intended to have made it without blunder, but I failed. }

\par{23. 帰るつもりだったが、 ${\overset{\textnormal{}}{\text{泊}}}$ まることになっちゃった。 \hfill\break
I intended to go home, but I ended up staying at a hotel. }

\par{${\overset{\textnormal{}}{\text{24. 君}}}$ の ${\overset{\textnormal{}}{\text{気持}}}$ ちは ${\overset{\textnormal{}}{\text{分}}}$ かっているつもり(だ)。(Colloquial) \hfill\break
I believe I know your feelings well enough. }
 
\par{25. コーヒーを ${\overset{\textnormal{}}{\text{一杯飲}}}$ んだつもりで、 ${\overset{\textnormal{}}{\text{特急電車}}}$ に ${\overset{\textnormal{}}{\text{乗}}}$ った。 \hfill\break
I took a limited express train, and I imagined that I had a cup of coffee. }
 
\par{26. よく読んだつもりでした。 \hfill\break
I was convinced that I had read it well. }

\par{ This usage of つもり is closer to "conviction". ~たつもり shows a defense to one's convictions despite the fact that there is overwhelming evidence to the contrary. This is always why it can sometimes show a selfish side. }

\par{27. 切手を貼ったつもりで、手紙をポストに入れてしまった。 \hfill\break
I accidentally put a letter into the postbox having thought I put a stamp on it (but I hadn't). }

\par{28. 死んだつもりで生きていこうと決心した。 \hfill\break
I was determined to live on having thought I would die. }

\par{29. いとこはもう大人のつもりだな。 \hfill\break
My cousin thinks he's already grownup, eh? \hfill\break
 \hfill\break
30. 「アメリカ人ですか」。「アメリカ人のつもりですけれど」。 \hfill\break
Are you American? I was American the last time I checked, but\dothyp{}\dothyp{}\dothyp{} }
      
\section{はず}
 
\par{ The noun はず shows obligation. The speaker may use this pattern to show that he or she is convinced of what should happen based on some sort of reasoning that is either built on personal judgment or on what he or she is quite sure of is the case. }

\par{ It follows the 連体形 of verbs or adjectives and nouns with の. It is like "supposed to". The negative is はずじゃない. Lastly, はずがない strongly denies when there is neither reason nor basis. }

\begin{ltabulary}{|P|P|}
\hline 

Nouns & 本気 \textbf{の }はず \\ \cline{1-2}

形容詞 & い \textbf{い }はず \\ \cline{1-2}

形容動詞 & 簡単 \textbf{な }はず \\ \cline{1-2}

Verbs & 着 \textbf{く }はず \\ \cline{1-2}

\end{ltabulary}
 
\begin{center}
\textbf{Examples }
\end{center}

\par{${\overset{\textnormal{}}{\text{31. 田中}}}$ さんという ${\overset{\textnormal{}}{\text{男}}}$ は ${\overset{\textnormal{}}{\text{顔見知}}}$ りのはずだ。 \hfill\break
That Mr. Tanaka is supposed to be an acquaintance. }
 
\par{32. ここが ${\overset{\textnormal{たから}}{\text{宝}}}$ の ${\overset{\textnormal{あ}}{\text{在}}}$ り ${\overset{\textnormal{か}}{\text{処}}}$ のはずだ。 \hfill\break
This is the whereabouts the treasure is supposed to be. }
 
\par{${\overset{\textnormal{}}{\text{33. 八時}}}$ までに ${\overset{\textnormal{}}{\text{宿題}}}$ をやったはずだよ。 \hfill\break
You're supposed to have already finished your homework by 8 o'clock. }
 
\par{${\overset{\textnormal{}}{\text{34. 手紙}}}$ はもう ${\overset{\textnormal{}}{\text{着}}}$ いたはずだ。 \hfill\break
The letter is supposed to have already arrived. }
 
\par{${\overset{\textnormal{}}{\text{35. 彼女}}}$ は3 ${\overset{\textnormal{}}{\text{時}}}$ に ${\overset{\textnormal{}}{\text{着}}}$ くはずだった。 \hfill\break
She was supposed to arrive at three. }
 
\par{${\overset{\textnormal{}}{\text{36. 彼}}}$ はすぐ ${\overset{\textnormal{}}{\text{戻}}}$ るはずだが。 \hfill\break
He's supposed to return soon but. }
 
\par{${\overset{\textnormal{}}{\text{37. 四日}}}$ で ${\overset{\textnormal{}}{\text{仕上が}}}$ るはずだ。 \hfill\break
We should finish in four days\slash on the fourth. }
 
\par{38. ヨーグルトは ${\overset{\textnormal{ちょう}}{\text{腸}}}$ にいいはずですよ。 \hfill\break
Yogurt is supposed to be good for your intestines. }
 
\par{${\overset{\textnormal{}}{\text{39. 君}}}$ は ${\overset{\textnormal{}}{\text{彼}}}$ の ${\overset{\textnormal{いばしょ}}{\text{居場所}}}$ を ${\overset{\textnormal{}}{\text{知}}}$ ってるはずさ。(Casual) \hfill\break
You ought to know his whereabouts. }
 
\par{40. それは ${\overset{\textnormal{}}{\text{本当}}}$ であるはずがない。 \hfill\break
It cannot be true. }

\par{41. ${\overset{\textnormal{うそ}}{\text{嘘}}}$ をついたはずがありません。 \hfill\break
He couldn't have told a lie. }
 
\par{${\overset{\textnormal{}}{\text{42. 今日家}}}$ にいるはずだ。 \hfill\break
He should be at home today. }
 
\par{43. デパートは8 ${\overset{\textnormal{}}{\text{時}}}$ に ${\overset{\textnormal{}}{\text{開}}}$ くはずです。 \hfill\break
The department store is supposed to open at 8. }

\par{44. 「あのレストランはいつも込んでますね」「ええ、でも、お昼前に行けば\{込んでいない・人が少ない・空いてい         る\}はずですよ。 \hfill\break
"That restaurant is always crowded". "Yes, but if you go before noun, it shouldn't be crowded". }

\par{45. 「井上さんは来るでしょうか」「ええ、さっき出かけると電話がありましたから、そろそろ\{来る・着く\}はずです」 \hfill\break
"Is Inoue coming?". "Yes, I got a call a while ago when we left, and (he\slash she) should be (coming\slash getting) here soon". }

\par{46a. 生きているということは、 ${\overset{\textnormal{たいおん}}{\text{体温}}}$ は35度から36度の間で ${\overset{\textnormal{いってい}}{\text{一定}}}$ しているはずです. \hfill\break
46b. 生き残るためには、体温を35度から36度の間で保持しなければならない。 \hfill\break
In order to live, your body temperature should be stabilized between 35 to 36 degrees. }
いつもならもっといいはずなのに、 Although it's supposed to be more better if it always (happens) 田中(たなか)さんという男(おとこ)は顔見知(かおみし)りのはずだ。 That Mr. Tanaka is supposed to be an acquaintance. ここが宝(たから)の在(あ)り処(か)のはずだ。 This is the whereabouts the treasure is supposed to be. もう八時(はちじ)までに宿題(しゅくだい)をやったはずだよ。 You're supposed to have already finished your homework by 8 o'clock. 手紙(てがみ)はもう着(つ)いたはずだ。 The letter is supposed to have already arrived. 彼女(かのじょ)は三時(さんじ)に着(つ)くはずだった。 She was supposed to arrive at three. 彼(かれ)はすぐ戻(もど)るはずだが。 He's supposed to return soon but. 四日(よっか)で仕上(しあ)げるはずだ。 We should finish in four days\slash on the fourth. ヨーグルトは腸(ちょう)にいいはずですよ。 Yogurt is supposed to be good for your intestines. 君(きみ)は彼(かれ)の居場所(いばしょ)を知(し)ってるはずさ。 You ought to know his whereabouts. 今日着(きょうつ)くはずです。 He's supposed to arrive today. それは本当(ほんとう)であるはずがない。 It cannot be true. 嘘(うそ)をついたはずがありません。 He couldn't have told a lie. 彼女(かのじょ)は来(く)るはずです。 She is supposed to come. 今日家(きょういえ)にいるはずだ。 He should be at home today. デパートは八時(はちじ)に開(ひら)くはずです。 The department store is supposed to open at 8. 生きているということは、 ${\overset{\textnormal{たいおん}}{\text{体温}}}$ は35度から36度の間で ${\overset{\textnormal{いってい}}{\text{一定}}}$ しているはずです. \hfill\break
生き ${\overset{\textnormal{}}{\text{残}}}$ るためには、体温を35度から36度の間で保持しなければならない。 \hfill\break
In order to live, your body temperature should be stabilized between 35 to 36 degrees. \hfill\break
      
\section{~べきだ}
 
\par{ ~べきだ is infrequently used. It shows strong subjective opinion of obligation. ~べきだ attaches to the ${\overset{\textnormal{}}{\text{終止形}}}$ of verbs. When ~べきだ attaches to する, you get す(る)べきだ. This also goes for ずる-Verbs. }

\begin{ltabulary}{|P|P|P|P|}
\hline 

To ought to feel & 感ず(る)べきだ & 感じるべきだ & 感じべきだ X \\ \cline{1-4}

To ought to esteem & 重んず(る)べきだ & 重んじるべきだ & 重んじべきだ X \\ \cline{1-4}

\end{ltabulary}

\par{ It's normally only used with verbs , but when not it shows a strong sense of "should". ~べし should follow the ${\overset{\textnormal{}}{\text{連体形 ~かる}}}$ of 形容詞 and the copula as なる for ${\overset{\textnormal{}}{\text{形容動詞}}}$ and nouns. As for -べきだ, it should follow ${\overset{\textnormal{}}{\text{形容詞}}}$ like in あたらし \textbf{くある }べきだ and after the copula である for 形容動詞 and nouns. }

\par{\textbf{Negative Note }: The negative form of this pattern should be ~べきじゃない. Don't feel bad if you are corrected for saying ~ないべきだ. ~ざるべし, which would be the predecessor of such a form, has existed in the past. Though the majority of natives believe that ~ないべきだ is grammatically incorrect, in spoken language, it is seen quite a lot. As a student, you should avoid it. }

\begin{center}
\textbf{Examples } 
\end{center}

\par{47. もっとご ${\overset{\textnormal{りょうしん}}{\text{両親}}}$ を ${\overset{\textnormal{うやま}}{\text{敬}}}$ う\{もの・べき\}です。 \hfill\break
You ought to be more respectful to your parents. }
 
\par{\textbf{Word Note }: 両親 alone is only used for "one's parents". }

\par{48. ${\overset{\textnormal{つい}}{\text{遂}}}$ に ${\overset{\textnormal{}}{\text{来}}}$ るべき ${\overset{\textnormal{}}{\text{時}}}$ が ${\overset{\textnormal{}}{\text{来}}}$ た。 \hfill\break
At last the time when we're supposed to come has arrived! }
 
\par{${\overset{\textnormal{}}{\text{49. 絶対}}}$ に ${\overset{\textnormal{}}{\text{守}}}$ るべき ${\overset{\textnormal{}}{\text{場所}}}$ 。 \hfill\break
A place that should be protected always. }

\par{50. ${\overset{\textnormal{うち}}{\text{内}}}$ でやるべきだ。 \hfill\break
Our (department\slash group) should do it. }

\par{52. ${\overset{\textnormal{そくざ}}{\text{即座}}}$ に ${\overset{\textnormal{}}{\text{戦}}}$ うべきだとする ${\overset{\textnormal{}}{\text{意見}}}$ が ${\overset{\textnormal{}}{\text{多数}}}$ を ${\overset{\textnormal{し}}{\text{占}}}$ める。 \hfill\break
The opinion that we should immediately fight holds the majority }
 
\par{${\overset{\textnormal{}}{\text{53. 学生}}}$ はまじめに ${\overset{\textnormal{}}{\text{勉強}}}$ すべきです。 \hfill\break
Students should study seriously. }
 
\par{${\overset{\textnormal{}}{\text{54. 若}}}$ い ${\overset{\textnormal{}}{\text{時}}}$ に、もっと ${\overset{\textnormal{}}{\text{勉強}}}$ するべきでした。 \hfill\break
I should have studied more when I was young. \hfill\break
 \hfill\break
 ${\overset{\textnormal{}}{\text{55. 彼女}}}$ を ${\overset{\textnormal{}}{\text{軽視}}}$ すべきではない。 \hfill\break
You shouldn't think lightly of her. }

\par{56. ${\overset{\textnormal{けんか}}{\text{喧嘩}}}$ すべきではなかった。 \hfill\break
I shouldn't have argued. }

\par{${\overset{\textnormal{}}{\text{57. 生}}}$ きるべきか、 ${\overset{\textnormal{}}{\text{死}}}$ ぬべきか、これが ${\overset{\textnormal{もんだいてん}}{\text{問題点}}}$ だ。 \hfill\break
To be or not or not be, that is the question. }

\par{58. ${\overset{\textnormal{みな}}{\text{皆}}}$ はもっと ${\overset{\textnormal{すいみん}}{\text{睡眠}}}$ を ${\overset{\textnormal{}}{\text{取}}}$ るべきだ。 \hfill\break
Everyone should get more sleep. }
 
\par{${\overset{\textnormal{}}{\text{59. 許}}}$ すべからざる ${\overset{\textnormal{}}{\text{行為}}}$ 。 \hfill\break
An action that should not be allowed\slash forgiven. }
 
\par{60. もっと ${\overset{\textnormal{}}{\text{本}}}$ を ${\overset{\textnormal{}}{\text{読}}}$ むべきです。 \hfill\break
You should read more books. }
 
\par{${\overset{\textnormal{}}{\text{61. 中国}}}$ に ${\overset{\textnormal{}}{\text{行}}}$ くべきではありませんよ。 \hfill\break
You should not go to China. }
 
\par{${\overset{\textnormal{}}{\text{62. 若}}}$ い ${\overset{\textnormal{}}{\text{時}}}$ に、もっと ${\overset{\textnormal{}}{\text{韓国語}}}$ を ${\overset{\textnormal{}}{\text{勉強}}}$ するべきだった。 \hfill\break
I should have studied Korean more when I was young. }
 
\par{63. よく考えるべきだ。 \hfill\break
You should consider it well. }

\par{64. 政府は失業者の ${\overset{\textnormal{ぜいふたん}}{\text{税負担}}}$ を ${\overset{\textnormal{めんじょ}}{\text{免除}}}$ すべきだ。 \hfill\break
The government should exempt the tax burden of the unemployed. }
 
\par{65. どう ${\overset{\textnormal{}}{\text{生}}}$ きるべきか。 \hfill\break
In what way should we live? }
 
\par{\textbf{Speech Style\slash Grammar Note }: In colloquial speech, だ may be dropped in ~べきだ. }
    