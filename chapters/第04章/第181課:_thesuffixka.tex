    
\chapter{The Suffix 化}

\begin{center}
\begin{Large}
第181課: The Suffix 化 
\end{Large}
\end{center}
 
\par{ Although “to become” is generally taken care of by the patterns になる in the intransitive sense and にする in the transitive sense, this is not the only way to go about expressing change with verbs. In the same way English has the suffixes “-ize,” “-ificate,” “-ify,” and “-ate,” Japanese has the suffix 化 that can attach to all sorts of nouns for the same effect. }

\par{ 化 is most commonly attached to Sino-Japanese words, but it is also frequently used with loan words from other languages. It may also seldom be used with native vocabulary. }

\begin{enumerate}

\item With Sino-Japanese words: 現代化 (modernization), 具体化 (materialization), 映画化 (making a book into a film), etc. \hfill\break

\item With Western language loans: グローバル化, カプセル化, イオン化, アメリカ化 \hfill\break

\item With native words: 畑化 
\end{enumerate}

\par{ Neologisms are also extremely easy to make with this, and so even if you were to memorize every current example of this that has been coined thus far, you might be surprised to see a totally new word coined with it tomorrow. }

\par{ As is with the case with any suffix, you will need to learn instances of them on a case by case basis. Even though the suffix 化 is used a lot, it\textquotesingle s not used with everything. For instance, you can\textquotesingle t say 必要化. Yet, you can say 必要性 (necessity), which utilizes another ending. However, you can\textquotesingle t look at this and say that suffix 化 can\textquotesingle t attach itself to things from adjectives. Examples like 多様化 demonstrate that that\textquotesingle s not true. }

\par{ In the example sentences below, you will find examples of the suffix 化 from very diverse backgrounds. All examples of this suffix can be used as stand alone nouns. They may also all be used as verbs. These resultant verbs may behave as intransitive, transitive or both intransitive and transitive verbs depending on the specific qualities of the word. }

\par{ Whenever you see a transitivity note that says you must change \emph{ }する to される to use the verb in an intransitive sense, the verb itself is not intrinsically intransitive. Therefore, you are simply creating the passive form of the verb itself. This means that you won\textquotesingle t have the ambiguity as to whether the passive nuance is literally meant or not as is the case with verbs that can either be intransitive or transitive. }

\par{ Whenever you see a transitivity note that says you must change \emph{ }する to させる to use the verb in a transitive sense, the verb itself is not intrinsically transitive. Therefore, you are simply creating the causative form of the verb itself. This means that you won\textquotesingle t have the ambiguity as to whether the causative nuance is literally meant or not as is the case with verbs that can either be intransitive or transitive. }
      
\section{Examples}
 
\par{1. ITの ${\overset{\textnormal{しんか}}{\text{進化}}}$ に ${\overset{\textnormal{ともな}}{\text{伴}}}$ って ${\overset{\textnormal{にちじょうせいかつ}}{\text{日常生活}}}$ のあらゆるものが ${\overset{\textnormal{きかいか}}{\text{機械化}}}$ \{して・されて\}きた。 \hfill\break
Following the evolution of IT, all sorts of things of everyday life have become mechanized. }

\par{\textbf{Transitivity Note }: The use of 機械化される for the intransitive use of this verb is done when one wishes to implicit hint at the agent. Meaning, in this example, people would have been actively working in the background to bring about the mechanization of everyday things via IT innovation. }

\par{2. ${\overset{\textnormal{さぎょう}}{\text{作業}}}$ は ${\overset{\textnormal{かんたん}}{\text{簡単}}}$ でも ${\overset{\textnormal{ひんど}}{\text{頻度}}}$ の ${\overset{\textnormal{たか}}{\text{高}}}$ い ${\overset{\textnormal{うんよう}}{\text{運用}}}$ ${\overset{\textnormal{たすく}}{\text{タスク}}}$ を ${\overset{\textnormal{じどうか}}{\text{自動化}}}$ した ${\overset{\textnormal{ばあい}}{\text{場合}}}$ 、 ${\overset{\textnormal{こうりつか}}{\text{効率化}}}$ が ${\overset{\textnormal{きたい}}{\text{期待}}}$ できます。 \hfill\break
In the situation that we automate the simple yet high-frequency operation tasks, we can expect optimization. }

\par{3. ${\overset{\textnormal{はっちゅうしょ}}{\text{発注書}}}$ の ${\overset{\textnormal{はっこう}}{\text{発行}}}$ が ${\overset{\textnormal{じどうか}}{\text{自動化}}}$ したことで、 ${\overset{\textnormal{しょうりょくか}}{\text{省力化}}}$ が ${\overset{\textnormal{すす}}{\text{進}}}$ んだ。 \hfill\break
By automatizing the issuance of work orders, it made progress in labor savings. }

\par{\textbf{Abbreviation Note }: In business, “work order” is frequently written as WO, following English standards. }

\par{4. もし ${\overset{\textnormal{ちゅうごく}}{\text{中国}}}$ が ${\overset{\textnormal{みんしゅか}}{\text{民主化}}}$ したらチベットやウイグル、 ${\overset{\textnormal{うち}}{\text{内}}}$ モンゴルは ${\overset{\textnormal{どくりつ}}{\text{独立}}}$ するんですか。 \hfill\break
If China were to become democratic, would Tibet, Uygur, and Inner Mongolia become independent? }

\par{5. ${\overset{\textnormal{ひづけ}}{\text{日付}}}$ の ${\overset{\textnormal{ひょうじ}}{\text{表示}}}$ を ${\overset{\textnormal{こくさいか}}{\text{国際化}}}$ しました。 \hfill\break
I\textquotesingle ve internationalized the display of the date. }

\par{6. ${\overset{\textnormal{にほん}}{\text{日本}}}$ は ${\overset{\textnormal{せいよう}}{\text{西洋}}}$ の ${\overset{\textnormal{せいじせいど}}{\text{政治制度}}}$ や ${\overset{\textnormal{ぶんか}}{\text{文化}}}$ を ${\overset{\textnormal{ゆにゅう}}{\text{輸入}}}$ して、( ${\overset{\textnormal{くに}}{\text{国}}}$ を) ${\overset{\textnormal{きんだいか}}{\text{近代化}}}$ しました。 \hfill\break
By importing a Western political system and Western culture, Japan modernized (its country). }

\par{7. ${\overset{\textnormal{じどううんてん}}{\text{自動運転}}}$ は、 ${\overset{\textnormal{かんぜん}}{\text{完全}}}$ に ${\overset{\textnormal{ごじゅう}}{\text{50}}}$ ${\overset{\textnormal{ねんご}}{\text{年後}}}$ に ${\overset{\textnormal{じつようか}}{\text{実用化}}}$ するでしょう。 \hfill\break
Automated driving will likely become completely practical fifty years from now. }

\par{8. ${\overset{\textnormal{おうしゅう}}{\text{欧州}}}$ は ${\overset{\textnormal{にいんせい}}{\text{二院制}}}$ のメリットを ${\overset{\textnormal{ごうりか}}{\text{合理化}}}$ している。 \hfill\break
Europe rationalizes the merits of the bicameral system. }

\par{\textbf{Transitivity Note }: In order to use 合理化する in an intransitive sense, you must use the form 合理化される. }

\par{9. ${\overset{\textnormal{ひししょくぶつ}}{\text{被子植物}}}$ は ${\overset{\textnormal{はくあきいこう}}{\text{白亜紀以降}}}$ に ${\overset{\textnormal{きゅうそく}}{\text{急速}}}$ に ${\overset{\textnormal{たようか}}{\text{多様化}}}$ しました。 \hfill\break
Flowering plants rapidly diversified from the Cretaceous period onward. }

\par{\textbf{Transitivity Note }: In order to use 多様化する in a transitive sense, you must use the form 多様化させる. }

\par{10. ${\overset{\textnormal{じだい}}{\text{時代}}}$ と ${\overset{\textnormal{とも}}{\text{共}}}$ に、 ${\overset{\textnormal{ことば}}{\text{言葉}}}$ の ${\overset{\textnormal{つか}}{\text{使}}}$ い ${\overset{\textnormal{かた}}{\text{方}}}$ や ${\overset{\textnormal{いみ}}{\text{意味}}}$ (など)が ${\overset{\textnormal{へんか}}{\text{変化}}}$ します。 \hfill\break
The usage and meanings of words change with time. }

\par{11. スーパーのウズラの ${\overset{\textnormal{たまご}}{\text{卵}}}$ には ${\overset{\textnormal{ゆうせいらん}}{\text{有精卵}}}$ が ${\overset{\textnormal{ま}}{\text{混}}}$ じっていて、ちゃんと ${\overset{\textnormal{あたた}}{\text{温}}}$ めれば ${\overset{\textnormal{ふか}}{\text{孵化}}}$ する ${\overset{\textnormal{かのうせい}}{\text{可能性}}}$ があります。 \hfill\break
Fertilized eggs are mixed with the quail eggs at the supermarket, and when you properly warm them up, there\textquotesingle s the possibility (of one) hatching. }

\par{\textbf{Spelling Note }: ウズラ may seldom be spelled as 鶉. }

\par{12. ヒトとチンパンジーが ${\overset{\textnormal{たねぶんか}}{\text{種分化}}}$ したのは ${\overset{\textnormal{ごひゃく}}{\text{500}}}$ ${\overset{\textnormal{まんねん}}{\text{万年}}}$ から ${\overset{\textnormal{ななひゃく}}{\text{700}}}$ ${\overset{\textnormal{まんねん}}{\text{万年}}}$ くらい ${\overset{\textnormal{まえ}}{\text{前}}}$ と ${\overset{\textnormal{い}}{\text{言}}}$ われています。 \hfill\break
It is said that humans and chimpanzees speciated from about 5 million to 7 million years ago. }

\par{13. ${\overset{\textnormal{さんか}}{\text{酸化}}}$ した ${\overset{\textnormal{あぶら}}{\text{油}}}$ は ${\overset{\textnormal{きけん}}{\text{危険}}}$ です。 \hfill\break
Oxidated oil is dangerous. }

\par{\textbf{Transitivity Note }: If the agent of oxidation is actively oxidizing the object, then 参加させる should be used. }

\par{14. セルロースは ${\overset{\textnormal{たんきかん}}{\text{短期間}}}$ で ${\overset{\textnormal{えきか}}{\text{液化}}}$ した。 \hfill\break
The cellulose liquefied in a short period of time. }

\par{\textbf{Transitivity Note }: If the agent of liquification is actively liquefying the object, then 液化させる should be used. }

\par{15. ${\overset{\textnormal{はいすい}}{\text{排水}}}$ を ${\overset{\textnormal{じょうか}}{\text{浄化}}}$ して ${\overset{\textnormal{さいりよう}}{\text{再利用}}}$ しています。 \hfill\break
We reuse drainage water by purifying it. }

\par{\textbf{Transitivity Note }: In order to use 浄化する in an intransitive sense, you must use the form 浄化される. }

\par{16. ${\overset{\textnormal{すな}}{\text{砂}}}$ は ${\overset{\textnormal{がんせき}}{\text{岩石}}}$ が ${\overset{\textnormal{ふうか}}{\text{風化}}}$ してできた ${\overset{\textnormal{もの}}{\text{物}}}$ である。 \hfill\break
Sand is what\textquotesingle s made from rocks eroding. }

\par{17. 他の ${\overset{\textnormal{みんぞく}}{\text{民族}}}$ を ${\overset{\textnormal{どうか}}{\text{同化}}}$ \{する・させる\}。 \hfill\break
To assimilate other ethnic groups. }

\par{\textbf{Transitivity Note }: Using 同化させる implies far more forceful assimilation. }

\par{\textbf{Reading Note }: 他 may be read as either た or ほか. }

\par{18. ${\overset{\textnormal{けいえいたいせい}}{\text{経営体制}}}$ を ${\overset{\textnormal{きょうか}}{\text{強化}}}$ しました。 \hfill\break
We\textquotesingle ve intensified our management structure. }

\par{\textbf{Transitivity Note }: In order to use 強化する in an intransitive sense, you must use the form 強化される. }

\par{19. ${\overset{\textnormal{ほそう}}{\text{舗装}}}$ が ${\overset{\textnormal{れっか}}{\text{劣化}}}$ している ${\overset{\textnormal{ほどう}}{\text{歩道}}}$ を ${\overset{\textnormal{なお}}{\text{直}}}$ してほしい。 \hfill\break
I\textquotesingle d like the sidewalks with the pavement deteriorating fixed. }

\par{\textbf{Transitivity Note }: In order to use 劣化する in a transitive sense, you must use the form 劣化させる. }

\par{20. ${\overset{\textnormal{にんげん}}{\text{人間}}}$ が ${\overset{\textnormal{くさ}}{\text{草}}}$ を ${\overset{\textnormal{しょうか}}{\text{消化}}}$ できないのは ${\overset{\textnormal{なぜ}}{\text{何故}}}$ ですか。 \hfill\break
Why is it that humans can\textquotesingle t digest grass? }

\par{\textbf{Transitivity Note }: In order to use 消化する in an intransitive sense, you must use the form 消化される. }

\par{21. ${\overset{\textnormal{た}}{\text{田}}}$ んぼをそのまま ${\overset{\textnormal{はたけか}}{\text{畑化}}}$ しても、 ${\overset{\textnormal{じゅうねんどしつ}}{\text{重粘土質}}}$ の ${\overset{\textnormal{どじょう}}{\text{土壌}}}$ なら ${\overset{\textnormal{さくもつ}}{\text{作物}}}$ の ${\overset{\textnormal{ね}}{\text{根}}}$ が ${\overset{\textnormal{ちっそく}}{\text{窒息}}}$ して ${\overset{\textnormal{か}}{\text{枯}}}$ れてしまいます。 \hfill\break
Even if you just change the rice paddy field into a (regular) field, if the soil is heavy with clay, the roots of your crop will suffocate and die. }

\par{\textbf{Spelling Note }: 田んぼ may also seldom be spelled as 田圃. }

\par{22. ${\overset{\textnormal{ないぶひばく}}{\text{内部被曝}}}$ は ${\overset{\textnormal{あっか}}{\text{悪化}}}$ するでしょう。 \hfill\break
Internal exposure will likely worsen. }

\par{\textbf{Transitivity Note }: In order to use 悪化する in a transitive sense, you must use the form 悪化させる. }

\par{23. インターネット ${\overset{\textnormal{じだい}}{\text{時代}}}$ には、 ${\overset{\textnormal{へいき}}{\text{兵器}}}$ も ${\overset{\textnormal{じょうほうか}}{\text{情報化}}}$ しているのです。 \hfill\break
In the Internet age, weapons are also being computerized. }

\par{24. ${\overset{\textnormal{がくじゅつようご}}{\text{学術用語}}}$ が ${\overset{\textnormal{いっぱんか}}{\text{一般化}}}$ して ${\overset{\textnormal{いみ}}{\text{意味}}}$ が ${\overset{\textnormal{ひろ}}{\text{広}}}$ がることが ${\overset{\textnormal{おお}}{\text{多}}}$ い。 \hfill\break
There are many instances in which technical terms become generalized and expand in meaning. }

\par{25. ${\overset{\textnormal{でんりょくこうり}}{\text{電力小売}}}$ を ${\overset{\textnormal{じゆうか}}{\text{自由化}}}$ しても ${\overset{\textnormal{りょうきん}}{\text{料金}}}$ は ${\overset{\textnormal{さ}}{\text{下}}}$ がらない。 \hfill\break
Even if you were to liberalize power retailing, fares won\textquotesingle t go down. }

\par{\textbf{Transitivity Note }: In order to use 自由化する in an intransitive sense, you must use the form \emph{ }自由化される. }

\par{26. ${\overset{\textnormal{みかく}}{\text{味覚}}}$ は ${\overset{\textnormal{かんぜん}}{\text{完全}}}$ に ${\overset{\textnormal{にほんか}}{\text{日本化}}}$ してるよね。 \hfill\break
Your sense of taste has become completely japanified, huh. }

\par{\textbf{Transitivity Note }: In order to use 日本化する in a transitive sense, you must use the form \emph{ }日本化させる. }

\par{27. Googleという ${\overset{\textnormal{たんご}}{\text{単語}}}$ は、 ${\overset{\textnormal{いっぱんどうしか}}{\text{一般動詞化}}}$ している。 \hfill\break
The word “Google” has turned into a standard verb. }

\par{28. ${\overset{\textnormal{のう}}{\text{脳}}}$ を ${\overset{\textnormal{かっせいか}}{\text{活性化}}}$ して ${\overset{\textnormal{べんきょう}}{\text{勉強}}}$ の ${\overset{\textnormal{こうりつ}}{\text{効率}}}$ をアップさせましょう。 \hfill\break
Increase the efficiency of your studying by stimulating your brain! }

\par{29. ${\overset{\textnormal{かんがいよう}}{\text{灌漑用}}}$ に ${\overset{\textnormal{みず}}{\text{水}}}$ を ${\overset{\textnormal{つか}}{\text{使}}}$ い ${\overset{\textnormal{す}}{\text{過}}}$ ぎて ${\overset{\textnormal{さばくか}}{\text{砂漠化}}}$ しているところもあります。 \hfill\break
There are also places that are turning into desert from overusing water for irrigation. }

\par{\textbf{Transitivity Note }: In order to use 砂漠化する in a transitive sense, you must use the form 砂漠化させる. }

\par{30. ${\overset{\textnormal{すいどうじぎょう}}{\text{水道事業}}}$ が ${\overset{\textnormal{みんえいか}}{\text{民営化}}}$ したら ${\overset{\textnormal{すいどうだい}}{\text{水道代}}}$ は ${\overset{\textnormal{はんぶん}}{\text{半分}}}$ になるって ${\overset{\textnormal{ほんとう}}{\text{本当}}}$ ですか。 \hfill\break
If the water supply were to privatize, would water bills really go down to half what they are? }

\par{31. ネット ${\overset{\textnormal{つうはん}}{\text{通販}}}$ の ${\overset{\textnormal{にもつ}}{\text{荷物}}}$ の ${\overset{\textnormal{ぞうか}}{\text{増加}}}$ と ${\overset{\textnormal{ひとでぶそく}}{\text{人手不足}}}$ で ${\overset{\textnormal{じゅうぎょういん}}{\text{従業員}}}$ の ${\overset{\textnormal{ちょうじかんろうどう}}{\text{長時間労働}}}$ が ${\overset{\textnormal{しんこくか}}{\text{深刻化}}}$ している。 \hfill\break
Long-hour labor is becoming more severe due to increases in packages from e-commerce and labor shortages. }

\par{\textbf{Transitivity Note }: In order to use 深刻化する in a transitive sense, you must use the form 深刻化させる. }

\par{32. ${\overset{\textnormal{とうしば}}{\text{東芝}}}$ の ${\overset{\textnormal{りんじかぶぬしそうかい}}{\text{臨時株主総会}}}$ で、 ${\overset{\textnormal{はんどうたいじぎょう}}{\text{半導体事業}}}$ を ${\overset{\textnormal{ぶんしゃか}}{\text{分社化}}}$ することについて、 ${\overset{\textnormal{さん}}{\text{3}}}$ ${\overset{\textnormal{ぷん}}{\text{分}}}$ の ${\overset{\textnormal{に}}{\text{2}}}$ ${\overset{\textnormal{いじょう}}{\text{以上}}}$ の ${\overset{\textnormal{かぶぬし}}{\text{株主}}}$ が ${\overset{\textnormal{さんせい}}{\text{賛成}}}$ して ${\overset{\textnormal{しょうにん}}{\text{承認}}}$ された。 \hfill\break
At the Special General Meeting of Shareholders, over two thirds of the shareholders supported the splitting off of the semiconductor business, which then (the split-off) was approved. }

\par{33. ${\overset{\textnormal{かごしまけん}}{\text{鹿児島県}}}$ の ${\overset{\textnormal{あまみおおしま}}{\text{奄美大島}}}$ と ${\overset{\textnormal{とくのしま}}{\text{徳之島}}}$ にだけ ${\overset{\textnormal{せいそく}}{\text{生息}}}$ する ${\overset{\textnormal{くに}}{\text{国}}}$ の ${\overset{\textnormal{とくべつてんねんきねんぶつ}}{\text{特別天然記念物}}}$ のアマミノクロウサギが ${\overset{\textnormal{やせいか}}{\text{野生化}}}$ した ${\overset{\textnormal{ねこ}}{\text{猫}}}$ に ${\overset{\textnormal{おそ}}{\text{襲}}}$ われるケースが ${\overset{\textnormal{あいつ}}{\text{相次}}}$ いでいる。 \hfill\break
Cases of the Amamino rabbit, a nationally protected species which only inhabits the islands of Amami Ōshima and Tokunoshima, being attacked by feral cats are happening one after another. }

\par{\textbf{Transitivity Note }: In order to use 野生化する in a transitive sense, you must use the form 野生化させる. }

\par{34. ライバルとの ${\overset{\textnormal{さべつか}}{\text{差別化}}}$ で ${\overset{\textnormal{じぎょう}}{\text{事業}}}$ を ${\overset{\textnormal{さら}}{\text{更}}}$ に ${\overset{\textnormal{きょうか}}{\text{強化}}}$ することが ${\overset{\textnormal{かだい}}{\text{課題}}}$ となっている。 \hfill\break
Further strengthening business via differentiation from rivals has become the task at hand. }

\par{35. ${\overset{\textnormal{こうれいか}}{\text{高齢化}}}$ や ${\overset{\textnormal{じんこう}}{\text{人口}}}$ の ${\overset{\textnormal{げんしょう}}{\text{減少}}}$ が ${\overset{\textnormal{すす}}{\text{進}}}$ んでいたところにイノシシによる ${\overset{\textnormal{たはた}}{\text{田畑}}}$ の ${\overset{\textnormal{ひがい}}{\text{被害}}}$ が ${\overset{\textnormal{あいつ}}{\text{相次}}}$ いだ。 \hfill\break
Just as population ageing and population decreasing has progressed, damage to fields by wild boars have occurred one after another. }

\par{\textbf{Spelling Note }: イノシシ \emph{ }may also be spelled as 猪. }

\par{\textbf{Transitivity Note }: In order to use 高齢化する in a transitive sense, you must use the form高齢化させる. }

\par{36. ${\overset{\textnormal{こんかい}}{\text{今回}}}$ の ${\overset{\textnormal{じこ}}{\text{事故}}}$ を ${\overset{\textnormal{う}}{\text{受}}}$ けて、 ${\overset{\textnormal{とちぎけんきょういくいいんかい}}{\text{栃木県教育委員会}}}$ は、ビーコンの ${\overset{\textnormal{ぎむか}}{\text{義務化}}}$ を ${\overset{\textnormal{けんとう}}{\text{検討}}}$ することにしている。 \hfill\break
In response to this incident, the Tochigi Prefecture Board of Education is making it a point to consider the mandating of beacons. }

\par{37. ${\overset{\textnormal{にほんいしん}}{\text{日本維新}}}$ の ${\overset{\textnormal{かい}}{\text{会}}}$ は、 ${\overset{\textnormal{きょねんはっぴょう}}{\text{去年発表}}}$ した ${\overset{\textnormal{けんぽうかいせいげんあん}}{\text{憲法改正原案}}}$ に、 ${\overset{\textnormal{きょういくむしょうか}}{\text{教育無償化}}}$ を ${\overset{\textnormal{さか}}{\text{盛}}}$ り ${\overset{\textnormal{こ}}{\text{込}}}$ んだ。 \hfill\break
The Japan Restoration Party incorporated making education free of charge in their original draft for constitutional reform, which they announced last year. }

\par{38. ${\overset{\textnormal{げんきん}}{\text{現金}}}$ をデジタル ${\overset{\textnormal{か}}{\text{化}}}$ しないと ${\overset{\textnormal{げんだい}}{\text{現代}}}$ の ${\overset{\textnormal{けっさいしゅだん}}{\text{決済手段}}}$ として ${\overset{\textnormal{い}}{\text{生}}}$ き ${\overset{\textnormal{のこ}}{\text{残}}}$ れない。 \hfill\break
If we don\textquotesingle t digitize cash, it won\textquotesingle t be able to last as a modern means of payment. }

\par{39. ${\overset{\textnormal{ほうそう}}{\text{包装}}}$ を ${\overset{\textnormal{かんそか}}{\text{簡素化}}}$ し、 ${\overset{\textnormal{せんでんひ}}{\text{宣伝費}}}$ も ${\overset{\textnormal{さくげん}}{\text{削減}}}$ するなどして ${\overset{\textnormal{おおて}}{\text{大手}}}$ メーカーより ${\overset{\textnormal{やす}}{\text{安}}}$ い ${\overset{\textnormal{しょうひん}}{\text{商品}}}$ を ${\overset{\textnormal{はんばい}}{\text{販売}}}$ しようとしている。 \hfill\break
(They) are trying to market merchandise cheaper than major manufacturers by simplifying packaging, reducing advertising expenses, and other means. }

\par{40. ${\overset{\textnormal{とうしっこうぶ}}{\text{党執行部}}}$ では、 ${\overset{\textnormal{げんぱつ}}{\text{原発}}}$ の ${\overset{\textnormal{かどう}}{\text{稼働}}}$ をゼロにする ${\overset{\textnormal{もくひょう}}{\text{目標}}}$ の ${\overset{\textnormal{じき}}{\text{時期}}}$ を「 ${\overset{\textnormal{にせんさんじゅうねん}}{\text{2030}}}$ ${\overset{\textnormal{ねん}}{\text{年}}}$ 」に ${\overset{\textnormal{じじつじょう}}{\text{事実上}}}$ 、 ${\overset{\textnormal{まえだお}}{\text{前倒}}}$ しして ${\overset{\textnormal{ほうあんか}}{\text{法案化}}}$ することを ${\overset{\textnormal{けんとう}}{\text{検討}}}$ していた。 \hfill\break
(They) had been considering legislating the goal period of reducing nuclear operations to zero, in actuality, ahead of schedule to 2030 among the executives of the party. }

\par{41. サイバー ${\overset{\textnormal{こうげき}}{\text{攻撃}}}$ もますます ${\overset{\textnormal{こうみょうか}}{\text{巧妙化}}}$ している。 \hfill\break
Cyber attacks are also gradually becoming more sophisticated. }

\par{\textbf{Transitivity Note }: In order to use 巧妙化する in a transitive sense, you must use the form 巧妙化させる. }

\par{42. ${\overset{\textnormal{けんこうじょうたい}}{\text{健康状態}}}$ などがデータ ${\overset{\textnormal{か}}{\text{化}}}$ できるようになった。 \hfill\break
Health status and the like have become convertible into data. }

\par{43. ${\overset{\textnormal{ぎょうむ}}{\text{業務}}}$ の ${\overset{\textnormal{ぶんさんか}}{\text{分散化}}}$ を ${\overset{\textnormal{はか}}{\text{図}}}$ っています。 \hfill\break
We are considering decentralizing business-work. }

\par{\textbf{Transitivity Note }: In order to use 分散化する in an intransitive sense, you must use the form 分散化される. }

\par{44. ${\overset{\textnormal{さいはいたつ}}{\text{再配達}}}$ の ${\overset{\textnormal{ゆうりょうか}}{\text{有料化}}}$ を ${\overset{\textnormal{かいけつさく}}{\text{解決策}}}$ に ${\overset{\textnormal{あ}}{\text{上}}}$ げる ${\overset{\textnormal{いけん}}{\text{意見}}}$ も ${\overset{\textnormal{めだ}}{\text{目立}}}$ った。 \hfill\break
Opinions calling for putting forward the charging of redelivering into the solution strategy also stood out. }

\par{45. ここ ${\overset{\textnormal{すうねん}}{\text{数年}}}$ は ${\overset{\textnormal{かかくきょうそう}}{\text{価格競争}}}$ の ${\overset{\textnormal{げきか}}{\text{激化}}}$ などで ${\overset{\textnormal{ぎょうせき}}{\text{業績}}}$ は ${\overset{\textnormal{の}}{\text{伸}}}$ び ${\overset{\textnormal{なや}}{\text{悩}}}$ んでいる。 \hfill\break
In these past several years, performance has been making little progress due to the intensification of price competition. }

\par{\textbf{Transitivity Note }: In order to use 激化する in a transitive sense, you must use the form 激化させる. }

\par{\textbf{Reading Note }: 激化 may also be read as げっか. }
    