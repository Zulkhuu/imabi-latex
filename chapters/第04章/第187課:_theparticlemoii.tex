    
\chapter{The Particle も II}

\begin{center}
\begin{Large}
第187課: The Particle も II 
\end{Large}
\end{center}
 
\par{ The particle も itself is a rather easy particle. If it doesn't mean “also” literally, it plays some sort of emphatic role in the sentence. The grammar that often accompanies it, however, needs a more detailed analysis. }
      
\section{連用形+もしない}
 
\par{ The purpose of this pattern is to emphasize the substantial meaning of the verb and forcefully negate it. Not surprisingly, you aren\textquotesingle t going to find this in honorifics, but you will find it in negative imperatives. }

\par{1. 働きもしないでぐずぐずするな。 \hfill\break
Don\textquotesingle t loiter around and not work! }

\par{2. 法案を読みもしないで論評することなんて馬鹿げた行動にすぎない。 \hfill\break
Commenting without having read the legislature is no more than a foolish act. }

\par{3. 切迫した現場にいもしないものが、状況判断もせず原則論のみで何をいうか。 \hfill\break
What does someone who isn't even in an urgent spot say with mere principle without assessing the situation? \hfill\break
From 光の雨 by 立松和平. }

\par{ Similarly, you can see ~なくもない. This is essentially a double negative turning into a positive. }

\par{知らなくもない。 \hfill\break
it's nothing that I don't know. }
      
\section{Light Exclamation}
 
\par{ This light exclamation is used to show something that may very well be the case all the time, but there is a certain kind of surprise about it all. This exclamation can be inclusive or something unpredictable. So, you have to view this exclamation in light of も\textquotesingle s multifaceted use. }

\par{4. お前も馬鹿だね。 \hfill\break
Wow, you're stupid. }

\par{ The translation is liberal in its interpretation of the sentence, but the exclamation does not take away from the fact that the speaker is showing surprise in the addressee being an idiot, which would relate the addressee to stupid people. }

\par{5. 今や春もたけなわ(だ)。 (Set Phrase) \hfill\break
Spring is now in full swing. }

\par{6. クリスマスも終わりですね。 \hfill\break
Christmas is over, isn't it? }

\par{7. あいつも太ったな。 \hfill\break
He\textquotesingle s gotten fat too, hasn't he? }

\par{8. 私も秋から成人です。 \hfill\break
I too will be an adult in fall. }

\par{ In these sentences, the speaker shows exclamation at recognizing that there has been a change in circumstance or attribute. One thing that has to be understood, though, is that for there to be a sense of exclamation, there has to be a second realization of the matter and that it was not in one\textquotesingle s supposition to that point. It\textquotesingle s about taking things in all at once and realizing the change. }

\par{ It may also be the fact that exclamation is brought on from the fact that something is counter to the norm. }

\par{9. 「あのばあちゃんもマラソンに参加するの?」「あ、ほんと!ばあちゃんも元気一杯だね」 \hfill\break
“That old lady\textquotesingle s going to be in the marathon?” “Ah, she is! Well, isn\textquotesingle t she full of energy?”. }

\par{ What if you\textquotesingle re showing that something\textquotesingle s useless? In such a case, although there may be something unexpected and be found akin to something else, no exclamation is inherently expressed by も. }

\par{10. あんなに頭がよかったのに、今度の試験では健も次郎もだめだ。 \hfill\break
Although they\textquotesingle re so smart, Ken and Jiro both won\textquotesingle t do well this next exam. }

\par{ So, in order for the exclamation function to work, there has to be an atemporal relation expressed. Otherwise, も may just be showing 同類性 (similarity) and or 意外性 (unpredictability). }
      
\section{The Conjunctive Particle も}
 
\par{ Although normally replaced by ても, the conjunctive particle も means "even though" or "no matter". This is typically old-fashioned except for instances like よくも, which you'll see below in the examples. }

\par{11. どんなに ${\overset{\textnormal{おお}}{\text{多}}}$ \textbf{くも }${\overset{\textnormal{じゅうにん}}{\text{十人}}}$ までだ。(Old-fashioned) \hfill\break
 \textbf{No matter }how many there are, there are going to be only up to 10 people. }

\par{12. 今日に ${\overset{\textnormal{いた}}{\text{至}}}$ る \textbf{も }${\overset{\textnormal{かんせい}}{\text{完成}}}$ を見ない。(Old-fashioned) \hfill\break
 \textbf{Even }if it reaches today, we won't see completion. }

\par{\textbf{Grammar Note }: This is actually an instance of も after the 連体形 of a verb, but in this set instance, 至るも is equivalent to 至っても. }

\par{13. よくも人を ${\overset{\textnormal{ばか}}{\text{馬鹿}}}$ にする。 \hfill\break
To dare make people feel stupid. }

\par{\textbf{Phrase Note }: よくも is used to criticize someone's action(s). Ex., よくもそんなことを!= How dare you! It can also show surprise\slash astonishment at someone's action(s). }

\par{\textbf{漢字 Note }: よくも can be written in 漢字 as 善くも. }

\par{14. よくもそんなに ${\overset{\textnormal{しゃべ}}{\text{喋}}}$ ることがありましたね。 \hfill\break
It's amazing that he had so much to talk about! }
      
\section{~も~たり}
 
\par{ "\dothyp{}\dothyp{}\dothyp{}も\dothyp{}\dothyp{}\dothyp{}たり" repeats to make expressions like the following. This usage is limited, but it is important to realize that this is an extension of も's use of showing 意外性. The result is ironically unexpected. }

\par{15. 走りも走ったり ${\overset{\textnormal{よんまんぽ}}{\text{四万歩}}}$ 。 \hfill\break
Running and running, it was (only) 40,000 steps. }
      
\section{Peculiar Set Phrases}
 
\par{ The particle も usually goes after something nominal, excluding its usage after て or the 連用形 of something. Consider, though, the following exceptions in set phrases. }

\par{16. ${\overset{\textnormal{お}}{\text{老}}}$ いも ${\overset{\textnormal{わか}}{\text{若}}}$ きも ${\overset{\textnormal{かんどう}}{\text{感動}}}$ した。 \hfill\break
Everyone was moved. \hfill\break
Literally: Even the young and old were moved. }

\par{\textbf{Old Grammar Note }: In the past, the 連体形 of verbs and adjectives could be used as nominal phrases and be followed by particles. }

\par{17. ${\overset{\textnormal{す}}{\text{酸}}}$ いも ${\overset{\textnormal{あま}}{\text{甘}}}$ いも ${\overset{\textnormal{か}}{\text{噛}}}$ み ${\overset{\textnormal{わ}}{\text{分}}}$ ける。 \hfill\break
To distinguish between anything. \hfill\break
Literally: To distinguish between sour and sweet. }

\par{\textbf{Grammar Note }: This phrase involves the modern 連体形, which is rather intriguing given the fact that you really can't take this old grammatical construction and apply it at will. }

\par{18. 聞くも ${\overset{\textnormal{なみだ}}{\text{涙}}}$ 、 ${\overset{\textnormal{かた}}{\text{語}}}$ るも涙の ${\overset{\textnormal{ものがたり}}{\text{物語}}}$ 。 \hfill\break
A story that when heard brings tears and when told brings tears. }

\par{19. もう勝ったも同然だ。 \hfill\break
It's almost the same as winning. }

\par{\textbf{Grammar Note }: ~も同然だ = As good as\slash almost the same as. Similar to other phrases in Japanese like ~に違いない, this phrase just tags along after something without any need to fix any grammatical connection before it. }
    