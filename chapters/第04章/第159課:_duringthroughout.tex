    
\chapter{~中}

\begin{center}
\begin{Large}
第159課: ~中: During\slash Throughout 
\end{Large}
\end{center}
 
\par{ You will often see ~中 used in time phrases, and it is typically translated as during or throughout. }
      
\section{~中}
 
\par{ Although both are written as 中, the suffixes ~ちゅう and ~じゅう are slightly different. }

\par{~ちゅう means "during" in the sense of "under way". It becomes ~じゅう with 今日, 昨日, etc. It may also be used to show duration of condition (Ex. お祈り中 = At\slash in prayer). It can also mean under(going)――Ex. 試験中 = undergoing an exam. }

\par{~じゅう can be used to show duration as well. ~間中(ずっと) is a frequently used phrase and means "during\slash while". It helps when you want to use ~じゅう when you can use 間. This shows start to finish. Phrases it can directly attach to include things such as ${\overset{\textnormal{ひとばん}}{\text{一晩}}}$ , ${\overset{\textnormal{いちねん}}{\text{一年}}}$ , etc. ~じゅう(に) may also mean "throughout" but in terms of distance and place.  For instance, say you get stung all over your body by bees. You can use 体じゅう to describe where you were stung. }

\par{Connecting Note: ~中 can attach to either Sino-Japanese or native words. However, the etymology of the word it is used with will not help you in using the right reading. You have to learn how it is read one phrase at a time. }

\begin{center}
 Examples 
\end{center}

\par{1. 弟は ${\overset{\textnormal{はるやす}}{\text{春休}}}$ み ${\overset{\textnormal{ちゅう}}{\text{中}}}$ 遊んでばかり\{でした・いた\}。 \hfill\break
My brother just played all throughout spring break. }

\par{Grammar Note: ~てばかりです shows incessant action. }

\par{2. ${\overset{\textnormal{きょうじゅう}}{\text{今日中}}}$ に ${\overset{\textnormal{しあ}}{\text{仕上}}}$ げてください。 \hfill\break
Please finish it up within (the end of) today. }

\par{3. ${\overset{\textnormal{きゅうぎょうちゅう}}{\text{休業中}}}$ の ${\overset{\textnormal{えいぎょうしょ}}{\text{営業所}}}$ \hfill\break
 A business office that's closed for the holidays }

\par{4. ${\overset{\textnormal{からだじゅう}}{\text{体中}}}$ が ${\overset{\textnormal{いた}}{\text{痛}}}$ んだ。 \hfill\break
It hurt all over my body. }

\par{5. ${\overset{\textnormal{みつりょう}}{\text{密漁}}}$ ${\overset{\textnormal{せん}}{\text{船}}}$ が ${\overset{\textnormal{そうぎょうちゅう}}{\text{操業中}}}$ に ${\overset{\textnormal{てん}}{\text{転}}}$ ${\overset{\textnormal{ぷく}}{\text{覆}}}$ した。 \hfill\break
A poaching ship capsized in operation. }

\par{6. ${\overset{\textnormal{ひゃく}}{\text{百}}}$ ${\overset{\textnormal{にんちゅう}}{\text{人中}}}$ ${\overset{\textnormal{はんすう}}{\text{半数}}}$ ${\overset{\textnormal{ひなん}}{\text{避難}}}$ しました。 \hfill\break
Out of the 100 people, half evacuated. }

\par{\textbf{Meaning Note }: You will often see ~中 used in the same sense as in Ex. 6. to mean "out of\dothyp{}\dothyp{}\dothyp{}". Note that it is attached to the \emph{counter }phrase 百人. }

\par{\textbf{Phrase Note }: One common phrase which is slightly grammatically questionable is 発売中. This is often used to emphasize having started to sell something. However, the definition of 発売 is "to begin selling". As you can see, it is an instantaneous verb, and instantaneous verbs normally should never take 中. For instance, you can't say 死亡中, 卒業中, or 結婚中. You can say, though, 婚約中 (engaged) and 出願中 (under application). These phrases are acceptable because there is an end to the state the verb brings about. So, while 発売中 is typically acceptable when the product has not been out long, after a long period of time, it becomes very strange Japanese to anyone. }

\par{ The following phrases all have ~中 with the reading ちゅう except 世界中, which is read as せかいじゅう. Remember, when ~中 means "during", it is read as ちゅう, and when it means "throughout", it is read as ~じゅう. }

\begin{ltabulary}{|P|P|P|P|P|P|}
\hline 

今年中 & During the year & 午前中 & During the morning & 今月中 & During the month \\ \cline{1-6}

来月中 & Within the next month & 在学中 & In schooling & 世界中 & Throughout the world \\ \cline{1-6}

販売中 & On sale (not as in discount) & 存命中 & While one is still alive & 空気中 & In the air \\ \cline{1-6}

\end{ltabulary}
\hfill\break
    