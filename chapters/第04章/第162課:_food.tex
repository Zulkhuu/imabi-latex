    
\chapter{Food}

\begin{center}
\begin{Large}
第162課: Food 
\end{Large}
\end{center}
 
\par{ Japanese food, ${\overset{\textnormal{にほんりょうり}}{\text{日本料理}}}$ ・ ${\overset{\textnormal{わしょく}}{\text{和食}}}$ ・日本食, is one of the most interesting things about Japan. At the end of this lesson, you will be able to go to a Japanese restaurant ( ${\overset{\textnormal{}}{\text{日本料理屋}}}$ ) and talk about what's in your food, how to ask for your food, etc. }
      
\section{At a Restaurant}
 
\begin{center}
 \textbf{お品書き }(Menu) 
\end{center}
 
\par{お好み焼き Okonomiyaki 700¥     天丼 Tendon 780¥     寿司 Sushi 1400¥   すき焼き Sukiyaki 1200¥ \hfill\break
刺し身 Sashimi 1350¥     狐そば Kitsune soba 700¥   狐うどん Kitsune udon 700¥ サラダ Salad 400¥ \hfill\break
お握りずし Onigirizushi 1000¥    ざるそば Zaru soba 550¥     しゃぶしゃぶ Shabushabu 1250¥ }
 
\par{\textbf{お飲み物 }}
 
\par{ビール Beer 310¥  お茶 Tea 300¥    水 Water 100¥    コーク Coke 250¥ \hfill\break
Qoo  300¥     焼酎 Shochu 300¥     ミルク Milk 250¥ }
 
\par{\textbf{デザート }}
 
\par{アイスクリーム Ice cream 240¥     チョコレートセーキ Chocolate shake 250¥ \hfill\break
アップルパイ Apple pie 300¥     チーズケーキ Cheese cake 350¥ }
 
\par{\textbf{Cultural Differences }: In the average Japanese restaurant, when food is done, it is sent out immediately regardless of whether or not the other people's food in one's party is ready. Refills are rare. Coffee might be refilled, but you may get charged double. When sitting down you are given a wet towel called a おしぼり to wipe your hands, a menu, and some tea and water. }

\begin{center}
 \textbf{Useful Expressions }
\end{center}

\par{1. メニューを ${\overset{\textnormal{}}{\text{見}}}$ せてください。 \hfill\break
Please show me a menu. }
 
\par{2. お ${\overset{\textnormal{なか}}{\text{腹}}}$ が ${\overset{\textnormal{あ}}{\text{空}}}$ きましたか。 \hfill\break
Are you hungry? }

\begin{ltabulary}{|P|P|P|P|P|P|P|P|}
\hline 

Bad & まずい & Sour & すっぱい & Sweet & 甘い & Bitter & 苦い \\ \cline{1-8}

Weak; thin & 薄い & Strong; thick & 濃い & Hot & 熱い & Cold & 冷たい \\ \cline{1-8}

\end{ltabulary}

\par{3. Xがおいしそうですね。 \hfill\break
X looks good. }
 
\par{${\overset{\textnormal{}}{\text{4. 安}}}$ いレストランです。 \hfill\break
It's a cheap restaurant. }
 
\par{5. じゃあ、それにしましょう。 \hfill\break
Well, let's go with that. }
 
\par{6. のどが ${\overset{\textnormal{かわ}}{\text{渇}}}$ きました。 \hfill\break
I'm thirsty. }

\par{7. ${\overset{\textnormal{すしににんまえ}}{\text{寿司二人前}}}$ お ${\overset{\textnormal{ねが}}{\text{願}}}$ いします。 \hfill\break
Sushi for two please. }
 
\par{${\overset{\textnormal{}}{\text{8. (お)酒}}}$ がお ${\overset{\textnormal{}}{\text{好}}}$ きですか。 \hfill\break
Do you like sake? }
 
\par{9. お ${\overset{\textnormal{}}{\text{飲}}}$ み ${\overset{\textnormal{}}{\text{物}}}$ は? \hfill\break
Drinks? }

\par{10. ${\overset{\textnormal{たんぱくしつ}}{\text{蛋白質}}}$ = Protein ${\overset{\textnormal{たんすいかぶつ}}{\text{炭水化物}}}$ = Carbohydrates ${\overset{\textnormal{しぼう}}{\text{脂肪}}}$ = Fat   }

\par{11. ${\overset{\textnormal{あいせき}}{\text{相席}}}$ お ${\overset{\textnormal{}}{\text{願}}}$ いします。 \hfill\break
Please let another party sit with you. }

\par{\textbf{Culture Note }: You may be asked this in Japan in inexpensive restaurants when it's really crowded and there's really nothing that you can do about it. }

\par{12. ${\overset{\textnormal{しょうしょう}}{\text{少々}}}$ お ${\overset{\textnormal{}}{\text{待}}}$ ちください。 \hfill\break
Please wait one moment. }
 
\par{13. いらっしゃいませ。 ${\overset{\textnormal{なんめい}}{\text{何名}}}$ ですか。こちらへどうぞ。 \hfill\break
Welcome. How many will there be? Please follow me. }
 
\par{14. ビールをもう ${\overset{\textnormal{}}{\text{一本}}}$ ください。 \hfill\break
Give me one more beer please. \hfill\break
\hfill\break
Note: Beer in this case is most likely 缶ビール. }
 
\par{15. てんぷらを ${\overset{\textnormal{}}{\text{一人前}}}$ お ${\overset{\textnormal{}}{\text{願}}}$ いします。 \hfill\break
Tempura for one please. }

\par{${\overset{\textnormal{}}{\text{16. 火}}}$ にかける。 \hfill\break
To put on the stove. }

\par{17. ${\overset{\textnormal{こしょう}}{\text{胡椒}}}$ を ${\overset{\textnormal{くわ}}{\text{加}}}$ えて ${\overset{\textnormal{あじ}}{\text{味}}}$ をつける。 \hfill\break
To put in flavor by adding pepper. }

\par{18. お ${\overset{\textnormal{かんじょう}}{\text{勘定}}}$ は○○になります。 \hfill\break
Your bill comes out to XX. }
 
\par{19. お ${\overset{\textnormal{}}{\text{待}}}$ たせしました。(Waiter) \hfill\break
I'm sorry for having made you wait. }
 
\par{20. 「ここでお ${\overset{\textnormal{め}}{\text{召}}}$ し ${\overset{\textnormal{}}{\text{上}}}$ がりですか、それともお ${\overset{\textnormal{}}{\text{持}}}$ ち ${\overset{\textnormal{}}{\text{帰}}}$ りですか」「ここで食べます。」 \hfill\break
“Is this to eat here or to-go?” “For here”. }
 
\par{21. すし ${\overset{\textnormal{や}}{\text{屋}}}$ へ ${\overset{\textnormal{}}{\text{行}}}$ きましょうか。 \hfill\break
How about going to the sushi shop? }
 
\par{22. たいてい ${\overset{\textnormal{}}{\text{何}}}$ を ${\overset{\textnormal{}}{\text{飲}}}$ みますか。 \hfill\break
What do you generally have to drink? }
 
\par{23. ご注文はお ${\overset{\textnormal{き}}{\text{決}}}$ まりですか。 \hfill\break
Have you decided on what you will order? }
 
\par{${\overset{\textnormal{}}{\text{24. 私}}}$ に ${\overset{\textnormal{はら}}{\text{払}}}$ わせてください。 \hfill\break
Allow me to pay. }
 
\par{25. はい、 ${\overset{\textnormal{ぜんぶ}}{\text{全部}}}$ でX ${\overset{\textnormal{えん}}{\text{円}}}$ になります。 \hfill\break
Yes, this comes out to in total to X yen. }
 
\par{${\overset{\textnormal{}}{\text{26. 何}}}$ に\{しましょう・なさいま\}か。 \hfill\break
What shall I get you? }
 
\par{27. お ${\overset{\textnormal{かんじょう}}{\text{勘定}}}$ をお ${\overset{\textnormal{ねが}}{\text{願}}}$ いします。 \hfill\break
Please bring us the bill. }
 
\par{28. 「お ${\overset{\textnormal{ちゃ}}{\text{茶}}}$ をもう ${\overset{\textnormal{いっぱい}}{\text{一杯}}}$ いかがでしょうか?」「ええ、いただきます」 \hfill\break
“Can I get you another glass of tea?” “Yes, thank you”. }
 
\par{29. コーヒーのお ${\overset{\textnormal{か}}{\text{代}}}$ わりはいかがですか。 \hfill\break
Would you like a refill of coffee? }
 
\par{\textbf{Culture Note }: When you are given something to eat or drink, say "いただきます". When leaving, you say "ご ${\overset{\textnormal{ちそう}}{\text{馳走}}}$ (さま(でした))". The hostess may respond with "お ${\overset{\textnormal{そまつ}}{\text{粗末}}}$ さまでした. }
 
\par{30. お ${\overset{\textnormal{}}{\text{水をください。}}}$ \hfill\break
Water, please. }
 
\par{\textbf{Female Speech Note }: お is more frequently used by women in common items such as these. Similar words include お ${\overset{\textnormal{にく}}{\text{肉}}}$ "meat" and お ${\overset{\textnormal{ひや}}{\text{冷}}}$ "cold water". }
 
\par{31. どこかおいしいレストランを ${\overset{\textnormal{おし}}{\text{教}}}$ えてください。 \hfill\break
Could you tell me of delicious restaurants anywhere? }
 
\par{${\overset{\textnormal{}}{\text{32. ずいぶん色々}}}$ なものがあるんですね。 \hfill\break
There are a lot of items. (In reference to the menu) }

\begin{center}
 \textbf{More Key Words }
\end{center}

\begin{ltabulary}{|P|P|P|P|P|P|P|P|}
\hline 

ナイフ & Knife & フォーク & Fork & スプーン & Spoon & はし & Chopsticks \\ \cline{1-8}

(お) ${\overset{\textnormal{さら}}{\text{皿}}}$ & Plate & ナプキン & Napkin & ウェイター & Waiter & ウェイトレス & Waitress \\ \cline{1-8}

\end{ltabulary}
      
\section{Rice and Noodles}
 
\par{ Rice is the most important food in Japan. ご ${\overset{\textnormal{はん}}{\text{飯}}}$ , cooked rice, is even synonymous to "meal". ${\overset{\textnormal{こめ}}{\text{米}}}$ is uncooked rice. Many things are made from rice. }
 
\par{\textbf{${\overset{\textnormal{}}{\text{酒}}}$ \textbf{: The Japanese Drink }}}
 
\par{ ${\overset{\textnormal{}}{\text{酒}}}$ , also known as ${\overset{\textnormal{にほんしゅ}}{\text{日本酒}}}$ , is a rice-based alcoholic beverage. It's served at room temperature or heated. It can be served in お ${\overset{\textnormal{ちょこ}}{\text{猪口}}}$ ,small cylindrical cups, ${\overset{\textnormal{さかずき}}{\text{杯}}}$ which are flat saucer-like cups, or ${\overset{\textnormal{ます}}{\text{枡}}}$ which are wooden box-like cups. One shouldn't fill one's own cup: it should be done for you. To ask: }
 
\par{33. (お) ${\overset{\textnormal{}}{\text{酒}}}$ をもう ${\overset{\textnormal{}}{\text{一杯}}}$ ください。 \hfill\break
One more cup please. }
 
\par{Be sure to hold your cup to the ${\overset{\textnormal{とっくり}}{\text{徳利}}}$ --the flask--as a gesture of acceptance. ${\overset{\textnormal{かんぱい}}{\text{乾杯}}}$ (Cheers)! }
 
\par{\textbf{More Rice }}
 
\par{${\overset{\textnormal{どんぶり}}{\text{丼}}}$ : 丼 is a bowl of hot steamed rice with toppings. \hfill\break
 ${\overset{\textnormal{せんべい}}{\text{煎餅}}}$ : Rice crackers in all shapes and flavors. \hfill\break
チャーハン・炒飯: Chinese fried rice. \hfill\break
お ${\overset{\textnormal{にぎ}}{\text{握}}}$ り: Balls of rice with filling. \hfill\break
 ${\overset{\textnormal{もち}}{\text{餅}}}$ : Rice cake. \hfill\break
すし: すし is vinegar rice topped\slash mixed with seafood \& vegetables. There are many kinds. }
 
\par{\textbf{Noodles ${\overset{\textnormal{めんるい}}{\text{麺類}}}$ }}
 
\par{うどん (Japanese): うどん is a wheat-flour noodle. There are several kinds of dishes. \hfill\break
そば (Japanese): そば is buckwheat. It is either served chilled with a dipping sauce or in a hot broth. \hfill\break
 ${\overset{\textnormal{そうめん}}{\text{素麺}}}$ (Japanese): そうめん are white thin wheat flour noodles dipped in めん ${\overset{\textnormal{つゆ}}{\text{汁}}}$ and served cold. The noodles are often placed in a flume in cold water and the diners have to catch them. \hfill\break
ラーメン (Chinese): ラーメン is made of Chinese-style wheat noodles and is served with a meat and often flavored with soy sauce or ${\overset{\textnormal{みそ}}{\text{味噌}}}$ (soybean paste). }
      
\section{Dishes}
  
\begin{ltabulary}{|P|P|P|P|P|P|}
\hline 

Deep-fry & 揚げ物(あげもの) & Pot cooking & 鍋物(なべもの) & Stews & 煮物(にもの) \\ \cline{1-6}

Grilled & 焼き物(やきもの) & Soups & 吸い物(すいもの) & Pickled & 漬物(つけもの) \\ \cline{1-6}

Stir-fried & 炒め物(いためもの) & Sashimi & 刺し身 & Soup (from juice) & 汁物(しるもの) \\ \cline{1-6}

\end{ltabulary}

\par{\textbf{Word Note }: ${\overset{\textnormal{つけもの}}{\text{漬物}}}$ also refers to salted foods. }
 
\par{${\overset{\textnormal{ふぐ}}{\text{河豚}}}$ (Sashimi): The フグ, puffer fish, is poisonous yet delicious. It is prepared with extreme caution to remove the toxic areas. The Emperor is forbidden to eat it. The liver is apparently the most delicious part, but it's also the part most likely to kill you. フグ is a delicacy ( ${\overset{\textnormal{ちんみ}}{\text{珍味}}}$ ). }
 
\par{ギョーザ (Yakimono): Chinese ravioli-dumplings usually filled with pork and vegetables. \hfill\break
 ${\overset{\textnormal{みそしる}}{\text{味噌汁}}}$ (Shirumono): みそ soup is made out of ${\overset{\textnormal{だし}}{\text{出汁}}}$ , stock, and みそ paste. \hfill\break
 ${\overset{\textnormal{うなぎ}}{\text{鰻}}}$ (Yakimono): ウナギ is freshwater eel. Saltwater eels are called ${\overset{\textnormal{あなご}}{\text{穴子}}}$ . \hfill\break
しゃぶしゃぶ (Nabemono): しゃぶしゃぶ is made with thinly sliced beef. It is usually served with ${\overset{\textnormal{とうふ}}{\text{豆腐}}}$ , ${\overset{\textnormal{はくさい}}{\text{白菜}}}$ (Chinese cabbage), 春菊 (edible chrysanthemum leaves), ${\overset{\textnormal{のり}}{\text{海苔}}}$ seaweed, onions, ${\overset{\textnormal{にんじん}}{\text{人参}}}$ carrots, ${\overset{\textnormal{しいたけ}}{\text{椎茸}}}$ and えのき ${\overset{\textnormal{だけ}}{\text{茸}}}$ mushrooms, etc. \hfill\break
 ${\overset{\textnormal{とん}}{\text{豚}}}$ カツ (Agemono): Breaded, deep fried pork cutlet. \hfill\break
 ${\overset{\textnormal{や}}{\text{焼}}}$ き ${\overset{\textnormal{とり}}{\text{鳥}}}$ (Yakimono): Skewered chicken, it can refer to skewered food in general. \hfill\break
 ${\overset{\textnormal{て}}{\text{照}}}$ り ${\overset{\textnormal{や}}{\text{焼}}}$ き (Yakimono): 照り焼き is grilled, broiled, or fried meat glazed in sweet soy sauce. \hfill\break
てんぷら (Agemono): Deep-fried prawns and vegetables. \hfill\break
 ${\overset{\textnormal{よ}}{\text{寄}}}$ せ ${\overset{\textnormal{なべ}}{\text{鍋}}}$ (Nabemono): Seafood hot pot. \hfill\break
 ${\overset{\textnormal{ぞうに}}{\text{雑煮}}}$ (Shirumono): A soup with ${\overset{\textnormal{もち}}{\text{餅}}}$ common in New Year's. \hfill\break
 ${\overset{\textnormal{とんじる}}{\text{豚汁}}}$ (Shirumono): Like みそ soup with pork. \hfill\break
 ${\overset{\textnormal{うめぼし}}{\text{梅干}}}$ (Tsukemono): 梅干 is pickled ${\overset{\textnormal{うめ}}{\text{梅}}}$ , which are like plums. }

\par{${\overset{\textnormal{からあ}}{\text{唐揚}}}$ げ (Agemono): 唐揚げ is bite-size chicken, fish, etc. deep fried. \hfill\break
お ${\overset{\textnormal{この}}{\text{好}}}$ み ${\overset{\textnormal{や}}{\text{焼}}}$ き (Yakimono): Consists of a flour batter, トロロ, ${\overset{\textnormal{みず}}{\text{水}}}$ \slash  ${\overset{\textnormal{だし}}{\text{出汁}}}$ , ${\overset{\textnormal{たまご}}{\text{卵}}}$ egg, and shredded キャベツ(cabbage), etc. It is often flavored with mayo. \hfill\break
すきやき (Nabemono): Thinly sliced ビーフ (beef) and vegetables cooked in ${\overset{\textnormal{しょうゆ}}{\text{醤油}}}$ soy sauce, 出汁 sugar, and sake. It's dipped into bowls of raw egg. \hfill\break
 ${\overset{\textnormal{いもに}}{\text{芋煮}}}$ (Suimono): A thick potato soup. \hfill\break
 ${\overset{\textnormal{うすづく}}{\text{薄作}}}$ り (Sashimi): Finely sliced raw fish. Plate is decorated with shredded ${\overset{\textnormal{だいこん}}{\text{大根}}}$ Japanese radish, lemon slice, ginger, and a ${\overset{\textnormal{だいこん}}{\text{大根}}}$ -chili mixture with scallions in the center. \hfill\break
 ${\overset{\textnormal{べんとう}}{\text{弁当}}}$ (Miscellaneous): Assorted lunches. }

\begin{center}
 \textbf{Delicacies }
\end{center}

\begin{ltabulary}{|P|P|P|P|}
\hline 

アンキモ & Anglerfish liver & カラスミ & Salted mullet roe \\ \cline{1-4}

このわた & Salted sea cucumber entrails & ウニ & Salt\slash pickled sea urchins  \\ \cline{1-4}

\end{ltabulary}
      
\section{Drinks}
 
\begin{ltabulary}{|P|P|P|P|P|P|}
\hline 

 お茶 & (Green) tea & 酒 & Alcohol & ワイン & Wine \\ \cline{1-6}

ビール & Beer & ウォッカ & Vodka & コーク & Coke \\ \cline{1-6}

抹茶 & Powdered green tea \hfill\break
& 麦茶 & Barley tea & 紅茶 & Black tea \\ \cline{1-6}

焼酎 & Shochu & 桜湯 & Cherry blossom tea & 水 & Water \\ \cline{1-6}

コーヒー & Coffee & ミルク & Milk & ジュース & Juice \\ \cline{1-6}

アイスティー & Ice tea & コーラ & Cola & スプライト & Sprite \\ \cline{1-6}

\end{ltabulary}

\begin{center}
 \textbf{Other Unique Drinks }
\end{center}

\begin{ltabulary}{|P|P|}
\hline 

クー & Qoo, a non-carbonated beverage with grape and orange flavors. \\ \cline{1-2}

ヤクルト & Yakult is a pro-biotic milk-like drink. \\ \cline{1-2}

カルピス & Calpis is a non-carbonated beverage with a milky taste. \\ \cline{1-2}

C.C Lemon & A soft drink known for its lemon flavor. \\ \cline{1-2}

ポカリスエット & A soft\slash sports drink that has a mild grapefruit flavor. \\ \cline{1-2}

ラムネ & A soft drink with many flavors. \\ \cline{1-2}

\end{ltabulary}

\par{\textbf{Word Note }: For liquor, "straight" is "ストレートで" and "on-ice" is " ${\overset{\textnormal{みずわ}}{\text{水割}}}$ りで". }
      
\section{Western Food  洋食}
  
\begin{ltabulary}{|P|P|P|P|}
\hline 

 カキフライ & Breaded oyster & カキエビ & Breaded shrimp \\ \cline{1-4}

ステーキ & Steak & ポークチョップ & Pork chop \\ \cline{1-4}

オクラ & Okra & アボカド & Avocado \\ \cline{1-4}

パイナップル & Pineapple & パパイヤ & Papaya \\ \cline{1-4}

ネクタリン & Nectarine & カボチャ & Pumpkin \hfill\break
\\ \cline{1-4}

ハンバーガー & Hamburger & ビフテキ & Beef steak \\ \cline{1-4}

ブドウ & Grape & ブラックベリー & Blackberry \\ \cline{1-4}

マンゴー & Mango & キーウィ & Kiwi \\ \cline{1-4}

プラム & Plum & ココナッツ & Coconut \\ \cline{1-4}

オレンジ & Orange & リンゴ & Apple \\ \cline{1-4}

ラスベリー & Raspberry & ブルーベリー & Blueberry \\ \cline{1-4}

ロースト & Roast & シチメンチョウ & Turkey \\ \cline{1-4}

フライドポテト & French fries & ピザ & Pizza \\ \cline{1-4}

タコス & Taco & ベーコン & Bacon \\ \cline{1-4}

スクワッシュ & Squash & ハム & Ham \\ \cline{1-4}

ホットドッグ & Hot dog & サンドウィッチ & Sandwich \\ \cline{1-4}

サラダ & Salad & スープ & Soup \\ \cline{1-4}

オムレツ & Omelet & ソーセージ & Sausage \\ \cline{1-4}

トマト & Tomato & ピーマン & Bell pepper \\ \cline{1-4}

\end{ltabulary}
       
\section{Snacks}
 
\par{  デザート (desserts) and おやつ (snacks) are very important, and there is a large variety that you can choose from in Japan.  }

\par{\textbf{${\overset{\textnormal{わがし}}{\text{和菓子}}}$ \textbf{(Japanese-style sweets) }}}

\begin{ltabulary}{|P|P|P|P|}
\hline 

団子 & Rice dumplings & カキ氷 & Shaved ice with syrup topping \hfill\break
\\ \cline{1-4}

こんぺいと & Crystal sugar candy & まんじゅう & Sticky rice surrounding a sweet bean center  \\ \cline{1-4}

\end{ltabulary}

\par{\textbf{${\overset{\textnormal{ようがし}}{\text{洋菓子}}}$ \textbf{(Western-style sweets) }}}

\begin{ltabulary}{|P|P|P|P|}
\hline 

カステラ & Iberian-style sponge cakes & アイスクリーム \hfill\break
& Ice cream \\ \cline{1-4}

ケーキ & Cake & クッキー & Cookie \\ \cline{1-4}

\end{ltabulary}
 
\begin{center}
\textbf{Common Snacks }
\end{center}

\begin{ltabulary}{|P|P|P|P|}
\hline 

ハイチュー & Edible chewing candy similar to gum. & ポッキー & Biscuit stick snack \\ \cline{1-4}

うまい棒 & Puff corn snacks similar to Cheetos & コアラのマーチ & Bite-sized cookie snacks \\ \cline{1-4}

\end{ltabulary}
      
\section{Seasoning 調味料}
 
\begin{ltabulary}{|P|P|P|P|}
\hline 

辛子 & Spicy mustard & 酢みそ & Vinegar miso sauce \\ \cline{1-4}

ケチャップ & Ketchup & 二杯酢 & Soy vinegar sauce \\ \cline{1-4}

米酢 & Rice vinegar & ふりかけ & Dry condiment sprinkled on rice \\ \cline{1-4}

しょう油 & Soy sauce & ポン酢 & Citrus-based sauce \\ \cline{1-4}

マヨネーズ & Mayonnaise & みりん & Low alcohol rice wine \\ \cline{1-4}

めんま & From dried bamboo & ラー油 & Chili-infused vegetable oil \\ \cline{1-4}

わさび & Wasabi & こしょう & Pepper \\ \cline{1-4}

塩 & Salt & 砂糖 & Sugar \\ \cline{1-4}

香辛料 & Spices & ショウガ & Ginger \\ \cline{1-4}

カレー粉 & Curry powder & マスタード & Mustard \\ \cline{1-4}

ソース & Worcestershire sauce & 油 & Oil\slash fat \\ \cline{1-4}

\end{ltabulary}
      
\section{Other Ingredients}
  
\begin{ltabulary}{|P|P|P|P|}
\hline 

 ミカン & Mandarin orange & アユ & Ayu fish \\ \cline{1-4}

ナマズ & Catfish & ヒジキ & Dark edible seaweed \hfill\break
\\ \cline{1-4}

かずのこ & Herring roe & ニラ & Chinese chives \\ \cline{1-4}

ナス & Eggplant & サケ & Salmon \\ \cline{1-4}

イカ & Squid \hfill\break
& ハマグリ & Clam \\ \cline{1-4}

豚肉 & Pork & サツマイモ & Sweet potato \\ \cline{1-4}

梨 & Nashi pear & 昆布 & Kombu \\ \cline{1-4}

キュウリ & Cucumber \hfill\break
& 豆 & Beans \\ \cline{1-4}

モモ & Peach & イチゴ & Strawberry \\ \cline{1-4}

バナナ & Banana & かも肉 & Duck \\ \cline{1-4}

インゲン & String bean & マッシュルーム & Mushroom \\ \cline{1-4}

ヤマイモ & Yam & 唐辛子 & Chili pepper \\ \cline{1-4}

レモン & Lemon & スイカ & Watermelon \\ \cline{1-4}

レタス & Lettuce & ザクロ & Pomegranate \\ \cline{1-4}

トウモロコシ & Corn & カブ & Turnip \\ \cline{1-4}

タコ & Octopus & クリ & Chestnut \\ \cline{1-4}

ホタテガイ & Scallop & カツオ & Bonito \\ \cline{1-4}

ホウレンソウ & Spinach & 酢 & Vinegar \\ \cline{1-4}

パン粉 & Dried bread crumbs \hfill\break
& マグロ & Tuna \\ \cline{1-4}

イクラ & Salmon roe & 魚肉 & Fish \\ \cline{1-4}

鶏肉 & Chicken & ヒラメ & Flounder \\ \cline{1-4}

イワシ & Sardine & エビ & Shrimp \\ \cline{1-4}

アズキ & Azuki red beans & ショウガ & Ginger \\ \cline{1-4}

\end{ltabulary}
       
\section{Cookware \& Utensils 台所用品}
 
\begin{ltabulary}{|P|P|P|P|P|P|}
\hline 

 皿 & Plate & 鍋 & Pot & ざる & Colander \\ \cline{1-6}

フライパン & Fry pan & 布巾 & Kitchen towel & ふた & Lid \\ \cline{1-6}

薬缶 & Kettle & (お)箸 & Chopsticks & 包丁 & Kitchen knife \\ \cline{1-6}

ナイフ & Knife & フォーク & Fork & スプーン & Spoon \\ \cline{1-6}

炊飯器 & Rice cooker & レンジ & Microwave & ガス台 & Gas stove \\ \cline{1-6}

まな板 & Cutting board & 杓文字 & Rice paddle & 缶切 & Can opener \\ \cline{1-6}

栓抜き & Bottle opener & 流し(台) & Sink & 換気扇 & Ventilation fan \\ \cline{1-6}

おたま & Ladle & ポット & Thermos bottle &  &  \\ \cline{1-6}

\end{ltabulary}
    