    
\chapter{Plants \& Animals}

\begin{center}
\begin{Large}
第156課: Plants \& Animals 
\end{Large}
\end{center}
 
\par{ We humans share the planet with many kinds of living things. We all probably know the names of hundreds in our native languages, but what about Japanese? This lesson does not introduce all things that lives, but you will finish knowing the names of several handfuls. }
      
\section{Biology in Context}
 
\par{ Although all plant and animal names have 漢字 associated with them for the most part, if the spelling is not as common than the カタカナ spelling, it will be left in parentheses for reference. }

\par{1. 東京の ${\overset{\textnormal{かっさい}}{\text{葛西}}}$ ${\overset{\textnormal{りんかい}}{\text{臨海}}}$ ${\overset{\textnormal{すいぞく}}{\text{水族}}}$ ${\overset{\textnormal{えん}}{\text{園}}}$ に ${\overset{\textnormal{おとず}}{\text{訪}}}$ れた人たちは ${\overset{\textnormal{いきお}}{\text{勢}}}$ いよく ${\overset{\textnormal{む}}{\text{群}}}$ れで泳ぐ \textbf{マグロ }(鮪)の ${\overset{\textnormal{すがた}}{\text{姿}}}$ を楽しんでいました。 \hfill\break
The people who visited the Kassai Marine Aquarium in Tokyo were enjoying seeing the schools of \textbf{tuna }swimming energetically. }

\par{2. ${\overset{\textnormal{やせい}}{\text{野生}}}$ の ${\overset{\textnormal{くま}}{\text{}}}$ が山から下りてきた。 \hfill\break
The wild \textbf{bear }came down from the mountain. }

\par{3. \textbf{バラ }(薔薇)の咲く ${\overset{\textnormal{らくえん}}{\text{楽園}}}$ へようこそ。 \hfill\break
Welcome to paradise where \textbf{roses }blossom. }
 
\par{4. \textbf{オオカミ }(狼)がいないと、 \textbf{ウサギ }(兎)が ${\overset{\textnormal{ほろ}}{\text{滅}}}$ びてしまう。 \hfill\break
If there were no \textbf{wolves }, \textbf{rabbits }would die out. }
 
\par{5. 道の ${\overset{\textnormal{まんなか}}{\text{真ん中}}}$ で ${\overset{\textnormal{へび}}{\text{}}}$ を ${\overset{\textnormal{ふ}}{\text{踏}}}$ んだらどうなりますか? \hfill\break
What happens when you step\slash stomp on a \textbf{snake }in the middle of the road? }
 
\par{6. \textbf{カエル }(蛙)を ${\overset{\textnormal{ひ}}{\text{轢}}}$ いたことがあります。 \hfill\break
I have run over a \textbf{frog }before. }
 
\par{7. その畑には多くの ${\overset{\textnormal{じゅもく}}{\text{}}}$ が ${\overset{\textnormal{なら}}{\text{並}}}$ んでいました。 \hfill\break
Many \textbf{trees }were lined up by each other in the field. }
 
\par{8. 近い ${\overset{\textnormal{しょうらい}}{\text{将来}}}$ 、海の \textbf{魚 }が ${\overset{\textnormal{ぜつめつ}}{\text{絶滅}}}$ ${\overset{\textnormal{じょうたい}}{\text{状態}}}$ になるかもしれない。 \hfill\break
In the near future, the \textbf{fishes }of the sea may become extinct. }
 
\par{\textbf{Reading Note }: 魚 may be read as さかな or うお. The first is typically more common, but the latter is required in certain expressions. The latter is actually from the original word for fish. }
 
\par{9. \textbf{馬 }に乗ったことがありますか? \hfill\break
Have you ever ridden a \textbf{horse }? }

\par{10. 庭に \textbf{竹 }を植えたいです。 \hfill\break
I want to plant \textbf{bamboo }in my yard. }
 
\par{11. \textbf{クモ }(蜘蛛)の巣は飛んでいる \textbf{虫 }を ${\overset{\textnormal{つか}}{\text{捕}}}$ まえる ${\overset{\textnormal{わな}}{\text{罠}}}$ です。 \hfill\break
A \textbf{spider }web is a trap to catch flying \textbf{insects }. }
 
\par{12. 引き上げる度に、10~20匹ずつ ${\overset{\textnormal{あみ}}{\text{網}}}$ に \textbf{カニ }が付いてくる。 \hfill\break
Each time I lift the net up, 10-20 \textbf{crabs }are in it. }
 
\par{13. 家の ${\overset{\textnormal{にわ}}{\text{庭}}}$ に ${\overset{\textnormal{かじゅ}}{\text{}}}$ や ${\overset{\textnormal{さくら}}{\text{}}}$ を ${\overset{\textnormal{うえ}}{\text{植え}}}$ てはいけない。 \hfill\break
You can't plant \textbf{fruit trees }or \textbf{cherry blossom trees }in your yard. }

\par{14. ${\overset{\textnormal{すうとう}}{\text{数頭}}}$ の \textbf{牛 }がそこの ${\overset{\textnormal{さく}}{\text{柵}}}$ を壊して ${\overset{\textnormal{に}}{\text{逃}}}$ げてしまった。 \hfill\break
Several \textbf{cows }broke that fence over there and escaped. }
 
\par{15. カメ(亀)に ${\overset{\textnormal{か}}{\text{噛}}}$ まれた時はどうすればよいでしょうか。 \hfill\break
What should you do when you're bitten by a \textbf{turtle }? }

\par{16. ${\overset{\textnormal{ひつじ}}{\text{}}}$ を ${\overset{\textnormal{かぞ}}{\text{数}}}$ えても ${\overset{\textnormal{ねむ}}{\text{眠}}}$ れない。 \hfill\break
I can't sleep even if I count \textbf{sheep }. }
 
\par{17. 日本には昔から \textbf{ウサギ }(兎)が月に ${\overset{\textnormal{す}}{\text{棲}}}$ むという ${\overset{\textnormal{せつわ}}{\text{説話}}}$ が ${\overset{\textnormal{つた}}{\text{伝}}}$ わっている。 \hfill\break
There is a legend in Japan that has been told since ancient times that \textbf{rabbits }live on the moon. }

\par{18. ${\overset{\textnormal{さる}}{\text{}}}$ は \textbf{木 }から落ちても \textbf{猿 }だが、 ${\overset{\textnormal{ぎいん}}{\text{議員}}}$ が ${\overset{\textnormal{せんきょ}}{\text{選挙}}}$ で落ちれば、ただの人なのだ。 \hfill\break
A \textbf{monkey }is still a \textbf{monkey }when he falls out of a \textbf{tree }, but an assemblyman is simply a regular man when he falls out of the election. }
 
\par{19. あたしは \textbf{トラ }(虎)になる夢を見る \textbf{猫 }ですにゃあ。 \hfill\break
I'm a \textbf{cat }who dreams of becoming a \textbf{tiger }. }
 20. \textbf{キツネ }(狐)を ${\overset{\textnormal{か}}{\text{飼}}}$ いたい。 \hfill\break
I want to raise a \textbf{fox }. 
\par{21. 現在、上野動物園には ${\overset{\textnormal{ぞう}}{\text{}}}$ は ${\overset{\textnormal{なんとう}}{\text{何頭}}}$ いるか知っていますか。 \hfill\break
How many \textbf{elephants }are there currently at Ueno Zoo? }

\par{22. たまに ${\overset{\textnormal{か}}{\text{}}}$ を手で殺した時に ${\overset{\textnormal{ち}}{\text{血}}}$ が ${\overset{\textnormal{ふちゃく}}{\text{付着}}}$ するけど、あれって僕らの血なの?それとも \textbf{蚊 }の血なの?誰か教えてください! \hfill\break
Occasionally blood gets on me when I kill a \textbf{mosquito }with my hand, but is that our blood? Or, is it the \textbf{mosquito }'s blood? Someone, please tell me. }

\par{23. \textbf{フグ }(河豚)を食べる国は日本 ${\overset{\textnormal{いがい}}{\text{以外}}}$ にはどれくらいありますか? \hfill\break
How many other countries are there aside from Japan where people eat \textbf{puffer fish }? }

\par{24. 小さい \textbf{トカゲ }(蜥蜴)が部屋の ${\overset{\textnormal{かべ}}{\text{壁}}}$ にくっついていた。 \hfill\break
There's a small \textbf{lizard }stuck on the wall inside the room. }

\par{25. ${\overset{\textnormal{にんしんちゅう}}{\text{妊娠中}}}$ は \textbf{イカ }(烏賊)や \textbf{タコ }(蛸)を食べてはいけない。 \hfill\break
You mustn't eat \textbf{squid }or \textbf{octopus }while pregnant. }

\par{26. \textbf{キリン }(麒麟)に乗れる場所を探しています。 \hfill\break
I'm looking for a place where I can ride a \textbf{giraffe }. }

\par{27. \textbf{ライオン }は \textbf{シマウマ }(縞馬)を食べますよね。 \hfill\break
 \textbf{Lions }eat \textbf{zebras }, right? }

\par{28. \textbf{パンダ }は中国にしかいない。 \hfill\break
 \textbf{Pandas }are only in China. }

\par{29. \textbf{シカ }(鹿)が多すぎる。 \hfill\break
There are too many \textbf{deer }. }

\par{30. \textbf{アリ }(蟻)を殺すと雨が降る。 \hfill\break
When you kill an \textbf{ant }, it rains. }

\par{31. \textbf{ペンギン }はなぜ ${\overset{\textnormal{みなみはんきゅう}}{\text{南半球}}}$ にしかいないの? \hfill\break
Why are \textbf{penguins }only in the Southern Hemisphere? }

\par{32. ${\overset{\textnormal{そぼ}}{\text{祖母}}}$ に ${\overset{\textnormal{かも}}{\text{}}}$ ${\overset{\textnormal{にく}}{\text{肉}}}$ を使ったレシピを教えてもらいました。 \hfill\break
I had my grandmother teach me a recipe that uses \textbf{duck }meat. }

\par{Note: A domesticated duck is called an アヒル. }

\par{33. ${\overset{\textnormal{ちょう}}{\text{}}}$ の ${\overset{\textnormal{じゅみょう}}{\text{寿命}}}$ は長くても ${\overset{\textnormal{すうかげつ}}{\text{数ヶ月}}}$ ${\overset{\textnormal{ていど}}{\text{程度}}}$ です。 \hfill\break
The lifespan of a butterfly, at the most, is around several months. }

\par{\textbf{Variation Note }: Butterfly may also be チョウチョウ(蝶々)or 蝶ちょ. }

\par{34. \textbf{うなぎ }(鰻) ${\overset{\textnormal{づ}}{\text{釣}}}$ りの ${\overset{\textnormal{えさ}}{\text{餌}}}$ は何がいい? \hfill\break
What sort of bait is best for fishing \textbf{eels }? }

\par{35. 本物の \textbf{クジラ }を見てみたい。 \hfill\break
I want to see an actual \textbf{whale }. }

\par{36. \textbf{タヌキ }(狸)は日本 ${\overset{\textnormal{とくゆう}}{\text{特有}}}$ の動物です。 \hfill\break
The \textbf{tanuki }(raccoon dogs) is a unique animal to Japan. }

\par{37. 日本ではかつて \textbf{ネズミ }${\overset{\textnormal{わな}}{\text{罠}}}$ を ${\overset{\textnormal{しか}}{\text{仕掛}}}$ けるとき、 ${\overset{\textnormal{あぶら}}{\text{油}}}$ ${\overset{\textnormal{あ}}{\text{揚}}}$ げを餌として ${\overset{\textnormal{もち}}{\text{用}}}$ いるのが ${\overset{\textnormal{いっぱんてき}}{\text{一般的}}}$ だった。 \hfill\break
In the past in Japan, it was commonplace to use deep-fried tofu as bait to trap \textbf{mice }. }

\par{38. \textbf{サメ }(鮫)は ${\overset{\textnormal{めった}}{\text{滅多}}}$ に ${\overset{\textnormal{にんげん}}{\text{}}}$ を ${\overset{\textnormal{おそ}}{\text{襲}}}$ わない。 \hfill\break
 \textbf{Sharks }seldom eat \textbf{people }. }

\par{\textbf{Usage Note }: Some people say フカ(鱶) for shark. This is predominantly a West Japanese word for it, and it traditionally refers to a large shark. Most sharks are large, so it might as well be the general word for shark. }
39. \textbf{ハチ }(蜂)や \textbf{スズメバチ }(雀蜂)に気をつけましょう! \hfill\break
Be careful of \textbf{bees }and \textbf{wasps }? 
\par{\textbf{Usage Note }: 蜂 is a general term for any kind of bee or wasp. ミツバチ refers to what Americans think of as being bees. アシナガバチ and スズメバチ both would be called wasps, hornets, or yellow jackets by English speakers, but the former has long legs as the name suggests. The Japanese equivalent of a bumblebee is a クマバチ. Some speakers call this as クマンバチ. Both words may also refer to a オオスズメバチ. The American version is マルハナバチ.The オオスズメバチ (giant hornet) is extremely dangerous. キイロスズメバチ may be called カメバチ (瓶蜂), トックリバチ (徳利蜂), or アカバチ (赤蜂). クロスズメバチ and シダクロスズメバチ, which would specifically be called wasps by most English speakers, are wasps known for building their nests in the ground. Thus, some Japanese speakers call them ジバチ (地蜂), ドバチ (土蜂), ハイバチ (灰蜂), ヘボ (used in the 東海地方), or スガレ・スガリ (used throughout 東北). Special attention is given to ハチ because all aside from bumblebees are especially dangerous in Japan. }
    