    
\chapter{Astronomy}

\begin{center}
\begin{Large}
第163課: Astronomy: The Planets \& More 
\end{Large}
\end{center}
 
\par{ In this vocabulary lesson, we will learn about the planets and basic astronomical terminology. Note that this lesson will be about ${\overset{\textnormal{てんもんがく}}{\text{天文学}}}$ (アストロノミー) and not astrology ${\overset{\textnormal{せんせいじゅつ}}{\text{占星術}}}$ (アストロロジー). The difference is that astronomy is a true science whereas astrology is a pseudo-science. Although fascinating with its own realm of terminology, this lesson will focus on the former as the terminology in the field is of practical use. }
      
\section{The Solar System}
 
\par{ Our solar system is called the ${\overset{\textnormal{たいようけい}}{\text{太陽系}}}$ . The Sun in Japanese is ${\overset{\textnormal{たいよう}}{\text{太陽}}}$ , but it also often just goes by 陽・日. The moon in Japanese is 月.The planets of our solar system are as follows: }

\begin{ltabulary}{|P|P|P|P|P|P|P|P|}
\hline 
 
  Mercury 
 &    ${\overset{\textnormal{すいせい}}{\text{水星}}}$ 
 &   Venus 
 &    ${\overset{\textnormal{きんせい}}{\text{金星}}}$ \hfill\break
 ${\overset{\textnormal{みょうじょう}}{\text{明星}}}$ 
 &   Earth 
 &    ${\overset{\textnormal{ちきゅう}}{\text{地球}}}$ 
 &   Mars 
 &    ${\overset{\textnormal{かせい}}{\text{火星}}}$ 
 \\ \cline{1-8} 
 
  Jupiter 
 &    ${\overset{\textnormal{もくせい}}{\text{木星}}}$ 
 &   September 
 &    ${\overset{\textnormal{どせい}}{\text{土星}}}$ 
 &   Uranus 
 &    ${\overset{\textnormal{てんのうせい}}{\text{天王星}}}$ 
 &   Neptune 
 &    ${\overset{\textnormal{かいおうせい}}{\text{海王星}}}$ 
 \\ \cline{1-8} 
 
\end{ltabulary}

\par{\hfill\break
Word Note: 明星 is a literary term and not an astronomical term. 金星 is the predominant name in the spoken language. }

\par{1. ${\overset{\textnormal{すいせい}}{\text{水星}}}$ は ${\overset{\textnormal{つき}}{\text{月}}}$ と ${\overset{\textnormal{に}}{\text{似}}}$ ている。 \hfill\break
Mercury resembles the moon. }

\par{2. ${\overset{\textnormal{きんせい}}{\text{金星}}}$ の ${\overset{\textnormal{たいき}}{\text{大気}}}$ は ${\overset{\textnormal{ほとん}}{\text{殆}}}$ どが ${\overset{\textnormal{にさんかたんそ}}{\text{二酸化炭素}}}$ から ${\overset{\textnormal{な}}{\text{成}}}$ っている。 \hfill\break
The atmosphere of Venus is almost completely made up of carbon dioxide. }

\par{3. ${\overset{\textnormal{わたし}}{\text{私}}}$ たち ${\overset{\textnormal{にんげん}}{\text{人間}}}$ が ${\overset{\textnormal{す}}{\text{住}}}$ む ${\overset{\textnormal{ちきゅう}}{\text{地球}}}$ も「 ${\overset{\textnormal{わくせい}}{\text{惑星}}}$ 」です。 \hfill\break
The Earth that we humans live on is also a “planet.” }

\par{4. ${\overset{\textnormal{かせい}}{\text{火星}}}$ に ${\overset{\textnormal{いじゅう}}{\text{移住}}}$ する ${\overset{\textnormal{けいかく}}{\text{計画}}}$ が ${\overset{\textnormal{じっさい}}{\text{実際}}}$ に ${\overset{\textnormal{しんこう}}{\text{進行}}}$ しているのはご ${\overset{\textnormal{ぞんじ}}{\text{存知}}}$ ですか。 \hfill\break
Did you know that plans to migrate to Mars are actually progressing? }

\par{5. ${\overset{\textnormal{たいようけい}}{\text{太陽系}}}$ で ${\overset{\textnormal{いちばんおお}}{\text{一番大}}}$ きい ${\overset{\textnormal{わくせい}}{\text{惑星}}}$ は ${\overset{\textnormal{もくせい}}{\text{木星}}}$ です。 \hfill\break
The largest planet in the solar system is Jupiter. }

\par{6. ${\overset{\textnormal{もくせい}}{\text{木星}}}$ と ${\overset{\textnormal{どせい}}{\text{土星}}}$ はガスで ${\overset{\textnormal{でき}}{\text{出来}}}$ ている。 \hfill\break
Jupiter and Saturn are made of gas. }

\par{7. ${\overset{\textnormal{てんのうせい}}{\text{天王星}}}$ は ${\overset{\textnormal{よこ}}{\text{横}}}$ を ${\overset{\textnormal{む}}{\text{向}}}$ いている。 \hfill\break
Uranus is on its side. }

\par{8. ${\overset{\textnormal{かいおうせい}}{\text{海王星}}}$ の ${\overset{\textnormal{ちゅうしんぶ}}{\text{中心部}}}$ は ${\overset{\textnormal{みず}}{\text{水}}}$ の ${\overset{\textnormal{こおり}}{\text{氷}}}$ とアンモニアから ${\overset{\textnormal{な}}{\text{成}}}$ っている。 \hfill\break
The center of Neptune is made up of water ice and ammonia. }
      
\section{Important Astronomical Terminology}
 
\par{ The chart below provides a healthy list of some of the most important terminology in the field of astronomy. These words encompass many of the words that pervade daily conversations regarding the heavens. Following the chart are plenty of example sentences for you to study past grammar as well as practice using the words introduced below.  }
 
\begin{ltabulary}{|P|P|P|P|}
\hline 
 
  Planet 
 &    ${\overset{\textnormal{わくせい}}{\text{惑星}}}$ 
 &   Dwarf Planet 
 &   準惑星 \hfill\break
矮惑星 
 \\ \cline{1-4} 
 
  Pluto 
 &    ${\overset{\textnormal{めいおうせい}}{\text{冥王星}}}$ 
 &   Kuiper Belt 
 &   カイパーベルト 
 \\ \cline{1-4} 
 
  Asteroid 
 &    ${\overset{\textnormal{しょうわくせい}}{\text{小惑星}}}$ 
 &   Asteroid Belt 
 &    ${\overset{\textnormal{しょうわくせいたい}}{\text{小惑星帯}}}$ 
 \\ \cline{1-4} 
 
  Star 
 &    ${\overset{\textnormal{こうせい}}{\text{恒星}}}$ \hfill\break
 ${\overset{\textnormal{ほし}}{\text{星}}}$ 
 &   Celestial Body 
 &    ${\overset{\textnormal{てんたい}}{\text{天体}}}$ 
 \\ \cline{1-4} 
 
  Celestial sphere 
 &    ${\overset{\textnormal{てんきゅう}}{\text{天球}}}$ 
 &   Nebula 
 &    ${\overset{\textnormal{せいうん}}{\text{星雲}}}$ 
 \\ \cline{1-4} 
 
  Milky Way 
 &    ${\overset{\textnormal{あま}}{\text{天}}}$ の ${\overset{\textnormal{がわ}}{\text{川}}}$ \hfill\break
 ${\overset{\textnormal{ぎんがけい}}{\text{銀河系}}}$ 
 &   Galaxy 
 &    ${\overset{\textnormal{ぎんが}}{\text{銀河}}}$ 
 \\ \cline{1-4} 
 
  Galaxy system 
 &    ${\overset{\textnormal{ぎんがけい}}{\text{銀河系}}}$ 
 &   Satellite 
 &    ${\overset{\textnormal{えいせい}}{\text{衛生}}}$ 
 \\ \cline{1-4} 
 
  Man-made satellite 
 &    ${\overset{\textnormal{じんこうえいせい}}{\text{人工衛星}}}$ 
 &   Star system 
 &    ${\overset{\textnormal{こうせい}}{\text{恒星}}}$ システム 
 \\ \cline{1-4} 
 
  Universe 
 &    ${\overset{\textnormal{うちゅう}}{\text{宇宙}}}$ 
 &   Extraterrestrial 
 &    ${\overset{\textnormal{うちゅうじん}}{\text{宇宙人}}}$ 
 \\ \cline{1-4} 
 
  Cosmology 
 &    ${\overset{\textnormal{うちゅうろん}}{\text{宇宙論}}}$ \hfill\break
コスモロジー 
 &   Space dust 
 &    ${\overset{\textnormal{うちゅうじん}}{\text{宇宙塵}}}$ 
 \\ \cline{1-4} 
 
  The Big Bang 
 &   ビッグバン 
 &   Space adaptation   syndrome 
 &    ${\overset{\textnormal{うちゅうよ}}{\text{宇宙酔}}}$ い 
 \\ \cline{1-4} 
 
  Space weather 
 &    ${\overset{\textnormal{うちゅうてんきよほう}}{\text{宇宙天気予報}}}$ 
 &   Observatory 
 &    ${\overset{\textnormal{てんもんだい}}{\text{天文台}}}$ 
 \\ \cline{1-4} 
 
  Spaceship 
 &    ${\overset{\textnormal{うちゅうせん}}{\text{宇宙船}}}$ 
 &   International Space   Station 
 &    ${\overset{\textnormal{こくさいうちゅう}}{\text{国際宇宙}}}$ ステーション 
 \\ \cline{1-4} 
 
  Orbit 
 &   軌道 
 &   Meteorite 
 &    ${\overset{\textnormal{いんせき}}{\text{隕石}}}$ 
 \\ \cline{1-4} 
 
  Crater 
 &   クレーター 
 &   Ozone layer 
 &   オゾン ${\overset{\textnormal{そう}}{\text{層}}}$ 
 \\ \cline{1-4} 
 
  Solar eclipse 
 &    ${\overset{\textnormal{にっしょく}}{\text{日蝕}}}$ ・ ${\overset{\textnormal{にっしょく}}{\text{日食}}}$ 
 &   Lunar eclipse 
 &    ${\overset{\textnormal{げっしょく}}{\text{月蝕}}}$ ・ ${\overset{\textnormal{げっしょく}}{\text{月食}}}$ 
 \\ \cline{1-4} 
 
  Telescope 
 &    ${\overset{\textnormal{ぼうえんきょう}}{\text{望遠鏡}}}$ 
 &   Comet 
 &    ${\overset{\textnormal{すいせい}}{\text{彗星}}}$ \hfill\break
ほうき ${\overset{\textnormal{ぼし}}{\text{星}}}$ 
 \\ \cline{1-4} 
 
  Revolution 
 &    ${\overset{\textnormal{こうてん}}{\text{公転}}}$ 
 &   Cycle 
 &   周期 
 \\ \cline{1-4} 
 
  Rotation 
 &    ${\overset{\textnormal{じてん}}{\text{自転}}}$ 
 &   Atmosphere 
 &    ${\overset{\textnormal{たいき}}{\text{大気}}}$ ( ${\overset{\textnormal{けん}}{\text{圏}}}$ ) 
 \\ \cline{1-4} 
 
  Oort Cloud 
 &   オールトの ${\overset{\textnormal{くも}}{\text{雲}}}$ 
 &   NASA 
 &   NASA \hfill\break
アメリカ ${\overset{\textnormal{こうくううちゅうきょく}}{\text{航空宇宙局}}}$ 
 \\ \cline{1-4} 
 
  Andromeda Galaxy 
 &   アンドロメダ ${\overset{\textnormal{ぎんが}}{\text{銀河}}}$ 
 &   Dwarf star 
 &    ${\overset{\textnormal{わいせい}}{\text{矮星}}}$ 
 \\ \cline{1-4} 
 
  Europa 
 &   エウロパ 
 &   Titan 
 &   タイタン 
 \\ \cline{1-4} 
 
  Blackhole 
 &   ブラックホール 
 &   Astronaut 
 &    ${\overset{\textnormal{うちゅうひこうし}}{\text{宇宙飛行士}}}$ 
 \\ \cline{1-4} 
 
  Meteor shower 
 &    ${\overset{\textnormal{りゅうせいぐん}}{\text{流星群}}}$ 
 &   Io 
 &   イオ 
 \\ \cline{1-4} 
 
  Red giant 
 &    ${\overset{\textnormal{せきしょくきょせい}}{\text{赤色巨星}}}$ 
 &   Superstar 
 &    ${\overset{\textnormal{きょせい}}{\text{巨星}}}$ 
 \\ \cline{1-4} 
 
\end{ltabulary}
 
\par{\hfill\break
\textbf{Word Notes }: }
 
\par{1. 恒星 is the astronomical term for a “star” whereas 星 is the common day word. Another difference is that 星 may refer to any celestial body, including the Earth and other planets. \hfill\break
2. 彗星 is both the technical word for “comet” and the most frequently used word for it in everyday speech. Colloquially, it may also be ほうき星. This is no different than the English phrase “shooting star.” \hfill\break
3. Colloquially, our galaxy is called 天の川. The technical term is 銀河系, which also refers to galaxy systems in general. \hfill\break
4. Eclipse is ${\overset{\textnormal{しょく}}{\text{食}}}$ . This, however, is the simplified spelling. It is traditionally spelled as 蝕. Both spellings are still used. In media, though, only the simplified spelling is typically employed. \hfill\break
5. Colloquially, 大気 suffices as the word for atmosphere, but 大気圏 encompasses all parts of the atmosphere up to the boundary between the outer edge and space itself whereas 大気 typically refers just to the part of the atmosphere that we view as air. }

\begin{center}
\textbf{Examples }
\end{center}
 
\par{9. ${\overset{\textnormal{すいせい}}{\text{水星}}}$ は、 ${\overset{\textnormal{たいよう}}{\text{太陽}}}$ に ${\overset{\textnormal{もっと}}{\text{最}}}$ も ${\overset{\textnormal{ちか}}{\text{近}}}$ く ${\overset{\textnormal{こうてん}}{\text{公転}}}$ している ${\overset{\textnormal{わくせい}}{\text{惑星}}}$ だ。 \hfill\break
Mercury is the planet that revolves closest to the Sun. }
 
\par{10. ${\overset{\textnormal{きょだい}}{\text{巨大}}}$ な ${\overset{\textnormal{いんせき}}{\text{隕石}}}$ が ${\overset{\textnormal{ちきゅう}}{\text{地球}}}$ に ${\overset{\textnormal{しょうとつ}}{\text{衝突}}}$ する ${\overset{\textnormal{かのうせい}}{\text{可能性}}}$ は ${\overset{\textnormal{なに}}{\text{何}}}$ パーセントですか。 \hfill\break
What percentage is the possibility that a large meteorite will collide with Earth? }
 
\par{11. オゾン ${\overset{\textnormal{そう}}{\text{層}}}$ が ${\overset{\textnormal{はかい}}{\text{破壊}}}$ されると、 ${\overset{\textnormal{ちきゅう}}{\text{地球}}}$ に ${\overset{\textnormal{とど}}{\text{届}}}$ く ${\overset{\textnormal{しがいせん}}{\text{紫外線}}}$ の ${\overset{\textnormal{りょう}}{\text{量}}}$ が ${\overset{\textnormal{ふ}}{\text{増}}}$ える。 \hfill\break
If the ozone layer is destroyed, the amount of ultra-violet rays that reach Earth will increase. }
 
\par{12. ほぼ ${\overset{\textnormal{ぜんいん}}{\text{全員}}}$ の ${\overset{\textnormal{うちゅうひこうし}}{\text{宇宙飛行士}}}$ がなんらかの ${\overset{\textnormal{うちゅうよ}}{\text{宇宙酔}}}$ いを ${\overset{\textnormal{けいけん}}{\text{経験}}}$ します。 \hfill\break
Almost all astronauts experience some amount of space adaptation syndrome. }
 
\par{13. ブラックホールとは ${\overset{\textnormal{こうみつど}}{\text{高密度}}}$ かつ ${\overset{\textnormal{だいしつりょう}}{\text{大質量}}}$ の ${\overset{\textnormal{てんたい}}{\text{天体}}}$ で、 ${\overset{\textnormal{ぶっしつ}}{\text{物質}}}$ だけでなく ${\overset{\textnormal{ひかり}}{\text{光}}}$ さえも吸い ${\overset{\textnormal{こ}}{\text{込}}}$ んでしまうほど ${\overset{\textnormal{きょうりょく}}{\text{強力}}}$ な ${\overset{\textnormal{じゅうりょく}}{\text{重力}}}$ を ${\overset{\textnormal{も}}{\text{持}}}$ っている。 \hfill\break
A blackhole is a celestial body with high density and great mass that holds such strong gravity that it not only swallows matter but even light. }
 
\par{14. ${\overset{\textnormal{ぼく}}{\text{僕}}}$ はひとり ${\overset{\textnormal{くるま}}{\text{車}}}$ で ${\overset{\textnormal{りゅうせいぐん}}{\text{流星群}}}$ を ${\overset{\textnormal{み}}{\text{見}}}$ に ${\overset{\textnormal{い}}{\text{行}}}$ きました。 \hfill\break
I went by myself via car to go see a meteor shower. }
 
\par{15. ${\overset{\textnormal{てんのうせい}}{\text{天王星}}}$ の ${\overset{\textnormal{じてんじく}}{\text{自転軸}}}$ が ${\overset{\textnormal{よこだお}}{\text{横倒}}}$ しになっている。 \hfill\break
The axle of Uranus is on its side. }
 
\par{16. みんなで ${\overset{\textnormal{かいきげっしょく}}{\text{皆既月食}}}$ を ${\overset{\textnormal{かんさつ}}{\text{観察}}}$ しましょう。 \hfill\break
Let\textquotesingle s all observe a total lunar eclipse together. }
 
\par{\textbf{Word Note }: The antonym of 皆既月食 is ${\overset{\textnormal{かいきにっしょく}}{\text{皆既日食}}}$ . }
 
\par{17. ${\overset{\textnormal{ぼうえんきょう}}{\text{望遠鏡}}}$ を覗き ${\overset{\textnormal{こ}}{\text{込}}}$ んで、ほうき ${\overset{\textnormal{ぼし}}{\text{星}}}$ を ${\overset{\textnormal{さが}}{\text{探}}}$ した。 \hfill\break
I looked into a telescope and searched for shooting stars. }
 
\par{18. ${\overset{\textnormal{よぞら}}{\text{夜空}}}$ を ${\overset{\textnormal{みあ}}{\text{見上}}}$ げたら、 ${\overset{\textnormal{あま}}{\text{天}}}$ の ${\overset{\textnormal{がわ}}{\text{川}}}$ が ${\overset{\textnormal{み}}{\text{見}}}$ えました。 \hfill\break
I looked up at the night sky and could see the Milky Way. }
 
\par{19. ${\overset{\textnormal{うちゅうせん}}{\text{宇宙船}}}$ に ${\overset{\textnormal{の}}{\text{乗}}}$ る ${\overset{\textnormal{ゆめ}}{\text{夢}}}$ を ${\overset{\textnormal{み}}{\text{見}}}$ ました。 \hfill\break
I dreamed of riding in a spaceship. }
 
\par{20. イオには、 ${\overset{\textnormal{じゅうはっ}}{\text{18}}}$ ${\overset{\textnormal{こ}}{\text{個}}}$ の ${\overset{\textnormal{かつかざん}}{\text{活火山}}}$ があります。 \hfill\break
There are eighteen active volcanoes on Io. }
 
\par{21. エウロパの ${\overset{\textnormal{ちか}}{\text{地下}}}$ に ${\overset{\textnormal{うみ}}{\text{海}}}$ があるかもしれない。 \hfill\break
There may be a sea below the surface of Europa. }
 
\par{22. ${\overset{\textnormal{つき}}{\text{月}}}$ の ${\overset{\textnormal{うらがわ}}{\text{裏側}}}$ には、 ${\overset{\textnormal{いんせきしょうとつ}}{\text{隕石衝突}}}$ でクレーターがボコボコと ${\overset{\textnormal{あ}}{\text{開}}}$ いている。 \hfill\break
On the backside of the moon, there are craters everywhere due to meteorite impacts. }
 
\par{23. ${\overset{\textnormal{てんもんだい}}{\text{天文台}}}$ で ${\overset{\textnormal{ほし}}{\text{星}}}$ を ${\overset{\textnormal{み}}{\text{観}}}$ たいです。 \hfill\break
I want to see stars at an observatory. }
 
\par{24. ${\overset{\textnormal{うちゅう}}{\text{宇宙}}}$ は ${\overset{\textnormal{ぼうちょう}}{\text{膨脹}}}$ し ${\overset{\textnormal{つづ}}{\text{続}}}$ けている。 \hfill\break
The universe continues to expand. }
 
\par{25. ${\overset{\textnormal{うちゅうじん}}{\text{宇宙人}}}$ が ${\overset{\textnormal{そんざい}}{\text{存在}}}$ する ${\overset{\textnormal{かくりつ}}{\text{確率}}}$ が ${\overset{\textnormal{たか}}{\text{高}}}$ いなら ${\overset{\textnormal{ちきゅう}}{\text{地球}}}$ に ${\overset{\textnormal{こ}}{\text{来}}}$ ないのは ${\overset{\textnormal{なぜ}}{\text{何故}}}$ だろうか。 \hfill\break
Why is it that if the probability that extraterrestrials is so high yet they haven\textquotesingle t come to Earth? }
 
\par{26. ${\overset{\textnormal{かせい}}{\text{火星}}}$ は ${\overset{\textnormal{かこ}}{\text{過去}}}$ にフォボスとデイモスのような ${\overset{\textnormal{ちい}}{\text{小}}}$ さい ${\overset{\textnormal{えいせい}}{\text{衛星}}}$ がいくつかあったとされています。 \hfill\break
It is said that Mars in the past had several small satellites like Phobos and Deimos. }
 
\par{27. ${\overset{\textnormal{てんたい}}{\text{天体}}}$ を ${\overset{\textnormal{み}}{\text{見}}}$ るのに ${\overset{\textnormal{てき}}{\text{適}}}$ した ${\overset{\textnormal{ばしょ}}{\text{場所}}}$ はどこですか。 \hfill\break
Where is a suitable place to see celestial bodies? }
 
\par{28. ${\overset{\textnormal{しょうわくせいたい}}{\text{小惑星帯}}}$ は ${\overset{\textnormal{げんし}}{\text{原始}}}$ の ${\overset{\textnormal{たいようけい}}{\text{太陽系}}}$ の ${\overset{\textnormal{なごり}}{\text{名残}}}$ に ${\overset{\textnormal{み}}{\text{見}}}$ えるが、 ${\overset{\textnormal{げんし}}{\text{原始}}}$ の ${\overset{\textnormal{じょうたい}}{\text{状態}}}$ を ${\overset{\textnormal{たも}}{\text{保}}}$ っているわけではない。 \hfill\break
The asteroid belt may look like relics of the primal solar system, but it is not the case that it maintains a primal condition. }
 
\par{29. ${\overset{\textnormal{こくさいうちゅう}}{\text{国際宇宙}}}$ ステーションとは、アメリカ ${\overset{\textnormal{がっしゅうこく}}{\text{合衆国}}}$ 、ロシア、 ${\overset{\textnormal{にっぽん}}{\text{日本}}}$ 、カナダ ${\overset{\textnormal{およ}}{\text{及}}}$ び ${\overset{\textnormal{おうしゅううちゅうきかん}}{\text{欧州宇宙機関}}}$ (ESA) が ${\overset{\textnormal{きょうりょく}}{\text{協力}}}$ して ${\overset{\textnormal{うんよう}}{\text{運用}}}$ している ${\overset{\textnormal{うちゅう}}{\text{宇宙}}}$ ステーションである。 \hfill\break
The International Space Station is a space station which is operated in collaboration of the United States, Russia, Japan, Canada and the European Space Agency (ESA). }
 
\par{30. ${\overset{\textnormal{うちゅうじん}}{\text{宇宙塵}}}$ は ${\overset{\textnormal{ちきゅう}}{\text{地球}}}$ に ${\overset{\textnormal{お}}{\text{降}}}$ り ${\overset{\textnormal{そそ}}{\text{注}}}$ ぐ ${\overset{\textnormal{うちゅう}}{\text{宇宙}}}$ の ${\overset{\textnormal{ちり}}{\text{塵}}}$ である。 \hfill\break
Space dust is dust from space that falls to Earth. }
 
\par{31. ${\overset{\textnormal{たいきけん}}{\text{大気圏}}}$ とは、 ${\overset{\textnormal{ちきゅう}}{\text{地球}}}$ を ${\overset{\textnormal{と}}{\text{取}}}$ り ${\overset{\textnormal{ま}}{\text{巻}}}$ く ${\overset{\textnormal{うす}}{\text{薄}}}$ い ${\overset{\textnormal{たいき}}{\text{大気}}}$ の ${\overset{\textnormal{そう}}{\text{層}}}$ のことである。 \hfill\break
The atmosphere is the thin stratum of air that surrounds Earth. }
 
\par{32. なぜ ${\overset{\textnormal{わくせい}}{\text{惑星}}}$ は ${\overset{\textnormal{こうてん}}{\text{公転}}}$ の ${\overset{\textnormal{きどう}}{\text{起動}}}$ から ${\overset{\textnormal{はず}}{\text{外}}}$ れないのか。 \hfill\break
Why is it that planets don\textquotesingle t leave their revolution orbits? }
 
\par{33. ビッグバンとは、 ${\overset{\textnormal{うちゅう}}{\text{宇宙}}}$ の ${\overset{\textnormal{かいびゃくちょくご}}{\text{開闢直後}}}$ に ${\overset{\textnormal{うちゅう}}{\text{宇宙}}}$ の ${\overset{\textnormal{ぼうちょう}}{\text{膨張}}}$ が ${\overset{\textnormal{はじ}}{\text{始}}}$ まった ${\overset{\textnormal{じてん}}{\text{時点}}}$ を ${\overset{\textnormal{さ}}{\text{指}}}$ します。 \hfill\break
The Big Bang indicates the point in time when universe expansion began immediately after the creation of the universe. }
 
\par{34. ${\overset{\textnormal{めいおうせい}}{\text{冥王星}}}$ は ${\overset{\textnormal{じゅんわくせい}}{\text{準惑星}}}$ の ${\overset{\textnormal{てんけいれい}}{\text{典型例}}}$ である。 \hfill\break
Pluto is a classic example of a dwarf planet. }
35. ハッブル ${\overset{\textnormal{うちゅうぼうえんきょう}}{\text{宇宙望遠鏡}}}$ の ${\overset{\textnormal{かんそく}}{\text{観測}}}$ データにより、アンドロメダ ${\overset{\textnormal{ぎんが}}{\text{銀河}}}$ の ${\overset{\textnormal{うご}}{\text{動}}}$ きが ${\overset{\textnormal{はあく}}{\text{把握}}}$ されるようになった。 \hfill\break
The movement of the Andromeda Galaxy was ascertained by means of observational data from the Hubble Space Telescope.      