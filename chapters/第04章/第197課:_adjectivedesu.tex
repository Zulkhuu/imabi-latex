    
\chapter{The Grammaticality of Adjective + です}

\begin{center}
\begin{Large}
第197課: The Grammaticality of Adjective + です 
\end{Large}
\end{center}
 
\par{ Although using です after adjectives is used by almost all speakers, you would be surprised to know that some speakers feel that it is ungrammatical. In this lesson, we learn as to why that is.  }
      
\section{形容詞+です、って正しい日本語なの?}
  First, let\textquotesingle s think back to what です is. It is most likely a contraction of でございます. Intermediate forms such as でござんす and でげす could be found in the speech of adults in the late 1800s and early 1900s in Tokyo. Historically, copula verbs in Japanese could only follow nominal phrases. Nominalized phrases count, and so in the past, using the 連体形 of verbs and adjectives allowed for a copula to follow. For instance, you could get phrase like 思ふなり (do not get confused by the words and spelling being different). This grammar is almost identical in meaning to ○○のだ and its variants.  
\par{1. 得るは、捨つるにあり。 Set phrase \hfill\break
When you receive something, you have to consequently throw something else away. }
 
\par{\textbf{Grammar Note }: 捨つる is the old 連体形 of 捨てる. }
 
\par{ In the history of Japanese, there wasn't ever an instance in which copulas could attach directly to an adjective without some form of nominalization. Something such as 新しいです would not be grammatical because there is no precedent for it to be. To make it grammatical with grammar from the past, it would need to be in its 連体形; however, this is now identical to its 終止形. Going back in time to use its original 連体形 would produce 新しきです, but this has never existed, making this potential fix anachronistic. }
 
\par{ The violation of Japanese grammar is apparent. Before the invent of "adjective + です," the pattern "adjective + のです" was used and quite popular in the early 1900s, and it is still used to this day following modern rules on the usage of のです. Due to the strong tone that のです can give, dropping the の to lessen the tone and as a result be more polite is likely the reasoning that brought about "adjective + です." }

\par{ Another option, however, to fix the inherent grammar issue with "adjective + です," is using 連用形+ある. With it, we get forms as follows: }
 
\par{2a. 新しくあります \hfill\break
2b. 新しくはあります \hfill\break
2c. 新しくありません \hfill\break
2d. 新しくはありません \hfill\break
2e. 新しくもありません \hfill\break
2f. 新しく(は・も)ありませんでした。 \hfill\break
2g. 新しゅうございます  }
 
\par{ The first option is stretching things as far as natural speech is concerned. Adding an intervening particle, though, does make it viable in speech. In the negative form, it is not really awkward, and using an intervening particle makes the option all the more viable and common. The last option is an old respectful form. We see this construction in older speech and in set phrases like ありがとうございます. However, because it has for the most part died out within the last 70 years or so, it is unfair and unwise to claim that this should be the one and only right answer to the grammar debate. }
 
\par{ Another option to avoid "adjective+です" is paraphrasing. The above situation is a form of paraphrasing, but consider the following with another adjective 暑い. }
 
\par{3a. 暑いです。 〇・△・X 3b. 暑い一日でした。 〇 \hfill\break
3c. 暑くなってきましたね。〇 \hfill\break
3d. 暑く感じます。〇 }
 
\par{  Now, let\textquotesingle s broaden our discussion to all potentially grammatically unsound combinations of adjective or adjective-like phrases with です. }
 
\par{ We have in addition to 新しいです, 新しかったです. We also see ~ないです, ~たいです, and even だったです. Some think that using the past tense of adjectives with です is grammatically the same as with non-past tense. If the speaker likes one, the speaker likes both and vice versa. As is the case with ~ないです, though, using です after an auxiliary gives off the sense of a last-minute addition to make one\textquotesingle s phrase polite. Thus, it clearly marks the phrase as a polite phrase found in casual speech that should not be mimicked in more formal situations such as being a program announcer on television. }
 
\par{  ~たいです does not normally get pointed out as being ungrammatical, but it too disappears in formalized speech. だったです is, unlike every other example, deemed to be wrong by a majority of speakers. This is because でした exists. Yet, you would be surprised how many people still use it. To be fair, it is grammatically correct to upwards of 30\% of speakers. }
 
\par{ People young and old use the pattern "adjective + です." In reality, register restrictions for when you use it are more important. In strict formal writing, it is almost non-existent. After all, such style of writing tends to be grammatically conservative. However, in the spoken language and writing styles such as those found in blogs, it is extremely common. }
 
\par{ The people who are more likely to think that this pattern is incorrect are those who are said to be more sensitive to proper language use, people reading and writing primarily in formal registers. These people are frequently eloquent in their manner of speaking. However, it is important to note that the dialogue sections of even early Modern-Japanese works which laid the foundation for Modern Japanese literature provide us many examples of "adjective+です." Some examples even go beyond what is typically accepted to be correct today. }
 
\par{4. けれど自覚と云うのは、自省ということをも含んでおるですからな、 ${\overset{\textnormal{むやみ}}{\text{無闇}}}$ に意志や自我を ${\overset{\textnormal{ふりまわ}}{\text{振廻}}}$ しては困るですよ。自分の ${\overset{\textnormal{や}}{\text{遣}}}$ ったことには自分が全責任を帯びる覚悟がなくては \hfill\break
But self-awareness also involves self-reflection, so you mustn\textquotesingle t simply go recklessly abusing your willpower and ego. You must possess the resignation that you have to bear full responsibility for your own actions. \hfill\break
From 蒲団 by 田山花袋  }
 
\par{5. ええ寝ていて空を見る方がいいですと答えて \hfill\break
I answered that it was best to rest well and look at the sky \hfill\break
From 坊ちゃん by 夏目漱石 }

\par{ At the end of the day, it is best to ignore these people because they are not willing to realize that the language has already changed for quite some time to allow it in (casual) polite speech titled towards the spoken language. }
    