    
\chapter{語尾 III}

\begin{center}
\begin{Large}
第185課: 語尾 III: かな, かしら, じゃん, い, け, が, こと, たら, \& や 
\end{Large}
\end{center}
 
\par{ Yes, there are more 語尾. There are even more out there in dialects and Classical Japanese that you will one day encounter. }
      
\section{The Final Particles かな \& かしら}
 
\par{ かな = "I wonder" and is a colloquial variant of ~でしょう. ~ないかな shows wishful thinking by stressing desire for something to happen already. かしら is a feminine version. }
 
\par{1. 終わるかな。 \hfill\break
I wonder if it's going to end. }
 
\par{2. 悲しいかな。 \hfill\break
How sad! }

\par{3. ${\overset{\textnormal{まど}}{\text{窓}}}$ を開けてくれないかな。 \hfill\break
Could you open up the window? }
 
\par{\textbf{Nuance Note }: As demonstrated, かな may also elicit a request to people familiar\slash close to you. }

\par{4. ${\overset{\textnormal{しゃべ}}{\text{喋}}}$ りすぎたのじゃないかしら。 \hfill\break
I'm afraid I did chatter too much. }
 
\par{5. どのくらい雪が ${\overset{\textnormal{つ}}{\text{積}}}$ もったかしらね。 \hfill\break
I wonder how much snow accumulated. }
 
\par{6. どうかしら? \hfill\break
How does it look? }
 
\par{7. これをいただけるかしら? \hfill\break
May I take this? }

\par{9. ${\overset{\textnormal{だいじょうぶ}}{\text{大丈夫}}}$ かしら。 \hfill\break
I wonder if he's OK. }
 
\par{10. あの男、 ${\overset{\textnormal{だれ}}{\text{誰}}}$ かしら。 \hfill\break
I wonder who that man is. }
 
\par{11a. 売らんかなの ${\overset{\textnormal{せんでん}}{\text{宣伝}}}$  (慣用句) \hfill\break
11b. 売ってなんぼの姿勢が見え隠れする宣伝 (普通の言い方) \hfill\break
Exploitation\slash sales talk }
 
\par{\textbf{Phrase Note }: 売らんかな is a set phrase made up of the verb 売る (to sell), the auxiliary verb ~ん, which is a contraction of ~む (a Classical ending that shows guess here), and かな. }
 
\par{13. 「トルコで ${\overset{\textnormal{もっと}}{\text{最}}}$ も危険な道路」と言われるのもむべなるかなである。 \hfill\break
It's also plausible that it's called "the most dangerous road in Turkey". \hfill\break
From 雨天炎天 by 村上春樹. }
 
\par{\textbf{Phrase Note }: むべなるかな is a set phrase made up an old 形容動詞 in its 連体形, むべなる, which is equivalent to もっともな (plausible; quite right), followed by かな. }
 
\par{${\overset{\textnormal{は}}{\text{果}}}$ たせるかな means "just as expected". It is interchangeable with 果たして, but it may be confusing to some simply because it has かな in it. }
 
\par{14. \{果たせるかな・案の ${\overset{\textnormal{じょう}}{\text{定}}}$ ・思った通り\} ${\overset{\textnormal{こころ}}{\text{試}}}$ みはことごとく ${\overset{\textnormal{しっぱい}}{\text{失敗}}}$ したんだ。 \hfill\break
As expected, the experiment failed altogether. }
      
\section{The Final Particle じゃん}
 
\par{ じゃん started out as a contraction of じゃないか. Once it reached Tokyo, it reached national stardom. Unlike what it came from, it is quite casual and is by no means blunt. }

\par{15. ${\overset{\textnormal{あしもと}}{\text{足元}}}$ が危なっかしいじゃん? \hfill\break
Isn't your footing unsteady? }
 
\par{\textbf{Word Note }: 危なっかしい is not a typo of 危ない (dangerous). The word is very similar and refers to a situation that is either unsteady or dangerous. It is very similar to the word "precarious". }
 
\par{16. いいじゃん。 \hfill\break
Isn't that good? }
      
\section{The Final Particle い}
 
\par{ い is quite explosive and normally rude. It is almost always used by guys but is most importantly used by people of high temperament. It is often seen after だ, や, or じゃ and is also seen often after the 命令形. }

\par{17. 当たり前だぜぃ。 \hfill\break
That is obvious! }

\par{18. おい、こりゃ何だい? \hfill\break
Hey, what is this!? }

\par{19. 何か飲むかい? \hfill\break
Will you drink anything? }

\par{20. こりゃ何じゃい? \hfill\break
What's that? }

\par{\textbf{Etymology Note }: This particle comes from the particle よ.  よ \textrightarrow  え \textrightarrow  い. }
      
\section{The Final Particle け}
 
\par{ (っ)け is casually used in attempt to recall something by jogging one's memory and perhaps also the listener(s). っけ can only follow either the plain non-past or the plain or polite past forms. So, ですっけ isn't used. It is most often used with the plain forms, probably due to the fact that it is a contraction. However, with adjectives and verbs in the non-past, んだ is almost always inserted. So, you will see 新しいんだっけ but not 新しいっけ. This, though, may be acceptable in certain dialects. }
 
\par{This restriction stems from the fact that its usage with the past tense has a much longer history. -た can show confirmation in sentences like "what was his name?". Obviously the person's name hasn't changed, but we still use the past tense form of the verb. }
 
\par{Similarly, with interrogatives it solicits a response from the listener, whether it's something that the listener(s) have actually said before or not. Some situations, then, require the past tense. For instance, if you're recalling an event that's already happened, だっけ is wrong. }
 
\begin{center}
\textbf{Examples } 
\end{center}

\par{21. よく ${\overset{\textnormal{けんか}}{\text{喧嘩}}}$ したっけねえ。 \hfill\break
We often bickered, didn't we? }
 
\par{22. このパンはもう ${\overset{\textnormal{しょうみきげん}}{\text{賞味期限}}}$ が切れてるんだっけ? \hfill\break
Isn't this bread already passed the expiration date? }
 
\par{23. いつ ${\overset{\textnormal{ぞう}}{\text{象}}}$ を ${\overset{\textnormal{つか}}{\text{捕}}}$ まえたっけ? \hfill\break
When did you catch the elephant? }
 
\par{24. アメリカでは、 ${\overset{\textnormal{ぜいきん}}{\text{税金}}}$ が去年ぐーんと引き上げられたっけ。 \hfill\break
Weren't taxes raised straight up in America last year? }
 
\par{25. いつだ(った)っけ。 \hfill\break
When is it? }
 
\par{\textbf{Historical Note }: ~たっけ comes from ~たりけり. It does not come from ~たかえ. ~かえ is now old-fashioned, but it is a feature of Eastern Japanese dialects similar to かい without the vulgarity. }
 
\par{26. 出来たかえ。 \hfill\break
Could you do it? \hfill\break
From ${\overset{\textnormal{わがはい}}{\text{我輩}}}$ は猫である by 漱石. }
 
\par{\textbf{Dialect Note }: In some dialects け can be a ruder or typical version of か. }
 
\par{27. これ要るけ。(京都弁) \hfill\break
Do you need this? }
 
\par{Also, in some dialects け \textrightarrow  かいな. In Standard Japanese, though, this is an old ending that stresses a thought with a sense of doubt or is a contraction of そうかな. }
 
\par{28. やったかいな (京都弁) = やったっけ }
      
\section{The Final Particle が}
 
\par{1. The final particle が shows malice by speaking ill of someone. This usage is very rude and may cause sharp backlashes. }

\par{29. ガキめが! \hfill\break
You brat! }

\par{2. The final particle が may hint at one's own thoughts rather than what's at hand. }

\par{30. 今日はもう閉店なんですが。 \hfill\break
But we're already closed for today. (Polite) }

\par{31. 引っ ${\overset{\textnormal{こ}}{\text{越}}}$ せばよかったんだが。 \hfill\break
It would be good if we had moved, but\dothyp{}\dothyp{}\dothyp{} }

\par{32. 僕も注意したんだが。 \hfill\break
I also warned them, but\dothyp{}\dothyp{}\dothyp{} }

\par{33. もしもし、鈴木ですが。 \hfill\break
Hello, this is Suzuki. (Polite) }

\par{34. 社長がお呼びですが。 (Respectful) \hfill\break
The company president has called for you. }
      
\section{The Final Particle こと}
 
\par{ The final particle こと is not the nominal noun 事. It has quite a few usages. }

\begin{itemize}

\item Shows a practical yet somewhat harsh command most often used by superiors. 
\item Shows slightly deep emotion in feminine speech. 
\item Shows a slight feeling of questioning in feminine speech. 
\item ことよ softens an affirmative statement in feminine speech. 
\end{itemize}
 
\par{Use the definitions above to figure out which usage is being used in the following examples. }

\par{35. ${\overset{\textnormal{めずら}}{\text{珍}}}$ しい動物だこと。 \hfill\break
It's a rare animal! }
 
\par{36. もういいこと! \hfill\break
Isn't it already OK? }
 
\par{37. 金を返すこと! \hfill\break
Pay back the money! }
 
\par{38. とにかく ${\overset{\textnormal{あやま}}{\text{謝}}}$ ること! \hfill\break
Anyhow, apologize! }
      
\section{The Final Particle たら}
 
\par{ This たら brings up something to someone's attention with a sense of surprise, criticism, or impatience. This has nothing to do with the conditionals, but don't be surprised when you hear it. It's not a good thing if such a statement is directed to you. }
 
\par{39. やめてったら! \hfill\break
Stop that! }
 
\par{40. だまってたら! \hfill\break
Shut it! }
      
\section{The Final Particle や}
 
\par{ The final particle や has 4 usages. }

\begin{itemize}

\item Expresses urgency to people at or below one's status. In masculine speech this can be used to coax action similarly to "shall we". 
\item Lightly declares. This could be out of an array of emotions. 
\item Expresses exclamation. 
\item Used to call a person\slash thing one has a close relationship or an underling. By elderly people, this can be used to soften the tone. 
\end{itemize}
 
\begin{center}
 \textbf{Examples }
\end{center}

\par{41. お ${\overset{\textnormal{じい}}{\text{祖父}}}$ さんや、今日はあなたの ${\overset{\textnormal{たんじょうび}}{\text{誕生日}}}$ ですよ。 \hfill\break
Today's your birthday, my dear old man. }
 
\par{42. まあ、座れや。(Old person; dialectical) \hfill\break
Well, sit down. }
 
\par{43. そんなこと、知らなかったや。 \hfill\break
I had no idea about that. }
 
\par{44. こりゃええや。(Dialectical) \hfill\break
This is good. }
 
\par{45. まあいいや。 \hfill\break
Ahh, forget it. }

\par{46. ${\overset{\textnormal{おそ}}{\text{恐}}}$ ろしいや。 \hfill\break
That's scary! }
      
\section{Multiple 語尾}
 
\par{ You will often see more than one 語尾 at once. There are several important combinations that you should be aware that you can make. Exact combinations may not be common in some parts of Japan. Plus, there are plenty of interjectory particles that are very regional. }
 
\begin{center}
\textbf{かね }
\end{center}
 
\par{かな is the combination of か and な. かね is also possible. It works the same way as かな, but it is definitely more common in certain age groups and or dialects in Japanese. It can also be used to soften a criticism like in "どうしてそんなことをしているのかね?". It is softer than かな. }
 
\begin{center}
\textbf{がな }
\end{center}
 
\par{This comes from the archaic particle もがな, which shows wishful thinking. This is equivalent to といいなあ. がな is still occasionally used. It may also be dialectal to emphasize (a reminder). }
 
\par{47. もう ${\overset{\textnormal{す}}{\text{済}}}$ んだがな! \hfill\break
It should already be done! }
 
\begin{center}
\textbf{かい }
\end{center}
 
\par{かい  is a more harsh version of か. Its use is declining and can be associated with foreigner speech as learners tend to overuse it. }
 
\par{48. もういいのかい。 \hfill\break
Isn't it already OK? }
 
\begin{center}
\textbf{やい }
\end{center}
 
\par{やい is used to harshly call out for someone. It may also make a curt statement. }
 
\par{${\overset{\textnormal{いくじ}}{\text{49. 意気地}}}$ なしやい! \hfill\break
You coward! }
 
\par{50. 俺じゃねーやい! \hfill\break
It's not me! }
 
\par{51. 高田君やい! \hfill\break
Takada! }

\begin{center}
 \textbf{わい }
\end{center}

\par{ わい is typical of (older) male speech, and it is more common in dialectical speech, especially in West Japanese Dialects. This ending shows exclamation. }

\par{52. こんな老人を目の前にして「お墓」の一語をさらりと口に出すというのはなかなか良い根性をしておる \textbf{わい }と平岡は思い一転して愉快な気分になってきた。 \hfill\break
Having such an old man in front of his eyes without hesitation express the word "grave" shows good tenacity, Hiraoka thought, and in turn, he came to feel delightful. \hfill\break
From 不可能 by 松浦寿輝. }

\par{53. 困ったことだわい。 \hfill\break
I'm troubled. }

\par{54. そんなことないわい。 \hfill\break
That's not it at all.  }
 
\begin{center}
\textbf{Other Important Combinations }
\end{center}
 
\par{There are still more combinations. よ is often followed by ね and な. わ is often followed by よ and ね. い is seen all the time after ぜ and ぞ. If 語尾 have opposing usages, they aren't going to be used together. You may hear のね but not のな. Many combinations are often limited to certain regions of Japan. }
    