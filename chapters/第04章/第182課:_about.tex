    
\chapter{About}

\begin{center}
\begin{Large}
第182課: About: ~について, ~に関して, \& ~をめぐって 
\end{Large}
\end{center}
 
\par{ It normally doesn't take long to find differences between similar items in Japanese. Sometimes there might be certain forms and meanings of one that the other doesn\textquotesingle t have, or there could be formality differences. These are the kinds of differences that you will learn about in regards to ~について, ~に関して, and ~をめぐって in this lesson. }
      
\section{~について \& ~に関して}
 
\par{ You have no doubt come across ~について and ~に関して and noticed that each time that they were both translated as “about”. Although this is quite true, most students don\textquotesingle t understand the differences between them, when and when not to attach は, and how to choose an appropriate 連体形 (attribute form). }

\par{ First, consider the following mistakes and corrections. }

\par{1a. 今から韓国に関して話してください。X \hfill\break
1b. 今から韓国について話してください。〇 \hfill\break
From now talk about Korea. }

\par{2. }

\par{A: 田山さんは家事はやりますか。 \hfill\break
B: いえ、家事について、家内任せです。X \hfill\break
B: いえ、家事については、家内任せです。〇 \hfill\break
A: Does Mr. Tayama do housework? \hfill\break
B: No, in regards to housework, he leaves it up to his wife. }

\par{ The fundamental usage of ~について and ~に関して is to emphasize an event\slash person as the topic\slash theme. Then, one states something about it. These phrases are also common in forming questions in this manner. }

\par{3. ${\overset{\textnormal{ぼしんせんそう}}{\text{戊辰戦争}}}$ について教えていただけませんか。 \hfill\break
Could you teach me about the Boshin War? }

\par{4. 彼らは ${\overset{\textnormal{ぼうえき}}{\text{貿易}}}$ に関して議論した。 \hfill\break
They argued about trade. }

\par{ Whereas ~について emphasizes the content at hand, ~に関して includes the surroundings related to it. Also, as ~について comes from the verb 付く, it is used a lot when referring to things being tied to acts of communication via speaking, writing, thinking, etc. As you might gather just from the few examples thus far, ~に関して is more formal and stiff. Given its nuance and formality, you should see why it was wrong in the first example. }

\par{5. 駐車違反取締り\{〇 についての・Xに関しての\}お知らせ \hfill\break
Notice about the management of parking violations }

\par{6. 説明会のタイトルは水不足の問題\{〇 について・Xに関して\}です。 \hfill\break
The title of this information session is about the water shortage problem. }

\par{ Given the defining difference between the two, which do you suppose would be used when a student is giving a speech about his\slash her home country? The answer is ~について. In fact, ~について is used in the introduction of speeches because you are telling the listeners what your speech is going to be about, not what your speech\slash content of speech has relation with. Individual examples might be related with something else, but that\textquotesingle s not your introduction either. }

\begin{center}
\textbf{読み物: Excerpt from a Student Group Speech about 敬語 }
\end{center}

\par{ 「皆さん、こんにちは。私たちのインタビュープロジェクトのトピックは敬語です。私たちは日本の方に色々な敬語に \textbf{ついての }質問をしてみたかったので、今回日本人の敬語の考え方をテーマにして発表することにしました。インタビューした人のうち、13人が女性で、8人が男性でした。大体皆さんは若者でした。たった2人だけが35歳以上でした。 ${\overset{\textnormal{しゅふ}}{\text{主婦}}}$ が8人いて、大学生が6人いて、大学院生が4人いて、仕事をしている人が4人いました。 }

\par{ 最初に聞いてみた2つの質問は、敬語学習の ${\overset{\textnormal{じき}}{\text{時期}}}$ \textbf{について }です。… }

\par{ 日本人の敬語の考え方 \textbf{について }いくつか質問しました。その中で私たちが思っていたのと違った答えだったのは、「敬語はフェイクで、敬語を使わなかったら、人ともっと近くなれる」というのでした。日本人がすべて「敬語が必要だ」という ${\overset{\textnormal{しゅちょう}}{\text{主張}}}$ に ${\overset{\textnormal{いっち}}{\text{一致}}}$ していないのにとてもびっくりしました。何故現代の日本社会で敬語が必要なのかは、文化と習慣からだけだというわけではなくて、誠に尊敬の意を表している日本人の心を代表としていると思います。言語は時代と ${\overset{\textnormal{とも}}{\text{共}}}$ に変わるにもかかわらず、 ${\overset{\textnormal{ことばづか}}{\text{言葉遣}}}$ いに気をつけないと、「日本人じゃない」と思われてしまうこともこの発表の結果を考えると、お分かりいただけるでしょう。敬語の使い方は日本人自身にとっても難しいけれど、この先もずっと大切に使われるであろうと言えるかもしれません。これで、敬語 \textbf{についての }発表を終わります。」 }

\par{1. What were the demographics of the Japanese people interviewed? \hfill\break
2. How many people were college students? \hfill\break
3. What were the first two questions about? \hfill\break
4. What is the attribute form of について? \hfill\break
5. What was the most surprising response? \hfill\break
6. Language changes with time, but what must you always pay attention to? \hfill\break
7. Translate the last two sentences. \hfill\break
8. True or False: 敬語 is thought to be part of Japanese culture and tradition according to the findings of this speech. }

\par{ In explanation these phrases are essentially interchangeable. The same kinds of words are used with them, and as far as the kinds of sentences they appear in, they're essentially parallel. So, what mainly causes problems is formality and the nuance difference mentioned earlier. }

\par{7. ${\overset{\textnormal{じんしゅさべつしゅぎ}}{\text{人種差別主義}}}$ に\{ついて・関して\}の問題は重大です。 \hfill\break
The problem against racism is serious. }

\par{8. 自由と ${\overset{\textnormal{あっせい}}{\text{圧制}}}$ について考察する。 \hfill\break
To inquire about freedom and oppression. }

\par{9. 彼女の顔についていえば、本当に美しいですね。 \hfill\break
Starting with her look, she's really beautiful, isn't she? }

\par{\textbf{Variant Usage }: Although not related, ~につき (a variant which will be mentioned again) and ~について may also mean "per" with numerical phrases. Lastly, ~につき may be like ~ため to show reasoning in formal situations where it is typically written in 漢字 as ~に就き. }

\par{10. ${\overset{\textnormal{ゆきなだれ}}{\text{雪雪崩}}}$ に ${\overset{\textnormal{つ}}{\text{就}}}$ き電車は ${\overset{\textnormal{ふつう}}{\text{不通}}}$ です。 \hfill\break
Train (service) is suspended due to an avalanche. }

\par{11. 牛肉は今一ポンドにつき6ドルだよ。 \hfill\break
Meat is six dollars per pound now. }

\par{12. 千円につき50円の ${\overset{\textnormal{てすうりょう}}{\text{手数料}}}$ がかかります。 \hfill\break
There is a fifty yen handling fee per one thousand yen. }

\begin{center}
\textbf{~については・に関しては }
\end{center}

\par{ は is added, as one would imagine, to raise something as the topic or thing of contrast. }

\par{13. }

\par{A: 秘書がお金を受け取ったんでしょう。 \hfill\break
B: そのこと\{については・関しては\}、私は何も知りません。 \hfill\break
A: Didn\textquotesingle t the secretary receive the money? \hfill\break
B: I don\textquotesingle t know anything about that. }

\par{14. 男性であるか女性であるか、年齢はどうかなど\{については・に関しては\}、それほど重要ではありません。 \hfill\break
Whether it be about them being male or female or their age, it\textquotesingle s not that important. }

\begin{center}
\textbf{Formality }
\end{center}

\par{ You can also find these expressions as ~につきまして and ~に関しまして in more polite settings. }

\par{15. ご質問につきましてお答えします。(謙譲語) \hfill\break
I will answer concerning your question. }

\par{16. この前の ${\overset{\textnormal{しょうかい}}{\text{照会}}}$ に関しまして (Very Formal) \hfill\break
Regarding your recent inquiry }

\par{\textbf{Variant Note }: について can also be seen as につき in very formal writing. }

\begin{center}
\textbf{Attribute Forms }
\end{center}

\par{ ~についての is the attribute form of ~について. As for ~に関して, it has the attribute forms ~に関しての and ~に関する, with the latter being more literary. }

\par{17. 政府に関しての ${\overset{\textnormal{とうろん}}{\text{討論}}}$ 。 \hfill\break
A debate about the government. }

\par{18. 平和に関する北朝鮮の ${\overset{\textnormal{ぎわく}}{\text{疑惑}}}$ は日本の ${\overset{\textnormal{ぼうえい}}{\text{防衛}}}$ を ${\overset{\textnormal{きょうい}}{\text{脅威}}}$ に ${\overset{\textnormal{さら}}{\text{晒}}}$ している。 \hfill\break
North Korean doubts on peace is threatening Japanese security. }

\par{19. 平和に関する会合を開く。 \hfill\break
To hold a meeting concerning peace. }

\par{\textbf{Meaning Note }: ~に関する and ~に関わる have overlapping meanings of "concerning\slash related to X'. The first shows a connection, but the other shows a direct effect. }

\begin{center}
 \textbf{A }\textbf{+の+連体形+の }
\end{center}

\par{In this pattern it moreover limits something in noting a condition about what is expressed by the noun. It's very similar to ~に ${\overset{\textnormal{かん}}{\text{関}}}$ して. }

\par{20. コーヒーの ${\overset{\textnormal{さ}}{\text{冷}}}$ めたの \hfill\break
Cold in regards to coffee }

\par{21. コーヒーに関しては冷たい。 \hfill\break
It's cold in relation to coffee. }

\par{22. オレンジの ${\overset{\textnormal{こぶ}}{\text{小振}}}$ りなの \hfill\break
A comparatively small orange }

\par{\textbf{Meaning Note }: The word 小振り is normally only used in reference to things like fruits, fish, etc. However, its usage may be expanded some in speaking. Nevertheless, something like 小振りなテレビ is very weird. You should use ${\overset{\textnormal{こがた}}{\text{小型}}}$ のテレビ instead. }
      
\section{~をめぐって・めぐり}
 
\par{ ~を巡って, also seen as ~を巡り in more stiff, literary language, is typically written with 巡, but this is important in understanding its usage. There are some instances where it is used in a more literal sense of passing through places. }

\par{23. 四国四十八ヶ所の ${\overset{\textnormal{めいしょきゅうせき}}{\text{名所旧跡}}}$ を巡る。 \hfill\break
To pass through the 48 famous landmarks of Shikoku. }

\par{24. 僕は ${\overset{\textnormal{ひさいち}}{\text{被災地}}}$ を巡るつもりだ。 \hfill\break
I plan to go around the devastated area. }

\par{25. 鎌倉の古寺を巡ろう。 \hfill\break
Let's go through the Kamakura temples! }

\par{26. 名月や池を巡りて夜もすがら \hfill\break
Going about the full moon and lakes, all night. \hfill\break
From ${\overset{\textnormal{ばしょう}}{\text{芭蕉}}}$ . }

\par{\textbf{Conjugation Note }: 巡りて is the Classical form of 巡って. }

\par{ It can also be used in the sense of something that had gone away has returned. In this sense, it can also be written as 廻る. }

\par{27. 季節が巡る。 \hfill\break
Seasons return. }

\par{28a. 血液が体内を巡るのは生きるために必要だ。 \hfill\break
28b. 血液が体内を巡らなければ、生きることができない。 \hfill\break
Blood returning through the body is necessary in order to be able to live. }

\par{29. 悪運が日本に巡ってきたようだ。 \hfill\break
It looks like bad luck has reached Japan. }

\par{ However, what will be used the most with 巡る here concerns the meaning of "surrounding" some X and describing its condition.  This is translated as "concerning". In a sense you are surrounding something in an interest and describing it. One grammatical issue is that the attribute form can only be ~を巡っての when the following noun phrase Y concerns a situation and not a person. This restriction does not exist for the attribute form ~を巡った. }

\par{30. ジュリエットを巡る ${\overset{\textnormal{こいがたき}}{\text{恋敵}}}$ だよ。 \hfill\break
They are rivals concerning Juliet. }

\par{31. ${\overset{\textnormal{しろ}}{\text{城}}}$ の周りを巡れ! \hfill\break
Enclose the surroundings of the castle! }

\par{32. アメリカでは ${\overset{\textnormal{けんぽう}}{\text{憲法}}}$ を巡る問題が重なっています。 \hfill\break
There are problems concerning the Constitution building up in America. }

\par{33. 資金を巡って政府は ${\overset{\textnormal{あんしょう}}{\text{暗礁}}}$ に乗り上げてしまいました。 \hfill\break
(The government) reached a deadlock concerning government funds. }

\begin{center}
\textbf{~を巡って VS ~ }\textbf{について・に関して }
\end{center}

\par{ Although largely interchangeable, since ~を巡って is coming from the speaker's perspective as an onlooker, when one is actually a person concerned with something, ~について・に関して should be used instead. }

\par{34. 日本の将来\{を巡って・について\}、 ${\overset{\textnormal{ゆうしきしゃ}}{\text{有識者}}}$ による ${\overset{\textnormal{とうろんかい}}{\text{討論会}}}$ が行われました。 \hfill\break
There was a debate opened by experts concerning the future of Japan. }

\par{35. 健康保険\{〇 について・X を巡って\}、論議を進めたいと思います。 \hfill\break
I would like to proceed with an argument\slash discussion about health insurance. }

\par{36. 来年の企画\{〇 について・〇 に関して・ X をめぐって\}話し合ってください。 \hfill\break
Please talk together about next year\textquotesingle s project. }
    