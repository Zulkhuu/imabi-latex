    
\chapter{Adjective Nominalization I}

\begin{center}
\begin{Large}
第165課: Adjective Nominalization I: ~さ \& ~み 
\end{Large}
\end{center}
 
\par{ There are some set phrases in Japanese that involve adjectives becoming nominalized by simply being left as is. As quick examples, consider the following set phrases. }

\par{1a. 酸いも甘いも嚙み分ける。 \hfill\break
1b. 酸っぱいも甘いもよく心得ている。 \hfill\break
To know full well about the world. \hfill\break
Literally: To fully distinguish between sour and sweet. }

\par{ Typically, however, ~さ and ~み are used to nominalize adjectives, with ~め and ~き two other methods. The resulting products of these endings are not completely interchangeable. }
      
\section{~さ}
 
\par{ ~さ nominalizes an adjective objectively. Nouns with ~さ can either be concrete or feel as if they are. Not only can you find it after 形容詞, but you can also find it after 形容動詞 of both native and Sino-Japanese origin as well as loans from other languages. }
 
\par{1. 長さはどれぐらいですか。 \hfill\break
How long is it? }
 
\par{2. ${\overset{\textnormal{うれ}}{\text{嬉}}}$ しさを ${\overset{\textnormal{か}}{\text{噛}}}$ み ${\overset{\textnormal{し}}{\text{締}}}$ める。 \hfill\break
To enjoy happiness. }
 
\par{3. ${\overset{\textnormal{むてっぽう}}{\text{無鉄砲}}}$ さだけは ${\overset{\textnormal{おやゆず}}{\text{親譲}}}$ り。  (皮肉的) \hfill\break
Only recklessness is from (his) parents. }

\par{4. 安定の早さ \hfill\break
Rapidity of stability }

\par{5. 薬の飲みにくさ \hfill\break
Difficulty of taking medicine }

\par{6. その花の美しさに感動しました。 \hfill\break
I was moved by the flower's beauty. }

\par{7. モダンさのコントラスト \hfill\break
Modernity contrast }

\par{8. 強さで ${\overset{\textnormal{な}}{\text{成}}}$ し ${\overset{\textnormal{と}}{\text{遂}}}$ げる。 \hfill\break
To accomplish with strength. }

\par{${\overset{\textnormal{とち}}{\text{9. 土地}}}$ の ${\overset{\textnormal{ねだん}}{\text{値段}}}$ は ${\overset{\textnormal{}}{\text{広}}}$ さと\{利便性・ ${\overset{\textnormal{べんり}}{\text{便利}}}$ さ △\}で変わります。 \hfill\break
The price of land changes according to the size and convenience. }
 
\par{10. 打撃の強さは木を倒すほどでした。 \hfill\break
The blow was strong enough to knock down a tree. }
 
\par{11. 設備投資の弱さは競争力に大きなマイナスになるだろう。 \hfill\break
The weakness of capital investment could became a big minus to competitiveness. }

\par{12. 幸登は故郷の讃岐うどんが食べたさに日帰りで帰省することにした。 \hfill\break
Wanting to eat his hometown's Sanuki udon, Yukito decided to go back home for the day. }

\par{\textbf{Grammar Note }: This shows that ~さ can be used with ~たい phrases. }

\par{13. 嵐の前の静けさだね。 \hfill\break
It's the lull before the storm. }

\par{\textbf{Form Note }: 静けさ is from the Classical Japanese  静けし. Note, though, that 静かさ does exist because 静けし and 静かなり were different words then. 静かさ describes the objective quietness of something whereas 静けさ describes quietness in the sense of tranquility, emphasizing that quietness is a prerequisite. }

\par{\textbf{Warning Note }: 大げさ  is not an example of this. The true spelling is 大袈裟, and this is not 当て字. Rather, 袈裟 in this phrase comes from 袈裟懸け, meaning "slashing someone diagonally from the shoulders" in this instance. This eventually led to its adjectival meaning of "grandiose". }
 
\par{14. 若さに免じて許す。 \hfill\break
To forgive out of consideration of (the person's) youth. }
      
\section{~み}
 
\par{ ~み nominalizes an adjective \emph{\textbf{subjectively }}. So, ~み shows that there is some sort of \textbf{condition or nature }. It may also show the "location" of a certain condition. For example, 高み means "elevated place". The 当て字 味 is often used to write ~み. }

\par{ Grammatically speaking, ~み is very limited as to what it may attach to. It must only attach to simple adjectives (単純形容詞) such as 苦い. Derived adjectives, compound adjectives, and adjectives with repeating parts like 美々しい, cannot be used with ~み. This is the complete opposite with ~さ. }

\par{ Most of the adjective it can be used with refer to sight, hearing, hues, significance, etc. However, even this is not good enough. The chart below has almost all existing examples in Modern Japanese. To really know how み works, you have to investigate into each one. Some commonalities exist between similar words, but that's about it. }

\begin{ltabulary}{|P|P|P|P|P|}
\hline 

赤み & 黒み & 白み & 黄み & 黄色み \\ \cline{1-5}

臭み & 深み & 浅み & 高み & 低み \\ \cline{1-5}

丸み & 重み & 厚み & 温かみ & 柔らかみ \\ \cline{1-5}

温み & 軽み & 強み & 弱み & 痛み \\ \cline{1-5}

苦み & 甘み & 旨み & 辛み & 酸っぱみ \\ \cline{1-5}

痒み & 悲しみ & 面白み & 楽しみ & しょっぱみ \\ \cline{1-5}

有り難み & おかしみ & 茂み & 明るみ & 暗み \\ \cline{1-5}

渋み & 苦しみ &  &  &  \\ \cline{1-5}

\end{ltabulary}

\par{ Some of these are so rare that they may not be recognized as existing words. Some may be viewed as coming from verbs (痛む \textrightarrow  痛み; 楽しむ \textrightarrow  楽しみ; 悲しむ \textrightarrow  悲しみ; 明るむ \textrightarrow  明るみ; 暗む \textrightarrow  暗み). }

\par{\textbf{Word Notes }: }

\par{1. 白み  may be read as 白身 to refer specifically to "egg white", and 白味 can be used to refer to "whiteness" or "egg white". }

\par{2. 黒み may also refer to "dark feeling" instead of just "dark hue". }

\par{3. 黄み spelled as 黄味 or 黄身 may mean "egg yolk" with the last option only meaning "egg yolk".   }

\par{4. 浅み (shallow place), 低み (low ground), 軽み (light hue) read as either かるみ or かろみ, and しょっぱみ are often deemed "ungrammatical" due to their rarely used status. It is to note that かろみ is actually a complex word from 芭蕉 works referring to a serious and smooth figure for discovering beauty. It's also important to note that although many Japanese teachers will tell you that しょっぱみ doesn't exist, it's not difficult to find examples of it made by native speakers. Perhaps this is due to confusion from み\textquotesingle s multiple usages in the Japanese lexicon. On one hand, しょっぱみ refers to salty flavor, not saltiness, which then it would need to be replaced with しょっぱさ. }

\par{5. 高み, 低み, 深み, and 浅み refer to places of a certain depth. }

\par{6. 臭み may be used figuratively. }

\par{7. うまみ has several meanings, which include "good taste", "the fifth category of taste 'umami'", "skill", and "profit (商売)". It's interesting to note that umami has been introduced into English. }

\par{8. 暗み no longer exists from the adjective 暗い, though it is productive when derived from 暗む. }

\par{9. ありがたさ = gratefulness; ありがたみ = value; importance. }

\par{10. 重み may refer to a weight that feels heavy, heavy in the feeling sense, or heavy in the abstract sense. Thus, as you can imagine, 強み and 弱み are based on feeling rather than on concrete strength. This sense is also true for 厚み, 温かみ, 柔らかみ, and 温み. }
 
\begin{center}
 \textbf{Examples }
\end{center}

\par{15. 赤味がかったもの \hfill\break
Something that's reddish }

\par{16. 深みにはまる。 \hfill\break
To go in too far. }

\par{17. 明るみに出された。 \hfill\break
It was brought to light. }

\par{18. ${\overset{\textnormal{へび}}{\text{蛇}}}$ が ${\overset{\textnormal{しげ}}{\text{茂}}}$ みの中でうごめいている。 \hfill\break
A snake is squirming in the grove. }

\par{\textbf{Word Note }: 茂み is an example where the original adjective is no longer in existence. }

\par{19. 母が亡くなって、初めて有難みを実感した。 \hfill\break
With the passing of my mother, I experienced gratefulness for the first time. }

\par{20. 新鮮味は欠けるが、多くの人を呼び寄せる企画だ。 \hfill\break
It lacks freshness, but it is a plan that will bring over a lot of people. }

\par{\textbf{Word Note }: 新鮮み may be viewed as an exceptional case, but 味 in this instance can also be viewed as literally being Sino-Japanese, in which case this would not be a suitable example. }

\par{21. 新鮮味に欠ける企画だ。 \hfill\break
It's a plan that lacks freshness. }
      
\section{~め・目}
 
\par{ ~め is yet another ending that may nominalize a subset of adjectives. It is even more restricted than み. It may be used with dimensional adjectives (次元形容詞) such as 長い, 短い, 太い, 近い, 遠い, 高い, 低い, 細い, 丸い, 厚い, 薄い, 深い, and 浅い. }

\par{ There are also other examples such as 多め, 早め, and 古め (not so common).It can also be with some color: 黒め; 白め. You have to be careful, though, because 赤目 does literally mean "red eye(s)". In speaking of which, 黒目 do mean "iris" and 白目 "white of the eye". So, using ひらがな instead is wise to differentiate meanings. These words in which there could be conflicting meanings are best used in phrases. In fact, all of thee words are most frequently used in larger phrases. }

\par{ The purpose of ~め is to show degree or tendency. And, it may also be seen after some verbs. Important examples include 控えめ and 落ち目. }

\par{22. 畑を深めに掘り返す。 \hfill\break
To tear up a field deep. }

\par{23. 肌が黒めの男性 \hfill\break
Dark-skinned males }

\par{24. 英語が多めのアニソン \hfill\break
Anime songs with a lot of English }

\par{25. 色素薄め \hfill\break
Light on pigment }

\par{26. 厚めの布 \hfill\break
A rather thick cloth }

\par{\textbf{Spelling Note }: There is no true difference in spelling if you use め or 目 aside from the color examples. However, it is less common when it attaches to adjectives. }
    