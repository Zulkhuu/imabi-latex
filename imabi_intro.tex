\chapter*{Introduction} % Introduction chapter suppressed from the table of contents

\begin{quote}
\begin{center}
The limits of my language are the limits of my world.\\
私の言語の限界は、私の世界の限界である。
\end{center}
\begin{flushright}
--Ludwig Wittgenstein
\end{flushright}
\end{quote}

\begin{comment}
\begin{quote}
\begin{center}
To have another language is to possess a second soul.\\
別の言語を話すことができるということは、第二の魂を持っているということ。
\end{center}
\begin{flushright}
--Charlemagne
\end{flushright}
\end{quote}
\end{comment}

\par{ }
\par{ The Japanese word for today is kyō. When written as 今日 with Chinese characters, it has other possible readings, one of which is oddly not used. That potential pronunciation became the source of inspiration for the name of this curriculum for learning Japanese. Changing "kyō" to "imabi" was done to symbolize a \textbf{new} today for learning Japanese.
}
\par{ IMABI has 421 lessons spanning from ground zero of knowing Japanese all the way to Classical Japanese if one so chooses to travel the entire journey. Curriculum content is constantly updated on a daily basis. Its usefulness is determined by not just the work I put into it, but also by the questions, concerns, and advice I get from users. If you wish for something to change to make things easier, that's where your voice is incredibly useful.
}